\chapter{EDA on Bearings Dataset}
\label{sec:bearings-eda}


\section{Introduction to EDA Chapter}

In this chapter, we discuss \ac{eda}, an indispensable step in data analysis. \ac{eda} focuses on gaining a comprehensive understanding of the data using descriptive statistics and graphical methods before the application of sophisticated statistical techniques or machine learning algorithms.

The purpose of our EDA is to delve into a dataset that includes measurements related to the performance of shaft bearings. These measurements include $Fr$ (radial force), $n$ (rotational speed), and $Lifetime$ (expected bearing lifetime). Through our \ac{eda}, we aim to understand the distributions of these variables and their interactions.

Such insights derived from \ac{eda} can be valuable for further data processing, hypothesis testing, feature selection, and machine learning model selection. In our specific context, understanding the factors that affect the lifetime of a bearing can help make informed decisions related to their operation and maintenance.

Our EDA process will be systematic and iterative, revising our understanding as new insights emerge. We will walk through an initial data inspection, followed by univariate and multivariate analyses, ultimately leading to feature engineering and hypothesis generation.


\section{Initial Data Inspection}

The first step in our \ac{eda} is the initial inspection of the dataset. This involves loading the data, examining a few records, identifying the type of each variable, checking for missing values, and noting the range of values each variable can take.

Our dataset is a structured one, with $Fr$ and $n$ recorded as integer values and $Lifetime$ as a floating-point number. A significant observation made at this point is the step-wise increment pattern seen in $Fr$ and $n$, and the corresponding decrement in $Lifetime$. Each unique combination of $Fr$ and $n$ represents different operational conditions for the bearing, and their influence on the bearing's expected $Lifetime$ is our primary point of interest.

No missing values are present in the dataset, which allows us to proceed with the analysis without the need for imputation or deletion strategies.


\section{Univariate Analysis}

Univariate analysis provides deeper insight into each feature independently. Its primary goal is to examine the distribution of each feature, detect potential outliers, and determine whether any transformations might be necessary.

As previously noted, the $Fr$ and $n$ variables show a structured, uniform distribution due to the step-wise pattern of data recording. This setup ensures a comprehensive coverage of various operational conditions.

The main focus of our univariate analysis is on the $Lifetime$ variable. A notable observation is that $Lifetime$ follows an exponential decrease pattern as $Fr$ and $n$ increase. This behavior suggests a potential non-linear relationship between $Lifetime$ and the operational parameters $Fr$ and $n$.

To better visualize the distribution of $Lifetime$, we employed histograms and boxplots (Figure \ref{fig:bearings-histogram} and Figure \ref{fig:bearings-boxplot}). Since $Lifetime$ features a long tail towards lower values, we transformed the data using a log function. This transformation helped us to bucket the $Lifetime$ data into more meaningful intervals for visualization, revealing a clearer distribution pattern.

\begin{figure}[ht]
    \centering
    \includegraphics[width=\textwidth]{assets/bearings-eda/histogram.png}
    \caption{Histogram of Lifetime}
    \label{fig:bearings-histogram}
\end{figure}

\begin{figure}[ht]
    \centering
    \includegraphics[width=\textwidth]{assets/bearings-eda/boxplot.png}
    \caption{Boxplot of Lifetime}
    \label{fig:bearings-boxplot}
\end{figure}

While the $Lifetime$ data features a wide range of values, this is to be expected given the nature of the measured attribute. The variation can be attributed to the range of operating conditions ($Fr$ and $n$) represented in the dataset. In this context, extreme values in $Lifetime$ are not necessarily "outliers" in the traditional sense, but rather indicate the broad spectrum of potential lifetimes for the shaft bearings under different conditions.

Overall, univariate analysis offers valuable insights into each feature's characteristics. It also paves the way for a more nuanced understanding of interactions between features in the subsequent multivariate analysis.


\section{Multivariate Analysis}

In the process of \ac{eda}, multivariate analysis plays a critical role in examining the interrelationships between different features. Simultaneously, the insights gleaned during this process can inform the creation of new features, enhancing our dataset and offering further explanatory power for our hypotheses.

To understand the relationships between the variables $Fr$, $n$, and $Lifetime$, we used various visualization techniques, including 3D plots, pair plots, and correlation matrices.

A 3D plot, representing the three variables, allowed us to perceive the nature of the interactions between them. We observed a significant decrease in $Lifetime$ with an increase in either $Fr$ or $n$. This trend is indicative of an inverse exponential relationship between these variables. The relationship was more pronounced with $Fr$, implying that $Fr$ had a greater impact on the $Lifetime$ compared to $n$.

\begin{figure}[ht]
    \centering
    \includegraphics[width=\textwidth]{assets/bearings-eda/3dplot.png}
    \caption{3D plot of Fr, n, and Lifetime}
    \label{fig:bearings-3dplot}
\end{figure}

Recognizing the potential exponential relationship, we engineered two new features, $Fr \cdot n$ and $Fr/n$, to further investigate these relationships. The choice of these features was motivated by their ability to reveal the interactions and the relative impact of $Fr$ and $n$ on $Lifetime$.

Subsequently, pair plots and correlation matrices (Figures \ref{fig:bearings-pairplot} and \ref{fig:bearings-corrmat}) were generated, incorporating the original and engineered features. These visualizations provided insights into the relationships and correlations between the variables, aiding in the derivation of the exponent in our hypothesis equation.

\begin{figure}[ht]
    \centering
    \includegraphics[width=\textwidth]{assets/bearings-eda/pairplot.png}
    \caption{Pairplot of variables}
    \label{fig:bearings-pairplot}
\end{figure}

\begin{figure}[ht]
    \centering
    \includegraphics[width=\textwidth]{assets/bearings-eda/correlation-matrix.png}
    \caption{Correlation matrix of variables}
    \label{fig:bearings-corrmat}
\end{figure}

The correlation matrix highlighted an intriguing characteristic of our engineered feature $Fr/n$. This feature exhibited a distinctive correlation with $Fr$, indicating a proportional relationship. From this, we could infer the relative weight of the exponents in our potential model equation by examining the inverse of this correlation. This gave us a critical clue towards the development of our hypothesis and shaped our understanding of the relationships within the dataset.

Through multivariate analysis and feature engineering, we've enriched our dataset and gained insights into the relationships between the variables. These findings will now serve as the foundation for the generation of our hypothesis in the next section.


\section{Hypothesis Generation}

During our \ac{eda}, we observed an inverse relationship between $Lifetime$ and both $Fr$ and $n$. This relationship suggested an exponential decay form, as depicted by the 3D plot. From this observation, we hypothesized that the relationship could be represented by a function of the form:

\begin{equation}
\label{eq:hypothesis}
{Lifetime} \approx c \cdot Fr^{-a} \cdot n^{-b}
\end{equation}

where $a$ and $b$ represent the relative influence of $Fr$ and $n$ on $Lifetime$. This exponential function indicates a multiplicative interaction between $Fr$ and $n$, each with an inverse relationship with $Lifetime$. The constant $c$ scales the function to the correct value range.

To further refine our hypothesis, we performed a sensitivity analysis. By investigating the response of the $Lifetime$ to a change in the $Fr$ and $n$ values, we found that when $Fr$ doubles, the $Lifetime$ reduces to approximately $\frac{1}{10.08}$ of the original. Conversely, when $n$ doubles, the $Lifetime$ becomes half. 

Using these observations, we can mathematically derive the weights $a$ and $b$. If doubling $Fr$ (replacing $Fr$ with $2Fr$) results in $Lifetime$ being roughly one-tenth of the original, it implies that $2^{-a} = \frac{1}{10.08}$. By solving this equation, we find $a = -\log_2(\frac{1}{10.08}) \approx \frac{10}{3}$. This value represents the weight of $Fr$ in our model function.

Similarly, if doubling $n$ (replacing $n$ with $2n$) results in $Lifetime$ being half of the original, it implies that $2^{-b} = 0.5$. Solving this equation, we find $b = -\log_2(0.5) = 1$.

Using these values for $a$ and $b$, we then calculated $c$ by comparing the actual $Lifetime$ values with $Fr^{-a} \cdot n^{-b}$ for the data point with the highest $Lifetime$. This gave us a constant value for $c$.

\begin{equation}
\label{eq:constant_c}
c \approx \frac{88445568.46}{200^{-10/3} \cdot 100^{-1}} \approx 4.13786 \cdot 10^{17}
\end{equation}

Our refined hypothesis for the function form thus became:

\begin{equation}
\label{eq:refined_hypothesis}
{Lifetime} \approx \frac{4.13786 \cdot 10^{17}}{Fr^{10/3} \cdot n}
\end{equation}

This hypothesis serves as a benchmark and starting point for our model. Moreover, by calculating the residuals between the $Lifetime$ and the function's output, we could assess the fit of our function. As shown in Figure (\ref{fig:bearings-residuals}), the residuals are generally extremely low in the range of the dataset, indicating that our function fits the data well. The z-axis, representing residuals, appears to show values between 6.232 and 6.222, but due to limitations in the plot's precision, these values are actually representing a range between $6.232\% \cdot 10^-5$ and $6.222\% \cdot 10^-5$. The correlation between the function's output and the actual $Lifetime$ is remarkably high, reaching a value of approximately 1.

\begin{figure}[ht]
    \centering
    \includegraphics[width=\textwidth]{assets/bearings-eda/3dplot-residuals.png}
    \caption{3D plot showing the residuals between the actual $Lifetime$ and the predicted values from our function}
    \label{fig:bearings-residuals}
\end{figure}

Observing the nearly constant difference between the actual $Lifetime$ and the function's output suggests that the approximation of the constant $c$ in our function is quite precise, given the minute scale of the residuals. This difference is in the order of $10^{-5}\%$, an incredibly small value. The minor variations in this difference may be attributed to the rounding of the $Lifetime$ values in the dataset. The extreme precision and predictability of the $Lifetime$ based on $Fr$ and $n$ alone suggest that the dataset may not originate from real-world measurements, but instead could be artificially generated. In a real-world scenario, other factors would likely contribute to the $Lifetime$ of the bearings, introducing more variability and uncertainty.

By establishing this function and assessing its fit to the data, we have a strong starting point and benchmark for any further modeling. In the subsequent stages of our analysis, we will build upon this foundation, adjusting and improving the model as required.
