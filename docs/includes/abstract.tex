\chapter*{Abstract}
\addcontentsline{toc}{chapter}{Abstract}

Predictive maintenance of machinery is critical for optimizing operational efficiency and minimizing downtime in various industries. With the advent of machine learning techniques, there is an increasing interest in harnessing these methods for predictive maintenance applications. This research explores the applicability of deep learning, specifically neural networks, in predicting the lifetime of ball bearings, which are crucial components in many mechanical systems. 
  
The study commences with an exploratory data analysis, which revealed an underlying equation for calculating the lifetime of ball bearings. This discovery was instrumental in generating an expanded artificial dataset for experimentation. Utilizing the PyTorch framework, a neural network model was developed and its performance was analyzed through systematic hyperparameter tuning via Grid Search. The TensorBoard visualization tool, along with the calculation of average performance metrics, was employed to assess the relationships between hyperparameters and the model's performance.
  
The findings demonstrate the potential of neural networks in categorizing ball bearings based on their expected lifetime, and thus serving as an effective tool for predictive maintenance. However, it is noted that the artificially generated dataset might not perfectly mimic real-world data, and as such, the performance of the model in practical scenarios remains unverified.
  
This research contributes to the field by providing a blueprint for developing, tuning, and evaluating neural network models for predictive maintenance, and underscores the importance of meticulous hyperparameter tuning. It is recommended that future research should focus on validating the model with real-world datasets and explore alternative machine learning approaches for comparative analysis.
