\chapter{Exploratory Data Analysis}
\label{sec:eda}

\ac{eda} is a statistical approach that involves analyzing and visualizing a dataset to understand its main characteristics and identify patterns and relationships within the data. The purpose of \ac{eda} is to gain insight into the data and detect any outliers, anomalies, or patterns that may be of interest. This is an essential step in the data analysis process, as it allows us to make informed decisions about the data and how it should be used in subsequent stages of the analysis.

This chapter introduces some common tools used in \ac{eda}, based on the iris data set as an example. One of these tools is histograms, which provide a visual representation of the distribution of the data. A histogram consists of a series of bars, where each bar represents a range of values and the height of the bar represents the frequency of those values within the data. This information helps to understand the overall shape of the data, such as whether it is symmetric, skewed, or multimodal, as shown in figure \ref{fig:histogram}.

\begin{figure}
  \begin{center}
    \resizebox{0.8\textwidth}{!}{%% Creator: Matplotlib, PGF backend
%%
%% To include the figure in your LaTeX document, write
%%   \input{<filename>.pgf}
%%
%% Make sure the required packages are loaded in your preamble
%%   \usepackage{pgf}
%%
%% Also ensure that all the required font packages are loaded; for instance,
%% the lmodern package is sometimes necessary when using math font.
%%   \usepackage{lmodern}
%%
%% Figures using additional raster images can only be included by \input if
%% they are in the same directory as the main LaTeX file. For loading figures
%% from other directories you can use the `import` package
%%   \usepackage{import}
%%
%% and then include the figures with
%%   \import{<path to file>}{<filename>.pgf}
%%
%% Matplotlib used the following preamble
%%
\begingroup%
\makeatletter%
\begin{pgfpicture}%
\pgfpathrectangle{\pgfpointorigin}{\pgfqpoint{6.400000in}{4.800000in}}%
\pgfusepath{use as bounding box, clip}%
\begin{pgfscope}%
\pgfsetbuttcap%
\pgfsetmiterjoin%
\definecolor{currentfill}{rgb}{1.000000,1.000000,1.000000}%
\pgfsetfillcolor{currentfill}%
\pgfsetlinewidth{0.000000pt}%
\definecolor{currentstroke}{rgb}{1.000000,1.000000,1.000000}%
\pgfsetstrokecolor{currentstroke}%
\pgfsetdash{}{0pt}%
\pgfpathmoveto{\pgfqpoint{0.000000in}{0.000000in}}%
\pgfpathlineto{\pgfqpoint{6.400000in}{0.000000in}}%
\pgfpathlineto{\pgfqpoint{6.400000in}{4.800000in}}%
\pgfpathlineto{\pgfqpoint{0.000000in}{4.800000in}}%
\pgfpathlineto{\pgfqpoint{0.000000in}{0.000000in}}%
\pgfpathclose%
\pgfusepath{fill}%
\end{pgfscope}%
\begin{pgfscope}%
\pgfsetbuttcap%
\pgfsetmiterjoin%
\definecolor{currentfill}{rgb}{1.000000,1.000000,1.000000}%
\pgfsetfillcolor{currentfill}%
\pgfsetlinewidth{0.000000pt}%
\definecolor{currentstroke}{rgb}{0.000000,0.000000,0.000000}%
\pgfsetstrokecolor{currentstroke}%
\pgfsetstrokeopacity{0.000000}%
\pgfsetdash{}{0pt}%
\pgfpathmoveto{\pgfqpoint{0.800000in}{2.617043in}}%
\pgfpathlineto{\pgfqpoint{2.956522in}{2.617043in}}%
\pgfpathlineto{\pgfqpoint{2.956522in}{4.224000in}}%
\pgfpathlineto{\pgfqpoint{0.800000in}{4.224000in}}%
\pgfpathlineto{\pgfqpoint{0.800000in}{2.617043in}}%
\pgfpathclose%
\pgfusepath{fill}%
\end{pgfscope}%
\begin{pgfscope}%
\pgfpathrectangle{\pgfqpoint{0.800000in}{2.617043in}}{\pgfqpoint{2.156522in}{1.606957in}}%
\pgfusepath{clip}%
\pgfsetbuttcap%
\pgfsetmiterjoin%
\definecolor{currentfill}{rgb}{0.121569,0.466667,0.705882}%
\pgfsetfillcolor{currentfill}%
\pgfsetlinewidth{0.000000pt}%
\definecolor{currentstroke}{rgb}{0.000000,0.000000,0.000000}%
\pgfsetstrokecolor{currentstroke}%
\pgfsetstrokeopacity{0.000000}%
\pgfsetdash{}{0pt}%
\pgfpathmoveto{\pgfqpoint{0.898024in}{2.617043in}}%
\pgfpathlineto{\pgfqpoint{1.094071in}{2.617043in}}%
\pgfpathlineto{\pgfqpoint{1.094071in}{3.127188in}}%
\pgfpathlineto{\pgfqpoint{0.898024in}{3.127188in}}%
\pgfpathlineto{\pgfqpoint{0.898024in}{2.617043in}}%
\pgfpathclose%
\pgfusepath{fill}%
\end{pgfscope}%
\begin{pgfscope}%
\pgfpathrectangle{\pgfqpoint{0.800000in}{2.617043in}}{\pgfqpoint{2.156522in}{1.606957in}}%
\pgfusepath{clip}%
\pgfsetbuttcap%
\pgfsetmiterjoin%
\definecolor{currentfill}{rgb}{0.121569,0.466667,0.705882}%
\pgfsetfillcolor{currentfill}%
\pgfsetlinewidth{0.000000pt}%
\definecolor{currentstroke}{rgb}{0.000000,0.000000,0.000000}%
\pgfsetstrokecolor{currentstroke}%
\pgfsetstrokeopacity{0.000000}%
\pgfsetdash{}{0pt}%
\pgfpathmoveto{\pgfqpoint{1.094071in}{2.617043in}}%
\pgfpathlineto{\pgfqpoint{1.290119in}{2.617043in}}%
\pgfpathlineto{\pgfqpoint{1.290119in}{3.920747in}}%
\pgfpathlineto{\pgfqpoint{1.094071in}{3.920747in}}%
\pgfpathlineto{\pgfqpoint{1.094071in}{2.617043in}}%
\pgfpathclose%
\pgfusepath{fill}%
\end{pgfscope}%
\begin{pgfscope}%
\pgfpathrectangle{\pgfqpoint{0.800000in}{2.617043in}}{\pgfqpoint{2.156522in}{1.606957in}}%
\pgfusepath{clip}%
\pgfsetbuttcap%
\pgfsetmiterjoin%
\definecolor{currentfill}{rgb}{0.121569,0.466667,0.705882}%
\pgfsetfillcolor{currentfill}%
\pgfsetlinewidth{0.000000pt}%
\definecolor{currentstroke}{rgb}{0.000000,0.000000,0.000000}%
\pgfsetstrokecolor{currentstroke}%
\pgfsetstrokeopacity{0.000000}%
\pgfsetdash{}{0pt}%
\pgfpathmoveto{\pgfqpoint{1.290119in}{2.617043in}}%
\pgfpathlineto{\pgfqpoint{1.486166in}{2.617043in}}%
\pgfpathlineto{\pgfqpoint{1.486166in}{3.410602in}}%
\pgfpathlineto{\pgfqpoint{1.290119in}{3.410602in}}%
\pgfpathlineto{\pgfqpoint{1.290119in}{2.617043in}}%
\pgfpathclose%
\pgfusepath{fill}%
\end{pgfscope}%
\begin{pgfscope}%
\pgfpathrectangle{\pgfqpoint{0.800000in}{2.617043in}}{\pgfqpoint{2.156522in}{1.606957in}}%
\pgfusepath{clip}%
\pgfsetbuttcap%
\pgfsetmiterjoin%
\definecolor{currentfill}{rgb}{0.121569,0.466667,0.705882}%
\pgfsetfillcolor{currentfill}%
\pgfsetlinewidth{0.000000pt}%
\definecolor{currentstroke}{rgb}{0.000000,0.000000,0.000000}%
\pgfsetstrokecolor{currentstroke}%
\pgfsetstrokeopacity{0.000000}%
\pgfsetdash{}{0pt}%
\pgfpathmoveto{\pgfqpoint{1.486166in}{2.617043in}}%
\pgfpathlineto{\pgfqpoint{1.682213in}{2.617043in}}%
\pgfpathlineto{\pgfqpoint{1.682213in}{4.147478in}}%
\pgfpathlineto{\pgfqpoint{1.486166in}{4.147478in}}%
\pgfpathlineto{\pgfqpoint{1.486166in}{2.617043in}}%
\pgfpathclose%
\pgfusepath{fill}%
\end{pgfscope}%
\begin{pgfscope}%
\pgfpathrectangle{\pgfqpoint{0.800000in}{2.617043in}}{\pgfqpoint{2.156522in}{1.606957in}}%
\pgfusepath{clip}%
\pgfsetbuttcap%
\pgfsetmiterjoin%
\definecolor{currentfill}{rgb}{0.121569,0.466667,0.705882}%
\pgfsetfillcolor{currentfill}%
\pgfsetlinewidth{0.000000pt}%
\definecolor{currentstroke}{rgb}{0.000000,0.000000,0.000000}%
\pgfsetstrokecolor{currentstroke}%
\pgfsetstrokeopacity{0.000000}%
\pgfsetdash{}{0pt}%
\pgfpathmoveto{\pgfqpoint{1.682213in}{2.617043in}}%
\pgfpathlineto{\pgfqpoint{1.878261in}{2.617043in}}%
\pgfpathlineto{\pgfqpoint{1.878261in}{3.523968in}}%
\pgfpathlineto{\pgfqpoint{1.682213in}{3.523968in}}%
\pgfpathlineto{\pgfqpoint{1.682213in}{2.617043in}}%
\pgfpathclose%
\pgfusepath{fill}%
\end{pgfscope}%
\begin{pgfscope}%
\pgfpathrectangle{\pgfqpoint{0.800000in}{2.617043in}}{\pgfqpoint{2.156522in}{1.606957in}}%
\pgfusepath{clip}%
\pgfsetbuttcap%
\pgfsetmiterjoin%
\definecolor{currentfill}{rgb}{0.121569,0.466667,0.705882}%
\pgfsetfillcolor{currentfill}%
\pgfsetlinewidth{0.000000pt}%
\definecolor{currentstroke}{rgb}{0.000000,0.000000,0.000000}%
\pgfsetstrokecolor{currentstroke}%
\pgfsetstrokeopacity{0.000000}%
\pgfsetdash{}{0pt}%
\pgfpathmoveto{\pgfqpoint{1.878261in}{2.617043in}}%
\pgfpathlineto{\pgfqpoint{2.074308in}{2.617043in}}%
\pgfpathlineto{\pgfqpoint{2.074308in}{4.090795in}}%
\pgfpathlineto{\pgfqpoint{1.878261in}{4.090795in}}%
\pgfpathlineto{\pgfqpoint{1.878261in}{2.617043in}}%
\pgfpathclose%
\pgfusepath{fill}%
\end{pgfscope}%
\begin{pgfscope}%
\pgfpathrectangle{\pgfqpoint{0.800000in}{2.617043in}}{\pgfqpoint{2.156522in}{1.606957in}}%
\pgfusepath{clip}%
\pgfsetbuttcap%
\pgfsetmiterjoin%
\definecolor{currentfill}{rgb}{0.121569,0.466667,0.705882}%
\pgfsetfillcolor{currentfill}%
\pgfsetlinewidth{0.000000pt}%
\definecolor{currentstroke}{rgb}{0.000000,0.000000,0.000000}%
\pgfsetstrokecolor{currentstroke}%
\pgfsetstrokeopacity{0.000000}%
\pgfsetdash{}{0pt}%
\pgfpathmoveto{\pgfqpoint{2.074308in}{2.617043in}}%
\pgfpathlineto{\pgfqpoint{2.270356in}{2.617043in}}%
\pgfpathlineto{\pgfqpoint{2.270356in}{3.637333in}}%
\pgfpathlineto{\pgfqpoint{2.074308in}{3.637333in}}%
\pgfpathlineto{\pgfqpoint{2.074308in}{2.617043in}}%
\pgfpathclose%
\pgfusepath{fill}%
\end{pgfscope}%
\begin{pgfscope}%
\pgfpathrectangle{\pgfqpoint{0.800000in}{2.617043in}}{\pgfqpoint{2.156522in}{1.606957in}}%
\pgfusepath{clip}%
\pgfsetbuttcap%
\pgfsetmiterjoin%
\definecolor{currentfill}{rgb}{0.121569,0.466667,0.705882}%
\pgfsetfillcolor{currentfill}%
\pgfsetlinewidth{0.000000pt}%
\definecolor{currentstroke}{rgb}{0.000000,0.000000,0.000000}%
\pgfsetstrokecolor{currentstroke}%
\pgfsetstrokeopacity{0.000000}%
\pgfsetdash{}{0pt}%
\pgfpathmoveto{\pgfqpoint{2.270356in}{2.617043in}}%
\pgfpathlineto{\pgfqpoint{2.466403in}{2.617043in}}%
\pgfpathlineto{\pgfqpoint{2.466403in}{2.957140in}}%
\pgfpathlineto{\pgfqpoint{2.270356in}{2.957140in}}%
\pgfpathlineto{\pgfqpoint{2.270356in}{2.617043in}}%
\pgfpathclose%
\pgfusepath{fill}%
\end{pgfscope}%
\begin{pgfscope}%
\pgfpathrectangle{\pgfqpoint{0.800000in}{2.617043in}}{\pgfqpoint{2.156522in}{1.606957in}}%
\pgfusepath{clip}%
\pgfsetbuttcap%
\pgfsetmiterjoin%
\definecolor{currentfill}{rgb}{0.121569,0.466667,0.705882}%
\pgfsetfillcolor{currentfill}%
\pgfsetlinewidth{0.000000pt}%
\definecolor{currentstroke}{rgb}{0.000000,0.000000,0.000000}%
\pgfsetstrokecolor{currentstroke}%
\pgfsetstrokeopacity{0.000000}%
\pgfsetdash{}{0pt}%
\pgfpathmoveto{\pgfqpoint{2.466403in}{2.617043in}}%
\pgfpathlineto{\pgfqpoint{2.662451in}{2.617043in}}%
\pgfpathlineto{\pgfqpoint{2.662451in}{2.900457in}}%
\pgfpathlineto{\pgfqpoint{2.466403in}{2.900457in}}%
\pgfpathlineto{\pgfqpoint{2.466403in}{2.617043in}}%
\pgfpathclose%
\pgfusepath{fill}%
\end{pgfscope}%
\begin{pgfscope}%
\pgfpathrectangle{\pgfqpoint{0.800000in}{2.617043in}}{\pgfqpoint{2.156522in}{1.606957in}}%
\pgfusepath{clip}%
\pgfsetbuttcap%
\pgfsetmiterjoin%
\definecolor{currentfill}{rgb}{0.121569,0.466667,0.705882}%
\pgfsetfillcolor{currentfill}%
\pgfsetlinewidth{0.000000pt}%
\definecolor{currentstroke}{rgb}{0.000000,0.000000,0.000000}%
\pgfsetstrokecolor{currentstroke}%
\pgfsetstrokeopacity{0.000000}%
\pgfsetdash{}{0pt}%
\pgfpathmoveto{\pgfqpoint{2.662451in}{2.617043in}}%
\pgfpathlineto{\pgfqpoint{2.858498in}{2.617043in}}%
\pgfpathlineto{\pgfqpoint{2.858498in}{2.957140in}}%
\pgfpathlineto{\pgfqpoint{2.662451in}{2.957140in}}%
\pgfpathlineto{\pgfqpoint{2.662451in}{2.617043in}}%
\pgfpathclose%
\pgfusepath{fill}%
\end{pgfscope}%
\begin{pgfscope}%
\pgfpathrectangle{\pgfqpoint{0.800000in}{2.617043in}}{\pgfqpoint{2.156522in}{1.606957in}}%
\pgfusepath{clip}%
\pgfsetrectcap%
\pgfsetroundjoin%
\pgfsetlinewidth{0.803000pt}%
\definecolor{currentstroke}{rgb}{0.690196,0.690196,0.690196}%
\pgfsetstrokecolor{currentstroke}%
\pgfsetdash{}{0pt}%
\pgfpathmoveto{\pgfqpoint{1.279227in}{2.617043in}}%
\pgfpathlineto{\pgfqpoint{1.279227in}{4.224000in}}%
\pgfusepath{stroke}%
\end{pgfscope}%
\begin{pgfscope}%
\pgfsetbuttcap%
\pgfsetroundjoin%
\definecolor{currentfill}{rgb}{0.000000,0.000000,0.000000}%
\pgfsetfillcolor{currentfill}%
\pgfsetlinewidth{0.803000pt}%
\definecolor{currentstroke}{rgb}{0.000000,0.000000,0.000000}%
\pgfsetstrokecolor{currentstroke}%
\pgfsetdash{}{0pt}%
\pgfsys@defobject{currentmarker}{\pgfqpoint{0.000000in}{-0.048611in}}{\pgfqpoint{0.000000in}{0.000000in}}{%
\pgfpathmoveto{\pgfqpoint{0.000000in}{0.000000in}}%
\pgfpathlineto{\pgfqpoint{0.000000in}{-0.048611in}}%
\pgfusepath{stroke,fill}%
}%
\begin{pgfscope}%
\pgfsys@transformshift{1.279227in}{2.617043in}%
\pgfsys@useobject{currentmarker}{}%
\end{pgfscope}%
\end{pgfscope}%
\begin{pgfscope}%
\definecolor{textcolor}{rgb}{0.000000,0.000000,0.000000}%
\pgfsetstrokecolor{textcolor}%
\pgfsetfillcolor{textcolor}%
\pgftext[x=1.279227in,y=2.519821in,,top]{\color{textcolor}\rmfamily\fontsize{10.000000}{12.000000}\selectfont \(\displaystyle {5}\)}%
\end{pgfscope}%
\begin{pgfscope}%
\pgfpathrectangle{\pgfqpoint{0.800000in}{2.617043in}}{\pgfqpoint{2.156522in}{1.606957in}}%
\pgfusepath{clip}%
\pgfsetrectcap%
\pgfsetroundjoin%
\pgfsetlinewidth{0.803000pt}%
\definecolor{currentstroke}{rgb}{0.690196,0.690196,0.690196}%
\pgfsetstrokecolor{currentstroke}%
\pgfsetdash{}{0pt}%
\pgfpathmoveto{\pgfqpoint{1.823803in}{2.617043in}}%
\pgfpathlineto{\pgfqpoint{1.823803in}{4.224000in}}%
\pgfusepath{stroke}%
\end{pgfscope}%
\begin{pgfscope}%
\pgfsetbuttcap%
\pgfsetroundjoin%
\definecolor{currentfill}{rgb}{0.000000,0.000000,0.000000}%
\pgfsetfillcolor{currentfill}%
\pgfsetlinewidth{0.803000pt}%
\definecolor{currentstroke}{rgb}{0.000000,0.000000,0.000000}%
\pgfsetstrokecolor{currentstroke}%
\pgfsetdash{}{0pt}%
\pgfsys@defobject{currentmarker}{\pgfqpoint{0.000000in}{-0.048611in}}{\pgfqpoint{0.000000in}{0.000000in}}{%
\pgfpathmoveto{\pgfqpoint{0.000000in}{0.000000in}}%
\pgfpathlineto{\pgfqpoint{0.000000in}{-0.048611in}}%
\pgfusepath{stroke,fill}%
}%
\begin{pgfscope}%
\pgfsys@transformshift{1.823803in}{2.617043in}%
\pgfsys@useobject{currentmarker}{}%
\end{pgfscope}%
\end{pgfscope}%
\begin{pgfscope}%
\definecolor{textcolor}{rgb}{0.000000,0.000000,0.000000}%
\pgfsetstrokecolor{textcolor}%
\pgfsetfillcolor{textcolor}%
\pgftext[x=1.823803in,y=2.519821in,,top]{\color{textcolor}\rmfamily\fontsize{10.000000}{12.000000}\selectfont \(\displaystyle {6}\)}%
\end{pgfscope}%
\begin{pgfscope}%
\pgfpathrectangle{\pgfqpoint{0.800000in}{2.617043in}}{\pgfqpoint{2.156522in}{1.606957in}}%
\pgfusepath{clip}%
\pgfsetrectcap%
\pgfsetroundjoin%
\pgfsetlinewidth{0.803000pt}%
\definecolor{currentstroke}{rgb}{0.690196,0.690196,0.690196}%
\pgfsetstrokecolor{currentstroke}%
\pgfsetdash{}{0pt}%
\pgfpathmoveto{\pgfqpoint{2.368379in}{2.617043in}}%
\pgfpathlineto{\pgfqpoint{2.368379in}{4.224000in}}%
\pgfusepath{stroke}%
\end{pgfscope}%
\begin{pgfscope}%
\pgfsetbuttcap%
\pgfsetroundjoin%
\definecolor{currentfill}{rgb}{0.000000,0.000000,0.000000}%
\pgfsetfillcolor{currentfill}%
\pgfsetlinewidth{0.803000pt}%
\definecolor{currentstroke}{rgb}{0.000000,0.000000,0.000000}%
\pgfsetstrokecolor{currentstroke}%
\pgfsetdash{}{0pt}%
\pgfsys@defobject{currentmarker}{\pgfqpoint{0.000000in}{-0.048611in}}{\pgfqpoint{0.000000in}{0.000000in}}{%
\pgfpathmoveto{\pgfqpoint{0.000000in}{0.000000in}}%
\pgfpathlineto{\pgfqpoint{0.000000in}{-0.048611in}}%
\pgfusepath{stroke,fill}%
}%
\begin{pgfscope}%
\pgfsys@transformshift{2.368379in}{2.617043in}%
\pgfsys@useobject{currentmarker}{}%
\end{pgfscope}%
\end{pgfscope}%
\begin{pgfscope}%
\definecolor{textcolor}{rgb}{0.000000,0.000000,0.000000}%
\pgfsetstrokecolor{textcolor}%
\pgfsetfillcolor{textcolor}%
\pgftext[x=2.368379in,y=2.519821in,,top]{\color{textcolor}\rmfamily\fontsize{10.000000}{12.000000}\selectfont \(\displaystyle {7}\)}%
\end{pgfscope}%
\begin{pgfscope}%
\pgfpathrectangle{\pgfqpoint{0.800000in}{2.617043in}}{\pgfqpoint{2.156522in}{1.606957in}}%
\pgfusepath{clip}%
\pgfsetrectcap%
\pgfsetroundjoin%
\pgfsetlinewidth{0.803000pt}%
\definecolor{currentstroke}{rgb}{0.690196,0.690196,0.690196}%
\pgfsetstrokecolor{currentstroke}%
\pgfsetdash{}{0pt}%
\pgfpathmoveto{\pgfqpoint{2.912956in}{2.617043in}}%
\pgfpathlineto{\pgfqpoint{2.912956in}{4.224000in}}%
\pgfusepath{stroke}%
\end{pgfscope}%
\begin{pgfscope}%
\pgfsetbuttcap%
\pgfsetroundjoin%
\definecolor{currentfill}{rgb}{0.000000,0.000000,0.000000}%
\pgfsetfillcolor{currentfill}%
\pgfsetlinewidth{0.803000pt}%
\definecolor{currentstroke}{rgb}{0.000000,0.000000,0.000000}%
\pgfsetstrokecolor{currentstroke}%
\pgfsetdash{}{0pt}%
\pgfsys@defobject{currentmarker}{\pgfqpoint{0.000000in}{-0.048611in}}{\pgfqpoint{0.000000in}{0.000000in}}{%
\pgfpathmoveto{\pgfqpoint{0.000000in}{0.000000in}}%
\pgfpathlineto{\pgfqpoint{0.000000in}{-0.048611in}}%
\pgfusepath{stroke,fill}%
}%
\begin{pgfscope}%
\pgfsys@transformshift{2.912956in}{2.617043in}%
\pgfsys@useobject{currentmarker}{}%
\end{pgfscope}%
\end{pgfscope}%
\begin{pgfscope}%
\definecolor{textcolor}{rgb}{0.000000,0.000000,0.000000}%
\pgfsetstrokecolor{textcolor}%
\pgfsetfillcolor{textcolor}%
\pgftext[x=2.912956in,y=2.519821in,,top]{\color{textcolor}\rmfamily\fontsize{10.000000}{12.000000}\selectfont \(\displaystyle {8}\)}%
\end{pgfscope}%
\begin{pgfscope}%
\pgfpathrectangle{\pgfqpoint{0.800000in}{2.617043in}}{\pgfqpoint{2.156522in}{1.606957in}}%
\pgfusepath{clip}%
\pgfsetrectcap%
\pgfsetroundjoin%
\pgfsetlinewidth{0.803000pt}%
\definecolor{currentstroke}{rgb}{0.690196,0.690196,0.690196}%
\pgfsetstrokecolor{currentstroke}%
\pgfsetdash{}{0pt}%
\pgfpathmoveto{\pgfqpoint{0.800000in}{2.617043in}}%
\pgfpathlineto{\pgfqpoint{2.956522in}{2.617043in}}%
\pgfusepath{stroke}%
\end{pgfscope}%
\begin{pgfscope}%
\pgfsetbuttcap%
\pgfsetroundjoin%
\definecolor{currentfill}{rgb}{0.000000,0.000000,0.000000}%
\pgfsetfillcolor{currentfill}%
\pgfsetlinewidth{0.803000pt}%
\definecolor{currentstroke}{rgb}{0.000000,0.000000,0.000000}%
\pgfsetstrokecolor{currentstroke}%
\pgfsetdash{}{0pt}%
\pgfsys@defobject{currentmarker}{\pgfqpoint{-0.048611in}{0.000000in}}{\pgfqpoint{-0.000000in}{0.000000in}}{%
\pgfpathmoveto{\pgfqpoint{-0.000000in}{0.000000in}}%
\pgfpathlineto{\pgfqpoint{-0.048611in}{0.000000in}}%
\pgfusepath{stroke,fill}%
}%
\begin{pgfscope}%
\pgfsys@transformshift{0.800000in}{2.617043in}%
\pgfsys@useobject{currentmarker}{}%
\end{pgfscope}%
\end{pgfscope}%
\begin{pgfscope}%
\definecolor{textcolor}{rgb}{0.000000,0.000000,0.000000}%
\pgfsetstrokecolor{textcolor}%
\pgfsetfillcolor{textcolor}%
\pgftext[x=0.633333in, y=2.568818in, left, base]{\color{textcolor}\rmfamily\fontsize{10.000000}{12.000000}\selectfont \(\displaystyle {0}\)}%
\end{pgfscope}%
\begin{pgfscope}%
\pgfpathrectangle{\pgfqpoint{0.800000in}{2.617043in}}{\pgfqpoint{2.156522in}{1.606957in}}%
\pgfusepath{clip}%
\pgfsetrectcap%
\pgfsetroundjoin%
\pgfsetlinewidth{0.803000pt}%
\definecolor{currentstroke}{rgb}{0.690196,0.690196,0.690196}%
\pgfsetstrokecolor{currentstroke}%
\pgfsetdash{}{0pt}%
\pgfpathmoveto{\pgfqpoint{0.800000in}{3.183871in}}%
\pgfpathlineto{\pgfqpoint{2.956522in}{3.183871in}}%
\pgfusepath{stroke}%
\end{pgfscope}%
\begin{pgfscope}%
\pgfsetbuttcap%
\pgfsetroundjoin%
\definecolor{currentfill}{rgb}{0.000000,0.000000,0.000000}%
\pgfsetfillcolor{currentfill}%
\pgfsetlinewidth{0.803000pt}%
\definecolor{currentstroke}{rgb}{0.000000,0.000000,0.000000}%
\pgfsetstrokecolor{currentstroke}%
\pgfsetdash{}{0pt}%
\pgfsys@defobject{currentmarker}{\pgfqpoint{-0.048611in}{0.000000in}}{\pgfqpoint{-0.000000in}{0.000000in}}{%
\pgfpathmoveto{\pgfqpoint{-0.000000in}{0.000000in}}%
\pgfpathlineto{\pgfqpoint{-0.048611in}{0.000000in}}%
\pgfusepath{stroke,fill}%
}%
\begin{pgfscope}%
\pgfsys@transformshift{0.800000in}{3.183871in}%
\pgfsys@useobject{currentmarker}{}%
\end{pgfscope}%
\end{pgfscope}%
\begin{pgfscope}%
\definecolor{textcolor}{rgb}{0.000000,0.000000,0.000000}%
\pgfsetstrokecolor{textcolor}%
\pgfsetfillcolor{textcolor}%
\pgftext[x=0.563888in, y=3.135646in, left, base]{\color{textcolor}\rmfamily\fontsize{10.000000}{12.000000}\selectfont \(\displaystyle {10}\)}%
\end{pgfscope}%
\begin{pgfscope}%
\pgfpathrectangle{\pgfqpoint{0.800000in}{2.617043in}}{\pgfqpoint{2.156522in}{1.606957in}}%
\pgfusepath{clip}%
\pgfsetrectcap%
\pgfsetroundjoin%
\pgfsetlinewidth{0.803000pt}%
\definecolor{currentstroke}{rgb}{0.690196,0.690196,0.690196}%
\pgfsetstrokecolor{currentstroke}%
\pgfsetdash{}{0pt}%
\pgfpathmoveto{\pgfqpoint{0.800000in}{3.750699in}}%
\pgfpathlineto{\pgfqpoint{2.956522in}{3.750699in}}%
\pgfusepath{stroke}%
\end{pgfscope}%
\begin{pgfscope}%
\pgfsetbuttcap%
\pgfsetroundjoin%
\definecolor{currentfill}{rgb}{0.000000,0.000000,0.000000}%
\pgfsetfillcolor{currentfill}%
\pgfsetlinewidth{0.803000pt}%
\definecolor{currentstroke}{rgb}{0.000000,0.000000,0.000000}%
\pgfsetstrokecolor{currentstroke}%
\pgfsetdash{}{0pt}%
\pgfsys@defobject{currentmarker}{\pgfqpoint{-0.048611in}{0.000000in}}{\pgfqpoint{-0.000000in}{0.000000in}}{%
\pgfpathmoveto{\pgfqpoint{-0.000000in}{0.000000in}}%
\pgfpathlineto{\pgfqpoint{-0.048611in}{0.000000in}}%
\pgfusepath{stroke,fill}%
}%
\begin{pgfscope}%
\pgfsys@transformshift{0.800000in}{3.750699in}%
\pgfsys@useobject{currentmarker}{}%
\end{pgfscope}%
\end{pgfscope}%
\begin{pgfscope}%
\definecolor{textcolor}{rgb}{0.000000,0.000000,0.000000}%
\pgfsetstrokecolor{textcolor}%
\pgfsetfillcolor{textcolor}%
\pgftext[x=0.563888in, y=3.702474in, left, base]{\color{textcolor}\rmfamily\fontsize{10.000000}{12.000000}\selectfont \(\displaystyle {20}\)}%
\end{pgfscope}%
\begin{pgfscope}%
\pgfsetrectcap%
\pgfsetmiterjoin%
\pgfsetlinewidth{0.803000pt}%
\definecolor{currentstroke}{rgb}{0.000000,0.000000,0.000000}%
\pgfsetstrokecolor{currentstroke}%
\pgfsetdash{}{0pt}%
\pgfpathmoveto{\pgfqpoint{0.800000in}{2.617043in}}%
\pgfpathlineto{\pgfqpoint{0.800000in}{4.224000in}}%
\pgfusepath{stroke}%
\end{pgfscope}%
\begin{pgfscope}%
\pgfsetrectcap%
\pgfsetmiterjoin%
\pgfsetlinewidth{0.803000pt}%
\definecolor{currentstroke}{rgb}{0.000000,0.000000,0.000000}%
\pgfsetstrokecolor{currentstroke}%
\pgfsetdash{}{0pt}%
\pgfpathmoveto{\pgfqpoint{2.956522in}{2.617043in}}%
\pgfpathlineto{\pgfqpoint{2.956522in}{4.224000in}}%
\pgfusepath{stroke}%
\end{pgfscope}%
\begin{pgfscope}%
\pgfsetrectcap%
\pgfsetmiterjoin%
\pgfsetlinewidth{0.803000pt}%
\definecolor{currentstroke}{rgb}{0.000000,0.000000,0.000000}%
\pgfsetstrokecolor{currentstroke}%
\pgfsetdash{}{0pt}%
\pgfpathmoveto{\pgfqpoint{0.800000in}{2.617043in}}%
\pgfpathlineto{\pgfqpoint{2.956522in}{2.617043in}}%
\pgfusepath{stroke}%
\end{pgfscope}%
\begin{pgfscope}%
\pgfsetrectcap%
\pgfsetmiterjoin%
\pgfsetlinewidth{0.803000pt}%
\definecolor{currentstroke}{rgb}{0.000000,0.000000,0.000000}%
\pgfsetstrokecolor{currentstroke}%
\pgfsetdash{}{0pt}%
\pgfpathmoveto{\pgfqpoint{0.800000in}{4.224000in}}%
\pgfpathlineto{\pgfqpoint{2.956522in}{4.224000in}}%
\pgfusepath{stroke}%
\end{pgfscope}%
\begin{pgfscope}%
\definecolor{textcolor}{rgb}{0.000000,0.000000,0.000000}%
\pgfsetstrokecolor{textcolor}%
\pgfsetfillcolor{textcolor}%
\pgftext[x=1.878261in,y=4.307333in,,base]{\color{textcolor}\rmfamily\fontsize{12.000000}{14.400000}\selectfont sepal\_length}%
\end{pgfscope}%
\begin{pgfscope}%
\pgfsetbuttcap%
\pgfsetmiterjoin%
\definecolor{currentfill}{rgb}{1.000000,1.000000,1.000000}%
\pgfsetfillcolor{currentfill}%
\pgfsetlinewidth{0.000000pt}%
\definecolor{currentstroke}{rgb}{0.000000,0.000000,0.000000}%
\pgfsetstrokecolor{currentstroke}%
\pgfsetstrokeopacity{0.000000}%
\pgfsetdash{}{0pt}%
\pgfpathmoveto{\pgfqpoint{3.603478in}{2.617043in}}%
\pgfpathlineto{\pgfqpoint{5.760000in}{2.617043in}}%
\pgfpathlineto{\pgfqpoint{5.760000in}{4.224000in}}%
\pgfpathlineto{\pgfqpoint{3.603478in}{4.224000in}}%
\pgfpathlineto{\pgfqpoint{3.603478in}{2.617043in}}%
\pgfpathclose%
\pgfusepath{fill}%
\end{pgfscope}%
\begin{pgfscope}%
\pgfpathrectangle{\pgfqpoint{3.603478in}{2.617043in}}{\pgfqpoint{2.156522in}{1.606957in}}%
\pgfusepath{clip}%
\pgfsetbuttcap%
\pgfsetmiterjoin%
\definecolor{currentfill}{rgb}{0.121569,0.466667,0.705882}%
\pgfsetfillcolor{currentfill}%
\pgfsetlinewidth{0.000000pt}%
\definecolor{currentstroke}{rgb}{0.000000,0.000000,0.000000}%
\pgfsetstrokecolor{currentstroke}%
\pgfsetstrokeopacity{0.000000}%
\pgfsetdash{}{0pt}%
\pgfpathmoveto{\pgfqpoint{3.701502in}{2.617043in}}%
\pgfpathlineto{\pgfqpoint{3.897549in}{2.617043in}}%
\pgfpathlineto{\pgfqpoint{3.897549in}{2.782496in}}%
\pgfpathlineto{\pgfqpoint{3.701502in}{2.782496in}}%
\pgfpathlineto{\pgfqpoint{3.701502in}{2.617043in}}%
\pgfpathclose%
\pgfusepath{fill}%
\end{pgfscope}%
\begin{pgfscope}%
\pgfpathrectangle{\pgfqpoint{3.603478in}{2.617043in}}{\pgfqpoint{2.156522in}{1.606957in}}%
\pgfusepath{clip}%
\pgfsetbuttcap%
\pgfsetmiterjoin%
\definecolor{currentfill}{rgb}{0.121569,0.466667,0.705882}%
\pgfsetfillcolor{currentfill}%
\pgfsetlinewidth{0.000000pt}%
\definecolor{currentstroke}{rgb}{0.000000,0.000000,0.000000}%
\pgfsetstrokecolor{currentstroke}%
\pgfsetstrokeopacity{0.000000}%
\pgfsetdash{}{0pt}%
\pgfpathmoveto{\pgfqpoint{3.897549in}{2.617043in}}%
\pgfpathlineto{\pgfqpoint{4.093597in}{2.617043in}}%
\pgfpathlineto{\pgfqpoint{4.093597in}{2.906585in}}%
\pgfpathlineto{\pgfqpoint{3.897549in}{2.906585in}}%
\pgfpathlineto{\pgfqpoint{3.897549in}{2.617043in}}%
\pgfpathclose%
\pgfusepath{fill}%
\end{pgfscope}%
\begin{pgfscope}%
\pgfpathrectangle{\pgfqpoint{3.603478in}{2.617043in}}{\pgfqpoint{2.156522in}{1.606957in}}%
\pgfusepath{clip}%
\pgfsetbuttcap%
\pgfsetmiterjoin%
\definecolor{currentfill}{rgb}{0.121569,0.466667,0.705882}%
\pgfsetfillcolor{currentfill}%
\pgfsetlinewidth{0.000000pt}%
\definecolor{currentstroke}{rgb}{0.000000,0.000000,0.000000}%
\pgfsetstrokecolor{currentstroke}%
\pgfsetstrokeopacity{0.000000}%
\pgfsetdash{}{0pt}%
\pgfpathmoveto{\pgfqpoint{4.093597in}{2.617043in}}%
\pgfpathlineto{\pgfqpoint{4.289644in}{2.617043in}}%
\pgfpathlineto{\pgfqpoint{4.289644in}{3.527032in}}%
\pgfpathlineto{\pgfqpoint{4.093597in}{3.527032in}}%
\pgfpathlineto{\pgfqpoint{4.093597in}{2.617043in}}%
\pgfpathclose%
\pgfusepath{fill}%
\end{pgfscope}%
\begin{pgfscope}%
\pgfpathrectangle{\pgfqpoint{3.603478in}{2.617043in}}{\pgfqpoint{2.156522in}{1.606957in}}%
\pgfusepath{clip}%
\pgfsetbuttcap%
\pgfsetmiterjoin%
\definecolor{currentfill}{rgb}{0.121569,0.466667,0.705882}%
\pgfsetfillcolor{currentfill}%
\pgfsetlinewidth{0.000000pt}%
\definecolor{currentstroke}{rgb}{0.000000,0.000000,0.000000}%
\pgfsetstrokecolor{currentstroke}%
\pgfsetstrokeopacity{0.000000}%
\pgfsetdash{}{0pt}%
\pgfpathmoveto{\pgfqpoint{4.289644in}{2.617043in}}%
\pgfpathlineto{\pgfqpoint{4.485692in}{2.617043in}}%
\pgfpathlineto{\pgfqpoint{4.485692in}{3.609758in}}%
\pgfpathlineto{\pgfqpoint{4.289644in}{3.609758in}}%
\pgfpathlineto{\pgfqpoint{4.289644in}{2.617043in}}%
\pgfpathclose%
\pgfusepath{fill}%
\end{pgfscope}%
\begin{pgfscope}%
\pgfpathrectangle{\pgfqpoint{3.603478in}{2.617043in}}{\pgfqpoint{2.156522in}{1.606957in}}%
\pgfusepath{clip}%
\pgfsetbuttcap%
\pgfsetmiterjoin%
\definecolor{currentfill}{rgb}{0.121569,0.466667,0.705882}%
\pgfsetfillcolor{currentfill}%
\pgfsetlinewidth{0.000000pt}%
\definecolor{currentstroke}{rgb}{0.000000,0.000000,0.000000}%
\pgfsetstrokecolor{currentstroke}%
\pgfsetstrokeopacity{0.000000}%
\pgfsetdash{}{0pt}%
\pgfpathmoveto{\pgfqpoint{4.485692in}{2.617043in}}%
\pgfpathlineto{\pgfqpoint{4.681739in}{2.617043in}}%
\pgfpathlineto{\pgfqpoint{4.681739in}{4.147478in}}%
\pgfpathlineto{\pgfqpoint{4.485692in}{4.147478in}}%
\pgfpathlineto{\pgfqpoint{4.485692in}{2.617043in}}%
\pgfpathclose%
\pgfusepath{fill}%
\end{pgfscope}%
\begin{pgfscope}%
\pgfpathrectangle{\pgfqpoint{3.603478in}{2.617043in}}{\pgfqpoint{2.156522in}{1.606957in}}%
\pgfusepath{clip}%
\pgfsetbuttcap%
\pgfsetmiterjoin%
\definecolor{currentfill}{rgb}{0.121569,0.466667,0.705882}%
\pgfsetfillcolor{currentfill}%
\pgfsetlinewidth{0.000000pt}%
\definecolor{currentstroke}{rgb}{0.000000,0.000000,0.000000}%
\pgfsetstrokecolor{currentstroke}%
\pgfsetstrokeopacity{0.000000}%
\pgfsetdash{}{0pt}%
\pgfpathmoveto{\pgfqpoint{4.681739in}{2.617043in}}%
\pgfpathlineto{\pgfqpoint{4.877787in}{2.617043in}}%
\pgfpathlineto{\pgfqpoint{4.877787in}{3.899300in}}%
\pgfpathlineto{\pgfqpoint{4.681739in}{3.899300in}}%
\pgfpathlineto{\pgfqpoint{4.681739in}{2.617043in}}%
\pgfpathclose%
\pgfusepath{fill}%
\end{pgfscope}%
\begin{pgfscope}%
\pgfpathrectangle{\pgfqpoint{3.603478in}{2.617043in}}{\pgfqpoint{2.156522in}{1.606957in}}%
\pgfusepath{clip}%
\pgfsetbuttcap%
\pgfsetmiterjoin%
\definecolor{currentfill}{rgb}{0.121569,0.466667,0.705882}%
\pgfsetfillcolor{currentfill}%
\pgfsetlinewidth{0.000000pt}%
\definecolor{currentstroke}{rgb}{0.000000,0.000000,0.000000}%
\pgfsetstrokecolor{currentstroke}%
\pgfsetstrokeopacity{0.000000}%
\pgfsetdash{}{0pt}%
\pgfpathmoveto{\pgfqpoint{4.877787in}{2.617043in}}%
\pgfpathlineto{\pgfqpoint{5.073834in}{2.617043in}}%
\pgfpathlineto{\pgfqpoint{5.073834in}{3.030675in}}%
\pgfpathlineto{\pgfqpoint{4.877787in}{3.030675in}}%
\pgfpathlineto{\pgfqpoint{4.877787in}{2.617043in}}%
\pgfpathclose%
\pgfusepath{fill}%
\end{pgfscope}%
\begin{pgfscope}%
\pgfpathrectangle{\pgfqpoint{3.603478in}{2.617043in}}{\pgfqpoint{2.156522in}{1.606957in}}%
\pgfusepath{clip}%
\pgfsetbuttcap%
\pgfsetmiterjoin%
\definecolor{currentfill}{rgb}{0.121569,0.466667,0.705882}%
\pgfsetfillcolor{currentfill}%
\pgfsetlinewidth{0.000000pt}%
\definecolor{currentstroke}{rgb}{0.000000,0.000000,0.000000}%
\pgfsetstrokecolor{currentstroke}%
\pgfsetstrokeopacity{0.000000}%
\pgfsetdash{}{0pt}%
\pgfpathmoveto{\pgfqpoint{5.073834in}{2.617043in}}%
\pgfpathlineto{\pgfqpoint{5.269881in}{2.617043in}}%
\pgfpathlineto{\pgfqpoint{5.269881in}{3.072038in}}%
\pgfpathlineto{\pgfqpoint{5.073834in}{3.072038in}}%
\pgfpathlineto{\pgfqpoint{5.073834in}{2.617043in}}%
\pgfpathclose%
\pgfusepath{fill}%
\end{pgfscope}%
\begin{pgfscope}%
\pgfpathrectangle{\pgfqpoint{3.603478in}{2.617043in}}{\pgfqpoint{2.156522in}{1.606957in}}%
\pgfusepath{clip}%
\pgfsetbuttcap%
\pgfsetmiterjoin%
\definecolor{currentfill}{rgb}{0.121569,0.466667,0.705882}%
\pgfsetfillcolor{currentfill}%
\pgfsetlinewidth{0.000000pt}%
\definecolor{currentstroke}{rgb}{0.000000,0.000000,0.000000}%
\pgfsetstrokecolor{currentstroke}%
\pgfsetstrokeopacity{0.000000}%
\pgfsetdash{}{0pt}%
\pgfpathmoveto{\pgfqpoint{5.269881in}{2.617043in}}%
\pgfpathlineto{\pgfqpoint{5.465929in}{2.617043in}}%
\pgfpathlineto{\pgfqpoint{5.465929in}{2.699770in}}%
\pgfpathlineto{\pgfqpoint{5.269881in}{2.699770in}}%
\pgfpathlineto{\pgfqpoint{5.269881in}{2.617043in}}%
\pgfpathclose%
\pgfusepath{fill}%
\end{pgfscope}%
\begin{pgfscope}%
\pgfpathrectangle{\pgfqpoint{3.603478in}{2.617043in}}{\pgfqpoint{2.156522in}{1.606957in}}%
\pgfusepath{clip}%
\pgfsetbuttcap%
\pgfsetmiterjoin%
\definecolor{currentfill}{rgb}{0.121569,0.466667,0.705882}%
\pgfsetfillcolor{currentfill}%
\pgfsetlinewidth{0.000000pt}%
\definecolor{currentstroke}{rgb}{0.000000,0.000000,0.000000}%
\pgfsetstrokecolor{currentstroke}%
\pgfsetstrokeopacity{0.000000}%
\pgfsetdash{}{0pt}%
\pgfpathmoveto{\pgfqpoint{5.465929in}{2.617043in}}%
\pgfpathlineto{\pgfqpoint{5.661976in}{2.617043in}}%
\pgfpathlineto{\pgfqpoint{5.661976in}{2.699770in}}%
\pgfpathlineto{\pgfqpoint{5.465929in}{2.699770in}}%
\pgfpathlineto{\pgfqpoint{5.465929in}{2.617043in}}%
\pgfpathclose%
\pgfusepath{fill}%
\end{pgfscope}%
\begin{pgfscope}%
\pgfpathrectangle{\pgfqpoint{3.603478in}{2.617043in}}{\pgfqpoint{2.156522in}{1.606957in}}%
\pgfusepath{clip}%
\pgfsetrectcap%
\pgfsetroundjoin%
\pgfsetlinewidth{0.803000pt}%
\definecolor{currentstroke}{rgb}{0.690196,0.690196,0.690196}%
\pgfsetstrokecolor{currentstroke}%
\pgfsetdash{}{0pt}%
\pgfpathmoveto{\pgfqpoint{3.701502in}{2.617043in}}%
\pgfpathlineto{\pgfqpoint{3.701502in}{4.224000in}}%
\pgfusepath{stroke}%
\end{pgfscope}%
\begin{pgfscope}%
\pgfsetbuttcap%
\pgfsetroundjoin%
\definecolor{currentfill}{rgb}{0.000000,0.000000,0.000000}%
\pgfsetfillcolor{currentfill}%
\pgfsetlinewidth{0.803000pt}%
\definecolor{currentstroke}{rgb}{0.000000,0.000000,0.000000}%
\pgfsetstrokecolor{currentstroke}%
\pgfsetdash{}{0pt}%
\pgfsys@defobject{currentmarker}{\pgfqpoint{0.000000in}{-0.048611in}}{\pgfqpoint{0.000000in}{0.000000in}}{%
\pgfpathmoveto{\pgfqpoint{0.000000in}{0.000000in}}%
\pgfpathlineto{\pgfqpoint{0.000000in}{-0.048611in}}%
\pgfusepath{stroke,fill}%
}%
\begin{pgfscope}%
\pgfsys@transformshift{3.701502in}{2.617043in}%
\pgfsys@useobject{currentmarker}{}%
\end{pgfscope}%
\end{pgfscope}%
\begin{pgfscope}%
\definecolor{textcolor}{rgb}{0.000000,0.000000,0.000000}%
\pgfsetstrokecolor{textcolor}%
\pgfsetfillcolor{textcolor}%
\pgftext[x=3.701502in,y=2.519821in,,top]{\color{textcolor}\rmfamily\fontsize{10.000000}{12.000000}\selectfont \(\displaystyle {2}\)}%
\end{pgfscope}%
\begin{pgfscope}%
\pgfpathrectangle{\pgfqpoint{3.603478in}{2.617043in}}{\pgfqpoint{2.156522in}{1.606957in}}%
\pgfusepath{clip}%
\pgfsetrectcap%
\pgfsetroundjoin%
\pgfsetlinewidth{0.803000pt}%
\definecolor{currentstroke}{rgb}{0.690196,0.690196,0.690196}%
\pgfsetstrokecolor{currentstroke}%
\pgfsetdash{}{0pt}%
\pgfpathmoveto{\pgfqpoint{4.518366in}{2.617043in}}%
\pgfpathlineto{\pgfqpoint{4.518366in}{4.224000in}}%
\pgfusepath{stroke}%
\end{pgfscope}%
\begin{pgfscope}%
\pgfsetbuttcap%
\pgfsetroundjoin%
\definecolor{currentfill}{rgb}{0.000000,0.000000,0.000000}%
\pgfsetfillcolor{currentfill}%
\pgfsetlinewidth{0.803000pt}%
\definecolor{currentstroke}{rgb}{0.000000,0.000000,0.000000}%
\pgfsetstrokecolor{currentstroke}%
\pgfsetdash{}{0pt}%
\pgfsys@defobject{currentmarker}{\pgfqpoint{0.000000in}{-0.048611in}}{\pgfqpoint{0.000000in}{0.000000in}}{%
\pgfpathmoveto{\pgfqpoint{0.000000in}{0.000000in}}%
\pgfpathlineto{\pgfqpoint{0.000000in}{-0.048611in}}%
\pgfusepath{stroke,fill}%
}%
\begin{pgfscope}%
\pgfsys@transformshift{4.518366in}{2.617043in}%
\pgfsys@useobject{currentmarker}{}%
\end{pgfscope}%
\end{pgfscope}%
\begin{pgfscope}%
\definecolor{textcolor}{rgb}{0.000000,0.000000,0.000000}%
\pgfsetstrokecolor{textcolor}%
\pgfsetfillcolor{textcolor}%
\pgftext[x=4.518366in,y=2.519821in,,top]{\color{textcolor}\rmfamily\fontsize{10.000000}{12.000000}\selectfont \(\displaystyle {3}\)}%
\end{pgfscope}%
\begin{pgfscope}%
\pgfpathrectangle{\pgfqpoint{3.603478in}{2.617043in}}{\pgfqpoint{2.156522in}{1.606957in}}%
\pgfusepath{clip}%
\pgfsetrectcap%
\pgfsetroundjoin%
\pgfsetlinewidth{0.803000pt}%
\definecolor{currentstroke}{rgb}{0.690196,0.690196,0.690196}%
\pgfsetstrokecolor{currentstroke}%
\pgfsetdash{}{0pt}%
\pgfpathmoveto{\pgfqpoint{5.335231in}{2.617043in}}%
\pgfpathlineto{\pgfqpoint{5.335231in}{4.224000in}}%
\pgfusepath{stroke}%
\end{pgfscope}%
\begin{pgfscope}%
\pgfsetbuttcap%
\pgfsetroundjoin%
\definecolor{currentfill}{rgb}{0.000000,0.000000,0.000000}%
\pgfsetfillcolor{currentfill}%
\pgfsetlinewidth{0.803000pt}%
\definecolor{currentstroke}{rgb}{0.000000,0.000000,0.000000}%
\pgfsetstrokecolor{currentstroke}%
\pgfsetdash{}{0pt}%
\pgfsys@defobject{currentmarker}{\pgfqpoint{0.000000in}{-0.048611in}}{\pgfqpoint{0.000000in}{0.000000in}}{%
\pgfpathmoveto{\pgfqpoint{0.000000in}{0.000000in}}%
\pgfpathlineto{\pgfqpoint{0.000000in}{-0.048611in}}%
\pgfusepath{stroke,fill}%
}%
\begin{pgfscope}%
\pgfsys@transformshift{5.335231in}{2.617043in}%
\pgfsys@useobject{currentmarker}{}%
\end{pgfscope}%
\end{pgfscope}%
\begin{pgfscope}%
\definecolor{textcolor}{rgb}{0.000000,0.000000,0.000000}%
\pgfsetstrokecolor{textcolor}%
\pgfsetfillcolor{textcolor}%
\pgftext[x=5.335231in,y=2.519821in,,top]{\color{textcolor}\rmfamily\fontsize{10.000000}{12.000000}\selectfont \(\displaystyle {4}\)}%
\end{pgfscope}%
\begin{pgfscope}%
\pgfpathrectangle{\pgfqpoint{3.603478in}{2.617043in}}{\pgfqpoint{2.156522in}{1.606957in}}%
\pgfusepath{clip}%
\pgfsetrectcap%
\pgfsetroundjoin%
\pgfsetlinewidth{0.803000pt}%
\definecolor{currentstroke}{rgb}{0.690196,0.690196,0.690196}%
\pgfsetstrokecolor{currentstroke}%
\pgfsetdash{}{0pt}%
\pgfpathmoveto{\pgfqpoint{3.603478in}{2.617043in}}%
\pgfpathlineto{\pgfqpoint{5.760000in}{2.617043in}}%
\pgfusepath{stroke}%
\end{pgfscope}%
\begin{pgfscope}%
\pgfsetbuttcap%
\pgfsetroundjoin%
\definecolor{currentfill}{rgb}{0.000000,0.000000,0.000000}%
\pgfsetfillcolor{currentfill}%
\pgfsetlinewidth{0.803000pt}%
\definecolor{currentstroke}{rgb}{0.000000,0.000000,0.000000}%
\pgfsetstrokecolor{currentstroke}%
\pgfsetdash{}{0pt}%
\pgfsys@defobject{currentmarker}{\pgfqpoint{-0.048611in}{0.000000in}}{\pgfqpoint{-0.000000in}{0.000000in}}{%
\pgfpathmoveto{\pgfqpoint{-0.000000in}{0.000000in}}%
\pgfpathlineto{\pgfqpoint{-0.048611in}{0.000000in}}%
\pgfusepath{stroke,fill}%
}%
\begin{pgfscope}%
\pgfsys@transformshift{3.603478in}{2.617043in}%
\pgfsys@useobject{currentmarker}{}%
\end{pgfscope}%
\end{pgfscope}%
\begin{pgfscope}%
\definecolor{textcolor}{rgb}{0.000000,0.000000,0.000000}%
\pgfsetstrokecolor{textcolor}%
\pgfsetfillcolor{textcolor}%
\pgftext[x=3.436811in, y=2.568818in, left, base]{\color{textcolor}\rmfamily\fontsize{10.000000}{12.000000}\selectfont \(\displaystyle {0}\)}%
\end{pgfscope}%
\begin{pgfscope}%
\pgfpathrectangle{\pgfqpoint{3.603478in}{2.617043in}}{\pgfqpoint{2.156522in}{1.606957in}}%
\pgfusepath{clip}%
\pgfsetrectcap%
\pgfsetroundjoin%
\pgfsetlinewidth{0.803000pt}%
\definecolor{currentstroke}{rgb}{0.690196,0.690196,0.690196}%
\pgfsetstrokecolor{currentstroke}%
\pgfsetdash{}{0pt}%
\pgfpathmoveto{\pgfqpoint{3.603478in}{3.030675in}}%
\pgfpathlineto{\pgfqpoint{5.760000in}{3.030675in}}%
\pgfusepath{stroke}%
\end{pgfscope}%
\begin{pgfscope}%
\pgfsetbuttcap%
\pgfsetroundjoin%
\definecolor{currentfill}{rgb}{0.000000,0.000000,0.000000}%
\pgfsetfillcolor{currentfill}%
\pgfsetlinewidth{0.803000pt}%
\definecolor{currentstroke}{rgb}{0.000000,0.000000,0.000000}%
\pgfsetstrokecolor{currentstroke}%
\pgfsetdash{}{0pt}%
\pgfsys@defobject{currentmarker}{\pgfqpoint{-0.048611in}{0.000000in}}{\pgfqpoint{-0.000000in}{0.000000in}}{%
\pgfpathmoveto{\pgfqpoint{-0.000000in}{0.000000in}}%
\pgfpathlineto{\pgfqpoint{-0.048611in}{0.000000in}}%
\pgfusepath{stroke,fill}%
}%
\begin{pgfscope}%
\pgfsys@transformshift{3.603478in}{3.030675in}%
\pgfsys@useobject{currentmarker}{}%
\end{pgfscope}%
\end{pgfscope}%
\begin{pgfscope}%
\definecolor{textcolor}{rgb}{0.000000,0.000000,0.000000}%
\pgfsetstrokecolor{textcolor}%
\pgfsetfillcolor{textcolor}%
\pgftext[x=3.367367in, y=2.982449in, left, base]{\color{textcolor}\rmfamily\fontsize{10.000000}{12.000000}\selectfont \(\displaystyle {10}\)}%
\end{pgfscope}%
\begin{pgfscope}%
\pgfpathrectangle{\pgfqpoint{3.603478in}{2.617043in}}{\pgfqpoint{2.156522in}{1.606957in}}%
\pgfusepath{clip}%
\pgfsetrectcap%
\pgfsetroundjoin%
\pgfsetlinewidth{0.803000pt}%
\definecolor{currentstroke}{rgb}{0.690196,0.690196,0.690196}%
\pgfsetstrokecolor{currentstroke}%
\pgfsetdash{}{0pt}%
\pgfpathmoveto{\pgfqpoint{3.603478in}{3.444306in}}%
\pgfpathlineto{\pgfqpoint{5.760000in}{3.444306in}}%
\pgfusepath{stroke}%
\end{pgfscope}%
\begin{pgfscope}%
\pgfsetbuttcap%
\pgfsetroundjoin%
\definecolor{currentfill}{rgb}{0.000000,0.000000,0.000000}%
\pgfsetfillcolor{currentfill}%
\pgfsetlinewidth{0.803000pt}%
\definecolor{currentstroke}{rgb}{0.000000,0.000000,0.000000}%
\pgfsetstrokecolor{currentstroke}%
\pgfsetdash{}{0pt}%
\pgfsys@defobject{currentmarker}{\pgfqpoint{-0.048611in}{0.000000in}}{\pgfqpoint{-0.000000in}{0.000000in}}{%
\pgfpathmoveto{\pgfqpoint{-0.000000in}{0.000000in}}%
\pgfpathlineto{\pgfqpoint{-0.048611in}{0.000000in}}%
\pgfusepath{stroke,fill}%
}%
\begin{pgfscope}%
\pgfsys@transformshift{3.603478in}{3.444306in}%
\pgfsys@useobject{currentmarker}{}%
\end{pgfscope}%
\end{pgfscope}%
\begin{pgfscope}%
\definecolor{textcolor}{rgb}{0.000000,0.000000,0.000000}%
\pgfsetstrokecolor{textcolor}%
\pgfsetfillcolor{textcolor}%
\pgftext[x=3.367367in, y=3.396080in, left, base]{\color{textcolor}\rmfamily\fontsize{10.000000}{12.000000}\selectfont \(\displaystyle {20}\)}%
\end{pgfscope}%
\begin{pgfscope}%
\pgfpathrectangle{\pgfqpoint{3.603478in}{2.617043in}}{\pgfqpoint{2.156522in}{1.606957in}}%
\pgfusepath{clip}%
\pgfsetrectcap%
\pgfsetroundjoin%
\pgfsetlinewidth{0.803000pt}%
\definecolor{currentstroke}{rgb}{0.690196,0.690196,0.690196}%
\pgfsetstrokecolor{currentstroke}%
\pgfsetdash{}{0pt}%
\pgfpathmoveto{\pgfqpoint{3.603478in}{3.857937in}}%
\pgfpathlineto{\pgfqpoint{5.760000in}{3.857937in}}%
\pgfusepath{stroke}%
\end{pgfscope}%
\begin{pgfscope}%
\pgfsetbuttcap%
\pgfsetroundjoin%
\definecolor{currentfill}{rgb}{0.000000,0.000000,0.000000}%
\pgfsetfillcolor{currentfill}%
\pgfsetlinewidth{0.803000pt}%
\definecolor{currentstroke}{rgb}{0.000000,0.000000,0.000000}%
\pgfsetstrokecolor{currentstroke}%
\pgfsetdash{}{0pt}%
\pgfsys@defobject{currentmarker}{\pgfqpoint{-0.048611in}{0.000000in}}{\pgfqpoint{-0.000000in}{0.000000in}}{%
\pgfpathmoveto{\pgfqpoint{-0.000000in}{0.000000in}}%
\pgfpathlineto{\pgfqpoint{-0.048611in}{0.000000in}}%
\pgfusepath{stroke,fill}%
}%
\begin{pgfscope}%
\pgfsys@transformshift{3.603478in}{3.857937in}%
\pgfsys@useobject{currentmarker}{}%
\end{pgfscope}%
\end{pgfscope}%
\begin{pgfscope}%
\definecolor{textcolor}{rgb}{0.000000,0.000000,0.000000}%
\pgfsetstrokecolor{textcolor}%
\pgfsetfillcolor{textcolor}%
\pgftext[x=3.367367in, y=3.809711in, left, base]{\color{textcolor}\rmfamily\fontsize{10.000000}{12.000000}\selectfont \(\displaystyle {30}\)}%
\end{pgfscope}%
\begin{pgfscope}%
\pgfsetrectcap%
\pgfsetmiterjoin%
\pgfsetlinewidth{0.803000pt}%
\definecolor{currentstroke}{rgb}{0.000000,0.000000,0.000000}%
\pgfsetstrokecolor{currentstroke}%
\pgfsetdash{}{0pt}%
\pgfpathmoveto{\pgfqpoint{3.603478in}{2.617043in}}%
\pgfpathlineto{\pgfqpoint{3.603478in}{4.224000in}}%
\pgfusepath{stroke}%
\end{pgfscope}%
\begin{pgfscope}%
\pgfsetrectcap%
\pgfsetmiterjoin%
\pgfsetlinewidth{0.803000pt}%
\definecolor{currentstroke}{rgb}{0.000000,0.000000,0.000000}%
\pgfsetstrokecolor{currentstroke}%
\pgfsetdash{}{0pt}%
\pgfpathmoveto{\pgfqpoint{5.760000in}{2.617043in}}%
\pgfpathlineto{\pgfqpoint{5.760000in}{4.224000in}}%
\pgfusepath{stroke}%
\end{pgfscope}%
\begin{pgfscope}%
\pgfsetrectcap%
\pgfsetmiterjoin%
\pgfsetlinewidth{0.803000pt}%
\definecolor{currentstroke}{rgb}{0.000000,0.000000,0.000000}%
\pgfsetstrokecolor{currentstroke}%
\pgfsetdash{}{0pt}%
\pgfpathmoveto{\pgfqpoint{3.603478in}{2.617043in}}%
\pgfpathlineto{\pgfqpoint{5.760000in}{2.617043in}}%
\pgfusepath{stroke}%
\end{pgfscope}%
\begin{pgfscope}%
\pgfsetrectcap%
\pgfsetmiterjoin%
\pgfsetlinewidth{0.803000pt}%
\definecolor{currentstroke}{rgb}{0.000000,0.000000,0.000000}%
\pgfsetstrokecolor{currentstroke}%
\pgfsetdash{}{0pt}%
\pgfpathmoveto{\pgfqpoint{3.603478in}{4.224000in}}%
\pgfpathlineto{\pgfqpoint{5.760000in}{4.224000in}}%
\pgfusepath{stroke}%
\end{pgfscope}%
\begin{pgfscope}%
\definecolor{textcolor}{rgb}{0.000000,0.000000,0.000000}%
\pgfsetstrokecolor{textcolor}%
\pgfsetfillcolor{textcolor}%
\pgftext[x=4.681739in,y=4.307333in,,base]{\color{textcolor}\rmfamily\fontsize{12.000000}{14.400000}\selectfont sepal\_width}%
\end{pgfscope}%
\begin{pgfscope}%
\pgfsetbuttcap%
\pgfsetmiterjoin%
\definecolor{currentfill}{rgb}{1.000000,1.000000,1.000000}%
\pgfsetfillcolor{currentfill}%
\pgfsetlinewidth{0.000000pt}%
\definecolor{currentstroke}{rgb}{0.000000,0.000000,0.000000}%
\pgfsetstrokecolor{currentstroke}%
\pgfsetstrokeopacity{0.000000}%
\pgfsetdash{}{0pt}%
\pgfpathmoveto{\pgfqpoint{0.800000in}{0.528000in}}%
\pgfpathlineto{\pgfqpoint{2.956522in}{0.528000in}}%
\pgfpathlineto{\pgfqpoint{2.956522in}{2.134957in}}%
\pgfpathlineto{\pgfqpoint{0.800000in}{2.134957in}}%
\pgfpathlineto{\pgfqpoint{0.800000in}{0.528000in}}%
\pgfpathclose%
\pgfusepath{fill}%
\end{pgfscope}%
\begin{pgfscope}%
\pgfpathrectangle{\pgfqpoint{0.800000in}{0.528000in}}{\pgfqpoint{2.156522in}{1.606957in}}%
\pgfusepath{clip}%
\pgfsetbuttcap%
\pgfsetmiterjoin%
\definecolor{currentfill}{rgb}{0.121569,0.466667,0.705882}%
\pgfsetfillcolor{currentfill}%
\pgfsetlinewidth{0.000000pt}%
\definecolor{currentstroke}{rgb}{0.000000,0.000000,0.000000}%
\pgfsetstrokecolor{currentstroke}%
\pgfsetstrokeopacity{0.000000}%
\pgfsetdash{}{0pt}%
\pgfpathmoveto{\pgfqpoint{0.898024in}{0.528000in}}%
\pgfpathlineto{\pgfqpoint{1.094071in}{0.528000in}}%
\pgfpathlineto{\pgfqpoint{1.094071in}{2.058435in}}%
\pgfpathlineto{\pgfqpoint{0.898024in}{2.058435in}}%
\pgfpathlineto{\pgfqpoint{0.898024in}{0.528000in}}%
\pgfpathclose%
\pgfusepath{fill}%
\end{pgfscope}%
\begin{pgfscope}%
\pgfpathrectangle{\pgfqpoint{0.800000in}{0.528000in}}{\pgfqpoint{2.156522in}{1.606957in}}%
\pgfusepath{clip}%
\pgfsetbuttcap%
\pgfsetmiterjoin%
\definecolor{currentfill}{rgb}{0.121569,0.466667,0.705882}%
\pgfsetfillcolor{currentfill}%
\pgfsetlinewidth{0.000000pt}%
\definecolor{currentstroke}{rgb}{0.000000,0.000000,0.000000}%
\pgfsetstrokecolor{currentstroke}%
\pgfsetstrokeopacity{0.000000}%
\pgfsetdash{}{0pt}%
\pgfpathmoveto{\pgfqpoint{1.094071in}{0.528000in}}%
\pgfpathlineto{\pgfqpoint{1.290119in}{0.528000in}}%
\pgfpathlineto{\pgfqpoint{1.290119in}{1.065720in}}%
\pgfpathlineto{\pgfqpoint{1.094071in}{1.065720in}}%
\pgfpathlineto{\pgfqpoint{1.094071in}{0.528000in}}%
\pgfpathclose%
\pgfusepath{fill}%
\end{pgfscope}%
\begin{pgfscope}%
\pgfpathrectangle{\pgfqpoint{0.800000in}{0.528000in}}{\pgfqpoint{2.156522in}{1.606957in}}%
\pgfusepath{clip}%
\pgfsetbuttcap%
\pgfsetmiterjoin%
\definecolor{currentfill}{rgb}{0.121569,0.466667,0.705882}%
\pgfsetfillcolor{currentfill}%
\pgfsetlinewidth{0.000000pt}%
\definecolor{currentstroke}{rgb}{0.000000,0.000000,0.000000}%
\pgfsetstrokecolor{currentstroke}%
\pgfsetstrokeopacity{0.000000}%
\pgfsetdash{}{0pt}%
\pgfpathmoveto{\pgfqpoint{1.290119in}{0.528000in}}%
\pgfpathlineto{\pgfqpoint{1.486166in}{0.528000in}}%
\pgfpathlineto{\pgfqpoint{1.486166in}{0.528000in}}%
\pgfpathlineto{\pgfqpoint{1.290119in}{0.528000in}}%
\pgfpathlineto{\pgfqpoint{1.290119in}{0.528000in}}%
\pgfpathclose%
\pgfusepath{fill}%
\end{pgfscope}%
\begin{pgfscope}%
\pgfpathrectangle{\pgfqpoint{0.800000in}{0.528000in}}{\pgfqpoint{2.156522in}{1.606957in}}%
\pgfusepath{clip}%
\pgfsetbuttcap%
\pgfsetmiterjoin%
\definecolor{currentfill}{rgb}{0.121569,0.466667,0.705882}%
\pgfsetfillcolor{currentfill}%
\pgfsetlinewidth{0.000000pt}%
\definecolor{currentstroke}{rgb}{0.000000,0.000000,0.000000}%
\pgfsetstrokecolor{currentstroke}%
\pgfsetstrokeopacity{0.000000}%
\pgfsetdash{}{0pt}%
\pgfpathmoveto{\pgfqpoint{1.486166in}{0.528000in}}%
\pgfpathlineto{\pgfqpoint{1.682213in}{0.528000in}}%
\pgfpathlineto{\pgfqpoint{1.682213in}{0.652089in}}%
\pgfpathlineto{\pgfqpoint{1.486166in}{0.652089in}}%
\pgfpathlineto{\pgfqpoint{1.486166in}{0.528000in}}%
\pgfpathclose%
\pgfusepath{fill}%
\end{pgfscope}%
\begin{pgfscope}%
\pgfpathrectangle{\pgfqpoint{0.800000in}{0.528000in}}{\pgfqpoint{2.156522in}{1.606957in}}%
\pgfusepath{clip}%
\pgfsetbuttcap%
\pgfsetmiterjoin%
\definecolor{currentfill}{rgb}{0.121569,0.466667,0.705882}%
\pgfsetfillcolor{currentfill}%
\pgfsetlinewidth{0.000000pt}%
\definecolor{currentstroke}{rgb}{0.000000,0.000000,0.000000}%
\pgfsetstrokecolor{currentstroke}%
\pgfsetstrokeopacity{0.000000}%
\pgfsetdash{}{0pt}%
\pgfpathmoveto{\pgfqpoint{1.682213in}{0.528000in}}%
\pgfpathlineto{\pgfqpoint{1.878261in}{0.528000in}}%
\pgfpathlineto{\pgfqpoint{1.878261in}{0.858905in}}%
\pgfpathlineto{\pgfqpoint{1.682213in}{0.858905in}}%
\pgfpathlineto{\pgfqpoint{1.682213in}{0.528000in}}%
\pgfpathclose%
\pgfusepath{fill}%
\end{pgfscope}%
\begin{pgfscope}%
\pgfpathrectangle{\pgfqpoint{0.800000in}{0.528000in}}{\pgfqpoint{2.156522in}{1.606957in}}%
\pgfusepath{clip}%
\pgfsetbuttcap%
\pgfsetmiterjoin%
\definecolor{currentfill}{rgb}{0.121569,0.466667,0.705882}%
\pgfsetfillcolor{currentfill}%
\pgfsetlinewidth{0.000000pt}%
\definecolor{currentstroke}{rgb}{0.000000,0.000000,0.000000}%
\pgfsetstrokecolor{currentstroke}%
\pgfsetstrokeopacity{0.000000}%
\pgfsetdash{}{0pt}%
\pgfpathmoveto{\pgfqpoint{1.878261in}{0.528000in}}%
\pgfpathlineto{\pgfqpoint{2.074308in}{0.528000in}}%
\pgfpathlineto{\pgfqpoint{2.074308in}{1.603441in}}%
\pgfpathlineto{\pgfqpoint{1.878261in}{1.603441in}}%
\pgfpathlineto{\pgfqpoint{1.878261in}{0.528000in}}%
\pgfpathclose%
\pgfusepath{fill}%
\end{pgfscope}%
\begin{pgfscope}%
\pgfpathrectangle{\pgfqpoint{0.800000in}{0.528000in}}{\pgfqpoint{2.156522in}{1.606957in}}%
\pgfusepath{clip}%
\pgfsetbuttcap%
\pgfsetmiterjoin%
\definecolor{currentfill}{rgb}{0.121569,0.466667,0.705882}%
\pgfsetfillcolor{currentfill}%
\pgfsetlinewidth{0.000000pt}%
\definecolor{currentstroke}{rgb}{0.000000,0.000000,0.000000}%
\pgfsetstrokecolor{currentstroke}%
\pgfsetstrokeopacity{0.000000}%
\pgfsetdash{}{0pt}%
\pgfpathmoveto{\pgfqpoint{2.074308in}{0.528000in}}%
\pgfpathlineto{\pgfqpoint{2.270356in}{0.528000in}}%
\pgfpathlineto{\pgfqpoint{2.270356in}{1.727530in}}%
\pgfpathlineto{\pgfqpoint{2.074308in}{1.727530in}}%
\pgfpathlineto{\pgfqpoint{2.074308in}{0.528000in}}%
\pgfpathclose%
\pgfusepath{fill}%
\end{pgfscope}%
\begin{pgfscope}%
\pgfpathrectangle{\pgfqpoint{0.800000in}{0.528000in}}{\pgfqpoint{2.156522in}{1.606957in}}%
\pgfusepath{clip}%
\pgfsetbuttcap%
\pgfsetmiterjoin%
\definecolor{currentfill}{rgb}{0.121569,0.466667,0.705882}%
\pgfsetfillcolor{currentfill}%
\pgfsetlinewidth{0.000000pt}%
\definecolor{currentstroke}{rgb}{0.000000,0.000000,0.000000}%
\pgfsetstrokecolor{currentstroke}%
\pgfsetstrokeopacity{0.000000}%
\pgfsetdash{}{0pt}%
\pgfpathmoveto{\pgfqpoint{2.270356in}{0.528000in}}%
\pgfpathlineto{\pgfqpoint{2.466403in}{0.528000in}}%
\pgfpathlineto{\pgfqpoint{2.466403in}{1.272536in}}%
\pgfpathlineto{\pgfqpoint{2.270356in}{1.272536in}}%
\pgfpathlineto{\pgfqpoint{2.270356in}{0.528000in}}%
\pgfpathclose%
\pgfusepath{fill}%
\end{pgfscope}%
\begin{pgfscope}%
\pgfpathrectangle{\pgfqpoint{0.800000in}{0.528000in}}{\pgfqpoint{2.156522in}{1.606957in}}%
\pgfusepath{clip}%
\pgfsetbuttcap%
\pgfsetmiterjoin%
\definecolor{currentfill}{rgb}{0.121569,0.466667,0.705882}%
\pgfsetfillcolor{currentfill}%
\pgfsetlinewidth{0.000000pt}%
\definecolor{currentstroke}{rgb}{0.000000,0.000000,0.000000}%
\pgfsetstrokecolor{currentstroke}%
\pgfsetstrokeopacity{0.000000}%
\pgfsetdash{}{0pt}%
\pgfpathmoveto{\pgfqpoint{2.466403in}{0.528000in}}%
\pgfpathlineto{\pgfqpoint{2.662451in}{0.528000in}}%
\pgfpathlineto{\pgfqpoint{2.662451in}{0.982994in}}%
\pgfpathlineto{\pgfqpoint{2.466403in}{0.982994in}}%
\pgfpathlineto{\pgfqpoint{2.466403in}{0.528000in}}%
\pgfpathclose%
\pgfusepath{fill}%
\end{pgfscope}%
\begin{pgfscope}%
\pgfpathrectangle{\pgfqpoint{0.800000in}{0.528000in}}{\pgfqpoint{2.156522in}{1.606957in}}%
\pgfusepath{clip}%
\pgfsetbuttcap%
\pgfsetmiterjoin%
\definecolor{currentfill}{rgb}{0.121569,0.466667,0.705882}%
\pgfsetfillcolor{currentfill}%
\pgfsetlinewidth{0.000000pt}%
\definecolor{currentstroke}{rgb}{0.000000,0.000000,0.000000}%
\pgfsetstrokecolor{currentstroke}%
\pgfsetstrokeopacity{0.000000}%
\pgfsetdash{}{0pt}%
\pgfpathmoveto{\pgfqpoint{2.662451in}{0.528000in}}%
\pgfpathlineto{\pgfqpoint{2.858498in}{0.528000in}}%
\pgfpathlineto{\pgfqpoint{2.858498in}{0.734816in}}%
\pgfpathlineto{\pgfqpoint{2.662451in}{0.734816in}}%
\pgfpathlineto{\pgfqpoint{2.662451in}{0.528000in}}%
\pgfpathclose%
\pgfusepath{fill}%
\end{pgfscope}%
\begin{pgfscope}%
\pgfpathrectangle{\pgfqpoint{0.800000in}{0.528000in}}{\pgfqpoint{2.156522in}{1.606957in}}%
\pgfusepath{clip}%
\pgfsetrectcap%
\pgfsetroundjoin%
\pgfsetlinewidth{0.803000pt}%
\definecolor{currentstroke}{rgb}{0.690196,0.690196,0.690196}%
\pgfsetstrokecolor{currentstroke}%
\pgfsetdash{}{0pt}%
\pgfpathmoveto{\pgfqpoint{1.230307in}{0.528000in}}%
\pgfpathlineto{\pgfqpoint{1.230307in}{2.134957in}}%
\pgfusepath{stroke}%
\end{pgfscope}%
\begin{pgfscope}%
\pgfsetbuttcap%
\pgfsetroundjoin%
\definecolor{currentfill}{rgb}{0.000000,0.000000,0.000000}%
\pgfsetfillcolor{currentfill}%
\pgfsetlinewidth{0.803000pt}%
\definecolor{currentstroke}{rgb}{0.000000,0.000000,0.000000}%
\pgfsetstrokecolor{currentstroke}%
\pgfsetdash{}{0pt}%
\pgfsys@defobject{currentmarker}{\pgfqpoint{0.000000in}{-0.048611in}}{\pgfqpoint{0.000000in}{0.000000in}}{%
\pgfpathmoveto{\pgfqpoint{0.000000in}{0.000000in}}%
\pgfpathlineto{\pgfqpoint{0.000000in}{-0.048611in}}%
\pgfusepath{stroke,fill}%
}%
\begin{pgfscope}%
\pgfsys@transformshift{1.230307in}{0.528000in}%
\pgfsys@useobject{currentmarker}{}%
\end{pgfscope}%
\end{pgfscope}%
\begin{pgfscope}%
\definecolor{textcolor}{rgb}{0.000000,0.000000,0.000000}%
\pgfsetstrokecolor{textcolor}%
\pgfsetfillcolor{textcolor}%
\pgftext[x=1.230307in,y=0.430778in,,top]{\color{textcolor}\rmfamily\fontsize{10.000000}{12.000000}\selectfont \(\displaystyle {2}\)}%
\end{pgfscope}%
\begin{pgfscope}%
\pgfpathrectangle{\pgfqpoint{0.800000in}{0.528000in}}{\pgfqpoint{2.156522in}{1.606957in}}%
\pgfusepath{clip}%
\pgfsetrectcap%
\pgfsetroundjoin%
\pgfsetlinewidth{0.803000pt}%
\definecolor{currentstroke}{rgb}{0.690196,0.690196,0.690196}%
\pgfsetstrokecolor{currentstroke}%
\pgfsetdash{}{0pt}%
\pgfpathmoveto{\pgfqpoint{1.894875in}{0.528000in}}%
\pgfpathlineto{\pgfqpoint{1.894875in}{2.134957in}}%
\pgfusepath{stroke}%
\end{pgfscope}%
\begin{pgfscope}%
\pgfsetbuttcap%
\pgfsetroundjoin%
\definecolor{currentfill}{rgb}{0.000000,0.000000,0.000000}%
\pgfsetfillcolor{currentfill}%
\pgfsetlinewidth{0.803000pt}%
\definecolor{currentstroke}{rgb}{0.000000,0.000000,0.000000}%
\pgfsetstrokecolor{currentstroke}%
\pgfsetdash{}{0pt}%
\pgfsys@defobject{currentmarker}{\pgfqpoint{0.000000in}{-0.048611in}}{\pgfqpoint{0.000000in}{0.000000in}}{%
\pgfpathmoveto{\pgfqpoint{0.000000in}{0.000000in}}%
\pgfpathlineto{\pgfqpoint{0.000000in}{-0.048611in}}%
\pgfusepath{stroke,fill}%
}%
\begin{pgfscope}%
\pgfsys@transformshift{1.894875in}{0.528000in}%
\pgfsys@useobject{currentmarker}{}%
\end{pgfscope}%
\end{pgfscope}%
\begin{pgfscope}%
\definecolor{textcolor}{rgb}{0.000000,0.000000,0.000000}%
\pgfsetstrokecolor{textcolor}%
\pgfsetfillcolor{textcolor}%
\pgftext[x=1.894875in,y=0.430778in,,top]{\color{textcolor}\rmfamily\fontsize{10.000000}{12.000000}\selectfont \(\displaystyle {4}\)}%
\end{pgfscope}%
\begin{pgfscope}%
\pgfpathrectangle{\pgfqpoint{0.800000in}{0.528000in}}{\pgfqpoint{2.156522in}{1.606957in}}%
\pgfusepath{clip}%
\pgfsetrectcap%
\pgfsetroundjoin%
\pgfsetlinewidth{0.803000pt}%
\definecolor{currentstroke}{rgb}{0.690196,0.690196,0.690196}%
\pgfsetstrokecolor{currentstroke}%
\pgfsetdash{}{0pt}%
\pgfpathmoveto{\pgfqpoint{2.559443in}{0.528000in}}%
\pgfpathlineto{\pgfqpoint{2.559443in}{2.134957in}}%
\pgfusepath{stroke}%
\end{pgfscope}%
\begin{pgfscope}%
\pgfsetbuttcap%
\pgfsetroundjoin%
\definecolor{currentfill}{rgb}{0.000000,0.000000,0.000000}%
\pgfsetfillcolor{currentfill}%
\pgfsetlinewidth{0.803000pt}%
\definecolor{currentstroke}{rgb}{0.000000,0.000000,0.000000}%
\pgfsetstrokecolor{currentstroke}%
\pgfsetdash{}{0pt}%
\pgfsys@defobject{currentmarker}{\pgfqpoint{0.000000in}{-0.048611in}}{\pgfqpoint{0.000000in}{0.000000in}}{%
\pgfpathmoveto{\pgfqpoint{0.000000in}{0.000000in}}%
\pgfpathlineto{\pgfqpoint{0.000000in}{-0.048611in}}%
\pgfusepath{stroke,fill}%
}%
\begin{pgfscope}%
\pgfsys@transformshift{2.559443in}{0.528000in}%
\pgfsys@useobject{currentmarker}{}%
\end{pgfscope}%
\end{pgfscope}%
\begin{pgfscope}%
\definecolor{textcolor}{rgb}{0.000000,0.000000,0.000000}%
\pgfsetstrokecolor{textcolor}%
\pgfsetfillcolor{textcolor}%
\pgftext[x=2.559443in,y=0.430778in,,top]{\color{textcolor}\rmfamily\fontsize{10.000000}{12.000000}\selectfont \(\displaystyle {6}\)}%
\end{pgfscope}%
\begin{pgfscope}%
\pgfpathrectangle{\pgfqpoint{0.800000in}{0.528000in}}{\pgfqpoint{2.156522in}{1.606957in}}%
\pgfusepath{clip}%
\pgfsetrectcap%
\pgfsetroundjoin%
\pgfsetlinewidth{0.803000pt}%
\definecolor{currentstroke}{rgb}{0.690196,0.690196,0.690196}%
\pgfsetstrokecolor{currentstroke}%
\pgfsetdash{}{0pt}%
\pgfpathmoveto{\pgfqpoint{0.800000in}{0.528000in}}%
\pgfpathlineto{\pgfqpoint{2.956522in}{0.528000in}}%
\pgfusepath{stroke}%
\end{pgfscope}%
\begin{pgfscope}%
\pgfsetbuttcap%
\pgfsetroundjoin%
\definecolor{currentfill}{rgb}{0.000000,0.000000,0.000000}%
\pgfsetfillcolor{currentfill}%
\pgfsetlinewidth{0.803000pt}%
\definecolor{currentstroke}{rgb}{0.000000,0.000000,0.000000}%
\pgfsetstrokecolor{currentstroke}%
\pgfsetdash{}{0pt}%
\pgfsys@defobject{currentmarker}{\pgfqpoint{-0.048611in}{0.000000in}}{\pgfqpoint{-0.000000in}{0.000000in}}{%
\pgfpathmoveto{\pgfqpoint{-0.000000in}{0.000000in}}%
\pgfpathlineto{\pgfqpoint{-0.048611in}{0.000000in}}%
\pgfusepath{stroke,fill}%
}%
\begin{pgfscope}%
\pgfsys@transformshift{0.800000in}{0.528000in}%
\pgfsys@useobject{currentmarker}{}%
\end{pgfscope}%
\end{pgfscope}%
\begin{pgfscope}%
\definecolor{textcolor}{rgb}{0.000000,0.000000,0.000000}%
\pgfsetstrokecolor{textcolor}%
\pgfsetfillcolor{textcolor}%
\pgftext[x=0.633333in, y=0.479775in, left, base]{\color{textcolor}\rmfamily\fontsize{10.000000}{12.000000}\selectfont \(\displaystyle {0}\)}%
\end{pgfscope}%
\begin{pgfscope}%
\pgfpathrectangle{\pgfqpoint{0.800000in}{0.528000in}}{\pgfqpoint{2.156522in}{1.606957in}}%
\pgfusepath{clip}%
\pgfsetrectcap%
\pgfsetroundjoin%
\pgfsetlinewidth{0.803000pt}%
\definecolor{currentstroke}{rgb}{0.690196,0.690196,0.690196}%
\pgfsetstrokecolor{currentstroke}%
\pgfsetdash{}{0pt}%
\pgfpathmoveto{\pgfqpoint{0.800000in}{0.941631in}}%
\pgfpathlineto{\pgfqpoint{2.956522in}{0.941631in}}%
\pgfusepath{stroke}%
\end{pgfscope}%
\begin{pgfscope}%
\pgfsetbuttcap%
\pgfsetroundjoin%
\definecolor{currentfill}{rgb}{0.000000,0.000000,0.000000}%
\pgfsetfillcolor{currentfill}%
\pgfsetlinewidth{0.803000pt}%
\definecolor{currentstroke}{rgb}{0.000000,0.000000,0.000000}%
\pgfsetstrokecolor{currentstroke}%
\pgfsetdash{}{0pt}%
\pgfsys@defobject{currentmarker}{\pgfqpoint{-0.048611in}{0.000000in}}{\pgfqpoint{-0.000000in}{0.000000in}}{%
\pgfpathmoveto{\pgfqpoint{-0.000000in}{0.000000in}}%
\pgfpathlineto{\pgfqpoint{-0.048611in}{0.000000in}}%
\pgfusepath{stroke,fill}%
}%
\begin{pgfscope}%
\pgfsys@transformshift{0.800000in}{0.941631in}%
\pgfsys@useobject{currentmarker}{}%
\end{pgfscope}%
\end{pgfscope}%
\begin{pgfscope}%
\definecolor{textcolor}{rgb}{0.000000,0.000000,0.000000}%
\pgfsetstrokecolor{textcolor}%
\pgfsetfillcolor{textcolor}%
\pgftext[x=0.563888in, y=0.893406in, left, base]{\color{textcolor}\rmfamily\fontsize{10.000000}{12.000000}\selectfont \(\displaystyle {10}\)}%
\end{pgfscope}%
\begin{pgfscope}%
\pgfpathrectangle{\pgfqpoint{0.800000in}{0.528000in}}{\pgfqpoint{2.156522in}{1.606957in}}%
\pgfusepath{clip}%
\pgfsetrectcap%
\pgfsetroundjoin%
\pgfsetlinewidth{0.803000pt}%
\definecolor{currentstroke}{rgb}{0.690196,0.690196,0.690196}%
\pgfsetstrokecolor{currentstroke}%
\pgfsetdash{}{0pt}%
\pgfpathmoveto{\pgfqpoint{0.800000in}{1.355262in}}%
\pgfpathlineto{\pgfqpoint{2.956522in}{1.355262in}}%
\pgfusepath{stroke}%
\end{pgfscope}%
\begin{pgfscope}%
\pgfsetbuttcap%
\pgfsetroundjoin%
\definecolor{currentfill}{rgb}{0.000000,0.000000,0.000000}%
\pgfsetfillcolor{currentfill}%
\pgfsetlinewidth{0.803000pt}%
\definecolor{currentstroke}{rgb}{0.000000,0.000000,0.000000}%
\pgfsetstrokecolor{currentstroke}%
\pgfsetdash{}{0pt}%
\pgfsys@defobject{currentmarker}{\pgfqpoint{-0.048611in}{0.000000in}}{\pgfqpoint{-0.000000in}{0.000000in}}{%
\pgfpathmoveto{\pgfqpoint{-0.000000in}{0.000000in}}%
\pgfpathlineto{\pgfqpoint{-0.048611in}{0.000000in}}%
\pgfusepath{stroke,fill}%
}%
\begin{pgfscope}%
\pgfsys@transformshift{0.800000in}{1.355262in}%
\pgfsys@useobject{currentmarker}{}%
\end{pgfscope}%
\end{pgfscope}%
\begin{pgfscope}%
\definecolor{textcolor}{rgb}{0.000000,0.000000,0.000000}%
\pgfsetstrokecolor{textcolor}%
\pgfsetfillcolor{textcolor}%
\pgftext[x=0.563888in, y=1.307037in, left, base]{\color{textcolor}\rmfamily\fontsize{10.000000}{12.000000}\selectfont \(\displaystyle {20}\)}%
\end{pgfscope}%
\begin{pgfscope}%
\pgfpathrectangle{\pgfqpoint{0.800000in}{0.528000in}}{\pgfqpoint{2.156522in}{1.606957in}}%
\pgfusepath{clip}%
\pgfsetrectcap%
\pgfsetroundjoin%
\pgfsetlinewidth{0.803000pt}%
\definecolor{currentstroke}{rgb}{0.690196,0.690196,0.690196}%
\pgfsetstrokecolor{currentstroke}%
\pgfsetdash{}{0pt}%
\pgfpathmoveto{\pgfqpoint{0.800000in}{1.768893in}}%
\pgfpathlineto{\pgfqpoint{2.956522in}{1.768893in}}%
\pgfusepath{stroke}%
\end{pgfscope}%
\begin{pgfscope}%
\pgfsetbuttcap%
\pgfsetroundjoin%
\definecolor{currentfill}{rgb}{0.000000,0.000000,0.000000}%
\pgfsetfillcolor{currentfill}%
\pgfsetlinewidth{0.803000pt}%
\definecolor{currentstroke}{rgb}{0.000000,0.000000,0.000000}%
\pgfsetstrokecolor{currentstroke}%
\pgfsetdash{}{0pt}%
\pgfsys@defobject{currentmarker}{\pgfqpoint{-0.048611in}{0.000000in}}{\pgfqpoint{-0.000000in}{0.000000in}}{%
\pgfpathmoveto{\pgfqpoint{-0.000000in}{0.000000in}}%
\pgfpathlineto{\pgfqpoint{-0.048611in}{0.000000in}}%
\pgfusepath{stroke,fill}%
}%
\begin{pgfscope}%
\pgfsys@transformshift{0.800000in}{1.768893in}%
\pgfsys@useobject{currentmarker}{}%
\end{pgfscope}%
\end{pgfscope}%
\begin{pgfscope}%
\definecolor{textcolor}{rgb}{0.000000,0.000000,0.000000}%
\pgfsetstrokecolor{textcolor}%
\pgfsetfillcolor{textcolor}%
\pgftext[x=0.563888in, y=1.720668in, left, base]{\color{textcolor}\rmfamily\fontsize{10.000000}{12.000000}\selectfont \(\displaystyle {30}\)}%
\end{pgfscope}%
\begin{pgfscope}%
\pgfsetrectcap%
\pgfsetmiterjoin%
\pgfsetlinewidth{0.803000pt}%
\definecolor{currentstroke}{rgb}{0.000000,0.000000,0.000000}%
\pgfsetstrokecolor{currentstroke}%
\pgfsetdash{}{0pt}%
\pgfpathmoveto{\pgfqpoint{0.800000in}{0.528000in}}%
\pgfpathlineto{\pgfqpoint{0.800000in}{2.134957in}}%
\pgfusepath{stroke}%
\end{pgfscope}%
\begin{pgfscope}%
\pgfsetrectcap%
\pgfsetmiterjoin%
\pgfsetlinewidth{0.803000pt}%
\definecolor{currentstroke}{rgb}{0.000000,0.000000,0.000000}%
\pgfsetstrokecolor{currentstroke}%
\pgfsetdash{}{0pt}%
\pgfpathmoveto{\pgfqpoint{2.956522in}{0.528000in}}%
\pgfpathlineto{\pgfqpoint{2.956522in}{2.134957in}}%
\pgfusepath{stroke}%
\end{pgfscope}%
\begin{pgfscope}%
\pgfsetrectcap%
\pgfsetmiterjoin%
\pgfsetlinewidth{0.803000pt}%
\definecolor{currentstroke}{rgb}{0.000000,0.000000,0.000000}%
\pgfsetstrokecolor{currentstroke}%
\pgfsetdash{}{0pt}%
\pgfpathmoveto{\pgfqpoint{0.800000in}{0.528000in}}%
\pgfpathlineto{\pgfqpoint{2.956522in}{0.528000in}}%
\pgfusepath{stroke}%
\end{pgfscope}%
\begin{pgfscope}%
\pgfsetrectcap%
\pgfsetmiterjoin%
\pgfsetlinewidth{0.803000pt}%
\definecolor{currentstroke}{rgb}{0.000000,0.000000,0.000000}%
\pgfsetstrokecolor{currentstroke}%
\pgfsetdash{}{0pt}%
\pgfpathmoveto{\pgfqpoint{0.800000in}{2.134957in}}%
\pgfpathlineto{\pgfqpoint{2.956522in}{2.134957in}}%
\pgfusepath{stroke}%
\end{pgfscope}%
\begin{pgfscope}%
\definecolor{textcolor}{rgb}{0.000000,0.000000,0.000000}%
\pgfsetstrokecolor{textcolor}%
\pgfsetfillcolor{textcolor}%
\pgftext[x=1.878261in,y=2.218290in,,base]{\color{textcolor}\rmfamily\fontsize{12.000000}{14.400000}\selectfont petal\_length}%
\end{pgfscope}%
\begin{pgfscope}%
\pgfsetbuttcap%
\pgfsetmiterjoin%
\definecolor{currentfill}{rgb}{1.000000,1.000000,1.000000}%
\pgfsetfillcolor{currentfill}%
\pgfsetlinewidth{0.000000pt}%
\definecolor{currentstroke}{rgb}{0.000000,0.000000,0.000000}%
\pgfsetstrokecolor{currentstroke}%
\pgfsetstrokeopacity{0.000000}%
\pgfsetdash{}{0pt}%
\pgfpathmoveto{\pgfqpoint{3.603478in}{0.528000in}}%
\pgfpathlineto{\pgfqpoint{5.760000in}{0.528000in}}%
\pgfpathlineto{\pgfqpoint{5.760000in}{2.134957in}}%
\pgfpathlineto{\pgfqpoint{3.603478in}{2.134957in}}%
\pgfpathlineto{\pgfqpoint{3.603478in}{0.528000in}}%
\pgfpathclose%
\pgfusepath{fill}%
\end{pgfscope}%
\begin{pgfscope}%
\pgfpathrectangle{\pgfqpoint{3.603478in}{0.528000in}}{\pgfqpoint{2.156522in}{1.606957in}}%
\pgfusepath{clip}%
\pgfsetbuttcap%
\pgfsetmiterjoin%
\definecolor{currentfill}{rgb}{0.121569,0.466667,0.705882}%
\pgfsetfillcolor{currentfill}%
\pgfsetlinewidth{0.000000pt}%
\definecolor{currentstroke}{rgb}{0.000000,0.000000,0.000000}%
\pgfsetstrokecolor{currentstroke}%
\pgfsetstrokeopacity{0.000000}%
\pgfsetdash{}{0pt}%
\pgfpathmoveto{\pgfqpoint{3.701502in}{0.528000in}}%
\pgfpathlineto{\pgfqpoint{3.897549in}{0.528000in}}%
\pgfpathlineto{\pgfqpoint{3.897549in}{2.058435in}}%
\pgfpathlineto{\pgfqpoint{3.701502in}{2.058435in}}%
\pgfpathlineto{\pgfqpoint{3.701502in}{0.528000in}}%
\pgfpathclose%
\pgfusepath{fill}%
\end{pgfscope}%
\begin{pgfscope}%
\pgfpathrectangle{\pgfqpoint{3.603478in}{0.528000in}}{\pgfqpoint{2.156522in}{1.606957in}}%
\pgfusepath{clip}%
\pgfsetbuttcap%
\pgfsetmiterjoin%
\definecolor{currentfill}{rgb}{0.121569,0.466667,0.705882}%
\pgfsetfillcolor{currentfill}%
\pgfsetlinewidth{0.000000pt}%
\definecolor{currentstroke}{rgb}{0.000000,0.000000,0.000000}%
\pgfsetstrokecolor{currentstroke}%
\pgfsetstrokeopacity{0.000000}%
\pgfsetdash{}{0pt}%
\pgfpathmoveto{\pgfqpoint{3.897549in}{0.528000in}}%
\pgfpathlineto{\pgfqpoint{4.093597in}{0.528000in}}%
\pgfpathlineto{\pgfqpoint{4.093597in}{0.826621in}}%
\pgfpathlineto{\pgfqpoint{3.897549in}{0.826621in}}%
\pgfpathlineto{\pgfqpoint{3.897549in}{0.528000in}}%
\pgfpathclose%
\pgfusepath{fill}%
\end{pgfscope}%
\begin{pgfscope}%
\pgfpathrectangle{\pgfqpoint{3.603478in}{0.528000in}}{\pgfqpoint{2.156522in}{1.606957in}}%
\pgfusepath{clip}%
\pgfsetbuttcap%
\pgfsetmiterjoin%
\definecolor{currentfill}{rgb}{0.121569,0.466667,0.705882}%
\pgfsetfillcolor{currentfill}%
\pgfsetlinewidth{0.000000pt}%
\definecolor{currentstroke}{rgb}{0.000000,0.000000,0.000000}%
\pgfsetstrokecolor{currentstroke}%
\pgfsetstrokeopacity{0.000000}%
\pgfsetdash{}{0pt}%
\pgfpathmoveto{\pgfqpoint{4.093597in}{0.528000in}}%
\pgfpathlineto{\pgfqpoint{4.289644in}{0.528000in}}%
\pgfpathlineto{\pgfqpoint{4.289644in}{0.565328in}}%
\pgfpathlineto{\pgfqpoint{4.093597in}{0.565328in}}%
\pgfpathlineto{\pgfqpoint{4.093597in}{0.528000in}}%
\pgfpathclose%
\pgfusepath{fill}%
\end{pgfscope}%
\begin{pgfscope}%
\pgfpathrectangle{\pgfqpoint{3.603478in}{0.528000in}}{\pgfqpoint{2.156522in}{1.606957in}}%
\pgfusepath{clip}%
\pgfsetbuttcap%
\pgfsetmiterjoin%
\definecolor{currentfill}{rgb}{0.121569,0.466667,0.705882}%
\pgfsetfillcolor{currentfill}%
\pgfsetlinewidth{0.000000pt}%
\definecolor{currentstroke}{rgb}{0.000000,0.000000,0.000000}%
\pgfsetstrokecolor{currentstroke}%
\pgfsetstrokeopacity{0.000000}%
\pgfsetdash{}{0pt}%
\pgfpathmoveto{\pgfqpoint{4.289644in}{0.528000in}}%
\pgfpathlineto{\pgfqpoint{4.485692in}{0.528000in}}%
\pgfpathlineto{\pgfqpoint{4.485692in}{0.789294in}}%
\pgfpathlineto{\pgfqpoint{4.289644in}{0.789294in}}%
\pgfpathlineto{\pgfqpoint{4.289644in}{0.528000in}}%
\pgfpathclose%
\pgfusepath{fill}%
\end{pgfscope}%
\begin{pgfscope}%
\pgfpathrectangle{\pgfqpoint{3.603478in}{0.528000in}}{\pgfqpoint{2.156522in}{1.606957in}}%
\pgfusepath{clip}%
\pgfsetbuttcap%
\pgfsetmiterjoin%
\definecolor{currentfill}{rgb}{0.121569,0.466667,0.705882}%
\pgfsetfillcolor{currentfill}%
\pgfsetlinewidth{0.000000pt}%
\definecolor{currentstroke}{rgb}{0.000000,0.000000,0.000000}%
\pgfsetstrokecolor{currentstroke}%
\pgfsetstrokeopacity{0.000000}%
\pgfsetdash{}{0pt}%
\pgfpathmoveto{\pgfqpoint{4.485692in}{0.528000in}}%
\pgfpathlineto{\pgfqpoint{4.681739in}{0.528000in}}%
\pgfpathlineto{\pgfqpoint{4.681739in}{0.826621in}}%
\pgfpathlineto{\pgfqpoint{4.485692in}{0.826621in}}%
\pgfpathlineto{\pgfqpoint{4.485692in}{0.528000in}}%
\pgfpathclose%
\pgfusepath{fill}%
\end{pgfscope}%
\begin{pgfscope}%
\pgfpathrectangle{\pgfqpoint{3.603478in}{0.528000in}}{\pgfqpoint{2.156522in}{1.606957in}}%
\pgfusepath{clip}%
\pgfsetbuttcap%
\pgfsetmiterjoin%
\definecolor{currentfill}{rgb}{0.121569,0.466667,0.705882}%
\pgfsetfillcolor{currentfill}%
\pgfsetlinewidth{0.000000pt}%
\definecolor{currentstroke}{rgb}{0.000000,0.000000,0.000000}%
\pgfsetstrokecolor{currentstroke}%
\pgfsetstrokeopacity{0.000000}%
\pgfsetdash{}{0pt}%
\pgfpathmoveto{\pgfqpoint{4.681739in}{0.528000in}}%
\pgfpathlineto{\pgfqpoint{4.877787in}{0.528000in}}%
\pgfpathlineto{\pgfqpoint{4.877787in}{1.759813in}}%
\pgfpathlineto{\pgfqpoint{4.681739in}{1.759813in}}%
\pgfpathlineto{\pgfqpoint{4.681739in}{0.528000in}}%
\pgfpathclose%
\pgfusepath{fill}%
\end{pgfscope}%
\begin{pgfscope}%
\pgfpathrectangle{\pgfqpoint{3.603478in}{0.528000in}}{\pgfqpoint{2.156522in}{1.606957in}}%
\pgfusepath{clip}%
\pgfsetbuttcap%
\pgfsetmiterjoin%
\definecolor{currentfill}{rgb}{0.121569,0.466667,0.705882}%
\pgfsetfillcolor{currentfill}%
\pgfsetlinewidth{0.000000pt}%
\definecolor{currentstroke}{rgb}{0.000000,0.000000,0.000000}%
\pgfsetstrokecolor{currentstroke}%
\pgfsetstrokeopacity{0.000000}%
\pgfsetdash{}{0pt}%
\pgfpathmoveto{\pgfqpoint{4.877787in}{0.528000in}}%
\pgfpathlineto{\pgfqpoint{5.073834in}{0.528000in}}%
\pgfpathlineto{\pgfqpoint{5.073834in}{0.751966in}}%
\pgfpathlineto{\pgfqpoint{4.877787in}{0.751966in}}%
\pgfpathlineto{\pgfqpoint{4.877787in}{0.528000in}}%
\pgfpathclose%
\pgfusepath{fill}%
\end{pgfscope}%
\begin{pgfscope}%
\pgfpathrectangle{\pgfqpoint{3.603478in}{0.528000in}}{\pgfqpoint{2.156522in}{1.606957in}}%
\pgfusepath{clip}%
\pgfsetbuttcap%
\pgfsetmiterjoin%
\definecolor{currentfill}{rgb}{0.121569,0.466667,0.705882}%
\pgfsetfillcolor{currentfill}%
\pgfsetlinewidth{0.000000pt}%
\definecolor{currentstroke}{rgb}{0.000000,0.000000,0.000000}%
\pgfsetstrokecolor{currentstroke}%
\pgfsetstrokeopacity{0.000000}%
\pgfsetdash{}{0pt}%
\pgfpathmoveto{\pgfqpoint{5.073834in}{0.528000in}}%
\pgfpathlineto{\pgfqpoint{5.269881in}{0.528000in}}%
\pgfpathlineto{\pgfqpoint{5.269881in}{1.386537in}}%
\pgfpathlineto{\pgfqpoint{5.073834in}{1.386537in}}%
\pgfpathlineto{\pgfqpoint{5.073834in}{0.528000in}}%
\pgfpathclose%
\pgfusepath{fill}%
\end{pgfscope}%
\begin{pgfscope}%
\pgfpathrectangle{\pgfqpoint{3.603478in}{0.528000in}}{\pgfqpoint{2.156522in}{1.606957in}}%
\pgfusepath{clip}%
\pgfsetbuttcap%
\pgfsetmiterjoin%
\definecolor{currentfill}{rgb}{0.121569,0.466667,0.705882}%
\pgfsetfillcolor{currentfill}%
\pgfsetlinewidth{0.000000pt}%
\definecolor{currentstroke}{rgb}{0.000000,0.000000,0.000000}%
\pgfsetstrokecolor{currentstroke}%
\pgfsetstrokeopacity{0.000000}%
\pgfsetdash{}{0pt}%
\pgfpathmoveto{\pgfqpoint{5.269881in}{0.528000in}}%
\pgfpathlineto{\pgfqpoint{5.465929in}{0.528000in}}%
\pgfpathlineto{\pgfqpoint{5.465929in}{0.863949in}}%
\pgfpathlineto{\pgfqpoint{5.269881in}{0.863949in}}%
\pgfpathlineto{\pgfqpoint{5.269881in}{0.528000in}}%
\pgfpathclose%
\pgfusepath{fill}%
\end{pgfscope}%
\begin{pgfscope}%
\pgfpathrectangle{\pgfqpoint{3.603478in}{0.528000in}}{\pgfqpoint{2.156522in}{1.606957in}}%
\pgfusepath{clip}%
\pgfsetbuttcap%
\pgfsetmiterjoin%
\definecolor{currentfill}{rgb}{0.121569,0.466667,0.705882}%
\pgfsetfillcolor{currentfill}%
\pgfsetlinewidth{0.000000pt}%
\definecolor{currentstroke}{rgb}{0.000000,0.000000,0.000000}%
\pgfsetstrokecolor{currentstroke}%
\pgfsetstrokeopacity{0.000000}%
\pgfsetdash{}{0pt}%
\pgfpathmoveto{\pgfqpoint{5.465929in}{0.528000in}}%
\pgfpathlineto{\pgfqpoint{5.661976in}{0.528000in}}%
\pgfpathlineto{\pgfqpoint{5.661976in}{1.050587in}}%
\pgfpathlineto{\pgfqpoint{5.465929in}{1.050587in}}%
\pgfpathlineto{\pgfqpoint{5.465929in}{0.528000in}}%
\pgfpathclose%
\pgfusepath{fill}%
\end{pgfscope}%
\begin{pgfscope}%
\pgfpathrectangle{\pgfqpoint{3.603478in}{0.528000in}}{\pgfqpoint{2.156522in}{1.606957in}}%
\pgfusepath{clip}%
\pgfsetrectcap%
\pgfsetroundjoin%
\pgfsetlinewidth{0.803000pt}%
\definecolor{currentstroke}{rgb}{0.690196,0.690196,0.690196}%
\pgfsetstrokecolor{currentstroke}%
\pgfsetdash{}{0pt}%
\pgfpathmoveto{\pgfqpoint{3.619816in}{0.528000in}}%
\pgfpathlineto{\pgfqpoint{3.619816in}{2.134957in}}%
\pgfusepath{stroke}%
\end{pgfscope}%
\begin{pgfscope}%
\pgfsetbuttcap%
\pgfsetroundjoin%
\definecolor{currentfill}{rgb}{0.000000,0.000000,0.000000}%
\pgfsetfillcolor{currentfill}%
\pgfsetlinewidth{0.803000pt}%
\definecolor{currentstroke}{rgb}{0.000000,0.000000,0.000000}%
\pgfsetstrokecolor{currentstroke}%
\pgfsetdash{}{0pt}%
\pgfsys@defobject{currentmarker}{\pgfqpoint{0.000000in}{-0.048611in}}{\pgfqpoint{0.000000in}{0.000000in}}{%
\pgfpathmoveto{\pgfqpoint{0.000000in}{0.000000in}}%
\pgfpathlineto{\pgfqpoint{0.000000in}{-0.048611in}}%
\pgfusepath{stroke,fill}%
}%
\begin{pgfscope}%
\pgfsys@transformshift{3.619816in}{0.528000in}%
\pgfsys@useobject{currentmarker}{}%
\end{pgfscope}%
\end{pgfscope}%
\begin{pgfscope}%
\definecolor{textcolor}{rgb}{0.000000,0.000000,0.000000}%
\pgfsetstrokecolor{textcolor}%
\pgfsetfillcolor{textcolor}%
\pgftext[x=3.619816in,y=0.430778in,,top]{\color{textcolor}\rmfamily\fontsize{10.000000}{12.000000}\selectfont \(\displaystyle {0}\)}%
\end{pgfscope}%
\begin{pgfscope}%
\pgfpathrectangle{\pgfqpoint{3.603478in}{0.528000in}}{\pgfqpoint{2.156522in}{1.606957in}}%
\pgfusepath{clip}%
\pgfsetrectcap%
\pgfsetroundjoin%
\pgfsetlinewidth{0.803000pt}%
\definecolor{currentstroke}{rgb}{0.690196,0.690196,0.690196}%
\pgfsetstrokecolor{currentstroke}%
\pgfsetdash{}{0pt}%
\pgfpathmoveto{\pgfqpoint{4.436680in}{0.528000in}}%
\pgfpathlineto{\pgfqpoint{4.436680in}{2.134957in}}%
\pgfusepath{stroke}%
\end{pgfscope}%
\begin{pgfscope}%
\pgfsetbuttcap%
\pgfsetroundjoin%
\definecolor{currentfill}{rgb}{0.000000,0.000000,0.000000}%
\pgfsetfillcolor{currentfill}%
\pgfsetlinewidth{0.803000pt}%
\definecolor{currentstroke}{rgb}{0.000000,0.000000,0.000000}%
\pgfsetstrokecolor{currentstroke}%
\pgfsetdash{}{0pt}%
\pgfsys@defobject{currentmarker}{\pgfqpoint{0.000000in}{-0.048611in}}{\pgfqpoint{0.000000in}{0.000000in}}{%
\pgfpathmoveto{\pgfqpoint{0.000000in}{0.000000in}}%
\pgfpathlineto{\pgfqpoint{0.000000in}{-0.048611in}}%
\pgfusepath{stroke,fill}%
}%
\begin{pgfscope}%
\pgfsys@transformshift{4.436680in}{0.528000in}%
\pgfsys@useobject{currentmarker}{}%
\end{pgfscope}%
\end{pgfscope}%
\begin{pgfscope}%
\definecolor{textcolor}{rgb}{0.000000,0.000000,0.000000}%
\pgfsetstrokecolor{textcolor}%
\pgfsetfillcolor{textcolor}%
\pgftext[x=4.436680in,y=0.430778in,,top]{\color{textcolor}\rmfamily\fontsize{10.000000}{12.000000}\selectfont \(\displaystyle {1}\)}%
\end{pgfscope}%
\begin{pgfscope}%
\pgfpathrectangle{\pgfqpoint{3.603478in}{0.528000in}}{\pgfqpoint{2.156522in}{1.606957in}}%
\pgfusepath{clip}%
\pgfsetrectcap%
\pgfsetroundjoin%
\pgfsetlinewidth{0.803000pt}%
\definecolor{currentstroke}{rgb}{0.690196,0.690196,0.690196}%
\pgfsetstrokecolor{currentstroke}%
\pgfsetdash{}{0pt}%
\pgfpathmoveto{\pgfqpoint{5.253544in}{0.528000in}}%
\pgfpathlineto{\pgfqpoint{5.253544in}{2.134957in}}%
\pgfusepath{stroke}%
\end{pgfscope}%
\begin{pgfscope}%
\pgfsetbuttcap%
\pgfsetroundjoin%
\definecolor{currentfill}{rgb}{0.000000,0.000000,0.000000}%
\pgfsetfillcolor{currentfill}%
\pgfsetlinewidth{0.803000pt}%
\definecolor{currentstroke}{rgb}{0.000000,0.000000,0.000000}%
\pgfsetstrokecolor{currentstroke}%
\pgfsetdash{}{0pt}%
\pgfsys@defobject{currentmarker}{\pgfqpoint{0.000000in}{-0.048611in}}{\pgfqpoint{0.000000in}{0.000000in}}{%
\pgfpathmoveto{\pgfqpoint{0.000000in}{0.000000in}}%
\pgfpathlineto{\pgfqpoint{0.000000in}{-0.048611in}}%
\pgfusepath{stroke,fill}%
}%
\begin{pgfscope}%
\pgfsys@transformshift{5.253544in}{0.528000in}%
\pgfsys@useobject{currentmarker}{}%
\end{pgfscope}%
\end{pgfscope}%
\begin{pgfscope}%
\definecolor{textcolor}{rgb}{0.000000,0.000000,0.000000}%
\pgfsetstrokecolor{textcolor}%
\pgfsetfillcolor{textcolor}%
\pgftext[x=5.253544in,y=0.430778in,,top]{\color{textcolor}\rmfamily\fontsize{10.000000}{12.000000}\selectfont \(\displaystyle {2}\)}%
\end{pgfscope}%
\begin{pgfscope}%
\pgfpathrectangle{\pgfqpoint{3.603478in}{0.528000in}}{\pgfqpoint{2.156522in}{1.606957in}}%
\pgfusepath{clip}%
\pgfsetrectcap%
\pgfsetroundjoin%
\pgfsetlinewidth{0.803000pt}%
\definecolor{currentstroke}{rgb}{0.690196,0.690196,0.690196}%
\pgfsetstrokecolor{currentstroke}%
\pgfsetdash{}{0pt}%
\pgfpathmoveto{\pgfqpoint{3.603478in}{0.528000in}}%
\pgfpathlineto{\pgfqpoint{5.760000in}{0.528000in}}%
\pgfusepath{stroke}%
\end{pgfscope}%
\begin{pgfscope}%
\pgfsetbuttcap%
\pgfsetroundjoin%
\definecolor{currentfill}{rgb}{0.000000,0.000000,0.000000}%
\pgfsetfillcolor{currentfill}%
\pgfsetlinewidth{0.803000pt}%
\definecolor{currentstroke}{rgb}{0.000000,0.000000,0.000000}%
\pgfsetstrokecolor{currentstroke}%
\pgfsetdash{}{0pt}%
\pgfsys@defobject{currentmarker}{\pgfqpoint{-0.048611in}{0.000000in}}{\pgfqpoint{-0.000000in}{0.000000in}}{%
\pgfpathmoveto{\pgfqpoint{-0.000000in}{0.000000in}}%
\pgfpathlineto{\pgfqpoint{-0.048611in}{0.000000in}}%
\pgfusepath{stroke,fill}%
}%
\begin{pgfscope}%
\pgfsys@transformshift{3.603478in}{0.528000in}%
\pgfsys@useobject{currentmarker}{}%
\end{pgfscope}%
\end{pgfscope}%
\begin{pgfscope}%
\definecolor{textcolor}{rgb}{0.000000,0.000000,0.000000}%
\pgfsetstrokecolor{textcolor}%
\pgfsetfillcolor{textcolor}%
\pgftext[x=3.436811in, y=0.479775in, left, base]{\color{textcolor}\rmfamily\fontsize{10.000000}{12.000000}\selectfont \(\displaystyle {0}\)}%
\end{pgfscope}%
\begin{pgfscope}%
\pgfpathrectangle{\pgfqpoint{3.603478in}{0.528000in}}{\pgfqpoint{2.156522in}{1.606957in}}%
\pgfusepath{clip}%
\pgfsetrectcap%
\pgfsetroundjoin%
\pgfsetlinewidth{0.803000pt}%
\definecolor{currentstroke}{rgb}{0.690196,0.690196,0.690196}%
\pgfsetstrokecolor{currentstroke}%
\pgfsetdash{}{0pt}%
\pgfpathmoveto{\pgfqpoint{3.603478in}{0.901277in}}%
\pgfpathlineto{\pgfqpoint{5.760000in}{0.901277in}}%
\pgfusepath{stroke}%
\end{pgfscope}%
\begin{pgfscope}%
\pgfsetbuttcap%
\pgfsetroundjoin%
\definecolor{currentfill}{rgb}{0.000000,0.000000,0.000000}%
\pgfsetfillcolor{currentfill}%
\pgfsetlinewidth{0.803000pt}%
\definecolor{currentstroke}{rgb}{0.000000,0.000000,0.000000}%
\pgfsetstrokecolor{currentstroke}%
\pgfsetdash{}{0pt}%
\pgfsys@defobject{currentmarker}{\pgfqpoint{-0.048611in}{0.000000in}}{\pgfqpoint{-0.000000in}{0.000000in}}{%
\pgfpathmoveto{\pgfqpoint{-0.000000in}{0.000000in}}%
\pgfpathlineto{\pgfqpoint{-0.048611in}{0.000000in}}%
\pgfusepath{stroke,fill}%
}%
\begin{pgfscope}%
\pgfsys@transformshift{3.603478in}{0.901277in}%
\pgfsys@useobject{currentmarker}{}%
\end{pgfscope}%
\end{pgfscope}%
\begin{pgfscope}%
\definecolor{textcolor}{rgb}{0.000000,0.000000,0.000000}%
\pgfsetstrokecolor{textcolor}%
\pgfsetfillcolor{textcolor}%
\pgftext[x=3.367367in, y=0.853051in, left, base]{\color{textcolor}\rmfamily\fontsize{10.000000}{12.000000}\selectfont \(\displaystyle {10}\)}%
\end{pgfscope}%
\begin{pgfscope}%
\pgfpathrectangle{\pgfqpoint{3.603478in}{0.528000in}}{\pgfqpoint{2.156522in}{1.606957in}}%
\pgfusepath{clip}%
\pgfsetrectcap%
\pgfsetroundjoin%
\pgfsetlinewidth{0.803000pt}%
\definecolor{currentstroke}{rgb}{0.690196,0.690196,0.690196}%
\pgfsetstrokecolor{currentstroke}%
\pgfsetdash{}{0pt}%
\pgfpathmoveto{\pgfqpoint{3.603478in}{1.274554in}}%
\pgfpathlineto{\pgfqpoint{5.760000in}{1.274554in}}%
\pgfusepath{stroke}%
\end{pgfscope}%
\begin{pgfscope}%
\pgfsetbuttcap%
\pgfsetroundjoin%
\definecolor{currentfill}{rgb}{0.000000,0.000000,0.000000}%
\pgfsetfillcolor{currentfill}%
\pgfsetlinewidth{0.803000pt}%
\definecolor{currentstroke}{rgb}{0.000000,0.000000,0.000000}%
\pgfsetstrokecolor{currentstroke}%
\pgfsetdash{}{0pt}%
\pgfsys@defobject{currentmarker}{\pgfqpoint{-0.048611in}{0.000000in}}{\pgfqpoint{-0.000000in}{0.000000in}}{%
\pgfpathmoveto{\pgfqpoint{-0.000000in}{0.000000in}}%
\pgfpathlineto{\pgfqpoint{-0.048611in}{0.000000in}}%
\pgfusepath{stroke,fill}%
}%
\begin{pgfscope}%
\pgfsys@transformshift{3.603478in}{1.274554in}%
\pgfsys@useobject{currentmarker}{}%
\end{pgfscope}%
\end{pgfscope}%
\begin{pgfscope}%
\definecolor{textcolor}{rgb}{0.000000,0.000000,0.000000}%
\pgfsetstrokecolor{textcolor}%
\pgfsetfillcolor{textcolor}%
\pgftext[x=3.367367in, y=1.226328in, left, base]{\color{textcolor}\rmfamily\fontsize{10.000000}{12.000000}\selectfont \(\displaystyle {20}\)}%
\end{pgfscope}%
\begin{pgfscope}%
\pgfpathrectangle{\pgfqpoint{3.603478in}{0.528000in}}{\pgfqpoint{2.156522in}{1.606957in}}%
\pgfusepath{clip}%
\pgfsetrectcap%
\pgfsetroundjoin%
\pgfsetlinewidth{0.803000pt}%
\definecolor{currentstroke}{rgb}{0.690196,0.690196,0.690196}%
\pgfsetstrokecolor{currentstroke}%
\pgfsetdash{}{0pt}%
\pgfpathmoveto{\pgfqpoint{3.603478in}{1.647830in}}%
\pgfpathlineto{\pgfqpoint{5.760000in}{1.647830in}}%
\pgfusepath{stroke}%
\end{pgfscope}%
\begin{pgfscope}%
\pgfsetbuttcap%
\pgfsetroundjoin%
\definecolor{currentfill}{rgb}{0.000000,0.000000,0.000000}%
\pgfsetfillcolor{currentfill}%
\pgfsetlinewidth{0.803000pt}%
\definecolor{currentstroke}{rgb}{0.000000,0.000000,0.000000}%
\pgfsetstrokecolor{currentstroke}%
\pgfsetdash{}{0pt}%
\pgfsys@defobject{currentmarker}{\pgfqpoint{-0.048611in}{0.000000in}}{\pgfqpoint{-0.000000in}{0.000000in}}{%
\pgfpathmoveto{\pgfqpoint{-0.000000in}{0.000000in}}%
\pgfpathlineto{\pgfqpoint{-0.048611in}{0.000000in}}%
\pgfusepath{stroke,fill}%
}%
\begin{pgfscope}%
\pgfsys@transformshift{3.603478in}{1.647830in}%
\pgfsys@useobject{currentmarker}{}%
\end{pgfscope}%
\end{pgfscope}%
\begin{pgfscope}%
\definecolor{textcolor}{rgb}{0.000000,0.000000,0.000000}%
\pgfsetstrokecolor{textcolor}%
\pgfsetfillcolor{textcolor}%
\pgftext[x=3.367367in, y=1.599605in, left, base]{\color{textcolor}\rmfamily\fontsize{10.000000}{12.000000}\selectfont \(\displaystyle {30}\)}%
\end{pgfscope}%
\begin{pgfscope}%
\pgfpathrectangle{\pgfqpoint{3.603478in}{0.528000in}}{\pgfqpoint{2.156522in}{1.606957in}}%
\pgfusepath{clip}%
\pgfsetrectcap%
\pgfsetroundjoin%
\pgfsetlinewidth{0.803000pt}%
\definecolor{currentstroke}{rgb}{0.690196,0.690196,0.690196}%
\pgfsetstrokecolor{currentstroke}%
\pgfsetdash{}{0pt}%
\pgfpathmoveto{\pgfqpoint{3.603478in}{2.021107in}}%
\pgfpathlineto{\pgfqpoint{5.760000in}{2.021107in}}%
\pgfusepath{stroke}%
\end{pgfscope}%
\begin{pgfscope}%
\pgfsetbuttcap%
\pgfsetroundjoin%
\definecolor{currentfill}{rgb}{0.000000,0.000000,0.000000}%
\pgfsetfillcolor{currentfill}%
\pgfsetlinewidth{0.803000pt}%
\definecolor{currentstroke}{rgb}{0.000000,0.000000,0.000000}%
\pgfsetstrokecolor{currentstroke}%
\pgfsetdash{}{0pt}%
\pgfsys@defobject{currentmarker}{\pgfqpoint{-0.048611in}{0.000000in}}{\pgfqpoint{-0.000000in}{0.000000in}}{%
\pgfpathmoveto{\pgfqpoint{-0.000000in}{0.000000in}}%
\pgfpathlineto{\pgfqpoint{-0.048611in}{0.000000in}}%
\pgfusepath{stroke,fill}%
}%
\begin{pgfscope}%
\pgfsys@transformshift{3.603478in}{2.021107in}%
\pgfsys@useobject{currentmarker}{}%
\end{pgfscope}%
\end{pgfscope}%
\begin{pgfscope}%
\definecolor{textcolor}{rgb}{0.000000,0.000000,0.000000}%
\pgfsetstrokecolor{textcolor}%
\pgfsetfillcolor{textcolor}%
\pgftext[x=3.367367in, y=1.972882in, left, base]{\color{textcolor}\rmfamily\fontsize{10.000000}{12.000000}\selectfont \(\displaystyle {40}\)}%
\end{pgfscope}%
\begin{pgfscope}%
\pgfsetrectcap%
\pgfsetmiterjoin%
\pgfsetlinewidth{0.803000pt}%
\definecolor{currentstroke}{rgb}{0.000000,0.000000,0.000000}%
\pgfsetstrokecolor{currentstroke}%
\pgfsetdash{}{0pt}%
\pgfpathmoveto{\pgfqpoint{3.603478in}{0.528000in}}%
\pgfpathlineto{\pgfqpoint{3.603478in}{2.134957in}}%
\pgfusepath{stroke}%
\end{pgfscope}%
\begin{pgfscope}%
\pgfsetrectcap%
\pgfsetmiterjoin%
\pgfsetlinewidth{0.803000pt}%
\definecolor{currentstroke}{rgb}{0.000000,0.000000,0.000000}%
\pgfsetstrokecolor{currentstroke}%
\pgfsetdash{}{0pt}%
\pgfpathmoveto{\pgfqpoint{5.760000in}{0.528000in}}%
\pgfpathlineto{\pgfqpoint{5.760000in}{2.134957in}}%
\pgfusepath{stroke}%
\end{pgfscope}%
\begin{pgfscope}%
\pgfsetrectcap%
\pgfsetmiterjoin%
\pgfsetlinewidth{0.803000pt}%
\definecolor{currentstroke}{rgb}{0.000000,0.000000,0.000000}%
\pgfsetstrokecolor{currentstroke}%
\pgfsetdash{}{0pt}%
\pgfpathmoveto{\pgfqpoint{3.603478in}{0.528000in}}%
\pgfpathlineto{\pgfqpoint{5.760000in}{0.528000in}}%
\pgfusepath{stroke}%
\end{pgfscope}%
\begin{pgfscope}%
\pgfsetrectcap%
\pgfsetmiterjoin%
\pgfsetlinewidth{0.803000pt}%
\definecolor{currentstroke}{rgb}{0.000000,0.000000,0.000000}%
\pgfsetstrokecolor{currentstroke}%
\pgfsetdash{}{0pt}%
\pgfpathmoveto{\pgfqpoint{3.603478in}{2.134957in}}%
\pgfpathlineto{\pgfqpoint{5.760000in}{2.134957in}}%
\pgfusepath{stroke}%
\end{pgfscope}%
\begin{pgfscope}%
\definecolor{textcolor}{rgb}{0.000000,0.000000,0.000000}%
\pgfsetstrokecolor{textcolor}%
\pgfsetfillcolor{textcolor}%
\pgftext[x=4.681739in,y=2.218290in,,base]{\color{textcolor}\rmfamily\fontsize{12.000000}{14.400000}\selectfont petal\_width}%
\end{pgfscope}%
\end{pgfpicture}%
\makeatother%
\endgroup%
}
  \end{center}
  \caption{Histogram of iris data.}
  \label{fig:histogram}
\end{figure}

Boxplots are another common tool in \ac{eda}. They provide a summary of the distribution of the data, represented by a box that represents the middle 50\% of the data, a line that represents the median, and whiskers that extend from the box to the minimum and maximum values of the data. Boxplots are useful for identifying outliers and understanding the spread and skewness of the data, as shown in figure \ref{fig:boxplot}.

\begin{figure}
  \begin{center}
    \resizebox{0.8\textwidth}{!}{%% Creator: Matplotlib, PGF backend
%%
%% To include the figure in your LaTeX document, write
%%   \input{<filename>.pgf}
%%
%% Make sure the required packages are loaded in your preamble
%%   \usepackage{pgf}
%%
%% Also ensure that all the required font packages are loaded; for instance,
%% the lmodern package is sometimes necessary when using math font.
%%   \usepackage{lmodern}
%%
%% Figures using additional raster images can only be included by \input if
%% they are in the same directory as the main LaTeX file. For loading figures
%% from other directories you can use the `import` package
%%   \usepackage{import}
%%
%% and then include the figures with
%%   \import{<path to file>}{<filename>.pgf}
%%
%% Matplotlib used the following preamble
%%
\begingroup%
\makeatletter%
\begin{pgfpicture}%
\pgfpathrectangle{\pgfpointorigin}{\pgfqpoint{6.400000in}{4.800000in}}%
\pgfusepath{use as bounding box, clip}%
\begin{pgfscope}%
\pgfsetbuttcap%
\pgfsetmiterjoin%
\definecolor{currentfill}{rgb}{1.000000,1.000000,1.000000}%
\pgfsetfillcolor{currentfill}%
\pgfsetlinewidth{0.000000pt}%
\definecolor{currentstroke}{rgb}{1.000000,1.000000,1.000000}%
\pgfsetstrokecolor{currentstroke}%
\pgfsetdash{}{0pt}%
\pgfpathmoveto{\pgfqpoint{0.000000in}{0.000000in}}%
\pgfpathlineto{\pgfqpoint{6.400000in}{0.000000in}}%
\pgfpathlineto{\pgfqpoint{6.400000in}{4.800000in}}%
\pgfpathlineto{\pgfqpoint{0.000000in}{4.800000in}}%
\pgfpathlineto{\pgfqpoint{0.000000in}{0.000000in}}%
\pgfpathclose%
\pgfusepath{fill}%
\end{pgfscope}%
\begin{pgfscope}%
\pgfsetbuttcap%
\pgfsetmiterjoin%
\definecolor{currentfill}{rgb}{1.000000,1.000000,1.000000}%
\pgfsetfillcolor{currentfill}%
\pgfsetlinewidth{0.000000pt}%
\definecolor{currentstroke}{rgb}{0.000000,0.000000,0.000000}%
\pgfsetstrokecolor{currentstroke}%
\pgfsetstrokeopacity{0.000000}%
\pgfsetdash{}{0pt}%
\pgfpathmoveto{\pgfqpoint{0.800000in}{0.528000in}}%
\pgfpathlineto{\pgfqpoint{5.760000in}{0.528000in}}%
\pgfpathlineto{\pgfqpoint{5.760000in}{4.224000in}}%
\pgfpathlineto{\pgfqpoint{0.800000in}{4.224000in}}%
\pgfpathlineto{\pgfqpoint{0.800000in}{0.528000in}}%
\pgfpathclose%
\pgfusepath{fill}%
\end{pgfscope}%
\begin{pgfscope}%
\pgfsetbuttcap%
\pgfsetroundjoin%
\definecolor{currentfill}{rgb}{0.000000,0.000000,0.000000}%
\pgfsetfillcolor{currentfill}%
\pgfsetlinewidth{0.803000pt}%
\definecolor{currentstroke}{rgb}{0.000000,0.000000,0.000000}%
\pgfsetstrokecolor{currentstroke}%
\pgfsetdash{}{0pt}%
\pgfsys@defobject{currentmarker}{\pgfqpoint{0.000000in}{-0.048611in}}{\pgfqpoint{0.000000in}{0.000000in}}{%
\pgfpathmoveto{\pgfqpoint{0.000000in}{0.000000in}}%
\pgfpathlineto{\pgfqpoint{0.000000in}{-0.048611in}}%
\pgfusepath{stroke,fill}%
}%
\begin{pgfscope}%
\pgfsys@transformshift{1.420000in}{0.528000in}%
\pgfsys@useobject{currentmarker}{}%
\end{pgfscope}%
\end{pgfscope}%
\begin{pgfscope}%
\definecolor{textcolor}{rgb}{0.000000,0.000000,0.000000}%
\pgfsetstrokecolor{textcolor}%
\pgfsetfillcolor{textcolor}%
\pgftext[x=1.420000in,y=0.430778in,,top]{\color{textcolor}\rmfamily\fontsize{10.000000}{12.000000}\selectfont sepal\_length}%
\end{pgfscope}%
\begin{pgfscope}%
\pgfsetbuttcap%
\pgfsetroundjoin%
\definecolor{currentfill}{rgb}{0.000000,0.000000,0.000000}%
\pgfsetfillcolor{currentfill}%
\pgfsetlinewidth{0.803000pt}%
\definecolor{currentstroke}{rgb}{0.000000,0.000000,0.000000}%
\pgfsetstrokecolor{currentstroke}%
\pgfsetdash{}{0pt}%
\pgfsys@defobject{currentmarker}{\pgfqpoint{0.000000in}{-0.048611in}}{\pgfqpoint{0.000000in}{0.000000in}}{%
\pgfpathmoveto{\pgfqpoint{0.000000in}{0.000000in}}%
\pgfpathlineto{\pgfqpoint{0.000000in}{-0.048611in}}%
\pgfusepath{stroke,fill}%
}%
\begin{pgfscope}%
\pgfsys@transformshift{2.660000in}{0.528000in}%
\pgfsys@useobject{currentmarker}{}%
\end{pgfscope}%
\end{pgfscope}%
\begin{pgfscope}%
\definecolor{textcolor}{rgb}{0.000000,0.000000,0.000000}%
\pgfsetstrokecolor{textcolor}%
\pgfsetfillcolor{textcolor}%
\pgftext[x=2.660000in,y=0.430778in,,top]{\color{textcolor}\rmfamily\fontsize{10.000000}{12.000000}\selectfont sepal\_width}%
\end{pgfscope}%
\begin{pgfscope}%
\pgfsetbuttcap%
\pgfsetroundjoin%
\definecolor{currentfill}{rgb}{0.000000,0.000000,0.000000}%
\pgfsetfillcolor{currentfill}%
\pgfsetlinewidth{0.803000pt}%
\definecolor{currentstroke}{rgb}{0.000000,0.000000,0.000000}%
\pgfsetstrokecolor{currentstroke}%
\pgfsetdash{}{0pt}%
\pgfsys@defobject{currentmarker}{\pgfqpoint{0.000000in}{-0.048611in}}{\pgfqpoint{0.000000in}{0.000000in}}{%
\pgfpathmoveto{\pgfqpoint{0.000000in}{0.000000in}}%
\pgfpathlineto{\pgfqpoint{0.000000in}{-0.048611in}}%
\pgfusepath{stroke,fill}%
}%
\begin{pgfscope}%
\pgfsys@transformshift{3.900000in}{0.528000in}%
\pgfsys@useobject{currentmarker}{}%
\end{pgfscope}%
\end{pgfscope}%
\begin{pgfscope}%
\definecolor{textcolor}{rgb}{0.000000,0.000000,0.000000}%
\pgfsetstrokecolor{textcolor}%
\pgfsetfillcolor{textcolor}%
\pgftext[x=3.900000in,y=0.430778in,,top]{\color{textcolor}\rmfamily\fontsize{10.000000}{12.000000}\selectfont petal\_length}%
\end{pgfscope}%
\begin{pgfscope}%
\pgfsetbuttcap%
\pgfsetroundjoin%
\definecolor{currentfill}{rgb}{0.000000,0.000000,0.000000}%
\pgfsetfillcolor{currentfill}%
\pgfsetlinewidth{0.803000pt}%
\definecolor{currentstroke}{rgb}{0.000000,0.000000,0.000000}%
\pgfsetstrokecolor{currentstroke}%
\pgfsetdash{}{0pt}%
\pgfsys@defobject{currentmarker}{\pgfqpoint{0.000000in}{-0.048611in}}{\pgfqpoint{0.000000in}{0.000000in}}{%
\pgfpathmoveto{\pgfqpoint{0.000000in}{0.000000in}}%
\pgfpathlineto{\pgfqpoint{0.000000in}{-0.048611in}}%
\pgfusepath{stroke,fill}%
}%
\begin{pgfscope}%
\pgfsys@transformshift{5.140000in}{0.528000in}%
\pgfsys@useobject{currentmarker}{}%
\end{pgfscope}%
\end{pgfscope}%
\begin{pgfscope}%
\definecolor{textcolor}{rgb}{0.000000,0.000000,0.000000}%
\pgfsetstrokecolor{textcolor}%
\pgfsetfillcolor{textcolor}%
\pgftext[x=5.140000in,y=0.430778in,,top]{\color{textcolor}\rmfamily\fontsize{10.000000}{12.000000}\selectfont petal\_width}%
\end{pgfscope}%
\begin{pgfscope}%
\pgfsetbuttcap%
\pgfsetroundjoin%
\definecolor{currentfill}{rgb}{0.000000,0.000000,0.000000}%
\pgfsetfillcolor{currentfill}%
\pgfsetlinewidth{0.803000pt}%
\definecolor{currentstroke}{rgb}{0.000000,0.000000,0.000000}%
\pgfsetstrokecolor{currentstroke}%
\pgfsetdash{}{0pt}%
\pgfsys@defobject{currentmarker}{\pgfqpoint{-0.048611in}{0.000000in}}{\pgfqpoint{-0.000000in}{0.000000in}}{%
\pgfpathmoveto{\pgfqpoint{-0.000000in}{0.000000in}}%
\pgfpathlineto{\pgfqpoint{-0.048611in}{0.000000in}}%
\pgfusepath{stroke,fill}%
}%
\begin{pgfscope}%
\pgfsys@transformshift{0.800000in}{0.652923in}%
\pgfsys@useobject{currentmarker}{}%
\end{pgfscope}%
\end{pgfscope}%
\begin{pgfscope}%
\definecolor{textcolor}{rgb}{0.000000,0.000000,0.000000}%
\pgfsetstrokecolor{textcolor}%
\pgfsetfillcolor{textcolor}%
\pgftext[x=0.633333in, y=0.604698in, left, base]{\color{textcolor}\rmfamily\fontsize{10.000000}{12.000000}\selectfont \(\displaystyle {0}\)}%
\end{pgfscope}%
\begin{pgfscope}%
\pgfsetbuttcap%
\pgfsetroundjoin%
\definecolor{currentfill}{rgb}{0.000000,0.000000,0.000000}%
\pgfsetfillcolor{currentfill}%
\pgfsetlinewidth{0.803000pt}%
\definecolor{currentstroke}{rgb}{0.000000,0.000000,0.000000}%
\pgfsetstrokecolor{currentstroke}%
\pgfsetdash{}{0pt}%
\pgfsys@defobject{currentmarker}{\pgfqpoint{-0.048611in}{0.000000in}}{\pgfqpoint{-0.000000in}{0.000000in}}{%
\pgfpathmoveto{\pgfqpoint{-0.000000in}{0.000000in}}%
\pgfpathlineto{\pgfqpoint{-0.048611in}{0.000000in}}%
\pgfusepath{stroke,fill}%
}%
\begin{pgfscope}%
\pgfsys@transformshift{0.800000in}{1.083692in}%
\pgfsys@useobject{currentmarker}{}%
\end{pgfscope}%
\end{pgfscope}%
\begin{pgfscope}%
\definecolor{textcolor}{rgb}{0.000000,0.000000,0.000000}%
\pgfsetstrokecolor{textcolor}%
\pgfsetfillcolor{textcolor}%
\pgftext[x=0.633333in, y=1.035467in, left, base]{\color{textcolor}\rmfamily\fontsize{10.000000}{12.000000}\selectfont \(\displaystyle {1}\)}%
\end{pgfscope}%
\begin{pgfscope}%
\pgfsetbuttcap%
\pgfsetroundjoin%
\definecolor{currentfill}{rgb}{0.000000,0.000000,0.000000}%
\pgfsetfillcolor{currentfill}%
\pgfsetlinewidth{0.803000pt}%
\definecolor{currentstroke}{rgb}{0.000000,0.000000,0.000000}%
\pgfsetstrokecolor{currentstroke}%
\pgfsetdash{}{0pt}%
\pgfsys@defobject{currentmarker}{\pgfqpoint{-0.048611in}{0.000000in}}{\pgfqpoint{-0.000000in}{0.000000in}}{%
\pgfpathmoveto{\pgfqpoint{-0.000000in}{0.000000in}}%
\pgfpathlineto{\pgfqpoint{-0.048611in}{0.000000in}}%
\pgfusepath{stroke,fill}%
}%
\begin{pgfscope}%
\pgfsys@transformshift{0.800000in}{1.514462in}%
\pgfsys@useobject{currentmarker}{}%
\end{pgfscope}%
\end{pgfscope}%
\begin{pgfscope}%
\definecolor{textcolor}{rgb}{0.000000,0.000000,0.000000}%
\pgfsetstrokecolor{textcolor}%
\pgfsetfillcolor{textcolor}%
\pgftext[x=0.633333in, y=1.466236in, left, base]{\color{textcolor}\rmfamily\fontsize{10.000000}{12.000000}\selectfont \(\displaystyle {2}\)}%
\end{pgfscope}%
\begin{pgfscope}%
\pgfsetbuttcap%
\pgfsetroundjoin%
\definecolor{currentfill}{rgb}{0.000000,0.000000,0.000000}%
\pgfsetfillcolor{currentfill}%
\pgfsetlinewidth{0.803000pt}%
\definecolor{currentstroke}{rgb}{0.000000,0.000000,0.000000}%
\pgfsetstrokecolor{currentstroke}%
\pgfsetdash{}{0pt}%
\pgfsys@defobject{currentmarker}{\pgfqpoint{-0.048611in}{0.000000in}}{\pgfqpoint{-0.000000in}{0.000000in}}{%
\pgfpathmoveto{\pgfqpoint{-0.000000in}{0.000000in}}%
\pgfpathlineto{\pgfqpoint{-0.048611in}{0.000000in}}%
\pgfusepath{stroke,fill}%
}%
\begin{pgfscope}%
\pgfsys@transformshift{0.800000in}{1.945231in}%
\pgfsys@useobject{currentmarker}{}%
\end{pgfscope}%
\end{pgfscope}%
\begin{pgfscope}%
\definecolor{textcolor}{rgb}{0.000000,0.000000,0.000000}%
\pgfsetstrokecolor{textcolor}%
\pgfsetfillcolor{textcolor}%
\pgftext[x=0.633333in, y=1.897005in, left, base]{\color{textcolor}\rmfamily\fontsize{10.000000}{12.000000}\selectfont \(\displaystyle {3}\)}%
\end{pgfscope}%
\begin{pgfscope}%
\pgfsetbuttcap%
\pgfsetroundjoin%
\definecolor{currentfill}{rgb}{0.000000,0.000000,0.000000}%
\pgfsetfillcolor{currentfill}%
\pgfsetlinewidth{0.803000pt}%
\definecolor{currentstroke}{rgb}{0.000000,0.000000,0.000000}%
\pgfsetstrokecolor{currentstroke}%
\pgfsetdash{}{0pt}%
\pgfsys@defobject{currentmarker}{\pgfqpoint{-0.048611in}{0.000000in}}{\pgfqpoint{-0.000000in}{0.000000in}}{%
\pgfpathmoveto{\pgfqpoint{-0.000000in}{0.000000in}}%
\pgfpathlineto{\pgfqpoint{-0.048611in}{0.000000in}}%
\pgfusepath{stroke,fill}%
}%
\begin{pgfscope}%
\pgfsys@transformshift{0.800000in}{2.376000in}%
\pgfsys@useobject{currentmarker}{}%
\end{pgfscope}%
\end{pgfscope}%
\begin{pgfscope}%
\definecolor{textcolor}{rgb}{0.000000,0.000000,0.000000}%
\pgfsetstrokecolor{textcolor}%
\pgfsetfillcolor{textcolor}%
\pgftext[x=0.633333in, y=2.327775in, left, base]{\color{textcolor}\rmfamily\fontsize{10.000000}{12.000000}\selectfont \(\displaystyle {4}\)}%
\end{pgfscope}%
\begin{pgfscope}%
\pgfsetbuttcap%
\pgfsetroundjoin%
\definecolor{currentfill}{rgb}{0.000000,0.000000,0.000000}%
\pgfsetfillcolor{currentfill}%
\pgfsetlinewidth{0.803000pt}%
\definecolor{currentstroke}{rgb}{0.000000,0.000000,0.000000}%
\pgfsetstrokecolor{currentstroke}%
\pgfsetdash{}{0pt}%
\pgfsys@defobject{currentmarker}{\pgfqpoint{-0.048611in}{0.000000in}}{\pgfqpoint{-0.000000in}{0.000000in}}{%
\pgfpathmoveto{\pgfqpoint{-0.000000in}{0.000000in}}%
\pgfpathlineto{\pgfqpoint{-0.048611in}{0.000000in}}%
\pgfusepath{stroke,fill}%
}%
\begin{pgfscope}%
\pgfsys@transformshift{0.800000in}{2.806769in}%
\pgfsys@useobject{currentmarker}{}%
\end{pgfscope}%
\end{pgfscope}%
\begin{pgfscope}%
\definecolor{textcolor}{rgb}{0.000000,0.000000,0.000000}%
\pgfsetstrokecolor{textcolor}%
\pgfsetfillcolor{textcolor}%
\pgftext[x=0.633333in, y=2.758544in, left, base]{\color{textcolor}\rmfamily\fontsize{10.000000}{12.000000}\selectfont \(\displaystyle {5}\)}%
\end{pgfscope}%
\begin{pgfscope}%
\pgfsetbuttcap%
\pgfsetroundjoin%
\definecolor{currentfill}{rgb}{0.000000,0.000000,0.000000}%
\pgfsetfillcolor{currentfill}%
\pgfsetlinewidth{0.803000pt}%
\definecolor{currentstroke}{rgb}{0.000000,0.000000,0.000000}%
\pgfsetstrokecolor{currentstroke}%
\pgfsetdash{}{0pt}%
\pgfsys@defobject{currentmarker}{\pgfqpoint{-0.048611in}{0.000000in}}{\pgfqpoint{-0.000000in}{0.000000in}}{%
\pgfpathmoveto{\pgfqpoint{-0.000000in}{0.000000in}}%
\pgfpathlineto{\pgfqpoint{-0.048611in}{0.000000in}}%
\pgfusepath{stroke,fill}%
}%
\begin{pgfscope}%
\pgfsys@transformshift{0.800000in}{3.237538in}%
\pgfsys@useobject{currentmarker}{}%
\end{pgfscope}%
\end{pgfscope}%
\begin{pgfscope}%
\definecolor{textcolor}{rgb}{0.000000,0.000000,0.000000}%
\pgfsetstrokecolor{textcolor}%
\pgfsetfillcolor{textcolor}%
\pgftext[x=0.633333in, y=3.189313in, left, base]{\color{textcolor}\rmfamily\fontsize{10.000000}{12.000000}\selectfont \(\displaystyle {6}\)}%
\end{pgfscope}%
\begin{pgfscope}%
\pgfsetbuttcap%
\pgfsetroundjoin%
\definecolor{currentfill}{rgb}{0.000000,0.000000,0.000000}%
\pgfsetfillcolor{currentfill}%
\pgfsetlinewidth{0.803000pt}%
\definecolor{currentstroke}{rgb}{0.000000,0.000000,0.000000}%
\pgfsetstrokecolor{currentstroke}%
\pgfsetdash{}{0pt}%
\pgfsys@defobject{currentmarker}{\pgfqpoint{-0.048611in}{0.000000in}}{\pgfqpoint{-0.000000in}{0.000000in}}{%
\pgfpathmoveto{\pgfqpoint{-0.000000in}{0.000000in}}%
\pgfpathlineto{\pgfqpoint{-0.048611in}{0.000000in}}%
\pgfusepath{stroke,fill}%
}%
\begin{pgfscope}%
\pgfsys@transformshift{0.800000in}{3.668308in}%
\pgfsys@useobject{currentmarker}{}%
\end{pgfscope}%
\end{pgfscope}%
\begin{pgfscope}%
\definecolor{textcolor}{rgb}{0.000000,0.000000,0.000000}%
\pgfsetstrokecolor{textcolor}%
\pgfsetfillcolor{textcolor}%
\pgftext[x=0.633333in, y=3.620082in, left, base]{\color{textcolor}\rmfamily\fontsize{10.000000}{12.000000}\selectfont \(\displaystyle {7}\)}%
\end{pgfscope}%
\begin{pgfscope}%
\pgfsetbuttcap%
\pgfsetroundjoin%
\definecolor{currentfill}{rgb}{0.000000,0.000000,0.000000}%
\pgfsetfillcolor{currentfill}%
\pgfsetlinewidth{0.803000pt}%
\definecolor{currentstroke}{rgb}{0.000000,0.000000,0.000000}%
\pgfsetstrokecolor{currentstroke}%
\pgfsetdash{}{0pt}%
\pgfsys@defobject{currentmarker}{\pgfqpoint{-0.048611in}{0.000000in}}{\pgfqpoint{-0.000000in}{0.000000in}}{%
\pgfpathmoveto{\pgfqpoint{-0.000000in}{0.000000in}}%
\pgfpathlineto{\pgfqpoint{-0.048611in}{0.000000in}}%
\pgfusepath{stroke,fill}%
}%
\begin{pgfscope}%
\pgfsys@transformshift{0.800000in}{4.099077in}%
\pgfsys@useobject{currentmarker}{}%
\end{pgfscope}%
\end{pgfscope}%
\begin{pgfscope}%
\definecolor{textcolor}{rgb}{0.000000,0.000000,0.000000}%
\pgfsetstrokecolor{textcolor}%
\pgfsetfillcolor{textcolor}%
\pgftext[x=0.633333in, y=4.050852in, left, base]{\color{textcolor}\rmfamily\fontsize{10.000000}{12.000000}\selectfont \(\displaystyle {8}\)}%
\end{pgfscope}%
\begin{pgfscope}%
\pgfpathrectangle{\pgfqpoint{0.800000in}{0.528000in}}{\pgfqpoint{4.960000in}{3.696000in}}%
\pgfusepath{clip}%
\pgfsetrectcap%
\pgfsetroundjoin%
\pgfsetlinewidth{1.003750pt}%
\definecolor{currentstroke}{rgb}{0.121569,0.466667,0.705882}%
\pgfsetstrokecolor{currentstroke}%
\pgfsetdash{}{0pt}%
\pgfpathmoveto{\pgfqpoint{1.141000in}{2.849846in}}%
\pgfpathlineto{\pgfqpoint{1.699000in}{2.849846in}}%
\pgfpathlineto{\pgfqpoint{1.699000in}{3.409846in}}%
\pgfpathlineto{\pgfqpoint{1.141000in}{3.409846in}}%
\pgfpathlineto{\pgfqpoint{1.141000in}{2.849846in}}%
\pgfusepath{stroke}%
\end{pgfscope}%
\begin{pgfscope}%
\pgfpathrectangle{\pgfqpoint{0.800000in}{0.528000in}}{\pgfqpoint{4.960000in}{3.696000in}}%
\pgfusepath{clip}%
\pgfsetrectcap%
\pgfsetroundjoin%
\pgfsetlinewidth{1.003750pt}%
\definecolor{currentstroke}{rgb}{0.121569,0.466667,0.705882}%
\pgfsetstrokecolor{currentstroke}%
\pgfsetdash{}{0pt}%
\pgfpathmoveto{\pgfqpoint{1.420000in}{2.849846in}}%
\pgfpathlineto{\pgfqpoint{1.420000in}{2.505231in}}%
\pgfusepath{stroke}%
\end{pgfscope}%
\begin{pgfscope}%
\pgfpathrectangle{\pgfqpoint{0.800000in}{0.528000in}}{\pgfqpoint{4.960000in}{3.696000in}}%
\pgfusepath{clip}%
\pgfsetrectcap%
\pgfsetroundjoin%
\pgfsetlinewidth{1.003750pt}%
\definecolor{currentstroke}{rgb}{0.121569,0.466667,0.705882}%
\pgfsetstrokecolor{currentstroke}%
\pgfsetdash{}{0pt}%
\pgfpathmoveto{\pgfqpoint{1.420000in}{3.409846in}}%
\pgfpathlineto{\pgfqpoint{1.420000in}{4.056000in}}%
\pgfusepath{stroke}%
\end{pgfscope}%
\begin{pgfscope}%
\pgfpathrectangle{\pgfqpoint{0.800000in}{0.528000in}}{\pgfqpoint{4.960000in}{3.696000in}}%
\pgfusepath{clip}%
\pgfsetrectcap%
\pgfsetroundjoin%
\pgfsetlinewidth{1.003750pt}%
\definecolor{currentstroke}{rgb}{0.121569,0.466667,0.705882}%
\pgfsetstrokecolor{currentstroke}%
\pgfsetdash{}{0pt}%
\pgfpathmoveto{\pgfqpoint{1.280500in}{2.505231in}}%
\pgfpathlineto{\pgfqpoint{1.559500in}{2.505231in}}%
\pgfusepath{stroke}%
\end{pgfscope}%
\begin{pgfscope}%
\pgfpathrectangle{\pgfqpoint{0.800000in}{0.528000in}}{\pgfqpoint{4.960000in}{3.696000in}}%
\pgfusepath{clip}%
\pgfsetrectcap%
\pgfsetroundjoin%
\pgfsetlinewidth{1.003750pt}%
\definecolor{currentstroke}{rgb}{0.121569,0.466667,0.705882}%
\pgfsetstrokecolor{currentstroke}%
\pgfsetdash{}{0pt}%
\pgfpathmoveto{\pgfqpoint{1.280500in}{4.056000in}}%
\pgfpathlineto{\pgfqpoint{1.559500in}{4.056000in}}%
\pgfusepath{stroke}%
\end{pgfscope}%
\begin{pgfscope}%
\pgfpathrectangle{\pgfqpoint{0.800000in}{0.528000in}}{\pgfqpoint{4.960000in}{3.696000in}}%
\pgfusepath{clip}%
\pgfsetrectcap%
\pgfsetroundjoin%
\pgfsetlinewidth{1.003750pt}%
\definecolor{currentstroke}{rgb}{0.121569,0.466667,0.705882}%
\pgfsetstrokecolor{currentstroke}%
\pgfsetdash{}{0pt}%
\pgfpathmoveto{\pgfqpoint{2.381000in}{1.859077in}}%
\pgfpathlineto{\pgfqpoint{2.939000in}{1.859077in}}%
\pgfpathlineto{\pgfqpoint{2.939000in}{2.074462in}}%
\pgfpathlineto{\pgfqpoint{2.381000in}{2.074462in}}%
\pgfpathlineto{\pgfqpoint{2.381000in}{1.859077in}}%
\pgfusepath{stroke}%
\end{pgfscope}%
\begin{pgfscope}%
\pgfpathrectangle{\pgfqpoint{0.800000in}{0.528000in}}{\pgfqpoint{4.960000in}{3.696000in}}%
\pgfusepath{clip}%
\pgfsetrectcap%
\pgfsetroundjoin%
\pgfsetlinewidth{1.003750pt}%
\definecolor{currentstroke}{rgb}{0.121569,0.466667,0.705882}%
\pgfsetstrokecolor{currentstroke}%
\pgfsetdash{}{0pt}%
\pgfpathmoveto{\pgfqpoint{2.660000in}{1.859077in}}%
\pgfpathlineto{\pgfqpoint{2.660000in}{1.600615in}}%
\pgfusepath{stroke}%
\end{pgfscope}%
\begin{pgfscope}%
\pgfpathrectangle{\pgfqpoint{0.800000in}{0.528000in}}{\pgfqpoint{4.960000in}{3.696000in}}%
\pgfusepath{clip}%
\pgfsetrectcap%
\pgfsetroundjoin%
\pgfsetlinewidth{1.003750pt}%
\definecolor{currentstroke}{rgb}{0.121569,0.466667,0.705882}%
\pgfsetstrokecolor{currentstroke}%
\pgfsetdash{}{0pt}%
\pgfpathmoveto{\pgfqpoint{2.660000in}{2.074462in}}%
\pgfpathlineto{\pgfqpoint{2.660000in}{2.376000in}}%
\pgfusepath{stroke}%
\end{pgfscope}%
\begin{pgfscope}%
\pgfpathrectangle{\pgfqpoint{0.800000in}{0.528000in}}{\pgfqpoint{4.960000in}{3.696000in}}%
\pgfusepath{clip}%
\pgfsetrectcap%
\pgfsetroundjoin%
\pgfsetlinewidth{1.003750pt}%
\definecolor{currentstroke}{rgb}{0.121569,0.466667,0.705882}%
\pgfsetstrokecolor{currentstroke}%
\pgfsetdash{}{0pt}%
\pgfpathmoveto{\pgfqpoint{2.520500in}{1.600615in}}%
\pgfpathlineto{\pgfqpoint{2.799500in}{1.600615in}}%
\pgfusepath{stroke}%
\end{pgfscope}%
\begin{pgfscope}%
\pgfpathrectangle{\pgfqpoint{0.800000in}{0.528000in}}{\pgfqpoint{4.960000in}{3.696000in}}%
\pgfusepath{clip}%
\pgfsetrectcap%
\pgfsetroundjoin%
\pgfsetlinewidth{1.003750pt}%
\definecolor{currentstroke}{rgb}{0.121569,0.466667,0.705882}%
\pgfsetstrokecolor{currentstroke}%
\pgfsetdash{}{0pt}%
\pgfpathmoveto{\pgfqpoint{2.520500in}{2.376000in}}%
\pgfpathlineto{\pgfqpoint{2.799500in}{2.376000in}}%
\pgfusepath{stroke}%
\end{pgfscope}%
\begin{pgfscope}%
\pgfpathrectangle{\pgfqpoint{0.800000in}{0.528000in}}{\pgfqpoint{4.960000in}{3.696000in}}%
\pgfusepath{clip}%
\pgfsetbuttcap%
\pgfsetroundjoin%
\definecolor{currentfill}{rgb}{0.000000,0.000000,0.000000}%
\pgfsetfillcolor{currentfill}%
\pgfsetfillopacity{0.000000}%
\pgfsetlinewidth{1.003750pt}%
\definecolor{currentstroke}{rgb}{0.000000,0.000000,0.000000}%
\pgfsetstrokecolor{currentstroke}%
\pgfsetdash{}{0pt}%
\pgfsys@defobject{currentmarker}{\pgfqpoint{-0.041667in}{-0.041667in}}{\pgfqpoint{0.041667in}{0.041667in}}{%
\pgfpathmoveto{\pgfqpoint{0.000000in}{-0.041667in}}%
\pgfpathcurveto{\pgfqpoint{0.011050in}{-0.041667in}}{\pgfqpoint{0.021649in}{-0.037276in}}{\pgfqpoint{0.029463in}{-0.029463in}}%
\pgfpathcurveto{\pgfqpoint{0.037276in}{-0.021649in}}{\pgfqpoint{0.041667in}{-0.011050in}}{\pgfqpoint{0.041667in}{0.000000in}}%
\pgfpathcurveto{\pgfqpoint{0.041667in}{0.011050in}}{\pgfqpoint{0.037276in}{0.021649in}}{\pgfqpoint{0.029463in}{0.029463in}}%
\pgfpathcurveto{\pgfqpoint{0.021649in}{0.037276in}}{\pgfqpoint{0.011050in}{0.041667in}}{\pgfqpoint{0.000000in}{0.041667in}}%
\pgfpathcurveto{\pgfqpoint{-0.011050in}{0.041667in}}{\pgfqpoint{-0.021649in}{0.037276in}}{\pgfqpoint{-0.029463in}{0.029463in}}%
\pgfpathcurveto{\pgfqpoint{-0.037276in}{0.021649in}}{\pgfqpoint{-0.041667in}{0.011050in}}{\pgfqpoint{-0.041667in}{0.000000in}}%
\pgfpathcurveto{\pgfqpoint{-0.041667in}{-0.011050in}}{\pgfqpoint{-0.037276in}{-0.021649in}}{\pgfqpoint{-0.029463in}{-0.029463in}}%
\pgfpathcurveto{\pgfqpoint{-0.021649in}{-0.037276in}}{\pgfqpoint{-0.011050in}{-0.041667in}}{\pgfqpoint{0.000000in}{-0.041667in}}%
\pgfpathlineto{\pgfqpoint{0.000000in}{-0.041667in}}%
\pgfpathclose%
\pgfusepath{stroke,fill}%
}%
\begin{pgfscope}%
\pgfsys@transformshift{2.660000in}{1.514462in}%
\pgfsys@useobject{currentmarker}{}%
\end{pgfscope}%
\begin{pgfscope}%
\pgfsys@transformshift{2.660000in}{2.548308in}%
\pgfsys@useobject{currentmarker}{}%
\end{pgfscope}%
\begin{pgfscope}%
\pgfsys@transformshift{2.660000in}{2.419077in}%
\pgfsys@useobject{currentmarker}{}%
\end{pgfscope}%
\begin{pgfscope}%
\pgfsys@transformshift{2.660000in}{2.462154in}%
\pgfsys@useobject{currentmarker}{}%
\end{pgfscope}%
\end{pgfscope}%
\begin{pgfscope}%
\pgfpathrectangle{\pgfqpoint{0.800000in}{0.528000in}}{\pgfqpoint{4.960000in}{3.696000in}}%
\pgfusepath{clip}%
\pgfsetrectcap%
\pgfsetroundjoin%
\pgfsetlinewidth{1.003750pt}%
\definecolor{currentstroke}{rgb}{0.121569,0.466667,0.705882}%
\pgfsetstrokecolor{currentstroke}%
\pgfsetdash{}{0pt}%
\pgfpathmoveto{\pgfqpoint{3.621000in}{1.342154in}}%
\pgfpathlineto{\pgfqpoint{4.179000in}{1.342154in}}%
\pgfpathlineto{\pgfqpoint{4.179000in}{2.849846in}}%
\pgfpathlineto{\pgfqpoint{3.621000in}{2.849846in}}%
\pgfpathlineto{\pgfqpoint{3.621000in}{1.342154in}}%
\pgfusepath{stroke}%
\end{pgfscope}%
\begin{pgfscope}%
\pgfpathrectangle{\pgfqpoint{0.800000in}{0.528000in}}{\pgfqpoint{4.960000in}{3.696000in}}%
\pgfusepath{clip}%
\pgfsetrectcap%
\pgfsetroundjoin%
\pgfsetlinewidth{1.003750pt}%
\definecolor{currentstroke}{rgb}{0.121569,0.466667,0.705882}%
\pgfsetstrokecolor{currentstroke}%
\pgfsetdash{}{0pt}%
\pgfpathmoveto{\pgfqpoint{3.900000in}{1.342154in}}%
\pgfpathlineto{\pgfqpoint{3.900000in}{1.083692in}}%
\pgfusepath{stroke}%
\end{pgfscope}%
\begin{pgfscope}%
\pgfpathrectangle{\pgfqpoint{0.800000in}{0.528000in}}{\pgfqpoint{4.960000in}{3.696000in}}%
\pgfusepath{clip}%
\pgfsetrectcap%
\pgfsetroundjoin%
\pgfsetlinewidth{1.003750pt}%
\definecolor{currentstroke}{rgb}{0.121569,0.466667,0.705882}%
\pgfsetstrokecolor{currentstroke}%
\pgfsetdash{}{0pt}%
\pgfpathmoveto{\pgfqpoint{3.900000in}{2.849846in}}%
\pgfpathlineto{\pgfqpoint{3.900000in}{3.625231in}}%
\pgfusepath{stroke}%
\end{pgfscope}%
\begin{pgfscope}%
\pgfpathrectangle{\pgfqpoint{0.800000in}{0.528000in}}{\pgfqpoint{4.960000in}{3.696000in}}%
\pgfusepath{clip}%
\pgfsetrectcap%
\pgfsetroundjoin%
\pgfsetlinewidth{1.003750pt}%
\definecolor{currentstroke}{rgb}{0.121569,0.466667,0.705882}%
\pgfsetstrokecolor{currentstroke}%
\pgfsetdash{}{0pt}%
\pgfpathmoveto{\pgfqpoint{3.760500in}{1.083692in}}%
\pgfpathlineto{\pgfqpoint{4.039500in}{1.083692in}}%
\pgfusepath{stroke}%
\end{pgfscope}%
\begin{pgfscope}%
\pgfpathrectangle{\pgfqpoint{0.800000in}{0.528000in}}{\pgfqpoint{4.960000in}{3.696000in}}%
\pgfusepath{clip}%
\pgfsetrectcap%
\pgfsetroundjoin%
\pgfsetlinewidth{1.003750pt}%
\definecolor{currentstroke}{rgb}{0.121569,0.466667,0.705882}%
\pgfsetstrokecolor{currentstroke}%
\pgfsetdash{}{0pt}%
\pgfpathmoveto{\pgfqpoint{3.760500in}{3.625231in}}%
\pgfpathlineto{\pgfqpoint{4.039500in}{3.625231in}}%
\pgfusepath{stroke}%
\end{pgfscope}%
\begin{pgfscope}%
\pgfpathrectangle{\pgfqpoint{0.800000in}{0.528000in}}{\pgfqpoint{4.960000in}{3.696000in}}%
\pgfusepath{clip}%
\pgfsetrectcap%
\pgfsetroundjoin%
\pgfsetlinewidth{1.003750pt}%
\definecolor{currentstroke}{rgb}{0.121569,0.466667,0.705882}%
\pgfsetstrokecolor{currentstroke}%
\pgfsetdash{}{0pt}%
\pgfpathmoveto{\pgfqpoint{4.861000in}{0.782154in}}%
\pgfpathlineto{\pgfqpoint{5.419000in}{0.782154in}}%
\pgfpathlineto{\pgfqpoint{5.419000in}{1.428308in}}%
\pgfpathlineto{\pgfqpoint{4.861000in}{1.428308in}}%
\pgfpathlineto{\pgfqpoint{4.861000in}{0.782154in}}%
\pgfusepath{stroke}%
\end{pgfscope}%
\begin{pgfscope}%
\pgfpathrectangle{\pgfqpoint{0.800000in}{0.528000in}}{\pgfqpoint{4.960000in}{3.696000in}}%
\pgfusepath{clip}%
\pgfsetrectcap%
\pgfsetroundjoin%
\pgfsetlinewidth{1.003750pt}%
\definecolor{currentstroke}{rgb}{0.121569,0.466667,0.705882}%
\pgfsetstrokecolor{currentstroke}%
\pgfsetdash{}{0pt}%
\pgfpathmoveto{\pgfqpoint{5.140000in}{0.782154in}}%
\pgfpathlineto{\pgfqpoint{5.140000in}{0.696000in}}%
\pgfusepath{stroke}%
\end{pgfscope}%
\begin{pgfscope}%
\pgfpathrectangle{\pgfqpoint{0.800000in}{0.528000in}}{\pgfqpoint{4.960000in}{3.696000in}}%
\pgfusepath{clip}%
\pgfsetrectcap%
\pgfsetroundjoin%
\pgfsetlinewidth{1.003750pt}%
\definecolor{currentstroke}{rgb}{0.121569,0.466667,0.705882}%
\pgfsetstrokecolor{currentstroke}%
\pgfsetdash{}{0pt}%
\pgfpathmoveto{\pgfqpoint{5.140000in}{1.428308in}}%
\pgfpathlineto{\pgfqpoint{5.140000in}{1.729846in}}%
\pgfusepath{stroke}%
\end{pgfscope}%
\begin{pgfscope}%
\pgfpathrectangle{\pgfqpoint{0.800000in}{0.528000in}}{\pgfqpoint{4.960000in}{3.696000in}}%
\pgfusepath{clip}%
\pgfsetrectcap%
\pgfsetroundjoin%
\pgfsetlinewidth{1.003750pt}%
\definecolor{currentstroke}{rgb}{0.121569,0.466667,0.705882}%
\pgfsetstrokecolor{currentstroke}%
\pgfsetdash{}{0pt}%
\pgfpathmoveto{\pgfqpoint{5.000500in}{0.696000in}}%
\pgfpathlineto{\pgfqpoint{5.279500in}{0.696000in}}%
\pgfusepath{stroke}%
\end{pgfscope}%
\begin{pgfscope}%
\pgfpathrectangle{\pgfqpoint{0.800000in}{0.528000in}}{\pgfqpoint{4.960000in}{3.696000in}}%
\pgfusepath{clip}%
\pgfsetrectcap%
\pgfsetroundjoin%
\pgfsetlinewidth{1.003750pt}%
\definecolor{currentstroke}{rgb}{0.121569,0.466667,0.705882}%
\pgfsetstrokecolor{currentstroke}%
\pgfsetdash{}{0pt}%
\pgfpathmoveto{\pgfqpoint{5.000500in}{1.729846in}}%
\pgfpathlineto{\pgfqpoint{5.279500in}{1.729846in}}%
\pgfusepath{stroke}%
\end{pgfscope}%
\begin{pgfscope}%
\pgfpathrectangle{\pgfqpoint{0.800000in}{0.528000in}}{\pgfqpoint{4.960000in}{3.696000in}}%
\pgfusepath{clip}%
\pgfsetrectcap%
\pgfsetroundjoin%
\pgfsetlinewidth{1.003750pt}%
\definecolor{currentstroke}{rgb}{0.172549,0.627451,0.172549}%
\pgfsetstrokecolor{currentstroke}%
\pgfsetdash{}{0pt}%
\pgfpathmoveto{\pgfqpoint{1.141000in}{3.151385in}}%
\pgfpathlineto{\pgfqpoint{1.699000in}{3.151385in}}%
\pgfusepath{stroke}%
\end{pgfscope}%
\begin{pgfscope}%
\pgfpathrectangle{\pgfqpoint{0.800000in}{0.528000in}}{\pgfqpoint{4.960000in}{3.696000in}}%
\pgfusepath{clip}%
\pgfsetrectcap%
\pgfsetroundjoin%
\pgfsetlinewidth{1.003750pt}%
\definecolor{currentstroke}{rgb}{0.172549,0.627451,0.172549}%
\pgfsetstrokecolor{currentstroke}%
\pgfsetdash{}{0pt}%
\pgfpathmoveto{\pgfqpoint{2.381000in}{1.945231in}}%
\pgfpathlineto{\pgfqpoint{2.939000in}{1.945231in}}%
\pgfusepath{stroke}%
\end{pgfscope}%
\begin{pgfscope}%
\pgfpathrectangle{\pgfqpoint{0.800000in}{0.528000in}}{\pgfqpoint{4.960000in}{3.696000in}}%
\pgfusepath{clip}%
\pgfsetrectcap%
\pgfsetroundjoin%
\pgfsetlinewidth{1.003750pt}%
\definecolor{currentstroke}{rgb}{0.172549,0.627451,0.172549}%
\pgfsetstrokecolor{currentstroke}%
\pgfsetdash{}{0pt}%
\pgfpathmoveto{\pgfqpoint{3.621000in}{2.526769in}}%
\pgfpathlineto{\pgfqpoint{4.179000in}{2.526769in}}%
\pgfusepath{stroke}%
\end{pgfscope}%
\begin{pgfscope}%
\pgfpathrectangle{\pgfqpoint{0.800000in}{0.528000in}}{\pgfqpoint{4.960000in}{3.696000in}}%
\pgfusepath{clip}%
\pgfsetrectcap%
\pgfsetroundjoin%
\pgfsetlinewidth{1.003750pt}%
\definecolor{currentstroke}{rgb}{0.172549,0.627451,0.172549}%
\pgfsetstrokecolor{currentstroke}%
\pgfsetdash{}{0pt}%
\pgfpathmoveto{\pgfqpoint{4.861000in}{1.212923in}}%
\pgfpathlineto{\pgfqpoint{5.419000in}{1.212923in}}%
\pgfusepath{stroke}%
\end{pgfscope}%
\begin{pgfscope}%
\pgfsetrectcap%
\pgfsetmiterjoin%
\pgfsetlinewidth{0.803000pt}%
\definecolor{currentstroke}{rgb}{0.000000,0.000000,0.000000}%
\pgfsetstrokecolor{currentstroke}%
\pgfsetdash{}{0pt}%
\pgfpathmoveto{\pgfqpoint{0.800000in}{0.528000in}}%
\pgfpathlineto{\pgfqpoint{0.800000in}{4.224000in}}%
\pgfusepath{stroke}%
\end{pgfscope}%
\begin{pgfscope}%
\pgfsetrectcap%
\pgfsetmiterjoin%
\pgfsetlinewidth{0.803000pt}%
\definecolor{currentstroke}{rgb}{0.000000,0.000000,0.000000}%
\pgfsetstrokecolor{currentstroke}%
\pgfsetdash{}{0pt}%
\pgfpathmoveto{\pgfqpoint{5.760000in}{0.528000in}}%
\pgfpathlineto{\pgfqpoint{5.760000in}{4.224000in}}%
\pgfusepath{stroke}%
\end{pgfscope}%
\begin{pgfscope}%
\pgfsetrectcap%
\pgfsetmiterjoin%
\pgfsetlinewidth{0.803000pt}%
\definecolor{currentstroke}{rgb}{0.000000,0.000000,0.000000}%
\pgfsetstrokecolor{currentstroke}%
\pgfsetdash{}{0pt}%
\pgfpathmoveto{\pgfqpoint{0.800000in}{0.528000in}}%
\pgfpathlineto{\pgfqpoint{5.760000in}{0.528000in}}%
\pgfusepath{stroke}%
\end{pgfscope}%
\begin{pgfscope}%
\pgfsetrectcap%
\pgfsetmiterjoin%
\pgfsetlinewidth{0.803000pt}%
\definecolor{currentstroke}{rgb}{0.000000,0.000000,0.000000}%
\pgfsetstrokecolor{currentstroke}%
\pgfsetdash{}{0pt}%
\pgfpathmoveto{\pgfqpoint{0.800000in}{4.224000in}}%
\pgfpathlineto{\pgfqpoint{5.760000in}{4.224000in}}%
\pgfusepath{stroke}%
\end{pgfscope}%
\end{pgfpicture}%
\makeatother%
\endgroup%
}
  \end{center}
  \caption{Boxplot of iris data.}
  \label{fig:boxplot}
\end{figure}

Pairplots allow us to visualize relationships between variables. A pairplot is a type of scatterplot matrix, where each variable is plotted against all other variables in the data. This information is useful for identifying patterns and correlations between variables and understanding relationships between multiple variables, as shown in figure \ref{fig:pairplot}.

\begin{figure}
  \begin{center}
    \resizebox{1\textwidth}{!}{%% Creator: Matplotlib, PGF backend
%%
%% To include the figure in your LaTeX document, write
%%   \input{<filename>.pgf}
%%
%% Make sure the required packages are loaded in your preamble
%%   \usepackage{pgf}
%%
%% Also ensure that all the required font packages are loaded; for instance,
%% the lmodern package is sometimes necessary when using math font.
%%   \usepackage{lmodern}
%%
%% Figures using additional raster images can only be included by \input if
%% they are in the same directory as the main LaTeX file. For loading figures
%% from other directories you can use the `import` package
%%   \usepackage{import}
%%
%% and then include the figures with
%%   \import{<path to file>}{<filename>.pgf}
%%
%% Matplotlib used the following preamble
%%
\begingroup%
\makeatletter%
\begin{pgfpicture}%
\pgfpathrectangle{\pgfpointorigin}{\pgfqpoint{11.072532in}{10.000000in}}%
\pgfusepath{use as bounding box, clip}%
\begin{pgfscope}%
\pgfsetbuttcap%
\pgfsetmiterjoin%
\definecolor{currentfill}{rgb}{1.000000,1.000000,1.000000}%
\pgfsetfillcolor{currentfill}%
\pgfsetlinewidth{0.000000pt}%
\definecolor{currentstroke}{rgb}{1.000000,1.000000,1.000000}%
\pgfsetstrokecolor{currentstroke}%
\pgfsetdash{}{0pt}%
\pgfpathmoveto{\pgfqpoint{0.000000in}{0.000000in}}%
\pgfpathlineto{\pgfqpoint{11.072532in}{0.000000in}}%
\pgfpathlineto{\pgfqpoint{11.072532in}{10.000000in}}%
\pgfpathlineto{\pgfqpoint{0.000000in}{10.000000in}}%
\pgfpathlineto{\pgfqpoint{0.000000in}{0.000000in}}%
\pgfpathclose%
\pgfusepath{fill}%
\end{pgfscope}%
\begin{pgfscope}%
\pgfsetbuttcap%
\pgfsetmiterjoin%
\definecolor{currentfill}{rgb}{1.000000,1.000000,1.000000}%
\pgfsetfillcolor{currentfill}%
\pgfsetlinewidth{0.000000pt}%
\definecolor{currentstroke}{rgb}{0.000000,0.000000,0.000000}%
\pgfsetstrokecolor{currentstroke}%
\pgfsetstrokeopacity{0.000000}%
\pgfsetdash{}{0pt}%
\pgfpathmoveto{\pgfqpoint{0.633874in}{7.624184in}}%
\pgfpathlineto{\pgfqpoint{2.811154in}{7.624184in}}%
\pgfpathlineto{\pgfqpoint{2.811154in}{9.825939in}}%
\pgfpathlineto{\pgfqpoint{0.633874in}{9.825939in}}%
\pgfpathlineto{\pgfqpoint{0.633874in}{7.624184in}}%
\pgfpathclose%
\pgfusepath{fill}%
\end{pgfscope}%
\begin{pgfscope}%
\pgfsetbuttcap%
\pgfsetroundjoin%
\definecolor{currentfill}{rgb}{0.000000,0.000000,0.000000}%
\pgfsetfillcolor{currentfill}%
\pgfsetlinewidth{0.803000pt}%
\definecolor{currentstroke}{rgb}{0.000000,0.000000,0.000000}%
\pgfsetstrokecolor{currentstroke}%
\pgfsetdash{}{0pt}%
\pgfsys@defobject{currentmarker}{\pgfqpoint{0.000000in}{-0.048611in}}{\pgfqpoint{0.000000in}{0.000000in}}{%
\pgfpathmoveto{\pgfqpoint{0.000000in}{0.000000in}}%
\pgfpathlineto{\pgfqpoint{0.000000in}{-0.048611in}}%
\pgfusepath{stroke,fill}%
}%
\begin{pgfscope}%
\pgfsys@transformshift{0.806163in}{7.624184in}%
\pgfsys@useobject{currentmarker}{}%
\end{pgfscope}%
\end{pgfscope}%
\begin{pgfscope}%
\pgfsetbuttcap%
\pgfsetroundjoin%
\definecolor{currentfill}{rgb}{0.000000,0.000000,0.000000}%
\pgfsetfillcolor{currentfill}%
\pgfsetlinewidth{0.803000pt}%
\definecolor{currentstroke}{rgb}{0.000000,0.000000,0.000000}%
\pgfsetstrokecolor{currentstroke}%
\pgfsetdash{}{0pt}%
\pgfsys@defobject{currentmarker}{\pgfqpoint{0.000000in}{-0.048611in}}{\pgfqpoint{0.000000in}{0.000000in}}{%
\pgfpathmoveto{\pgfqpoint{0.000000in}{0.000000in}}%
\pgfpathlineto{\pgfqpoint{0.000000in}{-0.048611in}}%
\pgfusepath{stroke,fill}%
}%
\begin{pgfscope}%
\pgfsys@transformshift{1.604937in}{7.624184in}%
\pgfsys@useobject{currentmarker}{}%
\end{pgfscope}%
\end{pgfscope}%
\begin{pgfscope}%
\pgfsetbuttcap%
\pgfsetroundjoin%
\definecolor{currentfill}{rgb}{0.000000,0.000000,0.000000}%
\pgfsetfillcolor{currentfill}%
\pgfsetlinewidth{0.803000pt}%
\definecolor{currentstroke}{rgb}{0.000000,0.000000,0.000000}%
\pgfsetstrokecolor{currentstroke}%
\pgfsetdash{}{0pt}%
\pgfsys@defobject{currentmarker}{\pgfqpoint{0.000000in}{-0.048611in}}{\pgfqpoint{0.000000in}{0.000000in}}{%
\pgfpathmoveto{\pgfqpoint{0.000000in}{0.000000in}}%
\pgfpathlineto{\pgfqpoint{0.000000in}{-0.048611in}}%
\pgfusepath{stroke,fill}%
}%
\begin{pgfscope}%
\pgfsys@transformshift{2.403711in}{7.624184in}%
\pgfsys@useobject{currentmarker}{}%
\end{pgfscope}%
\end{pgfscope}%
\begin{pgfscope}%
\pgfsetbuttcap%
\pgfsetroundjoin%
\definecolor{currentfill}{rgb}{0.000000,0.000000,0.000000}%
\pgfsetfillcolor{currentfill}%
\pgfsetlinewidth{0.803000pt}%
\definecolor{currentstroke}{rgb}{0.000000,0.000000,0.000000}%
\pgfsetstrokecolor{currentstroke}%
\pgfsetdash{}{0pt}%
\pgfsys@defobject{currentmarker}{\pgfqpoint{-0.048611in}{0.000000in}}{\pgfqpoint{-0.000000in}{0.000000in}}{%
\pgfpathmoveto{\pgfqpoint{-0.000000in}{0.000000in}}%
\pgfpathlineto{\pgfqpoint{-0.048611in}{0.000000in}}%
\pgfusepath{stroke,fill}%
}%
\begin{pgfscope}%
\pgfsys@transformshift{0.633874in}{8.113463in}%
\pgfsys@useobject{currentmarker}{}%
\end{pgfscope}%
\end{pgfscope}%
\begin{pgfscope}%
\definecolor{textcolor}{rgb}{0.000000,0.000000,0.000000}%
\pgfsetstrokecolor{textcolor}%
\pgfsetfillcolor{textcolor}%
\pgftext[x=0.467207in, y=8.065237in, left, base]{\color{textcolor}\rmfamily\fontsize{10.000000}{12.000000}\selectfont \(\displaystyle {5}\)}%
\end{pgfscope}%
\begin{pgfscope}%
\pgfsetbuttcap%
\pgfsetroundjoin%
\definecolor{currentfill}{rgb}{0.000000,0.000000,0.000000}%
\pgfsetfillcolor{currentfill}%
\pgfsetlinewidth{0.803000pt}%
\definecolor{currentstroke}{rgb}{0.000000,0.000000,0.000000}%
\pgfsetstrokecolor{currentstroke}%
\pgfsetdash{}{0pt}%
\pgfsys@defobject{currentmarker}{\pgfqpoint{-0.048611in}{0.000000in}}{\pgfqpoint{-0.000000in}{0.000000in}}{%
\pgfpathmoveto{\pgfqpoint{-0.000000in}{0.000000in}}%
\pgfpathlineto{\pgfqpoint{-0.048611in}{0.000000in}}%
\pgfusepath{stroke,fill}%
}%
\begin{pgfscope}%
\pgfsys@transformshift{0.633874in}{8.669461in}%
\pgfsys@useobject{currentmarker}{}%
\end{pgfscope}%
\end{pgfscope}%
\begin{pgfscope}%
\definecolor{textcolor}{rgb}{0.000000,0.000000,0.000000}%
\pgfsetstrokecolor{textcolor}%
\pgfsetfillcolor{textcolor}%
\pgftext[x=0.467207in, y=8.621236in, left, base]{\color{textcolor}\rmfamily\fontsize{10.000000}{12.000000}\selectfont \(\displaystyle {6}\)}%
\end{pgfscope}%
\begin{pgfscope}%
\pgfsetbuttcap%
\pgfsetroundjoin%
\definecolor{currentfill}{rgb}{0.000000,0.000000,0.000000}%
\pgfsetfillcolor{currentfill}%
\pgfsetlinewidth{0.803000pt}%
\definecolor{currentstroke}{rgb}{0.000000,0.000000,0.000000}%
\pgfsetstrokecolor{currentstroke}%
\pgfsetdash{}{0pt}%
\pgfsys@defobject{currentmarker}{\pgfqpoint{-0.048611in}{0.000000in}}{\pgfqpoint{-0.000000in}{0.000000in}}{%
\pgfpathmoveto{\pgfqpoint{-0.000000in}{0.000000in}}%
\pgfpathlineto{\pgfqpoint{-0.048611in}{0.000000in}}%
\pgfusepath{stroke,fill}%
}%
\begin{pgfscope}%
\pgfsys@transformshift{0.633874in}{9.225460in}%
\pgfsys@useobject{currentmarker}{}%
\end{pgfscope}%
\end{pgfscope}%
\begin{pgfscope}%
\definecolor{textcolor}{rgb}{0.000000,0.000000,0.000000}%
\pgfsetstrokecolor{textcolor}%
\pgfsetfillcolor{textcolor}%
\pgftext[x=0.467207in, y=9.177235in, left, base]{\color{textcolor}\rmfamily\fontsize{10.000000}{12.000000}\selectfont \(\displaystyle {7}\)}%
\end{pgfscope}%
\begin{pgfscope}%
\pgfsetbuttcap%
\pgfsetroundjoin%
\definecolor{currentfill}{rgb}{0.000000,0.000000,0.000000}%
\pgfsetfillcolor{currentfill}%
\pgfsetlinewidth{0.803000pt}%
\definecolor{currentstroke}{rgb}{0.000000,0.000000,0.000000}%
\pgfsetstrokecolor{currentstroke}%
\pgfsetdash{}{0pt}%
\pgfsys@defobject{currentmarker}{\pgfqpoint{-0.048611in}{0.000000in}}{\pgfqpoint{-0.000000in}{0.000000in}}{%
\pgfpathmoveto{\pgfqpoint{-0.000000in}{0.000000in}}%
\pgfpathlineto{\pgfqpoint{-0.048611in}{0.000000in}}%
\pgfusepath{stroke,fill}%
}%
\begin{pgfscope}%
\pgfsys@transformshift{0.633874in}{9.781459in}%
\pgfsys@useobject{currentmarker}{}%
\end{pgfscope}%
\end{pgfscope}%
\begin{pgfscope}%
\definecolor{textcolor}{rgb}{0.000000,0.000000,0.000000}%
\pgfsetstrokecolor{textcolor}%
\pgfsetfillcolor{textcolor}%
\pgftext[x=0.467207in, y=9.733234in, left, base]{\color{textcolor}\rmfamily\fontsize{10.000000}{12.000000}\selectfont \(\displaystyle {8}\)}%
\end{pgfscope}%
\begin{pgfscope}%
\definecolor{textcolor}{rgb}{0.000000,0.000000,0.000000}%
\pgfsetstrokecolor{textcolor}%
\pgfsetfillcolor{textcolor}%
\pgftext[x=0.411651in,y=8.725061in,,bottom,rotate=90.000000]{\color{textcolor}\rmfamily\fontsize{10.000000}{12.000000}\selectfont sepal\_length}%
\end{pgfscope}%
\begin{pgfscope}%
\pgfsetrectcap%
\pgfsetmiterjoin%
\pgfsetlinewidth{0.803000pt}%
\definecolor{currentstroke}{rgb}{0.000000,0.000000,0.000000}%
\pgfsetstrokecolor{currentstroke}%
\pgfsetdash{}{0pt}%
\pgfpathmoveto{\pgfqpoint{0.633874in}{7.624184in}}%
\pgfpathlineto{\pgfqpoint{0.633874in}{9.825939in}}%
\pgfusepath{stroke}%
\end{pgfscope}%
\begin{pgfscope}%
\pgfsetrectcap%
\pgfsetmiterjoin%
\pgfsetlinewidth{0.803000pt}%
\definecolor{currentstroke}{rgb}{0.000000,0.000000,0.000000}%
\pgfsetstrokecolor{currentstroke}%
\pgfsetdash{}{0pt}%
\pgfpathmoveto{\pgfqpoint{0.633874in}{7.624184in}}%
\pgfpathlineto{\pgfqpoint{2.811154in}{7.624184in}}%
\pgfusepath{stroke}%
\end{pgfscope}%
\begin{pgfscope}%
\pgfsetbuttcap%
\pgfsetmiterjoin%
\definecolor{currentfill}{rgb}{1.000000,1.000000,1.000000}%
\pgfsetfillcolor{currentfill}%
\pgfsetlinewidth{0.000000pt}%
\definecolor{currentstroke}{rgb}{0.000000,0.000000,0.000000}%
\pgfsetstrokecolor{currentstroke}%
\pgfsetstrokeopacity{0.000000}%
\pgfsetdash{}{0pt}%
\pgfpathmoveto{\pgfqpoint{2.963410in}{7.624184in}}%
\pgfpathlineto{\pgfqpoint{5.140690in}{7.624184in}}%
\pgfpathlineto{\pgfqpoint{5.140690in}{9.825939in}}%
\pgfpathlineto{\pgfqpoint{2.963410in}{9.825939in}}%
\pgfpathlineto{\pgfqpoint{2.963410in}{7.624184in}}%
\pgfpathclose%
\pgfusepath{fill}%
\end{pgfscope}%
\begin{pgfscope}%
\pgfpathrectangle{\pgfqpoint{2.963410in}{7.624184in}}{\pgfqpoint{2.177280in}{2.201755in}}%
\pgfusepath{clip}%
\pgfsetbuttcap%
\pgfsetroundjoin%
\definecolor{currentfill}{rgb}{0.121569,0.466667,0.705882}%
\pgfsetfillcolor{currentfill}%
\pgfsetlinewidth{0.481800pt}%
\definecolor{currentstroke}{rgb}{1.000000,1.000000,1.000000}%
\pgfsetstrokecolor{currentstroke}%
\pgfsetdash{}{0pt}%
\pgfpathmoveto{\pgfqpoint{4.202828in}{8.127396in}}%
\pgfpathcurveto{\pgfqpoint{4.213878in}{8.127396in}}{\pgfqpoint{4.224477in}{8.131786in}}{\pgfqpoint{4.232291in}{8.139600in}}%
\pgfpathcurveto{\pgfqpoint{4.240104in}{8.147413in}}{\pgfqpoint{4.244495in}{8.158012in}}{\pgfqpoint{4.244495in}{8.169063in}}%
\pgfpathcurveto{\pgfqpoint{4.244495in}{8.180113in}}{\pgfqpoint{4.240104in}{8.190712in}}{\pgfqpoint{4.232291in}{8.198525in}}%
\pgfpathcurveto{\pgfqpoint{4.224477in}{8.206339in}}{\pgfqpoint{4.213878in}{8.210729in}}{\pgfqpoint{4.202828in}{8.210729in}}%
\pgfpathcurveto{\pgfqpoint{4.191778in}{8.210729in}}{\pgfqpoint{4.181179in}{8.206339in}}{\pgfqpoint{4.173365in}{8.198525in}}%
\pgfpathcurveto{\pgfqpoint{4.165552in}{8.190712in}}{\pgfqpoint{4.161161in}{8.180113in}}{\pgfqpoint{4.161161in}{8.169063in}}%
\pgfpathcurveto{\pgfqpoint{4.161161in}{8.158012in}}{\pgfqpoint{4.165552in}{8.147413in}}{\pgfqpoint{4.173365in}{8.139600in}}%
\pgfpathcurveto{\pgfqpoint{4.181179in}{8.131786in}}{\pgfqpoint{4.191778in}{8.127396in}}{\pgfqpoint{4.202828in}{8.127396in}}%
\pgfpathlineto{\pgfqpoint{4.202828in}{8.127396in}}%
\pgfpathclose%
\pgfusepath{stroke,fill}%
\end{pgfscope}%
\begin{pgfscope}%
\pgfpathrectangle{\pgfqpoint{2.963410in}{7.624184in}}{\pgfqpoint{2.177280in}{2.201755in}}%
\pgfusepath{clip}%
\pgfsetbuttcap%
\pgfsetroundjoin%
\definecolor{currentfill}{rgb}{0.121569,0.466667,0.705882}%
\pgfsetfillcolor{currentfill}%
\pgfsetlinewidth{0.481800pt}%
\definecolor{currentstroke}{rgb}{1.000000,1.000000,1.000000}%
\pgfsetstrokecolor{currentstroke}%
\pgfsetdash{}{0pt}%
\pgfpathmoveto{\pgfqpoint{3.907452in}{8.016196in}}%
\pgfpathcurveto{\pgfqpoint{3.918502in}{8.016196in}}{\pgfqpoint{3.929101in}{8.020586in}}{\pgfqpoint{3.936915in}{8.028400in}}%
\pgfpathcurveto{\pgfqpoint{3.944728in}{8.036214in}}{\pgfqpoint{3.949118in}{8.046813in}}{\pgfqpoint{3.949118in}{8.057863in}}%
\pgfpathcurveto{\pgfqpoint{3.949118in}{8.068913in}}{\pgfqpoint{3.944728in}{8.079512in}}{\pgfqpoint{3.936915in}{8.087326in}}%
\pgfpathcurveto{\pgfqpoint{3.929101in}{8.095139in}}{\pgfqpoint{3.918502in}{8.099529in}}{\pgfqpoint{3.907452in}{8.099529in}}%
\pgfpathcurveto{\pgfqpoint{3.896402in}{8.099529in}}{\pgfqpoint{3.885803in}{8.095139in}}{\pgfqpoint{3.877989in}{8.087326in}}%
\pgfpathcurveto{\pgfqpoint{3.870175in}{8.079512in}}{\pgfqpoint{3.865785in}{8.068913in}}{\pgfqpoint{3.865785in}{8.057863in}}%
\pgfpathcurveto{\pgfqpoint{3.865785in}{8.046813in}}{\pgfqpoint{3.870175in}{8.036214in}}{\pgfqpoint{3.877989in}{8.028400in}}%
\pgfpathcurveto{\pgfqpoint{3.885803in}{8.020586in}}{\pgfqpoint{3.896402in}{8.016196in}}{\pgfqpoint{3.907452in}{8.016196in}}%
\pgfpathlineto{\pgfqpoint{3.907452in}{8.016196in}}%
\pgfpathclose%
\pgfusepath{stroke,fill}%
\end{pgfscope}%
\begin{pgfscope}%
\pgfpathrectangle{\pgfqpoint{2.963410in}{7.624184in}}{\pgfqpoint{2.177280in}{2.201755in}}%
\pgfusepath{clip}%
\pgfsetbuttcap%
\pgfsetroundjoin%
\definecolor{currentfill}{rgb}{0.121569,0.466667,0.705882}%
\pgfsetfillcolor{currentfill}%
\pgfsetlinewidth{0.481800pt}%
\definecolor{currentstroke}{rgb}{1.000000,1.000000,1.000000}%
\pgfsetstrokecolor{currentstroke}%
\pgfsetdash{}{0pt}%
\pgfpathmoveto{\pgfqpoint{4.025602in}{7.904996in}}%
\pgfpathcurveto{\pgfqpoint{4.036652in}{7.904996in}}{\pgfqpoint{4.047251in}{7.909387in}}{\pgfqpoint{4.055065in}{7.917200in}}%
\pgfpathcurveto{\pgfqpoint{4.062879in}{7.925014in}}{\pgfqpoint{4.067269in}{7.935613in}}{\pgfqpoint{4.067269in}{7.946663in}}%
\pgfpathcurveto{\pgfqpoint{4.067269in}{7.957713in}}{\pgfqpoint{4.062879in}{7.968312in}}{\pgfqpoint{4.055065in}{7.976126in}}%
\pgfpathcurveto{\pgfqpoint{4.047251in}{7.983939in}}{\pgfqpoint{4.036652in}{7.988330in}}{\pgfqpoint{4.025602in}{7.988330in}}%
\pgfpathcurveto{\pgfqpoint{4.014552in}{7.988330in}}{\pgfqpoint{4.003953in}{7.983939in}}{\pgfqpoint{3.996139in}{7.976126in}}%
\pgfpathcurveto{\pgfqpoint{3.988326in}{7.968312in}}{\pgfqpoint{3.983936in}{7.957713in}}{\pgfqpoint{3.983936in}{7.946663in}}%
\pgfpathcurveto{\pgfqpoint{3.983936in}{7.935613in}}{\pgfqpoint{3.988326in}{7.925014in}}{\pgfqpoint{3.996139in}{7.917200in}}%
\pgfpathcurveto{\pgfqpoint{4.003953in}{7.909387in}}{\pgfqpoint{4.014552in}{7.904996in}}{\pgfqpoint{4.025602in}{7.904996in}}%
\pgfpathlineto{\pgfqpoint{4.025602in}{7.904996in}}%
\pgfpathclose%
\pgfusepath{stroke,fill}%
\end{pgfscope}%
\begin{pgfscope}%
\pgfpathrectangle{\pgfqpoint{2.963410in}{7.624184in}}{\pgfqpoint{2.177280in}{2.201755in}}%
\pgfusepath{clip}%
\pgfsetbuttcap%
\pgfsetroundjoin%
\definecolor{currentfill}{rgb}{0.121569,0.466667,0.705882}%
\pgfsetfillcolor{currentfill}%
\pgfsetlinewidth{0.481800pt}%
\definecolor{currentstroke}{rgb}{1.000000,1.000000,1.000000}%
\pgfsetstrokecolor{currentstroke}%
\pgfsetdash{}{0pt}%
\pgfpathmoveto{\pgfqpoint{3.966527in}{7.849396in}}%
\pgfpathcurveto{\pgfqpoint{3.977577in}{7.849396in}}{\pgfqpoint{3.988176in}{7.853787in}}{\pgfqpoint{3.995990in}{7.861600in}}%
\pgfpathcurveto{\pgfqpoint{4.003803in}{7.869414in}}{\pgfqpoint{4.008194in}{7.880013in}}{\pgfqpoint{4.008194in}{7.891063in}}%
\pgfpathcurveto{\pgfqpoint{4.008194in}{7.902113in}}{\pgfqpoint{4.003803in}{7.912712in}}{\pgfqpoint{3.995990in}{7.920526in}}%
\pgfpathcurveto{\pgfqpoint{3.988176in}{7.928340in}}{\pgfqpoint{3.977577in}{7.932730in}}{\pgfqpoint{3.966527in}{7.932730in}}%
\pgfpathcurveto{\pgfqpoint{3.955477in}{7.932730in}}{\pgfqpoint{3.944878in}{7.928340in}}{\pgfqpoint{3.937064in}{7.920526in}}%
\pgfpathcurveto{\pgfqpoint{3.929251in}{7.912712in}}{\pgfqpoint{3.924860in}{7.902113in}}{\pgfqpoint{3.924860in}{7.891063in}}%
\pgfpathcurveto{\pgfqpoint{3.924860in}{7.880013in}}{\pgfqpoint{3.929251in}{7.869414in}}{\pgfqpoint{3.937064in}{7.861600in}}%
\pgfpathcurveto{\pgfqpoint{3.944878in}{7.853787in}}{\pgfqpoint{3.955477in}{7.849396in}}{\pgfqpoint{3.966527in}{7.849396in}}%
\pgfpathlineto{\pgfqpoint{3.966527in}{7.849396in}}%
\pgfpathclose%
\pgfusepath{stroke,fill}%
\end{pgfscope}%
\begin{pgfscope}%
\pgfpathrectangle{\pgfqpoint{2.963410in}{7.624184in}}{\pgfqpoint{2.177280in}{2.201755in}}%
\pgfusepath{clip}%
\pgfsetbuttcap%
\pgfsetroundjoin%
\definecolor{currentfill}{rgb}{0.121569,0.466667,0.705882}%
\pgfsetfillcolor{currentfill}%
\pgfsetlinewidth{0.481800pt}%
\definecolor{currentstroke}{rgb}{1.000000,1.000000,1.000000}%
\pgfsetstrokecolor{currentstroke}%
\pgfsetdash{}{0pt}%
\pgfpathmoveto{\pgfqpoint{4.261903in}{8.071796in}}%
\pgfpathcurveto{\pgfqpoint{4.272953in}{8.071796in}}{\pgfqpoint{4.283552in}{8.076186in}}{\pgfqpoint{4.291366in}{8.084000in}}%
\pgfpathcurveto{\pgfqpoint{4.299180in}{8.091813in}}{\pgfqpoint{4.303570in}{8.102413in}}{\pgfqpoint{4.303570in}{8.113463in}}%
\pgfpathcurveto{\pgfqpoint{4.303570in}{8.124513in}}{\pgfqpoint{4.299180in}{8.135112in}}{\pgfqpoint{4.291366in}{8.142925in}}%
\pgfpathcurveto{\pgfqpoint{4.283552in}{8.150739in}}{\pgfqpoint{4.272953in}{8.155129in}}{\pgfqpoint{4.261903in}{8.155129in}}%
\pgfpathcurveto{\pgfqpoint{4.250853in}{8.155129in}}{\pgfqpoint{4.240254in}{8.150739in}}{\pgfqpoint{4.232441in}{8.142925in}}%
\pgfpathcurveto{\pgfqpoint{4.224627in}{8.135112in}}{\pgfqpoint{4.220237in}{8.124513in}}{\pgfqpoint{4.220237in}{8.113463in}}%
\pgfpathcurveto{\pgfqpoint{4.220237in}{8.102413in}}{\pgfqpoint{4.224627in}{8.091813in}}{\pgfqpoint{4.232441in}{8.084000in}}%
\pgfpathcurveto{\pgfqpoint{4.240254in}{8.076186in}}{\pgfqpoint{4.250853in}{8.071796in}}{\pgfqpoint{4.261903in}{8.071796in}}%
\pgfpathlineto{\pgfqpoint{4.261903in}{8.071796in}}%
\pgfpathclose%
\pgfusepath{stroke,fill}%
\end{pgfscope}%
\begin{pgfscope}%
\pgfpathrectangle{\pgfqpoint{2.963410in}{7.624184in}}{\pgfqpoint{2.177280in}{2.201755in}}%
\pgfusepath{clip}%
\pgfsetbuttcap%
\pgfsetroundjoin%
\definecolor{currentfill}{rgb}{0.121569,0.466667,0.705882}%
\pgfsetfillcolor{currentfill}%
\pgfsetlinewidth{0.481800pt}%
\definecolor{currentstroke}{rgb}{1.000000,1.000000,1.000000}%
\pgfsetstrokecolor{currentstroke}%
\pgfsetdash{}{0pt}%
\pgfpathmoveto{\pgfqpoint{4.439129in}{8.294195in}}%
\pgfpathcurveto{\pgfqpoint{4.450179in}{8.294195in}}{\pgfqpoint{4.460778in}{8.298586in}}{\pgfqpoint{4.468592in}{8.306399in}}%
\pgfpathcurveto{\pgfqpoint{4.476406in}{8.314213in}}{\pgfqpoint{4.480796in}{8.324812in}}{\pgfqpoint{4.480796in}{8.335862in}}%
\pgfpathcurveto{\pgfqpoint{4.480796in}{8.346912in}}{\pgfqpoint{4.476406in}{8.357511in}}{\pgfqpoint{4.468592in}{8.365325in}}%
\pgfpathcurveto{\pgfqpoint{4.460778in}{8.373139in}}{\pgfqpoint{4.450179in}{8.377529in}}{\pgfqpoint{4.439129in}{8.377529in}}%
\pgfpathcurveto{\pgfqpoint{4.428079in}{8.377529in}}{\pgfqpoint{4.417480in}{8.373139in}}{\pgfqpoint{4.409666in}{8.365325in}}%
\pgfpathcurveto{\pgfqpoint{4.401853in}{8.357511in}}{\pgfqpoint{4.397462in}{8.346912in}}{\pgfqpoint{4.397462in}{8.335862in}}%
\pgfpathcurveto{\pgfqpoint{4.397462in}{8.324812in}}{\pgfqpoint{4.401853in}{8.314213in}}{\pgfqpoint{4.409666in}{8.306399in}}%
\pgfpathcurveto{\pgfqpoint{4.417480in}{8.298586in}}{\pgfqpoint{4.428079in}{8.294195in}}{\pgfqpoint{4.439129in}{8.294195in}}%
\pgfpathlineto{\pgfqpoint{4.439129in}{8.294195in}}%
\pgfpathclose%
\pgfusepath{stroke,fill}%
\end{pgfscope}%
\begin{pgfscope}%
\pgfpathrectangle{\pgfqpoint{2.963410in}{7.624184in}}{\pgfqpoint{2.177280in}{2.201755in}}%
\pgfusepath{clip}%
\pgfsetbuttcap%
\pgfsetroundjoin%
\definecolor{currentfill}{rgb}{0.121569,0.466667,0.705882}%
\pgfsetfillcolor{currentfill}%
\pgfsetlinewidth{0.481800pt}%
\definecolor{currentstroke}{rgb}{1.000000,1.000000,1.000000}%
\pgfsetstrokecolor{currentstroke}%
\pgfsetdash{}{0pt}%
\pgfpathmoveto{\pgfqpoint{4.143753in}{7.849396in}}%
\pgfpathcurveto{\pgfqpoint{4.154803in}{7.849396in}}{\pgfqpoint{4.165402in}{7.853787in}}{\pgfqpoint{4.173216in}{7.861600in}}%
\pgfpathcurveto{\pgfqpoint{4.181029in}{7.869414in}}{\pgfqpoint{4.185419in}{7.880013in}}{\pgfqpoint{4.185419in}{7.891063in}}%
\pgfpathcurveto{\pgfqpoint{4.185419in}{7.902113in}}{\pgfqpoint{4.181029in}{7.912712in}}{\pgfqpoint{4.173216in}{7.920526in}}%
\pgfpathcurveto{\pgfqpoint{4.165402in}{7.928340in}}{\pgfqpoint{4.154803in}{7.932730in}}{\pgfqpoint{4.143753in}{7.932730in}}%
\pgfpathcurveto{\pgfqpoint{4.132703in}{7.932730in}}{\pgfqpoint{4.122104in}{7.928340in}}{\pgfqpoint{4.114290in}{7.920526in}}%
\pgfpathcurveto{\pgfqpoint{4.106476in}{7.912712in}}{\pgfqpoint{4.102086in}{7.902113in}}{\pgfqpoint{4.102086in}{7.891063in}}%
\pgfpathcurveto{\pgfqpoint{4.102086in}{7.880013in}}{\pgfqpoint{4.106476in}{7.869414in}}{\pgfqpoint{4.114290in}{7.861600in}}%
\pgfpathcurveto{\pgfqpoint{4.122104in}{7.853787in}}{\pgfqpoint{4.132703in}{7.849396in}}{\pgfqpoint{4.143753in}{7.849396in}}%
\pgfpathlineto{\pgfqpoint{4.143753in}{7.849396in}}%
\pgfpathclose%
\pgfusepath{stroke,fill}%
\end{pgfscope}%
\begin{pgfscope}%
\pgfpathrectangle{\pgfqpoint{2.963410in}{7.624184in}}{\pgfqpoint{2.177280in}{2.201755in}}%
\pgfusepath{clip}%
\pgfsetbuttcap%
\pgfsetroundjoin%
\definecolor{currentfill}{rgb}{0.121569,0.466667,0.705882}%
\pgfsetfillcolor{currentfill}%
\pgfsetlinewidth{0.481800pt}%
\definecolor{currentstroke}{rgb}{1.000000,1.000000,1.000000}%
\pgfsetstrokecolor{currentstroke}%
\pgfsetdash{}{0pt}%
\pgfpathmoveto{\pgfqpoint{4.143753in}{8.071796in}}%
\pgfpathcurveto{\pgfqpoint{4.154803in}{8.071796in}}{\pgfqpoint{4.165402in}{8.076186in}}{\pgfqpoint{4.173216in}{8.084000in}}%
\pgfpathcurveto{\pgfqpoint{4.181029in}{8.091813in}}{\pgfqpoint{4.185419in}{8.102413in}}{\pgfqpoint{4.185419in}{8.113463in}}%
\pgfpathcurveto{\pgfqpoint{4.185419in}{8.124513in}}{\pgfqpoint{4.181029in}{8.135112in}}{\pgfqpoint{4.173216in}{8.142925in}}%
\pgfpathcurveto{\pgfqpoint{4.165402in}{8.150739in}}{\pgfqpoint{4.154803in}{8.155129in}}{\pgfqpoint{4.143753in}{8.155129in}}%
\pgfpathcurveto{\pgfqpoint{4.132703in}{8.155129in}}{\pgfqpoint{4.122104in}{8.150739in}}{\pgfqpoint{4.114290in}{8.142925in}}%
\pgfpathcurveto{\pgfqpoint{4.106476in}{8.135112in}}{\pgfqpoint{4.102086in}{8.124513in}}{\pgfqpoint{4.102086in}{8.113463in}}%
\pgfpathcurveto{\pgfqpoint{4.102086in}{8.102413in}}{\pgfqpoint{4.106476in}{8.091813in}}{\pgfqpoint{4.114290in}{8.084000in}}%
\pgfpathcurveto{\pgfqpoint{4.122104in}{8.076186in}}{\pgfqpoint{4.132703in}{8.071796in}}{\pgfqpoint{4.143753in}{8.071796in}}%
\pgfpathlineto{\pgfqpoint{4.143753in}{8.071796in}}%
\pgfpathclose%
\pgfusepath{stroke,fill}%
\end{pgfscope}%
\begin{pgfscope}%
\pgfpathrectangle{\pgfqpoint{2.963410in}{7.624184in}}{\pgfqpoint{2.177280in}{2.201755in}}%
\pgfusepath{clip}%
\pgfsetbuttcap%
\pgfsetroundjoin%
\definecolor{currentfill}{rgb}{0.121569,0.466667,0.705882}%
\pgfsetfillcolor{currentfill}%
\pgfsetlinewidth{0.481800pt}%
\definecolor{currentstroke}{rgb}{1.000000,1.000000,1.000000}%
\pgfsetstrokecolor{currentstroke}%
\pgfsetdash{}{0pt}%
\pgfpathmoveto{\pgfqpoint{3.848376in}{7.738197in}}%
\pgfpathcurveto{\pgfqpoint{3.859427in}{7.738197in}}{\pgfqpoint{3.870026in}{7.742587in}}{\pgfqpoint{3.877839in}{7.750401in}}%
\pgfpathcurveto{\pgfqpoint{3.885653in}{7.758214in}}{\pgfqpoint{3.890043in}{7.768813in}}{\pgfqpoint{3.890043in}{7.779863in}}%
\pgfpathcurveto{\pgfqpoint{3.890043in}{7.790914in}}{\pgfqpoint{3.885653in}{7.801513in}}{\pgfqpoint{3.877839in}{7.809326in}}%
\pgfpathcurveto{\pgfqpoint{3.870026in}{7.817140in}}{\pgfqpoint{3.859427in}{7.821530in}}{\pgfqpoint{3.848376in}{7.821530in}}%
\pgfpathcurveto{\pgfqpoint{3.837326in}{7.821530in}}{\pgfqpoint{3.826727in}{7.817140in}}{\pgfqpoint{3.818914in}{7.809326in}}%
\pgfpathcurveto{\pgfqpoint{3.811100in}{7.801513in}}{\pgfqpoint{3.806710in}{7.790914in}}{\pgfqpoint{3.806710in}{7.779863in}}%
\pgfpathcurveto{\pgfqpoint{3.806710in}{7.768813in}}{\pgfqpoint{3.811100in}{7.758214in}}{\pgfqpoint{3.818914in}{7.750401in}}%
\pgfpathcurveto{\pgfqpoint{3.826727in}{7.742587in}}{\pgfqpoint{3.837326in}{7.738197in}}{\pgfqpoint{3.848376in}{7.738197in}}%
\pgfpathlineto{\pgfqpoint{3.848376in}{7.738197in}}%
\pgfpathclose%
\pgfusepath{stroke,fill}%
\end{pgfscope}%
\begin{pgfscope}%
\pgfpathrectangle{\pgfqpoint{2.963410in}{7.624184in}}{\pgfqpoint{2.177280in}{2.201755in}}%
\pgfusepath{clip}%
\pgfsetbuttcap%
\pgfsetroundjoin%
\definecolor{currentfill}{rgb}{0.121569,0.466667,0.705882}%
\pgfsetfillcolor{currentfill}%
\pgfsetlinewidth{0.481800pt}%
\definecolor{currentstroke}{rgb}{1.000000,1.000000,1.000000}%
\pgfsetstrokecolor{currentstroke}%
\pgfsetdash{}{0pt}%
\pgfpathmoveto{\pgfqpoint{3.966527in}{8.016196in}}%
\pgfpathcurveto{\pgfqpoint{3.977577in}{8.016196in}}{\pgfqpoint{3.988176in}{8.020586in}}{\pgfqpoint{3.995990in}{8.028400in}}%
\pgfpathcurveto{\pgfqpoint{4.003803in}{8.036214in}}{\pgfqpoint{4.008194in}{8.046813in}}{\pgfqpoint{4.008194in}{8.057863in}}%
\pgfpathcurveto{\pgfqpoint{4.008194in}{8.068913in}}{\pgfqpoint{4.003803in}{8.079512in}}{\pgfqpoint{3.995990in}{8.087326in}}%
\pgfpathcurveto{\pgfqpoint{3.988176in}{8.095139in}}{\pgfqpoint{3.977577in}{8.099529in}}{\pgfqpoint{3.966527in}{8.099529in}}%
\pgfpathcurveto{\pgfqpoint{3.955477in}{8.099529in}}{\pgfqpoint{3.944878in}{8.095139in}}{\pgfqpoint{3.937064in}{8.087326in}}%
\pgfpathcurveto{\pgfqpoint{3.929251in}{8.079512in}}{\pgfqpoint{3.924860in}{8.068913in}}{\pgfqpoint{3.924860in}{8.057863in}}%
\pgfpathcurveto{\pgfqpoint{3.924860in}{8.046813in}}{\pgfqpoint{3.929251in}{8.036214in}}{\pgfqpoint{3.937064in}{8.028400in}}%
\pgfpathcurveto{\pgfqpoint{3.944878in}{8.020586in}}{\pgfqpoint{3.955477in}{8.016196in}}{\pgfqpoint{3.966527in}{8.016196in}}%
\pgfpathlineto{\pgfqpoint{3.966527in}{8.016196in}}%
\pgfpathclose%
\pgfusepath{stroke,fill}%
\end{pgfscope}%
\begin{pgfscope}%
\pgfpathrectangle{\pgfqpoint{2.963410in}{7.624184in}}{\pgfqpoint{2.177280in}{2.201755in}}%
\pgfusepath{clip}%
\pgfsetbuttcap%
\pgfsetroundjoin%
\definecolor{currentfill}{rgb}{0.121569,0.466667,0.705882}%
\pgfsetfillcolor{currentfill}%
\pgfsetlinewidth{0.481800pt}%
\definecolor{currentstroke}{rgb}{1.000000,1.000000,1.000000}%
\pgfsetstrokecolor{currentstroke}%
\pgfsetdash{}{0pt}%
\pgfpathmoveto{\pgfqpoint{4.320979in}{8.294195in}}%
\pgfpathcurveto{\pgfqpoint{4.332029in}{8.294195in}}{\pgfqpoint{4.342628in}{8.298586in}}{\pgfqpoint{4.350441in}{8.306399in}}%
\pgfpathcurveto{\pgfqpoint{4.358255in}{8.314213in}}{\pgfqpoint{4.362645in}{8.324812in}}{\pgfqpoint{4.362645in}{8.335862in}}%
\pgfpathcurveto{\pgfqpoint{4.362645in}{8.346912in}}{\pgfqpoint{4.358255in}{8.357511in}}{\pgfqpoint{4.350441in}{8.365325in}}%
\pgfpathcurveto{\pgfqpoint{4.342628in}{8.373139in}}{\pgfqpoint{4.332029in}{8.377529in}}{\pgfqpoint{4.320979in}{8.377529in}}%
\pgfpathcurveto{\pgfqpoint{4.309928in}{8.377529in}}{\pgfqpoint{4.299329in}{8.373139in}}{\pgfqpoint{4.291516in}{8.365325in}}%
\pgfpathcurveto{\pgfqpoint{4.283702in}{8.357511in}}{\pgfqpoint{4.279312in}{8.346912in}}{\pgfqpoint{4.279312in}{8.335862in}}%
\pgfpathcurveto{\pgfqpoint{4.279312in}{8.324812in}}{\pgfqpoint{4.283702in}{8.314213in}}{\pgfqpoint{4.291516in}{8.306399in}}%
\pgfpathcurveto{\pgfqpoint{4.299329in}{8.298586in}}{\pgfqpoint{4.309928in}{8.294195in}}{\pgfqpoint{4.320979in}{8.294195in}}%
\pgfpathlineto{\pgfqpoint{4.320979in}{8.294195in}}%
\pgfpathclose%
\pgfusepath{stroke,fill}%
\end{pgfscope}%
\begin{pgfscope}%
\pgfpathrectangle{\pgfqpoint{2.963410in}{7.624184in}}{\pgfqpoint{2.177280in}{2.201755in}}%
\pgfusepath{clip}%
\pgfsetbuttcap%
\pgfsetroundjoin%
\definecolor{currentfill}{rgb}{0.121569,0.466667,0.705882}%
\pgfsetfillcolor{currentfill}%
\pgfsetlinewidth{0.481800pt}%
\definecolor{currentstroke}{rgb}{1.000000,1.000000,1.000000}%
\pgfsetstrokecolor{currentstroke}%
\pgfsetdash{}{0pt}%
\pgfpathmoveto{\pgfqpoint{4.143753in}{7.960596in}}%
\pgfpathcurveto{\pgfqpoint{4.154803in}{7.960596in}}{\pgfqpoint{4.165402in}{7.964986in}}{\pgfqpoint{4.173216in}{7.972800in}}%
\pgfpathcurveto{\pgfqpoint{4.181029in}{7.980614in}}{\pgfqpoint{4.185419in}{7.991213in}}{\pgfqpoint{4.185419in}{8.002263in}}%
\pgfpathcurveto{\pgfqpoint{4.185419in}{8.013313in}}{\pgfqpoint{4.181029in}{8.023912in}}{\pgfqpoint{4.173216in}{8.031726in}}%
\pgfpathcurveto{\pgfqpoint{4.165402in}{8.039539in}}{\pgfqpoint{4.154803in}{8.043930in}}{\pgfqpoint{4.143753in}{8.043930in}}%
\pgfpathcurveto{\pgfqpoint{4.132703in}{8.043930in}}{\pgfqpoint{4.122104in}{8.039539in}}{\pgfqpoint{4.114290in}{8.031726in}}%
\pgfpathcurveto{\pgfqpoint{4.106476in}{8.023912in}}{\pgfqpoint{4.102086in}{8.013313in}}{\pgfqpoint{4.102086in}{8.002263in}}%
\pgfpathcurveto{\pgfqpoint{4.102086in}{7.991213in}}{\pgfqpoint{4.106476in}{7.980614in}}{\pgfqpoint{4.114290in}{7.972800in}}%
\pgfpathcurveto{\pgfqpoint{4.122104in}{7.964986in}}{\pgfqpoint{4.132703in}{7.960596in}}{\pgfqpoint{4.143753in}{7.960596in}}%
\pgfpathlineto{\pgfqpoint{4.143753in}{7.960596in}}%
\pgfpathclose%
\pgfusepath{stroke,fill}%
\end{pgfscope}%
\begin{pgfscope}%
\pgfpathrectangle{\pgfqpoint{2.963410in}{7.624184in}}{\pgfqpoint{2.177280in}{2.201755in}}%
\pgfusepath{clip}%
\pgfsetbuttcap%
\pgfsetroundjoin%
\definecolor{currentfill}{rgb}{0.121569,0.466667,0.705882}%
\pgfsetfillcolor{currentfill}%
\pgfsetlinewidth{0.481800pt}%
\definecolor{currentstroke}{rgb}{1.000000,1.000000,1.000000}%
\pgfsetstrokecolor{currentstroke}%
\pgfsetdash{}{0pt}%
\pgfpathmoveto{\pgfqpoint{3.907452in}{7.960596in}}%
\pgfpathcurveto{\pgfqpoint{3.918502in}{7.960596in}}{\pgfqpoint{3.929101in}{7.964986in}}{\pgfqpoint{3.936915in}{7.972800in}}%
\pgfpathcurveto{\pgfqpoint{3.944728in}{7.980614in}}{\pgfqpoint{3.949118in}{7.991213in}}{\pgfqpoint{3.949118in}{8.002263in}}%
\pgfpathcurveto{\pgfqpoint{3.949118in}{8.013313in}}{\pgfqpoint{3.944728in}{8.023912in}}{\pgfqpoint{3.936915in}{8.031726in}}%
\pgfpathcurveto{\pgfqpoint{3.929101in}{8.039539in}}{\pgfqpoint{3.918502in}{8.043930in}}{\pgfqpoint{3.907452in}{8.043930in}}%
\pgfpathcurveto{\pgfqpoint{3.896402in}{8.043930in}}{\pgfqpoint{3.885803in}{8.039539in}}{\pgfqpoint{3.877989in}{8.031726in}}%
\pgfpathcurveto{\pgfqpoint{3.870175in}{8.023912in}}{\pgfqpoint{3.865785in}{8.013313in}}{\pgfqpoint{3.865785in}{8.002263in}}%
\pgfpathcurveto{\pgfqpoint{3.865785in}{7.991213in}}{\pgfqpoint{3.870175in}{7.980614in}}{\pgfqpoint{3.877989in}{7.972800in}}%
\pgfpathcurveto{\pgfqpoint{3.885803in}{7.964986in}}{\pgfqpoint{3.896402in}{7.960596in}}{\pgfqpoint{3.907452in}{7.960596in}}%
\pgfpathlineto{\pgfqpoint{3.907452in}{7.960596in}}%
\pgfpathclose%
\pgfusepath{stroke,fill}%
\end{pgfscope}%
\begin{pgfscope}%
\pgfpathrectangle{\pgfqpoint{2.963410in}{7.624184in}}{\pgfqpoint{2.177280in}{2.201755in}}%
\pgfusepath{clip}%
\pgfsetbuttcap%
\pgfsetroundjoin%
\definecolor{currentfill}{rgb}{0.121569,0.466667,0.705882}%
\pgfsetfillcolor{currentfill}%
\pgfsetlinewidth{0.481800pt}%
\definecolor{currentstroke}{rgb}{1.000000,1.000000,1.000000}%
\pgfsetstrokecolor{currentstroke}%
\pgfsetdash{}{0pt}%
\pgfpathmoveto{\pgfqpoint{3.907452in}{7.682597in}}%
\pgfpathcurveto{\pgfqpoint{3.918502in}{7.682597in}}{\pgfqpoint{3.929101in}{7.686987in}}{\pgfqpoint{3.936915in}{7.694801in}}%
\pgfpathcurveto{\pgfqpoint{3.944728in}{7.702614in}}{\pgfqpoint{3.949118in}{7.713213in}}{\pgfqpoint{3.949118in}{7.724264in}}%
\pgfpathcurveto{\pgfqpoint{3.949118in}{7.735314in}}{\pgfqpoint{3.944728in}{7.745913in}}{\pgfqpoint{3.936915in}{7.753726in}}%
\pgfpathcurveto{\pgfqpoint{3.929101in}{7.761540in}}{\pgfqpoint{3.918502in}{7.765930in}}{\pgfqpoint{3.907452in}{7.765930in}}%
\pgfpathcurveto{\pgfqpoint{3.896402in}{7.765930in}}{\pgfqpoint{3.885803in}{7.761540in}}{\pgfqpoint{3.877989in}{7.753726in}}%
\pgfpathcurveto{\pgfqpoint{3.870175in}{7.745913in}}{\pgfqpoint{3.865785in}{7.735314in}}{\pgfqpoint{3.865785in}{7.724264in}}%
\pgfpathcurveto{\pgfqpoint{3.865785in}{7.713213in}}{\pgfqpoint{3.870175in}{7.702614in}}{\pgfqpoint{3.877989in}{7.694801in}}%
\pgfpathcurveto{\pgfqpoint{3.885803in}{7.686987in}}{\pgfqpoint{3.896402in}{7.682597in}}{\pgfqpoint{3.907452in}{7.682597in}}%
\pgfpathlineto{\pgfqpoint{3.907452in}{7.682597in}}%
\pgfpathclose%
\pgfusepath{stroke,fill}%
\end{pgfscope}%
\begin{pgfscope}%
\pgfpathrectangle{\pgfqpoint{2.963410in}{7.624184in}}{\pgfqpoint{2.177280in}{2.201755in}}%
\pgfusepath{clip}%
\pgfsetbuttcap%
\pgfsetroundjoin%
\definecolor{currentfill}{rgb}{0.121569,0.466667,0.705882}%
\pgfsetfillcolor{currentfill}%
\pgfsetlinewidth{0.481800pt}%
\definecolor{currentstroke}{rgb}{1.000000,1.000000,1.000000}%
\pgfsetstrokecolor{currentstroke}%
\pgfsetdash{}{0pt}%
\pgfpathmoveto{\pgfqpoint{4.498204in}{8.516595in}}%
\pgfpathcurveto{\pgfqpoint{4.509255in}{8.516595in}}{\pgfqpoint{4.519854in}{8.520985in}}{\pgfqpoint{4.527667in}{8.528799in}}%
\pgfpathcurveto{\pgfqpoint{4.535481in}{8.536612in}}{\pgfqpoint{4.539871in}{8.547212in}}{\pgfqpoint{4.539871in}{8.558262in}}%
\pgfpathcurveto{\pgfqpoint{4.539871in}{8.569312in}}{\pgfqpoint{4.535481in}{8.579911in}}{\pgfqpoint{4.527667in}{8.587724in}}%
\pgfpathcurveto{\pgfqpoint{4.519854in}{8.595538in}}{\pgfqpoint{4.509255in}{8.599928in}}{\pgfqpoint{4.498204in}{8.599928in}}%
\pgfpathcurveto{\pgfqpoint{4.487154in}{8.599928in}}{\pgfqpoint{4.476555in}{8.595538in}}{\pgfqpoint{4.468742in}{8.587724in}}%
\pgfpathcurveto{\pgfqpoint{4.460928in}{8.579911in}}{\pgfqpoint{4.456538in}{8.569312in}}{\pgfqpoint{4.456538in}{8.558262in}}%
\pgfpathcurveto{\pgfqpoint{4.456538in}{8.547212in}}{\pgfqpoint{4.460928in}{8.536612in}}{\pgfqpoint{4.468742in}{8.528799in}}%
\pgfpathcurveto{\pgfqpoint{4.476555in}{8.520985in}}{\pgfqpoint{4.487154in}{8.516595in}}{\pgfqpoint{4.498204in}{8.516595in}}%
\pgfpathlineto{\pgfqpoint{4.498204in}{8.516595in}}%
\pgfpathclose%
\pgfusepath{stroke,fill}%
\end{pgfscope}%
\begin{pgfscope}%
\pgfpathrectangle{\pgfqpoint{2.963410in}{7.624184in}}{\pgfqpoint{2.177280in}{2.201755in}}%
\pgfusepath{clip}%
\pgfsetbuttcap%
\pgfsetroundjoin%
\definecolor{currentfill}{rgb}{0.121569,0.466667,0.705882}%
\pgfsetfillcolor{currentfill}%
\pgfsetlinewidth{0.481800pt}%
\definecolor{currentstroke}{rgb}{1.000000,1.000000,1.000000}%
\pgfsetstrokecolor{currentstroke}%
\pgfsetdash{}{0pt}%
\pgfpathmoveto{\pgfqpoint{4.734505in}{8.460995in}}%
\pgfpathcurveto{\pgfqpoint{4.745556in}{8.460995in}}{\pgfqpoint{4.756155in}{8.465385in}}{\pgfqpoint{4.763968in}{8.473199in}}%
\pgfpathcurveto{\pgfqpoint{4.771782in}{8.481013in}}{\pgfqpoint{4.776172in}{8.491612in}}{\pgfqpoint{4.776172in}{8.502662in}}%
\pgfpathcurveto{\pgfqpoint{4.776172in}{8.513712in}}{\pgfqpoint{4.771782in}{8.524311in}}{\pgfqpoint{4.763968in}{8.532125in}}%
\pgfpathcurveto{\pgfqpoint{4.756155in}{8.539938in}}{\pgfqpoint{4.745556in}{8.544328in}}{\pgfqpoint{4.734505in}{8.544328in}}%
\pgfpathcurveto{\pgfqpoint{4.723455in}{8.544328in}}{\pgfqpoint{4.712856in}{8.539938in}}{\pgfqpoint{4.705043in}{8.532125in}}%
\pgfpathcurveto{\pgfqpoint{4.697229in}{8.524311in}}{\pgfqpoint{4.692839in}{8.513712in}}{\pgfqpoint{4.692839in}{8.502662in}}%
\pgfpathcurveto{\pgfqpoint{4.692839in}{8.491612in}}{\pgfqpoint{4.697229in}{8.481013in}}{\pgfqpoint{4.705043in}{8.473199in}}%
\pgfpathcurveto{\pgfqpoint{4.712856in}{8.465385in}}{\pgfqpoint{4.723455in}{8.460995in}}{\pgfqpoint{4.734505in}{8.460995in}}%
\pgfpathlineto{\pgfqpoint{4.734505in}{8.460995in}}%
\pgfpathclose%
\pgfusepath{stroke,fill}%
\end{pgfscope}%
\begin{pgfscope}%
\pgfpathrectangle{\pgfqpoint{2.963410in}{7.624184in}}{\pgfqpoint{2.177280in}{2.201755in}}%
\pgfusepath{clip}%
\pgfsetbuttcap%
\pgfsetroundjoin%
\definecolor{currentfill}{rgb}{0.121569,0.466667,0.705882}%
\pgfsetfillcolor{currentfill}%
\pgfsetlinewidth{0.481800pt}%
\definecolor{currentstroke}{rgb}{1.000000,1.000000,1.000000}%
\pgfsetstrokecolor{currentstroke}%
\pgfsetdash{}{0pt}%
\pgfpathmoveto{\pgfqpoint{4.439129in}{8.294195in}}%
\pgfpathcurveto{\pgfqpoint{4.450179in}{8.294195in}}{\pgfqpoint{4.460778in}{8.298586in}}{\pgfqpoint{4.468592in}{8.306399in}}%
\pgfpathcurveto{\pgfqpoint{4.476406in}{8.314213in}}{\pgfqpoint{4.480796in}{8.324812in}}{\pgfqpoint{4.480796in}{8.335862in}}%
\pgfpathcurveto{\pgfqpoint{4.480796in}{8.346912in}}{\pgfqpoint{4.476406in}{8.357511in}}{\pgfqpoint{4.468592in}{8.365325in}}%
\pgfpathcurveto{\pgfqpoint{4.460778in}{8.373139in}}{\pgfqpoint{4.450179in}{8.377529in}}{\pgfqpoint{4.439129in}{8.377529in}}%
\pgfpathcurveto{\pgfqpoint{4.428079in}{8.377529in}}{\pgfqpoint{4.417480in}{8.373139in}}{\pgfqpoint{4.409666in}{8.365325in}}%
\pgfpathcurveto{\pgfqpoint{4.401853in}{8.357511in}}{\pgfqpoint{4.397462in}{8.346912in}}{\pgfqpoint{4.397462in}{8.335862in}}%
\pgfpathcurveto{\pgfqpoint{4.397462in}{8.324812in}}{\pgfqpoint{4.401853in}{8.314213in}}{\pgfqpoint{4.409666in}{8.306399in}}%
\pgfpathcurveto{\pgfqpoint{4.417480in}{8.298586in}}{\pgfqpoint{4.428079in}{8.294195in}}{\pgfqpoint{4.439129in}{8.294195in}}%
\pgfpathlineto{\pgfqpoint{4.439129in}{8.294195in}}%
\pgfpathclose%
\pgfusepath{stroke,fill}%
\end{pgfscope}%
\begin{pgfscope}%
\pgfpathrectangle{\pgfqpoint{2.963410in}{7.624184in}}{\pgfqpoint{2.177280in}{2.201755in}}%
\pgfusepath{clip}%
\pgfsetbuttcap%
\pgfsetroundjoin%
\definecolor{currentfill}{rgb}{0.121569,0.466667,0.705882}%
\pgfsetfillcolor{currentfill}%
\pgfsetlinewidth{0.481800pt}%
\definecolor{currentstroke}{rgb}{1.000000,1.000000,1.000000}%
\pgfsetstrokecolor{currentstroke}%
\pgfsetdash{}{0pt}%
\pgfpathmoveto{\pgfqpoint{4.202828in}{8.127396in}}%
\pgfpathcurveto{\pgfqpoint{4.213878in}{8.127396in}}{\pgfqpoint{4.224477in}{8.131786in}}{\pgfqpoint{4.232291in}{8.139600in}}%
\pgfpathcurveto{\pgfqpoint{4.240104in}{8.147413in}}{\pgfqpoint{4.244495in}{8.158012in}}{\pgfqpoint{4.244495in}{8.169063in}}%
\pgfpathcurveto{\pgfqpoint{4.244495in}{8.180113in}}{\pgfqpoint{4.240104in}{8.190712in}}{\pgfqpoint{4.232291in}{8.198525in}}%
\pgfpathcurveto{\pgfqpoint{4.224477in}{8.206339in}}{\pgfqpoint{4.213878in}{8.210729in}}{\pgfqpoint{4.202828in}{8.210729in}}%
\pgfpathcurveto{\pgfqpoint{4.191778in}{8.210729in}}{\pgfqpoint{4.181179in}{8.206339in}}{\pgfqpoint{4.173365in}{8.198525in}}%
\pgfpathcurveto{\pgfqpoint{4.165552in}{8.190712in}}{\pgfqpoint{4.161161in}{8.180113in}}{\pgfqpoint{4.161161in}{8.169063in}}%
\pgfpathcurveto{\pgfqpoint{4.161161in}{8.158012in}}{\pgfqpoint{4.165552in}{8.147413in}}{\pgfqpoint{4.173365in}{8.139600in}}%
\pgfpathcurveto{\pgfqpoint{4.181179in}{8.131786in}}{\pgfqpoint{4.191778in}{8.127396in}}{\pgfqpoint{4.202828in}{8.127396in}}%
\pgfpathlineto{\pgfqpoint{4.202828in}{8.127396in}}%
\pgfpathclose%
\pgfusepath{stroke,fill}%
\end{pgfscope}%
\begin{pgfscope}%
\pgfpathrectangle{\pgfqpoint{2.963410in}{7.624184in}}{\pgfqpoint{2.177280in}{2.201755in}}%
\pgfusepath{clip}%
\pgfsetbuttcap%
\pgfsetroundjoin%
\definecolor{currentfill}{rgb}{0.121569,0.466667,0.705882}%
\pgfsetfillcolor{currentfill}%
\pgfsetlinewidth{0.481800pt}%
\definecolor{currentstroke}{rgb}{1.000000,1.000000,1.000000}%
\pgfsetstrokecolor{currentstroke}%
\pgfsetdash{}{0pt}%
\pgfpathmoveto{\pgfqpoint{4.380054in}{8.460995in}}%
\pgfpathcurveto{\pgfqpoint{4.391104in}{8.460995in}}{\pgfqpoint{4.401703in}{8.465385in}}{\pgfqpoint{4.409517in}{8.473199in}}%
\pgfpathcurveto{\pgfqpoint{4.417330in}{8.481013in}}{\pgfqpoint{4.421721in}{8.491612in}}{\pgfqpoint{4.421721in}{8.502662in}}%
\pgfpathcurveto{\pgfqpoint{4.421721in}{8.513712in}}{\pgfqpoint{4.417330in}{8.524311in}}{\pgfqpoint{4.409517in}{8.532125in}}%
\pgfpathcurveto{\pgfqpoint{4.401703in}{8.539938in}}{\pgfqpoint{4.391104in}{8.544328in}}{\pgfqpoint{4.380054in}{8.544328in}}%
\pgfpathcurveto{\pgfqpoint{4.369004in}{8.544328in}}{\pgfqpoint{4.358405in}{8.539938in}}{\pgfqpoint{4.350591in}{8.532125in}}%
\pgfpathcurveto{\pgfqpoint{4.342777in}{8.524311in}}{\pgfqpoint{4.338387in}{8.513712in}}{\pgfqpoint{4.338387in}{8.502662in}}%
\pgfpathcurveto{\pgfqpoint{4.338387in}{8.491612in}}{\pgfqpoint{4.342777in}{8.481013in}}{\pgfqpoint{4.350591in}{8.473199in}}%
\pgfpathcurveto{\pgfqpoint{4.358405in}{8.465385in}}{\pgfqpoint{4.369004in}{8.460995in}}{\pgfqpoint{4.380054in}{8.460995in}}%
\pgfpathlineto{\pgfqpoint{4.380054in}{8.460995in}}%
\pgfpathclose%
\pgfusepath{stroke,fill}%
\end{pgfscope}%
\begin{pgfscope}%
\pgfpathrectangle{\pgfqpoint{2.963410in}{7.624184in}}{\pgfqpoint{2.177280in}{2.201755in}}%
\pgfusepath{clip}%
\pgfsetbuttcap%
\pgfsetroundjoin%
\definecolor{currentfill}{rgb}{0.121569,0.466667,0.705882}%
\pgfsetfillcolor{currentfill}%
\pgfsetlinewidth{0.481800pt}%
\definecolor{currentstroke}{rgb}{1.000000,1.000000,1.000000}%
\pgfsetstrokecolor{currentstroke}%
\pgfsetdash{}{0pt}%
\pgfpathmoveto{\pgfqpoint{4.380054in}{8.127396in}}%
\pgfpathcurveto{\pgfqpoint{4.391104in}{8.127396in}}{\pgfqpoint{4.401703in}{8.131786in}}{\pgfqpoint{4.409517in}{8.139600in}}%
\pgfpathcurveto{\pgfqpoint{4.417330in}{8.147413in}}{\pgfqpoint{4.421721in}{8.158012in}}{\pgfqpoint{4.421721in}{8.169063in}}%
\pgfpathcurveto{\pgfqpoint{4.421721in}{8.180113in}}{\pgfqpoint{4.417330in}{8.190712in}}{\pgfqpoint{4.409517in}{8.198525in}}%
\pgfpathcurveto{\pgfqpoint{4.401703in}{8.206339in}}{\pgfqpoint{4.391104in}{8.210729in}}{\pgfqpoint{4.380054in}{8.210729in}}%
\pgfpathcurveto{\pgfqpoint{4.369004in}{8.210729in}}{\pgfqpoint{4.358405in}{8.206339in}}{\pgfqpoint{4.350591in}{8.198525in}}%
\pgfpathcurveto{\pgfqpoint{4.342777in}{8.190712in}}{\pgfqpoint{4.338387in}{8.180113in}}{\pgfqpoint{4.338387in}{8.169063in}}%
\pgfpathcurveto{\pgfqpoint{4.338387in}{8.158012in}}{\pgfqpoint{4.342777in}{8.147413in}}{\pgfqpoint{4.350591in}{8.139600in}}%
\pgfpathcurveto{\pgfqpoint{4.358405in}{8.131786in}}{\pgfqpoint{4.369004in}{8.127396in}}{\pgfqpoint{4.380054in}{8.127396in}}%
\pgfpathlineto{\pgfqpoint{4.380054in}{8.127396in}}%
\pgfpathclose%
\pgfusepath{stroke,fill}%
\end{pgfscope}%
\begin{pgfscope}%
\pgfpathrectangle{\pgfqpoint{2.963410in}{7.624184in}}{\pgfqpoint{2.177280in}{2.201755in}}%
\pgfusepath{clip}%
\pgfsetbuttcap%
\pgfsetroundjoin%
\definecolor{currentfill}{rgb}{0.121569,0.466667,0.705882}%
\pgfsetfillcolor{currentfill}%
\pgfsetlinewidth{0.481800pt}%
\definecolor{currentstroke}{rgb}{1.000000,1.000000,1.000000}%
\pgfsetstrokecolor{currentstroke}%
\pgfsetdash{}{0pt}%
\pgfpathmoveto{\pgfqpoint{4.143753in}{8.294195in}}%
\pgfpathcurveto{\pgfqpoint{4.154803in}{8.294195in}}{\pgfqpoint{4.165402in}{8.298586in}}{\pgfqpoint{4.173216in}{8.306399in}}%
\pgfpathcurveto{\pgfqpoint{4.181029in}{8.314213in}}{\pgfqpoint{4.185419in}{8.324812in}}{\pgfqpoint{4.185419in}{8.335862in}}%
\pgfpathcurveto{\pgfqpoint{4.185419in}{8.346912in}}{\pgfqpoint{4.181029in}{8.357511in}}{\pgfqpoint{4.173216in}{8.365325in}}%
\pgfpathcurveto{\pgfqpoint{4.165402in}{8.373139in}}{\pgfqpoint{4.154803in}{8.377529in}}{\pgfqpoint{4.143753in}{8.377529in}}%
\pgfpathcurveto{\pgfqpoint{4.132703in}{8.377529in}}{\pgfqpoint{4.122104in}{8.373139in}}{\pgfqpoint{4.114290in}{8.365325in}}%
\pgfpathcurveto{\pgfqpoint{4.106476in}{8.357511in}}{\pgfqpoint{4.102086in}{8.346912in}}{\pgfqpoint{4.102086in}{8.335862in}}%
\pgfpathcurveto{\pgfqpoint{4.102086in}{8.324812in}}{\pgfqpoint{4.106476in}{8.314213in}}{\pgfqpoint{4.114290in}{8.306399in}}%
\pgfpathcurveto{\pgfqpoint{4.122104in}{8.298586in}}{\pgfqpoint{4.132703in}{8.294195in}}{\pgfqpoint{4.143753in}{8.294195in}}%
\pgfpathlineto{\pgfqpoint{4.143753in}{8.294195in}}%
\pgfpathclose%
\pgfusepath{stroke,fill}%
\end{pgfscope}%
\begin{pgfscope}%
\pgfpathrectangle{\pgfqpoint{2.963410in}{7.624184in}}{\pgfqpoint{2.177280in}{2.201755in}}%
\pgfusepath{clip}%
\pgfsetbuttcap%
\pgfsetroundjoin%
\definecolor{currentfill}{rgb}{0.121569,0.466667,0.705882}%
\pgfsetfillcolor{currentfill}%
\pgfsetlinewidth{0.481800pt}%
\definecolor{currentstroke}{rgb}{1.000000,1.000000,1.000000}%
\pgfsetstrokecolor{currentstroke}%
\pgfsetdash{}{0pt}%
\pgfpathmoveto{\pgfqpoint{4.320979in}{8.127396in}}%
\pgfpathcurveto{\pgfqpoint{4.332029in}{8.127396in}}{\pgfqpoint{4.342628in}{8.131786in}}{\pgfqpoint{4.350441in}{8.139600in}}%
\pgfpathcurveto{\pgfqpoint{4.358255in}{8.147413in}}{\pgfqpoint{4.362645in}{8.158012in}}{\pgfqpoint{4.362645in}{8.169063in}}%
\pgfpathcurveto{\pgfqpoint{4.362645in}{8.180113in}}{\pgfqpoint{4.358255in}{8.190712in}}{\pgfqpoint{4.350441in}{8.198525in}}%
\pgfpathcurveto{\pgfqpoint{4.342628in}{8.206339in}}{\pgfqpoint{4.332029in}{8.210729in}}{\pgfqpoint{4.320979in}{8.210729in}}%
\pgfpathcurveto{\pgfqpoint{4.309928in}{8.210729in}}{\pgfqpoint{4.299329in}{8.206339in}}{\pgfqpoint{4.291516in}{8.198525in}}%
\pgfpathcurveto{\pgfqpoint{4.283702in}{8.190712in}}{\pgfqpoint{4.279312in}{8.180113in}}{\pgfqpoint{4.279312in}{8.169063in}}%
\pgfpathcurveto{\pgfqpoint{4.279312in}{8.158012in}}{\pgfqpoint{4.283702in}{8.147413in}}{\pgfqpoint{4.291516in}{8.139600in}}%
\pgfpathcurveto{\pgfqpoint{4.299329in}{8.131786in}}{\pgfqpoint{4.309928in}{8.127396in}}{\pgfqpoint{4.320979in}{8.127396in}}%
\pgfpathlineto{\pgfqpoint{4.320979in}{8.127396in}}%
\pgfpathclose%
\pgfusepath{stroke,fill}%
\end{pgfscope}%
\begin{pgfscope}%
\pgfpathrectangle{\pgfqpoint{2.963410in}{7.624184in}}{\pgfqpoint{2.177280in}{2.201755in}}%
\pgfusepath{clip}%
\pgfsetbuttcap%
\pgfsetroundjoin%
\definecolor{currentfill}{rgb}{0.121569,0.466667,0.705882}%
\pgfsetfillcolor{currentfill}%
\pgfsetlinewidth{0.481800pt}%
\definecolor{currentstroke}{rgb}{1.000000,1.000000,1.000000}%
\pgfsetstrokecolor{currentstroke}%
\pgfsetdash{}{0pt}%
\pgfpathmoveto{\pgfqpoint{4.261903in}{7.849396in}}%
\pgfpathcurveto{\pgfqpoint{4.272953in}{7.849396in}}{\pgfqpoint{4.283552in}{7.853787in}}{\pgfqpoint{4.291366in}{7.861600in}}%
\pgfpathcurveto{\pgfqpoint{4.299180in}{7.869414in}}{\pgfqpoint{4.303570in}{7.880013in}}{\pgfqpoint{4.303570in}{7.891063in}}%
\pgfpathcurveto{\pgfqpoint{4.303570in}{7.902113in}}{\pgfqpoint{4.299180in}{7.912712in}}{\pgfqpoint{4.291366in}{7.920526in}}%
\pgfpathcurveto{\pgfqpoint{4.283552in}{7.928340in}}{\pgfqpoint{4.272953in}{7.932730in}}{\pgfqpoint{4.261903in}{7.932730in}}%
\pgfpathcurveto{\pgfqpoint{4.250853in}{7.932730in}}{\pgfqpoint{4.240254in}{7.928340in}}{\pgfqpoint{4.232441in}{7.920526in}}%
\pgfpathcurveto{\pgfqpoint{4.224627in}{7.912712in}}{\pgfqpoint{4.220237in}{7.902113in}}{\pgfqpoint{4.220237in}{7.891063in}}%
\pgfpathcurveto{\pgfqpoint{4.220237in}{7.880013in}}{\pgfqpoint{4.224627in}{7.869414in}}{\pgfqpoint{4.232441in}{7.861600in}}%
\pgfpathcurveto{\pgfqpoint{4.240254in}{7.853787in}}{\pgfqpoint{4.250853in}{7.849396in}}{\pgfqpoint{4.261903in}{7.849396in}}%
\pgfpathlineto{\pgfqpoint{4.261903in}{7.849396in}}%
\pgfpathclose%
\pgfusepath{stroke,fill}%
\end{pgfscope}%
\begin{pgfscope}%
\pgfpathrectangle{\pgfqpoint{2.963410in}{7.624184in}}{\pgfqpoint{2.177280in}{2.201755in}}%
\pgfusepath{clip}%
\pgfsetbuttcap%
\pgfsetroundjoin%
\definecolor{currentfill}{rgb}{0.121569,0.466667,0.705882}%
\pgfsetfillcolor{currentfill}%
\pgfsetlinewidth{0.481800pt}%
\definecolor{currentstroke}{rgb}{1.000000,1.000000,1.000000}%
\pgfsetstrokecolor{currentstroke}%
\pgfsetdash{}{0pt}%
\pgfpathmoveto{\pgfqpoint{4.084678in}{8.127396in}}%
\pgfpathcurveto{\pgfqpoint{4.095728in}{8.127396in}}{\pgfqpoint{4.106327in}{8.131786in}}{\pgfqpoint{4.114140in}{8.139600in}}%
\pgfpathcurveto{\pgfqpoint{4.121954in}{8.147413in}}{\pgfqpoint{4.126344in}{8.158012in}}{\pgfqpoint{4.126344in}{8.169063in}}%
\pgfpathcurveto{\pgfqpoint{4.126344in}{8.180113in}}{\pgfqpoint{4.121954in}{8.190712in}}{\pgfqpoint{4.114140in}{8.198525in}}%
\pgfpathcurveto{\pgfqpoint{4.106327in}{8.206339in}}{\pgfqpoint{4.095728in}{8.210729in}}{\pgfqpoint{4.084678in}{8.210729in}}%
\pgfpathcurveto{\pgfqpoint{4.073627in}{8.210729in}}{\pgfqpoint{4.063028in}{8.206339in}}{\pgfqpoint{4.055215in}{8.198525in}}%
\pgfpathcurveto{\pgfqpoint{4.047401in}{8.190712in}}{\pgfqpoint{4.043011in}{8.180113in}}{\pgfqpoint{4.043011in}{8.169063in}}%
\pgfpathcurveto{\pgfqpoint{4.043011in}{8.158012in}}{\pgfqpoint{4.047401in}{8.147413in}}{\pgfqpoint{4.055215in}{8.139600in}}%
\pgfpathcurveto{\pgfqpoint{4.063028in}{8.131786in}}{\pgfqpoint{4.073627in}{8.127396in}}{\pgfqpoint{4.084678in}{8.127396in}}%
\pgfpathlineto{\pgfqpoint{4.084678in}{8.127396in}}%
\pgfpathclose%
\pgfusepath{stroke,fill}%
\end{pgfscope}%
\begin{pgfscope}%
\pgfpathrectangle{\pgfqpoint{2.963410in}{7.624184in}}{\pgfqpoint{2.177280in}{2.201755in}}%
\pgfusepath{clip}%
\pgfsetbuttcap%
\pgfsetroundjoin%
\definecolor{currentfill}{rgb}{0.121569,0.466667,0.705882}%
\pgfsetfillcolor{currentfill}%
\pgfsetlinewidth{0.481800pt}%
\definecolor{currentstroke}{rgb}{1.000000,1.000000,1.000000}%
\pgfsetstrokecolor{currentstroke}%
\pgfsetdash{}{0pt}%
\pgfpathmoveto{\pgfqpoint{4.143753in}{7.960596in}}%
\pgfpathcurveto{\pgfqpoint{4.154803in}{7.960596in}}{\pgfqpoint{4.165402in}{7.964986in}}{\pgfqpoint{4.173216in}{7.972800in}}%
\pgfpathcurveto{\pgfqpoint{4.181029in}{7.980614in}}{\pgfqpoint{4.185419in}{7.991213in}}{\pgfqpoint{4.185419in}{8.002263in}}%
\pgfpathcurveto{\pgfqpoint{4.185419in}{8.013313in}}{\pgfqpoint{4.181029in}{8.023912in}}{\pgfqpoint{4.173216in}{8.031726in}}%
\pgfpathcurveto{\pgfqpoint{4.165402in}{8.039539in}}{\pgfqpoint{4.154803in}{8.043930in}}{\pgfqpoint{4.143753in}{8.043930in}}%
\pgfpathcurveto{\pgfqpoint{4.132703in}{8.043930in}}{\pgfqpoint{4.122104in}{8.039539in}}{\pgfqpoint{4.114290in}{8.031726in}}%
\pgfpathcurveto{\pgfqpoint{4.106476in}{8.023912in}}{\pgfqpoint{4.102086in}{8.013313in}}{\pgfqpoint{4.102086in}{8.002263in}}%
\pgfpathcurveto{\pgfqpoint{4.102086in}{7.991213in}}{\pgfqpoint{4.106476in}{7.980614in}}{\pgfqpoint{4.114290in}{7.972800in}}%
\pgfpathcurveto{\pgfqpoint{4.122104in}{7.964986in}}{\pgfqpoint{4.132703in}{7.960596in}}{\pgfqpoint{4.143753in}{7.960596in}}%
\pgfpathlineto{\pgfqpoint{4.143753in}{7.960596in}}%
\pgfpathclose%
\pgfusepath{stroke,fill}%
\end{pgfscope}%
\begin{pgfscope}%
\pgfpathrectangle{\pgfqpoint{2.963410in}{7.624184in}}{\pgfqpoint{2.177280in}{2.201755in}}%
\pgfusepath{clip}%
\pgfsetbuttcap%
\pgfsetroundjoin%
\definecolor{currentfill}{rgb}{0.121569,0.466667,0.705882}%
\pgfsetfillcolor{currentfill}%
\pgfsetlinewidth{0.481800pt}%
\definecolor{currentstroke}{rgb}{1.000000,1.000000,1.000000}%
\pgfsetstrokecolor{currentstroke}%
\pgfsetdash{}{0pt}%
\pgfpathmoveto{\pgfqpoint{3.907452in}{8.071796in}}%
\pgfpathcurveto{\pgfqpoint{3.918502in}{8.071796in}}{\pgfqpoint{3.929101in}{8.076186in}}{\pgfqpoint{3.936915in}{8.084000in}}%
\pgfpathcurveto{\pgfqpoint{3.944728in}{8.091813in}}{\pgfqpoint{3.949118in}{8.102413in}}{\pgfqpoint{3.949118in}{8.113463in}}%
\pgfpathcurveto{\pgfqpoint{3.949118in}{8.124513in}}{\pgfqpoint{3.944728in}{8.135112in}}{\pgfqpoint{3.936915in}{8.142925in}}%
\pgfpathcurveto{\pgfqpoint{3.929101in}{8.150739in}}{\pgfqpoint{3.918502in}{8.155129in}}{\pgfqpoint{3.907452in}{8.155129in}}%
\pgfpathcurveto{\pgfqpoint{3.896402in}{8.155129in}}{\pgfqpoint{3.885803in}{8.150739in}}{\pgfqpoint{3.877989in}{8.142925in}}%
\pgfpathcurveto{\pgfqpoint{3.870175in}{8.135112in}}{\pgfqpoint{3.865785in}{8.124513in}}{\pgfqpoint{3.865785in}{8.113463in}}%
\pgfpathcurveto{\pgfqpoint{3.865785in}{8.102413in}}{\pgfqpoint{3.870175in}{8.091813in}}{\pgfqpoint{3.877989in}{8.084000in}}%
\pgfpathcurveto{\pgfqpoint{3.885803in}{8.076186in}}{\pgfqpoint{3.896402in}{8.071796in}}{\pgfqpoint{3.907452in}{8.071796in}}%
\pgfpathlineto{\pgfqpoint{3.907452in}{8.071796in}}%
\pgfpathclose%
\pgfusepath{stroke,fill}%
\end{pgfscope}%
\begin{pgfscope}%
\pgfpathrectangle{\pgfqpoint{2.963410in}{7.624184in}}{\pgfqpoint{2.177280in}{2.201755in}}%
\pgfusepath{clip}%
\pgfsetbuttcap%
\pgfsetroundjoin%
\definecolor{currentfill}{rgb}{0.121569,0.466667,0.705882}%
\pgfsetfillcolor{currentfill}%
\pgfsetlinewidth{0.481800pt}%
\definecolor{currentstroke}{rgb}{1.000000,1.000000,1.000000}%
\pgfsetstrokecolor{currentstroke}%
\pgfsetdash{}{0pt}%
\pgfpathmoveto{\pgfqpoint{4.143753in}{8.071796in}}%
\pgfpathcurveto{\pgfqpoint{4.154803in}{8.071796in}}{\pgfqpoint{4.165402in}{8.076186in}}{\pgfqpoint{4.173216in}{8.084000in}}%
\pgfpathcurveto{\pgfqpoint{4.181029in}{8.091813in}}{\pgfqpoint{4.185419in}{8.102413in}}{\pgfqpoint{4.185419in}{8.113463in}}%
\pgfpathcurveto{\pgfqpoint{4.185419in}{8.124513in}}{\pgfqpoint{4.181029in}{8.135112in}}{\pgfqpoint{4.173216in}{8.142925in}}%
\pgfpathcurveto{\pgfqpoint{4.165402in}{8.150739in}}{\pgfqpoint{4.154803in}{8.155129in}}{\pgfqpoint{4.143753in}{8.155129in}}%
\pgfpathcurveto{\pgfqpoint{4.132703in}{8.155129in}}{\pgfqpoint{4.122104in}{8.150739in}}{\pgfqpoint{4.114290in}{8.142925in}}%
\pgfpathcurveto{\pgfqpoint{4.106476in}{8.135112in}}{\pgfqpoint{4.102086in}{8.124513in}}{\pgfqpoint{4.102086in}{8.113463in}}%
\pgfpathcurveto{\pgfqpoint{4.102086in}{8.102413in}}{\pgfqpoint{4.106476in}{8.091813in}}{\pgfqpoint{4.114290in}{8.084000in}}%
\pgfpathcurveto{\pgfqpoint{4.122104in}{8.076186in}}{\pgfqpoint{4.132703in}{8.071796in}}{\pgfqpoint{4.143753in}{8.071796in}}%
\pgfpathlineto{\pgfqpoint{4.143753in}{8.071796in}}%
\pgfpathclose%
\pgfusepath{stroke,fill}%
\end{pgfscope}%
\begin{pgfscope}%
\pgfpathrectangle{\pgfqpoint{2.963410in}{7.624184in}}{\pgfqpoint{2.177280in}{2.201755in}}%
\pgfusepath{clip}%
\pgfsetbuttcap%
\pgfsetroundjoin%
\definecolor{currentfill}{rgb}{0.121569,0.466667,0.705882}%
\pgfsetfillcolor{currentfill}%
\pgfsetlinewidth{0.481800pt}%
\definecolor{currentstroke}{rgb}{1.000000,1.000000,1.000000}%
\pgfsetstrokecolor{currentstroke}%
\pgfsetdash{}{0pt}%
\pgfpathmoveto{\pgfqpoint{4.202828in}{8.182996in}}%
\pgfpathcurveto{\pgfqpoint{4.213878in}{8.182996in}}{\pgfqpoint{4.224477in}{8.187386in}}{\pgfqpoint{4.232291in}{8.195200in}}%
\pgfpathcurveto{\pgfqpoint{4.240104in}{8.203013in}}{\pgfqpoint{4.244495in}{8.213612in}}{\pgfqpoint{4.244495in}{8.224662in}}%
\pgfpathcurveto{\pgfqpoint{4.244495in}{8.235713in}}{\pgfqpoint{4.240104in}{8.246312in}}{\pgfqpoint{4.232291in}{8.254125in}}%
\pgfpathcurveto{\pgfqpoint{4.224477in}{8.261939in}}{\pgfqpoint{4.213878in}{8.266329in}}{\pgfqpoint{4.202828in}{8.266329in}}%
\pgfpathcurveto{\pgfqpoint{4.191778in}{8.266329in}}{\pgfqpoint{4.181179in}{8.261939in}}{\pgfqpoint{4.173365in}{8.254125in}}%
\pgfpathcurveto{\pgfqpoint{4.165552in}{8.246312in}}{\pgfqpoint{4.161161in}{8.235713in}}{\pgfqpoint{4.161161in}{8.224662in}}%
\pgfpathcurveto{\pgfqpoint{4.161161in}{8.213612in}}{\pgfqpoint{4.165552in}{8.203013in}}{\pgfqpoint{4.173365in}{8.195200in}}%
\pgfpathcurveto{\pgfqpoint{4.181179in}{8.187386in}}{\pgfqpoint{4.191778in}{8.182996in}}{\pgfqpoint{4.202828in}{8.182996in}}%
\pgfpathlineto{\pgfqpoint{4.202828in}{8.182996in}}%
\pgfpathclose%
\pgfusepath{stroke,fill}%
\end{pgfscope}%
\begin{pgfscope}%
\pgfpathrectangle{\pgfqpoint{2.963410in}{7.624184in}}{\pgfqpoint{2.177280in}{2.201755in}}%
\pgfusepath{clip}%
\pgfsetbuttcap%
\pgfsetroundjoin%
\definecolor{currentfill}{rgb}{0.121569,0.466667,0.705882}%
\pgfsetfillcolor{currentfill}%
\pgfsetlinewidth{0.481800pt}%
\definecolor{currentstroke}{rgb}{1.000000,1.000000,1.000000}%
\pgfsetstrokecolor{currentstroke}%
\pgfsetdash{}{0pt}%
\pgfpathmoveto{\pgfqpoint{4.143753in}{8.182996in}}%
\pgfpathcurveto{\pgfqpoint{4.154803in}{8.182996in}}{\pgfqpoint{4.165402in}{8.187386in}}{\pgfqpoint{4.173216in}{8.195200in}}%
\pgfpathcurveto{\pgfqpoint{4.181029in}{8.203013in}}{\pgfqpoint{4.185419in}{8.213612in}}{\pgfqpoint{4.185419in}{8.224662in}}%
\pgfpathcurveto{\pgfqpoint{4.185419in}{8.235713in}}{\pgfqpoint{4.181029in}{8.246312in}}{\pgfqpoint{4.173216in}{8.254125in}}%
\pgfpathcurveto{\pgfqpoint{4.165402in}{8.261939in}}{\pgfqpoint{4.154803in}{8.266329in}}{\pgfqpoint{4.143753in}{8.266329in}}%
\pgfpathcurveto{\pgfqpoint{4.132703in}{8.266329in}}{\pgfqpoint{4.122104in}{8.261939in}}{\pgfqpoint{4.114290in}{8.254125in}}%
\pgfpathcurveto{\pgfqpoint{4.106476in}{8.246312in}}{\pgfqpoint{4.102086in}{8.235713in}}{\pgfqpoint{4.102086in}{8.224662in}}%
\pgfpathcurveto{\pgfqpoint{4.102086in}{8.213612in}}{\pgfqpoint{4.106476in}{8.203013in}}{\pgfqpoint{4.114290in}{8.195200in}}%
\pgfpathcurveto{\pgfqpoint{4.122104in}{8.187386in}}{\pgfqpoint{4.132703in}{8.182996in}}{\pgfqpoint{4.143753in}{8.182996in}}%
\pgfpathlineto{\pgfqpoint{4.143753in}{8.182996in}}%
\pgfpathclose%
\pgfusepath{stroke,fill}%
\end{pgfscope}%
\begin{pgfscope}%
\pgfpathrectangle{\pgfqpoint{2.963410in}{7.624184in}}{\pgfqpoint{2.177280in}{2.201755in}}%
\pgfusepath{clip}%
\pgfsetbuttcap%
\pgfsetroundjoin%
\definecolor{currentfill}{rgb}{0.121569,0.466667,0.705882}%
\pgfsetfillcolor{currentfill}%
\pgfsetlinewidth{0.481800pt}%
\definecolor{currentstroke}{rgb}{1.000000,1.000000,1.000000}%
\pgfsetstrokecolor{currentstroke}%
\pgfsetdash{}{0pt}%
\pgfpathmoveto{\pgfqpoint{4.025602in}{7.904996in}}%
\pgfpathcurveto{\pgfqpoint{4.036652in}{7.904996in}}{\pgfqpoint{4.047251in}{7.909387in}}{\pgfqpoint{4.055065in}{7.917200in}}%
\pgfpathcurveto{\pgfqpoint{4.062879in}{7.925014in}}{\pgfqpoint{4.067269in}{7.935613in}}{\pgfqpoint{4.067269in}{7.946663in}}%
\pgfpathcurveto{\pgfqpoint{4.067269in}{7.957713in}}{\pgfqpoint{4.062879in}{7.968312in}}{\pgfqpoint{4.055065in}{7.976126in}}%
\pgfpathcurveto{\pgfqpoint{4.047251in}{7.983939in}}{\pgfqpoint{4.036652in}{7.988330in}}{\pgfqpoint{4.025602in}{7.988330in}}%
\pgfpathcurveto{\pgfqpoint{4.014552in}{7.988330in}}{\pgfqpoint{4.003953in}{7.983939in}}{\pgfqpoint{3.996139in}{7.976126in}}%
\pgfpathcurveto{\pgfqpoint{3.988326in}{7.968312in}}{\pgfqpoint{3.983936in}{7.957713in}}{\pgfqpoint{3.983936in}{7.946663in}}%
\pgfpathcurveto{\pgfqpoint{3.983936in}{7.935613in}}{\pgfqpoint{3.988326in}{7.925014in}}{\pgfqpoint{3.996139in}{7.917200in}}%
\pgfpathcurveto{\pgfqpoint{4.003953in}{7.909387in}}{\pgfqpoint{4.014552in}{7.904996in}}{\pgfqpoint{4.025602in}{7.904996in}}%
\pgfpathlineto{\pgfqpoint{4.025602in}{7.904996in}}%
\pgfpathclose%
\pgfusepath{stroke,fill}%
\end{pgfscope}%
\begin{pgfscope}%
\pgfpathrectangle{\pgfqpoint{2.963410in}{7.624184in}}{\pgfqpoint{2.177280in}{2.201755in}}%
\pgfusepath{clip}%
\pgfsetbuttcap%
\pgfsetroundjoin%
\definecolor{currentfill}{rgb}{0.121569,0.466667,0.705882}%
\pgfsetfillcolor{currentfill}%
\pgfsetlinewidth{0.481800pt}%
\definecolor{currentstroke}{rgb}{1.000000,1.000000,1.000000}%
\pgfsetstrokecolor{currentstroke}%
\pgfsetdash{}{0pt}%
\pgfpathmoveto{\pgfqpoint{3.966527in}{7.960596in}}%
\pgfpathcurveto{\pgfqpoint{3.977577in}{7.960596in}}{\pgfqpoint{3.988176in}{7.964986in}}{\pgfqpoint{3.995990in}{7.972800in}}%
\pgfpathcurveto{\pgfqpoint{4.003803in}{7.980614in}}{\pgfqpoint{4.008194in}{7.991213in}}{\pgfqpoint{4.008194in}{8.002263in}}%
\pgfpathcurveto{\pgfqpoint{4.008194in}{8.013313in}}{\pgfqpoint{4.003803in}{8.023912in}}{\pgfqpoint{3.995990in}{8.031726in}}%
\pgfpathcurveto{\pgfqpoint{3.988176in}{8.039539in}}{\pgfqpoint{3.977577in}{8.043930in}}{\pgfqpoint{3.966527in}{8.043930in}}%
\pgfpathcurveto{\pgfqpoint{3.955477in}{8.043930in}}{\pgfqpoint{3.944878in}{8.039539in}}{\pgfqpoint{3.937064in}{8.031726in}}%
\pgfpathcurveto{\pgfqpoint{3.929251in}{8.023912in}}{\pgfqpoint{3.924860in}{8.013313in}}{\pgfqpoint{3.924860in}{8.002263in}}%
\pgfpathcurveto{\pgfqpoint{3.924860in}{7.991213in}}{\pgfqpoint{3.929251in}{7.980614in}}{\pgfqpoint{3.937064in}{7.972800in}}%
\pgfpathcurveto{\pgfqpoint{3.944878in}{7.964986in}}{\pgfqpoint{3.955477in}{7.960596in}}{\pgfqpoint{3.966527in}{7.960596in}}%
\pgfpathlineto{\pgfqpoint{3.966527in}{7.960596in}}%
\pgfpathclose%
\pgfusepath{stroke,fill}%
\end{pgfscope}%
\begin{pgfscope}%
\pgfpathrectangle{\pgfqpoint{2.963410in}{7.624184in}}{\pgfqpoint{2.177280in}{2.201755in}}%
\pgfusepath{clip}%
\pgfsetbuttcap%
\pgfsetroundjoin%
\definecolor{currentfill}{rgb}{0.121569,0.466667,0.705882}%
\pgfsetfillcolor{currentfill}%
\pgfsetlinewidth{0.481800pt}%
\definecolor{currentstroke}{rgb}{1.000000,1.000000,1.000000}%
\pgfsetstrokecolor{currentstroke}%
\pgfsetdash{}{0pt}%
\pgfpathmoveto{\pgfqpoint{4.143753in}{8.294195in}}%
\pgfpathcurveto{\pgfqpoint{4.154803in}{8.294195in}}{\pgfqpoint{4.165402in}{8.298586in}}{\pgfqpoint{4.173216in}{8.306399in}}%
\pgfpathcurveto{\pgfqpoint{4.181029in}{8.314213in}}{\pgfqpoint{4.185419in}{8.324812in}}{\pgfqpoint{4.185419in}{8.335862in}}%
\pgfpathcurveto{\pgfqpoint{4.185419in}{8.346912in}}{\pgfqpoint{4.181029in}{8.357511in}}{\pgfqpoint{4.173216in}{8.365325in}}%
\pgfpathcurveto{\pgfqpoint{4.165402in}{8.373139in}}{\pgfqpoint{4.154803in}{8.377529in}}{\pgfqpoint{4.143753in}{8.377529in}}%
\pgfpathcurveto{\pgfqpoint{4.132703in}{8.377529in}}{\pgfqpoint{4.122104in}{8.373139in}}{\pgfqpoint{4.114290in}{8.365325in}}%
\pgfpathcurveto{\pgfqpoint{4.106476in}{8.357511in}}{\pgfqpoint{4.102086in}{8.346912in}}{\pgfqpoint{4.102086in}{8.335862in}}%
\pgfpathcurveto{\pgfqpoint{4.102086in}{8.324812in}}{\pgfqpoint{4.106476in}{8.314213in}}{\pgfqpoint{4.114290in}{8.306399in}}%
\pgfpathcurveto{\pgfqpoint{4.122104in}{8.298586in}}{\pgfqpoint{4.132703in}{8.294195in}}{\pgfqpoint{4.143753in}{8.294195in}}%
\pgfpathlineto{\pgfqpoint{4.143753in}{8.294195in}}%
\pgfpathclose%
\pgfusepath{stroke,fill}%
\end{pgfscope}%
\begin{pgfscope}%
\pgfpathrectangle{\pgfqpoint{2.963410in}{7.624184in}}{\pgfqpoint{2.177280in}{2.201755in}}%
\pgfusepath{clip}%
\pgfsetbuttcap%
\pgfsetroundjoin%
\definecolor{currentfill}{rgb}{0.121569,0.466667,0.705882}%
\pgfsetfillcolor{currentfill}%
\pgfsetlinewidth{0.481800pt}%
\definecolor{currentstroke}{rgb}{1.000000,1.000000,1.000000}%
\pgfsetstrokecolor{currentstroke}%
\pgfsetdash{}{0pt}%
\pgfpathmoveto{\pgfqpoint{4.557280in}{8.182996in}}%
\pgfpathcurveto{\pgfqpoint{4.568330in}{8.182996in}}{\pgfqpoint{4.578929in}{8.187386in}}{\pgfqpoint{4.586742in}{8.195200in}}%
\pgfpathcurveto{\pgfqpoint{4.594556in}{8.203013in}}{\pgfqpoint{4.598946in}{8.213612in}}{\pgfqpoint{4.598946in}{8.224662in}}%
\pgfpathcurveto{\pgfqpoint{4.598946in}{8.235713in}}{\pgfqpoint{4.594556in}{8.246312in}}{\pgfqpoint{4.586742in}{8.254125in}}%
\pgfpathcurveto{\pgfqpoint{4.578929in}{8.261939in}}{\pgfqpoint{4.568330in}{8.266329in}}{\pgfqpoint{4.557280in}{8.266329in}}%
\pgfpathcurveto{\pgfqpoint{4.546230in}{8.266329in}}{\pgfqpoint{4.535631in}{8.261939in}}{\pgfqpoint{4.527817in}{8.254125in}}%
\pgfpathcurveto{\pgfqpoint{4.520003in}{8.246312in}}{\pgfqpoint{4.515613in}{8.235713in}}{\pgfqpoint{4.515613in}{8.224662in}}%
\pgfpathcurveto{\pgfqpoint{4.515613in}{8.213612in}}{\pgfqpoint{4.520003in}{8.203013in}}{\pgfqpoint{4.527817in}{8.195200in}}%
\pgfpathcurveto{\pgfqpoint{4.535631in}{8.187386in}}{\pgfqpoint{4.546230in}{8.182996in}}{\pgfqpoint{4.557280in}{8.182996in}}%
\pgfpathlineto{\pgfqpoint{4.557280in}{8.182996in}}%
\pgfpathclose%
\pgfusepath{stroke,fill}%
\end{pgfscope}%
\begin{pgfscope}%
\pgfpathrectangle{\pgfqpoint{2.963410in}{7.624184in}}{\pgfqpoint{2.177280in}{2.201755in}}%
\pgfusepath{clip}%
\pgfsetbuttcap%
\pgfsetroundjoin%
\definecolor{currentfill}{rgb}{0.121569,0.466667,0.705882}%
\pgfsetfillcolor{currentfill}%
\pgfsetlinewidth{0.481800pt}%
\definecolor{currentstroke}{rgb}{1.000000,1.000000,1.000000}%
\pgfsetstrokecolor{currentstroke}%
\pgfsetdash{}{0pt}%
\pgfpathmoveto{\pgfqpoint{4.616355in}{8.349795in}}%
\pgfpathcurveto{\pgfqpoint{4.627405in}{8.349795in}}{\pgfqpoint{4.638004in}{8.354186in}}{\pgfqpoint{4.645818in}{8.361999in}}%
\pgfpathcurveto{\pgfqpoint{4.653631in}{8.369813in}}{\pgfqpoint{4.658022in}{8.380412in}}{\pgfqpoint{4.658022in}{8.391462in}}%
\pgfpathcurveto{\pgfqpoint{4.658022in}{8.402512in}}{\pgfqpoint{4.653631in}{8.413111in}}{\pgfqpoint{4.645818in}{8.420925in}}%
\pgfpathcurveto{\pgfqpoint{4.638004in}{8.428738in}}{\pgfqpoint{4.627405in}{8.433129in}}{\pgfqpoint{4.616355in}{8.433129in}}%
\pgfpathcurveto{\pgfqpoint{4.605305in}{8.433129in}}{\pgfqpoint{4.594706in}{8.428738in}}{\pgfqpoint{4.586892in}{8.420925in}}%
\pgfpathcurveto{\pgfqpoint{4.579079in}{8.413111in}}{\pgfqpoint{4.574688in}{8.402512in}}{\pgfqpoint{4.574688in}{8.391462in}}%
\pgfpathcurveto{\pgfqpoint{4.574688in}{8.380412in}}{\pgfqpoint{4.579079in}{8.369813in}}{\pgfqpoint{4.586892in}{8.361999in}}%
\pgfpathcurveto{\pgfqpoint{4.594706in}{8.354186in}}{\pgfqpoint{4.605305in}{8.349795in}}{\pgfqpoint{4.616355in}{8.349795in}}%
\pgfpathlineto{\pgfqpoint{4.616355in}{8.349795in}}%
\pgfpathclose%
\pgfusepath{stroke,fill}%
\end{pgfscope}%
\begin{pgfscope}%
\pgfpathrectangle{\pgfqpoint{2.963410in}{7.624184in}}{\pgfqpoint{2.177280in}{2.201755in}}%
\pgfusepath{clip}%
\pgfsetbuttcap%
\pgfsetroundjoin%
\definecolor{currentfill}{rgb}{0.121569,0.466667,0.705882}%
\pgfsetfillcolor{currentfill}%
\pgfsetlinewidth{0.481800pt}%
\definecolor{currentstroke}{rgb}{1.000000,1.000000,1.000000}%
\pgfsetstrokecolor{currentstroke}%
\pgfsetdash{}{0pt}%
\pgfpathmoveto{\pgfqpoint{3.966527in}{8.016196in}}%
\pgfpathcurveto{\pgfqpoint{3.977577in}{8.016196in}}{\pgfqpoint{3.988176in}{8.020586in}}{\pgfqpoint{3.995990in}{8.028400in}}%
\pgfpathcurveto{\pgfqpoint{4.003803in}{8.036214in}}{\pgfqpoint{4.008194in}{8.046813in}}{\pgfqpoint{4.008194in}{8.057863in}}%
\pgfpathcurveto{\pgfqpoint{4.008194in}{8.068913in}}{\pgfqpoint{4.003803in}{8.079512in}}{\pgfqpoint{3.995990in}{8.087326in}}%
\pgfpathcurveto{\pgfqpoint{3.988176in}{8.095139in}}{\pgfqpoint{3.977577in}{8.099529in}}{\pgfqpoint{3.966527in}{8.099529in}}%
\pgfpathcurveto{\pgfqpoint{3.955477in}{8.099529in}}{\pgfqpoint{3.944878in}{8.095139in}}{\pgfqpoint{3.937064in}{8.087326in}}%
\pgfpathcurveto{\pgfqpoint{3.929251in}{8.079512in}}{\pgfqpoint{3.924860in}{8.068913in}}{\pgfqpoint{3.924860in}{8.057863in}}%
\pgfpathcurveto{\pgfqpoint{3.924860in}{8.046813in}}{\pgfqpoint{3.929251in}{8.036214in}}{\pgfqpoint{3.937064in}{8.028400in}}%
\pgfpathcurveto{\pgfqpoint{3.944878in}{8.020586in}}{\pgfqpoint{3.955477in}{8.016196in}}{\pgfqpoint{3.966527in}{8.016196in}}%
\pgfpathlineto{\pgfqpoint{3.966527in}{8.016196in}}%
\pgfpathclose%
\pgfusepath{stroke,fill}%
\end{pgfscope}%
\begin{pgfscope}%
\pgfpathrectangle{\pgfqpoint{2.963410in}{7.624184in}}{\pgfqpoint{2.177280in}{2.201755in}}%
\pgfusepath{clip}%
\pgfsetbuttcap%
\pgfsetroundjoin%
\definecolor{currentfill}{rgb}{0.121569,0.466667,0.705882}%
\pgfsetfillcolor{currentfill}%
\pgfsetlinewidth{0.481800pt}%
\definecolor{currentstroke}{rgb}{1.000000,1.000000,1.000000}%
\pgfsetstrokecolor{currentstroke}%
\pgfsetdash{}{0pt}%
\pgfpathmoveto{\pgfqpoint{4.025602in}{8.071796in}}%
\pgfpathcurveto{\pgfqpoint{4.036652in}{8.071796in}}{\pgfqpoint{4.047251in}{8.076186in}}{\pgfqpoint{4.055065in}{8.084000in}}%
\pgfpathcurveto{\pgfqpoint{4.062879in}{8.091813in}}{\pgfqpoint{4.067269in}{8.102413in}}{\pgfqpoint{4.067269in}{8.113463in}}%
\pgfpathcurveto{\pgfqpoint{4.067269in}{8.124513in}}{\pgfqpoint{4.062879in}{8.135112in}}{\pgfqpoint{4.055065in}{8.142925in}}%
\pgfpathcurveto{\pgfqpoint{4.047251in}{8.150739in}}{\pgfqpoint{4.036652in}{8.155129in}}{\pgfqpoint{4.025602in}{8.155129in}}%
\pgfpathcurveto{\pgfqpoint{4.014552in}{8.155129in}}{\pgfqpoint{4.003953in}{8.150739in}}{\pgfqpoint{3.996139in}{8.142925in}}%
\pgfpathcurveto{\pgfqpoint{3.988326in}{8.135112in}}{\pgfqpoint{3.983936in}{8.124513in}}{\pgfqpoint{3.983936in}{8.113463in}}%
\pgfpathcurveto{\pgfqpoint{3.983936in}{8.102413in}}{\pgfqpoint{3.988326in}{8.091813in}}{\pgfqpoint{3.996139in}{8.084000in}}%
\pgfpathcurveto{\pgfqpoint{4.003953in}{8.076186in}}{\pgfqpoint{4.014552in}{8.071796in}}{\pgfqpoint{4.025602in}{8.071796in}}%
\pgfpathlineto{\pgfqpoint{4.025602in}{8.071796in}}%
\pgfpathclose%
\pgfusepath{stroke,fill}%
\end{pgfscope}%
\begin{pgfscope}%
\pgfpathrectangle{\pgfqpoint{2.963410in}{7.624184in}}{\pgfqpoint{2.177280in}{2.201755in}}%
\pgfusepath{clip}%
\pgfsetbuttcap%
\pgfsetroundjoin%
\definecolor{currentfill}{rgb}{0.121569,0.466667,0.705882}%
\pgfsetfillcolor{currentfill}%
\pgfsetlinewidth{0.481800pt}%
\definecolor{currentstroke}{rgb}{1.000000,1.000000,1.000000}%
\pgfsetstrokecolor{currentstroke}%
\pgfsetdash{}{0pt}%
\pgfpathmoveto{\pgfqpoint{4.202828in}{8.349795in}}%
\pgfpathcurveto{\pgfqpoint{4.213878in}{8.349795in}}{\pgfqpoint{4.224477in}{8.354186in}}{\pgfqpoint{4.232291in}{8.361999in}}%
\pgfpathcurveto{\pgfqpoint{4.240104in}{8.369813in}}{\pgfqpoint{4.244495in}{8.380412in}}{\pgfqpoint{4.244495in}{8.391462in}}%
\pgfpathcurveto{\pgfqpoint{4.244495in}{8.402512in}}{\pgfqpoint{4.240104in}{8.413111in}}{\pgfqpoint{4.232291in}{8.420925in}}%
\pgfpathcurveto{\pgfqpoint{4.224477in}{8.428738in}}{\pgfqpoint{4.213878in}{8.433129in}}{\pgfqpoint{4.202828in}{8.433129in}}%
\pgfpathcurveto{\pgfqpoint{4.191778in}{8.433129in}}{\pgfqpoint{4.181179in}{8.428738in}}{\pgfqpoint{4.173365in}{8.420925in}}%
\pgfpathcurveto{\pgfqpoint{4.165552in}{8.413111in}}{\pgfqpoint{4.161161in}{8.402512in}}{\pgfqpoint{4.161161in}{8.391462in}}%
\pgfpathcurveto{\pgfqpoint{4.161161in}{8.380412in}}{\pgfqpoint{4.165552in}{8.369813in}}{\pgfqpoint{4.173365in}{8.361999in}}%
\pgfpathcurveto{\pgfqpoint{4.181179in}{8.354186in}}{\pgfqpoint{4.191778in}{8.349795in}}{\pgfqpoint{4.202828in}{8.349795in}}%
\pgfpathlineto{\pgfqpoint{4.202828in}{8.349795in}}%
\pgfpathclose%
\pgfusepath{stroke,fill}%
\end{pgfscope}%
\begin{pgfscope}%
\pgfpathrectangle{\pgfqpoint{2.963410in}{7.624184in}}{\pgfqpoint{2.177280in}{2.201755in}}%
\pgfusepath{clip}%
\pgfsetbuttcap%
\pgfsetroundjoin%
\definecolor{currentfill}{rgb}{0.121569,0.466667,0.705882}%
\pgfsetfillcolor{currentfill}%
\pgfsetlinewidth{0.481800pt}%
\definecolor{currentstroke}{rgb}{1.000000,1.000000,1.000000}%
\pgfsetstrokecolor{currentstroke}%
\pgfsetdash{}{0pt}%
\pgfpathmoveto{\pgfqpoint{4.261903in}{8.016196in}}%
\pgfpathcurveto{\pgfqpoint{4.272953in}{8.016196in}}{\pgfqpoint{4.283552in}{8.020586in}}{\pgfqpoint{4.291366in}{8.028400in}}%
\pgfpathcurveto{\pgfqpoint{4.299180in}{8.036214in}}{\pgfqpoint{4.303570in}{8.046813in}}{\pgfqpoint{4.303570in}{8.057863in}}%
\pgfpathcurveto{\pgfqpoint{4.303570in}{8.068913in}}{\pgfqpoint{4.299180in}{8.079512in}}{\pgfqpoint{4.291366in}{8.087326in}}%
\pgfpathcurveto{\pgfqpoint{4.283552in}{8.095139in}}{\pgfqpoint{4.272953in}{8.099529in}}{\pgfqpoint{4.261903in}{8.099529in}}%
\pgfpathcurveto{\pgfqpoint{4.250853in}{8.099529in}}{\pgfqpoint{4.240254in}{8.095139in}}{\pgfqpoint{4.232441in}{8.087326in}}%
\pgfpathcurveto{\pgfqpoint{4.224627in}{8.079512in}}{\pgfqpoint{4.220237in}{8.068913in}}{\pgfqpoint{4.220237in}{8.057863in}}%
\pgfpathcurveto{\pgfqpoint{4.220237in}{8.046813in}}{\pgfqpoint{4.224627in}{8.036214in}}{\pgfqpoint{4.232441in}{8.028400in}}%
\pgfpathcurveto{\pgfqpoint{4.240254in}{8.020586in}}{\pgfqpoint{4.250853in}{8.016196in}}{\pgfqpoint{4.261903in}{8.016196in}}%
\pgfpathlineto{\pgfqpoint{4.261903in}{8.016196in}}%
\pgfpathclose%
\pgfusepath{stroke,fill}%
\end{pgfscope}%
\begin{pgfscope}%
\pgfpathrectangle{\pgfqpoint{2.963410in}{7.624184in}}{\pgfqpoint{2.177280in}{2.201755in}}%
\pgfusepath{clip}%
\pgfsetbuttcap%
\pgfsetroundjoin%
\definecolor{currentfill}{rgb}{0.121569,0.466667,0.705882}%
\pgfsetfillcolor{currentfill}%
\pgfsetlinewidth{0.481800pt}%
\definecolor{currentstroke}{rgb}{1.000000,1.000000,1.000000}%
\pgfsetstrokecolor{currentstroke}%
\pgfsetdash{}{0pt}%
\pgfpathmoveto{\pgfqpoint{3.907452in}{7.738197in}}%
\pgfpathcurveto{\pgfqpoint{3.918502in}{7.738197in}}{\pgfqpoint{3.929101in}{7.742587in}}{\pgfqpoint{3.936915in}{7.750401in}}%
\pgfpathcurveto{\pgfqpoint{3.944728in}{7.758214in}}{\pgfqpoint{3.949118in}{7.768813in}}{\pgfqpoint{3.949118in}{7.779863in}}%
\pgfpathcurveto{\pgfqpoint{3.949118in}{7.790914in}}{\pgfqpoint{3.944728in}{7.801513in}}{\pgfqpoint{3.936915in}{7.809326in}}%
\pgfpathcurveto{\pgfqpoint{3.929101in}{7.817140in}}{\pgfqpoint{3.918502in}{7.821530in}}{\pgfqpoint{3.907452in}{7.821530in}}%
\pgfpathcurveto{\pgfqpoint{3.896402in}{7.821530in}}{\pgfqpoint{3.885803in}{7.817140in}}{\pgfqpoint{3.877989in}{7.809326in}}%
\pgfpathcurveto{\pgfqpoint{3.870175in}{7.801513in}}{\pgfqpoint{3.865785in}{7.790914in}}{\pgfqpoint{3.865785in}{7.779863in}}%
\pgfpathcurveto{\pgfqpoint{3.865785in}{7.768813in}}{\pgfqpoint{3.870175in}{7.758214in}}{\pgfqpoint{3.877989in}{7.750401in}}%
\pgfpathcurveto{\pgfqpoint{3.885803in}{7.742587in}}{\pgfqpoint{3.896402in}{7.738197in}}{\pgfqpoint{3.907452in}{7.738197in}}%
\pgfpathlineto{\pgfqpoint{3.907452in}{7.738197in}}%
\pgfpathclose%
\pgfusepath{stroke,fill}%
\end{pgfscope}%
\begin{pgfscope}%
\pgfpathrectangle{\pgfqpoint{2.963410in}{7.624184in}}{\pgfqpoint{2.177280in}{2.201755in}}%
\pgfusepath{clip}%
\pgfsetbuttcap%
\pgfsetroundjoin%
\definecolor{currentfill}{rgb}{0.121569,0.466667,0.705882}%
\pgfsetfillcolor{currentfill}%
\pgfsetlinewidth{0.481800pt}%
\definecolor{currentstroke}{rgb}{1.000000,1.000000,1.000000}%
\pgfsetstrokecolor{currentstroke}%
\pgfsetdash{}{0pt}%
\pgfpathmoveto{\pgfqpoint{4.143753in}{8.127396in}}%
\pgfpathcurveto{\pgfqpoint{4.154803in}{8.127396in}}{\pgfqpoint{4.165402in}{8.131786in}}{\pgfqpoint{4.173216in}{8.139600in}}%
\pgfpathcurveto{\pgfqpoint{4.181029in}{8.147413in}}{\pgfqpoint{4.185419in}{8.158012in}}{\pgfqpoint{4.185419in}{8.169063in}}%
\pgfpathcurveto{\pgfqpoint{4.185419in}{8.180113in}}{\pgfqpoint{4.181029in}{8.190712in}}{\pgfqpoint{4.173216in}{8.198525in}}%
\pgfpathcurveto{\pgfqpoint{4.165402in}{8.206339in}}{\pgfqpoint{4.154803in}{8.210729in}}{\pgfqpoint{4.143753in}{8.210729in}}%
\pgfpathcurveto{\pgfqpoint{4.132703in}{8.210729in}}{\pgfqpoint{4.122104in}{8.206339in}}{\pgfqpoint{4.114290in}{8.198525in}}%
\pgfpathcurveto{\pgfqpoint{4.106476in}{8.190712in}}{\pgfqpoint{4.102086in}{8.180113in}}{\pgfqpoint{4.102086in}{8.169063in}}%
\pgfpathcurveto{\pgfqpoint{4.102086in}{8.158012in}}{\pgfqpoint{4.106476in}{8.147413in}}{\pgfqpoint{4.114290in}{8.139600in}}%
\pgfpathcurveto{\pgfqpoint{4.122104in}{8.131786in}}{\pgfqpoint{4.132703in}{8.127396in}}{\pgfqpoint{4.143753in}{8.127396in}}%
\pgfpathlineto{\pgfqpoint{4.143753in}{8.127396in}}%
\pgfpathclose%
\pgfusepath{stroke,fill}%
\end{pgfscope}%
\begin{pgfscope}%
\pgfpathrectangle{\pgfqpoint{2.963410in}{7.624184in}}{\pgfqpoint{2.177280in}{2.201755in}}%
\pgfusepath{clip}%
\pgfsetbuttcap%
\pgfsetroundjoin%
\definecolor{currentfill}{rgb}{0.121569,0.466667,0.705882}%
\pgfsetfillcolor{currentfill}%
\pgfsetlinewidth{0.481800pt}%
\definecolor{currentstroke}{rgb}{1.000000,1.000000,1.000000}%
\pgfsetstrokecolor{currentstroke}%
\pgfsetdash{}{0pt}%
\pgfpathmoveto{\pgfqpoint{4.202828in}{8.071796in}}%
\pgfpathcurveto{\pgfqpoint{4.213878in}{8.071796in}}{\pgfqpoint{4.224477in}{8.076186in}}{\pgfqpoint{4.232291in}{8.084000in}}%
\pgfpathcurveto{\pgfqpoint{4.240104in}{8.091813in}}{\pgfqpoint{4.244495in}{8.102413in}}{\pgfqpoint{4.244495in}{8.113463in}}%
\pgfpathcurveto{\pgfqpoint{4.244495in}{8.124513in}}{\pgfqpoint{4.240104in}{8.135112in}}{\pgfqpoint{4.232291in}{8.142925in}}%
\pgfpathcurveto{\pgfqpoint{4.224477in}{8.150739in}}{\pgfqpoint{4.213878in}{8.155129in}}{\pgfqpoint{4.202828in}{8.155129in}}%
\pgfpathcurveto{\pgfqpoint{4.191778in}{8.155129in}}{\pgfqpoint{4.181179in}{8.150739in}}{\pgfqpoint{4.173365in}{8.142925in}}%
\pgfpathcurveto{\pgfqpoint{4.165552in}{8.135112in}}{\pgfqpoint{4.161161in}{8.124513in}}{\pgfqpoint{4.161161in}{8.113463in}}%
\pgfpathcurveto{\pgfqpoint{4.161161in}{8.102413in}}{\pgfqpoint{4.165552in}{8.091813in}}{\pgfqpoint{4.173365in}{8.084000in}}%
\pgfpathcurveto{\pgfqpoint{4.181179in}{8.076186in}}{\pgfqpoint{4.191778in}{8.071796in}}{\pgfqpoint{4.202828in}{8.071796in}}%
\pgfpathlineto{\pgfqpoint{4.202828in}{8.071796in}}%
\pgfpathclose%
\pgfusepath{stroke,fill}%
\end{pgfscope}%
\begin{pgfscope}%
\pgfpathrectangle{\pgfqpoint{2.963410in}{7.624184in}}{\pgfqpoint{2.177280in}{2.201755in}}%
\pgfusepath{clip}%
\pgfsetbuttcap%
\pgfsetroundjoin%
\definecolor{currentfill}{rgb}{0.121569,0.466667,0.705882}%
\pgfsetfillcolor{currentfill}%
\pgfsetlinewidth{0.481800pt}%
\definecolor{currentstroke}{rgb}{1.000000,1.000000,1.000000}%
\pgfsetstrokecolor{currentstroke}%
\pgfsetdash{}{0pt}%
\pgfpathmoveto{\pgfqpoint{3.493925in}{7.793797in}}%
\pgfpathcurveto{\pgfqpoint{3.504975in}{7.793797in}}{\pgfqpoint{3.515574in}{7.798187in}}{\pgfqpoint{3.523388in}{7.806000in}}%
\pgfpathcurveto{\pgfqpoint{3.531201in}{7.813814in}}{\pgfqpoint{3.535592in}{7.824413in}}{\pgfqpoint{3.535592in}{7.835463in}}%
\pgfpathcurveto{\pgfqpoint{3.535592in}{7.846513in}}{\pgfqpoint{3.531201in}{7.857112in}}{\pgfqpoint{3.523388in}{7.864926in}}%
\pgfpathcurveto{\pgfqpoint{3.515574in}{7.872740in}}{\pgfqpoint{3.504975in}{7.877130in}}{\pgfqpoint{3.493925in}{7.877130in}}%
\pgfpathcurveto{\pgfqpoint{3.482875in}{7.877130in}}{\pgfqpoint{3.472276in}{7.872740in}}{\pgfqpoint{3.464462in}{7.864926in}}%
\pgfpathcurveto{\pgfqpoint{3.456648in}{7.857112in}}{\pgfqpoint{3.452258in}{7.846513in}}{\pgfqpoint{3.452258in}{7.835463in}}%
\pgfpathcurveto{\pgfqpoint{3.452258in}{7.824413in}}{\pgfqpoint{3.456648in}{7.813814in}}{\pgfqpoint{3.464462in}{7.806000in}}%
\pgfpathcurveto{\pgfqpoint{3.472276in}{7.798187in}}{\pgfqpoint{3.482875in}{7.793797in}}{\pgfqpoint{3.493925in}{7.793797in}}%
\pgfpathlineto{\pgfqpoint{3.493925in}{7.793797in}}%
\pgfpathclose%
\pgfusepath{stroke,fill}%
\end{pgfscope}%
\begin{pgfscope}%
\pgfpathrectangle{\pgfqpoint{2.963410in}{7.624184in}}{\pgfqpoint{2.177280in}{2.201755in}}%
\pgfusepath{clip}%
\pgfsetbuttcap%
\pgfsetroundjoin%
\definecolor{currentfill}{rgb}{0.121569,0.466667,0.705882}%
\pgfsetfillcolor{currentfill}%
\pgfsetlinewidth{0.481800pt}%
\definecolor{currentstroke}{rgb}{1.000000,1.000000,1.000000}%
\pgfsetstrokecolor{currentstroke}%
\pgfsetdash{}{0pt}%
\pgfpathmoveto{\pgfqpoint{4.025602in}{7.738197in}}%
\pgfpathcurveto{\pgfqpoint{4.036652in}{7.738197in}}{\pgfqpoint{4.047251in}{7.742587in}}{\pgfqpoint{4.055065in}{7.750401in}}%
\pgfpathcurveto{\pgfqpoint{4.062879in}{7.758214in}}{\pgfqpoint{4.067269in}{7.768813in}}{\pgfqpoint{4.067269in}{7.779863in}}%
\pgfpathcurveto{\pgfqpoint{4.067269in}{7.790914in}}{\pgfqpoint{4.062879in}{7.801513in}}{\pgfqpoint{4.055065in}{7.809326in}}%
\pgfpathcurveto{\pgfqpoint{4.047251in}{7.817140in}}{\pgfqpoint{4.036652in}{7.821530in}}{\pgfqpoint{4.025602in}{7.821530in}}%
\pgfpathcurveto{\pgfqpoint{4.014552in}{7.821530in}}{\pgfqpoint{4.003953in}{7.817140in}}{\pgfqpoint{3.996139in}{7.809326in}}%
\pgfpathcurveto{\pgfqpoint{3.988326in}{7.801513in}}{\pgfqpoint{3.983936in}{7.790914in}}{\pgfqpoint{3.983936in}{7.779863in}}%
\pgfpathcurveto{\pgfqpoint{3.983936in}{7.768813in}}{\pgfqpoint{3.988326in}{7.758214in}}{\pgfqpoint{3.996139in}{7.750401in}}%
\pgfpathcurveto{\pgfqpoint{4.003953in}{7.742587in}}{\pgfqpoint{4.014552in}{7.738197in}}{\pgfqpoint{4.025602in}{7.738197in}}%
\pgfpathlineto{\pgfqpoint{4.025602in}{7.738197in}}%
\pgfpathclose%
\pgfusepath{stroke,fill}%
\end{pgfscope}%
\begin{pgfscope}%
\pgfpathrectangle{\pgfqpoint{2.963410in}{7.624184in}}{\pgfqpoint{2.177280in}{2.201755in}}%
\pgfusepath{clip}%
\pgfsetbuttcap%
\pgfsetroundjoin%
\definecolor{currentfill}{rgb}{0.121569,0.466667,0.705882}%
\pgfsetfillcolor{currentfill}%
\pgfsetlinewidth{0.481800pt}%
\definecolor{currentstroke}{rgb}{1.000000,1.000000,1.000000}%
\pgfsetstrokecolor{currentstroke}%
\pgfsetdash{}{0pt}%
\pgfpathmoveto{\pgfqpoint{4.202828in}{8.071796in}}%
\pgfpathcurveto{\pgfqpoint{4.213878in}{8.071796in}}{\pgfqpoint{4.224477in}{8.076186in}}{\pgfqpoint{4.232291in}{8.084000in}}%
\pgfpathcurveto{\pgfqpoint{4.240104in}{8.091813in}}{\pgfqpoint{4.244495in}{8.102413in}}{\pgfqpoint{4.244495in}{8.113463in}}%
\pgfpathcurveto{\pgfqpoint{4.244495in}{8.124513in}}{\pgfqpoint{4.240104in}{8.135112in}}{\pgfqpoint{4.232291in}{8.142925in}}%
\pgfpathcurveto{\pgfqpoint{4.224477in}{8.150739in}}{\pgfqpoint{4.213878in}{8.155129in}}{\pgfqpoint{4.202828in}{8.155129in}}%
\pgfpathcurveto{\pgfqpoint{4.191778in}{8.155129in}}{\pgfqpoint{4.181179in}{8.150739in}}{\pgfqpoint{4.173365in}{8.142925in}}%
\pgfpathcurveto{\pgfqpoint{4.165552in}{8.135112in}}{\pgfqpoint{4.161161in}{8.124513in}}{\pgfqpoint{4.161161in}{8.113463in}}%
\pgfpathcurveto{\pgfqpoint{4.161161in}{8.102413in}}{\pgfqpoint{4.165552in}{8.091813in}}{\pgfqpoint{4.173365in}{8.084000in}}%
\pgfpathcurveto{\pgfqpoint{4.181179in}{8.076186in}}{\pgfqpoint{4.191778in}{8.071796in}}{\pgfqpoint{4.202828in}{8.071796in}}%
\pgfpathlineto{\pgfqpoint{4.202828in}{8.071796in}}%
\pgfpathclose%
\pgfusepath{stroke,fill}%
\end{pgfscope}%
\begin{pgfscope}%
\pgfpathrectangle{\pgfqpoint{2.963410in}{7.624184in}}{\pgfqpoint{2.177280in}{2.201755in}}%
\pgfusepath{clip}%
\pgfsetbuttcap%
\pgfsetroundjoin%
\definecolor{currentfill}{rgb}{0.121569,0.466667,0.705882}%
\pgfsetfillcolor{currentfill}%
\pgfsetlinewidth{0.481800pt}%
\definecolor{currentstroke}{rgb}{1.000000,1.000000,1.000000}%
\pgfsetstrokecolor{currentstroke}%
\pgfsetdash{}{0pt}%
\pgfpathmoveto{\pgfqpoint{4.380054in}{8.127396in}}%
\pgfpathcurveto{\pgfqpoint{4.391104in}{8.127396in}}{\pgfqpoint{4.401703in}{8.131786in}}{\pgfqpoint{4.409517in}{8.139600in}}%
\pgfpathcurveto{\pgfqpoint{4.417330in}{8.147413in}}{\pgfqpoint{4.421721in}{8.158012in}}{\pgfqpoint{4.421721in}{8.169063in}}%
\pgfpathcurveto{\pgfqpoint{4.421721in}{8.180113in}}{\pgfqpoint{4.417330in}{8.190712in}}{\pgfqpoint{4.409517in}{8.198525in}}%
\pgfpathcurveto{\pgfqpoint{4.401703in}{8.206339in}}{\pgfqpoint{4.391104in}{8.210729in}}{\pgfqpoint{4.380054in}{8.210729in}}%
\pgfpathcurveto{\pgfqpoint{4.369004in}{8.210729in}}{\pgfqpoint{4.358405in}{8.206339in}}{\pgfqpoint{4.350591in}{8.198525in}}%
\pgfpathcurveto{\pgfqpoint{4.342777in}{8.190712in}}{\pgfqpoint{4.338387in}{8.180113in}}{\pgfqpoint{4.338387in}{8.169063in}}%
\pgfpathcurveto{\pgfqpoint{4.338387in}{8.158012in}}{\pgfqpoint{4.342777in}{8.147413in}}{\pgfqpoint{4.350591in}{8.139600in}}%
\pgfpathcurveto{\pgfqpoint{4.358405in}{8.131786in}}{\pgfqpoint{4.369004in}{8.127396in}}{\pgfqpoint{4.380054in}{8.127396in}}%
\pgfpathlineto{\pgfqpoint{4.380054in}{8.127396in}}%
\pgfpathclose%
\pgfusepath{stroke,fill}%
\end{pgfscope}%
\begin{pgfscope}%
\pgfpathrectangle{\pgfqpoint{2.963410in}{7.624184in}}{\pgfqpoint{2.177280in}{2.201755in}}%
\pgfusepath{clip}%
\pgfsetbuttcap%
\pgfsetroundjoin%
\definecolor{currentfill}{rgb}{0.121569,0.466667,0.705882}%
\pgfsetfillcolor{currentfill}%
\pgfsetlinewidth{0.481800pt}%
\definecolor{currentstroke}{rgb}{1.000000,1.000000,1.000000}%
\pgfsetstrokecolor{currentstroke}%
\pgfsetdash{}{0pt}%
\pgfpathmoveto{\pgfqpoint{3.907452in}{7.960596in}}%
\pgfpathcurveto{\pgfqpoint{3.918502in}{7.960596in}}{\pgfqpoint{3.929101in}{7.964986in}}{\pgfqpoint{3.936915in}{7.972800in}}%
\pgfpathcurveto{\pgfqpoint{3.944728in}{7.980614in}}{\pgfqpoint{3.949118in}{7.991213in}}{\pgfqpoint{3.949118in}{8.002263in}}%
\pgfpathcurveto{\pgfqpoint{3.949118in}{8.013313in}}{\pgfqpoint{3.944728in}{8.023912in}}{\pgfqpoint{3.936915in}{8.031726in}}%
\pgfpathcurveto{\pgfqpoint{3.929101in}{8.039539in}}{\pgfqpoint{3.918502in}{8.043930in}}{\pgfqpoint{3.907452in}{8.043930in}}%
\pgfpathcurveto{\pgfqpoint{3.896402in}{8.043930in}}{\pgfqpoint{3.885803in}{8.039539in}}{\pgfqpoint{3.877989in}{8.031726in}}%
\pgfpathcurveto{\pgfqpoint{3.870175in}{8.023912in}}{\pgfqpoint{3.865785in}{8.013313in}}{\pgfqpoint{3.865785in}{8.002263in}}%
\pgfpathcurveto{\pgfqpoint{3.865785in}{7.991213in}}{\pgfqpoint{3.870175in}{7.980614in}}{\pgfqpoint{3.877989in}{7.972800in}}%
\pgfpathcurveto{\pgfqpoint{3.885803in}{7.964986in}}{\pgfqpoint{3.896402in}{7.960596in}}{\pgfqpoint{3.907452in}{7.960596in}}%
\pgfpathlineto{\pgfqpoint{3.907452in}{7.960596in}}%
\pgfpathclose%
\pgfusepath{stroke,fill}%
\end{pgfscope}%
\begin{pgfscope}%
\pgfpathrectangle{\pgfqpoint{2.963410in}{7.624184in}}{\pgfqpoint{2.177280in}{2.201755in}}%
\pgfusepath{clip}%
\pgfsetbuttcap%
\pgfsetroundjoin%
\definecolor{currentfill}{rgb}{0.121569,0.466667,0.705882}%
\pgfsetfillcolor{currentfill}%
\pgfsetlinewidth{0.481800pt}%
\definecolor{currentstroke}{rgb}{1.000000,1.000000,1.000000}%
\pgfsetstrokecolor{currentstroke}%
\pgfsetdash{}{0pt}%
\pgfpathmoveto{\pgfqpoint{4.380054in}{8.127396in}}%
\pgfpathcurveto{\pgfqpoint{4.391104in}{8.127396in}}{\pgfqpoint{4.401703in}{8.131786in}}{\pgfqpoint{4.409517in}{8.139600in}}%
\pgfpathcurveto{\pgfqpoint{4.417330in}{8.147413in}}{\pgfqpoint{4.421721in}{8.158012in}}{\pgfqpoint{4.421721in}{8.169063in}}%
\pgfpathcurveto{\pgfqpoint{4.421721in}{8.180113in}}{\pgfqpoint{4.417330in}{8.190712in}}{\pgfqpoint{4.409517in}{8.198525in}}%
\pgfpathcurveto{\pgfqpoint{4.401703in}{8.206339in}}{\pgfqpoint{4.391104in}{8.210729in}}{\pgfqpoint{4.380054in}{8.210729in}}%
\pgfpathcurveto{\pgfqpoint{4.369004in}{8.210729in}}{\pgfqpoint{4.358405in}{8.206339in}}{\pgfqpoint{4.350591in}{8.198525in}}%
\pgfpathcurveto{\pgfqpoint{4.342777in}{8.190712in}}{\pgfqpoint{4.338387in}{8.180113in}}{\pgfqpoint{4.338387in}{8.169063in}}%
\pgfpathcurveto{\pgfqpoint{4.338387in}{8.158012in}}{\pgfqpoint{4.342777in}{8.147413in}}{\pgfqpoint{4.350591in}{8.139600in}}%
\pgfpathcurveto{\pgfqpoint{4.358405in}{8.131786in}}{\pgfqpoint{4.369004in}{8.127396in}}{\pgfqpoint{4.380054in}{8.127396in}}%
\pgfpathlineto{\pgfqpoint{4.380054in}{8.127396in}}%
\pgfpathclose%
\pgfusepath{stroke,fill}%
\end{pgfscope}%
\begin{pgfscope}%
\pgfpathrectangle{\pgfqpoint{2.963410in}{7.624184in}}{\pgfqpoint{2.177280in}{2.201755in}}%
\pgfusepath{clip}%
\pgfsetbuttcap%
\pgfsetroundjoin%
\definecolor{currentfill}{rgb}{0.121569,0.466667,0.705882}%
\pgfsetfillcolor{currentfill}%
\pgfsetlinewidth{0.481800pt}%
\definecolor{currentstroke}{rgb}{1.000000,1.000000,1.000000}%
\pgfsetstrokecolor{currentstroke}%
\pgfsetdash{}{0pt}%
\pgfpathmoveto{\pgfqpoint{4.025602in}{7.849396in}}%
\pgfpathcurveto{\pgfqpoint{4.036652in}{7.849396in}}{\pgfqpoint{4.047251in}{7.853787in}}{\pgfqpoint{4.055065in}{7.861600in}}%
\pgfpathcurveto{\pgfqpoint{4.062879in}{7.869414in}}{\pgfqpoint{4.067269in}{7.880013in}}{\pgfqpoint{4.067269in}{7.891063in}}%
\pgfpathcurveto{\pgfqpoint{4.067269in}{7.902113in}}{\pgfqpoint{4.062879in}{7.912712in}}{\pgfqpoint{4.055065in}{7.920526in}}%
\pgfpathcurveto{\pgfqpoint{4.047251in}{7.928340in}}{\pgfqpoint{4.036652in}{7.932730in}}{\pgfqpoint{4.025602in}{7.932730in}}%
\pgfpathcurveto{\pgfqpoint{4.014552in}{7.932730in}}{\pgfqpoint{4.003953in}{7.928340in}}{\pgfqpoint{3.996139in}{7.920526in}}%
\pgfpathcurveto{\pgfqpoint{3.988326in}{7.912712in}}{\pgfqpoint{3.983936in}{7.902113in}}{\pgfqpoint{3.983936in}{7.891063in}}%
\pgfpathcurveto{\pgfqpoint{3.983936in}{7.880013in}}{\pgfqpoint{3.988326in}{7.869414in}}{\pgfqpoint{3.996139in}{7.861600in}}%
\pgfpathcurveto{\pgfqpoint{4.003953in}{7.853787in}}{\pgfqpoint{4.014552in}{7.849396in}}{\pgfqpoint{4.025602in}{7.849396in}}%
\pgfpathlineto{\pgfqpoint{4.025602in}{7.849396in}}%
\pgfpathclose%
\pgfusepath{stroke,fill}%
\end{pgfscope}%
\begin{pgfscope}%
\pgfpathrectangle{\pgfqpoint{2.963410in}{7.624184in}}{\pgfqpoint{2.177280in}{2.201755in}}%
\pgfusepath{clip}%
\pgfsetbuttcap%
\pgfsetroundjoin%
\definecolor{currentfill}{rgb}{0.121569,0.466667,0.705882}%
\pgfsetfillcolor{currentfill}%
\pgfsetlinewidth{0.481800pt}%
\definecolor{currentstroke}{rgb}{1.000000,1.000000,1.000000}%
\pgfsetstrokecolor{currentstroke}%
\pgfsetdash{}{0pt}%
\pgfpathmoveto{\pgfqpoint{4.320979in}{8.238596in}}%
\pgfpathcurveto{\pgfqpoint{4.332029in}{8.238596in}}{\pgfqpoint{4.342628in}{8.242986in}}{\pgfqpoint{4.350441in}{8.250799in}}%
\pgfpathcurveto{\pgfqpoint{4.358255in}{8.258613in}}{\pgfqpoint{4.362645in}{8.269212in}}{\pgfqpoint{4.362645in}{8.280262in}}%
\pgfpathcurveto{\pgfqpoint{4.362645in}{8.291312in}}{\pgfqpoint{4.358255in}{8.301911in}}{\pgfqpoint{4.350441in}{8.309725in}}%
\pgfpathcurveto{\pgfqpoint{4.342628in}{8.317539in}}{\pgfqpoint{4.332029in}{8.321929in}}{\pgfqpoint{4.320979in}{8.321929in}}%
\pgfpathcurveto{\pgfqpoint{4.309928in}{8.321929in}}{\pgfqpoint{4.299329in}{8.317539in}}{\pgfqpoint{4.291516in}{8.309725in}}%
\pgfpathcurveto{\pgfqpoint{4.283702in}{8.301911in}}{\pgfqpoint{4.279312in}{8.291312in}}{\pgfqpoint{4.279312in}{8.280262in}}%
\pgfpathcurveto{\pgfqpoint{4.279312in}{8.269212in}}{\pgfqpoint{4.283702in}{8.258613in}}{\pgfqpoint{4.291516in}{8.250799in}}%
\pgfpathcurveto{\pgfqpoint{4.299329in}{8.242986in}}{\pgfqpoint{4.309928in}{8.238596in}}{\pgfqpoint{4.320979in}{8.238596in}}%
\pgfpathlineto{\pgfqpoint{4.320979in}{8.238596in}}%
\pgfpathclose%
\pgfusepath{stroke,fill}%
\end{pgfscope}%
\begin{pgfscope}%
\pgfpathrectangle{\pgfqpoint{2.963410in}{7.624184in}}{\pgfqpoint{2.177280in}{2.201755in}}%
\pgfusepath{clip}%
\pgfsetbuttcap%
\pgfsetroundjoin%
\definecolor{currentfill}{rgb}{0.121569,0.466667,0.705882}%
\pgfsetfillcolor{currentfill}%
\pgfsetlinewidth{0.481800pt}%
\definecolor{currentstroke}{rgb}{1.000000,1.000000,1.000000}%
\pgfsetstrokecolor{currentstroke}%
\pgfsetdash{}{0pt}%
\pgfpathmoveto{\pgfqpoint{4.084678in}{8.071796in}}%
\pgfpathcurveto{\pgfqpoint{4.095728in}{8.071796in}}{\pgfqpoint{4.106327in}{8.076186in}}{\pgfqpoint{4.114140in}{8.084000in}}%
\pgfpathcurveto{\pgfqpoint{4.121954in}{8.091813in}}{\pgfqpoint{4.126344in}{8.102413in}}{\pgfqpoint{4.126344in}{8.113463in}}%
\pgfpathcurveto{\pgfqpoint{4.126344in}{8.124513in}}{\pgfqpoint{4.121954in}{8.135112in}}{\pgfqpoint{4.114140in}{8.142925in}}%
\pgfpathcurveto{\pgfqpoint{4.106327in}{8.150739in}}{\pgfqpoint{4.095728in}{8.155129in}}{\pgfqpoint{4.084678in}{8.155129in}}%
\pgfpathcurveto{\pgfqpoint{4.073627in}{8.155129in}}{\pgfqpoint{4.063028in}{8.150739in}}{\pgfqpoint{4.055215in}{8.142925in}}%
\pgfpathcurveto{\pgfqpoint{4.047401in}{8.135112in}}{\pgfqpoint{4.043011in}{8.124513in}}{\pgfqpoint{4.043011in}{8.113463in}}%
\pgfpathcurveto{\pgfqpoint{4.043011in}{8.102413in}}{\pgfqpoint{4.047401in}{8.091813in}}{\pgfqpoint{4.055215in}{8.084000in}}%
\pgfpathcurveto{\pgfqpoint{4.063028in}{8.076186in}}{\pgfqpoint{4.073627in}{8.071796in}}{\pgfqpoint{4.084678in}{8.071796in}}%
\pgfpathlineto{\pgfqpoint{4.084678in}{8.071796in}}%
\pgfpathclose%
\pgfusepath{stroke,fill}%
\end{pgfscope}%
\begin{pgfscope}%
\pgfpathrectangle{\pgfqpoint{2.963410in}{7.624184in}}{\pgfqpoint{2.177280in}{2.201755in}}%
\pgfusepath{clip}%
\pgfsetbuttcap%
\pgfsetroundjoin%
\definecolor{currentfill}{rgb}{1.000000,0.498039,0.054902}%
\pgfsetfillcolor{currentfill}%
\pgfsetlinewidth{0.481800pt}%
\definecolor{currentstroke}{rgb}{1.000000,1.000000,1.000000}%
\pgfsetstrokecolor{currentstroke}%
\pgfsetdash{}{0pt}%
\pgfpathmoveto{\pgfqpoint{4.025602in}{9.183793in}}%
\pgfpathcurveto{\pgfqpoint{4.036652in}{9.183793in}}{\pgfqpoint{4.047251in}{9.188184in}}{\pgfqpoint{4.055065in}{9.195997in}}%
\pgfpathcurveto{\pgfqpoint{4.062879in}{9.203811in}}{\pgfqpoint{4.067269in}{9.214410in}}{\pgfqpoint{4.067269in}{9.225460in}}%
\pgfpathcurveto{\pgfqpoint{4.067269in}{9.236510in}}{\pgfqpoint{4.062879in}{9.247109in}}{\pgfqpoint{4.055065in}{9.254923in}}%
\pgfpathcurveto{\pgfqpoint{4.047251in}{9.262737in}}{\pgfqpoint{4.036652in}{9.267127in}}{\pgfqpoint{4.025602in}{9.267127in}}%
\pgfpathcurveto{\pgfqpoint{4.014552in}{9.267127in}}{\pgfqpoint{4.003953in}{9.262737in}}{\pgfqpoint{3.996139in}{9.254923in}}%
\pgfpathcurveto{\pgfqpoint{3.988326in}{9.247109in}}{\pgfqpoint{3.983936in}{9.236510in}}{\pgfqpoint{3.983936in}{9.225460in}}%
\pgfpathcurveto{\pgfqpoint{3.983936in}{9.214410in}}{\pgfqpoint{3.988326in}{9.203811in}}{\pgfqpoint{3.996139in}{9.195997in}}%
\pgfpathcurveto{\pgfqpoint{4.003953in}{9.188184in}}{\pgfqpoint{4.014552in}{9.183793in}}{\pgfqpoint{4.025602in}{9.183793in}}%
\pgfpathlineto{\pgfqpoint{4.025602in}{9.183793in}}%
\pgfpathclose%
\pgfusepath{stroke,fill}%
\end{pgfscope}%
\begin{pgfscope}%
\pgfpathrectangle{\pgfqpoint{2.963410in}{7.624184in}}{\pgfqpoint{2.177280in}{2.201755in}}%
\pgfusepath{clip}%
\pgfsetbuttcap%
\pgfsetroundjoin%
\definecolor{currentfill}{rgb}{1.000000,0.498039,0.054902}%
\pgfsetfillcolor{currentfill}%
\pgfsetlinewidth{0.481800pt}%
\definecolor{currentstroke}{rgb}{1.000000,1.000000,1.000000}%
\pgfsetstrokecolor{currentstroke}%
\pgfsetdash{}{0pt}%
\pgfpathmoveto{\pgfqpoint{4.025602in}{8.850194in}}%
\pgfpathcurveto{\pgfqpoint{4.036652in}{8.850194in}}{\pgfqpoint{4.047251in}{8.854584in}}{\pgfqpoint{4.055065in}{8.862398in}}%
\pgfpathcurveto{\pgfqpoint{4.062879in}{8.870212in}}{\pgfqpoint{4.067269in}{8.880811in}}{\pgfqpoint{4.067269in}{8.891861in}}%
\pgfpathcurveto{\pgfqpoint{4.067269in}{8.902911in}}{\pgfqpoint{4.062879in}{8.913510in}}{\pgfqpoint{4.055065in}{8.921324in}}%
\pgfpathcurveto{\pgfqpoint{4.047251in}{8.929137in}}{\pgfqpoint{4.036652in}{8.933528in}}{\pgfqpoint{4.025602in}{8.933528in}}%
\pgfpathcurveto{\pgfqpoint{4.014552in}{8.933528in}}{\pgfqpoint{4.003953in}{8.929137in}}{\pgfqpoint{3.996139in}{8.921324in}}%
\pgfpathcurveto{\pgfqpoint{3.988326in}{8.913510in}}{\pgfqpoint{3.983936in}{8.902911in}}{\pgfqpoint{3.983936in}{8.891861in}}%
\pgfpathcurveto{\pgfqpoint{3.983936in}{8.880811in}}{\pgfqpoint{3.988326in}{8.870212in}}{\pgfqpoint{3.996139in}{8.862398in}}%
\pgfpathcurveto{\pgfqpoint{4.003953in}{8.854584in}}{\pgfqpoint{4.014552in}{8.850194in}}{\pgfqpoint{4.025602in}{8.850194in}}%
\pgfpathlineto{\pgfqpoint{4.025602in}{8.850194in}}%
\pgfpathclose%
\pgfusepath{stroke,fill}%
\end{pgfscope}%
\begin{pgfscope}%
\pgfpathrectangle{\pgfqpoint{2.963410in}{7.624184in}}{\pgfqpoint{2.177280in}{2.201755in}}%
\pgfusepath{clip}%
\pgfsetbuttcap%
\pgfsetroundjoin%
\definecolor{currentfill}{rgb}{1.000000,0.498039,0.054902}%
\pgfsetfillcolor{currentfill}%
\pgfsetlinewidth{0.481800pt}%
\definecolor{currentstroke}{rgb}{1.000000,1.000000,1.000000}%
\pgfsetstrokecolor{currentstroke}%
\pgfsetdash{}{0pt}%
\pgfpathmoveto{\pgfqpoint{3.966527in}{9.128194in}}%
\pgfpathcurveto{\pgfqpoint{3.977577in}{9.128194in}}{\pgfqpoint{3.988176in}{9.132584in}}{\pgfqpoint{3.995990in}{9.140397in}}%
\pgfpathcurveto{\pgfqpoint{4.003803in}{9.148211in}}{\pgfqpoint{4.008194in}{9.158810in}}{\pgfqpoint{4.008194in}{9.169860in}}%
\pgfpathcurveto{\pgfqpoint{4.008194in}{9.180910in}}{\pgfqpoint{4.003803in}{9.191509in}}{\pgfqpoint{3.995990in}{9.199323in}}%
\pgfpathcurveto{\pgfqpoint{3.988176in}{9.207137in}}{\pgfqpoint{3.977577in}{9.211527in}}{\pgfqpoint{3.966527in}{9.211527in}}%
\pgfpathcurveto{\pgfqpoint{3.955477in}{9.211527in}}{\pgfqpoint{3.944878in}{9.207137in}}{\pgfqpoint{3.937064in}{9.199323in}}%
\pgfpathcurveto{\pgfqpoint{3.929251in}{9.191509in}}{\pgfqpoint{3.924860in}{9.180910in}}{\pgfqpoint{3.924860in}{9.169860in}}%
\pgfpathcurveto{\pgfqpoint{3.924860in}{9.158810in}}{\pgfqpoint{3.929251in}{9.148211in}}{\pgfqpoint{3.937064in}{9.140397in}}%
\pgfpathcurveto{\pgfqpoint{3.944878in}{9.132584in}}{\pgfqpoint{3.955477in}{9.128194in}}{\pgfqpoint{3.966527in}{9.128194in}}%
\pgfpathlineto{\pgfqpoint{3.966527in}{9.128194in}}%
\pgfpathclose%
\pgfusepath{stroke,fill}%
\end{pgfscope}%
\begin{pgfscope}%
\pgfpathrectangle{\pgfqpoint{2.963410in}{7.624184in}}{\pgfqpoint{2.177280in}{2.201755in}}%
\pgfusepath{clip}%
\pgfsetbuttcap%
\pgfsetroundjoin%
\definecolor{currentfill}{rgb}{1.000000,0.498039,0.054902}%
\pgfsetfillcolor{currentfill}%
\pgfsetlinewidth{0.481800pt}%
\definecolor{currentstroke}{rgb}{1.000000,1.000000,1.000000}%
\pgfsetstrokecolor{currentstroke}%
\pgfsetdash{}{0pt}%
\pgfpathmoveto{\pgfqpoint{3.493925in}{8.349795in}}%
\pgfpathcurveto{\pgfqpoint{3.504975in}{8.349795in}}{\pgfqpoint{3.515574in}{8.354186in}}{\pgfqpoint{3.523388in}{8.361999in}}%
\pgfpathcurveto{\pgfqpoint{3.531201in}{8.369813in}}{\pgfqpoint{3.535592in}{8.380412in}}{\pgfqpoint{3.535592in}{8.391462in}}%
\pgfpathcurveto{\pgfqpoint{3.535592in}{8.402512in}}{\pgfqpoint{3.531201in}{8.413111in}}{\pgfqpoint{3.523388in}{8.420925in}}%
\pgfpathcurveto{\pgfqpoint{3.515574in}{8.428738in}}{\pgfqpoint{3.504975in}{8.433129in}}{\pgfqpoint{3.493925in}{8.433129in}}%
\pgfpathcurveto{\pgfqpoint{3.482875in}{8.433129in}}{\pgfqpoint{3.472276in}{8.428738in}}{\pgfqpoint{3.464462in}{8.420925in}}%
\pgfpathcurveto{\pgfqpoint{3.456648in}{8.413111in}}{\pgfqpoint{3.452258in}{8.402512in}}{\pgfqpoint{3.452258in}{8.391462in}}%
\pgfpathcurveto{\pgfqpoint{3.452258in}{8.380412in}}{\pgfqpoint{3.456648in}{8.369813in}}{\pgfqpoint{3.464462in}{8.361999in}}%
\pgfpathcurveto{\pgfqpoint{3.472276in}{8.354186in}}{\pgfqpoint{3.482875in}{8.349795in}}{\pgfqpoint{3.493925in}{8.349795in}}%
\pgfpathlineto{\pgfqpoint{3.493925in}{8.349795in}}%
\pgfpathclose%
\pgfusepath{stroke,fill}%
\end{pgfscope}%
\begin{pgfscope}%
\pgfpathrectangle{\pgfqpoint{2.963410in}{7.624184in}}{\pgfqpoint{2.177280in}{2.201755in}}%
\pgfusepath{clip}%
\pgfsetbuttcap%
\pgfsetroundjoin%
\definecolor{currentfill}{rgb}{1.000000,0.498039,0.054902}%
\pgfsetfillcolor{currentfill}%
\pgfsetlinewidth{0.481800pt}%
\definecolor{currentstroke}{rgb}{1.000000,1.000000,1.000000}%
\pgfsetstrokecolor{currentstroke}%
\pgfsetdash{}{0pt}%
\pgfpathmoveto{\pgfqpoint{3.789301in}{8.905794in}}%
\pgfpathcurveto{\pgfqpoint{3.800351in}{8.905794in}}{\pgfqpoint{3.810950in}{8.910184in}}{\pgfqpoint{3.818764in}{8.917998in}}%
\pgfpathcurveto{\pgfqpoint{3.826578in}{8.925812in}}{\pgfqpoint{3.830968in}{8.936411in}}{\pgfqpoint{3.830968in}{8.947461in}}%
\pgfpathcurveto{\pgfqpoint{3.830968in}{8.958511in}}{\pgfqpoint{3.826578in}{8.969110in}}{\pgfqpoint{3.818764in}{8.976924in}}%
\pgfpathcurveto{\pgfqpoint{3.810950in}{8.984737in}}{\pgfqpoint{3.800351in}{8.989127in}}{\pgfqpoint{3.789301in}{8.989127in}}%
\pgfpathcurveto{\pgfqpoint{3.778251in}{8.989127in}}{\pgfqpoint{3.767652in}{8.984737in}}{\pgfqpoint{3.759838in}{8.976924in}}%
\pgfpathcurveto{\pgfqpoint{3.752025in}{8.969110in}}{\pgfqpoint{3.747635in}{8.958511in}}{\pgfqpoint{3.747635in}{8.947461in}}%
\pgfpathcurveto{\pgfqpoint{3.747635in}{8.936411in}}{\pgfqpoint{3.752025in}{8.925812in}}{\pgfqpoint{3.759838in}{8.917998in}}%
\pgfpathcurveto{\pgfqpoint{3.767652in}{8.910184in}}{\pgfqpoint{3.778251in}{8.905794in}}{\pgfqpoint{3.789301in}{8.905794in}}%
\pgfpathlineto{\pgfqpoint{3.789301in}{8.905794in}}%
\pgfpathclose%
\pgfusepath{stroke,fill}%
\end{pgfscope}%
\begin{pgfscope}%
\pgfpathrectangle{\pgfqpoint{2.963410in}{7.624184in}}{\pgfqpoint{2.177280in}{2.201755in}}%
\pgfusepath{clip}%
\pgfsetbuttcap%
\pgfsetroundjoin%
\definecolor{currentfill}{rgb}{1.000000,0.498039,0.054902}%
\pgfsetfillcolor{currentfill}%
\pgfsetlinewidth{0.481800pt}%
\definecolor{currentstroke}{rgb}{1.000000,1.000000,1.000000}%
\pgfsetstrokecolor{currentstroke}%
\pgfsetdash{}{0pt}%
\pgfpathmoveto{\pgfqpoint{3.789301in}{8.460995in}}%
\pgfpathcurveto{\pgfqpoint{3.800351in}{8.460995in}}{\pgfqpoint{3.810950in}{8.465385in}}{\pgfqpoint{3.818764in}{8.473199in}}%
\pgfpathcurveto{\pgfqpoint{3.826578in}{8.481013in}}{\pgfqpoint{3.830968in}{8.491612in}}{\pgfqpoint{3.830968in}{8.502662in}}%
\pgfpathcurveto{\pgfqpoint{3.830968in}{8.513712in}}{\pgfqpoint{3.826578in}{8.524311in}}{\pgfqpoint{3.818764in}{8.532125in}}%
\pgfpathcurveto{\pgfqpoint{3.810950in}{8.539938in}}{\pgfqpoint{3.800351in}{8.544328in}}{\pgfqpoint{3.789301in}{8.544328in}}%
\pgfpathcurveto{\pgfqpoint{3.778251in}{8.544328in}}{\pgfqpoint{3.767652in}{8.539938in}}{\pgfqpoint{3.759838in}{8.532125in}}%
\pgfpathcurveto{\pgfqpoint{3.752025in}{8.524311in}}{\pgfqpoint{3.747635in}{8.513712in}}{\pgfqpoint{3.747635in}{8.502662in}}%
\pgfpathcurveto{\pgfqpoint{3.747635in}{8.491612in}}{\pgfqpoint{3.752025in}{8.481013in}}{\pgfqpoint{3.759838in}{8.473199in}}%
\pgfpathcurveto{\pgfqpoint{3.767652in}{8.465385in}}{\pgfqpoint{3.778251in}{8.460995in}}{\pgfqpoint{3.789301in}{8.460995in}}%
\pgfpathlineto{\pgfqpoint{3.789301in}{8.460995in}}%
\pgfpathclose%
\pgfusepath{stroke,fill}%
\end{pgfscope}%
\begin{pgfscope}%
\pgfpathrectangle{\pgfqpoint{2.963410in}{7.624184in}}{\pgfqpoint{2.177280in}{2.201755in}}%
\pgfusepath{clip}%
\pgfsetbuttcap%
\pgfsetroundjoin%
\definecolor{currentfill}{rgb}{1.000000,0.498039,0.054902}%
\pgfsetfillcolor{currentfill}%
\pgfsetlinewidth{0.481800pt}%
\definecolor{currentstroke}{rgb}{1.000000,1.000000,1.000000}%
\pgfsetstrokecolor{currentstroke}%
\pgfsetdash{}{0pt}%
\pgfpathmoveto{\pgfqpoint{4.084678in}{8.794594in}}%
\pgfpathcurveto{\pgfqpoint{4.095728in}{8.794594in}}{\pgfqpoint{4.106327in}{8.798985in}}{\pgfqpoint{4.114140in}{8.806798in}}%
\pgfpathcurveto{\pgfqpoint{4.121954in}{8.814612in}}{\pgfqpoint{4.126344in}{8.825211in}}{\pgfqpoint{4.126344in}{8.836261in}}%
\pgfpathcurveto{\pgfqpoint{4.126344in}{8.847311in}}{\pgfqpoint{4.121954in}{8.857910in}}{\pgfqpoint{4.114140in}{8.865724in}}%
\pgfpathcurveto{\pgfqpoint{4.106327in}{8.873537in}}{\pgfqpoint{4.095728in}{8.877928in}}{\pgfqpoint{4.084678in}{8.877928in}}%
\pgfpathcurveto{\pgfqpoint{4.073627in}{8.877928in}}{\pgfqpoint{4.063028in}{8.873537in}}{\pgfqpoint{4.055215in}{8.865724in}}%
\pgfpathcurveto{\pgfqpoint{4.047401in}{8.857910in}}{\pgfqpoint{4.043011in}{8.847311in}}{\pgfqpoint{4.043011in}{8.836261in}}%
\pgfpathcurveto{\pgfqpoint{4.043011in}{8.825211in}}{\pgfqpoint{4.047401in}{8.814612in}}{\pgfqpoint{4.055215in}{8.806798in}}%
\pgfpathcurveto{\pgfqpoint{4.063028in}{8.798985in}}{\pgfqpoint{4.073627in}{8.794594in}}{\pgfqpoint{4.084678in}{8.794594in}}%
\pgfpathlineto{\pgfqpoint{4.084678in}{8.794594in}}%
\pgfpathclose%
\pgfusepath{stroke,fill}%
\end{pgfscope}%
\begin{pgfscope}%
\pgfpathrectangle{\pgfqpoint{2.963410in}{7.624184in}}{\pgfqpoint{2.177280in}{2.201755in}}%
\pgfusepath{clip}%
\pgfsetbuttcap%
\pgfsetroundjoin%
\definecolor{currentfill}{rgb}{1.000000,0.498039,0.054902}%
\pgfsetfillcolor{currentfill}%
\pgfsetlinewidth{0.481800pt}%
\definecolor{currentstroke}{rgb}{1.000000,1.000000,1.000000}%
\pgfsetstrokecolor{currentstroke}%
\pgfsetdash{}{0pt}%
\pgfpathmoveto{\pgfqpoint{3.553000in}{8.016196in}}%
\pgfpathcurveto{\pgfqpoint{3.564050in}{8.016196in}}{\pgfqpoint{3.574649in}{8.020586in}}{\pgfqpoint{3.582463in}{8.028400in}}%
\pgfpathcurveto{\pgfqpoint{3.590277in}{8.036214in}}{\pgfqpoint{3.594667in}{8.046813in}}{\pgfqpoint{3.594667in}{8.057863in}}%
\pgfpathcurveto{\pgfqpoint{3.594667in}{8.068913in}}{\pgfqpoint{3.590277in}{8.079512in}}{\pgfqpoint{3.582463in}{8.087326in}}%
\pgfpathcurveto{\pgfqpoint{3.574649in}{8.095139in}}{\pgfqpoint{3.564050in}{8.099529in}}{\pgfqpoint{3.553000in}{8.099529in}}%
\pgfpathcurveto{\pgfqpoint{3.541950in}{8.099529in}}{\pgfqpoint{3.531351in}{8.095139in}}{\pgfqpoint{3.523537in}{8.087326in}}%
\pgfpathcurveto{\pgfqpoint{3.515724in}{8.079512in}}{\pgfqpoint{3.511333in}{8.068913in}}{\pgfqpoint{3.511333in}{8.057863in}}%
\pgfpathcurveto{\pgfqpoint{3.511333in}{8.046813in}}{\pgfqpoint{3.515724in}{8.036214in}}{\pgfqpoint{3.523537in}{8.028400in}}%
\pgfpathcurveto{\pgfqpoint{3.531351in}{8.020586in}}{\pgfqpoint{3.541950in}{8.016196in}}{\pgfqpoint{3.553000in}{8.016196in}}%
\pgfpathlineto{\pgfqpoint{3.553000in}{8.016196in}}%
\pgfpathclose%
\pgfusepath{stroke,fill}%
\end{pgfscope}%
\begin{pgfscope}%
\pgfpathrectangle{\pgfqpoint{2.963410in}{7.624184in}}{\pgfqpoint{2.177280in}{2.201755in}}%
\pgfusepath{clip}%
\pgfsetbuttcap%
\pgfsetroundjoin%
\definecolor{currentfill}{rgb}{1.000000,0.498039,0.054902}%
\pgfsetfillcolor{currentfill}%
\pgfsetlinewidth{0.481800pt}%
\definecolor{currentstroke}{rgb}{1.000000,1.000000,1.000000}%
\pgfsetstrokecolor{currentstroke}%
\pgfsetdash{}{0pt}%
\pgfpathmoveto{\pgfqpoint{3.848376in}{8.961394in}}%
\pgfpathcurveto{\pgfqpoint{3.859427in}{8.961394in}}{\pgfqpoint{3.870026in}{8.965784in}}{\pgfqpoint{3.877839in}{8.973598in}}%
\pgfpathcurveto{\pgfqpoint{3.885653in}{8.981411in}}{\pgfqpoint{3.890043in}{8.992011in}}{\pgfqpoint{3.890043in}{9.003061in}}%
\pgfpathcurveto{\pgfqpoint{3.890043in}{9.014111in}}{\pgfqpoint{3.885653in}{9.024710in}}{\pgfqpoint{3.877839in}{9.032523in}}%
\pgfpathcurveto{\pgfqpoint{3.870026in}{9.040337in}}{\pgfqpoint{3.859427in}{9.044727in}}{\pgfqpoint{3.848376in}{9.044727in}}%
\pgfpathcurveto{\pgfqpoint{3.837326in}{9.044727in}}{\pgfqpoint{3.826727in}{9.040337in}}{\pgfqpoint{3.818914in}{9.032523in}}%
\pgfpathcurveto{\pgfqpoint{3.811100in}{9.024710in}}{\pgfqpoint{3.806710in}{9.014111in}}{\pgfqpoint{3.806710in}{9.003061in}}%
\pgfpathcurveto{\pgfqpoint{3.806710in}{8.992011in}}{\pgfqpoint{3.811100in}{8.981411in}}{\pgfqpoint{3.818914in}{8.973598in}}%
\pgfpathcurveto{\pgfqpoint{3.826727in}{8.965784in}}{\pgfqpoint{3.837326in}{8.961394in}}{\pgfqpoint{3.848376in}{8.961394in}}%
\pgfpathlineto{\pgfqpoint{3.848376in}{8.961394in}}%
\pgfpathclose%
\pgfusepath{stroke,fill}%
\end{pgfscope}%
\begin{pgfscope}%
\pgfpathrectangle{\pgfqpoint{2.963410in}{7.624184in}}{\pgfqpoint{2.177280in}{2.201755in}}%
\pgfusepath{clip}%
\pgfsetbuttcap%
\pgfsetroundjoin%
\definecolor{currentfill}{rgb}{1.000000,0.498039,0.054902}%
\pgfsetfillcolor{currentfill}%
\pgfsetlinewidth{0.481800pt}%
\definecolor{currentstroke}{rgb}{1.000000,1.000000,1.000000}%
\pgfsetstrokecolor{currentstroke}%
\pgfsetdash{}{0pt}%
\pgfpathmoveto{\pgfqpoint{3.730226in}{8.182996in}}%
\pgfpathcurveto{\pgfqpoint{3.741276in}{8.182996in}}{\pgfqpoint{3.751875in}{8.187386in}}{\pgfqpoint{3.759689in}{8.195200in}}%
\pgfpathcurveto{\pgfqpoint{3.767502in}{8.203013in}}{\pgfqpoint{3.771893in}{8.213612in}}{\pgfqpoint{3.771893in}{8.224662in}}%
\pgfpathcurveto{\pgfqpoint{3.771893in}{8.235713in}}{\pgfqpoint{3.767502in}{8.246312in}}{\pgfqpoint{3.759689in}{8.254125in}}%
\pgfpathcurveto{\pgfqpoint{3.751875in}{8.261939in}}{\pgfqpoint{3.741276in}{8.266329in}}{\pgfqpoint{3.730226in}{8.266329in}}%
\pgfpathcurveto{\pgfqpoint{3.719176in}{8.266329in}}{\pgfqpoint{3.708577in}{8.261939in}}{\pgfqpoint{3.700763in}{8.254125in}}%
\pgfpathcurveto{\pgfqpoint{3.692950in}{8.246312in}}{\pgfqpoint{3.688559in}{8.235713in}}{\pgfqpoint{3.688559in}{8.224662in}}%
\pgfpathcurveto{\pgfqpoint{3.688559in}{8.213612in}}{\pgfqpoint{3.692950in}{8.203013in}}{\pgfqpoint{3.700763in}{8.195200in}}%
\pgfpathcurveto{\pgfqpoint{3.708577in}{8.187386in}}{\pgfqpoint{3.719176in}{8.182996in}}{\pgfqpoint{3.730226in}{8.182996in}}%
\pgfpathlineto{\pgfqpoint{3.730226in}{8.182996in}}%
\pgfpathclose%
\pgfusepath{stroke,fill}%
\end{pgfscope}%
\begin{pgfscope}%
\pgfpathrectangle{\pgfqpoint{2.963410in}{7.624184in}}{\pgfqpoint{2.177280in}{2.201755in}}%
\pgfusepath{clip}%
\pgfsetbuttcap%
\pgfsetroundjoin%
\definecolor{currentfill}{rgb}{1.000000,0.498039,0.054902}%
\pgfsetfillcolor{currentfill}%
\pgfsetlinewidth{0.481800pt}%
\definecolor{currentstroke}{rgb}{1.000000,1.000000,1.000000}%
\pgfsetstrokecolor{currentstroke}%
\pgfsetdash{}{0pt}%
\pgfpathmoveto{\pgfqpoint{3.316699in}{8.071796in}}%
\pgfpathcurveto{\pgfqpoint{3.327749in}{8.071796in}}{\pgfqpoint{3.338348in}{8.076186in}}{\pgfqpoint{3.346162in}{8.084000in}}%
\pgfpathcurveto{\pgfqpoint{3.353975in}{8.091813in}}{\pgfqpoint{3.358366in}{8.102413in}}{\pgfqpoint{3.358366in}{8.113463in}}%
\pgfpathcurveto{\pgfqpoint{3.358366in}{8.124513in}}{\pgfqpoint{3.353975in}{8.135112in}}{\pgfqpoint{3.346162in}{8.142925in}}%
\pgfpathcurveto{\pgfqpoint{3.338348in}{8.150739in}}{\pgfqpoint{3.327749in}{8.155129in}}{\pgfqpoint{3.316699in}{8.155129in}}%
\pgfpathcurveto{\pgfqpoint{3.305649in}{8.155129in}}{\pgfqpoint{3.295050in}{8.150739in}}{\pgfqpoint{3.287236in}{8.142925in}}%
\pgfpathcurveto{\pgfqpoint{3.279423in}{8.135112in}}{\pgfqpoint{3.275032in}{8.124513in}}{\pgfqpoint{3.275032in}{8.113463in}}%
\pgfpathcurveto{\pgfqpoint{3.275032in}{8.102413in}}{\pgfqpoint{3.279423in}{8.091813in}}{\pgfqpoint{3.287236in}{8.084000in}}%
\pgfpathcurveto{\pgfqpoint{3.295050in}{8.076186in}}{\pgfqpoint{3.305649in}{8.071796in}}{\pgfqpoint{3.316699in}{8.071796in}}%
\pgfpathlineto{\pgfqpoint{3.316699in}{8.071796in}}%
\pgfpathclose%
\pgfusepath{stroke,fill}%
\end{pgfscope}%
\begin{pgfscope}%
\pgfpathrectangle{\pgfqpoint{2.963410in}{7.624184in}}{\pgfqpoint{2.177280in}{2.201755in}}%
\pgfusepath{clip}%
\pgfsetbuttcap%
\pgfsetroundjoin%
\definecolor{currentfill}{rgb}{1.000000,0.498039,0.054902}%
\pgfsetfillcolor{currentfill}%
\pgfsetlinewidth{0.481800pt}%
\definecolor{currentstroke}{rgb}{1.000000,1.000000,1.000000}%
\pgfsetstrokecolor{currentstroke}%
\pgfsetdash{}{0pt}%
\pgfpathmoveto{\pgfqpoint{3.907452in}{8.572195in}}%
\pgfpathcurveto{\pgfqpoint{3.918502in}{8.572195in}}{\pgfqpoint{3.929101in}{8.576585in}}{\pgfqpoint{3.936915in}{8.584399in}}%
\pgfpathcurveto{\pgfqpoint{3.944728in}{8.592212in}}{\pgfqpoint{3.949118in}{8.602811in}}{\pgfqpoint{3.949118in}{8.613862in}}%
\pgfpathcurveto{\pgfqpoint{3.949118in}{8.624912in}}{\pgfqpoint{3.944728in}{8.635511in}}{\pgfqpoint{3.936915in}{8.643324in}}%
\pgfpathcurveto{\pgfqpoint{3.929101in}{8.651138in}}{\pgfqpoint{3.918502in}{8.655528in}}{\pgfqpoint{3.907452in}{8.655528in}}%
\pgfpathcurveto{\pgfqpoint{3.896402in}{8.655528in}}{\pgfqpoint{3.885803in}{8.651138in}}{\pgfqpoint{3.877989in}{8.643324in}}%
\pgfpathcurveto{\pgfqpoint{3.870175in}{8.635511in}}{\pgfqpoint{3.865785in}{8.624912in}}{\pgfqpoint{3.865785in}{8.613862in}}%
\pgfpathcurveto{\pgfqpoint{3.865785in}{8.602811in}}{\pgfqpoint{3.870175in}{8.592212in}}{\pgfqpoint{3.877989in}{8.584399in}}%
\pgfpathcurveto{\pgfqpoint{3.885803in}{8.576585in}}{\pgfqpoint{3.896402in}{8.572195in}}{\pgfqpoint{3.907452in}{8.572195in}}%
\pgfpathlineto{\pgfqpoint{3.907452in}{8.572195in}}%
\pgfpathclose%
\pgfusepath{stroke,fill}%
\end{pgfscope}%
\begin{pgfscope}%
\pgfpathrectangle{\pgfqpoint{2.963410in}{7.624184in}}{\pgfqpoint{2.177280in}{2.201755in}}%
\pgfusepath{clip}%
\pgfsetbuttcap%
\pgfsetroundjoin%
\definecolor{currentfill}{rgb}{1.000000,0.498039,0.054902}%
\pgfsetfillcolor{currentfill}%
\pgfsetlinewidth{0.481800pt}%
\definecolor{currentstroke}{rgb}{1.000000,1.000000,1.000000}%
\pgfsetstrokecolor{currentstroke}%
\pgfsetdash{}{0pt}%
\pgfpathmoveto{\pgfqpoint{3.434850in}{8.627795in}}%
\pgfpathcurveto{\pgfqpoint{3.445900in}{8.627795in}}{\pgfqpoint{3.456499in}{8.632185in}}{\pgfqpoint{3.464312in}{8.639999in}}%
\pgfpathcurveto{\pgfqpoint{3.472126in}{8.647812in}}{\pgfqpoint{3.476516in}{8.658411in}}{\pgfqpoint{3.476516in}{8.669461in}}%
\pgfpathcurveto{\pgfqpoint{3.476516in}{8.680512in}}{\pgfqpoint{3.472126in}{8.691111in}}{\pgfqpoint{3.464312in}{8.698924in}}%
\pgfpathcurveto{\pgfqpoint{3.456499in}{8.706738in}}{\pgfqpoint{3.445900in}{8.711128in}}{\pgfqpoint{3.434850in}{8.711128in}}%
\pgfpathcurveto{\pgfqpoint{3.423799in}{8.711128in}}{\pgfqpoint{3.413200in}{8.706738in}}{\pgfqpoint{3.405387in}{8.698924in}}%
\pgfpathcurveto{\pgfqpoint{3.397573in}{8.691111in}}{\pgfqpoint{3.393183in}{8.680512in}}{\pgfqpoint{3.393183in}{8.669461in}}%
\pgfpathcurveto{\pgfqpoint{3.393183in}{8.658411in}}{\pgfqpoint{3.397573in}{8.647812in}}{\pgfqpoint{3.405387in}{8.639999in}}%
\pgfpathcurveto{\pgfqpoint{3.413200in}{8.632185in}}{\pgfqpoint{3.423799in}{8.627795in}}{\pgfqpoint{3.434850in}{8.627795in}}%
\pgfpathlineto{\pgfqpoint{3.434850in}{8.627795in}}%
\pgfpathclose%
\pgfusepath{stroke,fill}%
\end{pgfscope}%
\begin{pgfscope}%
\pgfpathrectangle{\pgfqpoint{2.963410in}{7.624184in}}{\pgfqpoint{2.177280in}{2.201755in}}%
\pgfusepath{clip}%
\pgfsetbuttcap%
\pgfsetroundjoin%
\definecolor{currentfill}{rgb}{1.000000,0.498039,0.054902}%
\pgfsetfillcolor{currentfill}%
\pgfsetlinewidth{0.481800pt}%
\definecolor{currentstroke}{rgb}{1.000000,1.000000,1.000000}%
\pgfsetstrokecolor{currentstroke}%
\pgfsetdash{}{0pt}%
\pgfpathmoveto{\pgfqpoint{3.848376in}{8.683395in}}%
\pgfpathcurveto{\pgfqpoint{3.859427in}{8.683395in}}{\pgfqpoint{3.870026in}{8.687785in}}{\pgfqpoint{3.877839in}{8.695598in}}%
\pgfpathcurveto{\pgfqpoint{3.885653in}{8.703412in}}{\pgfqpoint{3.890043in}{8.714011in}}{\pgfqpoint{3.890043in}{8.725061in}}%
\pgfpathcurveto{\pgfqpoint{3.890043in}{8.736111in}}{\pgfqpoint{3.885653in}{8.746710in}}{\pgfqpoint{3.877839in}{8.754524in}}%
\pgfpathcurveto{\pgfqpoint{3.870026in}{8.762338in}}{\pgfqpoint{3.859427in}{8.766728in}}{\pgfqpoint{3.848376in}{8.766728in}}%
\pgfpathcurveto{\pgfqpoint{3.837326in}{8.766728in}}{\pgfqpoint{3.826727in}{8.762338in}}{\pgfqpoint{3.818914in}{8.754524in}}%
\pgfpathcurveto{\pgfqpoint{3.811100in}{8.746710in}}{\pgfqpoint{3.806710in}{8.736111in}}{\pgfqpoint{3.806710in}{8.725061in}}%
\pgfpathcurveto{\pgfqpoint{3.806710in}{8.714011in}}{\pgfqpoint{3.811100in}{8.703412in}}{\pgfqpoint{3.818914in}{8.695598in}}%
\pgfpathcurveto{\pgfqpoint{3.826727in}{8.687785in}}{\pgfqpoint{3.837326in}{8.683395in}}{\pgfqpoint{3.848376in}{8.683395in}}%
\pgfpathlineto{\pgfqpoint{3.848376in}{8.683395in}}%
\pgfpathclose%
\pgfusepath{stroke,fill}%
\end{pgfscope}%
\begin{pgfscope}%
\pgfpathrectangle{\pgfqpoint{2.963410in}{7.624184in}}{\pgfqpoint{2.177280in}{2.201755in}}%
\pgfusepath{clip}%
\pgfsetbuttcap%
\pgfsetroundjoin%
\definecolor{currentfill}{rgb}{1.000000,0.498039,0.054902}%
\pgfsetfillcolor{currentfill}%
\pgfsetlinewidth{0.481800pt}%
\definecolor{currentstroke}{rgb}{1.000000,1.000000,1.000000}%
\pgfsetstrokecolor{currentstroke}%
\pgfsetdash{}{0pt}%
\pgfpathmoveto{\pgfqpoint{3.848376in}{8.405395in}}%
\pgfpathcurveto{\pgfqpoint{3.859427in}{8.405395in}}{\pgfqpoint{3.870026in}{8.409785in}}{\pgfqpoint{3.877839in}{8.417599in}}%
\pgfpathcurveto{\pgfqpoint{3.885653in}{8.425413in}}{\pgfqpoint{3.890043in}{8.436012in}}{\pgfqpoint{3.890043in}{8.447062in}}%
\pgfpathcurveto{\pgfqpoint{3.890043in}{8.458112in}}{\pgfqpoint{3.885653in}{8.468711in}}{\pgfqpoint{3.877839in}{8.476525in}}%
\pgfpathcurveto{\pgfqpoint{3.870026in}{8.484338in}}{\pgfqpoint{3.859427in}{8.488729in}}{\pgfqpoint{3.848376in}{8.488729in}}%
\pgfpathcurveto{\pgfqpoint{3.837326in}{8.488729in}}{\pgfqpoint{3.826727in}{8.484338in}}{\pgfqpoint{3.818914in}{8.476525in}}%
\pgfpathcurveto{\pgfqpoint{3.811100in}{8.468711in}}{\pgfqpoint{3.806710in}{8.458112in}}{\pgfqpoint{3.806710in}{8.447062in}}%
\pgfpathcurveto{\pgfqpoint{3.806710in}{8.436012in}}{\pgfqpoint{3.811100in}{8.425413in}}{\pgfqpoint{3.818914in}{8.417599in}}%
\pgfpathcurveto{\pgfqpoint{3.826727in}{8.409785in}}{\pgfqpoint{3.837326in}{8.405395in}}{\pgfqpoint{3.848376in}{8.405395in}}%
\pgfpathlineto{\pgfqpoint{3.848376in}{8.405395in}}%
\pgfpathclose%
\pgfusepath{stroke,fill}%
\end{pgfscope}%
\begin{pgfscope}%
\pgfpathrectangle{\pgfqpoint{2.963410in}{7.624184in}}{\pgfqpoint{2.177280in}{2.201755in}}%
\pgfusepath{clip}%
\pgfsetbuttcap%
\pgfsetroundjoin%
\definecolor{currentfill}{rgb}{1.000000,0.498039,0.054902}%
\pgfsetfillcolor{currentfill}%
\pgfsetlinewidth{0.481800pt}%
\definecolor{currentstroke}{rgb}{1.000000,1.000000,1.000000}%
\pgfsetstrokecolor{currentstroke}%
\pgfsetdash{}{0pt}%
\pgfpathmoveto{\pgfqpoint{3.966527in}{9.016994in}}%
\pgfpathcurveto{\pgfqpoint{3.977577in}{9.016994in}}{\pgfqpoint{3.988176in}{9.021384in}}{\pgfqpoint{3.995990in}{9.029198in}}%
\pgfpathcurveto{\pgfqpoint{4.003803in}{9.037011in}}{\pgfqpoint{4.008194in}{9.047610in}}{\pgfqpoint{4.008194in}{9.058661in}}%
\pgfpathcurveto{\pgfqpoint{4.008194in}{9.069711in}}{\pgfqpoint{4.003803in}{9.080310in}}{\pgfqpoint{3.995990in}{9.088123in}}%
\pgfpathcurveto{\pgfqpoint{3.988176in}{9.095937in}}{\pgfqpoint{3.977577in}{9.100327in}}{\pgfqpoint{3.966527in}{9.100327in}}%
\pgfpathcurveto{\pgfqpoint{3.955477in}{9.100327in}}{\pgfqpoint{3.944878in}{9.095937in}}{\pgfqpoint{3.937064in}{9.088123in}}%
\pgfpathcurveto{\pgfqpoint{3.929251in}{9.080310in}}{\pgfqpoint{3.924860in}{9.069711in}}{\pgfqpoint{3.924860in}{9.058661in}}%
\pgfpathcurveto{\pgfqpoint{3.924860in}{9.047610in}}{\pgfqpoint{3.929251in}{9.037011in}}{\pgfqpoint{3.937064in}{9.029198in}}%
\pgfpathcurveto{\pgfqpoint{3.944878in}{9.021384in}}{\pgfqpoint{3.955477in}{9.016994in}}{\pgfqpoint{3.966527in}{9.016994in}}%
\pgfpathlineto{\pgfqpoint{3.966527in}{9.016994in}}%
\pgfpathclose%
\pgfusepath{stroke,fill}%
\end{pgfscope}%
\begin{pgfscope}%
\pgfpathrectangle{\pgfqpoint{2.963410in}{7.624184in}}{\pgfqpoint{2.177280in}{2.201755in}}%
\pgfusepath{clip}%
\pgfsetbuttcap%
\pgfsetroundjoin%
\definecolor{currentfill}{rgb}{1.000000,0.498039,0.054902}%
\pgfsetfillcolor{currentfill}%
\pgfsetlinewidth{0.481800pt}%
\definecolor{currentstroke}{rgb}{1.000000,1.000000,1.000000}%
\pgfsetstrokecolor{currentstroke}%
\pgfsetdash{}{0pt}%
\pgfpathmoveto{\pgfqpoint{3.907452in}{8.405395in}}%
\pgfpathcurveto{\pgfqpoint{3.918502in}{8.405395in}}{\pgfqpoint{3.929101in}{8.409785in}}{\pgfqpoint{3.936915in}{8.417599in}}%
\pgfpathcurveto{\pgfqpoint{3.944728in}{8.425413in}}{\pgfqpoint{3.949118in}{8.436012in}}{\pgfqpoint{3.949118in}{8.447062in}}%
\pgfpathcurveto{\pgfqpoint{3.949118in}{8.458112in}}{\pgfqpoint{3.944728in}{8.468711in}}{\pgfqpoint{3.936915in}{8.476525in}}%
\pgfpathcurveto{\pgfqpoint{3.929101in}{8.484338in}}{\pgfqpoint{3.918502in}{8.488729in}}{\pgfqpoint{3.907452in}{8.488729in}}%
\pgfpathcurveto{\pgfqpoint{3.896402in}{8.488729in}}{\pgfqpoint{3.885803in}{8.484338in}}{\pgfqpoint{3.877989in}{8.476525in}}%
\pgfpathcurveto{\pgfqpoint{3.870175in}{8.468711in}}{\pgfqpoint{3.865785in}{8.458112in}}{\pgfqpoint{3.865785in}{8.447062in}}%
\pgfpathcurveto{\pgfqpoint{3.865785in}{8.436012in}}{\pgfqpoint{3.870175in}{8.425413in}}{\pgfqpoint{3.877989in}{8.417599in}}%
\pgfpathcurveto{\pgfqpoint{3.885803in}{8.409785in}}{\pgfqpoint{3.896402in}{8.405395in}}{\pgfqpoint{3.907452in}{8.405395in}}%
\pgfpathlineto{\pgfqpoint{3.907452in}{8.405395in}}%
\pgfpathclose%
\pgfusepath{stroke,fill}%
\end{pgfscope}%
\begin{pgfscope}%
\pgfpathrectangle{\pgfqpoint{2.963410in}{7.624184in}}{\pgfqpoint{2.177280in}{2.201755in}}%
\pgfusepath{clip}%
\pgfsetbuttcap%
\pgfsetroundjoin%
\definecolor{currentfill}{rgb}{1.000000,0.498039,0.054902}%
\pgfsetfillcolor{currentfill}%
\pgfsetlinewidth{0.481800pt}%
\definecolor{currentstroke}{rgb}{1.000000,1.000000,1.000000}%
\pgfsetstrokecolor{currentstroke}%
\pgfsetdash{}{0pt}%
\pgfpathmoveto{\pgfqpoint{3.730226in}{8.516595in}}%
\pgfpathcurveto{\pgfqpoint{3.741276in}{8.516595in}}{\pgfqpoint{3.751875in}{8.520985in}}{\pgfqpoint{3.759689in}{8.528799in}}%
\pgfpathcurveto{\pgfqpoint{3.767502in}{8.536612in}}{\pgfqpoint{3.771893in}{8.547212in}}{\pgfqpoint{3.771893in}{8.558262in}}%
\pgfpathcurveto{\pgfqpoint{3.771893in}{8.569312in}}{\pgfqpoint{3.767502in}{8.579911in}}{\pgfqpoint{3.759689in}{8.587724in}}%
\pgfpathcurveto{\pgfqpoint{3.751875in}{8.595538in}}{\pgfqpoint{3.741276in}{8.599928in}}{\pgfqpoint{3.730226in}{8.599928in}}%
\pgfpathcurveto{\pgfqpoint{3.719176in}{8.599928in}}{\pgfqpoint{3.708577in}{8.595538in}}{\pgfqpoint{3.700763in}{8.587724in}}%
\pgfpathcurveto{\pgfqpoint{3.692950in}{8.579911in}}{\pgfqpoint{3.688559in}{8.569312in}}{\pgfqpoint{3.688559in}{8.558262in}}%
\pgfpathcurveto{\pgfqpoint{3.688559in}{8.547212in}}{\pgfqpoint{3.692950in}{8.536612in}}{\pgfqpoint{3.700763in}{8.528799in}}%
\pgfpathcurveto{\pgfqpoint{3.708577in}{8.520985in}}{\pgfqpoint{3.719176in}{8.516595in}}{\pgfqpoint{3.730226in}{8.516595in}}%
\pgfpathlineto{\pgfqpoint{3.730226in}{8.516595in}}%
\pgfpathclose%
\pgfusepath{stroke,fill}%
\end{pgfscope}%
\begin{pgfscope}%
\pgfpathrectangle{\pgfqpoint{2.963410in}{7.624184in}}{\pgfqpoint{2.177280in}{2.201755in}}%
\pgfusepath{clip}%
\pgfsetbuttcap%
\pgfsetroundjoin%
\definecolor{currentfill}{rgb}{1.000000,0.498039,0.054902}%
\pgfsetfillcolor{currentfill}%
\pgfsetlinewidth{0.481800pt}%
\definecolor{currentstroke}{rgb}{1.000000,1.000000,1.000000}%
\pgfsetstrokecolor{currentstroke}%
\pgfsetdash{}{0pt}%
\pgfpathmoveto{\pgfqpoint{3.434850in}{8.738994in}}%
\pgfpathcurveto{\pgfqpoint{3.445900in}{8.738994in}}{\pgfqpoint{3.456499in}{8.743385in}}{\pgfqpoint{3.464312in}{8.751198in}}%
\pgfpathcurveto{\pgfqpoint{3.472126in}{8.759012in}}{\pgfqpoint{3.476516in}{8.769611in}}{\pgfqpoint{3.476516in}{8.780661in}}%
\pgfpathcurveto{\pgfqpoint{3.476516in}{8.791711in}}{\pgfqpoint{3.472126in}{8.802310in}}{\pgfqpoint{3.464312in}{8.810124in}}%
\pgfpathcurveto{\pgfqpoint{3.456499in}{8.817938in}}{\pgfqpoint{3.445900in}{8.822328in}}{\pgfqpoint{3.434850in}{8.822328in}}%
\pgfpathcurveto{\pgfqpoint{3.423799in}{8.822328in}}{\pgfqpoint{3.413200in}{8.817938in}}{\pgfqpoint{3.405387in}{8.810124in}}%
\pgfpathcurveto{\pgfqpoint{3.397573in}{8.802310in}}{\pgfqpoint{3.393183in}{8.791711in}}{\pgfqpoint{3.393183in}{8.780661in}}%
\pgfpathcurveto{\pgfqpoint{3.393183in}{8.769611in}}{\pgfqpoint{3.397573in}{8.759012in}}{\pgfqpoint{3.405387in}{8.751198in}}%
\pgfpathcurveto{\pgfqpoint{3.413200in}{8.743385in}}{\pgfqpoint{3.423799in}{8.738994in}}{\pgfqpoint{3.434850in}{8.738994in}}%
\pgfpathlineto{\pgfqpoint{3.434850in}{8.738994in}}%
\pgfpathclose%
\pgfusepath{stroke,fill}%
\end{pgfscope}%
\begin{pgfscope}%
\pgfpathrectangle{\pgfqpoint{2.963410in}{7.624184in}}{\pgfqpoint{2.177280in}{2.201755in}}%
\pgfusepath{clip}%
\pgfsetbuttcap%
\pgfsetroundjoin%
\definecolor{currentfill}{rgb}{1.000000,0.498039,0.054902}%
\pgfsetfillcolor{currentfill}%
\pgfsetlinewidth{0.481800pt}%
\definecolor{currentstroke}{rgb}{1.000000,1.000000,1.000000}%
\pgfsetstrokecolor{currentstroke}%
\pgfsetdash{}{0pt}%
\pgfpathmoveto{\pgfqpoint{3.612075in}{8.405395in}}%
\pgfpathcurveto{\pgfqpoint{3.623126in}{8.405395in}}{\pgfqpoint{3.633725in}{8.409785in}}{\pgfqpoint{3.641538in}{8.417599in}}%
\pgfpathcurveto{\pgfqpoint{3.649352in}{8.425413in}}{\pgfqpoint{3.653742in}{8.436012in}}{\pgfqpoint{3.653742in}{8.447062in}}%
\pgfpathcurveto{\pgfqpoint{3.653742in}{8.458112in}}{\pgfqpoint{3.649352in}{8.468711in}}{\pgfqpoint{3.641538in}{8.476525in}}%
\pgfpathcurveto{\pgfqpoint{3.633725in}{8.484338in}}{\pgfqpoint{3.623126in}{8.488729in}}{\pgfqpoint{3.612075in}{8.488729in}}%
\pgfpathcurveto{\pgfqpoint{3.601025in}{8.488729in}}{\pgfqpoint{3.590426in}{8.484338in}}{\pgfqpoint{3.582613in}{8.476525in}}%
\pgfpathcurveto{\pgfqpoint{3.574799in}{8.468711in}}{\pgfqpoint{3.570409in}{8.458112in}}{\pgfqpoint{3.570409in}{8.447062in}}%
\pgfpathcurveto{\pgfqpoint{3.570409in}{8.436012in}}{\pgfqpoint{3.574799in}{8.425413in}}{\pgfqpoint{3.582613in}{8.417599in}}%
\pgfpathcurveto{\pgfqpoint{3.590426in}{8.409785in}}{\pgfqpoint{3.601025in}{8.405395in}}{\pgfqpoint{3.612075in}{8.405395in}}%
\pgfpathlineto{\pgfqpoint{3.612075in}{8.405395in}}%
\pgfpathclose%
\pgfusepath{stroke,fill}%
\end{pgfscope}%
\begin{pgfscope}%
\pgfpathrectangle{\pgfqpoint{2.963410in}{7.624184in}}{\pgfqpoint{2.177280in}{2.201755in}}%
\pgfusepath{clip}%
\pgfsetbuttcap%
\pgfsetroundjoin%
\definecolor{currentfill}{rgb}{1.000000,0.498039,0.054902}%
\pgfsetfillcolor{currentfill}%
\pgfsetlinewidth{0.481800pt}%
\definecolor{currentstroke}{rgb}{1.000000,1.000000,1.000000}%
\pgfsetstrokecolor{currentstroke}%
\pgfsetdash{}{0pt}%
\pgfpathmoveto{\pgfqpoint{4.025602in}{8.572195in}}%
\pgfpathcurveto{\pgfqpoint{4.036652in}{8.572195in}}{\pgfqpoint{4.047251in}{8.576585in}}{\pgfqpoint{4.055065in}{8.584399in}}%
\pgfpathcurveto{\pgfqpoint{4.062879in}{8.592212in}}{\pgfqpoint{4.067269in}{8.602811in}}{\pgfqpoint{4.067269in}{8.613862in}}%
\pgfpathcurveto{\pgfqpoint{4.067269in}{8.624912in}}{\pgfqpoint{4.062879in}{8.635511in}}{\pgfqpoint{4.055065in}{8.643324in}}%
\pgfpathcurveto{\pgfqpoint{4.047251in}{8.651138in}}{\pgfqpoint{4.036652in}{8.655528in}}{\pgfqpoint{4.025602in}{8.655528in}}%
\pgfpathcurveto{\pgfqpoint{4.014552in}{8.655528in}}{\pgfqpoint{4.003953in}{8.651138in}}{\pgfqpoint{3.996139in}{8.643324in}}%
\pgfpathcurveto{\pgfqpoint{3.988326in}{8.635511in}}{\pgfqpoint{3.983936in}{8.624912in}}{\pgfqpoint{3.983936in}{8.613862in}}%
\pgfpathcurveto{\pgfqpoint{3.983936in}{8.602811in}}{\pgfqpoint{3.988326in}{8.592212in}}{\pgfqpoint{3.996139in}{8.584399in}}%
\pgfpathcurveto{\pgfqpoint{4.003953in}{8.576585in}}{\pgfqpoint{4.014552in}{8.572195in}}{\pgfqpoint{4.025602in}{8.572195in}}%
\pgfpathlineto{\pgfqpoint{4.025602in}{8.572195in}}%
\pgfpathclose%
\pgfusepath{stroke,fill}%
\end{pgfscope}%
\begin{pgfscope}%
\pgfpathrectangle{\pgfqpoint{2.963410in}{7.624184in}}{\pgfqpoint{2.177280in}{2.201755in}}%
\pgfusepath{clip}%
\pgfsetbuttcap%
\pgfsetroundjoin%
\definecolor{currentfill}{rgb}{1.000000,0.498039,0.054902}%
\pgfsetfillcolor{currentfill}%
\pgfsetlinewidth{0.481800pt}%
\definecolor{currentstroke}{rgb}{1.000000,1.000000,1.000000}%
\pgfsetstrokecolor{currentstroke}%
\pgfsetdash{}{0pt}%
\pgfpathmoveto{\pgfqpoint{3.789301in}{8.683395in}}%
\pgfpathcurveto{\pgfqpoint{3.800351in}{8.683395in}}{\pgfqpoint{3.810950in}{8.687785in}}{\pgfqpoint{3.818764in}{8.695598in}}%
\pgfpathcurveto{\pgfqpoint{3.826578in}{8.703412in}}{\pgfqpoint{3.830968in}{8.714011in}}{\pgfqpoint{3.830968in}{8.725061in}}%
\pgfpathcurveto{\pgfqpoint{3.830968in}{8.736111in}}{\pgfqpoint{3.826578in}{8.746710in}}{\pgfqpoint{3.818764in}{8.754524in}}%
\pgfpathcurveto{\pgfqpoint{3.810950in}{8.762338in}}{\pgfqpoint{3.800351in}{8.766728in}}{\pgfqpoint{3.789301in}{8.766728in}}%
\pgfpathcurveto{\pgfqpoint{3.778251in}{8.766728in}}{\pgfqpoint{3.767652in}{8.762338in}}{\pgfqpoint{3.759838in}{8.754524in}}%
\pgfpathcurveto{\pgfqpoint{3.752025in}{8.746710in}}{\pgfqpoint{3.747635in}{8.736111in}}{\pgfqpoint{3.747635in}{8.725061in}}%
\pgfpathcurveto{\pgfqpoint{3.747635in}{8.714011in}}{\pgfqpoint{3.752025in}{8.703412in}}{\pgfqpoint{3.759838in}{8.695598in}}%
\pgfpathcurveto{\pgfqpoint{3.767652in}{8.687785in}}{\pgfqpoint{3.778251in}{8.683395in}}{\pgfqpoint{3.789301in}{8.683395in}}%
\pgfpathlineto{\pgfqpoint{3.789301in}{8.683395in}}%
\pgfpathclose%
\pgfusepath{stroke,fill}%
\end{pgfscope}%
\begin{pgfscope}%
\pgfpathrectangle{\pgfqpoint{2.963410in}{7.624184in}}{\pgfqpoint{2.177280in}{2.201755in}}%
\pgfusepath{clip}%
\pgfsetbuttcap%
\pgfsetroundjoin%
\definecolor{currentfill}{rgb}{1.000000,0.498039,0.054902}%
\pgfsetfillcolor{currentfill}%
\pgfsetlinewidth{0.481800pt}%
\definecolor{currentstroke}{rgb}{1.000000,1.000000,1.000000}%
\pgfsetstrokecolor{currentstroke}%
\pgfsetdash{}{0pt}%
\pgfpathmoveto{\pgfqpoint{3.612075in}{8.794594in}}%
\pgfpathcurveto{\pgfqpoint{3.623126in}{8.794594in}}{\pgfqpoint{3.633725in}{8.798985in}}{\pgfqpoint{3.641538in}{8.806798in}}%
\pgfpathcurveto{\pgfqpoint{3.649352in}{8.814612in}}{\pgfqpoint{3.653742in}{8.825211in}}{\pgfqpoint{3.653742in}{8.836261in}}%
\pgfpathcurveto{\pgfqpoint{3.653742in}{8.847311in}}{\pgfqpoint{3.649352in}{8.857910in}}{\pgfqpoint{3.641538in}{8.865724in}}%
\pgfpathcurveto{\pgfqpoint{3.633725in}{8.873537in}}{\pgfqpoint{3.623126in}{8.877928in}}{\pgfqpoint{3.612075in}{8.877928in}}%
\pgfpathcurveto{\pgfqpoint{3.601025in}{8.877928in}}{\pgfqpoint{3.590426in}{8.873537in}}{\pgfqpoint{3.582613in}{8.865724in}}%
\pgfpathcurveto{\pgfqpoint{3.574799in}{8.857910in}}{\pgfqpoint{3.570409in}{8.847311in}}{\pgfqpoint{3.570409in}{8.836261in}}%
\pgfpathcurveto{\pgfqpoint{3.570409in}{8.825211in}}{\pgfqpoint{3.574799in}{8.814612in}}{\pgfqpoint{3.582613in}{8.806798in}}%
\pgfpathcurveto{\pgfqpoint{3.590426in}{8.798985in}}{\pgfqpoint{3.601025in}{8.794594in}}{\pgfqpoint{3.612075in}{8.794594in}}%
\pgfpathlineto{\pgfqpoint{3.612075in}{8.794594in}}%
\pgfpathclose%
\pgfusepath{stroke,fill}%
\end{pgfscope}%
\begin{pgfscope}%
\pgfpathrectangle{\pgfqpoint{2.963410in}{7.624184in}}{\pgfqpoint{2.177280in}{2.201755in}}%
\pgfusepath{clip}%
\pgfsetbuttcap%
\pgfsetroundjoin%
\definecolor{currentfill}{rgb}{1.000000,0.498039,0.054902}%
\pgfsetfillcolor{currentfill}%
\pgfsetlinewidth{0.481800pt}%
\definecolor{currentstroke}{rgb}{1.000000,1.000000,1.000000}%
\pgfsetstrokecolor{currentstroke}%
\pgfsetdash{}{0pt}%
\pgfpathmoveto{\pgfqpoint{3.789301in}{8.683395in}}%
\pgfpathcurveto{\pgfqpoint{3.800351in}{8.683395in}}{\pgfqpoint{3.810950in}{8.687785in}}{\pgfqpoint{3.818764in}{8.695598in}}%
\pgfpathcurveto{\pgfqpoint{3.826578in}{8.703412in}}{\pgfqpoint{3.830968in}{8.714011in}}{\pgfqpoint{3.830968in}{8.725061in}}%
\pgfpathcurveto{\pgfqpoint{3.830968in}{8.736111in}}{\pgfqpoint{3.826578in}{8.746710in}}{\pgfqpoint{3.818764in}{8.754524in}}%
\pgfpathcurveto{\pgfqpoint{3.810950in}{8.762338in}}{\pgfqpoint{3.800351in}{8.766728in}}{\pgfqpoint{3.789301in}{8.766728in}}%
\pgfpathcurveto{\pgfqpoint{3.778251in}{8.766728in}}{\pgfqpoint{3.767652in}{8.762338in}}{\pgfqpoint{3.759838in}{8.754524in}}%
\pgfpathcurveto{\pgfqpoint{3.752025in}{8.746710in}}{\pgfqpoint{3.747635in}{8.736111in}}{\pgfqpoint{3.747635in}{8.725061in}}%
\pgfpathcurveto{\pgfqpoint{3.747635in}{8.714011in}}{\pgfqpoint{3.752025in}{8.703412in}}{\pgfqpoint{3.759838in}{8.695598in}}%
\pgfpathcurveto{\pgfqpoint{3.767652in}{8.687785in}}{\pgfqpoint{3.778251in}{8.683395in}}{\pgfqpoint{3.789301in}{8.683395in}}%
\pgfpathlineto{\pgfqpoint{3.789301in}{8.683395in}}%
\pgfpathclose%
\pgfusepath{stroke,fill}%
\end{pgfscope}%
\begin{pgfscope}%
\pgfpathrectangle{\pgfqpoint{2.963410in}{7.624184in}}{\pgfqpoint{2.177280in}{2.201755in}}%
\pgfusepath{clip}%
\pgfsetbuttcap%
\pgfsetroundjoin%
\definecolor{currentfill}{rgb}{1.000000,0.498039,0.054902}%
\pgfsetfillcolor{currentfill}%
\pgfsetlinewidth{0.481800pt}%
\definecolor{currentstroke}{rgb}{1.000000,1.000000,1.000000}%
\pgfsetstrokecolor{currentstroke}%
\pgfsetdash{}{0pt}%
\pgfpathmoveto{\pgfqpoint{3.848376in}{8.850194in}}%
\pgfpathcurveto{\pgfqpoint{3.859427in}{8.850194in}}{\pgfqpoint{3.870026in}{8.854584in}}{\pgfqpoint{3.877839in}{8.862398in}}%
\pgfpathcurveto{\pgfqpoint{3.885653in}{8.870212in}}{\pgfqpoint{3.890043in}{8.880811in}}{\pgfqpoint{3.890043in}{8.891861in}}%
\pgfpathcurveto{\pgfqpoint{3.890043in}{8.902911in}}{\pgfqpoint{3.885653in}{8.913510in}}{\pgfqpoint{3.877839in}{8.921324in}}%
\pgfpathcurveto{\pgfqpoint{3.870026in}{8.929137in}}{\pgfqpoint{3.859427in}{8.933528in}}{\pgfqpoint{3.848376in}{8.933528in}}%
\pgfpathcurveto{\pgfqpoint{3.837326in}{8.933528in}}{\pgfqpoint{3.826727in}{8.929137in}}{\pgfqpoint{3.818914in}{8.921324in}}%
\pgfpathcurveto{\pgfqpoint{3.811100in}{8.913510in}}{\pgfqpoint{3.806710in}{8.902911in}}{\pgfqpoint{3.806710in}{8.891861in}}%
\pgfpathcurveto{\pgfqpoint{3.806710in}{8.880811in}}{\pgfqpoint{3.811100in}{8.870212in}}{\pgfqpoint{3.818914in}{8.862398in}}%
\pgfpathcurveto{\pgfqpoint{3.826727in}{8.854584in}}{\pgfqpoint{3.837326in}{8.850194in}}{\pgfqpoint{3.848376in}{8.850194in}}%
\pgfpathlineto{\pgfqpoint{3.848376in}{8.850194in}}%
\pgfpathclose%
\pgfusepath{stroke,fill}%
\end{pgfscope}%
\begin{pgfscope}%
\pgfpathrectangle{\pgfqpoint{2.963410in}{7.624184in}}{\pgfqpoint{2.177280in}{2.201755in}}%
\pgfusepath{clip}%
\pgfsetbuttcap%
\pgfsetroundjoin%
\definecolor{currentfill}{rgb}{1.000000,0.498039,0.054902}%
\pgfsetfillcolor{currentfill}%
\pgfsetlinewidth{0.481800pt}%
\definecolor{currentstroke}{rgb}{1.000000,1.000000,1.000000}%
\pgfsetstrokecolor{currentstroke}%
\pgfsetdash{}{0pt}%
\pgfpathmoveto{\pgfqpoint{3.907452in}{8.961394in}}%
\pgfpathcurveto{\pgfqpoint{3.918502in}{8.961394in}}{\pgfqpoint{3.929101in}{8.965784in}}{\pgfqpoint{3.936915in}{8.973598in}}%
\pgfpathcurveto{\pgfqpoint{3.944728in}{8.981411in}}{\pgfqpoint{3.949118in}{8.992011in}}{\pgfqpoint{3.949118in}{9.003061in}}%
\pgfpathcurveto{\pgfqpoint{3.949118in}{9.014111in}}{\pgfqpoint{3.944728in}{9.024710in}}{\pgfqpoint{3.936915in}{9.032523in}}%
\pgfpathcurveto{\pgfqpoint{3.929101in}{9.040337in}}{\pgfqpoint{3.918502in}{9.044727in}}{\pgfqpoint{3.907452in}{9.044727in}}%
\pgfpathcurveto{\pgfqpoint{3.896402in}{9.044727in}}{\pgfqpoint{3.885803in}{9.040337in}}{\pgfqpoint{3.877989in}{9.032523in}}%
\pgfpathcurveto{\pgfqpoint{3.870175in}{9.024710in}}{\pgfqpoint{3.865785in}{9.014111in}}{\pgfqpoint{3.865785in}{9.003061in}}%
\pgfpathcurveto{\pgfqpoint{3.865785in}{8.992011in}}{\pgfqpoint{3.870175in}{8.981411in}}{\pgfqpoint{3.877989in}{8.973598in}}%
\pgfpathcurveto{\pgfqpoint{3.885803in}{8.965784in}}{\pgfqpoint{3.896402in}{8.961394in}}{\pgfqpoint{3.907452in}{8.961394in}}%
\pgfpathlineto{\pgfqpoint{3.907452in}{8.961394in}}%
\pgfpathclose%
\pgfusepath{stroke,fill}%
\end{pgfscope}%
\begin{pgfscope}%
\pgfpathrectangle{\pgfqpoint{2.963410in}{7.624184in}}{\pgfqpoint{2.177280in}{2.201755in}}%
\pgfusepath{clip}%
\pgfsetbuttcap%
\pgfsetroundjoin%
\definecolor{currentfill}{rgb}{1.000000,0.498039,0.054902}%
\pgfsetfillcolor{currentfill}%
\pgfsetlinewidth{0.481800pt}%
\definecolor{currentstroke}{rgb}{1.000000,1.000000,1.000000}%
\pgfsetstrokecolor{currentstroke}%
\pgfsetdash{}{0pt}%
\pgfpathmoveto{\pgfqpoint{3.789301in}{9.072594in}}%
\pgfpathcurveto{\pgfqpoint{3.800351in}{9.072594in}}{\pgfqpoint{3.810950in}{9.076984in}}{\pgfqpoint{3.818764in}{9.084798in}}%
\pgfpathcurveto{\pgfqpoint{3.826578in}{9.092611in}}{\pgfqpoint{3.830968in}{9.103210in}}{\pgfqpoint{3.830968in}{9.114260in}}%
\pgfpathcurveto{\pgfqpoint{3.830968in}{9.125311in}}{\pgfqpoint{3.826578in}{9.135910in}}{\pgfqpoint{3.818764in}{9.143723in}}%
\pgfpathcurveto{\pgfqpoint{3.810950in}{9.151537in}}{\pgfqpoint{3.800351in}{9.155927in}}{\pgfqpoint{3.789301in}{9.155927in}}%
\pgfpathcurveto{\pgfqpoint{3.778251in}{9.155927in}}{\pgfqpoint{3.767652in}{9.151537in}}{\pgfqpoint{3.759838in}{9.143723in}}%
\pgfpathcurveto{\pgfqpoint{3.752025in}{9.135910in}}{\pgfqpoint{3.747635in}{9.125311in}}{\pgfqpoint{3.747635in}{9.114260in}}%
\pgfpathcurveto{\pgfqpoint{3.747635in}{9.103210in}}{\pgfqpoint{3.752025in}{9.092611in}}{\pgfqpoint{3.759838in}{9.084798in}}%
\pgfpathcurveto{\pgfqpoint{3.767652in}{9.076984in}}{\pgfqpoint{3.778251in}{9.072594in}}{\pgfqpoint{3.789301in}{9.072594in}}%
\pgfpathlineto{\pgfqpoint{3.789301in}{9.072594in}}%
\pgfpathclose%
\pgfusepath{stroke,fill}%
\end{pgfscope}%
\begin{pgfscope}%
\pgfpathrectangle{\pgfqpoint{2.963410in}{7.624184in}}{\pgfqpoint{2.177280in}{2.201755in}}%
\pgfusepath{clip}%
\pgfsetbuttcap%
\pgfsetroundjoin%
\definecolor{currentfill}{rgb}{1.000000,0.498039,0.054902}%
\pgfsetfillcolor{currentfill}%
\pgfsetlinewidth{0.481800pt}%
\definecolor{currentstroke}{rgb}{1.000000,1.000000,1.000000}%
\pgfsetstrokecolor{currentstroke}%
\pgfsetdash{}{0pt}%
\pgfpathmoveto{\pgfqpoint{3.907452in}{9.016994in}}%
\pgfpathcurveto{\pgfqpoint{3.918502in}{9.016994in}}{\pgfqpoint{3.929101in}{9.021384in}}{\pgfqpoint{3.936915in}{9.029198in}}%
\pgfpathcurveto{\pgfqpoint{3.944728in}{9.037011in}}{\pgfqpoint{3.949118in}{9.047610in}}{\pgfqpoint{3.949118in}{9.058661in}}%
\pgfpathcurveto{\pgfqpoint{3.949118in}{9.069711in}}{\pgfqpoint{3.944728in}{9.080310in}}{\pgfqpoint{3.936915in}{9.088123in}}%
\pgfpathcurveto{\pgfqpoint{3.929101in}{9.095937in}}{\pgfqpoint{3.918502in}{9.100327in}}{\pgfqpoint{3.907452in}{9.100327in}}%
\pgfpathcurveto{\pgfqpoint{3.896402in}{9.100327in}}{\pgfqpoint{3.885803in}{9.095937in}}{\pgfqpoint{3.877989in}{9.088123in}}%
\pgfpathcurveto{\pgfqpoint{3.870175in}{9.080310in}}{\pgfqpoint{3.865785in}{9.069711in}}{\pgfqpoint{3.865785in}{9.058661in}}%
\pgfpathcurveto{\pgfqpoint{3.865785in}{9.047610in}}{\pgfqpoint{3.870175in}{9.037011in}}{\pgfqpoint{3.877989in}{9.029198in}}%
\pgfpathcurveto{\pgfqpoint{3.885803in}{9.021384in}}{\pgfqpoint{3.896402in}{9.016994in}}{\pgfqpoint{3.907452in}{9.016994in}}%
\pgfpathlineto{\pgfqpoint{3.907452in}{9.016994in}}%
\pgfpathclose%
\pgfusepath{stroke,fill}%
\end{pgfscope}%
\begin{pgfscope}%
\pgfpathrectangle{\pgfqpoint{2.963410in}{7.624184in}}{\pgfqpoint{2.177280in}{2.201755in}}%
\pgfusepath{clip}%
\pgfsetbuttcap%
\pgfsetroundjoin%
\definecolor{currentfill}{rgb}{1.000000,0.498039,0.054902}%
\pgfsetfillcolor{currentfill}%
\pgfsetlinewidth{0.481800pt}%
\definecolor{currentstroke}{rgb}{1.000000,1.000000,1.000000}%
\pgfsetstrokecolor{currentstroke}%
\pgfsetdash{}{0pt}%
\pgfpathmoveto{\pgfqpoint{3.848376in}{8.627795in}}%
\pgfpathcurveto{\pgfqpoint{3.859427in}{8.627795in}}{\pgfqpoint{3.870026in}{8.632185in}}{\pgfqpoint{3.877839in}{8.639999in}}%
\pgfpathcurveto{\pgfqpoint{3.885653in}{8.647812in}}{\pgfqpoint{3.890043in}{8.658411in}}{\pgfqpoint{3.890043in}{8.669461in}}%
\pgfpathcurveto{\pgfqpoint{3.890043in}{8.680512in}}{\pgfqpoint{3.885653in}{8.691111in}}{\pgfqpoint{3.877839in}{8.698924in}}%
\pgfpathcurveto{\pgfqpoint{3.870026in}{8.706738in}}{\pgfqpoint{3.859427in}{8.711128in}}{\pgfqpoint{3.848376in}{8.711128in}}%
\pgfpathcurveto{\pgfqpoint{3.837326in}{8.711128in}}{\pgfqpoint{3.826727in}{8.706738in}}{\pgfqpoint{3.818914in}{8.698924in}}%
\pgfpathcurveto{\pgfqpoint{3.811100in}{8.691111in}}{\pgfqpoint{3.806710in}{8.680512in}}{\pgfqpoint{3.806710in}{8.669461in}}%
\pgfpathcurveto{\pgfqpoint{3.806710in}{8.658411in}}{\pgfqpoint{3.811100in}{8.647812in}}{\pgfqpoint{3.818914in}{8.639999in}}%
\pgfpathcurveto{\pgfqpoint{3.826727in}{8.632185in}}{\pgfqpoint{3.837326in}{8.627795in}}{\pgfqpoint{3.848376in}{8.627795in}}%
\pgfpathlineto{\pgfqpoint{3.848376in}{8.627795in}}%
\pgfpathclose%
\pgfusepath{stroke,fill}%
\end{pgfscope}%
\begin{pgfscope}%
\pgfpathrectangle{\pgfqpoint{2.963410in}{7.624184in}}{\pgfqpoint{2.177280in}{2.201755in}}%
\pgfusepath{clip}%
\pgfsetbuttcap%
\pgfsetroundjoin%
\definecolor{currentfill}{rgb}{1.000000,0.498039,0.054902}%
\pgfsetfillcolor{currentfill}%
\pgfsetlinewidth{0.481800pt}%
\definecolor{currentstroke}{rgb}{1.000000,1.000000,1.000000}%
\pgfsetstrokecolor{currentstroke}%
\pgfsetdash{}{0pt}%
\pgfpathmoveto{\pgfqpoint{3.671151in}{8.460995in}}%
\pgfpathcurveto{\pgfqpoint{3.682201in}{8.460995in}}{\pgfqpoint{3.692800in}{8.465385in}}{\pgfqpoint{3.700613in}{8.473199in}}%
\pgfpathcurveto{\pgfqpoint{3.708427in}{8.481013in}}{\pgfqpoint{3.712817in}{8.491612in}}{\pgfqpoint{3.712817in}{8.502662in}}%
\pgfpathcurveto{\pgfqpoint{3.712817in}{8.513712in}}{\pgfqpoint{3.708427in}{8.524311in}}{\pgfqpoint{3.700613in}{8.532125in}}%
\pgfpathcurveto{\pgfqpoint{3.692800in}{8.539938in}}{\pgfqpoint{3.682201in}{8.544328in}}{\pgfqpoint{3.671151in}{8.544328in}}%
\pgfpathcurveto{\pgfqpoint{3.660101in}{8.544328in}}{\pgfqpoint{3.649501in}{8.539938in}}{\pgfqpoint{3.641688in}{8.532125in}}%
\pgfpathcurveto{\pgfqpoint{3.633874in}{8.524311in}}{\pgfqpoint{3.629484in}{8.513712in}}{\pgfqpoint{3.629484in}{8.502662in}}%
\pgfpathcurveto{\pgfqpoint{3.629484in}{8.491612in}}{\pgfqpoint{3.633874in}{8.481013in}}{\pgfqpoint{3.641688in}{8.473199in}}%
\pgfpathcurveto{\pgfqpoint{3.649501in}{8.465385in}}{\pgfqpoint{3.660101in}{8.460995in}}{\pgfqpoint{3.671151in}{8.460995in}}%
\pgfpathlineto{\pgfqpoint{3.671151in}{8.460995in}}%
\pgfpathclose%
\pgfusepath{stroke,fill}%
\end{pgfscope}%
\begin{pgfscope}%
\pgfpathrectangle{\pgfqpoint{2.963410in}{7.624184in}}{\pgfqpoint{2.177280in}{2.201755in}}%
\pgfusepath{clip}%
\pgfsetbuttcap%
\pgfsetroundjoin%
\definecolor{currentfill}{rgb}{1.000000,0.498039,0.054902}%
\pgfsetfillcolor{currentfill}%
\pgfsetlinewidth{0.481800pt}%
\definecolor{currentstroke}{rgb}{1.000000,1.000000,1.000000}%
\pgfsetstrokecolor{currentstroke}%
\pgfsetdash{}{0pt}%
\pgfpathmoveto{\pgfqpoint{3.553000in}{8.349795in}}%
\pgfpathcurveto{\pgfqpoint{3.564050in}{8.349795in}}{\pgfqpoint{3.574649in}{8.354186in}}{\pgfqpoint{3.582463in}{8.361999in}}%
\pgfpathcurveto{\pgfqpoint{3.590277in}{8.369813in}}{\pgfqpoint{3.594667in}{8.380412in}}{\pgfqpoint{3.594667in}{8.391462in}}%
\pgfpathcurveto{\pgfqpoint{3.594667in}{8.402512in}}{\pgfqpoint{3.590277in}{8.413111in}}{\pgfqpoint{3.582463in}{8.420925in}}%
\pgfpathcurveto{\pgfqpoint{3.574649in}{8.428738in}}{\pgfqpoint{3.564050in}{8.433129in}}{\pgfqpoint{3.553000in}{8.433129in}}%
\pgfpathcurveto{\pgfqpoint{3.541950in}{8.433129in}}{\pgfqpoint{3.531351in}{8.428738in}}{\pgfqpoint{3.523537in}{8.420925in}}%
\pgfpathcurveto{\pgfqpoint{3.515724in}{8.413111in}}{\pgfqpoint{3.511333in}{8.402512in}}{\pgfqpoint{3.511333in}{8.391462in}}%
\pgfpathcurveto{\pgfqpoint{3.511333in}{8.380412in}}{\pgfqpoint{3.515724in}{8.369813in}}{\pgfqpoint{3.523537in}{8.361999in}}%
\pgfpathcurveto{\pgfqpoint{3.531351in}{8.354186in}}{\pgfqpoint{3.541950in}{8.349795in}}{\pgfqpoint{3.553000in}{8.349795in}}%
\pgfpathlineto{\pgfqpoint{3.553000in}{8.349795in}}%
\pgfpathclose%
\pgfusepath{stroke,fill}%
\end{pgfscope}%
\begin{pgfscope}%
\pgfpathrectangle{\pgfqpoint{2.963410in}{7.624184in}}{\pgfqpoint{2.177280in}{2.201755in}}%
\pgfusepath{clip}%
\pgfsetbuttcap%
\pgfsetroundjoin%
\definecolor{currentfill}{rgb}{1.000000,0.498039,0.054902}%
\pgfsetfillcolor{currentfill}%
\pgfsetlinewidth{0.481800pt}%
\definecolor{currentstroke}{rgb}{1.000000,1.000000,1.000000}%
\pgfsetstrokecolor{currentstroke}%
\pgfsetdash{}{0pt}%
\pgfpathmoveto{\pgfqpoint{3.553000in}{8.349795in}}%
\pgfpathcurveto{\pgfqpoint{3.564050in}{8.349795in}}{\pgfqpoint{3.574649in}{8.354186in}}{\pgfqpoint{3.582463in}{8.361999in}}%
\pgfpathcurveto{\pgfqpoint{3.590277in}{8.369813in}}{\pgfqpoint{3.594667in}{8.380412in}}{\pgfqpoint{3.594667in}{8.391462in}}%
\pgfpathcurveto{\pgfqpoint{3.594667in}{8.402512in}}{\pgfqpoint{3.590277in}{8.413111in}}{\pgfqpoint{3.582463in}{8.420925in}}%
\pgfpathcurveto{\pgfqpoint{3.574649in}{8.428738in}}{\pgfqpoint{3.564050in}{8.433129in}}{\pgfqpoint{3.553000in}{8.433129in}}%
\pgfpathcurveto{\pgfqpoint{3.541950in}{8.433129in}}{\pgfqpoint{3.531351in}{8.428738in}}{\pgfqpoint{3.523537in}{8.420925in}}%
\pgfpathcurveto{\pgfqpoint{3.515724in}{8.413111in}}{\pgfqpoint{3.511333in}{8.402512in}}{\pgfqpoint{3.511333in}{8.391462in}}%
\pgfpathcurveto{\pgfqpoint{3.511333in}{8.380412in}}{\pgfqpoint{3.515724in}{8.369813in}}{\pgfqpoint{3.523537in}{8.361999in}}%
\pgfpathcurveto{\pgfqpoint{3.531351in}{8.354186in}}{\pgfqpoint{3.541950in}{8.349795in}}{\pgfqpoint{3.553000in}{8.349795in}}%
\pgfpathlineto{\pgfqpoint{3.553000in}{8.349795in}}%
\pgfpathclose%
\pgfusepath{stroke,fill}%
\end{pgfscope}%
\begin{pgfscope}%
\pgfpathrectangle{\pgfqpoint{2.963410in}{7.624184in}}{\pgfqpoint{2.177280in}{2.201755in}}%
\pgfusepath{clip}%
\pgfsetbuttcap%
\pgfsetroundjoin%
\definecolor{currentfill}{rgb}{1.000000,0.498039,0.054902}%
\pgfsetfillcolor{currentfill}%
\pgfsetlinewidth{0.481800pt}%
\definecolor{currentstroke}{rgb}{1.000000,1.000000,1.000000}%
\pgfsetstrokecolor{currentstroke}%
\pgfsetdash{}{0pt}%
\pgfpathmoveto{\pgfqpoint{3.730226in}{8.516595in}}%
\pgfpathcurveto{\pgfqpoint{3.741276in}{8.516595in}}{\pgfqpoint{3.751875in}{8.520985in}}{\pgfqpoint{3.759689in}{8.528799in}}%
\pgfpathcurveto{\pgfqpoint{3.767502in}{8.536612in}}{\pgfqpoint{3.771893in}{8.547212in}}{\pgfqpoint{3.771893in}{8.558262in}}%
\pgfpathcurveto{\pgfqpoint{3.771893in}{8.569312in}}{\pgfqpoint{3.767502in}{8.579911in}}{\pgfqpoint{3.759689in}{8.587724in}}%
\pgfpathcurveto{\pgfqpoint{3.751875in}{8.595538in}}{\pgfqpoint{3.741276in}{8.599928in}}{\pgfqpoint{3.730226in}{8.599928in}}%
\pgfpathcurveto{\pgfqpoint{3.719176in}{8.599928in}}{\pgfqpoint{3.708577in}{8.595538in}}{\pgfqpoint{3.700763in}{8.587724in}}%
\pgfpathcurveto{\pgfqpoint{3.692950in}{8.579911in}}{\pgfqpoint{3.688559in}{8.569312in}}{\pgfqpoint{3.688559in}{8.558262in}}%
\pgfpathcurveto{\pgfqpoint{3.688559in}{8.547212in}}{\pgfqpoint{3.692950in}{8.536612in}}{\pgfqpoint{3.700763in}{8.528799in}}%
\pgfpathcurveto{\pgfqpoint{3.708577in}{8.520985in}}{\pgfqpoint{3.719176in}{8.516595in}}{\pgfqpoint{3.730226in}{8.516595in}}%
\pgfpathlineto{\pgfqpoint{3.730226in}{8.516595in}}%
\pgfpathclose%
\pgfusepath{stroke,fill}%
\end{pgfscope}%
\begin{pgfscope}%
\pgfpathrectangle{\pgfqpoint{2.963410in}{7.624184in}}{\pgfqpoint{2.177280in}{2.201755in}}%
\pgfusepath{clip}%
\pgfsetbuttcap%
\pgfsetroundjoin%
\definecolor{currentfill}{rgb}{1.000000,0.498039,0.054902}%
\pgfsetfillcolor{currentfill}%
\pgfsetlinewidth{0.481800pt}%
\definecolor{currentstroke}{rgb}{1.000000,1.000000,1.000000}%
\pgfsetstrokecolor{currentstroke}%
\pgfsetdash{}{0pt}%
\pgfpathmoveto{\pgfqpoint{3.730226in}{8.627795in}}%
\pgfpathcurveto{\pgfqpoint{3.741276in}{8.627795in}}{\pgfqpoint{3.751875in}{8.632185in}}{\pgfqpoint{3.759689in}{8.639999in}}%
\pgfpathcurveto{\pgfqpoint{3.767502in}{8.647812in}}{\pgfqpoint{3.771893in}{8.658411in}}{\pgfqpoint{3.771893in}{8.669461in}}%
\pgfpathcurveto{\pgfqpoint{3.771893in}{8.680512in}}{\pgfqpoint{3.767502in}{8.691111in}}{\pgfqpoint{3.759689in}{8.698924in}}%
\pgfpathcurveto{\pgfqpoint{3.751875in}{8.706738in}}{\pgfqpoint{3.741276in}{8.711128in}}{\pgfqpoint{3.730226in}{8.711128in}}%
\pgfpathcurveto{\pgfqpoint{3.719176in}{8.711128in}}{\pgfqpoint{3.708577in}{8.706738in}}{\pgfqpoint{3.700763in}{8.698924in}}%
\pgfpathcurveto{\pgfqpoint{3.692950in}{8.691111in}}{\pgfqpoint{3.688559in}{8.680512in}}{\pgfqpoint{3.688559in}{8.669461in}}%
\pgfpathcurveto{\pgfqpoint{3.688559in}{8.658411in}}{\pgfqpoint{3.692950in}{8.647812in}}{\pgfqpoint{3.700763in}{8.639999in}}%
\pgfpathcurveto{\pgfqpoint{3.708577in}{8.632185in}}{\pgfqpoint{3.719176in}{8.627795in}}{\pgfqpoint{3.730226in}{8.627795in}}%
\pgfpathlineto{\pgfqpoint{3.730226in}{8.627795in}}%
\pgfpathclose%
\pgfusepath{stroke,fill}%
\end{pgfscope}%
\begin{pgfscope}%
\pgfpathrectangle{\pgfqpoint{2.963410in}{7.624184in}}{\pgfqpoint{2.177280in}{2.201755in}}%
\pgfusepath{clip}%
\pgfsetbuttcap%
\pgfsetroundjoin%
\definecolor{currentfill}{rgb}{1.000000,0.498039,0.054902}%
\pgfsetfillcolor{currentfill}%
\pgfsetlinewidth{0.481800pt}%
\definecolor{currentstroke}{rgb}{1.000000,1.000000,1.000000}%
\pgfsetstrokecolor{currentstroke}%
\pgfsetdash{}{0pt}%
\pgfpathmoveto{\pgfqpoint{3.907452in}{8.294195in}}%
\pgfpathcurveto{\pgfqpoint{3.918502in}{8.294195in}}{\pgfqpoint{3.929101in}{8.298586in}}{\pgfqpoint{3.936915in}{8.306399in}}%
\pgfpathcurveto{\pgfqpoint{3.944728in}{8.314213in}}{\pgfqpoint{3.949118in}{8.324812in}}{\pgfqpoint{3.949118in}{8.335862in}}%
\pgfpathcurveto{\pgfqpoint{3.949118in}{8.346912in}}{\pgfqpoint{3.944728in}{8.357511in}}{\pgfqpoint{3.936915in}{8.365325in}}%
\pgfpathcurveto{\pgfqpoint{3.929101in}{8.373139in}}{\pgfqpoint{3.918502in}{8.377529in}}{\pgfqpoint{3.907452in}{8.377529in}}%
\pgfpathcurveto{\pgfqpoint{3.896402in}{8.377529in}}{\pgfqpoint{3.885803in}{8.373139in}}{\pgfqpoint{3.877989in}{8.365325in}}%
\pgfpathcurveto{\pgfqpoint{3.870175in}{8.357511in}}{\pgfqpoint{3.865785in}{8.346912in}}{\pgfqpoint{3.865785in}{8.335862in}}%
\pgfpathcurveto{\pgfqpoint{3.865785in}{8.324812in}}{\pgfqpoint{3.870175in}{8.314213in}}{\pgfqpoint{3.877989in}{8.306399in}}%
\pgfpathcurveto{\pgfqpoint{3.885803in}{8.298586in}}{\pgfqpoint{3.896402in}{8.294195in}}{\pgfqpoint{3.907452in}{8.294195in}}%
\pgfpathlineto{\pgfqpoint{3.907452in}{8.294195in}}%
\pgfpathclose%
\pgfusepath{stroke,fill}%
\end{pgfscope}%
\begin{pgfscope}%
\pgfpathrectangle{\pgfqpoint{2.963410in}{7.624184in}}{\pgfqpoint{2.177280in}{2.201755in}}%
\pgfusepath{clip}%
\pgfsetbuttcap%
\pgfsetroundjoin%
\definecolor{currentfill}{rgb}{1.000000,0.498039,0.054902}%
\pgfsetfillcolor{currentfill}%
\pgfsetlinewidth{0.481800pt}%
\definecolor{currentstroke}{rgb}{1.000000,1.000000,1.000000}%
\pgfsetstrokecolor{currentstroke}%
\pgfsetdash{}{0pt}%
\pgfpathmoveto{\pgfqpoint{4.143753in}{8.627795in}}%
\pgfpathcurveto{\pgfqpoint{4.154803in}{8.627795in}}{\pgfqpoint{4.165402in}{8.632185in}}{\pgfqpoint{4.173216in}{8.639999in}}%
\pgfpathcurveto{\pgfqpoint{4.181029in}{8.647812in}}{\pgfqpoint{4.185419in}{8.658411in}}{\pgfqpoint{4.185419in}{8.669461in}}%
\pgfpathcurveto{\pgfqpoint{4.185419in}{8.680512in}}{\pgfqpoint{4.181029in}{8.691111in}}{\pgfqpoint{4.173216in}{8.698924in}}%
\pgfpathcurveto{\pgfqpoint{4.165402in}{8.706738in}}{\pgfqpoint{4.154803in}{8.711128in}}{\pgfqpoint{4.143753in}{8.711128in}}%
\pgfpathcurveto{\pgfqpoint{4.132703in}{8.711128in}}{\pgfqpoint{4.122104in}{8.706738in}}{\pgfqpoint{4.114290in}{8.698924in}}%
\pgfpathcurveto{\pgfqpoint{4.106476in}{8.691111in}}{\pgfqpoint{4.102086in}{8.680512in}}{\pgfqpoint{4.102086in}{8.669461in}}%
\pgfpathcurveto{\pgfqpoint{4.102086in}{8.658411in}}{\pgfqpoint{4.106476in}{8.647812in}}{\pgfqpoint{4.114290in}{8.639999in}}%
\pgfpathcurveto{\pgfqpoint{4.122104in}{8.632185in}}{\pgfqpoint{4.132703in}{8.627795in}}{\pgfqpoint{4.143753in}{8.627795in}}%
\pgfpathlineto{\pgfqpoint{4.143753in}{8.627795in}}%
\pgfpathclose%
\pgfusepath{stroke,fill}%
\end{pgfscope}%
\begin{pgfscope}%
\pgfpathrectangle{\pgfqpoint{2.963410in}{7.624184in}}{\pgfqpoint{2.177280in}{2.201755in}}%
\pgfusepath{clip}%
\pgfsetbuttcap%
\pgfsetroundjoin%
\definecolor{currentfill}{rgb}{1.000000,0.498039,0.054902}%
\pgfsetfillcolor{currentfill}%
\pgfsetlinewidth{0.481800pt}%
\definecolor{currentstroke}{rgb}{1.000000,1.000000,1.000000}%
\pgfsetstrokecolor{currentstroke}%
\pgfsetdash{}{0pt}%
\pgfpathmoveto{\pgfqpoint{3.966527in}{9.016994in}}%
\pgfpathcurveto{\pgfqpoint{3.977577in}{9.016994in}}{\pgfqpoint{3.988176in}{9.021384in}}{\pgfqpoint{3.995990in}{9.029198in}}%
\pgfpathcurveto{\pgfqpoint{4.003803in}{9.037011in}}{\pgfqpoint{4.008194in}{9.047610in}}{\pgfqpoint{4.008194in}{9.058661in}}%
\pgfpathcurveto{\pgfqpoint{4.008194in}{9.069711in}}{\pgfqpoint{4.003803in}{9.080310in}}{\pgfqpoint{3.995990in}{9.088123in}}%
\pgfpathcurveto{\pgfqpoint{3.988176in}{9.095937in}}{\pgfqpoint{3.977577in}{9.100327in}}{\pgfqpoint{3.966527in}{9.100327in}}%
\pgfpathcurveto{\pgfqpoint{3.955477in}{9.100327in}}{\pgfqpoint{3.944878in}{9.095937in}}{\pgfqpoint{3.937064in}{9.088123in}}%
\pgfpathcurveto{\pgfqpoint{3.929251in}{9.080310in}}{\pgfqpoint{3.924860in}{9.069711in}}{\pgfqpoint{3.924860in}{9.058661in}}%
\pgfpathcurveto{\pgfqpoint{3.924860in}{9.047610in}}{\pgfqpoint{3.929251in}{9.037011in}}{\pgfqpoint{3.937064in}{9.029198in}}%
\pgfpathcurveto{\pgfqpoint{3.944878in}{9.021384in}}{\pgfqpoint{3.955477in}{9.016994in}}{\pgfqpoint{3.966527in}{9.016994in}}%
\pgfpathlineto{\pgfqpoint{3.966527in}{9.016994in}}%
\pgfpathclose%
\pgfusepath{stroke,fill}%
\end{pgfscope}%
\begin{pgfscope}%
\pgfpathrectangle{\pgfqpoint{2.963410in}{7.624184in}}{\pgfqpoint{2.177280in}{2.201755in}}%
\pgfusepath{clip}%
\pgfsetbuttcap%
\pgfsetroundjoin%
\definecolor{currentfill}{rgb}{1.000000,0.498039,0.054902}%
\pgfsetfillcolor{currentfill}%
\pgfsetlinewidth{0.481800pt}%
\definecolor{currentstroke}{rgb}{1.000000,1.000000,1.000000}%
\pgfsetstrokecolor{currentstroke}%
\pgfsetdash{}{0pt}%
\pgfpathmoveto{\pgfqpoint{3.493925in}{8.794594in}}%
\pgfpathcurveto{\pgfqpoint{3.504975in}{8.794594in}}{\pgfqpoint{3.515574in}{8.798985in}}{\pgfqpoint{3.523388in}{8.806798in}}%
\pgfpathcurveto{\pgfqpoint{3.531201in}{8.814612in}}{\pgfqpoint{3.535592in}{8.825211in}}{\pgfqpoint{3.535592in}{8.836261in}}%
\pgfpathcurveto{\pgfqpoint{3.535592in}{8.847311in}}{\pgfqpoint{3.531201in}{8.857910in}}{\pgfqpoint{3.523388in}{8.865724in}}%
\pgfpathcurveto{\pgfqpoint{3.515574in}{8.873537in}}{\pgfqpoint{3.504975in}{8.877928in}}{\pgfqpoint{3.493925in}{8.877928in}}%
\pgfpathcurveto{\pgfqpoint{3.482875in}{8.877928in}}{\pgfqpoint{3.472276in}{8.873537in}}{\pgfqpoint{3.464462in}{8.865724in}}%
\pgfpathcurveto{\pgfqpoint{3.456648in}{8.857910in}}{\pgfqpoint{3.452258in}{8.847311in}}{\pgfqpoint{3.452258in}{8.836261in}}%
\pgfpathcurveto{\pgfqpoint{3.452258in}{8.825211in}}{\pgfqpoint{3.456648in}{8.814612in}}{\pgfqpoint{3.464462in}{8.806798in}}%
\pgfpathcurveto{\pgfqpoint{3.472276in}{8.798985in}}{\pgfqpoint{3.482875in}{8.794594in}}{\pgfqpoint{3.493925in}{8.794594in}}%
\pgfpathlineto{\pgfqpoint{3.493925in}{8.794594in}}%
\pgfpathclose%
\pgfusepath{stroke,fill}%
\end{pgfscope}%
\begin{pgfscope}%
\pgfpathrectangle{\pgfqpoint{2.963410in}{7.624184in}}{\pgfqpoint{2.177280in}{2.201755in}}%
\pgfusepath{clip}%
\pgfsetbuttcap%
\pgfsetroundjoin%
\definecolor{currentfill}{rgb}{1.000000,0.498039,0.054902}%
\pgfsetfillcolor{currentfill}%
\pgfsetlinewidth{0.481800pt}%
\definecolor{currentstroke}{rgb}{1.000000,1.000000,1.000000}%
\pgfsetstrokecolor{currentstroke}%
\pgfsetdash{}{0pt}%
\pgfpathmoveto{\pgfqpoint{3.907452in}{8.405395in}}%
\pgfpathcurveto{\pgfqpoint{3.918502in}{8.405395in}}{\pgfqpoint{3.929101in}{8.409785in}}{\pgfqpoint{3.936915in}{8.417599in}}%
\pgfpathcurveto{\pgfqpoint{3.944728in}{8.425413in}}{\pgfqpoint{3.949118in}{8.436012in}}{\pgfqpoint{3.949118in}{8.447062in}}%
\pgfpathcurveto{\pgfqpoint{3.949118in}{8.458112in}}{\pgfqpoint{3.944728in}{8.468711in}}{\pgfqpoint{3.936915in}{8.476525in}}%
\pgfpathcurveto{\pgfqpoint{3.929101in}{8.484338in}}{\pgfqpoint{3.918502in}{8.488729in}}{\pgfqpoint{3.907452in}{8.488729in}}%
\pgfpathcurveto{\pgfqpoint{3.896402in}{8.488729in}}{\pgfqpoint{3.885803in}{8.484338in}}{\pgfqpoint{3.877989in}{8.476525in}}%
\pgfpathcurveto{\pgfqpoint{3.870175in}{8.468711in}}{\pgfqpoint{3.865785in}{8.458112in}}{\pgfqpoint{3.865785in}{8.447062in}}%
\pgfpathcurveto{\pgfqpoint{3.865785in}{8.436012in}}{\pgfqpoint{3.870175in}{8.425413in}}{\pgfqpoint{3.877989in}{8.417599in}}%
\pgfpathcurveto{\pgfqpoint{3.885803in}{8.409785in}}{\pgfqpoint{3.896402in}{8.405395in}}{\pgfqpoint{3.907452in}{8.405395in}}%
\pgfpathlineto{\pgfqpoint{3.907452in}{8.405395in}}%
\pgfpathclose%
\pgfusepath{stroke,fill}%
\end{pgfscope}%
\begin{pgfscope}%
\pgfpathrectangle{\pgfqpoint{2.963410in}{7.624184in}}{\pgfqpoint{2.177280in}{2.201755in}}%
\pgfusepath{clip}%
\pgfsetbuttcap%
\pgfsetroundjoin%
\definecolor{currentfill}{rgb}{1.000000,0.498039,0.054902}%
\pgfsetfillcolor{currentfill}%
\pgfsetlinewidth{0.481800pt}%
\definecolor{currentstroke}{rgb}{1.000000,1.000000,1.000000}%
\pgfsetstrokecolor{currentstroke}%
\pgfsetdash{}{0pt}%
\pgfpathmoveto{\pgfqpoint{3.612075in}{8.349795in}}%
\pgfpathcurveto{\pgfqpoint{3.623126in}{8.349795in}}{\pgfqpoint{3.633725in}{8.354186in}}{\pgfqpoint{3.641538in}{8.361999in}}%
\pgfpathcurveto{\pgfqpoint{3.649352in}{8.369813in}}{\pgfqpoint{3.653742in}{8.380412in}}{\pgfqpoint{3.653742in}{8.391462in}}%
\pgfpathcurveto{\pgfqpoint{3.653742in}{8.402512in}}{\pgfqpoint{3.649352in}{8.413111in}}{\pgfqpoint{3.641538in}{8.420925in}}%
\pgfpathcurveto{\pgfqpoint{3.633725in}{8.428738in}}{\pgfqpoint{3.623126in}{8.433129in}}{\pgfqpoint{3.612075in}{8.433129in}}%
\pgfpathcurveto{\pgfqpoint{3.601025in}{8.433129in}}{\pgfqpoint{3.590426in}{8.428738in}}{\pgfqpoint{3.582613in}{8.420925in}}%
\pgfpathcurveto{\pgfqpoint{3.574799in}{8.413111in}}{\pgfqpoint{3.570409in}{8.402512in}}{\pgfqpoint{3.570409in}{8.391462in}}%
\pgfpathcurveto{\pgfqpoint{3.570409in}{8.380412in}}{\pgfqpoint{3.574799in}{8.369813in}}{\pgfqpoint{3.582613in}{8.361999in}}%
\pgfpathcurveto{\pgfqpoint{3.590426in}{8.354186in}}{\pgfqpoint{3.601025in}{8.349795in}}{\pgfqpoint{3.612075in}{8.349795in}}%
\pgfpathlineto{\pgfqpoint{3.612075in}{8.349795in}}%
\pgfpathclose%
\pgfusepath{stroke,fill}%
\end{pgfscope}%
\begin{pgfscope}%
\pgfpathrectangle{\pgfqpoint{2.963410in}{7.624184in}}{\pgfqpoint{2.177280in}{2.201755in}}%
\pgfusepath{clip}%
\pgfsetbuttcap%
\pgfsetroundjoin%
\definecolor{currentfill}{rgb}{1.000000,0.498039,0.054902}%
\pgfsetfillcolor{currentfill}%
\pgfsetlinewidth{0.481800pt}%
\definecolor{currentstroke}{rgb}{1.000000,1.000000,1.000000}%
\pgfsetstrokecolor{currentstroke}%
\pgfsetdash{}{0pt}%
\pgfpathmoveto{\pgfqpoint{3.671151in}{8.349795in}}%
\pgfpathcurveto{\pgfqpoint{3.682201in}{8.349795in}}{\pgfqpoint{3.692800in}{8.354186in}}{\pgfqpoint{3.700613in}{8.361999in}}%
\pgfpathcurveto{\pgfqpoint{3.708427in}{8.369813in}}{\pgfqpoint{3.712817in}{8.380412in}}{\pgfqpoint{3.712817in}{8.391462in}}%
\pgfpathcurveto{\pgfqpoint{3.712817in}{8.402512in}}{\pgfqpoint{3.708427in}{8.413111in}}{\pgfqpoint{3.700613in}{8.420925in}}%
\pgfpathcurveto{\pgfqpoint{3.692800in}{8.428738in}}{\pgfqpoint{3.682201in}{8.433129in}}{\pgfqpoint{3.671151in}{8.433129in}}%
\pgfpathcurveto{\pgfqpoint{3.660101in}{8.433129in}}{\pgfqpoint{3.649501in}{8.428738in}}{\pgfqpoint{3.641688in}{8.420925in}}%
\pgfpathcurveto{\pgfqpoint{3.633874in}{8.413111in}}{\pgfqpoint{3.629484in}{8.402512in}}{\pgfqpoint{3.629484in}{8.391462in}}%
\pgfpathcurveto{\pgfqpoint{3.629484in}{8.380412in}}{\pgfqpoint{3.633874in}{8.369813in}}{\pgfqpoint{3.641688in}{8.361999in}}%
\pgfpathcurveto{\pgfqpoint{3.649501in}{8.354186in}}{\pgfqpoint{3.660101in}{8.349795in}}{\pgfqpoint{3.671151in}{8.349795in}}%
\pgfpathlineto{\pgfqpoint{3.671151in}{8.349795in}}%
\pgfpathclose%
\pgfusepath{stroke,fill}%
\end{pgfscope}%
\begin{pgfscope}%
\pgfpathrectangle{\pgfqpoint{2.963410in}{7.624184in}}{\pgfqpoint{2.177280in}{2.201755in}}%
\pgfusepath{clip}%
\pgfsetbuttcap%
\pgfsetroundjoin%
\definecolor{currentfill}{rgb}{1.000000,0.498039,0.054902}%
\pgfsetfillcolor{currentfill}%
\pgfsetlinewidth{0.481800pt}%
\definecolor{currentstroke}{rgb}{1.000000,1.000000,1.000000}%
\pgfsetstrokecolor{currentstroke}%
\pgfsetdash{}{0pt}%
\pgfpathmoveto{\pgfqpoint{3.907452in}{8.683395in}}%
\pgfpathcurveto{\pgfqpoint{3.918502in}{8.683395in}}{\pgfqpoint{3.929101in}{8.687785in}}{\pgfqpoint{3.936915in}{8.695598in}}%
\pgfpathcurveto{\pgfqpoint{3.944728in}{8.703412in}}{\pgfqpoint{3.949118in}{8.714011in}}{\pgfqpoint{3.949118in}{8.725061in}}%
\pgfpathcurveto{\pgfqpoint{3.949118in}{8.736111in}}{\pgfqpoint{3.944728in}{8.746710in}}{\pgfqpoint{3.936915in}{8.754524in}}%
\pgfpathcurveto{\pgfqpoint{3.929101in}{8.762338in}}{\pgfqpoint{3.918502in}{8.766728in}}{\pgfqpoint{3.907452in}{8.766728in}}%
\pgfpathcurveto{\pgfqpoint{3.896402in}{8.766728in}}{\pgfqpoint{3.885803in}{8.762338in}}{\pgfqpoint{3.877989in}{8.754524in}}%
\pgfpathcurveto{\pgfqpoint{3.870175in}{8.746710in}}{\pgfqpoint{3.865785in}{8.736111in}}{\pgfqpoint{3.865785in}{8.725061in}}%
\pgfpathcurveto{\pgfqpoint{3.865785in}{8.714011in}}{\pgfqpoint{3.870175in}{8.703412in}}{\pgfqpoint{3.877989in}{8.695598in}}%
\pgfpathcurveto{\pgfqpoint{3.885803in}{8.687785in}}{\pgfqpoint{3.896402in}{8.683395in}}{\pgfqpoint{3.907452in}{8.683395in}}%
\pgfpathlineto{\pgfqpoint{3.907452in}{8.683395in}}%
\pgfpathclose%
\pgfusepath{stroke,fill}%
\end{pgfscope}%
\begin{pgfscope}%
\pgfpathrectangle{\pgfqpoint{2.963410in}{7.624184in}}{\pgfqpoint{2.177280in}{2.201755in}}%
\pgfusepath{clip}%
\pgfsetbuttcap%
\pgfsetroundjoin%
\definecolor{currentfill}{rgb}{1.000000,0.498039,0.054902}%
\pgfsetfillcolor{currentfill}%
\pgfsetlinewidth{0.481800pt}%
\definecolor{currentstroke}{rgb}{1.000000,1.000000,1.000000}%
\pgfsetstrokecolor{currentstroke}%
\pgfsetdash{}{0pt}%
\pgfpathmoveto{\pgfqpoint{3.671151in}{8.516595in}}%
\pgfpathcurveto{\pgfqpoint{3.682201in}{8.516595in}}{\pgfqpoint{3.692800in}{8.520985in}}{\pgfqpoint{3.700613in}{8.528799in}}%
\pgfpathcurveto{\pgfqpoint{3.708427in}{8.536612in}}{\pgfqpoint{3.712817in}{8.547212in}}{\pgfqpoint{3.712817in}{8.558262in}}%
\pgfpathcurveto{\pgfqpoint{3.712817in}{8.569312in}}{\pgfqpoint{3.708427in}{8.579911in}}{\pgfqpoint{3.700613in}{8.587724in}}%
\pgfpathcurveto{\pgfqpoint{3.692800in}{8.595538in}}{\pgfqpoint{3.682201in}{8.599928in}}{\pgfqpoint{3.671151in}{8.599928in}}%
\pgfpathcurveto{\pgfqpoint{3.660101in}{8.599928in}}{\pgfqpoint{3.649501in}{8.595538in}}{\pgfqpoint{3.641688in}{8.587724in}}%
\pgfpathcurveto{\pgfqpoint{3.633874in}{8.579911in}}{\pgfqpoint{3.629484in}{8.569312in}}{\pgfqpoint{3.629484in}{8.558262in}}%
\pgfpathcurveto{\pgfqpoint{3.629484in}{8.547212in}}{\pgfqpoint{3.633874in}{8.536612in}}{\pgfqpoint{3.641688in}{8.528799in}}%
\pgfpathcurveto{\pgfqpoint{3.649501in}{8.520985in}}{\pgfqpoint{3.660101in}{8.516595in}}{\pgfqpoint{3.671151in}{8.516595in}}%
\pgfpathlineto{\pgfqpoint{3.671151in}{8.516595in}}%
\pgfpathclose%
\pgfusepath{stroke,fill}%
\end{pgfscope}%
\begin{pgfscope}%
\pgfpathrectangle{\pgfqpoint{2.963410in}{7.624184in}}{\pgfqpoint{2.177280in}{2.201755in}}%
\pgfusepath{clip}%
\pgfsetbuttcap%
\pgfsetroundjoin%
\definecolor{currentfill}{rgb}{1.000000,0.498039,0.054902}%
\pgfsetfillcolor{currentfill}%
\pgfsetlinewidth{0.481800pt}%
\definecolor{currentstroke}{rgb}{1.000000,1.000000,1.000000}%
\pgfsetstrokecolor{currentstroke}%
\pgfsetdash{}{0pt}%
\pgfpathmoveto{\pgfqpoint{3.493925in}{8.071796in}}%
\pgfpathcurveto{\pgfqpoint{3.504975in}{8.071796in}}{\pgfqpoint{3.515574in}{8.076186in}}{\pgfqpoint{3.523388in}{8.084000in}}%
\pgfpathcurveto{\pgfqpoint{3.531201in}{8.091813in}}{\pgfqpoint{3.535592in}{8.102413in}}{\pgfqpoint{3.535592in}{8.113463in}}%
\pgfpathcurveto{\pgfqpoint{3.535592in}{8.124513in}}{\pgfqpoint{3.531201in}{8.135112in}}{\pgfqpoint{3.523388in}{8.142925in}}%
\pgfpathcurveto{\pgfqpoint{3.515574in}{8.150739in}}{\pgfqpoint{3.504975in}{8.155129in}}{\pgfqpoint{3.493925in}{8.155129in}}%
\pgfpathcurveto{\pgfqpoint{3.482875in}{8.155129in}}{\pgfqpoint{3.472276in}{8.150739in}}{\pgfqpoint{3.464462in}{8.142925in}}%
\pgfpathcurveto{\pgfqpoint{3.456648in}{8.135112in}}{\pgfqpoint{3.452258in}{8.124513in}}{\pgfqpoint{3.452258in}{8.113463in}}%
\pgfpathcurveto{\pgfqpoint{3.452258in}{8.102413in}}{\pgfqpoint{3.456648in}{8.091813in}}{\pgfqpoint{3.464462in}{8.084000in}}%
\pgfpathcurveto{\pgfqpoint{3.472276in}{8.076186in}}{\pgfqpoint{3.482875in}{8.071796in}}{\pgfqpoint{3.493925in}{8.071796in}}%
\pgfpathlineto{\pgfqpoint{3.493925in}{8.071796in}}%
\pgfpathclose%
\pgfusepath{stroke,fill}%
\end{pgfscope}%
\begin{pgfscope}%
\pgfpathrectangle{\pgfqpoint{2.963410in}{7.624184in}}{\pgfqpoint{2.177280in}{2.201755in}}%
\pgfusepath{clip}%
\pgfsetbuttcap%
\pgfsetroundjoin%
\definecolor{currentfill}{rgb}{1.000000,0.498039,0.054902}%
\pgfsetfillcolor{currentfill}%
\pgfsetlinewidth{0.481800pt}%
\definecolor{currentstroke}{rgb}{1.000000,1.000000,1.000000}%
\pgfsetstrokecolor{currentstroke}%
\pgfsetdash{}{0pt}%
\pgfpathmoveto{\pgfqpoint{3.730226in}{8.405395in}}%
\pgfpathcurveto{\pgfqpoint{3.741276in}{8.405395in}}{\pgfqpoint{3.751875in}{8.409785in}}{\pgfqpoint{3.759689in}{8.417599in}}%
\pgfpathcurveto{\pgfqpoint{3.767502in}{8.425413in}}{\pgfqpoint{3.771893in}{8.436012in}}{\pgfqpoint{3.771893in}{8.447062in}}%
\pgfpathcurveto{\pgfqpoint{3.771893in}{8.458112in}}{\pgfqpoint{3.767502in}{8.468711in}}{\pgfqpoint{3.759689in}{8.476525in}}%
\pgfpathcurveto{\pgfqpoint{3.751875in}{8.484338in}}{\pgfqpoint{3.741276in}{8.488729in}}{\pgfqpoint{3.730226in}{8.488729in}}%
\pgfpathcurveto{\pgfqpoint{3.719176in}{8.488729in}}{\pgfqpoint{3.708577in}{8.484338in}}{\pgfqpoint{3.700763in}{8.476525in}}%
\pgfpathcurveto{\pgfqpoint{3.692950in}{8.468711in}}{\pgfqpoint{3.688559in}{8.458112in}}{\pgfqpoint{3.688559in}{8.447062in}}%
\pgfpathcurveto{\pgfqpoint{3.688559in}{8.436012in}}{\pgfqpoint{3.692950in}{8.425413in}}{\pgfqpoint{3.700763in}{8.417599in}}%
\pgfpathcurveto{\pgfqpoint{3.708577in}{8.409785in}}{\pgfqpoint{3.719176in}{8.405395in}}{\pgfqpoint{3.730226in}{8.405395in}}%
\pgfpathlineto{\pgfqpoint{3.730226in}{8.405395in}}%
\pgfpathclose%
\pgfusepath{stroke,fill}%
\end{pgfscope}%
\begin{pgfscope}%
\pgfpathrectangle{\pgfqpoint{2.963410in}{7.624184in}}{\pgfqpoint{2.177280in}{2.201755in}}%
\pgfusepath{clip}%
\pgfsetbuttcap%
\pgfsetroundjoin%
\definecolor{currentfill}{rgb}{1.000000,0.498039,0.054902}%
\pgfsetfillcolor{currentfill}%
\pgfsetlinewidth{0.481800pt}%
\definecolor{currentstroke}{rgb}{1.000000,1.000000,1.000000}%
\pgfsetstrokecolor{currentstroke}%
\pgfsetdash{}{0pt}%
\pgfpathmoveto{\pgfqpoint{3.907452in}{8.460995in}}%
\pgfpathcurveto{\pgfqpoint{3.918502in}{8.460995in}}{\pgfqpoint{3.929101in}{8.465385in}}{\pgfqpoint{3.936915in}{8.473199in}}%
\pgfpathcurveto{\pgfqpoint{3.944728in}{8.481013in}}{\pgfqpoint{3.949118in}{8.491612in}}{\pgfqpoint{3.949118in}{8.502662in}}%
\pgfpathcurveto{\pgfqpoint{3.949118in}{8.513712in}}{\pgfqpoint{3.944728in}{8.524311in}}{\pgfqpoint{3.936915in}{8.532125in}}%
\pgfpathcurveto{\pgfqpoint{3.929101in}{8.539938in}}{\pgfqpoint{3.918502in}{8.544328in}}{\pgfqpoint{3.907452in}{8.544328in}}%
\pgfpathcurveto{\pgfqpoint{3.896402in}{8.544328in}}{\pgfqpoint{3.885803in}{8.539938in}}{\pgfqpoint{3.877989in}{8.532125in}}%
\pgfpathcurveto{\pgfqpoint{3.870175in}{8.524311in}}{\pgfqpoint{3.865785in}{8.513712in}}{\pgfqpoint{3.865785in}{8.502662in}}%
\pgfpathcurveto{\pgfqpoint{3.865785in}{8.491612in}}{\pgfqpoint{3.870175in}{8.481013in}}{\pgfqpoint{3.877989in}{8.473199in}}%
\pgfpathcurveto{\pgfqpoint{3.885803in}{8.465385in}}{\pgfqpoint{3.896402in}{8.460995in}}{\pgfqpoint{3.907452in}{8.460995in}}%
\pgfpathlineto{\pgfqpoint{3.907452in}{8.460995in}}%
\pgfpathclose%
\pgfusepath{stroke,fill}%
\end{pgfscope}%
\begin{pgfscope}%
\pgfpathrectangle{\pgfqpoint{2.963410in}{7.624184in}}{\pgfqpoint{2.177280in}{2.201755in}}%
\pgfusepath{clip}%
\pgfsetbuttcap%
\pgfsetroundjoin%
\definecolor{currentfill}{rgb}{1.000000,0.498039,0.054902}%
\pgfsetfillcolor{currentfill}%
\pgfsetlinewidth{0.481800pt}%
\definecolor{currentstroke}{rgb}{1.000000,1.000000,1.000000}%
\pgfsetstrokecolor{currentstroke}%
\pgfsetdash{}{0pt}%
\pgfpathmoveto{\pgfqpoint{3.848376in}{8.460995in}}%
\pgfpathcurveto{\pgfqpoint{3.859427in}{8.460995in}}{\pgfqpoint{3.870026in}{8.465385in}}{\pgfqpoint{3.877839in}{8.473199in}}%
\pgfpathcurveto{\pgfqpoint{3.885653in}{8.481013in}}{\pgfqpoint{3.890043in}{8.491612in}}{\pgfqpoint{3.890043in}{8.502662in}}%
\pgfpathcurveto{\pgfqpoint{3.890043in}{8.513712in}}{\pgfqpoint{3.885653in}{8.524311in}}{\pgfqpoint{3.877839in}{8.532125in}}%
\pgfpathcurveto{\pgfqpoint{3.870026in}{8.539938in}}{\pgfqpoint{3.859427in}{8.544328in}}{\pgfqpoint{3.848376in}{8.544328in}}%
\pgfpathcurveto{\pgfqpoint{3.837326in}{8.544328in}}{\pgfqpoint{3.826727in}{8.539938in}}{\pgfqpoint{3.818914in}{8.532125in}}%
\pgfpathcurveto{\pgfqpoint{3.811100in}{8.524311in}}{\pgfqpoint{3.806710in}{8.513712in}}{\pgfqpoint{3.806710in}{8.502662in}}%
\pgfpathcurveto{\pgfqpoint{3.806710in}{8.491612in}}{\pgfqpoint{3.811100in}{8.481013in}}{\pgfqpoint{3.818914in}{8.473199in}}%
\pgfpathcurveto{\pgfqpoint{3.826727in}{8.465385in}}{\pgfqpoint{3.837326in}{8.460995in}}{\pgfqpoint{3.848376in}{8.460995in}}%
\pgfpathlineto{\pgfqpoint{3.848376in}{8.460995in}}%
\pgfpathclose%
\pgfusepath{stroke,fill}%
\end{pgfscope}%
\begin{pgfscope}%
\pgfpathrectangle{\pgfqpoint{2.963410in}{7.624184in}}{\pgfqpoint{2.177280in}{2.201755in}}%
\pgfusepath{clip}%
\pgfsetbuttcap%
\pgfsetroundjoin%
\definecolor{currentfill}{rgb}{1.000000,0.498039,0.054902}%
\pgfsetfillcolor{currentfill}%
\pgfsetlinewidth{0.481800pt}%
\definecolor{currentstroke}{rgb}{1.000000,1.000000,1.000000}%
\pgfsetstrokecolor{currentstroke}%
\pgfsetdash{}{0pt}%
\pgfpathmoveto{\pgfqpoint{3.848376in}{8.738994in}}%
\pgfpathcurveto{\pgfqpoint{3.859427in}{8.738994in}}{\pgfqpoint{3.870026in}{8.743385in}}{\pgfqpoint{3.877839in}{8.751198in}}%
\pgfpathcurveto{\pgfqpoint{3.885653in}{8.759012in}}{\pgfqpoint{3.890043in}{8.769611in}}{\pgfqpoint{3.890043in}{8.780661in}}%
\pgfpathcurveto{\pgfqpoint{3.890043in}{8.791711in}}{\pgfqpoint{3.885653in}{8.802310in}}{\pgfqpoint{3.877839in}{8.810124in}}%
\pgfpathcurveto{\pgfqpoint{3.870026in}{8.817938in}}{\pgfqpoint{3.859427in}{8.822328in}}{\pgfqpoint{3.848376in}{8.822328in}}%
\pgfpathcurveto{\pgfqpoint{3.837326in}{8.822328in}}{\pgfqpoint{3.826727in}{8.817938in}}{\pgfqpoint{3.818914in}{8.810124in}}%
\pgfpathcurveto{\pgfqpoint{3.811100in}{8.802310in}}{\pgfqpoint{3.806710in}{8.791711in}}{\pgfqpoint{3.806710in}{8.780661in}}%
\pgfpathcurveto{\pgfqpoint{3.806710in}{8.769611in}}{\pgfqpoint{3.811100in}{8.759012in}}{\pgfqpoint{3.818914in}{8.751198in}}%
\pgfpathcurveto{\pgfqpoint{3.826727in}{8.743385in}}{\pgfqpoint{3.837326in}{8.738994in}}{\pgfqpoint{3.848376in}{8.738994in}}%
\pgfpathlineto{\pgfqpoint{3.848376in}{8.738994in}}%
\pgfpathclose%
\pgfusepath{stroke,fill}%
\end{pgfscope}%
\begin{pgfscope}%
\pgfpathrectangle{\pgfqpoint{2.963410in}{7.624184in}}{\pgfqpoint{2.177280in}{2.201755in}}%
\pgfusepath{clip}%
\pgfsetbuttcap%
\pgfsetroundjoin%
\definecolor{currentfill}{rgb}{1.000000,0.498039,0.054902}%
\pgfsetfillcolor{currentfill}%
\pgfsetlinewidth{0.481800pt}%
\definecolor{currentstroke}{rgb}{1.000000,1.000000,1.000000}%
\pgfsetstrokecolor{currentstroke}%
\pgfsetdash{}{0pt}%
\pgfpathmoveto{\pgfqpoint{3.612075in}{8.127396in}}%
\pgfpathcurveto{\pgfqpoint{3.623126in}{8.127396in}}{\pgfqpoint{3.633725in}{8.131786in}}{\pgfqpoint{3.641538in}{8.139600in}}%
\pgfpathcurveto{\pgfqpoint{3.649352in}{8.147413in}}{\pgfqpoint{3.653742in}{8.158012in}}{\pgfqpoint{3.653742in}{8.169063in}}%
\pgfpathcurveto{\pgfqpoint{3.653742in}{8.180113in}}{\pgfqpoint{3.649352in}{8.190712in}}{\pgfqpoint{3.641538in}{8.198525in}}%
\pgfpathcurveto{\pgfqpoint{3.633725in}{8.206339in}}{\pgfqpoint{3.623126in}{8.210729in}}{\pgfqpoint{3.612075in}{8.210729in}}%
\pgfpathcurveto{\pgfqpoint{3.601025in}{8.210729in}}{\pgfqpoint{3.590426in}{8.206339in}}{\pgfqpoint{3.582613in}{8.198525in}}%
\pgfpathcurveto{\pgfqpoint{3.574799in}{8.190712in}}{\pgfqpoint{3.570409in}{8.180113in}}{\pgfqpoint{3.570409in}{8.169063in}}%
\pgfpathcurveto{\pgfqpoint{3.570409in}{8.158012in}}{\pgfqpoint{3.574799in}{8.147413in}}{\pgfqpoint{3.582613in}{8.139600in}}%
\pgfpathcurveto{\pgfqpoint{3.590426in}{8.131786in}}{\pgfqpoint{3.601025in}{8.127396in}}{\pgfqpoint{3.612075in}{8.127396in}}%
\pgfpathlineto{\pgfqpoint{3.612075in}{8.127396in}}%
\pgfpathclose%
\pgfusepath{stroke,fill}%
\end{pgfscope}%
\begin{pgfscope}%
\pgfpathrectangle{\pgfqpoint{2.963410in}{7.624184in}}{\pgfqpoint{2.177280in}{2.201755in}}%
\pgfusepath{clip}%
\pgfsetbuttcap%
\pgfsetroundjoin%
\definecolor{currentfill}{rgb}{1.000000,0.498039,0.054902}%
\pgfsetfillcolor{currentfill}%
\pgfsetlinewidth{0.481800pt}%
\definecolor{currentstroke}{rgb}{1.000000,1.000000,1.000000}%
\pgfsetstrokecolor{currentstroke}%
\pgfsetdash{}{0pt}%
\pgfpathmoveto{\pgfqpoint{3.789301in}{8.460995in}}%
\pgfpathcurveto{\pgfqpoint{3.800351in}{8.460995in}}{\pgfqpoint{3.810950in}{8.465385in}}{\pgfqpoint{3.818764in}{8.473199in}}%
\pgfpathcurveto{\pgfqpoint{3.826578in}{8.481013in}}{\pgfqpoint{3.830968in}{8.491612in}}{\pgfqpoint{3.830968in}{8.502662in}}%
\pgfpathcurveto{\pgfqpoint{3.830968in}{8.513712in}}{\pgfqpoint{3.826578in}{8.524311in}}{\pgfqpoint{3.818764in}{8.532125in}}%
\pgfpathcurveto{\pgfqpoint{3.810950in}{8.539938in}}{\pgfqpoint{3.800351in}{8.544328in}}{\pgfqpoint{3.789301in}{8.544328in}}%
\pgfpathcurveto{\pgfqpoint{3.778251in}{8.544328in}}{\pgfqpoint{3.767652in}{8.539938in}}{\pgfqpoint{3.759838in}{8.532125in}}%
\pgfpathcurveto{\pgfqpoint{3.752025in}{8.524311in}}{\pgfqpoint{3.747635in}{8.513712in}}{\pgfqpoint{3.747635in}{8.502662in}}%
\pgfpathcurveto{\pgfqpoint{3.747635in}{8.491612in}}{\pgfqpoint{3.752025in}{8.481013in}}{\pgfqpoint{3.759838in}{8.473199in}}%
\pgfpathcurveto{\pgfqpoint{3.767652in}{8.465385in}}{\pgfqpoint{3.778251in}{8.460995in}}{\pgfqpoint{3.789301in}{8.460995in}}%
\pgfpathlineto{\pgfqpoint{3.789301in}{8.460995in}}%
\pgfpathclose%
\pgfusepath{stroke,fill}%
\end{pgfscope}%
\begin{pgfscope}%
\pgfpathrectangle{\pgfqpoint{2.963410in}{7.624184in}}{\pgfqpoint{2.177280in}{2.201755in}}%
\pgfusepath{clip}%
\pgfsetbuttcap%
\pgfsetroundjoin%
\definecolor{currentfill}{rgb}{0.172549,0.627451,0.172549}%
\pgfsetfillcolor{currentfill}%
\pgfsetlinewidth{0.481800pt}%
\definecolor{currentstroke}{rgb}{1.000000,1.000000,1.000000}%
\pgfsetstrokecolor{currentstroke}%
\pgfsetdash{}{0pt}%
\pgfpathmoveto{\pgfqpoint{4.084678in}{8.794594in}}%
\pgfpathcurveto{\pgfqpoint{4.095728in}{8.794594in}}{\pgfqpoint{4.106327in}{8.798985in}}{\pgfqpoint{4.114140in}{8.806798in}}%
\pgfpathcurveto{\pgfqpoint{4.121954in}{8.814612in}}{\pgfqpoint{4.126344in}{8.825211in}}{\pgfqpoint{4.126344in}{8.836261in}}%
\pgfpathcurveto{\pgfqpoint{4.126344in}{8.847311in}}{\pgfqpoint{4.121954in}{8.857910in}}{\pgfqpoint{4.114140in}{8.865724in}}%
\pgfpathcurveto{\pgfqpoint{4.106327in}{8.873537in}}{\pgfqpoint{4.095728in}{8.877928in}}{\pgfqpoint{4.084678in}{8.877928in}}%
\pgfpathcurveto{\pgfqpoint{4.073627in}{8.877928in}}{\pgfqpoint{4.063028in}{8.873537in}}{\pgfqpoint{4.055215in}{8.865724in}}%
\pgfpathcurveto{\pgfqpoint{4.047401in}{8.857910in}}{\pgfqpoint{4.043011in}{8.847311in}}{\pgfqpoint{4.043011in}{8.836261in}}%
\pgfpathcurveto{\pgfqpoint{4.043011in}{8.825211in}}{\pgfqpoint{4.047401in}{8.814612in}}{\pgfqpoint{4.055215in}{8.806798in}}%
\pgfpathcurveto{\pgfqpoint{4.063028in}{8.798985in}}{\pgfqpoint{4.073627in}{8.794594in}}{\pgfqpoint{4.084678in}{8.794594in}}%
\pgfpathlineto{\pgfqpoint{4.084678in}{8.794594in}}%
\pgfpathclose%
\pgfusepath{stroke,fill}%
\end{pgfscope}%
\begin{pgfscope}%
\pgfpathrectangle{\pgfqpoint{2.963410in}{7.624184in}}{\pgfqpoint{2.177280in}{2.201755in}}%
\pgfusepath{clip}%
\pgfsetbuttcap%
\pgfsetroundjoin%
\definecolor{currentfill}{rgb}{0.172549,0.627451,0.172549}%
\pgfsetfillcolor{currentfill}%
\pgfsetlinewidth{0.481800pt}%
\definecolor{currentstroke}{rgb}{1.000000,1.000000,1.000000}%
\pgfsetstrokecolor{currentstroke}%
\pgfsetdash{}{0pt}%
\pgfpathmoveto{\pgfqpoint{3.730226in}{8.516595in}}%
\pgfpathcurveto{\pgfqpoint{3.741276in}{8.516595in}}{\pgfqpoint{3.751875in}{8.520985in}}{\pgfqpoint{3.759689in}{8.528799in}}%
\pgfpathcurveto{\pgfqpoint{3.767502in}{8.536612in}}{\pgfqpoint{3.771893in}{8.547212in}}{\pgfqpoint{3.771893in}{8.558262in}}%
\pgfpathcurveto{\pgfqpoint{3.771893in}{8.569312in}}{\pgfqpoint{3.767502in}{8.579911in}}{\pgfqpoint{3.759689in}{8.587724in}}%
\pgfpathcurveto{\pgfqpoint{3.751875in}{8.595538in}}{\pgfqpoint{3.741276in}{8.599928in}}{\pgfqpoint{3.730226in}{8.599928in}}%
\pgfpathcurveto{\pgfqpoint{3.719176in}{8.599928in}}{\pgfqpoint{3.708577in}{8.595538in}}{\pgfqpoint{3.700763in}{8.587724in}}%
\pgfpathcurveto{\pgfqpoint{3.692950in}{8.579911in}}{\pgfqpoint{3.688559in}{8.569312in}}{\pgfqpoint{3.688559in}{8.558262in}}%
\pgfpathcurveto{\pgfqpoint{3.688559in}{8.547212in}}{\pgfqpoint{3.692950in}{8.536612in}}{\pgfqpoint{3.700763in}{8.528799in}}%
\pgfpathcurveto{\pgfqpoint{3.708577in}{8.520985in}}{\pgfqpoint{3.719176in}{8.516595in}}{\pgfqpoint{3.730226in}{8.516595in}}%
\pgfpathlineto{\pgfqpoint{3.730226in}{8.516595in}}%
\pgfpathclose%
\pgfusepath{stroke,fill}%
\end{pgfscope}%
\begin{pgfscope}%
\pgfpathrectangle{\pgfqpoint{2.963410in}{7.624184in}}{\pgfqpoint{2.177280in}{2.201755in}}%
\pgfusepath{clip}%
\pgfsetbuttcap%
\pgfsetroundjoin%
\definecolor{currentfill}{rgb}{0.172549,0.627451,0.172549}%
\pgfsetfillcolor{currentfill}%
\pgfsetlinewidth{0.481800pt}%
\definecolor{currentstroke}{rgb}{1.000000,1.000000,1.000000}%
\pgfsetstrokecolor{currentstroke}%
\pgfsetdash{}{0pt}%
\pgfpathmoveto{\pgfqpoint{3.907452in}{9.239393in}}%
\pgfpathcurveto{\pgfqpoint{3.918502in}{9.239393in}}{\pgfqpoint{3.929101in}{9.243784in}}{\pgfqpoint{3.936915in}{9.251597in}}%
\pgfpathcurveto{\pgfqpoint{3.944728in}{9.259411in}}{\pgfqpoint{3.949118in}{9.270010in}}{\pgfqpoint{3.949118in}{9.281060in}}%
\pgfpathcurveto{\pgfqpoint{3.949118in}{9.292110in}}{\pgfqpoint{3.944728in}{9.302709in}}{\pgfqpoint{3.936915in}{9.310523in}}%
\pgfpathcurveto{\pgfqpoint{3.929101in}{9.318336in}}{\pgfqpoint{3.918502in}{9.322727in}}{\pgfqpoint{3.907452in}{9.322727in}}%
\pgfpathcurveto{\pgfqpoint{3.896402in}{9.322727in}}{\pgfqpoint{3.885803in}{9.318336in}}{\pgfqpoint{3.877989in}{9.310523in}}%
\pgfpathcurveto{\pgfqpoint{3.870175in}{9.302709in}}{\pgfqpoint{3.865785in}{9.292110in}}{\pgfqpoint{3.865785in}{9.281060in}}%
\pgfpathcurveto{\pgfqpoint{3.865785in}{9.270010in}}{\pgfqpoint{3.870175in}{9.259411in}}{\pgfqpoint{3.877989in}{9.251597in}}%
\pgfpathcurveto{\pgfqpoint{3.885803in}{9.243784in}}{\pgfqpoint{3.896402in}{9.239393in}}{\pgfqpoint{3.907452in}{9.239393in}}%
\pgfpathlineto{\pgfqpoint{3.907452in}{9.239393in}}%
\pgfpathclose%
\pgfusepath{stroke,fill}%
\end{pgfscope}%
\begin{pgfscope}%
\pgfpathrectangle{\pgfqpoint{2.963410in}{7.624184in}}{\pgfqpoint{2.177280in}{2.201755in}}%
\pgfusepath{clip}%
\pgfsetbuttcap%
\pgfsetroundjoin%
\definecolor{currentfill}{rgb}{0.172549,0.627451,0.172549}%
\pgfsetfillcolor{currentfill}%
\pgfsetlinewidth{0.481800pt}%
\definecolor{currentstroke}{rgb}{1.000000,1.000000,1.000000}%
\pgfsetstrokecolor{currentstroke}%
\pgfsetdash{}{0pt}%
\pgfpathmoveto{\pgfqpoint{3.848376in}{8.794594in}}%
\pgfpathcurveto{\pgfqpoint{3.859427in}{8.794594in}}{\pgfqpoint{3.870026in}{8.798985in}}{\pgfqpoint{3.877839in}{8.806798in}}%
\pgfpathcurveto{\pgfqpoint{3.885653in}{8.814612in}}{\pgfqpoint{3.890043in}{8.825211in}}{\pgfqpoint{3.890043in}{8.836261in}}%
\pgfpathcurveto{\pgfqpoint{3.890043in}{8.847311in}}{\pgfqpoint{3.885653in}{8.857910in}}{\pgfqpoint{3.877839in}{8.865724in}}%
\pgfpathcurveto{\pgfqpoint{3.870026in}{8.873537in}}{\pgfqpoint{3.859427in}{8.877928in}}{\pgfqpoint{3.848376in}{8.877928in}}%
\pgfpathcurveto{\pgfqpoint{3.837326in}{8.877928in}}{\pgfqpoint{3.826727in}{8.873537in}}{\pgfqpoint{3.818914in}{8.865724in}}%
\pgfpathcurveto{\pgfqpoint{3.811100in}{8.857910in}}{\pgfqpoint{3.806710in}{8.847311in}}{\pgfqpoint{3.806710in}{8.836261in}}%
\pgfpathcurveto{\pgfqpoint{3.806710in}{8.825211in}}{\pgfqpoint{3.811100in}{8.814612in}}{\pgfqpoint{3.818914in}{8.806798in}}%
\pgfpathcurveto{\pgfqpoint{3.826727in}{8.798985in}}{\pgfqpoint{3.837326in}{8.794594in}}{\pgfqpoint{3.848376in}{8.794594in}}%
\pgfpathlineto{\pgfqpoint{3.848376in}{8.794594in}}%
\pgfpathclose%
\pgfusepath{stroke,fill}%
\end{pgfscope}%
\begin{pgfscope}%
\pgfpathrectangle{\pgfqpoint{2.963410in}{7.624184in}}{\pgfqpoint{2.177280in}{2.201755in}}%
\pgfusepath{clip}%
\pgfsetbuttcap%
\pgfsetroundjoin%
\definecolor{currentfill}{rgb}{0.172549,0.627451,0.172549}%
\pgfsetfillcolor{currentfill}%
\pgfsetlinewidth{0.481800pt}%
\definecolor{currentstroke}{rgb}{1.000000,1.000000,1.000000}%
\pgfsetstrokecolor{currentstroke}%
\pgfsetdash{}{0pt}%
\pgfpathmoveto{\pgfqpoint{3.907452in}{8.905794in}}%
\pgfpathcurveto{\pgfqpoint{3.918502in}{8.905794in}}{\pgfqpoint{3.929101in}{8.910184in}}{\pgfqpoint{3.936915in}{8.917998in}}%
\pgfpathcurveto{\pgfqpoint{3.944728in}{8.925812in}}{\pgfqpoint{3.949118in}{8.936411in}}{\pgfqpoint{3.949118in}{8.947461in}}%
\pgfpathcurveto{\pgfqpoint{3.949118in}{8.958511in}}{\pgfqpoint{3.944728in}{8.969110in}}{\pgfqpoint{3.936915in}{8.976924in}}%
\pgfpathcurveto{\pgfqpoint{3.929101in}{8.984737in}}{\pgfqpoint{3.918502in}{8.989127in}}{\pgfqpoint{3.907452in}{8.989127in}}%
\pgfpathcurveto{\pgfqpoint{3.896402in}{8.989127in}}{\pgfqpoint{3.885803in}{8.984737in}}{\pgfqpoint{3.877989in}{8.976924in}}%
\pgfpathcurveto{\pgfqpoint{3.870175in}{8.969110in}}{\pgfqpoint{3.865785in}{8.958511in}}{\pgfqpoint{3.865785in}{8.947461in}}%
\pgfpathcurveto{\pgfqpoint{3.865785in}{8.936411in}}{\pgfqpoint{3.870175in}{8.925812in}}{\pgfqpoint{3.877989in}{8.917998in}}%
\pgfpathcurveto{\pgfqpoint{3.885803in}{8.910184in}}{\pgfqpoint{3.896402in}{8.905794in}}{\pgfqpoint{3.907452in}{8.905794in}}%
\pgfpathlineto{\pgfqpoint{3.907452in}{8.905794in}}%
\pgfpathclose%
\pgfusepath{stroke,fill}%
\end{pgfscope}%
\begin{pgfscope}%
\pgfpathrectangle{\pgfqpoint{2.963410in}{7.624184in}}{\pgfqpoint{2.177280in}{2.201755in}}%
\pgfusepath{clip}%
\pgfsetbuttcap%
\pgfsetroundjoin%
\definecolor{currentfill}{rgb}{0.172549,0.627451,0.172549}%
\pgfsetfillcolor{currentfill}%
\pgfsetlinewidth{0.481800pt}%
\definecolor{currentstroke}{rgb}{1.000000,1.000000,1.000000}%
\pgfsetstrokecolor{currentstroke}%
\pgfsetdash{}{0pt}%
\pgfpathmoveto{\pgfqpoint{3.907452in}{9.517393in}}%
\pgfpathcurveto{\pgfqpoint{3.918502in}{9.517393in}}{\pgfqpoint{3.929101in}{9.521783in}}{\pgfqpoint{3.936915in}{9.529597in}}%
\pgfpathcurveto{\pgfqpoint{3.944728in}{9.537410in}}{\pgfqpoint{3.949118in}{9.548009in}}{\pgfqpoint{3.949118in}{9.559059in}}%
\pgfpathcurveto{\pgfqpoint{3.949118in}{9.570110in}}{\pgfqpoint{3.944728in}{9.580709in}}{\pgfqpoint{3.936915in}{9.588522in}}%
\pgfpathcurveto{\pgfqpoint{3.929101in}{9.596336in}}{\pgfqpoint{3.918502in}{9.600726in}}{\pgfqpoint{3.907452in}{9.600726in}}%
\pgfpathcurveto{\pgfqpoint{3.896402in}{9.600726in}}{\pgfqpoint{3.885803in}{9.596336in}}{\pgfqpoint{3.877989in}{9.588522in}}%
\pgfpathcurveto{\pgfqpoint{3.870175in}{9.580709in}}{\pgfqpoint{3.865785in}{9.570110in}}{\pgfqpoint{3.865785in}{9.559059in}}%
\pgfpathcurveto{\pgfqpoint{3.865785in}{9.548009in}}{\pgfqpoint{3.870175in}{9.537410in}}{\pgfqpoint{3.877989in}{9.529597in}}%
\pgfpathcurveto{\pgfqpoint{3.885803in}{9.521783in}}{\pgfqpoint{3.896402in}{9.517393in}}{\pgfqpoint{3.907452in}{9.517393in}}%
\pgfpathlineto{\pgfqpoint{3.907452in}{9.517393in}}%
\pgfpathclose%
\pgfusepath{stroke,fill}%
\end{pgfscope}%
\begin{pgfscope}%
\pgfpathrectangle{\pgfqpoint{2.963410in}{7.624184in}}{\pgfqpoint{2.177280in}{2.201755in}}%
\pgfusepath{clip}%
\pgfsetbuttcap%
\pgfsetroundjoin%
\definecolor{currentfill}{rgb}{0.172549,0.627451,0.172549}%
\pgfsetfillcolor{currentfill}%
\pgfsetlinewidth{0.481800pt}%
\definecolor{currentstroke}{rgb}{1.000000,1.000000,1.000000}%
\pgfsetstrokecolor{currentstroke}%
\pgfsetdash{}{0pt}%
\pgfpathmoveto{\pgfqpoint{3.612075in}{8.016196in}}%
\pgfpathcurveto{\pgfqpoint{3.623126in}{8.016196in}}{\pgfqpoint{3.633725in}{8.020586in}}{\pgfqpoint{3.641538in}{8.028400in}}%
\pgfpathcurveto{\pgfqpoint{3.649352in}{8.036214in}}{\pgfqpoint{3.653742in}{8.046813in}}{\pgfqpoint{3.653742in}{8.057863in}}%
\pgfpathcurveto{\pgfqpoint{3.653742in}{8.068913in}}{\pgfqpoint{3.649352in}{8.079512in}}{\pgfqpoint{3.641538in}{8.087326in}}%
\pgfpathcurveto{\pgfqpoint{3.633725in}{8.095139in}}{\pgfqpoint{3.623126in}{8.099529in}}{\pgfqpoint{3.612075in}{8.099529in}}%
\pgfpathcurveto{\pgfqpoint{3.601025in}{8.099529in}}{\pgfqpoint{3.590426in}{8.095139in}}{\pgfqpoint{3.582613in}{8.087326in}}%
\pgfpathcurveto{\pgfqpoint{3.574799in}{8.079512in}}{\pgfqpoint{3.570409in}{8.068913in}}{\pgfqpoint{3.570409in}{8.057863in}}%
\pgfpathcurveto{\pgfqpoint{3.570409in}{8.046813in}}{\pgfqpoint{3.574799in}{8.036214in}}{\pgfqpoint{3.582613in}{8.028400in}}%
\pgfpathcurveto{\pgfqpoint{3.590426in}{8.020586in}}{\pgfqpoint{3.601025in}{8.016196in}}{\pgfqpoint{3.612075in}{8.016196in}}%
\pgfpathlineto{\pgfqpoint{3.612075in}{8.016196in}}%
\pgfpathclose%
\pgfusepath{stroke,fill}%
\end{pgfscope}%
\begin{pgfscope}%
\pgfpathrectangle{\pgfqpoint{2.963410in}{7.624184in}}{\pgfqpoint{2.177280in}{2.201755in}}%
\pgfusepath{clip}%
\pgfsetbuttcap%
\pgfsetroundjoin%
\definecolor{currentfill}{rgb}{0.172549,0.627451,0.172549}%
\pgfsetfillcolor{currentfill}%
\pgfsetlinewidth{0.481800pt}%
\definecolor{currentstroke}{rgb}{1.000000,1.000000,1.000000}%
\pgfsetstrokecolor{currentstroke}%
\pgfsetdash{}{0pt}%
\pgfpathmoveto{\pgfqpoint{3.848376in}{9.350593in}}%
\pgfpathcurveto{\pgfqpoint{3.859427in}{9.350593in}}{\pgfqpoint{3.870026in}{9.354983in}}{\pgfqpoint{3.877839in}{9.362797in}}%
\pgfpathcurveto{\pgfqpoint{3.885653in}{9.370611in}}{\pgfqpoint{3.890043in}{9.381210in}}{\pgfqpoint{3.890043in}{9.392260in}}%
\pgfpathcurveto{\pgfqpoint{3.890043in}{9.403310in}}{\pgfqpoint{3.885653in}{9.413909in}}{\pgfqpoint{3.877839in}{9.421723in}}%
\pgfpathcurveto{\pgfqpoint{3.870026in}{9.429536in}}{\pgfqpoint{3.859427in}{9.433926in}}{\pgfqpoint{3.848376in}{9.433926in}}%
\pgfpathcurveto{\pgfqpoint{3.837326in}{9.433926in}}{\pgfqpoint{3.826727in}{9.429536in}}{\pgfqpoint{3.818914in}{9.421723in}}%
\pgfpathcurveto{\pgfqpoint{3.811100in}{9.413909in}}{\pgfqpoint{3.806710in}{9.403310in}}{\pgfqpoint{3.806710in}{9.392260in}}%
\pgfpathcurveto{\pgfqpoint{3.806710in}{9.381210in}}{\pgfqpoint{3.811100in}{9.370611in}}{\pgfqpoint{3.818914in}{9.362797in}}%
\pgfpathcurveto{\pgfqpoint{3.826727in}{9.354983in}}{\pgfqpoint{3.837326in}{9.350593in}}{\pgfqpoint{3.848376in}{9.350593in}}%
\pgfpathlineto{\pgfqpoint{3.848376in}{9.350593in}}%
\pgfpathclose%
\pgfusepath{stroke,fill}%
\end{pgfscope}%
\begin{pgfscope}%
\pgfpathrectangle{\pgfqpoint{2.963410in}{7.624184in}}{\pgfqpoint{2.177280in}{2.201755in}}%
\pgfusepath{clip}%
\pgfsetbuttcap%
\pgfsetroundjoin%
\definecolor{currentfill}{rgb}{0.172549,0.627451,0.172549}%
\pgfsetfillcolor{currentfill}%
\pgfsetlinewidth{0.481800pt}%
\definecolor{currentstroke}{rgb}{1.000000,1.000000,1.000000}%
\pgfsetstrokecolor{currentstroke}%
\pgfsetdash{}{0pt}%
\pgfpathmoveto{\pgfqpoint{3.612075in}{9.016994in}}%
\pgfpathcurveto{\pgfqpoint{3.623126in}{9.016994in}}{\pgfqpoint{3.633725in}{9.021384in}}{\pgfqpoint{3.641538in}{9.029198in}}%
\pgfpathcurveto{\pgfqpoint{3.649352in}{9.037011in}}{\pgfqpoint{3.653742in}{9.047610in}}{\pgfqpoint{3.653742in}{9.058661in}}%
\pgfpathcurveto{\pgfqpoint{3.653742in}{9.069711in}}{\pgfqpoint{3.649352in}{9.080310in}}{\pgfqpoint{3.641538in}{9.088123in}}%
\pgfpathcurveto{\pgfqpoint{3.633725in}{9.095937in}}{\pgfqpoint{3.623126in}{9.100327in}}{\pgfqpoint{3.612075in}{9.100327in}}%
\pgfpathcurveto{\pgfqpoint{3.601025in}{9.100327in}}{\pgfqpoint{3.590426in}{9.095937in}}{\pgfqpoint{3.582613in}{9.088123in}}%
\pgfpathcurveto{\pgfqpoint{3.574799in}{9.080310in}}{\pgfqpoint{3.570409in}{9.069711in}}{\pgfqpoint{3.570409in}{9.058661in}}%
\pgfpathcurveto{\pgfqpoint{3.570409in}{9.047610in}}{\pgfqpoint{3.574799in}{9.037011in}}{\pgfqpoint{3.582613in}{9.029198in}}%
\pgfpathcurveto{\pgfqpoint{3.590426in}{9.021384in}}{\pgfqpoint{3.601025in}{9.016994in}}{\pgfqpoint{3.612075in}{9.016994in}}%
\pgfpathlineto{\pgfqpoint{3.612075in}{9.016994in}}%
\pgfpathclose%
\pgfusepath{stroke,fill}%
\end{pgfscope}%
\begin{pgfscope}%
\pgfpathrectangle{\pgfqpoint{2.963410in}{7.624184in}}{\pgfqpoint{2.177280in}{2.201755in}}%
\pgfusepath{clip}%
\pgfsetbuttcap%
\pgfsetroundjoin%
\definecolor{currentfill}{rgb}{0.172549,0.627451,0.172549}%
\pgfsetfillcolor{currentfill}%
\pgfsetlinewidth{0.481800pt}%
\definecolor{currentstroke}{rgb}{1.000000,1.000000,1.000000}%
\pgfsetstrokecolor{currentstroke}%
\pgfsetdash{}{0pt}%
\pgfpathmoveto{\pgfqpoint{4.261903in}{9.294993in}}%
\pgfpathcurveto{\pgfqpoint{4.272953in}{9.294993in}}{\pgfqpoint{4.283552in}{9.299384in}}{\pgfqpoint{4.291366in}{9.307197in}}%
\pgfpathcurveto{\pgfqpoint{4.299180in}{9.315011in}}{\pgfqpoint{4.303570in}{9.325610in}}{\pgfqpoint{4.303570in}{9.336660in}}%
\pgfpathcurveto{\pgfqpoint{4.303570in}{9.347710in}}{\pgfqpoint{4.299180in}{9.358309in}}{\pgfqpoint{4.291366in}{9.366123in}}%
\pgfpathcurveto{\pgfqpoint{4.283552in}{9.373936in}}{\pgfqpoint{4.272953in}{9.378327in}}{\pgfqpoint{4.261903in}{9.378327in}}%
\pgfpathcurveto{\pgfqpoint{4.250853in}{9.378327in}}{\pgfqpoint{4.240254in}{9.373936in}}{\pgfqpoint{4.232441in}{9.366123in}}%
\pgfpathcurveto{\pgfqpoint{4.224627in}{9.358309in}}{\pgfqpoint{4.220237in}{9.347710in}}{\pgfqpoint{4.220237in}{9.336660in}}%
\pgfpathcurveto{\pgfqpoint{4.220237in}{9.325610in}}{\pgfqpoint{4.224627in}{9.315011in}}{\pgfqpoint{4.232441in}{9.307197in}}%
\pgfpathcurveto{\pgfqpoint{4.240254in}{9.299384in}}{\pgfqpoint{4.250853in}{9.294993in}}{\pgfqpoint{4.261903in}{9.294993in}}%
\pgfpathlineto{\pgfqpoint{4.261903in}{9.294993in}}%
\pgfpathclose%
\pgfusepath{stroke,fill}%
\end{pgfscope}%
\begin{pgfscope}%
\pgfpathrectangle{\pgfqpoint{2.963410in}{7.624184in}}{\pgfqpoint{2.177280in}{2.201755in}}%
\pgfusepath{clip}%
\pgfsetbuttcap%
\pgfsetroundjoin%
\definecolor{currentfill}{rgb}{0.172549,0.627451,0.172549}%
\pgfsetfillcolor{currentfill}%
\pgfsetlinewidth{0.481800pt}%
\definecolor{currentstroke}{rgb}{1.000000,1.000000,1.000000}%
\pgfsetstrokecolor{currentstroke}%
\pgfsetdash{}{0pt}%
\pgfpathmoveto{\pgfqpoint{4.025602in}{8.905794in}}%
\pgfpathcurveto{\pgfqpoint{4.036652in}{8.905794in}}{\pgfqpoint{4.047251in}{8.910184in}}{\pgfqpoint{4.055065in}{8.917998in}}%
\pgfpathcurveto{\pgfqpoint{4.062879in}{8.925812in}}{\pgfqpoint{4.067269in}{8.936411in}}{\pgfqpoint{4.067269in}{8.947461in}}%
\pgfpathcurveto{\pgfqpoint{4.067269in}{8.958511in}}{\pgfqpoint{4.062879in}{8.969110in}}{\pgfqpoint{4.055065in}{8.976924in}}%
\pgfpathcurveto{\pgfqpoint{4.047251in}{8.984737in}}{\pgfqpoint{4.036652in}{8.989127in}}{\pgfqpoint{4.025602in}{8.989127in}}%
\pgfpathcurveto{\pgfqpoint{4.014552in}{8.989127in}}{\pgfqpoint{4.003953in}{8.984737in}}{\pgfqpoint{3.996139in}{8.976924in}}%
\pgfpathcurveto{\pgfqpoint{3.988326in}{8.969110in}}{\pgfqpoint{3.983936in}{8.958511in}}{\pgfqpoint{3.983936in}{8.947461in}}%
\pgfpathcurveto{\pgfqpoint{3.983936in}{8.936411in}}{\pgfqpoint{3.988326in}{8.925812in}}{\pgfqpoint{3.996139in}{8.917998in}}%
\pgfpathcurveto{\pgfqpoint{4.003953in}{8.910184in}}{\pgfqpoint{4.014552in}{8.905794in}}{\pgfqpoint{4.025602in}{8.905794in}}%
\pgfpathlineto{\pgfqpoint{4.025602in}{8.905794in}}%
\pgfpathclose%
\pgfusepath{stroke,fill}%
\end{pgfscope}%
\begin{pgfscope}%
\pgfpathrectangle{\pgfqpoint{2.963410in}{7.624184in}}{\pgfqpoint{2.177280in}{2.201755in}}%
\pgfusepath{clip}%
\pgfsetbuttcap%
\pgfsetroundjoin%
\definecolor{currentfill}{rgb}{0.172549,0.627451,0.172549}%
\pgfsetfillcolor{currentfill}%
\pgfsetlinewidth{0.481800pt}%
\definecolor{currentstroke}{rgb}{1.000000,1.000000,1.000000}%
\pgfsetstrokecolor{currentstroke}%
\pgfsetdash{}{0pt}%
\pgfpathmoveto{\pgfqpoint{3.730226in}{8.850194in}}%
\pgfpathcurveto{\pgfqpoint{3.741276in}{8.850194in}}{\pgfqpoint{3.751875in}{8.854584in}}{\pgfqpoint{3.759689in}{8.862398in}}%
\pgfpathcurveto{\pgfqpoint{3.767502in}{8.870212in}}{\pgfqpoint{3.771893in}{8.880811in}}{\pgfqpoint{3.771893in}{8.891861in}}%
\pgfpathcurveto{\pgfqpoint{3.771893in}{8.902911in}}{\pgfqpoint{3.767502in}{8.913510in}}{\pgfqpoint{3.759689in}{8.921324in}}%
\pgfpathcurveto{\pgfqpoint{3.751875in}{8.929137in}}{\pgfqpoint{3.741276in}{8.933528in}}{\pgfqpoint{3.730226in}{8.933528in}}%
\pgfpathcurveto{\pgfqpoint{3.719176in}{8.933528in}}{\pgfqpoint{3.708577in}{8.929137in}}{\pgfqpoint{3.700763in}{8.921324in}}%
\pgfpathcurveto{\pgfqpoint{3.692950in}{8.913510in}}{\pgfqpoint{3.688559in}{8.902911in}}{\pgfqpoint{3.688559in}{8.891861in}}%
\pgfpathcurveto{\pgfqpoint{3.688559in}{8.880811in}}{\pgfqpoint{3.692950in}{8.870212in}}{\pgfqpoint{3.700763in}{8.862398in}}%
\pgfpathcurveto{\pgfqpoint{3.708577in}{8.854584in}}{\pgfqpoint{3.719176in}{8.850194in}}{\pgfqpoint{3.730226in}{8.850194in}}%
\pgfpathlineto{\pgfqpoint{3.730226in}{8.850194in}}%
\pgfpathclose%
\pgfusepath{stroke,fill}%
\end{pgfscope}%
\begin{pgfscope}%
\pgfpathrectangle{\pgfqpoint{2.963410in}{7.624184in}}{\pgfqpoint{2.177280in}{2.201755in}}%
\pgfusepath{clip}%
\pgfsetbuttcap%
\pgfsetroundjoin%
\definecolor{currentfill}{rgb}{0.172549,0.627451,0.172549}%
\pgfsetfillcolor{currentfill}%
\pgfsetlinewidth{0.481800pt}%
\definecolor{currentstroke}{rgb}{1.000000,1.000000,1.000000}%
\pgfsetstrokecolor{currentstroke}%
\pgfsetdash{}{0pt}%
\pgfpathmoveto{\pgfqpoint{3.907452in}{9.072594in}}%
\pgfpathcurveto{\pgfqpoint{3.918502in}{9.072594in}}{\pgfqpoint{3.929101in}{9.076984in}}{\pgfqpoint{3.936915in}{9.084798in}}%
\pgfpathcurveto{\pgfqpoint{3.944728in}{9.092611in}}{\pgfqpoint{3.949118in}{9.103210in}}{\pgfqpoint{3.949118in}{9.114260in}}%
\pgfpathcurveto{\pgfqpoint{3.949118in}{9.125311in}}{\pgfqpoint{3.944728in}{9.135910in}}{\pgfqpoint{3.936915in}{9.143723in}}%
\pgfpathcurveto{\pgfqpoint{3.929101in}{9.151537in}}{\pgfqpoint{3.918502in}{9.155927in}}{\pgfqpoint{3.907452in}{9.155927in}}%
\pgfpathcurveto{\pgfqpoint{3.896402in}{9.155927in}}{\pgfqpoint{3.885803in}{9.151537in}}{\pgfqpoint{3.877989in}{9.143723in}}%
\pgfpathcurveto{\pgfqpoint{3.870175in}{9.135910in}}{\pgfqpoint{3.865785in}{9.125311in}}{\pgfqpoint{3.865785in}{9.114260in}}%
\pgfpathcurveto{\pgfqpoint{3.865785in}{9.103210in}}{\pgfqpoint{3.870175in}{9.092611in}}{\pgfqpoint{3.877989in}{9.084798in}}%
\pgfpathcurveto{\pgfqpoint{3.885803in}{9.076984in}}{\pgfqpoint{3.896402in}{9.072594in}}{\pgfqpoint{3.907452in}{9.072594in}}%
\pgfpathlineto{\pgfqpoint{3.907452in}{9.072594in}}%
\pgfpathclose%
\pgfusepath{stroke,fill}%
\end{pgfscope}%
\begin{pgfscope}%
\pgfpathrectangle{\pgfqpoint{2.963410in}{7.624184in}}{\pgfqpoint{2.177280in}{2.201755in}}%
\pgfusepath{clip}%
\pgfsetbuttcap%
\pgfsetroundjoin%
\definecolor{currentfill}{rgb}{0.172549,0.627451,0.172549}%
\pgfsetfillcolor{currentfill}%
\pgfsetlinewidth{0.481800pt}%
\definecolor{currentstroke}{rgb}{1.000000,1.000000,1.000000}%
\pgfsetstrokecolor{currentstroke}%
\pgfsetdash{}{0pt}%
\pgfpathmoveto{\pgfqpoint{3.612075in}{8.460995in}}%
\pgfpathcurveto{\pgfqpoint{3.623126in}{8.460995in}}{\pgfqpoint{3.633725in}{8.465385in}}{\pgfqpoint{3.641538in}{8.473199in}}%
\pgfpathcurveto{\pgfqpoint{3.649352in}{8.481013in}}{\pgfqpoint{3.653742in}{8.491612in}}{\pgfqpoint{3.653742in}{8.502662in}}%
\pgfpathcurveto{\pgfqpoint{3.653742in}{8.513712in}}{\pgfqpoint{3.649352in}{8.524311in}}{\pgfqpoint{3.641538in}{8.532125in}}%
\pgfpathcurveto{\pgfqpoint{3.633725in}{8.539938in}}{\pgfqpoint{3.623126in}{8.544328in}}{\pgfqpoint{3.612075in}{8.544328in}}%
\pgfpathcurveto{\pgfqpoint{3.601025in}{8.544328in}}{\pgfqpoint{3.590426in}{8.539938in}}{\pgfqpoint{3.582613in}{8.532125in}}%
\pgfpathcurveto{\pgfqpoint{3.574799in}{8.524311in}}{\pgfqpoint{3.570409in}{8.513712in}}{\pgfqpoint{3.570409in}{8.502662in}}%
\pgfpathcurveto{\pgfqpoint{3.570409in}{8.491612in}}{\pgfqpoint{3.574799in}{8.481013in}}{\pgfqpoint{3.582613in}{8.473199in}}%
\pgfpathcurveto{\pgfqpoint{3.590426in}{8.465385in}}{\pgfqpoint{3.601025in}{8.460995in}}{\pgfqpoint{3.612075in}{8.460995in}}%
\pgfpathlineto{\pgfqpoint{3.612075in}{8.460995in}}%
\pgfpathclose%
\pgfusepath{stroke,fill}%
\end{pgfscope}%
\begin{pgfscope}%
\pgfpathrectangle{\pgfqpoint{2.963410in}{7.624184in}}{\pgfqpoint{2.177280in}{2.201755in}}%
\pgfusepath{clip}%
\pgfsetbuttcap%
\pgfsetroundjoin%
\definecolor{currentfill}{rgb}{0.172549,0.627451,0.172549}%
\pgfsetfillcolor{currentfill}%
\pgfsetlinewidth{0.481800pt}%
\definecolor{currentstroke}{rgb}{1.000000,1.000000,1.000000}%
\pgfsetstrokecolor{currentstroke}%
\pgfsetdash{}{0pt}%
\pgfpathmoveto{\pgfqpoint{3.789301in}{8.516595in}}%
\pgfpathcurveto{\pgfqpoint{3.800351in}{8.516595in}}{\pgfqpoint{3.810950in}{8.520985in}}{\pgfqpoint{3.818764in}{8.528799in}}%
\pgfpathcurveto{\pgfqpoint{3.826578in}{8.536612in}}{\pgfqpoint{3.830968in}{8.547212in}}{\pgfqpoint{3.830968in}{8.558262in}}%
\pgfpathcurveto{\pgfqpoint{3.830968in}{8.569312in}}{\pgfqpoint{3.826578in}{8.579911in}}{\pgfqpoint{3.818764in}{8.587724in}}%
\pgfpathcurveto{\pgfqpoint{3.810950in}{8.595538in}}{\pgfqpoint{3.800351in}{8.599928in}}{\pgfqpoint{3.789301in}{8.599928in}}%
\pgfpathcurveto{\pgfqpoint{3.778251in}{8.599928in}}{\pgfqpoint{3.767652in}{8.595538in}}{\pgfqpoint{3.759838in}{8.587724in}}%
\pgfpathcurveto{\pgfqpoint{3.752025in}{8.579911in}}{\pgfqpoint{3.747635in}{8.569312in}}{\pgfqpoint{3.747635in}{8.558262in}}%
\pgfpathcurveto{\pgfqpoint{3.747635in}{8.547212in}}{\pgfqpoint{3.752025in}{8.536612in}}{\pgfqpoint{3.759838in}{8.528799in}}%
\pgfpathcurveto{\pgfqpoint{3.767652in}{8.520985in}}{\pgfqpoint{3.778251in}{8.516595in}}{\pgfqpoint{3.789301in}{8.516595in}}%
\pgfpathlineto{\pgfqpoint{3.789301in}{8.516595in}}%
\pgfpathclose%
\pgfusepath{stroke,fill}%
\end{pgfscope}%
\begin{pgfscope}%
\pgfpathrectangle{\pgfqpoint{2.963410in}{7.624184in}}{\pgfqpoint{2.177280in}{2.201755in}}%
\pgfusepath{clip}%
\pgfsetbuttcap%
\pgfsetroundjoin%
\definecolor{currentfill}{rgb}{0.172549,0.627451,0.172549}%
\pgfsetfillcolor{currentfill}%
\pgfsetlinewidth{0.481800pt}%
\definecolor{currentstroke}{rgb}{1.000000,1.000000,1.000000}%
\pgfsetstrokecolor{currentstroke}%
\pgfsetdash{}{0pt}%
\pgfpathmoveto{\pgfqpoint{4.025602in}{8.850194in}}%
\pgfpathcurveto{\pgfqpoint{4.036652in}{8.850194in}}{\pgfqpoint{4.047251in}{8.854584in}}{\pgfqpoint{4.055065in}{8.862398in}}%
\pgfpathcurveto{\pgfqpoint{4.062879in}{8.870212in}}{\pgfqpoint{4.067269in}{8.880811in}}{\pgfqpoint{4.067269in}{8.891861in}}%
\pgfpathcurveto{\pgfqpoint{4.067269in}{8.902911in}}{\pgfqpoint{4.062879in}{8.913510in}}{\pgfqpoint{4.055065in}{8.921324in}}%
\pgfpathcurveto{\pgfqpoint{4.047251in}{8.929137in}}{\pgfqpoint{4.036652in}{8.933528in}}{\pgfqpoint{4.025602in}{8.933528in}}%
\pgfpathcurveto{\pgfqpoint{4.014552in}{8.933528in}}{\pgfqpoint{4.003953in}{8.929137in}}{\pgfqpoint{3.996139in}{8.921324in}}%
\pgfpathcurveto{\pgfqpoint{3.988326in}{8.913510in}}{\pgfqpoint{3.983936in}{8.902911in}}{\pgfqpoint{3.983936in}{8.891861in}}%
\pgfpathcurveto{\pgfqpoint{3.983936in}{8.880811in}}{\pgfqpoint{3.988326in}{8.870212in}}{\pgfqpoint{3.996139in}{8.862398in}}%
\pgfpathcurveto{\pgfqpoint{4.003953in}{8.854584in}}{\pgfqpoint{4.014552in}{8.850194in}}{\pgfqpoint{4.025602in}{8.850194in}}%
\pgfpathlineto{\pgfqpoint{4.025602in}{8.850194in}}%
\pgfpathclose%
\pgfusepath{stroke,fill}%
\end{pgfscope}%
\begin{pgfscope}%
\pgfpathrectangle{\pgfqpoint{2.963410in}{7.624184in}}{\pgfqpoint{2.177280in}{2.201755in}}%
\pgfusepath{clip}%
\pgfsetbuttcap%
\pgfsetroundjoin%
\definecolor{currentfill}{rgb}{0.172549,0.627451,0.172549}%
\pgfsetfillcolor{currentfill}%
\pgfsetlinewidth{0.481800pt}%
\definecolor{currentstroke}{rgb}{1.000000,1.000000,1.000000}%
\pgfsetstrokecolor{currentstroke}%
\pgfsetdash{}{0pt}%
\pgfpathmoveto{\pgfqpoint{3.907452in}{8.905794in}}%
\pgfpathcurveto{\pgfqpoint{3.918502in}{8.905794in}}{\pgfqpoint{3.929101in}{8.910184in}}{\pgfqpoint{3.936915in}{8.917998in}}%
\pgfpathcurveto{\pgfqpoint{3.944728in}{8.925812in}}{\pgfqpoint{3.949118in}{8.936411in}}{\pgfqpoint{3.949118in}{8.947461in}}%
\pgfpathcurveto{\pgfqpoint{3.949118in}{8.958511in}}{\pgfqpoint{3.944728in}{8.969110in}}{\pgfqpoint{3.936915in}{8.976924in}}%
\pgfpathcurveto{\pgfqpoint{3.929101in}{8.984737in}}{\pgfqpoint{3.918502in}{8.989127in}}{\pgfqpoint{3.907452in}{8.989127in}}%
\pgfpathcurveto{\pgfqpoint{3.896402in}{8.989127in}}{\pgfqpoint{3.885803in}{8.984737in}}{\pgfqpoint{3.877989in}{8.976924in}}%
\pgfpathcurveto{\pgfqpoint{3.870175in}{8.969110in}}{\pgfqpoint{3.865785in}{8.958511in}}{\pgfqpoint{3.865785in}{8.947461in}}%
\pgfpathcurveto{\pgfqpoint{3.865785in}{8.936411in}}{\pgfqpoint{3.870175in}{8.925812in}}{\pgfqpoint{3.877989in}{8.917998in}}%
\pgfpathcurveto{\pgfqpoint{3.885803in}{8.910184in}}{\pgfqpoint{3.896402in}{8.905794in}}{\pgfqpoint{3.907452in}{8.905794in}}%
\pgfpathlineto{\pgfqpoint{3.907452in}{8.905794in}}%
\pgfpathclose%
\pgfusepath{stroke,fill}%
\end{pgfscope}%
\begin{pgfscope}%
\pgfpathrectangle{\pgfqpoint{2.963410in}{7.624184in}}{\pgfqpoint{2.177280in}{2.201755in}}%
\pgfusepath{clip}%
\pgfsetbuttcap%
\pgfsetroundjoin%
\definecolor{currentfill}{rgb}{0.172549,0.627451,0.172549}%
\pgfsetfillcolor{currentfill}%
\pgfsetlinewidth{0.481800pt}%
\definecolor{currentstroke}{rgb}{1.000000,1.000000,1.000000}%
\pgfsetstrokecolor{currentstroke}%
\pgfsetdash{}{0pt}%
\pgfpathmoveto{\pgfqpoint{4.380054in}{9.572993in}}%
\pgfpathcurveto{\pgfqpoint{4.391104in}{9.572993in}}{\pgfqpoint{4.401703in}{9.577383in}}{\pgfqpoint{4.409517in}{9.585197in}}%
\pgfpathcurveto{\pgfqpoint{4.417330in}{9.593010in}}{\pgfqpoint{4.421721in}{9.603609in}}{\pgfqpoint{4.421721in}{9.614659in}}%
\pgfpathcurveto{\pgfqpoint{4.421721in}{9.625709in}}{\pgfqpoint{4.417330in}{9.636308in}}{\pgfqpoint{4.409517in}{9.644122in}}%
\pgfpathcurveto{\pgfqpoint{4.401703in}{9.651936in}}{\pgfqpoint{4.391104in}{9.656326in}}{\pgfqpoint{4.380054in}{9.656326in}}%
\pgfpathcurveto{\pgfqpoint{4.369004in}{9.656326in}}{\pgfqpoint{4.358405in}{9.651936in}}{\pgfqpoint{4.350591in}{9.644122in}}%
\pgfpathcurveto{\pgfqpoint{4.342777in}{9.636308in}}{\pgfqpoint{4.338387in}{9.625709in}}{\pgfqpoint{4.338387in}{9.614659in}}%
\pgfpathcurveto{\pgfqpoint{4.338387in}{9.603609in}}{\pgfqpoint{4.342777in}{9.593010in}}{\pgfqpoint{4.350591in}{9.585197in}}%
\pgfpathcurveto{\pgfqpoint{4.358405in}{9.577383in}}{\pgfqpoint{4.369004in}{9.572993in}}{\pgfqpoint{4.380054in}{9.572993in}}%
\pgfpathlineto{\pgfqpoint{4.380054in}{9.572993in}}%
\pgfpathclose%
\pgfusepath{stroke,fill}%
\end{pgfscope}%
\begin{pgfscope}%
\pgfpathrectangle{\pgfqpoint{2.963410in}{7.624184in}}{\pgfqpoint{2.177280in}{2.201755in}}%
\pgfusepath{clip}%
\pgfsetbuttcap%
\pgfsetroundjoin%
\definecolor{currentfill}{rgb}{0.172549,0.627451,0.172549}%
\pgfsetfillcolor{currentfill}%
\pgfsetlinewidth{0.481800pt}%
\definecolor{currentstroke}{rgb}{1.000000,1.000000,1.000000}%
\pgfsetstrokecolor{currentstroke}%
\pgfsetdash{}{0pt}%
\pgfpathmoveto{\pgfqpoint{3.671151in}{9.572993in}}%
\pgfpathcurveto{\pgfqpoint{3.682201in}{9.572993in}}{\pgfqpoint{3.692800in}{9.577383in}}{\pgfqpoint{3.700613in}{9.585197in}}%
\pgfpathcurveto{\pgfqpoint{3.708427in}{9.593010in}}{\pgfqpoint{3.712817in}{9.603609in}}{\pgfqpoint{3.712817in}{9.614659in}}%
\pgfpathcurveto{\pgfqpoint{3.712817in}{9.625709in}}{\pgfqpoint{3.708427in}{9.636308in}}{\pgfqpoint{3.700613in}{9.644122in}}%
\pgfpathcurveto{\pgfqpoint{3.692800in}{9.651936in}}{\pgfqpoint{3.682201in}{9.656326in}}{\pgfqpoint{3.671151in}{9.656326in}}%
\pgfpathcurveto{\pgfqpoint{3.660101in}{9.656326in}}{\pgfqpoint{3.649501in}{9.651936in}}{\pgfqpoint{3.641688in}{9.644122in}}%
\pgfpathcurveto{\pgfqpoint{3.633874in}{9.636308in}}{\pgfqpoint{3.629484in}{9.625709in}}{\pgfqpoint{3.629484in}{9.614659in}}%
\pgfpathcurveto{\pgfqpoint{3.629484in}{9.603609in}}{\pgfqpoint{3.633874in}{9.593010in}}{\pgfqpoint{3.641688in}{9.585197in}}%
\pgfpathcurveto{\pgfqpoint{3.649501in}{9.577383in}}{\pgfqpoint{3.660101in}{9.572993in}}{\pgfqpoint{3.671151in}{9.572993in}}%
\pgfpathlineto{\pgfqpoint{3.671151in}{9.572993in}}%
\pgfpathclose%
\pgfusepath{stroke,fill}%
\end{pgfscope}%
\begin{pgfscope}%
\pgfpathrectangle{\pgfqpoint{2.963410in}{7.624184in}}{\pgfqpoint{2.177280in}{2.201755in}}%
\pgfusepath{clip}%
\pgfsetbuttcap%
\pgfsetroundjoin%
\definecolor{currentfill}{rgb}{0.172549,0.627451,0.172549}%
\pgfsetfillcolor{currentfill}%
\pgfsetlinewidth{0.481800pt}%
\definecolor{currentstroke}{rgb}{1.000000,1.000000,1.000000}%
\pgfsetstrokecolor{currentstroke}%
\pgfsetdash{}{0pt}%
\pgfpathmoveto{\pgfqpoint{3.434850in}{8.627795in}}%
\pgfpathcurveto{\pgfqpoint{3.445900in}{8.627795in}}{\pgfqpoint{3.456499in}{8.632185in}}{\pgfqpoint{3.464312in}{8.639999in}}%
\pgfpathcurveto{\pgfqpoint{3.472126in}{8.647812in}}{\pgfqpoint{3.476516in}{8.658411in}}{\pgfqpoint{3.476516in}{8.669461in}}%
\pgfpathcurveto{\pgfqpoint{3.476516in}{8.680512in}}{\pgfqpoint{3.472126in}{8.691111in}}{\pgfqpoint{3.464312in}{8.698924in}}%
\pgfpathcurveto{\pgfqpoint{3.456499in}{8.706738in}}{\pgfqpoint{3.445900in}{8.711128in}}{\pgfqpoint{3.434850in}{8.711128in}}%
\pgfpathcurveto{\pgfqpoint{3.423799in}{8.711128in}}{\pgfqpoint{3.413200in}{8.706738in}}{\pgfqpoint{3.405387in}{8.698924in}}%
\pgfpathcurveto{\pgfqpoint{3.397573in}{8.691111in}}{\pgfqpoint{3.393183in}{8.680512in}}{\pgfqpoint{3.393183in}{8.669461in}}%
\pgfpathcurveto{\pgfqpoint{3.393183in}{8.658411in}}{\pgfqpoint{3.397573in}{8.647812in}}{\pgfqpoint{3.405387in}{8.639999in}}%
\pgfpathcurveto{\pgfqpoint{3.413200in}{8.632185in}}{\pgfqpoint{3.423799in}{8.627795in}}{\pgfqpoint{3.434850in}{8.627795in}}%
\pgfpathlineto{\pgfqpoint{3.434850in}{8.627795in}}%
\pgfpathclose%
\pgfusepath{stroke,fill}%
\end{pgfscope}%
\begin{pgfscope}%
\pgfpathrectangle{\pgfqpoint{2.963410in}{7.624184in}}{\pgfqpoint{2.177280in}{2.201755in}}%
\pgfusepath{clip}%
\pgfsetbuttcap%
\pgfsetroundjoin%
\definecolor{currentfill}{rgb}{0.172549,0.627451,0.172549}%
\pgfsetfillcolor{currentfill}%
\pgfsetlinewidth{0.481800pt}%
\definecolor{currentstroke}{rgb}{1.000000,1.000000,1.000000}%
\pgfsetstrokecolor{currentstroke}%
\pgfsetdash{}{0pt}%
\pgfpathmoveto{\pgfqpoint{4.025602in}{9.128194in}}%
\pgfpathcurveto{\pgfqpoint{4.036652in}{9.128194in}}{\pgfqpoint{4.047251in}{9.132584in}}{\pgfqpoint{4.055065in}{9.140397in}}%
\pgfpathcurveto{\pgfqpoint{4.062879in}{9.148211in}}{\pgfqpoint{4.067269in}{9.158810in}}{\pgfqpoint{4.067269in}{9.169860in}}%
\pgfpathcurveto{\pgfqpoint{4.067269in}{9.180910in}}{\pgfqpoint{4.062879in}{9.191509in}}{\pgfqpoint{4.055065in}{9.199323in}}%
\pgfpathcurveto{\pgfqpoint{4.047251in}{9.207137in}}{\pgfqpoint{4.036652in}{9.211527in}}{\pgfqpoint{4.025602in}{9.211527in}}%
\pgfpathcurveto{\pgfqpoint{4.014552in}{9.211527in}}{\pgfqpoint{4.003953in}{9.207137in}}{\pgfqpoint{3.996139in}{9.199323in}}%
\pgfpathcurveto{\pgfqpoint{3.988326in}{9.191509in}}{\pgfqpoint{3.983936in}{9.180910in}}{\pgfqpoint{3.983936in}{9.169860in}}%
\pgfpathcurveto{\pgfqpoint{3.983936in}{9.158810in}}{\pgfqpoint{3.988326in}{9.148211in}}{\pgfqpoint{3.996139in}{9.140397in}}%
\pgfpathcurveto{\pgfqpoint{4.003953in}{9.132584in}}{\pgfqpoint{4.014552in}{9.128194in}}{\pgfqpoint{4.025602in}{9.128194in}}%
\pgfpathlineto{\pgfqpoint{4.025602in}{9.128194in}}%
\pgfpathclose%
\pgfusepath{stroke,fill}%
\end{pgfscope}%
\begin{pgfscope}%
\pgfpathrectangle{\pgfqpoint{2.963410in}{7.624184in}}{\pgfqpoint{2.177280in}{2.201755in}}%
\pgfusepath{clip}%
\pgfsetbuttcap%
\pgfsetroundjoin%
\definecolor{currentfill}{rgb}{0.172549,0.627451,0.172549}%
\pgfsetfillcolor{currentfill}%
\pgfsetlinewidth{0.481800pt}%
\definecolor{currentstroke}{rgb}{1.000000,1.000000,1.000000}%
\pgfsetstrokecolor{currentstroke}%
\pgfsetdash{}{0pt}%
\pgfpathmoveto{\pgfqpoint{3.789301in}{8.405395in}}%
\pgfpathcurveto{\pgfqpoint{3.800351in}{8.405395in}}{\pgfqpoint{3.810950in}{8.409785in}}{\pgfqpoint{3.818764in}{8.417599in}}%
\pgfpathcurveto{\pgfqpoint{3.826578in}{8.425413in}}{\pgfqpoint{3.830968in}{8.436012in}}{\pgfqpoint{3.830968in}{8.447062in}}%
\pgfpathcurveto{\pgfqpoint{3.830968in}{8.458112in}}{\pgfqpoint{3.826578in}{8.468711in}}{\pgfqpoint{3.818764in}{8.476525in}}%
\pgfpathcurveto{\pgfqpoint{3.810950in}{8.484338in}}{\pgfqpoint{3.800351in}{8.488729in}}{\pgfqpoint{3.789301in}{8.488729in}}%
\pgfpathcurveto{\pgfqpoint{3.778251in}{8.488729in}}{\pgfqpoint{3.767652in}{8.484338in}}{\pgfqpoint{3.759838in}{8.476525in}}%
\pgfpathcurveto{\pgfqpoint{3.752025in}{8.468711in}}{\pgfqpoint{3.747635in}{8.458112in}}{\pgfqpoint{3.747635in}{8.447062in}}%
\pgfpathcurveto{\pgfqpoint{3.747635in}{8.436012in}}{\pgfqpoint{3.752025in}{8.425413in}}{\pgfqpoint{3.759838in}{8.417599in}}%
\pgfpathcurveto{\pgfqpoint{3.767652in}{8.409785in}}{\pgfqpoint{3.778251in}{8.405395in}}{\pgfqpoint{3.789301in}{8.405395in}}%
\pgfpathlineto{\pgfqpoint{3.789301in}{8.405395in}}%
\pgfpathclose%
\pgfusepath{stroke,fill}%
\end{pgfscope}%
\begin{pgfscope}%
\pgfpathrectangle{\pgfqpoint{2.963410in}{7.624184in}}{\pgfqpoint{2.177280in}{2.201755in}}%
\pgfusepath{clip}%
\pgfsetbuttcap%
\pgfsetroundjoin%
\definecolor{currentfill}{rgb}{0.172549,0.627451,0.172549}%
\pgfsetfillcolor{currentfill}%
\pgfsetlinewidth{0.481800pt}%
\definecolor{currentstroke}{rgb}{1.000000,1.000000,1.000000}%
\pgfsetstrokecolor{currentstroke}%
\pgfsetdash{}{0pt}%
\pgfpathmoveto{\pgfqpoint{3.789301in}{9.572993in}}%
\pgfpathcurveto{\pgfqpoint{3.800351in}{9.572993in}}{\pgfqpoint{3.810950in}{9.577383in}}{\pgfqpoint{3.818764in}{9.585197in}}%
\pgfpathcurveto{\pgfqpoint{3.826578in}{9.593010in}}{\pgfqpoint{3.830968in}{9.603609in}}{\pgfqpoint{3.830968in}{9.614659in}}%
\pgfpathcurveto{\pgfqpoint{3.830968in}{9.625709in}}{\pgfqpoint{3.826578in}{9.636308in}}{\pgfqpoint{3.818764in}{9.644122in}}%
\pgfpathcurveto{\pgfqpoint{3.810950in}{9.651936in}}{\pgfqpoint{3.800351in}{9.656326in}}{\pgfqpoint{3.789301in}{9.656326in}}%
\pgfpathcurveto{\pgfqpoint{3.778251in}{9.656326in}}{\pgfqpoint{3.767652in}{9.651936in}}{\pgfqpoint{3.759838in}{9.644122in}}%
\pgfpathcurveto{\pgfqpoint{3.752025in}{9.636308in}}{\pgfqpoint{3.747635in}{9.625709in}}{\pgfqpoint{3.747635in}{9.614659in}}%
\pgfpathcurveto{\pgfqpoint{3.747635in}{9.603609in}}{\pgfqpoint{3.752025in}{9.593010in}}{\pgfqpoint{3.759838in}{9.585197in}}%
\pgfpathcurveto{\pgfqpoint{3.767652in}{9.577383in}}{\pgfqpoint{3.778251in}{9.572993in}}{\pgfqpoint{3.789301in}{9.572993in}}%
\pgfpathlineto{\pgfqpoint{3.789301in}{9.572993in}}%
\pgfpathclose%
\pgfusepath{stroke,fill}%
\end{pgfscope}%
\begin{pgfscope}%
\pgfpathrectangle{\pgfqpoint{2.963410in}{7.624184in}}{\pgfqpoint{2.177280in}{2.201755in}}%
\pgfusepath{clip}%
\pgfsetbuttcap%
\pgfsetroundjoin%
\definecolor{currentfill}{rgb}{0.172549,0.627451,0.172549}%
\pgfsetfillcolor{currentfill}%
\pgfsetlinewidth{0.481800pt}%
\definecolor{currentstroke}{rgb}{1.000000,1.000000,1.000000}%
\pgfsetstrokecolor{currentstroke}%
\pgfsetdash{}{0pt}%
\pgfpathmoveto{\pgfqpoint{3.730226in}{8.794594in}}%
\pgfpathcurveto{\pgfqpoint{3.741276in}{8.794594in}}{\pgfqpoint{3.751875in}{8.798985in}}{\pgfqpoint{3.759689in}{8.806798in}}%
\pgfpathcurveto{\pgfqpoint{3.767502in}{8.814612in}}{\pgfqpoint{3.771893in}{8.825211in}}{\pgfqpoint{3.771893in}{8.836261in}}%
\pgfpathcurveto{\pgfqpoint{3.771893in}{8.847311in}}{\pgfqpoint{3.767502in}{8.857910in}}{\pgfqpoint{3.759689in}{8.865724in}}%
\pgfpathcurveto{\pgfqpoint{3.751875in}{8.873537in}}{\pgfqpoint{3.741276in}{8.877928in}}{\pgfqpoint{3.730226in}{8.877928in}}%
\pgfpathcurveto{\pgfqpoint{3.719176in}{8.877928in}}{\pgfqpoint{3.708577in}{8.873537in}}{\pgfqpoint{3.700763in}{8.865724in}}%
\pgfpathcurveto{\pgfqpoint{3.692950in}{8.857910in}}{\pgfqpoint{3.688559in}{8.847311in}}{\pgfqpoint{3.688559in}{8.836261in}}%
\pgfpathcurveto{\pgfqpoint{3.688559in}{8.825211in}}{\pgfqpoint{3.692950in}{8.814612in}}{\pgfqpoint{3.700763in}{8.806798in}}%
\pgfpathcurveto{\pgfqpoint{3.708577in}{8.798985in}}{\pgfqpoint{3.719176in}{8.794594in}}{\pgfqpoint{3.730226in}{8.794594in}}%
\pgfpathlineto{\pgfqpoint{3.730226in}{8.794594in}}%
\pgfpathclose%
\pgfusepath{stroke,fill}%
\end{pgfscope}%
\begin{pgfscope}%
\pgfpathrectangle{\pgfqpoint{2.963410in}{7.624184in}}{\pgfqpoint{2.177280in}{2.201755in}}%
\pgfusepath{clip}%
\pgfsetbuttcap%
\pgfsetroundjoin%
\definecolor{currentfill}{rgb}{0.172549,0.627451,0.172549}%
\pgfsetfillcolor{currentfill}%
\pgfsetlinewidth{0.481800pt}%
\definecolor{currentstroke}{rgb}{1.000000,1.000000,1.000000}%
\pgfsetstrokecolor{currentstroke}%
\pgfsetdash{}{0pt}%
\pgfpathmoveto{\pgfqpoint{4.084678in}{9.016994in}}%
\pgfpathcurveto{\pgfqpoint{4.095728in}{9.016994in}}{\pgfqpoint{4.106327in}{9.021384in}}{\pgfqpoint{4.114140in}{9.029198in}}%
\pgfpathcurveto{\pgfqpoint{4.121954in}{9.037011in}}{\pgfqpoint{4.126344in}{9.047610in}}{\pgfqpoint{4.126344in}{9.058661in}}%
\pgfpathcurveto{\pgfqpoint{4.126344in}{9.069711in}}{\pgfqpoint{4.121954in}{9.080310in}}{\pgfqpoint{4.114140in}{9.088123in}}%
\pgfpathcurveto{\pgfqpoint{4.106327in}{9.095937in}}{\pgfqpoint{4.095728in}{9.100327in}}{\pgfqpoint{4.084678in}{9.100327in}}%
\pgfpathcurveto{\pgfqpoint{4.073627in}{9.100327in}}{\pgfqpoint{4.063028in}{9.095937in}}{\pgfqpoint{4.055215in}{9.088123in}}%
\pgfpathcurveto{\pgfqpoint{4.047401in}{9.080310in}}{\pgfqpoint{4.043011in}{9.069711in}}{\pgfqpoint{4.043011in}{9.058661in}}%
\pgfpathcurveto{\pgfqpoint{4.043011in}{9.047610in}}{\pgfqpoint{4.047401in}{9.037011in}}{\pgfqpoint{4.055215in}{9.029198in}}%
\pgfpathcurveto{\pgfqpoint{4.063028in}{9.021384in}}{\pgfqpoint{4.073627in}{9.016994in}}{\pgfqpoint{4.084678in}{9.016994in}}%
\pgfpathlineto{\pgfqpoint{4.084678in}{9.016994in}}%
\pgfpathclose%
\pgfusepath{stroke,fill}%
\end{pgfscope}%
\begin{pgfscope}%
\pgfpathrectangle{\pgfqpoint{2.963410in}{7.624184in}}{\pgfqpoint{2.177280in}{2.201755in}}%
\pgfusepath{clip}%
\pgfsetbuttcap%
\pgfsetroundjoin%
\definecolor{currentfill}{rgb}{0.172549,0.627451,0.172549}%
\pgfsetfillcolor{currentfill}%
\pgfsetlinewidth{0.481800pt}%
\definecolor{currentstroke}{rgb}{1.000000,1.000000,1.000000}%
\pgfsetstrokecolor{currentstroke}%
\pgfsetdash{}{0pt}%
\pgfpathmoveto{\pgfqpoint{4.025602in}{9.294993in}}%
\pgfpathcurveto{\pgfqpoint{4.036652in}{9.294993in}}{\pgfqpoint{4.047251in}{9.299384in}}{\pgfqpoint{4.055065in}{9.307197in}}%
\pgfpathcurveto{\pgfqpoint{4.062879in}{9.315011in}}{\pgfqpoint{4.067269in}{9.325610in}}{\pgfqpoint{4.067269in}{9.336660in}}%
\pgfpathcurveto{\pgfqpoint{4.067269in}{9.347710in}}{\pgfqpoint{4.062879in}{9.358309in}}{\pgfqpoint{4.055065in}{9.366123in}}%
\pgfpathcurveto{\pgfqpoint{4.047251in}{9.373936in}}{\pgfqpoint{4.036652in}{9.378327in}}{\pgfqpoint{4.025602in}{9.378327in}}%
\pgfpathcurveto{\pgfqpoint{4.014552in}{9.378327in}}{\pgfqpoint{4.003953in}{9.373936in}}{\pgfqpoint{3.996139in}{9.366123in}}%
\pgfpathcurveto{\pgfqpoint{3.988326in}{9.358309in}}{\pgfqpoint{3.983936in}{9.347710in}}{\pgfqpoint{3.983936in}{9.336660in}}%
\pgfpathcurveto{\pgfqpoint{3.983936in}{9.325610in}}{\pgfqpoint{3.988326in}{9.315011in}}{\pgfqpoint{3.996139in}{9.307197in}}%
\pgfpathcurveto{\pgfqpoint{4.003953in}{9.299384in}}{\pgfqpoint{4.014552in}{9.294993in}}{\pgfqpoint{4.025602in}{9.294993in}}%
\pgfpathlineto{\pgfqpoint{4.025602in}{9.294993in}}%
\pgfpathclose%
\pgfusepath{stroke,fill}%
\end{pgfscope}%
\begin{pgfscope}%
\pgfpathrectangle{\pgfqpoint{2.963410in}{7.624184in}}{\pgfqpoint{2.177280in}{2.201755in}}%
\pgfusepath{clip}%
\pgfsetbuttcap%
\pgfsetroundjoin%
\definecolor{currentfill}{rgb}{0.172549,0.627451,0.172549}%
\pgfsetfillcolor{currentfill}%
\pgfsetlinewidth{0.481800pt}%
\definecolor{currentstroke}{rgb}{1.000000,1.000000,1.000000}%
\pgfsetstrokecolor{currentstroke}%
\pgfsetdash{}{0pt}%
\pgfpathmoveto{\pgfqpoint{3.789301in}{8.738994in}}%
\pgfpathcurveto{\pgfqpoint{3.800351in}{8.738994in}}{\pgfqpoint{3.810950in}{8.743385in}}{\pgfqpoint{3.818764in}{8.751198in}}%
\pgfpathcurveto{\pgfqpoint{3.826578in}{8.759012in}}{\pgfqpoint{3.830968in}{8.769611in}}{\pgfqpoint{3.830968in}{8.780661in}}%
\pgfpathcurveto{\pgfqpoint{3.830968in}{8.791711in}}{\pgfqpoint{3.826578in}{8.802310in}}{\pgfqpoint{3.818764in}{8.810124in}}%
\pgfpathcurveto{\pgfqpoint{3.810950in}{8.817938in}}{\pgfqpoint{3.800351in}{8.822328in}}{\pgfqpoint{3.789301in}{8.822328in}}%
\pgfpathcurveto{\pgfqpoint{3.778251in}{8.822328in}}{\pgfqpoint{3.767652in}{8.817938in}}{\pgfqpoint{3.759838in}{8.810124in}}%
\pgfpathcurveto{\pgfqpoint{3.752025in}{8.802310in}}{\pgfqpoint{3.747635in}{8.791711in}}{\pgfqpoint{3.747635in}{8.780661in}}%
\pgfpathcurveto{\pgfqpoint{3.747635in}{8.769611in}}{\pgfqpoint{3.752025in}{8.759012in}}{\pgfqpoint{3.759838in}{8.751198in}}%
\pgfpathcurveto{\pgfqpoint{3.767652in}{8.743385in}}{\pgfqpoint{3.778251in}{8.738994in}}{\pgfqpoint{3.789301in}{8.738994in}}%
\pgfpathlineto{\pgfqpoint{3.789301in}{8.738994in}}%
\pgfpathclose%
\pgfusepath{stroke,fill}%
\end{pgfscope}%
\begin{pgfscope}%
\pgfpathrectangle{\pgfqpoint{2.963410in}{7.624184in}}{\pgfqpoint{2.177280in}{2.201755in}}%
\pgfusepath{clip}%
\pgfsetbuttcap%
\pgfsetroundjoin%
\definecolor{currentfill}{rgb}{0.172549,0.627451,0.172549}%
\pgfsetfillcolor{currentfill}%
\pgfsetlinewidth{0.481800pt}%
\definecolor{currentstroke}{rgb}{1.000000,1.000000,1.000000}%
\pgfsetstrokecolor{currentstroke}%
\pgfsetdash{}{0pt}%
\pgfpathmoveto{\pgfqpoint{3.907452in}{8.683395in}}%
\pgfpathcurveto{\pgfqpoint{3.918502in}{8.683395in}}{\pgfqpoint{3.929101in}{8.687785in}}{\pgfqpoint{3.936915in}{8.695598in}}%
\pgfpathcurveto{\pgfqpoint{3.944728in}{8.703412in}}{\pgfqpoint{3.949118in}{8.714011in}}{\pgfqpoint{3.949118in}{8.725061in}}%
\pgfpathcurveto{\pgfqpoint{3.949118in}{8.736111in}}{\pgfqpoint{3.944728in}{8.746710in}}{\pgfqpoint{3.936915in}{8.754524in}}%
\pgfpathcurveto{\pgfqpoint{3.929101in}{8.762338in}}{\pgfqpoint{3.918502in}{8.766728in}}{\pgfqpoint{3.907452in}{8.766728in}}%
\pgfpathcurveto{\pgfqpoint{3.896402in}{8.766728in}}{\pgfqpoint{3.885803in}{8.762338in}}{\pgfqpoint{3.877989in}{8.754524in}}%
\pgfpathcurveto{\pgfqpoint{3.870175in}{8.746710in}}{\pgfqpoint{3.865785in}{8.736111in}}{\pgfqpoint{3.865785in}{8.725061in}}%
\pgfpathcurveto{\pgfqpoint{3.865785in}{8.714011in}}{\pgfqpoint{3.870175in}{8.703412in}}{\pgfqpoint{3.877989in}{8.695598in}}%
\pgfpathcurveto{\pgfqpoint{3.885803in}{8.687785in}}{\pgfqpoint{3.896402in}{8.683395in}}{\pgfqpoint{3.907452in}{8.683395in}}%
\pgfpathlineto{\pgfqpoint{3.907452in}{8.683395in}}%
\pgfpathclose%
\pgfusepath{stroke,fill}%
\end{pgfscope}%
\begin{pgfscope}%
\pgfpathrectangle{\pgfqpoint{2.963410in}{7.624184in}}{\pgfqpoint{2.177280in}{2.201755in}}%
\pgfusepath{clip}%
\pgfsetbuttcap%
\pgfsetroundjoin%
\definecolor{currentfill}{rgb}{0.172549,0.627451,0.172549}%
\pgfsetfillcolor{currentfill}%
\pgfsetlinewidth{0.481800pt}%
\definecolor{currentstroke}{rgb}{1.000000,1.000000,1.000000}%
\pgfsetstrokecolor{currentstroke}%
\pgfsetdash{}{0pt}%
\pgfpathmoveto{\pgfqpoint{3.789301in}{8.850194in}}%
\pgfpathcurveto{\pgfqpoint{3.800351in}{8.850194in}}{\pgfqpoint{3.810950in}{8.854584in}}{\pgfqpoint{3.818764in}{8.862398in}}%
\pgfpathcurveto{\pgfqpoint{3.826578in}{8.870212in}}{\pgfqpoint{3.830968in}{8.880811in}}{\pgfqpoint{3.830968in}{8.891861in}}%
\pgfpathcurveto{\pgfqpoint{3.830968in}{8.902911in}}{\pgfqpoint{3.826578in}{8.913510in}}{\pgfqpoint{3.818764in}{8.921324in}}%
\pgfpathcurveto{\pgfqpoint{3.810950in}{8.929137in}}{\pgfqpoint{3.800351in}{8.933528in}}{\pgfqpoint{3.789301in}{8.933528in}}%
\pgfpathcurveto{\pgfqpoint{3.778251in}{8.933528in}}{\pgfqpoint{3.767652in}{8.929137in}}{\pgfqpoint{3.759838in}{8.921324in}}%
\pgfpathcurveto{\pgfqpoint{3.752025in}{8.913510in}}{\pgfqpoint{3.747635in}{8.902911in}}{\pgfqpoint{3.747635in}{8.891861in}}%
\pgfpathcurveto{\pgfqpoint{3.747635in}{8.880811in}}{\pgfqpoint{3.752025in}{8.870212in}}{\pgfqpoint{3.759838in}{8.862398in}}%
\pgfpathcurveto{\pgfqpoint{3.767652in}{8.854584in}}{\pgfqpoint{3.778251in}{8.850194in}}{\pgfqpoint{3.789301in}{8.850194in}}%
\pgfpathlineto{\pgfqpoint{3.789301in}{8.850194in}}%
\pgfpathclose%
\pgfusepath{stroke,fill}%
\end{pgfscope}%
\begin{pgfscope}%
\pgfpathrectangle{\pgfqpoint{2.963410in}{7.624184in}}{\pgfqpoint{2.177280in}{2.201755in}}%
\pgfusepath{clip}%
\pgfsetbuttcap%
\pgfsetroundjoin%
\definecolor{currentfill}{rgb}{0.172549,0.627451,0.172549}%
\pgfsetfillcolor{currentfill}%
\pgfsetlinewidth{0.481800pt}%
\definecolor{currentstroke}{rgb}{1.000000,1.000000,1.000000}%
\pgfsetstrokecolor{currentstroke}%
\pgfsetdash{}{0pt}%
\pgfpathmoveto{\pgfqpoint{3.907452in}{9.294993in}}%
\pgfpathcurveto{\pgfqpoint{3.918502in}{9.294993in}}{\pgfqpoint{3.929101in}{9.299384in}}{\pgfqpoint{3.936915in}{9.307197in}}%
\pgfpathcurveto{\pgfqpoint{3.944728in}{9.315011in}}{\pgfqpoint{3.949118in}{9.325610in}}{\pgfqpoint{3.949118in}{9.336660in}}%
\pgfpathcurveto{\pgfqpoint{3.949118in}{9.347710in}}{\pgfqpoint{3.944728in}{9.358309in}}{\pgfqpoint{3.936915in}{9.366123in}}%
\pgfpathcurveto{\pgfqpoint{3.929101in}{9.373936in}}{\pgfqpoint{3.918502in}{9.378327in}}{\pgfqpoint{3.907452in}{9.378327in}}%
\pgfpathcurveto{\pgfqpoint{3.896402in}{9.378327in}}{\pgfqpoint{3.885803in}{9.373936in}}{\pgfqpoint{3.877989in}{9.366123in}}%
\pgfpathcurveto{\pgfqpoint{3.870175in}{9.358309in}}{\pgfqpoint{3.865785in}{9.347710in}}{\pgfqpoint{3.865785in}{9.336660in}}%
\pgfpathcurveto{\pgfqpoint{3.865785in}{9.325610in}}{\pgfqpoint{3.870175in}{9.315011in}}{\pgfqpoint{3.877989in}{9.307197in}}%
\pgfpathcurveto{\pgfqpoint{3.885803in}{9.299384in}}{\pgfqpoint{3.896402in}{9.294993in}}{\pgfqpoint{3.907452in}{9.294993in}}%
\pgfpathlineto{\pgfqpoint{3.907452in}{9.294993in}}%
\pgfpathclose%
\pgfusepath{stroke,fill}%
\end{pgfscope}%
\begin{pgfscope}%
\pgfpathrectangle{\pgfqpoint{2.963410in}{7.624184in}}{\pgfqpoint{2.177280in}{2.201755in}}%
\pgfusepath{clip}%
\pgfsetbuttcap%
\pgfsetroundjoin%
\definecolor{currentfill}{rgb}{0.172549,0.627451,0.172549}%
\pgfsetfillcolor{currentfill}%
\pgfsetlinewidth{0.481800pt}%
\definecolor{currentstroke}{rgb}{1.000000,1.000000,1.000000}%
\pgfsetstrokecolor{currentstroke}%
\pgfsetdash{}{0pt}%
\pgfpathmoveto{\pgfqpoint{3.789301in}{9.406193in}}%
\pgfpathcurveto{\pgfqpoint{3.800351in}{9.406193in}}{\pgfqpoint{3.810950in}{9.410583in}}{\pgfqpoint{3.818764in}{9.418397in}}%
\pgfpathcurveto{\pgfqpoint{3.826578in}{9.426211in}}{\pgfqpoint{3.830968in}{9.436810in}}{\pgfqpoint{3.830968in}{9.447860in}}%
\pgfpathcurveto{\pgfqpoint{3.830968in}{9.458910in}}{\pgfqpoint{3.826578in}{9.469509in}}{\pgfqpoint{3.818764in}{9.477322in}}%
\pgfpathcurveto{\pgfqpoint{3.810950in}{9.485136in}}{\pgfqpoint{3.800351in}{9.489526in}}{\pgfqpoint{3.789301in}{9.489526in}}%
\pgfpathcurveto{\pgfqpoint{3.778251in}{9.489526in}}{\pgfqpoint{3.767652in}{9.485136in}}{\pgfqpoint{3.759838in}{9.477322in}}%
\pgfpathcurveto{\pgfqpoint{3.752025in}{9.469509in}}{\pgfqpoint{3.747635in}{9.458910in}}{\pgfqpoint{3.747635in}{9.447860in}}%
\pgfpathcurveto{\pgfqpoint{3.747635in}{9.436810in}}{\pgfqpoint{3.752025in}{9.426211in}}{\pgfqpoint{3.759838in}{9.418397in}}%
\pgfpathcurveto{\pgfqpoint{3.767652in}{9.410583in}}{\pgfqpoint{3.778251in}{9.406193in}}{\pgfqpoint{3.789301in}{9.406193in}}%
\pgfpathlineto{\pgfqpoint{3.789301in}{9.406193in}}%
\pgfpathclose%
\pgfusepath{stroke,fill}%
\end{pgfscope}%
\begin{pgfscope}%
\pgfpathrectangle{\pgfqpoint{2.963410in}{7.624184in}}{\pgfqpoint{2.177280in}{2.201755in}}%
\pgfusepath{clip}%
\pgfsetbuttcap%
\pgfsetroundjoin%
\definecolor{currentfill}{rgb}{0.172549,0.627451,0.172549}%
\pgfsetfillcolor{currentfill}%
\pgfsetlinewidth{0.481800pt}%
\definecolor{currentstroke}{rgb}{1.000000,1.000000,1.000000}%
\pgfsetstrokecolor{currentstroke}%
\pgfsetdash{}{0pt}%
\pgfpathmoveto{\pgfqpoint{4.380054in}{9.684192in}}%
\pgfpathcurveto{\pgfqpoint{4.391104in}{9.684192in}}{\pgfqpoint{4.401703in}{9.688583in}}{\pgfqpoint{4.409517in}{9.696396in}}%
\pgfpathcurveto{\pgfqpoint{4.417330in}{9.704210in}}{\pgfqpoint{4.421721in}{9.714809in}}{\pgfqpoint{4.421721in}{9.725859in}}%
\pgfpathcurveto{\pgfqpoint{4.421721in}{9.736909in}}{\pgfqpoint{4.417330in}{9.747508in}}{\pgfqpoint{4.409517in}{9.755322in}}%
\pgfpathcurveto{\pgfqpoint{4.401703in}{9.763135in}}{\pgfqpoint{4.391104in}{9.767526in}}{\pgfqpoint{4.380054in}{9.767526in}}%
\pgfpathcurveto{\pgfqpoint{4.369004in}{9.767526in}}{\pgfqpoint{4.358405in}{9.763135in}}{\pgfqpoint{4.350591in}{9.755322in}}%
\pgfpathcurveto{\pgfqpoint{4.342777in}{9.747508in}}{\pgfqpoint{4.338387in}{9.736909in}}{\pgfqpoint{4.338387in}{9.725859in}}%
\pgfpathcurveto{\pgfqpoint{4.338387in}{9.714809in}}{\pgfqpoint{4.342777in}{9.704210in}}{\pgfqpoint{4.350591in}{9.696396in}}%
\pgfpathcurveto{\pgfqpoint{4.358405in}{9.688583in}}{\pgfqpoint{4.369004in}{9.684192in}}{\pgfqpoint{4.380054in}{9.684192in}}%
\pgfpathlineto{\pgfqpoint{4.380054in}{9.684192in}}%
\pgfpathclose%
\pgfusepath{stroke,fill}%
\end{pgfscope}%
\begin{pgfscope}%
\pgfpathrectangle{\pgfqpoint{2.963410in}{7.624184in}}{\pgfqpoint{2.177280in}{2.201755in}}%
\pgfusepath{clip}%
\pgfsetbuttcap%
\pgfsetroundjoin%
\definecolor{currentfill}{rgb}{0.172549,0.627451,0.172549}%
\pgfsetfillcolor{currentfill}%
\pgfsetlinewidth{0.481800pt}%
\definecolor{currentstroke}{rgb}{1.000000,1.000000,1.000000}%
\pgfsetstrokecolor{currentstroke}%
\pgfsetdash{}{0pt}%
\pgfpathmoveto{\pgfqpoint{3.789301in}{8.850194in}}%
\pgfpathcurveto{\pgfqpoint{3.800351in}{8.850194in}}{\pgfqpoint{3.810950in}{8.854584in}}{\pgfqpoint{3.818764in}{8.862398in}}%
\pgfpathcurveto{\pgfqpoint{3.826578in}{8.870212in}}{\pgfqpoint{3.830968in}{8.880811in}}{\pgfqpoint{3.830968in}{8.891861in}}%
\pgfpathcurveto{\pgfqpoint{3.830968in}{8.902911in}}{\pgfqpoint{3.826578in}{8.913510in}}{\pgfqpoint{3.818764in}{8.921324in}}%
\pgfpathcurveto{\pgfqpoint{3.810950in}{8.929137in}}{\pgfqpoint{3.800351in}{8.933528in}}{\pgfqpoint{3.789301in}{8.933528in}}%
\pgfpathcurveto{\pgfqpoint{3.778251in}{8.933528in}}{\pgfqpoint{3.767652in}{8.929137in}}{\pgfqpoint{3.759838in}{8.921324in}}%
\pgfpathcurveto{\pgfqpoint{3.752025in}{8.913510in}}{\pgfqpoint{3.747635in}{8.902911in}}{\pgfqpoint{3.747635in}{8.891861in}}%
\pgfpathcurveto{\pgfqpoint{3.747635in}{8.880811in}}{\pgfqpoint{3.752025in}{8.870212in}}{\pgfqpoint{3.759838in}{8.862398in}}%
\pgfpathcurveto{\pgfqpoint{3.767652in}{8.854584in}}{\pgfqpoint{3.778251in}{8.850194in}}{\pgfqpoint{3.789301in}{8.850194in}}%
\pgfpathlineto{\pgfqpoint{3.789301in}{8.850194in}}%
\pgfpathclose%
\pgfusepath{stroke,fill}%
\end{pgfscope}%
\begin{pgfscope}%
\pgfpathrectangle{\pgfqpoint{2.963410in}{7.624184in}}{\pgfqpoint{2.177280in}{2.201755in}}%
\pgfusepath{clip}%
\pgfsetbuttcap%
\pgfsetroundjoin%
\definecolor{currentfill}{rgb}{0.172549,0.627451,0.172549}%
\pgfsetfillcolor{currentfill}%
\pgfsetlinewidth{0.481800pt}%
\definecolor{currentstroke}{rgb}{1.000000,1.000000,1.000000}%
\pgfsetstrokecolor{currentstroke}%
\pgfsetdash{}{0pt}%
\pgfpathmoveto{\pgfqpoint{3.789301in}{8.794594in}}%
\pgfpathcurveto{\pgfqpoint{3.800351in}{8.794594in}}{\pgfqpoint{3.810950in}{8.798985in}}{\pgfqpoint{3.818764in}{8.806798in}}%
\pgfpathcurveto{\pgfqpoint{3.826578in}{8.814612in}}{\pgfqpoint{3.830968in}{8.825211in}}{\pgfqpoint{3.830968in}{8.836261in}}%
\pgfpathcurveto{\pgfqpoint{3.830968in}{8.847311in}}{\pgfqpoint{3.826578in}{8.857910in}}{\pgfqpoint{3.818764in}{8.865724in}}%
\pgfpathcurveto{\pgfqpoint{3.810950in}{8.873537in}}{\pgfqpoint{3.800351in}{8.877928in}}{\pgfqpoint{3.789301in}{8.877928in}}%
\pgfpathcurveto{\pgfqpoint{3.778251in}{8.877928in}}{\pgfqpoint{3.767652in}{8.873537in}}{\pgfqpoint{3.759838in}{8.865724in}}%
\pgfpathcurveto{\pgfqpoint{3.752025in}{8.857910in}}{\pgfqpoint{3.747635in}{8.847311in}}{\pgfqpoint{3.747635in}{8.836261in}}%
\pgfpathcurveto{\pgfqpoint{3.747635in}{8.825211in}}{\pgfqpoint{3.752025in}{8.814612in}}{\pgfqpoint{3.759838in}{8.806798in}}%
\pgfpathcurveto{\pgfqpoint{3.767652in}{8.798985in}}{\pgfqpoint{3.778251in}{8.794594in}}{\pgfqpoint{3.789301in}{8.794594in}}%
\pgfpathlineto{\pgfqpoint{3.789301in}{8.794594in}}%
\pgfpathclose%
\pgfusepath{stroke,fill}%
\end{pgfscope}%
\begin{pgfscope}%
\pgfpathrectangle{\pgfqpoint{2.963410in}{7.624184in}}{\pgfqpoint{2.177280in}{2.201755in}}%
\pgfusepath{clip}%
\pgfsetbuttcap%
\pgfsetroundjoin%
\definecolor{currentfill}{rgb}{0.172549,0.627451,0.172549}%
\pgfsetfillcolor{currentfill}%
\pgfsetlinewidth{0.481800pt}%
\definecolor{currentstroke}{rgb}{1.000000,1.000000,1.000000}%
\pgfsetstrokecolor{currentstroke}%
\pgfsetdash{}{0pt}%
\pgfpathmoveto{\pgfqpoint{3.671151in}{8.683395in}}%
\pgfpathcurveto{\pgfqpoint{3.682201in}{8.683395in}}{\pgfqpoint{3.692800in}{8.687785in}}{\pgfqpoint{3.700613in}{8.695598in}}%
\pgfpathcurveto{\pgfqpoint{3.708427in}{8.703412in}}{\pgfqpoint{3.712817in}{8.714011in}}{\pgfqpoint{3.712817in}{8.725061in}}%
\pgfpathcurveto{\pgfqpoint{3.712817in}{8.736111in}}{\pgfqpoint{3.708427in}{8.746710in}}{\pgfqpoint{3.700613in}{8.754524in}}%
\pgfpathcurveto{\pgfqpoint{3.692800in}{8.762338in}}{\pgfqpoint{3.682201in}{8.766728in}}{\pgfqpoint{3.671151in}{8.766728in}}%
\pgfpathcurveto{\pgfqpoint{3.660101in}{8.766728in}}{\pgfqpoint{3.649501in}{8.762338in}}{\pgfqpoint{3.641688in}{8.754524in}}%
\pgfpathcurveto{\pgfqpoint{3.633874in}{8.746710in}}{\pgfqpoint{3.629484in}{8.736111in}}{\pgfqpoint{3.629484in}{8.725061in}}%
\pgfpathcurveto{\pgfqpoint{3.629484in}{8.714011in}}{\pgfqpoint{3.633874in}{8.703412in}}{\pgfqpoint{3.641688in}{8.695598in}}%
\pgfpathcurveto{\pgfqpoint{3.649501in}{8.687785in}}{\pgfqpoint{3.660101in}{8.683395in}}{\pgfqpoint{3.671151in}{8.683395in}}%
\pgfpathlineto{\pgfqpoint{3.671151in}{8.683395in}}%
\pgfpathclose%
\pgfusepath{stroke,fill}%
\end{pgfscope}%
\begin{pgfscope}%
\pgfpathrectangle{\pgfqpoint{2.963410in}{7.624184in}}{\pgfqpoint{2.177280in}{2.201755in}}%
\pgfusepath{clip}%
\pgfsetbuttcap%
\pgfsetroundjoin%
\definecolor{currentfill}{rgb}{0.172549,0.627451,0.172549}%
\pgfsetfillcolor{currentfill}%
\pgfsetlinewidth{0.481800pt}%
\definecolor{currentstroke}{rgb}{1.000000,1.000000,1.000000}%
\pgfsetstrokecolor{currentstroke}%
\pgfsetdash{}{0pt}%
\pgfpathmoveto{\pgfqpoint{3.907452in}{9.572993in}}%
\pgfpathcurveto{\pgfqpoint{3.918502in}{9.572993in}}{\pgfqpoint{3.929101in}{9.577383in}}{\pgfqpoint{3.936915in}{9.585197in}}%
\pgfpathcurveto{\pgfqpoint{3.944728in}{9.593010in}}{\pgfqpoint{3.949118in}{9.603609in}}{\pgfqpoint{3.949118in}{9.614659in}}%
\pgfpathcurveto{\pgfqpoint{3.949118in}{9.625709in}}{\pgfqpoint{3.944728in}{9.636308in}}{\pgfqpoint{3.936915in}{9.644122in}}%
\pgfpathcurveto{\pgfqpoint{3.929101in}{9.651936in}}{\pgfqpoint{3.918502in}{9.656326in}}{\pgfqpoint{3.907452in}{9.656326in}}%
\pgfpathcurveto{\pgfqpoint{3.896402in}{9.656326in}}{\pgfqpoint{3.885803in}{9.651936in}}{\pgfqpoint{3.877989in}{9.644122in}}%
\pgfpathcurveto{\pgfqpoint{3.870175in}{9.636308in}}{\pgfqpoint{3.865785in}{9.625709in}}{\pgfqpoint{3.865785in}{9.614659in}}%
\pgfpathcurveto{\pgfqpoint{3.865785in}{9.603609in}}{\pgfqpoint{3.870175in}{9.593010in}}{\pgfqpoint{3.877989in}{9.585197in}}%
\pgfpathcurveto{\pgfqpoint{3.885803in}{9.577383in}}{\pgfqpoint{3.896402in}{9.572993in}}{\pgfqpoint{3.907452in}{9.572993in}}%
\pgfpathlineto{\pgfqpoint{3.907452in}{9.572993in}}%
\pgfpathclose%
\pgfusepath{stroke,fill}%
\end{pgfscope}%
\begin{pgfscope}%
\pgfpathrectangle{\pgfqpoint{2.963410in}{7.624184in}}{\pgfqpoint{2.177280in}{2.201755in}}%
\pgfusepath{clip}%
\pgfsetbuttcap%
\pgfsetroundjoin%
\definecolor{currentfill}{rgb}{0.172549,0.627451,0.172549}%
\pgfsetfillcolor{currentfill}%
\pgfsetlinewidth{0.481800pt}%
\definecolor{currentstroke}{rgb}{1.000000,1.000000,1.000000}%
\pgfsetstrokecolor{currentstroke}%
\pgfsetdash{}{0pt}%
\pgfpathmoveto{\pgfqpoint{4.143753in}{8.794594in}}%
\pgfpathcurveto{\pgfqpoint{4.154803in}{8.794594in}}{\pgfqpoint{4.165402in}{8.798985in}}{\pgfqpoint{4.173216in}{8.806798in}}%
\pgfpathcurveto{\pgfqpoint{4.181029in}{8.814612in}}{\pgfqpoint{4.185419in}{8.825211in}}{\pgfqpoint{4.185419in}{8.836261in}}%
\pgfpathcurveto{\pgfqpoint{4.185419in}{8.847311in}}{\pgfqpoint{4.181029in}{8.857910in}}{\pgfqpoint{4.173216in}{8.865724in}}%
\pgfpathcurveto{\pgfqpoint{4.165402in}{8.873537in}}{\pgfqpoint{4.154803in}{8.877928in}}{\pgfqpoint{4.143753in}{8.877928in}}%
\pgfpathcurveto{\pgfqpoint{4.132703in}{8.877928in}}{\pgfqpoint{4.122104in}{8.873537in}}{\pgfqpoint{4.114290in}{8.865724in}}%
\pgfpathcurveto{\pgfqpoint{4.106476in}{8.857910in}}{\pgfqpoint{4.102086in}{8.847311in}}{\pgfqpoint{4.102086in}{8.836261in}}%
\pgfpathcurveto{\pgfqpoint{4.102086in}{8.825211in}}{\pgfqpoint{4.106476in}{8.814612in}}{\pgfqpoint{4.114290in}{8.806798in}}%
\pgfpathcurveto{\pgfqpoint{4.122104in}{8.798985in}}{\pgfqpoint{4.132703in}{8.794594in}}{\pgfqpoint{4.143753in}{8.794594in}}%
\pgfpathlineto{\pgfqpoint{4.143753in}{8.794594in}}%
\pgfpathclose%
\pgfusepath{stroke,fill}%
\end{pgfscope}%
\begin{pgfscope}%
\pgfpathrectangle{\pgfqpoint{2.963410in}{7.624184in}}{\pgfqpoint{2.177280in}{2.201755in}}%
\pgfusepath{clip}%
\pgfsetbuttcap%
\pgfsetroundjoin%
\definecolor{currentfill}{rgb}{0.172549,0.627451,0.172549}%
\pgfsetfillcolor{currentfill}%
\pgfsetlinewidth{0.481800pt}%
\definecolor{currentstroke}{rgb}{1.000000,1.000000,1.000000}%
\pgfsetstrokecolor{currentstroke}%
\pgfsetdash{}{0pt}%
\pgfpathmoveto{\pgfqpoint{3.966527in}{8.850194in}}%
\pgfpathcurveto{\pgfqpoint{3.977577in}{8.850194in}}{\pgfqpoint{3.988176in}{8.854584in}}{\pgfqpoint{3.995990in}{8.862398in}}%
\pgfpathcurveto{\pgfqpoint{4.003803in}{8.870212in}}{\pgfqpoint{4.008194in}{8.880811in}}{\pgfqpoint{4.008194in}{8.891861in}}%
\pgfpathcurveto{\pgfqpoint{4.008194in}{8.902911in}}{\pgfqpoint{4.003803in}{8.913510in}}{\pgfqpoint{3.995990in}{8.921324in}}%
\pgfpathcurveto{\pgfqpoint{3.988176in}{8.929137in}}{\pgfqpoint{3.977577in}{8.933528in}}{\pgfqpoint{3.966527in}{8.933528in}}%
\pgfpathcurveto{\pgfqpoint{3.955477in}{8.933528in}}{\pgfqpoint{3.944878in}{8.929137in}}{\pgfqpoint{3.937064in}{8.921324in}}%
\pgfpathcurveto{\pgfqpoint{3.929251in}{8.913510in}}{\pgfqpoint{3.924860in}{8.902911in}}{\pgfqpoint{3.924860in}{8.891861in}}%
\pgfpathcurveto{\pgfqpoint{3.924860in}{8.880811in}}{\pgfqpoint{3.929251in}{8.870212in}}{\pgfqpoint{3.937064in}{8.862398in}}%
\pgfpathcurveto{\pgfqpoint{3.944878in}{8.854584in}}{\pgfqpoint{3.955477in}{8.850194in}}{\pgfqpoint{3.966527in}{8.850194in}}%
\pgfpathlineto{\pgfqpoint{3.966527in}{8.850194in}}%
\pgfpathclose%
\pgfusepath{stroke,fill}%
\end{pgfscope}%
\begin{pgfscope}%
\pgfpathrectangle{\pgfqpoint{2.963410in}{7.624184in}}{\pgfqpoint{2.177280in}{2.201755in}}%
\pgfusepath{clip}%
\pgfsetbuttcap%
\pgfsetroundjoin%
\definecolor{currentfill}{rgb}{0.172549,0.627451,0.172549}%
\pgfsetfillcolor{currentfill}%
\pgfsetlinewidth{0.481800pt}%
\definecolor{currentstroke}{rgb}{1.000000,1.000000,1.000000}%
\pgfsetstrokecolor{currentstroke}%
\pgfsetdash{}{0pt}%
\pgfpathmoveto{\pgfqpoint{3.907452in}{8.627795in}}%
\pgfpathcurveto{\pgfqpoint{3.918502in}{8.627795in}}{\pgfqpoint{3.929101in}{8.632185in}}{\pgfqpoint{3.936915in}{8.639999in}}%
\pgfpathcurveto{\pgfqpoint{3.944728in}{8.647812in}}{\pgfqpoint{3.949118in}{8.658411in}}{\pgfqpoint{3.949118in}{8.669461in}}%
\pgfpathcurveto{\pgfqpoint{3.949118in}{8.680512in}}{\pgfqpoint{3.944728in}{8.691111in}}{\pgfqpoint{3.936915in}{8.698924in}}%
\pgfpathcurveto{\pgfqpoint{3.929101in}{8.706738in}}{\pgfqpoint{3.918502in}{8.711128in}}{\pgfqpoint{3.907452in}{8.711128in}}%
\pgfpathcurveto{\pgfqpoint{3.896402in}{8.711128in}}{\pgfqpoint{3.885803in}{8.706738in}}{\pgfqpoint{3.877989in}{8.698924in}}%
\pgfpathcurveto{\pgfqpoint{3.870175in}{8.691111in}}{\pgfqpoint{3.865785in}{8.680512in}}{\pgfqpoint{3.865785in}{8.669461in}}%
\pgfpathcurveto{\pgfqpoint{3.865785in}{8.658411in}}{\pgfqpoint{3.870175in}{8.647812in}}{\pgfqpoint{3.877989in}{8.639999in}}%
\pgfpathcurveto{\pgfqpoint{3.885803in}{8.632185in}}{\pgfqpoint{3.896402in}{8.627795in}}{\pgfqpoint{3.907452in}{8.627795in}}%
\pgfpathlineto{\pgfqpoint{3.907452in}{8.627795in}}%
\pgfpathclose%
\pgfusepath{stroke,fill}%
\end{pgfscope}%
\begin{pgfscope}%
\pgfpathrectangle{\pgfqpoint{2.963410in}{7.624184in}}{\pgfqpoint{2.177280in}{2.201755in}}%
\pgfusepath{clip}%
\pgfsetbuttcap%
\pgfsetroundjoin%
\definecolor{currentfill}{rgb}{0.172549,0.627451,0.172549}%
\pgfsetfillcolor{currentfill}%
\pgfsetlinewidth{0.481800pt}%
\definecolor{currentstroke}{rgb}{1.000000,1.000000,1.000000}%
\pgfsetstrokecolor{currentstroke}%
\pgfsetdash{}{0pt}%
\pgfpathmoveto{\pgfqpoint{3.966527in}{9.128194in}}%
\pgfpathcurveto{\pgfqpoint{3.977577in}{9.128194in}}{\pgfqpoint{3.988176in}{9.132584in}}{\pgfqpoint{3.995990in}{9.140397in}}%
\pgfpathcurveto{\pgfqpoint{4.003803in}{9.148211in}}{\pgfqpoint{4.008194in}{9.158810in}}{\pgfqpoint{4.008194in}{9.169860in}}%
\pgfpathcurveto{\pgfqpoint{4.008194in}{9.180910in}}{\pgfqpoint{4.003803in}{9.191509in}}{\pgfqpoint{3.995990in}{9.199323in}}%
\pgfpathcurveto{\pgfqpoint{3.988176in}{9.207137in}}{\pgfqpoint{3.977577in}{9.211527in}}{\pgfqpoint{3.966527in}{9.211527in}}%
\pgfpathcurveto{\pgfqpoint{3.955477in}{9.211527in}}{\pgfqpoint{3.944878in}{9.207137in}}{\pgfqpoint{3.937064in}{9.199323in}}%
\pgfpathcurveto{\pgfqpoint{3.929251in}{9.191509in}}{\pgfqpoint{3.924860in}{9.180910in}}{\pgfqpoint{3.924860in}{9.169860in}}%
\pgfpathcurveto{\pgfqpoint{3.924860in}{9.158810in}}{\pgfqpoint{3.929251in}{9.148211in}}{\pgfqpoint{3.937064in}{9.140397in}}%
\pgfpathcurveto{\pgfqpoint{3.944878in}{9.132584in}}{\pgfqpoint{3.955477in}{9.128194in}}{\pgfqpoint{3.966527in}{9.128194in}}%
\pgfpathlineto{\pgfqpoint{3.966527in}{9.128194in}}%
\pgfpathclose%
\pgfusepath{stroke,fill}%
\end{pgfscope}%
\begin{pgfscope}%
\pgfpathrectangle{\pgfqpoint{2.963410in}{7.624184in}}{\pgfqpoint{2.177280in}{2.201755in}}%
\pgfusepath{clip}%
\pgfsetbuttcap%
\pgfsetroundjoin%
\definecolor{currentfill}{rgb}{0.172549,0.627451,0.172549}%
\pgfsetfillcolor{currentfill}%
\pgfsetlinewidth{0.481800pt}%
\definecolor{currentstroke}{rgb}{1.000000,1.000000,1.000000}%
\pgfsetstrokecolor{currentstroke}%
\pgfsetdash{}{0pt}%
\pgfpathmoveto{\pgfqpoint{3.966527in}{9.016994in}}%
\pgfpathcurveto{\pgfqpoint{3.977577in}{9.016994in}}{\pgfqpoint{3.988176in}{9.021384in}}{\pgfqpoint{3.995990in}{9.029198in}}%
\pgfpathcurveto{\pgfqpoint{4.003803in}{9.037011in}}{\pgfqpoint{4.008194in}{9.047610in}}{\pgfqpoint{4.008194in}{9.058661in}}%
\pgfpathcurveto{\pgfqpoint{4.008194in}{9.069711in}}{\pgfqpoint{4.003803in}{9.080310in}}{\pgfqpoint{3.995990in}{9.088123in}}%
\pgfpathcurveto{\pgfqpoint{3.988176in}{9.095937in}}{\pgfqpoint{3.977577in}{9.100327in}}{\pgfqpoint{3.966527in}{9.100327in}}%
\pgfpathcurveto{\pgfqpoint{3.955477in}{9.100327in}}{\pgfqpoint{3.944878in}{9.095937in}}{\pgfqpoint{3.937064in}{9.088123in}}%
\pgfpathcurveto{\pgfqpoint{3.929251in}{9.080310in}}{\pgfqpoint{3.924860in}{9.069711in}}{\pgfqpoint{3.924860in}{9.058661in}}%
\pgfpathcurveto{\pgfqpoint{3.924860in}{9.047610in}}{\pgfqpoint{3.929251in}{9.037011in}}{\pgfqpoint{3.937064in}{9.029198in}}%
\pgfpathcurveto{\pgfqpoint{3.944878in}{9.021384in}}{\pgfqpoint{3.955477in}{9.016994in}}{\pgfqpoint{3.966527in}{9.016994in}}%
\pgfpathlineto{\pgfqpoint{3.966527in}{9.016994in}}%
\pgfpathclose%
\pgfusepath{stroke,fill}%
\end{pgfscope}%
\begin{pgfscope}%
\pgfpathrectangle{\pgfqpoint{2.963410in}{7.624184in}}{\pgfqpoint{2.177280in}{2.201755in}}%
\pgfusepath{clip}%
\pgfsetbuttcap%
\pgfsetroundjoin%
\definecolor{currentfill}{rgb}{0.172549,0.627451,0.172549}%
\pgfsetfillcolor{currentfill}%
\pgfsetlinewidth{0.481800pt}%
\definecolor{currentstroke}{rgb}{1.000000,1.000000,1.000000}%
\pgfsetstrokecolor{currentstroke}%
\pgfsetdash{}{0pt}%
\pgfpathmoveto{\pgfqpoint{3.966527in}{9.128194in}}%
\pgfpathcurveto{\pgfqpoint{3.977577in}{9.128194in}}{\pgfqpoint{3.988176in}{9.132584in}}{\pgfqpoint{3.995990in}{9.140397in}}%
\pgfpathcurveto{\pgfqpoint{4.003803in}{9.148211in}}{\pgfqpoint{4.008194in}{9.158810in}}{\pgfqpoint{4.008194in}{9.169860in}}%
\pgfpathcurveto{\pgfqpoint{4.008194in}{9.180910in}}{\pgfqpoint{4.003803in}{9.191509in}}{\pgfqpoint{3.995990in}{9.199323in}}%
\pgfpathcurveto{\pgfqpoint{3.988176in}{9.207137in}}{\pgfqpoint{3.977577in}{9.211527in}}{\pgfqpoint{3.966527in}{9.211527in}}%
\pgfpathcurveto{\pgfqpoint{3.955477in}{9.211527in}}{\pgfqpoint{3.944878in}{9.207137in}}{\pgfqpoint{3.937064in}{9.199323in}}%
\pgfpathcurveto{\pgfqpoint{3.929251in}{9.191509in}}{\pgfqpoint{3.924860in}{9.180910in}}{\pgfqpoint{3.924860in}{9.169860in}}%
\pgfpathcurveto{\pgfqpoint{3.924860in}{9.158810in}}{\pgfqpoint{3.929251in}{9.148211in}}{\pgfqpoint{3.937064in}{9.140397in}}%
\pgfpathcurveto{\pgfqpoint{3.944878in}{9.132584in}}{\pgfqpoint{3.955477in}{9.128194in}}{\pgfqpoint{3.966527in}{9.128194in}}%
\pgfpathlineto{\pgfqpoint{3.966527in}{9.128194in}}%
\pgfpathclose%
\pgfusepath{stroke,fill}%
\end{pgfscope}%
\begin{pgfscope}%
\pgfpathrectangle{\pgfqpoint{2.963410in}{7.624184in}}{\pgfqpoint{2.177280in}{2.201755in}}%
\pgfusepath{clip}%
\pgfsetbuttcap%
\pgfsetroundjoin%
\definecolor{currentfill}{rgb}{0.172549,0.627451,0.172549}%
\pgfsetfillcolor{currentfill}%
\pgfsetlinewidth{0.481800pt}%
\definecolor{currentstroke}{rgb}{1.000000,1.000000,1.000000}%
\pgfsetstrokecolor{currentstroke}%
\pgfsetdash{}{0pt}%
\pgfpathmoveto{\pgfqpoint{3.730226in}{8.516595in}}%
\pgfpathcurveto{\pgfqpoint{3.741276in}{8.516595in}}{\pgfqpoint{3.751875in}{8.520985in}}{\pgfqpoint{3.759689in}{8.528799in}}%
\pgfpathcurveto{\pgfqpoint{3.767502in}{8.536612in}}{\pgfqpoint{3.771893in}{8.547212in}}{\pgfqpoint{3.771893in}{8.558262in}}%
\pgfpathcurveto{\pgfqpoint{3.771893in}{8.569312in}}{\pgfqpoint{3.767502in}{8.579911in}}{\pgfqpoint{3.759689in}{8.587724in}}%
\pgfpathcurveto{\pgfqpoint{3.751875in}{8.595538in}}{\pgfqpoint{3.741276in}{8.599928in}}{\pgfqpoint{3.730226in}{8.599928in}}%
\pgfpathcurveto{\pgfqpoint{3.719176in}{8.599928in}}{\pgfqpoint{3.708577in}{8.595538in}}{\pgfqpoint{3.700763in}{8.587724in}}%
\pgfpathcurveto{\pgfqpoint{3.692950in}{8.579911in}}{\pgfqpoint{3.688559in}{8.569312in}}{\pgfqpoint{3.688559in}{8.558262in}}%
\pgfpathcurveto{\pgfqpoint{3.688559in}{8.547212in}}{\pgfqpoint{3.692950in}{8.536612in}}{\pgfqpoint{3.700763in}{8.528799in}}%
\pgfpathcurveto{\pgfqpoint{3.708577in}{8.520985in}}{\pgfqpoint{3.719176in}{8.516595in}}{\pgfqpoint{3.730226in}{8.516595in}}%
\pgfpathlineto{\pgfqpoint{3.730226in}{8.516595in}}%
\pgfpathclose%
\pgfusepath{stroke,fill}%
\end{pgfscope}%
\begin{pgfscope}%
\pgfpathrectangle{\pgfqpoint{2.963410in}{7.624184in}}{\pgfqpoint{2.177280in}{2.201755in}}%
\pgfusepath{clip}%
\pgfsetbuttcap%
\pgfsetroundjoin%
\definecolor{currentfill}{rgb}{0.172549,0.627451,0.172549}%
\pgfsetfillcolor{currentfill}%
\pgfsetlinewidth{0.481800pt}%
\definecolor{currentstroke}{rgb}{1.000000,1.000000,1.000000}%
\pgfsetstrokecolor{currentstroke}%
\pgfsetdash{}{0pt}%
\pgfpathmoveto{\pgfqpoint{4.025602in}{9.072594in}}%
\pgfpathcurveto{\pgfqpoint{4.036652in}{9.072594in}}{\pgfqpoint{4.047251in}{9.076984in}}{\pgfqpoint{4.055065in}{9.084798in}}%
\pgfpathcurveto{\pgfqpoint{4.062879in}{9.092611in}}{\pgfqpoint{4.067269in}{9.103210in}}{\pgfqpoint{4.067269in}{9.114260in}}%
\pgfpathcurveto{\pgfqpoint{4.067269in}{9.125311in}}{\pgfqpoint{4.062879in}{9.135910in}}{\pgfqpoint{4.055065in}{9.143723in}}%
\pgfpathcurveto{\pgfqpoint{4.047251in}{9.151537in}}{\pgfqpoint{4.036652in}{9.155927in}}{\pgfqpoint{4.025602in}{9.155927in}}%
\pgfpathcurveto{\pgfqpoint{4.014552in}{9.155927in}}{\pgfqpoint{4.003953in}{9.151537in}}{\pgfqpoint{3.996139in}{9.143723in}}%
\pgfpathcurveto{\pgfqpoint{3.988326in}{9.135910in}}{\pgfqpoint{3.983936in}{9.125311in}}{\pgfqpoint{3.983936in}{9.114260in}}%
\pgfpathcurveto{\pgfqpoint{3.983936in}{9.103210in}}{\pgfqpoint{3.988326in}{9.092611in}}{\pgfqpoint{3.996139in}{9.084798in}}%
\pgfpathcurveto{\pgfqpoint{4.003953in}{9.076984in}}{\pgfqpoint{4.014552in}{9.072594in}}{\pgfqpoint{4.025602in}{9.072594in}}%
\pgfpathlineto{\pgfqpoint{4.025602in}{9.072594in}}%
\pgfpathclose%
\pgfusepath{stroke,fill}%
\end{pgfscope}%
\begin{pgfscope}%
\pgfpathrectangle{\pgfqpoint{2.963410in}{7.624184in}}{\pgfqpoint{2.177280in}{2.201755in}}%
\pgfusepath{clip}%
\pgfsetbuttcap%
\pgfsetroundjoin%
\definecolor{currentfill}{rgb}{0.172549,0.627451,0.172549}%
\pgfsetfillcolor{currentfill}%
\pgfsetlinewidth{0.481800pt}%
\definecolor{currentstroke}{rgb}{1.000000,1.000000,1.000000}%
\pgfsetstrokecolor{currentstroke}%
\pgfsetdash{}{0pt}%
\pgfpathmoveto{\pgfqpoint{4.084678in}{9.016994in}}%
\pgfpathcurveto{\pgfqpoint{4.095728in}{9.016994in}}{\pgfqpoint{4.106327in}{9.021384in}}{\pgfqpoint{4.114140in}{9.029198in}}%
\pgfpathcurveto{\pgfqpoint{4.121954in}{9.037011in}}{\pgfqpoint{4.126344in}{9.047610in}}{\pgfqpoint{4.126344in}{9.058661in}}%
\pgfpathcurveto{\pgfqpoint{4.126344in}{9.069711in}}{\pgfqpoint{4.121954in}{9.080310in}}{\pgfqpoint{4.114140in}{9.088123in}}%
\pgfpathcurveto{\pgfqpoint{4.106327in}{9.095937in}}{\pgfqpoint{4.095728in}{9.100327in}}{\pgfqpoint{4.084678in}{9.100327in}}%
\pgfpathcurveto{\pgfqpoint{4.073627in}{9.100327in}}{\pgfqpoint{4.063028in}{9.095937in}}{\pgfqpoint{4.055215in}{9.088123in}}%
\pgfpathcurveto{\pgfqpoint{4.047401in}{9.080310in}}{\pgfqpoint{4.043011in}{9.069711in}}{\pgfqpoint{4.043011in}{9.058661in}}%
\pgfpathcurveto{\pgfqpoint{4.043011in}{9.047610in}}{\pgfqpoint{4.047401in}{9.037011in}}{\pgfqpoint{4.055215in}{9.029198in}}%
\pgfpathcurveto{\pgfqpoint{4.063028in}{9.021384in}}{\pgfqpoint{4.073627in}{9.016994in}}{\pgfqpoint{4.084678in}{9.016994in}}%
\pgfpathlineto{\pgfqpoint{4.084678in}{9.016994in}}%
\pgfpathclose%
\pgfusepath{stroke,fill}%
\end{pgfscope}%
\begin{pgfscope}%
\pgfpathrectangle{\pgfqpoint{2.963410in}{7.624184in}}{\pgfqpoint{2.177280in}{2.201755in}}%
\pgfusepath{clip}%
\pgfsetbuttcap%
\pgfsetroundjoin%
\definecolor{currentfill}{rgb}{0.172549,0.627451,0.172549}%
\pgfsetfillcolor{currentfill}%
\pgfsetlinewidth{0.481800pt}%
\definecolor{currentstroke}{rgb}{1.000000,1.000000,1.000000}%
\pgfsetstrokecolor{currentstroke}%
\pgfsetdash{}{0pt}%
\pgfpathmoveto{\pgfqpoint{3.907452in}{9.016994in}}%
\pgfpathcurveto{\pgfqpoint{3.918502in}{9.016994in}}{\pgfqpoint{3.929101in}{9.021384in}}{\pgfqpoint{3.936915in}{9.029198in}}%
\pgfpathcurveto{\pgfqpoint{3.944728in}{9.037011in}}{\pgfqpoint{3.949118in}{9.047610in}}{\pgfqpoint{3.949118in}{9.058661in}}%
\pgfpathcurveto{\pgfqpoint{3.949118in}{9.069711in}}{\pgfqpoint{3.944728in}{9.080310in}}{\pgfqpoint{3.936915in}{9.088123in}}%
\pgfpathcurveto{\pgfqpoint{3.929101in}{9.095937in}}{\pgfqpoint{3.918502in}{9.100327in}}{\pgfqpoint{3.907452in}{9.100327in}}%
\pgfpathcurveto{\pgfqpoint{3.896402in}{9.100327in}}{\pgfqpoint{3.885803in}{9.095937in}}{\pgfqpoint{3.877989in}{9.088123in}}%
\pgfpathcurveto{\pgfqpoint{3.870175in}{9.080310in}}{\pgfqpoint{3.865785in}{9.069711in}}{\pgfqpoint{3.865785in}{9.058661in}}%
\pgfpathcurveto{\pgfqpoint{3.865785in}{9.047610in}}{\pgfqpoint{3.870175in}{9.037011in}}{\pgfqpoint{3.877989in}{9.029198in}}%
\pgfpathcurveto{\pgfqpoint{3.885803in}{9.021384in}}{\pgfqpoint{3.896402in}{9.016994in}}{\pgfqpoint{3.907452in}{9.016994in}}%
\pgfpathlineto{\pgfqpoint{3.907452in}{9.016994in}}%
\pgfpathclose%
\pgfusepath{stroke,fill}%
\end{pgfscope}%
\begin{pgfscope}%
\pgfpathrectangle{\pgfqpoint{2.963410in}{7.624184in}}{\pgfqpoint{2.177280in}{2.201755in}}%
\pgfusepath{clip}%
\pgfsetbuttcap%
\pgfsetroundjoin%
\definecolor{currentfill}{rgb}{0.172549,0.627451,0.172549}%
\pgfsetfillcolor{currentfill}%
\pgfsetlinewidth{0.481800pt}%
\definecolor{currentstroke}{rgb}{1.000000,1.000000,1.000000}%
\pgfsetstrokecolor{currentstroke}%
\pgfsetdash{}{0pt}%
\pgfpathmoveto{\pgfqpoint{3.612075in}{8.794594in}}%
\pgfpathcurveto{\pgfqpoint{3.623126in}{8.794594in}}{\pgfqpoint{3.633725in}{8.798985in}}{\pgfqpoint{3.641538in}{8.806798in}}%
\pgfpathcurveto{\pgfqpoint{3.649352in}{8.814612in}}{\pgfqpoint{3.653742in}{8.825211in}}{\pgfqpoint{3.653742in}{8.836261in}}%
\pgfpathcurveto{\pgfqpoint{3.653742in}{8.847311in}}{\pgfqpoint{3.649352in}{8.857910in}}{\pgfqpoint{3.641538in}{8.865724in}}%
\pgfpathcurveto{\pgfqpoint{3.633725in}{8.873537in}}{\pgfqpoint{3.623126in}{8.877928in}}{\pgfqpoint{3.612075in}{8.877928in}}%
\pgfpathcurveto{\pgfqpoint{3.601025in}{8.877928in}}{\pgfqpoint{3.590426in}{8.873537in}}{\pgfqpoint{3.582613in}{8.865724in}}%
\pgfpathcurveto{\pgfqpoint{3.574799in}{8.857910in}}{\pgfqpoint{3.570409in}{8.847311in}}{\pgfqpoint{3.570409in}{8.836261in}}%
\pgfpathcurveto{\pgfqpoint{3.570409in}{8.825211in}}{\pgfqpoint{3.574799in}{8.814612in}}{\pgfqpoint{3.582613in}{8.806798in}}%
\pgfpathcurveto{\pgfqpoint{3.590426in}{8.798985in}}{\pgfqpoint{3.601025in}{8.794594in}}{\pgfqpoint{3.612075in}{8.794594in}}%
\pgfpathlineto{\pgfqpoint{3.612075in}{8.794594in}}%
\pgfpathclose%
\pgfusepath{stroke,fill}%
\end{pgfscope}%
\begin{pgfscope}%
\pgfpathrectangle{\pgfqpoint{2.963410in}{7.624184in}}{\pgfqpoint{2.177280in}{2.201755in}}%
\pgfusepath{clip}%
\pgfsetbuttcap%
\pgfsetroundjoin%
\definecolor{currentfill}{rgb}{0.172549,0.627451,0.172549}%
\pgfsetfillcolor{currentfill}%
\pgfsetlinewidth{0.481800pt}%
\definecolor{currentstroke}{rgb}{1.000000,1.000000,1.000000}%
\pgfsetstrokecolor{currentstroke}%
\pgfsetdash{}{0pt}%
\pgfpathmoveto{\pgfqpoint{3.907452in}{8.905794in}}%
\pgfpathcurveto{\pgfqpoint{3.918502in}{8.905794in}}{\pgfqpoint{3.929101in}{8.910184in}}{\pgfqpoint{3.936915in}{8.917998in}}%
\pgfpathcurveto{\pgfqpoint{3.944728in}{8.925812in}}{\pgfqpoint{3.949118in}{8.936411in}}{\pgfqpoint{3.949118in}{8.947461in}}%
\pgfpathcurveto{\pgfqpoint{3.949118in}{8.958511in}}{\pgfqpoint{3.944728in}{8.969110in}}{\pgfqpoint{3.936915in}{8.976924in}}%
\pgfpathcurveto{\pgfqpoint{3.929101in}{8.984737in}}{\pgfqpoint{3.918502in}{8.989127in}}{\pgfqpoint{3.907452in}{8.989127in}}%
\pgfpathcurveto{\pgfqpoint{3.896402in}{8.989127in}}{\pgfqpoint{3.885803in}{8.984737in}}{\pgfqpoint{3.877989in}{8.976924in}}%
\pgfpathcurveto{\pgfqpoint{3.870175in}{8.969110in}}{\pgfqpoint{3.865785in}{8.958511in}}{\pgfqpoint{3.865785in}{8.947461in}}%
\pgfpathcurveto{\pgfqpoint{3.865785in}{8.936411in}}{\pgfqpoint{3.870175in}{8.925812in}}{\pgfqpoint{3.877989in}{8.917998in}}%
\pgfpathcurveto{\pgfqpoint{3.885803in}{8.910184in}}{\pgfqpoint{3.896402in}{8.905794in}}{\pgfqpoint{3.907452in}{8.905794in}}%
\pgfpathlineto{\pgfqpoint{3.907452in}{8.905794in}}%
\pgfpathclose%
\pgfusepath{stroke,fill}%
\end{pgfscope}%
\begin{pgfscope}%
\pgfpathrectangle{\pgfqpoint{2.963410in}{7.624184in}}{\pgfqpoint{2.177280in}{2.201755in}}%
\pgfusepath{clip}%
\pgfsetbuttcap%
\pgfsetroundjoin%
\definecolor{currentfill}{rgb}{0.172549,0.627451,0.172549}%
\pgfsetfillcolor{currentfill}%
\pgfsetlinewidth{0.481800pt}%
\definecolor{currentstroke}{rgb}{1.000000,1.000000,1.000000}%
\pgfsetstrokecolor{currentstroke}%
\pgfsetdash{}{0pt}%
\pgfpathmoveto{\pgfqpoint{4.143753in}{8.738994in}}%
\pgfpathcurveto{\pgfqpoint{4.154803in}{8.738994in}}{\pgfqpoint{4.165402in}{8.743385in}}{\pgfqpoint{4.173216in}{8.751198in}}%
\pgfpathcurveto{\pgfqpoint{4.181029in}{8.759012in}}{\pgfqpoint{4.185419in}{8.769611in}}{\pgfqpoint{4.185419in}{8.780661in}}%
\pgfpathcurveto{\pgfqpoint{4.185419in}{8.791711in}}{\pgfqpoint{4.181029in}{8.802310in}}{\pgfqpoint{4.173216in}{8.810124in}}%
\pgfpathcurveto{\pgfqpoint{4.165402in}{8.817938in}}{\pgfqpoint{4.154803in}{8.822328in}}{\pgfqpoint{4.143753in}{8.822328in}}%
\pgfpathcurveto{\pgfqpoint{4.132703in}{8.822328in}}{\pgfqpoint{4.122104in}{8.817938in}}{\pgfqpoint{4.114290in}{8.810124in}}%
\pgfpathcurveto{\pgfqpoint{4.106476in}{8.802310in}}{\pgfqpoint{4.102086in}{8.791711in}}{\pgfqpoint{4.102086in}{8.780661in}}%
\pgfpathcurveto{\pgfqpoint{4.102086in}{8.769611in}}{\pgfqpoint{4.106476in}{8.759012in}}{\pgfqpoint{4.114290in}{8.751198in}}%
\pgfpathcurveto{\pgfqpoint{4.122104in}{8.743385in}}{\pgfqpoint{4.132703in}{8.738994in}}{\pgfqpoint{4.143753in}{8.738994in}}%
\pgfpathlineto{\pgfqpoint{4.143753in}{8.738994in}}%
\pgfpathclose%
\pgfusepath{stroke,fill}%
\end{pgfscope}%
\begin{pgfscope}%
\pgfpathrectangle{\pgfqpoint{2.963410in}{7.624184in}}{\pgfqpoint{2.177280in}{2.201755in}}%
\pgfusepath{clip}%
\pgfsetbuttcap%
\pgfsetroundjoin%
\definecolor{currentfill}{rgb}{0.172549,0.627451,0.172549}%
\pgfsetfillcolor{currentfill}%
\pgfsetlinewidth{0.481800pt}%
\definecolor{currentstroke}{rgb}{1.000000,1.000000,1.000000}%
\pgfsetstrokecolor{currentstroke}%
\pgfsetdash{}{0pt}%
\pgfpathmoveto{\pgfqpoint{3.907452in}{8.572195in}}%
\pgfpathcurveto{\pgfqpoint{3.918502in}{8.572195in}}{\pgfqpoint{3.929101in}{8.576585in}}{\pgfqpoint{3.936915in}{8.584399in}}%
\pgfpathcurveto{\pgfqpoint{3.944728in}{8.592212in}}{\pgfqpoint{3.949118in}{8.602811in}}{\pgfqpoint{3.949118in}{8.613862in}}%
\pgfpathcurveto{\pgfqpoint{3.949118in}{8.624912in}}{\pgfqpoint{3.944728in}{8.635511in}}{\pgfqpoint{3.936915in}{8.643324in}}%
\pgfpathcurveto{\pgfqpoint{3.929101in}{8.651138in}}{\pgfqpoint{3.918502in}{8.655528in}}{\pgfqpoint{3.907452in}{8.655528in}}%
\pgfpathcurveto{\pgfqpoint{3.896402in}{8.655528in}}{\pgfqpoint{3.885803in}{8.651138in}}{\pgfqpoint{3.877989in}{8.643324in}}%
\pgfpathcurveto{\pgfqpoint{3.870175in}{8.635511in}}{\pgfqpoint{3.865785in}{8.624912in}}{\pgfqpoint{3.865785in}{8.613862in}}%
\pgfpathcurveto{\pgfqpoint{3.865785in}{8.602811in}}{\pgfqpoint{3.870175in}{8.592212in}}{\pgfqpoint{3.877989in}{8.584399in}}%
\pgfpathcurveto{\pgfqpoint{3.885803in}{8.576585in}}{\pgfqpoint{3.896402in}{8.572195in}}{\pgfqpoint{3.907452in}{8.572195in}}%
\pgfpathlineto{\pgfqpoint{3.907452in}{8.572195in}}%
\pgfpathclose%
\pgfusepath{stroke,fill}%
\end{pgfscope}%
\begin{pgfscope}%
\pgfpathrectangle{\pgfqpoint{2.963410in}{7.624184in}}{\pgfqpoint{2.177280in}{2.201755in}}%
\pgfusepath{clip}%
\pgfsetbuttcap%
\pgfsetroundjoin%
\definecolor{currentfill}{rgb}{0.121569,0.466667,0.705882}%
\pgfsetfillcolor{currentfill}%
\pgfsetlinewidth{1.003750pt}%
\definecolor{currentstroke}{rgb}{0.121569,0.466667,0.705882}%
\pgfsetstrokecolor{currentstroke}%
\pgfsetdash{}{0pt}%
\pgfsys@defobject{currentmarker}{\pgfqpoint{-0.041667in}{-0.041667in}}{\pgfqpoint{0.041667in}{0.041667in}}{%
\pgfpathmoveto{\pgfqpoint{0.000000in}{-0.041667in}}%
\pgfpathcurveto{\pgfqpoint{0.011050in}{-0.041667in}}{\pgfqpoint{0.021649in}{-0.037276in}}{\pgfqpoint{0.029463in}{-0.029463in}}%
\pgfpathcurveto{\pgfqpoint{0.037276in}{-0.021649in}}{\pgfqpoint{0.041667in}{-0.011050in}}{\pgfqpoint{0.041667in}{0.000000in}}%
\pgfpathcurveto{\pgfqpoint{0.041667in}{0.011050in}}{\pgfqpoint{0.037276in}{0.021649in}}{\pgfqpoint{0.029463in}{0.029463in}}%
\pgfpathcurveto{\pgfqpoint{0.021649in}{0.037276in}}{\pgfqpoint{0.011050in}{0.041667in}}{\pgfqpoint{0.000000in}{0.041667in}}%
\pgfpathcurveto{\pgfqpoint{-0.011050in}{0.041667in}}{\pgfqpoint{-0.021649in}{0.037276in}}{\pgfqpoint{-0.029463in}{0.029463in}}%
\pgfpathcurveto{\pgfqpoint{-0.037276in}{0.021649in}}{\pgfqpoint{-0.041667in}{0.011050in}}{\pgfqpoint{-0.041667in}{0.000000in}}%
\pgfpathcurveto{\pgfqpoint{-0.041667in}{-0.011050in}}{\pgfqpoint{-0.037276in}{-0.021649in}}{\pgfqpoint{-0.029463in}{-0.029463in}}%
\pgfpathcurveto{\pgfqpoint{-0.021649in}{-0.037276in}}{\pgfqpoint{-0.011050in}{-0.041667in}}{\pgfqpoint{0.000000in}{-0.041667in}}%
\pgfpathlineto{\pgfqpoint{0.000000in}{-0.041667in}}%
\pgfpathclose%
\pgfusepath{stroke,fill}%
}%
\end{pgfscope}%
\begin{pgfscope}%
\pgfpathrectangle{\pgfqpoint{2.963410in}{7.624184in}}{\pgfqpoint{2.177280in}{2.201755in}}%
\pgfusepath{clip}%
\pgfsetbuttcap%
\pgfsetroundjoin%
\definecolor{currentfill}{rgb}{1.000000,0.498039,0.054902}%
\pgfsetfillcolor{currentfill}%
\pgfsetlinewidth{1.003750pt}%
\definecolor{currentstroke}{rgb}{1.000000,0.498039,0.054902}%
\pgfsetstrokecolor{currentstroke}%
\pgfsetdash{}{0pt}%
\pgfsys@defobject{currentmarker}{\pgfqpoint{-0.041667in}{-0.041667in}}{\pgfqpoint{0.041667in}{0.041667in}}{%
\pgfpathmoveto{\pgfqpoint{0.000000in}{-0.041667in}}%
\pgfpathcurveto{\pgfqpoint{0.011050in}{-0.041667in}}{\pgfqpoint{0.021649in}{-0.037276in}}{\pgfqpoint{0.029463in}{-0.029463in}}%
\pgfpathcurveto{\pgfqpoint{0.037276in}{-0.021649in}}{\pgfqpoint{0.041667in}{-0.011050in}}{\pgfqpoint{0.041667in}{0.000000in}}%
\pgfpathcurveto{\pgfqpoint{0.041667in}{0.011050in}}{\pgfqpoint{0.037276in}{0.021649in}}{\pgfqpoint{0.029463in}{0.029463in}}%
\pgfpathcurveto{\pgfqpoint{0.021649in}{0.037276in}}{\pgfqpoint{0.011050in}{0.041667in}}{\pgfqpoint{0.000000in}{0.041667in}}%
\pgfpathcurveto{\pgfqpoint{-0.011050in}{0.041667in}}{\pgfqpoint{-0.021649in}{0.037276in}}{\pgfqpoint{-0.029463in}{0.029463in}}%
\pgfpathcurveto{\pgfqpoint{-0.037276in}{0.021649in}}{\pgfqpoint{-0.041667in}{0.011050in}}{\pgfqpoint{-0.041667in}{0.000000in}}%
\pgfpathcurveto{\pgfqpoint{-0.041667in}{-0.011050in}}{\pgfqpoint{-0.037276in}{-0.021649in}}{\pgfqpoint{-0.029463in}{-0.029463in}}%
\pgfpathcurveto{\pgfqpoint{-0.021649in}{-0.037276in}}{\pgfqpoint{-0.011050in}{-0.041667in}}{\pgfqpoint{0.000000in}{-0.041667in}}%
\pgfpathlineto{\pgfqpoint{0.000000in}{-0.041667in}}%
\pgfpathclose%
\pgfusepath{stroke,fill}%
}%
\end{pgfscope}%
\begin{pgfscope}%
\pgfpathrectangle{\pgfqpoint{2.963410in}{7.624184in}}{\pgfqpoint{2.177280in}{2.201755in}}%
\pgfusepath{clip}%
\pgfsetbuttcap%
\pgfsetroundjoin%
\definecolor{currentfill}{rgb}{0.172549,0.627451,0.172549}%
\pgfsetfillcolor{currentfill}%
\pgfsetlinewidth{1.003750pt}%
\definecolor{currentstroke}{rgb}{0.172549,0.627451,0.172549}%
\pgfsetstrokecolor{currentstroke}%
\pgfsetdash{}{0pt}%
\pgfsys@defobject{currentmarker}{\pgfqpoint{-0.041667in}{-0.041667in}}{\pgfqpoint{0.041667in}{0.041667in}}{%
\pgfpathmoveto{\pgfqpoint{0.000000in}{-0.041667in}}%
\pgfpathcurveto{\pgfqpoint{0.011050in}{-0.041667in}}{\pgfqpoint{0.021649in}{-0.037276in}}{\pgfqpoint{0.029463in}{-0.029463in}}%
\pgfpathcurveto{\pgfqpoint{0.037276in}{-0.021649in}}{\pgfqpoint{0.041667in}{-0.011050in}}{\pgfqpoint{0.041667in}{0.000000in}}%
\pgfpathcurveto{\pgfqpoint{0.041667in}{0.011050in}}{\pgfqpoint{0.037276in}{0.021649in}}{\pgfqpoint{0.029463in}{0.029463in}}%
\pgfpathcurveto{\pgfqpoint{0.021649in}{0.037276in}}{\pgfqpoint{0.011050in}{0.041667in}}{\pgfqpoint{0.000000in}{0.041667in}}%
\pgfpathcurveto{\pgfqpoint{-0.011050in}{0.041667in}}{\pgfqpoint{-0.021649in}{0.037276in}}{\pgfqpoint{-0.029463in}{0.029463in}}%
\pgfpathcurveto{\pgfqpoint{-0.037276in}{0.021649in}}{\pgfqpoint{-0.041667in}{0.011050in}}{\pgfqpoint{-0.041667in}{0.000000in}}%
\pgfpathcurveto{\pgfqpoint{-0.041667in}{-0.011050in}}{\pgfqpoint{-0.037276in}{-0.021649in}}{\pgfqpoint{-0.029463in}{-0.029463in}}%
\pgfpathcurveto{\pgfqpoint{-0.021649in}{-0.037276in}}{\pgfqpoint{-0.011050in}{-0.041667in}}{\pgfqpoint{0.000000in}{-0.041667in}}%
\pgfpathlineto{\pgfqpoint{0.000000in}{-0.041667in}}%
\pgfpathclose%
\pgfusepath{stroke,fill}%
}%
\end{pgfscope}%
\begin{pgfscope}%
\pgfsetbuttcap%
\pgfsetroundjoin%
\definecolor{currentfill}{rgb}{0.000000,0.000000,0.000000}%
\pgfsetfillcolor{currentfill}%
\pgfsetlinewidth{0.803000pt}%
\definecolor{currentstroke}{rgb}{0.000000,0.000000,0.000000}%
\pgfsetstrokecolor{currentstroke}%
\pgfsetdash{}{0pt}%
\pgfsys@defobject{currentmarker}{\pgfqpoint{0.000000in}{-0.048611in}}{\pgfqpoint{0.000000in}{0.000000in}}{%
\pgfpathmoveto{\pgfqpoint{0.000000in}{0.000000in}}%
\pgfpathlineto{\pgfqpoint{0.000000in}{-0.048611in}}%
\pgfusepath{stroke,fill}%
}%
\begin{pgfscope}%
\pgfsys@transformshift{3.316699in}{7.624184in}%
\pgfsys@useobject{currentmarker}{}%
\end{pgfscope}%
\end{pgfscope}%
\begin{pgfscope}%
\pgfsetbuttcap%
\pgfsetroundjoin%
\definecolor{currentfill}{rgb}{0.000000,0.000000,0.000000}%
\pgfsetfillcolor{currentfill}%
\pgfsetlinewidth{0.803000pt}%
\definecolor{currentstroke}{rgb}{0.000000,0.000000,0.000000}%
\pgfsetstrokecolor{currentstroke}%
\pgfsetdash{}{0pt}%
\pgfsys@defobject{currentmarker}{\pgfqpoint{0.000000in}{-0.048611in}}{\pgfqpoint{0.000000in}{0.000000in}}{%
\pgfpathmoveto{\pgfqpoint{0.000000in}{0.000000in}}%
\pgfpathlineto{\pgfqpoint{0.000000in}{-0.048611in}}%
\pgfusepath{stroke,fill}%
}%
\begin{pgfscope}%
\pgfsys@transformshift{3.907452in}{7.624184in}%
\pgfsys@useobject{currentmarker}{}%
\end{pgfscope}%
\end{pgfscope}%
\begin{pgfscope}%
\pgfsetbuttcap%
\pgfsetroundjoin%
\definecolor{currentfill}{rgb}{0.000000,0.000000,0.000000}%
\pgfsetfillcolor{currentfill}%
\pgfsetlinewidth{0.803000pt}%
\definecolor{currentstroke}{rgb}{0.000000,0.000000,0.000000}%
\pgfsetstrokecolor{currentstroke}%
\pgfsetdash{}{0pt}%
\pgfsys@defobject{currentmarker}{\pgfqpoint{0.000000in}{-0.048611in}}{\pgfqpoint{0.000000in}{0.000000in}}{%
\pgfpathmoveto{\pgfqpoint{0.000000in}{0.000000in}}%
\pgfpathlineto{\pgfqpoint{0.000000in}{-0.048611in}}%
\pgfusepath{stroke,fill}%
}%
\begin{pgfscope}%
\pgfsys@transformshift{4.498204in}{7.624184in}%
\pgfsys@useobject{currentmarker}{}%
\end{pgfscope}%
\end{pgfscope}%
\begin{pgfscope}%
\pgfsetbuttcap%
\pgfsetroundjoin%
\definecolor{currentfill}{rgb}{0.000000,0.000000,0.000000}%
\pgfsetfillcolor{currentfill}%
\pgfsetlinewidth{0.803000pt}%
\definecolor{currentstroke}{rgb}{0.000000,0.000000,0.000000}%
\pgfsetstrokecolor{currentstroke}%
\pgfsetdash{}{0pt}%
\pgfsys@defobject{currentmarker}{\pgfqpoint{0.000000in}{-0.048611in}}{\pgfqpoint{0.000000in}{0.000000in}}{%
\pgfpathmoveto{\pgfqpoint{0.000000in}{0.000000in}}%
\pgfpathlineto{\pgfqpoint{0.000000in}{-0.048611in}}%
\pgfusepath{stroke,fill}%
}%
\begin{pgfscope}%
\pgfsys@transformshift{5.088957in}{7.624184in}%
\pgfsys@useobject{currentmarker}{}%
\end{pgfscope}%
\end{pgfscope}%
\begin{pgfscope}%
\pgfsetbuttcap%
\pgfsetroundjoin%
\definecolor{currentfill}{rgb}{0.000000,0.000000,0.000000}%
\pgfsetfillcolor{currentfill}%
\pgfsetlinewidth{0.803000pt}%
\definecolor{currentstroke}{rgb}{0.000000,0.000000,0.000000}%
\pgfsetstrokecolor{currentstroke}%
\pgfsetdash{}{0pt}%
\pgfsys@defobject{currentmarker}{\pgfqpoint{-0.048611in}{0.000000in}}{\pgfqpoint{-0.000000in}{0.000000in}}{%
\pgfpathmoveto{\pgfqpoint{-0.000000in}{0.000000in}}%
\pgfpathlineto{\pgfqpoint{-0.048611in}{0.000000in}}%
\pgfusepath{stroke,fill}%
}%
\begin{pgfscope}%
\pgfsys@transformshift{2.963410in}{8.113463in}%
\pgfsys@useobject{currentmarker}{}%
\end{pgfscope}%
\end{pgfscope}%
\begin{pgfscope}%
\pgfsetbuttcap%
\pgfsetroundjoin%
\definecolor{currentfill}{rgb}{0.000000,0.000000,0.000000}%
\pgfsetfillcolor{currentfill}%
\pgfsetlinewidth{0.803000pt}%
\definecolor{currentstroke}{rgb}{0.000000,0.000000,0.000000}%
\pgfsetstrokecolor{currentstroke}%
\pgfsetdash{}{0pt}%
\pgfsys@defobject{currentmarker}{\pgfqpoint{-0.048611in}{0.000000in}}{\pgfqpoint{-0.000000in}{0.000000in}}{%
\pgfpathmoveto{\pgfqpoint{-0.000000in}{0.000000in}}%
\pgfpathlineto{\pgfqpoint{-0.048611in}{0.000000in}}%
\pgfusepath{stroke,fill}%
}%
\begin{pgfscope}%
\pgfsys@transformshift{2.963410in}{8.669461in}%
\pgfsys@useobject{currentmarker}{}%
\end{pgfscope}%
\end{pgfscope}%
\begin{pgfscope}%
\pgfsetbuttcap%
\pgfsetroundjoin%
\definecolor{currentfill}{rgb}{0.000000,0.000000,0.000000}%
\pgfsetfillcolor{currentfill}%
\pgfsetlinewidth{0.803000pt}%
\definecolor{currentstroke}{rgb}{0.000000,0.000000,0.000000}%
\pgfsetstrokecolor{currentstroke}%
\pgfsetdash{}{0pt}%
\pgfsys@defobject{currentmarker}{\pgfqpoint{-0.048611in}{0.000000in}}{\pgfqpoint{-0.000000in}{0.000000in}}{%
\pgfpathmoveto{\pgfqpoint{-0.000000in}{0.000000in}}%
\pgfpathlineto{\pgfqpoint{-0.048611in}{0.000000in}}%
\pgfusepath{stroke,fill}%
}%
\begin{pgfscope}%
\pgfsys@transformshift{2.963410in}{9.225460in}%
\pgfsys@useobject{currentmarker}{}%
\end{pgfscope}%
\end{pgfscope}%
\begin{pgfscope}%
\pgfsetbuttcap%
\pgfsetroundjoin%
\definecolor{currentfill}{rgb}{0.000000,0.000000,0.000000}%
\pgfsetfillcolor{currentfill}%
\pgfsetlinewidth{0.803000pt}%
\definecolor{currentstroke}{rgb}{0.000000,0.000000,0.000000}%
\pgfsetstrokecolor{currentstroke}%
\pgfsetdash{}{0pt}%
\pgfsys@defobject{currentmarker}{\pgfqpoint{-0.048611in}{0.000000in}}{\pgfqpoint{-0.000000in}{0.000000in}}{%
\pgfpathmoveto{\pgfqpoint{-0.000000in}{0.000000in}}%
\pgfpathlineto{\pgfqpoint{-0.048611in}{0.000000in}}%
\pgfusepath{stroke,fill}%
}%
\begin{pgfscope}%
\pgfsys@transformshift{2.963410in}{9.781459in}%
\pgfsys@useobject{currentmarker}{}%
\end{pgfscope}%
\end{pgfscope}%
\begin{pgfscope}%
\pgfsetrectcap%
\pgfsetmiterjoin%
\pgfsetlinewidth{0.803000pt}%
\definecolor{currentstroke}{rgb}{0.000000,0.000000,0.000000}%
\pgfsetstrokecolor{currentstroke}%
\pgfsetdash{}{0pt}%
\pgfpathmoveto{\pgfqpoint{2.963410in}{7.624184in}}%
\pgfpathlineto{\pgfqpoint{2.963410in}{9.825939in}}%
\pgfusepath{stroke}%
\end{pgfscope}%
\begin{pgfscope}%
\pgfsetrectcap%
\pgfsetmiterjoin%
\pgfsetlinewidth{0.803000pt}%
\definecolor{currentstroke}{rgb}{0.000000,0.000000,0.000000}%
\pgfsetstrokecolor{currentstroke}%
\pgfsetdash{}{0pt}%
\pgfpathmoveto{\pgfqpoint{2.963410in}{7.624184in}}%
\pgfpathlineto{\pgfqpoint{5.140690in}{7.624184in}}%
\pgfusepath{stroke}%
\end{pgfscope}%
\begin{pgfscope}%
\pgfsetbuttcap%
\pgfsetmiterjoin%
\definecolor{currentfill}{rgb}{1.000000,1.000000,1.000000}%
\pgfsetfillcolor{currentfill}%
\pgfsetlinewidth{0.000000pt}%
\definecolor{currentstroke}{rgb}{0.000000,0.000000,0.000000}%
\pgfsetstrokecolor{currentstroke}%
\pgfsetstrokeopacity{0.000000}%
\pgfsetdash{}{0pt}%
\pgfpathmoveto{\pgfqpoint{5.292946in}{7.624184in}}%
\pgfpathlineto{\pgfqpoint{7.470226in}{7.624184in}}%
\pgfpathlineto{\pgfqpoint{7.470226in}{9.825939in}}%
\pgfpathlineto{\pgfqpoint{5.292946in}{9.825939in}}%
\pgfpathlineto{\pgfqpoint{5.292946in}{7.624184in}}%
\pgfpathclose%
\pgfusepath{fill}%
\end{pgfscope}%
\begin{pgfscope}%
\pgfpathrectangle{\pgfqpoint{5.292946in}{7.624184in}}{\pgfqpoint{2.177280in}{2.201755in}}%
\pgfusepath{clip}%
\pgfsetbuttcap%
\pgfsetroundjoin%
\definecolor{currentfill}{rgb}{0.121569,0.466667,0.705882}%
\pgfsetfillcolor{currentfill}%
\pgfsetlinewidth{0.481800pt}%
\definecolor{currentstroke}{rgb}{1.000000,1.000000,1.000000}%
\pgfsetstrokecolor{currentstroke}%
\pgfsetdash{}{0pt}%
\pgfpathmoveto{\pgfqpoint{5.575125in}{8.127396in}}%
\pgfpathcurveto{\pgfqpoint{5.586176in}{8.127396in}}{\pgfqpoint{5.596775in}{8.131786in}}{\pgfqpoint{5.604588in}{8.139600in}}%
\pgfpathcurveto{\pgfqpoint{5.612402in}{8.147413in}}{\pgfqpoint{5.616792in}{8.158012in}}{\pgfqpoint{5.616792in}{8.169063in}}%
\pgfpathcurveto{\pgfqpoint{5.616792in}{8.180113in}}{\pgfqpoint{5.612402in}{8.190712in}}{\pgfqpoint{5.604588in}{8.198525in}}%
\pgfpathcurveto{\pgfqpoint{5.596775in}{8.206339in}}{\pgfqpoint{5.586176in}{8.210729in}}{\pgfqpoint{5.575125in}{8.210729in}}%
\pgfpathcurveto{\pgfqpoint{5.564075in}{8.210729in}}{\pgfqpoint{5.553476in}{8.206339in}}{\pgfqpoint{5.545663in}{8.198525in}}%
\pgfpathcurveto{\pgfqpoint{5.537849in}{8.190712in}}{\pgfqpoint{5.533459in}{8.180113in}}{\pgfqpoint{5.533459in}{8.169063in}}%
\pgfpathcurveto{\pgfqpoint{5.533459in}{8.158012in}}{\pgfqpoint{5.537849in}{8.147413in}}{\pgfqpoint{5.545663in}{8.139600in}}%
\pgfpathcurveto{\pgfqpoint{5.553476in}{8.131786in}}{\pgfqpoint{5.564075in}{8.127396in}}{\pgfqpoint{5.575125in}{8.127396in}}%
\pgfpathlineto{\pgfqpoint{5.575125in}{8.127396in}}%
\pgfpathclose%
\pgfusepath{stroke,fill}%
\end{pgfscope}%
\begin{pgfscope}%
\pgfpathrectangle{\pgfqpoint{5.292946in}{7.624184in}}{\pgfqpoint{2.177280in}{2.201755in}}%
\pgfusepath{clip}%
\pgfsetbuttcap%
\pgfsetroundjoin%
\definecolor{currentfill}{rgb}{0.121569,0.466667,0.705882}%
\pgfsetfillcolor{currentfill}%
\pgfsetlinewidth{0.481800pt}%
\definecolor{currentstroke}{rgb}{1.000000,1.000000,1.000000}%
\pgfsetstrokecolor{currentstroke}%
\pgfsetdash{}{0pt}%
\pgfpathmoveto{\pgfqpoint{5.575125in}{8.016196in}}%
\pgfpathcurveto{\pgfqpoint{5.586176in}{8.016196in}}{\pgfqpoint{5.596775in}{8.020586in}}{\pgfqpoint{5.604588in}{8.028400in}}%
\pgfpathcurveto{\pgfqpoint{5.612402in}{8.036214in}}{\pgfqpoint{5.616792in}{8.046813in}}{\pgfqpoint{5.616792in}{8.057863in}}%
\pgfpathcurveto{\pgfqpoint{5.616792in}{8.068913in}}{\pgfqpoint{5.612402in}{8.079512in}}{\pgfqpoint{5.604588in}{8.087326in}}%
\pgfpathcurveto{\pgfqpoint{5.596775in}{8.095139in}}{\pgfqpoint{5.586176in}{8.099529in}}{\pgfqpoint{5.575125in}{8.099529in}}%
\pgfpathcurveto{\pgfqpoint{5.564075in}{8.099529in}}{\pgfqpoint{5.553476in}{8.095139in}}{\pgfqpoint{5.545663in}{8.087326in}}%
\pgfpathcurveto{\pgfqpoint{5.537849in}{8.079512in}}{\pgfqpoint{5.533459in}{8.068913in}}{\pgfqpoint{5.533459in}{8.057863in}}%
\pgfpathcurveto{\pgfqpoint{5.533459in}{8.046813in}}{\pgfqpoint{5.537849in}{8.036214in}}{\pgfqpoint{5.545663in}{8.028400in}}%
\pgfpathcurveto{\pgfqpoint{5.553476in}{8.020586in}}{\pgfqpoint{5.564075in}{8.016196in}}{\pgfqpoint{5.575125in}{8.016196in}}%
\pgfpathlineto{\pgfqpoint{5.575125in}{8.016196in}}%
\pgfpathclose%
\pgfusepath{stroke,fill}%
\end{pgfscope}%
\begin{pgfscope}%
\pgfpathrectangle{\pgfqpoint{5.292946in}{7.624184in}}{\pgfqpoint{2.177280in}{2.201755in}}%
\pgfusepath{clip}%
\pgfsetbuttcap%
\pgfsetroundjoin%
\definecolor{currentfill}{rgb}{0.121569,0.466667,0.705882}%
\pgfsetfillcolor{currentfill}%
\pgfsetlinewidth{0.481800pt}%
\definecolor{currentstroke}{rgb}{1.000000,1.000000,1.000000}%
\pgfsetstrokecolor{currentstroke}%
\pgfsetdash{}{0pt}%
\pgfpathmoveto{\pgfqpoint{5.546420in}{7.904996in}}%
\pgfpathcurveto{\pgfqpoint{5.557470in}{7.904996in}}{\pgfqpoint{5.568069in}{7.909387in}}{\pgfqpoint{5.575883in}{7.917200in}}%
\pgfpathcurveto{\pgfqpoint{5.583697in}{7.925014in}}{\pgfqpoint{5.588087in}{7.935613in}}{\pgfqpoint{5.588087in}{7.946663in}}%
\pgfpathcurveto{\pgfqpoint{5.588087in}{7.957713in}}{\pgfqpoint{5.583697in}{7.968312in}}{\pgfqpoint{5.575883in}{7.976126in}}%
\pgfpathcurveto{\pgfqpoint{5.568069in}{7.983939in}}{\pgfqpoint{5.557470in}{7.988330in}}{\pgfqpoint{5.546420in}{7.988330in}}%
\pgfpathcurveto{\pgfqpoint{5.535370in}{7.988330in}}{\pgfqpoint{5.524771in}{7.983939in}}{\pgfqpoint{5.516957in}{7.976126in}}%
\pgfpathcurveto{\pgfqpoint{5.509144in}{7.968312in}}{\pgfqpoint{5.504753in}{7.957713in}}{\pgfqpoint{5.504753in}{7.946663in}}%
\pgfpathcurveto{\pgfqpoint{5.504753in}{7.935613in}}{\pgfqpoint{5.509144in}{7.925014in}}{\pgfqpoint{5.516957in}{7.917200in}}%
\pgfpathcurveto{\pgfqpoint{5.524771in}{7.909387in}}{\pgfqpoint{5.535370in}{7.904996in}}{\pgfqpoint{5.546420in}{7.904996in}}%
\pgfpathlineto{\pgfqpoint{5.546420in}{7.904996in}}%
\pgfpathclose%
\pgfusepath{stroke,fill}%
\end{pgfscope}%
\begin{pgfscope}%
\pgfpathrectangle{\pgfqpoint{5.292946in}{7.624184in}}{\pgfqpoint{2.177280in}{2.201755in}}%
\pgfusepath{clip}%
\pgfsetbuttcap%
\pgfsetroundjoin%
\definecolor{currentfill}{rgb}{0.121569,0.466667,0.705882}%
\pgfsetfillcolor{currentfill}%
\pgfsetlinewidth{0.481800pt}%
\definecolor{currentstroke}{rgb}{1.000000,1.000000,1.000000}%
\pgfsetstrokecolor{currentstroke}%
\pgfsetdash{}{0pt}%
\pgfpathmoveto{\pgfqpoint{5.603831in}{7.849396in}}%
\pgfpathcurveto{\pgfqpoint{5.614881in}{7.849396in}}{\pgfqpoint{5.625480in}{7.853787in}}{\pgfqpoint{5.633294in}{7.861600in}}%
\pgfpathcurveto{\pgfqpoint{5.641107in}{7.869414in}}{\pgfqpoint{5.645497in}{7.880013in}}{\pgfqpoint{5.645497in}{7.891063in}}%
\pgfpathcurveto{\pgfqpoint{5.645497in}{7.902113in}}{\pgfqpoint{5.641107in}{7.912712in}}{\pgfqpoint{5.633294in}{7.920526in}}%
\pgfpathcurveto{\pgfqpoint{5.625480in}{7.928340in}}{\pgfqpoint{5.614881in}{7.932730in}}{\pgfqpoint{5.603831in}{7.932730in}}%
\pgfpathcurveto{\pgfqpoint{5.592781in}{7.932730in}}{\pgfqpoint{5.582182in}{7.928340in}}{\pgfqpoint{5.574368in}{7.920526in}}%
\pgfpathcurveto{\pgfqpoint{5.566554in}{7.912712in}}{\pgfqpoint{5.562164in}{7.902113in}}{\pgfqpoint{5.562164in}{7.891063in}}%
\pgfpathcurveto{\pgfqpoint{5.562164in}{7.880013in}}{\pgfqpoint{5.566554in}{7.869414in}}{\pgfqpoint{5.574368in}{7.861600in}}%
\pgfpathcurveto{\pgfqpoint{5.582182in}{7.853787in}}{\pgfqpoint{5.592781in}{7.849396in}}{\pgfqpoint{5.603831in}{7.849396in}}%
\pgfpathlineto{\pgfqpoint{5.603831in}{7.849396in}}%
\pgfpathclose%
\pgfusepath{stroke,fill}%
\end{pgfscope}%
\begin{pgfscope}%
\pgfpathrectangle{\pgfqpoint{5.292946in}{7.624184in}}{\pgfqpoint{2.177280in}{2.201755in}}%
\pgfusepath{clip}%
\pgfsetbuttcap%
\pgfsetroundjoin%
\definecolor{currentfill}{rgb}{0.121569,0.466667,0.705882}%
\pgfsetfillcolor{currentfill}%
\pgfsetlinewidth{0.481800pt}%
\definecolor{currentstroke}{rgb}{1.000000,1.000000,1.000000}%
\pgfsetstrokecolor{currentstroke}%
\pgfsetdash{}{0pt}%
\pgfpathmoveto{\pgfqpoint{5.575125in}{8.071796in}}%
\pgfpathcurveto{\pgfqpoint{5.586176in}{8.071796in}}{\pgfqpoint{5.596775in}{8.076186in}}{\pgfqpoint{5.604588in}{8.084000in}}%
\pgfpathcurveto{\pgfqpoint{5.612402in}{8.091813in}}{\pgfqpoint{5.616792in}{8.102413in}}{\pgfqpoint{5.616792in}{8.113463in}}%
\pgfpathcurveto{\pgfqpoint{5.616792in}{8.124513in}}{\pgfqpoint{5.612402in}{8.135112in}}{\pgfqpoint{5.604588in}{8.142925in}}%
\pgfpathcurveto{\pgfqpoint{5.596775in}{8.150739in}}{\pgfqpoint{5.586176in}{8.155129in}}{\pgfqpoint{5.575125in}{8.155129in}}%
\pgfpathcurveto{\pgfqpoint{5.564075in}{8.155129in}}{\pgfqpoint{5.553476in}{8.150739in}}{\pgfqpoint{5.545663in}{8.142925in}}%
\pgfpathcurveto{\pgfqpoint{5.537849in}{8.135112in}}{\pgfqpoint{5.533459in}{8.124513in}}{\pgfqpoint{5.533459in}{8.113463in}}%
\pgfpathcurveto{\pgfqpoint{5.533459in}{8.102413in}}{\pgfqpoint{5.537849in}{8.091813in}}{\pgfqpoint{5.545663in}{8.084000in}}%
\pgfpathcurveto{\pgfqpoint{5.553476in}{8.076186in}}{\pgfqpoint{5.564075in}{8.071796in}}{\pgfqpoint{5.575125in}{8.071796in}}%
\pgfpathlineto{\pgfqpoint{5.575125in}{8.071796in}}%
\pgfpathclose%
\pgfusepath{stroke,fill}%
\end{pgfscope}%
\begin{pgfscope}%
\pgfpathrectangle{\pgfqpoint{5.292946in}{7.624184in}}{\pgfqpoint{2.177280in}{2.201755in}}%
\pgfusepath{clip}%
\pgfsetbuttcap%
\pgfsetroundjoin%
\definecolor{currentfill}{rgb}{0.121569,0.466667,0.705882}%
\pgfsetfillcolor{currentfill}%
\pgfsetlinewidth{0.481800pt}%
\definecolor{currentstroke}{rgb}{1.000000,1.000000,1.000000}%
\pgfsetstrokecolor{currentstroke}%
\pgfsetdash{}{0pt}%
\pgfpathmoveto{\pgfqpoint{5.661241in}{8.294195in}}%
\pgfpathcurveto{\pgfqpoint{5.672291in}{8.294195in}}{\pgfqpoint{5.682890in}{8.298586in}}{\pgfqpoint{5.690704in}{8.306399in}}%
\pgfpathcurveto{\pgfqpoint{5.698518in}{8.314213in}}{\pgfqpoint{5.702908in}{8.324812in}}{\pgfqpoint{5.702908in}{8.335862in}}%
\pgfpathcurveto{\pgfqpoint{5.702908in}{8.346912in}}{\pgfqpoint{5.698518in}{8.357511in}}{\pgfqpoint{5.690704in}{8.365325in}}%
\pgfpathcurveto{\pgfqpoint{5.682890in}{8.373139in}}{\pgfqpoint{5.672291in}{8.377529in}}{\pgfqpoint{5.661241in}{8.377529in}}%
\pgfpathcurveto{\pgfqpoint{5.650191in}{8.377529in}}{\pgfqpoint{5.639592in}{8.373139in}}{\pgfqpoint{5.631779in}{8.365325in}}%
\pgfpathcurveto{\pgfqpoint{5.623965in}{8.357511in}}{\pgfqpoint{5.619575in}{8.346912in}}{\pgfqpoint{5.619575in}{8.335862in}}%
\pgfpathcurveto{\pgfqpoint{5.619575in}{8.324812in}}{\pgfqpoint{5.623965in}{8.314213in}}{\pgfqpoint{5.631779in}{8.306399in}}%
\pgfpathcurveto{\pgfqpoint{5.639592in}{8.298586in}}{\pgfqpoint{5.650191in}{8.294195in}}{\pgfqpoint{5.661241in}{8.294195in}}%
\pgfpathlineto{\pgfqpoint{5.661241in}{8.294195in}}%
\pgfpathclose%
\pgfusepath{stroke,fill}%
\end{pgfscope}%
\begin{pgfscope}%
\pgfpathrectangle{\pgfqpoint{5.292946in}{7.624184in}}{\pgfqpoint{2.177280in}{2.201755in}}%
\pgfusepath{clip}%
\pgfsetbuttcap%
\pgfsetroundjoin%
\definecolor{currentfill}{rgb}{0.121569,0.466667,0.705882}%
\pgfsetfillcolor{currentfill}%
\pgfsetlinewidth{0.481800pt}%
\definecolor{currentstroke}{rgb}{1.000000,1.000000,1.000000}%
\pgfsetstrokecolor{currentstroke}%
\pgfsetdash{}{0pt}%
\pgfpathmoveto{\pgfqpoint{5.575125in}{7.849396in}}%
\pgfpathcurveto{\pgfqpoint{5.586176in}{7.849396in}}{\pgfqpoint{5.596775in}{7.853787in}}{\pgfqpoint{5.604588in}{7.861600in}}%
\pgfpathcurveto{\pgfqpoint{5.612402in}{7.869414in}}{\pgfqpoint{5.616792in}{7.880013in}}{\pgfqpoint{5.616792in}{7.891063in}}%
\pgfpathcurveto{\pgfqpoint{5.616792in}{7.902113in}}{\pgfqpoint{5.612402in}{7.912712in}}{\pgfqpoint{5.604588in}{7.920526in}}%
\pgfpathcurveto{\pgfqpoint{5.596775in}{7.928340in}}{\pgfqpoint{5.586176in}{7.932730in}}{\pgfqpoint{5.575125in}{7.932730in}}%
\pgfpathcurveto{\pgfqpoint{5.564075in}{7.932730in}}{\pgfqpoint{5.553476in}{7.928340in}}{\pgfqpoint{5.545663in}{7.920526in}}%
\pgfpathcurveto{\pgfqpoint{5.537849in}{7.912712in}}{\pgfqpoint{5.533459in}{7.902113in}}{\pgfqpoint{5.533459in}{7.891063in}}%
\pgfpathcurveto{\pgfqpoint{5.533459in}{7.880013in}}{\pgfqpoint{5.537849in}{7.869414in}}{\pgfqpoint{5.545663in}{7.861600in}}%
\pgfpathcurveto{\pgfqpoint{5.553476in}{7.853787in}}{\pgfqpoint{5.564075in}{7.849396in}}{\pgfqpoint{5.575125in}{7.849396in}}%
\pgfpathlineto{\pgfqpoint{5.575125in}{7.849396in}}%
\pgfpathclose%
\pgfusepath{stroke,fill}%
\end{pgfscope}%
\begin{pgfscope}%
\pgfpathrectangle{\pgfqpoint{5.292946in}{7.624184in}}{\pgfqpoint{2.177280in}{2.201755in}}%
\pgfusepath{clip}%
\pgfsetbuttcap%
\pgfsetroundjoin%
\definecolor{currentfill}{rgb}{0.121569,0.466667,0.705882}%
\pgfsetfillcolor{currentfill}%
\pgfsetlinewidth{0.481800pt}%
\definecolor{currentstroke}{rgb}{1.000000,1.000000,1.000000}%
\pgfsetstrokecolor{currentstroke}%
\pgfsetdash{}{0pt}%
\pgfpathmoveto{\pgfqpoint{5.603831in}{8.071796in}}%
\pgfpathcurveto{\pgfqpoint{5.614881in}{8.071796in}}{\pgfqpoint{5.625480in}{8.076186in}}{\pgfqpoint{5.633294in}{8.084000in}}%
\pgfpathcurveto{\pgfqpoint{5.641107in}{8.091813in}}{\pgfqpoint{5.645497in}{8.102413in}}{\pgfqpoint{5.645497in}{8.113463in}}%
\pgfpathcurveto{\pgfqpoint{5.645497in}{8.124513in}}{\pgfqpoint{5.641107in}{8.135112in}}{\pgfqpoint{5.633294in}{8.142925in}}%
\pgfpathcurveto{\pgfqpoint{5.625480in}{8.150739in}}{\pgfqpoint{5.614881in}{8.155129in}}{\pgfqpoint{5.603831in}{8.155129in}}%
\pgfpathcurveto{\pgfqpoint{5.592781in}{8.155129in}}{\pgfqpoint{5.582182in}{8.150739in}}{\pgfqpoint{5.574368in}{8.142925in}}%
\pgfpathcurveto{\pgfqpoint{5.566554in}{8.135112in}}{\pgfqpoint{5.562164in}{8.124513in}}{\pgfqpoint{5.562164in}{8.113463in}}%
\pgfpathcurveto{\pgfqpoint{5.562164in}{8.102413in}}{\pgfqpoint{5.566554in}{8.091813in}}{\pgfqpoint{5.574368in}{8.084000in}}%
\pgfpathcurveto{\pgfqpoint{5.582182in}{8.076186in}}{\pgfqpoint{5.592781in}{8.071796in}}{\pgfqpoint{5.603831in}{8.071796in}}%
\pgfpathlineto{\pgfqpoint{5.603831in}{8.071796in}}%
\pgfpathclose%
\pgfusepath{stroke,fill}%
\end{pgfscope}%
\begin{pgfscope}%
\pgfpathrectangle{\pgfqpoint{5.292946in}{7.624184in}}{\pgfqpoint{2.177280in}{2.201755in}}%
\pgfusepath{clip}%
\pgfsetbuttcap%
\pgfsetroundjoin%
\definecolor{currentfill}{rgb}{0.121569,0.466667,0.705882}%
\pgfsetfillcolor{currentfill}%
\pgfsetlinewidth{0.481800pt}%
\definecolor{currentstroke}{rgb}{1.000000,1.000000,1.000000}%
\pgfsetstrokecolor{currentstroke}%
\pgfsetdash{}{0pt}%
\pgfpathmoveto{\pgfqpoint{5.575125in}{7.738197in}}%
\pgfpathcurveto{\pgfqpoint{5.586176in}{7.738197in}}{\pgfqpoint{5.596775in}{7.742587in}}{\pgfqpoint{5.604588in}{7.750401in}}%
\pgfpathcurveto{\pgfqpoint{5.612402in}{7.758214in}}{\pgfqpoint{5.616792in}{7.768813in}}{\pgfqpoint{5.616792in}{7.779863in}}%
\pgfpathcurveto{\pgfqpoint{5.616792in}{7.790914in}}{\pgfqpoint{5.612402in}{7.801513in}}{\pgfqpoint{5.604588in}{7.809326in}}%
\pgfpathcurveto{\pgfqpoint{5.596775in}{7.817140in}}{\pgfqpoint{5.586176in}{7.821530in}}{\pgfqpoint{5.575125in}{7.821530in}}%
\pgfpathcurveto{\pgfqpoint{5.564075in}{7.821530in}}{\pgfqpoint{5.553476in}{7.817140in}}{\pgfqpoint{5.545663in}{7.809326in}}%
\pgfpathcurveto{\pgfqpoint{5.537849in}{7.801513in}}{\pgfqpoint{5.533459in}{7.790914in}}{\pgfqpoint{5.533459in}{7.779863in}}%
\pgfpathcurveto{\pgfqpoint{5.533459in}{7.768813in}}{\pgfqpoint{5.537849in}{7.758214in}}{\pgfqpoint{5.545663in}{7.750401in}}%
\pgfpathcurveto{\pgfqpoint{5.553476in}{7.742587in}}{\pgfqpoint{5.564075in}{7.738197in}}{\pgfqpoint{5.575125in}{7.738197in}}%
\pgfpathlineto{\pgfqpoint{5.575125in}{7.738197in}}%
\pgfpathclose%
\pgfusepath{stroke,fill}%
\end{pgfscope}%
\begin{pgfscope}%
\pgfpathrectangle{\pgfqpoint{5.292946in}{7.624184in}}{\pgfqpoint{2.177280in}{2.201755in}}%
\pgfusepath{clip}%
\pgfsetbuttcap%
\pgfsetroundjoin%
\definecolor{currentfill}{rgb}{0.121569,0.466667,0.705882}%
\pgfsetfillcolor{currentfill}%
\pgfsetlinewidth{0.481800pt}%
\definecolor{currentstroke}{rgb}{1.000000,1.000000,1.000000}%
\pgfsetstrokecolor{currentstroke}%
\pgfsetdash{}{0pt}%
\pgfpathmoveto{\pgfqpoint{5.603831in}{8.016196in}}%
\pgfpathcurveto{\pgfqpoint{5.614881in}{8.016196in}}{\pgfqpoint{5.625480in}{8.020586in}}{\pgfqpoint{5.633294in}{8.028400in}}%
\pgfpathcurveto{\pgfqpoint{5.641107in}{8.036214in}}{\pgfqpoint{5.645497in}{8.046813in}}{\pgfqpoint{5.645497in}{8.057863in}}%
\pgfpathcurveto{\pgfqpoint{5.645497in}{8.068913in}}{\pgfqpoint{5.641107in}{8.079512in}}{\pgfqpoint{5.633294in}{8.087326in}}%
\pgfpathcurveto{\pgfqpoint{5.625480in}{8.095139in}}{\pgfqpoint{5.614881in}{8.099529in}}{\pgfqpoint{5.603831in}{8.099529in}}%
\pgfpathcurveto{\pgfqpoint{5.592781in}{8.099529in}}{\pgfqpoint{5.582182in}{8.095139in}}{\pgfqpoint{5.574368in}{8.087326in}}%
\pgfpathcurveto{\pgfqpoint{5.566554in}{8.079512in}}{\pgfqpoint{5.562164in}{8.068913in}}{\pgfqpoint{5.562164in}{8.057863in}}%
\pgfpathcurveto{\pgfqpoint{5.562164in}{8.046813in}}{\pgfqpoint{5.566554in}{8.036214in}}{\pgfqpoint{5.574368in}{8.028400in}}%
\pgfpathcurveto{\pgfqpoint{5.582182in}{8.020586in}}{\pgfqpoint{5.592781in}{8.016196in}}{\pgfqpoint{5.603831in}{8.016196in}}%
\pgfpathlineto{\pgfqpoint{5.603831in}{8.016196in}}%
\pgfpathclose%
\pgfusepath{stroke,fill}%
\end{pgfscope}%
\begin{pgfscope}%
\pgfpathrectangle{\pgfqpoint{5.292946in}{7.624184in}}{\pgfqpoint{2.177280in}{2.201755in}}%
\pgfusepath{clip}%
\pgfsetbuttcap%
\pgfsetroundjoin%
\definecolor{currentfill}{rgb}{0.121569,0.466667,0.705882}%
\pgfsetfillcolor{currentfill}%
\pgfsetlinewidth{0.481800pt}%
\definecolor{currentstroke}{rgb}{1.000000,1.000000,1.000000}%
\pgfsetstrokecolor{currentstroke}%
\pgfsetdash{}{0pt}%
\pgfpathmoveto{\pgfqpoint{5.603831in}{8.294195in}}%
\pgfpathcurveto{\pgfqpoint{5.614881in}{8.294195in}}{\pgfqpoint{5.625480in}{8.298586in}}{\pgfqpoint{5.633294in}{8.306399in}}%
\pgfpathcurveto{\pgfqpoint{5.641107in}{8.314213in}}{\pgfqpoint{5.645497in}{8.324812in}}{\pgfqpoint{5.645497in}{8.335862in}}%
\pgfpathcurveto{\pgfqpoint{5.645497in}{8.346912in}}{\pgfqpoint{5.641107in}{8.357511in}}{\pgfqpoint{5.633294in}{8.365325in}}%
\pgfpathcurveto{\pgfqpoint{5.625480in}{8.373139in}}{\pgfqpoint{5.614881in}{8.377529in}}{\pgfqpoint{5.603831in}{8.377529in}}%
\pgfpathcurveto{\pgfqpoint{5.592781in}{8.377529in}}{\pgfqpoint{5.582182in}{8.373139in}}{\pgfqpoint{5.574368in}{8.365325in}}%
\pgfpathcurveto{\pgfqpoint{5.566554in}{8.357511in}}{\pgfqpoint{5.562164in}{8.346912in}}{\pgfqpoint{5.562164in}{8.335862in}}%
\pgfpathcurveto{\pgfqpoint{5.562164in}{8.324812in}}{\pgfqpoint{5.566554in}{8.314213in}}{\pgfqpoint{5.574368in}{8.306399in}}%
\pgfpathcurveto{\pgfqpoint{5.582182in}{8.298586in}}{\pgfqpoint{5.592781in}{8.294195in}}{\pgfqpoint{5.603831in}{8.294195in}}%
\pgfpathlineto{\pgfqpoint{5.603831in}{8.294195in}}%
\pgfpathclose%
\pgfusepath{stroke,fill}%
\end{pgfscope}%
\begin{pgfscope}%
\pgfpathrectangle{\pgfqpoint{5.292946in}{7.624184in}}{\pgfqpoint{2.177280in}{2.201755in}}%
\pgfusepath{clip}%
\pgfsetbuttcap%
\pgfsetroundjoin%
\definecolor{currentfill}{rgb}{0.121569,0.466667,0.705882}%
\pgfsetfillcolor{currentfill}%
\pgfsetlinewidth{0.481800pt}%
\definecolor{currentstroke}{rgb}{1.000000,1.000000,1.000000}%
\pgfsetstrokecolor{currentstroke}%
\pgfsetdash{}{0pt}%
\pgfpathmoveto{\pgfqpoint{5.632536in}{7.960596in}}%
\pgfpathcurveto{\pgfqpoint{5.643586in}{7.960596in}}{\pgfqpoint{5.654185in}{7.964986in}}{\pgfqpoint{5.661999in}{7.972800in}}%
\pgfpathcurveto{\pgfqpoint{5.669812in}{7.980614in}}{\pgfqpoint{5.674203in}{7.991213in}}{\pgfqpoint{5.674203in}{8.002263in}}%
\pgfpathcurveto{\pgfqpoint{5.674203in}{8.013313in}}{\pgfqpoint{5.669812in}{8.023912in}}{\pgfqpoint{5.661999in}{8.031726in}}%
\pgfpathcurveto{\pgfqpoint{5.654185in}{8.039539in}}{\pgfqpoint{5.643586in}{8.043930in}}{\pgfqpoint{5.632536in}{8.043930in}}%
\pgfpathcurveto{\pgfqpoint{5.621486in}{8.043930in}}{\pgfqpoint{5.610887in}{8.039539in}}{\pgfqpoint{5.603073in}{8.031726in}}%
\pgfpathcurveto{\pgfqpoint{5.595260in}{8.023912in}}{\pgfqpoint{5.590869in}{8.013313in}}{\pgfqpoint{5.590869in}{8.002263in}}%
\pgfpathcurveto{\pgfqpoint{5.590869in}{7.991213in}}{\pgfqpoint{5.595260in}{7.980614in}}{\pgfqpoint{5.603073in}{7.972800in}}%
\pgfpathcurveto{\pgfqpoint{5.610887in}{7.964986in}}{\pgfqpoint{5.621486in}{7.960596in}}{\pgfqpoint{5.632536in}{7.960596in}}%
\pgfpathlineto{\pgfqpoint{5.632536in}{7.960596in}}%
\pgfpathclose%
\pgfusepath{stroke,fill}%
\end{pgfscope}%
\begin{pgfscope}%
\pgfpathrectangle{\pgfqpoint{5.292946in}{7.624184in}}{\pgfqpoint{2.177280in}{2.201755in}}%
\pgfusepath{clip}%
\pgfsetbuttcap%
\pgfsetroundjoin%
\definecolor{currentfill}{rgb}{0.121569,0.466667,0.705882}%
\pgfsetfillcolor{currentfill}%
\pgfsetlinewidth{0.481800pt}%
\definecolor{currentstroke}{rgb}{1.000000,1.000000,1.000000}%
\pgfsetstrokecolor{currentstroke}%
\pgfsetdash{}{0pt}%
\pgfpathmoveto{\pgfqpoint{5.575125in}{7.960596in}}%
\pgfpathcurveto{\pgfqpoint{5.586176in}{7.960596in}}{\pgfqpoint{5.596775in}{7.964986in}}{\pgfqpoint{5.604588in}{7.972800in}}%
\pgfpathcurveto{\pgfqpoint{5.612402in}{7.980614in}}{\pgfqpoint{5.616792in}{7.991213in}}{\pgfqpoint{5.616792in}{8.002263in}}%
\pgfpathcurveto{\pgfqpoint{5.616792in}{8.013313in}}{\pgfqpoint{5.612402in}{8.023912in}}{\pgfqpoint{5.604588in}{8.031726in}}%
\pgfpathcurveto{\pgfqpoint{5.596775in}{8.039539in}}{\pgfqpoint{5.586176in}{8.043930in}}{\pgfqpoint{5.575125in}{8.043930in}}%
\pgfpathcurveto{\pgfqpoint{5.564075in}{8.043930in}}{\pgfqpoint{5.553476in}{8.039539in}}{\pgfqpoint{5.545663in}{8.031726in}}%
\pgfpathcurveto{\pgfqpoint{5.537849in}{8.023912in}}{\pgfqpoint{5.533459in}{8.013313in}}{\pgfqpoint{5.533459in}{8.002263in}}%
\pgfpathcurveto{\pgfqpoint{5.533459in}{7.991213in}}{\pgfqpoint{5.537849in}{7.980614in}}{\pgfqpoint{5.545663in}{7.972800in}}%
\pgfpathcurveto{\pgfqpoint{5.553476in}{7.964986in}}{\pgfqpoint{5.564075in}{7.960596in}}{\pgfqpoint{5.575125in}{7.960596in}}%
\pgfpathlineto{\pgfqpoint{5.575125in}{7.960596in}}%
\pgfpathclose%
\pgfusepath{stroke,fill}%
\end{pgfscope}%
\begin{pgfscope}%
\pgfpathrectangle{\pgfqpoint{5.292946in}{7.624184in}}{\pgfqpoint{2.177280in}{2.201755in}}%
\pgfusepath{clip}%
\pgfsetbuttcap%
\pgfsetroundjoin%
\definecolor{currentfill}{rgb}{0.121569,0.466667,0.705882}%
\pgfsetfillcolor{currentfill}%
\pgfsetlinewidth{0.481800pt}%
\definecolor{currentstroke}{rgb}{1.000000,1.000000,1.000000}%
\pgfsetstrokecolor{currentstroke}%
\pgfsetdash{}{0pt}%
\pgfpathmoveto{\pgfqpoint{5.489010in}{7.682597in}}%
\pgfpathcurveto{\pgfqpoint{5.500060in}{7.682597in}}{\pgfqpoint{5.510659in}{7.686987in}}{\pgfqpoint{5.518472in}{7.694801in}}%
\pgfpathcurveto{\pgfqpoint{5.526286in}{7.702614in}}{\pgfqpoint{5.530676in}{7.713213in}}{\pgfqpoint{5.530676in}{7.724264in}}%
\pgfpathcurveto{\pgfqpoint{5.530676in}{7.735314in}}{\pgfqpoint{5.526286in}{7.745913in}}{\pgfqpoint{5.518472in}{7.753726in}}%
\pgfpathcurveto{\pgfqpoint{5.510659in}{7.761540in}}{\pgfqpoint{5.500060in}{7.765930in}}{\pgfqpoint{5.489010in}{7.765930in}}%
\pgfpathcurveto{\pgfqpoint{5.477959in}{7.765930in}}{\pgfqpoint{5.467360in}{7.761540in}}{\pgfqpoint{5.459547in}{7.753726in}}%
\pgfpathcurveto{\pgfqpoint{5.451733in}{7.745913in}}{\pgfqpoint{5.447343in}{7.735314in}}{\pgfqpoint{5.447343in}{7.724264in}}%
\pgfpathcurveto{\pgfqpoint{5.447343in}{7.713213in}}{\pgfqpoint{5.451733in}{7.702614in}}{\pgfqpoint{5.459547in}{7.694801in}}%
\pgfpathcurveto{\pgfqpoint{5.467360in}{7.686987in}}{\pgfqpoint{5.477959in}{7.682597in}}{\pgfqpoint{5.489010in}{7.682597in}}%
\pgfpathlineto{\pgfqpoint{5.489010in}{7.682597in}}%
\pgfpathclose%
\pgfusepath{stroke,fill}%
\end{pgfscope}%
\begin{pgfscope}%
\pgfpathrectangle{\pgfqpoint{5.292946in}{7.624184in}}{\pgfqpoint{2.177280in}{2.201755in}}%
\pgfusepath{clip}%
\pgfsetbuttcap%
\pgfsetroundjoin%
\definecolor{currentfill}{rgb}{0.121569,0.466667,0.705882}%
\pgfsetfillcolor{currentfill}%
\pgfsetlinewidth{0.481800pt}%
\definecolor{currentstroke}{rgb}{1.000000,1.000000,1.000000}%
\pgfsetstrokecolor{currentstroke}%
\pgfsetdash{}{0pt}%
\pgfpathmoveto{\pgfqpoint{5.517715in}{8.516595in}}%
\pgfpathcurveto{\pgfqpoint{5.528765in}{8.516595in}}{\pgfqpoint{5.539364in}{8.520985in}}{\pgfqpoint{5.547178in}{8.528799in}}%
\pgfpathcurveto{\pgfqpoint{5.554991in}{8.536612in}}{\pgfqpoint{5.559382in}{8.547212in}}{\pgfqpoint{5.559382in}{8.558262in}}%
\pgfpathcurveto{\pgfqpoint{5.559382in}{8.569312in}}{\pgfqpoint{5.554991in}{8.579911in}}{\pgfqpoint{5.547178in}{8.587724in}}%
\pgfpathcurveto{\pgfqpoint{5.539364in}{8.595538in}}{\pgfqpoint{5.528765in}{8.599928in}}{\pgfqpoint{5.517715in}{8.599928in}}%
\pgfpathcurveto{\pgfqpoint{5.506665in}{8.599928in}}{\pgfqpoint{5.496066in}{8.595538in}}{\pgfqpoint{5.488252in}{8.587724in}}%
\pgfpathcurveto{\pgfqpoint{5.480438in}{8.579911in}}{\pgfqpoint{5.476048in}{8.569312in}}{\pgfqpoint{5.476048in}{8.558262in}}%
\pgfpathcurveto{\pgfqpoint{5.476048in}{8.547212in}}{\pgfqpoint{5.480438in}{8.536612in}}{\pgfqpoint{5.488252in}{8.528799in}}%
\pgfpathcurveto{\pgfqpoint{5.496066in}{8.520985in}}{\pgfqpoint{5.506665in}{8.516595in}}{\pgfqpoint{5.517715in}{8.516595in}}%
\pgfpathlineto{\pgfqpoint{5.517715in}{8.516595in}}%
\pgfpathclose%
\pgfusepath{stroke,fill}%
\end{pgfscope}%
\begin{pgfscope}%
\pgfpathrectangle{\pgfqpoint{5.292946in}{7.624184in}}{\pgfqpoint{2.177280in}{2.201755in}}%
\pgfusepath{clip}%
\pgfsetbuttcap%
\pgfsetroundjoin%
\definecolor{currentfill}{rgb}{0.121569,0.466667,0.705882}%
\pgfsetfillcolor{currentfill}%
\pgfsetlinewidth{0.481800pt}%
\definecolor{currentstroke}{rgb}{1.000000,1.000000,1.000000}%
\pgfsetstrokecolor{currentstroke}%
\pgfsetdash{}{0pt}%
\pgfpathmoveto{\pgfqpoint{5.603831in}{8.460995in}}%
\pgfpathcurveto{\pgfqpoint{5.614881in}{8.460995in}}{\pgfqpoint{5.625480in}{8.465385in}}{\pgfqpoint{5.633294in}{8.473199in}}%
\pgfpathcurveto{\pgfqpoint{5.641107in}{8.481013in}}{\pgfqpoint{5.645497in}{8.491612in}}{\pgfqpoint{5.645497in}{8.502662in}}%
\pgfpathcurveto{\pgfqpoint{5.645497in}{8.513712in}}{\pgfqpoint{5.641107in}{8.524311in}}{\pgfqpoint{5.633294in}{8.532125in}}%
\pgfpathcurveto{\pgfqpoint{5.625480in}{8.539938in}}{\pgfqpoint{5.614881in}{8.544328in}}{\pgfqpoint{5.603831in}{8.544328in}}%
\pgfpathcurveto{\pgfqpoint{5.592781in}{8.544328in}}{\pgfqpoint{5.582182in}{8.539938in}}{\pgfqpoint{5.574368in}{8.532125in}}%
\pgfpathcurveto{\pgfqpoint{5.566554in}{8.524311in}}{\pgfqpoint{5.562164in}{8.513712in}}{\pgfqpoint{5.562164in}{8.502662in}}%
\pgfpathcurveto{\pgfqpoint{5.562164in}{8.491612in}}{\pgfqpoint{5.566554in}{8.481013in}}{\pgfqpoint{5.574368in}{8.473199in}}%
\pgfpathcurveto{\pgfqpoint{5.582182in}{8.465385in}}{\pgfqpoint{5.592781in}{8.460995in}}{\pgfqpoint{5.603831in}{8.460995in}}%
\pgfpathlineto{\pgfqpoint{5.603831in}{8.460995in}}%
\pgfpathclose%
\pgfusepath{stroke,fill}%
\end{pgfscope}%
\begin{pgfscope}%
\pgfpathrectangle{\pgfqpoint{5.292946in}{7.624184in}}{\pgfqpoint{2.177280in}{2.201755in}}%
\pgfusepath{clip}%
\pgfsetbuttcap%
\pgfsetroundjoin%
\definecolor{currentfill}{rgb}{0.121569,0.466667,0.705882}%
\pgfsetfillcolor{currentfill}%
\pgfsetlinewidth{0.481800pt}%
\definecolor{currentstroke}{rgb}{1.000000,1.000000,1.000000}%
\pgfsetstrokecolor{currentstroke}%
\pgfsetdash{}{0pt}%
\pgfpathmoveto{\pgfqpoint{5.546420in}{8.294195in}}%
\pgfpathcurveto{\pgfqpoint{5.557470in}{8.294195in}}{\pgfqpoint{5.568069in}{8.298586in}}{\pgfqpoint{5.575883in}{8.306399in}}%
\pgfpathcurveto{\pgfqpoint{5.583697in}{8.314213in}}{\pgfqpoint{5.588087in}{8.324812in}}{\pgfqpoint{5.588087in}{8.335862in}}%
\pgfpathcurveto{\pgfqpoint{5.588087in}{8.346912in}}{\pgfqpoint{5.583697in}{8.357511in}}{\pgfqpoint{5.575883in}{8.365325in}}%
\pgfpathcurveto{\pgfqpoint{5.568069in}{8.373139in}}{\pgfqpoint{5.557470in}{8.377529in}}{\pgfqpoint{5.546420in}{8.377529in}}%
\pgfpathcurveto{\pgfqpoint{5.535370in}{8.377529in}}{\pgfqpoint{5.524771in}{8.373139in}}{\pgfqpoint{5.516957in}{8.365325in}}%
\pgfpathcurveto{\pgfqpoint{5.509144in}{8.357511in}}{\pgfqpoint{5.504753in}{8.346912in}}{\pgfqpoint{5.504753in}{8.335862in}}%
\pgfpathcurveto{\pgfqpoint{5.504753in}{8.324812in}}{\pgfqpoint{5.509144in}{8.314213in}}{\pgfqpoint{5.516957in}{8.306399in}}%
\pgfpathcurveto{\pgfqpoint{5.524771in}{8.298586in}}{\pgfqpoint{5.535370in}{8.294195in}}{\pgfqpoint{5.546420in}{8.294195in}}%
\pgfpathlineto{\pgfqpoint{5.546420in}{8.294195in}}%
\pgfpathclose%
\pgfusepath{stroke,fill}%
\end{pgfscope}%
\begin{pgfscope}%
\pgfpathrectangle{\pgfqpoint{5.292946in}{7.624184in}}{\pgfqpoint{2.177280in}{2.201755in}}%
\pgfusepath{clip}%
\pgfsetbuttcap%
\pgfsetroundjoin%
\definecolor{currentfill}{rgb}{0.121569,0.466667,0.705882}%
\pgfsetfillcolor{currentfill}%
\pgfsetlinewidth{0.481800pt}%
\definecolor{currentstroke}{rgb}{1.000000,1.000000,1.000000}%
\pgfsetstrokecolor{currentstroke}%
\pgfsetdash{}{0pt}%
\pgfpathmoveto{\pgfqpoint{5.575125in}{8.127396in}}%
\pgfpathcurveto{\pgfqpoint{5.586176in}{8.127396in}}{\pgfqpoint{5.596775in}{8.131786in}}{\pgfqpoint{5.604588in}{8.139600in}}%
\pgfpathcurveto{\pgfqpoint{5.612402in}{8.147413in}}{\pgfqpoint{5.616792in}{8.158012in}}{\pgfqpoint{5.616792in}{8.169063in}}%
\pgfpathcurveto{\pgfqpoint{5.616792in}{8.180113in}}{\pgfqpoint{5.612402in}{8.190712in}}{\pgfqpoint{5.604588in}{8.198525in}}%
\pgfpathcurveto{\pgfqpoint{5.596775in}{8.206339in}}{\pgfqpoint{5.586176in}{8.210729in}}{\pgfqpoint{5.575125in}{8.210729in}}%
\pgfpathcurveto{\pgfqpoint{5.564075in}{8.210729in}}{\pgfqpoint{5.553476in}{8.206339in}}{\pgfqpoint{5.545663in}{8.198525in}}%
\pgfpathcurveto{\pgfqpoint{5.537849in}{8.190712in}}{\pgfqpoint{5.533459in}{8.180113in}}{\pgfqpoint{5.533459in}{8.169063in}}%
\pgfpathcurveto{\pgfqpoint{5.533459in}{8.158012in}}{\pgfqpoint{5.537849in}{8.147413in}}{\pgfqpoint{5.545663in}{8.139600in}}%
\pgfpathcurveto{\pgfqpoint{5.553476in}{8.131786in}}{\pgfqpoint{5.564075in}{8.127396in}}{\pgfqpoint{5.575125in}{8.127396in}}%
\pgfpathlineto{\pgfqpoint{5.575125in}{8.127396in}}%
\pgfpathclose%
\pgfusepath{stroke,fill}%
\end{pgfscope}%
\begin{pgfscope}%
\pgfpathrectangle{\pgfqpoint{5.292946in}{7.624184in}}{\pgfqpoint{2.177280in}{2.201755in}}%
\pgfusepath{clip}%
\pgfsetbuttcap%
\pgfsetroundjoin%
\definecolor{currentfill}{rgb}{0.121569,0.466667,0.705882}%
\pgfsetfillcolor{currentfill}%
\pgfsetlinewidth{0.481800pt}%
\definecolor{currentstroke}{rgb}{1.000000,1.000000,1.000000}%
\pgfsetstrokecolor{currentstroke}%
\pgfsetdash{}{0pt}%
\pgfpathmoveto{\pgfqpoint{5.661241in}{8.460995in}}%
\pgfpathcurveto{\pgfqpoint{5.672291in}{8.460995in}}{\pgfqpoint{5.682890in}{8.465385in}}{\pgfqpoint{5.690704in}{8.473199in}}%
\pgfpathcurveto{\pgfqpoint{5.698518in}{8.481013in}}{\pgfqpoint{5.702908in}{8.491612in}}{\pgfqpoint{5.702908in}{8.502662in}}%
\pgfpathcurveto{\pgfqpoint{5.702908in}{8.513712in}}{\pgfqpoint{5.698518in}{8.524311in}}{\pgfqpoint{5.690704in}{8.532125in}}%
\pgfpathcurveto{\pgfqpoint{5.682890in}{8.539938in}}{\pgfqpoint{5.672291in}{8.544328in}}{\pgfqpoint{5.661241in}{8.544328in}}%
\pgfpathcurveto{\pgfqpoint{5.650191in}{8.544328in}}{\pgfqpoint{5.639592in}{8.539938in}}{\pgfqpoint{5.631779in}{8.532125in}}%
\pgfpathcurveto{\pgfqpoint{5.623965in}{8.524311in}}{\pgfqpoint{5.619575in}{8.513712in}}{\pgfqpoint{5.619575in}{8.502662in}}%
\pgfpathcurveto{\pgfqpoint{5.619575in}{8.491612in}}{\pgfqpoint{5.623965in}{8.481013in}}{\pgfqpoint{5.631779in}{8.473199in}}%
\pgfpathcurveto{\pgfqpoint{5.639592in}{8.465385in}}{\pgfqpoint{5.650191in}{8.460995in}}{\pgfqpoint{5.661241in}{8.460995in}}%
\pgfpathlineto{\pgfqpoint{5.661241in}{8.460995in}}%
\pgfpathclose%
\pgfusepath{stroke,fill}%
\end{pgfscope}%
\begin{pgfscope}%
\pgfpathrectangle{\pgfqpoint{5.292946in}{7.624184in}}{\pgfqpoint{2.177280in}{2.201755in}}%
\pgfusepath{clip}%
\pgfsetbuttcap%
\pgfsetroundjoin%
\definecolor{currentfill}{rgb}{0.121569,0.466667,0.705882}%
\pgfsetfillcolor{currentfill}%
\pgfsetlinewidth{0.481800pt}%
\definecolor{currentstroke}{rgb}{1.000000,1.000000,1.000000}%
\pgfsetstrokecolor{currentstroke}%
\pgfsetdash{}{0pt}%
\pgfpathmoveto{\pgfqpoint{5.603831in}{8.127396in}}%
\pgfpathcurveto{\pgfqpoint{5.614881in}{8.127396in}}{\pgfqpoint{5.625480in}{8.131786in}}{\pgfqpoint{5.633294in}{8.139600in}}%
\pgfpathcurveto{\pgfqpoint{5.641107in}{8.147413in}}{\pgfqpoint{5.645497in}{8.158012in}}{\pgfqpoint{5.645497in}{8.169063in}}%
\pgfpathcurveto{\pgfqpoint{5.645497in}{8.180113in}}{\pgfqpoint{5.641107in}{8.190712in}}{\pgfqpoint{5.633294in}{8.198525in}}%
\pgfpathcurveto{\pgfqpoint{5.625480in}{8.206339in}}{\pgfqpoint{5.614881in}{8.210729in}}{\pgfqpoint{5.603831in}{8.210729in}}%
\pgfpathcurveto{\pgfqpoint{5.592781in}{8.210729in}}{\pgfqpoint{5.582182in}{8.206339in}}{\pgfqpoint{5.574368in}{8.198525in}}%
\pgfpathcurveto{\pgfqpoint{5.566554in}{8.190712in}}{\pgfqpoint{5.562164in}{8.180113in}}{\pgfqpoint{5.562164in}{8.169063in}}%
\pgfpathcurveto{\pgfqpoint{5.562164in}{8.158012in}}{\pgfqpoint{5.566554in}{8.147413in}}{\pgfqpoint{5.574368in}{8.139600in}}%
\pgfpathcurveto{\pgfqpoint{5.582182in}{8.131786in}}{\pgfqpoint{5.592781in}{8.127396in}}{\pgfqpoint{5.603831in}{8.127396in}}%
\pgfpathlineto{\pgfqpoint{5.603831in}{8.127396in}}%
\pgfpathclose%
\pgfusepath{stroke,fill}%
\end{pgfscope}%
\begin{pgfscope}%
\pgfpathrectangle{\pgfqpoint{5.292946in}{7.624184in}}{\pgfqpoint{2.177280in}{2.201755in}}%
\pgfusepath{clip}%
\pgfsetbuttcap%
\pgfsetroundjoin%
\definecolor{currentfill}{rgb}{0.121569,0.466667,0.705882}%
\pgfsetfillcolor{currentfill}%
\pgfsetlinewidth{0.481800pt}%
\definecolor{currentstroke}{rgb}{1.000000,1.000000,1.000000}%
\pgfsetstrokecolor{currentstroke}%
\pgfsetdash{}{0pt}%
\pgfpathmoveto{\pgfqpoint{5.661241in}{8.294195in}}%
\pgfpathcurveto{\pgfqpoint{5.672291in}{8.294195in}}{\pgfqpoint{5.682890in}{8.298586in}}{\pgfqpoint{5.690704in}{8.306399in}}%
\pgfpathcurveto{\pgfqpoint{5.698518in}{8.314213in}}{\pgfqpoint{5.702908in}{8.324812in}}{\pgfqpoint{5.702908in}{8.335862in}}%
\pgfpathcurveto{\pgfqpoint{5.702908in}{8.346912in}}{\pgfqpoint{5.698518in}{8.357511in}}{\pgfqpoint{5.690704in}{8.365325in}}%
\pgfpathcurveto{\pgfqpoint{5.682890in}{8.373139in}}{\pgfqpoint{5.672291in}{8.377529in}}{\pgfqpoint{5.661241in}{8.377529in}}%
\pgfpathcurveto{\pgfqpoint{5.650191in}{8.377529in}}{\pgfqpoint{5.639592in}{8.373139in}}{\pgfqpoint{5.631779in}{8.365325in}}%
\pgfpathcurveto{\pgfqpoint{5.623965in}{8.357511in}}{\pgfqpoint{5.619575in}{8.346912in}}{\pgfqpoint{5.619575in}{8.335862in}}%
\pgfpathcurveto{\pgfqpoint{5.619575in}{8.324812in}}{\pgfqpoint{5.623965in}{8.314213in}}{\pgfqpoint{5.631779in}{8.306399in}}%
\pgfpathcurveto{\pgfqpoint{5.639592in}{8.298586in}}{\pgfqpoint{5.650191in}{8.294195in}}{\pgfqpoint{5.661241in}{8.294195in}}%
\pgfpathlineto{\pgfqpoint{5.661241in}{8.294195in}}%
\pgfpathclose%
\pgfusepath{stroke,fill}%
\end{pgfscope}%
\begin{pgfscope}%
\pgfpathrectangle{\pgfqpoint{5.292946in}{7.624184in}}{\pgfqpoint{2.177280in}{2.201755in}}%
\pgfusepath{clip}%
\pgfsetbuttcap%
\pgfsetroundjoin%
\definecolor{currentfill}{rgb}{0.121569,0.466667,0.705882}%
\pgfsetfillcolor{currentfill}%
\pgfsetlinewidth{0.481800pt}%
\definecolor{currentstroke}{rgb}{1.000000,1.000000,1.000000}%
\pgfsetstrokecolor{currentstroke}%
\pgfsetdash{}{0pt}%
\pgfpathmoveto{\pgfqpoint{5.603831in}{8.127396in}}%
\pgfpathcurveto{\pgfqpoint{5.614881in}{8.127396in}}{\pgfqpoint{5.625480in}{8.131786in}}{\pgfqpoint{5.633294in}{8.139600in}}%
\pgfpathcurveto{\pgfqpoint{5.641107in}{8.147413in}}{\pgfqpoint{5.645497in}{8.158012in}}{\pgfqpoint{5.645497in}{8.169063in}}%
\pgfpathcurveto{\pgfqpoint{5.645497in}{8.180113in}}{\pgfqpoint{5.641107in}{8.190712in}}{\pgfqpoint{5.633294in}{8.198525in}}%
\pgfpathcurveto{\pgfqpoint{5.625480in}{8.206339in}}{\pgfqpoint{5.614881in}{8.210729in}}{\pgfqpoint{5.603831in}{8.210729in}}%
\pgfpathcurveto{\pgfqpoint{5.592781in}{8.210729in}}{\pgfqpoint{5.582182in}{8.206339in}}{\pgfqpoint{5.574368in}{8.198525in}}%
\pgfpathcurveto{\pgfqpoint{5.566554in}{8.190712in}}{\pgfqpoint{5.562164in}{8.180113in}}{\pgfqpoint{5.562164in}{8.169063in}}%
\pgfpathcurveto{\pgfqpoint{5.562164in}{8.158012in}}{\pgfqpoint{5.566554in}{8.147413in}}{\pgfqpoint{5.574368in}{8.139600in}}%
\pgfpathcurveto{\pgfqpoint{5.582182in}{8.131786in}}{\pgfqpoint{5.592781in}{8.127396in}}{\pgfqpoint{5.603831in}{8.127396in}}%
\pgfpathlineto{\pgfqpoint{5.603831in}{8.127396in}}%
\pgfpathclose%
\pgfusepath{stroke,fill}%
\end{pgfscope}%
\begin{pgfscope}%
\pgfpathrectangle{\pgfqpoint{5.292946in}{7.624184in}}{\pgfqpoint{2.177280in}{2.201755in}}%
\pgfusepath{clip}%
\pgfsetbuttcap%
\pgfsetroundjoin%
\definecolor{currentfill}{rgb}{0.121569,0.466667,0.705882}%
\pgfsetfillcolor{currentfill}%
\pgfsetlinewidth{0.481800pt}%
\definecolor{currentstroke}{rgb}{1.000000,1.000000,1.000000}%
\pgfsetstrokecolor{currentstroke}%
\pgfsetdash{}{0pt}%
\pgfpathmoveto{\pgfqpoint{5.460304in}{7.849396in}}%
\pgfpathcurveto{\pgfqpoint{5.471354in}{7.849396in}}{\pgfqpoint{5.481953in}{7.853787in}}{\pgfqpoint{5.489767in}{7.861600in}}%
\pgfpathcurveto{\pgfqpoint{5.497581in}{7.869414in}}{\pgfqpoint{5.501971in}{7.880013in}}{\pgfqpoint{5.501971in}{7.891063in}}%
\pgfpathcurveto{\pgfqpoint{5.501971in}{7.902113in}}{\pgfqpoint{5.497581in}{7.912712in}}{\pgfqpoint{5.489767in}{7.920526in}}%
\pgfpathcurveto{\pgfqpoint{5.481953in}{7.928340in}}{\pgfqpoint{5.471354in}{7.932730in}}{\pgfqpoint{5.460304in}{7.932730in}}%
\pgfpathcurveto{\pgfqpoint{5.449254in}{7.932730in}}{\pgfqpoint{5.438655in}{7.928340in}}{\pgfqpoint{5.430842in}{7.920526in}}%
\pgfpathcurveto{\pgfqpoint{5.423028in}{7.912712in}}{\pgfqpoint{5.418638in}{7.902113in}}{\pgfqpoint{5.418638in}{7.891063in}}%
\pgfpathcurveto{\pgfqpoint{5.418638in}{7.880013in}}{\pgfqpoint{5.423028in}{7.869414in}}{\pgfqpoint{5.430842in}{7.861600in}}%
\pgfpathcurveto{\pgfqpoint{5.438655in}{7.853787in}}{\pgfqpoint{5.449254in}{7.849396in}}{\pgfqpoint{5.460304in}{7.849396in}}%
\pgfpathlineto{\pgfqpoint{5.460304in}{7.849396in}}%
\pgfpathclose%
\pgfusepath{stroke,fill}%
\end{pgfscope}%
\begin{pgfscope}%
\pgfpathrectangle{\pgfqpoint{5.292946in}{7.624184in}}{\pgfqpoint{2.177280in}{2.201755in}}%
\pgfusepath{clip}%
\pgfsetbuttcap%
\pgfsetroundjoin%
\definecolor{currentfill}{rgb}{0.121569,0.466667,0.705882}%
\pgfsetfillcolor{currentfill}%
\pgfsetlinewidth{0.481800pt}%
\definecolor{currentstroke}{rgb}{1.000000,1.000000,1.000000}%
\pgfsetstrokecolor{currentstroke}%
\pgfsetdash{}{0pt}%
\pgfpathmoveto{\pgfqpoint{5.661241in}{8.127396in}}%
\pgfpathcurveto{\pgfqpoint{5.672291in}{8.127396in}}{\pgfqpoint{5.682890in}{8.131786in}}{\pgfqpoint{5.690704in}{8.139600in}}%
\pgfpathcurveto{\pgfqpoint{5.698518in}{8.147413in}}{\pgfqpoint{5.702908in}{8.158012in}}{\pgfqpoint{5.702908in}{8.169063in}}%
\pgfpathcurveto{\pgfqpoint{5.702908in}{8.180113in}}{\pgfqpoint{5.698518in}{8.190712in}}{\pgfqpoint{5.690704in}{8.198525in}}%
\pgfpathcurveto{\pgfqpoint{5.682890in}{8.206339in}}{\pgfqpoint{5.672291in}{8.210729in}}{\pgfqpoint{5.661241in}{8.210729in}}%
\pgfpathcurveto{\pgfqpoint{5.650191in}{8.210729in}}{\pgfqpoint{5.639592in}{8.206339in}}{\pgfqpoint{5.631779in}{8.198525in}}%
\pgfpathcurveto{\pgfqpoint{5.623965in}{8.190712in}}{\pgfqpoint{5.619575in}{8.180113in}}{\pgfqpoint{5.619575in}{8.169063in}}%
\pgfpathcurveto{\pgfqpoint{5.619575in}{8.158012in}}{\pgfqpoint{5.623965in}{8.147413in}}{\pgfqpoint{5.631779in}{8.139600in}}%
\pgfpathcurveto{\pgfqpoint{5.639592in}{8.131786in}}{\pgfqpoint{5.650191in}{8.127396in}}{\pgfqpoint{5.661241in}{8.127396in}}%
\pgfpathlineto{\pgfqpoint{5.661241in}{8.127396in}}%
\pgfpathclose%
\pgfusepath{stroke,fill}%
\end{pgfscope}%
\begin{pgfscope}%
\pgfpathrectangle{\pgfqpoint{5.292946in}{7.624184in}}{\pgfqpoint{2.177280in}{2.201755in}}%
\pgfusepath{clip}%
\pgfsetbuttcap%
\pgfsetroundjoin%
\definecolor{currentfill}{rgb}{0.121569,0.466667,0.705882}%
\pgfsetfillcolor{currentfill}%
\pgfsetlinewidth{0.481800pt}%
\definecolor{currentstroke}{rgb}{1.000000,1.000000,1.000000}%
\pgfsetstrokecolor{currentstroke}%
\pgfsetdash{}{0pt}%
\pgfpathmoveto{\pgfqpoint{5.718652in}{7.960596in}}%
\pgfpathcurveto{\pgfqpoint{5.729702in}{7.960596in}}{\pgfqpoint{5.740301in}{7.964986in}}{\pgfqpoint{5.748115in}{7.972800in}}%
\pgfpathcurveto{\pgfqpoint{5.755928in}{7.980614in}}{\pgfqpoint{5.760319in}{7.991213in}}{\pgfqpoint{5.760319in}{8.002263in}}%
\pgfpathcurveto{\pgfqpoint{5.760319in}{8.013313in}}{\pgfqpoint{5.755928in}{8.023912in}}{\pgfqpoint{5.748115in}{8.031726in}}%
\pgfpathcurveto{\pgfqpoint{5.740301in}{8.039539in}}{\pgfqpoint{5.729702in}{8.043930in}}{\pgfqpoint{5.718652in}{8.043930in}}%
\pgfpathcurveto{\pgfqpoint{5.707602in}{8.043930in}}{\pgfqpoint{5.697003in}{8.039539in}}{\pgfqpoint{5.689189in}{8.031726in}}%
\pgfpathcurveto{\pgfqpoint{5.681375in}{8.023912in}}{\pgfqpoint{5.676985in}{8.013313in}}{\pgfqpoint{5.676985in}{8.002263in}}%
\pgfpathcurveto{\pgfqpoint{5.676985in}{7.991213in}}{\pgfqpoint{5.681375in}{7.980614in}}{\pgfqpoint{5.689189in}{7.972800in}}%
\pgfpathcurveto{\pgfqpoint{5.697003in}{7.964986in}}{\pgfqpoint{5.707602in}{7.960596in}}{\pgfqpoint{5.718652in}{7.960596in}}%
\pgfpathlineto{\pgfqpoint{5.718652in}{7.960596in}}%
\pgfpathclose%
\pgfusepath{stroke,fill}%
\end{pgfscope}%
\begin{pgfscope}%
\pgfpathrectangle{\pgfqpoint{5.292946in}{7.624184in}}{\pgfqpoint{2.177280in}{2.201755in}}%
\pgfusepath{clip}%
\pgfsetbuttcap%
\pgfsetroundjoin%
\definecolor{currentfill}{rgb}{0.121569,0.466667,0.705882}%
\pgfsetfillcolor{currentfill}%
\pgfsetlinewidth{0.481800pt}%
\definecolor{currentstroke}{rgb}{1.000000,1.000000,1.000000}%
\pgfsetstrokecolor{currentstroke}%
\pgfsetdash{}{0pt}%
\pgfpathmoveto{\pgfqpoint{5.632536in}{8.071796in}}%
\pgfpathcurveto{\pgfqpoint{5.643586in}{8.071796in}}{\pgfqpoint{5.654185in}{8.076186in}}{\pgfqpoint{5.661999in}{8.084000in}}%
\pgfpathcurveto{\pgfqpoint{5.669812in}{8.091813in}}{\pgfqpoint{5.674203in}{8.102413in}}{\pgfqpoint{5.674203in}{8.113463in}}%
\pgfpathcurveto{\pgfqpoint{5.674203in}{8.124513in}}{\pgfqpoint{5.669812in}{8.135112in}}{\pgfqpoint{5.661999in}{8.142925in}}%
\pgfpathcurveto{\pgfqpoint{5.654185in}{8.150739in}}{\pgfqpoint{5.643586in}{8.155129in}}{\pgfqpoint{5.632536in}{8.155129in}}%
\pgfpathcurveto{\pgfqpoint{5.621486in}{8.155129in}}{\pgfqpoint{5.610887in}{8.150739in}}{\pgfqpoint{5.603073in}{8.142925in}}%
\pgfpathcurveto{\pgfqpoint{5.595260in}{8.135112in}}{\pgfqpoint{5.590869in}{8.124513in}}{\pgfqpoint{5.590869in}{8.113463in}}%
\pgfpathcurveto{\pgfqpoint{5.590869in}{8.102413in}}{\pgfqpoint{5.595260in}{8.091813in}}{\pgfqpoint{5.603073in}{8.084000in}}%
\pgfpathcurveto{\pgfqpoint{5.610887in}{8.076186in}}{\pgfqpoint{5.621486in}{8.071796in}}{\pgfqpoint{5.632536in}{8.071796in}}%
\pgfpathlineto{\pgfqpoint{5.632536in}{8.071796in}}%
\pgfpathclose%
\pgfusepath{stroke,fill}%
\end{pgfscope}%
\begin{pgfscope}%
\pgfpathrectangle{\pgfqpoint{5.292946in}{7.624184in}}{\pgfqpoint{2.177280in}{2.201755in}}%
\pgfusepath{clip}%
\pgfsetbuttcap%
\pgfsetroundjoin%
\definecolor{currentfill}{rgb}{0.121569,0.466667,0.705882}%
\pgfsetfillcolor{currentfill}%
\pgfsetlinewidth{0.481800pt}%
\definecolor{currentstroke}{rgb}{1.000000,1.000000,1.000000}%
\pgfsetstrokecolor{currentstroke}%
\pgfsetdash{}{0pt}%
\pgfpathmoveto{\pgfqpoint{5.632536in}{8.071796in}}%
\pgfpathcurveto{\pgfqpoint{5.643586in}{8.071796in}}{\pgfqpoint{5.654185in}{8.076186in}}{\pgfqpoint{5.661999in}{8.084000in}}%
\pgfpathcurveto{\pgfqpoint{5.669812in}{8.091813in}}{\pgfqpoint{5.674203in}{8.102413in}}{\pgfqpoint{5.674203in}{8.113463in}}%
\pgfpathcurveto{\pgfqpoint{5.674203in}{8.124513in}}{\pgfqpoint{5.669812in}{8.135112in}}{\pgfqpoint{5.661999in}{8.142925in}}%
\pgfpathcurveto{\pgfqpoint{5.654185in}{8.150739in}}{\pgfqpoint{5.643586in}{8.155129in}}{\pgfqpoint{5.632536in}{8.155129in}}%
\pgfpathcurveto{\pgfqpoint{5.621486in}{8.155129in}}{\pgfqpoint{5.610887in}{8.150739in}}{\pgfqpoint{5.603073in}{8.142925in}}%
\pgfpathcurveto{\pgfqpoint{5.595260in}{8.135112in}}{\pgfqpoint{5.590869in}{8.124513in}}{\pgfqpoint{5.590869in}{8.113463in}}%
\pgfpathcurveto{\pgfqpoint{5.590869in}{8.102413in}}{\pgfqpoint{5.595260in}{8.091813in}}{\pgfqpoint{5.603073in}{8.084000in}}%
\pgfpathcurveto{\pgfqpoint{5.610887in}{8.076186in}}{\pgfqpoint{5.621486in}{8.071796in}}{\pgfqpoint{5.632536in}{8.071796in}}%
\pgfpathlineto{\pgfqpoint{5.632536in}{8.071796in}}%
\pgfpathclose%
\pgfusepath{stroke,fill}%
\end{pgfscope}%
\begin{pgfscope}%
\pgfpathrectangle{\pgfqpoint{5.292946in}{7.624184in}}{\pgfqpoint{2.177280in}{2.201755in}}%
\pgfusepath{clip}%
\pgfsetbuttcap%
\pgfsetroundjoin%
\definecolor{currentfill}{rgb}{0.121569,0.466667,0.705882}%
\pgfsetfillcolor{currentfill}%
\pgfsetlinewidth{0.481800pt}%
\definecolor{currentstroke}{rgb}{1.000000,1.000000,1.000000}%
\pgfsetstrokecolor{currentstroke}%
\pgfsetdash{}{0pt}%
\pgfpathmoveto{\pgfqpoint{5.603831in}{8.182996in}}%
\pgfpathcurveto{\pgfqpoint{5.614881in}{8.182996in}}{\pgfqpoint{5.625480in}{8.187386in}}{\pgfqpoint{5.633294in}{8.195200in}}%
\pgfpathcurveto{\pgfqpoint{5.641107in}{8.203013in}}{\pgfqpoint{5.645497in}{8.213612in}}{\pgfqpoint{5.645497in}{8.224662in}}%
\pgfpathcurveto{\pgfqpoint{5.645497in}{8.235713in}}{\pgfqpoint{5.641107in}{8.246312in}}{\pgfqpoint{5.633294in}{8.254125in}}%
\pgfpathcurveto{\pgfqpoint{5.625480in}{8.261939in}}{\pgfqpoint{5.614881in}{8.266329in}}{\pgfqpoint{5.603831in}{8.266329in}}%
\pgfpathcurveto{\pgfqpoint{5.592781in}{8.266329in}}{\pgfqpoint{5.582182in}{8.261939in}}{\pgfqpoint{5.574368in}{8.254125in}}%
\pgfpathcurveto{\pgfqpoint{5.566554in}{8.246312in}}{\pgfqpoint{5.562164in}{8.235713in}}{\pgfqpoint{5.562164in}{8.224662in}}%
\pgfpathcurveto{\pgfqpoint{5.562164in}{8.213612in}}{\pgfqpoint{5.566554in}{8.203013in}}{\pgfqpoint{5.574368in}{8.195200in}}%
\pgfpathcurveto{\pgfqpoint{5.582182in}{8.187386in}}{\pgfqpoint{5.592781in}{8.182996in}}{\pgfqpoint{5.603831in}{8.182996in}}%
\pgfpathlineto{\pgfqpoint{5.603831in}{8.182996in}}%
\pgfpathclose%
\pgfusepath{stroke,fill}%
\end{pgfscope}%
\begin{pgfscope}%
\pgfpathrectangle{\pgfqpoint{5.292946in}{7.624184in}}{\pgfqpoint{2.177280in}{2.201755in}}%
\pgfusepath{clip}%
\pgfsetbuttcap%
\pgfsetroundjoin%
\definecolor{currentfill}{rgb}{0.121569,0.466667,0.705882}%
\pgfsetfillcolor{currentfill}%
\pgfsetlinewidth{0.481800pt}%
\definecolor{currentstroke}{rgb}{1.000000,1.000000,1.000000}%
\pgfsetstrokecolor{currentstroke}%
\pgfsetdash{}{0pt}%
\pgfpathmoveto{\pgfqpoint{5.575125in}{8.182996in}}%
\pgfpathcurveto{\pgfqpoint{5.586176in}{8.182996in}}{\pgfqpoint{5.596775in}{8.187386in}}{\pgfqpoint{5.604588in}{8.195200in}}%
\pgfpathcurveto{\pgfqpoint{5.612402in}{8.203013in}}{\pgfqpoint{5.616792in}{8.213612in}}{\pgfqpoint{5.616792in}{8.224662in}}%
\pgfpathcurveto{\pgfqpoint{5.616792in}{8.235713in}}{\pgfqpoint{5.612402in}{8.246312in}}{\pgfqpoint{5.604588in}{8.254125in}}%
\pgfpathcurveto{\pgfqpoint{5.596775in}{8.261939in}}{\pgfqpoint{5.586176in}{8.266329in}}{\pgfqpoint{5.575125in}{8.266329in}}%
\pgfpathcurveto{\pgfqpoint{5.564075in}{8.266329in}}{\pgfqpoint{5.553476in}{8.261939in}}{\pgfqpoint{5.545663in}{8.254125in}}%
\pgfpathcurveto{\pgfqpoint{5.537849in}{8.246312in}}{\pgfqpoint{5.533459in}{8.235713in}}{\pgfqpoint{5.533459in}{8.224662in}}%
\pgfpathcurveto{\pgfqpoint{5.533459in}{8.213612in}}{\pgfqpoint{5.537849in}{8.203013in}}{\pgfqpoint{5.545663in}{8.195200in}}%
\pgfpathcurveto{\pgfqpoint{5.553476in}{8.187386in}}{\pgfqpoint{5.564075in}{8.182996in}}{\pgfqpoint{5.575125in}{8.182996in}}%
\pgfpathlineto{\pgfqpoint{5.575125in}{8.182996in}}%
\pgfpathclose%
\pgfusepath{stroke,fill}%
\end{pgfscope}%
\begin{pgfscope}%
\pgfpathrectangle{\pgfqpoint{5.292946in}{7.624184in}}{\pgfqpoint{2.177280in}{2.201755in}}%
\pgfusepath{clip}%
\pgfsetbuttcap%
\pgfsetroundjoin%
\definecolor{currentfill}{rgb}{0.121569,0.466667,0.705882}%
\pgfsetfillcolor{currentfill}%
\pgfsetlinewidth{0.481800pt}%
\definecolor{currentstroke}{rgb}{1.000000,1.000000,1.000000}%
\pgfsetstrokecolor{currentstroke}%
\pgfsetdash{}{0pt}%
\pgfpathmoveto{\pgfqpoint{5.632536in}{7.904996in}}%
\pgfpathcurveto{\pgfqpoint{5.643586in}{7.904996in}}{\pgfqpoint{5.654185in}{7.909387in}}{\pgfqpoint{5.661999in}{7.917200in}}%
\pgfpathcurveto{\pgfqpoint{5.669812in}{7.925014in}}{\pgfqpoint{5.674203in}{7.935613in}}{\pgfqpoint{5.674203in}{7.946663in}}%
\pgfpathcurveto{\pgfqpoint{5.674203in}{7.957713in}}{\pgfqpoint{5.669812in}{7.968312in}}{\pgfqpoint{5.661999in}{7.976126in}}%
\pgfpathcurveto{\pgfqpoint{5.654185in}{7.983939in}}{\pgfqpoint{5.643586in}{7.988330in}}{\pgfqpoint{5.632536in}{7.988330in}}%
\pgfpathcurveto{\pgfqpoint{5.621486in}{7.988330in}}{\pgfqpoint{5.610887in}{7.983939in}}{\pgfqpoint{5.603073in}{7.976126in}}%
\pgfpathcurveto{\pgfqpoint{5.595260in}{7.968312in}}{\pgfqpoint{5.590869in}{7.957713in}}{\pgfqpoint{5.590869in}{7.946663in}}%
\pgfpathcurveto{\pgfqpoint{5.590869in}{7.935613in}}{\pgfqpoint{5.595260in}{7.925014in}}{\pgfqpoint{5.603073in}{7.917200in}}%
\pgfpathcurveto{\pgfqpoint{5.610887in}{7.909387in}}{\pgfqpoint{5.621486in}{7.904996in}}{\pgfqpoint{5.632536in}{7.904996in}}%
\pgfpathlineto{\pgfqpoint{5.632536in}{7.904996in}}%
\pgfpathclose%
\pgfusepath{stroke,fill}%
\end{pgfscope}%
\begin{pgfscope}%
\pgfpathrectangle{\pgfqpoint{5.292946in}{7.624184in}}{\pgfqpoint{2.177280in}{2.201755in}}%
\pgfusepath{clip}%
\pgfsetbuttcap%
\pgfsetroundjoin%
\definecolor{currentfill}{rgb}{0.121569,0.466667,0.705882}%
\pgfsetfillcolor{currentfill}%
\pgfsetlinewidth{0.481800pt}%
\definecolor{currentstroke}{rgb}{1.000000,1.000000,1.000000}%
\pgfsetstrokecolor{currentstroke}%
\pgfsetdash{}{0pt}%
\pgfpathmoveto{\pgfqpoint{5.632536in}{7.960596in}}%
\pgfpathcurveto{\pgfqpoint{5.643586in}{7.960596in}}{\pgfqpoint{5.654185in}{7.964986in}}{\pgfqpoint{5.661999in}{7.972800in}}%
\pgfpathcurveto{\pgfqpoint{5.669812in}{7.980614in}}{\pgfqpoint{5.674203in}{7.991213in}}{\pgfqpoint{5.674203in}{8.002263in}}%
\pgfpathcurveto{\pgfqpoint{5.674203in}{8.013313in}}{\pgfqpoint{5.669812in}{8.023912in}}{\pgfqpoint{5.661999in}{8.031726in}}%
\pgfpathcurveto{\pgfqpoint{5.654185in}{8.039539in}}{\pgfqpoint{5.643586in}{8.043930in}}{\pgfqpoint{5.632536in}{8.043930in}}%
\pgfpathcurveto{\pgfqpoint{5.621486in}{8.043930in}}{\pgfqpoint{5.610887in}{8.039539in}}{\pgfqpoint{5.603073in}{8.031726in}}%
\pgfpathcurveto{\pgfqpoint{5.595260in}{8.023912in}}{\pgfqpoint{5.590869in}{8.013313in}}{\pgfqpoint{5.590869in}{8.002263in}}%
\pgfpathcurveto{\pgfqpoint{5.590869in}{7.991213in}}{\pgfqpoint{5.595260in}{7.980614in}}{\pgfqpoint{5.603073in}{7.972800in}}%
\pgfpathcurveto{\pgfqpoint{5.610887in}{7.964986in}}{\pgfqpoint{5.621486in}{7.960596in}}{\pgfqpoint{5.632536in}{7.960596in}}%
\pgfpathlineto{\pgfqpoint{5.632536in}{7.960596in}}%
\pgfpathclose%
\pgfusepath{stroke,fill}%
\end{pgfscope}%
\begin{pgfscope}%
\pgfpathrectangle{\pgfqpoint{5.292946in}{7.624184in}}{\pgfqpoint{2.177280in}{2.201755in}}%
\pgfusepath{clip}%
\pgfsetbuttcap%
\pgfsetroundjoin%
\definecolor{currentfill}{rgb}{0.121569,0.466667,0.705882}%
\pgfsetfillcolor{currentfill}%
\pgfsetlinewidth{0.481800pt}%
\definecolor{currentstroke}{rgb}{1.000000,1.000000,1.000000}%
\pgfsetstrokecolor{currentstroke}%
\pgfsetdash{}{0pt}%
\pgfpathmoveto{\pgfqpoint{5.603831in}{8.294195in}}%
\pgfpathcurveto{\pgfqpoint{5.614881in}{8.294195in}}{\pgfqpoint{5.625480in}{8.298586in}}{\pgfqpoint{5.633294in}{8.306399in}}%
\pgfpathcurveto{\pgfqpoint{5.641107in}{8.314213in}}{\pgfqpoint{5.645497in}{8.324812in}}{\pgfqpoint{5.645497in}{8.335862in}}%
\pgfpathcurveto{\pgfqpoint{5.645497in}{8.346912in}}{\pgfqpoint{5.641107in}{8.357511in}}{\pgfqpoint{5.633294in}{8.365325in}}%
\pgfpathcurveto{\pgfqpoint{5.625480in}{8.373139in}}{\pgfqpoint{5.614881in}{8.377529in}}{\pgfqpoint{5.603831in}{8.377529in}}%
\pgfpathcurveto{\pgfqpoint{5.592781in}{8.377529in}}{\pgfqpoint{5.582182in}{8.373139in}}{\pgfqpoint{5.574368in}{8.365325in}}%
\pgfpathcurveto{\pgfqpoint{5.566554in}{8.357511in}}{\pgfqpoint{5.562164in}{8.346912in}}{\pgfqpoint{5.562164in}{8.335862in}}%
\pgfpathcurveto{\pgfqpoint{5.562164in}{8.324812in}}{\pgfqpoint{5.566554in}{8.314213in}}{\pgfqpoint{5.574368in}{8.306399in}}%
\pgfpathcurveto{\pgfqpoint{5.582182in}{8.298586in}}{\pgfqpoint{5.592781in}{8.294195in}}{\pgfqpoint{5.603831in}{8.294195in}}%
\pgfpathlineto{\pgfqpoint{5.603831in}{8.294195in}}%
\pgfpathclose%
\pgfusepath{stroke,fill}%
\end{pgfscope}%
\begin{pgfscope}%
\pgfpathrectangle{\pgfqpoint{5.292946in}{7.624184in}}{\pgfqpoint{2.177280in}{2.201755in}}%
\pgfusepath{clip}%
\pgfsetbuttcap%
\pgfsetroundjoin%
\definecolor{currentfill}{rgb}{0.121569,0.466667,0.705882}%
\pgfsetfillcolor{currentfill}%
\pgfsetlinewidth{0.481800pt}%
\definecolor{currentstroke}{rgb}{1.000000,1.000000,1.000000}%
\pgfsetstrokecolor{currentstroke}%
\pgfsetdash{}{0pt}%
\pgfpathmoveto{\pgfqpoint{5.603831in}{8.182996in}}%
\pgfpathcurveto{\pgfqpoint{5.614881in}{8.182996in}}{\pgfqpoint{5.625480in}{8.187386in}}{\pgfqpoint{5.633294in}{8.195200in}}%
\pgfpathcurveto{\pgfqpoint{5.641107in}{8.203013in}}{\pgfqpoint{5.645497in}{8.213612in}}{\pgfqpoint{5.645497in}{8.224662in}}%
\pgfpathcurveto{\pgfqpoint{5.645497in}{8.235713in}}{\pgfqpoint{5.641107in}{8.246312in}}{\pgfqpoint{5.633294in}{8.254125in}}%
\pgfpathcurveto{\pgfqpoint{5.625480in}{8.261939in}}{\pgfqpoint{5.614881in}{8.266329in}}{\pgfqpoint{5.603831in}{8.266329in}}%
\pgfpathcurveto{\pgfqpoint{5.592781in}{8.266329in}}{\pgfqpoint{5.582182in}{8.261939in}}{\pgfqpoint{5.574368in}{8.254125in}}%
\pgfpathcurveto{\pgfqpoint{5.566554in}{8.246312in}}{\pgfqpoint{5.562164in}{8.235713in}}{\pgfqpoint{5.562164in}{8.224662in}}%
\pgfpathcurveto{\pgfqpoint{5.562164in}{8.213612in}}{\pgfqpoint{5.566554in}{8.203013in}}{\pgfqpoint{5.574368in}{8.195200in}}%
\pgfpathcurveto{\pgfqpoint{5.582182in}{8.187386in}}{\pgfqpoint{5.592781in}{8.182996in}}{\pgfqpoint{5.603831in}{8.182996in}}%
\pgfpathlineto{\pgfqpoint{5.603831in}{8.182996in}}%
\pgfpathclose%
\pgfusepath{stroke,fill}%
\end{pgfscope}%
\begin{pgfscope}%
\pgfpathrectangle{\pgfqpoint{5.292946in}{7.624184in}}{\pgfqpoint{2.177280in}{2.201755in}}%
\pgfusepath{clip}%
\pgfsetbuttcap%
\pgfsetroundjoin%
\definecolor{currentfill}{rgb}{0.121569,0.466667,0.705882}%
\pgfsetfillcolor{currentfill}%
\pgfsetlinewidth{0.481800pt}%
\definecolor{currentstroke}{rgb}{1.000000,1.000000,1.000000}%
\pgfsetstrokecolor{currentstroke}%
\pgfsetdash{}{0pt}%
\pgfpathmoveto{\pgfqpoint{5.575125in}{8.349795in}}%
\pgfpathcurveto{\pgfqpoint{5.586176in}{8.349795in}}{\pgfqpoint{5.596775in}{8.354186in}}{\pgfqpoint{5.604588in}{8.361999in}}%
\pgfpathcurveto{\pgfqpoint{5.612402in}{8.369813in}}{\pgfqpoint{5.616792in}{8.380412in}}{\pgfqpoint{5.616792in}{8.391462in}}%
\pgfpathcurveto{\pgfqpoint{5.616792in}{8.402512in}}{\pgfqpoint{5.612402in}{8.413111in}}{\pgfqpoint{5.604588in}{8.420925in}}%
\pgfpathcurveto{\pgfqpoint{5.596775in}{8.428738in}}{\pgfqpoint{5.586176in}{8.433129in}}{\pgfqpoint{5.575125in}{8.433129in}}%
\pgfpathcurveto{\pgfqpoint{5.564075in}{8.433129in}}{\pgfqpoint{5.553476in}{8.428738in}}{\pgfqpoint{5.545663in}{8.420925in}}%
\pgfpathcurveto{\pgfqpoint{5.537849in}{8.413111in}}{\pgfqpoint{5.533459in}{8.402512in}}{\pgfqpoint{5.533459in}{8.391462in}}%
\pgfpathcurveto{\pgfqpoint{5.533459in}{8.380412in}}{\pgfqpoint{5.537849in}{8.369813in}}{\pgfqpoint{5.545663in}{8.361999in}}%
\pgfpathcurveto{\pgfqpoint{5.553476in}{8.354186in}}{\pgfqpoint{5.564075in}{8.349795in}}{\pgfqpoint{5.575125in}{8.349795in}}%
\pgfpathlineto{\pgfqpoint{5.575125in}{8.349795in}}%
\pgfpathclose%
\pgfusepath{stroke,fill}%
\end{pgfscope}%
\begin{pgfscope}%
\pgfpathrectangle{\pgfqpoint{5.292946in}{7.624184in}}{\pgfqpoint{2.177280in}{2.201755in}}%
\pgfusepath{clip}%
\pgfsetbuttcap%
\pgfsetroundjoin%
\definecolor{currentfill}{rgb}{0.121569,0.466667,0.705882}%
\pgfsetfillcolor{currentfill}%
\pgfsetlinewidth{0.481800pt}%
\definecolor{currentstroke}{rgb}{1.000000,1.000000,1.000000}%
\pgfsetstrokecolor{currentstroke}%
\pgfsetdash{}{0pt}%
\pgfpathmoveto{\pgfqpoint{5.603831in}{8.016196in}}%
\pgfpathcurveto{\pgfqpoint{5.614881in}{8.016196in}}{\pgfqpoint{5.625480in}{8.020586in}}{\pgfqpoint{5.633294in}{8.028400in}}%
\pgfpathcurveto{\pgfqpoint{5.641107in}{8.036214in}}{\pgfqpoint{5.645497in}{8.046813in}}{\pgfqpoint{5.645497in}{8.057863in}}%
\pgfpathcurveto{\pgfqpoint{5.645497in}{8.068913in}}{\pgfqpoint{5.641107in}{8.079512in}}{\pgfqpoint{5.633294in}{8.087326in}}%
\pgfpathcurveto{\pgfqpoint{5.625480in}{8.095139in}}{\pgfqpoint{5.614881in}{8.099529in}}{\pgfqpoint{5.603831in}{8.099529in}}%
\pgfpathcurveto{\pgfqpoint{5.592781in}{8.099529in}}{\pgfqpoint{5.582182in}{8.095139in}}{\pgfqpoint{5.574368in}{8.087326in}}%
\pgfpathcurveto{\pgfqpoint{5.566554in}{8.079512in}}{\pgfqpoint{5.562164in}{8.068913in}}{\pgfqpoint{5.562164in}{8.057863in}}%
\pgfpathcurveto{\pgfqpoint{5.562164in}{8.046813in}}{\pgfqpoint{5.566554in}{8.036214in}}{\pgfqpoint{5.574368in}{8.028400in}}%
\pgfpathcurveto{\pgfqpoint{5.582182in}{8.020586in}}{\pgfqpoint{5.592781in}{8.016196in}}{\pgfqpoint{5.603831in}{8.016196in}}%
\pgfpathlineto{\pgfqpoint{5.603831in}{8.016196in}}%
\pgfpathclose%
\pgfusepath{stroke,fill}%
\end{pgfscope}%
\begin{pgfscope}%
\pgfpathrectangle{\pgfqpoint{5.292946in}{7.624184in}}{\pgfqpoint{2.177280in}{2.201755in}}%
\pgfusepath{clip}%
\pgfsetbuttcap%
\pgfsetroundjoin%
\definecolor{currentfill}{rgb}{0.121569,0.466667,0.705882}%
\pgfsetfillcolor{currentfill}%
\pgfsetlinewidth{0.481800pt}%
\definecolor{currentstroke}{rgb}{1.000000,1.000000,1.000000}%
\pgfsetstrokecolor{currentstroke}%
\pgfsetdash{}{0pt}%
\pgfpathmoveto{\pgfqpoint{5.517715in}{8.071796in}}%
\pgfpathcurveto{\pgfqpoint{5.528765in}{8.071796in}}{\pgfqpoint{5.539364in}{8.076186in}}{\pgfqpoint{5.547178in}{8.084000in}}%
\pgfpathcurveto{\pgfqpoint{5.554991in}{8.091813in}}{\pgfqpoint{5.559382in}{8.102413in}}{\pgfqpoint{5.559382in}{8.113463in}}%
\pgfpathcurveto{\pgfqpoint{5.559382in}{8.124513in}}{\pgfqpoint{5.554991in}{8.135112in}}{\pgfqpoint{5.547178in}{8.142925in}}%
\pgfpathcurveto{\pgfqpoint{5.539364in}{8.150739in}}{\pgfqpoint{5.528765in}{8.155129in}}{\pgfqpoint{5.517715in}{8.155129in}}%
\pgfpathcurveto{\pgfqpoint{5.506665in}{8.155129in}}{\pgfqpoint{5.496066in}{8.150739in}}{\pgfqpoint{5.488252in}{8.142925in}}%
\pgfpathcurveto{\pgfqpoint{5.480438in}{8.135112in}}{\pgfqpoint{5.476048in}{8.124513in}}{\pgfqpoint{5.476048in}{8.113463in}}%
\pgfpathcurveto{\pgfqpoint{5.476048in}{8.102413in}}{\pgfqpoint{5.480438in}{8.091813in}}{\pgfqpoint{5.488252in}{8.084000in}}%
\pgfpathcurveto{\pgfqpoint{5.496066in}{8.076186in}}{\pgfqpoint{5.506665in}{8.071796in}}{\pgfqpoint{5.517715in}{8.071796in}}%
\pgfpathlineto{\pgfqpoint{5.517715in}{8.071796in}}%
\pgfpathclose%
\pgfusepath{stroke,fill}%
\end{pgfscope}%
\begin{pgfscope}%
\pgfpathrectangle{\pgfqpoint{5.292946in}{7.624184in}}{\pgfqpoint{2.177280in}{2.201755in}}%
\pgfusepath{clip}%
\pgfsetbuttcap%
\pgfsetroundjoin%
\definecolor{currentfill}{rgb}{0.121569,0.466667,0.705882}%
\pgfsetfillcolor{currentfill}%
\pgfsetlinewidth{0.481800pt}%
\definecolor{currentstroke}{rgb}{1.000000,1.000000,1.000000}%
\pgfsetstrokecolor{currentstroke}%
\pgfsetdash{}{0pt}%
\pgfpathmoveto{\pgfqpoint{5.546420in}{8.349795in}}%
\pgfpathcurveto{\pgfqpoint{5.557470in}{8.349795in}}{\pgfqpoint{5.568069in}{8.354186in}}{\pgfqpoint{5.575883in}{8.361999in}}%
\pgfpathcurveto{\pgfqpoint{5.583697in}{8.369813in}}{\pgfqpoint{5.588087in}{8.380412in}}{\pgfqpoint{5.588087in}{8.391462in}}%
\pgfpathcurveto{\pgfqpoint{5.588087in}{8.402512in}}{\pgfqpoint{5.583697in}{8.413111in}}{\pgfqpoint{5.575883in}{8.420925in}}%
\pgfpathcurveto{\pgfqpoint{5.568069in}{8.428738in}}{\pgfqpoint{5.557470in}{8.433129in}}{\pgfqpoint{5.546420in}{8.433129in}}%
\pgfpathcurveto{\pgfqpoint{5.535370in}{8.433129in}}{\pgfqpoint{5.524771in}{8.428738in}}{\pgfqpoint{5.516957in}{8.420925in}}%
\pgfpathcurveto{\pgfqpoint{5.509144in}{8.413111in}}{\pgfqpoint{5.504753in}{8.402512in}}{\pgfqpoint{5.504753in}{8.391462in}}%
\pgfpathcurveto{\pgfqpoint{5.504753in}{8.380412in}}{\pgfqpoint{5.509144in}{8.369813in}}{\pgfqpoint{5.516957in}{8.361999in}}%
\pgfpathcurveto{\pgfqpoint{5.524771in}{8.354186in}}{\pgfqpoint{5.535370in}{8.349795in}}{\pgfqpoint{5.546420in}{8.349795in}}%
\pgfpathlineto{\pgfqpoint{5.546420in}{8.349795in}}%
\pgfpathclose%
\pgfusepath{stroke,fill}%
\end{pgfscope}%
\begin{pgfscope}%
\pgfpathrectangle{\pgfqpoint{5.292946in}{7.624184in}}{\pgfqpoint{2.177280in}{2.201755in}}%
\pgfusepath{clip}%
\pgfsetbuttcap%
\pgfsetroundjoin%
\definecolor{currentfill}{rgb}{0.121569,0.466667,0.705882}%
\pgfsetfillcolor{currentfill}%
\pgfsetlinewidth{0.481800pt}%
\definecolor{currentstroke}{rgb}{1.000000,1.000000,1.000000}%
\pgfsetstrokecolor{currentstroke}%
\pgfsetdash{}{0pt}%
\pgfpathmoveto{\pgfqpoint{5.575125in}{8.016196in}}%
\pgfpathcurveto{\pgfqpoint{5.586176in}{8.016196in}}{\pgfqpoint{5.596775in}{8.020586in}}{\pgfqpoint{5.604588in}{8.028400in}}%
\pgfpathcurveto{\pgfqpoint{5.612402in}{8.036214in}}{\pgfqpoint{5.616792in}{8.046813in}}{\pgfqpoint{5.616792in}{8.057863in}}%
\pgfpathcurveto{\pgfqpoint{5.616792in}{8.068913in}}{\pgfqpoint{5.612402in}{8.079512in}}{\pgfqpoint{5.604588in}{8.087326in}}%
\pgfpathcurveto{\pgfqpoint{5.596775in}{8.095139in}}{\pgfqpoint{5.586176in}{8.099529in}}{\pgfqpoint{5.575125in}{8.099529in}}%
\pgfpathcurveto{\pgfqpoint{5.564075in}{8.099529in}}{\pgfqpoint{5.553476in}{8.095139in}}{\pgfqpoint{5.545663in}{8.087326in}}%
\pgfpathcurveto{\pgfqpoint{5.537849in}{8.079512in}}{\pgfqpoint{5.533459in}{8.068913in}}{\pgfqpoint{5.533459in}{8.057863in}}%
\pgfpathcurveto{\pgfqpoint{5.533459in}{8.046813in}}{\pgfqpoint{5.537849in}{8.036214in}}{\pgfqpoint{5.545663in}{8.028400in}}%
\pgfpathcurveto{\pgfqpoint{5.553476in}{8.020586in}}{\pgfqpoint{5.564075in}{8.016196in}}{\pgfqpoint{5.575125in}{8.016196in}}%
\pgfpathlineto{\pgfqpoint{5.575125in}{8.016196in}}%
\pgfpathclose%
\pgfusepath{stroke,fill}%
\end{pgfscope}%
\begin{pgfscope}%
\pgfpathrectangle{\pgfqpoint{5.292946in}{7.624184in}}{\pgfqpoint{2.177280in}{2.201755in}}%
\pgfusepath{clip}%
\pgfsetbuttcap%
\pgfsetroundjoin%
\definecolor{currentfill}{rgb}{0.121569,0.466667,0.705882}%
\pgfsetfillcolor{currentfill}%
\pgfsetlinewidth{0.481800pt}%
\definecolor{currentstroke}{rgb}{1.000000,1.000000,1.000000}%
\pgfsetstrokecolor{currentstroke}%
\pgfsetdash{}{0pt}%
\pgfpathmoveto{\pgfqpoint{5.546420in}{7.738197in}}%
\pgfpathcurveto{\pgfqpoint{5.557470in}{7.738197in}}{\pgfqpoint{5.568069in}{7.742587in}}{\pgfqpoint{5.575883in}{7.750401in}}%
\pgfpathcurveto{\pgfqpoint{5.583697in}{7.758214in}}{\pgfqpoint{5.588087in}{7.768813in}}{\pgfqpoint{5.588087in}{7.779863in}}%
\pgfpathcurveto{\pgfqpoint{5.588087in}{7.790914in}}{\pgfqpoint{5.583697in}{7.801513in}}{\pgfqpoint{5.575883in}{7.809326in}}%
\pgfpathcurveto{\pgfqpoint{5.568069in}{7.817140in}}{\pgfqpoint{5.557470in}{7.821530in}}{\pgfqpoint{5.546420in}{7.821530in}}%
\pgfpathcurveto{\pgfqpoint{5.535370in}{7.821530in}}{\pgfqpoint{5.524771in}{7.817140in}}{\pgfqpoint{5.516957in}{7.809326in}}%
\pgfpathcurveto{\pgfqpoint{5.509144in}{7.801513in}}{\pgfqpoint{5.504753in}{7.790914in}}{\pgfqpoint{5.504753in}{7.779863in}}%
\pgfpathcurveto{\pgfqpoint{5.504753in}{7.768813in}}{\pgfqpoint{5.509144in}{7.758214in}}{\pgfqpoint{5.516957in}{7.750401in}}%
\pgfpathcurveto{\pgfqpoint{5.524771in}{7.742587in}}{\pgfqpoint{5.535370in}{7.738197in}}{\pgfqpoint{5.546420in}{7.738197in}}%
\pgfpathlineto{\pgfqpoint{5.546420in}{7.738197in}}%
\pgfpathclose%
\pgfusepath{stroke,fill}%
\end{pgfscope}%
\begin{pgfscope}%
\pgfpathrectangle{\pgfqpoint{5.292946in}{7.624184in}}{\pgfqpoint{2.177280in}{2.201755in}}%
\pgfusepath{clip}%
\pgfsetbuttcap%
\pgfsetroundjoin%
\definecolor{currentfill}{rgb}{0.121569,0.466667,0.705882}%
\pgfsetfillcolor{currentfill}%
\pgfsetlinewidth{0.481800pt}%
\definecolor{currentstroke}{rgb}{1.000000,1.000000,1.000000}%
\pgfsetstrokecolor{currentstroke}%
\pgfsetdash{}{0pt}%
\pgfpathmoveto{\pgfqpoint{5.603831in}{8.127396in}}%
\pgfpathcurveto{\pgfqpoint{5.614881in}{8.127396in}}{\pgfqpoint{5.625480in}{8.131786in}}{\pgfqpoint{5.633294in}{8.139600in}}%
\pgfpathcurveto{\pgfqpoint{5.641107in}{8.147413in}}{\pgfqpoint{5.645497in}{8.158012in}}{\pgfqpoint{5.645497in}{8.169063in}}%
\pgfpathcurveto{\pgfqpoint{5.645497in}{8.180113in}}{\pgfqpoint{5.641107in}{8.190712in}}{\pgfqpoint{5.633294in}{8.198525in}}%
\pgfpathcurveto{\pgfqpoint{5.625480in}{8.206339in}}{\pgfqpoint{5.614881in}{8.210729in}}{\pgfqpoint{5.603831in}{8.210729in}}%
\pgfpathcurveto{\pgfqpoint{5.592781in}{8.210729in}}{\pgfqpoint{5.582182in}{8.206339in}}{\pgfqpoint{5.574368in}{8.198525in}}%
\pgfpathcurveto{\pgfqpoint{5.566554in}{8.190712in}}{\pgfqpoint{5.562164in}{8.180113in}}{\pgfqpoint{5.562164in}{8.169063in}}%
\pgfpathcurveto{\pgfqpoint{5.562164in}{8.158012in}}{\pgfqpoint{5.566554in}{8.147413in}}{\pgfqpoint{5.574368in}{8.139600in}}%
\pgfpathcurveto{\pgfqpoint{5.582182in}{8.131786in}}{\pgfqpoint{5.592781in}{8.127396in}}{\pgfqpoint{5.603831in}{8.127396in}}%
\pgfpathlineto{\pgfqpoint{5.603831in}{8.127396in}}%
\pgfpathclose%
\pgfusepath{stroke,fill}%
\end{pgfscope}%
\begin{pgfscope}%
\pgfpathrectangle{\pgfqpoint{5.292946in}{7.624184in}}{\pgfqpoint{2.177280in}{2.201755in}}%
\pgfusepath{clip}%
\pgfsetbuttcap%
\pgfsetroundjoin%
\definecolor{currentfill}{rgb}{0.121569,0.466667,0.705882}%
\pgfsetfillcolor{currentfill}%
\pgfsetlinewidth{0.481800pt}%
\definecolor{currentstroke}{rgb}{1.000000,1.000000,1.000000}%
\pgfsetstrokecolor{currentstroke}%
\pgfsetdash{}{0pt}%
\pgfpathmoveto{\pgfqpoint{5.546420in}{8.071796in}}%
\pgfpathcurveto{\pgfqpoint{5.557470in}{8.071796in}}{\pgfqpoint{5.568069in}{8.076186in}}{\pgfqpoint{5.575883in}{8.084000in}}%
\pgfpathcurveto{\pgfqpoint{5.583697in}{8.091813in}}{\pgfqpoint{5.588087in}{8.102413in}}{\pgfqpoint{5.588087in}{8.113463in}}%
\pgfpathcurveto{\pgfqpoint{5.588087in}{8.124513in}}{\pgfqpoint{5.583697in}{8.135112in}}{\pgfqpoint{5.575883in}{8.142925in}}%
\pgfpathcurveto{\pgfqpoint{5.568069in}{8.150739in}}{\pgfqpoint{5.557470in}{8.155129in}}{\pgfqpoint{5.546420in}{8.155129in}}%
\pgfpathcurveto{\pgfqpoint{5.535370in}{8.155129in}}{\pgfqpoint{5.524771in}{8.150739in}}{\pgfqpoint{5.516957in}{8.142925in}}%
\pgfpathcurveto{\pgfqpoint{5.509144in}{8.135112in}}{\pgfqpoint{5.504753in}{8.124513in}}{\pgfqpoint{5.504753in}{8.113463in}}%
\pgfpathcurveto{\pgfqpoint{5.504753in}{8.102413in}}{\pgfqpoint{5.509144in}{8.091813in}}{\pgfqpoint{5.516957in}{8.084000in}}%
\pgfpathcurveto{\pgfqpoint{5.524771in}{8.076186in}}{\pgfqpoint{5.535370in}{8.071796in}}{\pgfqpoint{5.546420in}{8.071796in}}%
\pgfpathlineto{\pgfqpoint{5.546420in}{8.071796in}}%
\pgfpathclose%
\pgfusepath{stroke,fill}%
\end{pgfscope}%
\begin{pgfscope}%
\pgfpathrectangle{\pgfqpoint{5.292946in}{7.624184in}}{\pgfqpoint{2.177280in}{2.201755in}}%
\pgfusepath{clip}%
\pgfsetbuttcap%
\pgfsetroundjoin%
\definecolor{currentfill}{rgb}{0.121569,0.466667,0.705882}%
\pgfsetfillcolor{currentfill}%
\pgfsetlinewidth{0.481800pt}%
\definecolor{currentstroke}{rgb}{1.000000,1.000000,1.000000}%
\pgfsetstrokecolor{currentstroke}%
\pgfsetdash{}{0pt}%
\pgfpathmoveto{\pgfqpoint{5.546420in}{7.793797in}}%
\pgfpathcurveto{\pgfqpoint{5.557470in}{7.793797in}}{\pgfqpoint{5.568069in}{7.798187in}}{\pgfqpoint{5.575883in}{7.806000in}}%
\pgfpathcurveto{\pgfqpoint{5.583697in}{7.813814in}}{\pgfqpoint{5.588087in}{7.824413in}}{\pgfqpoint{5.588087in}{7.835463in}}%
\pgfpathcurveto{\pgfqpoint{5.588087in}{7.846513in}}{\pgfqpoint{5.583697in}{7.857112in}}{\pgfqpoint{5.575883in}{7.864926in}}%
\pgfpathcurveto{\pgfqpoint{5.568069in}{7.872740in}}{\pgfqpoint{5.557470in}{7.877130in}}{\pgfqpoint{5.546420in}{7.877130in}}%
\pgfpathcurveto{\pgfqpoint{5.535370in}{7.877130in}}{\pgfqpoint{5.524771in}{7.872740in}}{\pgfqpoint{5.516957in}{7.864926in}}%
\pgfpathcurveto{\pgfqpoint{5.509144in}{7.857112in}}{\pgfqpoint{5.504753in}{7.846513in}}{\pgfqpoint{5.504753in}{7.835463in}}%
\pgfpathcurveto{\pgfqpoint{5.504753in}{7.824413in}}{\pgfqpoint{5.509144in}{7.813814in}}{\pgfqpoint{5.516957in}{7.806000in}}%
\pgfpathcurveto{\pgfqpoint{5.524771in}{7.798187in}}{\pgfqpoint{5.535370in}{7.793797in}}{\pgfqpoint{5.546420in}{7.793797in}}%
\pgfpathlineto{\pgfqpoint{5.546420in}{7.793797in}}%
\pgfpathclose%
\pgfusepath{stroke,fill}%
\end{pgfscope}%
\begin{pgfscope}%
\pgfpathrectangle{\pgfqpoint{5.292946in}{7.624184in}}{\pgfqpoint{2.177280in}{2.201755in}}%
\pgfusepath{clip}%
\pgfsetbuttcap%
\pgfsetroundjoin%
\definecolor{currentfill}{rgb}{0.121569,0.466667,0.705882}%
\pgfsetfillcolor{currentfill}%
\pgfsetlinewidth{0.481800pt}%
\definecolor{currentstroke}{rgb}{1.000000,1.000000,1.000000}%
\pgfsetstrokecolor{currentstroke}%
\pgfsetdash{}{0pt}%
\pgfpathmoveto{\pgfqpoint{5.546420in}{7.738197in}}%
\pgfpathcurveto{\pgfqpoint{5.557470in}{7.738197in}}{\pgfqpoint{5.568069in}{7.742587in}}{\pgfqpoint{5.575883in}{7.750401in}}%
\pgfpathcurveto{\pgfqpoint{5.583697in}{7.758214in}}{\pgfqpoint{5.588087in}{7.768813in}}{\pgfqpoint{5.588087in}{7.779863in}}%
\pgfpathcurveto{\pgfqpoint{5.588087in}{7.790914in}}{\pgfqpoint{5.583697in}{7.801513in}}{\pgfqpoint{5.575883in}{7.809326in}}%
\pgfpathcurveto{\pgfqpoint{5.568069in}{7.817140in}}{\pgfqpoint{5.557470in}{7.821530in}}{\pgfqpoint{5.546420in}{7.821530in}}%
\pgfpathcurveto{\pgfqpoint{5.535370in}{7.821530in}}{\pgfqpoint{5.524771in}{7.817140in}}{\pgfqpoint{5.516957in}{7.809326in}}%
\pgfpathcurveto{\pgfqpoint{5.509144in}{7.801513in}}{\pgfqpoint{5.504753in}{7.790914in}}{\pgfqpoint{5.504753in}{7.779863in}}%
\pgfpathcurveto{\pgfqpoint{5.504753in}{7.768813in}}{\pgfqpoint{5.509144in}{7.758214in}}{\pgfqpoint{5.516957in}{7.750401in}}%
\pgfpathcurveto{\pgfqpoint{5.524771in}{7.742587in}}{\pgfqpoint{5.535370in}{7.738197in}}{\pgfqpoint{5.546420in}{7.738197in}}%
\pgfpathlineto{\pgfqpoint{5.546420in}{7.738197in}}%
\pgfpathclose%
\pgfusepath{stroke,fill}%
\end{pgfscope}%
\begin{pgfscope}%
\pgfpathrectangle{\pgfqpoint{5.292946in}{7.624184in}}{\pgfqpoint{2.177280in}{2.201755in}}%
\pgfusepath{clip}%
\pgfsetbuttcap%
\pgfsetroundjoin%
\definecolor{currentfill}{rgb}{0.121569,0.466667,0.705882}%
\pgfsetfillcolor{currentfill}%
\pgfsetlinewidth{0.481800pt}%
\definecolor{currentstroke}{rgb}{1.000000,1.000000,1.000000}%
\pgfsetstrokecolor{currentstroke}%
\pgfsetdash{}{0pt}%
\pgfpathmoveto{\pgfqpoint{5.632536in}{8.071796in}}%
\pgfpathcurveto{\pgfqpoint{5.643586in}{8.071796in}}{\pgfqpoint{5.654185in}{8.076186in}}{\pgfqpoint{5.661999in}{8.084000in}}%
\pgfpathcurveto{\pgfqpoint{5.669812in}{8.091813in}}{\pgfqpoint{5.674203in}{8.102413in}}{\pgfqpoint{5.674203in}{8.113463in}}%
\pgfpathcurveto{\pgfqpoint{5.674203in}{8.124513in}}{\pgfqpoint{5.669812in}{8.135112in}}{\pgfqpoint{5.661999in}{8.142925in}}%
\pgfpathcurveto{\pgfqpoint{5.654185in}{8.150739in}}{\pgfqpoint{5.643586in}{8.155129in}}{\pgfqpoint{5.632536in}{8.155129in}}%
\pgfpathcurveto{\pgfqpoint{5.621486in}{8.155129in}}{\pgfqpoint{5.610887in}{8.150739in}}{\pgfqpoint{5.603073in}{8.142925in}}%
\pgfpathcurveto{\pgfqpoint{5.595260in}{8.135112in}}{\pgfqpoint{5.590869in}{8.124513in}}{\pgfqpoint{5.590869in}{8.113463in}}%
\pgfpathcurveto{\pgfqpoint{5.590869in}{8.102413in}}{\pgfqpoint{5.595260in}{8.091813in}}{\pgfqpoint{5.603073in}{8.084000in}}%
\pgfpathcurveto{\pgfqpoint{5.610887in}{8.076186in}}{\pgfqpoint{5.621486in}{8.071796in}}{\pgfqpoint{5.632536in}{8.071796in}}%
\pgfpathlineto{\pgfqpoint{5.632536in}{8.071796in}}%
\pgfpathclose%
\pgfusepath{stroke,fill}%
\end{pgfscope}%
\begin{pgfscope}%
\pgfpathrectangle{\pgfqpoint{5.292946in}{7.624184in}}{\pgfqpoint{2.177280in}{2.201755in}}%
\pgfusepath{clip}%
\pgfsetbuttcap%
\pgfsetroundjoin%
\definecolor{currentfill}{rgb}{0.121569,0.466667,0.705882}%
\pgfsetfillcolor{currentfill}%
\pgfsetlinewidth{0.481800pt}%
\definecolor{currentstroke}{rgb}{1.000000,1.000000,1.000000}%
\pgfsetstrokecolor{currentstroke}%
\pgfsetdash{}{0pt}%
\pgfpathmoveto{\pgfqpoint{5.718652in}{8.127396in}}%
\pgfpathcurveto{\pgfqpoint{5.729702in}{8.127396in}}{\pgfqpoint{5.740301in}{8.131786in}}{\pgfqpoint{5.748115in}{8.139600in}}%
\pgfpathcurveto{\pgfqpoint{5.755928in}{8.147413in}}{\pgfqpoint{5.760319in}{8.158012in}}{\pgfqpoint{5.760319in}{8.169063in}}%
\pgfpathcurveto{\pgfqpoint{5.760319in}{8.180113in}}{\pgfqpoint{5.755928in}{8.190712in}}{\pgfqpoint{5.748115in}{8.198525in}}%
\pgfpathcurveto{\pgfqpoint{5.740301in}{8.206339in}}{\pgfqpoint{5.729702in}{8.210729in}}{\pgfqpoint{5.718652in}{8.210729in}}%
\pgfpathcurveto{\pgfqpoint{5.707602in}{8.210729in}}{\pgfqpoint{5.697003in}{8.206339in}}{\pgfqpoint{5.689189in}{8.198525in}}%
\pgfpathcurveto{\pgfqpoint{5.681375in}{8.190712in}}{\pgfqpoint{5.676985in}{8.180113in}}{\pgfqpoint{5.676985in}{8.169063in}}%
\pgfpathcurveto{\pgfqpoint{5.676985in}{8.158012in}}{\pgfqpoint{5.681375in}{8.147413in}}{\pgfqpoint{5.689189in}{8.139600in}}%
\pgfpathcurveto{\pgfqpoint{5.697003in}{8.131786in}}{\pgfqpoint{5.707602in}{8.127396in}}{\pgfqpoint{5.718652in}{8.127396in}}%
\pgfpathlineto{\pgfqpoint{5.718652in}{8.127396in}}%
\pgfpathclose%
\pgfusepath{stroke,fill}%
\end{pgfscope}%
\begin{pgfscope}%
\pgfpathrectangle{\pgfqpoint{5.292946in}{7.624184in}}{\pgfqpoint{2.177280in}{2.201755in}}%
\pgfusepath{clip}%
\pgfsetbuttcap%
\pgfsetroundjoin%
\definecolor{currentfill}{rgb}{0.121569,0.466667,0.705882}%
\pgfsetfillcolor{currentfill}%
\pgfsetlinewidth{0.481800pt}%
\definecolor{currentstroke}{rgb}{1.000000,1.000000,1.000000}%
\pgfsetstrokecolor{currentstroke}%
\pgfsetdash{}{0pt}%
\pgfpathmoveto{\pgfqpoint{5.575125in}{7.960596in}}%
\pgfpathcurveto{\pgfqpoint{5.586176in}{7.960596in}}{\pgfqpoint{5.596775in}{7.964986in}}{\pgfqpoint{5.604588in}{7.972800in}}%
\pgfpathcurveto{\pgfqpoint{5.612402in}{7.980614in}}{\pgfqpoint{5.616792in}{7.991213in}}{\pgfqpoint{5.616792in}{8.002263in}}%
\pgfpathcurveto{\pgfqpoint{5.616792in}{8.013313in}}{\pgfqpoint{5.612402in}{8.023912in}}{\pgfqpoint{5.604588in}{8.031726in}}%
\pgfpathcurveto{\pgfqpoint{5.596775in}{8.039539in}}{\pgfqpoint{5.586176in}{8.043930in}}{\pgfqpoint{5.575125in}{8.043930in}}%
\pgfpathcurveto{\pgfqpoint{5.564075in}{8.043930in}}{\pgfqpoint{5.553476in}{8.039539in}}{\pgfqpoint{5.545663in}{8.031726in}}%
\pgfpathcurveto{\pgfqpoint{5.537849in}{8.023912in}}{\pgfqpoint{5.533459in}{8.013313in}}{\pgfqpoint{5.533459in}{8.002263in}}%
\pgfpathcurveto{\pgfqpoint{5.533459in}{7.991213in}}{\pgfqpoint{5.537849in}{7.980614in}}{\pgfqpoint{5.545663in}{7.972800in}}%
\pgfpathcurveto{\pgfqpoint{5.553476in}{7.964986in}}{\pgfqpoint{5.564075in}{7.960596in}}{\pgfqpoint{5.575125in}{7.960596in}}%
\pgfpathlineto{\pgfqpoint{5.575125in}{7.960596in}}%
\pgfpathclose%
\pgfusepath{stroke,fill}%
\end{pgfscope}%
\begin{pgfscope}%
\pgfpathrectangle{\pgfqpoint{5.292946in}{7.624184in}}{\pgfqpoint{2.177280in}{2.201755in}}%
\pgfusepath{clip}%
\pgfsetbuttcap%
\pgfsetroundjoin%
\definecolor{currentfill}{rgb}{0.121569,0.466667,0.705882}%
\pgfsetfillcolor{currentfill}%
\pgfsetlinewidth{0.481800pt}%
\definecolor{currentstroke}{rgb}{1.000000,1.000000,1.000000}%
\pgfsetstrokecolor{currentstroke}%
\pgfsetdash{}{0pt}%
\pgfpathmoveto{\pgfqpoint{5.632536in}{8.127396in}}%
\pgfpathcurveto{\pgfqpoint{5.643586in}{8.127396in}}{\pgfqpoint{5.654185in}{8.131786in}}{\pgfqpoint{5.661999in}{8.139600in}}%
\pgfpathcurveto{\pgfqpoint{5.669812in}{8.147413in}}{\pgfqpoint{5.674203in}{8.158012in}}{\pgfqpoint{5.674203in}{8.169063in}}%
\pgfpathcurveto{\pgfqpoint{5.674203in}{8.180113in}}{\pgfqpoint{5.669812in}{8.190712in}}{\pgfqpoint{5.661999in}{8.198525in}}%
\pgfpathcurveto{\pgfqpoint{5.654185in}{8.206339in}}{\pgfqpoint{5.643586in}{8.210729in}}{\pgfqpoint{5.632536in}{8.210729in}}%
\pgfpathcurveto{\pgfqpoint{5.621486in}{8.210729in}}{\pgfqpoint{5.610887in}{8.206339in}}{\pgfqpoint{5.603073in}{8.198525in}}%
\pgfpathcurveto{\pgfqpoint{5.595260in}{8.190712in}}{\pgfqpoint{5.590869in}{8.180113in}}{\pgfqpoint{5.590869in}{8.169063in}}%
\pgfpathcurveto{\pgfqpoint{5.590869in}{8.158012in}}{\pgfqpoint{5.595260in}{8.147413in}}{\pgfqpoint{5.603073in}{8.139600in}}%
\pgfpathcurveto{\pgfqpoint{5.610887in}{8.131786in}}{\pgfqpoint{5.621486in}{8.127396in}}{\pgfqpoint{5.632536in}{8.127396in}}%
\pgfpathlineto{\pgfqpoint{5.632536in}{8.127396in}}%
\pgfpathclose%
\pgfusepath{stroke,fill}%
\end{pgfscope}%
\begin{pgfscope}%
\pgfpathrectangle{\pgfqpoint{5.292946in}{7.624184in}}{\pgfqpoint{2.177280in}{2.201755in}}%
\pgfusepath{clip}%
\pgfsetbuttcap%
\pgfsetroundjoin%
\definecolor{currentfill}{rgb}{0.121569,0.466667,0.705882}%
\pgfsetfillcolor{currentfill}%
\pgfsetlinewidth{0.481800pt}%
\definecolor{currentstroke}{rgb}{1.000000,1.000000,1.000000}%
\pgfsetstrokecolor{currentstroke}%
\pgfsetdash{}{0pt}%
\pgfpathmoveto{\pgfqpoint{5.575125in}{7.849396in}}%
\pgfpathcurveto{\pgfqpoint{5.586176in}{7.849396in}}{\pgfqpoint{5.596775in}{7.853787in}}{\pgfqpoint{5.604588in}{7.861600in}}%
\pgfpathcurveto{\pgfqpoint{5.612402in}{7.869414in}}{\pgfqpoint{5.616792in}{7.880013in}}{\pgfqpoint{5.616792in}{7.891063in}}%
\pgfpathcurveto{\pgfqpoint{5.616792in}{7.902113in}}{\pgfqpoint{5.612402in}{7.912712in}}{\pgfqpoint{5.604588in}{7.920526in}}%
\pgfpathcurveto{\pgfqpoint{5.596775in}{7.928340in}}{\pgfqpoint{5.586176in}{7.932730in}}{\pgfqpoint{5.575125in}{7.932730in}}%
\pgfpathcurveto{\pgfqpoint{5.564075in}{7.932730in}}{\pgfqpoint{5.553476in}{7.928340in}}{\pgfqpoint{5.545663in}{7.920526in}}%
\pgfpathcurveto{\pgfqpoint{5.537849in}{7.912712in}}{\pgfqpoint{5.533459in}{7.902113in}}{\pgfqpoint{5.533459in}{7.891063in}}%
\pgfpathcurveto{\pgfqpoint{5.533459in}{7.880013in}}{\pgfqpoint{5.537849in}{7.869414in}}{\pgfqpoint{5.545663in}{7.861600in}}%
\pgfpathcurveto{\pgfqpoint{5.553476in}{7.853787in}}{\pgfqpoint{5.564075in}{7.849396in}}{\pgfqpoint{5.575125in}{7.849396in}}%
\pgfpathlineto{\pgfqpoint{5.575125in}{7.849396in}}%
\pgfpathclose%
\pgfusepath{stroke,fill}%
\end{pgfscope}%
\begin{pgfscope}%
\pgfpathrectangle{\pgfqpoint{5.292946in}{7.624184in}}{\pgfqpoint{2.177280in}{2.201755in}}%
\pgfusepath{clip}%
\pgfsetbuttcap%
\pgfsetroundjoin%
\definecolor{currentfill}{rgb}{0.121569,0.466667,0.705882}%
\pgfsetfillcolor{currentfill}%
\pgfsetlinewidth{0.481800pt}%
\definecolor{currentstroke}{rgb}{1.000000,1.000000,1.000000}%
\pgfsetstrokecolor{currentstroke}%
\pgfsetdash{}{0pt}%
\pgfpathmoveto{\pgfqpoint{5.603831in}{8.238596in}}%
\pgfpathcurveto{\pgfqpoint{5.614881in}{8.238596in}}{\pgfqpoint{5.625480in}{8.242986in}}{\pgfqpoint{5.633294in}{8.250799in}}%
\pgfpathcurveto{\pgfqpoint{5.641107in}{8.258613in}}{\pgfqpoint{5.645497in}{8.269212in}}{\pgfqpoint{5.645497in}{8.280262in}}%
\pgfpathcurveto{\pgfqpoint{5.645497in}{8.291312in}}{\pgfqpoint{5.641107in}{8.301911in}}{\pgfqpoint{5.633294in}{8.309725in}}%
\pgfpathcurveto{\pgfqpoint{5.625480in}{8.317539in}}{\pgfqpoint{5.614881in}{8.321929in}}{\pgfqpoint{5.603831in}{8.321929in}}%
\pgfpathcurveto{\pgfqpoint{5.592781in}{8.321929in}}{\pgfqpoint{5.582182in}{8.317539in}}{\pgfqpoint{5.574368in}{8.309725in}}%
\pgfpathcurveto{\pgfqpoint{5.566554in}{8.301911in}}{\pgfqpoint{5.562164in}{8.291312in}}{\pgfqpoint{5.562164in}{8.280262in}}%
\pgfpathcurveto{\pgfqpoint{5.562164in}{8.269212in}}{\pgfqpoint{5.566554in}{8.258613in}}{\pgfqpoint{5.574368in}{8.250799in}}%
\pgfpathcurveto{\pgfqpoint{5.582182in}{8.242986in}}{\pgfqpoint{5.592781in}{8.238596in}}{\pgfqpoint{5.603831in}{8.238596in}}%
\pgfpathlineto{\pgfqpoint{5.603831in}{8.238596in}}%
\pgfpathclose%
\pgfusepath{stroke,fill}%
\end{pgfscope}%
\begin{pgfscope}%
\pgfpathrectangle{\pgfqpoint{5.292946in}{7.624184in}}{\pgfqpoint{2.177280in}{2.201755in}}%
\pgfusepath{clip}%
\pgfsetbuttcap%
\pgfsetroundjoin%
\definecolor{currentfill}{rgb}{0.121569,0.466667,0.705882}%
\pgfsetfillcolor{currentfill}%
\pgfsetlinewidth{0.481800pt}%
\definecolor{currentstroke}{rgb}{1.000000,1.000000,1.000000}%
\pgfsetstrokecolor{currentstroke}%
\pgfsetdash{}{0pt}%
\pgfpathmoveto{\pgfqpoint{5.575125in}{8.071796in}}%
\pgfpathcurveto{\pgfqpoint{5.586176in}{8.071796in}}{\pgfqpoint{5.596775in}{8.076186in}}{\pgfqpoint{5.604588in}{8.084000in}}%
\pgfpathcurveto{\pgfqpoint{5.612402in}{8.091813in}}{\pgfqpoint{5.616792in}{8.102413in}}{\pgfqpoint{5.616792in}{8.113463in}}%
\pgfpathcurveto{\pgfqpoint{5.616792in}{8.124513in}}{\pgfqpoint{5.612402in}{8.135112in}}{\pgfqpoint{5.604588in}{8.142925in}}%
\pgfpathcurveto{\pgfqpoint{5.596775in}{8.150739in}}{\pgfqpoint{5.586176in}{8.155129in}}{\pgfqpoint{5.575125in}{8.155129in}}%
\pgfpathcurveto{\pgfqpoint{5.564075in}{8.155129in}}{\pgfqpoint{5.553476in}{8.150739in}}{\pgfqpoint{5.545663in}{8.142925in}}%
\pgfpathcurveto{\pgfqpoint{5.537849in}{8.135112in}}{\pgfqpoint{5.533459in}{8.124513in}}{\pgfqpoint{5.533459in}{8.113463in}}%
\pgfpathcurveto{\pgfqpoint{5.533459in}{8.102413in}}{\pgfqpoint{5.537849in}{8.091813in}}{\pgfqpoint{5.545663in}{8.084000in}}%
\pgfpathcurveto{\pgfqpoint{5.553476in}{8.076186in}}{\pgfqpoint{5.564075in}{8.071796in}}{\pgfqpoint{5.575125in}{8.071796in}}%
\pgfpathlineto{\pgfqpoint{5.575125in}{8.071796in}}%
\pgfpathclose%
\pgfusepath{stroke,fill}%
\end{pgfscope}%
\begin{pgfscope}%
\pgfpathrectangle{\pgfqpoint{5.292946in}{7.624184in}}{\pgfqpoint{2.177280in}{2.201755in}}%
\pgfusepath{clip}%
\pgfsetbuttcap%
\pgfsetroundjoin%
\definecolor{currentfill}{rgb}{1.000000,0.498039,0.054902}%
\pgfsetfillcolor{currentfill}%
\pgfsetlinewidth{0.481800pt}%
\definecolor{currentstroke}{rgb}{1.000000,1.000000,1.000000}%
\pgfsetstrokecolor{currentstroke}%
\pgfsetdash{}{0pt}%
\pgfpathmoveto{\pgfqpoint{6.522400in}{9.183793in}}%
\pgfpathcurveto{\pgfqpoint{6.533450in}{9.183793in}}{\pgfqpoint{6.544049in}{9.188184in}}{\pgfqpoint{6.551863in}{9.195997in}}%
\pgfpathcurveto{\pgfqpoint{6.559676in}{9.203811in}}{\pgfqpoint{6.564067in}{9.214410in}}{\pgfqpoint{6.564067in}{9.225460in}}%
\pgfpathcurveto{\pgfqpoint{6.564067in}{9.236510in}}{\pgfqpoint{6.559676in}{9.247109in}}{\pgfqpoint{6.551863in}{9.254923in}}%
\pgfpathcurveto{\pgfqpoint{6.544049in}{9.262737in}}{\pgfqpoint{6.533450in}{9.267127in}}{\pgfqpoint{6.522400in}{9.267127in}}%
\pgfpathcurveto{\pgfqpoint{6.511350in}{9.267127in}}{\pgfqpoint{6.500751in}{9.262737in}}{\pgfqpoint{6.492937in}{9.254923in}}%
\pgfpathcurveto{\pgfqpoint{6.485123in}{9.247109in}}{\pgfqpoint{6.480733in}{9.236510in}}{\pgfqpoint{6.480733in}{9.225460in}}%
\pgfpathcurveto{\pgfqpoint{6.480733in}{9.214410in}}{\pgfqpoint{6.485123in}{9.203811in}}{\pgfqpoint{6.492937in}{9.195997in}}%
\pgfpathcurveto{\pgfqpoint{6.500751in}{9.188184in}}{\pgfqpoint{6.511350in}{9.183793in}}{\pgfqpoint{6.522400in}{9.183793in}}%
\pgfpathlineto{\pgfqpoint{6.522400in}{9.183793in}}%
\pgfpathclose%
\pgfusepath{stroke,fill}%
\end{pgfscope}%
\begin{pgfscope}%
\pgfpathrectangle{\pgfqpoint{5.292946in}{7.624184in}}{\pgfqpoint{2.177280in}{2.201755in}}%
\pgfusepath{clip}%
\pgfsetbuttcap%
\pgfsetroundjoin%
\definecolor{currentfill}{rgb}{1.000000,0.498039,0.054902}%
\pgfsetfillcolor{currentfill}%
\pgfsetlinewidth{0.481800pt}%
\definecolor{currentstroke}{rgb}{1.000000,1.000000,1.000000}%
\pgfsetstrokecolor{currentstroke}%
\pgfsetdash{}{0pt}%
\pgfpathmoveto{\pgfqpoint{6.464989in}{8.850194in}}%
\pgfpathcurveto{\pgfqpoint{6.476039in}{8.850194in}}{\pgfqpoint{6.486638in}{8.854584in}}{\pgfqpoint{6.494452in}{8.862398in}}%
\pgfpathcurveto{\pgfqpoint{6.502266in}{8.870212in}}{\pgfqpoint{6.506656in}{8.880811in}}{\pgfqpoint{6.506656in}{8.891861in}}%
\pgfpathcurveto{\pgfqpoint{6.506656in}{8.902911in}}{\pgfqpoint{6.502266in}{8.913510in}}{\pgfqpoint{6.494452in}{8.921324in}}%
\pgfpathcurveto{\pgfqpoint{6.486638in}{8.929137in}}{\pgfqpoint{6.476039in}{8.933528in}}{\pgfqpoint{6.464989in}{8.933528in}}%
\pgfpathcurveto{\pgfqpoint{6.453939in}{8.933528in}}{\pgfqpoint{6.443340in}{8.929137in}}{\pgfqpoint{6.435527in}{8.921324in}}%
\pgfpathcurveto{\pgfqpoint{6.427713in}{8.913510in}}{\pgfqpoint{6.423323in}{8.902911in}}{\pgfqpoint{6.423323in}{8.891861in}}%
\pgfpathcurveto{\pgfqpoint{6.423323in}{8.880811in}}{\pgfqpoint{6.427713in}{8.870212in}}{\pgfqpoint{6.435527in}{8.862398in}}%
\pgfpathcurveto{\pgfqpoint{6.443340in}{8.854584in}}{\pgfqpoint{6.453939in}{8.850194in}}{\pgfqpoint{6.464989in}{8.850194in}}%
\pgfpathlineto{\pgfqpoint{6.464989in}{8.850194in}}%
\pgfpathclose%
\pgfusepath{stroke,fill}%
\end{pgfscope}%
\begin{pgfscope}%
\pgfpathrectangle{\pgfqpoint{5.292946in}{7.624184in}}{\pgfqpoint{2.177280in}{2.201755in}}%
\pgfusepath{clip}%
\pgfsetbuttcap%
\pgfsetroundjoin%
\definecolor{currentfill}{rgb}{1.000000,0.498039,0.054902}%
\pgfsetfillcolor{currentfill}%
\pgfsetlinewidth{0.481800pt}%
\definecolor{currentstroke}{rgb}{1.000000,1.000000,1.000000}%
\pgfsetstrokecolor{currentstroke}%
\pgfsetdash{}{0pt}%
\pgfpathmoveto{\pgfqpoint{6.579810in}{9.128194in}}%
\pgfpathcurveto{\pgfqpoint{6.590861in}{9.128194in}}{\pgfqpoint{6.601460in}{9.132584in}}{\pgfqpoint{6.609273in}{9.140397in}}%
\pgfpathcurveto{\pgfqpoint{6.617087in}{9.148211in}}{\pgfqpoint{6.621477in}{9.158810in}}{\pgfqpoint{6.621477in}{9.169860in}}%
\pgfpathcurveto{\pgfqpoint{6.621477in}{9.180910in}}{\pgfqpoint{6.617087in}{9.191509in}}{\pgfqpoint{6.609273in}{9.199323in}}%
\pgfpathcurveto{\pgfqpoint{6.601460in}{9.207137in}}{\pgfqpoint{6.590861in}{9.211527in}}{\pgfqpoint{6.579810in}{9.211527in}}%
\pgfpathcurveto{\pgfqpoint{6.568760in}{9.211527in}}{\pgfqpoint{6.558161in}{9.207137in}}{\pgfqpoint{6.550348in}{9.199323in}}%
\pgfpathcurveto{\pgfqpoint{6.542534in}{9.191509in}}{\pgfqpoint{6.538144in}{9.180910in}}{\pgfqpoint{6.538144in}{9.169860in}}%
\pgfpathcurveto{\pgfqpoint{6.538144in}{9.158810in}}{\pgfqpoint{6.542534in}{9.148211in}}{\pgfqpoint{6.550348in}{9.140397in}}%
\pgfpathcurveto{\pgfqpoint{6.558161in}{9.132584in}}{\pgfqpoint{6.568760in}{9.128194in}}{\pgfqpoint{6.579810in}{9.128194in}}%
\pgfpathlineto{\pgfqpoint{6.579810in}{9.128194in}}%
\pgfpathclose%
\pgfusepath{stroke,fill}%
\end{pgfscope}%
\begin{pgfscope}%
\pgfpathrectangle{\pgfqpoint{5.292946in}{7.624184in}}{\pgfqpoint{2.177280in}{2.201755in}}%
\pgfusepath{clip}%
\pgfsetbuttcap%
\pgfsetroundjoin%
\definecolor{currentfill}{rgb}{1.000000,0.498039,0.054902}%
\pgfsetfillcolor{currentfill}%
\pgfsetlinewidth{0.481800pt}%
\definecolor{currentstroke}{rgb}{1.000000,1.000000,1.000000}%
\pgfsetstrokecolor{currentstroke}%
\pgfsetdash{}{0pt}%
\pgfpathmoveto{\pgfqpoint{6.321463in}{8.349795in}}%
\pgfpathcurveto{\pgfqpoint{6.332513in}{8.349795in}}{\pgfqpoint{6.343112in}{8.354186in}}{\pgfqpoint{6.350926in}{8.361999in}}%
\pgfpathcurveto{\pgfqpoint{6.358739in}{8.369813in}}{\pgfqpoint{6.363130in}{8.380412in}}{\pgfqpoint{6.363130in}{8.391462in}}%
\pgfpathcurveto{\pgfqpoint{6.363130in}{8.402512in}}{\pgfqpoint{6.358739in}{8.413111in}}{\pgfqpoint{6.350926in}{8.420925in}}%
\pgfpathcurveto{\pgfqpoint{6.343112in}{8.428738in}}{\pgfqpoint{6.332513in}{8.433129in}}{\pgfqpoint{6.321463in}{8.433129in}}%
\pgfpathcurveto{\pgfqpoint{6.310413in}{8.433129in}}{\pgfqpoint{6.299814in}{8.428738in}}{\pgfqpoint{6.292000in}{8.420925in}}%
\pgfpathcurveto{\pgfqpoint{6.284186in}{8.413111in}}{\pgfqpoint{6.279796in}{8.402512in}}{\pgfqpoint{6.279796in}{8.391462in}}%
\pgfpathcurveto{\pgfqpoint{6.279796in}{8.380412in}}{\pgfqpoint{6.284186in}{8.369813in}}{\pgfqpoint{6.292000in}{8.361999in}}%
\pgfpathcurveto{\pgfqpoint{6.299814in}{8.354186in}}{\pgfqpoint{6.310413in}{8.349795in}}{\pgfqpoint{6.321463in}{8.349795in}}%
\pgfpathlineto{\pgfqpoint{6.321463in}{8.349795in}}%
\pgfpathclose%
\pgfusepath{stroke,fill}%
\end{pgfscope}%
\begin{pgfscope}%
\pgfpathrectangle{\pgfqpoint{5.292946in}{7.624184in}}{\pgfqpoint{2.177280in}{2.201755in}}%
\pgfusepath{clip}%
\pgfsetbuttcap%
\pgfsetroundjoin%
\definecolor{currentfill}{rgb}{1.000000,0.498039,0.054902}%
\pgfsetfillcolor{currentfill}%
\pgfsetlinewidth{0.481800pt}%
\definecolor{currentstroke}{rgb}{1.000000,1.000000,1.000000}%
\pgfsetstrokecolor{currentstroke}%
\pgfsetdash{}{0pt}%
\pgfpathmoveto{\pgfqpoint{6.493695in}{8.905794in}}%
\pgfpathcurveto{\pgfqpoint{6.504745in}{8.905794in}}{\pgfqpoint{6.515344in}{8.910184in}}{\pgfqpoint{6.523157in}{8.917998in}}%
\pgfpathcurveto{\pgfqpoint{6.530971in}{8.925812in}}{\pgfqpoint{6.535361in}{8.936411in}}{\pgfqpoint{6.535361in}{8.947461in}}%
\pgfpathcurveto{\pgfqpoint{6.535361in}{8.958511in}}{\pgfqpoint{6.530971in}{8.969110in}}{\pgfqpoint{6.523157in}{8.976924in}}%
\pgfpathcurveto{\pgfqpoint{6.515344in}{8.984737in}}{\pgfqpoint{6.504745in}{8.989127in}}{\pgfqpoint{6.493695in}{8.989127in}}%
\pgfpathcurveto{\pgfqpoint{6.482644in}{8.989127in}}{\pgfqpoint{6.472045in}{8.984737in}}{\pgfqpoint{6.464232in}{8.976924in}}%
\pgfpathcurveto{\pgfqpoint{6.456418in}{8.969110in}}{\pgfqpoint{6.452028in}{8.958511in}}{\pgfqpoint{6.452028in}{8.947461in}}%
\pgfpathcurveto{\pgfqpoint{6.452028in}{8.936411in}}{\pgfqpoint{6.456418in}{8.925812in}}{\pgfqpoint{6.464232in}{8.917998in}}%
\pgfpathcurveto{\pgfqpoint{6.472045in}{8.910184in}}{\pgfqpoint{6.482644in}{8.905794in}}{\pgfqpoint{6.493695in}{8.905794in}}%
\pgfpathlineto{\pgfqpoint{6.493695in}{8.905794in}}%
\pgfpathclose%
\pgfusepath{stroke,fill}%
\end{pgfscope}%
\begin{pgfscope}%
\pgfpathrectangle{\pgfqpoint{5.292946in}{7.624184in}}{\pgfqpoint{2.177280in}{2.201755in}}%
\pgfusepath{clip}%
\pgfsetbuttcap%
\pgfsetroundjoin%
\definecolor{currentfill}{rgb}{1.000000,0.498039,0.054902}%
\pgfsetfillcolor{currentfill}%
\pgfsetlinewidth{0.481800pt}%
\definecolor{currentstroke}{rgb}{1.000000,1.000000,1.000000}%
\pgfsetstrokecolor{currentstroke}%
\pgfsetdash{}{0pt}%
\pgfpathmoveto{\pgfqpoint{6.464989in}{8.460995in}}%
\pgfpathcurveto{\pgfqpoint{6.476039in}{8.460995in}}{\pgfqpoint{6.486638in}{8.465385in}}{\pgfqpoint{6.494452in}{8.473199in}}%
\pgfpathcurveto{\pgfqpoint{6.502266in}{8.481013in}}{\pgfqpoint{6.506656in}{8.491612in}}{\pgfqpoint{6.506656in}{8.502662in}}%
\pgfpathcurveto{\pgfqpoint{6.506656in}{8.513712in}}{\pgfqpoint{6.502266in}{8.524311in}}{\pgfqpoint{6.494452in}{8.532125in}}%
\pgfpathcurveto{\pgfqpoint{6.486638in}{8.539938in}}{\pgfqpoint{6.476039in}{8.544328in}}{\pgfqpoint{6.464989in}{8.544328in}}%
\pgfpathcurveto{\pgfqpoint{6.453939in}{8.544328in}}{\pgfqpoint{6.443340in}{8.539938in}}{\pgfqpoint{6.435527in}{8.532125in}}%
\pgfpathcurveto{\pgfqpoint{6.427713in}{8.524311in}}{\pgfqpoint{6.423323in}{8.513712in}}{\pgfqpoint{6.423323in}{8.502662in}}%
\pgfpathcurveto{\pgfqpoint{6.423323in}{8.491612in}}{\pgfqpoint{6.427713in}{8.481013in}}{\pgfqpoint{6.435527in}{8.473199in}}%
\pgfpathcurveto{\pgfqpoint{6.443340in}{8.465385in}}{\pgfqpoint{6.453939in}{8.460995in}}{\pgfqpoint{6.464989in}{8.460995in}}%
\pgfpathlineto{\pgfqpoint{6.464989in}{8.460995in}}%
\pgfpathclose%
\pgfusepath{stroke,fill}%
\end{pgfscope}%
\begin{pgfscope}%
\pgfpathrectangle{\pgfqpoint{5.292946in}{7.624184in}}{\pgfqpoint{2.177280in}{2.201755in}}%
\pgfusepath{clip}%
\pgfsetbuttcap%
\pgfsetroundjoin%
\definecolor{currentfill}{rgb}{1.000000,0.498039,0.054902}%
\pgfsetfillcolor{currentfill}%
\pgfsetlinewidth{0.481800pt}%
\definecolor{currentstroke}{rgb}{1.000000,1.000000,1.000000}%
\pgfsetstrokecolor{currentstroke}%
\pgfsetdash{}{0pt}%
\pgfpathmoveto{\pgfqpoint{6.522400in}{8.794594in}}%
\pgfpathcurveto{\pgfqpoint{6.533450in}{8.794594in}}{\pgfqpoint{6.544049in}{8.798985in}}{\pgfqpoint{6.551863in}{8.806798in}}%
\pgfpathcurveto{\pgfqpoint{6.559676in}{8.814612in}}{\pgfqpoint{6.564067in}{8.825211in}}{\pgfqpoint{6.564067in}{8.836261in}}%
\pgfpathcurveto{\pgfqpoint{6.564067in}{8.847311in}}{\pgfqpoint{6.559676in}{8.857910in}}{\pgfqpoint{6.551863in}{8.865724in}}%
\pgfpathcurveto{\pgfqpoint{6.544049in}{8.873537in}}{\pgfqpoint{6.533450in}{8.877928in}}{\pgfqpoint{6.522400in}{8.877928in}}%
\pgfpathcurveto{\pgfqpoint{6.511350in}{8.877928in}}{\pgfqpoint{6.500751in}{8.873537in}}{\pgfqpoint{6.492937in}{8.865724in}}%
\pgfpathcurveto{\pgfqpoint{6.485123in}{8.857910in}}{\pgfqpoint{6.480733in}{8.847311in}}{\pgfqpoint{6.480733in}{8.836261in}}%
\pgfpathcurveto{\pgfqpoint{6.480733in}{8.825211in}}{\pgfqpoint{6.485123in}{8.814612in}}{\pgfqpoint{6.492937in}{8.806798in}}%
\pgfpathcurveto{\pgfqpoint{6.500751in}{8.798985in}}{\pgfqpoint{6.511350in}{8.794594in}}{\pgfqpoint{6.522400in}{8.794594in}}%
\pgfpathlineto{\pgfqpoint{6.522400in}{8.794594in}}%
\pgfpathclose%
\pgfusepath{stroke,fill}%
\end{pgfscope}%
\begin{pgfscope}%
\pgfpathrectangle{\pgfqpoint{5.292946in}{7.624184in}}{\pgfqpoint{2.177280in}{2.201755in}}%
\pgfusepath{clip}%
\pgfsetbuttcap%
\pgfsetroundjoin%
\definecolor{currentfill}{rgb}{1.000000,0.498039,0.054902}%
\pgfsetfillcolor{currentfill}%
\pgfsetlinewidth{0.481800pt}%
\definecolor{currentstroke}{rgb}{1.000000,1.000000,1.000000}%
\pgfsetstrokecolor{currentstroke}%
\pgfsetdash{}{0pt}%
\pgfpathmoveto{\pgfqpoint{6.120526in}{8.016196in}}%
\pgfpathcurveto{\pgfqpoint{6.131576in}{8.016196in}}{\pgfqpoint{6.142175in}{8.020586in}}{\pgfqpoint{6.149989in}{8.028400in}}%
\pgfpathcurveto{\pgfqpoint{6.157802in}{8.036214in}}{\pgfqpoint{6.162193in}{8.046813in}}{\pgfqpoint{6.162193in}{8.057863in}}%
\pgfpathcurveto{\pgfqpoint{6.162193in}{8.068913in}}{\pgfqpoint{6.157802in}{8.079512in}}{\pgfqpoint{6.149989in}{8.087326in}}%
\pgfpathcurveto{\pgfqpoint{6.142175in}{8.095139in}}{\pgfqpoint{6.131576in}{8.099529in}}{\pgfqpoint{6.120526in}{8.099529in}}%
\pgfpathcurveto{\pgfqpoint{6.109476in}{8.099529in}}{\pgfqpoint{6.098877in}{8.095139in}}{\pgfqpoint{6.091063in}{8.087326in}}%
\pgfpathcurveto{\pgfqpoint{6.083249in}{8.079512in}}{\pgfqpoint{6.078859in}{8.068913in}}{\pgfqpoint{6.078859in}{8.057863in}}%
\pgfpathcurveto{\pgfqpoint{6.078859in}{8.046813in}}{\pgfqpoint{6.083249in}{8.036214in}}{\pgfqpoint{6.091063in}{8.028400in}}%
\pgfpathcurveto{\pgfqpoint{6.098877in}{8.020586in}}{\pgfqpoint{6.109476in}{8.016196in}}{\pgfqpoint{6.120526in}{8.016196in}}%
\pgfpathlineto{\pgfqpoint{6.120526in}{8.016196in}}%
\pgfpathclose%
\pgfusepath{stroke,fill}%
\end{pgfscope}%
\begin{pgfscope}%
\pgfpathrectangle{\pgfqpoint{5.292946in}{7.624184in}}{\pgfqpoint{2.177280in}{2.201755in}}%
\pgfusepath{clip}%
\pgfsetbuttcap%
\pgfsetroundjoin%
\definecolor{currentfill}{rgb}{1.000000,0.498039,0.054902}%
\pgfsetfillcolor{currentfill}%
\pgfsetlinewidth{0.481800pt}%
\definecolor{currentstroke}{rgb}{1.000000,1.000000,1.000000}%
\pgfsetstrokecolor{currentstroke}%
\pgfsetdash{}{0pt}%
\pgfpathmoveto{\pgfqpoint{6.493695in}{8.961394in}}%
\pgfpathcurveto{\pgfqpoint{6.504745in}{8.961394in}}{\pgfqpoint{6.515344in}{8.965784in}}{\pgfqpoint{6.523157in}{8.973598in}}%
\pgfpathcurveto{\pgfqpoint{6.530971in}{8.981411in}}{\pgfqpoint{6.535361in}{8.992011in}}{\pgfqpoint{6.535361in}{9.003061in}}%
\pgfpathcurveto{\pgfqpoint{6.535361in}{9.014111in}}{\pgfqpoint{6.530971in}{9.024710in}}{\pgfqpoint{6.523157in}{9.032523in}}%
\pgfpathcurveto{\pgfqpoint{6.515344in}{9.040337in}}{\pgfqpoint{6.504745in}{9.044727in}}{\pgfqpoint{6.493695in}{9.044727in}}%
\pgfpathcurveto{\pgfqpoint{6.482644in}{9.044727in}}{\pgfqpoint{6.472045in}{9.040337in}}{\pgfqpoint{6.464232in}{9.032523in}}%
\pgfpathcurveto{\pgfqpoint{6.456418in}{9.024710in}}{\pgfqpoint{6.452028in}{9.014111in}}{\pgfqpoint{6.452028in}{9.003061in}}%
\pgfpathcurveto{\pgfqpoint{6.452028in}{8.992011in}}{\pgfqpoint{6.456418in}{8.981411in}}{\pgfqpoint{6.464232in}{8.973598in}}%
\pgfpathcurveto{\pgfqpoint{6.472045in}{8.965784in}}{\pgfqpoint{6.482644in}{8.961394in}}{\pgfqpoint{6.493695in}{8.961394in}}%
\pgfpathlineto{\pgfqpoint{6.493695in}{8.961394in}}%
\pgfpathclose%
\pgfusepath{stroke,fill}%
\end{pgfscope}%
\begin{pgfscope}%
\pgfpathrectangle{\pgfqpoint{5.292946in}{7.624184in}}{\pgfqpoint{2.177280in}{2.201755in}}%
\pgfusepath{clip}%
\pgfsetbuttcap%
\pgfsetroundjoin%
\definecolor{currentfill}{rgb}{1.000000,0.498039,0.054902}%
\pgfsetfillcolor{currentfill}%
\pgfsetlinewidth{0.481800pt}%
\definecolor{currentstroke}{rgb}{1.000000,1.000000,1.000000}%
\pgfsetstrokecolor{currentstroke}%
\pgfsetdash{}{0pt}%
\pgfpathmoveto{\pgfqpoint{6.292758in}{8.182996in}}%
\pgfpathcurveto{\pgfqpoint{6.303808in}{8.182996in}}{\pgfqpoint{6.314407in}{8.187386in}}{\pgfqpoint{6.322220in}{8.195200in}}%
\pgfpathcurveto{\pgfqpoint{6.330034in}{8.203013in}}{\pgfqpoint{6.334424in}{8.213612in}}{\pgfqpoint{6.334424in}{8.224662in}}%
\pgfpathcurveto{\pgfqpoint{6.334424in}{8.235713in}}{\pgfqpoint{6.330034in}{8.246312in}}{\pgfqpoint{6.322220in}{8.254125in}}%
\pgfpathcurveto{\pgfqpoint{6.314407in}{8.261939in}}{\pgfqpoint{6.303808in}{8.266329in}}{\pgfqpoint{6.292758in}{8.266329in}}%
\pgfpathcurveto{\pgfqpoint{6.281707in}{8.266329in}}{\pgfqpoint{6.271108in}{8.261939in}}{\pgfqpoint{6.263295in}{8.254125in}}%
\pgfpathcurveto{\pgfqpoint{6.255481in}{8.246312in}}{\pgfqpoint{6.251091in}{8.235713in}}{\pgfqpoint{6.251091in}{8.224662in}}%
\pgfpathcurveto{\pgfqpoint{6.251091in}{8.213612in}}{\pgfqpoint{6.255481in}{8.203013in}}{\pgfqpoint{6.263295in}{8.195200in}}%
\pgfpathcurveto{\pgfqpoint{6.271108in}{8.187386in}}{\pgfqpoint{6.281707in}{8.182996in}}{\pgfqpoint{6.292758in}{8.182996in}}%
\pgfpathlineto{\pgfqpoint{6.292758in}{8.182996in}}%
\pgfpathclose%
\pgfusepath{stroke,fill}%
\end{pgfscope}%
\begin{pgfscope}%
\pgfpathrectangle{\pgfqpoint{5.292946in}{7.624184in}}{\pgfqpoint{2.177280in}{2.201755in}}%
\pgfusepath{clip}%
\pgfsetbuttcap%
\pgfsetroundjoin%
\definecolor{currentfill}{rgb}{1.000000,0.498039,0.054902}%
\pgfsetfillcolor{currentfill}%
\pgfsetlinewidth{0.481800pt}%
\definecolor{currentstroke}{rgb}{1.000000,1.000000,1.000000}%
\pgfsetstrokecolor{currentstroke}%
\pgfsetdash{}{0pt}%
\pgfpathmoveto{\pgfqpoint{6.177936in}{8.071796in}}%
\pgfpathcurveto{\pgfqpoint{6.188987in}{8.071796in}}{\pgfqpoint{6.199586in}{8.076186in}}{\pgfqpoint{6.207399in}{8.084000in}}%
\pgfpathcurveto{\pgfqpoint{6.215213in}{8.091813in}}{\pgfqpoint{6.219603in}{8.102413in}}{\pgfqpoint{6.219603in}{8.113463in}}%
\pgfpathcurveto{\pgfqpoint{6.219603in}{8.124513in}}{\pgfqpoint{6.215213in}{8.135112in}}{\pgfqpoint{6.207399in}{8.142925in}}%
\pgfpathcurveto{\pgfqpoint{6.199586in}{8.150739in}}{\pgfqpoint{6.188987in}{8.155129in}}{\pgfqpoint{6.177936in}{8.155129in}}%
\pgfpathcurveto{\pgfqpoint{6.166886in}{8.155129in}}{\pgfqpoint{6.156287in}{8.150739in}}{\pgfqpoint{6.148474in}{8.142925in}}%
\pgfpathcurveto{\pgfqpoint{6.140660in}{8.135112in}}{\pgfqpoint{6.136270in}{8.124513in}}{\pgfqpoint{6.136270in}{8.113463in}}%
\pgfpathcurveto{\pgfqpoint{6.136270in}{8.102413in}}{\pgfqpoint{6.140660in}{8.091813in}}{\pgfqpoint{6.148474in}{8.084000in}}%
\pgfpathcurveto{\pgfqpoint{6.156287in}{8.076186in}}{\pgfqpoint{6.166886in}{8.071796in}}{\pgfqpoint{6.177936in}{8.071796in}}%
\pgfpathlineto{\pgfqpoint{6.177936in}{8.071796in}}%
\pgfpathclose%
\pgfusepath{stroke,fill}%
\end{pgfscope}%
\begin{pgfscope}%
\pgfpathrectangle{\pgfqpoint{5.292946in}{7.624184in}}{\pgfqpoint{2.177280in}{2.201755in}}%
\pgfusepath{clip}%
\pgfsetbuttcap%
\pgfsetroundjoin%
\definecolor{currentfill}{rgb}{1.000000,0.498039,0.054902}%
\pgfsetfillcolor{currentfill}%
\pgfsetlinewidth{0.481800pt}%
\definecolor{currentstroke}{rgb}{1.000000,1.000000,1.000000}%
\pgfsetstrokecolor{currentstroke}%
\pgfsetdash{}{0pt}%
\pgfpathmoveto{\pgfqpoint{6.378873in}{8.572195in}}%
\pgfpathcurveto{\pgfqpoint{6.389924in}{8.572195in}}{\pgfqpoint{6.400523in}{8.576585in}}{\pgfqpoint{6.408336in}{8.584399in}}%
\pgfpathcurveto{\pgfqpoint{6.416150in}{8.592212in}}{\pgfqpoint{6.420540in}{8.602811in}}{\pgfqpoint{6.420540in}{8.613862in}}%
\pgfpathcurveto{\pgfqpoint{6.420540in}{8.624912in}}{\pgfqpoint{6.416150in}{8.635511in}}{\pgfqpoint{6.408336in}{8.643324in}}%
\pgfpathcurveto{\pgfqpoint{6.400523in}{8.651138in}}{\pgfqpoint{6.389924in}{8.655528in}}{\pgfqpoint{6.378873in}{8.655528in}}%
\pgfpathcurveto{\pgfqpoint{6.367823in}{8.655528in}}{\pgfqpoint{6.357224in}{8.651138in}}{\pgfqpoint{6.349411in}{8.643324in}}%
\pgfpathcurveto{\pgfqpoint{6.341597in}{8.635511in}}{\pgfqpoint{6.337207in}{8.624912in}}{\pgfqpoint{6.337207in}{8.613862in}}%
\pgfpathcurveto{\pgfqpoint{6.337207in}{8.602811in}}{\pgfqpoint{6.341597in}{8.592212in}}{\pgfqpoint{6.349411in}{8.584399in}}%
\pgfpathcurveto{\pgfqpoint{6.357224in}{8.576585in}}{\pgfqpoint{6.367823in}{8.572195in}}{\pgfqpoint{6.378873in}{8.572195in}}%
\pgfpathlineto{\pgfqpoint{6.378873in}{8.572195in}}%
\pgfpathclose%
\pgfusepath{stroke,fill}%
\end{pgfscope}%
\begin{pgfscope}%
\pgfpathrectangle{\pgfqpoint{5.292946in}{7.624184in}}{\pgfqpoint{2.177280in}{2.201755in}}%
\pgfusepath{clip}%
\pgfsetbuttcap%
\pgfsetroundjoin%
\definecolor{currentfill}{rgb}{1.000000,0.498039,0.054902}%
\pgfsetfillcolor{currentfill}%
\pgfsetlinewidth{0.481800pt}%
\definecolor{currentstroke}{rgb}{1.000000,1.000000,1.000000}%
\pgfsetstrokecolor{currentstroke}%
\pgfsetdash{}{0pt}%
\pgfpathmoveto{\pgfqpoint{6.321463in}{8.627795in}}%
\pgfpathcurveto{\pgfqpoint{6.332513in}{8.627795in}}{\pgfqpoint{6.343112in}{8.632185in}}{\pgfqpoint{6.350926in}{8.639999in}}%
\pgfpathcurveto{\pgfqpoint{6.358739in}{8.647812in}}{\pgfqpoint{6.363130in}{8.658411in}}{\pgfqpoint{6.363130in}{8.669461in}}%
\pgfpathcurveto{\pgfqpoint{6.363130in}{8.680512in}}{\pgfqpoint{6.358739in}{8.691111in}}{\pgfqpoint{6.350926in}{8.698924in}}%
\pgfpathcurveto{\pgfqpoint{6.343112in}{8.706738in}}{\pgfqpoint{6.332513in}{8.711128in}}{\pgfqpoint{6.321463in}{8.711128in}}%
\pgfpathcurveto{\pgfqpoint{6.310413in}{8.711128in}}{\pgfqpoint{6.299814in}{8.706738in}}{\pgfqpoint{6.292000in}{8.698924in}}%
\pgfpathcurveto{\pgfqpoint{6.284186in}{8.691111in}}{\pgfqpoint{6.279796in}{8.680512in}}{\pgfqpoint{6.279796in}{8.669461in}}%
\pgfpathcurveto{\pgfqpoint{6.279796in}{8.658411in}}{\pgfqpoint{6.284186in}{8.647812in}}{\pgfqpoint{6.292000in}{8.639999in}}%
\pgfpathcurveto{\pgfqpoint{6.299814in}{8.632185in}}{\pgfqpoint{6.310413in}{8.627795in}}{\pgfqpoint{6.321463in}{8.627795in}}%
\pgfpathlineto{\pgfqpoint{6.321463in}{8.627795in}}%
\pgfpathclose%
\pgfusepath{stroke,fill}%
\end{pgfscope}%
\begin{pgfscope}%
\pgfpathrectangle{\pgfqpoint{5.292946in}{7.624184in}}{\pgfqpoint{2.177280in}{2.201755in}}%
\pgfusepath{clip}%
\pgfsetbuttcap%
\pgfsetroundjoin%
\definecolor{currentfill}{rgb}{1.000000,0.498039,0.054902}%
\pgfsetfillcolor{currentfill}%
\pgfsetlinewidth{0.481800pt}%
\definecolor{currentstroke}{rgb}{1.000000,1.000000,1.000000}%
\pgfsetstrokecolor{currentstroke}%
\pgfsetdash{}{0pt}%
\pgfpathmoveto{\pgfqpoint{6.522400in}{8.683395in}}%
\pgfpathcurveto{\pgfqpoint{6.533450in}{8.683395in}}{\pgfqpoint{6.544049in}{8.687785in}}{\pgfqpoint{6.551863in}{8.695598in}}%
\pgfpathcurveto{\pgfqpoint{6.559676in}{8.703412in}}{\pgfqpoint{6.564067in}{8.714011in}}{\pgfqpoint{6.564067in}{8.725061in}}%
\pgfpathcurveto{\pgfqpoint{6.564067in}{8.736111in}}{\pgfqpoint{6.559676in}{8.746710in}}{\pgfqpoint{6.551863in}{8.754524in}}%
\pgfpathcurveto{\pgfqpoint{6.544049in}{8.762338in}}{\pgfqpoint{6.533450in}{8.766728in}}{\pgfqpoint{6.522400in}{8.766728in}}%
\pgfpathcurveto{\pgfqpoint{6.511350in}{8.766728in}}{\pgfqpoint{6.500751in}{8.762338in}}{\pgfqpoint{6.492937in}{8.754524in}}%
\pgfpathcurveto{\pgfqpoint{6.485123in}{8.746710in}}{\pgfqpoint{6.480733in}{8.736111in}}{\pgfqpoint{6.480733in}{8.725061in}}%
\pgfpathcurveto{\pgfqpoint{6.480733in}{8.714011in}}{\pgfqpoint{6.485123in}{8.703412in}}{\pgfqpoint{6.492937in}{8.695598in}}%
\pgfpathcurveto{\pgfqpoint{6.500751in}{8.687785in}}{\pgfqpoint{6.511350in}{8.683395in}}{\pgfqpoint{6.522400in}{8.683395in}}%
\pgfpathlineto{\pgfqpoint{6.522400in}{8.683395in}}%
\pgfpathclose%
\pgfusepath{stroke,fill}%
\end{pgfscope}%
\begin{pgfscope}%
\pgfpathrectangle{\pgfqpoint{5.292946in}{7.624184in}}{\pgfqpoint{2.177280in}{2.201755in}}%
\pgfusepath{clip}%
\pgfsetbuttcap%
\pgfsetroundjoin%
\definecolor{currentfill}{rgb}{1.000000,0.498039,0.054902}%
\pgfsetfillcolor{currentfill}%
\pgfsetlinewidth{0.481800pt}%
\definecolor{currentstroke}{rgb}{1.000000,1.000000,1.000000}%
\pgfsetstrokecolor{currentstroke}%
\pgfsetdash{}{0pt}%
\pgfpathmoveto{\pgfqpoint{6.206642in}{8.405395in}}%
\pgfpathcurveto{\pgfqpoint{6.217692in}{8.405395in}}{\pgfqpoint{6.228291in}{8.409785in}}{\pgfqpoint{6.236104in}{8.417599in}}%
\pgfpathcurveto{\pgfqpoint{6.243918in}{8.425413in}}{\pgfqpoint{6.248308in}{8.436012in}}{\pgfqpoint{6.248308in}{8.447062in}}%
\pgfpathcurveto{\pgfqpoint{6.248308in}{8.458112in}}{\pgfqpoint{6.243918in}{8.468711in}}{\pgfqpoint{6.236104in}{8.476525in}}%
\pgfpathcurveto{\pgfqpoint{6.228291in}{8.484338in}}{\pgfqpoint{6.217692in}{8.488729in}}{\pgfqpoint{6.206642in}{8.488729in}}%
\pgfpathcurveto{\pgfqpoint{6.195592in}{8.488729in}}{\pgfqpoint{6.184993in}{8.484338in}}{\pgfqpoint{6.177179in}{8.476525in}}%
\pgfpathcurveto{\pgfqpoint{6.169365in}{8.468711in}}{\pgfqpoint{6.164975in}{8.458112in}}{\pgfqpoint{6.164975in}{8.447062in}}%
\pgfpathcurveto{\pgfqpoint{6.164975in}{8.436012in}}{\pgfqpoint{6.169365in}{8.425413in}}{\pgfqpoint{6.177179in}{8.417599in}}%
\pgfpathcurveto{\pgfqpoint{6.184993in}{8.409785in}}{\pgfqpoint{6.195592in}{8.405395in}}{\pgfqpoint{6.206642in}{8.405395in}}%
\pgfpathlineto{\pgfqpoint{6.206642in}{8.405395in}}%
\pgfpathclose%
\pgfusepath{stroke,fill}%
\end{pgfscope}%
\begin{pgfscope}%
\pgfpathrectangle{\pgfqpoint{5.292946in}{7.624184in}}{\pgfqpoint{2.177280in}{2.201755in}}%
\pgfusepath{clip}%
\pgfsetbuttcap%
\pgfsetroundjoin%
\definecolor{currentfill}{rgb}{1.000000,0.498039,0.054902}%
\pgfsetfillcolor{currentfill}%
\pgfsetlinewidth{0.481800pt}%
\definecolor{currentstroke}{rgb}{1.000000,1.000000,1.000000}%
\pgfsetstrokecolor{currentstroke}%
\pgfsetdash{}{0pt}%
\pgfpathmoveto{\pgfqpoint{6.436284in}{9.016994in}}%
\pgfpathcurveto{\pgfqpoint{6.447334in}{9.016994in}}{\pgfqpoint{6.457933in}{9.021384in}}{\pgfqpoint{6.465747in}{9.029198in}}%
\pgfpathcurveto{\pgfqpoint{6.473560in}{9.037011in}}{\pgfqpoint{6.477951in}{9.047610in}}{\pgfqpoint{6.477951in}{9.058661in}}%
\pgfpathcurveto{\pgfqpoint{6.477951in}{9.069711in}}{\pgfqpoint{6.473560in}{9.080310in}}{\pgfqpoint{6.465747in}{9.088123in}}%
\pgfpathcurveto{\pgfqpoint{6.457933in}{9.095937in}}{\pgfqpoint{6.447334in}{9.100327in}}{\pgfqpoint{6.436284in}{9.100327in}}%
\pgfpathcurveto{\pgfqpoint{6.425234in}{9.100327in}}{\pgfqpoint{6.414635in}{9.095937in}}{\pgfqpoint{6.406821in}{9.088123in}}%
\pgfpathcurveto{\pgfqpoint{6.399008in}{9.080310in}}{\pgfqpoint{6.394617in}{9.069711in}}{\pgfqpoint{6.394617in}{9.058661in}}%
\pgfpathcurveto{\pgfqpoint{6.394617in}{9.047610in}}{\pgfqpoint{6.399008in}{9.037011in}}{\pgfqpoint{6.406821in}{9.029198in}}%
\pgfpathcurveto{\pgfqpoint{6.414635in}{9.021384in}}{\pgfqpoint{6.425234in}{9.016994in}}{\pgfqpoint{6.436284in}{9.016994in}}%
\pgfpathlineto{\pgfqpoint{6.436284in}{9.016994in}}%
\pgfpathclose%
\pgfusepath{stroke,fill}%
\end{pgfscope}%
\begin{pgfscope}%
\pgfpathrectangle{\pgfqpoint{5.292946in}{7.624184in}}{\pgfqpoint{2.177280in}{2.201755in}}%
\pgfusepath{clip}%
\pgfsetbuttcap%
\pgfsetroundjoin%
\definecolor{currentfill}{rgb}{1.000000,0.498039,0.054902}%
\pgfsetfillcolor{currentfill}%
\pgfsetlinewidth{0.481800pt}%
\definecolor{currentstroke}{rgb}{1.000000,1.000000,1.000000}%
\pgfsetstrokecolor{currentstroke}%
\pgfsetdash{}{0pt}%
\pgfpathmoveto{\pgfqpoint{6.464989in}{8.405395in}}%
\pgfpathcurveto{\pgfqpoint{6.476039in}{8.405395in}}{\pgfqpoint{6.486638in}{8.409785in}}{\pgfqpoint{6.494452in}{8.417599in}}%
\pgfpathcurveto{\pgfqpoint{6.502266in}{8.425413in}}{\pgfqpoint{6.506656in}{8.436012in}}{\pgfqpoint{6.506656in}{8.447062in}}%
\pgfpathcurveto{\pgfqpoint{6.506656in}{8.458112in}}{\pgfqpoint{6.502266in}{8.468711in}}{\pgfqpoint{6.494452in}{8.476525in}}%
\pgfpathcurveto{\pgfqpoint{6.486638in}{8.484338in}}{\pgfqpoint{6.476039in}{8.488729in}}{\pgfqpoint{6.464989in}{8.488729in}}%
\pgfpathcurveto{\pgfqpoint{6.453939in}{8.488729in}}{\pgfqpoint{6.443340in}{8.484338in}}{\pgfqpoint{6.435527in}{8.476525in}}%
\pgfpathcurveto{\pgfqpoint{6.427713in}{8.468711in}}{\pgfqpoint{6.423323in}{8.458112in}}{\pgfqpoint{6.423323in}{8.447062in}}%
\pgfpathcurveto{\pgfqpoint{6.423323in}{8.436012in}}{\pgfqpoint{6.427713in}{8.425413in}}{\pgfqpoint{6.435527in}{8.417599in}}%
\pgfpathcurveto{\pgfqpoint{6.443340in}{8.409785in}}{\pgfqpoint{6.453939in}{8.405395in}}{\pgfqpoint{6.464989in}{8.405395in}}%
\pgfpathlineto{\pgfqpoint{6.464989in}{8.405395in}}%
\pgfpathclose%
\pgfusepath{stroke,fill}%
\end{pgfscope}%
\begin{pgfscope}%
\pgfpathrectangle{\pgfqpoint{5.292946in}{7.624184in}}{\pgfqpoint{2.177280in}{2.201755in}}%
\pgfusepath{clip}%
\pgfsetbuttcap%
\pgfsetroundjoin%
\definecolor{currentfill}{rgb}{1.000000,0.498039,0.054902}%
\pgfsetfillcolor{currentfill}%
\pgfsetlinewidth{0.481800pt}%
\definecolor{currentstroke}{rgb}{1.000000,1.000000,1.000000}%
\pgfsetstrokecolor{currentstroke}%
\pgfsetdash{}{0pt}%
\pgfpathmoveto{\pgfqpoint{6.350168in}{8.516595in}}%
\pgfpathcurveto{\pgfqpoint{6.361218in}{8.516595in}}{\pgfqpoint{6.371817in}{8.520985in}}{\pgfqpoint{6.379631in}{8.528799in}}%
\pgfpathcurveto{\pgfqpoint{6.387445in}{8.536612in}}{\pgfqpoint{6.391835in}{8.547212in}}{\pgfqpoint{6.391835in}{8.558262in}}%
\pgfpathcurveto{\pgfqpoint{6.391835in}{8.569312in}}{\pgfqpoint{6.387445in}{8.579911in}}{\pgfqpoint{6.379631in}{8.587724in}}%
\pgfpathcurveto{\pgfqpoint{6.371817in}{8.595538in}}{\pgfqpoint{6.361218in}{8.599928in}}{\pgfqpoint{6.350168in}{8.599928in}}%
\pgfpathcurveto{\pgfqpoint{6.339118in}{8.599928in}}{\pgfqpoint{6.328519in}{8.595538in}}{\pgfqpoint{6.320705in}{8.587724in}}%
\pgfpathcurveto{\pgfqpoint{6.312892in}{8.579911in}}{\pgfqpoint{6.308501in}{8.569312in}}{\pgfqpoint{6.308501in}{8.558262in}}%
\pgfpathcurveto{\pgfqpoint{6.308501in}{8.547212in}}{\pgfqpoint{6.312892in}{8.536612in}}{\pgfqpoint{6.320705in}{8.528799in}}%
\pgfpathcurveto{\pgfqpoint{6.328519in}{8.520985in}}{\pgfqpoint{6.339118in}{8.516595in}}{\pgfqpoint{6.350168in}{8.516595in}}%
\pgfpathlineto{\pgfqpoint{6.350168in}{8.516595in}}%
\pgfpathclose%
\pgfusepath{stroke,fill}%
\end{pgfscope}%
\begin{pgfscope}%
\pgfpathrectangle{\pgfqpoint{5.292946in}{7.624184in}}{\pgfqpoint{2.177280in}{2.201755in}}%
\pgfusepath{clip}%
\pgfsetbuttcap%
\pgfsetroundjoin%
\definecolor{currentfill}{rgb}{1.000000,0.498039,0.054902}%
\pgfsetfillcolor{currentfill}%
\pgfsetlinewidth{0.481800pt}%
\definecolor{currentstroke}{rgb}{1.000000,1.000000,1.000000}%
\pgfsetstrokecolor{currentstroke}%
\pgfsetdash{}{0pt}%
\pgfpathmoveto{\pgfqpoint{6.464989in}{8.738994in}}%
\pgfpathcurveto{\pgfqpoint{6.476039in}{8.738994in}}{\pgfqpoint{6.486638in}{8.743385in}}{\pgfqpoint{6.494452in}{8.751198in}}%
\pgfpathcurveto{\pgfqpoint{6.502266in}{8.759012in}}{\pgfqpoint{6.506656in}{8.769611in}}{\pgfqpoint{6.506656in}{8.780661in}}%
\pgfpathcurveto{\pgfqpoint{6.506656in}{8.791711in}}{\pgfqpoint{6.502266in}{8.802310in}}{\pgfqpoint{6.494452in}{8.810124in}}%
\pgfpathcurveto{\pgfqpoint{6.486638in}{8.817938in}}{\pgfqpoint{6.476039in}{8.822328in}}{\pgfqpoint{6.464989in}{8.822328in}}%
\pgfpathcurveto{\pgfqpoint{6.453939in}{8.822328in}}{\pgfqpoint{6.443340in}{8.817938in}}{\pgfqpoint{6.435527in}{8.810124in}}%
\pgfpathcurveto{\pgfqpoint{6.427713in}{8.802310in}}{\pgfqpoint{6.423323in}{8.791711in}}{\pgfqpoint{6.423323in}{8.780661in}}%
\pgfpathcurveto{\pgfqpoint{6.423323in}{8.769611in}}{\pgfqpoint{6.427713in}{8.759012in}}{\pgfqpoint{6.435527in}{8.751198in}}%
\pgfpathcurveto{\pgfqpoint{6.443340in}{8.743385in}}{\pgfqpoint{6.453939in}{8.738994in}}{\pgfqpoint{6.464989in}{8.738994in}}%
\pgfpathlineto{\pgfqpoint{6.464989in}{8.738994in}}%
\pgfpathclose%
\pgfusepath{stroke,fill}%
\end{pgfscope}%
\begin{pgfscope}%
\pgfpathrectangle{\pgfqpoint{5.292946in}{7.624184in}}{\pgfqpoint{2.177280in}{2.201755in}}%
\pgfusepath{clip}%
\pgfsetbuttcap%
\pgfsetroundjoin%
\definecolor{currentfill}{rgb}{1.000000,0.498039,0.054902}%
\pgfsetfillcolor{currentfill}%
\pgfsetlinewidth{0.481800pt}%
\definecolor{currentstroke}{rgb}{1.000000,1.000000,1.000000}%
\pgfsetstrokecolor{currentstroke}%
\pgfsetdash{}{0pt}%
\pgfpathmoveto{\pgfqpoint{6.292758in}{8.405395in}}%
\pgfpathcurveto{\pgfqpoint{6.303808in}{8.405395in}}{\pgfqpoint{6.314407in}{8.409785in}}{\pgfqpoint{6.322220in}{8.417599in}}%
\pgfpathcurveto{\pgfqpoint{6.330034in}{8.425413in}}{\pgfqpoint{6.334424in}{8.436012in}}{\pgfqpoint{6.334424in}{8.447062in}}%
\pgfpathcurveto{\pgfqpoint{6.334424in}{8.458112in}}{\pgfqpoint{6.330034in}{8.468711in}}{\pgfqpoint{6.322220in}{8.476525in}}%
\pgfpathcurveto{\pgfqpoint{6.314407in}{8.484338in}}{\pgfqpoint{6.303808in}{8.488729in}}{\pgfqpoint{6.292758in}{8.488729in}}%
\pgfpathcurveto{\pgfqpoint{6.281707in}{8.488729in}}{\pgfqpoint{6.271108in}{8.484338in}}{\pgfqpoint{6.263295in}{8.476525in}}%
\pgfpathcurveto{\pgfqpoint{6.255481in}{8.468711in}}{\pgfqpoint{6.251091in}{8.458112in}}{\pgfqpoint{6.251091in}{8.447062in}}%
\pgfpathcurveto{\pgfqpoint{6.251091in}{8.436012in}}{\pgfqpoint{6.255481in}{8.425413in}}{\pgfqpoint{6.263295in}{8.417599in}}%
\pgfpathcurveto{\pgfqpoint{6.271108in}{8.409785in}}{\pgfqpoint{6.281707in}{8.405395in}}{\pgfqpoint{6.292758in}{8.405395in}}%
\pgfpathlineto{\pgfqpoint{6.292758in}{8.405395in}}%
\pgfpathclose%
\pgfusepath{stroke,fill}%
\end{pgfscope}%
\begin{pgfscope}%
\pgfpathrectangle{\pgfqpoint{5.292946in}{7.624184in}}{\pgfqpoint{2.177280in}{2.201755in}}%
\pgfusepath{clip}%
\pgfsetbuttcap%
\pgfsetroundjoin%
\definecolor{currentfill}{rgb}{1.000000,0.498039,0.054902}%
\pgfsetfillcolor{currentfill}%
\pgfsetlinewidth{0.481800pt}%
\definecolor{currentstroke}{rgb}{1.000000,1.000000,1.000000}%
\pgfsetstrokecolor{currentstroke}%
\pgfsetdash{}{0pt}%
\pgfpathmoveto{\pgfqpoint{6.551105in}{8.572195in}}%
\pgfpathcurveto{\pgfqpoint{6.562155in}{8.572195in}}{\pgfqpoint{6.572754in}{8.576585in}}{\pgfqpoint{6.580568in}{8.584399in}}%
\pgfpathcurveto{\pgfqpoint{6.588382in}{8.592212in}}{\pgfqpoint{6.592772in}{8.602811in}}{\pgfqpoint{6.592772in}{8.613862in}}%
\pgfpathcurveto{\pgfqpoint{6.592772in}{8.624912in}}{\pgfqpoint{6.588382in}{8.635511in}}{\pgfqpoint{6.580568in}{8.643324in}}%
\pgfpathcurveto{\pgfqpoint{6.572754in}{8.651138in}}{\pgfqpoint{6.562155in}{8.655528in}}{\pgfqpoint{6.551105in}{8.655528in}}%
\pgfpathcurveto{\pgfqpoint{6.540055in}{8.655528in}}{\pgfqpoint{6.529456in}{8.651138in}}{\pgfqpoint{6.521642in}{8.643324in}}%
\pgfpathcurveto{\pgfqpoint{6.513829in}{8.635511in}}{\pgfqpoint{6.509438in}{8.624912in}}{\pgfqpoint{6.509438in}{8.613862in}}%
\pgfpathcurveto{\pgfqpoint{6.509438in}{8.602811in}}{\pgfqpoint{6.513829in}{8.592212in}}{\pgfqpoint{6.521642in}{8.584399in}}%
\pgfpathcurveto{\pgfqpoint{6.529456in}{8.576585in}}{\pgfqpoint{6.540055in}{8.572195in}}{\pgfqpoint{6.551105in}{8.572195in}}%
\pgfpathlineto{\pgfqpoint{6.551105in}{8.572195in}}%
\pgfpathclose%
\pgfusepath{stroke,fill}%
\end{pgfscope}%
\begin{pgfscope}%
\pgfpathrectangle{\pgfqpoint{5.292946in}{7.624184in}}{\pgfqpoint{2.177280in}{2.201755in}}%
\pgfusepath{clip}%
\pgfsetbuttcap%
\pgfsetroundjoin%
\definecolor{currentfill}{rgb}{1.000000,0.498039,0.054902}%
\pgfsetfillcolor{currentfill}%
\pgfsetlinewidth{0.481800pt}%
\definecolor{currentstroke}{rgb}{1.000000,1.000000,1.000000}%
\pgfsetstrokecolor{currentstroke}%
\pgfsetdash{}{0pt}%
\pgfpathmoveto{\pgfqpoint{6.321463in}{8.683395in}}%
\pgfpathcurveto{\pgfqpoint{6.332513in}{8.683395in}}{\pgfqpoint{6.343112in}{8.687785in}}{\pgfqpoint{6.350926in}{8.695598in}}%
\pgfpathcurveto{\pgfqpoint{6.358739in}{8.703412in}}{\pgfqpoint{6.363130in}{8.714011in}}{\pgfqpoint{6.363130in}{8.725061in}}%
\pgfpathcurveto{\pgfqpoint{6.363130in}{8.736111in}}{\pgfqpoint{6.358739in}{8.746710in}}{\pgfqpoint{6.350926in}{8.754524in}}%
\pgfpathcurveto{\pgfqpoint{6.343112in}{8.762338in}}{\pgfqpoint{6.332513in}{8.766728in}}{\pgfqpoint{6.321463in}{8.766728in}}%
\pgfpathcurveto{\pgfqpoint{6.310413in}{8.766728in}}{\pgfqpoint{6.299814in}{8.762338in}}{\pgfqpoint{6.292000in}{8.754524in}}%
\pgfpathcurveto{\pgfqpoint{6.284186in}{8.746710in}}{\pgfqpoint{6.279796in}{8.736111in}}{\pgfqpoint{6.279796in}{8.725061in}}%
\pgfpathcurveto{\pgfqpoint{6.279796in}{8.714011in}}{\pgfqpoint{6.284186in}{8.703412in}}{\pgfqpoint{6.292000in}{8.695598in}}%
\pgfpathcurveto{\pgfqpoint{6.299814in}{8.687785in}}{\pgfqpoint{6.310413in}{8.683395in}}{\pgfqpoint{6.321463in}{8.683395in}}%
\pgfpathlineto{\pgfqpoint{6.321463in}{8.683395in}}%
\pgfpathclose%
\pgfusepath{stroke,fill}%
\end{pgfscope}%
\begin{pgfscope}%
\pgfpathrectangle{\pgfqpoint{5.292946in}{7.624184in}}{\pgfqpoint{2.177280in}{2.201755in}}%
\pgfusepath{clip}%
\pgfsetbuttcap%
\pgfsetroundjoin%
\definecolor{currentfill}{rgb}{1.000000,0.498039,0.054902}%
\pgfsetfillcolor{currentfill}%
\pgfsetlinewidth{0.481800pt}%
\definecolor{currentstroke}{rgb}{1.000000,1.000000,1.000000}%
\pgfsetstrokecolor{currentstroke}%
\pgfsetdash{}{0pt}%
\pgfpathmoveto{\pgfqpoint{6.579810in}{8.794594in}}%
\pgfpathcurveto{\pgfqpoint{6.590861in}{8.794594in}}{\pgfqpoint{6.601460in}{8.798985in}}{\pgfqpoint{6.609273in}{8.806798in}}%
\pgfpathcurveto{\pgfqpoint{6.617087in}{8.814612in}}{\pgfqpoint{6.621477in}{8.825211in}}{\pgfqpoint{6.621477in}{8.836261in}}%
\pgfpathcurveto{\pgfqpoint{6.621477in}{8.847311in}}{\pgfqpoint{6.617087in}{8.857910in}}{\pgfqpoint{6.609273in}{8.865724in}}%
\pgfpathcurveto{\pgfqpoint{6.601460in}{8.873537in}}{\pgfqpoint{6.590861in}{8.877928in}}{\pgfqpoint{6.579810in}{8.877928in}}%
\pgfpathcurveto{\pgfqpoint{6.568760in}{8.877928in}}{\pgfqpoint{6.558161in}{8.873537in}}{\pgfqpoint{6.550348in}{8.865724in}}%
\pgfpathcurveto{\pgfqpoint{6.542534in}{8.857910in}}{\pgfqpoint{6.538144in}{8.847311in}}{\pgfqpoint{6.538144in}{8.836261in}}%
\pgfpathcurveto{\pgfqpoint{6.538144in}{8.825211in}}{\pgfqpoint{6.542534in}{8.814612in}}{\pgfqpoint{6.550348in}{8.806798in}}%
\pgfpathcurveto{\pgfqpoint{6.558161in}{8.798985in}}{\pgfqpoint{6.568760in}{8.794594in}}{\pgfqpoint{6.579810in}{8.794594in}}%
\pgfpathlineto{\pgfqpoint{6.579810in}{8.794594in}}%
\pgfpathclose%
\pgfusepath{stroke,fill}%
\end{pgfscope}%
\begin{pgfscope}%
\pgfpathrectangle{\pgfqpoint{5.292946in}{7.624184in}}{\pgfqpoint{2.177280in}{2.201755in}}%
\pgfusepath{clip}%
\pgfsetbuttcap%
\pgfsetroundjoin%
\definecolor{currentfill}{rgb}{1.000000,0.498039,0.054902}%
\pgfsetfillcolor{currentfill}%
\pgfsetlinewidth{0.481800pt}%
\definecolor{currentstroke}{rgb}{1.000000,1.000000,1.000000}%
\pgfsetstrokecolor{currentstroke}%
\pgfsetdash{}{0pt}%
\pgfpathmoveto{\pgfqpoint{6.522400in}{8.683395in}}%
\pgfpathcurveto{\pgfqpoint{6.533450in}{8.683395in}}{\pgfqpoint{6.544049in}{8.687785in}}{\pgfqpoint{6.551863in}{8.695598in}}%
\pgfpathcurveto{\pgfqpoint{6.559676in}{8.703412in}}{\pgfqpoint{6.564067in}{8.714011in}}{\pgfqpoint{6.564067in}{8.725061in}}%
\pgfpathcurveto{\pgfqpoint{6.564067in}{8.736111in}}{\pgfqpoint{6.559676in}{8.746710in}}{\pgfqpoint{6.551863in}{8.754524in}}%
\pgfpathcurveto{\pgfqpoint{6.544049in}{8.762338in}}{\pgfqpoint{6.533450in}{8.766728in}}{\pgfqpoint{6.522400in}{8.766728in}}%
\pgfpathcurveto{\pgfqpoint{6.511350in}{8.766728in}}{\pgfqpoint{6.500751in}{8.762338in}}{\pgfqpoint{6.492937in}{8.754524in}}%
\pgfpathcurveto{\pgfqpoint{6.485123in}{8.746710in}}{\pgfqpoint{6.480733in}{8.736111in}}{\pgfqpoint{6.480733in}{8.725061in}}%
\pgfpathcurveto{\pgfqpoint{6.480733in}{8.714011in}}{\pgfqpoint{6.485123in}{8.703412in}}{\pgfqpoint{6.492937in}{8.695598in}}%
\pgfpathcurveto{\pgfqpoint{6.500751in}{8.687785in}}{\pgfqpoint{6.511350in}{8.683395in}}{\pgfqpoint{6.522400in}{8.683395in}}%
\pgfpathlineto{\pgfqpoint{6.522400in}{8.683395in}}%
\pgfpathclose%
\pgfusepath{stroke,fill}%
\end{pgfscope}%
\begin{pgfscope}%
\pgfpathrectangle{\pgfqpoint{5.292946in}{7.624184in}}{\pgfqpoint{2.177280in}{2.201755in}}%
\pgfusepath{clip}%
\pgfsetbuttcap%
\pgfsetroundjoin%
\definecolor{currentfill}{rgb}{1.000000,0.498039,0.054902}%
\pgfsetfillcolor{currentfill}%
\pgfsetlinewidth{0.481800pt}%
\definecolor{currentstroke}{rgb}{1.000000,1.000000,1.000000}%
\pgfsetstrokecolor{currentstroke}%
\pgfsetdash{}{0pt}%
\pgfpathmoveto{\pgfqpoint{6.407579in}{8.850194in}}%
\pgfpathcurveto{\pgfqpoint{6.418629in}{8.850194in}}{\pgfqpoint{6.429228in}{8.854584in}}{\pgfqpoint{6.437041in}{8.862398in}}%
\pgfpathcurveto{\pgfqpoint{6.444855in}{8.870212in}}{\pgfqpoint{6.449245in}{8.880811in}}{\pgfqpoint{6.449245in}{8.891861in}}%
\pgfpathcurveto{\pgfqpoint{6.449245in}{8.902911in}}{\pgfqpoint{6.444855in}{8.913510in}}{\pgfqpoint{6.437041in}{8.921324in}}%
\pgfpathcurveto{\pgfqpoint{6.429228in}{8.929137in}}{\pgfqpoint{6.418629in}{8.933528in}}{\pgfqpoint{6.407579in}{8.933528in}}%
\pgfpathcurveto{\pgfqpoint{6.396529in}{8.933528in}}{\pgfqpoint{6.385930in}{8.929137in}}{\pgfqpoint{6.378116in}{8.921324in}}%
\pgfpathcurveto{\pgfqpoint{6.370302in}{8.913510in}}{\pgfqpoint{6.365912in}{8.902911in}}{\pgfqpoint{6.365912in}{8.891861in}}%
\pgfpathcurveto{\pgfqpoint{6.365912in}{8.880811in}}{\pgfqpoint{6.370302in}{8.870212in}}{\pgfqpoint{6.378116in}{8.862398in}}%
\pgfpathcurveto{\pgfqpoint{6.385930in}{8.854584in}}{\pgfqpoint{6.396529in}{8.850194in}}{\pgfqpoint{6.407579in}{8.850194in}}%
\pgfpathlineto{\pgfqpoint{6.407579in}{8.850194in}}%
\pgfpathclose%
\pgfusepath{stroke,fill}%
\end{pgfscope}%
\begin{pgfscope}%
\pgfpathrectangle{\pgfqpoint{5.292946in}{7.624184in}}{\pgfqpoint{2.177280in}{2.201755in}}%
\pgfusepath{clip}%
\pgfsetbuttcap%
\pgfsetroundjoin%
\definecolor{currentfill}{rgb}{1.000000,0.498039,0.054902}%
\pgfsetfillcolor{currentfill}%
\pgfsetlinewidth{0.481800pt}%
\definecolor{currentstroke}{rgb}{1.000000,1.000000,1.000000}%
\pgfsetstrokecolor{currentstroke}%
\pgfsetdash{}{0pt}%
\pgfpathmoveto{\pgfqpoint{6.436284in}{8.961394in}}%
\pgfpathcurveto{\pgfqpoint{6.447334in}{8.961394in}}{\pgfqpoint{6.457933in}{8.965784in}}{\pgfqpoint{6.465747in}{8.973598in}}%
\pgfpathcurveto{\pgfqpoint{6.473560in}{8.981411in}}{\pgfqpoint{6.477951in}{8.992011in}}{\pgfqpoint{6.477951in}{9.003061in}}%
\pgfpathcurveto{\pgfqpoint{6.477951in}{9.014111in}}{\pgfqpoint{6.473560in}{9.024710in}}{\pgfqpoint{6.465747in}{9.032523in}}%
\pgfpathcurveto{\pgfqpoint{6.457933in}{9.040337in}}{\pgfqpoint{6.447334in}{9.044727in}}{\pgfqpoint{6.436284in}{9.044727in}}%
\pgfpathcurveto{\pgfqpoint{6.425234in}{9.044727in}}{\pgfqpoint{6.414635in}{9.040337in}}{\pgfqpoint{6.406821in}{9.032523in}}%
\pgfpathcurveto{\pgfqpoint{6.399008in}{9.024710in}}{\pgfqpoint{6.394617in}{9.014111in}}{\pgfqpoint{6.394617in}{9.003061in}}%
\pgfpathcurveto{\pgfqpoint{6.394617in}{8.992011in}}{\pgfqpoint{6.399008in}{8.981411in}}{\pgfqpoint{6.406821in}{8.973598in}}%
\pgfpathcurveto{\pgfqpoint{6.414635in}{8.965784in}}{\pgfqpoint{6.425234in}{8.961394in}}{\pgfqpoint{6.436284in}{8.961394in}}%
\pgfpathlineto{\pgfqpoint{6.436284in}{8.961394in}}%
\pgfpathclose%
\pgfusepath{stroke,fill}%
\end{pgfscope}%
\begin{pgfscope}%
\pgfpathrectangle{\pgfqpoint{5.292946in}{7.624184in}}{\pgfqpoint{2.177280in}{2.201755in}}%
\pgfusepath{clip}%
\pgfsetbuttcap%
\pgfsetroundjoin%
\definecolor{currentfill}{rgb}{1.000000,0.498039,0.054902}%
\pgfsetfillcolor{currentfill}%
\pgfsetlinewidth{0.481800pt}%
\definecolor{currentstroke}{rgb}{1.000000,1.000000,1.000000}%
\pgfsetstrokecolor{currentstroke}%
\pgfsetdash{}{0pt}%
\pgfpathmoveto{\pgfqpoint{6.551105in}{9.072594in}}%
\pgfpathcurveto{\pgfqpoint{6.562155in}{9.072594in}}{\pgfqpoint{6.572754in}{9.076984in}}{\pgfqpoint{6.580568in}{9.084798in}}%
\pgfpathcurveto{\pgfqpoint{6.588382in}{9.092611in}}{\pgfqpoint{6.592772in}{9.103210in}}{\pgfqpoint{6.592772in}{9.114260in}}%
\pgfpathcurveto{\pgfqpoint{6.592772in}{9.125311in}}{\pgfqpoint{6.588382in}{9.135910in}}{\pgfqpoint{6.580568in}{9.143723in}}%
\pgfpathcurveto{\pgfqpoint{6.572754in}{9.151537in}}{\pgfqpoint{6.562155in}{9.155927in}}{\pgfqpoint{6.551105in}{9.155927in}}%
\pgfpathcurveto{\pgfqpoint{6.540055in}{9.155927in}}{\pgfqpoint{6.529456in}{9.151537in}}{\pgfqpoint{6.521642in}{9.143723in}}%
\pgfpathcurveto{\pgfqpoint{6.513829in}{9.135910in}}{\pgfqpoint{6.509438in}{9.125311in}}{\pgfqpoint{6.509438in}{9.114260in}}%
\pgfpathcurveto{\pgfqpoint{6.509438in}{9.103210in}}{\pgfqpoint{6.513829in}{9.092611in}}{\pgfqpoint{6.521642in}{9.084798in}}%
\pgfpathcurveto{\pgfqpoint{6.529456in}{9.076984in}}{\pgfqpoint{6.540055in}{9.072594in}}{\pgfqpoint{6.551105in}{9.072594in}}%
\pgfpathlineto{\pgfqpoint{6.551105in}{9.072594in}}%
\pgfpathclose%
\pgfusepath{stroke,fill}%
\end{pgfscope}%
\begin{pgfscope}%
\pgfpathrectangle{\pgfqpoint{5.292946in}{7.624184in}}{\pgfqpoint{2.177280in}{2.201755in}}%
\pgfusepath{clip}%
\pgfsetbuttcap%
\pgfsetroundjoin%
\definecolor{currentfill}{rgb}{1.000000,0.498039,0.054902}%
\pgfsetfillcolor{currentfill}%
\pgfsetlinewidth{0.481800pt}%
\definecolor{currentstroke}{rgb}{1.000000,1.000000,1.000000}%
\pgfsetstrokecolor{currentstroke}%
\pgfsetdash{}{0pt}%
\pgfpathmoveto{\pgfqpoint{6.608516in}{9.016994in}}%
\pgfpathcurveto{\pgfqpoint{6.619566in}{9.016994in}}{\pgfqpoint{6.630165in}{9.021384in}}{\pgfqpoint{6.637978in}{9.029198in}}%
\pgfpathcurveto{\pgfqpoint{6.645792in}{9.037011in}}{\pgfqpoint{6.650182in}{9.047610in}}{\pgfqpoint{6.650182in}{9.058661in}}%
\pgfpathcurveto{\pgfqpoint{6.650182in}{9.069711in}}{\pgfqpoint{6.645792in}{9.080310in}}{\pgfqpoint{6.637978in}{9.088123in}}%
\pgfpathcurveto{\pgfqpoint{6.630165in}{9.095937in}}{\pgfqpoint{6.619566in}{9.100327in}}{\pgfqpoint{6.608516in}{9.100327in}}%
\pgfpathcurveto{\pgfqpoint{6.597466in}{9.100327in}}{\pgfqpoint{6.586867in}{9.095937in}}{\pgfqpoint{6.579053in}{9.088123in}}%
\pgfpathcurveto{\pgfqpoint{6.571239in}{9.080310in}}{\pgfqpoint{6.566849in}{9.069711in}}{\pgfqpoint{6.566849in}{9.058661in}}%
\pgfpathcurveto{\pgfqpoint{6.566849in}{9.047610in}}{\pgfqpoint{6.571239in}{9.037011in}}{\pgfqpoint{6.579053in}{9.029198in}}%
\pgfpathcurveto{\pgfqpoint{6.586867in}{9.021384in}}{\pgfqpoint{6.597466in}{9.016994in}}{\pgfqpoint{6.608516in}{9.016994in}}%
\pgfpathlineto{\pgfqpoint{6.608516in}{9.016994in}}%
\pgfpathclose%
\pgfusepath{stroke,fill}%
\end{pgfscope}%
\begin{pgfscope}%
\pgfpathrectangle{\pgfqpoint{5.292946in}{7.624184in}}{\pgfqpoint{2.177280in}{2.201755in}}%
\pgfusepath{clip}%
\pgfsetbuttcap%
\pgfsetroundjoin%
\definecolor{currentfill}{rgb}{1.000000,0.498039,0.054902}%
\pgfsetfillcolor{currentfill}%
\pgfsetlinewidth{0.481800pt}%
\definecolor{currentstroke}{rgb}{1.000000,1.000000,1.000000}%
\pgfsetstrokecolor{currentstroke}%
\pgfsetdash{}{0pt}%
\pgfpathmoveto{\pgfqpoint{6.464989in}{8.627795in}}%
\pgfpathcurveto{\pgfqpoint{6.476039in}{8.627795in}}{\pgfqpoint{6.486638in}{8.632185in}}{\pgfqpoint{6.494452in}{8.639999in}}%
\pgfpathcurveto{\pgfqpoint{6.502266in}{8.647812in}}{\pgfqpoint{6.506656in}{8.658411in}}{\pgfqpoint{6.506656in}{8.669461in}}%
\pgfpathcurveto{\pgfqpoint{6.506656in}{8.680512in}}{\pgfqpoint{6.502266in}{8.691111in}}{\pgfqpoint{6.494452in}{8.698924in}}%
\pgfpathcurveto{\pgfqpoint{6.486638in}{8.706738in}}{\pgfqpoint{6.476039in}{8.711128in}}{\pgfqpoint{6.464989in}{8.711128in}}%
\pgfpathcurveto{\pgfqpoint{6.453939in}{8.711128in}}{\pgfqpoint{6.443340in}{8.706738in}}{\pgfqpoint{6.435527in}{8.698924in}}%
\pgfpathcurveto{\pgfqpoint{6.427713in}{8.691111in}}{\pgfqpoint{6.423323in}{8.680512in}}{\pgfqpoint{6.423323in}{8.669461in}}%
\pgfpathcurveto{\pgfqpoint{6.423323in}{8.658411in}}{\pgfqpoint{6.427713in}{8.647812in}}{\pgfqpoint{6.435527in}{8.639999in}}%
\pgfpathcurveto{\pgfqpoint{6.443340in}{8.632185in}}{\pgfqpoint{6.453939in}{8.627795in}}{\pgfqpoint{6.464989in}{8.627795in}}%
\pgfpathlineto{\pgfqpoint{6.464989in}{8.627795in}}%
\pgfpathclose%
\pgfusepath{stroke,fill}%
\end{pgfscope}%
\begin{pgfscope}%
\pgfpathrectangle{\pgfqpoint{5.292946in}{7.624184in}}{\pgfqpoint{2.177280in}{2.201755in}}%
\pgfusepath{clip}%
\pgfsetbuttcap%
\pgfsetroundjoin%
\definecolor{currentfill}{rgb}{1.000000,0.498039,0.054902}%
\pgfsetfillcolor{currentfill}%
\pgfsetlinewidth{0.481800pt}%
\definecolor{currentstroke}{rgb}{1.000000,1.000000,1.000000}%
\pgfsetstrokecolor{currentstroke}%
\pgfsetdash{}{0pt}%
\pgfpathmoveto{\pgfqpoint{6.177936in}{8.460995in}}%
\pgfpathcurveto{\pgfqpoint{6.188987in}{8.460995in}}{\pgfqpoint{6.199586in}{8.465385in}}{\pgfqpoint{6.207399in}{8.473199in}}%
\pgfpathcurveto{\pgfqpoint{6.215213in}{8.481013in}}{\pgfqpoint{6.219603in}{8.491612in}}{\pgfqpoint{6.219603in}{8.502662in}}%
\pgfpathcurveto{\pgfqpoint{6.219603in}{8.513712in}}{\pgfqpoint{6.215213in}{8.524311in}}{\pgfqpoint{6.207399in}{8.532125in}}%
\pgfpathcurveto{\pgfqpoint{6.199586in}{8.539938in}}{\pgfqpoint{6.188987in}{8.544328in}}{\pgfqpoint{6.177936in}{8.544328in}}%
\pgfpathcurveto{\pgfqpoint{6.166886in}{8.544328in}}{\pgfqpoint{6.156287in}{8.539938in}}{\pgfqpoint{6.148474in}{8.532125in}}%
\pgfpathcurveto{\pgfqpoint{6.140660in}{8.524311in}}{\pgfqpoint{6.136270in}{8.513712in}}{\pgfqpoint{6.136270in}{8.502662in}}%
\pgfpathcurveto{\pgfqpoint{6.136270in}{8.491612in}}{\pgfqpoint{6.140660in}{8.481013in}}{\pgfqpoint{6.148474in}{8.473199in}}%
\pgfpathcurveto{\pgfqpoint{6.156287in}{8.465385in}}{\pgfqpoint{6.166886in}{8.460995in}}{\pgfqpoint{6.177936in}{8.460995in}}%
\pgfpathlineto{\pgfqpoint{6.177936in}{8.460995in}}%
\pgfpathclose%
\pgfusepath{stroke,fill}%
\end{pgfscope}%
\begin{pgfscope}%
\pgfpathrectangle{\pgfqpoint{5.292946in}{7.624184in}}{\pgfqpoint{2.177280in}{2.201755in}}%
\pgfusepath{clip}%
\pgfsetbuttcap%
\pgfsetroundjoin%
\definecolor{currentfill}{rgb}{1.000000,0.498039,0.054902}%
\pgfsetfillcolor{currentfill}%
\pgfsetlinewidth{0.481800pt}%
\definecolor{currentstroke}{rgb}{1.000000,1.000000,1.000000}%
\pgfsetstrokecolor{currentstroke}%
\pgfsetdash{}{0pt}%
\pgfpathmoveto{\pgfqpoint{6.264052in}{8.349795in}}%
\pgfpathcurveto{\pgfqpoint{6.275102in}{8.349795in}}{\pgfqpoint{6.285701in}{8.354186in}}{\pgfqpoint{6.293515in}{8.361999in}}%
\pgfpathcurveto{\pgfqpoint{6.301329in}{8.369813in}}{\pgfqpoint{6.305719in}{8.380412in}}{\pgfqpoint{6.305719in}{8.391462in}}%
\pgfpathcurveto{\pgfqpoint{6.305719in}{8.402512in}}{\pgfqpoint{6.301329in}{8.413111in}}{\pgfqpoint{6.293515in}{8.420925in}}%
\pgfpathcurveto{\pgfqpoint{6.285701in}{8.428738in}}{\pgfqpoint{6.275102in}{8.433129in}}{\pgfqpoint{6.264052in}{8.433129in}}%
\pgfpathcurveto{\pgfqpoint{6.253002in}{8.433129in}}{\pgfqpoint{6.242403in}{8.428738in}}{\pgfqpoint{6.234590in}{8.420925in}}%
\pgfpathcurveto{\pgfqpoint{6.226776in}{8.413111in}}{\pgfqpoint{6.222386in}{8.402512in}}{\pgfqpoint{6.222386in}{8.391462in}}%
\pgfpathcurveto{\pgfqpoint{6.222386in}{8.380412in}}{\pgfqpoint{6.226776in}{8.369813in}}{\pgfqpoint{6.234590in}{8.361999in}}%
\pgfpathcurveto{\pgfqpoint{6.242403in}{8.354186in}}{\pgfqpoint{6.253002in}{8.349795in}}{\pgfqpoint{6.264052in}{8.349795in}}%
\pgfpathlineto{\pgfqpoint{6.264052in}{8.349795in}}%
\pgfpathclose%
\pgfusepath{stroke,fill}%
\end{pgfscope}%
\begin{pgfscope}%
\pgfpathrectangle{\pgfqpoint{5.292946in}{7.624184in}}{\pgfqpoint{2.177280in}{2.201755in}}%
\pgfusepath{clip}%
\pgfsetbuttcap%
\pgfsetroundjoin%
\definecolor{currentfill}{rgb}{1.000000,0.498039,0.054902}%
\pgfsetfillcolor{currentfill}%
\pgfsetlinewidth{0.481800pt}%
\definecolor{currentstroke}{rgb}{1.000000,1.000000,1.000000}%
\pgfsetstrokecolor{currentstroke}%
\pgfsetdash{}{0pt}%
\pgfpathmoveto{\pgfqpoint{6.235347in}{8.349795in}}%
\pgfpathcurveto{\pgfqpoint{6.246397in}{8.349795in}}{\pgfqpoint{6.256996in}{8.354186in}}{\pgfqpoint{6.264810in}{8.361999in}}%
\pgfpathcurveto{\pgfqpoint{6.272623in}{8.369813in}}{\pgfqpoint{6.277014in}{8.380412in}}{\pgfqpoint{6.277014in}{8.391462in}}%
\pgfpathcurveto{\pgfqpoint{6.277014in}{8.402512in}}{\pgfqpoint{6.272623in}{8.413111in}}{\pgfqpoint{6.264810in}{8.420925in}}%
\pgfpathcurveto{\pgfqpoint{6.256996in}{8.428738in}}{\pgfqpoint{6.246397in}{8.433129in}}{\pgfqpoint{6.235347in}{8.433129in}}%
\pgfpathcurveto{\pgfqpoint{6.224297in}{8.433129in}}{\pgfqpoint{6.213698in}{8.428738in}}{\pgfqpoint{6.205884in}{8.420925in}}%
\pgfpathcurveto{\pgfqpoint{6.198071in}{8.413111in}}{\pgfqpoint{6.193680in}{8.402512in}}{\pgfqpoint{6.193680in}{8.391462in}}%
\pgfpathcurveto{\pgfqpoint{6.193680in}{8.380412in}}{\pgfqpoint{6.198071in}{8.369813in}}{\pgfqpoint{6.205884in}{8.361999in}}%
\pgfpathcurveto{\pgfqpoint{6.213698in}{8.354186in}}{\pgfqpoint{6.224297in}{8.349795in}}{\pgfqpoint{6.235347in}{8.349795in}}%
\pgfpathlineto{\pgfqpoint{6.235347in}{8.349795in}}%
\pgfpathclose%
\pgfusepath{stroke,fill}%
\end{pgfscope}%
\begin{pgfscope}%
\pgfpathrectangle{\pgfqpoint{5.292946in}{7.624184in}}{\pgfqpoint{2.177280in}{2.201755in}}%
\pgfusepath{clip}%
\pgfsetbuttcap%
\pgfsetroundjoin%
\definecolor{currentfill}{rgb}{1.000000,0.498039,0.054902}%
\pgfsetfillcolor{currentfill}%
\pgfsetlinewidth{0.481800pt}%
\definecolor{currentstroke}{rgb}{1.000000,1.000000,1.000000}%
\pgfsetstrokecolor{currentstroke}%
\pgfsetdash{}{0pt}%
\pgfpathmoveto{\pgfqpoint{6.292758in}{8.516595in}}%
\pgfpathcurveto{\pgfqpoint{6.303808in}{8.516595in}}{\pgfqpoint{6.314407in}{8.520985in}}{\pgfqpoint{6.322220in}{8.528799in}}%
\pgfpathcurveto{\pgfqpoint{6.330034in}{8.536612in}}{\pgfqpoint{6.334424in}{8.547212in}}{\pgfqpoint{6.334424in}{8.558262in}}%
\pgfpathcurveto{\pgfqpoint{6.334424in}{8.569312in}}{\pgfqpoint{6.330034in}{8.579911in}}{\pgfqpoint{6.322220in}{8.587724in}}%
\pgfpathcurveto{\pgfqpoint{6.314407in}{8.595538in}}{\pgfqpoint{6.303808in}{8.599928in}}{\pgfqpoint{6.292758in}{8.599928in}}%
\pgfpathcurveto{\pgfqpoint{6.281707in}{8.599928in}}{\pgfqpoint{6.271108in}{8.595538in}}{\pgfqpoint{6.263295in}{8.587724in}}%
\pgfpathcurveto{\pgfqpoint{6.255481in}{8.579911in}}{\pgfqpoint{6.251091in}{8.569312in}}{\pgfqpoint{6.251091in}{8.558262in}}%
\pgfpathcurveto{\pgfqpoint{6.251091in}{8.547212in}}{\pgfqpoint{6.255481in}{8.536612in}}{\pgfqpoint{6.263295in}{8.528799in}}%
\pgfpathcurveto{\pgfqpoint{6.271108in}{8.520985in}}{\pgfqpoint{6.281707in}{8.516595in}}{\pgfqpoint{6.292758in}{8.516595in}}%
\pgfpathlineto{\pgfqpoint{6.292758in}{8.516595in}}%
\pgfpathclose%
\pgfusepath{stroke,fill}%
\end{pgfscope}%
\begin{pgfscope}%
\pgfpathrectangle{\pgfqpoint{5.292946in}{7.624184in}}{\pgfqpoint{2.177280in}{2.201755in}}%
\pgfusepath{clip}%
\pgfsetbuttcap%
\pgfsetroundjoin%
\definecolor{currentfill}{rgb}{1.000000,0.498039,0.054902}%
\pgfsetfillcolor{currentfill}%
\pgfsetlinewidth{0.481800pt}%
\definecolor{currentstroke}{rgb}{1.000000,1.000000,1.000000}%
\pgfsetstrokecolor{currentstroke}%
\pgfsetdash{}{0pt}%
\pgfpathmoveto{\pgfqpoint{6.637221in}{8.627795in}}%
\pgfpathcurveto{\pgfqpoint{6.648271in}{8.627795in}}{\pgfqpoint{6.658870in}{8.632185in}}{\pgfqpoint{6.666684in}{8.639999in}}%
\pgfpathcurveto{\pgfqpoint{6.674497in}{8.647812in}}{\pgfqpoint{6.678888in}{8.658411in}}{\pgfqpoint{6.678888in}{8.669461in}}%
\pgfpathcurveto{\pgfqpoint{6.678888in}{8.680512in}}{\pgfqpoint{6.674497in}{8.691111in}}{\pgfqpoint{6.666684in}{8.698924in}}%
\pgfpathcurveto{\pgfqpoint{6.658870in}{8.706738in}}{\pgfqpoint{6.648271in}{8.711128in}}{\pgfqpoint{6.637221in}{8.711128in}}%
\pgfpathcurveto{\pgfqpoint{6.626171in}{8.711128in}}{\pgfqpoint{6.615572in}{8.706738in}}{\pgfqpoint{6.607758in}{8.698924in}}%
\pgfpathcurveto{\pgfqpoint{6.599945in}{8.691111in}}{\pgfqpoint{6.595554in}{8.680512in}}{\pgfqpoint{6.595554in}{8.669461in}}%
\pgfpathcurveto{\pgfqpoint{6.595554in}{8.658411in}}{\pgfqpoint{6.599945in}{8.647812in}}{\pgfqpoint{6.607758in}{8.639999in}}%
\pgfpathcurveto{\pgfqpoint{6.615572in}{8.632185in}}{\pgfqpoint{6.626171in}{8.627795in}}{\pgfqpoint{6.637221in}{8.627795in}}%
\pgfpathlineto{\pgfqpoint{6.637221in}{8.627795in}}%
\pgfpathclose%
\pgfusepath{stroke,fill}%
\end{pgfscope}%
\begin{pgfscope}%
\pgfpathrectangle{\pgfqpoint{5.292946in}{7.624184in}}{\pgfqpoint{2.177280in}{2.201755in}}%
\pgfusepath{clip}%
\pgfsetbuttcap%
\pgfsetroundjoin%
\definecolor{currentfill}{rgb}{1.000000,0.498039,0.054902}%
\pgfsetfillcolor{currentfill}%
\pgfsetlinewidth{0.481800pt}%
\definecolor{currentstroke}{rgb}{1.000000,1.000000,1.000000}%
\pgfsetstrokecolor{currentstroke}%
\pgfsetdash{}{0pt}%
\pgfpathmoveto{\pgfqpoint{6.464989in}{8.294195in}}%
\pgfpathcurveto{\pgfqpoint{6.476039in}{8.294195in}}{\pgfqpoint{6.486638in}{8.298586in}}{\pgfqpoint{6.494452in}{8.306399in}}%
\pgfpathcurveto{\pgfqpoint{6.502266in}{8.314213in}}{\pgfqpoint{6.506656in}{8.324812in}}{\pgfqpoint{6.506656in}{8.335862in}}%
\pgfpathcurveto{\pgfqpoint{6.506656in}{8.346912in}}{\pgfqpoint{6.502266in}{8.357511in}}{\pgfqpoint{6.494452in}{8.365325in}}%
\pgfpathcurveto{\pgfqpoint{6.486638in}{8.373139in}}{\pgfqpoint{6.476039in}{8.377529in}}{\pgfqpoint{6.464989in}{8.377529in}}%
\pgfpathcurveto{\pgfqpoint{6.453939in}{8.377529in}}{\pgfqpoint{6.443340in}{8.373139in}}{\pgfqpoint{6.435527in}{8.365325in}}%
\pgfpathcurveto{\pgfqpoint{6.427713in}{8.357511in}}{\pgfqpoint{6.423323in}{8.346912in}}{\pgfqpoint{6.423323in}{8.335862in}}%
\pgfpathcurveto{\pgfqpoint{6.423323in}{8.324812in}}{\pgfqpoint{6.427713in}{8.314213in}}{\pgfqpoint{6.435527in}{8.306399in}}%
\pgfpathcurveto{\pgfqpoint{6.443340in}{8.298586in}}{\pgfqpoint{6.453939in}{8.294195in}}{\pgfqpoint{6.464989in}{8.294195in}}%
\pgfpathlineto{\pgfqpoint{6.464989in}{8.294195in}}%
\pgfpathclose%
\pgfusepath{stroke,fill}%
\end{pgfscope}%
\begin{pgfscope}%
\pgfpathrectangle{\pgfqpoint{5.292946in}{7.624184in}}{\pgfqpoint{2.177280in}{2.201755in}}%
\pgfusepath{clip}%
\pgfsetbuttcap%
\pgfsetroundjoin%
\definecolor{currentfill}{rgb}{1.000000,0.498039,0.054902}%
\pgfsetfillcolor{currentfill}%
\pgfsetlinewidth{0.481800pt}%
\definecolor{currentstroke}{rgb}{1.000000,1.000000,1.000000}%
\pgfsetstrokecolor{currentstroke}%
\pgfsetdash{}{0pt}%
\pgfpathmoveto{\pgfqpoint{6.464989in}{8.627795in}}%
\pgfpathcurveto{\pgfqpoint{6.476039in}{8.627795in}}{\pgfqpoint{6.486638in}{8.632185in}}{\pgfqpoint{6.494452in}{8.639999in}}%
\pgfpathcurveto{\pgfqpoint{6.502266in}{8.647812in}}{\pgfqpoint{6.506656in}{8.658411in}}{\pgfqpoint{6.506656in}{8.669461in}}%
\pgfpathcurveto{\pgfqpoint{6.506656in}{8.680512in}}{\pgfqpoint{6.502266in}{8.691111in}}{\pgfqpoint{6.494452in}{8.698924in}}%
\pgfpathcurveto{\pgfqpoint{6.486638in}{8.706738in}}{\pgfqpoint{6.476039in}{8.711128in}}{\pgfqpoint{6.464989in}{8.711128in}}%
\pgfpathcurveto{\pgfqpoint{6.453939in}{8.711128in}}{\pgfqpoint{6.443340in}{8.706738in}}{\pgfqpoint{6.435527in}{8.698924in}}%
\pgfpathcurveto{\pgfqpoint{6.427713in}{8.691111in}}{\pgfqpoint{6.423323in}{8.680512in}}{\pgfqpoint{6.423323in}{8.669461in}}%
\pgfpathcurveto{\pgfqpoint{6.423323in}{8.658411in}}{\pgfqpoint{6.427713in}{8.647812in}}{\pgfqpoint{6.435527in}{8.639999in}}%
\pgfpathcurveto{\pgfqpoint{6.443340in}{8.632185in}}{\pgfqpoint{6.453939in}{8.627795in}}{\pgfqpoint{6.464989in}{8.627795in}}%
\pgfpathlineto{\pgfqpoint{6.464989in}{8.627795in}}%
\pgfpathclose%
\pgfusepath{stroke,fill}%
\end{pgfscope}%
\begin{pgfscope}%
\pgfpathrectangle{\pgfqpoint{5.292946in}{7.624184in}}{\pgfqpoint{2.177280in}{2.201755in}}%
\pgfusepath{clip}%
\pgfsetbuttcap%
\pgfsetroundjoin%
\definecolor{currentfill}{rgb}{1.000000,0.498039,0.054902}%
\pgfsetfillcolor{currentfill}%
\pgfsetlinewidth{0.481800pt}%
\definecolor{currentstroke}{rgb}{1.000000,1.000000,1.000000}%
\pgfsetstrokecolor{currentstroke}%
\pgfsetdash{}{0pt}%
\pgfpathmoveto{\pgfqpoint{6.522400in}{9.016994in}}%
\pgfpathcurveto{\pgfqpoint{6.533450in}{9.016994in}}{\pgfqpoint{6.544049in}{9.021384in}}{\pgfqpoint{6.551863in}{9.029198in}}%
\pgfpathcurveto{\pgfqpoint{6.559676in}{9.037011in}}{\pgfqpoint{6.564067in}{9.047610in}}{\pgfqpoint{6.564067in}{9.058661in}}%
\pgfpathcurveto{\pgfqpoint{6.564067in}{9.069711in}}{\pgfqpoint{6.559676in}{9.080310in}}{\pgfqpoint{6.551863in}{9.088123in}}%
\pgfpathcurveto{\pgfqpoint{6.544049in}{9.095937in}}{\pgfqpoint{6.533450in}{9.100327in}}{\pgfqpoint{6.522400in}{9.100327in}}%
\pgfpathcurveto{\pgfqpoint{6.511350in}{9.100327in}}{\pgfqpoint{6.500751in}{9.095937in}}{\pgfqpoint{6.492937in}{9.088123in}}%
\pgfpathcurveto{\pgfqpoint{6.485123in}{9.080310in}}{\pgfqpoint{6.480733in}{9.069711in}}{\pgfqpoint{6.480733in}{9.058661in}}%
\pgfpathcurveto{\pgfqpoint{6.480733in}{9.047610in}}{\pgfqpoint{6.485123in}{9.037011in}}{\pgfqpoint{6.492937in}{9.029198in}}%
\pgfpathcurveto{\pgfqpoint{6.500751in}{9.021384in}}{\pgfqpoint{6.511350in}{9.016994in}}{\pgfqpoint{6.522400in}{9.016994in}}%
\pgfpathlineto{\pgfqpoint{6.522400in}{9.016994in}}%
\pgfpathclose%
\pgfusepath{stroke,fill}%
\end{pgfscope}%
\begin{pgfscope}%
\pgfpathrectangle{\pgfqpoint{5.292946in}{7.624184in}}{\pgfqpoint{2.177280in}{2.201755in}}%
\pgfusepath{clip}%
\pgfsetbuttcap%
\pgfsetroundjoin%
\definecolor{currentfill}{rgb}{1.000000,0.498039,0.054902}%
\pgfsetfillcolor{currentfill}%
\pgfsetlinewidth{0.481800pt}%
\definecolor{currentstroke}{rgb}{1.000000,1.000000,1.000000}%
\pgfsetstrokecolor{currentstroke}%
\pgfsetdash{}{0pt}%
\pgfpathmoveto{\pgfqpoint{6.436284in}{8.794594in}}%
\pgfpathcurveto{\pgfqpoint{6.447334in}{8.794594in}}{\pgfqpoint{6.457933in}{8.798985in}}{\pgfqpoint{6.465747in}{8.806798in}}%
\pgfpathcurveto{\pgfqpoint{6.473560in}{8.814612in}}{\pgfqpoint{6.477951in}{8.825211in}}{\pgfqpoint{6.477951in}{8.836261in}}%
\pgfpathcurveto{\pgfqpoint{6.477951in}{8.847311in}}{\pgfqpoint{6.473560in}{8.857910in}}{\pgfqpoint{6.465747in}{8.865724in}}%
\pgfpathcurveto{\pgfqpoint{6.457933in}{8.873537in}}{\pgfqpoint{6.447334in}{8.877928in}}{\pgfqpoint{6.436284in}{8.877928in}}%
\pgfpathcurveto{\pgfqpoint{6.425234in}{8.877928in}}{\pgfqpoint{6.414635in}{8.873537in}}{\pgfqpoint{6.406821in}{8.865724in}}%
\pgfpathcurveto{\pgfqpoint{6.399008in}{8.857910in}}{\pgfqpoint{6.394617in}{8.847311in}}{\pgfqpoint{6.394617in}{8.836261in}}%
\pgfpathcurveto{\pgfqpoint{6.394617in}{8.825211in}}{\pgfqpoint{6.399008in}{8.814612in}}{\pgfqpoint{6.406821in}{8.806798in}}%
\pgfpathcurveto{\pgfqpoint{6.414635in}{8.798985in}}{\pgfqpoint{6.425234in}{8.794594in}}{\pgfqpoint{6.436284in}{8.794594in}}%
\pgfpathlineto{\pgfqpoint{6.436284in}{8.794594in}}%
\pgfpathclose%
\pgfusepath{stroke,fill}%
\end{pgfscope}%
\begin{pgfscope}%
\pgfpathrectangle{\pgfqpoint{5.292946in}{7.624184in}}{\pgfqpoint{2.177280in}{2.201755in}}%
\pgfusepath{clip}%
\pgfsetbuttcap%
\pgfsetroundjoin%
\definecolor{currentfill}{rgb}{1.000000,0.498039,0.054902}%
\pgfsetfillcolor{currentfill}%
\pgfsetlinewidth{0.481800pt}%
\definecolor{currentstroke}{rgb}{1.000000,1.000000,1.000000}%
\pgfsetstrokecolor{currentstroke}%
\pgfsetdash{}{0pt}%
\pgfpathmoveto{\pgfqpoint{6.350168in}{8.405395in}}%
\pgfpathcurveto{\pgfqpoint{6.361218in}{8.405395in}}{\pgfqpoint{6.371817in}{8.409785in}}{\pgfqpoint{6.379631in}{8.417599in}}%
\pgfpathcurveto{\pgfqpoint{6.387445in}{8.425413in}}{\pgfqpoint{6.391835in}{8.436012in}}{\pgfqpoint{6.391835in}{8.447062in}}%
\pgfpathcurveto{\pgfqpoint{6.391835in}{8.458112in}}{\pgfqpoint{6.387445in}{8.468711in}}{\pgfqpoint{6.379631in}{8.476525in}}%
\pgfpathcurveto{\pgfqpoint{6.371817in}{8.484338in}}{\pgfqpoint{6.361218in}{8.488729in}}{\pgfqpoint{6.350168in}{8.488729in}}%
\pgfpathcurveto{\pgfqpoint{6.339118in}{8.488729in}}{\pgfqpoint{6.328519in}{8.484338in}}{\pgfqpoint{6.320705in}{8.476525in}}%
\pgfpathcurveto{\pgfqpoint{6.312892in}{8.468711in}}{\pgfqpoint{6.308501in}{8.458112in}}{\pgfqpoint{6.308501in}{8.447062in}}%
\pgfpathcurveto{\pgfqpoint{6.308501in}{8.436012in}}{\pgfqpoint{6.312892in}{8.425413in}}{\pgfqpoint{6.320705in}{8.417599in}}%
\pgfpathcurveto{\pgfqpoint{6.328519in}{8.409785in}}{\pgfqpoint{6.339118in}{8.405395in}}{\pgfqpoint{6.350168in}{8.405395in}}%
\pgfpathlineto{\pgfqpoint{6.350168in}{8.405395in}}%
\pgfpathclose%
\pgfusepath{stroke,fill}%
\end{pgfscope}%
\begin{pgfscope}%
\pgfpathrectangle{\pgfqpoint{5.292946in}{7.624184in}}{\pgfqpoint{2.177280in}{2.201755in}}%
\pgfusepath{clip}%
\pgfsetbuttcap%
\pgfsetroundjoin%
\definecolor{currentfill}{rgb}{1.000000,0.498039,0.054902}%
\pgfsetfillcolor{currentfill}%
\pgfsetlinewidth{0.481800pt}%
\definecolor{currentstroke}{rgb}{1.000000,1.000000,1.000000}%
\pgfsetstrokecolor{currentstroke}%
\pgfsetdash{}{0pt}%
\pgfpathmoveto{\pgfqpoint{6.321463in}{8.349795in}}%
\pgfpathcurveto{\pgfqpoint{6.332513in}{8.349795in}}{\pgfqpoint{6.343112in}{8.354186in}}{\pgfqpoint{6.350926in}{8.361999in}}%
\pgfpathcurveto{\pgfqpoint{6.358739in}{8.369813in}}{\pgfqpoint{6.363130in}{8.380412in}}{\pgfqpoint{6.363130in}{8.391462in}}%
\pgfpathcurveto{\pgfqpoint{6.363130in}{8.402512in}}{\pgfqpoint{6.358739in}{8.413111in}}{\pgfqpoint{6.350926in}{8.420925in}}%
\pgfpathcurveto{\pgfqpoint{6.343112in}{8.428738in}}{\pgfqpoint{6.332513in}{8.433129in}}{\pgfqpoint{6.321463in}{8.433129in}}%
\pgfpathcurveto{\pgfqpoint{6.310413in}{8.433129in}}{\pgfqpoint{6.299814in}{8.428738in}}{\pgfqpoint{6.292000in}{8.420925in}}%
\pgfpathcurveto{\pgfqpoint{6.284186in}{8.413111in}}{\pgfqpoint{6.279796in}{8.402512in}}{\pgfqpoint{6.279796in}{8.391462in}}%
\pgfpathcurveto{\pgfqpoint{6.279796in}{8.380412in}}{\pgfqpoint{6.284186in}{8.369813in}}{\pgfqpoint{6.292000in}{8.361999in}}%
\pgfpathcurveto{\pgfqpoint{6.299814in}{8.354186in}}{\pgfqpoint{6.310413in}{8.349795in}}{\pgfqpoint{6.321463in}{8.349795in}}%
\pgfpathlineto{\pgfqpoint{6.321463in}{8.349795in}}%
\pgfpathclose%
\pgfusepath{stroke,fill}%
\end{pgfscope}%
\begin{pgfscope}%
\pgfpathrectangle{\pgfqpoint{5.292946in}{7.624184in}}{\pgfqpoint{2.177280in}{2.201755in}}%
\pgfusepath{clip}%
\pgfsetbuttcap%
\pgfsetroundjoin%
\definecolor{currentfill}{rgb}{1.000000,0.498039,0.054902}%
\pgfsetfillcolor{currentfill}%
\pgfsetlinewidth{0.481800pt}%
\definecolor{currentstroke}{rgb}{1.000000,1.000000,1.000000}%
\pgfsetstrokecolor{currentstroke}%
\pgfsetdash{}{0pt}%
\pgfpathmoveto{\pgfqpoint{6.436284in}{8.349795in}}%
\pgfpathcurveto{\pgfqpoint{6.447334in}{8.349795in}}{\pgfqpoint{6.457933in}{8.354186in}}{\pgfqpoint{6.465747in}{8.361999in}}%
\pgfpathcurveto{\pgfqpoint{6.473560in}{8.369813in}}{\pgfqpoint{6.477951in}{8.380412in}}{\pgfqpoint{6.477951in}{8.391462in}}%
\pgfpathcurveto{\pgfqpoint{6.477951in}{8.402512in}}{\pgfqpoint{6.473560in}{8.413111in}}{\pgfqpoint{6.465747in}{8.420925in}}%
\pgfpathcurveto{\pgfqpoint{6.457933in}{8.428738in}}{\pgfqpoint{6.447334in}{8.433129in}}{\pgfqpoint{6.436284in}{8.433129in}}%
\pgfpathcurveto{\pgfqpoint{6.425234in}{8.433129in}}{\pgfqpoint{6.414635in}{8.428738in}}{\pgfqpoint{6.406821in}{8.420925in}}%
\pgfpathcurveto{\pgfqpoint{6.399008in}{8.413111in}}{\pgfqpoint{6.394617in}{8.402512in}}{\pgfqpoint{6.394617in}{8.391462in}}%
\pgfpathcurveto{\pgfqpoint{6.394617in}{8.380412in}}{\pgfqpoint{6.399008in}{8.369813in}}{\pgfqpoint{6.406821in}{8.361999in}}%
\pgfpathcurveto{\pgfqpoint{6.414635in}{8.354186in}}{\pgfqpoint{6.425234in}{8.349795in}}{\pgfqpoint{6.436284in}{8.349795in}}%
\pgfpathlineto{\pgfqpoint{6.436284in}{8.349795in}}%
\pgfpathclose%
\pgfusepath{stroke,fill}%
\end{pgfscope}%
\begin{pgfscope}%
\pgfpathrectangle{\pgfqpoint{5.292946in}{7.624184in}}{\pgfqpoint{2.177280in}{2.201755in}}%
\pgfusepath{clip}%
\pgfsetbuttcap%
\pgfsetroundjoin%
\definecolor{currentfill}{rgb}{1.000000,0.498039,0.054902}%
\pgfsetfillcolor{currentfill}%
\pgfsetlinewidth{0.481800pt}%
\definecolor{currentstroke}{rgb}{1.000000,1.000000,1.000000}%
\pgfsetstrokecolor{currentstroke}%
\pgfsetdash{}{0pt}%
\pgfpathmoveto{\pgfqpoint{6.493695in}{8.683395in}}%
\pgfpathcurveto{\pgfqpoint{6.504745in}{8.683395in}}{\pgfqpoint{6.515344in}{8.687785in}}{\pgfqpoint{6.523157in}{8.695598in}}%
\pgfpathcurveto{\pgfqpoint{6.530971in}{8.703412in}}{\pgfqpoint{6.535361in}{8.714011in}}{\pgfqpoint{6.535361in}{8.725061in}}%
\pgfpathcurveto{\pgfqpoint{6.535361in}{8.736111in}}{\pgfqpoint{6.530971in}{8.746710in}}{\pgfqpoint{6.523157in}{8.754524in}}%
\pgfpathcurveto{\pgfqpoint{6.515344in}{8.762338in}}{\pgfqpoint{6.504745in}{8.766728in}}{\pgfqpoint{6.493695in}{8.766728in}}%
\pgfpathcurveto{\pgfqpoint{6.482644in}{8.766728in}}{\pgfqpoint{6.472045in}{8.762338in}}{\pgfqpoint{6.464232in}{8.754524in}}%
\pgfpathcurveto{\pgfqpoint{6.456418in}{8.746710in}}{\pgfqpoint{6.452028in}{8.736111in}}{\pgfqpoint{6.452028in}{8.725061in}}%
\pgfpathcurveto{\pgfqpoint{6.452028in}{8.714011in}}{\pgfqpoint{6.456418in}{8.703412in}}{\pgfqpoint{6.464232in}{8.695598in}}%
\pgfpathcurveto{\pgfqpoint{6.472045in}{8.687785in}}{\pgfqpoint{6.482644in}{8.683395in}}{\pgfqpoint{6.493695in}{8.683395in}}%
\pgfpathlineto{\pgfqpoint{6.493695in}{8.683395in}}%
\pgfpathclose%
\pgfusepath{stroke,fill}%
\end{pgfscope}%
\begin{pgfscope}%
\pgfpathrectangle{\pgfqpoint{5.292946in}{7.624184in}}{\pgfqpoint{2.177280in}{2.201755in}}%
\pgfusepath{clip}%
\pgfsetbuttcap%
\pgfsetroundjoin%
\definecolor{currentfill}{rgb}{1.000000,0.498039,0.054902}%
\pgfsetfillcolor{currentfill}%
\pgfsetlinewidth{0.481800pt}%
\definecolor{currentstroke}{rgb}{1.000000,1.000000,1.000000}%
\pgfsetstrokecolor{currentstroke}%
\pgfsetdash{}{0pt}%
\pgfpathmoveto{\pgfqpoint{6.321463in}{8.516595in}}%
\pgfpathcurveto{\pgfqpoint{6.332513in}{8.516595in}}{\pgfqpoint{6.343112in}{8.520985in}}{\pgfqpoint{6.350926in}{8.528799in}}%
\pgfpathcurveto{\pgfqpoint{6.358739in}{8.536612in}}{\pgfqpoint{6.363130in}{8.547212in}}{\pgfqpoint{6.363130in}{8.558262in}}%
\pgfpathcurveto{\pgfqpoint{6.363130in}{8.569312in}}{\pgfqpoint{6.358739in}{8.579911in}}{\pgfqpoint{6.350926in}{8.587724in}}%
\pgfpathcurveto{\pgfqpoint{6.343112in}{8.595538in}}{\pgfqpoint{6.332513in}{8.599928in}}{\pgfqpoint{6.321463in}{8.599928in}}%
\pgfpathcurveto{\pgfqpoint{6.310413in}{8.599928in}}{\pgfqpoint{6.299814in}{8.595538in}}{\pgfqpoint{6.292000in}{8.587724in}}%
\pgfpathcurveto{\pgfqpoint{6.284186in}{8.579911in}}{\pgfqpoint{6.279796in}{8.569312in}}{\pgfqpoint{6.279796in}{8.558262in}}%
\pgfpathcurveto{\pgfqpoint{6.279796in}{8.547212in}}{\pgfqpoint{6.284186in}{8.536612in}}{\pgfqpoint{6.292000in}{8.528799in}}%
\pgfpathcurveto{\pgfqpoint{6.299814in}{8.520985in}}{\pgfqpoint{6.310413in}{8.516595in}}{\pgfqpoint{6.321463in}{8.516595in}}%
\pgfpathlineto{\pgfqpoint{6.321463in}{8.516595in}}%
\pgfpathclose%
\pgfusepath{stroke,fill}%
\end{pgfscope}%
\begin{pgfscope}%
\pgfpathrectangle{\pgfqpoint{5.292946in}{7.624184in}}{\pgfqpoint{2.177280in}{2.201755in}}%
\pgfusepath{clip}%
\pgfsetbuttcap%
\pgfsetroundjoin%
\definecolor{currentfill}{rgb}{1.000000,0.498039,0.054902}%
\pgfsetfillcolor{currentfill}%
\pgfsetlinewidth{0.481800pt}%
\definecolor{currentstroke}{rgb}{1.000000,1.000000,1.000000}%
\pgfsetstrokecolor{currentstroke}%
\pgfsetdash{}{0pt}%
\pgfpathmoveto{\pgfqpoint{6.120526in}{8.071796in}}%
\pgfpathcurveto{\pgfqpoint{6.131576in}{8.071796in}}{\pgfqpoint{6.142175in}{8.076186in}}{\pgfqpoint{6.149989in}{8.084000in}}%
\pgfpathcurveto{\pgfqpoint{6.157802in}{8.091813in}}{\pgfqpoint{6.162193in}{8.102413in}}{\pgfqpoint{6.162193in}{8.113463in}}%
\pgfpathcurveto{\pgfqpoint{6.162193in}{8.124513in}}{\pgfqpoint{6.157802in}{8.135112in}}{\pgfqpoint{6.149989in}{8.142925in}}%
\pgfpathcurveto{\pgfqpoint{6.142175in}{8.150739in}}{\pgfqpoint{6.131576in}{8.155129in}}{\pgfqpoint{6.120526in}{8.155129in}}%
\pgfpathcurveto{\pgfqpoint{6.109476in}{8.155129in}}{\pgfqpoint{6.098877in}{8.150739in}}{\pgfqpoint{6.091063in}{8.142925in}}%
\pgfpathcurveto{\pgfqpoint{6.083249in}{8.135112in}}{\pgfqpoint{6.078859in}{8.124513in}}{\pgfqpoint{6.078859in}{8.113463in}}%
\pgfpathcurveto{\pgfqpoint{6.078859in}{8.102413in}}{\pgfqpoint{6.083249in}{8.091813in}}{\pgfqpoint{6.091063in}{8.084000in}}%
\pgfpathcurveto{\pgfqpoint{6.098877in}{8.076186in}}{\pgfqpoint{6.109476in}{8.071796in}}{\pgfqpoint{6.120526in}{8.071796in}}%
\pgfpathlineto{\pgfqpoint{6.120526in}{8.071796in}}%
\pgfpathclose%
\pgfusepath{stroke,fill}%
\end{pgfscope}%
\begin{pgfscope}%
\pgfpathrectangle{\pgfqpoint{5.292946in}{7.624184in}}{\pgfqpoint{2.177280in}{2.201755in}}%
\pgfusepath{clip}%
\pgfsetbuttcap%
\pgfsetroundjoin%
\definecolor{currentfill}{rgb}{1.000000,0.498039,0.054902}%
\pgfsetfillcolor{currentfill}%
\pgfsetlinewidth{0.481800pt}%
\definecolor{currentstroke}{rgb}{1.000000,1.000000,1.000000}%
\pgfsetstrokecolor{currentstroke}%
\pgfsetdash{}{0pt}%
\pgfpathmoveto{\pgfqpoint{6.378873in}{8.405395in}}%
\pgfpathcurveto{\pgfqpoint{6.389924in}{8.405395in}}{\pgfqpoint{6.400523in}{8.409785in}}{\pgfqpoint{6.408336in}{8.417599in}}%
\pgfpathcurveto{\pgfqpoint{6.416150in}{8.425413in}}{\pgfqpoint{6.420540in}{8.436012in}}{\pgfqpoint{6.420540in}{8.447062in}}%
\pgfpathcurveto{\pgfqpoint{6.420540in}{8.458112in}}{\pgfqpoint{6.416150in}{8.468711in}}{\pgfqpoint{6.408336in}{8.476525in}}%
\pgfpathcurveto{\pgfqpoint{6.400523in}{8.484338in}}{\pgfqpoint{6.389924in}{8.488729in}}{\pgfqpoint{6.378873in}{8.488729in}}%
\pgfpathcurveto{\pgfqpoint{6.367823in}{8.488729in}}{\pgfqpoint{6.357224in}{8.484338in}}{\pgfqpoint{6.349411in}{8.476525in}}%
\pgfpathcurveto{\pgfqpoint{6.341597in}{8.468711in}}{\pgfqpoint{6.337207in}{8.458112in}}{\pgfqpoint{6.337207in}{8.447062in}}%
\pgfpathcurveto{\pgfqpoint{6.337207in}{8.436012in}}{\pgfqpoint{6.341597in}{8.425413in}}{\pgfqpoint{6.349411in}{8.417599in}}%
\pgfpathcurveto{\pgfqpoint{6.357224in}{8.409785in}}{\pgfqpoint{6.367823in}{8.405395in}}{\pgfqpoint{6.378873in}{8.405395in}}%
\pgfpathlineto{\pgfqpoint{6.378873in}{8.405395in}}%
\pgfpathclose%
\pgfusepath{stroke,fill}%
\end{pgfscope}%
\begin{pgfscope}%
\pgfpathrectangle{\pgfqpoint{5.292946in}{7.624184in}}{\pgfqpoint{2.177280in}{2.201755in}}%
\pgfusepath{clip}%
\pgfsetbuttcap%
\pgfsetroundjoin%
\definecolor{currentfill}{rgb}{1.000000,0.498039,0.054902}%
\pgfsetfillcolor{currentfill}%
\pgfsetlinewidth{0.481800pt}%
\definecolor{currentstroke}{rgb}{1.000000,1.000000,1.000000}%
\pgfsetstrokecolor{currentstroke}%
\pgfsetdash{}{0pt}%
\pgfpathmoveto{\pgfqpoint{6.378873in}{8.460995in}}%
\pgfpathcurveto{\pgfqpoint{6.389924in}{8.460995in}}{\pgfqpoint{6.400523in}{8.465385in}}{\pgfqpoint{6.408336in}{8.473199in}}%
\pgfpathcurveto{\pgfqpoint{6.416150in}{8.481013in}}{\pgfqpoint{6.420540in}{8.491612in}}{\pgfqpoint{6.420540in}{8.502662in}}%
\pgfpathcurveto{\pgfqpoint{6.420540in}{8.513712in}}{\pgfqpoint{6.416150in}{8.524311in}}{\pgfqpoint{6.408336in}{8.532125in}}%
\pgfpathcurveto{\pgfqpoint{6.400523in}{8.539938in}}{\pgfqpoint{6.389924in}{8.544328in}}{\pgfqpoint{6.378873in}{8.544328in}}%
\pgfpathcurveto{\pgfqpoint{6.367823in}{8.544328in}}{\pgfqpoint{6.357224in}{8.539938in}}{\pgfqpoint{6.349411in}{8.532125in}}%
\pgfpathcurveto{\pgfqpoint{6.341597in}{8.524311in}}{\pgfqpoint{6.337207in}{8.513712in}}{\pgfqpoint{6.337207in}{8.502662in}}%
\pgfpathcurveto{\pgfqpoint{6.337207in}{8.491612in}}{\pgfqpoint{6.341597in}{8.481013in}}{\pgfqpoint{6.349411in}{8.473199in}}%
\pgfpathcurveto{\pgfqpoint{6.357224in}{8.465385in}}{\pgfqpoint{6.367823in}{8.460995in}}{\pgfqpoint{6.378873in}{8.460995in}}%
\pgfpathlineto{\pgfqpoint{6.378873in}{8.460995in}}%
\pgfpathclose%
\pgfusepath{stroke,fill}%
\end{pgfscope}%
\begin{pgfscope}%
\pgfpathrectangle{\pgfqpoint{5.292946in}{7.624184in}}{\pgfqpoint{2.177280in}{2.201755in}}%
\pgfusepath{clip}%
\pgfsetbuttcap%
\pgfsetroundjoin%
\definecolor{currentfill}{rgb}{1.000000,0.498039,0.054902}%
\pgfsetfillcolor{currentfill}%
\pgfsetlinewidth{0.481800pt}%
\definecolor{currentstroke}{rgb}{1.000000,1.000000,1.000000}%
\pgfsetstrokecolor{currentstroke}%
\pgfsetdash{}{0pt}%
\pgfpathmoveto{\pgfqpoint{6.378873in}{8.460995in}}%
\pgfpathcurveto{\pgfqpoint{6.389924in}{8.460995in}}{\pgfqpoint{6.400523in}{8.465385in}}{\pgfqpoint{6.408336in}{8.473199in}}%
\pgfpathcurveto{\pgfqpoint{6.416150in}{8.481013in}}{\pgfqpoint{6.420540in}{8.491612in}}{\pgfqpoint{6.420540in}{8.502662in}}%
\pgfpathcurveto{\pgfqpoint{6.420540in}{8.513712in}}{\pgfqpoint{6.416150in}{8.524311in}}{\pgfqpoint{6.408336in}{8.532125in}}%
\pgfpathcurveto{\pgfqpoint{6.400523in}{8.539938in}}{\pgfqpoint{6.389924in}{8.544328in}}{\pgfqpoint{6.378873in}{8.544328in}}%
\pgfpathcurveto{\pgfqpoint{6.367823in}{8.544328in}}{\pgfqpoint{6.357224in}{8.539938in}}{\pgfqpoint{6.349411in}{8.532125in}}%
\pgfpathcurveto{\pgfqpoint{6.341597in}{8.524311in}}{\pgfqpoint{6.337207in}{8.513712in}}{\pgfqpoint{6.337207in}{8.502662in}}%
\pgfpathcurveto{\pgfqpoint{6.337207in}{8.491612in}}{\pgfqpoint{6.341597in}{8.481013in}}{\pgfqpoint{6.349411in}{8.473199in}}%
\pgfpathcurveto{\pgfqpoint{6.357224in}{8.465385in}}{\pgfqpoint{6.367823in}{8.460995in}}{\pgfqpoint{6.378873in}{8.460995in}}%
\pgfpathlineto{\pgfqpoint{6.378873in}{8.460995in}}%
\pgfpathclose%
\pgfusepath{stroke,fill}%
\end{pgfscope}%
\begin{pgfscope}%
\pgfpathrectangle{\pgfqpoint{5.292946in}{7.624184in}}{\pgfqpoint{2.177280in}{2.201755in}}%
\pgfusepath{clip}%
\pgfsetbuttcap%
\pgfsetroundjoin%
\definecolor{currentfill}{rgb}{1.000000,0.498039,0.054902}%
\pgfsetfillcolor{currentfill}%
\pgfsetlinewidth{0.481800pt}%
\definecolor{currentstroke}{rgb}{1.000000,1.000000,1.000000}%
\pgfsetstrokecolor{currentstroke}%
\pgfsetdash{}{0pt}%
\pgfpathmoveto{\pgfqpoint{6.407579in}{8.738994in}}%
\pgfpathcurveto{\pgfqpoint{6.418629in}{8.738994in}}{\pgfqpoint{6.429228in}{8.743385in}}{\pgfqpoint{6.437041in}{8.751198in}}%
\pgfpathcurveto{\pgfqpoint{6.444855in}{8.759012in}}{\pgfqpoint{6.449245in}{8.769611in}}{\pgfqpoint{6.449245in}{8.780661in}}%
\pgfpathcurveto{\pgfqpoint{6.449245in}{8.791711in}}{\pgfqpoint{6.444855in}{8.802310in}}{\pgfqpoint{6.437041in}{8.810124in}}%
\pgfpathcurveto{\pgfqpoint{6.429228in}{8.817938in}}{\pgfqpoint{6.418629in}{8.822328in}}{\pgfqpoint{6.407579in}{8.822328in}}%
\pgfpathcurveto{\pgfqpoint{6.396529in}{8.822328in}}{\pgfqpoint{6.385930in}{8.817938in}}{\pgfqpoint{6.378116in}{8.810124in}}%
\pgfpathcurveto{\pgfqpoint{6.370302in}{8.802310in}}{\pgfqpoint{6.365912in}{8.791711in}}{\pgfqpoint{6.365912in}{8.780661in}}%
\pgfpathcurveto{\pgfqpoint{6.365912in}{8.769611in}}{\pgfqpoint{6.370302in}{8.759012in}}{\pgfqpoint{6.378116in}{8.751198in}}%
\pgfpathcurveto{\pgfqpoint{6.385930in}{8.743385in}}{\pgfqpoint{6.396529in}{8.738994in}}{\pgfqpoint{6.407579in}{8.738994in}}%
\pgfpathlineto{\pgfqpoint{6.407579in}{8.738994in}}%
\pgfpathclose%
\pgfusepath{stroke,fill}%
\end{pgfscope}%
\begin{pgfscope}%
\pgfpathrectangle{\pgfqpoint{5.292946in}{7.624184in}}{\pgfqpoint{2.177280in}{2.201755in}}%
\pgfusepath{clip}%
\pgfsetbuttcap%
\pgfsetroundjoin%
\definecolor{currentfill}{rgb}{1.000000,0.498039,0.054902}%
\pgfsetfillcolor{currentfill}%
\pgfsetlinewidth{0.481800pt}%
\definecolor{currentstroke}{rgb}{1.000000,1.000000,1.000000}%
\pgfsetstrokecolor{currentstroke}%
\pgfsetdash{}{0pt}%
\pgfpathmoveto{\pgfqpoint{6.034410in}{8.127396in}}%
\pgfpathcurveto{\pgfqpoint{6.045460in}{8.127396in}}{\pgfqpoint{6.056059in}{8.131786in}}{\pgfqpoint{6.063873in}{8.139600in}}%
\pgfpathcurveto{\pgfqpoint{6.071686in}{8.147413in}}{\pgfqpoint{6.076077in}{8.158012in}}{\pgfqpoint{6.076077in}{8.169063in}}%
\pgfpathcurveto{\pgfqpoint{6.076077in}{8.180113in}}{\pgfqpoint{6.071686in}{8.190712in}}{\pgfqpoint{6.063873in}{8.198525in}}%
\pgfpathcurveto{\pgfqpoint{6.056059in}{8.206339in}}{\pgfqpoint{6.045460in}{8.210729in}}{\pgfqpoint{6.034410in}{8.210729in}}%
\pgfpathcurveto{\pgfqpoint{6.023360in}{8.210729in}}{\pgfqpoint{6.012761in}{8.206339in}}{\pgfqpoint{6.004947in}{8.198525in}}%
\pgfpathcurveto{\pgfqpoint{5.997134in}{8.190712in}}{\pgfqpoint{5.992743in}{8.180113in}}{\pgfqpoint{5.992743in}{8.169063in}}%
\pgfpathcurveto{\pgfqpoint{5.992743in}{8.158012in}}{\pgfqpoint{5.997134in}{8.147413in}}{\pgfqpoint{6.004947in}{8.139600in}}%
\pgfpathcurveto{\pgfqpoint{6.012761in}{8.131786in}}{\pgfqpoint{6.023360in}{8.127396in}}{\pgfqpoint{6.034410in}{8.127396in}}%
\pgfpathlineto{\pgfqpoint{6.034410in}{8.127396in}}%
\pgfpathclose%
\pgfusepath{stroke,fill}%
\end{pgfscope}%
\begin{pgfscope}%
\pgfpathrectangle{\pgfqpoint{5.292946in}{7.624184in}}{\pgfqpoint{2.177280in}{2.201755in}}%
\pgfusepath{clip}%
\pgfsetbuttcap%
\pgfsetroundjoin%
\definecolor{currentfill}{rgb}{1.000000,0.498039,0.054902}%
\pgfsetfillcolor{currentfill}%
\pgfsetlinewidth{0.481800pt}%
\definecolor{currentstroke}{rgb}{1.000000,1.000000,1.000000}%
\pgfsetstrokecolor{currentstroke}%
\pgfsetdash{}{0pt}%
\pgfpathmoveto{\pgfqpoint{6.350168in}{8.460995in}}%
\pgfpathcurveto{\pgfqpoint{6.361218in}{8.460995in}}{\pgfqpoint{6.371817in}{8.465385in}}{\pgfqpoint{6.379631in}{8.473199in}}%
\pgfpathcurveto{\pgfqpoint{6.387445in}{8.481013in}}{\pgfqpoint{6.391835in}{8.491612in}}{\pgfqpoint{6.391835in}{8.502662in}}%
\pgfpathcurveto{\pgfqpoint{6.391835in}{8.513712in}}{\pgfqpoint{6.387445in}{8.524311in}}{\pgfqpoint{6.379631in}{8.532125in}}%
\pgfpathcurveto{\pgfqpoint{6.371817in}{8.539938in}}{\pgfqpoint{6.361218in}{8.544328in}}{\pgfqpoint{6.350168in}{8.544328in}}%
\pgfpathcurveto{\pgfqpoint{6.339118in}{8.544328in}}{\pgfqpoint{6.328519in}{8.539938in}}{\pgfqpoint{6.320705in}{8.532125in}}%
\pgfpathcurveto{\pgfqpoint{6.312892in}{8.524311in}}{\pgfqpoint{6.308501in}{8.513712in}}{\pgfqpoint{6.308501in}{8.502662in}}%
\pgfpathcurveto{\pgfqpoint{6.308501in}{8.491612in}}{\pgfqpoint{6.312892in}{8.481013in}}{\pgfqpoint{6.320705in}{8.473199in}}%
\pgfpathcurveto{\pgfqpoint{6.328519in}{8.465385in}}{\pgfqpoint{6.339118in}{8.460995in}}{\pgfqpoint{6.350168in}{8.460995in}}%
\pgfpathlineto{\pgfqpoint{6.350168in}{8.460995in}}%
\pgfpathclose%
\pgfusepath{stroke,fill}%
\end{pgfscope}%
\begin{pgfscope}%
\pgfpathrectangle{\pgfqpoint{5.292946in}{7.624184in}}{\pgfqpoint{2.177280in}{2.201755in}}%
\pgfusepath{clip}%
\pgfsetbuttcap%
\pgfsetroundjoin%
\definecolor{currentfill}{rgb}{0.172549,0.627451,0.172549}%
\pgfsetfillcolor{currentfill}%
\pgfsetlinewidth{0.481800pt}%
\definecolor{currentstroke}{rgb}{1.000000,1.000000,1.000000}%
\pgfsetstrokecolor{currentstroke}%
\pgfsetdash{}{0pt}%
\pgfpathmoveto{\pgfqpoint{6.895569in}{8.794594in}}%
\pgfpathcurveto{\pgfqpoint{6.906619in}{8.794594in}}{\pgfqpoint{6.917218in}{8.798985in}}{\pgfqpoint{6.925031in}{8.806798in}}%
\pgfpathcurveto{\pgfqpoint{6.932845in}{8.814612in}}{\pgfqpoint{6.937235in}{8.825211in}}{\pgfqpoint{6.937235in}{8.836261in}}%
\pgfpathcurveto{\pgfqpoint{6.937235in}{8.847311in}}{\pgfqpoint{6.932845in}{8.857910in}}{\pgfqpoint{6.925031in}{8.865724in}}%
\pgfpathcurveto{\pgfqpoint{6.917218in}{8.873537in}}{\pgfqpoint{6.906619in}{8.877928in}}{\pgfqpoint{6.895569in}{8.877928in}}%
\pgfpathcurveto{\pgfqpoint{6.884518in}{8.877928in}}{\pgfqpoint{6.873919in}{8.873537in}}{\pgfqpoint{6.866106in}{8.865724in}}%
\pgfpathcurveto{\pgfqpoint{6.858292in}{8.857910in}}{\pgfqpoint{6.853902in}{8.847311in}}{\pgfqpoint{6.853902in}{8.836261in}}%
\pgfpathcurveto{\pgfqpoint{6.853902in}{8.825211in}}{\pgfqpoint{6.858292in}{8.814612in}}{\pgfqpoint{6.866106in}{8.806798in}}%
\pgfpathcurveto{\pgfqpoint{6.873919in}{8.798985in}}{\pgfqpoint{6.884518in}{8.794594in}}{\pgfqpoint{6.895569in}{8.794594in}}%
\pgfpathlineto{\pgfqpoint{6.895569in}{8.794594in}}%
\pgfpathclose%
\pgfusepath{stroke,fill}%
\end{pgfscope}%
\begin{pgfscope}%
\pgfpathrectangle{\pgfqpoint{5.292946in}{7.624184in}}{\pgfqpoint{2.177280in}{2.201755in}}%
\pgfusepath{clip}%
\pgfsetbuttcap%
\pgfsetroundjoin%
\definecolor{currentfill}{rgb}{0.172549,0.627451,0.172549}%
\pgfsetfillcolor{currentfill}%
\pgfsetlinewidth{0.481800pt}%
\definecolor{currentstroke}{rgb}{1.000000,1.000000,1.000000}%
\pgfsetstrokecolor{currentstroke}%
\pgfsetdash{}{0pt}%
\pgfpathmoveto{\pgfqpoint{6.637221in}{8.516595in}}%
\pgfpathcurveto{\pgfqpoint{6.648271in}{8.516595in}}{\pgfqpoint{6.658870in}{8.520985in}}{\pgfqpoint{6.666684in}{8.528799in}}%
\pgfpathcurveto{\pgfqpoint{6.674497in}{8.536612in}}{\pgfqpoint{6.678888in}{8.547212in}}{\pgfqpoint{6.678888in}{8.558262in}}%
\pgfpathcurveto{\pgfqpoint{6.678888in}{8.569312in}}{\pgfqpoint{6.674497in}{8.579911in}}{\pgfqpoint{6.666684in}{8.587724in}}%
\pgfpathcurveto{\pgfqpoint{6.658870in}{8.595538in}}{\pgfqpoint{6.648271in}{8.599928in}}{\pgfqpoint{6.637221in}{8.599928in}}%
\pgfpathcurveto{\pgfqpoint{6.626171in}{8.599928in}}{\pgfqpoint{6.615572in}{8.595538in}}{\pgfqpoint{6.607758in}{8.587724in}}%
\pgfpathcurveto{\pgfqpoint{6.599945in}{8.579911in}}{\pgfqpoint{6.595554in}{8.569312in}}{\pgfqpoint{6.595554in}{8.558262in}}%
\pgfpathcurveto{\pgfqpoint{6.595554in}{8.547212in}}{\pgfqpoint{6.599945in}{8.536612in}}{\pgfqpoint{6.607758in}{8.528799in}}%
\pgfpathcurveto{\pgfqpoint{6.615572in}{8.520985in}}{\pgfqpoint{6.626171in}{8.516595in}}{\pgfqpoint{6.637221in}{8.516595in}}%
\pgfpathlineto{\pgfqpoint{6.637221in}{8.516595in}}%
\pgfpathclose%
\pgfusepath{stroke,fill}%
\end{pgfscope}%
\begin{pgfscope}%
\pgfpathrectangle{\pgfqpoint{5.292946in}{7.624184in}}{\pgfqpoint{2.177280in}{2.201755in}}%
\pgfusepath{clip}%
\pgfsetbuttcap%
\pgfsetroundjoin%
\definecolor{currentfill}{rgb}{0.172549,0.627451,0.172549}%
\pgfsetfillcolor{currentfill}%
\pgfsetlinewidth{0.481800pt}%
\definecolor{currentstroke}{rgb}{1.000000,1.000000,1.000000}%
\pgfsetstrokecolor{currentstroke}%
\pgfsetdash{}{0pt}%
\pgfpathmoveto{\pgfqpoint{6.866863in}{9.239393in}}%
\pgfpathcurveto{\pgfqpoint{6.877913in}{9.239393in}}{\pgfqpoint{6.888512in}{9.243784in}}{\pgfqpoint{6.896326in}{9.251597in}}%
\pgfpathcurveto{\pgfqpoint{6.904140in}{9.259411in}}{\pgfqpoint{6.908530in}{9.270010in}}{\pgfqpoint{6.908530in}{9.281060in}}%
\pgfpathcurveto{\pgfqpoint{6.908530in}{9.292110in}}{\pgfqpoint{6.904140in}{9.302709in}}{\pgfqpoint{6.896326in}{9.310523in}}%
\pgfpathcurveto{\pgfqpoint{6.888512in}{9.318336in}}{\pgfqpoint{6.877913in}{9.322727in}}{\pgfqpoint{6.866863in}{9.322727in}}%
\pgfpathcurveto{\pgfqpoint{6.855813in}{9.322727in}}{\pgfqpoint{6.845214in}{9.318336in}}{\pgfqpoint{6.837401in}{9.310523in}}%
\pgfpathcurveto{\pgfqpoint{6.829587in}{9.302709in}}{\pgfqpoint{6.825197in}{9.292110in}}{\pgfqpoint{6.825197in}{9.281060in}}%
\pgfpathcurveto{\pgfqpoint{6.825197in}{9.270010in}}{\pgfqpoint{6.829587in}{9.259411in}}{\pgfqpoint{6.837401in}{9.251597in}}%
\pgfpathcurveto{\pgfqpoint{6.845214in}{9.243784in}}{\pgfqpoint{6.855813in}{9.239393in}}{\pgfqpoint{6.866863in}{9.239393in}}%
\pgfpathlineto{\pgfqpoint{6.866863in}{9.239393in}}%
\pgfpathclose%
\pgfusepath{stroke,fill}%
\end{pgfscope}%
\begin{pgfscope}%
\pgfpathrectangle{\pgfqpoint{5.292946in}{7.624184in}}{\pgfqpoint{2.177280in}{2.201755in}}%
\pgfusepath{clip}%
\pgfsetbuttcap%
\pgfsetroundjoin%
\definecolor{currentfill}{rgb}{0.172549,0.627451,0.172549}%
\pgfsetfillcolor{currentfill}%
\pgfsetlinewidth{0.481800pt}%
\definecolor{currentstroke}{rgb}{1.000000,1.000000,1.000000}%
\pgfsetstrokecolor{currentstroke}%
\pgfsetdash{}{0pt}%
\pgfpathmoveto{\pgfqpoint{6.780747in}{8.794594in}}%
\pgfpathcurveto{\pgfqpoint{6.791798in}{8.794594in}}{\pgfqpoint{6.802397in}{8.798985in}}{\pgfqpoint{6.810210in}{8.806798in}}%
\pgfpathcurveto{\pgfqpoint{6.818024in}{8.814612in}}{\pgfqpoint{6.822414in}{8.825211in}}{\pgfqpoint{6.822414in}{8.836261in}}%
\pgfpathcurveto{\pgfqpoint{6.822414in}{8.847311in}}{\pgfqpoint{6.818024in}{8.857910in}}{\pgfqpoint{6.810210in}{8.865724in}}%
\pgfpathcurveto{\pgfqpoint{6.802397in}{8.873537in}}{\pgfqpoint{6.791798in}{8.877928in}}{\pgfqpoint{6.780747in}{8.877928in}}%
\pgfpathcurveto{\pgfqpoint{6.769697in}{8.877928in}}{\pgfqpoint{6.759098in}{8.873537in}}{\pgfqpoint{6.751285in}{8.865724in}}%
\pgfpathcurveto{\pgfqpoint{6.743471in}{8.857910in}}{\pgfqpoint{6.739081in}{8.847311in}}{\pgfqpoint{6.739081in}{8.836261in}}%
\pgfpathcurveto{\pgfqpoint{6.739081in}{8.825211in}}{\pgfqpoint{6.743471in}{8.814612in}}{\pgfqpoint{6.751285in}{8.806798in}}%
\pgfpathcurveto{\pgfqpoint{6.759098in}{8.798985in}}{\pgfqpoint{6.769697in}{8.794594in}}{\pgfqpoint{6.780747in}{8.794594in}}%
\pgfpathlineto{\pgfqpoint{6.780747in}{8.794594in}}%
\pgfpathclose%
\pgfusepath{stroke,fill}%
\end{pgfscope}%
\begin{pgfscope}%
\pgfpathrectangle{\pgfqpoint{5.292946in}{7.624184in}}{\pgfqpoint{2.177280in}{2.201755in}}%
\pgfusepath{clip}%
\pgfsetbuttcap%
\pgfsetroundjoin%
\definecolor{currentfill}{rgb}{0.172549,0.627451,0.172549}%
\pgfsetfillcolor{currentfill}%
\pgfsetlinewidth{0.481800pt}%
\definecolor{currentstroke}{rgb}{1.000000,1.000000,1.000000}%
\pgfsetstrokecolor{currentstroke}%
\pgfsetdash{}{0pt}%
\pgfpathmoveto{\pgfqpoint{6.838158in}{8.905794in}}%
\pgfpathcurveto{\pgfqpoint{6.849208in}{8.905794in}}{\pgfqpoint{6.859807in}{8.910184in}}{\pgfqpoint{6.867621in}{8.917998in}}%
\pgfpathcurveto{\pgfqpoint{6.875434in}{8.925812in}}{\pgfqpoint{6.879825in}{8.936411in}}{\pgfqpoint{6.879825in}{8.947461in}}%
\pgfpathcurveto{\pgfqpoint{6.879825in}{8.958511in}}{\pgfqpoint{6.875434in}{8.969110in}}{\pgfqpoint{6.867621in}{8.976924in}}%
\pgfpathcurveto{\pgfqpoint{6.859807in}{8.984737in}}{\pgfqpoint{6.849208in}{8.989127in}}{\pgfqpoint{6.838158in}{8.989127in}}%
\pgfpathcurveto{\pgfqpoint{6.827108in}{8.989127in}}{\pgfqpoint{6.816509in}{8.984737in}}{\pgfqpoint{6.808695in}{8.976924in}}%
\pgfpathcurveto{\pgfqpoint{6.800882in}{8.969110in}}{\pgfqpoint{6.796491in}{8.958511in}}{\pgfqpoint{6.796491in}{8.947461in}}%
\pgfpathcurveto{\pgfqpoint{6.796491in}{8.936411in}}{\pgfqpoint{6.800882in}{8.925812in}}{\pgfqpoint{6.808695in}{8.917998in}}%
\pgfpathcurveto{\pgfqpoint{6.816509in}{8.910184in}}{\pgfqpoint{6.827108in}{8.905794in}}{\pgfqpoint{6.838158in}{8.905794in}}%
\pgfpathlineto{\pgfqpoint{6.838158in}{8.905794in}}%
\pgfpathclose%
\pgfusepath{stroke,fill}%
\end{pgfscope}%
\begin{pgfscope}%
\pgfpathrectangle{\pgfqpoint{5.292946in}{7.624184in}}{\pgfqpoint{2.177280in}{2.201755in}}%
\pgfusepath{clip}%
\pgfsetbuttcap%
\pgfsetroundjoin%
\definecolor{currentfill}{rgb}{0.172549,0.627451,0.172549}%
\pgfsetfillcolor{currentfill}%
\pgfsetlinewidth{0.481800pt}%
\definecolor{currentstroke}{rgb}{1.000000,1.000000,1.000000}%
\pgfsetstrokecolor{currentstroke}%
\pgfsetdash{}{0pt}%
\pgfpathmoveto{\pgfqpoint{7.067800in}{9.517393in}}%
\pgfpathcurveto{\pgfqpoint{7.078850in}{9.517393in}}{\pgfqpoint{7.089449in}{9.521783in}}{\pgfqpoint{7.097263in}{9.529597in}}%
\pgfpathcurveto{\pgfqpoint{7.105077in}{9.537410in}}{\pgfqpoint{7.109467in}{9.548009in}}{\pgfqpoint{7.109467in}{9.559059in}}%
\pgfpathcurveto{\pgfqpoint{7.109467in}{9.570110in}}{\pgfqpoint{7.105077in}{9.580709in}}{\pgfqpoint{7.097263in}{9.588522in}}%
\pgfpathcurveto{\pgfqpoint{7.089449in}{9.596336in}}{\pgfqpoint{7.078850in}{9.600726in}}{\pgfqpoint{7.067800in}{9.600726in}}%
\pgfpathcurveto{\pgfqpoint{7.056750in}{9.600726in}}{\pgfqpoint{7.046151in}{9.596336in}}{\pgfqpoint{7.038338in}{9.588522in}}%
\pgfpathcurveto{\pgfqpoint{7.030524in}{9.580709in}}{\pgfqpoint{7.026134in}{9.570110in}}{\pgfqpoint{7.026134in}{9.559059in}}%
\pgfpathcurveto{\pgfqpoint{7.026134in}{9.548009in}}{\pgfqpoint{7.030524in}{9.537410in}}{\pgfqpoint{7.038338in}{9.529597in}}%
\pgfpathcurveto{\pgfqpoint{7.046151in}{9.521783in}}{\pgfqpoint{7.056750in}{9.517393in}}{\pgfqpoint{7.067800in}{9.517393in}}%
\pgfpathlineto{\pgfqpoint{7.067800in}{9.517393in}}%
\pgfpathclose%
\pgfusepath{stroke,fill}%
\end{pgfscope}%
\begin{pgfscope}%
\pgfpathrectangle{\pgfqpoint{5.292946in}{7.624184in}}{\pgfqpoint{2.177280in}{2.201755in}}%
\pgfusepath{clip}%
\pgfsetbuttcap%
\pgfsetroundjoin%
\definecolor{currentfill}{rgb}{0.172549,0.627451,0.172549}%
\pgfsetfillcolor{currentfill}%
\pgfsetlinewidth{0.481800pt}%
\definecolor{currentstroke}{rgb}{1.000000,1.000000,1.000000}%
\pgfsetstrokecolor{currentstroke}%
\pgfsetdash{}{0pt}%
\pgfpathmoveto{\pgfqpoint{6.464989in}{8.016196in}}%
\pgfpathcurveto{\pgfqpoint{6.476039in}{8.016196in}}{\pgfqpoint{6.486638in}{8.020586in}}{\pgfqpoint{6.494452in}{8.028400in}}%
\pgfpathcurveto{\pgfqpoint{6.502266in}{8.036214in}}{\pgfqpoint{6.506656in}{8.046813in}}{\pgfqpoint{6.506656in}{8.057863in}}%
\pgfpathcurveto{\pgfqpoint{6.506656in}{8.068913in}}{\pgfqpoint{6.502266in}{8.079512in}}{\pgfqpoint{6.494452in}{8.087326in}}%
\pgfpathcurveto{\pgfqpoint{6.486638in}{8.095139in}}{\pgfqpoint{6.476039in}{8.099529in}}{\pgfqpoint{6.464989in}{8.099529in}}%
\pgfpathcurveto{\pgfqpoint{6.453939in}{8.099529in}}{\pgfqpoint{6.443340in}{8.095139in}}{\pgfqpoint{6.435527in}{8.087326in}}%
\pgfpathcurveto{\pgfqpoint{6.427713in}{8.079512in}}{\pgfqpoint{6.423323in}{8.068913in}}{\pgfqpoint{6.423323in}{8.057863in}}%
\pgfpathcurveto{\pgfqpoint{6.423323in}{8.046813in}}{\pgfqpoint{6.427713in}{8.036214in}}{\pgfqpoint{6.435527in}{8.028400in}}%
\pgfpathcurveto{\pgfqpoint{6.443340in}{8.020586in}}{\pgfqpoint{6.453939in}{8.016196in}}{\pgfqpoint{6.464989in}{8.016196in}}%
\pgfpathlineto{\pgfqpoint{6.464989in}{8.016196in}}%
\pgfpathclose%
\pgfusepath{stroke,fill}%
\end{pgfscope}%
\begin{pgfscope}%
\pgfpathrectangle{\pgfqpoint{5.292946in}{7.624184in}}{\pgfqpoint{2.177280in}{2.201755in}}%
\pgfusepath{clip}%
\pgfsetbuttcap%
\pgfsetroundjoin%
\definecolor{currentfill}{rgb}{0.172549,0.627451,0.172549}%
\pgfsetfillcolor{currentfill}%
\pgfsetlinewidth{0.481800pt}%
\definecolor{currentstroke}{rgb}{1.000000,1.000000,1.000000}%
\pgfsetstrokecolor{currentstroke}%
\pgfsetdash{}{0pt}%
\pgfpathmoveto{\pgfqpoint{6.981684in}{9.350593in}}%
\pgfpathcurveto{\pgfqpoint{6.992735in}{9.350593in}}{\pgfqpoint{7.003334in}{9.354983in}}{\pgfqpoint{7.011147in}{9.362797in}}%
\pgfpathcurveto{\pgfqpoint{7.018961in}{9.370611in}}{\pgfqpoint{7.023351in}{9.381210in}}{\pgfqpoint{7.023351in}{9.392260in}}%
\pgfpathcurveto{\pgfqpoint{7.023351in}{9.403310in}}{\pgfqpoint{7.018961in}{9.413909in}}{\pgfqpoint{7.011147in}{9.421723in}}%
\pgfpathcurveto{\pgfqpoint{7.003334in}{9.429536in}}{\pgfqpoint{6.992735in}{9.433926in}}{\pgfqpoint{6.981684in}{9.433926in}}%
\pgfpathcurveto{\pgfqpoint{6.970634in}{9.433926in}}{\pgfqpoint{6.960035in}{9.429536in}}{\pgfqpoint{6.952222in}{9.421723in}}%
\pgfpathcurveto{\pgfqpoint{6.944408in}{9.413909in}}{\pgfqpoint{6.940018in}{9.403310in}}{\pgfqpoint{6.940018in}{9.392260in}}%
\pgfpathcurveto{\pgfqpoint{6.940018in}{9.381210in}}{\pgfqpoint{6.944408in}{9.370611in}}{\pgfqpoint{6.952222in}{9.362797in}}%
\pgfpathcurveto{\pgfqpoint{6.960035in}{9.354983in}}{\pgfqpoint{6.970634in}{9.350593in}}{\pgfqpoint{6.981684in}{9.350593in}}%
\pgfpathlineto{\pgfqpoint{6.981684in}{9.350593in}}%
\pgfpathclose%
\pgfusepath{stroke,fill}%
\end{pgfscope}%
\begin{pgfscope}%
\pgfpathrectangle{\pgfqpoint{5.292946in}{7.624184in}}{\pgfqpoint{2.177280in}{2.201755in}}%
\pgfusepath{clip}%
\pgfsetbuttcap%
\pgfsetroundjoin%
\definecolor{currentfill}{rgb}{0.172549,0.627451,0.172549}%
\pgfsetfillcolor{currentfill}%
\pgfsetlinewidth{0.481800pt}%
\definecolor{currentstroke}{rgb}{1.000000,1.000000,1.000000}%
\pgfsetstrokecolor{currentstroke}%
\pgfsetdash{}{0pt}%
\pgfpathmoveto{\pgfqpoint{6.838158in}{9.016994in}}%
\pgfpathcurveto{\pgfqpoint{6.849208in}{9.016994in}}{\pgfqpoint{6.859807in}{9.021384in}}{\pgfqpoint{6.867621in}{9.029198in}}%
\pgfpathcurveto{\pgfqpoint{6.875434in}{9.037011in}}{\pgfqpoint{6.879825in}{9.047610in}}{\pgfqpoint{6.879825in}{9.058661in}}%
\pgfpathcurveto{\pgfqpoint{6.879825in}{9.069711in}}{\pgfqpoint{6.875434in}{9.080310in}}{\pgfqpoint{6.867621in}{9.088123in}}%
\pgfpathcurveto{\pgfqpoint{6.859807in}{9.095937in}}{\pgfqpoint{6.849208in}{9.100327in}}{\pgfqpoint{6.838158in}{9.100327in}}%
\pgfpathcurveto{\pgfqpoint{6.827108in}{9.100327in}}{\pgfqpoint{6.816509in}{9.095937in}}{\pgfqpoint{6.808695in}{9.088123in}}%
\pgfpathcurveto{\pgfqpoint{6.800882in}{9.080310in}}{\pgfqpoint{6.796491in}{9.069711in}}{\pgfqpoint{6.796491in}{9.058661in}}%
\pgfpathcurveto{\pgfqpoint{6.796491in}{9.047610in}}{\pgfqpoint{6.800882in}{9.037011in}}{\pgfqpoint{6.808695in}{9.029198in}}%
\pgfpathcurveto{\pgfqpoint{6.816509in}{9.021384in}}{\pgfqpoint{6.827108in}{9.016994in}}{\pgfqpoint{6.838158in}{9.016994in}}%
\pgfpathlineto{\pgfqpoint{6.838158in}{9.016994in}}%
\pgfpathclose%
\pgfusepath{stroke,fill}%
\end{pgfscope}%
\begin{pgfscope}%
\pgfpathrectangle{\pgfqpoint{5.292946in}{7.624184in}}{\pgfqpoint{2.177280in}{2.201755in}}%
\pgfusepath{clip}%
\pgfsetbuttcap%
\pgfsetroundjoin%
\definecolor{currentfill}{rgb}{0.172549,0.627451,0.172549}%
\pgfsetfillcolor{currentfill}%
\pgfsetlinewidth{0.481800pt}%
\definecolor{currentstroke}{rgb}{1.000000,1.000000,1.000000}%
\pgfsetstrokecolor{currentstroke}%
\pgfsetdash{}{0pt}%
\pgfpathmoveto{\pgfqpoint{6.924274in}{9.294993in}}%
\pgfpathcurveto{\pgfqpoint{6.935324in}{9.294993in}}{\pgfqpoint{6.945923in}{9.299384in}}{\pgfqpoint{6.953737in}{9.307197in}}%
\pgfpathcurveto{\pgfqpoint{6.961550in}{9.315011in}}{\pgfqpoint{6.965941in}{9.325610in}}{\pgfqpoint{6.965941in}{9.336660in}}%
\pgfpathcurveto{\pgfqpoint{6.965941in}{9.347710in}}{\pgfqpoint{6.961550in}{9.358309in}}{\pgfqpoint{6.953737in}{9.366123in}}%
\pgfpathcurveto{\pgfqpoint{6.945923in}{9.373936in}}{\pgfqpoint{6.935324in}{9.378327in}}{\pgfqpoint{6.924274in}{9.378327in}}%
\pgfpathcurveto{\pgfqpoint{6.913224in}{9.378327in}}{\pgfqpoint{6.902625in}{9.373936in}}{\pgfqpoint{6.894811in}{9.366123in}}%
\pgfpathcurveto{\pgfqpoint{6.886997in}{9.358309in}}{\pgfqpoint{6.882607in}{9.347710in}}{\pgfqpoint{6.882607in}{9.336660in}}%
\pgfpathcurveto{\pgfqpoint{6.882607in}{9.325610in}}{\pgfqpoint{6.886997in}{9.315011in}}{\pgfqpoint{6.894811in}{9.307197in}}%
\pgfpathcurveto{\pgfqpoint{6.902625in}{9.299384in}}{\pgfqpoint{6.913224in}{9.294993in}}{\pgfqpoint{6.924274in}{9.294993in}}%
\pgfpathlineto{\pgfqpoint{6.924274in}{9.294993in}}%
\pgfpathclose%
\pgfusepath{stroke,fill}%
\end{pgfscope}%
\begin{pgfscope}%
\pgfpathrectangle{\pgfqpoint{5.292946in}{7.624184in}}{\pgfqpoint{2.177280in}{2.201755in}}%
\pgfusepath{clip}%
\pgfsetbuttcap%
\pgfsetroundjoin%
\definecolor{currentfill}{rgb}{0.172549,0.627451,0.172549}%
\pgfsetfillcolor{currentfill}%
\pgfsetlinewidth{0.481800pt}%
\definecolor{currentstroke}{rgb}{1.000000,1.000000,1.000000}%
\pgfsetstrokecolor{currentstroke}%
\pgfsetdash{}{0pt}%
\pgfpathmoveto{\pgfqpoint{6.637221in}{8.905794in}}%
\pgfpathcurveto{\pgfqpoint{6.648271in}{8.905794in}}{\pgfqpoint{6.658870in}{8.910184in}}{\pgfqpoint{6.666684in}{8.917998in}}%
\pgfpathcurveto{\pgfqpoint{6.674497in}{8.925812in}}{\pgfqpoint{6.678888in}{8.936411in}}{\pgfqpoint{6.678888in}{8.947461in}}%
\pgfpathcurveto{\pgfqpoint{6.678888in}{8.958511in}}{\pgfqpoint{6.674497in}{8.969110in}}{\pgfqpoint{6.666684in}{8.976924in}}%
\pgfpathcurveto{\pgfqpoint{6.658870in}{8.984737in}}{\pgfqpoint{6.648271in}{8.989127in}}{\pgfqpoint{6.637221in}{8.989127in}}%
\pgfpathcurveto{\pgfqpoint{6.626171in}{8.989127in}}{\pgfqpoint{6.615572in}{8.984737in}}{\pgfqpoint{6.607758in}{8.976924in}}%
\pgfpathcurveto{\pgfqpoint{6.599945in}{8.969110in}}{\pgfqpoint{6.595554in}{8.958511in}}{\pgfqpoint{6.595554in}{8.947461in}}%
\pgfpathcurveto{\pgfqpoint{6.595554in}{8.936411in}}{\pgfqpoint{6.599945in}{8.925812in}}{\pgfqpoint{6.607758in}{8.917998in}}%
\pgfpathcurveto{\pgfqpoint{6.615572in}{8.910184in}}{\pgfqpoint{6.626171in}{8.905794in}}{\pgfqpoint{6.637221in}{8.905794in}}%
\pgfpathlineto{\pgfqpoint{6.637221in}{8.905794in}}%
\pgfpathclose%
\pgfusepath{stroke,fill}%
\end{pgfscope}%
\begin{pgfscope}%
\pgfpathrectangle{\pgfqpoint{5.292946in}{7.624184in}}{\pgfqpoint{2.177280in}{2.201755in}}%
\pgfusepath{clip}%
\pgfsetbuttcap%
\pgfsetroundjoin%
\definecolor{currentfill}{rgb}{0.172549,0.627451,0.172549}%
\pgfsetfillcolor{currentfill}%
\pgfsetlinewidth{0.481800pt}%
\definecolor{currentstroke}{rgb}{1.000000,1.000000,1.000000}%
\pgfsetstrokecolor{currentstroke}%
\pgfsetdash{}{0pt}%
\pgfpathmoveto{\pgfqpoint{6.694632in}{8.850194in}}%
\pgfpathcurveto{\pgfqpoint{6.705682in}{8.850194in}}{\pgfqpoint{6.716281in}{8.854584in}}{\pgfqpoint{6.724094in}{8.862398in}}%
\pgfpathcurveto{\pgfqpoint{6.731908in}{8.870212in}}{\pgfqpoint{6.736298in}{8.880811in}}{\pgfqpoint{6.736298in}{8.891861in}}%
\pgfpathcurveto{\pgfqpoint{6.736298in}{8.902911in}}{\pgfqpoint{6.731908in}{8.913510in}}{\pgfqpoint{6.724094in}{8.921324in}}%
\pgfpathcurveto{\pgfqpoint{6.716281in}{8.929137in}}{\pgfqpoint{6.705682in}{8.933528in}}{\pgfqpoint{6.694632in}{8.933528in}}%
\pgfpathcurveto{\pgfqpoint{6.683581in}{8.933528in}}{\pgfqpoint{6.672982in}{8.929137in}}{\pgfqpoint{6.665169in}{8.921324in}}%
\pgfpathcurveto{\pgfqpoint{6.657355in}{8.913510in}}{\pgfqpoint{6.652965in}{8.902911in}}{\pgfqpoint{6.652965in}{8.891861in}}%
\pgfpathcurveto{\pgfqpoint{6.652965in}{8.880811in}}{\pgfqpoint{6.657355in}{8.870212in}}{\pgfqpoint{6.665169in}{8.862398in}}%
\pgfpathcurveto{\pgfqpoint{6.672982in}{8.854584in}}{\pgfqpoint{6.683581in}{8.850194in}}{\pgfqpoint{6.694632in}{8.850194in}}%
\pgfpathlineto{\pgfqpoint{6.694632in}{8.850194in}}%
\pgfpathclose%
\pgfusepath{stroke,fill}%
\end{pgfscope}%
\begin{pgfscope}%
\pgfpathrectangle{\pgfqpoint{5.292946in}{7.624184in}}{\pgfqpoint{2.177280in}{2.201755in}}%
\pgfusepath{clip}%
\pgfsetbuttcap%
\pgfsetroundjoin%
\definecolor{currentfill}{rgb}{0.172549,0.627451,0.172549}%
\pgfsetfillcolor{currentfill}%
\pgfsetlinewidth{0.481800pt}%
\definecolor{currentstroke}{rgb}{1.000000,1.000000,1.000000}%
\pgfsetstrokecolor{currentstroke}%
\pgfsetdash{}{0pt}%
\pgfpathmoveto{\pgfqpoint{6.752042in}{9.072594in}}%
\pgfpathcurveto{\pgfqpoint{6.763092in}{9.072594in}}{\pgfqpoint{6.773691in}{9.076984in}}{\pgfqpoint{6.781505in}{9.084798in}}%
\pgfpathcurveto{\pgfqpoint{6.789319in}{9.092611in}}{\pgfqpoint{6.793709in}{9.103210in}}{\pgfqpoint{6.793709in}{9.114260in}}%
\pgfpathcurveto{\pgfqpoint{6.793709in}{9.125311in}}{\pgfqpoint{6.789319in}{9.135910in}}{\pgfqpoint{6.781505in}{9.143723in}}%
\pgfpathcurveto{\pgfqpoint{6.773691in}{9.151537in}}{\pgfqpoint{6.763092in}{9.155927in}}{\pgfqpoint{6.752042in}{9.155927in}}%
\pgfpathcurveto{\pgfqpoint{6.740992in}{9.155927in}}{\pgfqpoint{6.730393in}{9.151537in}}{\pgfqpoint{6.722579in}{9.143723in}}%
\pgfpathcurveto{\pgfqpoint{6.714766in}{9.135910in}}{\pgfqpoint{6.710375in}{9.125311in}}{\pgfqpoint{6.710375in}{9.114260in}}%
\pgfpathcurveto{\pgfqpoint{6.710375in}{9.103210in}}{\pgfqpoint{6.714766in}{9.092611in}}{\pgfqpoint{6.722579in}{9.084798in}}%
\pgfpathcurveto{\pgfqpoint{6.730393in}{9.076984in}}{\pgfqpoint{6.740992in}{9.072594in}}{\pgfqpoint{6.752042in}{9.072594in}}%
\pgfpathlineto{\pgfqpoint{6.752042in}{9.072594in}}%
\pgfpathclose%
\pgfusepath{stroke,fill}%
\end{pgfscope}%
\begin{pgfscope}%
\pgfpathrectangle{\pgfqpoint{5.292946in}{7.624184in}}{\pgfqpoint{2.177280in}{2.201755in}}%
\pgfusepath{clip}%
\pgfsetbuttcap%
\pgfsetroundjoin%
\definecolor{currentfill}{rgb}{0.172549,0.627451,0.172549}%
\pgfsetfillcolor{currentfill}%
\pgfsetlinewidth{0.481800pt}%
\definecolor{currentstroke}{rgb}{1.000000,1.000000,1.000000}%
\pgfsetstrokecolor{currentstroke}%
\pgfsetdash{}{0pt}%
\pgfpathmoveto{\pgfqpoint{6.608516in}{8.460995in}}%
\pgfpathcurveto{\pgfqpoint{6.619566in}{8.460995in}}{\pgfqpoint{6.630165in}{8.465385in}}{\pgfqpoint{6.637978in}{8.473199in}}%
\pgfpathcurveto{\pgfqpoint{6.645792in}{8.481013in}}{\pgfqpoint{6.650182in}{8.491612in}}{\pgfqpoint{6.650182in}{8.502662in}}%
\pgfpathcurveto{\pgfqpoint{6.650182in}{8.513712in}}{\pgfqpoint{6.645792in}{8.524311in}}{\pgfqpoint{6.637978in}{8.532125in}}%
\pgfpathcurveto{\pgfqpoint{6.630165in}{8.539938in}}{\pgfqpoint{6.619566in}{8.544328in}}{\pgfqpoint{6.608516in}{8.544328in}}%
\pgfpathcurveto{\pgfqpoint{6.597466in}{8.544328in}}{\pgfqpoint{6.586867in}{8.539938in}}{\pgfqpoint{6.579053in}{8.532125in}}%
\pgfpathcurveto{\pgfqpoint{6.571239in}{8.524311in}}{\pgfqpoint{6.566849in}{8.513712in}}{\pgfqpoint{6.566849in}{8.502662in}}%
\pgfpathcurveto{\pgfqpoint{6.566849in}{8.491612in}}{\pgfqpoint{6.571239in}{8.481013in}}{\pgfqpoint{6.579053in}{8.473199in}}%
\pgfpathcurveto{\pgfqpoint{6.586867in}{8.465385in}}{\pgfqpoint{6.597466in}{8.460995in}}{\pgfqpoint{6.608516in}{8.460995in}}%
\pgfpathlineto{\pgfqpoint{6.608516in}{8.460995in}}%
\pgfpathclose%
\pgfusepath{stroke,fill}%
\end{pgfscope}%
\begin{pgfscope}%
\pgfpathrectangle{\pgfqpoint{5.292946in}{7.624184in}}{\pgfqpoint{2.177280in}{2.201755in}}%
\pgfusepath{clip}%
\pgfsetbuttcap%
\pgfsetroundjoin%
\definecolor{currentfill}{rgb}{0.172549,0.627451,0.172549}%
\pgfsetfillcolor{currentfill}%
\pgfsetlinewidth{0.481800pt}%
\definecolor{currentstroke}{rgb}{1.000000,1.000000,1.000000}%
\pgfsetstrokecolor{currentstroke}%
\pgfsetdash{}{0pt}%
\pgfpathmoveto{\pgfqpoint{6.637221in}{8.516595in}}%
\pgfpathcurveto{\pgfqpoint{6.648271in}{8.516595in}}{\pgfqpoint{6.658870in}{8.520985in}}{\pgfqpoint{6.666684in}{8.528799in}}%
\pgfpathcurveto{\pgfqpoint{6.674497in}{8.536612in}}{\pgfqpoint{6.678888in}{8.547212in}}{\pgfqpoint{6.678888in}{8.558262in}}%
\pgfpathcurveto{\pgfqpoint{6.678888in}{8.569312in}}{\pgfqpoint{6.674497in}{8.579911in}}{\pgfqpoint{6.666684in}{8.587724in}}%
\pgfpathcurveto{\pgfqpoint{6.658870in}{8.595538in}}{\pgfqpoint{6.648271in}{8.599928in}}{\pgfqpoint{6.637221in}{8.599928in}}%
\pgfpathcurveto{\pgfqpoint{6.626171in}{8.599928in}}{\pgfqpoint{6.615572in}{8.595538in}}{\pgfqpoint{6.607758in}{8.587724in}}%
\pgfpathcurveto{\pgfqpoint{6.599945in}{8.579911in}}{\pgfqpoint{6.595554in}{8.569312in}}{\pgfqpoint{6.595554in}{8.558262in}}%
\pgfpathcurveto{\pgfqpoint{6.595554in}{8.547212in}}{\pgfqpoint{6.599945in}{8.536612in}}{\pgfqpoint{6.607758in}{8.528799in}}%
\pgfpathcurveto{\pgfqpoint{6.615572in}{8.520985in}}{\pgfqpoint{6.626171in}{8.516595in}}{\pgfqpoint{6.637221in}{8.516595in}}%
\pgfpathlineto{\pgfqpoint{6.637221in}{8.516595in}}%
\pgfpathclose%
\pgfusepath{stroke,fill}%
\end{pgfscope}%
\begin{pgfscope}%
\pgfpathrectangle{\pgfqpoint{5.292946in}{7.624184in}}{\pgfqpoint{2.177280in}{2.201755in}}%
\pgfusepath{clip}%
\pgfsetbuttcap%
\pgfsetroundjoin%
\definecolor{currentfill}{rgb}{0.172549,0.627451,0.172549}%
\pgfsetfillcolor{currentfill}%
\pgfsetlinewidth{0.481800pt}%
\definecolor{currentstroke}{rgb}{1.000000,1.000000,1.000000}%
\pgfsetstrokecolor{currentstroke}%
\pgfsetdash{}{0pt}%
\pgfpathmoveto{\pgfqpoint{6.694632in}{8.850194in}}%
\pgfpathcurveto{\pgfqpoint{6.705682in}{8.850194in}}{\pgfqpoint{6.716281in}{8.854584in}}{\pgfqpoint{6.724094in}{8.862398in}}%
\pgfpathcurveto{\pgfqpoint{6.731908in}{8.870212in}}{\pgfqpoint{6.736298in}{8.880811in}}{\pgfqpoint{6.736298in}{8.891861in}}%
\pgfpathcurveto{\pgfqpoint{6.736298in}{8.902911in}}{\pgfqpoint{6.731908in}{8.913510in}}{\pgfqpoint{6.724094in}{8.921324in}}%
\pgfpathcurveto{\pgfqpoint{6.716281in}{8.929137in}}{\pgfqpoint{6.705682in}{8.933528in}}{\pgfqpoint{6.694632in}{8.933528in}}%
\pgfpathcurveto{\pgfqpoint{6.683581in}{8.933528in}}{\pgfqpoint{6.672982in}{8.929137in}}{\pgfqpoint{6.665169in}{8.921324in}}%
\pgfpathcurveto{\pgfqpoint{6.657355in}{8.913510in}}{\pgfqpoint{6.652965in}{8.902911in}}{\pgfqpoint{6.652965in}{8.891861in}}%
\pgfpathcurveto{\pgfqpoint{6.652965in}{8.880811in}}{\pgfqpoint{6.657355in}{8.870212in}}{\pgfqpoint{6.665169in}{8.862398in}}%
\pgfpathcurveto{\pgfqpoint{6.672982in}{8.854584in}}{\pgfqpoint{6.683581in}{8.850194in}}{\pgfqpoint{6.694632in}{8.850194in}}%
\pgfpathlineto{\pgfqpoint{6.694632in}{8.850194in}}%
\pgfpathclose%
\pgfusepath{stroke,fill}%
\end{pgfscope}%
\begin{pgfscope}%
\pgfpathrectangle{\pgfqpoint{5.292946in}{7.624184in}}{\pgfqpoint{2.177280in}{2.201755in}}%
\pgfusepath{clip}%
\pgfsetbuttcap%
\pgfsetroundjoin%
\definecolor{currentfill}{rgb}{0.172549,0.627451,0.172549}%
\pgfsetfillcolor{currentfill}%
\pgfsetlinewidth{0.481800pt}%
\definecolor{currentstroke}{rgb}{1.000000,1.000000,1.000000}%
\pgfsetstrokecolor{currentstroke}%
\pgfsetdash{}{0pt}%
\pgfpathmoveto{\pgfqpoint{6.752042in}{8.905794in}}%
\pgfpathcurveto{\pgfqpoint{6.763092in}{8.905794in}}{\pgfqpoint{6.773691in}{8.910184in}}{\pgfqpoint{6.781505in}{8.917998in}}%
\pgfpathcurveto{\pgfqpoint{6.789319in}{8.925812in}}{\pgfqpoint{6.793709in}{8.936411in}}{\pgfqpoint{6.793709in}{8.947461in}}%
\pgfpathcurveto{\pgfqpoint{6.793709in}{8.958511in}}{\pgfqpoint{6.789319in}{8.969110in}}{\pgfqpoint{6.781505in}{8.976924in}}%
\pgfpathcurveto{\pgfqpoint{6.773691in}{8.984737in}}{\pgfqpoint{6.763092in}{8.989127in}}{\pgfqpoint{6.752042in}{8.989127in}}%
\pgfpathcurveto{\pgfqpoint{6.740992in}{8.989127in}}{\pgfqpoint{6.730393in}{8.984737in}}{\pgfqpoint{6.722579in}{8.976924in}}%
\pgfpathcurveto{\pgfqpoint{6.714766in}{8.969110in}}{\pgfqpoint{6.710375in}{8.958511in}}{\pgfqpoint{6.710375in}{8.947461in}}%
\pgfpathcurveto{\pgfqpoint{6.710375in}{8.936411in}}{\pgfqpoint{6.714766in}{8.925812in}}{\pgfqpoint{6.722579in}{8.917998in}}%
\pgfpathcurveto{\pgfqpoint{6.730393in}{8.910184in}}{\pgfqpoint{6.740992in}{8.905794in}}{\pgfqpoint{6.752042in}{8.905794in}}%
\pgfpathlineto{\pgfqpoint{6.752042in}{8.905794in}}%
\pgfpathclose%
\pgfusepath{stroke,fill}%
\end{pgfscope}%
\begin{pgfscope}%
\pgfpathrectangle{\pgfqpoint{5.292946in}{7.624184in}}{\pgfqpoint{2.177280in}{2.201755in}}%
\pgfusepath{clip}%
\pgfsetbuttcap%
\pgfsetroundjoin%
\definecolor{currentfill}{rgb}{0.172549,0.627451,0.172549}%
\pgfsetfillcolor{currentfill}%
\pgfsetlinewidth{0.481800pt}%
\definecolor{currentstroke}{rgb}{1.000000,1.000000,1.000000}%
\pgfsetstrokecolor{currentstroke}%
\pgfsetdash{}{0pt}%
\pgfpathmoveto{\pgfqpoint{7.096506in}{9.572993in}}%
\pgfpathcurveto{\pgfqpoint{7.107556in}{9.572993in}}{\pgfqpoint{7.118155in}{9.577383in}}{\pgfqpoint{7.125968in}{9.585197in}}%
\pgfpathcurveto{\pgfqpoint{7.133782in}{9.593010in}}{\pgfqpoint{7.138172in}{9.603609in}}{\pgfqpoint{7.138172in}{9.614659in}}%
\pgfpathcurveto{\pgfqpoint{7.138172in}{9.625709in}}{\pgfqpoint{7.133782in}{9.636308in}}{\pgfqpoint{7.125968in}{9.644122in}}%
\pgfpathcurveto{\pgfqpoint{7.118155in}{9.651936in}}{\pgfqpoint{7.107556in}{9.656326in}}{\pgfqpoint{7.096506in}{9.656326in}}%
\pgfpathcurveto{\pgfqpoint{7.085455in}{9.656326in}}{\pgfqpoint{7.074856in}{9.651936in}}{\pgfqpoint{7.067043in}{9.644122in}}%
\pgfpathcurveto{\pgfqpoint{7.059229in}{9.636308in}}{\pgfqpoint{7.054839in}{9.625709in}}{\pgfqpoint{7.054839in}{9.614659in}}%
\pgfpathcurveto{\pgfqpoint{7.054839in}{9.603609in}}{\pgfqpoint{7.059229in}{9.593010in}}{\pgfqpoint{7.067043in}{9.585197in}}%
\pgfpathcurveto{\pgfqpoint{7.074856in}{9.577383in}}{\pgfqpoint{7.085455in}{9.572993in}}{\pgfqpoint{7.096506in}{9.572993in}}%
\pgfpathlineto{\pgfqpoint{7.096506in}{9.572993in}}%
\pgfpathclose%
\pgfusepath{stroke,fill}%
\end{pgfscope}%
\begin{pgfscope}%
\pgfpathrectangle{\pgfqpoint{5.292946in}{7.624184in}}{\pgfqpoint{2.177280in}{2.201755in}}%
\pgfusepath{clip}%
\pgfsetbuttcap%
\pgfsetroundjoin%
\definecolor{currentfill}{rgb}{0.172549,0.627451,0.172549}%
\pgfsetfillcolor{currentfill}%
\pgfsetlinewidth{0.481800pt}%
\definecolor{currentstroke}{rgb}{1.000000,1.000000,1.000000}%
\pgfsetstrokecolor{currentstroke}%
\pgfsetdash{}{0pt}%
\pgfpathmoveto{\pgfqpoint{7.153916in}{9.572993in}}%
\pgfpathcurveto{\pgfqpoint{7.164966in}{9.572993in}}{\pgfqpoint{7.175565in}{9.577383in}}{\pgfqpoint{7.183379in}{9.585197in}}%
\pgfpathcurveto{\pgfqpoint{7.191193in}{9.593010in}}{\pgfqpoint{7.195583in}{9.603609in}}{\pgfqpoint{7.195583in}{9.614659in}}%
\pgfpathcurveto{\pgfqpoint{7.195583in}{9.625709in}}{\pgfqpoint{7.191193in}{9.636308in}}{\pgfqpoint{7.183379in}{9.644122in}}%
\pgfpathcurveto{\pgfqpoint{7.175565in}{9.651936in}}{\pgfqpoint{7.164966in}{9.656326in}}{\pgfqpoint{7.153916in}{9.656326in}}%
\pgfpathcurveto{\pgfqpoint{7.142866in}{9.656326in}}{\pgfqpoint{7.132267in}{9.651936in}}{\pgfqpoint{7.124453in}{9.644122in}}%
\pgfpathcurveto{\pgfqpoint{7.116640in}{9.636308in}}{\pgfqpoint{7.112249in}{9.625709in}}{\pgfqpoint{7.112249in}{9.614659in}}%
\pgfpathcurveto{\pgfqpoint{7.112249in}{9.603609in}}{\pgfqpoint{7.116640in}{9.593010in}}{\pgfqpoint{7.124453in}{9.585197in}}%
\pgfpathcurveto{\pgfqpoint{7.132267in}{9.577383in}}{\pgfqpoint{7.142866in}{9.572993in}}{\pgfqpoint{7.153916in}{9.572993in}}%
\pgfpathlineto{\pgfqpoint{7.153916in}{9.572993in}}%
\pgfpathclose%
\pgfusepath{stroke,fill}%
\end{pgfscope}%
\begin{pgfscope}%
\pgfpathrectangle{\pgfqpoint{5.292946in}{7.624184in}}{\pgfqpoint{2.177280in}{2.201755in}}%
\pgfusepath{clip}%
\pgfsetbuttcap%
\pgfsetroundjoin%
\definecolor{currentfill}{rgb}{0.172549,0.627451,0.172549}%
\pgfsetfillcolor{currentfill}%
\pgfsetlinewidth{0.481800pt}%
\definecolor{currentstroke}{rgb}{1.000000,1.000000,1.000000}%
\pgfsetstrokecolor{currentstroke}%
\pgfsetdash{}{0pt}%
\pgfpathmoveto{\pgfqpoint{6.608516in}{8.627795in}}%
\pgfpathcurveto{\pgfqpoint{6.619566in}{8.627795in}}{\pgfqpoint{6.630165in}{8.632185in}}{\pgfqpoint{6.637978in}{8.639999in}}%
\pgfpathcurveto{\pgfqpoint{6.645792in}{8.647812in}}{\pgfqpoint{6.650182in}{8.658411in}}{\pgfqpoint{6.650182in}{8.669461in}}%
\pgfpathcurveto{\pgfqpoint{6.650182in}{8.680512in}}{\pgfqpoint{6.645792in}{8.691111in}}{\pgfqpoint{6.637978in}{8.698924in}}%
\pgfpathcurveto{\pgfqpoint{6.630165in}{8.706738in}}{\pgfqpoint{6.619566in}{8.711128in}}{\pgfqpoint{6.608516in}{8.711128in}}%
\pgfpathcurveto{\pgfqpoint{6.597466in}{8.711128in}}{\pgfqpoint{6.586867in}{8.706738in}}{\pgfqpoint{6.579053in}{8.698924in}}%
\pgfpathcurveto{\pgfqpoint{6.571239in}{8.691111in}}{\pgfqpoint{6.566849in}{8.680512in}}{\pgfqpoint{6.566849in}{8.669461in}}%
\pgfpathcurveto{\pgfqpoint{6.566849in}{8.658411in}}{\pgfqpoint{6.571239in}{8.647812in}}{\pgfqpoint{6.579053in}{8.639999in}}%
\pgfpathcurveto{\pgfqpoint{6.586867in}{8.632185in}}{\pgfqpoint{6.597466in}{8.627795in}}{\pgfqpoint{6.608516in}{8.627795in}}%
\pgfpathlineto{\pgfqpoint{6.608516in}{8.627795in}}%
\pgfpathclose%
\pgfusepath{stroke,fill}%
\end{pgfscope}%
\begin{pgfscope}%
\pgfpathrectangle{\pgfqpoint{5.292946in}{7.624184in}}{\pgfqpoint{2.177280in}{2.201755in}}%
\pgfusepath{clip}%
\pgfsetbuttcap%
\pgfsetroundjoin%
\definecolor{currentfill}{rgb}{0.172549,0.627451,0.172549}%
\pgfsetfillcolor{currentfill}%
\pgfsetlinewidth{0.481800pt}%
\definecolor{currentstroke}{rgb}{1.000000,1.000000,1.000000}%
\pgfsetstrokecolor{currentstroke}%
\pgfsetdash{}{0pt}%
\pgfpathmoveto{\pgfqpoint{6.809453in}{9.128194in}}%
\pgfpathcurveto{\pgfqpoint{6.820503in}{9.128194in}}{\pgfqpoint{6.831102in}{9.132584in}}{\pgfqpoint{6.838915in}{9.140397in}}%
\pgfpathcurveto{\pgfqpoint{6.846729in}{9.148211in}}{\pgfqpoint{6.851119in}{9.158810in}}{\pgfqpoint{6.851119in}{9.169860in}}%
\pgfpathcurveto{\pgfqpoint{6.851119in}{9.180910in}}{\pgfqpoint{6.846729in}{9.191509in}}{\pgfqpoint{6.838915in}{9.199323in}}%
\pgfpathcurveto{\pgfqpoint{6.831102in}{9.207137in}}{\pgfqpoint{6.820503in}{9.211527in}}{\pgfqpoint{6.809453in}{9.211527in}}%
\pgfpathcurveto{\pgfqpoint{6.798403in}{9.211527in}}{\pgfqpoint{6.787804in}{9.207137in}}{\pgfqpoint{6.779990in}{9.199323in}}%
\pgfpathcurveto{\pgfqpoint{6.772176in}{9.191509in}}{\pgfqpoint{6.767786in}{9.180910in}}{\pgfqpoint{6.767786in}{9.169860in}}%
\pgfpathcurveto{\pgfqpoint{6.767786in}{9.158810in}}{\pgfqpoint{6.772176in}{9.148211in}}{\pgfqpoint{6.779990in}{9.140397in}}%
\pgfpathcurveto{\pgfqpoint{6.787804in}{9.132584in}}{\pgfqpoint{6.798403in}{9.128194in}}{\pgfqpoint{6.809453in}{9.128194in}}%
\pgfpathlineto{\pgfqpoint{6.809453in}{9.128194in}}%
\pgfpathclose%
\pgfusepath{stroke,fill}%
\end{pgfscope}%
\begin{pgfscope}%
\pgfpathrectangle{\pgfqpoint{5.292946in}{7.624184in}}{\pgfqpoint{2.177280in}{2.201755in}}%
\pgfusepath{clip}%
\pgfsetbuttcap%
\pgfsetroundjoin%
\definecolor{currentfill}{rgb}{0.172549,0.627451,0.172549}%
\pgfsetfillcolor{currentfill}%
\pgfsetlinewidth{0.481800pt}%
\definecolor{currentstroke}{rgb}{1.000000,1.000000,1.000000}%
\pgfsetstrokecolor{currentstroke}%
\pgfsetdash{}{0pt}%
\pgfpathmoveto{\pgfqpoint{6.579810in}{8.405395in}}%
\pgfpathcurveto{\pgfqpoint{6.590861in}{8.405395in}}{\pgfqpoint{6.601460in}{8.409785in}}{\pgfqpoint{6.609273in}{8.417599in}}%
\pgfpathcurveto{\pgfqpoint{6.617087in}{8.425413in}}{\pgfqpoint{6.621477in}{8.436012in}}{\pgfqpoint{6.621477in}{8.447062in}}%
\pgfpathcurveto{\pgfqpoint{6.621477in}{8.458112in}}{\pgfqpoint{6.617087in}{8.468711in}}{\pgfqpoint{6.609273in}{8.476525in}}%
\pgfpathcurveto{\pgfqpoint{6.601460in}{8.484338in}}{\pgfqpoint{6.590861in}{8.488729in}}{\pgfqpoint{6.579810in}{8.488729in}}%
\pgfpathcurveto{\pgfqpoint{6.568760in}{8.488729in}}{\pgfqpoint{6.558161in}{8.484338in}}{\pgfqpoint{6.550348in}{8.476525in}}%
\pgfpathcurveto{\pgfqpoint{6.542534in}{8.468711in}}{\pgfqpoint{6.538144in}{8.458112in}}{\pgfqpoint{6.538144in}{8.447062in}}%
\pgfpathcurveto{\pgfqpoint{6.538144in}{8.436012in}}{\pgfqpoint{6.542534in}{8.425413in}}{\pgfqpoint{6.550348in}{8.417599in}}%
\pgfpathcurveto{\pgfqpoint{6.558161in}{8.409785in}}{\pgfqpoint{6.568760in}{8.405395in}}{\pgfqpoint{6.579810in}{8.405395in}}%
\pgfpathlineto{\pgfqpoint{6.579810in}{8.405395in}}%
\pgfpathclose%
\pgfusepath{stroke,fill}%
\end{pgfscope}%
\begin{pgfscope}%
\pgfpathrectangle{\pgfqpoint{5.292946in}{7.624184in}}{\pgfqpoint{2.177280in}{2.201755in}}%
\pgfusepath{clip}%
\pgfsetbuttcap%
\pgfsetroundjoin%
\definecolor{currentfill}{rgb}{0.172549,0.627451,0.172549}%
\pgfsetfillcolor{currentfill}%
\pgfsetlinewidth{0.481800pt}%
\definecolor{currentstroke}{rgb}{1.000000,1.000000,1.000000}%
\pgfsetstrokecolor{currentstroke}%
\pgfsetdash{}{0pt}%
\pgfpathmoveto{\pgfqpoint{7.096506in}{9.572993in}}%
\pgfpathcurveto{\pgfqpoint{7.107556in}{9.572993in}}{\pgfqpoint{7.118155in}{9.577383in}}{\pgfqpoint{7.125968in}{9.585197in}}%
\pgfpathcurveto{\pgfqpoint{7.133782in}{9.593010in}}{\pgfqpoint{7.138172in}{9.603609in}}{\pgfqpoint{7.138172in}{9.614659in}}%
\pgfpathcurveto{\pgfqpoint{7.138172in}{9.625709in}}{\pgfqpoint{7.133782in}{9.636308in}}{\pgfqpoint{7.125968in}{9.644122in}}%
\pgfpathcurveto{\pgfqpoint{7.118155in}{9.651936in}}{\pgfqpoint{7.107556in}{9.656326in}}{\pgfqpoint{7.096506in}{9.656326in}}%
\pgfpathcurveto{\pgfqpoint{7.085455in}{9.656326in}}{\pgfqpoint{7.074856in}{9.651936in}}{\pgfqpoint{7.067043in}{9.644122in}}%
\pgfpathcurveto{\pgfqpoint{7.059229in}{9.636308in}}{\pgfqpoint{7.054839in}{9.625709in}}{\pgfqpoint{7.054839in}{9.614659in}}%
\pgfpathcurveto{\pgfqpoint{7.054839in}{9.603609in}}{\pgfqpoint{7.059229in}{9.593010in}}{\pgfqpoint{7.067043in}{9.585197in}}%
\pgfpathcurveto{\pgfqpoint{7.074856in}{9.577383in}}{\pgfqpoint{7.085455in}{9.572993in}}{\pgfqpoint{7.096506in}{9.572993in}}%
\pgfpathlineto{\pgfqpoint{7.096506in}{9.572993in}}%
\pgfpathclose%
\pgfusepath{stroke,fill}%
\end{pgfscope}%
\begin{pgfscope}%
\pgfpathrectangle{\pgfqpoint{5.292946in}{7.624184in}}{\pgfqpoint{2.177280in}{2.201755in}}%
\pgfusepath{clip}%
\pgfsetbuttcap%
\pgfsetroundjoin%
\definecolor{currentfill}{rgb}{0.172549,0.627451,0.172549}%
\pgfsetfillcolor{currentfill}%
\pgfsetlinewidth{0.481800pt}%
\definecolor{currentstroke}{rgb}{1.000000,1.000000,1.000000}%
\pgfsetstrokecolor{currentstroke}%
\pgfsetdash{}{0pt}%
\pgfpathmoveto{\pgfqpoint{6.579810in}{8.794594in}}%
\pgfpathcurveto{\pgfqpoint{6.590861in}{8.794594in}}{\pgfqpoint{6.601460in}{8.798985in}}{\pgfqpoint{6.609273in}{8.806798in}}%
\pgfpathcurveto{\pgfqpoint{6.617087in}{8.814612in}}{\pgfqpoint{6.621477in}{8.825211in}}{\pgfqpoint{6.621477in}{8.836261in}}%
\pgfpathcurveto{\pgfqpoint{6.621477in}{8.847311in}}{\pgfqpoint{6.617087in}{8.857910in}}{\pgfqpoint{6.609273in}{8.865724in}}%
\pgfpathcurveto{\pgfqpoint{6.601460in}{8.873537in}}{\pgfqpoint{6.590861in}{8.877928in}}{\pgfqpoint{6.579810in}{8.877928in}}%
\pgfpathcurveto{\pgfqpoint{6.568760in}{8.877928in}}{\pgfqpoint{6.558161in}{8.873537in}}{\pgfqpoint{6.550348in}{8.865724in}}%
\pgfpathcurveto{\pgfqpoint{6.542534in}{8.857910in}}{\pgfqpoint{6.538144in}{8.847311in}}{\pgfqpoint{6.538144in}{8.836261in}}%
\pgfpathcurveto{\pgfqpoint{6.538144in}{8.825211in}}{\pgfqpoint{6.542534in}{8.814612in}}{\pgfqpoint{6.550348in}{8.806798in}}%
\pgfpathcurveto{\pgfqpoint{6.558161in}{8.798985in}}{\pgfqpoint{6.568760in}{8.794594in}}{\pgfqpoint{6.579810in}{8.794594in}}%
\pgfpathlineto{\pgfqpoint{6.579810in}{8.794594in}}%
\pgfpathclose%
\pgfusepath{stroke,fill}%
\end{pgfscope}%
\begin{pgfscope}%
\pgfpathrectangle{\pgfqpoint{5.292946in}{7.624184in}}{\pgfqpoint{2.177280in}{2.201755in}}%
\pgfusepath{clip}%
\pgfsetbuttcap%
\pgfsetroundjoin%
\definecolor{currentfill}{rgb}{0.172549,0.627451,0.172549}%
\pgfsetfillcolor{currentfill}%
\pgfsetlinewidth{0.481800pt}%
\definecolor{currentstroke}{rgb}{1.000000,1.000000,1.000000}%
\pgfsetstrokecolor{currentstroke}%
\pgfsetdash{}{0pt}%
\pgfpathmoveto{\pgfqpoint{6.809453in}{9.016994in}}%
\pgfpathcurveto{\pgfqpoint{6.820503in}{9.016994in}}{\pgfqpoint{6.831102in}{9.021384in}}{\pgfqpoint{6.838915in}{9.029198in}}%
\pgfpathcurveto{\pgfqpoint{6.846729in}{9.037011in}}{\pgfqpoint{6.851119in}{9.047610in}}{\pgfqpoint{6.851119in}{9.058661in}}%
\pgfpathcurveto{\pgfqpoint{6.851119in}{9.069711in}}{\pgfqpoint{6.846729in}{9.080310in}}{\pgfqpoint{6.838915in}{9.088123in}}%
\pgfpathcurveto{\pgfqpoint{6.831102in}{9.095937in}}{\pgfqpoint{6.820503in}{9.100327in}}{\pgfqpoint{6.809453in}{9.100327in}}%
\pgfpathcurveto{\pgfqpoint{6.798403in}{9.100327in}}{\pgfqpoint{6.787804in}{9.095937in}}{\pgfqpoint{6.779990in}{9.088123in}}%
\pgfpathcurveto{\pgfqpoint{6.772176in}{9.080310in}}{\pgfqpoint{6.767786in}{9.069711in}}{\pgfqpoint{6.767786in}{9.058661in}}%
\pgfpathcurveto{\pgfqpoint{6.767786in}{9.047610in}}{\pgfqpoint{6.772176in}{9.037011in}}{\pgfqpoint{6.779990in}{9.029198in}}%
\pgfpathcurveto{\pgfqpoint{6.787804in}{9.021384in}}{\pgfqpoint{6.798403in}{9.016994in}}{\pgfqpoint{6.809453in}{9.016994in}}%
\pgfpathlineto{\pgfqpoint{6.809453in}{9.016994in}}%
\pgfpathclose%
\pgfusepath{stroke,fill}%
\end{pgfscope}%
\begin{pgfscope}%
\pgfpathrectangle{\pgfqpoint{5.292946in}{7.624184in}}{\pgfqpoint{2.177280in}{2.201755in}}%
\pgfusepath{clip}%
\pgfsetbuttcap%
\pgfsetroundjoin%
\definecolor{currentfill}{rgb}{0.172549,0.627451,0.172549}%
\pgfsetfillcolor{currentfill}%
\pgfsetlinewidth{0.481800pt}%
\definecolor{currentstroke}{rgb}{1.000000,1.000000,1.000000}%
\pgfsetstrokecolor{currentstroke}%
\pgfsetdash{}{0pt}%
\pgfpathmoveto{\pgfqpoint{6.895569in}{9.294993in}}%
\pgfpathcurveto{\pgfqpoint{6.906619in}{9.294993in}}{\pgfqpoint{6.917218in}{9.299384in}}{\pgfqpoint{6.925031in}{9.307197in}}%
\pgfpathcurveto{\pgfqpoint{6.932845in}{9.315011in}}{\pgfqpoint{6.937235in}{9.325610in}}{\pgfqpoint{6.937235in}{9.336660in}}%
\pgfpathcurveto{\pgfqpoint{6.937235in}{9.347710in}}{\pgfqpoint{6.932845in}{9.358309in}}{\pgfqpoint{6.925031in}{9.366123in}}%
\pgfpathcurveto{\pgfqpoint{6.917218in}{9.373936in}}{\pgfqpoint{6.906619in}{9.378327in}}{\pgfqpoint{6.895569in}{9.378327in}}%
\pgfpathcurveto{\pgfqpoint{6.884518in}{9.378327in}}{\pgfqpoint{6.873919in}{9.373936in}}{\pgfqpoint{6.866106in}{9.366123in}}%
\pgfpathcurveto{\pgfqpoint{6.858292in}{9.358309in}}{\pgfqpoint{6.853902in}{9.347710in}}{\pgfqpoint{6.853902in}{9.336660in}}%
\pgfpathcurveto{\pgfqpoint{6.853902in}{9.325610in}}{\pgfqpoint{6.858292in}{9.315011in}}{\pgfqpoint{6.866106in}{9.307197in}}%
\pgfpathcurveto{\pgfqpoint{6.873919in}{9.299384in}}{\pgfqpoint{6.884518in}{9.294993in}}{\pgfqpoint{6.895569in}{9.294993in}}%
\pgfpathlineto{\pgfqpoint{6.895569in}{9.294993in}}%
\pgfpathclose%
\pgfusepath{stroke,fill}%
\end{pgfscope}%
\begin{pgfscope}%
\pgfpathrectangle{\pgfqpoint{5.292946in}{7.624184in}}{\pgfqpoint{2.177280in}{2.201755in}}%
\pgfusepath{clip}%
\pgfsetbuttcap%
\pgfsetroundjoin%
\definecolor{currentfill}{rgb}{0.172549,0.627451,0.172549}%
\pgfsetfillcolor{currentfill}%
\pgfsetlinewidth{0.481800pt}%
\definecolor{currentstroke}{rgb}{1.000000,1.000000,1.000000}%
\pgfsetstrokecolor{currentstroke}%
\pgfsetdash{}{0pt}%
\pgfpathmoveto{\pgfqpoint{6.551105in}{8.738994in}}%
\pgfpathcurveto{\pgfqpoint{6.562155in}{8.738994in}}{\pgfqpoint{6.572754in}{8.743385in}}{\pgfqpoint{6.580568in}{8.751198in}}%
\pgfpathcurveto{\pgfqpoint{6.588382in}{8.759012in}}{\pgfqpoint{6.592772in}{8.769611in}}{\pgfqpoint{6.592772in}{8.780661in}}%
\pgfpathcurveto{\pgfqpoint{6.592772in}{8.791711in}}{\pgfqpoint{6.588382in}{8.802310in}}{\pgfqpoint{6.580568in}{8.810124in}}%
\pgfpathcurveto{\pgfqpoint{6.572754in}{8.817938in}}{\pgfqpoint{6.562155in}{8.822328in}}{\pgfqpoint{6.551105in}{8.822328in}}%
\pgfpathcurveto{\pgfqpoint{6.540055in}{8.822328in}}{\pgfqpoint{6.529456in}{8.817938in}}{\pgfqpoint{6.521642in}{8.810124in}}%
\pgfpathcurveto{\pgfqpoint{6.513829in}{8.802310in}}{\pgfqpoint{6.509438in}{8.791711in}}{\pgfqpoint{6.509438in}{8.780661in}}%
\pgfpathcurveto{\pgfqpoint{6.509438in}{8.769611in}}{\pgfqpoint{6.513829in}{8.759012in}}{\pgfqpoint{6.521642in}{8.751198in}}%
\pgfpathcurveto{\pgfqpoint{6.529456in}{8.743385in}}{\pgfqpoint{6.540055in}{8.738994in}}{\pgfqpoint{6.551105in}{8.738994in}}%
\pgfpathlineto{\pgfqpoint{6.551105in}{8.738994in}}%
\pgfpathclose%
\pgfusepath{stroke,fill}%
\end{pgfscope}%
\begin{pgfscope}%
\pgfpathrectangle{\pgfqpoint{5.292946in}{7.624184in}}{\pgfqpoint{2.177280in}{2.201755in}}%
\pgfusepath{clip}%
\pgfsetbuttcap%
\pgfsetroundjoin%
\definecolor{currentfill}{rgb}{0.172549,0.627451,0.172549}%
\pgfsetfillcolor{currentfill}%
\pgfsetlinewidth{0.481800pt}%
\definecolor{currentstroke}{rgb}{1.000000,1.000000,1.000000}%
\pgfsetstrokecolor{currentstroke}%
\pgfsetdash{}{0pt}%
\pgfpathmoveto{\pgfqpoint{6.579810in}{8.683395in}}%
\pgfpathcurveto{\pgfqpoint{6.590861in}{8.683395in}}{\pgfqpoint{6.601460in}{8.687785in}}{\pgfqpoint{6.609273in}{8.695598in}}%
\pgfpathcurveto{\pgfqpoint{6.617087in}{8.703412in}}{\pgfqpoint{6.621477in}{8.714011in}}{\pgfqpoint{6.621477in}{8.725061in}}%
\pgfpathcurveto{\pgfqpoint{6.621477in}{8.736111in}}{\pgfqpoint{6.617087in}{8.746710in}}{\pgfqpoint{6.609273in}{8.754524in}}%
\pgfpathcurveto{\pgfqpoint{6.601460in}{8.762338in}}{\pgfqpoint{6.590861in}{8.766728in}}{\pgfqpoint{6.579810in}{8.766728in}}%
\pgfpathcurveto{\pgfqpoint{6.568760in}{8.766728in}}{\pgfqpoint{6.558161in}{8.762338in}}{\pgfqpoint{6.550348in}{8.754524in}}%
\pgfpathcurveto{\pgfqpoint{6.542534in}{8.746710in}}{\pgfqpoint{6.538144in}{8.736111in}}{\pgfqpoint{6.538144in}{8.725061in}}%
\pgfpathcurveto{\pgfqpoint{6.538144in}{8.714011in}}{\pgfqpoint{6.542534in}{8.703412in}}{\pgfqpoint{6.550348in}{8.695598in}}%
\pgfpathcurveto{\pgfqpoint{6.558161in}{8.687785in}}{\pgfqpoint{6.568760in}{8.683395in}}{\pgfqpoint{6.579810in}{8.683395in}}%
\pgfpathlineto{\pgfqpoint{6.579810in}{8.683395in}}%
\pgfpathclose%
\pgfusepath{stroke,fill}%
\end{pgfscope}%
\begin{pgfscope}%
\pgfpathrectangle{\pgfqpoint{5.292946in}{7.624184in}}{\pgfqpoint{2.177280in}{2.201755in}}%
\pgfusepath{clip}%
\pgfsetbuttcap%
\pgfsetroundjoin%
\definecolor{currentfill}{rgb}{0.172549,0.627451,0.172549}%
\pgfsetfillcolor{currentfill}%
\pgfsetlinewidth{0.481800pt}%
\definecolor{currentstroke}{rgb}{1.000000,1.000000,1.000000}%
\pgfsetstrokecolor{currentstroke}%
\pgfsetdash{}{0pt}%
\pgfpathmoveto{\pgfqpoint{6.780747in}{8.850194in}}%
\pgfpathcurveto{\pgfqpoint{6.791798in}{8.850194in}}{\pgfqpoint{6.802397in}{8.854584in}}{\pgfqpoint{6.810210in}{8.862398in}}%
\pgfpathcurveto{\pgfqpoint{6.818024in}{8.870212in}}{\pgfqpoint{6.822414in}{8.880811in}}{\pgfqpoint{6.822414in}{8.891861in}}%
\pgfpathcurveto{\pgfqpoint{6.822414in}{8.902911in}}{\pgfqpoint{6.818024in}{8.913510in}}{\pgfqpoint{6.810210in}{8.921324in}}%
\pgfpathcurveto{\pgfqpoint{6.802397in}{8.929137in}}{\pgfqpoint{6.791798in}{8.933528in}}{\pgfqpoint{6.780747in}{8.933528in}}%
\pgfpathcurveto{\pgfqpoint{6.769697in}{8.933528in}}{\pgfqpoint{6.759098in}{8.929137in}}{\pgfqpoint{6.751285in}{8.921324in}}%
\pgfpathcurveto{\pgfqpoint{6.743471in}{8.913510in}}{\pgfqpoint{6.739081in}{8.902911in}}{\pgfqpoint{6.739081in}{8.891861in}}%
\pgfpathcurveto{\pgfqpoint{6.739081in}{8.880811in}}{\pgfqpoint{6.743471in}{8.870212in}}{\pgfqpoint{6.751285in}{8.862398in}}%
\pgfpathcurveto{\pgfqpoint{6.759098in}{8.854584in}}{\pgfqpoint{6.769697in}{8.850194in}}{\pgfqpoint{6.780747in}{8.850194in}}%
\pgfpathlineto{\pgfqpoint{6.780747in}{8.850194in}}%
\pgfpathclose%
\pgfusepath{stroke,fill}%
\end{pgfscope}%
\begin{pgfscope}%
\pgfpathrectangle{\pgfqpoint{5.292946in}{7.624184in}}{\pgfqpoint{2.177280in}{2.201755in}}%
\pgfusepath{clip}%
\pgfsetbuttcap%
\pgfsetroundjoin%
\definecolor{currentfill}{rgb}{0.172549,0.627451,0.172549}%
\pgfsetfillcolor{currentfill}%
\pgfsetlinewidth{0.481800pt}%
\definecolor{currentstroke}{rgb}{1.000000,1.000000,1.000000}%
\pgfsetstrokecolor{currentstroke}%
\pgfsetdash{}{0pt}%
\pgfpathmoveto{\pgfqpoint{6.838158in}{9.294993in}}%
\pgfpathcurveto{\pgfqpoint{6.849208in}{9.294993in}}{\pgfqpoint{6.859807in}{9.299384in}}{\pgfqpoint{6.867621in}{9.307197in}}%
\pgfpathcurveto{\pgfqpoint{6.875434in}{9.315011in}}{\pgfqpoint{6.879825in}{9.325610in}}{\pgfqpoint{6.879825in}{9.336660in}}%
\pgfpathcurveto{\pgfqpoint{6.879825in}{9.347710in}}{\pgfqpoint{6.875434in}{9.358309in}}{\pgfqpoint{6.867621in}{9.366123in}}%
\pgfpathcurveto{\pgfqpoint{6.859807in}{9.373936in}}{\pgfqpoint{6.849208in}{9.378327in}}{\pgfqpoint{6.838158in}{9.378327in}}%
\pgfpathcurveto{\pgfqpoint{6.827108in}{9.378327in}}{\pgfqpoint{6.816509in}{9.373936in}}{\pgfqpoint{6.808695in}{9.366123in}}%
\pgfpathcurveto{\pgfqpoint{6.800882in}{9.358309in}}{\pgfqpoint{6.796491in}{9.347710in}}{\pgfqpoint{6.796491in}{9.336660in}}%
\pgfpathcurveto{\pgfqpoint{6.796491in}{9.325610in}}{\pgfqpoint{6.800882in}{9.315011in}}{\pgfqpoint{6.808695in}{9.307197in}}%
\pgfpathcurveto{\pgfqpoint{6.816509in}{9.299384in}}{\pgfqpoint{6.827108in}{9.294993in}}{\pgfqpoint{6.838158in}{9.294993in}}%
\pgfpathlineto{\pgfqpoint{6.838158in}{9.294993in}}%
\pgfpathclose%
\pgfusepath{stroke,fill}%
\end{pgfscope}%
\begin{pgfscope}%
\pgfpathrectangle{\pgfqpoint{5.292946in}{7.624184in}}{\pgfqpoint{2.177280in}{2.201755in}}%
\pgfusepath{clip}%
\pgfsetbuttcap%
\pgfsetroundjoin%
\definecolor{currentfill}{rgb}{0.172549,0.627451,0.172549}%
\pgfsetfillcolor{currentfill}%
\pgfsetlinewidth{0.481800pt}%
\definecolor{currentstroke}{rgb}{1.000000,1.000000,1.000000}%
\pgfsetstrokecolor{currentstroke}%
\pgfsetdash{}{0pt}%
\pgfpathmoveto{\pgfqpoint{6.924274in}{9.406193in}}%
\pgfpathcurveto{\pgfqpoint{6.935324in}{9.406193in}}{\pgfqpoint{6.945923in}{9.410583in}}{\pgfqpoint{6.953737in}{9.418397in}}%
\pgfpathcurveto{\pgfqpoint{6.961550in}{9.426211in}}{\pgfqpoint{6.965941in}{9.436810in}}{\pgfqpoint{6.965941in}{9.447860in}}%
\pgfpathcurveto{\pgfqpoint{6.965941in}{9.458910in}}{\pgfqpoint{6.961550in}{9.469509in}}{\pgfqpoint{6.953737in}{9.477322in}}%
\pgfpathcurveto{\pgfqpoint{6.945923in}{9.485136in}}{\pgfqpoint{6.935324in}{9.489526in}}{\pgfqpoint{6.924274in}{9.489526in}}%
\pgfpathcurveto{\pgfqpoint{6.913224in}{9.489526in}}{\pgfqpoint{6.902625in}{9.485136in}}{\pgfqpoint{6.894811in}{9.477322in}}%
\pgfpathcurveto{\pgfqpoint{6.886997in}{9.469509in}}{\pgfqpoint{6.882607in}{9.458910in}}{\pgfqpoint{6.882607in}{9.447860in}}%
\pgfpathcurveto{\pgfqpoint{6.882607in}{9.436810in}}{\pgfqpoint{6.886997in}{9.426211in}}{\pgfqpoint{6.894811in}{9.418397in}}%
\pgfpathcurveto{\pgfqpoint{6.902625in}{9.410583in}}{\pgfqpoint{6.913224in}{9.406193in}}{\pgfqpoint{6.924274in}{9.406193in}}%
\pgfpathlineto{\pgfqpoint{6.924274in}{9.406193in}}%
\pgfpathclose%
\pgfusepath{stroke,fill}%
\end{pgfscope}%
\begin{pgfscope}%
\pgfpathrectangle{\pgfqpoint{5.292946in}{7.624184in}}{\pgfqpoint{2.177280in}{2.201755in}}%
\pgfusepath{clip}%
\pgfsetbuttcap%
\pgfsetroundjoin%
\definecolor{currentfill}{rgb}{0.172549,0.627451,0.172549}%
\pgfsetfillcolor{currentfill}%
\pgfsetlinewidth{0.481800pt}%
\definecolor{currentstroke}{rgb}{1.000000,1.000000,1.000000}%
\pgfsetstrokecolor{currentstroke}%
\pgfsetdash{}{0pt}%
\pgfpathmoveto{\pgfqpoint{7.010390in}{9.684192in}}%
\pgfpathcurveto{\pgfqpoint{7.021440in}{9.684192in}}{\pgfqpoint{7.032039in}{9.688583in}}{\pgfqpoint{7.039852in}{9.696396in}}%
\pgfpathcurveto{\pgfqpoint{7.047666in}{9.704210in}}{\pgfqpoint{7.052056in}{9.714809in}}{\pgfqpoint{7.052056in}{9.725859in}}%
\pgfpathcurveto{\pgfqpoint{7.052056in}{9.736909in}}{\pgfqpoint{7.047666in}{9.747508in}}{\pgfqpoint{7.039852in}{9.755322in}}%
\pgfpathcurveto{\pgfqpoint{7.032039in}{9.763135in}}{\pgfqpoint{7.021440in}{9.767526in}}{\pgfqpoint{7.010390in}{9.767526in}}%
\pgfpathcurveto{\pgfqpoint{6.999340in}{9.767526in}}{\pgfqpoint{6.988741in}{9.763135in}}{\pgfqpoint{6.980927in}{9.755322in}}%
\pgfpathcurveto{\pgfqpoint{6.973113in}{9.747508in}}{\pgfqpoint{6.968723in}{9.736909in}}{\pgfqpoint{6.968723in}{9.725859in}}%
\pgfpathcurveto{\pgfqpoint{6.968723in}{9.714809in}}{\pgfqpoint{6.973113in}{9.704210in}}{\pgfqpoint{6.980927in}{9.696396in}}%
\pgfpathcurveto{\pgfqpoint{6.988741in}{9.688583in}}{\pgfqpoint{6.999340in}{9.684192in}}{\pgfqpoint{7.010390in}{9.684192in}}%
\pgfpathlineto{\pgfqpoint{7.010390in}{9.684192in}}%
\pgfpathclose%
\pgfusepath{stroke,fill}%
\end{pgfscope}%
\begin{pgfscope}%
\pgfpathrectangle{\pgfqpoint{5.292946in}{7.624184in}}{\pgfqpoint{2.177280in}{2.201755in}}%
\pgfusepath{clip}%
\pgfsetbuttcap%
\pgfsetroundjoin%
\definecolor{currentfill}{rgb}{0.172549,0.627451,0.172549}%
\pgfsetfillcolor{currentfill}%
\pgfsetlinewidth{0.481800pt}%
\definecolor{currentstroke}{rgb}{1.000000,1.000000,1.000000}%
\pgfsetstrokecolor{currentstroke}%
\pgfsetdash{}{0pt}%
\pgfpathmoveto{\pgfqpoint{6.780747in}{8.850194in}}%
\pgfpathcurveto{\pgfqpoint{6.791798in}{8.850194in}}{\pgfqpoint{6.802397in}{8.854584in}}{\pgfqpoint{6.810210in}{8.862398in}}%
\pgfpathcurveto{\pgfqpoint{6.818024in}{8.870212in}}{\pgfqpoint{6.822414in}{8.880811in}}{\pgfqpoint{6.822414in}{8.891861in}}%
\pgfpathcurveto{\pgfqpoint{6.822414in}{8.902911in}}{\pgfqpoint{6.818024in}{8.913510in}}{\pgfqpoint{6.810210in}{8.921324in}}%
\pgfpathcurveto{\pgfqpoint{6.802397in}{8.929137in}}{\pgfqpoint{6.791798in}{8.933528in}}{\pgfqpoint{6.780747in}{8.933528in}}%
\pgfpathcurveto{\pgfqpoint{6.769697in}{8.933528in}}{\pgfqpoint{6.759098in}{8.929137in}}{\pgfqpoint{6.751285in}{8.921324in}}%
\pgfpathcurveto{\pgfqpoint{6.743471in}{8.913510in}}{\pgfqpoint{6.739081in}{8.902911in}}{\pgfqpoint{6.739081in}{8.891861in}}%
\pgfpathcurveto{\pgfqpoint{6.739081in}{8.880811in}}{\pgfqpoint{6.743471in}{8.870212in}}{\pgfqpoint{6.751285in}{8.862398in}}%
\pgfpathcurveto{\pgfqpoint{6.759098in}{8.854584in}}{\pgfqpoint{6.769697in}{8.850194in}}{\pgfqpoint{6.780747in}{8.850194in}}%
\pgfpathlineto{\pgfqpoint{6.780747in}{8.850194in}}%
\pgfpathclose%
\pgfusepath{stroke,fill}%
\end{pgfscope}%
\begin{pgfscope}%
\pgfpathrectangle{\pgfqpoint{5.292946in}{7.624184in}}{\pgfqpoint{2.177280in}{2.201755in}}%
\pgfusepath{clip}%
\pgfsetbuttcap%
\pgfsetroundjoin%
\definecolor{currentfill}{rgb}{0.172549,0.627451,0.172549}%
\pgfsetfillcolor{currentfill}%
\pgfsetlinewidth{0.481800pt}%
\definecolor{currentstroke}{rgb}{1.000000,1.000000,1.000000}%
\pgfsetstrokecolor{currentstroke}%
\pgfsetdash{}{0pt}%
\pgfpathmoveto{\pgfqpoint{6.637221in}{8.794594in}}%
\pgfpathcurveto{\pgfqpoint{6.648271in}{8.794594in}}{\pgfqpoint{6.658870in}{8.798985in}}{\pgfqpoint{6.666684in}{8.806798in}}%
\pgfpathcurveto{\pgfqpoint{6.674497in}{8.814612in}}{\pgfqpoint{6.678888in}{8.825211in}}{\pgfqpoint{6.678888in}{8.836261in}}%
\pgfpathcurveto{\pgfqpoint{6.678888in}{8.847311in}}{\pgfqpoint{6.674497in}{8.857910in}}{\pgfqpoint{6.666684in}{8.865724in}}%
\pgfpathcurveto{\pgfqpoint{6.658870in}{8.873537in}}{\pgfqpoint{6.648271in}{8.877928in}}{\pgfqpoint{6.637221in}{8.877928in}}%
\pgfpathcurveto{\pgfqpoint{6.626171in}{8.877928in}}{\pgfqpoint{6.615572in}{8.873537in}}{\pgfqpoint{6.607758in}{8.865724in}}%
\pgfpathcurveto{\pgfqpoint{6.599945in}{8.857910in}}{\pgfqpoint{6.595554in}{8.847311in}}{\pgfqpoint{6.595554in}{8.836261in}}%
\pgfpathcurveto{\pgfqpoint{6.595554in}{8.825211in}}{\pgfqpoint{6.599945in}{8.814612in}}{\pgfqpoint{6.607758in}{8.806798in}}%
\pgfpathcurveto{\pgfqpoint{6.615572in}{8.798985in}}{\pgfqpoint{6.626171in}{8.794594in}}{\pgfqpoint{6.637221in}{8.794594in}}%
\pgfpathlineto{\pgfqpoint{6.637221in}{8.794594in}}%
\pgfpathclose%
\pgfusepath{stroke,fill}%
\end{pgfscope}%
\begin{pgfscope}%
\pgfpathrectangle{\pgfqpoint{5.292946in}{7.624184in}}{\pgfqpoint{2.177280in}{2.201755in}}%
\pgfusepath{clip}%
\pgfsetbuttcap%
\pgfsetroundjoin%
\definecolor{currentfill}{rgb}{0.172549,0.627451,0.172549}%
\pgfsetfillcolor{currentfill}%
\pgfsetlinewidth{0.481800pt}%
\definecolor{currentstroke}{rgb}{1.000000,1.000000,1.000000}%
\pgfsetstrokecolor{currentstroke}%
\pgfsetdash{}{0pt}%
\pgfpathmoveto{\pgfqpoint{6.780747in}{8.683395in}}%
\pgfpathcurveto{\pgfqpoint{6.791798in}{8.683395in}}{\pgfqpoint{6.802397in}{8.687785in}}{\pgfqpoint{6.810210in}{8.695598in}}%
\pgfpathcurveto{\pgfqpoint{6.818024in}{8.703412in}}{\pgfqpoint{6.822414in}{8.714011in}}{\pgfqpoint{6.822414in}{8.725061in}}%
\pgfpathcurveto{\pgfqpoint{6.822414in}{8.736111in}}{\pgfqpoint{6.818024in}{8.746710in}}{\pgfqpoint{6.810210in}{8.754524in}}%
\pgfpathcurveto{\pgfqpoint{6.802397in}{8.762338in}}{\pgfqpoint{6.791798in}{8.766728in}}{\pgfqpoint{6.780747in}{8.766728in}}%
\pgfpathcurveto{\pgfqpoint{6.769697in}{8.766728in}}{\pgfqpoint{6.759098in}{8.762338in}}{\pgfqpoint{6.751285in}{8.754524in}}%
\pgfpathcurveto{\pgfqpoint{6.743471in}{8.746710in}}{\pgfqpoint{6.739081in}{8.736111in}}{\pgfqpoint{6.739081in}{8.725061in}}%
\pgfpathcurveto{\pgfqpoint{6.739081in}{8.714011in}}{\pgfqpoint{6.743471in}{8.703412in}}{\pgfqpoint{6.751285in}{8.695598in}}%
\pgfpathcurveto{\pgfqpoint{6.759098in}{8.687785in}}{\pgfqpoint{6.769697in}{8.683395in}}{\pgfqpoint{6.780747in}{8.683395in}}%
\pgfpathlineto{\pgfqpoint{6.780747in}{8.683395in}}%
\pgfpathclose%
\pgfusepath{stroke,fill}%
\end{pgfscope}%
\begin{pgfscope}%
\pgfpathrectangle{\pgfqpoint{5.292946in}{7.624184in}}{\pgfqpoint{2.177280in}{2.201755in}}%
\pgfusepath{clip}%
\pgfsetbuttcap%
\pgfsetroundjoin%
\definecolor{currentfill}{rgb}{0.172549,0.627451,0.172549}%
\pgfsetfillcolor{currentfill}%
\pgfsetlinewidth{0.481800pt}%
\definecolor{currentstroke}{rgb}{1.000000,1.000000,1.000000}%
\pgfsetstrokecolor{currentstroke}%
\pgfsetdash{}{0pt}%
\pgfpathmoveto{\pgfqpoint{6.924274in}{9.572993in}}%
\pgfpathcurveto{\pgfqpoint{6.935324in}{9.572993in}}{\pgfqpoint{6.945923in}{9.577383in}}{\pgfqpoint{6.953737in}{9.585197in}}%
\pgfpathcurveto{\pgfqpoint{6.961550in}{9.593010in}}{\pgfqpoint{6.965941in}{9.603609in}}{\pgfqpoint{6.965941in}{9.614659in}}%
\pgfpathcurveto{\pgfqpoint{6.965941in}{9.625709in}}{\pgfqpoint{6.961550in}{9.636308in}}{\pgfqpoint{6.953737in}{9.644122in}}%
\pgfpathcurveto{\pgfqpoint{6.945923in}{9.651936in}}{\pgfqpoint{6.935324in}{9.656326in}}{\pgfqpoint{6.924274in}{9.656326in}}%
\pgfpathcurveto{\pgfqpoint{6.913224in}{9.656326in}}{\pgfqpoint{6.902625in}{9.651936in}}{\pgfqpoint{6.894811in}{9.644122in}}%
\pgfpathcurveto{\pgfqpoint{6.886997in}{9.636308in}}{\pgfqpoint{6.882607in}{9.625709in}}{\pgfqpoint{6.882607in}{9.614659in}}%
\pgfpathcurveto{\pgfqpoint{6.882607in}{9.603609in}}{\pgfqpoint{6.886997in}{9.593010in}}{\pgfqpoint{6.894811in}{9.585197in}}%
\pgfpathcurveto{\pgfqpoint{6.902625in}{9.577383in}}{\pgfqpoint{6.913224in}{9.572993in}}{\pgfqpoint{6.924274in}{9.572993in}}%
\pgfpathlineto{\pgfqpoint{6.924274in}{9.572993in}}%
\pgfpathclose%
\pgfusepath{stroke,fill}%
\end{pgfscope}%
\begin{pgfscope}%
\pgfpathrectangle{\pgfqpoint{5.292946in}{7.624184in}}{\pgfqpoint{2.177280in}{2.201755in}}%
\pgfusepath{clip}%
\pgfsetbuttcap%
\pgfsetroundjoin%
\definecolor{currentfill}{rgb}{0.172549,0.627451,0.172549}%
\pgfsetfillcolor{currentfill}%
\pgfsetlinewidth{0.481800pt}%
\definecolor{currentstroke}{rgb}{1.000000,1.000000,1.000000}%
\pgfsetstrokecolor{currentstroke}%
\pgfsetdash{}{0pt}%
\pgfpathmoveto{\pgfqpoint{6.780747in}{8.794594in}}%
\pgfpathcurveto{\pgfqpoint{6.791798in}{8.794594in}}{\pgfqpoint{6.802397in}{8.798985in}}{\pgfqpoint{6.810210in}{8.806798in}}%
\pgfpathcurveto{\pgfqpoint{6.818024in}{8.814612in}}{\pgfqpoint{6.822414in}{8.825211in}}{\pgfqpoint{6.822414in}{8.836261in}}%
\pgfpathcurveto{\pgfqpoint{6.822414in}{8.847311in}}{\pgfqpoint{6.818024in}{8.857910in}}{\pgfqpoint{6.810210in}{8.865724in}}%
\pgfpathcurveto{\pgfqpoint{6.802397in}{8.873537in}}{\pgfqpoint{6.791798in}{8.877928in}}{\pgfqpoint{6.780747in}{8.877928in}}%
\pgfpathcurveto{\pgfqpoint{6.769697in}{8.877928in}}{\pgfqpoint{6.759098in}{8.873537in}}{\pgfqpoint{6.751285in}{8.865724in}}%
\pgfpathcurveto{\pgfqpoint{6.743471in}{8.857910in}}{\pgfqpoint{6.739081in}{8.847311in}}{\pgfqpoint{6.739081in}{8.836261in}}%
\pgfpathcurveto{\pgfqpoint{6.739081in}{8.825211in}}{\pgfqpoint{6.743471in}{8.814612in}}{\pgfqpoint{6.751285in}{8.806798in}}%
\pgfpathcurveto{\pgfqpoint{6.759098in}{8.798985in}}{\pgfqpoint{6.769697in}{8.794594in}}{\pgfqpoint{6.780747in}{8.794594in}}%
\pgfpathlineto{\pgfqpoint{6.780747in}{8.794594in}}%
\pgfpathclose%
\pgfusepath{stroke,fill}%
\end{pgfscope}%
\begin{pgfscope}%
\pgfpathrectangle{\pgfqpoint{5.292946in}{7.624184in}}{\pgfqpoint{2.177280in}{2.201755in}}%
\pgfusepath{clip}%
\pgfsetbuttcap%
\pgfsetroundjoin%
\definecolor{currentfill}{rgb}{0.172549,0.627451,0.172549}%
\pgfsetfillcolor{currentfill}%
\pgfsetlinewidth{0.481800pt}%
\definecolor{currentstroke}{rgb}{1.000000,1.000000,1.000000}%
\pgfsetstrokecolor{currentstroke}%
\pgfsetdash{}{0pt}%
\pgfpathmoveto{\pgfqpoint{6.752042in}{8.850194in}}%
\pgfpathcurveto{\pgfqpoint{6.763092in}{8.850194in}}{\pgfqpoint{6.773691in}{8.854584in}}{\pgfqpoint{6.781505in}{8.862398in}}%
\pgfpathcurveto{\pgfqpoint{6.789319in}{8.870212in}}{\pgfqpoint{6.793709in}{8.880811in}}{\pgfqpoint{6.793709in}{8.891861in}}%
\pgfpathcurveto{\pgfqpoint{6.793709in}{8.902911in}}{\pgfqpoint{6.789319in}{8.913510in}}{\pgfqpoint{6.781505in}{8.921324in}}%
\pgfpathcurveto{\pgfqpoint{6.773691in}{8.929137in}}{\pgfqpoint{6.763092in}{8.933528in}}{\pgfqpoint{6.752042in}{8.933528in}}%
\pgfpathcurveto{\pgfqpoint{6.740992in}{8.933528in}}{\pgfqpoint{6.730393in}{8.929137in}}{\pgfqpoint{6.722579in}{8.921324in}}%
\pgfpathcurveto{\pgfqpoint{6.714766in}{8.913510in}}{\pgfqpoint{6.710375in}{8.902911in}}{\pgfqpoint{6.710375in}{8.891861in}}%
\pgfpathcurveto{\pgfqpoint{6.710375in}{8.880811in}}{\pgfqpoint{6.714766in}{8.870212in}}{\pgfqpoint{6.722579in}{8.862398in}}%
\pgfpathcurveto{\pgfqpoint{6.730393in}{8.854584in}}{\pgfqpoint{6.740992in}{8.850194in}}{\pgfqpoint{6.752042in}{8.850194in}}%
\pgfpathlineto{\pgfqpoint{6.752042in}{8.850194in}}%
\pgfpathclose%
\pgfusepath{stroke,fill}%
\end{pgfscope}%
\begin{pgfscope}%
\pgfpathrectangle{\pgfqpoint{5.292946in}{7.624184in}}{\pgfqpoint{2.177280in}{2.201755in}}%
\pgfusepath{clip}%
\pgfsetbuttcap%
\pgfsetroundjoin%
\definecolor{currentfill}{rgb}{0.172549,0.627451,0.172549}%
\pgfsetfillcolor{currentfill}%
\pgfsetlinewidth{0.481800pt}%
\definecolor{currentstroke}{rgb}{1.000000,1.000000,1.000000}%
\pgfsetstrokecolor{currentstroke}%
\pgfsetdash{}{0pt}%
\pgfpathmoveto{\pgfqpoint{6.551105in}{8.627795in}}%
\pgfpathcurveto{\pgfqpoint{6.562155in}{8.627795in}}{\pgfqpoint{6.572754in}{8.632185in}}{\pgfqpoint{6.580568in}{8.639999in}}%
\pgfpathcurveto{\pgfqpoint{6.588382in}{8.647812in}}{\pgfqpoint{6.592772in}{8.658411in}}{\pgfqpoint{6.592772in}{8.669461in}}%
\pgfpathcurveto{\pgfqpoint{6.592772in}{8.680512in}}{\pgfqpoint{6.588382in}{8.691111in}}{\pgfqpoint{6.580568in}{8.698924in}}%
\pgfpathcurveto{\pgfqpoint{6.572754in}{8.706738in}}{\pgfqpoint{6.562155in}{8.711128in}}{\pgfqpoint{6.551105in}{8.711128in}}%
\pgfpathcurveto{\pgfqpoint{6.540055in}{8.711128in}}{\pgfqpoint{6.529456in}{8.706738in}}{\pgfqpoint{6.521642in}{8.698924in}}%
\pgfpathcurveto{\pgfqpoint{6.513829in}{8.691111in}}{\pgfqpoint{6.509438in}{8.680512in}}{\pgfqpoint{6.509438in}{8.669461in}}%
\pgfpathcurveto{\pgfqpoint{6.509438in}{8.658411in}}{\pgfqpoint{6.513829in}{8.647812in}}{\pgfqpoint{6.521642in}{8.639999in}}%
\pgfpathcurveto{\pgfqpoint{6.529456in}{8.632185in}}{\pgfqpoint{6.540055in}{8.627795in}}{\pgfqpoint{6.551105in}{8.627795in}}%
\pgfpathlineto{\pgfqpoint{6.551105in}{8.627795in}}%
\pgfpathclose%
\pgfusepath{stroke,fill}%
\end{pgfscope}%
\begin{pgfscope}%
\pgfpathrectangle{\pgfqpoint{5.292946in}{7.624184in}}{\pgfqpoint{2.177280in}{2.201755in}}%
\pgfusepath{clip}%
\pgfsetbuttcap%
\pgfsetroundjoin%
\definecolor{currentfill}{rgb}{0.172549,0.627451,0.172549}%
\pgfsetfillcolor{currentfill}%
\pgfsetlinewidth{0.481800pt}%
\definecolor{currentstroke}{rgb}{1.000000,1.000000,1.000000}%
\pgfsetstrokecolor{currentstroke}%
\pgfsetdash{}{0pt}%
\pgfpathmoveto{\pgfqpoint{6.723337in}{9.128194in}}%
\pgfpathcurveto{\pgfqpoint{6.734387in}{9.128194in}}{\pgfqpoint{6.744986in}{9.132584in}}{\pgfqpoint{6.752800in}{9.140397in}}%
\pgfpathcurveto{\pgfqpoint{6.760613in}{9.148211in}}{\pgfqpoint{6.765004in}{9.158810in}}{\pgfqpoint{6.765004in}{9.169860in}}%
\pgfpathcurveto{\pgfqpoint{6.765004in}{9.180910in}}{\pgfqpoint{6.760613in}{9.191509in}}{\pgfqpoint{6.752800in}{9.199323in}}%
\pgfpathcurveto{\pgfqpoint{6.744986in}{9.207137in}}{\pgfqpoint{6.734387in}{9.211527in}}{\pgfqpoint{6.723337in}{9.211527in}}%
\pgfpathcurveto{\pgfqpoint{6.712287in}{9.211527in}}{\pgfqpoint{6.701688in}{9.207137in}}{\pgfqpoint{6.693874in}{9.199323in}}%
\pgfpathcurveto{\pgfqpoint{6.686060in}{9.191509in}}{\pgfqpoint{6.681670in}{9.180910in}}{\pgfqpoint{6.681670in}{9.169860in}}%
\pgfpathcurveto{\pgfqpoint{6.681670in}{9.158810in}}{\pgfqpoint{6.686060in}{9.148211in}}{\pgfqpoint{6.693874in}{9.140397in}}%
\pgfpathcurveto{\pgfqpoint{6.701688in}{9.132584in}}{\pgfqpoint{6.712287in}{9.128194in}}{\pgfqpoint{6.723337in}{9.128194in}}%
\pgfpathlineto{\pgfqpoint{6.723337in}{9.128194in}}%
\pgfpathclose%
\pgfusepath{stroke,fill}%
\end{pgfscope}%
\begin{pgfscope}%
\pgfpathrectangle{\pgfqpoint{5.292946in}{7.624184in}}{\pgfqpoint{2.177280in}{2.201755in}}%
\pgfusepath{clip}%
\pgfsetbuttcap%
\pgfsetroundjoin%
\definecolor{currentfill}{rgb}{0.172549,0.627451,0.172549}%
\pgfsetfillcolor{currentfill}%
\pgfsetlinewidth{0.481800pt}%
\definecolor{currentstroke}{rgb}{1.000000,1.000000,1.000000}%
\pgfsetstrokecolor{currentstroke}%
\pgfsetdash{}{0pt}%
\pgfpathmoveto{\pgfqpoint{6.780747in}{9.016994in}}%
\pgfpathcurveto{\pgfqpoint{6.791798in}{9.016994in}}{\pgfqpoint{6.802397in}{9.021384in}}{\pgfqpoint{6.810210in}{9.029198in}}%
\pgfpathcurveto{\pgfqpoint{6.818024in}{9.037011in}}{\pgfqpoint{6.822414in}{9.047610in}}{\pgfqpoint{6.822414in}{9.058661in}}%
\pgfpathcurveto{\pgfqpoint{6.822414in}{9.069711in}}{\pgfqpoint{6.818024in}{9.080310in}}{\pgfqpoint{6.810210in}{9.088123in}}%
\pgfpathcurveto{\pgfqpoint{6.802397in}{9.095937in}}{\pgfqpoint{6.791798in}{9.100327in}}{\pgfqpoint{6.780747in}{9.100327in}}%
\pgfpathcurveto{\pgfqpoint{6.769697in}{9.100327in}}{\pgfqpoint{6.759098in}{9.095937in}}{\pgfqpoint{6.751285in}{9.088123in}}%
\pgfpathcurveto{\pgfqpoint{6.743471in}{9.080310in}}{\pgfqpoint{6.739081in}{9.069711in}}{\pgfqpoint{6.739081in}{9.058661in}}%
\pgfpathcurveto{\pgfqpoint{6.739081in}{9.047610in}}{\pgfqpoint{6.743471in}{9.037011in}}{\pgfqpoint{6.751285in}{9.029198in}}%
\pgfpathcurveto{\pgfqpoint{6.759098in}{9.021384in}}{\pgfqpoint{6.769697in}{9.016994in}}{\pgfqpoint{6.780747in}{9.016994in}}%
\pgfpathlineto{\pgfqpoint{6.780747in}{9.016994in}}%
\pgfpathclose%
\pgfusepath{stroke,fill}%
\end{pgfscope}%
\begin{pgfscope}%
\pgfpathrectangle{\pgfqpoint{5.292946in}{7.624184in}}{\pgfqpoint{2.177280in}{2.201755in}}%
\pgfusepath{clip}%
\pgfsetbuttcap%
\pgfsetroundjoin%
\definecolor{currentfill}{rgb}{0.172549,0.627451,0.172549}%
\pgfsetfillcolor{currentfill}%
\pgfsetlinewidth{0.481800pt}%
\definecolor{currentstroke}{rgb}{1.000000,1.000000,1.000000}%
\pgfsetstrokecolor{currentstroke}%
\pgfsetdash{}{0pt}%
\pgfpathmoveto{\pgfqpoint{6.637221in}{9.128194in}}%
\pgfpathcurveto{\pgfqpoint{6.648271in}{9.128194in}}{\pgfqpoint{6.658870in}{9.132584in}}{\pgfqpoint{6.666684in}{9.140397in}}%
\pgfpathcurveto{\pgfqpoint{6.674497in}{9.148211in}}{\pgfqpoint{6.678888in}{9.158810in}}{\pgfqpoint{6.678888in}{9.169860in}}%
\pgfpathcurveto{\pgfqpoint{6.678888in}{9.180910in}}{\pgfqpoint{6.674497in}{9.191509in}}{\pgfqpoint{6.666684in}{9.199323in}}%
\pgfpathcurveto{\pgfqpoint{6.658870in}{9.207137in}}{\pgfqpoint{6.648271in}{9.211527in}}{\pgfqpoint{6.637221in}{9.211527in}}%
\pgfpathcurveto{\pgfqpoint{6.626171in}{9.211527in}}{\pgfqpoint{6.615572in}{9.207137in}}{\pgfqpoint{6.607758in}{9.199323in}}%
\pgfpathcurveto{\pgfqpoint{6.599945in}{9.191509in}}{\pgfqpoint{6.595554in}{9.180910in}}{\pgfqpoint{6.595554in}{9.169860in}}%
\pgfpathcurveto{\pgfqpoint{6.595554in}{9.158810in}}{\pgfqpoint{6.599945in}{9.148211in}}{\pgfqpoint{6.607758in}{9.140397in}}%
\pgfpathcurveto{\pgfqpoint{6.615572in}{9.132584in}}{\pgfqpoint{6.626171in}{9.128194in}}{\pgfqpoint{6.637221in}{9.128194in}}%
\pgfpathlineto{\pgfqpoint{6.637221in}{9.128194in}}%
\pgfpathclose%
\pgfusepath{stroke,fill}%
\end{pgfscope}%
\begin{pgfscope}%
\pgfpathrectangle{\pgfqpoint{5.292946in}{7.624184in}}{\pgfqpoint{2.177280in}{2.201755in}}%
\pgfusepath{clip}%
\pgfsetbuttcap%
\pgfsetroundjoin%
\definecolor{currentfill}{rgb}{0.172549,0.627451,0.172549}%
\pgfsetfillcolor{currentfill}%
\pgfsetlinewidth{0.481800pt}%
\definecolor{currentstroke}{rgb}{1.000000,1.000000,1.000000}%
\pgfsetstrokecolor{currentstroke}%
\pgfsetdash{}{0pt}%
\pgfpathmoveto{\pgfqpoint{6.637221in}{8.516595in}}%
\pgfpathcurveto{\pgfqpoint{6.648271in}{8.516595in}}{\pgfqpoint{6.658870in}{8.520985in}}{\pgfqpoint{6.666684in}{8.528799in}}%
\pgfpathcurveto{\pgfqpoint{6.674497in}{8.536612in}}{\pgfqpoint{6.678888in}{8.547212in}}{\pgfqpoint{6.678888in}{8.558262in}}%
\pgfpathcurveto{\pgfqpoint{6.678888in}{8.569312in}}{\pgfqpoint{6.674497in}{8.579911in}}{\pgfqpoint{6.666684in}{8.587724in}}%
\pgfpathcurveto{\pgfqpoint{6.658870in}{8.595538in}}{\pgfqpoint{6.648271in}{8.599928in}}{\pgfqpoint{6.637221in}{8.599928in}}%
\pgfpathcurveto{\pgfqpoint{6.626171in}{8.599928in}}{\pgfqpoint{6.615572in}{8.595538in}}{\pgfqpoint{6.607758in}{8.587724in}}%
\pgfpathcurveto{\pgfqpoint{6.599945in}{8.579911in}}{\pgfqpoint{6.595554in}{8.569312in}}{\pgfqpoint{6.595554in}{8.558262in}}%
\pgfpathcurveto{\pgfqpoint{6.595554in}{8.547212in}}{\pgfqpoint{6.599945in}{8.536612in}}{\pgfqpoint{6.607758in}{8.528799in}}%
\pgfpathcurveto{\pgfqpoint{6.615572in}{8.520985in}}{\pgfqpoint{6.626171in}{8.516595in}}{\pgfqpoint{6.637221in}{8.516595in}}%
\pgfpathlineto{\pgfqpoint{6.637221in}{8.516595in}}%
\pgfpathclose%
\pgfusepath{stroke,fill}%
\end{pgfscope}%
\begin{pgfscope}%
\pgfpathrectangle{\pgfqpoint{5.292946in}{7.624184in}}{\pgfqpoint{2.177280in}{2.201755in}}%
\pgfusepath{clip}%
\pgfsetbuttcap%
\pgfsetroundjoin%
\definecolor{currentfill}{rgb}{0.172549,0.627451,0.172549}%
\pgfsetfillcolor{currentfill}%
\pgfsetlinewidth{0.481800pt}%
\definecolor{currentstroke}{rgb}{1.000000,1.000000,1.000000}%
\pgfsetstrokecolor{currentstroke}%
\pgfsetdash{}{0pt}%
\pgfpathmoveto{\pgfqpoint{6.866863in}{9.072594in}}%
\pgfpathcurveto{\pgfqpoint{6.877913in}{9.072594in}}{\pgfqpoint{6.888512in}{9.076984in}}{\pgfqpoint{6.896326in}{9.084798in}}%
\pgfpathcurveto{\pgfqpoint{6.904140in}{9.092611in}}{\pgfqpoint{6.908530in}{9.103210in}}{\pgfqpoint{6.908530in}{9.114260in}}%
\pgfpathcurveto{\pgfqpoint{6.908530in}{9.125311in}}{\pgfqpoint{6.904140in}{9.135910in}}{\pgfqpoint{6.896326in}{9.143723in}}%
\pgfpathcurveto{\pgfqpoint{6.888512in}{9.151537in}}{\pgfqpoint{6.877913in}{9.155927in}}{\pgfqpoint{6.866863in}{9.155927in}}%
\pgfpathcurveto{\pgfqpoint{6.855813in}{9.155927in}}{\pgfqpoint{6.845214in}{9.151537in}}{\pgfqpoint{6.837401in}{9.143723in}}%
\pgfpathcurveto{\pgfqpoint{6.829587in}{9.135910in}}{\pgfqpoint{6.825197in}{9.125311in}}{\pgfqpoint{6.825197in}{9.114260in}}%
\pgfpathcurveto{\pgfqpoint{6.825197in}{9.103210in}}{\pgfqpoint{6.829587in}{9.092611in}}{\pgfqpoint{6.837401in}{9.084798in}}%
\pgfpathcurveto{\pgfqpoint{6.845214in}{9.076984in}}{\pgfqpoint{6.855813in}{9.072594in}}{\pgfqpoint{6.866863in}{9.072594in}}%
\pgfpathlineto{\pgfqpoint{6.866863in}{9.072594in}}%
\pgfpathclose%
\pgfusepath{stroke,fill}%
\end{pgfscope}%
\begin{pgfscope}%
\pgfpathrectangle{\pgfqpoint{5.292946in}{7.624184in}}{\pgfqpoint{2.177280in}{2.201755in}}%
\pgfusepath{clip}%
\pgfsetbuttcap%
\pgfsetroundjoin%
\definecolor{currentfill}{rgb}{0.172549,0.627451,0.172549}%
\pgfsetfillcolor{currentfill}%
\pgfsetlinewidth{0.481800pt}%
\definecolor{currentstroke}{rgb}{1.000000,1.000000,1.000000}%
\pgfsetstrokecolor{currentstroke}%
\pgfsetdash{}{0pt}%
\pgfpathmoveto{\pgfqpoint{6.809453in}{9.016994in}}%
\pgfpathcurveto{\pgfqpoint{6.820503in}{9.016994in}}{\pgfqpoint{6.831102in}{9.021384in}}{\pgfqpoint{6.838915in}{9.029198in}}%
\pgfpathcurveto{\pgfqpoint{6.846729in}{9.037011in}}{\pgfqpoint{6.851119in}{9.047610in}}{\pgfqpoint{6.851119in}{9.058661in}}%
\pgfpathcurveto{\pgfqpoint{6.851119in}{9.069711in}}{\pgfqpoint{6.846729in}{9.080310in}}{\pgfqpoint{6.838915in}{9.088123in}}%
\pgfpathcurveto{\pgfqpoint{6.831102in}{9.095937in}}{\pgfqpoint{6.820503in}{9.100327in}}{\pgfqpoint{6.809453in}{9.100327in}}%
\pgfpathcurveto{\pgfqpoint{6.798403in}{9.100327in}}{\pgfqpoint{6.787804in}{9.095937in}}{\pgfqpoint{6.779990in}{9.088123in}}%
\pgfpathcurveto{\pgfqpoint{6.772176in}{9.080310in}}{\pgfqpoint{6.767786in}{9.069711in}}{\pgfqpoint{6.767786in}{9.058661in}}%
\pgfpathcurveto{\pgfqpoint{6.767786in}{9.047610in}}{\pgfqpoint{6.772176in}{9.037011in}}{\pgfqpoint{6.779990in}{9.029198in}}%
\pgfpathcurveto{\pgfqpoint{6.787804in}{9.021384in}}{\pgfqpoint{6.798403in}{9.016994in}}{\pgfqpoint{6.809453in}{9.016994in}}%
\pgfpathlineto{\pgfqpoint{6.809453in}{9.016994in}}%
\pgfpathclose%
\pgfusepath{stroke,fill}%
\end{pgfscope}%
\begin{pgfscope}%
\pgfpathrectangle{\pgfqpoint{5.292946in}{7.624184in}}{\pgfqpoint{2.177280in}{2.201755in}}%
\pgfusepath{clip}%
\pgfsetbuttcap%
\pgfsetroundjoin%
\definecolor{currentfill}{rgb}{0.172549,0.627451,0.172549}%
\pgfsetfillcolor{currentfill}%
\pgfsetlinewidth{0.481800pt}%
\definecolor{currentstroke}{rgb}{1.000000,1.000000,1.000000}%
\pgfsetstrokecolor{currentstroke}%
\pgfsetdash{}{0pt}%
\pgfpathmoveto{\pgfqpoint{6.665926in}{9.016994in}}%
\pgfpathcurveto{\pgfqpoint{6.676976in}{9.016994in}}{\pgfqpoint{6.687575in}{9.021384in}}{\pgfqpoint{6.695389in}{9.029198in}}%
\pgfpathcurveto{\pgfqpoint{6.703203in}{9.037011in}}{\pgfqpoint{6.707593in}{9.047610in}}{\pgfqpoint{6.707593in}{9.058661in}}%
\pgfpathcurveto{\pgfqpoint{6.707593in}{9.069711in}}{\pgfqpoint{6.703203in}{9.080310in}}{\pgfqpoint{6.695389in}{9.088123in}}%
\pgfpathcurveto{\pgfqpoint{6.687575in}{9.095937in}}{\pgfqpoint{6.676976in}{9.100327in}}{\pgfqpoint{6.665926in}{9.100327in}}%
\pgfpathcurveto{\pgfqpoint{6.654876in}{9.100327in}}{\pgfqpoint{6.644277in}{9.095937in}}{\pgfqpoint{6.636464in}{9.088123in}}%
\pgfpathcurveto{\pgfqpoint{6.628650in}{9.080310in}}{\pgfqpoint{6.624260in}{9.069711in}}{\pgfqpoint{6.624260in}{9.058661in}}%
\pgfpathcurveto{\pgfqpoint{6.624260in}{9.047610in}}{\pgfqpoint{6.628650in}{9.037011in}}{\pgfqpoint{6.636464in}{9.029198in}}%
\pgfpathcurveto{\pgfqpoint{6.644277in}{9.021384in}}{\pgfqpoint{6.654876in}{9.016994in}}{\pgfqpoint{6.665926in}{9.016994in}}%
\pgfpathlineto{\pgfqpoint{6.665926in}{9.016994in}}%
\pgfpathclose%
\pgfusepath{stroke,fill}%
\end{pgfscope}%
\begin{pgfscope}%
\pgfpathrectangle{\pgfqpoint{5.292946in}{7.624184in}}{\pgfqpoint{2.177280in}{2.201755in}}%
\pgfusepath{clip}%
\pgfsetbuttcap%
\pgfsetroundjoin%
\definecolor{currentfill}{rgb}{0.172549,0.627451,0.172549}%
\pgfsetfillcolor{currentfill}%
\pgfsetlinewidth{0.481800pt}%
\definecolor{currentstroke}{rgb}{1.000000,1.000000,1.000000}%
\pgfsetstrokecolor{currentstroke}%
\pgfsetdash{}{0pt}%
\pgfpathmoveto{\pgfqpoint{6.608516in}{8.794594in}}%
\pgfpathcurveto{\pgfqpoint{6.619566in}{8.794594in}}{\pgfqpoint{6.630165in}{8.798985in}}{\pgfqpoint{6.637978in}{8.806798in}}%
\pgfpathcurveto{\pgfqpoint{6.645792in}{8.814612in}}{\pgfqpoint{6.650182in}{8.825211in}}{\pgfqpoint{6.650182in}{8.836261in}}%
\pgfpathcurveto{\pgfqpoint{6.650182in}{8.847311in}}{\pgfqpoint{6.645792in}{8.857910in}}{\pgfqpoint{6.637978in}{8.865724in}}%
\pgfpathcurveto{\pgfqpoint{6.630165in}{8.873537in}}{\pgfqpoint{6.619566in}{8.877928in}}{\pgfqpoint{6.608516in}{8.877928in}}%
\pgfpathcurveto{\pgfqpoint{6.597466in}{8.877928in}}{\pgfqpoint{6.586867in}{8.873537in}}{\pgfqpoint{6.579053in}{8.865724in}}%
\pgfpathcurveto{\pgfqpoint{6.571239in}{8.857910in}}{\pgfqpoint{6.566849in}{8.847311in}}{\pgfqpoint{6.566849in}{8.836261in}}%
\pgfpathcurveto{\pgfqpoint{6.566849in}{8.825211in}}{\pgfqpoint{6.571239in}{8.814612in}}{\pgfqpoint{6.579053in}{8.806798in}}%
\pgfpathcurveto{\pgfqpoint{6.586867in}{8.798985in}}{\pgfqpoint{6.597466in}{8.794594in}}{\pgfqpoint{6.608516in}{8.794594in}}%
\pgfpathlineto{\pgfqpoint{6.608516in}{8.794594in}}%
\pgfpathclose%
\pgfusepath{stroke,fill}%
\end{pgfscope}%
\begin{pgfscope}%
\pgfpathrectangle{\pgfqpoint{5.292946in}{7.624184in}}{\pgfqpoint{2.177280in}{2.201755in}}%
\pgfusepath{clip}%
\pgfsetbuttcap%
\pgfsetroundjoin%
\definecolor{currentfill}{rgb}{0.172549,0.627451,0.172549}%
\pgfsetfillcolor{currentfill}%
\pgfsetlinewidth{0.481800pt}%
\definecolor{currentstroke}{rgb}{1.000000,1.000000,1.000000}%
\pgfsetstrokecolor{currentstroke}%
\pgfsetdash{}{0pt}%
\pgfpathmoveto{\pgfqpoint{6.665926in}{8.905794in}}%
\pgfpathcurveto{\pgfqpoint{6.676976in}{8.905794in}}{\pgfqpoint{6.687575in}{8.910184in}}{\pgfqpoint{6.695389in}{8.917998in}}%
\pgfpathcurveto{\pgfqpoint{6.703203in}{8.925812in}}{\pgfqpoint{6.707593in}{8.936411in}}{\pgfqpoint{6.707593in}{8.947461in}}%
\pgfpathcurveto{\pgfqpoint{6.707593in}{8.958511in}}{\pgfqpoint{6.703203in}{8.969110in}}{\pgfqpoint{6.695389in}{8.976924in}}%
\pgfpathcurveto{\pgfqpoint{6.687575in}{8.984737in}}{\pgfqpoint{6.676976in}{8.989127in}}{\pgfqpoint{6.665926in}{8.989127in}}%
\pgfpathcurveto{\pgfqpoint{6.654876in}{8.989127in}}{\pgfqpoint{6.644277in}{8.984737in}}{\pgfqpoint{6.636464in}{8.976924in}}%
\pgfpathcurveto{\pgfqpoint{6.628650in}{8.969110in}}{\pgfqpoint{6.624260in}{8.958511in}}{\pgfqpoint{6.624260in}{8.947461in}}%
\pgfpathcurveto{\pgfqpoint{6.624260in}{8.936411in}}{\pgfqpoint{6.628650in}{8.925812in}}{\pgfqpoint{6.636464in}{8.917998in}}%
\pgfpathcurveto{\pgfqpoint{6.644277in}{8.910184in}}{\pgfqpoint{6.654876in}{8.905794in}}{\pgfqpoint{6.665926in}{8.905794in}}%
\pgfpathlineto{\pgfqpoint{6.665926in}{8.905794in}}%
\pgfpathclose%
\pgfusepath{stroke,fill}%
\end{pgfscope}%
\begin{pgfscope}%
\pgfpathrectangle{\pgfqpoint{5.292946in}{7.624184in}}{\pgfqpoint{2.177280in}{2.201755in}}%
\pgfusepath{clip}%
\pgfsetbuttcap%
\pgfsetroundjoin%
\definecolor{currentfill}{rgb}{0.172549,0.627451,0.172549}%
\pgfsetfillcolor{currentfill}%
\pgfsetlinewidth{0.481800pt}%
\definecolor{currentstroke}{rgb}{1.000000,1.000000,1.000000}%
\pgfsetstrokecolor{currentstroke}%
\pgfsetdash{}{0pt}%
\pgfpathmoveto{\pgfqpoint{6.723337in}{8.738994in}}%
\pgfpathcurveto{\pgfqpoint{6.734387in}{8.738994in}}{\pgfqpoint{6.744986in}{8.743385in}}{\pgfqpoint{6.752800in}{8.751198in}}%
\pgfpathcurveto{\pgfqpoint{6.760613in}{8.759012in}}{\pgfqpoint{6.765004in}{8.769611in}}{\pgfqpoint{6.765004in}{8.780661in}}%
\pgfpathcurveto{\pgfqpoint{6.765004in}{8.791711in}}{\pgfqpoint{6.760613in}{8.802310in}}{\pgfqpoint{6.752800in}{8.810124in}}%
\pgfpathcurveto{\pgfqpoint{6.744986in}{8.817938in}}{\pgfqpoint{6.734387in}{8.822328in}}{\pgfqpoint{6.723337in}{8.822328in}}%
\pgfpathcurveto{\pgfqpoint{6.712287in}{8.822328in}}{\pgfqpoint{6.701688in}{8.817938in}}{\pgfqpoint{6.693874in}{8.810124in}}%
\pgfpathcurveto{\pgfqpoint{6.686060in}{8.802310in}}{\pgfqpoint{6.681670in}{8.791711in}}{\pgfqpoint{6.681670in}{8.780661in}}%
\pgfpathcurveto{\pgfqpoint{6.681670in}{8.769611in}}{\pgfqpoint{6.686060in}{8.759012in}}{\pgfqpoint{6.693874in}{8.751198in}}%
\pgfpathcurveto{\pgfqpoint{6.701688in}{8.743385in}}{\pgfqpoint{6.712287in}{8.738994in}}{\pgfqpoint{6.723337in}{8.738994in}}%
\pgfpathlineto{\pgfqpoint{6.723337in}{8.738994in}}%
\pgfpathclose%
\pgfusepath{stroke,fill}%
\end{pgfscope}%
\begin{pgfscope}%
\pgfpathrectangle{\pgfqpoint{5.292946in}{7.624184in}}{\pgfqpoint{2.177280in}{2.201755in}}%
\pgfusepath{clip}%
\pgfsetbuttcap%
\pgfsetroundjoin%
\definecolor{currentfill}{rgb}{0.172549,0.627451,0.172549}%
\pgfsetfillcolor{currentfill}%
\pgfsetlinewidth{0.481800pt}%
\definecolor{currentstroke}{rgb}{1.000000,1.000000,1.000000}%
\pgfsetstrokecolor{currentstroke}%
\pgfsetdash{}{0pt}%
\pgfpathmoveto{\pgfqpoint{6.637221in}{8.572195in}}%
\pgfpathcurveto{\pgfqpoint{6.648271in}{8.572195in}}{\pgfqpoint{6.658870in}{8.576585in}}{\pgfqpoint{6.666684in}{8.584399in}}%
\pgfpathcurveto{\pgfqpoint{6.674497in}{8.592212in}}{\pgfqpoint{6.678888in}{8.602811in}}{\pgfqpoint{6.678888in}{8.613862in}}%
\pgfpathcurveto{\pgfqpoint{6.678888in}{8.624912in}}{\pgfqpoint{6.674497in}{8.635511in}}{\pgfqpoint{6.666684in}{8.643324in}}%
\pgfpathcurveto{\pgfqpoint{6.658870in}{8.651138in}}{\pgfqpoint{6.648271in}{8.655528in}}{\pgfqpoint{6.637221in}{8.655528in}}%
\pgfpathcurveto{\pgfqpoint{6.626171in}{8.655528in}}{\pgfqpoint{6.615572in}{8.651138in}}{\pgfqpoint{6.607758in}{8.643324in}}%
\pgfpathcurveto{\pgfqpoint{6.599945in}{8.635511in}}{\pgfqpoint{6.595554in}{8.624912in}}{\pgfqpoint{6.595554in}{8.613862in}}%
\pgfpathcurveto{\pgfqpoint{6.595554in}{8.602811in}}{\pgfqpoint{6.599945in}{8.592212in}}{\pgfqpoint{6.607758in}{8.584399in}}%
\pgfpathcurveto{\pgfqpoint{6.615572in}{8.576585in}}{\pgfqpoint{6.626171in}{8.572195in}}{\pgfqpoint{6.637221in}{8.572195in}}%
\pgfpathlineto{\pgfqpoint{6.637221in}{8.572195in}}%
\pgfpathclose%
\pgfusepath{stroke,fill}%
\end{pgfscope}%
\begin{pgfscope}%
\pgfpathrectangle{\pgfqpoint{5.292946in}{7.624184in}}{\pgfqpoint{2.177280in}{2.201755in}}%
\pgfusepath{clip}%
\pgfsetbuttcap%
\pgfsetroundjoin%
\definecolor{currentfill}{rgb}{0.121569,0.466667,0.705882}%
\pgfsetfillcolor{currentfill}%
\pgfsetlinewidth{1.003750pt}%
\definecolor{currentstroke}{rgb}{0.121569,0.466667,0.705882}%
\pgfsetstrokecolor{currentstroke}%
\pgfsetdash{}{0pt}%
\pgfsys@defobject{currentmarker}{\pgfqpoint{-0.041667in}{-0.041667in}}{\pgfqpoint{0.041667in}{0.041667in}}{%
\pgfpathmoveto{\pgfqpoint{0.000000in}{-0.041667in}}%
\pgfpathcurveto{\pgfqpoint{0.011050in}{-0.041667in}}{\pgfqpoint{0.021649in}{-0.037276in}}{\pgfqpoint{0.029463in}{-0.029463in}}%
\pgfpathcurveto{\pgfqpoint{0.037276in}{-0.021649in}}{\pgfqpoint{0.041667in}{-0.011050in}}{\pgfqpoint{0.041667in}{0.000000in}}%
\pgfpathcurveto{\pgfqpoint{0.041667in}{0.011050in}}{\pgfqpoint{0.037276in}{0.021649in}}{\pgfqpoint{0.029463in}{0.029463in}}%
\pgfpathcurveto{\pgfqpoint{0.021649in}{0.037276in}}{\pgfqpoint{0.011050in}{0.041667in}}{\pgfqpoint{0.000000in}{0.041667in}}%
\pgfpathcurveto{\pgfqpoint{-0.011050in}{0.041667in}}{\pgfqpoint{-0.021649in}{0.037276in}}{\pgfqpoint{-0.029463in}{0.029463in}}%
\pgfpathcurveto{\pgfqpoint{-0.037276in}{0.021649in}}{\pgfqpoint{-0.041667in}{0.011050in}}{\pgfqpoint{-0.041667in}{0.000000in}}%
\pgfpathcurveto{\pgfqpoint{-0.041667in}{-0.011050in}}{\pgfqpoint{-0.037276in}{-0.021649in}}{\pgfqpoint{-0.029463in}{-0.029463in}}%
\pgfpathcurveto{\pgfqpoint{-0.021649in}{-0.037276in}}{\pgfqpoint{-0.011050in}{-0.041667in}}{\pgfqpoint{0.000000in}{-0.041667in}}%
\pgfpathlineto{\pgfqpoint{0.000000in}{-0.041667in}}%
\pgfpathclose%
\pgfusepath{stroke,fill}%
}%
\end{pgfscope}%
\begin{pgfscope}%
\pgfpathrectangle{\pgfqpoint{5.292946in}{7.624184in}}{\pgfqpoint{2.177280in}{2.201755in}}%
\pgfusepath{clip}%
\pgfsetbuttcap%
\pgfsetroundjoin%
\definecolor{currentfill}{rgb}{1.000000,0.498039,0.054902}%
\pgfsetfillcolor{currentfill}%
\pgfsetlinewidth{1.003750pt}%
\definecolor{currentstroke}{rgb}{1.000000,0.498039,0.054902}%
\pgfsetstrokecolor{currentstroke}%
\pgfsetdash{}{0pt}%
\pgfsys@defobject{currentmarker}{\pgfqpoint{-0.041667in}{-0.041667in}}{\pgfqpoint{0.041667in}{0.041667in}}{%
\pgfpathmoveto{\pgfqpoint{0.000000in}{-0.041667in}}%
\pgfpathcurveto{\pgfqpoint{0.011050in}{-0.041667in}}{\pgfqpoint{0.021649in}{-0.037276in}}{\pgfqpoint{0.029463in}{-0.029463in}}%
\pgfpathcurveto{\pgfqpoint{0.037276in}{-0.021649in}}{\pgfqpoint{0.041667in}{-0.011050in}}{\pgfqpoint{0.041667in}{0.000000in}}%
\pgfpathcurveto{\pgfqpoint{0.041667in}{0.011050in}}{\pgfqpoint{0.037276in}{0.021649in}}{\pgfqpoint{0.029463in}{0.029463in}}%
\pgfpathcurveto{\pgfqpoint{0.021649in}{0.037276in}}{\pgfqpoint{0.011050in}{0.041667in}}{\pgfqpoint{0.000000in}{0.041667in}}%
\pgfpathcurveto{\pgfqpoint{-0.011050in}{0.041667in}}{\pgfqpoint{-0.021649in}{0.037276in}}{\pgfqpoint{-0.029463in}{0.029463in}}%
\pgfpathcurveto{\pgfqpoint{-0.037276in}{0.021649in}}{\pgfqpoint{-0.041667in}{0.011050in}}{\pgfqpoint{-0.041667in}{0.000000in}}%
\pgfpathcurveto{\pgfqpoint{-0.041667in}{-0.011050in}}{\pgfqpoint{-0.037276in}{-0.021649in}}{\pgfqpoint{-0.029463in}{-0.029463in}}%
\pgfpathcurveto{\pgfqpoint{-0.021649in}{-0.037276in}}{\pgfqpoint{-0.011050in}{-0.041667in}}{\pgfqpoint{0.000000in}{-0.041667in}}%
\pgfpathlineto{\pgfqpoint{0.000000in}{-0.041667in}}%
\pgfpathclose%
\pgfusepath{stroke,fill}%
}%
\end{pgfscope}%
\begin{pgfscope}%
\pgfpathrectangle{\pgfqpoint{5.292946in}{7.624184in}}{\pgfqpoint{2.177280in}{2.201755in}}%
\pgfusepath{clip}%
\pgfsetbuttcap%
\pgfsetroundjoin%
\definecolor{currentfill}{rgb}{0.172549,0.627451,0.172549}%
\pgfsetfillcolor{currentfill}%
\pgfsetlinewidth{1.003750pt}%
\definecolor{currentstroke}{rgb}{0.172549,0.627451,0.172549}%
\pgfsetstrokecolor{currentstroke}%
\pgfsetdash{}{0pt}%
\pgfsys@defobject{currentmarker}{\pgfqpoint{-0.041667in}{-0.041667in}}{\pgfqpoint{0.041667in}{0.041667in}}{%
\pgfpathmoveto{\pgfqpoint{0.000000in}{-0.041667in}}%
\pgfpathcurveto{\pgfqpoint{0.011050in}{-0.041667in}}{\pgfqpoint{0.021649in}{-0.037276in}}{\pgfqpoint{0.029463in}{-0.029463in}}%
\pgfpathcurveto{\pgfqpoint{0.037276in}{-0.021649in}}{\pgfqpoint{0.041667in}{-0.011050in}}{\pgfqpoint{0.041667in}{0.000000in}}%
\pgfpathcurveto{\pgfqpoint{0.041667in}{0.011050in}}{\pgfqpoint{0.037276in}{0.021649in}}{\pgfqpoint{0.029463in}{0.029463in}}%
\pgfpathcurveto{\pgfqpoint{0.021649in}{0.037276in}}{\pgfqpoint{0.011050in}{0.041667in}}{\pgfqpoint{0.000000in}{0.041667in}}%
\pgfpathcurveto{\pgfqpoint{-0.011050in}{0.041667in}}{\pgfqpoint{-0.021649in}{0.037276in}}{\pgfqpoint{-0.029463in}{0.029463in}}%
\pgfpathcurveto{\pgfqpoint{-0.037276in}{0.021649in}}{\pgfqpoint{-0.041667in}{0.011050in}}{\pgfqpoint{-0.041667in}{0.000000in}}%
\pgfpathcurveto{\pgfqpoint{-0.041667in}{-0.011050in}}{\pgfqpoint{-0.037276in}{-0.021649in}}{\pgfqpoint{-0.029463in}{-0.029463in}}%
\pgfpathcurveto{\pgfqpoint{-0.021649in}{-0.037276in}}{\pgfqpoint{-0.011050in}{-0.041667in}}{\pgfqpoint{0.000000in}{-0.041667in}}%
\pgfpathlineto{\pgfqpoint{0.000000in}{-0.041667in}}%
\pgfpathclose%
\pgfusepath{stroke,fill}%
}%
\end{pgfscope}%
\begin{pgfscope}%
\pgfsetbuttcap%
\pgfsetroundjoin%
\definecolor{currentfill}{rgb}{0.000000,0.000000,0.000000}%
\pgfsetfillcolor{currentfill}%
\pgfsetlinewidth{0.803000pt}%
\definecolor{currentstroke}{rgb}{0.000000,0.000000,0.000000}%
\pgfsetstrokecolor{currentstroke}%
\pgfsetdash{}{0pt}%
\pgfsys@defobject{currentmarker}{\pgfqpoint{0.000000in}{-0.048611in}}{\pgfqpoint{0.000000in}{0.000000in}}{%
\pgfpathmoveto{\pgfqpoint{0.000000in}{0.000000in}}%
\pgfpathlineto{\pgfqpoint{0.000000in}{-0.048611in}}%
\pgfusepath{stroke,fill}%
}%
\begin{pgfscope}%
\pgfsys@transformshift{5.747357in}{7.624184in}%
\pgfsys@useobject{currentmarker}{}%
\end{pgfscope}%
\end{pgfscope}%
\begin{pgfscope}%
\pgfsetbuttcap%
\pgfsetroundjoin%
\definecolor{currentfill}{rgb}{0.000000,0.000000,0.000000}%
\pgfsetfillcolor{currentfill}%
\pgfsetlinewidth{0.803000pt}%
\definecolor{currentstroke}{rgb}{0.000000,0.000000,0.000000}%
\pgfsetstrokecolor{currentstroke}%
\pgfsetdash{}{0pt}%
\pgfsys@defobject{currentmarker}{\pgfqpoint{0.000000in}{-0.048611in}}{\pgfqpoint{0.000000in}{0.000000in}}{%
\pgfpathmoveto{\pgfqpoint{0.000000in}{0.000000in}}%
\pgfpathlineto{\pgfqpoint{0.000000in}{-0.048611in}}%
\pgfusepath{stroke,fill}%
}%
\begin{pgfscope}%
\pgfsys@transformshift{6.321463in}{7.624184in}%
\pgfsys@useobject{currentmarker}{}%
\end{pgfscope}%
\end{pgfscope}%
\begin{pgfscope}%
\pgfsetbuttcap%
\pgfsetroundjoin%
\definecolor{currentfill}{rgb}{0.000000,0.000000,0.000000}%
\pgfsetfillcolor{currentfill}%
\pgfsetlinewidth{0.803000pt}%
\definecolor{currentstroke}{rgb}{0.000000,0.000000,0.000000}%
\pgfsetstrokecolor{currentstroke}%
\pgfsetdash{}{0pt}%
\pgfsys@defobject{currentmarker}{\pgfqpoint{0.000000in}{-0.048611in}}{\pgfqpoint{0.000000in}{0.000000in}}{%
\pgfpathmoveto{\pgfqpoint{0.000000in}{0.000000in}}%
\pgfpathlineto{\pgfqpoint{0.000000in}{-0.048611in}}%
\pgfusepath{stroke,fill}%
}%
\begin{pgfscope}%
\pgfsys@transformshift{6.895569in}{7.624184in}%
\pgfsys@useobject{currentmarker}{}%
\end{pgfscope}%
\end{pgfscope}%
\begin{pgfscope}%
\pgfsetbuttcap%
\pgfsetroundjoin%
\definecolor{currentfill}{rgb}{0.000000,0.000000,0.000000}%
\pgfsetfillcolor{currentfill}%
\pgfsetlinewidth{0.803000pt}%
\definecolor{currentstroke}{rgb}{0.000000,0.000000,0.000000}%
\pgfsetstrokecolor{currentstroke}%
\pgfsetdash{}{0pt}%
\pgfsys@defobject{currentmarker}{\pgfqpoint{0.000000in}{-0.048611in}}{\pgfqpoint{0.000000in}{0.000000in}}{%
\pgfpathmoveto{\pgfqpoint{0.000000in}{0.000000in}}%
\pgfpathlineto{\pgfqpoint{0.000000in}{-0.048611in}}%
\pgfusepath{stroke,fill}%
}%
\begin{pgfscope}%
\pgfsys@transformshift{7.469674in}{7.624184in}%
\pgfsys@useobject{currentmarker}{}%
\end{pgfscope}%
\end{pgfscope}%
\begin{pgfscope}%
\pgfsetbuttcap%
\pgfsetroundjoin%
\definecolor{currentfill}{rgb}{0.000000,0.000000,0.000000}%
\pgfsetfillcolor{currentfill}%
\pgfsetlinewidth{0.803000pt}%
\definecolor{currentstroke}{rgb}{0.000000,0.000000,0.000000}%
\pgfsetstrokecolor{currentstroke}%
\pgfsetdash{}{0pt}%
\pgfsys@defobject{currentmarker}{\pgfqpoint{-0.048611in}{0.000000in}}{\pgfqpoint{-0.000000in}{0.000000in}}{%
\pgfpathmoveto{\pgfqpoint{-0.000000in}{0.000000in}}%
\pgfpathlineto{\pgfqpoint{-0.048611in}{0.000000in}}%
\pgfusepath{stroke,fill}%
}%
\begin{pgfscope}%
\pgfsys@transformshift{5.292946in}{8.113463in}%
\pgfsys@useobject{currentmarker}{}%
\end{pgfscope}%
\end{pgfscope}%
\begin{pgfscope}%
\pgfsetbuttcap%
\pgfsetroundjoin%
\definecolor{currentfill}{rgb}{0.000000,0.000000,0.000000}%
\pgfsetfillcolor{currentfill}%
\pgfsetlinewidth{0.803000pt}%
\definecolor{currentstroke}{rgb}{0.000000,0.000000,0.000000}%
\pgfsetstrokecolor{currentstroke}%
\pgfsetdash{}{0pt}%
\pgfsys@defobject{currentmarker}{\pgfqpoint{-0.048611in}{0.000000in}}{\pgfqpoint{-0.000000in}{0.000000in}}{%
\pgfpathmoveto{\pgfqpoint{-0.000000in}{0.000000in}}%
\pgfpathlineto{\pgfqpoint{-0.048611in}{0.000000in}}%
\pgfusepath{stroke,fill}%
}%
\begin{pgfscope}%
\pgfsys@transformshift{5.292946in}{8.669461in}%
\pgfsys@useobject{currentmarker}{}%
\end{pgfscope}%
\end{pgfscope}%
\begin{pgfscope}%
\pgfsetbuttcap%
\pgfsetroundjoin%
\definecolor{currentfill}{rgb}{0.000000,0.000000,0.000000}%
\pgfsetfillcolor{currentfill}%
\pgfsetlinewidth{0.803000pt}%
\definecolor{currentstroke}{rgb}{0.000000,0.000000,0.000000}%
\pgfsetstrokecolor{currentstroke}%
\pgfsetdash{}{0pt}%
\pgfsys@defobject{currentmarker}{\pgfqpoint{-0.048611in}{0.000000in}}{\pgfqpoint{-0.000000in}{0.000000in}}{%
\pgfpathmoveto{\pgfqpoint{-0.000000in}{0.000000in}}%
\pgfpathlineto{\pgfqpoint{-0.048611in}{0.000000in}}%
\pgfusepath{stroke,fill}%
}%
\begin{pgfscope}%
\pgfsys@transformshift{5.292946in}{9.225460in}%
\pgfsys@useobject{currentmarker}{}%
\end{pgfscope}%
\end{pgfscope}%
\begin{pgfscope}%
\pgfsetbuttcap%
\pgfsetroundjoin%
\definecolor{currentfill}{rgb}{0.000000,0.000000,0.000000}%
\pgfsetfillcolor{currentfill}%
\pgfsetlinewidth{0.803000pt}%
\definecolor{currentstroke}{rgb}{0.000000,0.000000,0.000000}%
\pgfsetstrokecolor{currentstroke}%
\pgfsetdash{}{0pt}%
\pgfsys@defobject{currentmarker}{\pgfqpoint{-0.048611in}{0.000000in}}{\pgfqpoint{-0.000000in}{0.000000in}}{%
\pgfpathmoveto{\pgfqpoint{-0.000000in}{0.000000in}}%
\pgfpathlineto{\pgfqpoint{-0.048611in}{0.000000in}}%
\pgfusepath{stroke,fill}%
}%
\begin{pgfscope}%
\pgfsys@transformshift{5.292946in}{9.781459in}%
\pgfsys@useobject{currentmarker}{}%
\end{pgfscope}%
\end{pgfscope}%
\begin{pgfscope}%
\pgfsetrectcap%
\pgfsetmiterjoin%
\pgfsetlinewidth{0.803000pt}%
\definecolor{currentstroke}{rgb}{0.000000,0.000000,0.000000}%
\pgfsetstrokecolor{currentstroke}%
\pgfsetdash{}{0pt}%
\pgfpathmoveto{\pgfqpoint{5.292946in}{7.624184in}}%
\pgfpathlineto{\pgfqpoint{5.292946in}{9.825939in}}%
\pgfusepath{stroke}%
\end{pgfscope}%
\begin{pgfscope}%
\pgfsetrectcap%
\pgfsetmiterjoin%
\pgfsetlinewidth{0.803000pt}%
\definecolor{currentstroke}{rgb}{0.000000,0.000000,0.000000}%
\pgfsetstrokecolor{currentstroke}%
\pgfsetdash{}{0pt}%
\pgfpathmoveto{\pgfqpoint{5.292946in}{7.624184in}}%
\pgfpathlineto{\pgfqpoint{7.470226in}{7.624184in}}%
\pgfusepath{stroke}%
\end{pgfscope}%
\begin{pgfscope}%
\pgfsetbuttcap%
\pgfsetmiterjoin%
\definecolor{currentfill}{rgb}{1.000000,1.000000,1.000000}%
\pgfsetfillcolor{currentfill}%
\pgfsetlinewidth{0.000000pt}%
\definecolor{currentstroke}{rgb}{0.000000,0.000000,0.000000}%
\pgfsetstrokecolor{currentstroke}%
\pgfsetstrokeopacity{0.000000}%
\pgfsetdash{}{0pt}%
\pgfpathmoveto{\pgfqpoint{7.622482in}{7.624184in}}%
\pgfpathlineto{\pgfqpoint{9.799762in}{7.624184in}}%
\pgfpathlineto{\pgfqpoint{9.799762in}{9.825939in}}%
\pgfpathlineto{\pgfqpoint{7.622482in}{9.825939in}}%
\pgfpathlineto{\pgfqpoint{7.622482in}{7.624184in}}%
\pgfpathclose%
\pgfusepath{fill}%
\end{pgfscope}%
\begin{pgfscope}%
\pgfpathrectangle{\pgfqpoint{7.622482in}{7.624184in}}{\pgfqpoint{2.177280in}{2.201755in}}%
\pgfusepath{clip}%
\pgfsetbuttcap%
\pgfsetroundjoin%
\definecolor{currentfill}{rgb}{0.121569,0.466667,0.705882}%
\pgfsetfillcolor{currentfill}%
\pgfsetlinewidth{0.481800pt}%
\definecolor{currentstroke}{rgb}{1.000000,1.000000,1.000000}%
\pgfsetstrokecolor{currentstroke}%
\pgfsetdash{}{0pt}%
\pgfpathmoveto{\pgfqpoint{7.887162in}{8.127396in}}%
\pgfpathcurveto{\pgfqpoint{7.898212in}{8.127396in}}{\pgfqpoint{7.908811in}{8.131786in}}{\pgfqpoint{7.916625in}{8.139600in}}%
\pgfpathcurveto{\pgfqpoint{7.924438in}{8.147413in}}{\pgfqpoint{7.928828in}{8.158012in}}{\pgfqpoint{7.928828in}{8.169063in}}%
\pgfpathcurveto{\pgfqpoint{7.928828in}{8.180113in}}{\pgfqpoint{7.924438in}{8.190712in}}{\pgfqpoint{7.916625in}{8.198525in}}%
\pgfpathcurveto{\pgfqpoint{7.908811in}{8.206339in}}{\pgfqpoint{7.898212in}{8.210729in}}{\pgfqpoint{7.887162in}{8.210729in}}%
\pgfpathcurveto{\pgfqpoint{7.876112in}{8.210729in}}{\pgfqpoint{7.865513in}{8.206339in}}{\pgfqpoint{7.857699in}{8.198525in}}%
\pgfpathcurveto{\pgfqpoint{7.849885in}{8.190712in}}{\pgfqpoint{7.845495in}{8.180113in}}{\pgfqpoint{7.845495in}{8.169063in}}%
\pgfpathcurveto{\pgfqpoint{7.845495in}{8.158012in}}{\pgfqpoint{7.849885in}{8.147413in}}{\pgfqpoint{7.857699in}{8.139600in}}%
\pgfpathcurveto{\pgfqpoint{7.865513in}{8.131786in}}{\pgfqpoint{7.876112in}{8.127396in}}{\pgfqpoint{7.887162in}{8.127396in}}%
\pgfpathlineto{\pgfqpoint{7.887162in}{8.127396in}}%
\pgfpathclose%
\pgfusepath{stroke,fill}%
\end{pgfscope}%
\begin{pgfscope}%
\pgfpathrectangle{\pgfqpoint{7.622482in}{7.624184in}}{\pgfqpoint{2.177280in}{2.201755in}}%
\pgfusepath{clip}%
\pgfsetbuttcap%
\pgfsetroundjoin%
\definecolor{currentfill}{rgb}{0.121569,0.466667,0.705882}%
\pgfsetfillcolor{currentfill}%
\pgfsetlinewidth{0.481800pt}%
\definecolor{currentstroke}{rgb}{1.000000,1.000000,1.000000}%
\pgfsetstrokecolor{currentstroke}%
\pgfsetdash{}{0pt}%
\pgfpathmoveto{\pgfqpoint{7.887162in}{8.016196in}}%
\pgfpathcurveto{\pgfqpoint{7.898212in}{8.016196in}}{\pgfqpoint{7.908811in}{8.020586in}}{\pgfqpoint{7.916625in}{8.028400in}}%
\pgfpathcurveto{\pgfqpoint{7.924438in}{8.036214in}}{\pgfqpoint{7.928828in}{8.046813in}}{\pgfqpoint{7.928828in}{8.057863in}}%
\pgfpathcurveto{\pgfqpoint{7.928828in}{8.068913in}}{\pgfqpoint{7.924438in}{8.079512in}}{\pgfqpoint{7.916625in}{8.087326in}}%
\pgfpathcurveto{\pgfqpoint{7.908811in}{8.095139in}}{\pgfqpoint{7.898212in}{8.099529in}}{\pgfqpoint{7.887162in}{8.099529in}}%
\pgfpathcurveto{\pgfqpoint{7.876112in}{8.099529in}}{\pgfqpoint{7.865513in}{8.095139in}}{\pgfqpoint{7.857699in}{8.087326in}}%
\pgfpathcurveto{\pgfqpoint{7.849885in}{8.079512in}}{\pgfqpoint{7.845495in}{8.068913in}}{\pgfqpoint{7.845495in}{8.057863in}}%
\pgfpathcurveto{\pgfqpoint{7.845495in}{8.046813in}}{\pgfqpoint{7.849885in}{8.036214in}}{\pgfqpoint{7.857699in}{8.028400in}}%
\pgfpathcurveto{\pgfqpoint{7.865513in}{8.020586in}}{\pgfqpoint{7.876112in}{8.016196in}}{\pgfqpoint{7.887162in}{8.016196in}}%
\pgfpathlineto{\pgfqpoint{7.887162in}{8.016196in}}%
\pgfpathclose%
\pgfusepath{stroke,fill}%
\end{pgfscope}%
\begin{pgfscope}%
\pgfpathrectangle{\pgfqpoint{7.622482in}{7.624184in}}{\pgfqpoint{2.177280in}{2.201755in}}%
\pgfusepath{clip}%
\pgfsetbuttcap%
\pgfsetroundjoin%
\definecolor{currentfill}{rgb}{0.121569,0.466667,0.705882}%
\pgfsetfillcolor{currentfill}%
\pgfsetlinewidth{0.481800pt}%
\definecolor{currentstroke}{rgb}{1.000000,1.000000,1.000000}%
\pgfsetstrokecolor{currentstroke}%
\pgfsetdash{}{0pt}%
\pgfpathmoveto{\pgfqpoint{7.887162in}{7.904996in}}%
\pgfpathcurveto{\pgfqpoint{7.898212in}{7.904996in}}{\pgfqpoint{7.908811in}{7.909387in}}{\pgfqpoint{7.916625in}{7.917200in}}%
\pgfpathcurveto{\pgfqpoint{7.924438in}{7.925014in}}{\pgfqpoint{7.928828in}{7.935613in}}{\pgfqpoint{7.928828in}{7.946663in}}%
\pgfpathcurveto{\pgfqpoint{7.928828in}{7.957713in}}{\pgfqpoint{7.924438in}{7.968312in}}{\pgfqpoint{7.916625in}{7.976126in}}%
\pgfpathcurveto{\pgfqpoint{7.908811in}{7.983939in}}{\pgfqpoint{7.898212in}{7.988330in}}{\pgfqpoint{7.887162in}{7.988330in}}%
\pgfpathcurveto{\pgfqpoint{7.876112in}{7.988330in}}{\pgfqpoint{7.865513in}{7.983939in}}{\pgfqpoint{7.857699in}{7.976126in}}%
\pgfpathcurveto{\pgfqpoint{7.849885in}{7.968312in}}{\pgfqpoint{7.845495in}{7.957713in}}{\pgfqpoint{7.845495in}{7.946663in}}%
\pgfpathcurveto{\pgfqpoint{7.845495in}{7.935613in}}{\pgfqpoint{7.849885in}{7.925014in}}{\pgfqpoint{7.857699in}{7.917200in}}%
\pgfpathcurveto{\pgfqpoint{7.865513in}{7.909387in}}{\pgfqpoint{7.876112in}{7.904996in}}{\pgfqpoint{7.887162in}{7.904996in}}%
\pgfpathlineto{\pgfqpoint{7.887162in}{7.904996in}}%
\pgfpathclose%
\pgfusepath{stroke,fill}%
\end{pgfscope}%
\begin{pgfscope}%
\pgfpathrectangle{\pgfqpoint{7.622482in}{7.624184in}}{\pgfqpoint{2.177280in}{2.201755in}}%
\pgfusepath{clip}%
\pgfsetbuttcap%
\pgfsetroundjoin%
\definecolor{currentfill}{rgb}{0.121569,0.466667,0.705882}%
\pgfsetfillcolor{currentfill}%
\pgfsetlinewidth{0.481800pt}%
\definecolor{currentstroke}{rgb}{1.000000,1.000000,1.000000}%
\pgfsetstrokecolor{currentstroke}%
\pgfsetdash{}{0pt}%
\pgfpathmoveto{\pgfqpoint{7.887162in}{7.849396in}}%
\pgfpathcurveto{\pgfqpoint{7.898212in}{7.849396in}}{\pgfqpoint{7.908811in}{7.853787in}}{\pgfqpoint{7.916625in}{7.861600in}}%
\pgfpathcurveto{\pgfqpoint{7.924438in}{7.869414in}}{\pgfqpoint{7.928828in}{7.880013in}}{\pgfqpoint{7.928828in}{7.891063in}}%
\pgfpathcurveto{\pgfqpoint{7.928828in}{7.902113in}}{\pgfqpoint{7.924438in}{7.912712in}}{\pgfqpoint{7.916625in}{7.920526in}}%
\pgfpathcurveto{\pgfqpoint{7.908811in}{7.928340in}}{\pgfqpoint{7.898212in}{7.932730in}}{\pgfqpoint{7.887162in}{7.932730in}}%
\pgfpathcurveto{\pgfqpoint{7.876112in}{7.932730in}}{\pgfqpoint{7.865513in}{7.928340in}}{\pgfqpoint{7.857699in}{7.920526in}}%
\pgfpathcurveto{\pgfqpoint{7.849885in}{7.912712in}}{\pgfqpoint{7.845495in}{7.902113in}}{\pgfqpoint{7.845495in}{7.891063in}}%
\pgfpathcurveto{\pgfqpoint{7.845495in}{7.880013in}}{\pgfqpoint{7.849885in}{7.869414in}}{\pgfqpoint{7.857699in}{7.861600in}}%
\pgfpathcurveto{\pgfqpoint{7.865513in}{7.853787in}}{\pgfqpoint{7.876112in}{7.849396in}}{\pgfqpoint{7.887162in}{7.849396in}}%
\pgfpathlineto{\pgfqpoint{7.887162in}{7.849396in}}%
\pgfpathclose%
\pgfusepath{stroke,fill}%
\end{pgfscope}%
\begin{pgfscope}%
\pgfpathrectangle{\pgfqpoint{7.622482in}{7.624184in}}{\pgfqpoint{2.177280in}{2.201755in}}%
\pgfusepath{clip}%
\pgfsetbuttcap%
\pgfsetroundjoin%
\definecolor{currentfill}{rgb}{0.121569,0.466667,0.705882}%
\pgfsetfillcolor{currentfill}%
\pgfsetlinewidth{0.481800pt}%
\definecolor{currentstroke}{rgb}{1.000000,1.000000,1.000000}%
\pgfsetstrokecolor{currentstroke}%
\pgfsetdash{}{0pt}%
\pgfpathmoveto{\pgfqpoint{7.887162in}{8.071796in}}%
\pgfpathcurveto{\pgfqpoint{7.898212in}{8.071796in}}{\pgfqpoint{7.908811in}{8.076186in}}{\pgfqpoint{7.916625in}{8.084000in}}%
\pgfpathcurveto{\pgfqpoint{7.924438in}{8.091813in}}{\pgfqpoint{7.928828in}{8.102413in}}{\pgfqpoint{7.928828in}{8.113463in}}%
\pgfpathcurveto{\pgfqpoint{7.928828in}{8.124513in}}{\pgfqpoint{7.924438in}{8.135112in}}{\pgfqpoint{7.916625in}{8.142925in}}%
\pgfpathcurveto{\pgfqpoint{7.908811in}{8.150739in}}{\pgfqpoint{7.898212in}{8.155129in}}{\pgfqpoint{7.887162in}{8.155129in}}%
\pgfpathcurveto{\pgfqpoint{7.876112in}{8.155129in}}{\pgfqpoint{7.865513in}{8.150739in}}{\pgfqpoint{7.857699in}{8.142925in}}%
\pgfpathcurveto{\pgfqpoint{7.849885in}{8.135112in}}{\pgfqpoint{7.845495in}{8.124513in}}{\pgfqpoint{7.845495in}{8.113463in}}%
\pgfpathcurveto{\pgfqpoint{7.845495in}{8.102413in}}{\pgfqpoint{7.849885in}{8.091813in}}{\pgfqpoint{7.857699in}{8.084000in}}%
\pgfpathcurveto{\pgfqpoint{7.865513in}{8.076186in}}{\pgfqpoint{7.876112in}{8.071796in}}{\pgfqpoint{7.887162in}{8.071796in}}%
\pgfpathlineto{\pgfqpoint{7.887162in}{8.071796in}}%
\pgfpathclose%
\pgfusepath{stroke,fill}%
\end{pgfscope}%
\begin{pgfscope}%
\pgfpathrectangle{\pgfqpoint{7.622482in}{7.624184in}}{\pgfqpoint{2.177280in}{2.201755in}}%
\pgfusepath{clip}%
\pgfsetbuttcap%
\pgfsetroundjoin%
\definecolor{currentfill}{rgb}{0.121569,0.466667,0.705882}%
\pgfsetfillcolor{currentfill}%
\pgfsetlinewidth{0.481800pt}%
\definecolor{currentstroke}{rgb}{1.000000,1.000000,1.000000}%
\pgfsetstrokecolor{currentstroke}%
\pgfsetdash{}{0pt}%
\pgfpathmoveto{\pgfqpoint{8.022670in}{8.294195in}}%
\pgfpathcurveto{\pgfqpoint{8.033720in}{8.294195in}}{\pgfqpoint{8.044319in}{8.298586in}}{\pgfqpoint{8.052132in}{8.306399in}}%
\pgfpathcurveto{\pgfqpoint{8.059946in}{8.314213in}}{\pgfqpoint{8.064336in}{8.324812in}}{\pgfqpoint{8.064336in}{8.335862in}}%
\pgfpathcurveto{\pgfqpoint{8.064336in}{8.346912in}}{\pgfqpoint{8.059946in}{8.357511in}}{\pgfqpoint{8.052132in}{8.365325in}}%
\pgfpathcurveto{\pgfqpoint{8.044319in}{8.373139in}}{\pgfqpoint{8.033720in}{8.377529in}}{\pgfqpoint{8.022670in}{8.377529in}}%
\pgfpathcurveto{\pgfqpoint{8.011619in}{8.377529in}}{\pgfqpoint{8.001020in}{8.373139in}}{\pgfqpoint{7.993207in}{8.365325in}}%
\pgfpathcurveto{\pgfqpoint{7.985393in}{8.357511in}}{\pgfqpoint{7.981003in}{8.346912in}}{\pgfqpoint{7.981003in}{8.335862in}}%
\pgfpathcurveto{\pgfqpoint{7.981003in}{8.324812in}}{\pgfqpoint{7.985393in}{8.314213in}}{\pgfqpoint{7.993207in}{8.306399in}}%
\pgfpathcurveto{\pgfqpoint{8.001020in}{8.298586in}}{\pgfqpoint{8.011619in}{8.294195in}}{\pgfqpoint{8.022670in}{8.294195in}}%
\pgfpathlineto{\pgfqpoint{8.022670in}{8.294195in}}%
\pgfpathclose%
\pgfusepath{stroke,fill}%
\end{pgfscope}%
\begin{pgfscope}%
\pgfpathrectangle{\pgfqpoint{7.622482in}{7.624184in}}{\pgfqpoint{2.177280in}{2.201755in}}%
\pgfusepath{clip}%
\pgfsetbuttcap%
\pgfsetroundjoin%
\definecolor{currentfill}{rgb}{0.121569,0.466667,0.705882}%
\pgfsetfillcolor{currentfill}%
\pgfsetlinewidth{0.481800pt}%
\definecolor{currentstroke}{rgb}{1.000000,1.000000,1.000000}%
\pgfsetstrokecolor{currentstroke}%
\pgfsetdash{}{0pt}%
\pgfpathmoveto{\pgfqpoint{7.954916in}{7.849396in}}%
\pgfpathcurveto{\pgfqpoint{7.965966in}{7.849396in}}{\pgfqpoint{7.976565in}{7.853787in}}{\pgfqpoint{7.984378in}{7.861600in}}%
\pgfpathcurveto{\pgfqpoint{7.992192in}{7.869414in}}{\pgfqpoint{7.996582in}{7.880013in}}{\pgfqpoint{7.996582in}{7.891063in}}%
\pgfpathcurveto{\pgfqpoint{7.996582in}{7.902113in}}{\pgfqpoint{7.992192in}{7.912712in}}{\pgfqpoint{7.984378in}{7.920526in}}%
\pgfpathcurveto{\pgfqpoint{7.976565in}{7.928340in}}{\pgfqpoint{7.965966in}{7.932730in}}{\pgfqpoint{7.954916in}{7.932730in}}%
\pgfpathcurveto{\pgfqpoint{7.943866in}{7.932730in}}{\pgfqpoint{7.933267in}{7.928340in}}{\pgfqpoint{7.925453in}{7.920526in}}%
\pgfpathcurveto{\pgfqpoint{7.917639in}{7.912712in}}{\pgfqpoint{7.913249in}{7.902113in}}{\pgfqpoint{7.913249in}{7.891063in}}%
\pgfpathcurveto{\pgfqpoint{7.913249in}{7.880013in}}{\pgfqpoint{7.917639in}{7.869414in}}{\pgfqpoint{7.925453in}{7.861600in}}%
\pgfpathcurveto{\pgfqpoint{7.933267in}{7.853787in}}{\pgfqpoint{7.943866in}{7.849396in}}{\pgfqpoint{7.954916in}{7.849396in}}%
\pgfpathlineto{\pgfqpoint{7.954916in}{7.849396in}}%
\pgfpathclose%
\pgfusepath{stroke,fill}%
\end{pgfscope}%
\begin{pgfscope}%
\pgfpathrectangle{\pgfqpoint{7.622482in}{7.624184in}}{\pgfqpoint{2.177280in}{2.201755in}}%
\pgfusepath{clip}%
\pgfsetbuttcap%
\pgfsetroundjoin%
\definecolor{currentfill}{rgb}{0.121569,0.466667,0.705882}%
\pgfsetfillcolor{currentfill}%
\pgfsetlinewidth{0.481800pt}%
\definecolor{currentstroke}{rgb}{1.000000,1.000000,1.000000}%
\pgfsetstrokecolor{currentstroke}%
\pgfsetdash{}{0pt}%
\pgfpathmoveto{\pgfqpoint{7.887162in}{8.071796in}}%
\pgfpathcurveto{\pgfqpoint{7.898212in}{8.071796in}}{\pgfqpoint{7.908811in}{8.076186in}}{\pgfqpoint{7.916625in}{8.084000in}}%
\pgfpathcurveto{\pgfqpoint{7.924438in}{8.091813in}}{\pgfqpoint{7.928828in}{8.102413in}}{\pgfqpoint{7.928828in}{8.113463in}}%
\pgfpathcurveto{\pgfqpoint{7.928828in}{8.124513in}}{\pgfqpoint{7.924438in}{8.135112in}}{\pgfqpoint{7.916625in}{8.142925in}}%
\pgfpathcurveto{\pgfqpoint{7.908811in}{8.150739in}}{\pgfqpoint{7.898212in}{8.155129in}}{\pgfqpoint{7.887162in}{8.155129in}}%
\pgfpathcurveto{\pgfqpoint{7.876112in}{8.155129in}}{\pgfqpoint{7.865513in}{8.150739in}}{\pgfqpoint{7.857699in}{8.142925in}}%
\pgfpathcurveto{\pgfqpoint{7.849885in}{8.135112in}}{\pgfqpoint{7.845495in}{8.124513in}}{\pgfqpoint{7.845495in}{8.113463in}}%
\pgfpathcurveto{\pgfqpoint{7.845495in}{8.102413in}}{\pgfqpoint{7.849885in}{8.091813in}}{\pgfqpoint{7.857699in}{8.084000in}}%
\pgfpathcurveto{\pgfqpoint{7.865513in}{8.076186in}}{\pgfqpoint{7.876112in}{8.071796in}}{\pgfqpoint{7.887162in}{8.071796in}}%
\pgfpathlineto{\pgfqpoint{7.887162in}{8.071796in}}%
\pgfpathclose%
\pgfusepath{stroke,fill}%
\end{pgfscope}%
\begin{pgfscope}%
\pgfpathrectangle{\pgfqpoint{7.622482in}{7.624184in}}{\pgfqpoint{2.177280in}{2.201755in}}%
\pgfusepath{clip}%
\pgfsetbuttcap%
\pgfsetroundjoin%
\definecolor{currentfill}{rgb}{0.121569,0.466667,0.705882}%
\pgfsetfillcolor{currentfill}%
\pgfsetlinewidth{0.481800pt}%
\definecolor{currentstroke}{rgb}{1.000000,1.000000,1.000000}%
\pgfsetstrokecolor{currentstroke}%
\pgfsetdash{}{0pt}%
\pgfpathmoveto{\pgfqpoint{7.887162in}{7.738197in}}%
\pgfpathcurveto{\pgfqpoint{7.898212in}{7.738197in}}{\pgfqpoint{7.908811in}{7.742587in}}{\pgfqpoint{7.916625in}{7.750401in}}%
\pgfpathcurveto{\pgfqpoint{7.924438in}{7.758214in}}{\pgfqpoint{7.928828in}{7.768813in}}{\pgfqpoint{7.928828in}{7.779863in}}%
\pgfpathcurveto{\pgfqpoint{7.928828in}{7.790914in}}{\pgfqpoint{7.924438in}{7.801513in}}{\pgfqpoint{7.916625in}{7.809326in}}%
\pgfpathcurveto{\pgfqpoint{7.908811in}{7.817140in}}{\pgfqpoint{7.898212in}{7.821530in}}{\pgfqpoint{7.887162in}{7.821530in}}%
\pgfpathcurveto{\pgfqpoint{7.876112in}{7.821530in}}{\pgfqpoint{7.865513in}{7.817140in}}{\pgfqpoint{7.857699in}{7.809326in}}%
\pgfpathcurveto{\pgfqpoint{7.849885in}{7.801513in}}{\pgfqpoint{7.845495in}{7.790914in}}{\pgfqpoint{7.845495in}{7.779863in}}%
\pgfpathcurveto{\pgfqpoint{7.845495in}{7.768813in}}{\pgfqpoint{7.849885in}{7.758214in}}{\pgfqpoint{7.857699in}{7.750401in}}%
\pgfpathcurveto{\pgfqpoint{7.865513in}{7.742587in}}{\pgfqpoint{7.876112in}{7.738197in}}{\pgfqpoint{7.887162in}{7.738197in}}%
\pgfpathlineto{\pgfqpoint{7.887162in}{7.738197in}}%
\pgfpathclose%
\pgfusepath{stroke,fill}%
\end{pgfscope}%
\begin{pgfscope}%
\pgfpathrectangle{\pgfqpoint{7.622482in}{7.624184in}}{\pgfqpoint{2.177280in}{2.201755in}}%
\pgfusepath{clip}%
\pgfsetbuttcap%
\pgfsetroundjoin%
\definecolor{currentfill}{rgb}{0.121569,0.466667,0.705882}%
\pgfsetfillcolor{currentfill}%
\pgfsetlinewidth{0.481800pt}%
\definecolor{currentstroke}{rgb}{1.000000,1.000000,1.000000}%
\pgfsetstrokecolor{currentstroke}%
\pgfsetdash{}{0pt}%
\pgfpathmoveto{\pgfqpoint{7.819408in}{8.016196in}}%
\pgfpathcurveto{\pgfqpoint{7.830458in}{8.016196in}}{\pgfqpoint{7.841057in}{8.020586in}}{\pgfqpoint{7.848871in}{8.028400in}}%
\pgfpathcurveto{\pgfqpoint{7.856684in}{8.036214in}}{\pgfqpoint{7.861075in}{8.046813in}}{\pgfqpoint{7.861075in}{8.057863in}}%
\pgfpathcurveto{\pgfqpoint{7.861075in}{8.068913in}}{\pgfqpoint{7.856684in}{8.079512in}}{\pgfqpoint{7.848871in}{8.087326in}}%
\pgfpathcurveto{\pgfqpoint{7.841057in}{8.095139in}}{\pgfqpoint{7.830458in}{8.099529in}}{\pgfqpoint{7.819408in}{8.099529in}}%
\pgfpathcurveto{\pgfqpoint{7.808358in}{8.099529in}}{\pgfqpoint{7.797759in}{8.095139in}}{\pgfqpoint{7.789945in}{8.087326in}}%
\pgfpathcurveto{\pgfqpoint{7.782132in}{8.079512in}}{\pgfqpoint{7.777741in}{8.068913in}}{\pgfqpoint{7.777741in}{8.057863in}}%
\pgfpathcurveto{\pgfqpoint{7.777741in}{8.046813in}}{\pgfqpoint{7.782132in}{8.036214in}}{\pgfqpoint{7.789945in}{8.028400in}}%
\pgfpathcurveto{\pgfqpoint{7.797759in}{8.020586in}}{\pgfqpoint{7.808358in}{8.016196in}}{\pgfqpoint{7.819408in}{8.016196in}}%
\pgfpathlineto{\pgfqpoint{7.819408in}{8.016196in}}%
\pgfpathclose%
\pgfusepath{stroke,fill}%
\end{pgfscope}%
\begin{pgfscope}%
\pgfpathrectangle{\pgfqpoint{7.622482in}{7.624184in}}{\pgfqpoint{2.177280in}{2.201755in}}%
\pgfusepath{clip}%
\pgfsetbuttcap%
\pgfsetroundjoin%
\definecolor{currentfill}{rgb}{0.121569,0.466667,0.705882}%
\pgfsetfillcolor{currentfill}%
\pgfsetlinewidth{0.481800pt}%
\definecolor{currentstroke}{rgb}{1.000000,1.000000,1.000000}%
\pgfsetstrokecolor{currentstroke}%
\pgfsetdash{}{0pt}%
\pgfpathmoveto{\pgfqpoint{7.887162in}{8.294195in}}%
\pgfpathcurveto{\pgfqpoint{7.898212in}{8.294195in}}{\pgfqpoint{7.908811in}{8.298586in}}{\pgfqpoint{7.916625in}{8.306399in}}%
\pgfpathcurveto{\pgfqpoint{7.924438in}{8.314213in}}{\pgfqpoint{7.928828in}{8.324812in}}{\pgfqpoint{7.928828in}{8.335862in}}%
\pgfpathcurveto{\pgfqpoint{7.928828in}{8.346912in}}{\pgfqpoint{7.924438in}{8.357511in}}{\pgfqpoint{7.916625in}{8.365325in}}%
\pgfpathcurveto{\pgfqpoint{7.908811in}{8.373139in}}{\pgfqpoint{7.898212in}{8.377529in}}{\pgfqpoint{7.887162in}{8.377529in}}%
\pgfpathcurveto{\pgfqpoint{7.876112in}{8.377529in}}{\pgfqpoint{7.865513in}{8.373139in}}{\pgfqpoint{7.857699in}{8.365325in}}%
\pgfpathcurveto{\pgfqpoint{7.849885in}{8.357511in}}{\pgfqpoint{7.845495in}{8.346912in}}{\pgfqpoint{7.845495in}{8.335862in}}%
\pgfpathcurveto{\pgfqpoint{7.845495in}{8.324812in}}{\pgfqpoint{7.849885in}{8.314213in}}{\pgfqpoint{7.857699in}{8.306399in}}%
\pgfpathcurveto{\pgfqpoint{7.865513in}{8.298586in}}{\pgfqpoint{7.876112in}{8.294195in}}{\pgfqpoint{7.887162in}{8.294195in}}%
\pgfpathlineto{\pgfqpoint{7.887162in}{8.294195in}}%
\pgfpathclose%
\pgfusepath{stroke,fill}%
\end{pgfscope}%
\begin{pgfscope}%
\pgfpathrectangle{\pgfqpoint{7.622482in}{7.624184in}}{\pgfqpoint{2.177280in}{2.201755in}}%
\pgfusepath{clip}%
\pgfsetbuttcap%
\pgfsetroundjoin%
\definecolor{currentfill}{rgb}{0.121569,0.466667,0.705882}%
\pgfsetfillcolor{currentfill}%
\pgfsetlinewidth{0.481800pt}%
\definecolor{currentstroke}{rgb}{1.000000,1.000000,1.000000}%
\pgfsetstrokecolor{currentstroke}%
\pgfsetdash{}{0pt}%
\pgfpathmoveto{\pgfqpoint{7.887162in}{7.960596in}}%
\pgfpathcurveto{\pgfqpoint{7.898212in}{7.960596in}}{\pgfqpoint{7.908811in}{7.964986in}}{\pgfqpoint{7.916625in}{7.972800in}}%
\pgfpathcurveto{\pgfqpoint{7.924438in}{7.980614in}}{\pgfqpoint{7.928828in}{7.991213in}}{\pgfqpoint{7.928828in}{8.002263in}}%
\pgfpathcurveto{\pgfqpoint{7.928828in}{8.013313in}}{\pgfqpoint{7.924438in}{8.023912in}}{\pgfqpoint{7.916625in}{8.031726in}}%
\pgfpathcurveto{\pgfqpoint{7.908811in}{8.039539in}}{\pgfqpoint{7.898212in}{8.043930in}}{\pgfqpoint{7.887162in}{8.043930in}}%
\pgfpathcurveto{\pgfqpoint{7.876112in}{8.043930in}}{\pgfqpoint{7.865513in}{8.039539in}}{\pgfqpoint{7.857699in}{8.031726in}}%
\pgfpathcurveto{\pgfqpoint{7.849885in}{8.023912in}}{\pgfqpoint{7.845495in}{8.013313in}}{\pgfqpoint{7.845495in}{8.002263in}}%
\pgfpathcurveto{\pgfqpoint{7.845495in}{7.991213in}}{\pgfqpoint{7.849885in}{7.980614in}}{\pgfqpoint{7.857699in}{7.972800in}}%
\pgfpathcurveto{\pgfqpoint{7.865513in}{7.964986in}}{\pgfqpoint{7.876112in}{7.960596in}}{\pgfqpoint{7.887162in}{7.960596in}}%
\pgfpathlineto{\pgfqpoint{7.887162in}{7.960596in}}%
\pgfpathclose%
\pgfusepath{stroke,fill}%
\end{pgfscope}%
\begin{pgfscope}%
\pgfpathrectangle{\pgfqpoint{7.622482in}{7.624184in}}{\pgfqpoint{2.177280in}{2.201755in}}%
\pgfusepath{clip}%
\pgfsetbuttcap%
\pgfsetroundjoin%
\definecolor{currentfill}{rgb}{0.121569,0.466667,0.705882}%
\pgfsetfillcolor{currentfill}%
\pgfsetlinewidth{0.481800pt}%
\definecolor{currentstroke}{rgb}{1.000000,1.000000,1.000000}%
\pgfsetstrokecolor{currentstroke}%
\pgfsetdash{}{0pt}%
\pgfpathmoveto{\pgfqpoint{7.819408in}{7.960596in}}%
\pgfpathcurveto{\pgfqpoint{7.830458in}{7.960596in}}{\pgfqpoint{7.841057in}{7.964986in}}{\pgfqpoint{7.848871in}{7.972800in}}%
\pgfpathcurveto{\pgfqpoint{7.856684in}{7.980614in}}{\pgfqpoint{7.861075in}{7.991213in}}{\pgfqpoint{7.861075in}{8.002263in}}%
\pgfpathcurveto{\pgfqpoint{7.861075in}{8.013313in}}{\pgfqpoint{7.856684in}{8.023912in}}{\pgfqpoint{7.848871in}{8.031726in}}%
\pgfpathcurveto{\pgfqpoint{7.841057in}{8.039539in}}{\pgfqpoint{7.830458in}{8.043930in}}{\pgfqpoint{7.819408in}{8.043930in}}%
\pgfpathcurveto{\pgfqpoint{7.808358in}{8.043930in}}{\pgfqpoint{7.797759in}{8.039539in}}{\pgfqpoint{7.789945in}{8.031726in}}%
\pgfpathcurveto{\pgfqpoint{7.782132in}{8.023912in}}{\pgfqpoint{7.777741in}{8.013313in}}{\pgfqpoint{7.777741in}{8.002263in}}%
\pgfpathcurveto{\pgfqpoint{7.777741in}{7.991213in}}{\pgfqpoint{7.782132in}{7.980614in}}{\pgfqpoint{7.789945in}{7.972800in}}%
\pgfpathcurveto{\pgfqpoint{7.797759in}{7.964986in}}{\pgfqpoint{7.808358in}{7.960596in}}{\pgfqpoint{7.819408in}{7.960596in}}%
\pgfpathlineto{\pgfqpoint{7.819408in}{7.960596in}}%
\pgfpathclose%
\pgfusepath{stroke,fill}%
\end{pgfscope}%
\begin{pgfscope}%
\pgfpathrectangle{\pgfqpoint{7.622482in}{7.624184in}}{\pgfqpoint{2.177280in}{2.201755in}}%
\pgfusepath{clip}%
\pgfsetbuttcap%
\pgfsetroundjoin%
\definecolor{currentfill}{rgb}{0.121569,0.466667,0.705882}%
\pgfsetfillcolor{currentfill}%
\pgfsetlinewidth{0.481800pt}%
\definecolor{currentstroke}{rgb}{1.000000,1.000000,1.000000}%
\pgfsetstrokecolor{currentstroke}%
\pgfsetdash{}{0pt}%
\pgfpathmoveto{\pgfqpoint{7.819408in}{7.682597in}}%
\pgfpathcurveto{\pgfqpoint{7.830458in}{7.682597in}}{\pgfqpoint{7.841057in}{7.686987in}}{\pgfqpoint{7.848871in}{7.694801in}}%
\pgfpathcurveto{\pgfqpoint{7.856684in}{7.702614in}}{\pgfqpoint{7.861075in}{7.713213in}}{\pgfqpoint{7.861075in}{7.724264in}}%
\pgfpathcurveto{\pgfqpoint{7.861075in}{7.735314in}}{\pgfqpoint{7.856684in}{7.745913in}}{\pgfqpoint{7.848871in}{7.753726in}}%
\pgfpathcurveto{\pgfqpoint{7.841057in}{7.761540in}}{\pgfqpoint{7.830458in}{7.765930in}}{\pgfqpoint{7.819408in}{7.765930in}}%
\pgfpathcurveto{\pgfqpoint{7.808358in}{7.765930in}}{\pgfqpoint{7.797759in}{7.761540in}}{\pgfqpoint{7.789945in}{7.753726in}}%
\pgfpathcurveto{\pgfqpoint{7.782132in}{7.745913in}}{\pgfqpoint{7.777741in}{7.735314in}}{\pgfqpoint{7.777741in}{7.724264in}}%
\pgfpathcurveto{\pgfqpoint{7.777741in}{7.713213in}}{\pgfqpoint{7.782132in}{7.702614in}}{\pgfqpoint{7.789945in}{7.694801in}}%
\pgfpathcurveto{\pgfqpoint{7.797759in}{7.686987in}}{\pgfqpoint{7.808358in}{7.682597in}}{\pgfqpoint{7.819408in}{7.682597in}}%
\pgfpathlineto{\pgfqpoint{7.819408in}{7.682597in}}%
\pgfpathclose%
\pgfusepath{stroke,fill}%
\end{pgfscope}%
\begin{pgfscope}%
\pgfpathrectangle{\pgfqpoint{7.622482in}{7.624184in}}{\pgfqpoint{2.177280in}{2.201755in}}%
\pgfusepath{clip}%
\pgfsetbuttcap%
\pgfsetroundjoin%
\definecolor{currentfill}{rgb}{0.121569,0.466667,0.705882}%
\pgfsetfillcolor{currentfill}%
\pgfsetlinewidth{0.481800pt}%
\definecolor{currentstroke}{rgb}{1.000000,1.000000,1.000000}%
\pgfsetstrokecolor{currentstroke}%
\pgfsetdash{}{0pt}%
\pgfpathmoveto{\pgfqpoint{7.887162in}{8.516595in}}%
\pgfpathcurveto{\pgfqpoint{7.898212in}{8.516595in}}{\pgfqpoint{7.908811in}{8.520985in}}{\pgfqpoint{7.916625in}{8.528799in}}%
\pgfpathcurveto{\pgfqpoint{7.924438in}{8.536612in}}{\pgfqpoint{7.928828in}{8.547212in}}{\pgfqpoint{7.928828in}{8.558262in}}%
\pgfpathcurveto{\pgfqpoint{7.928828in}{8.569312in}}{\pgfqpoint{7.924438in}{8.579911in}}{\pgfqpoint{7.916625in}{8.587724in}}%
\pgfpathcurveto{\pgfqpoint{7.908811in}{8.595538in}}{\pgfqpoint{7.898212in}{8.599928in}}{\pgfqpoint{7.887162in}{8.599928in}}%
\pgfpathcurveto{\pgfqpoint{7.876112in}{8.599928in}}{\pgfqpoint{7.865513in}{8.595538in}}{\pgfqpoint{7.857699in}{8.587724in}}%
\pgfpathcurveto{\pgfqpoint{7.849885in}{8.579911in}}{\pgfqpoint{7.845495in}{8.569312in}}{\pgfqpoint{7.845495in}{8.558262in}}%
\pgfpathcurveto{\pgfqpoint{7.845495in}{8.547212in}}{\pgfqpoint{7.849885in}{8.536612in}}{\pgfqpoint{7.857699in}{8.528799in}}%
\pgfpathcurveto{\pgfqpoint{7.865513in}{8.520985in}}{\pgfqpoint{7.876112in}{8.516595in}}{\pgfqpoint{7.887162in}{8.516595in}}%
\pgfpathlineto{\pgfqpoint{7.887162in}{8.516595in}}%
\pgfpathclose%
\pgfusepath{stroke,fill}%
\end{pgfscope}%
\begin{pgfscope}%
\pgfpathrectangle{\pgfqpoint{7.622482in}{7.624184in}}{\pgfqpoint{2.177280in}{2.201755in}}%
\pgfusepath{clip}%
\pgfsetbuttcap%
\pgfsetroundjoin%
\definecolor{currentfill}{rgb}{0.121569,0.466667,0.705882}%
\pgfsetfillcolor{currentfill}%
\pgfsetlinewidth{0.481800pt}%
\definecolor{currentstroke}{rgb}{1.000000,1.000000,1.000000}%
\pgfsetstrokecolor{currentstroke}%
\pgfsetdash{}{0pt}%
\pgfpathmoveto{\pgfqpoint{8.022670in}{8.460995in}}%
\pgfpathcurveto{\pgfqpoint{8.033720in}{8.460995in}}{\pgfqpoint{8.044319in}{8.465385in}}{\pgfqpoint{8.052132in}{8.473199in}}%
\pgfpathcurveto{\pgfqpoint{8.059946in}{8.481013in}}{\pgfqpoint{8.064336in}{8.491612in}}{\pgfqpoint{8.064336in}{8.502662in}}%
\pgfpathcurveto{\pgfqpoint{8.064336in}{8.513712in}}{\pgfqpoint{8.059946in}{8.524311in}}{\pgfqpoint{8.052132in}{8.532125in}}%
\pgfpathcurveto{\pgfqpoint{8.044319in}{8.539938in}}{\pgfqpoint{8.033720in}{8.544328in}}{\pgfqpoint{8.022670in}{8.544328in}}%
\pgfpathcurveto{\pgfqpoint{8.011619in}{8.544328in}}{\pgfqpoint{8.001020in}{8.539938in}}{\pgfqpoint{7.993207in}{8.532125in}}%
\pgfpathcurveto{\pgfqpoint{7.985393in}{8.524311in}}{\pgfqpoint{7.981003in}{8.513712in}}{\pgfqpoint{7.981003in}{8.502662in}}%
\pgfpathcurveto{\pgfqpoint{7.981003in}{8.491612in}}{\pgfqpoint{7.985393in}{8.481013in}}{\pgfqpoint{7.993207in}{8.473199in}}%
\pgfpathcurveto{\pgfqpoint{8.001020in}{8.465385in}}{\pgfqpoint{8.011619in}{8.460995in}}{\pgfqpoint{8.022670in}{8.460995in}}%
\pgfpathlineto{\pgfqpoint{8.022670in}{8.460995in}}%
\pgfpathclose%
\pgfusepath{stroke,fill}%
\end{pgfscope}%
\begin{pgfscope}%
\pgfpathrectangle{\pgfqpoint{7.622482in}{7.624184in}}{\pgfqpoint{2.177280in}{2.201755in}}%
\pgfusepath{clip}%
\pgfsetbuttcap%
\pgfsetroundjoin%
\definecolor{currentfill}{rgb}{0.121569,0.466667,0.705882}%
\pgfsetfillcolor{currentfill}%
\pgfsetlinewidth{0.481800pt}%
\definecolor{currentstroke}{rgb}{1.000000,1.000000,1.000000}%
\pgfsetstrokecolor{currentstroke}%
\pgfsetdash{}{0pt}%
\pgfpathmoveto{\pgfqpoint{8.022670in}{8.294195in}}%
\pgfpathcurveto{\pgfqpoint{8.033720in}{8.294195in}}{\pgfqpoint{8.044319in}{8.298586in}}{\pgfqpoint{8.052132in}{8.306399in}}%
\pgfpathcurveto{\pgfqpoint{8.059946in}{8.314213in}}{\pgfqpoint{8.064336in}{8.324812in}}{\pgfqpoint{8.064336in}{8.335862in}}%
\pgfpathcurveto{\pgfqpoint{8.064336in}{8.346912in}}{\pgfqpoint{8.059946in}{8.357511in}}{\pgfqpoint{8.052132in}{8.365325in}}%
\pgfpathcurveto{\pgfqpoint{8.044319in}{8.373139in}}{\pgfqpoint{8.033720in}{8.377529in}}{\pgfqpoint{8.022670in}{8.377529in}}%
\pgfpathcurveto{\pgfqpoint{8.011619in}{8.377529in}}{\pgfqpoint{8.001020in}{8.373139in}}{\pgfqpoint{7.993207in}{8.365325in}}%
\pgfpathcurveto{\pgfqpoint{7.985393in}{8.357511in}}{\pgfqpoint{7.981003in}{8.346912in}}{\pgfqpoint{7.981003in}{8.335862in}}%
\pgfpathcurveto{\pgfqpoint{7.981003in}{8.324812in}}{\pgfqpoint{7.985393in}{8.314213in}}{\pgfqpoint{7.993207in}{8.306399in}}%
\pgfpathcurveto{\pgfqpoint{8.001020in}{8.298586in}}{\pgfqpoint{8.011619in}{8.294195in}}{\pgfqpoint{8.022670in}{8.294195in}}%
\pgfpathlineto{\pgfqpoint{8.022670in}{8.294195in}}%
\pgfpathclose%
\pgfusepath{stroke,fill}%
\end{pgfscope}%
\begin{pgfscope}%
\pgfpathrectangle{\pgfqpoint{7.622482in}{7.624184in}}{\pgfqpoint{2.177280in}{2.201755in}}%
\pgfusepath{clip}%
\pgfsetbuttcap%
\pgfsetroundjoin%
\definecolor{currentfill}{rgb}{0.121569,0.466667,0.705882}%
\pgfsetfillcolor{currentfill}%
\pgfsetlinewidth{0.481800pt}%
\definecolor{currentstroke}{rgb}{1.000000,1.000000,1.000000}%
\pgfsetstrokecolor{currentstroke}%
\pgfsetdash{}{0pt}%
\pgfpathmoveto{\pgfqpoint{7.954916in}{8.127396in}}%
\pgfpathcurveto{\pgfqpoint{7.965966in}{8.127396in}}{\pgfqpoint{7.976565in}{8.131786in}}{\pgfqpoint{7.984378in}{8.139600in}}%
\pgfpathcurveto{\pgfqpoint{7.992192in}{8.147413in}}{\pgfqpoint{7.996582in}{8.158012in}}{\pgfqpoint{7.996582in}{8.169063in}}%
\pgfpathcurveto{\pgfqpoint{7.996582in}{8.180113in}}{\pgfqpoint{7.992192in}{8.190712in}}{\pgfqpoint{7.984378in}{8.198525in}}%
\pgfpathcurveto{\pgfqpoint{7.976565in}{8.206339in}}{\pgfqpoint{7.965966in}{8.210729in}}{\pgfqpoint{7.954916in}{8.210729in}}%
\pgfpathcurveto{\pgfqpoint{7.943866in}{8.210729in}}{\pgfqpoint{7.933267in}{8.206339in}}{\pgfqpoint{7.925453in}{8.198525in}}%
\pgfpathcurveto{\pgfqpoint{7.917639in}{8.190712in}}{\pgfqpoint{7.913249in}{8.180113in}}{\pgfqpoint{7.913249in}{8.169063in}}%
\pgfpathcurveto{\pgfqpoint{7.913249in}{8.158012in}}{\pgfqpoint{7.917639in}{8.147413in}}{\pgfqpoint{7.925453in}{8.139600in}}%
\pgfpathcurveto{\pgfqpoint{7.933267in}{8.131786in}}{\pgfqpoint{7.943866in}{8.127396in}}{\pgfqpoint{7.954916in}{8.127396in}}%
\pgfpathlineto{\pgfqpoint{7.954916in}{8.127396in}}%
\pgfpathclose%
\pgfusepath{stroke,fill}%
\end{pgfscope}%
\begin{pgfscope}%
\pgfpathrectangle{\pgfqpoint{7.622482in}{7.624184in}}{\pgfqpoint{2.177280in}{2.201755in}}%
\pgfusepath{clip}%
\pgfsetbuttcap%
\pgfsetroundjoin%
\definecolor{currentfill}{rgb}{0.121569,0.466667,0.705882}%
\pgfsetfillcolor{currentfill}%
\pgfsetlinewidth{0.481800pt}%
\definecolor{currentstroke}{rgb}{1.000000,1.000000,1.000000}%
\pgfsetstrokecolor{currentstroke}%
\pgfsetdash{}{0pt}%
\pgfpathmoveto{\pgfqpoint{7.954916in}{8.460995in}}%
\pgfpathcurveto{\pgfqpoint{7.965966in}{8.460995in}}{\pgfqpoint{7.976565in}{8.465385in}}{\pgfqpoint{7.984378in}{8.473199in}}%
\pgfpathcurveto{\pgfqpoint{7.992192in}{8.481013in}}{\pgfqpoint{7.996582in}{8.491612in}}{\pgfqpoint{7.996582in}{8.502662in}}%
\pgfpathcurveto{\pgfqpoint{7.996582in}{8.513712in}}{\pgfqpoint{7.992192in}{8.524311in}}{\pgfqpoint{7.984378in}{8.532125in}}%
\pgfpathcurveto{\pgfqpoint{7.976565in}{8.539938in}}{\pgfqpoint{7.965966in}{8.544328in}}{\pgfqpoint{7.954916in}{8.544328in}}%
\pgfpathcurveto{\pgfqpoint{7.943866in}{8.544328in}}{\pgfqpoint{7.933267in}{8.539938in}}{\pgfqpoint{7.925453in}{8.532125in}}%
\pgfpathcurveto{\pgfqpoint{7.917639in}{8.524311in}}{\pgfqpoint{7.913249in}{8.513712in}}{\pgfqpoint{7.913249in}{8.502662in}}%
\pgfpathcurveto{\pgfqpoint{7.913249in}{8.491612in}}{\pgfqpoint{7.917639in}{8.481013in}}{\pgfqpoint{7.925453in}{8.473199in}}%
\pgfpathcurveto{\pgfqpoint{7.933267in}{8.465385in}}{\pgfqpoint{7.943866in}{8.460995in}}{\pgfqpoint{7.954916in}{8.460995in}}%
\pgfpathlineto{\pgfqpoint{7.954916in}{8.460995in}}%
\pgfpathclose%
\pgfusepath{stroke,fill}%
\end{pgfscope}%
\begin{pgfscope}%
\pgfpathrectangle{\pgfqpoint{7.622482in}{7.624184in}}{\pgfqpoint{2.177280in}{2.201755in}}%
\pgfusepath{clip}%
\pgfsetbuttcap%
\pgfsetroundjoin%
\definecolor{currentfill}{rgb}{0.121569,0.466667,0.705882}%
\pgfsetfillcolor{currentfill}%
\pgfsetlinewidth{0.481800pt}%
\definecolor{currentstroke}{rgb}{1.000000,1.000000,1.000000}%
\pgfsetstrokecolor{currentstroke}%
\pgfsetdash{}{0pt}%
\pgfpathmoveto{\pgfqpoint{7.954916in}{8.127396in}}%
\pgfpathcurveto{\pgfqpoint{7.965966in}{8.127396in}}{\pgfqpoint{7.976565in}{8.131786in}}{\pgfqpoint{7.984378in}{8.139600in}}%
\pgfpathcurveto{\pgfqpoint{7.992192in}{8.147413in}}{\pgfqpoint{7.996582in}{8.158012in}}{\pgfqpoint{7.996582in}{8.169063in}}%
\pgfpathcurveto{\pgfqpoint{7.996582in}{8.180113in}}{\pgfqpoint{7.992192in}{8.190712in}}{\pgfqpoint{7.984378in}{8.198525in}}%
\pgfpathcurveto{\pgfqpoint{7.976565in}{8.206339in}}{\pgfqpoint{7.965966in}{8.210729in}}{\pgfqpoint{7.954916in}{8.210729in}}%
\pgfpathcurveto{\pgfqpoint{7.943866in}{8.210729in}}{\pgfqpoint{7.933267in}{8.206339in}}{\pgfqpoint{7.925453in}{8.198525in}}%
\pgfpathcurveto{\pgfqpoint{7.917639in}{8.190712in}}{\pgfqpoint{7.913249in}{8.180113in}}{\pgfqpoint{7.913249in}{8.169063in}}%
\pgfpathcurveto{\pgfqpoint{7.913249in}{8.158012in}}{\pgfqpoint{7.917639in}{8.147413in}}{\pgfqpoint{7.925453in}{8.139600in}}%
\pgfpathcurveto{\pgfqpoint{7.933267in}{8.131786in}}{\pgfqpoint{7.943866in}{8.127396in}}{\pgfqpoint{7.954916in}{8.127396in}}%
\pgfpathlineto{\pgfqpoint{7.954916in}{8.127396in}}%
\pgfpathclose%
\pgfusepath{stroke,fill}%
\end{pgfscope}%
\begin{pgfscope}%
\pgfpathrectangle{\pgfqpoint{7.622482in}{7.624184in}}{\pgfqpoint{2.177280in}{2.201755in}}%
\pgfusepath{clip}%
\pgfsetbuttcap%
\pgfsetroundjoin%
\definecolor{currentfill}{rgb}{0.121569,0.466667,0.705882}%
\pgfsetfillcolor{currentfill}%
\pgfsetlinewidth{0.481800pt}%
\definecolor{currentstroke}{rgb}{1.000000,1.000000,1.000000}%
\pgfsetstrokecolor{currentstroke}%
\pgfsetdash{}{0pt}%
\pgfpathmoveto{\pgfqpoint{7.887162in}{8.294195in}}%
\pgfpathcurveto{\pgfqpoint{7.898212in}{8.294195in}}{\pgfqpoint{7.908811in}{8.298586in}}{\pgfqpoint{7.916625in}{8.306399in}}%
\pgfpathcurveto{\pgfqpoint{7.924438in}{8.314213in}}{\pgfqpoint{7.928828in}{8.324812in}}{\pgfqpoint{7.928828in}{8.335862in}}%
\pgfpathcurveto{\pgfqpoint{7.928828in}{8.346912in}}{\pgfqpoint{7.924438in}{8.357511in}}{\pgfqpoint{7.916625in}{8.365325in}}%
\pgfpathcurveto{\pgfqpoint{7.908811in}{8.373139in}}{\pgfqpoint{7.898212in}{8.377529in}}{\pgfqpoint{7.887162in}{8.377529in}}%
\pgfpathcurveto{\pgfqpoint{7.876112in}{8.377529in}}{\pgfqpoint{7.865513in}{8.373139in}}{\pgfqpoint{7.857699in}{8.365325in}}%
\pgfpathcurveto{\pgfqpoint{7.849885in}{8.357511in}}{\pgfqpoint{7.845495in}{8.346912in}}{\pgfqpoint{7.845495in}{8.335862in}}%
\pgfpathcurveto{\pgfqpoint{7.845495in}{8.324812in}}{\pgfqpoint{7.849885in}{8.314213in}}{\pgfqpoint{7.857699in}{8.306399in}}%
\pgfpathcurveto{\pgfqpoint{7.865513in}{8.298586in}}{\pgfqpoint{7.876112in}{8.294195in}}{\pgfqpoint{7.887162in}{8.294195in}}%
\pgfpathlineto{\pgfqpoint{7.887162in}{8.294195in}}%
\pgfpathclose%
\pgfusepath{stroke,fill}%
\end{pgfscope}%
\begin{pgfscope}%
\pgfpathrectangle{\pgfqpoint{7.622482in}{7.624184in}}{\pgfqpoint{2.177280in}{2.201755in}}%
\pgfusepath{clip}%
\pgfsetbuttcap%
\pgfsetroundjoin%
\definecolor{currentfill}{rgb}{0.121569,0.466667,0.705882}%
\pgfsetfillcolor{currentfill}%
\pgfsetlinewidth{0.481800pt}%
\definecolor{currentstroke}{rgb}{1.000000,1.000000,1.000000}%
\pgfsetstrokecolor{currentstroke}%
\pgfsetdash{}{0pt}%
\pgfpathmoveto{\pgfqpoint{8.022670in}{8.127396in}}%
\pgfpathcurveto{\pgfqpoint{8.033720in}{8.127396in}}{\pgfqpoint{8.044319in}{8.131786in}}{\pgfqpoint{8.052132in}{8.139600in}}%
\pgfpathcurveto{\pgfqpoint{8.059946in}{8.147413in}}{\pgfqpoint{8.064336in}{8.158012in}}{\pgfqpoint{8.064336in}{8.169063in}}%
\pgfpathcurveto{\pgfqpoint{8.064336in}{8.180113in}}{\pgfqpoint{8.059946in}{8.190712in}}{\pgfqpoint{8.052132in}{8.198525in}}%
\pgfpathcurveto{\pgfqpoint{8.044319in}{8.206339in}}{\pgfqpoint{8.033720in}{8.210729in}}{\pgfqpoint{8.022670in}{8.210729in}}%
\pgfpathcurveto{\pgfqpoint{8.011619in}{8.210729in}}{\pgfqpoint{8.001020in}{8.206339in}}{\pgfqpoint{7.993207in}{8.198525in}}%
\pgfpathcurveto{\pgfqpoint{7.985393in}{8.190712in}}{\pgfqpoint{7.981003in}{8.180113in}}{\pgfqpoint{7.981003in}{8.169063in}}%
\pgfpathcurveto{\pgfqpoint{7.981003in}{8.158012in}}{\pgfqpoint{7.985393in}{8.147413in}}{\pgfqpoint{7.993207in}{8.139600in}}%
\pgfpathcurveto{\pgfqpoint{8.001020in}{8.131786in}}{\pgfqpoint{8.011619in}{8.127396in}}{\pgfqpoint{8.022670in}{8.127396in}}%
\pgfpathlineto{\pgfqpoint{8.022670in}{8.127396in}}%
\pgfpathclose%
\pgfusepath{stroke,fill}%
\end{pgfscope}%
\begin{pgfscope}%
\pgfpathrectangle{\pgfqpoint{7.622482in}{7.624184in}}{\pgfqpoint{2.177280in}{2.201755in}}%
\pgfusepath{clip}%
\pgfsetbuttcap%
\pgfsetroundjoin%
\definecolor{currentfill}{rgb}{0.121569,0.466667,0.705882}%
\pgfsetfillcolor{currentfill}%
\pgfsetlinewidth{0.481800pt}%
\definecolor{currentstroke}{rgb}{1.000000,1.000000,1.000000}%
\pgfsetstrokecolor{currentstroke}%
\pgfsetdash{}{0pt}%
\pgfpathmoveto{\pgfqpoint{7.887162in}{7.849396in}}%
\pgfpathcurveto{\pgfqpoint{7.898212in}{7.849396in}}{\pgfqpoint{7.908811in}{7.853787in}}{\pgfqpoint{7.916625in}{7.861600in}}%
\pgfpathcurveto{\pgfqpoint{7.924438in}{7.869414in}}{\pgfqpoint{7.928828in}{7.880013in}}{\pgfqpoint{7.928828in}{7.891063in}}%
\pgfpathcurveto{\pgfqpoint{7.928828in}{7.902113in}}{\pgfqpoint{7.924438in}{7.912712in}}{\pgfqpoint{7.916625in}{7.920526in}}%
\pgfpathcurveto{\pgfqpoint{7.908811in}{7.928340in}}{\pgfqpoint{7.898212in}{7.932730in}}{\pgfqpoint{7.887162in}{7.932730in}}%
\pgfpathcurveto{\pgfqpoint{7.876112in}{7.932730in}}{\pgfqpoint{7.865513in}{7.928340in}}{\pgfqpoint{7.857699in}{7.920526in}}%
\pgfpathcurveto{\pgfqpoint{7.849885in}{7.912712in}}{\pgfqpoint{7.845495in}{7.902113in}}{\pgfqpoint{7.845495in}{7.891063in}}%
\pgfpathcurveto{\pgfqpoint{7.845495in}{7.880013in}}{\pgfqpoint{7.849885in}{7.869414in}}{\pgfqpoint{7.857699in}{7.861600in}}%
\pgfpathcurveto{\pgfqpoint{7.865513in}{7.853787in}}{\pgfqpoint{7.876112in}{7.849396in}}{\pgfqpoint{7.887162in}{7.849396in}}%
\pgfpathlineto{\pgfqpoint{7.887162in}{7.849396in}}%
\pgfpathclose%
\pgfusepath{stroke,fill}%
\end{pgfscope}%
\begin{pgfscope}%
\pgfpathrectangle{\pgfqpoint{7.622482in}{7.624184in}}{\pgfqpoint{2.177280in}{2.201755in}}%
\pgfusepath{clip}%
\pgfsetbuttcap%
\pgfsetroundjoin%
\definecolor{currentfill}{rgb}{0.121569,0.466667,0.705882}%
\pgfsetfillcolor{currentfill}%
\pgfsetlinewidth{0.481800pt}%
\definecolor{currentstroke}{rgb}{1.000000,1.000000,1.000000}%
\pgfsetstrokecolor{currentstroke}%
\pgfsetdash{}{0pt}%
\pgfpathmoveto{\pgfqpoint{8.090423in}{8.127396in}}%
\pgfpathcurveto{\pgfqpoint{8.101474in}{8.127396in}}{\pgfqpoint{8.112073in}{8.131786in}}{\pgfqpoint{8.119886in}{8.139600in}}%
\pgfpathcurveto{\pgfqpoint{8.127700in}{8.147413in}}{\pgfqpoint{8.132090in}{8.158012in}}{\pgfqpoint{8.132090in}{8.169063in}}%
\pgfpathcurveto{\pgfqpoint{8.132090in}{8.180113in}}{\pgfqpoint{8.127700in}{8.190712in}}{\pgfqpoint{8.119886in}{8.198525in}}%
\pgfpathcurveto{\pgfqpoint{8.112073in}{8.206339in}}{\pgfqpoint{8.101474in}{8.210729in}}{\pgfqpoint{8.090423in}{8.210729in}}%
\pgfpathcurveto{\pgfqpoint{8.079373in}{8.210729in}}{\pgfqpoint{8.068774in}{8.206339in}}{\pgfqpoint{8.060961in}{8.198525in}}%
\pgfpathcurveto{\pgfqpoint{8.053147in}{8.190712in}}{\pgfqpoint{8.048757in}{8.180113in}}{\pgfqpoint{8.048757in}{8.169063in}}%
\pgfpathcurveto{\pgfqpoint{8.048757in}{8.158012in}}{\pgfqpoint{8.053147in}{8.147413in}}{\pgfqpoint{8.060961in}{8.139600in}}%
\pgfpathcurveto{\pgfqpoint{8.068774in}{8.131786in}}{\pgfqpoint{8.079373in}{8.127396in}}{\pgfqpoint{8.090423in}{8.127396in}}%
\pgfpathlineto{\pgfqpoint{8.090423in}{8.127396in}}%
\pgfpathclose%
\pgfusepath{stroke,fill}%
\end{pgfscope}%
\begin{pgfscope}%
\pgfpathrectangle{\pgfqpoint{7.622482in}{7.624184in}}{\pgfqpoint{2.177280in}{2.201755in}}%
\pgfusepath{clip}%
\pgfsetbuttcap%
\pgfsetroundjoin%
\definecolor{currentfill}{rgb}{0.121569,0.466667,0.705882}%
\pgfsetfillcolor{currentfill}%
\pgfsetlinewidth{0.481800pt}%
\definecolor{currentstroke}{rgb}{1.000000,1.000000,1.000000}%
\pgfsetstrokecolor{currentstroke}%
\pgfsetdash{}{0pt}%
\pgfpathmoveto{\pgfqpoint{7.887162in}{7.960596in}}%
\pgfpathcurveto{\pgfqpoint{7.898212in}{7.960596in}}{\pgfqpoint{7.908811in}{7.964986in}}{\pgfqpoint{7.916625in}{7.972800in}}%
\pgfpathcurveto{\pgfqpoint{7.924438in}{7.980614in}}{\pgfqpoint{7.928828in}{7.991213in}}{\pgfqpoint{7.928828in}{8.002263in}}%
\pgfpathcurveto{\pgfqpoint{7.928828in}{8.013313in}}{\pgfqpoint{7.924438in}{8.023912in}}{\pgfqpoint{7.916625in}{8.031726in}}%
\pgfpathcurveto{\pgfqpoint{7.908811in}{8.039539in}}{\pgfqpoint{7.898212in}{8.043930in}}{\pgfqpoint{7.887162in}{8.043930in}}%
\pgfpathcurveto{\pgfqpoint{7.876112in}{8.043930in}}{\pgfqpoint{7.865513in}{8.039539in}}{\pgfqpoint{7.857699in}{8.031726in}}%
\pgfpathcurveto{\pgfqpoint{7.849885in}{8.023912in}}{\pgfqpoint{7.845495in}{8.013313in}}{\pgfqpoint{7.845495in}{8.002263in}}%
\pgfpathcurveto{\pgfqpoint{7.845495in}{7.991213in}}{\pgfqpoint{7.849885in}{7.980614in}}{\pgfqpoint{7.857699in}{7.972800in}}%
\pgfpathcurveto{\pgfqpoint{7.865513in}{7.964986in}}{\pgfqpoint{7.876112in}{7.960596in}}{\pgfqpoint{7.887162in}{7.960596in}}%
\pgfpathlineto{\pgfqpoint{7.887162in}{7.960596in}}%
\pgfpathclose%
\pgfusepath{stroke,fill}%
\end{pgfscope}%
\begin{pgfscope}%
\pgfpathrectangle{\pgfqpoint{7.622482in}{7.624184in}}{\pgfqpoint{2.177280in}{2.201755in}}%
\pgfusepath{clip}%
\pgfsetbuttcap%
\pgfsetroundjoin%
\definecolor{currentfill}{rgb}{0.121569,0.466667,0.705882}%
\pgfsetfillcolor{currentfill}%
\pgfsetlinewidth{0.481800pt}%
\definecolor{currentstroke}{rgb}{1.000000,1.000000,1.000000}%
\pgfsetstrokecolor{currentstroke}%
\pgfsetdash{}{0pt}%
\pgfpathmoveto{\pgfqpoint{7.887162in}{8.071796in}}%
\pgfpathcurveto{\pgfqpoint{7.898212in}{8.071796in}}{\pgfqpoint{7.908811in}{8.076186in}}{\pgfqpoint{7.916625in}{8.084000in}}%
\pgfpathcurveto{\pgfqpoint{7.924438in}{8.091813in}}{\pgfqpoint{7.928828in}{8.102413in}}{\pgfqpoint{7.928828in}{8.113463in}}%
\pgfpathcurveto{\pgfqpoint{7.928828in}{8.124513in}}{\pgfqpoint{7.924438in}{8.135112in}}{\pgfqpoint{7.916625in}{8.142925in}}%
\pgfpathcurveto{\pgfqpoint{7.908811in}{8.150739in}}{\pgfqpoint{7.898212in}{8.155129in}}{\pgfqpoint{7.887162in}{8.155129in}}%
\pgfpathcurveto{\pgfqpoint{7.876112in}{8.155129in}}{\pgfqpoint{7.865513in}{8.150739in}}{\pgfqpoint{7.857699in}{8.142925in}}%
\pgfpathcurveto{\pgfqpoint{7.849885in}{8.135112in}}{\pgfqpoint{7.845495in}{8.124513in}}{\pgfqpoint{7.845495in}{8.113463in}}%
\pgfpathcurveto{\pgfqpoint{7.845495in}{8.102413in}}{\pgfqpoint{7.849885in}{8.091813in}}{\pgfqpoint{7.857699in}{8.084000in}}%
\pgfpathcurveto{\pgfqpoint{7.865513in}{8.076186in}}{\pgfqpoint{7.876112in}{8.071796in}}{\pgfqpoint{7.887162in}{8.071796in}}%
\pgfpathlineto{\pgfqpoint{7.887162in}{8.071796in}}%
\pgfpathclose%
\pgfusepath{stroke,fill}%
\end{pgfscope}%
\begin{pgfscope}%
\pgfpathrectangle{\pgfqpoint{7.622482in}{7.624184in}}{\pgfqpoint{2.177280in}{2.201755in}}%
\pgfusepath{clip}%
\pgfsetbuttcap%
\pgfsetroundjoin%
\definecolor{currentfill}{rgb}{0.121569,0.466667,0.705882}%
\pgfsetfillcolor{currentfill}%
\pgfsetlinewidth{0.481800pt}%
\definecolor{currentstroke}{rgb}{1.000000,1.000000,1.000000}%
\pgfsetstrokecolor{currentstroke}%
\pgfsetdash{}{0pt}%
\pgfpathmoveto{\pgfqpoint{8.022670in}{8.071796in}}%
\pgfpathcurveto{\pgfqpoint{8.033720in}{8.071796in}}{\pgfqpoint{8.044319in}{8.076186in}}{\pgfqpoint{8.052132in}{8.084000in}}%
\pgfpathcurveto{\pgfqpoint{8.059946in}{8.091813in}}{\pgfqpoint{8.064336in}{8.102413in}}{\pgfqpoint{8.064336in}{8.113463in}}%
\pgfpathcurveto{\pgfqpoint{8.064336in}{8.124513in}}{\pgfqpoint{8.059946in}{8.135112in}}{\pgfqpoint{8.052132in}{8.142925in}}%
\pgfpathcurveto{\pgfqpoint{8.044319in}{8.150739in}}{\pgfqpoint{8.033720in}{8.155129in}}{\pgfqpoint{8.022670in}{8.155129in}}%
\pgfpathcurveto{\pgfqpoint{8.011619in}{8.155129in}}{\pgfqpoint{8.001020in}{8.150739in}}{\pgfqpoint{7.993207in}{8.142925in}}%
\pgfpathcurveto{\pgfqpoint{7.985393in}{8.135112in}}{\pgfqpoint{7.981003in}{8.124513in}}{\pgfqpoint{7.981003in}{8.113463in}}%
\pgfpathcurveto{\pgfqpoint{7.981003in}{8.102413in}}{\pgfqpoint{7.985393in}{8.091813in}}{\pgfqpoint{7.993207in}{8.084000in}}%
\pgfpathcurveto{\pgfqpoint{8.001020in}{8.076186in}}{\pgfqpoint{8.011619in}{8.071796in}}{\pgfqpoint{8.022670in}{8.071796in}}%
\pgfpathlineto{\pgfqpoint{8.022670in}{8.071796in}}%
\pgfpathclose%
\pgfusepath{stroke,fill}%
\end{pgfscope}%
\begin{pgfscope}%
\pgfpathrectangle{\pgfqpoint{7.622482in}{7.624184in}}{\pgfqpoint{2.177280in}{2.201755in}}%
\pgfusepath{clip}%
\pgfsetbuttcap%
\pgfsetroundjoin%
\definecolor{currentfill}{rgb}{0.121569,0.466667,0.705882}%
\pgfsetfillcolor{currentfill}%
\pgfsetlinewidth{0.481800pt}%
\definecolor{currentstroke}{rgb}{1.000000,1.000000,1.000000}%
\pgfsetstrokecolor{currentstroke}%
\pgfsetdash{}{0pt}%
\pgfpathmoveto{\pgfqpoint{7.887162in}{8.182996in}}%
\pgfpathcurveto{\pgfqpoint{7.898212in}{8.182996in}}{\pgfqpoint{7.908811in}{8.187386in}}{\pgfqpoint{7.916625in}{8.195200in}}%
\pgfpathcurveto{\pgfqpoint{7.924438in}{8.203013in}}{\pgfqpoint{7.928828in}{8.213612in}}{\pgfqpoint{7.928828in}{8.224662in}}%
\pgfpathcurveto{\pgfqpoint{7.928828in}{8.235713in}}{\pgfqpoint{7.924438in}{8.246312in}}{\pgfqpoint{7.916625in}{8.254125in}}%
\pgfpathcurveto{\pgfqpoint{7.908811in}{8.261939in}}{\pgfqpoint{7.898212in}{8.266329in}}{\pgfqpoint{7.887162in}{8.266329in}}%
\pgfpathcurveto{\pgfqpoint{7.876112in}{8.266329in}}{\pgfqpoint{7.865513in}{8.261939in}}{\pgfqpoint{7.857699in}{8.254125in}}%
\pgfpathcurveto{\pgfqpoint{7.849885in}{8.246312in}}{\pgfqpoint{7.845495in}{8.235713in}}{\pgfqpoint{7.845495in}{8.224662in}}%
\pgfpathcurveto{\pgfqpoint{7.845495in}{8.213612in}}{\pgfqpoint{7.849885in}{8.203013in}}{\pgfqpoint{7.857699in}{8.195200in}}%
\pgfpathcurveto{\pgfqpoint{7.865513in}{8.187386in}}{\pgfqpoint{7.876112in}{8.182996in}}{\pgfqpoint{7.887162in}{8.182996in}}%
\pgfpathlineto{\pgfqpoint{7.887162in}{8.182996in}}%
\pgfpathclose%
\pgfusepath{stroke,fill}%
\end{pgfscope}%
\begin{pgfscope}%
\pgfpathrectangle{\pgfqpoint{7.622482in}{7.624184in}}{\pgfqpoint{2.177280in}{2.201755in}}%
\pgfusepath{clip}%
\pgfsetbuttcap%
\pgfsetroundjoin%
\definecolor{currentfill}{rgb}{0.121569,0.466667,0.705882}%
\pgfsetfillcolor{currentfill}%
\pgfsetlinewidth{0.481800pt}%
\definecolor{currentstroke}{rgb}{1.000000,1.000000,1.000000}%
\pgfsetstrokecolor{currentstroke}%
\pgfsetdash{}{0pt}%
\pgfpathmoveto{\pgfqpoint{7.887162in}{8.182996in}}%
\pgfpathcurveto{\pgfqpoint{7.898212in}{8.182996in}}{\pgfqpoint{7.908811in}{8.187386in}}{\pgfqpoint{7.916625in}{8.195200in}}%
\pgfpathcurveto{\pgfqpoint{7.924438in}{8.203013in}}{\pgfqpoint{7.928828in}{8.213612in}}{\pgfqpoint{7.928828in}{8.224662in}}%
\pgfpathcurveto{\pgfqpoint{7.928828in}{8.235713in}}{\pgfqpoint{7.924438in}{8.246312in}}{\pgfqpoint{7.916625in}{8.254125in}}%
\pgfpathcurveto{\pgfqpoint{7.908811in}{8.261939in}}{\pgfqpoint{7.898212in}{8.266329in}}{\pgfqpoint{7.887162in}{8.266329in}}%
\pgfpathcurveto{\pgfqpoint{7.876112in}{8.266329in}}{\pgfqpoint{7.865513in}{8.261939in}}{\pgfqpoint{7.857699in}{8.254125in}}%
\pgfpathcurveto{\pgfqpoint{7.849885in}{8.246312in}}{\pgfqpoint{7.845495in}{8.235713in}}{\pgfqpoint{7.845495in}{8.224662in}}%
\pgfpathcurveto{\pgfqpoint{7.845495in}{8.213612in}}{\pgfqpoint{7.849885in}{8.203013in}}{\pgfqpoint{7.857699in}{8.195200in}}%
\pgfpathcurveto{\pgfqpoint{7.865513in}{8.187386in}}{\pgfqpoint{7.876112in}{8.182996in}}{\pgfqpoint{7.887162in}{8.182996in}}%
\pgfpathlineto{\pgfqpoint{7.887162in}{8.182996in}}%
\pgfpathclose%
\pgfusepath{stroke,fill}%
\end{pgfscope}%
\begin{pgfscope}%
\pgfpathrectangle{\pgfqpoint{7.622482in}{7.624184in}}{\pgfqpoint{2.177280in}{2.201755in}}%
\pgfusepath{clip}%
\pgfsetbuttcap%
\pgfsetroundjoin%
\definecolor{currentfill}{rgb}{0.121569,0.466667,0.705882}%
\pgfsetfillcolor{currentfill}%
\pgfsetlinewidth{0.481800pt}%
\definecolor{currentstroke}{rgb}{1.000000,1.000000,1.000000}%
\pgfsetstrokecolor{currentstroke}%
\pgfsetdash{}{0pt}%
\pgfpathmoveto{\pgfqpoint{7.887162in}{7.904996in}}%
\pgfpathcurveto{\pgfqpoint{7.898212in}{7.904996in}}{\pgfqpoint{7.908811in}{7.909387in}}{\pgfqpoint{7.916625in}{7.917200in}}%
\pgfpathcurveto{\pgfqpoint{7.924438in}{7.925014in}}{\pgfqpoint{7.928828in}{7.935613in}}{\pgfqpoint{7.928828in}{7.946663in}}%
\pgfpathcurveto{\pgfqpoint{7.928828in}{7.957713in}}{\pgfqpoint{7.924438in}{7.968312in}}{\pgfqpoint{7.916625in}{7.976126in}}%
\pgfpathcurveto{\pgfqpoint{7.908811in}{7.983939in}}{\pgfqpoint{7.898212in}{7.988330in}}{\pgfqpoint{7.887162in}{7.988330in}}%
\pgfpathcurveto{\pgfqpoint{7.876112in}{7.988330in}}{\pgfqpoint{7.865513in}{7.983939in}}{\pgfqpoint{7.857699in}{7.976126in}}%
\pgfpathcurveto{\pgfqpoint{7.849885in}{7.968312in}}{\pgfqpoint{7.845495in}{7.957713in}}{\pgfqpoint{7.845495in}{7.946663in}}%
\pgfpathcurveto{\pgfqpoint{7.845495in}{7.935613in}}{\pgfqpoint{7.849885in}{7.925014in}}{\pgfqpoint{7.857699in}{7.917200in}}%
\pgfpathcurveto{\pgfqpoint{7.865513in}{7.909387in}}{\pgfqpoint{7.876112in}{7.904996in}}{\pgfqpoint{7.887162in}{7.904996in}}%
\pgfpathlineto{\pgfqpoint{7.887162in}{7.904996in}}%
\pgfpathclose%
\pgfusepath{stroke,fill}%
\end{pgfscope}%
\begin{pgfscope}%
\pgfpathrectangle{\pgfqpoint{7.622482in}{7.624184in}}{\pgfqpoint{2.177280in}{2.201755in}}%
\pgfusepath{clip}%
\pgfsetbuttcap%
\pgfsetroundjoin%
\definecolor{currentfill}{rgb}{0.121569,0.466667,0.705882}%
\pgfsetfillcolor{currentfill}%
\pgfsetlinewidth{0.481800pt}%
\definecolor{currentstroke}{rgb}{1.000000,1.000000,1.000000}%
\pgfsetstrokecolor{currentstroke}%
\pgfsetdash{}{0pt}%
\pgfpathmoveto{\pgfqpoint{7.887162in}{7.960596in}}%
\pgfpathcurveto{\pgfqpoint{7.898212in}{7.960596in}}{\pgfqpoint{7.908811in}{7.964986in}}{\pgfqpoint{7.916625in}{7.972800in}}%
\pgfpathcurveto{\pgfqpoint{7.924438in}{7.980614in}}{\pgfqpoint{7.928828in}{7.991213in}}{\pgfqpoint{7.928828in}{8.002263in}}%
\pgfpathcurveto{\pgfqpoint{7.928828in}{8.013313in}}{\pgfqpoint{7.924438in}{8.023912in}}{\pgfqpoint{7.916625in}{8.031726in}}%
\pgfpathcurveto{\pgfqpoint{7.908811in}{8.039539in}}{\pgfqpoint{7.898212in}{8.043930in}}{\pgfqpoint{7.887162in}{8.043930in}}%
\pgfpathcurveto{\pgfqpoint{7.876112in}{8.043930in}}{\pgfqpoint{7.865513in}{8.039539in}}{\pgfqpoint{7.857699in}{8.031726in}}%
\pgfpathcurveto{\pgfqpoint{7.849885in}{8.023912in}}{\pgfqpoint{7.845495in}{8.013313in}}{\pgfqpoint{7.845495in}{8.002263in}}%
\pgfpathcurveto{\pgfqpoint{7.845495in}{7.991213in}}{\pgfqpoint{7.849885in}{7.980614in}}{\pgfqpoint{7.857699in}{7.972800in}}%
\pgfpathcurveto{\pgfqpoint{7.865513in}{7.964986in}}{\pgfqpoint{7.876112in}{7.960596in}}{\pgfqpoint{7.887162in}{7.960596in}}%
\pgfpathlineto{\pgfqpoint{7.887162in}{7.960596in}}%
\pgfpathclose%
\pgfusepath{stroke,fill}%
\end{pgfscope}%
\begin{pgfscope}%
\pgfpathrectangle{\pgfqpoint{7.622482in}{7.624184in}}{\pgfqpoint{2.177280in}{2.201755in}}%
\pgfusepath{clip}%
\pgfsetbuttcap%
\pgfsetroundjoin%
\definecolor{currentfill}{rgb}{0.121569,0.466667,0.705882}%
\pgfsetfillcolor{currentfill}%
\pgfsetlinewidth{0.481800pt}%
\definecolor{currentstroke}{rgb}{1.000000,1.000000,1.000000}%
\pgfsetstrokecolor{currentstroke}%
\pgfsetdash{}{0pt}%
\pgfpathmoveto{\pgfqpoint{8.022670in}{8.294195in}}%
\pgfpathcurveto{\pgfqpoint{8.033720in}{8.294195in}}{\pgfqpoint{8.044319in}{8.298586in}}{\pgfqpoint{8.052132in}{8.306399in}}%
\pgfpathcurveto{\pgfqpoint{8.059946in}{8.314213in}}{\pgfqpoint{8.064336in}{8.324812in}}{\pgfqpoint{8.064336in}{8.335862in}}%
\pgfpathcurveto{\pgfqpoint{8.064336in}{8.346912in}}{\pgfqpoint{8.059946in}{8.357511in}}{\pgfqpoint{8.052132in}{8.365325in}}%
\pgfpathcurveto{\pgfqpoint{8.044319in}{8.373139in}}{\pgfqpoint{8.033720in}{8.377529in}}{\pgfqpoint{8.022670in}{8.377529in}}%
\pgfpathcurveto{\pgfqpoint{8.011619in}{8.377529in}}{\pgfqpoint{8.001020in}{8.373139in}}{\pgfqpoint{7.993207in}{8.365325in}}%
\pgfpathcurveto{\pgfqpoint{7.985393in}{8.357511in}}{\pgfqpoint{7.981003in}{8.346912in}}{\pgfqpoint{7.981003in}{8.335862in}}%
\pgfpathcurveto{\pgfqpoint{7.981003in}{8.324812in}}{\pgfqpoint{7.985393in}{8.314213in}}{\pgfqpoint{7.993207in}{8.306399in}}%
\pgfpathcurveto{\pgfqpoint{8.001020in}{8.298586in}}{\pgfqpoint{8.011619in}{8.294195in}}{\pgfqpoint{8.022670in}{8.294195in}}%
\pgfpathlineto{\pgfqpoint{8.022670in}{8.294195in}}%
\pgfpathclose%
\pgfusepath{stroke,fill}%
\end{pgfscope}%
\begin{pgfscope}%
\pgfpathrectangle{\pgfqpoint{7.622482in}{7.624184in}}{\pgfqpoint{2.177280in}{2.201755in}}%
\pgfusepath{clip}%
\pgfsetbuttcap%
\pgfsetroundjoin%
\definecolor{currentfill}{rgb}{0.121569,0.466667,0.705882}%
\pgfsetfillcolor{currentfill}%
\pgfsetlinewidth{0.481800pt}%
\definecolor{currentstroke}{rgb}{1.000000,1.000000,1.000000}%
\pgfsetstrokecolor{currentstroke}%
\pgfsetdash{}{0pt}%
\pgfpathmoveto{\pgfqpoint{7.819408in}{8.182996in}}%
\pgfpathcurveto{\pgfqpoint{7.830458in}{8.182996in}}{\pgfqpoint{7.841057in}{8.187386in}}{\pgfqpoint{7.848871in}{8.195200in}}%
\pgfpathcurveto{\pgfqpoint{7.856684in}{8.203013in}}{\pgfqpoint{7.861075in}{8.213612in}}{\pgfqpoint{7.861075in}{8.224662in}}%
\pgfpathcurveto{\pgfqpoint{7.861075in}{8.235713in}}{\pgfqpoint{7.856684in}{8.246312in}}{\pgfqpoint{7.848871in}{8.254125in}}%
\pgfpathcurveto{\pgfqpoint{7.841057in}{8.261939in}}{\pgfqpoint{7.830458in}{8.266329in}}{\pgfqpoint{7.819408in}{8.266329in}}%
\pgfpathcurveto{\pgfqpoint{7.808358in}{8.266329in}}{\pgfqpoint{7.797759in}{8.261939in}}{\pgfqpoint{7.789945in}{8.254125in}}%
\pgfpathcurveto{\pgfqpoint{7.782132in}{8.246312in}}{\pgfqpoint{7.777741in}{8.235713in}}{\pgfqpoint{7.777741in}{8.224662in}}%
\pgfpathcurveto{\pgfqpoint{7.777741in}{8.213612in}}{\pgfqpoint{7.782132in}{8.203013in}}{\pgfqpoint{7.789945in}{8.195200in}}%
\pgfpathcurveto{\pgfqpoint{7.797759in}{8.187386in}}{\pgfqpoint{7.808358in}{8.182996in}}{\pgfqpoint{7.819408in}{8.182996in}}%
\pgfpathlineto{\pgfqpoint{7.819408in}{8.182996in}}%
\pgfpathclose%
\pgfusepath{stroke,fill}%
\end{pgfscope}%
\begin{pgfscope}%
\pgfpathrectangle{\pgfqpoint{7.622482in}{7.624184in}}{\pgfqpoint{2.177280in}{2.201755in}}%
\pgfusepath{clip}%
\pgfsetbuttcap%
\pgfsetroundjoin%
\definecolor{currentfill}{rgb}{0.121569,0.466667,0.705882}%
\pgfsetfillcolor{currentfill}%
\pgfsetlinewidth{0.481800pt}%
\definecolor{currentstroke}{rgb}{1.000000,1.000000,1.000000}%
\pgfsetstrokecolor{currentstroke}%
\pgfsetdash{}{0pt}%
\pgfpathmoveto{\pgfqpoint{7.887162in}{8.349795in}}%
\pgfpathcurveto{\pgfqpoint{7.898212in}{8.349795in}}{\pgfqpoint{7.908811in}{8.354186in}}{\pgfqpoint{7.916625in}{8.361999in}}%
\pgfpathcurveto{\pgfqpoint{7.924438in}{8.369813in}}{\pgfqpoint{7.928828in}{8.380412in}}{\pgfqpoint{7.928828in}{8.391462in}}%
\pgfpathcurveto{\pgfqpoint{7.928828in}{8.402512in}}{\pgfqpoint{7.924438in}{8.413111in}}{\pgfqpoint{7.916625in}{8.420925in}}%
\pgfpathcurveto{\pgfqpoint{7.908811in}{8.428738in}}{\pgfqpoint{7.898212in}{8.433129in}}{\pgfqpoint{7.887162in}{8.433129in}}%
\pgfpathcurveto{\pgfqpoint{7.876112in}{8.433129in}}{\pgfqpoint{7.865513in}{8.428738in}}{\pgfqpoint{7.857699in}{8.420925in}}%
\pgfpathcurveto{\pgfqpoint{7.849885in}{8.413111in}}{\pgfqpoint{7.845495in}{8.402512in}}{\pgfqpoint{7.845495in}{8.391462in}}%
\pgfpathcurveto{\pgfqpoint{7.845495in}{8.380412in}}{\pgfqpoint{7.849885in}{8.369813in}}{\pgfqpoint{7.857699in}{8.361999in}}%
\pgfpathcurveto{\pgfqpoint{7.865513in}{8.354186in}}{\pgfqpoint{7.876112in}{8.349795in}}{\pgfqpoint{7.887162in}{8.349795in}}%
\pgfpathlineto{\pgfqpoint{7.887162in}{8.349795in}}%
\pgfpathclose%
\pgfusepath{stroke,fill}%
\end{pgfscope}%
\begin{pgfscope}%
\pgfpathrectangle{\pgfqpoint{7.622482in}{7.624184in}}{\pgfqpoint{2.177280in}{2.201755in}}%
\pgfusepath{clip}%
\pgfsetbuttcap%
\pgfsetroundjoin%
\definecolor{currentfill}{rgb}{0.121569,0.466667,0.705882}%
\pgfsetfillcolor{currentfill}%
\pgfsetlinewidth{0.481800pt}%
\definecolor{currentstroke}{rgb}{1.000000,1.000000,1.000000}%
\pgfsetstrokecolor{currentstroke}%
\pgfsetdash{}{0pt}%
\pgfpathmoveto{\pgfqpoint{7.887162in}{8.016196in}}%
\pgfpathcurveto{\pgfqpoint{7.898212in}{8.016196in}}{\pgfqpoint{7.908811in}{8.020586in}}{\pgfqpoint{7.916625in}{8.028400in}}%
\pgfpathcurveto{\pgfqpoint{7.924438in}{8.036214in}}{\pgfqpoint{7.928828in}{8.046813in}}{\pgfqpoint{7.928828in}{8.057863in}}%
\pgfpathcurveto{\pgfqpoint{7.928828in}{8.068913in}}{\pgfqpoint{7.924438in}{8.079512in}}{\pgfqpoint{7.916625in}{8.087326in}}%
\pgfpathcurveto{\pgfqpoint{7.908811in}{8.095139in}}{\pgfqpoint{7.898212in}{8.099529in}}{\pgfqpoint{7.887162in}{8.099529in}}%
\pgfpathcurveto{\pgfqpoint{7.876112in}{8.099529in}}{\pgfqpoint{7.865513in}{8.095139in}}{\pgfqpoint{7.857699in}{8.087326in}}%
\pgfpathcurveto{\pgfqpoint{7.849885in}{8.079512in}}{\pgfqpoint{7.845495in}{8.068913in}}{\pgfqpoint{7.845495in}{8.057863in}}%
\pgfpathcurveto{\pgfqpoint{7.845495in}{8.046813in}}{\pgfqpoint{7.849885in}{8.036214in}}{\pgfqpoint{7.857699in}{8.028400in}}%
\pgfpathcurveto{\pgfqpoint{7.865513in}{8.020586in}}{\pgfqpoint{7.876112in}{8.016196in}}{\pgfqpoint{7.887162in}{8.016196in}}%
\pgfpathlineto{\pgfqpoint{7.887162in}{8.016196in}}%
\pgfpathclose%
\pgfusepath{stroke,fill}%
\end{pgfscope}%
\begin{pgfscope}%
\pgfpathrectangle{\pgfqpoint{7.622482in}{7.624184in}}{\pgfqpoint{2.177280in}{2.201755in}}%
\pgfusepath{clip}%
\pgfsetbuttcap%
\pgfsetroundjoin%
\definecolor{currentfill}{rgb}{0.121569,0.466667,0.705882}%
\pgfsetfillcolor{currentfill}%
\pgfsetlinewidth{0.481800pt}%
\definecolor{currentstroke}{rgb}{1.000000,1.000000,1.000000}%
\pgfsetstrokecolor{currentstroke}%
\pgfsetdash{}{0pt}%
\pgfpathmoveto{\pgfqpoint{7.887162in}{8.071796in}}%
\pgfpathcurveto{\pgfqpoint{7.898212in}{8.071796in}}{\pgfqpoint{7.908811in}{8.076186in}}{\pgfqpoint{7.916625in}{8.084000in}}%
\pgfpathcurveto{\pgfqpoint{7.924438in}{8.091813in}}{\pgfqpoint{7.928828in}{8.102413in}}{\pgfqpoint{7.928828in}{8.113463in}}%
\pgfpathcurveto{\pgfqpoint{7.928828in}{8.124513in}}{\pgfqpoint{7.924438in}{8.135112in}}{\pgfqpoint{7.916625in}{8.142925in}}%
\pgfpathcurveto{\pgfqpoint{7.908811in}{8.150739in}}{\pgfqpoint{7.898212in}{8.155129in}}{\pgfqpoint{7.887162in}{8.155129in}}%
\pgfpathcurveto{\pgfqpoint{7.876112in}{8.155129in}}{\pgfqpoint{7.865513in}{8.150739in}}{\pgfqpoint{7.857699in}{8.142925in}}%
\pgfpathcurveto{\pgfqpoint{7.849885in}{8.135112in}}{\pgfqpoint{7.845495in}{8.124513in}}{\pgfqpoint{7.845495in}{8.113463in}}%
\pgfpathcurveto{\pgfqpoint{7.845495in}{8.102413in}}{\pgfqpoint{7.849885in}{8.091813in}}{\pgfqpoint{7.857699in}{8.084000in}}%
\pgfpathcurveto{\pgfqpoint{7.865513in}{8.076186in}}{\pgfqpoint{7.876112in}{8.071796in}}{\pgfqpoint{7.887162in}{8.071796in}}%
\pgfpathlineto{\pgfqpoint{7.887162in}{8.071796in}}%
\pgfpathclose%
\pgfusepath{stroke,fill}%
\end{pgfscope}%
\begin{pgfscope}%
\pgfpathrectangle{\pgfqpoint{7.622482in}{7.624184in}}{\pgfqpoint{2.177280in}{2.201755in}}%
\pgfusepath{clip}%
\pgfsetbuttcap%
\pgfsetroundjoin%
\definecolor{currentfill}{rgb}{0.121569,0.466667,0.705882}%
\pgfsetfillcolor{currentfill}%
\pgfsetlinewidth{0.481800pt}%
\definecolor{currentstroke}{rgb}{1.000000,1.000000,1.000000}%
\pgfsetstrokecolor{currentstroke}%
\pgfsetdash{}{0pt}%
\pgfpathmoveto{\pgfqpoint{7.887162in}{8.349795in}}%
\pgfpathcurveto{\pgfqpoint{7.898212in}{8.349795in}}{\pgfqpoint{7.908811in}{8.354186in}}{\pgfqpoint{7.916625in}{8.361999in}}%
\pgfpathcurveto{\pgfqpoint{7.924438in}{8.369813in}}{\pgfqpoint{7.928828in}{8.380412in}}{\pgfqpoint{7.928828in}{8.391462in}}%
\pgfpathcurveto{\pgfqpoint{7.928828in}{8.402512in}}{\pgfqpoint{7.924438in}{8.413111in}}{\pgfqpoint{7.916625in}{8.420925in}}%
\pgfpathcurveto{\pgfqpoint{7.908811in}{8.428738in}}{\pgfqpoint{7.898212in}{8.433129in}}{\pgfqpoint{7.887162in}{8.433129in}}%
\pgfpathcurveto{\pgfqpoint{7.876112in}{8.433129in}}{\pgfqpoint{7.865513in}{8.428738in}}{\pgfqpoint{7.857699in}{8.420925in}}%
\pgfpathcurveto{\pgfqpoint{7.849885in}{8.413111in}}{\pgfqpoint{7.845495in}{8.402512in}}{\pgfqpoint{7.845495in}{8.391462in}}%
\pgfpathcurveto{\pgfqpoint{7.845495in}{8.380412in}}{\pgfqpoint{7.849885in}{8.369813in}}{\pgfqpoint{7.857699in}{8.361999in}}%
\pgfpathcurveto{\pgfqpoint{7.865513in}{8.354186in}}{\pgfqpoint{7.876112in}{8.349795in}}{\pgfqpoint{7.887162in}{8.349795in}}%
\pgfpathlineto{\pgfqpoint{7.887162in}{8.349795in}}%
\pgfpathclose%
\pgfusepath{stroke,fill}%
\end{pgfscope}%
\begin{pgfscope}%
\pgfpathrectangle{\pgfqpoint{7.622482in}{7.624184in}}{\pgfqpoint{2.177280in}{2.201755in}}%
\pgfusepath{clip}%
\pgfsetbuttcap%
\pgfsetroundjoin%
\definecolor{currentfill}{rgb}{0.121569,0.466667,0.705882}%
\pgfsetfillcolor{currentfill}%
\pgfsetlinewidth{0.481800pt}%
\definecolor{currentstroke}{rgb}{1.000000,1.000000,1.000000}%
\pgfsetstrokecolor{currentstroke}%
\pgfsetdash{}{0pt}%
\pgfpathmoveto{\pgfqpoint{7.819408in}{8.016196in}}%
\pgfpathcurveto{\pgfqpoint{7.830458in}{8.016196in}}{\pgfqpoint{7.841057in}{8.020586in}}{\pgfqpoint{7.848871in}{8.028400in}}%
\pgfpathcurveto{\pgfqpoint{7.856684in}{8.036214in}}{\pgfqpoint{7.861075in}{8.046813in}}{\pgfqpoint{7.861075in}{8.057863in}}%
\pgfpathcurveto{\pgfqpoint{7.861075in}{8.068913in}}{\pgfqpoint{7.856684in}{8.079512in}}{\pgfqpoint{7.848871in}{8.087326in}}%
\pgfpathcurveto{\pgfqpoint{7.841057in}{8.095139in}}{\pgfqpoint{7.830458in}{8.099529in}}{\pgfqpoint{7.819408in}{8.099529in}}%
\pgfpathcurveto{\pgfqpoint{7.808358in}{8.099529in}}{\pgfqpoint{7.797759in}{8.095139in}}{\pgfqpoint{7.789945in}{8.087326in}}%
\pgfpathcurveto{\pgfqpoint{7.782132in}{8.079512in}}{\pgfqpoint{7.777741in}{8.068913in}}{\pgfqpoint{7.777741in}{8.057863in}}%
\pgfpathcurveto{\pgfqpoint{7.777741in}{8.046813in}}{\pgfqpoint{7.782132in}{8.036214in}}{\pgfqpoint{7.789945in}{8.028400in}}%
\pgfpathcurveto{\pgfqpoint{7.797759in}{8.020586in}}{\pgfqpoint{7.808358in}{8.016196in}}{\pgfqpoint{7.819408in}{8.016196in}}%
\pgfpathlineto{\pgfqpoint{7.819408in}{8.016196in}}%
\pgfpathclose%
\pgfusepath{stroke,fill}%
\end{pgfscope}%
\begin{pgfscope}%
\pgfpathrectangle{\pgfqpoint{7.622482in}{7.624184in}}{\pgfqpoint{2.177280in}{2.201755in}}%
\pgfusepath{clip}%
\pgfsetbuttcap%
\pgfsetroundjoin%
\definecolor{currentfill}{rgb}{0.121569,0.466667,0.705882}%
\pgfsetfillcolor{currentfill}%
\pgfsetlinewidth{0.481800pt}%
\definecolor{currentstroke}{rgb}{1.000000,1.000000,1.000000}%
\pgfsetstrokecolor{currentstroke}%
\pgfsetdash{}{0pt}%
\pgfpathmoveto{\pgfqpoint{7.887162in}{7.738197in}}%
\pgfpathcurveto{\pgfqpoint{7.898212in}{7.738197in}}{\pgfqpoint{7.908811in}{7.742587in}}{\pgfqpoint{7.916625in}{7.750401in}}%
\pgfpathcurveto{\pgfqpoint{7.924438in}{7.758214in}}{\pgfqpoint{7.928828in}{7.768813in}}{\pgfqpoint{7.928828in}{7.779863in}}%
\pgfpathcurveto{\pgfqpoint{7.928828in}{7.790914in}}{\pgfqpoint{7.924438in}{7.801513in}}{\pgfqpoint{7.916625in}{7.809326in}}%
\pgfpathcurveto{\pgfqpoint{7.908811in}{7.817140in}}{\pgfqpoint{7.898212in}{7.821530in}}{\pgfqpoint{7.887162in}{7.821530in}}%
\pgfpathcurveto{\pgfqpoint{7.876112in}{7.821530in}}{\pgfqpoint{7.865513in}{7.817140in}}{\pgfqpoint{7.857699in}{7.809326in}}%
\pgfpathcurveto{\pgfqpoint{7.849885in}{7.801513in}}{\pgfqpoint{7.845495in}{7.790914in}}{\pgfqpoint{7.845495in}{7.779863in}}%
\pgfpathcurveto{\pgfqpoint{7.845495in}{7.768813in}}{\pgfqpoint{7.849885in}{7.758214in}}{\pgfqpoint{7.857699in}{7.750401in}}%
\pgfpathcurveto{\pgfqpoint{7.865513in}{7.742587in}}{\pgfqpoint{7.876112in}{7.738197in}}{\pgfqpoint{7.887162in}{7.738197in}}%
\pgfpathlineto{\pgfqpoint{7.887162in}{7.738197in}}%
\pgfpathclose%
\pgfusepath{stroke,fill}%
\end{pgfscope}%
\begin{pgfscope}%
\pgfpathrectangle{\pgfqpoint{7.622482in}{7.624184in}}{\pgfqpoint{2.177280in}{2.201755in}}%
\pgfusepath{clip}%
\pgfsetbuttcap%
\pgfsetroundjoin%
\definecolor{currentfill}{rgb}{0.121569,0.466667,0.705882}%
\pgfsetfillcolor{currentfill}%
\pgfsetlinewidth{0.481800pt}%
\definecolor{currentstroke}{rgb}{1.000000,1.000000,1.000000}%
\pgfsetstrokecolor{currentstroke}%
\pgfsetdash{}{0pt}%
\pgfpathmoveto{\pgfqpoint{7.887162in}{8.127396in}}%
\pgfpathcurveto{\pgfqpoint{7.898212in}{8.127396in}}{\pgfqpoint{7.908811in}{8.131786in}}{\pgfqpoint{7.916625in}{8.139600in}}%
\pgfpathcurveto{\pgfqpoint{7.924438in}{8.147413in}}{\pgfqpoint{7.928828in}{8.158012in}}{\pgfqpoint{7.928828in}{8.169063in}}%
\pgfpathcurveto{\pgfqpoint{7.928828in}{8.180113in}}{\pgfqpoint{7.924438in}{8.190712in}}{\pgfqpoint{7.916625in}{8.198525in}}%
\pgfpathcurveto{\pgfqpoint{7.908811in}{8.206339in}}{\pgfqpoint{7.898212in}{8.210729in}}{\pgfqpoint{7.887162in}{8.210729in}}%
\pgfpathcurveto{\pgfqpoint{7.876112in}{8.210729in}}{\pgfqpoint{7.865513in}{8.206339in}}{\pgfqpoint{7.857699in}{8.198525in}}%
\pgfpathcurveto{\pgfqpoint{7.849885in}{8.190712in}}{\pgfqpoint{7.845495in}{8.180113in}}{\pgfqpoint{7.845495in}{8.169063in}}%
\pgfpathcurveto{\pgfqpoint{7.845495in}{8.158012in}}{\pgfqpoint{7.849885in}{8.147413in}}{\pgfqpoint{7.857699in}{8.139600in}}%
\pgfpathcurveto{\pgfqpoint{7.865513in}{8.131786in}}{\pgfqpoint{7.876112in}{8.127396in}}{\pgfqpoint{7.887162in}{8.127396in}}%
\pgfpathlineto{\pgfqpoint{7.887162in}{8.127396in}}%
\pgfpathclose%
\pgfusepath{stroke,fill}%
\end{pgfscope}%
\begin{pgfscope}%
\pgfpathrectangle{\pgfqpoint{7.622482in}{7.624184in}}{\pgfqpoint{2.177280in}{2.201755in}}%
\pgfusepath{clip}%
\pgfsetbuttcap%
\pgfsetroundjoin%
\definecolor{currentfill}{rgb}{0.121569,0.466667,0.705882}%
\pgfsetfillcolor{currentfill}%
\pgfsetlinewidth{0.481800pt}%
\definecolor{currentstroke}{rgb}{1.000000,1.000000,1.000000}%
\pgfsetstrokecolor{currentstroke}%
\pgfsetdash{}{0pt}%
\pgfpathmoveto{\pgfqpoint{7.954916in}{8.071796in}}%
\pgfpathcurveto{\pgfqpoint{7.965966in}{8.071796in}}{\pgfqpoint{7.976565in}{8.076186in}}{\pgfqpoint{7.984378in}{8.084000in}}%
\pgfpathcurveto{\pgfqpoint{7.992192in}{8.091813in}}{\pgfqpoint{7.996582in}{8.102413in}}{\pgfqpoint{7.996582in}{8.113463in}}%
\pgfpathcurveto{\pgfqpoint{7.996582in}{8.124513in}}{\pgfqpoint{7.992192in}{8.135112in}}{\pgfqpoint{7.984378in}{8.142925in}}%
\pgfpathcurveto{\pgfqpoint{7.976565in}{8.150739in}}{\pgfqpoint{7.965966in}{8.155129in}}{\pgfqpoint{7.954916in}{8.155129in}}%
\pgfpathcurveto{\pgfqpoint{7.943866in}{8.155129in}}{\pgfqpoint{7.933267in}{8.150739in}}{\pgfqpoint{7.925453in}{8.142925in}}%
\pgfpathcurveto{\pgfqpoint{7.917639in}{8.135112in}}{\pgfqpoint{7.913249in}{8.124513in}}{\pgfqpoint{7.913249in}{8.113463in}}%
\pgfpathcurveto{\pgfqpoint{7.913249in}{8.102413in}}{\pgfqpoint{7.917639in}{8.091813in}}{\pgfqpoint{7.925453in}{8.084000in}}%
\pgfpathcurveto{\pgfqpoint{7.933267in}{8.076186in}}{\pgfqpoint{7.943866in}{8.071796in}}{\pgfqpoint{7.954916in}{8.071796in}}%
\pgfpathlineto{\pgfqpoint{7.954916in}{8.071796in}}%
\pgfpathclose%
\pgfusepath{stroke,fill}%
\end{pgfscope}%
\begin{pgfscope}%
\pgfpathrectangle{\pgfqpoint{7.622482in}{7.624184in}}{\pgfqpoint{2.177280in}{2.201755in}}%
\pgfusepath{clip}%
\pgfsetbuttcap%
\pgfsetroundjoin%
\definecolor{currentfill}{rgb}{0.121569,0.466667,0.705882}%
\pgfsetfillcolor{currentfill}%
\pgfsetlinewidth{0.481800pt}%
\definecolor{currentstroke}{rgb}{1.000000,1.000000,1.000000}%
\pgfsetstrokecolor{currentstroke}%
\pgfsetdash{}{0pt}%
\pgfpathmoveto{\pgfqpoint{7.954916in}{7.793797in}}%
\pgfpathcurveto{\pgfqpoint{7.965966in}{7.793797in}}{\pgfqpoint{7.976565in}{7.798187in}}{\pgfqpoint{7.984378in}{7.806000in}}%
\pgfpathcurveto{\pgfqpoint{7.992192in}{7.813814in}}{\pgfqpoint{7.996582in}{7.824413in}}{\pgfqpoint{7.996582in}{7.835463in}}%
\pgfpathcurveto{\pgfqpoint{7.996582in}{7.846513in}}{\pgfqpoint{7.992192in}{7.857112in}}{\pgfqpoint{7.984378in}{7.864926in}}%
\pgfpathcurveto{\pgfqpoint{7.976565in}{7.872740in}}{\pgfqpoint{7.965966in}{7.877130in}}{\pgfqpoint{7.954916in}{7.877130in}}%
\pgfpathcurveto{\pgfqpoint{7.943866in}{7.877130in}}{\pgfqpoint{7.933267in}{7.872740in}}{\pgfqpoint{7.925453in}{7.864926in}}%
\pgfpathcurveto{\pgfqpoint{7.917639in}{7.857112in}}{\pgfqpoint{7.913249in}{7.846513in}}{\pgfqpoint{7.913249in}{7.835463in}}%
\pgfpathcurveto{\pgfqpoint{7.913249in}{7.824413in}}{\pgfqpoint{7.917639in}{7.813814in}}{\pgfqpoint{7.925453in}{7.806000in}}%
\pgfpathcurveto{\pgfqpoint{7.933267in}{7.798187in}}{\pgfqpoint{7.943866in}{7.793797in}}{\pgfqpoint{7.954916in}{7.793797in}}%
\pgfpathlineto{\pgfqpoint{7.954916in}{7.793797in}}%
\pgfpathclose%
\pgfusepath{stroke,fill}%
\end{pgfscope}%
\begin{pgfscope}%
\pgfpathrectangle{\pgfqpoint{7.622482in}{7.624184in}}{\pgfqpoint{2.177280in}{2.201755in}}%
\pgfusepath{clip}%
\pgfsetbuttcap%
\pgfsetroundjoin%
\definecolor{currentfill}{rgb}{0.121569,0.466667,0.705882}%
\pgfsetfillcolor{currentfill}%
\pgfsetlinewidth{0.481800pt}%
\definecolor{currentstroke}{rgb}{1.000000,1.000000,1.000000}%
\pgfsetstrokecolor{currentstroke}%
\pgfsetdash{}{0pt}%
\pgfpathmoveto{\pgfqpoint{7.887162in}{7.738197in}}%
\pgfpathcurveto{\pgfqpoint{7.898212in}{7.738197in}}{\pgfqpoint{7.908811in}{7.742587in}}{\pgfqpoint{7.916625in}{7.750401in}}%
\pgfpathcurveto{\pgfqpoint{7.924438in}{7.758214in}}{\pgfqpoint{7.928828in}{7.768813in}}{\pgfqpoint{7.928828in}{7.779863in}}%
\pgfpathcurveto{\pgfqpoint{7.928828in}{7.790914in}}{\pgfqpoint{7.924438in}{7.801513in}}{\pgfqpoint{7.916625in}{7.809326in}}%
\pgfpathcurveto{\pgfqpoint{7.908811in}{7.817140in}}{\pgfqpoint{7.898212in}{7.821530in}}{\pgfqpoint{7.887162in}{7.821530in}}%
\pgfpathcurveto{\pgfqpoint{7.876112in}{7.821530in}}{\pgfqpoint{7.865513in}{7.817140in}}{\pgfqpoint{7.857699in}{7.809326in}}%
\pgfpathcurveto{\pgfqpoint{7.849885in}{7.801513in}}{\pgfqpoint{7.845495in}{7.790914in}}{\pgfqpoint{7.845495in}{7.779863in}}%
\pgfpathcurveto{\pgfqpoint{7.845495in}{7.768813in}}{\pgfqpoint{7.849885in}{7.758214in}}{\pgfqpoint{7.857699in}{7.750401in}}%
\pgfpathcurveto{\pgfqpoint{7.865513in}{7.742587in}}{\pgfqpoint{7.876112in}{7.738197in}}{\pgfqpoint{7.887162in}{7.738197in}}%
\pgfpathlineto{\pgfqpoint{7.887162in}{7.738197in}}%
\pgfpathclose%
\pgfusepath{stroke,fill}%
\end{pgfscope}%
\begin{pgfscope}%
\pgfpathrectangle{\pgfqpoint{7.622482in}{7.624184in}}{\pgfqpoint{2.177280in}{2.201755in}}%
\pgfusepath{clip}%
\pgfsetbuttcap%
\pgfsetroundjoin%
\definecolor{currentfill}{rgb}{0.121569,0.466667,0.705882}%
\pgfsetfillcolor{currentfill}%
\pgfsetlinewidth{0.481800pt}%
\definecolor{currentstroke}{rgb}{1.000000,1.000000,1.000000}%
\pgfsetstrokecolor{currentstroke}%
\pgfsetdash{}{0pt}%
\pgfpathmoveto{\pgfqpoint{8.158177in}{8.071796in}}%
\pgfpathcurveto{\pgfqpoint{8.169227in}{8.071796in}}{\pgfqpoint{8.179826in}{8.076186in}}{\pgfqpoint{8.187640in}{8.084000in}}%
\pgfpathcurveto{\pgfqpoint{8.195454in}{8.091813in}}{\pgfqpoint{8.199844in}{8.102413in}}{\pgfqpoint{8.199844in}{8.113463in}}%
\pgfpathcurveto{\pgfqpoint{8.199844in}{8.124513in}}{\pgfqpoint{8.195454in}{8.135112in}}{\pgfqpoint{8.187640in}{8.142925in}}%
\pgfpathcurveto{\pgfqpoint{8.179826in}{8.150739in}}{\pgfqpoint{8.169227in}{8.155129in}}{\pgfqpoint{8.158177in}{8.155129in}}%
\pgfpathcurveto{\pgfqpoint{8.147127in}{8.155129in}}{\pgfqpoint{8.136528in}{8.150739in}}{\pgfqpoint{8.128714in}{8.142925in}}%
\pgfpathcurveto{\pgfqpoint{8.120901in}{8.135112in}}{\pgfqpoint{8.116511in}{8.124513in}}{\pgfqpoint{8.116511in}{8.113463in}}%
\pgfpathcurveto{\pgfqpoint{8.116511in}{8.102413in}}{\pgfqpoint{8.120901in}{8.091813in}}{\pgfqpoint{8.128714in}{8.084000in}}%
\pgfpathcurveto{\pgfqpoint{8.136528in}{8.076186in}}{\pgfqpoint{8.147127in}{8.071796in}}{\pgfqpoint{8.158177in}{8.071796in}}%
\pgfpathlineto{\pgfqpoint{8.158177in}{8.071796in}}%
\pgfpathclose%
\pgfusepath{stroke,fill}%
\end{pgfscope}%
\begin{pgfscope}%
\pgfpathrectangle{\pgfqpoint{7.622482in}{7.624184in}}{\pgfqpoint{2.177280in}{2.201755in}}%
\pgfusepath{clip}%
\pgfsetbuttcap%
\pgfsetroundjoin%
\definecolor{currentfill}{rgb}{0.121569,0.466667,0.705882}%
\pgfsetfillcolor{currentfill}%
\pgfsetlinewidth{0.481800pt}%
\definecolor{currentstroke}{rgb}{1.000000,1.000000,1.000000}%
\pgfsetstrokecolor{currentstroke}%
\pgfsetdash{}{0pt}%
\pgfpathmoveto{\pgfqpoint{8.022670in}{8.127396in}}%
\pgfpathcurveto{\pgfqpoint{8.033720in}{8.127396in}}{\pgfqpoint{8.044319in}{8.131786in}}{\pgfqpoint{8.052132in}{8.139600in}}%
\pgfpathcurveto{\pgfqpoint{8.059946in}{8.147413in}}{\pgfqpoint{8.064336in}{8.158012in}}{\pgfqpoint{8.064336in}{8.169063in}}%
\pgfpathcurveto{\pgfqpoint{8.064336in}{8.180113in}}{\pgfqpoint{8.059946in}{8.190712in}}{\pgfqpoint{8.052132in}{8.198525in}}%
\pgfpathcurveto{\pgfqpoint{8.044319in}{8.206339in}}{\pgfqpoint{8.033720in}{8.210729in}}{\pgfqpoint{8.022670in}{8.210729in}}%
\pgfpathcurveto{\pgfqpoint{8.011619in}{8.210729in}}{\pgfqpoint{8.001020in}{8.206339in}}{\pgfqpoint{7.993207in}{8.198525in}}%
\pgfpathcurveto{\pgfqpoint{7.985393in}{8.190712in}}{\pgfqpoint{7.981003in}{8.180113in}}{\pgfqpoint{7.981003in}{8.169063in}}%
\pgfpathcurveto{\pgfqpoint{7.981003in}{8.158012in}}{\pgfqpoint{7.985393in}{8.147413in}}{\pgfqpoint{7.993207in}{8.139600in}}%
\pgfpathcurveto{\pgfqpoint{8.001020in}{8.131786in}}{\pgfqpoint{8.011619in}{8.127396in}}{\pgfqpoint{8.022670in}{8.127396in}}%
\pgfpathlineto{\pgfqpoint{8.022670in}{8.127396in}}%
\pgfpathclose%
\pgfusepath{stroke,fill}%
\end{pgfscope}%
\begin{pgfscope}%
\pgfpathrectangle{\pgfqpoint{7.622482in}{7.624184in}}{\pgfqpoint{2.177280in}{2.201755in}}%
\pgfusepath{clip}%
\pgfsetbuttcap%
\pgfsetroundjoin%
\definecolor{currentfill}{rgb}{0.121569,0.466667,0.705882}%
\pgfsetfillcolor{currentfill}%
\pgfsetlinewidth{0.481800pt}%
\definecolor{currentstroke}{rgb}{1.000000,1.000000,1.000000}%
\pgfsetstrokecolor{currentstroke}%
\pgfsetdash{}{0pt}%
\pgfpathmoveto{\pgfqpoint{7.954916in}{7.960596in}}%
\pgfpathcurveto{\pgfqpoint{7.965966in}{7.960596in}}{\pgfqpoint{7.976565in}{7.964986in}}{\pgfqpoint{7.984378in}{7.972800in}}%
\pgfpathcurveto{\pgfqpoint{7.992192in}{7.980614in}}{\pgfqpoint{7.996582in}{7.991213in}}{\pgfqpoint{7.996582in}{8.002263in}}%
\pgfpathcurveto{\pgfqpoint{7.996582in}{8.013313in}}{\pgfqpoint{7.992192in}{8.023912in}}{\pgfqpoint{7.984378in}{8.031726in}}%
\pgfpathcurveto{\pgfqpoint{7.976565in}{8.039539in}}{\pgfqpoint{7.965966in}{8.043930in}}{\pgfqpoint{7.954916in}{8.043930in}}%
\pgfpathcurveto{\pgfqpoint{7.943866in}{8.043930in}}{\pgfqpoint{7.933267in}{8.039539in}}{\pgfqpoint{7.925453in}{8.031726in}}%
\pgfpathcurveto{\pgfqpoint{7.917639in}{8.023912in}}{\pgfqpoint{7.913249in}{8.013313in}}{\pgfqpoint{7.913249in}{8.002263in}}%
\pgfpathcurveto{\pgfqpoint{7.913249in}{7.991213in}}{\pgfqpoint{7.917639in}{7.980614in}}{\pgfqpoint{7.925453in}{7.972800in}}%
\pgfpathcurveto{\pgfqpoint{7.933267in}{7.964986in}}{\pgfqpoint{7.943866in}{7.960596in}}{\pgfqpoint{7.954916in}{7.960596in}}%
\pgfpathlineto{\pgfqpoint{7.954916in}{7.960596in}}%
\pgfpathclose%
\pgfusepath{stroke,fill}%
\end{pgfscope}%
\begin{pgfscope}%
\pgfpathrectangle{\pgfqpoint{7.622482in}{7.624184in}}{\pgfqpoint{2.177280in}{2.201755in}}%
\pgfusepath{clip}%
\pgfsetbuttcap%
\pgfsetroundjoin%
\definecolor{currentfill}{rgb}{0.121569,0.466667,0.705882}%
\pgfsetfillcolor{currentfill}%
\pgfsetlinewidth{0.481800pt}%
\definecolor{currentstroke}{rgb}{1.000000,1.000000,1.000000}%
\pgfsetstrokecolor{currentstroke}%
\pgfsetdash{}{0pt}%
\pgfpathmoveto{\pgfqpoint{7.887162in}{8.127396in}}%
\pgfpathcurveto{\pgfqpoint{7.898212in}{8.127396in}}{\pgfqpoint{7.908811in}{8.131786in}}{\pgfqpoint{7.916625in}{8.139600in}}%
\pgfpathcurveto{\pgfqpoint{7.924438in}{8.147413in}}{\pgfqpoint{7.928828in}{8.158012in}}{\pgfqpoint{7.928828in}{8.169063in}}%
\pgfpathcurveto{\pgfqpoint{7.928828in}{8.180113in}}{\pgfqpoint{7.924438in}{8.190712in}}{\pgfqpoint{7.916625in}{8.198525in}}%
\pgfpathcurveto{\pgfqpoint{7.908811in}{8.206339in}}{\pgfqpoint{7.898212in}{8.210729in}}{\pgfqpoint{7.887162in}{8.210729in}}%
\pgfpathcurveto{\pgfqpoint{7.876112in}{8.210729in}}{\pgfqpoint{7.865513in}{8.206339in}}{\pgfqpoint{7.857699in}{8.198525in}}%
\pgfpathcurveto{\pgfqpoint{7.849885in}{8.190712in}}{\pgfqpoint{7.845495in}{8.180113in}}{\pgfqpoint{7.845495in}{8.169063in}}%
\pgfpathcurveto{\pgfqpoint{7.845495in}{8.158012in}}{\pgfqpoint{7.849885in}{8.147413in}}{\pgfqpoint{7.857699in}{8.139600in}}%
\pgfpathcurveto{\pgfqpoint{7.865513in}{8.131786in}}{\pgfqpoint{7.876112in}{8.127396in}}{\pgfqpoint{7.887162in}{8.127396in}}%
\pgfpathlineto{\pgfqpoint{7.887162in}{8.127396in}}%
\pgfpathclose%
\pgfusepath{stroke,fill}%
\end{pgfscope}%
\begin{pgfscope}%
\pgfpathrectangle{\pgfqpoint{7.622482in}{7.624184in}}{\pgfqpoint{2.177280in}{2.201755in}}%
\pgfusepath{clip}%
\pgfsetbuttcap%
\pgfsetroundjoin%
\definecolor{currentfill}{rgb}{0.121569,0.466667,0.705882}%
\pgfsetfillcolor{currentfill}%
\pgfsetlinewidth{0.481800pt}%
\definecolor{currentstroke}{rgb}{1.000000,1.000000,1.000000}%
\pgfsetstrokecolor{currentstroke}%
\pgfsetdash{}{0pt}%
\pgfpathmoveto{\pgfqpoint{7.887162in}{7.849396in}}%
\pgfpathcurveto{\pgfqpoint{7.898212in}{7.849396in}}{\pgfqpoint{7.908811in}{7.853787in}}{\pgfqpoint{7.916625in}{7.861600in}}%
\pgfpathcurveto{\pgfqpoint{7.924438in}{7.869414in}}{\pgfqpoint{7.928828in}{7.880013in}}{\pgfqpoint{7.928828in}{7.891063in}}%
\pgfpathcurveto{\pgfqpoint{7.928828in}{7.902113in}}{\pgfqpoint{7.924438in}{7.912712in}}{\pgfqpoint{7.916625in}{7.920526in}}%
\pgfpathcurveto{\pgfqpoint{7.908811in}{7.928340in}}{\pgfqpoint{7.898212in}{7.932730in}}{\pgfqpoint{7.887162in}{7.932730in}}%
\pgfpathcurveto{\pgfqpoint{7.876112in}{7.932730in}}{\pgfqpoint{7.865513in}{7.928340in}}{\pgfqpoint{7.857699in}{7.920526in}}%
\pgfpathcurveto{\pgfqpoint{7.849885in}{7.912712in}}{\pgfqpoint{7.845495in}{7.902113in}}{\pgfqpoint{7.845495in}{7.891063in}}%
\pgfpathcurveto{\pgfqpoint{7.845495in}{7.880013in}}{\pgfqpoint{7.849885in}{7.869414in}}{\pgfqpoint{7.857699in}{7.861600in}}%
\pgfpathcurveto{\pgfqpoint{7.865513in}{7.853787in}}{\pgfqpoint{7.876112in}{7.849396in}}{\pgfqpoint{7.887162in}{7.849396in}}%
\pgfpathlineto{\pgfqpoint{7.887162in}{7.849396in}}%
\pgfpathclose%
\pgfusepath{stroke,fill}%
\end{pgfscope}%
\begin{pgfscope}%
\pgfpathrectangle{\pgfqpoint{7.622482in}{7.624184in}}{\pgfqpoint{2.177280in}{2.201755in}}%
\pgfusepath{clip}%
\pgfsetbuttcap%
\pgfsetroundjoin%
\definecolor{currentfill}{rgb}{0.121569,0.466667,0.705882}%
\pgfsetfillcolor{currentfill}%
\pgfsetlinewidth{0.481800pt}%
\definecolor{currentstroke}{rgb}{1.000000,1.000000,1.000000}%
\pgfsetstrokecolor{currentstroke}%
\pgfsetdash{}{0pt}%
\pgfpathmoveto{\pgfqpoint{7.887162in}{8.238596in}}%
\pgfpathcurveto{\pgfqpoint{7.898212in}{8.238596in}}{\pgfqpoint{7.908811in}{8.242986in}}{\pgfqpoint{7.916625in}{8.250799in}}%
\pgfpathcurveto{\pgfqpoint{7.924438in}{8.258613in}}{\pgfqpoint{7.928828in}{8.269212in}}{\pgfqpoint{7.928828in}{8.280262in}}%
\pgfpathcurveto{\pgfqpoint{7.928828in}{8.291312in}}{\pgfqpoint{7.924438in}{8.301911in}}{\pgfqpoint{7.916625in}{8.309725in}}%
\pgfpathcurveto{\pgfqpoint{7.908811in}{8.317539in}}{\pgfqpoint{7.898212in}{8.321929in}}{\pgfqpoint{7.887162in}{8.321929in}}%
\pgfpathcurveto{\pgfqpoint{7.876112in}{8.321929in}}{\pgfqpoint{7.865513in}{8.317539in}}{\pgfqpoint{7.857699in}{8.309725in}}%
\pgfpathcurveto{\pgfqpoint{7.849885in}{8.301911in}}{\pgfqpoint{7.845495in}{8.291312in}}{\pgfqpoint{7.845495in}{8.280262in}}%
\pgfpathcurveto{\pgfqpoint{7.845495in}{8.269212in}}{\pgfqpoint{7.849885in}{8.258613in}}{\pgfqpoint{7.857699in}{8.250799in}}%
\pgfpathcurveto{\pgfqpoint{7.865513in}{8.242986in}}{\pgfqpoint{7.876112in}{8.238596in}}{\pgfqpoint{7.887162in}{8.238596in}}%
\pgfpathlineto{\pgfqpoint{7.887162in}{8.238596in}}%
\pgfpathclose%
\pgfusepath{stroke,fill}%
\end{pgfscope}%
\begin{pgfscope}%
\pgfpathrectangle{\pgfqpoint{7.622482in}{7.624184in}}{\pgfqpoint{2.177280in}{2.201755in}}%
\pgfusepath{clip}%
\pgfsetbuttcap%
\pgfsetroundjoin%
\definecolor{currentfill}{rgb}{0.121569,0.466667,0.705882}%
\pgfsetfillcolor{currentfill}%
\pgfsetlinewidth{0.481800pt}%
\definecolor{currentstroke}{rgb}{1.000000,1.000000,1.000000}%
\pgfsetstrokecolor{currentstroke}%
\pgfsetdash{}{0pt}%
\pgfpathmoveto{\pgfqpoint{7.887162in}{8.071796in}}%
\pgfpathcurveto{\pgfqpoint{7.898212in}{8.071796in}}{\pgfqpoint{7.908811in}{8.076186in}}{\pgfqpoint{7.916625in}{8.084000in}}%
\pgfpathcurveto{\pgfqpoint{7.924438in}{8.091813in}}{\pgfqpoint{7.928828in}{8.102413in}}{\pgfqpoint{7.928828in}{8.113463in}}%
\pgfpathcurveto{\pgfqpoint{7.928828in}{8.124513in}}{\pgfqpoint{7.924438in}{8.135112in}}{\pgfqpoint{7.916625in}{8.142925in}}%
\pgfpathcurveto{\pgfqpoint{7.908811in}{8.150739in}}{\pgfqpoint{7.898212in}{8.155129in}}{\pgfqpoint{7.887162in}{8.155129in}}%
\pgfpathcurveto{\pgfqpoint{7.876112in}{8.155129in}}{\pgfqpoint{7.865513in}{8.150739in}}{\pgfqpoint{7.857699in}{8.142925in}}%
\pgfpathcurveto{\pgfqpoint{7.849885in}{8.135112in}}{\pgfqpoint{7.845495in}{8.124513in}}{\pgfqpoint{7.845495in}{8.113463in}}%
\pgfpathcurveto{\pgfqpoint{7.845495in}{8.102413in}}{\pgfqpoint{7.849885in}{8.091813in}}{\pgfqpoint{7.857699in}{8.084000in}}%
\pgfpathcurveto{\pgfqpoint{7.865513in}{8.076186in}}{\pgfqpoint{7.876112in}{8.071796in}}{\pgfqpoint{7.887162in}{8.071796in}}%
\pgfpathlineto{\pgfqpoint{7.887162in}{8.071796in}}%
\pgfpathclose%
\pgfusepath{stroke,fill}%
\end{pgfscope}%
\begin{pgfscope}%
\pgfpathrectangle{\pgfqpoint{7.622482in}{7.624184in}}{\pgfqpoint{2.177280in}{2.201755in}}%
\pgfusepath{clip}%
\pgfsetbuttcap%
\pgfsetroundjoin%
\definecolor{currentfill}{rgb}{1.000000,0.498039,0.054902}%
\pgfsetfillcolor{currentfill}%
\pgfsetlinewidth{0.481800pt}%
\definecolor{currentstroke}{rgb}{1.000000,1.000000,1.000000}%
\pgfsetstrokecolor{currentstroke}%
\pgfsetdash{}{0pt}%
\pgfpathmoveto{\pgfqpoint{8.700208in}{9.183793in}}%
\pgfpathcurveto{\pgfqpoint{8.711258in}{9.183793in}}{\pgfqpoint{8.721857in}{9.188184in}}{\pgfqpoint{8.729671in}{9.195997in}}%
\pgfpathcurveto{\pgfqpoint{8.737485in}{9.203811in}}{\pgfqpoint{8.741875in}{9.214410in}}{\pgfqpoint{8.741875in}{9.225460in}}%
\pgfpathcurveto{\pgfqpoint{8.741875in}{9.236510in}}{\pgfqpoint{8.737485in}{9.247109in}}{\pgfqpoint{8.729671in}{9.254923in}}%
\pgfpathcurveto{\pgfqpoint{8.721857in}{9.262737in}}{\pgfqpoint{8.711258in}{9.267127in}}{\pgfqpoint{8.700208in}{9.267127in}}%
\pgfpathcurveto{\pgfqpoint{8.689158in}{9.267127in}}{\pgfqpoint{8.678559in}{9.262737in}}{\pgfqpoint{8.670745in}{9.254923in}}%
\pgfpathcurveto{\pgfqpoint{8.662932in}{9.247109in}}{\pgfqpoint{8.658542in}{9.236510in}}{\pgfqpoint{8.658542in}{9.225460in}}%
\pgfpathcurveto{\pgfqpoint{8.658542in}{9.214410in}}{\pgfqpoint{8.662932in}{9.203811in}}{\pgfqpoint{8.670745in}{9.195997in}}%
\pgfpathcurveto{\pgfqpoint{8.678559in}{9.188184in}}{\pgfqpoint{8.689158in}{9.183793in}}{\pgfqpoint{8.700208in}{9.183793in}}%
\pgfpathlineto{\pgfqpoint{8.700208in}{9.183793in}}%
\pgfpathclose%
\pgfusepath{stroke,fill}%
\end{pgfscope}%
\begin{pgfscope}%
\pgfpathrectangle{\pgfqpoint{7.622482in}{7.624184in}}{\pgfqpoint{2.177280in}{2.201755in}}%
\pgfusepath{clip}%
\pgfsetbuttcap%
\pgfsetroundjoin%
\definecolor{currentfill}{rgb}{1.000000,0.498039,0.054902}%
\pgfsetfillcolor{currentfill}%
\pgfsetlinewidth{0.481800pt}%
\definecolor{currentstroke}{rgb}{1.000000,1.000000,1.000000}%
\pgfsetstrokecolor{currentstroke}%
\pgfsetdash{}{0pt}%
\pgfpathmoveto{\pgfqpoint{8.767962in}{8.850194in}}%
\pgfpathcurveto{\pgfqpoint{8.779012in}{8.850194in}}{\pgfqpoint{8.789611in}{8.854584in}}{\pgfqpoint{8.797425in}{8.862398in}}%
\pgfpathcurveto{\pgfqpoint{8.805238in}{8.870212in}}{\pgfqpoint{8.809629in}{8.880811in}}{\pgfqpoint{8.809629in}{8.891861in}}%
\pgfpathcurveto{\pgfqpoint{8.809629in}{8.902911in}}{\pgfqpoint{8.805238in}{8.913510in}}{\pgfqpoint{8.797425in}{8.921324in}}%
\pgfpathcurveto{\pgfqpoint{8.789611in}{8.929137in}}{\pgfqpoint{8.779012in}{8.933528in}}{\pgfqpoint{8.767962in}{8.933528in}}%
\pgfpathcurveto{\pgfqpoint{8.756912in}{8.933528in}}{\pgfqpoint{8.746313in}{8.929137in}}{\pgfqpoint{8.738499in}{8.921324in}}%
\pgfpathcurveto{\pgfqpoint{8.730686in}{8.913510in}}{\pgfqpoint{8.726295in}{8.902911in}}{\pgfqpoint{8.726295in}{8.891861in}}%
\pgfpathcurveto{\pgfqpoint{8.726295in}{8.880811in}}{\pgfqpoint{8.730686in}{8.870212in}}{\pgfqpoint{8.738499in}{8.862398in}}%
\pgfpathcurveto{\pgfqpoint{8.746313in}{8.854584in}}{\pgfqpoint{8.756912in}{8.850194in}}{\pgfqpoint{8.767962in}{8.850194in}}%
\pgfpathlineto{\pgfqpoint{8.767962in}{8.850194in}}%
\pgfpathclose%
\pgfusepath{stroke,fill}%
\end{pgfscope}%
\begin{pgfscope}%
\pgfpathrectangle{\pgfqpoint{7.622482in}{7.624184in}}{\pgfqpoint{2.177280in}{2.201755in}}%
\pgfusepath{clip}%
\pgfsetbuttcap%
\pgfsetroundjoin%
\definecolor{currentfill}{rgb}{1.000000,0.498039,0.054902}%
\pgfsetfillcolor{currentfill}%
\pgfsetlinewidth{0.481800pt}%
\definecolor{currentstroke}{rgb}{1.000000,1.000000,1.000000}%
\pgfsetstrokecolor{currentstroke}%
\pgfsetdash{}{0pt}%
\pgfpathmoveto{\pgfqpoint{8.767962in}{9.128194in}}%
\pgfpathcurveto{\pgfqpoint{8.779012in}{9.128194in}}{\pgfqpoint{8.789611in}{9.132584in}}{\pgfqpoint{8.797425in}{9.140397in}}%
\pgfpathcurveto{\pgfqpoint{8.805238in}{9.148211in}}{\pgfqpoint{8.809629in}{9.158810in}}{\pgfqpoint{8.809629in}{9.169860in}}%
\pgfpathcurveto{\pgfqpoint{8.809629in}{9.180910in}}{\pgfqpoint{8.805238in}{9.191509in}}{\pgfqpoint{8.797425in}{9.199323in}}%
\pgfpathcurveto{\pgfqpoint{8.789611in}{9.207137in}}{\pgfqpoint{8.779012in}{9.211527in}}{\pgfqpoint{8.767962in}{9.211527in}}%
\pgfpathcurveto{\pgfqpoint{8.756912in}{9.211527in}}{\pgfqpoint{8.746313in}{9.207137in}}{\pgfqpoint{8.738499in}{9.199323in}}%
\pgfpathcurveto{\pgfqpoint{8.730686in}{9.191509in}}{\pgfqpoint{8.726295in}{9.180910in}}{\pgfqpoint{8.726295in}{9.169860in}}%
\pgfpathcurveto{\pgfqpoint{8.726295in}{9.158810in}}{\pgfqpoint{8.730686in}{9.148211in}}{\pgfqpoint{8.738499in}{9.140397in}}%
\pgfpathcurveto{\pgfqpoint{8.746313in}{9.132584in}}{\pgfqpoint{8.756912in}{9.128194in}}{\pgfqpoint{8.767962in}{9.128194in}}%
\pgfpathlineto{\pgfqpoint{8.767962in}{9.128194in}}%
\pgfpathclose%
\pgfusepath{stroke,fill}%
\end{pgfscope}%
\begin{pgfscope}%
\pgfpathrectangle{\pgfqpoint{7.622482in}{7.624184in}}{\pgfqpoint{2.177280in}{2.201755in}}%
\pgfusepath{clip}%
\pgfsetbuttcap%
\pgfsetroundjoin%
\definecolor{currentfill}{rgb}{1.000000,0.498039,0.054902}%
\pgfsetfillcolor{currentfill}%
\pgfsetlinewidth{0.481800pt}%
\definecolor{currentstroke}{rgb}{1.000000,1.000000,1.000000}%
\pgfsetstrokecolor{currentstroke}%
\pgfsetdash{}{0pt}%
\pgfpathmoveto{\pgfqpoint{8.632454in}{8.349795in}}%
\pgfpathcurveto{\pgfqpoint{8.643504in}{8.349795in}}{\pgfqpoint{8.654104in}{8.354186in}}{\pgfqpoint{8.661917in}{8.361999in}}%
\pgfpathcurveto{\pgfqpoint{8.669731in}{8.369813in}}{\pgfqpoint{8.674121in}{8.380412in}}{\pgfqpoint{8.674121in}{8.391462in}}%
\pgfpathcurveto{\pgfqpoint{8.674121in}{8.402512in}}{\pgfqpoint{8.669731in}{8.413111in}}{\pgfqpoint{8.661917in}{8.420925in}}%
\pgfpathcurveto{\pgfqpoint{8.654104in}{8.428738in}}{\pgfqpoint{8.643504in}{8.433129in}}{\pgfqpoint{8.632454in}{8.433129in}}%
\pgfpathcurveto{\pgfqpoint{8.621404in}{8.433129in}}{\pgfqpoint{8.610805in}{8.428738in}}{\pgfqpoint{8.602992in}{8.420925in}}%
\pgfpathcurveto{\pgfqpoint{8.595178in}{8.413111in}}{\pgfqpoint{8.590788in}{8.402512in}}{\pgfqpoint{8.590788in}{8.391462in}}%
\pgfpathcurveto{\pgfqpoint{8.590788in}{8.380412in}}{\pgfqpoint{8.595178in}{8.369813in}}{\pgfqpoint{8.602992in}{8.361999in}}%
\pgfpathcurveto{\pgfqpoint{8.610805in}{8.354186in}}{\pgfqpoint{8.621404in}{8.349795in}}{\pgfqpoint{8.632454in}{8.349795in}}%
\pgfpathlineto{\pgfqpoint{8.632454in}{8.349795in}}%
\pgfpathclose%
\pgfusepath{stroke,fill}%
\end{pgfscope}%
\begin{pgfscope}%
\pgfpathrectangle{\pgfqpoint{7.622482in}{7.624184in}}{\pgfqpoint{2.177280in}{2.201755in}}%
\pgfusepath{clip}%
\pgfsetbuttcap%
\pgfsetroundjoin%
\definecolor{currentfill}{rgb}{1.000000,0.498039,0.054902}%
\pgfsetfillcolor{currentfill}%
\pgfsetlinewidth{0.481800pt}%
\definecolor{currentstroke}{rgb}{1.000000,1.000000,1.000000}%
\pgfsetstrokecolor{currentstroke}%
\pgfsetdash{}{0pt}%
\pgfpathmoveto{\pgfqpoint{8.767962in}{8.905794in}}%
\pgfpathcurveto{\pgfqpoint{8.779012in}{8.905794in}}{\pgfqpoint{8.789611in}{8.910184in}}{\pgfqpoint{8.797425in}{8.917998in}}%
\pgfpathcurveto{\pgfqpoint{8.805238in}{8.925812in}}{\pgfqpoint{8.809629in}{8.936411in}}{\pgfqpoint{8.809629in}{8.947461in}}%
\pgfpathcurveto{\pgfqpoint{8.809629in}{8.958511in}}{\pgfqpoint{8.805238in}{8.969110in}}{\pgfqpoint{8.797425in}{8.976924in}}%
\pgfpathcurveto{\pgfqpoint{8.789611in}{8.984737in}}{\pgfqpoint{8.779012in}{8.989127in}}{\pgfqpoint{8.767962in}{8.989127in}}%
\pgfpathcurveto{\pgfqpoint{8.756912in}{8.989127in}}{\pgfqpoint{8.746313in}{8.984737in}}{\pgfqpoint{8.738499in}{8.976924in}}%
\pgfpathcurveto{\pgfqpoint{8.730686in}{8.969110in}}{\pgfqpoint{8.726295in}{8.958511in}}{\pgfqpoint{8.726295in}{8.947461in}}%
\pgfpathcurveto{\pgfqpoint{8.726295in}{8.936411in}}{\pgfqpoint{8.730686in}{8.925812in}}{\pgfqpoint{8.738499in}{8.917998in}}%
\pgfpathcurveto{\pgfqpoint{8.746313in}{8.910184in}}{\pgfqpoint{8.756912in}{8.905794in}}{\pgfqpoint{8.767962in}{8.905794in}}%
\pgfpathlineto{\pgfqpoint{8.767962in}{8.905794in}}%
\pgfpathclose%
\pgfusepath{stroke,fill}%
\end{pgfscope}%
\begin{pgfscope}%
\pgfpathrectangle{\pgfqpoint{7.622482in}{7.624184in}}{\pgfqpoint{2.177280in}{2.201755in}}%
\pgfusepath{clip}%
\pgfsetbuttcap%
\pgfsetroundjoin%
\definecolor{currentfill}{rgb}{1.000000,0.498039,0.054902}%
\pgfsetfillcolor{currentfill}%
\pgfsetlinewidth{0.481800pt}%
\definecolor{currentstroke}{rgb}{1.000000,1.000000,1.000000}%
\pgfsetstrokecolor{currentstroke}%
\pgfsetdash{}{0pt}%
\pgfpathmoveto{\pgfqpoint{8.632454in}{8.460995in}}%
\pgfpathcurveto{\pgfqpoint{8.643504in}{8.460995in}}{\pgfqpoint{8.654104in}{8.465385in}}{\pgfqpoint{8.661917in}{8.473199in}}%
\pgfpathcurveto{\pgfqpoint{8.669731in}{8.481013in}}{\pgfqpoint{8.674121in}{8.491612in}}{\pgfqpoint{8.674121in}{8.502662in}}%
\pgfpathcurveto{\pgfqpoint{8.674121in}{8.513712in}}{\pgfqpoint{8.669731in}{8.524311in}}{\pgfqpoint{8.661917in}{8.532125in}}%
\pgfpathcurveto{\pgfqpoint{8.654104in}{8.539938in}}{\pgfqpoint{8.643504in}{8.544328in}}{\pgfqpoint{8.632454in}{8.544328in}}%
\pgfpathcurveto{\pgfqpoint{8.621404in}{8.544328in}}{\pgfqpoint{8.610805in}{8.539938in}}{\pgfqpoint{8.602992in}{8.532125in}}%
\pgfpathcurveto{\pgfqpoint{8.595178in}{8.524311in}}{\pgfqpoint{8.590788in}{8.513712in}}{\pgfqpoint{8.590788in}{8.502662in}}%
\pgfpathcurveto{\pgfqpoint{8.590788in}{8.491612in}}{\pgfqpoint{8.595178in}{8.481013in}}{\pgfqpoint{8.602992in}{8.473199in}}%
\pgfpathcurveto{\pgfqpoint{8.610805in}{8.465385in}}{\pgfqpoint{8.621404in}{8.460995in}}{\pgfqpoint{8.632454in}{8.460995in}}%
\pgfpathlineto{\pgfqpoint{8.632454in}{8.460995in}}%
\pgfpathclose%
\pgfusepath{stroke,fill}%
\end{pgfscope}%
\begin{pgfscope}%
\pgfpathrectangle{\pgfqpoint{7.622482in}{7.624184in}}{\pgfqpoint{2.177280in}{2.201755in}}%
\pgfusepath{clip}%
\pgfsetbuttcap%
\pgfsetroundjoin%
\definecolor{currentfill}{rgb}{1.000000,0.498039,0.054902}%
\pgfsetfillcolor{currentfill}%
\pgfsetlinewidth{0.481800pt}%
\definecolor{currentstroke}{rgb}{1.000000,1.000000,1.000000}%
\pgfsetstrokecolor{currentstroke}%
\pgfsetdash{}{0pt}%
\pgfpathmoveto{\pgfqpoint{8.835716in}{8.794594in}}%
\pgfpathcurveto{\pgfqpoint{8.846766in}{8.794594in}}{\pgfqpoint{8.857365in}{8.798985in}}{\pgfqpoint{8.865179in}{8.806798in}}%
\pgfpathcurveto{\pgfqpoint{8.872992in}{8.814612in}}{\pgfqpoint{8.877383in}{8.825211in}}{\pgfqpoint{8.877383in}{8.836261in}}%
\pgfpathcurveto{\pgfqpoint{8.877383in}{8.847311in}}{\pgfqpoint{8.872992in}{8.857910in}}{\pgfqpoint{8.865179in}{8.865724in}}%
\pgfpathcurveto{\pgfqpoint{8.857365in}{8.873537in}}{\pgfqpoint{8.846766in}{8.877928in}}{\pgfqpoint{8.835716in}{8.877928in}}%
\pgfpathcurveto{\pgfqpoint{8.824666in}{8.877928in}}{\pgfqpoint{8.814067in}{8.873537in}}{\pgfqpoint{8.806253in}{8.865724in}}%
\pgfpathcurveto{\pgfqpoint{8.798440in}{8.857910in}}{\pgfqpoint{8.794049in}{8.847311in}}{\pgfqpoint{8.794049in}{8.836261in}}%
\pgfpathcurveto{\pgfqpoint{8.794049in}{8.825211in}}{\pgfqpoint{8.798440in}{8.814612in}}{\pgfqpoint{8.806253in}{8.806798in}}%
\pgfpathcurveto{\pgfqpoint{8.814067in}{8.798985in}}{\pgfqpoint{8.824666in}{8.794594in}}{\pgfqpoint{8.835716in}{8.794594in}}%
\pgfpathlineto{\pgfqpoint{8.835716in}{8.794594in}}%
\pgfpathclose%
\pgfusepath{stroke,fill}%
\end{pgfscope}%
\begin{pgfscope}%
\pgfpathrectangle{\pgfqpoint{7.622482in}{7.624184in}}{\pgfqpoint{2.177280in}{2.201755in}}%
\pgfusepath{clip}%
\pgfsetbuttcap%
\pgfsetroundjoin%
\definecolor{currentfill}{rgb}{1.000000,0.498039,0.054902}%
\pgfsetfillcolor{currentfill}%
\pgfsetlinewidth{0.481800pt}%
\definecolor{currentstroke}{rgb}{1.000000,1.000000,1.000000}%
\pgfsetstrokecolor{currentstroke}%
\pgfsetdash{}{0pt}%
\pgfpathmoveto{\pgfqpoint{8.429193in}{8.016196in}}%
\pgfpathcurveto{\pgfqpoint{8.440243in}{8.016196in}}{\pgfqpoint{8.450842in}{8.020586in}}{\pgfqpoint{8.458656in}{8.028400in}}%
\pgfpathcurveto{\pgfqpoint{8.466469in}{8.036214in}}{\pgfqpoint{8.470859in}{8.046813in}}{\pgfqpoint{8.470859in}{8.057863in}}%
\pgfpathcurveto{\pgfqpoint{8.470859in}{8.068913in}}{\pgfqpoint{8.466469in}{8.079512in}}{\pgfqpoint{8.458656in}{8.087326in}}%
\pgfpathcurveto{\pgfqpoint{8.450842in}{8.095139in}}{\pgfqpoint{8.440243in}{8.099529in}}{\pgfqpoint{8.429193in}{8.099529in}}%
\pgfpathcurveto{\pgfqpoint{8.418143in}{8.099529in}}{\pgfqpoint{8.407544in}{8.095139in}}{\pgfqpoint{8.399730in}{8.087326in}}%
\pgfpathcurveto{\pgfqpoint{8.391916in}{8.079512in}}{\pgfqpoint{8.387526in}{8.068913in}}{\pgfqpoint{8.387526in}{8.057863in}}%
\pgfpathcurveto{\pgfqpoint{8.387526in}{8.046813in}}{\pgfqpoint{8.391916in}{8.036214in}}{\pgfqpoint{8.399730in}{8.028400in}}%
\pgfpathcurveto{\pgfqpoint{8.407544in}{8.020586in}}{\pgfqpoint{8.418143in}{8.016196in}}{\pgfqpoint{8.429193in}{8.016196in}}%
\pgfpathlineto{\pgfqpoint{8.429193in}{8.016196in}}%
\pgfpathclose%
\pgfusepath{stroke,fill}%
\end{pgfscope}%
\begin{pgfscope}%
\pgfpathrectangle{\pgfqpoint{7.622482in}{7.624184in}}{\pgfqpoint{2.177280in}{2.201755in}}%
\pgfusepath{clip}%
\pgfsetbuttcap%
\pgfsetroundjoin%
\definecolor{currentfill}{rgb}{1.000000,0.498039,0.054902}%
\pgfsetfillcolor{currentfill}%
\pgfsetlinewidth{0.481800pt}%
\definecolor{currentstroke}{rgb}{1.000000,1.000000,1.000000}%
\pgfsetstrokecolor{currentstroke}%
\pgfsetdash{}{0pt}%
\pgfpathmoveto{\pgfqpoint{8.632454in}{8.961394in}}%
\pgfpathcurveto{\pgfqpoint{8.643504in}{8.961394in}}{\pgfqpoint{8.654104in}{8.965784in}}{\pgfqpoint{8.661917in}{8.973598in}}%
\pgfpathcurveto{\pgfqpoint{8.669731in}{8.981411in}}{\pgfqpoint{8.674121in}{8.992011in}}{\pgfqpoint{8.674121in}{9.003061in}}%
\pgfpathcurveto{\pgfqpoint{8.674121in}{9.014111in}}{\pgfqpoint{8.669731in}{9.024710in}}{\pgfqpoint{8.661917in}{9.032523in}}%
\pgfpathcurveto{\pgfqpoint{8.654104in}{9.040337in}}{\pgfqpoint{8.643504in}{9.044727in}}{\pgfqpoint{8.632454in}{9.044727in}}%
\pgfpathcurveto{\pgfqpoint{8.621404in}{9.044727in}}{\pgfqpoint{8.610805in}{9.040337in}}{\pgfqpoint{8.602992in}{9.032523in}}%
\pgfpathcurveto{\pgfqpoint{8.595178in}{9.024710in}}{\pgfqpoint{8.590788in}{9.014111in}}{\pgfqpoint{8.590788in}{9.003061in}}%
\pgfpathcurveto{\pgfqpoint{8.590788in}{8.992011in}}{\pgfqpoint{8.595178in}{8.981411in}}{\pgfqpoint{8.602992in}{8.973598in}}%
\pgfpathcurveto{\pgfqpoint{8.610805in}{8.965784in}}{\pgfqpoint{8.621404in}{8.961394in}}{\pgfqpoint{8.632454in}{8.961394in}}%
\pgfpathlineto{\pgfqpoint{8.632454in}{8.961394in}}%
\pgfpathclose%
\pgfusepath{stroke,fill}%
\end{pgfscope}%
\begin{pgfscope}%
\pgfpathrectangle{\pgfqpoint{7.622482in}{7.624184in}}{\pgfqpoint{2.177280in}{2.201755in}}%
\pgfusepath{clip}%
\pgfsetbuttcap%
\pgfsetroundjoin%
\definecolor{currentfill}{rgb}{1.000000,0.498039,0.054902}%
\pgfsetfillcolor{currentfill}%
\pgfsetlinewidth{0.481800pt}%
\definecolor{currentstroke}{rgb}{1.000000,1.000000,1.000000}%
\pgfsetstrokecolor{currentstroke}%
\pgfsetdash{}{0pt}%
\pgfpathmoveto{\pgfqpoint{8.700208in}{8.182996in}}%
\pgfpathcurveto{\pgfqpoint{8.711258in}{8.182996in}}{\pgfqpoint{8.721857in}{8.187386in}}{\pgfqpoint{8.729671in}{8.195200in}}%
\pgfpathcurveto{\pgfqpoint{8.737485in}{8.203013in}}{\pgfqpoint{8.741875in}{8.213612in}}{\pgfqpoint{8.741875in}{8.224662in}}%
\pgfpathcurveto{\pgfqpoint{8.741875in}{8.235713in}}{\pgfqpoint{8.737485in}{8.246312in}}{\pgfqpoint{8.729671in}{8.254125in}}%
\pgfpathcurveto{\pgfqpoint{8.721857in}{8.261939in}}{\pgfqpoint{8.711258in}{8.266329in}}{\pgfqpoint{8.700208in}{8.266329in}}%
\pgfpathcurveto{\pgfqpoint{8.689158in}{8.266329in}}{\pgfqpoint{8.678559in}{8.261939in}}{\pgfqpoint{8.670745in}{8.254125in}}%
\pgfpathcurveto{\pgfqpoint{8.662932in}{8.246312in}}{\pgfqpoint{8.658542in}{8.235713in}}{\pgfqpoint{8.658542in}{8.224662in}}%
\pgfpathcurveto{\pgfqpoint{8.658542in}{8.213612in}}{\pgfqpoint{8.662932in}{8.203013in}}{\pgfqpoint{8.670745in}{8.195200in}}%
\pgfpathcurveto{\pgfqpoint{8.678559in}{8.187386in}}{\pgfqpoint{8.689158in}{8.182996in}}{\pgfqpoint{8.700208in}{8.182996in}}%
\pgfpathlineto{\pgfqpoint{8.700208in}{8.182996in}}%
\pgfpathclose%
\pgfusepath{stroke,fill}%
\end{pgfscope}%
\begin{pgfscope}%
\pgfpathrectangle{\pgfqpoint{7.622482in}{7.624184in}}{\pgfqpoint{2.177280in}{2.201755in}}%
\pgfusepath{clip}%
\pgfsetbuttcap%
\pgfsetroundjoin%
\definecolor{currentfill}{rgb}{1.000000,0.498039,0.054902}%
\pgfsetfillcolor{currentfill}%
\pgfsetlinewidth{0.481800pt}%
\definecolor{currentstroke}{rgb}{1.000000,1.000000,1.000000}%
\pgfsetstrokecolor{currentstroke}%
\pgfsetdash{}{0pt}%
\pgfpathmoveto{\pgfqpoint{8.429193in}{8.071796in}}%
\pgfpathcurveto{\pgfqpoint{8.440243in}{8.071796in}}{\pgfqpoint{8.450842in}{8.076186in}}{\pgfqpoint{8.458656in}{8.084000in}}%
\pgfpathcurveto{\pgfqpoint{8.466469in}{8.091813in}}{\pgfqpoint{8.470859in}{8.102413in}}{\pgfqpoint{8.470859in}{8.113463in}}%
\pgfpathcurveto{\pgfqpoint{8.470859in}{8.124513in}}{\pgfqpoint{8.466469in}{8.135112in}}{\pgfqpoint{8.458656in}{8.142925in}}%
\pgfpathcurveto{\pgfqpoint{8.450842in}{8.150739in}}{\pgfqpoint{8.440243in}{8.155129in}}{\pgfqpoint{8.429193in}{8.155129in}}%
\pgfpathcurveto{\pgfqpoint{8.418143in}{8.155129in}}{\pgfqpoint{8.407544in}{8.150739in}}{\pgfqpoint{8.399730in}{8.142925in}}%
\pgfpathcurveto{\pgfqpoint{8.391916in}{8.135112in}}{\pgfqpoint{8.387526in}{8.124513in}}{\pgfqpoint{8.387526in}{8.113463in}}%
\pgfpathcurveto{\pgfqpoint{8.387526in}{8.102413in}}{\pgfqpoint{8.391916in}{8.091813in}}{\pgfqpoint{8.399730in}{8.084000in}}%
\pgfpathcurveto{\pgfqpoint{8.407544in}{8.076186in}}{\pgfqpoint{8.418143in}{8.071796in}}{\pgfqpoint{8.429193in}{8.071796in}}%
\pgfpathlineto{\pgfqpoint{8.429193in}{8.071796in}}%
\pgfpathclose%
\pgfusepath{stroke,fill}%
\end{pgfscope}%
\begin{pgfscope}%
\pgfpathrectangle{\pgfqpoint{7.622482in}{7.624184in}}{\pgfqpoint{2.177280in}{2.201755in}}%
\pgfusepath{clip}%
\pgfsetbuttcap%
\pgfsetroundjoin%
\definecolor{currentfill}{rgb}{1.000000,0.498039,0.054902}%
\pgfsetfillcolor{currentfill}%
\pgfsetlinewidth{0.481800pt}%
\definecolor{currentstroke}{rgb}{1.000000,1.000000,1.000000}%
\pgfsetstrokecolor{currentstroke}%
\pgfsetdash{}{0pt}%
\pgfpathmoveto{\pgfqpoint{8.767962in}{8.572195in}}%
\pgfpathcurveto{\pgfqpoint{8.779012in}{8.572195in}}{\pgfqpoint{8.789611in}{8.576585in}}{\pgfqpoint{8.797425in}{8.584399in}}%
\pgfpathcurveto{\pgfqpoint{8.805238in}{8.592212in}}{\pgfqpoint{8.809629in}{8.602811in}}{\pgfqpoint{8.809629in}{8.613862in}}%
\pgfpathcurveto{\pgfqpoint{8.809629in}{8.624912in}}{\pgfqpoint{8.805238in}{8.635511in}}{\pgfqpoint{8.797425in}{8.643324in}}%
\pgfpathcurveto{\pgfqpoint{8.789611in}{8.651138in}}{\pgfqpoint{8.779012in}{8.655528in}}{\pgfqpoint{8.767962in}{8.655528in}}%
\pgfpathcurveto{\pgfqpoint{8.756912in}{8.655528in}}{\pgfqpoint{8.746313in}{8.651138in}}{\pgfqpoint{8.738499in}{8.643324in}}%
\pgfpathcurveto{\pgfqpoint{8.730686in}{8.635511in}}{\pgfqpoint{8.726295in}{8.624912in}}{\pgfqpoint{8.726295in}{8.613862in}}%
\pgfpathcurveto{\pgfqpoint{8.726295in}{8.602811in}}{\pgfqpoint{8.730686in}{8.592212in}}{\pgfqpoint{8.738499in}{8.584399in}}%
\pgfpathcurveto{\pgfqpoint{8.746313in}{8.576585in}}{\pgfqpoint{8.756912in}{8.572195in}}{\pgfqpoint{8.767962in}{8.572195in}}%
\pgfpathlineto{\pgfqpoint{8.767962in}{8.572195in}}%
\pgfpathclose%
\pgfusepath{stroke,fill}%
\end{pgfscope}%
\begin{pgfscope}%
\pgfpathrectangle{\pgfqpoint{7.622482in}{7.624184in}}{\pgfqpoint{2.177280in}{2.201755in}}%
\pgfusepath{clip}%
\pgfsetbuttcap%
\pgfsetroundjoin%
\definecolor{currentfill}{rgb}{1.000000,0.498039,0.054902}%
\pgfsetfillcolor{currentfill}%
\pgfsetlinewidth{0.481800pt}%
\definecolor{currentstroke}{rgb}{1.000000,1.000000,1.000000}%
\pgfsetstrokecolor{currentstroke}%
\pgfsetdash{}{0pt}%
\pgfpathmoveto{\pgfqpoint{8.429193in}{8.627795in}}%
\pgfpathcurveto{\pgfqpoint{8.440243in}{8.627795in}}{\pgfqpoint{8.450842in}{8.632185in}}{\pgfqpoint{8.458656in}{8.639999in}}%
\pgfpathcurveto{\pgfqpoint{8.466469in}{8.647812in}}{\pgfqpoint{8.470859in}{8.658411in}}{\pgfqpoint{8.470859in}{8.669461in}}%
\pgfpathcurveto{\pgfqpoint{8.470859in}{8.680512in}}{\pgfqpoint{8.466469in}{8.691111in}}{\pgfqpoint{8.458656in}{8.698924in}}%
\pgfpathcurveto{\pgfqpoint{8.450842in}{8.706738in}}{\pgfqpoint{8.440243in}{8.711128in}}{\pgfqpoint{8.429193in}{8.711128in}}%
\pgfpathcurveto{\pgfqpoint{8.418143in}{8.711128in}}{\pgfqpoint{8.407544in}{8.706738in}}{\pgfqpoint{8.399730in}{8.698924in}}%
\pgfpathcurveto{\pgfqpoint{8.391916in}{8.691111in}}{\pgfqpoint{8.387526in}{8.680512in}}{\pgfqpoint{8.387526in}{8.669461in}}%
\pgfpathcurveto{\pgfqpoint{8.387526in}{8.658411in}}{\pgfqpoint{8.391916in}{8.647812in}}{\pgfqpoint{8.399730in}{8.639999in}}%
\pgfpathcurveto{\pgfqpoint{8.407544in}{8.632185in}}{\pgfqpoint{8.418143in}{8.627795in}}{\pgfqpoint{8.429193in}{8.627795in}}%
\pgfpathlineto{\pgfqpoint{8.429193in}{8.627795in}}%
\pgfpathclose%
\pgfusepath{stroke,fill}%
\end{pgfscope}%
\begin{pgfscope}%
\pgfpathrectangle{\pgfqpoint{7.622482in}{7.624184in}}{\pgfqpoint{2.177280in}{2.201755in}}%
\pgfusepath{clip}%
\pgfsetbuttcap%
\pgfsetroundjoin%
\definecolor{currentfill}{rgb}{1.000000,0.498039,0.054902}%
\pgfsetfillcolor{currentfill}%
\pgfsetlinewidth{0.481800pt}%
\definecolor{currentstroke}{rgb}{1.000000,1.000000,1.000000}%
\pgfsetstrokecolor{currentstroke}%
\pgfsetdash{}{0pt}%
\pgfpathmoveto{\pgfqpoint{8.700208in}{8.683395in}}%
\pgfpathcurveto{\pgfqpoint{8.711258in}{8.683395in}}{\pgfqpoint{8.721857in}{8.687785in}}{\pgfqpoint{8.729671in}{8.695598in}}%
\pgfpathcurveto{\pgfqpoint{8.737485in}{8.703412in}}{\pgfqpoint{8.741875in}{8.714011in}}{\pgfqpoint{8.741875in}{8.725061in}}%
\pgfpathcurveto{\pgfqpoint{8.741875in}{8.736111in}}{\pgfqpoint{8.737485in}{8.746710in}}{\pgfqpoint{8.729671in}{8.754524in}}%
\pgfpathcurveto{\pgfqpoint{8.721857in}{8.762338in}}{\pgfqpoint{8.711258in}{8.766728in}}{\pgfqpoint{8.700208in}{8.766728in}}%
\pgfpathcurveto{\pgfqpoint{8.689158in}{8.766728in}}{\pgfqpoint{8.678559in}{8.762338in}}{\pgfqpoint{8.670745in}{8.754524in}}%
\pgfpathcurveto{\pgfqpoint{8.662932in}{8.746710in}}{\pgfqpoint{8.658542in}{8.736111in}}{\pgfqpoint{8.658542in}{8.725061in}}%
\pgfpathcurveto{\pgfqpoint{8.658542in}{8.714011in}}{\pgfqpoint{8.662932in}{8.703412in}}{\pgfqpoint{8.670745in}{8.695598in}}%
\pgfpathcurveto{\pgfqpoint{8.678559in}{8.687785in}}{\pgfqpoint{8.689158in}{8.683395in}}{\pgfqpoint{8.700208in}{8.683395in}}%
\pgfpathlineto{\pgfqpoint{8.700208in}{8.683395in}}%
\pgfpathclose%
\pgfusepath{stroke,fill}%
\end{pgfscope}%
\begin{pgfscope}%
\pgfpathrectangle{\pgfqpoint{7.622482in}{7.624184in}}{\pgfqpoint{2.177280in}{2.201755in}}%
\pgfusepath{clip}%
\pgfsetbuttcap%
\pgfsetroundjoin%
\definecolor{currentfill}{rgb}{1.000000,0.498039,0.054902}%
\pgfsetfillcolor{currentfill}%
\pgfsetlinewidth{0.481800pt}%
\definecolor{currentstroke}{rgb}{1.000000,1.000000,1.000000}%
\pgfsetstrokecolor{currentstroke}%
\pgfsetdash{}{0pt}%
\pgfpathmoveto{\pgfqpoint{8.632454in}{8.405395in}}%
\pgfpathcurveto{\pgfqpoint{8.643504in}{8.405395in}}{\pgfqpoint{8.654104in}{8.409785in}}{\pgfqpoint{8.661917in}{8.417599in}}%
\pgfpathcurveto{\pgfqpoint{8.669731in}{8.425413in}}{\pgfqpoint{8.674121in}{8.436012in}}{\pgfqpoint{8.674121in}{8.447062in}}%
\pgfpathcurveto{\pgfqpoint{8.674121in}{8.458112in}}{\pgfqpoint{8.669731in}{8.468711in}}{\pgfqpoint{8.661917in}{8.476525in}}%
\pgfpathcurveto{\pgfqpoint{8.654104in}{8.484338in}}{\pgfqpoint{8.643504in}{8.488729in}}{\pgfqpoint{8.632454in}{8.488729in}}%
\pgfpathcurveto{\pgfqpoint{8.621404in}{8.488729in}}{\pgfqpoint{8.610805in}{8.484338in}}{\pgfqpoint{8.602992in}{8.476525in}}%
\pgfpathcurveto{\pgfqpoint{8.595178in}{8.468711in}}{\pgfqpoint{8.590788in}{8.458112in}}{\pgfqpoint{8.590788in}{8.447062in}}%
\pgfpathcurveto{\pgfqpoint{8.590788in}{8.436012in}}{\pgfqpoint{8.595178in}{8.425413in}}{\pgfqpoint{8.602992in}{8.417599in}}%
\pgfpathcurveto{\pgfqpoint{8.610805in}{8.409785in}}{\pgfqpoint{8.621404in}{8.405395in}}{\pgfqpoint{8.632454in}{8.405395in}}%
\pgfpathlineto{\pgfqpoint{8.632454in}{8.405395in}}%
\pgfpathclose%
\pgfusepath{stroke,fill}%
\end{pgfscope}%
\begin{pgfscope}%
\pgfpathrectangle{\pgfqpoint{7.622482in}{7.624184in}}{\pgfqpoint{2.177280in}{2.201755in}}%
\pgfusepath{clip}%
\pgfsetbuttcap%
\pgfsetroundjoin%
\definecolor{currentfill}{rgb}{1.000000,0.498039,0.054902}%
\pgfsetfillcolor{currentfill}%
\pgfsetlinewidth{0.481800pt}%
\definecolor{currentstroke}{rgb}{1.000000,1.000000,1.000000}%
\pgfsetstrokecolor{currentstroke}%
\pgfsetdash{}{0pt}%
\pgfpathmoveto{\pgfqpoint{8.700208in}{9.016994in}}%
\pgfpathcurveto{\pgfqpoint{8.711258in}{9.016994in}}{\pgfqpoint{8.721857in}{9.021384in}}{\pgfqpoint{8.729671in}{9.029198in}}%
\pgfpathcurveto{\pgfqpoint{8.737485in}{9.037011in}}{\pgfqpoint{8.741875in}{9.047610in}}{\pgfqpoint{8.741875in}{9.058661in}}%
\pgfpathcurveto{\pgfqpoint{8.741875in}{9.069711in}}{\pgfqpoint{8.737485in}{9.080310in}}{\pgfqpoint{8.729671in}{9.088123in}}%
\pgfpathcurveto{\pgfqpoint{8.721857in}{9.095937in}}{\pgfqpoint{8.711258in}{9.100327in}}{\pgfqpoint{8.700208in}{9.100327in}}%
\pgfpathcurveto{\pgfqpoint{8.689158in}{9.100327in}}{\pgfqpoint{8.678559in}{9.095937in}}{\pgfqpoint{8.670745in}{9.088123in}}%
\pgfpathcurveto{\pgfqpoint{8.662932in}{9.080310in}}{\pgfqpoint{8.658542in}{9.069711in}}{\pgfqpoint{8.658542in}{9.058661in}}%
\pgfpathcurveto{\pgfqpoint{8.658542in}{9.047610in}}{\pgfqpoint{8.662932in}{9.037011in}}{\pgfqpoint{8.670745in}{9.029198in}}%
\pgfpathcurveto{\pgfqpoint{8.678559in}{9.021384in}}{\pgfqpoint{8.689158in}{9.016994in}}{\pgfqpoint{8.700208in}{9.016994in}}%
\pgfpathlineto{\pgfqpoint{8.700208in}{9.016994in}}%
\pgfpathclose%
\pgfusepath{stroke,fill}%
\end{pgfscope}%
\begin{pgfscope}%
\pgfpathrectangle{\pgfqpoint{7.622482in}{7.624184in}}{\pgfqpoint{2.177280in}{2.201755in}}%
\pgfusepath{clip}%
\pgfsetbuttcap%
\pgfsetroundjoin%
\definecolor{currentfill}{rgb}{1.000000,0.498039,0.054902}%
\pgfsetfillcolor{currentfill}%
\pgfsetlinewidth{0.481800pt}%
\definecolor{currentstroke}{rgb}{1.000000,1.000000,1.000000}%
\pgfsetstrokecolor{currentstroke}%
\pgfsetdash{}{0pt}%
\pgfpathmoveto{\pgfqpoint{8.767962in}{8.405395in}}%
\pgfpathcurveto{\pgfqpoint{8.779012in}{8.405395in}}{\pgfqpoint{8.789611in}{8.409785in}}{\pgfqpoint{8.797425in}{8.417599in}}%
\pgfpathcurveto{\pgfqpoint{8.805238in}{8.425413in}}{\pgfqpoint{8.809629in}{8.436012in}}{\pgfqpoint{8.809629in}{8.447062in}}%
\pgfpathcurveto{\pgfqpoint{8.809629in}{8.458112in}}{\pgfqpoint{8.805238in}{8.468711in}}{\pgfqpoint{8.797425in}{8.476525in}}%
\pgfpathcurveto{\pgfqpoint{8.789611in}{8.484338in}}{\pgfqpoint{8.779012in}{8.488729in}}{\pgfqpoint{8.767962in}{8.488729in}}%
\pgfpathcurveto{\pgfqpoint{8.756912in}{8.488729in}}{\pgfqpoint{8.746313in}{8.484338in}}{\pgfqpoint{8.738499in}{8.476525in}}%
\pgfpathcurveto{\pgfqpoint{8.730686in}{8.468711in}}{\pgfqpoint{8.726295in}{8.458112in}}{\pgfqpoint{8.726295in}{8.447062in}}%
\pgfpathcurveto{\pgfqpoint{8.726295in}{8.436012in}}{\pgfqpoint{8.730686in}{8.425413in}}{\pgfqpoint{8.738499in}{8.417599in}}%
\pgfpathcurveto{\pgfqpoint{8.746313in}{8.409785in}}{\pgfqpoint{8.756912in}{8.405395in}}{\pgfqpoint{8.767962in}{8.405395in}}%
\pgfpathlineto{\pgfqpoint{8.767962in}{8.405395in}}%
\pgfpathclose%
\pgfusepath{stroke,fill}%
\end{pgfscope}%
\begin{pgfscope}%
\pgfpathrectangle{\pgfqpoint{7.622482in}{7.624184in}}{\pgfqpoint{2.177280in}{2.201755in}}%
\pgfusepath{clip}%
\pgfsetbuttcap%
\pgfsetroundjoin%
\definecolor{currentfill}{rgb}{1.000000,0.498039,0.054902}%
\pgfsetfillcolor{currentfill}%
\pgfsetlinewidth{0.481800pt}%
\definecolor{currentstroke}{rgb}{1.000000,1.000000,1.000000}%
\pgfsetstrokecolor{currentstroke}%
\pgfsetdash{}{0pt}%
\pgfpathmoveto{\pgfqpoint{8.429193in}{8.516595in}}%
\pgfpathcurveto{\pgfqpoint{8.440243in}{8.516595in}}{\pgfqpoint{8.450842in}{8.520985in}}{\pgfqpoint{8.458656in}{8.528799in}}%
\pgfpathcurveto{\pgfqpoint{8.466469in}{8.536612in}}{\pgfqpoint{8.470859in}{8.547212in}}{\pgfqpoint{8.470859in}{8.558262in}}%
\pgfpathcurveto{\pgfqpoint{8.470859in}{8.569312in}}{\pgfqpoint{8.466469in}{8.579911in}}{\pgfqpoint{8.458656in}{8.587724in}}%
\pgfpathcurveto{\pgfqpoint{8.450842in}{8.595538in}}{\pgfqpoint{8.440243in}{8.599928in}}{\pgfqpoint{8.429193in}{8.599928in}}%
\pgfpathcurveto{\pgfqpoint{8.418143in}{8.599928in}}{\pgfqpoint{8.407544in}{8.595538in}}{\pgfqpoint{8.399730in}{8.587724in}}%
\pgfpathcurveto{\pgfqpoint{8.391916in}{8.579911in}}{\pgfqpoint{8.387526in}{8.569312in}}{\pgfqpoint{8.387526in}{8.558262in}}%
\pgfpathcurveto{\pgfqpoint{8.387526in}{8.547212in}}{\pgfqpoint{8.391916in}{8.536612in}}{\pgfqpoint{8.399730in}{8.528799in}}%
\pgfpathcurveto{\pgfqpoint{8.407544in}{8.520985in}}{\pgfqpoint{8.418143in}{8.516595in}}{\pgfqpoint{8.429193in}{8.516595in}}%
\pgfpathlineto{\pgfqpoint{8.429193in}{8.516595in}}%
\pgfpathclose%
\pgfusepath{stroke,fill}%
\end{pgfscope}%
\begin{pgfscope}%
\pgfpathrectangle{\pgfqpoint{7.622482in}{7.624184in}}{\pgfqpoint{2.177280in}{2.201755in}}%
\pgfusepath{clip}%
\pgfsetbuttcap%
\pgfsetroundjoin%
\definecolor{currentfill}{rgb}{1.000000,0.498039,0.054902}%
\pgfsetfillcolor{currentfill}%
\pgfsetlinewidth{0.481800pt}%
\definecolor{currentstroke}{rgb}{1.000000,1.000000,1.000000}%
\pgfsetstrokecolor{currentstroke}%
\pgfsetdash{}{0pt}%
\pgfpathmoveto{\pgfqpoint{8.767962in}{8.738994in}}%
\pgfpathcurveto{\pgfqpoint{8.779012in}{8.738994in}}{\pgfqpoint{8.789611in}{8.743385in}}{\pgfqpoint{8.797425in}{8.751198in}}%
\pgfpathcurveto{\pgfqpoint{8.805238in}{8.759012in}}{\pgfqpoint{8.809629in}{8.769611in}}{\pgfqpoint{8.809629in}{8.780661in}}%
\pgfpathcurveto{\pgfqpoint{8.809629in}{8.791711in}}{\pgfqpoint{8.805238in}{8.802310in}}{\pgfqpoint{8.797425in}{8.810124in}}%
\pgfpathcurveto{\pgfqpoint{8.789611in}{8.817938in}}{\pgfqpoint{8.779012in}{8.822328in}}{\pgfqpoint{8.767962in}{8.822328in}}%
\pgfpathcurveto{\pgfqpoint{8.756912in}{8.822328in}}{\pgfqpoint{8.746313in}{8.817938in}}{\pgfqpoint{8.738499in}{8.810124in}}%
\pgfpathcurveto{\pgfqpoint{8.730686in}{8.802310in}}{\pgfqpoint{8.726295in}{8.791711in}}{\pgfqpoint{8.726295in}{8.780661in}}%
\pgfpathcurveto{\pgfqpoint{8.726295in}{8.769611in}}{\pgfqpoint{8.730686in}{8.759012in}}{\pgfqpoint{8.738499in}{8.751198in}}%
\pgfpathcurveto{\pgfqpoint{8.746313in}{8.743385in}}{\pgfqpoint{8.756912in}{8.738994in}}{\pgfqpoint{8.767962in}{8.738994in}}%
\pgfpathlineto{\pgfqpoint{8.767962in}{8.738994in}}%
\pgfpathclose%
\pgfusepath{stroke,fill}%
\end{pgfscope}%
\begin{pgfscope}%
\pgfpathrectangle{\pgfqpoint{7.622482in}{7.624184in}}{\pgfqpoint{2.177280in}{2.201755in}}%
\pgfusepath{clip}%
\pgfsetbuttcap%
\pgfsetroundjoin%
\definecolor{currentfill}{rgb}{1.000000,0.498039,0.054902}%
\pgfsetfillcolor{currentfill}%
\pgfsetlinewidth{0.481800pt}%
\definecolor{currentstroke}{rgb}{1.000000,1.000000,1.000000}%
\pgfsetstrokecolor{currentstroke}%
\pgfsetdash{}{0pt}%
\pgfpathmoveto{\pgfqpoint{8.496947in}{8.405395in}}%
\pgfpathcurveto{\pgfqpoint{8.507997in}{8.405395in}}{\pgfqpoint{8.518596in}{8.409785in}}{\pgfqpoint{8.526409in}{8.417599in}}%
\pgfpathcurveto{\pgfqpoint{8.534223in}{8.425413in}}{\pgfqpoint{8.538613in}{8.436012in}}{\pgfqpoint{8.538613in}{8.447062in}}%
\pgfpathcurveto{\pgfqpoint{8.538613in}{8.458112in}}{\pgfqpoint{8.534223in}{8.468711in}}{\pgfqpoint{8.526409in}{8.476525in}}%
\pgfpathcurveto{\pgfqpoint{8.518596in}{8.484338in}}{\pgfqpoint{8.507997in}{8.488729in}}{\pgfqpoint{8.496947in}{8.488729in}}%
\pgfpathcurveto{\pgfqpoint{8.485896in}{8.488729in}}{\pgfqpoint{8.475297in}{8.484338in}}{\pgfqpoint{8.467484in}{8.476525in}}%
\pgfpathcurveto{\pgfqpoint{8.459670in}{8.468711in}}{\pgfqpoint{8.455280in}{8.458112in}}{\pgfqpoint{8.455280in}{8.447062in}}%
\pgfpathcurveto{\pgfqpoint{8.455280in}{8.436012in}}{\pgfqpoint{8.459670in}{8.425413in}}{\pgfqpoint{8.467484in}{8.417599in}}%
\pgfpathcurveto{\pgfqpoint{8.475297in}{8.409785in}}{\pgfqpoint{8.485896in}{8.405395in}}{\pgfqpoint{8.496947in}{8.405395in}}%
\pgfpathlineto{\pgfqpoint{8.496947in}{8.405395in}}%
\pgfpathclose%
\pgfusepath{stroke,fill}%
\end{pgfscope}%
\begin{pgfscope}%
\pgfpathrectangle{\pgfqpoint{7.622482in}{7.624184in}}{\pgfqpoint{2.177280in}{2.201755in}}%
\pgfusepath{clip}%
\pgfsetbuttcap%
\pgfsetroundjoin%
\definecolor{currentfill}{rgb}{1.000000,0.498039,0.054902}%
\pgfsetfillcolor{currentfill}%
\pgfsetlinewidth{0.481800pt}%
\definecolor{currentstroke}{rgb}{1.000000,1.000000,1.000000}%
\pgfsetstrokecolor{currentstroke}%
\pgfsetdash{}{0pt}%
\pgfpathmoveto{\pgfqpoint{8.971224in}{8.572195in}}%
\pgfpathcurveto{\pgfqpoint{8.982274in}{8.572195in}}{\pgfqpoint{8.992873in}{8.576585in}}{\pgfqpoint{9.000686in}{8.584399in}}%
\pgfpathcurveto{\pgfqpoint{9.008500in}{8.592212in}}{\pgfqpoint{9.012890in}{8.602811in}}{\pgfqpoint{9.012890in}{8.613862in}}%
\pgfpathcurveto{\pgfqpoint{9.012890in}{8.624912in}}{\pgfqpoint{9.008500in}{8.635511in}}{\pgfqpoint{9.000686in}{8.643324in}}%
\pgfpathcurveto{\pgfqpoint{8.992873in}{8.651138in}}{\pgfqpoint{8.982274in}{8.655528in}}{\pgfqpoint{8.971224in}{8.655528in}}%
\pgfpathcurveto{\pgfqpoint{8.960174in}{8.655528in}}{\pgfqpoint{8.949575in}{8.651138in}}{\pgfqpoint{8.941761in}{8.643324in}}%
\pgfpathcurveto{\pgfqpoint{8.933947in}{8.635511in}}{\pgfqpoint{8.929557in}{8.624912in}}{\pgfqpoint{8.929557in}{8.613862in}}%
\pgfpathcurveto{\pgfqpoint{8.929557in}{8.602811in}}{\pgfqpoint{8.933947in}{8.592212in}}{\pgfqpoint{8.941761in}{8.584399in}}%
\pgfpathcurveto{\pgfqpoint{8.949575in}{8.576585in}}{\pgfqpoint{8.960174in}{8.572195in}}{\pgfqpoint{8.971224in}{8.572195in}}%
\pgfpathlineto{\pgfqpoint{8.971224in}{8.572195in}}%
\pgfpathclose%
\pgfusepath{stroke,fill}%
\end{pgfscope}%
\begin{pgfscope}%
\pgfpathrectangle{\pgfqpoint{7.622482in}{7.624184in}}{\pgfqpoint{2.177280in}{2.201755in}}%
\pgfusepath{clip}%
\pgfsetbuttcap%
\pgfsetroundjoin%
\definecolor{currentfill}{rgb}{1.000000,0.498039,0.054902}%
\pgfsetfillcolor{currentfill}%
\pgfsetlinewidth{0.481800pt}%
\definecolor{currentstroke}{rgb}{1.000000,1.000000,1.000000}%
\pgfsetstrokecolor{currentstroke}%
\pgfsetdash{}{0pt}%
\pgfpathmoveto{\pgfqpoint{8.632454in}{8.683395in}}%
\pgfpathcurveto{\pgfqpoint{8.643504in}{8.683395in}}{\pgfqpoint{8.654104in}{8.687785in}}{\pgfqpoint{8.661917in}{8.695598in}}%
\pgfpathcurveto{\pgfqpoint{8.669731in}{8.703412in}}{\pgfqpoint{8.674121in}{8.714011in}}{\pgfqpoint{8.674121in}{8.725061in}}%
\pgfpathcurveto{\pgfqpoint{8.674121in}{8.736111in}}{\pgfqpoint{8.669731in}{8.746710in}}{\pgfqpoint{8.661917in}{8.754524in}}%
\pgfpathcurveto{\pgfqpoint{8.654104in}{8.762338in}}{\pgfqpoint{8.643504in}{8.766728in}}{\pgfqpoint{8.632454in}{8.766728in}}%
\pgfpathcurveto{\pgfqpoint{8.621404in}{8.766728in}}{\pgfqpoint{8.610805in}{8.762338in}}{\pgfqpoint{8.602992in}{8.754524in}}%
\pgfpathcurveto{\pgfqpoint{8.595178in}{8.746710in}}{\pgfqpoint{8.590788in}{8.736111in}}{\pgfqpoint{8.590788in}{8.725061in}}%
\pgfpathcurveto{\pgfqpoint{8.590788in}{8.714011in}}{\pgfqpoint{8.595178in}{8.703412in}}{\pgfqpoint{8.602992in}{8.695598in}}%
\pgfpathcurveto{\pgfqpoint{8.610805in}{8.687785in}}{\pgfqpoint{8.621404in}{8.683395in}}{\pgfqpoint{8.632454in}{8.683395in}}%
\pgfpathlineto{\pgfqpoint{8.632454in}{8.683395in}}%
\pgfpathclose%
\pgfusepath{stroke,fill}%
\end{pgfscope}%
\begin{pgfscope}%
\pgfpathrectangle{\pgfqpoint{7.622482in}{7.624184in}}{\pgfqpoint{2.177280in}{2.201755in}}%
\pgfusepath{clip}%
\pgfsetbuttcap%
\pgfsetroundjoin%
\definecolor{currentfill}{rgb}{1.000000,0.498039,0.054902}%
\pgfsetfillcolor{currentfill}%
\pgfsetlinewidth{0.481800pt}%
\definecolor{currentstroke}{rgb}{1.000000,1.000000,1.000000}%
\pgfsetstrokecolor{currentstroke}%
\pgfsetdash{}{0pt}%
\pgfpathmoveto{\pgfqpoint{8.767962in}{8.794594in}}%
\pgfpathcurveto{\pgfqpoint{8.779012in}{8.794594in}}{\pgfqpoint{8.789611in}{8.798985in}}{\pgfqpoint{8.797425in}{8.806798in}}%
\pgfpathcurveto{\pgfqpoint{8.805238in}{8.814612in}}{\pgfqpoint{8.809629in}{8.825211in}}{\pgfqpoint{8.809629in}{8.836261in}}%
\pgfpathcurveto{\pgfqpoint{8.809629in}{8.847311in}}{\pgfqpoint{8.805238in}{8.857910in}}{\pgfqpoint{8.797425in}{8.865724in}}%
\pgfpathcurveto{\pgfqpoint{8.789611in}{8.873537in}}{\pgfqpoint{8.779012in}{8.877928in}}{\pgfqpoint{8.767962in}{8.877928in}}%
\pgfpathcurveto{\pgfqpoint{8.756912in}{8.877928in}}{\pgfqpoint{8.746313in}{8.873537in}}{\pgfqpoint{8.738499in}{8.865724in}}%
\pgfpathcurveto{\pgfqpoint{8.730686in}{8.857910in}}{\pgfqpoint{8.726295in}{8.847311in}}{\pgfqpoint{8.726295in}{8.836261in}}%
\pgfpathcurveto{\pgfqpoint{8.726295in}{8.825211in}}{\pgfqpoint{8.730686in}{8.814612in}}{\pgfqpoint{8.738499in}{8.806798in}}%
\pgfpathcurveto{\pgfqpoint{8.746313in}{8.798985in}}{\pgfqpoint{8.756912in}{8.794594in}}{\pgfqpoint{8.767962in}{8.794594in}}%
\pgfpathlineto{\pgfqpoint{8.767962in}{8.794594in}}%
\pgfpathclose%
\pgfusepath{stroke,fill}%
\end{pgfscope}%
\begin{pgfscope}%
\pgfpathrectangle{\pgfqpoint{7.622482in}{7.624184in}}{\pgfqpoint{2.177280in}{2.201755in}}%
\pgfusepath{clip}%
\pgfsetbuttcap%
\pgfsetroundjoin%
\definecolor{currentfill}{rgb}{1.000000,0.498039,0.054902}%
\pgfsetfillcolor{currentfill}%
\pgfsetlinewidth{0.481800pt}%
\definecolor{currentstroke}{rgb}{1.000000,1.000000,1.000000}%
\pgfsetstrokecolor{currentstroke}%
\pgfsetdash{}{0pt}%
\pgfpathmoveto{\pgfqpoint{8.564700in}{8.683395in}}%
\pgfpathcurveto{\pgfqpoint{8.575751in}{8.683395in}}{\pgfqpoint{8.586350in}{8.687785in}}{\pgfqpoint{8.594163in}{8.695598in}}%
\pgfpathcurveto{\pgfqpoint{8.601977in}{8.703412in}}{\pgfqpoint{8.606367in}{8.714011in}}{\pgfqpoint{8.606367in}{8.725061in}}%
\pgfpathcurveto{\pgfqpoint{8.606367in}{8.736111in}}{\pgfqpoint{8.601977in}{8.746710in}}{\pgfqpoint{8.594163in}{8.754524in}}%
\pgfpathcurveto{\pgfqpoint{8.586350in}{8.762338in}}{\pgfqpoint{8.575751in}{8.766728in}}{\pgfqpoint{8.564700in}{8.766728in}}%
\pgfpathcurveto{\pgfqpoint{8.553650in}{8.766728in}}{\pgfqpoint{8.543051in}{8.762338in}}{\pgfqpoint{8.535238in}{8.754524in}}%
\pgfpathcurveto{\pgfqpoint{8.527424in}{8.746710in}}{\pgfqpoint{8.523034in}{8.736111in}}{\pgfqpoint{8.523034in}{8.725061in}}%
\pgfpathcurveto{\pgfqpoint{8.523034in}{8.714011in}}{\pgfqpoint{8.527424in}{8.703412in}}{\pgfqpoint{8.535238in}{8.695598in}}%
\pgfpathcurveto{\pgfqpoint{8.543051in}{8.687785in}}{\pgfqpoint{8.553650in}{8.683395in}}{\pgfqpoint{8.564700in}{8.683395in}}%
\pgfpathlineto{\pgfqpoint{8.564700in}{8.683395in}}%
\pgfpathclose%
\pgfusepath{stroke,fill}%
\end{pgfscope}%
\begin{pgfscope}%
\pgfpathrectangle{\pgfqpoint{7.622482in}{7.624184in}}{\pgfqpoint{2.177280in}{2.201755in}}%
\pgfusepath{clip}%
\pgfsetbuttcap%
\pgfsetroundjoin%
\definecolor{currentfill}{rgb}{1.000000,0.498039,0.054902}%
\pgfsetfillcolor{currentfill}%
\pgfsetlinewidth{0.481800pt}%
\definecolor{currentstroke}{rgb}{1.000000,1.000000,1.000000}%
\pgfsetstrokecolor{currentstroke}%
\pgfsetdash{}{0pt}%
\pgfpathmoveto{\pgfqpoint{8.632454in}{8.850194in}}%
\pgfpathcurveto{\pgfqpoint{8.643504in}{8.850194in}}{\pgfqpoint{8.654104in}{8.854584in}}{\pgfqpoint{8.661917in}{8.862398in}}%
\pgfpathcurveto{\pgfqpoint{8.669731in}{8.870212in}}{\pgfqpoint{8.674121in}{8.880811in}}{\pgfqpoint{8.674121in}{8.891861in}}%
\pgfpathcurveto{\pgfqpoint{8.674121in}{8.902911in}}{\pgfqpoint{8.669731in}{8.913510in}}{\pgfqpoint{8.661917in}{8.921324in}}%
\pgfpathcurveto{\pgfqpoint{8.654104in}{8.929137in}}{\pgfqpoint{8.643504in}{8.933528in}}{\pgfqpoint{8.632454in}{8.933528in}}%
\pgfpathcurveto{\pgfqpoint{8.621404in}{8.933528in}}{\pgfqpoint{8.610805in}{8.929137in}}{\pgfqpoint{8.602992in}{8.921324in}}%
\pgfpathcurveto{\pgfqpoint{8.595178in}{8.913510in}}{\pgfqpoint{8.590788in}{8.902911in}}{\pgfqpoint{8.590788in}{8.891861in}}%
\pgfpathcurveto{\pgfqpoint{8.590788in}{8.880811in}}{\pgfqpoint{8.595178in}{8.870212in}}{\pgfqpoint{8.602992in}{8.862398in}}%
\pgfpathcurveto{\pgfqpoint{8.610805in}{8.854584in}}{\pgfqpoint{8.621404in}{8.850194in}}{\pgfqpoint{8.632454in}{8.850194in}}%
\pgfpathlineto{\pgfqpoint{8.632454in}{8.850194in}}%
\pgfpathclose%
\pgfusepath{stroke,fill}%
\end{pgfscope}%
\begin{pgfscope}%
\pgfpathrectangle{\pgfqpoint{7.622482in}{7.624184in}}{\pgfqpoint{2.177280in}{2.201755in}}%
\pgfusepath{clip}%
\pgfsetbuttcap%
\pgfsetroundjoin%
\definecolor{currentfill}{rgb}{1.000000,0.498039,0.054902}%
\pgfsetfillcolor{currentfill}%
\pgfsetlinewidth{0.481800pt}%
\definecolor{currentstroke}{rgb}{1.000000,1.000000,1.000000}%
\pgfsetstrokecolor{currentstroke}%
\pgfsetdash{}{0pt}%
\pgfpathmoveto{\pgfqpoint{8.700208in}{8.961394in}}%
\pgfpathcurveto{\pgfqpoint{8.711258in}{8.961394in}}{\pgfqpoint{8.721857in}{8.965784in}}{\pgfqpoint{8.729671in}{8.973598in}}%
\pgfpathcurveto{\pgfqpoint{8.737485in}{8.981411in}}{\pgfqpoint{8.741875in}{8.992011in}}{\pgfqpoint{8.741875in}{9.003061in}}%
\pgfpathcurveto{\pgfqpoint{8.741875in}{9.014111in}}{\pgfqpoint{8.737485in}{9.024710in}}{\pgfqpoint{8.729671in}{9.032523in}}%
\pgfpathcurveto{\pgfqpoint{8.721857in}{9.040337in}}{\pgfqpoint{8.711258in}{9.044727in}}{\pgfqpoint{8.700208in}{9.044727in}}%
\pgfpathcurveto{\pgfqpoint{8.689158in}{9.044727in}}{\pgfqpoint{8.678559in}{9.040337in}}{\pgfqpoint{8.670745in}{9.032523in}}%
\pgfpathcurveto{\pgfqpoint{8.662932in}{9.024710in}}{\pgfqpoint{8.658542in}{9.014111in}}{\pgfqpoint{8.658542in}{9.003061in}}%
\pgfpathcurveto{\pgfqpoint{8.658542in}{8.992011in}}{\pgfqpoint{8.662932in}{8.981411in}}{\pgfqpoint{8.670745in}{8.973598in}}%
\pgfpathcurveto{\pgfqpoint{8.678559in}{8.965784in}}{\pgfqpoint{8.689158in}{8.961394in}}{\pgfqpoint{8.700208in}{8.961394in}}%
\pgfpathlineto{\pgfqpoint{8.700208in}{8.961394in}}%
\pgfpathclose%
\pgfusepath{stroke,fill}%
\end{pgfscope}%
\begin{pgfscope}%
\pgfpathrectangle{\pgfqpoint{7.622482in}{7.624184in}}{\pgfqpoint{2.177280in}{2.201755in}}%
\pgfusepath{clip}%
\pgfsetbuttcap%
\pgfsetroundjoin%
\definecolor{currentfill}{rgb}{1.000000,0.498039,0.054902}%
\pgfsetfillcolor{currentfill}%
\pgfsetlinewidth{0.481800pt}%
\definecolor{currentstroke}{rgb}{1.000000,1.000000,1.000000}%
\pgfsetstrokecolor{currentstroke}%
\pgfsetdash{}{0pt}%
\pgfpathmoveto{\pgfqpoint{8.700208in}{9.072594in}}%
\pgfpathcurveto{\pgfqpoint{8.711258in}{9.072594in}}{\pgfqpoint{8.721857in}{9.076984in}}{\pgfqpoint{8.729671in}{9.084798in}}%
\pgfpathcurveto{\pgfqpoint{8.737485in}{9.092611in}}{\pgfqpoint{8.741875in}{9.103210in}}{\pgfqpoint{8.741875in}{9.114260in}}%
\pgfpathcurveto{\pgfqpoint{8.741875in}{9.125311in}}{\pgfqpoint{8.737485in}{9.135910in}}{\pgfqpoint{8.729671in}{9.143723in}}%
\pgfpathcurveto{\pgfqpoint{8.721857in}{9.151537in}}{\pgfqpoint{8.711258in}{9.155927in}}{\pgfqpoint{8.700208in}{9.155927in}}%
\pgfpathcurveto{\pgfqpoint{8.689158in}{9.155927in}}{\pgfqpoint{8.678559in}{9.151537in}}{\pgfqpoint{8.670745in}{9.143723in}}%
\pgfpathcurveto{\pgfqpoint{8.662932in}{9.135910in}}{\pgfqpoint{8.658542in}{9.125311in}}{\pgfqpoint{8.658542in}{9.114260in}}%
\pgfpathcurveto{\pgfqpoint{8.658542in}{9.103210in}}{\pgfqpoint{8.662932in}{9.092611in}}{\pgfqpoint{8.670745in}{9.084798in}}%
\pgfpathcurveto{\pgfqpoint{8.678559in}{9.076984in}}{\pgfqpoint{8.689158in}{9.072594in}}{\pgfqpoint{8.700208in}{9.072594in}}%
\pgfpathlineto{\pgfqpoint{8.700208in}{9.072594in}}%
\pgfpathclose%
\pgfusepath{stroke,fill}%
\end{pgfscope}%
\begin{pgfscope}%
\pgfpathrectangle{\pgfqpoint{7.622482in}{7.624184in}}{\pgfqpoint{2.177280in}{2.201755in}}%
\pgfusepath{clip}%
\pgfsetbuttcap%
\pgfsetroundjoin%
\definecolor{currentfill}{rgb}{1.000000,0.498039,0.054902}%
\pgfsetfillcolor{currentfill}%
\pgfsetlinewidth{0.481800pt}%
\definecolor{currentstroke}{rgb}{1.000000,1.000000,1.000000}%
\pgfsetstrokecolor{currentstroke}%
\pgfsetdash{}{0pt}%
\pgfpathmoveto{\pgfqpoint{8.903470in}{9.016994in}}%
\pgfpathcurveto{\pgfqpoint{8.914520in}{9.016994in}}{\pgfqpoint{8.925119in}{9.021384in}}{\pgfqpoint{8.932933in}{9.029198in}}%
\pgfpathcurveto{\pgfqpoint{8.940746in}{9.037011in}}{\pgfqpoint{8.945136in}{9.047610in}}{\pgfqpoint{8.945136in}{9.058661in}}%
\pgfpathcurveto{\pgfqpoint{8.945136in}{9.069711in}}{\pgfqpoint{8.940746in}{9.080310in}}{\pgfqpoint{8.932933in}{9.088123in}}%
\pgfpathcurveto{\pgfqpoint{8.925119in}{9.095937in}}{\pgfqpoint{8.914520in}{9.100327in}}{\pgfqpoint{8.903470in}{9.100327in}}%
\pgfpathcurveto{\pgfqpoint{8.892420in}{9.100327in}}{\pgfqpoint{8.881821in}{9.095937in}}{\pgfqpoint{8.874007in}{9.088123in}}%
\pgfpathcurveto{\pgfqpoint{8.866193in}{9.080310in}}{\pgfqpoint{8.861803in}{9.069711in}}{\pgfqpoint{8.861803in}{9.058661in}}%
\pgfpathcurveto{\pgfqpoint{8.861803in}{9.047610in}}{\pgfqpoint{8.866193in}{9.037011in}}{\pgfqpoint{8.874007in}{9.029198in}}%
\pgfpathcurveto{\pgfqpoint{8.881821in}{9.021384in}}{\pgfqpoint{8.892420in}{9.016994in}}{\pgfqpoint{8.903470in}{9.016994in}}%
\pgfpathlineto{\pgfqpoint{8.903470in}{9.016994in}}%
\pgfpathclose%
\pgfusepath{stroke,fill}%
\end{pgfscope}%
\begin{pgfscope}%
\pgfpathrectangle{\pgfqpoint{7.622482in}{7.624184in}}{\pgfqpoint{2.177280in}{2.201755in}}%
\pgfusepath{clip}%
\pgfsetbuttcap%
\pgfsetroundjoin%
\definecolor{currentfill}{rgb}{1.000000,0.498039,0.054902}%
\pgfsetfillcolor{currentfill}%
\pgfsetlinewidth{0.481800pt}%
\definecolor{currentstroke}{rgb}{1.000000,1.000000,1.000000}%
\pgfsetstrokecolor{currentstroke}%
\pgfsetdash{}{0pt}%
\pgfpathmoveto{\pgfqpoint{8.767962in}{8.627795in}}%
\pgfpathcurveto{\pgfqpoint{8.779012in}{8.627795in}}{\pgfqpoint{8.789611in}{8.632185in}}{\pgfqpoint{8.797425in}{8.639999in}}%
\pgfpathcurveto{\pgfqpoint{8.805238in}{8.647812in}}{\pgfqpoint{8.809629in}{8.658411in}}{\pgfqpoint{8.809629in}{8.669461in}}%
\pgfpathcurveto{\pgfqpoint{8.809629in}{8.680512in}}{\pgfqpoint{8.805238in}{8.691111in}}{\pgfqpoint{8.797425in}{8.698924in}}%
\pgfpathcurveto{\pgfqpoint{8.789611in}{8.706738in}}{\pgfqpoint{8.779012in}{8.711128in}}{\pgfqpoint{8.767962in}{8.711128in}}%
\pgfpathcurveto{\pgfqpoint{8.756912in}{8.711128in}}{\pgfqpoint{8.746313in}{8.706738in}}{\pgfqpoint{8.738499in}{8.698924in}}%
\pgfpathcurveto{\pgfqpoint{8.730686in}{8.691111in}}{\pgfqpoint{8.726295in}{8.680512in}}{\pgfqpoint{8.726295in}{8.669461in}}%
\pgfpathcurveto{\pgfqpoint{8.726295in}{8.658411in}}{\pgfqpoint{8.730686in}{8.647812in}}{\pgfqpoint{8.738499in}{8.639999in}}%
\pgfpathcurveto{\pgfqpoint{8.746313in}{8.632185in}}{\pgfqpoint{8.756912in}{8.627795in}}{\pgfqpoint{8.767962in}{8.627795in}}%
\pgfpathlineto{\pgfqpoint{8.767962in}{8.627795in}}%
\pgfpathclose%
\pgfusepath{stroke,fill}%
\end{pgfscope}%
\begin{pgfscope}%
\pgfpathrectangle{\pgfqpoint{7.622482in}{7.624184in}}{\pgfqpoint{2.177280in}{2.201755in}}%
\pgfusepath{clip}%
\pgfsetbuttcap%
\pgfsetroundjoin%
\definecolor{currentfill}{rgb}{1.000000,0.498039,0.054902}%
\pgfsetfillcolor{currentfill}%
\pgfsetlinewidth{0.481800pt}%
\definecolor{currentstroke}{rgb}{1.000000,1.000000,1.000000}%
\pgfsetstrokecolor{currentstroke}%
\pgfsetdash{}{0pt}%
\pgfpathmoveto{\pgfqpoint{8.429193in}{8.460995in}}%
\pgfpathcurveto{\pgfqpoint{8.440243in}{8.460995in}}{\pgfqpoint{8.450842in}{8.465385in}}{\pgfqpoint{8.458656in}{8.473199in}}%
\pgfpathcurveto{\pgfqpoint{8.466469in}{8.481013in}}{\pgfqpoint{8.470859in}{8.491612in}}{\pgfqpoint{8.470859in}{8.502662in}}%
\pgfpathcurveto{\pgfqpoint{8.470859in}{8.513712in}}{\pgfqpoint{8.466469in}{8.524311in}}{\pgfqpoint{8.458656in}{8.532125in}}%
\pgfpathcurveto{\pgfqpoint{8.450842in}{8.539938in}}{\pgfqpoint{8.440243in}{8.544328in}}{\pgfqpoint{8.429193in}{8.544328in}}%
\pgfpathcurveto{\pgfqpoint{8.418143in}{8.544328in}}{\pgfqpoint{8.407544in}{8.539938in}}{\pgfqpoint{8.399730in}{8.532125in}}%
\pgfpathcurveto{\pgfqpoint{8.391916in}{8.524311in}}{\pgfqpoint{8.387526in}{8.513712in}}{\pgfqpoint{8.387526in}{8.502662in}}%
\pgfpathcurveto{\pgfqpoint{8.387526in}{8.491612in}}{\pgfqpoint{8.391916in}{8.481013in}}{\pgfqpoint{8.399730in}{8.473199in}}%
\pgfpathcurveto{\pgfqpoint{8.407544in}{8.465385in}}{\pgfqpoint{8.418143in}{8.460995in}}{\pgfqpoint{8.429193in}{8.460995in}}%
\pgfpathlineto{\pgfqpoint{8.429193in}{8.460995in}}%
\pgfpathclose%
\pgfusepath{stroke,fill}%
\end{pgfscope}%
\begin{pgfscope}%
\pgfpathrectangle{\pgfqpoint{7.622482in}{7.624184in}}{\pgfqpoint{2.177280in}{2.201755in}}%
\pgfusepath{clip}%
\pgfsetbuttcap%
\pgfsetroundjoin%
\definecolor{currentfill}{rgb}{1.000000,0.498039,0.054902}%
\pgfsetfillcolor{currentfill}%
\pgfsetlinewidth{0.481800pt}%
\definecolor{currentstroke}{rgb}{1.000000,1.000000,1.000000}%
\pgfsetstrokecolor{currentstroke}%
\pgfsetdash{}{0pt}%
\pgfpathmoveto{\pgfqpoint{8.496947in}{8.349795in}}%
\pgfpathcurveto{\pgfqpoint{8.507997in}{8.349795in}}{\pgfqpoint{8.518596in}{8.354186in}}{\pgfqpoint{8.526409in}{8.361999in}}%
\pgfpathcurveto{\pgfqpoint{8.534223in}{8.369813in}}{\pgfqpoint{8.538613in}{8.380412in}}{\pgfqpoint{8.538613in}{8.391462in}}%
\pgfpathcurveto{\pgfqpoint{8.538613in}{8.402512in}}{\pgfqpoint{8.534223in}{8.413111in}}{\pgfqpoint{8.526409in}{8.420925in}}%
\pgfpathcurveto{\pgfqpoint{8.518596in}{8.428738in}}{\pgfqpoint{8.507997in}{8.433129in}}{\pgfqpoint{8.496947in}{8.433129in}}%
\pgfpathcurveto{\pgfqpoint{8.485896in}{8.433129in}}{\pgfqpoint{8.475297in}{8.428738in}}{\pgfqpoint{8.467484in}{8.420925in}}%
\pgfpathcurveto{\pgfqpoint{8.459670in}{8.413111in}}{\pgfqpoint{8.455280in}{8.402512in}}{\pgfqpoint{8.455280in}{8.391462in}}%
\pgfpathcurveto{\pgfqpoint{8.455280in}{8.380412in}}{\pgfqpoint{8.459670in}{8.369813in}}{\pgfqpoint{8.467484in}{8.361999in}}%
\pgfpathcurveto{\pgfqpoint{8.475297in}{8.354186in}}{\pgfqpoint{8.485896in}{8.349795in}}{\pgfqpoint{8.496947in}{8.349795in}}%
\pgfpathlineto{\pgfqpoint{8.496947in}{8.349795in}}%
\pgfpathclose%
\pgfusepath{stroke,fill}%
\end{pgfscope}%
\begin{pgfscope}%
\pgfpathrectangle{\pgfqpoint{7.622482in}{7.624184in}}{\pgfqpoint{2.177280in}{2.201755in}}%
\pgfusepath{clip}%
\pgfsetbuttcap%
\pgfsetroundjoin%
\definecolor{currentfill}{rgb}{1.000000,0.498039,0.054902}%
\pgfsetfillcolor{currentfill}%
\pgfsetlinewidth{0.481800pt}%
\definecolor{currentstroke}{rgb}{1.000000,1.000000,1.000000}%
\pgfsetstrokecolor{currentstroke}%
\pgfsetdash{}{0pt}%
\pgfpathmoveto{\pgfqpoint{8.429193in}{8.349795in}}%
\pgfpathcurveto{\pgfqpoint{8.440243in}{8.349795in}}{\pgfqpoint{8.450842in}{8.354186in}}{\pgfqpoint{8.458656in}{8.361999in}}%
\pgfpathcurveto{\pgfqpoint{8.466469in}{8.369813in}}{\pgfqpoint{8.470859in}{8.380412in}}{\pgfqpoint{8.470859in}{8.391462in}}%
\pgfpathcurveto{\pgfqpoint{8.470859in}{8.402512in}}{\pgfqpoint{8.466469in}{8.413111in}}{\pgfqpoint{8.458656in}{8.420925in}}%
\pgfpathcurveto{\pgfqpoint{8.450842in}{8.428738in}}{\pgfqpoint{8.440243in}{8.433129in}}{\pgfqpoint{8.429193in}{8.433129in}}%
\pgfpathcurveto{\pgfqpoint{8.418143in}{8.433129in}}{\pgfqpoint{8.407544in}{8.428738in}}{\pgfqpoint{8.399730in}{8.420925in}}%
\pgfpathcurveto{\pgfqpoint{8.391916in}{8.413111in}}{\pgfqpoint{8.387526in}{8.402512in}}{\pgfqpoint{8.387526in}{8.391462in}}%
\pgfpathcurveto{\pgfqpoint{8.387526in}{8.380412in}}{\pgfqpoint{8.391916in}{8.369813in}}{\pgfqpoint{8.399730in}{8.361999in}}%
\pgfpathcurveto{\pgfqpoint{8.407544in}{8.354186in}}{\pgfqpoint{8.418143in}{8.349795in}}{\pgfqpoint{8.429193in}{8.349795in}}%
\pgfpathlineto{\pgfqpoint{8.429193in}{8.349795in}}%
\pgfpathclose%
\pgfusepath{stroke,fill}%
\end{pgfscope}%
\begin{pgfscope}%
\pgfpathrectangle{\pgfqpoint{7.622482in}{7.624184in}}{\pgfqpoint{2.177280in}{2.201755in}}%
\pgfusepath{clip}%
\pgfsetbuttcap%
\pgfsetroundjoin%
\definecolor{currentfill}{rgb}{1.000000,0.498039,0.054902}%
\pgfsetfillcolor{currentfill}%
\pgfsetlinewidth{0.481800pt}%
\definecolor{currentstroke}{rgb}{1.000000,1.000000,1.000000}%
\pgfsetstrokecolor{currentstroke}%
\pgfsetdash{}{0pt}%
\pgfpathmoveto{\pgfqpoint{8.564700in}{8.516595in}}%
\pgfpathcurveto{\pgfqpoint{8.575751in}{8.516595in}}{\pgfqpoint{8.586350in}{8.520985in}}{\pgfqpoint{8.594163in}{8.528799in}}%
\pgfpathcurveto{\pgfqpoint{8.601977in}{8.536612in}}{\pgfqpoint{8.606367in}{8.547212in}}{\pgfqpoint{8.606367in}{8.558262in}}%
\pgfpathcurveto{\pgfqpoint{8.606367in}{8.569312in}}{\pgfqpoint{8.601977in}{8.579911in}}{\pgfqpoint{8.594163in}{8.587724in}}%
\pgfpathcurveto{\pgfqpoint{8.586350in}{8.595538in}}{\pgfqpoint{8.575751in}{8.599928in}}{\pgfqpoint{8.564700in}{8.599928in}}%
\pgfpathcurveto{\pgfqpoint{8.553650in}{8.599928in}}{\pgfqpoint{8.543051in}{8.595538in}}{\pgfqpoint{8.535238in}{8.587724in}}%
\pgfpathcurveto{\pgfqpoint{8.527424in}{8.579911in}}{\pgfqpoint{8.523034in}{8.569312in}}{\pgfqpoint{8.523034in}{8.558262in}}%
\pgfpathcurveto{\pgfqpoint{8.523034in}{8.547212in}}{\pgfqpoint{8.527424in}{8.536612in}}{\pgfqpoint{8.535238in}{8.528799in}}%
\pgfpathcurveto{\pgfqpoint{8.543051in}{8.520985in}}{\pgfqpoint{8.553650in}{8.516595in}}{\pgfqpoint{8.564700in}{8.516595in}}%
\pgfpathlineto{\pgfqpoint{8.564700in}{8.516595in}}%
\pgfpathclose%
\pgfusepath{stroke,fill}%
\end{pgfscope}%
\begin{pgfscope}%
\pgfpathrectangle{\pgfqpoint{7.622482in}{7.624184in}}{\pgfqpoint{2.177280in}{2.201755in}}%
\pgfusepath{clip}%
\pgfsetbuttcap%
\pgfsetroundjoin%
\definecolor{currentfill}{rgb}{1.000000,0.498039,0.054902}%
\pgfsetfillcolor{currentfill}%
\pgfsetlinewidth{0.481800pt}%
\definecolor{currentstroke}{rgb}{1.000000,1.000000,1.000000}%
\pgfsetstrokecolor{currentstroke}%
\pgfsetdash{}{0pt}%
\pgfpathmoveto{\pgfqpoint{8.835716in}{8.627795in}}%
\pgfpathcurveto{\pgfqpoint{8.846766in}{8.627795in}}{\pgfqpoint{8.857365in}{8.632185in}}{\pgfqpoint{8.865179in}{8.639999in}}%
\pgfpathcurveto{\pgfqpoint{8.872992in}{8.647812in}}{\pgfqpoint{8.877383in}{8.658411in}}{\pgfqpoint{8.877383in}{8.669461in}}%
\pgfpathcurveto{\pgfqpoint{8.877383in}{8.680512in}}{\pgfqpoint{8.872992in}{8.691111in}}{\pgfqpoint{8.865179in}{8.698924in}}%
\pgfpathcurveto{\pgfqpoint{8.857365in}{8.706738in}}{\pgfqpoint{8.846766in}{8.711128in}}{\pgfqpoint{8.835716in}{8.711128in}}%
\pgfpathcurveto{\pgfqpoint{8.824666in}{8.711128in}}{\pgfqpoint{8.814067in}{8.706738in}}{\pgfqpoint{8.806253in}{8.698924in}}%
\pgfpathcurveto{\pgfqpoint{8.798440in}{8.691111in}}{\pgfqpoint{8.794049in}{8.680512in}}{\pgfqpoint{8.794049in}{8.669461in}}%
\pgfpathcurveto{\pgfqpoint{8.794049in}{8.658411in}}{\pgfqpoint{8.798440in}{8.647812in}}{\pgfqpoint{8.806253in}{8.639999in}}%
\pgfpathcurveto{\pgfqpoint{8.814067in}{8.632185in}}{\pgfqpoint{8.824666in}{8.627795in}}{\pgfqpoint{8.835716in}{8.627795in}}%
\pgfpathlineto{\pgfqpoint{8.835716in}{8.627795in}}%
\pgfpathclose%
\pgfusepath{stroke,fill}%
\end{pgfscope}%
\begin{pgfscope}%
\pgfpathrectangle{\pgfqpoint{7.622482in}{7.624184in}}{\pgfqpoint{2.177280in}{2.201755in}}%
\pgfusepath{clip}%
\pgfsetbuttcap%
\pgfsetroundjoin%
\definecolor{currentfill}{rgb}{1.000000,0.498039,0.054902}%
\pgfsetfillcolor{currentfill}%
\pgfsetlinewidth{0.481800pt}%
\definecolor{currentstroke}{rgb}{1.000000,1.000000,1.000000}%
\pgfsetstrokecolor{currentstroke}%
\pgfsetdash{}{0pt}%
\pgfpathmoveto{\pgfqpoint{8.767962in}{8.294195in}}%
\pgfpathcurveto{\pgfqpoint{8.779012in}{8.294195in}}{\pgfqpoint{8.789611in}{8.298586in}}{\pgfqpoint{8.797425in}{8.306399in}}%
\pgfpathcurveto{\pgfqpoint{8.805238in}{8.314213in}}{\pgfqpoint{8.809629in}{8.324812in}}{\pgfqpoint{8.809629in}{8.335862in}}%
\pgfpathcurveto{\pgfqpoint{8.809629in}{8.346912in}}{\pgfqpoint{8.805238in}{8.357511in}}{\pgfqpoint{8.797425in}{8.365325in}}%
\pgfpathcurveto{\pgfqpoint{8.789611in}{8.373139in}}{\pgfqpoint{8.779012in}{8.377529in}}{\pgfqpoint{8.767962in}{8.377529in}}%
\pgfpathcurveto{\pgfqpoint{8.756912in}{8.377529in}}{\pgfqpoint{8.746313in}{8.373139in}}{\pgfqpoint{8.738499in}{8.365325in}}%
\pgfpathcurveto{\pgfqpoint{8.730686in}{8.357511in}}{\pgfqpoint{8.726295in}{8.346912in}}{\pgfqpoint{8.726295in}{8.335862in}}%
\pgfpathcurveto{\pgfqpoint{8.726295in}{8.324812in}}{\pgfqpoint{8.730686in}{8.314213in}}{\pgfqpoint{8.738499in}{8.306399in}}%
\pgfpathcurveto{\pgfqpoint{8.746313in}{8.298586in}}{\pgfqpoint{8.756912in}{8.294195in}}{\pgfqpoint{8.767962in}{8.294195in}}%
\pgfpathlineto{\pgfqpoint{8.767962in}{8.294195in}}%
\pgfpathclose%
\pgfusepath{stroke,fill}%
\end{pgfscope}%
\begin{pgfscope}%
\pgfpathrectangle{\pgfqpoint{7.622482in}{7.624184in}}{\pgfqpoint{2.177280in}{2.201755in}}%
\pgfusepath{clip}%
\pgfsetbuttcap%
\pgfsetroundjoin%
\definecolor{currentfill}{rgb}{1.000000,0.498039,0.054902}%
\pgfsetfillcolor{currentfill}%
\pgfsetlinewidth{0.481800pt}%
\definecolor{currentstroke}{rgb}{1.000000,1.000000,1.000000}%
\pgfsetstrokecolor{currentstroke}%
\pgfsetdash{}{0pt}%
\pgfpathmoveto{\pgfqpoint{8.835716in}{8.627795in}}%
\pgfpathcurveto{\pgfqpoint{8.846766in}{8.627795in}}{\pgfqpoint{8.857365in}{8.632185in}}{\pgfqpoint{8.865179in}{8.639999in}}%
\pgfpathcurveto{\pgfqpoint{8.872992in}{8.647812in}}{\pgfqpoint{8.877383in}{8.658411in}}{\pgfqpoint{8.877383in}{8.669461in}}%
\pgfpathcurveto{\pgfqpoint{8.877383in}{8.680512in}}{\pgfqpoint{8.872992in}{8.691111in}}{\pgfqpoint{8.865179in}{8.698924in}}%
\pgfpathcurveto{\pgfqpoint{8.857365in}{8.706738in}}{\pgfqpoint{8.846766in}{8.711128in}}{\pgfqpoint{8.835716in}{8.711128in}}%
\pgfpathcurveto{\pgfqpoint{8.824666in}{8.711128in}}{\pgfqpoint{8.814067in}{8.706738in}}{\pgfqpoint{8.806253in}{8.698924in}}%
\pgfpathcurveto{\pgfqpoint{8.798440in}{8.691111in}}{\pgfqpoint{8.794049in}{8.680512in}}{\pgfqpoint{8.794049in}{8.669461in}}%
\pgfpathcurveto{\pgfqpoint{8.794049in}{8.658411in}}{\pgfqpoint{8.798440in}{8.647812in}}{\pgfqpoint{8.806253in}{8.639999in}}%
\pgfpathcurveto{\pgfqpoint{8.814067in}{8.632185in}}{\pgfqpoint{8.824666in}{8.627795in}}{\pgfqpoint{8.835716in}{8.627795in}}%
\pgfpathlineto{\pgfqpoint{8.835716in}{8.627795in}}%
\pgfpathclose%
\pgfusepath{stroke,fill}%
\end{pgfscope}%
\begin{pgfscope}%
\pgfpathrectangle{\pgfqpoint{7.622482in}{7.624184in}}{\pgfqpoint{2.177280in}{2.201755in}}%
\pgfusepath{clip}%
\pgfsetbuttcap%
\pgfsetroundjoin%
\definecolor{currentfill}{rgb}{1.000000,0.498039,0.054902}%
\pgfsetfillcolor{currentfill}%
\pgfsetlinewidth{0.481800pt}%
\definecolor{currentstroke}{rgb}{1.000000,1.000000,1.000000}%
\pgfsetstrokecolor{currentstroke}%
\pgfsetdash{}{0pt}%
\pgfpathmoveto{\pgfqpoint{8.767962in}{9.016994in}}%
\pgfpathcurveto{\pgfqpoint{8.779012in}{9.016994in}}{\pgfqpoint{8.789611in}{9.021384in}}{\pgfqpoint{8.797425in}{9.029198in}}%
\pgfpathcurveto{\pgfqpoint{8.805238in}{9.037011in}}{\pgfqpoint{8.809629in}{9.047610in}}{\pgfqpoint{8.809629in}{9.058661in}}%
\pgfpathcurveto{\pgfqpoint{8.809629in}{9.069711in}}{\pgfqpoint{8.805238in}{9.080310in}}{\pgfqpoint{8.797425in}{9.088123in}}%
\pgfpathcurveto{\pgfqpoint{8.789611in}{9.095937in}}{\pgfqpoint{8.779012in}{9.100327in}}{\pgfqpoint{8.767962in}{9.100327in}}%
\pgfpathcurveto{\pgfqpoint{8.756912in}{9.100327in}}{\pgfqpoint{8.746313in}{9.095937in}}{\pgfqpoint{8.738499in}{9.088123in}}%
\pgfpathcurveto{\pgfqpoint{8.730686in}{9.080310in}}{\pgfqpoint{8.726295in}{9.069711in}}{\pgfqpoint{8.726295in}{9.058661in}}%
\pgfpathcurveto{\pgfqpoint{8.726295in}{9.047610in}}{\pgfqpoint{8.730686in}{9.037011in}}{\pgfqpoint{8.738499in}{9.029198in}}%
\pgfpathcurveto{\pgfqpoint{8.746313in}{9.021384in}}{\pgfqpoint{8.756912in}{9.016994in}}{\pgfqpoint{8.767962in}{9.016994in}}%
\pgfpathlineto{\pgfqpoint{8.767962in}{9.016994in}}%
\pgfpathclose%
\pgfusepath{stroke,fill}%
\end{pgfscope}%
\begin{pgfscope}%
\pgfpathrectangle{\pgfqpoint{7.622482in}{7.624184in}}{\pgfqpoint{2.177280in}{2.201755in}}%
\pgfusepath{clip}%
\pgfsetbuttcap%
\pgfsetroundjoin%
\definecolor{currentfill}{rgb}{1.000000,0.498039,0.054902}%
\pgfsetfillcolor{currentfill}%
\pgfsetlinewidth{0.481800pt}%
\definecolor{currentstroke}{rgb}{1.000000,1.000000,1.000000}%
\pgfsetstrokecolor{currentstroke}%
\pgfsetdash{}{0pt}%
\pgfpathmoveto{\pgfqpoint{8.632454in}{8.794594in}}%
\pgfpathcurveto{\pgfqpoint{8.643504in}{8.794594in}}{\pgfqpoint{8.654104in}{8.798985in}}{\pgfqpoint{8.661917in}{8.806798in}}%
\pgfpathcurveto{\pgfqpoint{8.669731in}{8.814612in}}{\pgfqpoint{8.674121in}{8.825211in}}{\pgfqpoint{8.674121in}{8.836261in}}%
\pgfpathcurveto{\pgfqpoint{8.674121in}{8.847311in}}{\pgfqpoint{8.669731in}{8.857910in}}{\pgfqpoint{8.661917in}{8.865724in}}%
\pgfpathcurveto{\pgfqpoint{8.654104in}{8.873537in}}{\pgfqpoint{8.643504in}{8.877928in}}{\pgfqpoint{8.632454in}{8.877928in}}%
\pgfpathcurveto{\pgfqpoint{8.621404in}{8.877928in}}{\pgfqpoint{8.610805in}{8.873537in}}{\pgfqpoint{8.602992in}{8.865724in}}%
\pgfpathcurveto{\pgfqpoint{8.595178in}{8.857910in}}{\pgfqpoint{8.590788in}{8.847311in}}{\pgfqpoint{8.590788in}{8.836261in}}%
\pgfpathcurveto{\pgfqpoint{8.590788in}{8.825211in}}{\pgfqpoint{8.595178in}{8.814612in}}{\pgfqpoint{8.602992in}{8.806798in}}%
\pgfpathcurveto{\pgfqpoint{8.610805in}{8.798985in}}{\pgfqpoint{8.621404in}{8.794594in}}{\pgfqpoint{8.632454in}{8.794594in}}%
\pgfpathlineto{\pgfqpoint{8.632454in}{8.794594in}}%
\pgfpathclose%
\pgfusepath{stroke,fill}%
\end{pgfscope}%
\begin{pgfscope}%
\pgfpathrectangle{\pgfqpoint{7.622482in}{7.624184in}}{\pgfqpoint{2.177280in}{2.201755in}}%
\pgfusepath{clip}%
\pgfsetbuttcap%
\pgfsetroundjoin%
\definecolor{currentfill}{rgb}{1.000000,0.498039,0.054902}%
\pgfsetfillcolor{currentfill}%
\pgfsetlinewidth{0.481800pt}%
\definecolor{currentstroke}{rgb}{1.000000,1.000000,1.000000}%
\pgfsetstrokecolor{currentstroke}%
\pgfsetdash{}{0pt}%
\pgfpathmoveto{\pgfqpoint{8.632454in}{8.405395in}}%
\pgfpathcurveto{\pgfqpoint{8.643504in}{8.405395in}}{\pgfqpoint{8.654104in}{8.409785in}}{\pgfqpoint{8.661917in}{8.417599in}}%
\pgfpathcurveto{\pgfqpoint{8.669731in}{8.425413in}}{\pgfqpoint{8.674121in}{8.436012in}}{\pgfqpoint{8.674121in}{8.447062in}}%
\pgfpathcurveto{\pgfqpoint{8.674121in}{8.458112in}}{\pgfqpoint{8.669731in}{8.468711in}}{\pgfqpoint{8.661917in}{8.476525in}}%
\pgfpathcurveto{\pgfqpoint{8.654104in}{8.484338in}}{\pgfqpoint{8.643504in}{8.488729in}}{\pgfqpoint{8.632454in}{8.488729in}}%
\pgfpathcurveto{\pgfqpoint{8.621404in}{8.488729in}}{\pgfqpoint{8.610805in}{8.484338in}}{\pgfqpoint{8.602992in}{8.476525in}}%
\pgfpathcurveto{\pgfqpoint{8.595178in}{8.468711in}}{\pgfqpoint{8.590788in}{8.458112in}}{\pgfqpoint{8.590788in}{8.447062in}}%
\pgfpathcurveto{\pgfqpoint{8.590788in}{8.436012in}}{\pgfqpoint{8.595178in}{8.425413in}}{\pgfqpoint{8.602992in}{8.417599in}}%
\pgfpathcurveto{\pgfqpoint{8.610805in}{8.409785in}}{\pgfqpoint{8.621404in}{8.405395in}}{\pgfqpoint{8.632454in}{8.405395in}}%
\pgfpathlineto{\pgfqpoint{8.632454in}{8.405395in}}%
\pgfpathclose%
\pgfusepath{stroke,fill}%
\end{pgfscope}%
\begin{pgfscope}%
\pgfpathrectangle{\pgfqpoint{7.622482in}{7.624184in}}{\pgfqpoint{2.177280in}{2.201755in}}%
\pgfusepath{clip}%
\pgfsetbuttcap%
\pgfsetroundjoin%
\definecolor{currentfill}{rgb}{1.000000,0.498039,0.054902}%
\pgfsetfillcolor{currentfill}%
\pgfsetlinewidth{0.481800pt}%
\definecolor{currentstroke}{rgb}{1.000000,1.000000,1.000000}%
\pgfsetstrokecolor{currentstroke}%
\pgfsetdash{}{0pt}%
\pgfpathmoveto{\pgfqpoint{8.632454in}{8.349795in}}%
\pgfpathcurveto{\pgfqpoint{8.643504in}{8.349795in}}{\pgfqpoint{8.654104in}{8.354186in}}{\pgfqpoint{8.661917in}{8.361999in}}%
\pgfpathcurveto{\pgfqpoint{8.669731in}{8.369813in}}{\pgfqpoint{8.674121in}{8.380412in}}{\pgfqpoint{8.674121in}{8.391462in}}%
\pgfpathcurveto{\pgfqpoint{8.674121in}{8.402512in}}{\pgfqpoint{8.669731in}{8.413111in}}{\pgfqpoint{8.661917in}{8.420925in}}%
\pgfpathcurveto{\pgfqpoint{8.654104in}{8.428738in}}{\pgfqpoint{8.643504in}{8.433129in}}{\pgfqpoint{8.632454in}{8.433129in}}%
\pgfpathcurveto{\pgfqpoint{8.621404in}{8.433129in}}{\pgfqpoint{8.610805in}{8.428738in}}{\pgfqpoint{8.602992in}{8.420925in}}%
\pgfpathcurveto{\pgfqpoint{8.595178in}{8.413111in}}{\pgfqpoint{8.590788in}{8.402512in}}{\pgfqpoint{8.590788in}{8.391462in}}%
\pgfpathcurveto{\pgfqpoint{8.590788in}{8.380412in}}{\pgfqpoint{8.595178in}{8.369813in}}{\pgfqpoint{8.602992in}{8.361999in}}%
\pgfpathcurveto{\pgfqpoint{8.610805in}{8.354186in}}{\pgfqpoint{8.621404in}{8.349795in}}{\pgfqpoint{8.632454in}{8.349795in}}%
\pgfpathlineto{\pgfqpoint{8.632454in}{8.349795in}}%
\pgfpathclose%
\pgfusepath{stroke,fill}%
\end{pgfscope}%
\begin{pgfscope}%
\pgfpathrectangle{\pgfqpoint{7.622482in}{7.624184in}}{\pgfqpoint{2.177280in}{2.201755in}}%
\pgfusepath{clip}%
\pgfsetbuttcap%
\pgfsetroundjoin%
\definecolor{currentfill}{rgb}{1.000000,0.498039,0.054902}%
\pgfsetfillcolor{currentfill}%
\pgfsetlinewidth{0.481800pt}%
\definecolor{currentstroke}{rgb}{1.000000,1.000000,1.000000}%
\pgfsetstrokecolor{currentstroke}%
\pgfsetdash{}{0pt}%
\pgfpathmoveto{\pgfqpoint{8.564700in}{8.349795in}}%
\pgfpathcurveto{\pgfqpoint{8.575751in}{8.349795in}}{\pgfqpoint{8.586350in}{8.354186in}}{\pgfqpoint{8.594163in}{8.361999in}}%
\pgfpathcurveto{\pgfqpoint{8.601977in}{8.369813in}}{\pgfqpoint{8.606367in}{8.380412in}}{\pgfqpoint{8.606367in}{8.391462in}}%
\pgfpathcurveto{\pgfqpoint{8.606367in}{8.402512in}}{\pgfqpoint{8.601977in}{8.413111in}}{\pgfqpoint{8.594163in}{8.420925in}}%
\pgfpathcurveto{\pgfqpoint{8.586350in}{8.428738in}}{\pgfqpoint{8.575751in}{8.433129in}}{\pgfqpoint{8.564700in}{8.433129in}}%
\pgfpathcurveto{\pgfqpoint{8.553650in}{8.433129in}}{\pgfqpoint{8.543051in}{8.428738in}}{\pgfqpoint{8.535238in}{8.420925in}}%
\pgfpathcurveto{\pgfqpoint{8.527424in}{8.413111in}}{\pgfqpoint{8.523034in}{8.402512in}}{\pgfqpoint{8.523034in}{8.391462in}}%
\pgfpathcurveto{\pgfqpoint{8.523034in}{8.380412in}}{\pgfqpoint{8.527424in}{8.369813in}}{\pgfqpoint{8.535238in}{8.361999in}}%
\pgfpathcurveto{\pgfqpoint{8.543051in}{8.354186in}}{\pgfqpoint{8.553650in}{8.349795in}}{\pgfqpoint{8.564700in}{8.349795in}}%
\pgfpathlineto{\pgfqpoint{8.564700in}{8.349795in}}%
\pgfpathclose%
\pgfusepath{stroke,fill}%
\end{pgfscope}%
\begin{pgfscope}%
\pgfpathrectangle{\pgfqpoint{7.622482in}{7.624184in}}{\pgfqpoint{2.177280in}{2.201755in}}%
\pgfusepath{clip}%
\pgfsetbuttcap%
\pgfsetroundjoin%
\definecolor{currentfill}{rgb}{1.000000,0.498039,0.054902}%
\pgfsetfillcolor{currentfill}%
\pgfsetlinewidth{0.481800pt}%
\definecolor{currentstroke}{rgb}{1.000000,1.000000,1.000000}%
\pgfsetstrokecolor{currentstroke}%
\pgfsetdash{}{0pt}%
\pgfpathmoveto{\pgfqpoint{8.700208in}{8.683395in}}%
\pgfpathcurveto{\pgfqpoint{8.711258in}{8.683395in}}{\pgfqpoint{8.721857in}{8.687785in}}{\pgfqpoint{8.729671in}{8.695598in}}%
\pgfpathcurveto{\pgfqpoint{8.737485in}{8.703412in}}{\pgfqpoint{8.741875in}{8.714011in}}{\pgfqpoint{8.741875in}{8.725061in}}%
\pgfpathcurveto{\pgfqpoint{8.741875in}{8.736111in}}{\pgfqpoint{8.737485in}{8.746710in}}{\pgfqpoint{8.729671in}{8.754524in}}%
\pgfpathcurveto{\pgfqpoint{8.721857in}{8.762338in}}{\pgfqpoint{8.711258in}{8.766728in}}{\pgfqpoint{8.700208in}{8.766728in}}%
\pgfpathcurveto{\pgfqpoint{8.689158in}{8.766728in}}{\pgfqpoint{8.678559in}{8.762338in}}{\pgfqpoint{8.670745in}{8.754524in}}%
\pgfpathcurveto{\pgfqpoint{8.662932in}{8.746710in}}{\pgfqpoint{8.658542in}{8.736111in}}{\pgfqpoint{8.658542in}{8.725061in}}%
\pgfpathcurveto{\pgfqpoint{8.658542in}{8.714011in}}{\pgfqpoint{8.662932in}{8.703412in}}{\pgfqpoint{8.670745in}{8.695598in}}%
\pgfpathcurveto{\pgfqpoint{8.678559in}{8.687785in}}{\pgfqpoint{8.689158in}{8.683395in}}{\pgfqpoint{8.700208in}{8.683395in}}%
\pgfpathlineto{\pgfqpoint{8.700208in}{8.683395in}}%
\pgfpathclose%
\pgfusepath{stroke,fill}%
\end{pgfscope}%
\begin{pgfscope}%
\pgfpathrectangle{\pgfqpoint{7.622482in}{7.624184in}}{\pgfqpoint{2.177280in}{2.201755in}}%
\pgfusepath{clip}%
\pgfsetbuttcap%
\pgfsetroundjoin%
\definecolor{currentfill}{rgb}{1.000000,0.498039,0.054902}%
\pgfsetfillcolor{currentfill}%
\pgfsetlinewidth{0.481800pt}%
\definecolor{currentstroke}{rgb}{1.000000,1.000000,1.000000}%
\pgfsetstrokecolor{currentstroke}%
\pgfsetdash{}{0pt}%
\pgfpathmoveto{\pgfqpoint{8.564700in}{8.516595in}}%
\pgfpathcurveto{\pgfqpoint{8.575751in}{8.516595in}}{\pgfqpoint{8.586350in}{8.520985in}}{\pgfqpoint{8.594163in}{8.528799in}}%
\pgfpathcurveto{\pgfqpoint{8.601977in}{8.536612in}}{\pgfqpoint{8.606367in}{8.547212in}}{\pgfqpoint{8.606367in}{8.558262in}}%
\pgfpathcurveto{\pgfqpoint{8.606367in}{8.569312in}}{\pgfqpoint{8.601977in}{8.579911in}}{\pgfqpoint{8.594163in}{8.587724in}}%
\pgfpathcurveto{\pgfqpoint{8.586350in}{8.595538in}}{\pgfqpoint{8.575751in}{8.599928in}}{\pgfqpoint{8.564700in}{8.599928in}}%
\pgfpathcurveto{\pgfqpoint{8.553650in}{8.599928in}}{\pgfqpoint{8.543051in}{8.595538in}}{\pgfqpoint{8.535238in}{8.587724in}}%
\pgfpathcurveto{\pgfqpoint{8.527424in}{8.579911in}}{\pgfqpoint{8.523034in}{8.569312in}}{\pgfqpoint{8.523034in}{8.558262in}}%
\pgfpathcurveto{\pgfqpoint{8.523034in}{8.547212in}}{\pgfqpoint{8.527424in}{8.536612in}}{\pgfqpoint{8.535238in}{8.528799in}}%
\pgfpathcurveto{\pgfqpoint{8.543051in}{8.520985in}}{\pgfqpoint{8.553650in}{8.516595in}}{\pgfqpoint{8.564700in}{8.516595in}}%
\pgfpathlineto{\pgfqpoint{8.564700in}{8.516595in}}%
\pgfpathclose%
\pgfusepath{stroke,fill}%
\end{pgfscope}%
\begin{pgfscope}%
\pgfpathrectangle{\pgfqpoint{7.622482in}{7.624184in}}{\pgfqpoint{2.177280in}{2.201755in}}%
\pgfusepath{clip}%
\pgfsetbuttcap%
\pgfsetroundjoin%
\definecolor{currentfill}{rgb}{1.000000,0.498039,0.054902}%
\pgfsetfillcolor{currentfill}%
\pgfsetlinewidth{0.481800pt}%
\definecolor{currentstroke}{rgb}{1.000000,1.000000,1.000000}%
\pgfsetstrokecolor{currentstroke}%
\pgfsetdash{}{0pt}%
\pgfpathmoveto{\pgfqpoint{8.429193in}{8.071796in}}%
\pgfpathcurveto{\pgfqpoint{8.440243in}{8.071796in}}{\pgfqpoint{8.450842in}{8.076186in}}{\pgfqpoint{8.458656in}{8.084000in}}%
\pgfpathcurveto{\pgfqpoint{8.466469in}{8.091813in}}{\pgfqpoint{8.470859in}{8.102413in}}{\pgfqpoint{8.470859in}{8.113463in}}%
\pgfpathcurveto{\pgfqpoint{8.470859in}{8.124513in}}{\pgfqpoint{8.466469in}{8.135112in}}{\pgfqpoint{8.458656in}{8.142925in}}%
\pgfpathcurveto{\pgfqpoint{8.450842in}{8.150739in}}{\pgfqpoint{8.440243in}{8.155129in}}{\pgfqpoint{8.429193in}{8.155129in}}%
\pgfpathcurveto{\pgfqpoint{8.418143in}{8.155129in}}{\pgfqpoint{8.407544in}{8.150739in}}{\pgfqpoint{8.399730in}{8.142925in}}%
\pgfpathcurveto{\pgfqpoint{8.391916in}{8.135112in}}{\pgfqpoint{8.387526in}{8.124513in}}{\pgfqpoint{8.387526in}{8.113463in}}%
\pgfpathcurveto{\pgfqpoint{8.387526in}{8.102413in}}{\pgfqpoint{8.391916in}{8.091813in}}{\pgfqpoint{8.399730in}{8.084000in}}%
\pgfpathcurveto{\pgfqpoint{8.407544in}{8.076186in}}{\pgfqpoint{8.418143in}{8.071796in}}{\pgfqpoint{8.429193in}{8.071796in}}%
\pgfpathlineto{\pgfqpoint{8.429193in}{8.071796in}}%
\pgfpathclose%
\pgfusepath{stroke,fill}%
\end{pgfscope}%
\begin{pgfscope}%
\pgfpathrectangle{\pgfqpoint{7.622482in}{7.624184in}}{\pgfqpoint{2.177280in}{2.201755in}}%
\pgfusepath{clip}%
\pgfsetbuttcap%
\pgfsetroundjoin%
\definecolor{currentfill}{rgb}{1.000000,0.498039,0.054902}%
\pgfsetfillcolor{currentfill}%
\pgfsetlinewidth{0.481800pt}%
\definecolor{currentstroke}{rgb}{1.000000,1.000000,1.000000}%
\pgfsetstrokecolor{currentstroke}%
\pgfsetdash{}{0pt}%
\pgfpathmoveto{\pgfqpoint{8.632454in}{8.405395in}}%
\pgfpathcurveto{\pgfqpoint{8.643504in}{8.405395in}}{\pgfqpoint{8.654104in}{8.409785in}}{\pgfqpoint{8.661917in}{8.417599in}}%
\pgfpathcurveto{\pgfqpoint{8.669731in}{8.425413in}}{\pgfqpoint{8.674121in}{8.436012in}}{\pgfqpoint{8.674121in}{8.447062in}}%
\pgfpathcurveto{\pgfqpoint{8.674121in}{8.458112in}}{\pgfqpoint{8.669731in}{8.468711in}}{\pgfqpoint{8.661917in}{8.476525in}}%
\pgfpathcurveto{\pgfqpoint{8.654104in}{8.484338in}}{\pgfqpoint{8.643504in}{8.488729in}}{\pgfqpoint{8.632454in}{8.488729in}}%
\pgfpathcurveto{\pgfqpoint{8.621404in}{8.488729in}}{\pgfqpoint{8.610805in}{8.484338in}}{\pgfqpoint{8.602992in}{8.476525in}}%
\pgfpathcurveto{\pgfqpoint{8.595178in}{8.468711in}}{\pgfqpoint{8.590788in}{8.458112in}}{\pgfqpoint{8.590788in}{8.447062in}}%
\pgfpathcurveto{\pgfqpoint{8.590788in}{8.436012in}}{\pgfqpoint{8.595178in}{8.425413in}}{\pgfqpoint{8.602992in}{8.417599in}}%
\pgfpathcurveto{\pgfqpoint{8.610805in}{8.409785in}}{\pgfqpoint{8.621404in}{8.405395in}}{\pgfqpoint{8.632454in}{8.405395in}}%
\pgfpathlineto{\pgfqpoint{8.632454in}{8.405395in}}%
\pgfpathclose%
\pgfusepath{stroke,fill}%
\end{pgfscope}%
\begin{pgfscope}%
\pgfpathrectangle{\pgfqpoint{7.622482in}{7.624184in}}{\pgfqpoint{2.177280in}{2.201755in}}%
\pgfusepath{clip}%
\pgfsetbuttcap%
\pgfsetroundjoin%
\definecolor{currentfill}{rgb}{1.000000,0.498039,0.054902}%
\pgfsetfillcolor{currentfill}%
\pgfsetlinewidth{0.481800pt}%
\definecolor{currentstroke}{rgb}{1.000000,1.000000,1.000000}%
\pgfsetstrokecolor{currentstroke}%
\pgfsetdash{}{0pt}%
\pgfpathmoveto{\pgfqpoint{8.564700in}{8.460995in}}%
\pgfpathcurveto{\pgfqpoint{8.575751in}{8.460995in}}{\pgfqpoint{8.586350in}{8.465385in}}{\pgfqpoint{8.594163in}{8.473199in}}%
\pgfpathcurveto{\pgfqpoint{8.601977in}{8.481013in}}{\pgfqpoint{8.606367in}{8.491612in}}{\pgfqpoint{8.606367in}{8.502662in}}%
\pgfpathcurveto{\pgfqpoint{8.606367in}{8.513712in}}{\pgfqpoint{8.601977in}{8.524311in}}{\pgfqpoint{8.594163in}{8.532125in}}%
\pgfpathcurveto{\pgfqpoint{8.586350in}{8.539938in}}{\pgfqpoint{8.575751in}{8.544328in}}{\pgfqpoint{8.564700in}{8.544328in}}%
\pgfpathcurveto{\pgfqpoint{8.553650in}{8.544328in}}{\pgfqpoint{8.543051in}{8.539938in}}{\pgfqpoint{8.535238in}{8.532125in}}%
\pgfpathcurveto{\pgfqpoint{8.527424in}{8.524311in}}{\pgfqpoint{8.523034in}{8.513712in}}{\pgfqpoint{8.523034in}{8.502662in}}%
\pgfpathcurveto{\pgfqpoint{8.523034in}{8.491612in}}{\pgfqpoint{8.527424in}{8.481013in}}{\pgfqpoint{8.535238in}{8.473199in}}%
\pgfpathcurveto{\pgfqpoint{8.543051in}{8.465385in}}{\pgfqpoint{8.553650in}{8.460995in}}{\pgfqpoint{8.564700in}{8.460995in}}%
\pgfpathlineto{\pgfqpoint{8.564700in}{8.460995in}}%
\pgfpathclose%
\pgfusepath{stroke,fill}%
\end{pgfscope}%
\begin{pgfscope}%
\pgfpathrectangle{\pgfqpoint{7.622482in}{7.624184in}}{\pgfqpoint{2.177280in}{2.201755in}}%
\pgfusepath{clip}%
\pgfsetbuttcap%
\pgfsetroundjoin%
\definecolor{currentfill}{rgb}{1.000000,0.498039,0.054902}%
\pgfsetfillcolor{currentfill}%
\pgfsetlinewidth{0.481800pt}%
\definecolor{currentstroke}{rgb}{1.000000,1.000000,1.000000}%
\pgfsetstrokecolor{currentstroke}%
\pgfsetdash{}{0pt}%
\pgfpathmoveto{\pgfqpoint{8.632454in}{8.460995in}}%
\pgfpathcurveto{\pgfqpoint{8.643504in}{8.460995in}}{\pgfqpoint{8.654104in}{8.465385in}}{\pgfqpoint{8.661917in}{8.473199in}}%
\pgfpathcurveto{\pgfqpoint{8.669731in}{8.481013in}}{\pgfqpoint{8.674121in}{8.491612in}}{\pgfqpoint{8.674121in}{8.502662in}}%
\pgfpathcurveto{\pgfqpoint{8.674121in}{8.513712in}}{\pgfqpoint{8.669731in}{8.524311in}}{\pgfqpoint{8.661917in}{8.532125in}}%
\pgfpathcurveto{\pgfqpoint{8.654104in}{8.539938in}}{\pgfqpoint{8.643504in}{8.544328in}}{\pgfqpoint{8.632454in}{8.544328in}}%
\pgfpathcurveto{\pgfqpoint{8.621404in}{8.544328in}}{\pgfqpoint{8.610805in}{8.539938in}}{\pgfqpoint{8.602992in}{8.532125in}}%
\pgfpathcurveto{\pgfqpoint{8.595178in}{8.524311in}}{\pgfqpoint{8.590788in}{8.513712in}}{\pgfqpoint{8.590788in}{8.502662in}}%
\pgfpathcurveto{\pgfqpoint{8.590788in}{8.491612in}}{\pgfqpoint{8.595178in}{8.481013in}}{\pgfqpoint{8.602992in}{8.473199in}}%
\pgfpathcurveto{\pgfqpoint{8.610805in}{8.465385in}}{\pgfqpoint{8.621404in}{8.460995in}}{\pgfqpoint{8.632454in}{8.460995in}}%
\pgfpathlineto{\pgfqpoint{8.632454in}{8.460995in}}%
\pgfpathclose%
\pgfusepath{stroke,fill}%
\end{pgfscope}%
\begin{pgfscope}%
\pgfpathrectangle{\pgfqpoint{7.622482in}{7.624184in}}{\pgfqpoint{2.177280in}{2.201755in}}%
\pgfusepath{clip}%
\pgfsetbuttcap%
\pgfsetroundjoin%
\definecolor{currentfill}{rgb}{1.000000,0.498039,0.054902}%
\pgfsetfillcolor{currentfill}%
\pgfsetlinewidth{0.481800pt}%
\definecolor{currentstroke}{rgb}{1.000000,1.000000,1.000000}%
\pgfsetstrokecolor{currentstroke}%
\pgfsetdash{}{0pt}%
\pgfpathmoveto{\pgfqpoint{8.632454in}{8.738994in}}%
\pgfpathcurveto{\pgfqpoint{8.643504in}{8.738994in}}{\pgfqpoint{8.654104in}{8.743385in}}{\pgfqpoint{8.661917in}{8.751198in}}%
\pgfpathcurveto{\pgfqpoint{8.669731in}{8.759012in}}{\pgfqpoint{8.674121in}{8.769611in}}{\pgfqpoint{8.674121in}{8.780661in}}%
\pgfpathcurveto{\pgfqpoint{8.674121in}{8.791711in}}{\pgfqpoint{8.669731in}{8.802310in}}{\pgfqpoint{8.661917in}{8.810124in}}%
\pgfpathcurveto{\pgfqpoint{8.654104in}{8.817938in}}{\pgfqpoint{8.643504in}{8.822328in}}{\pgfqpoint{8.632454in}{8.822328in}}%
\pgfpathcurveto{\pgfqpoint{8.621404in}{8.822328in}}{\pgfqpoint{8.610805in}{8.817938in}}{\pgfqpoint{8.602992in}{8.810124in}}%
\pgfpathcurveto{\pgfqpoint{8.595178in}{8.802310in}}{\pgfqpoint{8.590788in}{8.791711in}}{\pgfqpoint{8.590788in}{8.780661in}}%
\pgfpathcurveto{\pgfqpoint{8.590788in}{8.769611in}}{\pgfqpoint{8.595178in}{8.759012in}}{\pgfqpoint{8.602992in}{8.751198in}}%
\pgfpathcurveto{\pgfqpoint{8.610805in}{8.743385in}}{\pgfqpoint{8.621404in}{8.738994in}}{\pgfqpoint{8.632454in}{8.738994in}}%
\pgfpathlineto{\pgfqpoint{8.632454in}{8.738994in}}%
\pgfpathclose%
\pgfusepath{stroke,fill}%
\end{pgfscope}%
\begin{pgfscope}%
\pgfpathrectangle{\pgfqpoint{7.622482in}{7.624184in}}{\pgfqpoint{2.177280in}{2.201755in}}%
\pgfusepath{clip}%
\pgfsetbuttcap%
\pgfsetroundjoin%
\definecolor{currentfill}{rgb}{1.000000,0.498039,0.054902}%
\pgfsetfillcolor{currentfill}%
\pgfsetlinewidth{0.481800pt}%
\definecolor{currentstroke}{rgb}{1.000000,1.000000,1.000000}%
\pgfsetstrokecolor{currentstroke}%
\pgfsetdash{}{0pt}%
\pgfpathmoveto{\pgfqpoint{8.496947in}{8.127396in}}%
\pgfpathcurveto{\pgfqpoint{8.507997in}{8.127396in}}{\pgfqpoint{8.518596in}{8.131786in}}{\pgfqpoint{8.526409in}{8.139600in}}%
\pgfpathcurveto{\pgfqpoint{8.534223in}{8.147413in}}{\pgfqpoint{8.538613in}{8.158012in}}{\pgfqpoint{8.538613in}{8.169063in}}%
\pgfpathcurveto{\pgfqpoint{8.538613in}{8.180113in}}{\pgfqpoint{8.534223in}{8.190712in}}{\pgfqpoint{8.526409in}{8.198525in}}%
\pgfpathcurveto{\pgfqpoint{8.518596in}{8.206339in}}{\pgfqpoint{8.507997in}{8.210729in}}{\pgfqpoint{8.496947in}{8.210729in}}%
\pgfpathcurveto{\pgfqpoint{8.485896in}{8.210729in}}{\pgfqpoint{8.475297in}{8.206339in}}{\pgfqpoint{8.467484in}{8.198525in}}%
\pgfpathcurveto{\pgfqpoint{8.459670in}{8.190712in}}{\pgfqpoint{8.455280in}{8.180113in}}{\pgfqpoint{8.455280in}{8.169063in}}%
\pgfpathcurveto{\pgfqpoint{8.455280in}{8.158012in}}{\pgfqpoint{8.459670in}{8.147413in}}{\pgfqpoint{8.467484in}{8.139600in}}%
\pgfpathcurveto{\pgfqpoint{8.475297in}{8.131786in}}{\pgfqpoint{8.485896in}{8.127396in}}{\pgfqpoint{8.496947in}{8.127396in}}%
\pgfpathlineto{\pgfqpoint{8.496947in}{8.127396in}}%
\pgfpathclose%
\pgfusepath{stroke,fill}%
\end{pgfscope}%
\begin{pgfscope}%
\pgfpathrectangle{\pgfqpoint{7.622482in}{7.624184in}}{\pgfqpoint{2.177280in}{2.201755in}}%
\pgfusepath{clip}%
\pgfsetbuttcap%
\pgfsetroundjoin%
\definecolor{currentfill}{rgb}{1.000000,0.498039,0.054902}%
\pgfsetfillcolor{currentfill}%
\pgfsetlinewidth{0.481800pt}%
\definecolor{currentstroke}{rgb}{1.000000,1.000000,1.000000}%
\pgfsetstrokecolor{currentstroke}%
\pgfsetdash{}{0pt}%
\pgfpathmoveto{\pgfqpoint{8.632454in}{8.460995in}}%
\pgfpathcurveto{\pgfqpoint{8.643504in}{8.460995in}}{\pgfqpoint{8.654104in}{8.465385in}}{\pgfqpoint{8.661917in}{8.473199in}}%
\pgfpathcurveto{\pgfqpoint{8.669731in}{8.481013in}}{\pgfqpoint{8.674121in}{8.491612in}}{\pgfqpoint{8.674121in}{8.502662in}}%
\pgfpathcurveto{\pgfqpoint{8.674121in}{8.513712in}}{\pgfqpoint{8.669731in}{8.524311in}}{\pgfqpoint{8.661917in}{8.532125in}}%
\pgfpathcurveto{\pgfqpoint{8.654104in}{8.539938in}}{\pgfqpoint{8.643504in}{8.544328in}}{\pgfqpoint{8.632454in}{8.544328in}}%
\pgfpathcurveto{\pgfqpoint{8.621404in}{8.544328in}}{\pgfqpoint{8.610805in}{8.539938in}}{\pgfqpoint{8.602992in}{8.532125in}}%
\pgfpathcurveto{\pgfqpoint{8.595178in}{8.524311in}}{\pgfqpoint{8.590788in}{8.513712in}}{\pgfqpoint{8.590788in}{8.502662in}}%
\pgfpathcurveto{\pgfqpoint{8.590788in}{8.491612in}}{\pgfqpoint{8.595178in}{8.481013in}}{\pgfqpoint{8.602992in}{8.473199in}}%
\pgfpathcurveto{\pgfqpoint{8.610805in}{8.465385in}}{\pgfqpoint{8.621404in}{8.460995in}}{\pgfqpoint{8.632454in}{8.460995in}}%
\pgfpathlineto{\pgfqpoint{8.632454in}{8.460995in}}%
\pgfpathclose%
\pgfusepath{stroke,fill}%
\end{pgfscope}%
\begin{pgfscope}%
\pgfpathrectangle{\pgfqpoint{7.622482in}{7.624184in}}{\pgfqpoint{2.177280in}{2.201755in}}%
\pgfusepath{clip}%
\pgfsetbuttcap%
\pgfsetroundjoin%
\definecolor{currentfill}{rgb}{0.172549,0.627451,0.172549}%
\pgfsetfillcolor{currentfill}%
\pgfsetlinewidth{0.481800pt}%
\definecolor{currentstroke}{rgb}{1.000000,1.000000,1.000000}%
\pgfsetstrokecolor{currentstroke}%
\pgfsetdash{}{0pt}%
\pgfpathmoveto{\pgfqpoint{9.445501in}{8.794594in}}%
\pgfpathcurveto{\pgfqpoint{9.456551in}{8.794594in}}{\pgfqpoint{9.467150in}{8.798985in}}{\pgfqpoint{9.474964in}{8.806798in}}%
\pgfpathcurveto{\pgfqpoint{9.482777in}{8.814612in}}{\pgfqpoint{9.487167in}{8.825211in}}{\pgfqpoint{9.487167in}{8.836261in}}%
\pgfpathcurveto{\pgfqpoint{9.487167in}{8.847311in}}{\pgfqpoint{9.482777in}{8.857910in}}{\pgfqpoint{9.474964in}{8.865724in}}%
\pgfpathcurveto{\pgfqpoint{9.467150in}{8.873537in}}{\pgfqpoint{9.456551in}{8.877928in}}{\pgfqpoint{9.445501in}{8.877928in}}%
\pgfpathcurveto{\pgfqpoint{9.434451in}{8.877928in}}{\pgfqpoint{9.423852in}{8.873537in}}{\pgfqpoint{9.416038in}{8.865724in}}%
\pgfpathcurveto{\pgfqpoint{9.408224in}{8.857910in}}{\pgfqpoint{9.403834in}{8.847311in}}{\pgfqpoint{9.403834in}{8.836261in}}%
\pgfpathcurveto{\pgfqpoint{9.403834in}{8.825211in}}{\pgfqpoint{9.408224in}{8.814612in}}{\pgfqpoint{9.416038in}{8.806798in}}%
\pgfpathcurveto{\pgfqpoint{9.423852in}{8.798985in}}{\pgfqpoint{9.434451in}{8.794594in}}{\pgfqpoint{9.445501in}{8.794594in}}%
\pgfpathlineto{\pgfqpoint{9.445501in}{8.794594in}}%
\pgfpathclose%
\pgfusepath{stroke,fill}%
\end{pgfscope}%
\begin{pgfscope}%
\pgfpathrectangle{\pgfqpoint{7.622482in}{7.624184in}}{\pgfqpoint{2.177280in}{2.201755in}}%
\pgfusepath{clip}%
\pgfsetbuttcap%
\pgfsetroundjoin%
\definecolor{currentfill}{rgb}{0.172549,0.627451,0.172549}%
\pgfsetfillcolor{currentfill}%
\pgfsetlinewidth{0.481800pt}%
\definecolor{currentstroke}{rgb}{1.000000,1.000000,1.000000}%
\pgfsetstrokecolor{currentstroke}%
\pgfsetdash{}{0pt}%
\pgfpathmoveto{\pgfqpoint{9.038978in}{8.516595in}}%
\pgfpathcurveto{\pgfqpoint{9.050028in}{8.516595in}}{\pgfqpoint{9.060627in}{8.520985in}}{\pgfqpoint{9.068440in}{8.528799in}}%
\pgfpathcurveto{\pgfqpoint{9.076254in}{8.536612in}}{\pgfqpoint{9.080644in}{8.547212in}}{\pgfqpoint{9.080644in}{8.558262in}}%
\pgfpathcurveto{\pgfqpoint{9.080644in}{8.569312in}}{\pgfqpoint{9.076254in}{8.579911in}}{\pgfqpoint{9.068440in}{8.587724in}}%
\pgfpathcurveto{\pgfqpoint{9.060627in}{8.595538in}}{\pgfqpoint{9.050028in}{8.599928in}}{\pgfqpoint{9.038978in}{8.599928in}}%
\pgfpathcurveto{\pgfqpoint{9.027927in}{8.599928in}}{\pgfqpoint{9.017328in}{8.595538in}}{\pgfqpoint{9.009515in}{8.587724in}}%
\pgfpathcurveto{\pgfqpoint{9.001701in}{8.579911in}}{\pgfqpoint{8.997311in}{8.569312in}}{\pgfqpoint{8.997311in}{8.558262in}}%
\pgfpathcurveto{\pgfqpoint{8.997311in}{8.547212in}}{\pgfqpoint{9.001701in}{8.536612in}}{\pgfqpoint{9.009515in}{8.528799in}}%
\pgfpathcurveto{\pgfqpoint{9.017328in}{8.520985in}}{\pgfqpoint{9.027927in}{8.516595in}}{\pgfqpoint{9.038978in}{8.516595in}}%
\pgfpathlineto{\pgfqpoint{9.038978in}{8.516595in}}%
\pgfpathclose%
\pgfusepath{stroke,fill}%
\end{pgfscope}%
\begin{pgfscope}%
\pgfpathrectangle{\pgfqpoint{7.622482in}{7.624184in}}{\pgfqpoint{2.177280in}{2.201755in}}%
\pgfusepath{clip}%
\pgfsetbuttcap%
\pgfsetroundjoin%
\definecolor{currentfill}{rgb}{0.172549,0.627451,0.172549}%
\pgfsetfillcolor{currentfill}%
\pgfsetlinewidth{0.481800pt}%
\definecolor{currentstroke}{rgb}{1.000000,1.000000,1.000000}%
\pgfsetstrokecolor{currentstroke}%
\pgfsetdash{}{0pt}%
\pgfpathmoveto{\pgfqpoint{9.174485in}{9.239393in}}%
\pgfpathcurveto{\pgfqpoint{9.185535in}{9.239393in}}{\pgfqpoint{9.196134in}{9.243784in}}{\pgfqpoint{9.203948in}{9.251597in}}%
\pgfpathcurveto{\pgfqpoint{9.211762in}{9.259411in}}{\pgfqpoint{9.216152in}{9.270010in}}{\pgfqpoint{9.216152in}{9.281060in}}%
\pgfpathcurveto{\pgfqpoint{9.216152in}{9.292110in}}{\pgfqpoint{9.211762in}{9.302709in}}{\pgfqpoint{9.203948in}{9.310523in}}%
\pgfpathcurveto{\pgfqpoint{9.196134in}{9.318336in}}{\pgfqpoint{9.185535in}{9.322727in}}{\pgfqpoint{9.174485in}{9.322727in}}%
\pgfpathcurveto{\pgfqpoint{9.163435in}{9.322727in}}{\pgfqpoint{9.152836in}{9.318336in}}{\pgfqpoint{9.145023in}{9.310523in}}%
\pgfpathcurveto{\pgfqpoint{9.137209in}{9.302709in}}{\pgfqpoint{9.132819in}{9.292110in}}{\pgfqpoint{9.132819in}{9.281060in}}%
\pgfpathcurveto{\pgfqpoint{9.132819in}{9.270010in}}{\pgfqpoint{9.137209in}{9.259411in}}{\pgfqpoint{9.145023in}{9.251597in}}%
\pgfpathcurveto{\pgfqpoint{9.152836in}{9.243784in}}{\pgfqpoint{9.163435in}{9.239393in}}{\pgfqpoint{9.174485in}{9.239393in}}%
\pgfpathlineto{\pgfqpoint{9.174485in}{9.239393in}}%
\pgfpathclose%
\pgfusepath{stroke,fill}%
\end{pgfscope}%
\begin{pgfscope}%
\pgfpathrectangle{\pgfqpoint{7.622482in}{7.624184in}}{\pgfqpoint{2.177280in}{2.201755in}}%
\pgfusepath{clip}%
\pgfsetbuttcap%
\pgfsetroundjoin%
\definecolor{currentfill}{rgb}{0.172549,0.627451,0.172549}%
\pgfsetfillcolor{currentfill}%
\pgfsetlinewidth{0.481800pt}%
\definecolor{currentstroke}{rgb}{1.000000,1.000000,1.000000}%
\pgfsetstrokecolor{currentstroke}%
\pgfsetdash{}{0pt}%
\pgfpathmoveto{\pgfqpoint{8.971224in}{8.794594in}}%
\pgfpathcurveto{\pgfqpoint{8.982274in}{8.794594in}}{\pgfqpoint{8.992873in}{8.798985in}}{\pgfqpoint{9.000686in}{8.806798in}}%
\pgfpathcurveto{\pgfqpoint{9.008500in}{8.814612in}}{\pgfqpoint{9.012890in}{8.825211in}}{\pgfqpoint{9.012890in}{8.836261in}}%
\pgfpathcurveto{\pgfqpoint{9.012890in}{8.847311in}}{\pgfqpoint{9.008500in}{8.857910in}}{\pgfqpoint{9.000686in}{8.865724in}}%
\pgfpathcurveto{\pgfqpoint{8.992873in}{8.873537in}}{\pgfqpoint{8.982274in}{8.877928in}}{\pgfqpoint{8.971224in}{8.877928in}}%
\pgfpathcurveto{\pgfqpoint{8.960174in}{8.877928in}}{\pgfqpoint{8.949575in}{8.873537in}}{\pgfqpoint{8.941761in}{8.865724in}}%
\pgfpathcurveto{\pgfqpoint{8.933947in}{8.857910in}}{\pgfqpoint{8.929557in}{8.847311in}}{\pgfqpoint{8.929557in}{8.836261in}}%
\pgfpathcurveto{\pgfqpoint{8.929557in}{8.825211in}}{\pgfqpoint{8.933947in}{8.814612in}}{\pgfqpoint{8.941761in}{8.806798in}}%
\pgfpathcurveto{\pgfqpoint{8.949575in}{8.798985in}}{\pgfqpoint{8.960174in}{8.794594in}}{\pgfqpoint{8.971224in}{8.794594in}}%
\pgfpathlineto{\pgfqpoint{8.971224in}{8.794594in}}%
\pgfpathclose%
\pgfusepath{stroke,fill}%
\end{pgfscope}%
\begin{pgfscope}%
\pgfpathrectangle{\pgfqpoint{7.622482in}{7.624184in}}{\pgfqpoint{2.177280in}{2.201755in}}%
\pgfusepath{clip}%
\pgfsetbuttcap%
\pgfsetroundjoin%
\definecolor{currentfill}{rgb}{0.172549,0.627451,0.172549}%
\pgfsetfillcolor{currentfill}%
\pgfsetlinewidth{0.481800pt}%
\definecolor{currentstroke}{rgb}{1.000000,1.000000,1.000000}%
\pgfsetstrokecolor{currentstroke}%
\pgfsetdash{}{0pt}%
\pgfpathmoveto{\pgfqpoint{9.242239in}{8.905794in}}%
\pgfpathcurveto{\pgfqpoint{9.253289in}{8.905794in}}{\pgfqpoint{9.263888in}{8.910184in}}{\pgfqpoint{9.271702in}{8.917998in}}%
\pgfpathcurveto{\pgfqpoint{9.279516in}{8.925812in}}{\pgfqpoint{9.283906in}{8.936411in}}{\pgfqpoint{9.283906in}{8.947461in}}%
\pgfpathcurveto{\pgfqpoint{9.283906in}{8.958511in}}{\pgfqpoint{9.279516in}{8.969110in}}{\pgfqpoint{9.271702in}{8.976924in}}%
\pgfpathcurveto{\pgfqpoint{9.263888in}{8.984737in}}{\pgfqpoint{9.253289in}{8.989127in}}{\pgfqpoint{9.242239in}{8.989127in}}%
\pgfpathcurveto{\pgfqpoint{9.231189in}{8.989127in}}{\pgfqpoint{9.220590in}{8.984737in}}{\pgfqpoint{9.212776in}{8.976924in}}%
\pgfpathcurveto{\pgfqpoint{9.204963in}{8.969110in}}{\pgfqpoint{9.200573in}{8.958511in}}{\pgfqpoint{9.200573in}{8.947461in}}%
\pgfpathcurveto{\pgfqpoint{9.200573in}{8.936411in}}{\pgfqpoint{9.204963in}{8.925812in}}{\pgfqpoint{9.212776in}{8.917998in}}%
\pgfpathcurveto{\pgfqpoint{9.220590in}{8.910184in}}{\pgfqpoint{9.231189in}{8.905794in}}{\pgfqpoint{9.242239in}{8.905794in}}%
\pgfpathlineto{\pgfqpoint{9.242239in}{8.905794in}}%
\pgfpathclose%
\pgfusepath{stroke,fill}%
\end{pgfscope}%
\begin{pgfscope}%
\pgfpathrectangle{\pgfqpoint{7.622482in}{7.624184in}}{\pgfqpoint{2.177280in}{2.201755in}}%
\pgfusepath{clip}%
\pgfsetbuttcap%
\pgfsetroundjoin%
\definecolor{currentfill}{rgb}{0.172549,0.627451,0.172549}%
\pgfsetfillcolor{currentfill}%
\pgfsetlinewidth{0.481800pt}%
\definecolor{currentstroke}{rgb}{1.000000,1.000000,1.000000}%
\pgfsetstrokecolor{currentstroke}%
\pgfsetdash{}{0pt}%
\pgfpathmoveto{\pgfqpoint{9.174485in}{9.517393in}}%
\pgfpathcurveto{\pgfqpoint{9.185535in}{9.517393in}}{\pgfqpoint{9.196134in}{9.521783in}}{\pgfqpoint{9.203948in}{9.529597in}}%
\pgfpathcurveto{\pgfqpoint{9.211762in}{9.537410in}}{\pgfqpoint{9.216152in}{9.548009in}}{\pgfqpoint{9.216152in}{9.559059in}}%
\pgfpathcurveto{\pgfqpoint{9.216152in}{9.570110in}}{\pgfqpoint{9.211762in}{9.580709in}}{\pgfqpoint{9.203948in}{9.588522in}}%
\pgfpathcurveto{\pgfqpoint{9.196134in}{9.596336in}}{\pgfqpoint{9.185535in}{9.600726in}}{\pgfqpoint{9.174485in}{9.600726in}}%
\pgfpathcurveto{\pgfqpoint{9.163435in}{9.600726in}}{\pgfqpoint{9.152836in}{9.596336in}}{\pgfqpoint{9.145023in}{9.588522in}}%
\pgfpathcurveto{\pgfqpoint{9.137209in}{9.580709in}}{\pgfqpoint{9.132819in}{9.570110in}}{\pgfqpoint{9.132819in}{9.559059in}}%
\pgfpathcurveto{\pgfqpoint{9.132819in}{9.548009in}}{\pgfqpoint{9.137209in}{9.537410in}}{\pgfqpoint{9.145023in}{9.529597in}}%
\pgfpathcurveto{\pgfqpoint{9.152836in}{9.521783in}}{\pgfqpoint{9.163435in}{9.517393in}}{\pgfqpoint{9.174485in}{9.517393in}}%
\pgfpathlineto{\pgfqpoint{9.174485in}{9.517393in}}%
\pgfpathclose%
\pgfusepath{stroke,fill}%
\end{pgfscope}%
\begin{pgfscope}%
\pgfpathrectangle{\pgfqpoint{7.622482in}{7.624184in}}{\pgfqpoint{2.177280in}{2.201755in}}%
\pgfusepath{clip}%
\pgfsetbuttcap%
\pgfsetroundjoin%
\definecolor{currentfill}{rgb}{0.172549,0.627451,0.172549}%
\pgfsetfillcolor{currentfill}%
\pgfsetlinewidth{0.481800pt}%
\definecolor{currentstroke}{rgb}{1.000000,1.000000,1.000000}%
\pgfsetstrokecolor{currentstroke}%
\pgfsetdash{}{0pt}%
\pgfpathmoveto{\pgfqpoint{8.903470in}{8.016196in}}%
\pgfpathcurveto{\pgfqpoint{8.914520in}{8.016196in}}{\pgfqpoint{8.925119in}{8.020586in}}{\pgfqpoint{8.932933in}{8.028400in}}%
\pgfpathcurveto{\pgfqpoint{8.940746in}{8.036214in}}{\pgfqpoint{8.945136in}{8.046813in}}{\pgfqpoint{8.945136in}{8.057863in}}%
\pgfpathcurveto{\pgfqpoint{8.945136in}{8.068913in}}{\pgfqpoint{8.940746in}{8.079512in}}{\pgfqpoint{8.932933in}{8.087326in}}%
\pgfpathcurveto{\pgfqpoint{8.925119in}{8.095139in}}{\pgfqpoint{8.914520in}{8.099529in}}{\pgfqpoint{8.903470in}{8.099529in}}%
\pgfpathcurveto{\pgfqpoint{8.892420in}{8.099529in}}{\pgfqpoint{8.881821in}{8.095139in}}{\pgfqpoint{8.874007in}{8.087326in}}%
\pgfpathcurveto{\pgfqpoint{8.866193in}{8.079512in}}{\pgfqpoint{8.861803in}{8.068913in}}{\pgfqpoint{8.861803in}{8.057863in}}%
\pgfpathcurveto{\pgfqpoint{8.861803in}{8.046813in}}{\pgfqpoint{8.866193in}{8.036214in}}{\pgfqpoint{8.874007in}{8.028400in}}%
\pgfpathcurveto{\pgfqpoint{8.881821in}{8.020586in}}{\pgfqpoint{8.892420in}{8.016196in}}{\pgfqpoint{8.903470in}{8.016196in}}%
\pgfpathlineto{\pgfqpoint{8.903470in}{8.016196in}}%
\pgfpathclose%
\pgfusepath{stroke,fill}%
\end{pgfscope}%
\begin{pgfscope}%
\pgfpathrectangle{\pgfqpoint{7.622482in}{7.624184in}}{\pgfqpoint{2.177280in}{2.201755in}}%
\pgfusepath{clip}%
\pgfsetbuttcap%
\pgfsetroundjoin%
\definecolor{currentfill}{rgb}{0.172549,0.627451,0.172549}%
\pgfsetfillcolor{currentfill}%
\pgfsetlinewidth{0.481800pt}%
\definecolor{currentstroke}{rgb}{1.000000,1.000000,1.000000}%
\pgfsetstrokecolor{currentstroke}%
\pgfsetdash{}{0pt}%
\pgfpathmoveto{\pgfqpoint{8.971224in}{9.350593in}}%
\pgfpathcurveto{\pgfqpoint{8.982274in}{9.350593in}}{\pgfqpoint{8.992873in}{9.354983in}}{\pgfqpoint{9.000686in}{9.362797in}}%
\pgfpathcurveto{\pgfqpoint{9.008500in}{9.370611in}}{\pgfqpoint{9.012890in}{9.381210in}}{\pgfqpoint{9.012890in}{9.392260in}}%
\pgfpathcurveto{\pgfqpoint{9.012890in}{9.403310in}}{\pgfqpoint{9.008500in}{9.413909in}}{\pgfqpoint{9.000686in}{9.421723in}}%
\pgfpathcurveto{\pgfqpoint{8.992873in}{9.429536in}}{\pgfqpoint{8.982274in}{9.433926in}}{\pgfqpoint{8.971224in}{9.433926in}}%
\pgfpathcurveto{\pgfqpoint{8.960174in}{9.433926in}}{\pgfqpoint{8.949575in}{9.429536in}}{\pgfqpoint{8.941761in}{9.421723in}}%
\pgfpathcurveto{\pgfqpoint{8.933947in}{9.413909in}}{\pgfqpoint{8.929557in}{9.403310in}}{\pgfqpoint{8.929557in}{9.392260in}}%
\pgfpathcurveto{\pgfqpoint{8.929557in}{9.381210in}}{\pgfqpoint{8.933947in}{9.370611in}}{\pgfqpoint{8.941761in}{9.362797in}}%
\pgfpathcurveto{\pgfqpoint{8.949575in}{9.354983in}}{\pgfqpoint{8.960174in}{9.350593in}}{\pgfqpoint{8.971224in}{9.350593in}}%
\pgfpathlineto{\pgfqpoint{8.971224in}{9.350593in}}%
\pgfpathclose%
\pgfusepath{stroke,fill}%
\end{pgfscope}%
\begin{pgfscope}%
\pgfpathrectangle{\pgfqpoint{7.622482in}{7.624184in}}{\pgfqpoint{2.177280in}{2.201755in}}%
\pgfusepath{clip}%
\pgfsetbuttcap%
\pgfsetroundjoin%
\definecolor{currentfill}{rgb}{0.172549,0.627451,0.172549}%
\pgfsetfillcolor{currentfill}%
\pgfsetlinewidth{0.481800pt}%
\definecolor{currentstroke}{rgb}{1.000000,1.000000,1.000000}%
\pgfsetstrokecolor{currentstroke}%
\pgfsetdash{}{0pt}%
\pgfpathmoveto{\pgfqpoint{8.971224in}{9.016994in}}%
\pgfpathcurveto{\pgfqpoint{8.982274in}{9.016994in}}{\pgfqpoint{8.992873in}{9.021384in}}{\pgfqpoint{9.000686in}{9.029198in}}%
\pgfpathcurveto{\pgfqpoint{9.008500in}{9.037011in}}{\pgfqpoint{9.012890in}{9.047610in}}{\pgfqpoint{9.012890in}{9.058661in}}%
\pgfpathcurveto{\pgfqpoint{9.012890in}{9.069711in}}{\pgfqpoint{9.008500in}{9.080310in}}{\pgfqpoint{9.000686in}{9.088123in}}%
\pgfpathcurveto{\pgfqpoint{8.992873in}{9.095937in}}{\pgfqpoint{8.982274in}{9.100327in}}{\pgfqpoint{8.971224in}{9.100327in}}%
\pgfpathcurveto{\pgfqpoint{8.960174in}{9.100327in}}{\pgfqpoint{8.949575in}{9.095937in}}{\pgfqpoint{8.941761in}{9.088123in}}%
\pgfpathcurveto{\pgfqpoint{8.933947in}{9.080310in}}{\pgfqpoint{8.929557in}{9.069711in}}{\pgfqpoint{8.929557in}{9.058661in}}%
\pgfpathcurveto{\pgfqpoint{8.929557in}{9.047610in}}{\pgfqpoint{8.933947in}{9.037011in}}{\pgfqpoint{8.941761in}{9.029198in}}%
\pgfpathcurveto{\pgfqpoint{8.949575in}{9.021384in}}{\pgfqpoint{8.960174in}{9.016994in}}{\pgfqpoint{8.971224in}{9.016994in}}%
\pgfpathlineto{\pgfqpoint{8.971224in}{9.016994in}}%
\pgfpathclose%
\pgfusepath{stroke,fill}%
\end{pgfscope}%
\begin{pgfscope}%
\pgfpathrectangle{\pgfqpoint{7.622482in}{7.624184in}}{\pgfqpoint{2.177280in}{2.201755in}}%
\pgfusepath{clip}%
\pgfsetbuttcap%
\pgfsetroundjoin%
\definecolor{currentfill}{rgb}{0.172549,0.627451,0.172549}%
\pgfsetfillcolor{currentfill}%
\pgfsetlinewidth{0.481800pt}%
\definecolor{currentstroke}{rgb}{1.000000,1.000000,1.000000}%
\pgfsetstrokecolor{currentstroke}%
\pgfsetdash{}{0pt}%
\pgfpathmoveto{\pgfqpoint{9.445501in}{9.294993in}}%
\pgfpathcurveto{\pgfqpoint{9.456551in}{9.294993in}}{\pgfqpoint{9.467150in}{9.299384in}}{\pgfqpoint{9.474964in}{9.307197in}}%
\pgfpathcurveto{\pgfqpoint{9.482777in}{9.315011in}}{\pgfqpoint{9.487167in}{9.325610in}}{\pgfqpoint{9.487167in}{9.336660in}}%
\pgfpathcurveto{\pgfqpoint{9.487167in}{9.347710in}}{\pgfqpoint{9.482777in}{9.358309in}}{\pgfqpoint{9.474964in}{9.366123in}}%
\pgfpathcurveto{\pgfqpoint{9.467150in}{9.373936in}}{\pgfqpoint{9.456551in}{9.378327in}}{\pgfqpoint{9.445501in}{9.378327in}}%
\pgfpathcurveto{\pgfqpoint{9.434451in}{9.378327in}}{\pgfqpoint{9.423852in}{9.373936in}}{\pgfqpoint{9.416038in}{9.366123in}}%
\pgfpathcurveto{\pgfqpoint{9.408224in}{9.358309in}}{\pgfqpoint{9.403834in}{9.347710in}}{\pgfqpoint{9.403834in}{9.336660in}}%
\pgfpathcurveto{\pgfqpoint{9.403834in}{9.325610in}}{\pgfqpoint{9.408224in}{9.315011in}}{\pgfqpoint{9.416038in}{9.307197in}}%
\pgfpathcurveto{\pgfqpoint{9.423852in}{9.299384in}}{\pgfqpoint{9.434451in}{9.294993in}}{\pgfqpoint{9.445501in}{9.294993in}}%
\pgfpathlineto{\pgfqpoint{9.445501in}{9.294993in}}%
\pgfpathclose%
\pgfusepath{stroke,fill}%
\end{pgfscope}%
\begin{pgfscope}%
\pgfpathrectangle{\pgfqpoint{7.622482in}{7.624184in}}{\pgfqpoint{2.177280in}{2.201755in}}%
\pgfusepath{clip}%
\pgfsetbuttcap%
\pgfsetroundjoin%
\definecolor{currentfill}{rgb}{0.172549,0.627451,0.172549}%
\pgfsetfillcolor{currentfill}%
\pgfsetlinewidth{0.481800pt}%
\definecolor{currentstroke}{rgb}{1.000000,1.000000,1.000000}%
\pgfsetstrokecolor{currentstroke}%
\pgfsetdash{}{0pt}%
\pgfpathmoveto{\pgfqpoint{9.106731in}{8.905794in}}%
\pgfpathcurveto{\pgfqpoint{9.117782in}{8.905794in}}{\pgfqpoint{9.128381in}{8.910184in}}{\pgfqpoint{9.136194in}{8.917998in}}%
\pgfpathcurveto{\pgfqpoint{9.144008in}{8.925812in}}{\pgfqpoint{9.148398in}{8.936411in}}{\pgfqpoint{9.148398in}{8.947461in}}%
\pgfpathcurveto{\pgfqpoint{9.148398in}{8.958511in}}{\pgfqpoint{9.144008in}{8.969110in}}{\pgfqpoint{9.136194in}{8.976924in}}%
\pgfpathcurveto{\pgfqpoint{9.128381in}{8.984737in}}{\pgfqpoint{9.117782in}{8.989127in}}{\pgfqpoint{9.106731in}{8.989127in}}%
\pgfpathcurveto{\pgfqpoint{9.095681in}{8.989127in}}{\pgfqpoint{9.085082in}{8.984737in}}{\pgfqpoint{9.077269in}{8.976924in}}%
\pgfpathcurveto{\pgfqpoint{9.069455in}{8.969110in}}{\pgfqpoint{9.065065in}{8.958511in}}{\pgfqpoint{9.065065in}{8.947461in}}%
\pgfpathcurveto{\pgfqpoint{9.065065in}{8.936411in}}{\pgfqpoint{9.069455in}{8.925812in}}{\pgfqpoint{9.077269in}{8.917998in}}%
\pgfpathcurveto{\pgfqpoint{9.085082in}{8.910184in}}{\pgfqpoint{9.095681in}{8.905794in}}{\pgfqpoint{9.106731in}{8.905794in}}%
\pgfpathlineto{\pgfqpoint{9.106731in}{8.905794in}}%
\pgfpathclose%
\pgfusepath{stroke,fill}%
\end{pgfscope}%
\begin{pgfscope}%
\pgfpathrectangle{\pgfqpoint{7.622482in}{7.624184in}}{\pgfqpoint{2.177280in}{2.201755in}}%
\pgfusepath{clip}%
\pgfsetbuttcap%
\pgfsetroundjoin%
\definecolor{currentfill}{rgb}{0.172549,0.627451,0.172549}%
\pgfsetfillcolor{currentfill}%
\pgfsetlinewidth{0.481800pt}%
\definecolor{currentstroke}{rgb}{1.000000,1.000000,1.000000}%
\pgfsetstrokecolor{currentstroke}%
\pgfsetdash{}{0pt}%
\pgfpathmoveto{\pgfqpoint{9.038978in}{8.850194in}}%
\pgfpathcurveto{\pgfqpoint{9.050028in}{8.850194in}}{\pgfqpoint{9.060627in}{8.854584in}}{\pgfqpoint{9.068440in}{8.862398in}}%
\pgfpathcurveto{\pgfqpoint{9.076254in}{8.870212in}}{\pgfqpoint{9.080644in}{8.880811in}}{\pgfqpoint{9.080644in}{8.891861in}}%
\pgfpathcurveto{\pgfqpoint{9.080644in}{8.902911in}}{\pgfqpoint{9.076254in}{8.913510in}}{\pgfqpoint{9.068440in}{8.921324in}}%
\pgfpathcurveto{\pgfqpoint{9.060627in}{8.929137in}}{\pgfqpoint{9.050028in}{8.933528in}}{\pgfqpoint{9.038978in}{8.933528in}}%
\pgfpathcurveto{\pgfqpoint{9.027927in}{8.933528in}}{\pgfqpoint{9.017328in}{8.929137in}}{\pgfqpoint{9.009515in}{8.921324in}}%
\pgfpathcurveto{\pgfqpoint{9.001701in}{8.913510in}}{\pgfqpoint{8.997311in}{8.902911in}}{\pgfqpoint{8.997311in}{8.891861in}}%
\pgfpathcurveto{\pgfqpoint{8.997311in}{8.880811in}}{\pgfqpoint{9.001701in}{8.870212in}}{\pgfqpoint{9.009515in}{8.862398in}}%
\pgfpathcurveto{\pgfqpoint{9.017328in}{8.854584in}}{\pgfqpoint{9.027927in}{8.850194in}}{\pgfqpoint{9.038978in}{8.850194in}}%
\pgfpathlineto{\pgfqpoint{9.038978in}{8.850194in}}%
\pgfpathclose%
\pgfusepath{stroke,fill}%
\end{pgfscope}%
\begin{pgfscope}%
\pgfpathrectangle{\pgfqpoint{7.622482in}{7.624184in}}{\pgfqpoint{2.177280in}{2.201755in}}%
\pgfusepath{clip}%
\pgfsetbuttcap%
\pgfsetroundjoin%
\definecolor{currentfill}{rgb}{0.172549,0.627451,0.172549}%
\pgfsetfillcolor{currentfill}%
\pgfsetlinewidth{0.481800pt}%
\definecolor{currentstroke}{rgb}{1.000000,1.000000,1.000000}%
\pgfsetstrokecolor{currentstroke}%
\pgfsetdash{}{0pt}%
\pgfpathmoveto{\pgfqpoint{9.174485in}{9.072594in}}%
\pgfpathcurveto{\pgfqpoint{9.185535in}{9.072594in}}{\pgfqpoint{9.196134in}{9.076984in}}{\pgfqpoint{9.203948in}{9.084798in}}%
\pgfpathcurveto{\pgfqpoint{9.211762in}{9.092611in}}{\pgfqpoint{9.216152in}{9.103210in}}{\pgfqpoint{9.216152in}{9.114260in}}%
\pgfpathcurveto{\pgfqpoint{9.216152in}{9.125311in}}{\pgfqpoint{9.211762in}{9.135910in}}{\pgfqpoint{9.203948in}{9.143723in}}%
\pgfpathcurveto{\pgfqpoint{9.196134in}{9.151537in}}{\pgfqpoint{9.185535in}{9.155927in}}{\pgfqpoint{9.174485in}{9.155927in}}%
\pgfpathcurveto{\pgfqpoint{9.163435in}{9.155927in}}{\pgfqpoint{9.152836in}{9.151537in}}{\pgfqpoint{9.145023in}{9.143723in}}%
\pgfpathcurveto{\pgfqpoint{9.137209in}{9.135910in}}{\pgfqpoint{9.132819in}{9.125311in}}{\pgfqpoint{9.132819in}{9.114260in}}%
\pgfpathcurveto{\pgfqpoint{9.132819in}{9.103210in}}{\pgfqpoint{9.137209in}{9.092611in}}{\pgfqpoint{9.145023in}{9.084798in}}%
\pgfpathcurveto{\pgfqpoint{9.152836in}{9.076984in}}{\pgfqpoint{9.163435in}{9.072594in}}{\pgfqpoint{9.174485in}{9.072594in}}%
\pgfpathlineto{\pgfqpoint{9.174485in}{9.072594in}}%
\pgfpathclose%
\pgfusepath{stroke,fill}%
\end{pgfscope}%
\begin{pgfscope}%
\pgfpathrectangle{\pgfqpoint{7.622482in}{7.624184in}}{\pgfqpoint{2.177280in}{2.201755in}}%
\pgfusepath{clip}%
\pgfsetbuttcap%
\pgfsetroundjoin%
\definecolor{currentfill}{rgb}{0.172549,0.627451,0.172549}%
\pgfsetfillcolor{currentfill}%
\pgfsetlinewidth{0.481800pt}%
\definecolor{currentstroke}{rgb}{1.000000,1.000000,1.000000}%
\pgfsetstrokecolor{currentstroke}%
\pgfsetdash{}{0pt}%
\pgfpathmoveto{\pgfqpoint{9.106731in}{8.460995in}}%
\pgfpathcurveto{\pgfqpoint{9.117782in}{8.460995in}}{\pgfqpoint{9.128381in}{8.465385in}}{\pgfqpoint{9.136194in}{8.473199in}}%
\pgfpathcurveto{\pgfqpoint{9.144008in}{8.481013in}}{\pgfqpoint{9.148398in}{8.491612in}}{\pgfqpoint{9.148398in}{8.502662in}}%
\pgfpathcurveto{\pgfqpoint{9.148398in}{8.513712in}}{\pgfqpoint{9.144008in}{8.524311in}}{\pgfqpoint{9.136194in}{8.532125in}}%
\pgfpathcurveto{\pgfqpoint{9.128381in}{8.539938in}}{\pgfqpoint{9.117782in}{8.544328in}}{\pgfqpoint{9.106731in}{8.544328in}}%
\pgfpathcurveto{\pgfqpoint{9.095681in}{8.544328in}}{\pgfqpoint{9.085082in}{8.539938in}}{\pgfqpoint{9.077269in}{8.532125in}}%
\pgfpathcurveto{\pgfqpoint{9.069455in}{8.524311in}}{\pgfqpoint{9.065065in}{8.513712in}}{\pgfqpoint{9.065065in}{8.502662in}}%
\pgfpathcurveto{\pgfqpoint{9.065065in}{8.491612in}}{\pgfqpoint{9.069455in}{8.481013in}}{\pgfqpoint{9.077269in}{8.473199in}}%
\pgfpathcurveto{\pgfqpoint{9.085082in}{8.465385in}}{\pgfqpoint{9.095681in}{8.460995in}}{\pgfqpoint{9.106731in}{8.460995in}}%
\pgfpathlineto{\pgfqpoint{9.106731in}{8.460995in}}%
\pgfpathclose%
\pgfusepath{stroke,fill}%
\end{pgfscope}%
\begin{pgfscope}%
\pgfpathrectangle{\pgfqpoint{7.622482in}{7.624184in}}{\pgfqpoint{2.177280in}{2.201755in}}%
\pgfusepath{clip}%
\pgfsetbuttcap%
\pgfsetroundjoin%
\definecolor{currentfill}{rgb}{0.172549,0.627451,0.172549}%
\pgfsetfillcolor{currentfill}%
\pgfsetlinewidth{0.481800pt}%
\definecolor{currentstroke}{rgb}{1.000000,1.000000,1.000000}%
\pgfsetstrokecolor{currentstroke}%
\pgfsetdash{}{0pt}%
\pgfpathmoveto{\pgfqpoint{9.377747in}{8.516595in}}%
\pgfpathcurveto{\pgfqpoint{9.388797in}{8.516595in}}{\pgfqpoint{9.399396in}{8.520985in}}{\pgfqpoint{9.407210in}{8.528799in}}%
\pgfpathcurveto{\pgfqpoint{9.415023in}{8.536612in}}{\pgfqpoint{9.419414in}{8.547212in}}{\pgfqpoint{9.419414in}{8.558262in}}%
\pgfpathcurveto{\pgfqpoint{9.419414in}{8.569312in}}{\pgfqpoint{9.415023in}{8.579911in}}{\pgfqpoint{9.407210in}{8.587724in}}%
\pgfpathcurveto{\pgfqpoint{9.399396in}{8.595538in}}{\pgfqpoint{9.388797in}{8.599928in}}{\pgfqpoint{9.377747in}{8.599928in}}%
\pgfpathcurveto{\pgfqpoint{9.366697in}{8.599928in}}{\pgfqpoint{9.356098in}{8.595538in}}{\pgfqpoint{9.348284in}{8.587724in}}%
\pgfpathcurveto{\pgfqpoint{9.340471in}{8.579911in}}{\pgfqpoint{9.336080in}{8.569312in}}{\pgfqpoint{9.336080in}{8.558262in}}%
\pgfpathcurveto{\pgfqpoint{9.336080in}{8.547212in}}{\pgfqpoint{9.340471in}{8.536612in}}{\pgfqpoint{9.348284in}{8.528799in}}%
\pgfpathcurveto{\pgfqpoint{9.356098in}{8.520985in}}{\pgfqpoint{9.366697in}{8.516595in}}{\pgfqpoint{9.377747in}{8.516595in}}%
\pgfpathlineto{\pgfqpoint{9.377747in}{8.516595in}}%
\pgfpathclose%
\pgfusepath{stroke,fill}%
\end{pgfscope}%
\begin{pgfscope}%
\pgfpathrectangle{\pgfqpoint{7.622482in}{7.624184in}}{\pgfqpoint{2.177280in}{2.201755in}}%
\pgfusepath{clip}%
\pgfsetbuttcap%
\pgfsetroundjoin%
\definecolor{currentfill}{rgb}{0.172549,0.627451,0.172549}%
\pgfsetfillcolor{currentfill}%
\pgfsetlinewidth{0.481800pt}%
\definecolor{currentstroke}{rgb}{1.000000,1.000000,1.000000}%
\pgfsetstrokecolor{currentstroke}%
\pgfsetdash{}{0pt}%
\pgfpathmoveto{\pgfqpoint{9.309993in}{8.850194in}}%
\pgfpathcurveto{\pgfqpoint{9.321043in}{8.850194in}}{\pgfqpoint{9.331642in}{8.854584in}}{\pgfqpoint{9.339456in}{8.862398in}}%
\pgfpathcurveto{\pgfqpoint{9.347269in}{8.870212in}}{\pgfqpoint{9.351660in}{8.880811in}}{\pgfqpoint{9.351660in}{8.891861in}}%
\pgfpathcurveto{\pgfqpoint{9.351660in}{8.902911in}}{\pgfqpoint{9.347269in}{8.913510in}}{\pgfqpoint{9.339456in}{8.921324in}}%
\pgfpathcurveto{\pgfqpoint{9.331642in}{8.929137in}}{\pgfqpoint{9.321043in}{8.933528in}}{\pgfqpoint{9.309993in}{8.933528in}}%
\pgfpathcurveto{\pgfqpoint{9.298943in}{8.933528in}}{\pgfqpoint{9.288344in}{8.929137in}}{\pgfqpoint{9.280530in}{8.921324in}}%
\pgfpathcurveto{\pgfqpoint{9.272717in}{8.913510in}}{\pgfqpoint{9.268326in}{8.902911in}}{\pgfqpoint{9.268326in}{8.891861in}}%
\pgfpathcurveto{\pgfqpoint{9.268326in}{8.880811in}}{\pgfqpoint{9.272717in}{8.870212in}}{\pgfqpoint{9.280530in}{8.862398in}}%
\pgfpathcurveto{\pgfqpoint{9.288344in}{8.854584in}}{\pgfqpoint{9.298943in}{8.850194in}}{\pgfqpoint{9.309993in}{8.850194in}}%
\pgfpathlineto{\pgfqpoint{9.309993in}{8.850194in}}%
\pgfpathclose%
\pgfusepath{stroke,fill}%
\end{pgfscope}%
\begin{pgfscope}%
\pgfpathrectangle{\pgfqpoint{7.622482in}{7.624184in}}{\pgfqpoint{2.177280in}{2.201755in}}%
\pgfusepath{clip}%
\pgfsetbuttcap%
\pgfsetroundjoin%
\definecolor{currentfill}{rgb}{0.172549,0.627451,0.172549}%
\pgfsetfillcolor{currentfill}%
\pgfsetlinewidth{0.481800pt}%
\definecolor{currentstroke}{rgb}{1.000000,1.000000,1.000000}%
\pgfsetstrokecolor{currentstroke}%
\pgfsetdash{}{0pt}%
\pgfpathmoveto{\pgfqpoint{8.971224in}{8.905794in}}%
\pgfpathcurveto{\pgfqpoint{8.982274in}{8.905794in}}{\pgfqpoint{8.992873in}{8.910184in}}{\pgfqpoint{9.000686in}{8.917998in}}%
\pgfpathcurveto{\pgfqpoint{9.008500in}{8.925812in}}{\pgfqpoint{9.012890in}{8.936411in}}{\pgfqpoint{9.012890in}{8.947461in}}%
\pgfpathcurveto{\pgfqpoint{9.012890in}{8.958511in}}{\pgfqpoint{9.008500in}{8.969110in}}{\pgfqpoint{9.000686in}{8.976924in}}%
\pgfpathcurveto{\pgfqpoint{8.992873in}{8.984737in}}{\pgfqpoint{8.982274in}{8.989127in}}{\pgfqpoint{8.971224in}{8.989127in}}%
\pgfpathcurveto{\pgfqpoint{8.960174in}{8.989127in}}{\pgfqpoint{8.949575in}{8.984737in}}{\pgfqpoint{8.941761in}{8.976924in}}%
\pgfpathcurveto{\pgfqpoint{8.933947in}{8.969110in}}{\pgfqpoint{8.929557in}{8.958511in}}{\pgfqpoint{8.929557in}{8.947461in}}%
\pgfpathcurveto{\pgfqpoint{8.929557in}{8.936411in}}{\pgfqpoint{8.933947in}{8.925812in}}{\pgfqpoint{8.941761in}{8.917998in}}%
\pgfpathcurveto{\pgfqpoint{8.949575in}{8.910184in}}{\pgfqpoint{8.960174in}{8.905794in}}{\pgfqpoint{8.971224in}{8.905794in}}%
\pgfpathlineto{\pgfqpoint{8.971224in}{8.905794in}}%
\pgfpathclose%
\pgfusepath{stroke,fill}%
\end{pgfscope}%
\begin{pgfscope}%
\pgfpathrectangle{\pgfqpoint{7.622482in}{7.624184in}}{\pgfqpoint{2.177280in}{2.201755in}}%
\pgfusepath{clip}%
\pgfsetbuttcap%
\pgfsetroundjoin%
\definecolor{currentfill}{rgb}{0.172549,0.627451,0.172549}%
\pgfsetfillcolor{currentfill}%
\pgfsetlinewidth{0.481800pt}%
\definecolor{currentstroke}{rgb}{1.000000,1.000000,1.000000}%
\pgfsetstrokecolor{currentstroke}%
\pgfsetdash{}{0pt}%
\pgfpathmoveto{\pgfqpoint{9.242239in}{9.572993in}}%
\pgfpathcurveto{\pgfqpoint{9.253289in}{9.572993in}}{\pgfqpoint{9.263888in}{9.577383in}}{\pgfqpoint{9.271702in}{9.585197in}}%
\pgfpathcurveto{\pgfqpoint{9.279516in}{9.593010in}}{\pgfqpoint{9.283906in}{9.603609in}}{\pgfqpoint{9.283906in}{9.614659in}}%
\pgfpathcurveto{\pgfqpoint{9.283906in}{9.625709in}}{\pgfqpoint{9.279516in}{9.636308in}}{\pgfqpoint{9.271702in}{9.644122in}}%
\pgfpathcurveto{\pgfqpoint{9.263888in}{9.651936in}}{\pgfqpoint{9.253289in}{9.656326in}}{\pgfqpoint{9.242239in}{9.656326in}}%
\pgfpathcurveto{\pgfqpoint{9.231189in}{9.656326in}}{\pgfqpoint{9.220590in}{9.651936in}}{\pgfqpoint{9.212776in}{9.644122in}}%
\pgfpathcurveto{\pgfqpoint{9.204963in}{9.636308in}}{\pgfqpoint{9.200573in}{9.625709in}}{\pgfqpoint{9.200573in}{9.614659in}}%
\pgfpathcurveto{\pgfqpoint{9.200573in}{9.603609in}}{\pgfqpoint{9.204963in}{9.593010in}}{\pgfqpoint{9.212776in}{9.585197in}}%
\pgfpathcurveto{\pgfqpoint{9.220590in}{9.577383in}}{\pgfqpoint{9.231189in}{9.572993in}}{\pgfqpoint{9.242239in}{9.572993in}}%
\pgfpathlineto{\pgfqpoint{9.242239in}{9.572993in}}%
\pgfpathclose%
\pgfusepath{stroke,fill}%
\end{pgfscope}%
\begin{pgfscope}%
\pgfpathrectangle{\pgfqpoint{7.622482in}{7.624184in}}{\pgfqpoint{2.177280in}{2.201755in}}%
\pgfusepath{clip}%
\pgfsetbuttcap%
\pgfsetroundjoin%
\definecolor{currentfill}{rgb}{0.172549,0.627451,0.172549}%
\pgfsetfillcolor{currentfill}%
\pgfsetlinewidth{0.481800pt}%
\definecolor{currentstroke}{rgb}{1.000000,1.000000,1.000000}%
\pgfsetstrokecolor{currentstroke}%
\pgfsetdash{}{0pt}%
\pgfpathmoveto{\pgfqpoint{9.309993in}{9.572993in}}%
\pgfpathcurveto{\pgfqpoint{9.321043in}{9.572993in}}{\pgfqpoint{9.331642in}{9.577383in}}{\pgfqpoint{9.339456in}{9.585197in}}%
\pgfpathcurveto{\pgfqpoint{9.347269in}{9.593010in}}{\pgfqpoint{9.351660in}{9.603609in}}{\pgfqpoint{9.351660in}{9.614659in}}%
\pgfpathcurveto{\pgfqpoint{9.351660in}{9.625709in}}{\pgfqpoint{9.347269in}{9.636308in}}{\pgfqpoint{9.339456in}{9.644122in}}%
\pgfpathcurveto{\pgfqpoint{9.331642in}{9.651936in}}{\pgfqpoint{9.321043in}{9.656326in}}{\pgfqpoint{9.309993in}{9.656326in}}%
\pgfpathcurveto{\pgfqpoint{9.298943in}{9.656326in}}{\pgfqpoint{9.288344in}{9.651936in}}{\pgfqpoint{9.280530in}{9.644122in}}%
\pgfpathcurveto{\pgfqpoint{9.272717in}{9.636308in}}{\pgfqpoint{9.268326in}{9.625709in}}{\pgfqpoint{9.268326in}{9.614659in}}%
\pgfpathcurveto{\pgfqpoint{9.268326in}{9.603609in}}{\pgfqpoint{9.272717in}{9.593010in}}{\pgfqpoint{9.280530in}{9.585197in}}%
\pgfpathcurveto{\pgfqpoint{9.288344in}{9.577383in}}{\pgfqpoint{9.298943in}{9.572993in}}{\pgfqpoint{9.309993in}{9.572993in}}%
\pgfpathlineto{\pgfqpoint{9.309993in}{9.572993in}}%
\pgfpathclose%
\pgfusepath{stroke,fill}%
\end{pgfscope}%
\begin{pgfscope}%
\pgfpathrectangle{\pgfqpoint{7.622482in}{7.624184in}}{\pgfqpoint{2.177280in}{2.201755in}}%
\pgfusepath{clip}%
\pgfsetbuttcap%
\pgfsetroundjoin%
\definecolor{currentfill}{rgb}{0.172549,0.627451,0.172549}%
\pgfsetfillcolor{currentfill}%
\pgfsetlinewidth{0.481800pt}%
\definecolor{currentstroke}{rgb}{1.000000,1.000000,1.000000}%
\pgfsetstrokecolor{currentstroke}%
\pgfsetdash{}{0pt}%
\pgfpathmoveto{\pgfqpoint{8.767962in}{8.627795in}}%
\pgfpathcurveto{\pgfqpoint{8.779012in}{8.627795in}}{\pgfqpoint{8.789611in}{8.632185in}}{\pgfqpoint{8.797425in}{8.639999in}}%
\pgfpathcurveto{\pgfqpoint{8.805238in}{8.647812in}}{\pgfqpoint{8.809629in}{8.658411in}}{\pgfqpoint{8.809629in}{8.669461in}}%
\pgfpathcurveto{\pgfqpoint{8.809629in}{8.680512in}}{\pgfqpoint{8.805238in}{8.691111in}}{\pgfqpoint{8.797425in}{8.698924in}}%
\pgfpathcurveto{\pgfqpoint{8.789611in}{8.706738in}}{\pgfqpoint{8.779012in}{8.711128in}}{\pgfqpoint{8.767962in}{8.711128in}}%
\pgfpathcurveto{\pgfqpoint{8.756912in}{8.711128in}}{\pgfqpoint{8.746313in}{8.706738in}}{\pgfqpoint{8.738499in}{8.698924in}}%
\pgfpathcurveto{\pgfqpoint{8.730686in}{8.691111in}}{\pgfqpoint{8.726295in}{8.680512in}}{\pgfqpoint{8.726295in}{8.669461in}}%
\pgfpathcurveto{\pgfqpoint{8.726295in}{8.658411in}}{\pgfqpoint{8.730686in}{8.647812in}}{\pgfqpoint{8.738499in}{8.639999in}}%
\pgfpathcurveto{\pgfqpoint{8.746313in}{8.632185in}}{\pgfqpoint{8.756912in}{8.627795in}}{\pgfqpoint{8.767962in}{8.627795in}}%
\pgfpathlineto{\pgfqpoint{8.767962in}{8.627795in}}%
\pgfpathclose%
\pgfusepath{stroke,fill}%
\end{pgfscope}%
\begin{pgfscope}%
\pgfpathrectangle{\pgfqpoint{7.622482in}{7.624184in}}{\pgfqpoint{2.177280in}{2.201755in}}%
\pgfusepath{clip}%
\pgfsetbuttcap%
\pgfsetroundjoin%
\definecolor{currentfill}{rgb}{0.172549,0.627451,0.172549}%
\pgfsetfillcolor{currentfill}%
\pgfsetlinewidth{0.481800pt}%
\definecolor{currentstroke}{rgb}{1.000000,1.000000,1.000000}%
\pgfsetstrokecolor{currentstroke}%
\pgfsetdash{}{0pt}%
\pgfpathmoveto{\pgfqpoint{9.309993in}{9.128194in}}%
\pgfpathcurveto{\pgfqpoint{9.321043in}{9.128194in}}{\pgfqpoint{9.331642in}{9.132584in}}{\pgfqpoint{9.339456in}{9.140397in}}%
\pgfpathcurveto{\pgfqpoint{9.347269in}{9.148211in}}{\pgfqpoint{9.351660in}{9.158810in}}{\pgfqpoint{9.351660in}{9.169860in}}%
\pgfpathcurveto{\pgfqpoint{9.351660in}{9.180910in}}{\pgfqpoint{9.347269in}{9.191509in}}{\pgfqpoint{9.339456in}{9.199323in}}%
\pgfpathcurveto{\pgfqpoint{9.331642in}{9.207137in}}{\pgfqpoint{9.321043in}{9.211527in}}{\pgfqpoint{9.309993in}{9.211527in}}%
\pgfpathcurveto{\pgfqpoint{9.298943in}{9.211527in}}{\pgfqpoint{9.288344in}{9.207137in}}{\pgfqpoint{9.280530in}{9.199323in}}%
\pgfpathcurveto{\pgfqpoint{9.272717in}{9.191509in}}{\pgfqpoint{9.268326in}{9.180910in}}{\pgfqpoint{9.268326in}{9.169860in}}%
\pgfpathcurveto{\pgfqpoint{9.268326in}{9.158810in}}{\pgfqpoint{9.272717in}{9.148211in}}{\pgfqpoint{9.280530in}{9.140397in}}%
\pgfpathcurveto{\pgfqpoint{9.288344in}{9.132584in}}{\pgfqpoint{9.298943in}{9.128194in}}{\pgfqpoint{9.309993in}{9.128194in}}%
\pgfpathlineto{\pgfqpoint{9.309993in}{9.128194in}}%
\pgfpathclose%
\pgfusepath{stroke,fill}%
\end{pgfscope}%
\begin{pgfscope}%
\pgfpathrectangle{\pgfqpoint{7.622482in}{7.624184in}}{\pgfqpoint{2.177280in}{2.201755in}}%
\pgfusepath{clip}%
\pgfsetbuttcap%
\pgfsetroundjoin%
\definecolor{currentfill}{rgb}{0.172549,0.627451,0.172549}%
\pgfsetfillcolor{currentfill}%
\pgfsetlinewidth{0.481800pt}%
\definecolor{currentstroke}{rgb}{1.000000,1.000000,1.000000}%
\pgfsetstrokecolor{currentstroke}%
\pgfsetdash{}{0pt}%
\pgfpathmoveto{\pgfqpoint{9.106731in}{8.405395in}}%
\pgfpathcurveto{\pgfqpoint{9.117782in}{8.405395in}}{\pgfqpoint{9.128381in}{8.409785in}}{\pgfqpoint{9.136194in}{8.417599in}}%
\pgfpathcurveto{\pgfqpoint{9.144008in}{8.425413in}}{\pgfqpoint{9.148398in}{8.436012in}}{\pgfqpoint{9.148398in}{8.447062in}}%
\pgfpathcurveto{\pgfqpoint{9.148398in}{8.458112in}}{\pgfqpoint{9.144008in}{8.468711in}}{\pgfqpoint{9.136194in}{8.476525in}}%
\pgfpathcurveto{\pgfqpoint{9.128381in}{8.484338in}}{\pgfqpoint{9.117782in}{8.488729in}}{\pgfqpoint{9.106731in}{8.488729in}}%
\pgfpathcurveto{\pgfqpoint{9.095681in}{8.488729in}}{\pgfqpoint{9.085082in}{8.484338in}}{\pgfqpoint{9.077269in}{8.476525in}}%
\pgfpathcurveto{\pgfqpoint{9.069455in}{8.468711in}}{\pgfqpoint{9.065065in}{8.458112in}}{\pgfqpoint{9.065065in}{8.447062in}}%
\pgfpathcurveto{\pgfqpoint{9.065065in}{8.436012in}}{\pgfqpoint{9.069455in}{8.425413in}}{\pgfqpoint{9.077269in}{8.417599in}}%
\pgfpathcurveto{\pgfqpoint{9.085082in}{8.409785in}}{\pgfqpoint{9.095681in}{8.405395in}}{\pgfqpoint{9.106731in}{8.405395in}}%
\pgfpathlineto{\pgfqpoint{9.106731in}{8.405395in}}%
\pgfpathclose%
\pgfusepath{stroke,fill}%
\end{pgfscope}%
\begin{pgfscope}%
\pgfpathrectangle{\pgfqpoint{7.622482in}{7.624184in}}{\pgfqpoint{2.177280in}{2.201755in}}%
\pgfusepath{clip}%
\pgfsetbuttcap%
\pgfsetroundjoin%
\definecolor{currentfill}{rgb}{0.172549,0.627451,0.172549}%
\pgfsetfillcolor{currentfill}%
\pgfsetlinewidth{0.481800pt}%
\definecolor{currentstroke}{rgb}{1.000000,1.000000,1.000000}%
\pgfsetstrokecolor{currentstroke}%
\pgfsetdash{}{0pt}%
\pgfpathmoveto{\pgfqpoint{9.106731in}{9.572993in}}%
\pgfpathcurveto{\pgfqpoint{9.117782in}{9.572993in}}{\pgfqpoint{9.128381in}{9.577383in}}{\pgfqpoint{9.136194in}{9.585197in}}%
\pgfpathcurveto{\pgfqpoint{9.144008in}{9.593010in}}{\pgfqpoint{9.148398in}{9.603609in}}{\pgfqpoint{9.148398in}{9.614659in}}%
\pgfpathcurveto{\pgfqpoint{9.148398in}{9.625709in}}{\pgfqpoint{9.144008in}{9.636308in}}{\pgfqpoint{9.136194in}{9.644122in}}%
\pgfpathcurveto{\pgfqpoint{9.128381in}{9.651936in}}{\pgfqpoint{9.117782in}{9.656326in}}{\pgfqpoint{9.106731in}{9.656326in}}%
\pgfpathcurveto{\pgfqpoint{9.095681in}{9.656326in}}{\pgfqpoint{9.085082in}{9.651936in}}{\pgfqpoint{9.077269in}{9.644122in}}%
\pgfpathcurveto{\pgfqpoint{9.069455in}{9.636308in}}{\pgfqpoint{9.065065in}{9.625709in}}{\pgfqpoint{9.065065in}{9.614659in}}%
\pgfpathcurveto{\pgfqpoint{9.065065in}{9.603609in}}{\pgfqpoint{9.069455in}{9.593010in}}{\pgfqpoint{9.077269in}{9.585197in}}%
\pgfpathcurveto{\pgfqpoint{9.085082in}{9.577383in}}{\pgfqpoint{9.095681in}{9.572993in}}{\pgfqpoint{9.106731in}{9.572993in}}%
\pgfpathlineto{\pgfqpoint{9.106731in}{9.572993in}}%
\pgfpathclose%
\pgfusepath{stroke,fill}%
\end{pgfscope}%
\begin{pgfscope}%
\pgfpathrectangle{\pgfqpoint{7.622482in}{7.624184in}}{\pgfqpoint{2.177280in}{2.201755in}}%
\pgfusepath{clip}%
\pgfsetbuttcap%
\pgfsetroundjoin%
\definecolor{currentfill}{rgb}{0.172549,0.627451,0.172549}%
\pgfsetfillcolor{currentfill}%
\pgfsetlinewidth{0.481800pt}%
\definecolor{currentstroke}{rgb}{1.000000,1.000000,1.000000}%
\pgfsetstrokecolor{currentstroke}%
\pgfsetdash{}{0pt}%
\pgfpathmoveto{\pgfqpoint{8.971224in}{8.794594in}}%
\pgfpathcurveto{\pgfqpoint{8.982274in}{8.794594in}}{\pgfqpoint{8.992873in}{8.798985in}}{\pgfqpoint{9.000686in}{8.806798in}}%
\pgfpathcurveto{\pgfqpoint{9.008500in}{8.814612in}}{\pgfqpoint{9.012890in}{8.825211in}}{\pgfqpoint{9.012890in}{8.836261in}}%
\pgfpathcurveto{\pgfqpoint{9.012890in}{8.847311in}}{\pgfqpoint{9.008500in}{8.857910in}}{\pgfqpoint{9.000686in}{8.865724in}}%
\pgfpathcurveto{\pgfqpoint{8.992873in}{8.873537in}}{\pgfqpoint{8.982274in}{8.877928in}}{\pgfqpoint{8.971224in}{8.877928in}}%
\pgfpathcurveto{\pgfqpoint{8.960174in}{8.877928in}}{\pgfqpoint{8.949575in}{8.873537in}}{\pgfqpoint{8.941761in}{8.865724in}}%
\pgfpathcurveto{\pgfqpoint{8.933947in}{8.857910in}}{\pgfqpoint{8.929557in}{8.847311in}}{\pgfqpoint{8.929557in}{8.836261in}}%
\pgfpathcurveto{\pgfqpoint{8.929557in}{8.825211in}}{\pgfqpoint{8.933947in}{8.814612in}}{\pgfqpoint{8.941761in}{8.806798in}}%
\pgfpathcurveto{\pgfqpoint{8.949575in}{8.798985in}}{\pgfqpoint{8.960174in}{8.794594in}}{\pgfqpoint{8.971224in}{8.794594in}}%
\pgfpathlineto{\pgfqpoint{8.971224in}{8.794594in}}%
\pgfpathclose%
\pgfusepath{stroke,fill}%
\end{pgfscope}%
\begin{pgfscope}%
\pgfpathrectangle{\pgfqpoint{7.622482in}{7.624184in}}{\pgfqpoint{2.177280in}{2.201755in}}%
\pgfusepath{clip}%
\pgfsetbuttcap%
\pgfsetroundjoin%
\definecolor{currentfill}{rgb}{0.172549,0.627451,0.172549}%
\pgfsetfillcolor{currentfill}%
\pgfsetlinewidth{0.481800pt}%
\definecolor{currentstroke}{rgb}{1.000000,1.000000,1.000000}%
\pgfsetstrokecolor{currentstroke}%
\pgfsetdash{}{0pt}%
\pgfpathmoveto{\pgfqpoint{9.174485in}{9.016994in}}%
\pgfpathcurveto{\pgfqpoint{9.185535in}{9.016994in}}{\pgfqpoint{9.196134in}{9.021384in}}{\pgfqpoint{9.203948in}{9.029198in}}%
\pgfpathcurveto{\pgfqpoint{9.211762in}{9.037011in}}{\pgfqpoint{9.216152in}{9.047610in}}{\pgfqpoint{9.216152in}{9.058661in}}%
\pgfpathcurveto{\pgfqpoint{9.216152in}{9.069711in}}{\pgfqpoint{9.211762in}{9.080310in}}{\pgfqpoint{9.203948in}{9.088123in}}%
\pgfpathcurveto{\pgfqpoint{9.196134in}{9.095937in}}{\pgfqpoint{9.185535in}{9.100327in}}{\pgfqpoint{9.174485in}{9.100327in}}%
\pgfpathcurveto{\pgfqpoint{9.163435in}{9.100327in}}{\pgfqpoint{9.152836in}{9.095937in}}{\pgfqpoint{9.145023in}{9.088123in}}%
\pgfpathcurveto{\pgfqpoint{9.137209in}{9.080310in}}{\pgfqpoint{9.132819in}{9.069711in}}{\pgfqpoint{9.132819in}{9.058661in}}%
\pgfpathcurveto{\pgfqpoint{9.132819in}{9.047610in}}{\pgfqpoint{9.137209in}{9.037011in}}{\pgfqpoint{9.145023in}{9.029198in}}%
\pgfpathcurveto{\pgfqpoint{9.152836in}{9.021384in}}{\pgfqpoint{9.163435in}{9.016994in}}{\pgfqpoint{9.174485in}{9.016994in}}%
\pgfpathlineto{\pgfqpoint{9.174485in}{9.016994in}}%
\pgfpathclose%
\pgfusepath{stroke,fill}%
\end{pgfscope}%
\begin{pgfscope}%
\pgfpathrectangle{\pgfqpoint{7.622482in}{7.624184in}}{\pgfqpoint{2.177280in}{2.201755in}}%
\pgfusepath{clip}%
\pgfsetbuttcap%
\pgfsetroundjoin%
\definecolor{currentfill}{rgb}{0.172549,0.627451,0.172549}%
\pgfsetfillcolor{currentfill}%
\pgfsetlinewidth{0.481800pt}%
\definecolor{currentstroke}{rgb}{1.000000,1.000000,1.000000}%
\pgfsetstrokecolor{currentstroke}%
\pgfsetdash{}{0pt}%
\pgfpathmoveto{\pgfqpoint{8.971224in}{9.294993in}}%
\pgfpathcurveto{\pgfqpoint{8.982274in}{9.294993in}}{\pgfqpoint{8.992873in}{9.299384in}}{\pgfqpoint{9.000686in}{9.307197in}}%
\pgfpathcurveto{\pgfqpoint{9.008500in}{9.315011in}}{\pgfqpoint{9.012890in}{9.325610in}}{\pgfqpoint{9.012890in}{9.336660in}}%
\pgfpathcurveto{\pgfqpoint{9.012890in}{9.347710in}}{\pgfqpoint{9.008500in}{9.358309in}}{\pgfqpoint{9.000686in}{9.366123in}}%
\pgfpathcurveto{\pgfqpoint{8.992873in}{9.373936in}}{\pgfqpoint{8.982274in}{9.378327in}}{\pgfqpoint{8.971224in}{9.378327in}}%
\pgfpathcurveto{\pgfqpoint{8.960174in}{9.378327in}}{\pgfqpoint{8.949575in}{9.373936in}}{\pgfqpoint{8.941761in}{9.366123in}}%
\pgfpathcurveto{\pgfqpoint{8.933947in}{9.358309in}}{\pgfqpoint{8.929557in}{9.347710in}}{\pgfqpoint{8.929557in}{9.336660in}}%
\pgfpathcurveto{\pgfqpoint{8.929557in}{9.325610in}}{\pgfqpoint{8.933947in}{9.315011in}}{\pgfqpoint{8.941761in}{9.307197in}}%
\pgfpathcurveto{\pgfqpoint{8.949575in}{9.299384in}}{\pgfqpoint{8.960174in}{9.294993in}}{\pgfqpoint{8.971224in}{9.294993in}}%
\pgfpathlineto{\pgfqpoint{8.971224in}{9.294993in}}%
\pgfpathclose%
\pgfusepath{stroke,fill}%
\end{pgfscope}%
\begin{pgfscope}%
\pgfpathrectangle{\pgfqpoint{7.622482in}{7.624184in}}{\pgfqpoint{2.177280in}{2.201755in}}%
\pgfusepath{clip}%
\pgfsetbuttcap%
\pgfsetroundjoin%
\definecolor{currentfill}{rgb}{0.172549,0.627451,0.172549}%
\pgfsetfillcolor{currentfill}%
\pgfsetlinewidth{0.481800pt}%
\definecolor{currentstroke}{rgb}{1.000000,1.000000,1.000000}%
\pgfsetstrokecolor{currentstroke}%
\pgfsetdash{}{0pt}%
\pgfpathmoveto{\pgfqpoint{8.971224in}{8.738994in}}%
\pgfpathcurveto{\pgfqpoint{8.982274in}{8.738994in}}{\pgfqpoint{8.992873in}{8.743385in}}{\pgfqpoint{9.000686in}{8.751198in}}%
\pgfpathcurveto{\pgfqpoint{9.008500in}{8.759012in}}{\pgfqpoint{9.012890in}{8.769611in}}{\pgfqpoint{9.012890in}{8.780661in}}%
\pgfpathcurveto{\pgfqpoint{9.012890in}{8.791711in}}{\pgfqpoint{9.008500in}{8.802310in}}{\pgfqpoint{9.000686in}{8.810124in}}%
\pgfpathcurveto{\pgfqpoint{8.992873in}{8.817938in}}{\pgfqpoint{8.982274in}{8.822328in}}{\pgfqpoint{8.971224in}{8.822328in}}%
\pgfpathcurveto{\pgfqpoint{8.960174in}{8.822328in}}{\pgfqpoint{8.949575in}{8.817938in}}{\pgfqpoint{8.941761in}{8.810124in}}%
\pgfpathcurveto{\pgfqpoint{8.933947in}{8.802310in}}{\pgfqpoint{8.929557in}{8.791711in}}{\pgfqpoint{8.929557in}{8.780661in}}%
\pgfpathcurveto{\pgfqpoint{8.929557in}{8.769611in}}{\pgfqpoint{8.933947in}{8.759012in}}{\pgfqpoint{8.941761in}{8.751198in}}%
\pgfpathcurveto{\pgfqpoint{8.949575in}{8.743385in}}{\pgfqpoint{8.960174in}{8.738994in}}{\pgfqpoint{8.971224in}{8.738994in}}%
\pgfpathlineto{\pgfqpoint{8.971224in}{8.738994in}}%
\pgfpathclose%
\pgfusepath{stroke,fill}%
\end{pgfscope}%
\begin{pgfscope}%
\pgfpathrectangle{\pgfqpoint{7.622482in}{7.624184in}}{\pgfqpoint{2.177280in}{2.201755in}}%
\pgfusepath{clip}%
\pgfsetbuttcap%
\pgfsetroundjoin%
\definecolor{currentfill}{rgb}{0.172549,0.627451,0.172549}%
\pgfsetfillcolor{currentfill}%
\pgfsetlinewidth{0.481800pt}%
\definecolor{currentstroke}{rgb}{1.000000,1.000000,1.000000}%
\pgfsetstrokecolor{currentstroke}%
\pgfsetdash{}{0pt}%
\pgfpathmoveto{\pgfqpoint{8.971224in}{8.683395in}}%
\pgfpathcurveto{\pgfqpoint{8.982274in}{8.683395in}}{\pgfqpoint{8.992873in}{8.687785in}}{\pgfqpoint{9.000686in}{8.695598in}}%
\pgfpathcurveto{\pgfqpoint{9.008500in}{8.703412in}}{\pgfqpoint{9.012890in}{8.714011in}}{\pgfqpoint{9.012890in}{8.725061in}}%
\pgfpathcurveto{\pgfqpoint{9.012890in}{8.736111in}}{\pgfqpoint{9.008500in}{8.746710in}}{\pgfqpoint{9.000686in}{8.754524in}}%
\pgfpathcurveto{\pgfqpoint{8.992873in}{8.762338in}}{\pgfqpoint{8.982274in}{8.766728in}}{\pgfqpoint{8.971224in}{8.766728in}}%
\pgfpathcurveto{\pgfqpoint{8.960174in}{8.766728in}}{\pgfqpoint{8.949575in}{8.762338in}}{\pgfqpoint{8.941761in}{8.754524in}}%
\pgfpathcurveto{\pgfqpoint{8.933947in}{8.746710in}}{\pgfqpoint{8.929557in}{8.736111in}}{\pgfqpoint{8.929557in}{8.725061in}}%
\pgfpathcurveto{\pgfqpoint{8.929557in}{8.714011in}}{\pgfqpoint{8.933947in}{8.703412in}}{\pgfqpoint{8.941761in}{8.695598in}}%
\pgfpathcurveto{\pgfqpoint{8.949575in}{8.687785in}}{\pgfqpoint{8.960174in}{8.683395in}}{\pgfqpoint{8.971224in}{8.683395in}}%
\pgfpathlineto{\pgfqpoint{8.971224in}{8.683395in}}%
\pgfpathclose%
\pgfusepath{stroke,fill}%
\end{pgfscope}%
\begin{pgfscope}%
\pgfpathrectangle{\pgfqpoint{7.622482in}{7.624184in}}{\pgfqpoint{2.177280in}{2.201755in}}%
\pgfusepath{clip}%
\pgfsetbuttcap%
\pgfsetroundjoin%
\definecolor{currentfill}{rgb}{0.172549,0.627451,0.172549}%
\pgfsetfillcolor{currentfill}%
\pgfsetlinewidth{0.481800pt}%
\definecolor{currentstroke}{rgb}{1.000000,1.000000,1.000000}%
\pgfsetstrokecolor{currentstroke}%
\pgfsetdash{}{0pt}%
\pgfpathmoveto{\pgfqpoint{9.174485in}{8.850194in}}%
\pgfpathcurveto{\pgfqpoint{9.185535in}{8.850194in}}{\pgfqpoint{9.196134in}{8.854584in}}{\pgfqpoint{9.203948in}{8.862398in}}%
\pgfpathcurveto{\pgfqpoint{9.211762in}{8.870212in}}{\pgfqpoint{9.216152in}{8.880811in}}{\pgfqpoint{9.216152in}{8.891861in}}%
\pgfpathcurveto{\pgfqpoint{9.216152in}{8.902911in}}{\pgfqpoint{9.211762in}{8.913510in}}{\pgfqpoint{9.203948in}{8.921324in}}%
\pgfpathcurveto{\pgfqpoint{9.196134in}{8.929137in}}{\pgfqpoint{9.185535in}{8.933528in}}{\pgfqpoint{9.174485in}{8.933528in}}%
\pgfpathcurveto{\pgfqpoint{9.163435in}{8.933528in}}{\pgfqpoint{9.152836in}{8.929137in}}{\pgfqpoint{9.145023in}{8.921324in}}%
\pgfpathcurveto{\pgfqpoint{9.137209in}{8.913510in}}{\pgfqpoint{9.132819in}{8.902911in}}{\pgfqpoint{9.132819in}{8.891861in}}%
\pgfpathcurveto{\pgfqpoint{9.132819in}{8.880811in}}{\pgfqpoint{9.137209in}{8.870212in}}{\pgfqpoint{9.145023in}{8.862398in}}%
\pgfpathcurveto{\pgfqpoint{9.152836in}{8.854584in}}{\pgfqpoint{9.163435in}{8.850194in}}{\pgfqpoint{9.174485in}{8.850194in}}%
\pgfpathlineto{\pgfqpoint{9.174485in}{8.850194in}}%
\pgfpathclose%
\pgfusepath{stroke,fill}%
\end{pgfscope}%
\begin{pgfscope}%
\pgfpathrectangle{\pgfqpoint{7.622482in}{7.624184in}}{\pgfqpoint{2.177280in}{2.201755in}}%
\pgfusepath{clip}%
\pgfsetbuttcap%
\pgfsetroundjoin%
\definecolor{currentfill}{rgb}{0.172549,0.627451,0.172549}%
\pgfsetfillcolor{currentfill}%
\pgfsetlinewidth{0.481800pt}%
\definecolor{currentstroke}{rgb}{1.000000,1.000000,1.000000}%
\pgfsetstrokecolor{currentstroke}%
\pgfsetdash{}{0pt}%
\pgfpathmoveto{\pgfqpoint{8.835716in}{9.294993in}}%
\pgfpathcurveto{\pgfqpoint{8.846766in}{9.294993in}}{\pgfqpoint{8.857365in}{9.299384in}}{\pgfqpoint{8.865179in}{9.307197in}}%
\pgfpathcurveto{\pgfqpoint{8.872992in}{9.315011in}}{\pgfqpoint{8.877383in}{9.325610in}}{\pgfqpoint{8.877383in}{9.336660in}}%
\pgfpathcurveto{\pgfqpoint{8.877383in}{9.347710in}}{\pgfqpoint{8.872992in}{9.358309in}}{\pgfqpoint{8.865179in}{9.366123in}}%
\pgfpathcurveto{\pgfqpoint{8.857365in}{9.373936in}}{\pgfqpoint{8.846766in}{9.378327in}}{\pgfqpoint{8.835716in}{9.378327in}}%
\pgfpathcurveto{\pgfqpoint{8.824666in}{9.378327in}}{\pgfqpoint{8.814067in}{9.373936in}}{\pgfqpoint{8.806253in}{9.366123in}}%
\pgfpathcurveto{\pgfqpoint{8.798440in}{9.358309in}}{\pgfqpoint{8.794049in}{9.347710in}}{\pgfqpoint{8.794049in}{9.336660in}}%
\pgfpathcurveto{\pgfqpoint{8.794049in}{9.325610in}}{\pgfqpoint{8.798440in}{9.315011in}}{\pgfqpoint{8.806253in}{9.307197in}}%
\pgfpathcurveto{\pgfqpoint{8.814067in}{9.299384in}}{\pgfqpoint{8.824666in}{9.294993in}}{\pgfqpoint{8.835716in}{9.294993in}}%
\pgfpathlineto{\pgfqpoint{8.835716in}{9.294993in}}%
\pgfpathclose%
\pgfusepath{stroke,fill}%
\end{pgfscope}%
\begin{pgfscope}%
\pgfpathrectangle{\pgfqpoint{7.622482in}{7.624184in}}{\pgfqpoint{2.177280in}{2.201755in}}%
\pgfusepath{clip}%
\pgfsetbuttcap%
\pgfsetroundjoin%
\definecolor{currentfill}{rgb}{0.172549,0.627451,0.172549}%
\pgfsetfillcolor{currentfill}%
\pgfsetlinewidth{0.481800pt}%
\definecolor{currentstroke}{rgb}{1.000000,1.000000,1.000000}%
\pgfsetstrokecolor{currentstroke}%
\pgfsetdash{}{0pt}%
\pgfpathmoveto{\pgfqpoint{9.038978in}{9.406193in}}%
\pgfpathcurveto{\pgfqpoint{9.050028in}{9.406193in}}{\pgfqpoint{9.060627in}{9.410583in}}{\pgfqpoint{9.068440in}{9.418397in}}%
\pgfpathcurveto{\pgfqpoint{9.076254in}{9.426211in}}{\pgfqpoint{9.080644in}{9.436810in}}{\pgfqpoint{9.080644in}{9.447860in}}%
\pgfpathcurveto{\pgfqpoint{9.080644in}{9.458910in}}{\pgfqpoint{9.076254in}{9.469509in}}{\pgfqpoint{9.068440in}{9.477322in}}%
\pgfpathcurveto{\pgfqpoint{9.060627in}{9.485136in}}{\pgfqpoint{9.050028in}{9.489526in}}{\pgfqpoint{9.038978in}{9.489526in}}%
\pgfpathcurveto{\pgfqpoint{9.027927in}{9.489526in}}{\pgfqpoint{9.017328in}{9.485136in}}{\pgfqpoint{9.009515in}{9.477322in}}%
\pgfpathcurveto{\pgfqpoint{9.001701in}{9.469509in}}{\pgfqpoint{8.997311in}{9.458910in}}{\pgfqpoint{8.997311in}{9.447860in}}%
\pgfpathcurveto{\pgfqpoint{8.997311in}{9.436810in}}{\pgfqpoint{9.001701in}{9.426211in}}{\pgfqpoint{9.009515in}{9.418397in}}%
\pgfpathcurveto{\pgfqpoint{9.017328in}{9.410583in}}{\pgfqpoint{9.027927in}{9.406193in}}{\pgfqpoint{9.038978in}{9.406193in}}%
\pgfpathlineto{\pgfqpoint{9.038978in}{9.406193in}}%
\pgfpathclose%
\pgfusepath{stroke,fill}%
\end{pgfscope}%
\begin{pgfscope}%
\pgfpathrectangle{\pgfqpoint{7.622482in}{7.624184in}}{\pgfqpoint{2.177280in}{2.201755in}}%
\pgfusepath{clip}%
\pgfsetbuttcap%
\pgfsetroundjoin%
\definecolor{currentfill}{rgb}{0.172549,0.627451,0.172549}%
\pgfsetfillcolor{currentfill}%
\pgfsetlinewidth{0.481800pt}%
\definecolor{currentstroke}{rgb}{1.000000,1.000000,1.000000}%
\pgfsetstrokecolor{currentstroke}%
\pgfsetdash{}{0pt}%
\pgfpathmoveto{\pgfqpoint{9.106731in}{9.684192in}}%
\pgfpathcurveto{\pgfqpoint{9.117782in}{9.684192in}}{\pgfqpoint{9.128381in}{9.688583in}}{\pgfqpoint{9.136194in}{9.696396in}}%
\pgfpathcurveto{\pgfqpoint{9.144008in}{9.704210in}}{\pgfqpoint{9.148398in}{9.714809in}}{\pgfqpoint{9.148398in}{9.725859in}}%
\pgfpathcurveto{\pgfqpoint{9.148398in}{9.736909in}}{\pgfqpoint{9.144008in}{9.747508in}}{\pgfqpoint{9.136194in}{9.755322in}}%
\pgfpathcurveto{\pgfqpoint{9.128381in}{9.763135in}}{\pgfqpoint{9.117782in}{9.767526in}}{\pgfqpoint{9.106731in}{9.767526in}}%
\pgfpathcurveto{\pgfqpoint{9.095681in}{9.767526in}}{\pgfqpoint{9.085082in}{9.763135in}}{\pgfqpoint{9.077269in}{9.755322in}}%
\pgfpathcurveto{\pgfqpoint{9.069455in}{9.747508in}}{\pgfqpoint{9.065065in}{9.736909in}}{\pgfqpoint{9.065065in}{9.725859in}}%
\pgfpathcurveto{\pgfqpoint{9.065065in}{9.714809in}}{\pgfqpoint{9.069455in}{9.704210in}}{\pgfqpoint{9.077269in}{9.696396in}}%
\pgfpathcurveto{\pgfqpoint{9.085082in}{9.688583in}}{\pgfqpoint{9.095681in}{9.684192in}}{\pgfqpoint{9.106731in}{9.684192in}}%
\pgfpathlineto{\pgfqpoint{9.106731in}{9.684192in}}%
\pgfpathclose%
\pgfusepath{stroke,fill}%
\end{pgfscope}%
\begin{pgfscope}%
\pgfpathrectangle{\pgfqpoint{7.622482in}{7.624184in}}{\pgfqpoint{2.177280in}{2.201755in}}%
\pgfusepath{clip}%
\pgfsetbuttcap%
\pgfsetroundjoin%
\definecolor{currentfill}{rgb}{0.172549,0.627451,0.172549}%
\pgfsetfillcolor{currentfill}%
\pgfsetlinewidth{0.481800pt}%
\definecolor{currentstroke}{rgb}{1.000000,1.000000,1.000000}%
\pgfsetstrokecolor{currentstroke}%
\pgfsetdash{}{0pt}%
\pgfpathmoveto{\pgfqpoint{9.242239in}{8.850194in}}%
\pgfpathcurveto{\pgfqpoint{9.253289in}{8.850194in}}{\pgfqpoint{9.263888in}{8.854584in}}{\pgfqpoint{9.271702in}{8.862398in}}%
\pgfpathcurveto{\pgfqpoint{9.279516in}{8.870212in}}{\pgfqpoint{9.283906in}{8.880811in}}{\pgfqpoint{9.283906in}{8.891861in}}%
\pgfpathcurveto{\pgfqpoint{9.283906in}{8.902911in}}{\pgfqpoint{9.279516in}{8.913510in}}{\pgfqpoint{9.271702in}{8.921324in}}%
\pgfpathcurveto{\pgfqpoint{9.263888in}{8.929137in}}{\pgfqpoint{9.253289in}{8.933528in}}{\pgfqpoint{9.242239in}{8.933528in}}%
\pgfpathcurveto{\pgfqpoint{9.231189in}{8.933528in}}{\pgfqpoint{9.220590in}{8.929137in}}{\pgfqpoint{9.212776in}{8.921324in}}%
\pgfpathcurveto{\pgfqpoint{9.204963in}{8.913510in}}{\pgfqpoint{9.200573in}{8.902911in}}{\pgfqpoint{9.200573in}{8.891861in}}%
\pgfpathcurveto{\pgfqpoint{9.200573in}{8.880811in}}{\pgfqpoint{9.204963in}{8.870212in}}{\pgfqpoint{9.212776in}{8.862398in}}%
\pgfpathcurveto{\pgfqpoint{9.220590in}{8.854584in}}{\pgfqpoint{9.231189in}{8.850194in}}{\pgfqpoint{9.242239in}{8.850194in}}%
\pgfpathlineto{\pgfqpoint{9.242239in}{8.850194in}}%
\pgfpathclose%
\pgfusepath{stroke,fill}%
\end{pgfscope}%
\begin{pgfscope}%
\pgfpathrectangle{\pgfqpoint{7.622482in}{7.624184in}}{\pgfqpoint{2.177280in}{2.201755in}}%
\pgfusepath{clip}%
\pgfsetbuttcap%
\pgfsetroundjoin%
\definecolor{currentfill}{rgb}{0.172549,0.627451,0.172549}%
\pgfsetfillcolor{currentfill}%
\pgfsetlinewidth{0.481800pt}%
\definecolor{currentstroke}{rgb}{1.000000,1.000000,1.000000}%
\pgfsetstrokecolor{currentstroke}%
\pgfsetdash{}{0pt}%
\pgfpathmoveto{\pgfqpoint{8.767962in}{8.794594in}}%
\pgfpathcurveto{\pgfqpoint{8.779012in}{8.794594in}}{\pgfqpoint{8.789611in}{8.798985in}}{\pgfqpoint{8.797425in}{8.806798in}}%
\pgfpathcurveto{\pgfqpoint{8.805238in}{8.814612in}}{\pgfqpoint{8.809629in}{8.825211in}}{\pgfqpoint{8.809629in}{8.836261in}}%
\pgfpathcurveto{\pgfqpoint{8.809629in}{8.847311in}}{\pgfqpoint{8.805238in}{8.857910in}}{\pgfqpoint{8.797425in}{8.865724in}}%
\pgfpathcurveto{\pgfqpoint{8.789611in}{8.873537in}}{\pgfqpoint{8.779012in}{8.877928in}}{\pgfqpoint{8.767962in}{8.877928in}}%
\pgfpathcurveto{\pgfqpoint{8.756912in}{8.877928in}}{\pgfqpoint{8.746313in}{8.873537in}}{\pgfqpoint{8.738499in}{8.865724in}}%
\pgfpathcurveto{\pgfqpoint{8.730686in}{8.857910in}}{\pgfqpoint{8.726295in}{8.847311in}}{\pgfqpoint{8.726295in}{8.836261in}}%
\pgfpathcurveto{\pgfqpoint{8.726295in}{8.825211in}}{\pgfqpoint{8.730686in}{8.814612in}}{\pgfqpoint{8.738499in}{8.806798in}}%
\pgfpathcurveto{\pgfqpoint{8.746313in}{8.798985in}}{\pgfqpoint{8.756912in}{8.794594in}}{\pgfqpoint{8.767962in}{8.794594in}}%
\pgfpathlineto{\pgfqpoint{8.767962in}{8.794594in}}%
\pgfpathclose%
\pgfusepath{stroke,fill}%
\end{pgfscope}%
\begin{pgfscope}%
\pgfpathrectangle{\pgfqpoint{7.622482in}{7.624184in}}{\pgfqpoint{2.177280in}{2.201755in}}%
\pgfusepath{clip}%
\pgfsetbuttcap%
\pgfsetroundjoin%
\definecolor{currentfill}{rgb}{0.172549,0.627451,0.172549}%
\pgfsetfillcolor{currentfill}%
\pgfsetlinewidth{0.481800pt}%
\definecolor{currentstroke}{rgb}{1.000000,1.000000,1.000000}%
\pgfsetstrokecolor{currentstroke}%
\pgfsetdash{}{0pt}%
\pgfpathmoveto{\pgfqpoint{8.700208in}{8.683395in}}%
\pgfpathcurveto{\pgfqpoint{8.711258in}{8.683395in}}{\pgfqpoint{8.721857in}{8.687785in}}{\pgfqpoint{8.729671in}{8.695598in}}%
\pgfpathcurveto{\pgfqpoint{8.737485in}{8.703412in}}{\pgfqpoint{8.741875in}{8.714011in}}{\pgfqpoint{8.741875in}{8.725061in}}%
\pgfpathcurveto{\pgfqpoint{8.741875in}{8.736111in}}{\pgfqpoint{8.737485in}{8.746710in}}{\pgfqpoint{8.729671in}{8.754524in}}%
\pgfpathcurveto{\pgfqpoint{8.721857in}{8.762338in}}{\pgfqpoint{8.711258in}{8.766728in}}{\pgfqpoint{8.700208in}{8.766728in}}%
\pgfpathcurveto{\pgfqpoint{8.689158in}{8.766728in}}{\pgfqpoint{8.678559in}{8.762338in}}{\pgfqpoint{8.670745in}{8.754524in}}%
\pgfpathcurveto{\pgfqpoint{8.662932in}{8.746710in}}{\pgfqpoint{8.658542in}{8.736111in}}{\pgfqpoint{8.658542in}{8.725061in}}%
\pgfpathcurveto{\pgfqpoint{8.658542in}{8.714011in}}{\pgfqpoint{8.662932in}{8.703412in}}{\pgfqpoint{8.670745in}{8.695598in}}%
\pgfpathcurveto{\pgfqpoint{8.678559in}{8.687785in}}{\pgfqpoint{8.689158in}{8.683395in}}{\pgfqpoint{8.700208in}{8.683395in}}%
\pgfpathlineto{\pgfqpoint{8.700208in}{8.683395in}}%
\pgfpathclose%
\pgfusepath{stroke,fill}%
\end{pgfscope}%
\begin{pgfscope}%
\pgfpathrectangle{\pgfqpoint{7.622482in}{7.624184in}}{\pgfqpoint{2.177280in}{2.201755in}}%
\pgfusepath{clip}%
\pgfsetbuttcap%
\pgfsetroundjoin%
\definecolor{currentfill}{rgb}{0.172549,0.627451,0.172549}%
\pgfsetfillcolor{currentfill}%
\pgfsetlinewidth{0.481800pt}%
\definecolor{currentstroke}{rgb}{1.000000,1.000000,1.000000}%
\pgfsetstrokecolor{currentstroke}%
\pgfsetdash{}{0pt}%
\pgfpathmoveto{\pgfqpoint{9.309993in}{9.572993in}}%
\pgfpathcurveto{\pgfqpoint{9.321043in}{9.572993in}}{\pgfqpoint{9.331642in}{9.577383in}}{\pgfqpoint{9.339456in}{9.585197in}}%
\pgfpathcurveto{\pgfqpoint{9.347269in}{9.593010in}}{\pgfqpoint{9.351660in}{9.603609in}}{\pgfqpoint{9.351660in}{9.614659in}}%
\pgfpathcurveto{\pgfqpoint{9.351660in}{9.625709in}}{\pgfqpoint{9.347269in}{9.636308in}}{\pgfqpoint{9.339456in}{9.644122in}}%
\pgfpathcurveto{\pgfqpoint{9.331642in}{9.651936in}}{\pgfqpoint{9.321043in}{9.656326in}}{\pgfqpoint{9.309993in}{9.656326in}}%
\pgfpathcurveto{\pgfqpoint{9.298943in}{9.656326in}}{\pgfqpoint{9.288344in}{9.651936in}}{\pgfqpoint{9.280530in}{9.644122in}}%
\pgfpathcurveto{\pgfqpoint{9.272717in}{9.636308in}}{\pgfqpoint{9.268326in}{9.625709in}}{\pgfqpoint{9.268326in}{9.614659in}}%
\pgfpathcurveto{\pgfqpoint{9.268326in}{9.603609in}}{\pgfqpoint{9.272717in}{9.593010in}}{\pgfqpoint{9.280530in}{9.585197in}}%
\pgfpathcurveto{\pgfqpoint{9.288344in}{9.577383in}}{\pgfqpoint{9.298943in}{9.572993in}}{\pgfqpoint{9.309993in}{9.572993in}}%
\pgfpathlineto{\pgfqpoint{9.309993in}{9.572993in}}%
\pgfpathclose%
\pgfusepath{stroke,fill}%
\end{pgfscope}%
\begin{pgfscope}%
\pgfpathrectangle{\pgfqpoint{7.622482in}{7.624184in}}{\pgfqpoint{2.177280in}{2.201755in}}%
\pgfusepath{clip}%
\pgfsetbuttcap%
\pgfsetroundjoin%
\definecolor{currentfill}{rgb}{0.172549,0.627451,0.172549}%
\pgfsetfillcolor{currentfill}%
\pgfsetlinewidth{0.481800pt}%
\definecolor{currentstroke}{rgb}{1.000000,1.000000,1.000000}%
\pgfsetstrokecolor{currentstroke}%
\pgfsetdash{}{0pt}%
\pgfpathmoveto{\pgfqpoint{9.377747in}{8.794594in}}%
\pgfpathcurveto{\pgfqpoint{9.388797in}{8.794594in}}{\pgfqpoint{9.399396in}{8.798985in}}{\pgfqpoint{9.407210in}{8.806798in}}%
\pgfpathcurveto{\pgfqpoint{9.415023in}{8.814612in}}{\pgfqpoint{9.419414in}{8.825211in}}{\pgfqpoint{9.419414in}{8.836261in}}%
\pgfpathcurveto{\pgfqpoint{9.419414in}{8.847311in}}{\pgfqpoint{9.415023in}{8.857910in}}{\pgfqpoint{9.407210in}{8.865724in}}%
\pgfpathcurveto{\pgfqpoint{9.399396in}{8.873537in}}{\pgfqpoint{9.388797in}{8.877928in}}{\pgfqpoint{9.377747in}{8.877928in}}%
\pgfpathcurveto{\pgfqpoint{9.366697in}{8.877928in}}{\pgfqpoint{9.356098in}{8.873537in}}{\pgfqpoint{9.348284in}{8.865724in}}%
\pgfpathcurveto{\pgfqpoint{9.340471in}{8.857910in}}{\pgfqpoint{9.336080in}{8.847311in}}{\pgfqpoint{9.336080in}{8.836261in}}%
\pgfpathcurveto{\pgfqpoint{9.336080in}{8.825211in}}{\pgfqpoint{9.340471in}{8.814612in}}{\pgfqpoint{9.348284in}{8.806798in}}%
\pgfpathcurveto{\pgfqpoint{9.356098in}{8.798985in}}{\pgfqpoint{9.366697in}{8.794594in}}{\pgfqpoint{9.377747in}{8.794594in}}%
\pgfpathlineto{\pgfqpoint{9.377747in}{8.794594in}}%
\pgfpathclose%
\pgfusepath{stroke,fill}%
\end{pgfscope}%
\begin{pgfscope}%
\pgfpathrectangle{\pgfqpoint{7.622482in}{7.624184in}}{\pgfqpoint{2.177280in}{2.201755in}}%
\pgfusepath{clip}%
\pgfsetbuttcap%
\pgfsetroundjoin%
\definecolor{currentfill}{rgb}{0.172549,0.627451,0.172549}%
\pgfsetfillcolor{currentfill}%
\pgfsetlinewidth{0.481800pt}%
\definecolor{currentstroke}{rgb}{1.000000,1.000000,1.000000}%
\pgfsetstrokecolor{currentstroke}%
\pgfsetdash{}{0pt}%
\pgfpathmoveto{\pgfqpoint{8.971224in}{8.850194in}}%
\pgfpathcurveto{\pgfqpoint{8.982274in}{8.850194in}}{\pgfqpoint{8.992873in}{8.854584in}}{\pgfqpoint{9.000686in}{8.862398in}}%
\pgfpathcurveto{\pgfqpoint{9.008500in}{8.870212in}}{\pgfqpoint{9.012890in}{8.880811in}}{\pgfqpoint{9.012890in}{8.891861in}}%
\pgfpathcurveto{\pgfqpoint{9.012890in}{8.902911in}}{\pgfqpoint{9.008500in}{8.913510in}}{\pgfqpoint{9.000686in}{8.921324in}}%
\pgfpathcurveto{\pgfqpoint{8.992873in}{8.929137in}}{\pgfqpoint{8.982274in}{8.933528in}}{\pgfqpoint{8.971224in}{8.933528in}}%
\pgfpathcurveto{\pgfqpoint{8.960174in}{8.933528in}}{\pgfqpoint{8.949575in}{8.929137in}}{\pgfqpoint{8.941761in}{8.921324in}}%
\pgfpathcurveto{\pgfqpoint{8.933947in}{8.913510in}}{\pgfqpoint{8.929557in}{8.902911in}}{\pgfqpoint{8.929557in}{8.891861in}}%
\pgfpathcurveto{\pgfqpoint{8.929557in}{8.880811in}}{\pgfqpoint{8.933947in}{8.870212in}}{\pgfqpoint{8.941761in}{8.862398in}}%
\pgfpathcurveto{\pgfqpoint{8.949575in}{8.854584in}}{\pgfqpoint{8.960174in}{8.850194in}}{\pgfqpoint{8.971224in}{8.850194in}}%
\pgfpathlineto{\pgfqpoint{8.971224in}{8.850194in}}%
\pgfpathclose%
\pgfusepath{stroke,fill}%
\end{pgfscope}%
\begin{pgfscope}%
\pgfpathrectangle{\pgfqpoint{7.622482in}{7.624184in}}{\pgfqpoint{2.177280in}{2.201755in}}%
\pgfusepath{clip}%
\pgfsetbuttcap%
\pgfsetroundjoin%
\definecolor{currentfill}{rgb}{0.172549,0.627451,0.172549}%
\pgfsetfillcolor{currentfill}%
\pgfsetlinewidth{0.481800pt}%
\definecolor{currentstroke}{rgb}{1.000000,1.000000,1.000000}%
\pgfsetstrokecolor{currentstroke}%
\pgfsetdash{}{0pt}%
\pgfpathmoveto{\pgfqpoint{8.971224in}{8.627795in}}%
\pgfpathcurveto{\pgfqpoint{8.982274in}{8.627795in}}{\pgfqpoint{8.992873in}{8.632185in}}{\pgfqpoint{9.000686in}{8.639999in}}%
\pgfpathcurveto{\pgfqpoint{9.008500in}{8.647812in}}{\pgfqpoint{9.012890in}{8.658411in}}{\pgfqpoint{9.012890in}{8.669461in}}%
\pgfpathcurveto{\pgfqpoint{9.012890in}{8.680512in}}{\pgfqpoint{9.008500in}{8.691111in}}{\pgfqpoint{9.000686in}{8.698924in}}%
\pgfpathcurveto{\pgfqpoint{8.992873in}{8.706738in}}{\pgfqpoint{8.982274in}{8.711128in}}{\pgfqpoint{8.971224in}{8.711128in}}%
\pgfpathcurveto{\pgfqpoint{8.960174in}{8.711128in}}{\pgfqpoint{8.949575in}{8.706738in}}{\pgfqpoint{8.941761in}{8.698924in}}%
\pgfpathcurveto{\pgfqpoint{8.933947in}{8.691111in}}{\pgfqpoint{8.929557in}{8.680512in}}{\pgfqpoint{8.929557in}{8.669461in}}%
\pgfpathcurveto{\pgfqpoint{8.929557in}{8.658411in}}{\pgfqpoint{8.933947in}{8.647812in}}{\pgfqpoint{8.941761in}{8.639999in}}%
\pgfpathcurveto{\pgfqpoint{8.949575in}{8.632185in}}{\pgfqpoint{8.960174in}{8.627795in}}{\pgfqpoint{8.971224in}{8.627795in}}%
\pgfpathlineto{\pgfqpoint{8.971224in}{8.627795in}}%
\pgfpathclose%
\pgfusepath{stroke,fill}%
\end{pgfscope}%
\begin{pgfscope}%
\pgfpathrectangle{\pgfqpoint{7.622482in}{7.624184in}}{\pgfqpoint{2.177280in}{2.201755in}}%
\pgfusepath{clip}%
\pgfsetbuttcap%
\pgfsetroundjoin%
\definecolor{currentfill}{rgb}{0.172549,0.627451,0.172549}%
\pgfsetfillcolor{currentfill}%
\pgfsetlinewidth{0.481800pt}%
\definecolor{currentstroke}{rgb}{1.000000,1.000000,1.000000}%
\pgfsetstrokecolor{currentstroke}%
\pgfsetdash{}{0pt}%
\pgfpathmoveto{\pgfqpoint{9.174485in}{9.128194in}}%
\pgfpathcurveto{\pgfqpoint{9.185535in}{9.128194in}}{\pgfqpoint{9.196134in}{9.132584in}}{\pgfqpoint{9.203948in}{9.140397in}}%
\pgfpathcurveto{\pgfqpoint{9.211762in}{9.148211in}}{\pgfqpoint{9.216152in}{9.158810in}}{\pgfqpoint{9.216152in}{9.169860in}}%
\pgfpathcurveto{\pgfqpoint{9.216152in}{9.180910in}}{\pgfqpoint{9.211762in}{9.191509in}}{\pgfqpoint{9.203948in}{9.199323in}}%
\pgfpathcurveto{\pgfqpoint{9.196134in}{9.207137in}}{\pgfqpoint{9.185535in}{9.211527in}}{\pgfqpoint{9.174485in}{9.211527in}}%
\pgfpathcurveto{\pgfqpoint{9.163435in}{9.211527in}}{\pgfqpoint{9.152836in}{9.207137in}}{\pgfqpoint{9.145023in}{9.199323in}}%
\pgfpathcurveto{\pgfqpoint{9.137209in}{9.191509in}}{\pgfqpoint{9.132819in}{9.180910in}}{\pgfqpoint{9.132819in}{9.169860in}}%
\pgfpathcurveto{\pgfqpoint{9.132819in}{9.158810in}}{\pgfqpoint{9.137209in}{9.148211in}}{\pgfqpoint{9.145023in}{9.140397in}}%
\pgfpathcurveto{\pgfqpoint{9.152836in}{9.132584in}}{\pgfqpoint{9.163435in}{9.128194in}}{\pgfqpoint{9.174485in}{9.128194in}}%
\pgfpathlineto{\pgfqpoint{9.174485in}{9.128194in}}%
\pgfpathclose%
\pgfusepath{stroke,fill}%
\end{pgfscope}%
\begin{pgfscope}%
\pgfpathrectangle{\pgfqpoint{7.622482in}{7.624184in}}{\pgfqpoint{2.177280in}{2.201755in}}%
\pgfusepath{clip}%
\pgfsetbuttcap%
\pgfsetroundjoin%
\definecolor{currentfill}{rgb}{0.172549,0.627451,0.172549}%
\pgfsetfillcolor{currentfill}%
\pgfsetlinewidth{0.481800pt}%
\definecolor{currentstroke}{rgb}{1.000000,1.000000,1.000000}%
\pgfsetstrokecolor{currentstroke}%
\pgfsetdash{}{0pt}%
\pgfpathmoveto{\pgfqpoint{9.377747in}{9.016994in}}%
\pgfpathcurveto{\pgfqpoint{9.388797in}{9.016994in}}{\pgfqpoint{9.399396in}{9.021384in}}{\pgfqpoint{9.407210in}{9.029198in}}%
\pgfpathcurveto{\pgfqpoint{9.415023in}{9.037011in}}{\pgfqpoint{9.419414in}{9.047610in}}{\pgfqpoint{9.419414in}{9.058661in}}%
\pgfpathcurveto{\pgfqpoint{9.419414in}{9.069711in}}{\pgfqpoint{9.415023in}{9.080310in}}{\pgfqpoint{9.407210in}{9.088123in}}%
\pgfpathcurveto{\pgfqpoint{9.399396in}{9.095937in}}{\pgfqpoint{9.388797in}{9.100327in}}{\pgfqpoint{9.377747in}{9.100327in}}%
\pgfpathcurveto{\pgfqpoint{9.366697in}{9.100327in}}{\pgfqpoint{9.356098in}{9.095937in}}{\pgfqpoint{9.348284in}{9.088123in}}%
\pgfpathcurveto{\pgfqpoint{9.340471in}{9.080310in}}{\pgfqpoint{9.336080in}{9.069711in}}{\pgfqpoint{9.336080in}{9.058661in}}%
\pgfpathcurveto{\pgfqpoint{9.336080in}{9.047610in}}{\pgfqpoint{9.340471in}{9.037011in}}{\pgfqpoint{9.348284in}{9.029198in}}%
\pgfpathcurveto{\pgfqpoint{9.356098in}{9.021384in}}{\pgfqpoint{9.366697in}{9.016994in}}{\pgfqpoint{9.377747in}{9.016994in}}%
\pgfpathlineto{\pgfqpoint{9.377747in}{9.016994in}}%
\pgfpathclose%
\pgfusepath{stroke,fill}%
\end{pgfscope}%
\begin{pgfscope}%
\pgfpathrectangle{\pgfqpoint{7.622482in}{7.624184in}}{\pgfqpoint{2.177280in}{2.201755in}}%
\pgfusepath{clip}%
\pgfsetbuttcap%
\pgfsetroundjoin%
\definecolor{currentfill}{rgb}{0.172549,0.627451,0.172549}%
\pgfsetfillcolor{currentfill}%
\pgfsetlinewidth{0.481800pt}%
\definecolor{currentstroke}{rgb}{1.000000,1.000000,1.000000}%
\pgfsetstrokecolor{currentstroke}%
\pgfsetdash{}{0pt}%
\pgfpathmoveto{\pgfqpoint{9.309993in}{9.128194in}}%
\pgfpathcurveto{\pgfqpoint{9.321043in}{9.128194in}}{\pgfqpoint{9.331642in}{9.132584in}}{\pgfqpoint{9.339456in}{9.140397in}}%
\pgfpathcurveto{\pgfqpoint{9.347269in}{9.148211in}}{\pgfqpoint{9.351660in}{9.158810in}}{\pgfqpoint{9.351660in}{9.169860in}}%
\pgfpathcurveto{\pgfqpoint{9.351660in}{9.180910in}}{\pgfqpoint{9.347269in}{9.191509in}}{\pgfqpoint{9.339456in}{9.199323in}}%
\pgfpathcurveto{\pgfqpoint{9.331642in}{9.207137in}}{\pgfqpoint{9.321043in}{9.211527in}}{\pgfqpoint{9.309993in}{9.211527in}}%
\pgfpathcurveto{\pgfqpoint{9.298943in}{9.211527in}}{\pgfqpoint{9.288344in}{9.207137in}}{\pgfqpoint{9.280530in}{9.199323in}}%
\pgfpathcurveto{\pgfqpoint{9.272717in}{9.191509in}}{\pgfqpoint{9.268326in}{9.180910in}}{\pgfqpoint{9.268326in}{9.169860in}}%
\pgfpathcurveto{\pgfqpoint{9.268326in}{9.158810in}}{\pgfqpoint{9.272717in}{9.148211in}}{\pgfqpoint{9.280530in}{9.140397in}}%
\pgfpathcurveto{\pgfqpoint{9.288344in}{9.132584in}}{\pgfqpoint{9.298943in}{9.128194in}}{\pgfqpoint{9.309993in}{9.128194in}}%
\pgfpathlineto{\pgfqpoint{9.309993in}{9.128194in}}%
\pgfpathclose%
\pgfusepath{stroke,fill}%
\end{pgfscope}%
\begin{pgfscope}%
\pgfpathrectangle{\pgfqpoint{7.622482in}{7.624184in}}{\pgfqpoint{2.177280in}{2.201755in}}%
\pgfusepath{clip}%
\pgfsetbuttcap%
\pgfsetroundjoin%
\definecolor{currentfill}{rgb}{0.172549,0.627451,0.172549}%
\pgfsetfillcolor{currentfill}%
\pgfsetlinewidth{0.481800pt}%
\definecolor{currentstroke}{rgb}{1.000000,1.000000,1.000000}%
\pgfsetstrokecolor{currentstroke}%
\pgfsetdash{}{0pt}%
\pgfpathmoveto{\pgfqpoint{9.038978in}{8.516595in}}%
\pgfpathcurveto{\pgfqpoint{9.050028in}{8.516595in}}{\pgfqpoint{9.060627in}{8.520985in}}{\pgfqpoint{9.068440in}{8.528799in}}%
\pgfpathcurveto{\pgfqpoint{9.076254in}{8.536612in}}{\pgfqpoint{9.080644in}{8.547212in}}{\pgfqpoint{9.080644in}{8.558262in}}%
\pgfpathcurveto{\pgfqpoint{9.080644in}{8.569312in}}{\pgfqpoint{9.076254in}{8.579911in}}{\pgfqpoint{9.068440in}{8.587724in}}%
\pgfpathcurveto{\pgfqpoint{9.060627in}{8.595538in}}{\pgfqpoint{9.050028in}{8.599928in}}{\pgfqpoint{9.038978in}{8.599928in}}%
\pgfpathcurveto{\pgfqpoint{9.027927in}{8.599928in}}{\pgfqpoint{9.017328in}{8.595538in}}{\pgfqpoint{9.009515in}{8.587724in}}%
\pgfpathcurveto{\pgfqpoint{9.001701in}{8.579911in}}{\pgfqpoint{8.997311in}{8.569312in}}{\pgfqpoint{8.997311in}{8.558262in}}%
\pgfpathcurveto{\pgfqpoint{8.997311in}{8.547212in}}{\pgfqpoint{9.001701in}{8.536612in}}{\pgfqpoint{9.009515in}{8.528799in}}%
\pgfpathcurveto{\pgfqpoint{9.017328in}{8.520985in}}{\pgfqpoint{9.027927in}{8.516595in}}{\pgfqpoint{9.038978in}{8.516595in}}%
\pgfpathlineto{\pgfqpoint{9.038978in}{8.516595in}}%
\pgfpathclose%
\pgfusepath{stroke,fill}%
\end{pgfscope}%
\begin{pgfscope}%
\pgfpathrectangle{\pgfqpoint{7.622482in}{7.624184in}}{\pgfqpoint{2.177280in}{2.201755in}}%
\pgfusepath{clip}%
\pgfsetbuttcap%
\pgfsetroundjoin%
\definecolor{currentfill}{rgb}{0.172549,0.627451,0.172549}%
\pgfsetfillcolor{currentfill}%
\pgfsetlinewidth{0.481800pt}%
\definecolor{currentstroke}{rgb}{1.000000,1.000000,1.000000}%
\pgfsetstrokecolor{currentstroke}%
\pgfsetdash{}{0pt}%
\pgfpathmoveto{\pgfqpoint{9.309993in}{9.072594in}}%
\pgfpathcurveto{\pgfqpoint{9.321043in}{9.072594in}}{\pgfqpoint{9.331642in}{9.076984in}}{\pgfqpoint{9.339456in}{9.084798in}}%
\pgfpathcurveto{\pgfqpoint{9.347269in}{9.092611in}}{\pgfqpoint{9.351660in}{9.103210in}}{\pgfqpoint{9.351660in}{9.114260in}}%
\pgfpathcurveto{\pgfqpoint{9.351660in}{9.125311in}}{\pgfqpoint{9.347269in}{9.135910in}}{\pgfqpoint{9.339456in}{9.143723in}}%
\pgfpathcurveto{\pgfqpoint{9.331642in}{9.151537in}}{\pgfqpoint{9.321043in}{9.155927in}}{\pgfqpoint{9.309993in}{9.155927in}}%
\pgfpathcurveto{\pgfqpoint{9.298943in}{9.155927in}}{\pgfqpoint{9.288344in}{9.151537in}}{\pgfqpoint{9.280530in}{9.143723in}}%
\pgfpathcurveto{\pgfqpoint{9.272717in}{9.135910in}}{\pgfqpoint{9.268326in}{9.125311in}}{\pgfqpoint{9.268326in}{9.114260in}}%
\pgfpathcurveto{\pgfqpoint{9.268326in}{9.103210in}}{\pgfqpoint{9.272717in}{9.092611in}}{\pgfqpoint{9.280530in}{9.084798in}}%
\pgfpathcurveto{\pgfqpoint{9.288344in}{9.076984in}}{\pgfqpoint{9.298943in}{9.072594in}}{\pgfqpoint{9.309993in}{9.072594in}}%
\pgfpathlineto{\pgfqpoint{9.309993in}{9.072594in}}%
\pgfpathclose%
\pgfusepath{stroke,fill}%
\end{pgfscope}%
\begin{pgfscope}%
\pgfpathrectangle{\pgfqpoint{7.622482in}{7.624184in}}{\pgfqpoint{2.177280in}{2.201755in}}%
\pgfusepath{clip}%
\pgfsetbuttcap%
\pgfsetroundjoin%
\definecolor{currentfill}{rgb}{0.172549,0.627451,0.172549}%
\pgfsetfillcolor{currentfill}%
\pgfsetlinewidth{0.481800pt}%
\definecolor{currentstroke}{rgb}{1.000000,1.000000,1.000000}%
\pgfsetstrokecolor{currentstroke}%
\pgfsetdash{}{0pt}%
\pgfpathmoveto{\pgfqpoint{9.445501in}{9.016994in}}%
\pgfpathcurveto{\pgfqpoint{9.456551in}{9.016994in}}{\pgfqpoint{9.467150in}{9.021384in}}{\pgfqpoint{9.474964in}{9.029198in}}%
\pgfpathcurveto{\pgfqpoint{9.482777in}{9.037011in}}{\pgfqpoint{9.487167in}{9.047610in}}{\pgfqpoint{9.487167in}{9.058661in}}%
\pgfpathcurveto{\pgfqpoint{9.487167in}{9.069711in}}{\pgfqpoint{9.482777in}{9.080310in}}{\pgfqpoint{9.474964in}{9.088123in}}%
\pgfpathcurveto{\pgfqpoint{9.467150in}{9.095937in}}{\pgfqpoint{9.456551in}{9.100327in}}{\pgfqpoint{9.445501in}{9.100327in}}%
\pgfpathcurveto{\pgfqpoint{9.434451in}{9.100327in}}{\pgfqpoint{9.423852in}{9.095937in}}{\pgfqpoint{9.416038in}{9.088123in}}%
\pgfpathcurveto{\pgfqpoint{9.408224in}{9.080310in}}{\pgfqpoint{9.403834in}{9.069711in}}{\pgfqpoint{9.403834in}{9.058661in}}%
\pgfpathcurveto{\pgfqpoint{9.403834in}{9.047610in}}{\pgfqpoint{9.408224in}{9.037011in}}{\pgfqpoint{9.416038in}{9.029198in}}%
\pgfpathcurveto{\pgfqpoint{9.423852in}{9.021384in}}{\pgfqpoint{9.434451in}{9.016994in}}{\pgfqpoint{9.445501in}{9.016994in}}%
\pgfpathlineto{\pgfqpoint{9.445501in}{9.016994in}}%
\pgfpathclose%
\pgfusepath{stroke,fill}%
\end{pgfscope}%
\begin{pgfscope}%
\pgfpathrectangle{\pgfqpoint{7.622482in}{7.624184in}}{\pgfqpoint{2.177280in}{2.201755in}}%
\pgfusepath{clip}%
\pgfsetbuttcap%
\pgfsetroundjoin%
\definecolor{currentfill}{rgb}{0.172549,0.627451,0.172549}%
\pgfsetfillcolor{currentfill}%
\pgfsetlinewidth{0.481800pt}%
\definecolor{currentstroke}{rgb}{1.000000,1.000000,1.000000}%
\pgfsetstrokecolor{currentstroke}%
\pgfsetdash{}{0pt}%
\pgfpathmoveto{\pgfqpoint{9.309993in}{9.016994in}}%
\pgfpathcurveto{\pgfqpoint{9.321043in}{9.016994in}}{\pgfqpoint{9.331642in}{9.021384in}}{\pgfqpoint{9.339456in}{9.029198in}}%
\pgfpathcurveto{\pgfqpoint{9.347269in}{9.037011in}}{\pgfqpoint{9.351660in}{9.047610in}}{\pgfqpoint{9.351660in}{9.058661in}}%
\pgfpathcurveto{\pgfqpoint{9.351660in}{9.069711in}}{\pgfqpoint{9.347269in}{9.080310in}}{\pgfqpoint{9.339456in}{9.088123in}}%
\pgfpathcurveto{\pgfqpoint{9.331642in}{9.095937in}}{\pgfqpoint{9.321043in}{9.100327in}}{\pgfqpoint{9.309993in}{9.100327in}}%
\pgfpathcurveto{\pgfqpoint{9.298943in}{9.100327in}}{\pgfqpoint{9.288344in}{9.095937in}}{\pgfqpoint{9.280530in}{9.088123in}}%
\pgfpathcurveto{\pgfqpoint{9.272717in}{9.080310in}}{\pgfqpoint{9.268326in}{9.069711in}}{\pgfqpoint{9.268326in}{9.058661in}}%
\pgfpathcurveto{\pgfqpoint{9.268326in}{9.047610in}}{\pgfqpoint{9.272717in}{9.037011in}}{\pgfqpoint{9.280530in}{9.029198in}}%
\pgfpathcurveto{\pgfqpoint{9.288344in}{9.021384in}}{\pgfqpoint{9.298943in}{9.016994in}}{\pgfqpoint{9.309993in}{9.016994in}}%
\pgfpathlineto{\pgfqpoint{9.309993in}{9.016994in}}%
\pgfpathclose%
\pgfusepath{stroke,fill}%
\end{pgfscope}%
\begin{pgfscope}%
\pgfpathrectangle{\pgfqpoint{7.622482in}{7.624184in}}{\pgfqpoint{2.177280in}{2.201755in}}%
\pgfusepath{clip}%
\pgfsetbuttcap%
\pgfsetroundjoin%
\definecolor{currentfill}{rgb}{0.172549,0.627451,0.172549}%
\pgfsetfillcolor{currentfill}%
\pgfsetlinewidth{0.481800pt}%
\definecolor{currentstroke}{rgb}{1.000000,1.000000,1.000000}%
\pgfsetstrokecolor{currentstroke}%
\pgfsetdash{}{0pt}%
\pgfpathmoveto{\pgfqpoint{9.038978in}{8.794594in}}%
\pgfpathcurveto{\pgfqpoint{9.050028in}{8.794594in}}{\pgfqpoint{9.060627in}{8.798985in}}{\pgfqpoint{9.068440in}{8.806798in}}%
\pgfpathcurveto{\pgfqpoint{9.076254in}{8.814612in}}{\pgfqpoint{9.080644in}{8.825211in}}{\pgfqpoint{9.080644in}{8.836261in}}%
\pgfpathcurveto{\pgfqpoint{9.080644in}{8.847311in}}{\pgfqpoint{9.076254in}{8.857910in}}{\pgfqpoint{9.068440in}{8.865724in}}%
\pgfpathcurveto{\pgfqpoint{9.060627in}{8.873537in}}{\pgfqpoint{9.050028in}{8.877928in}}{\pgfqpoint{9.038978in}{8.877928in}}%
\pgfpathcurveto{\pgfqpoint{9.027927in}{8.877928in}}{\pgfqpoint{9.017328in}{8.873537in}}{\pgfqpoint{9.009515in}{8.865724in}}%
\pgfpathcurveto{\pgfqpoint{9.001701in}{8.857910in}}{\pgfqpoint{8.997311in}{8.847311in}}{\pgfqpoint{8.997311in}{8.836261in}}%
\pgfpathcurveto{\pgfqpoint{8.997311in}{8.825211in}}{\pgfqpoint{9.001701in}{8.814612in}}{\pgfqpoint{9.009515in}{8.806798in}}%
\pgfpathcurveto{\pgfqpoint{9.017328in}{8.798985in}}{\pgfqpoint{9.027927in}{8.794594in}}{\pgfqpoint{9.038978in}{8.794594in}}%
\pgfpathlineto{\pgfqpoint{9.038978in}{8.794594in}}%
\pgfpathclose%
\pgfusepath{stroke,fill}%
\end{pgfscope}%
\begin{pgfscope}%
\pgfpathrectangle{\pgfqpoint{7.622482in}{7.624184in}}{\pgfqpoint{2.177280in}{2.201755in}}%
\pgfusepath{clip}%
\pgfsetbuttcap%
\pgfsetroundjoin%
\definecolor{currentfill}{rgb}{0.172549,0.627451,0.172549}%
\pgfsetfillcolor{currentfill}%
\pgfsetlinewidth{0.481800pt}%
\definecolor{currentstroke}{rgb}{1.000000,1.000000,1.000000}%
\pgfsetstrokecolor{currentstroke}%
\pgfsetdash{}{0pt}%
\pgfpathmoveto{\pgfqpoint{9.106731in}{8.905794in}}%
\pgfpathcurveto{\pgfqpoint{9.117782in}{8.905794in}}{\pgfqpoint{9.128381in}{8.910184in}}{\pgfqpoint{9.136194in}{8.917998in}}%
\pgfpathcurveto{\pgfqpoint{9.144008in}{8.925812in}}{\pgfqpoint{9.148398in}{8.936411in}}{\pgfqpoint{9.148398in}{8.947461in}}%
\pgfpathcurveto{\pgfqpoint{9.148398in}{8.958511in}}{\pgfqpoint{9.144008in}{8.969110in}}{\pgfqpoint{9.136194in}{8.976924in}}%
\pgfpathcurveto{\pgfqpoint{9.128381in}{8.984737in}}{\pgfqpoint{9.117782in}{8.989127in}}{\pgfqpoint{9.106731in}{8.989127in}}%
\pgfpathcurveto{\pgfqpoint{9.095681in}{8.989127in}}{\pgfqpoint{9.085082in}{8.984737in}}{\pgfqpoint{9.077269in}{8.976924in}}%
\pgfpathcurveto{\pgfqpoint{9.069455in}{8.969110in}}{\pgfqpoint{9.065065in}{8.958511in}}{\pgfqpoint{9.065065in}{8.947461in}}%
\pgfpathcurveto{\pgfqpoint{9.065065in}{8.936411in}}{\pgfqpoint{9.069455in}{8.925812in}}{\pgfqpoint{9.077269in}{8.917998in}}%
\pgfpathcurveto{\pgfqpoint{9.085082in}{8.910184in}}{\pgfqpoint{9.095681in}{8.905794in}}{\pgfqpoint{9.106731in}{8.905794in}}%
\pgfpathlineto{\pgfqpoint{9.106731in}{8.905794in}}%
\pgfpathclose%
\pgfusepath{stroke,fill}%
\end{pgfscope}%
\begin{pgfscope}%
\pgfpathrectangle{\pgfqpoint{7.622482in}{7.624184in}}{\pgfqpoint{2.177280in}{2.201755in}}%
\pgfusepath{clip}%
\pgfsetbuttcap%
\pgfsetroundjoin%
\definecolor{currentfill}{rgb}{0.172549,0.627451,0.172549}%
\pgfsetfillcolor{currentfill}%
\pgfsetlinewidth{0.481800pt}%
\definecolor{currentstroke}{rgb}{1.000000,1.000000,1.000000}%
\pgfsetstrokecolor{currentstroke}%
\pgfsetdash{}{0pt}%
\pgfpathmoveto{\pgfqpoint{9.309993in}{8.738994in}}%
\pgfpathcurveto{\pgfqpoint{9.321043in}{8.738994in}}{\pgfqpoint{9.331642in}{8.743385in}}{\pgfqpoint{9.339456in}{8.751198in}}%
\pgfpathcurveto{\pgfqpoint{9.347269in}{8.759012in}}{\pgfqpoint{9.351660in}{8.769611in}}{\pgfqpoint{9.351660in}{8.780661in}}%
\pgfpathcurveto{\pgfqpoint{9.351660in}{8.791711in}}{\pgfqpoint{9.347269in}{8.802310in}}{\pgfqpoint{9.339456in}{8.810124in}}%
\pgfpathcurveto{\pgfqpoint{9.331642in}{8.817938in}}{\pgfqpoint{9.321043in}{8.822328in}}{\pgfqpoint{9.309993in}{8.822328in}}%
\pgfpathcurveto{\pgfqpoint{9.298943in}{8.822328in}}{\pgfqpoint{9.288344in}{8.817938in}}{\pgfqpoint{9.280530in}{8.810124in}}%
\pgfpathcurveto{\pgfqpoint{9.272717in}{8.802310in}}{\pgfqpoint{9.268326in}{8.791711in}}{\pgfqpoint{9.268326in}{8.780661in}}%
\pgfpathcurveto{\pgfqpoint{9.268326in}{8.769611in}}{\pgfqpoint{9.272717in}{8.759012in}}{\pgfqpoint{9.280530in}{8.751198in}}%
\pgfpathcurveto{\pgfqpoint{9.288344in}{8.743385in}}{\pgfqpoint{9.298943in}{8.738994in}}{\pgfqpoint{9.309993in}{8.738994in}}%
\pgfpathlineto{\pgfqpoint{9.309993in}{8.738994in}}%
\pgfpathclose%
\pgfusepath{stroke,fill}%
\end{pgfscope}%
\begin{pgfscope}%
\pgfpathrectangle{\pgfqpoint{7.622482in}{7.624184in}}{\pgfqpoint{2.177280in}{2.201755in}}%
\pgfusepath{clip}%
\pgfsetbuttcap%
\pgfsetroundjoin%
\definecolor{currentfill}{rgb}{0.172549,0.627451,0.172549}%
\pgfsetfillcolor{currentfill}%
\pgfsetlinewidth{0.481800pt}%
\definecolor{currentstroke}{rgb}{1.000000,1.000000,1.000000}%
\pgfsetstrokecolor{currentstroke}%
\pgfsetdash{}{0pt}%
\pgfpathmoveto{\pgfqpoint{8.971224in}{8.572195in}}%
\pgfpathcurveto{\pgfqpoint{8.982274in}{8.572195in}}{\pgfqpoint{8.992873in}{8.576585in}}{\pgfqpoint{9.000686in}{8.584399in}}%
\pgfpathcurveto{\pgfqpoint{9.008500in}{8.592212in}}{\pgfqpoint{9.012890in}{8.602811in}}{\pgfqpoint{9.012890in}{8.613862in}}%
\pgfpathcurveto{\pgfqpoint{9.012890in}{8.624912in}}{\pgfqpoint{9.008500in}{8.635511in}}{\pgfqpoint{9.000686in}{8.643324in}}%
\pgfpathcurveto{\pgfqpoint{8.992873in}{8.651138in}}{\pgfqpoint{8.982274in}{8.655528in}}{\pgfqpoint{8.971224in}{8.655528in}}%
\pgfpathcurveto{\pgfqpoint{8.960174in}{8.655528in}}{\pgfqpoint{8.949575in}{8.651138in}}{\pgfqpoint{8.941761in}{8.643324in}}%
\pgfpathcurveto{\pgfqpoint{8.933947in}{8.635511in}}{\pgfqpoint{8.929557in}{8.624912in}}{\pgfqpoint{8.929557in}{8.613862in}}%
\pgfpathcurveto{\pgfqpoint{8.929557in}{8.602811in}}{\pgfqpoint{8.933947in}{8.592212in}}{\pgfqpoint{8.941761in}{8.584399in}}%
\pgfpathcurveto{\pgfqpoint{8.949575in}{8.576585in}}{\pgfqpoint{8.960174in}{8.572195in}}{\pgfqpoint{8.971224in}{8.572195in}}%
\pgfpathlineto{\pgfqpoint{8.971224in}{8.572195in}}%
\pgfpathclose%
\pgfusepath{stroke,fill}%
\end{pgfscope}%
\begin{pgfscope}%
\pgfpathrectangle{\pgfqpoint{7.622482in}{7.624184in}}{\pgfqpoint{2.177280in}{2.201755in}}%
\pgfusepath{clip}%
\pgfsetbuttcap%
\pgfsetroundjoin%
\definecolor{currentfill}{rgb}{0.121569,0.466667,0.705882}%
\pgfsetfillcolor{currentfill}%
\pgfsetlinewidth{1.003750pt}%
\definecolor{currentstroke}{rgb}{0.121569,0.466667,0.705882}%
\pgfsetstrokecolor{currentstroke}%
\pgfsetdash{}{0pt}%
\pgfsys@defobject{currentmarker}{\pgfqpoint{-0.041667in}{-0.041667in}}{\pgfqpoint{0.041667in}{0.041667in}}{%
\pgfpathmoveto{\pgfqpoint{0.000000in}{-0.041667in}}%
\pgfpathcurveto{\pgfqpoint{0.011050in}{-0.041667in}}{\pgfqpoint{0.021649in}{-0.037276in}}{\pgfqpoint{0.029463in}{-0.029463in}}%
\pgfpathcurveto{\pgfqpoint{0.037276in}{-0.021649in}}{\pgfqpoint{0.041667in}{-0.011050in}}{\pgfqpoint{0.041667in}{0.000000in}}%
\pgfpathcurveto{\pgfqpoint{0.041667in}{0.011050in}}{\pgfqpoint{0.037276in}{0.021649in}}{\pgfqpoint{0.029463in}{0.029463in}}%
\pgfpathcurveto{\pgfqpoint{0.021649in}{0.037276in}}{\pgfqpoint{0.011050in}{0.041667in}}{\pgfqpoint{0.000000in}{0.041667in}}%
\pgfpathcurveto{\pgfqpoint{-0.011050in}{0.041667in}}{\pgfqpoint{-0.021649in}{0.037276in}}{\pgfqpoint{-0.029463in}{0.029463in}}%
\pgfpathcurveto{\pgfqpoint{-0.037276in}{0.021649in}}{\pgfqpoint{-0.041667in}{0.011050in}}{\pgfqpoint{-0.041667in}{0.000000in}}%
\pgfpathcurveto{\pgfqpoint{-0.041667in}{-0.011050in}}{\pgfqpoint{-0.037276in}{-0.021649in}}{\pgfqpoint{-0.029463in}{-0.029463in}}%
\pgfpathcurveto{\pgfqpoint{-0.021649in}{-0.037276in}}{\pgfqpoint{-0.011050in}{-0.041667in}}{\pgfqpoint{0.000000in}{-0.041667in}}%
\pgfpathlineto{\pgfqpoint{0.000000in}{-0.041667in}}%
\pgfpathclose%
\pgfusepath{stroke,fill}%
}%
\end{pgfscope}%
\begin{pgfscope}%
\pgfpathrectangle{\pgfqpoint{7.622482in}{7.624184in}}{\pgfqpoint{2.177280in}{2.201755in}}%
\pgfusepath{clip}%
\pgfsetbuttcap%
\pgfsetroundjoin%
\definecolor{currentfill}{rgb}{1.000000,0.498039,0.054902}%
\pgfsetfillcolor{currentfill}%
\pgfsetlinewidth{1.003750pt}%
\definecolor{currentstroke}{rgb}{1.000000,0.498039,0.054902}%
\pgfsetstrokecolor{currentstroke}%
\pgfsetdash{}{0pt}%
\pgfsys@defobject{currentmarker}{\pgfqpoint{-0.041667in}{-0.041667in}}{\pgfqpoint{0.041667in}{0.041667in}}{%
\pgfpathmoveto{\pgfqpoint{0.000000in}{-0.041667in}}%
\pgfpathcurveto{\pgfqpoint{0.011050in}{-0.041667in}}{\pgfqpoint{0.021649in}{-0.037276in}}{\pgfqpoint{0.029463in}{-0.029463in}}%
\pgfpathcurveto{\pgfqpoint{0.037276in}{-0.021649in}}{\pgfqpoint{0.041667in}{-0.011050in}}{\pgfqpoint{0.041667in}{0.000000in}}%
\pgfpathcurveto{\pgfqpoint{0.041667in}{0.011050in}}{\pgfqpoint{0.037276in}{0.021649in}}{\pgfqpoint{0.029463in}{0.029463in}}%
\pgfpathcurveto{\pgfqpoint{0.021649in}{0.037276in}}{\pgfqpoint{0.011050in}{0.041667in}}{\pgfqpoint{0.000000in}{0.041667in}}%
\pgfpathcurveto{\pgfqpoint{-0.011050in}{0.041667in}}{\pgfqpoint{-0.021649in}{0.037276in}}{\pgfqpoint{-0.029463in}{0.029463in}}%
\pgfpathcurveto{\pgfqpoint{-0.037276in}{0.021649in}}{\pgfqpoint{-0.041667in}{0.011050in}}{\pgfqpoint{-0.041667in}{0.000000in}}%
\pgfpathcurveto{\pgfqpoint{-0.041667in}{-0.011050in}}{\pgfqpoint{-0.037276in}{-0.021649in}}{\pgfqpoint{-0.029463in}{-0.029463in}}%
\pgfpathcurveto{\pgfqpoint{-0.021649in}{-0.037276in}}{\pgfqpoint{-0.011050in}{-0.041667in}}{\pgfqpoint{0.000000in}{-0.041667in}}%
\pgfpathlineto{\pgfqpoint{0.000000in}{-0.041667in}}%
\pgfpathclose%
\pgfusepath{stroke,fill}%
}%
\end{pgfscope}%
\begin{pgfscope}%
\pgfpathrectangle{\pgfqpoint{7.622482in}{7.624184in}}{\pgfqpoint{2.177280in}{2.201755in}}%
\pgfusepath{clip}%
\pgfsetbuttcap%
\pgfsetroundjoin%
\definecolor{currentfill}{rgb}{0.172549,0.627451,0.172549}%
\pgfsetfillcolor{currentfill}%
\pgfsetlinewidth{1.003750pt}%
\definecolor{currentstroke}{rgb}{0.172549,0.627451,0.172549}%
\pgfsetstrokecolor{currentstroke}%
\pgfsetdash{}{0pt}%
\pgfsys@defobject{currentmarker}{\pgfqpoint{-0.041667in}{-0.041667in}}{\pgfqpoint{0.041667in}{0.041667in}}{%
\pgfpathmoveto{\pgfqpoint{0.000000in}{-0.041667in}}%
\pgfpathcurveto{\pgfqpoint{0.011050in}{-0.041667in}}{\pgfqpoint{0.021649in}{-0.037276in}}{\pgfqpoint{0.029463in}{-0.029463in}}%
\pgfpathcurveto{\pgfqpoint{0.037276in}{-0.021649in}}{\pgfqpoint{0.041667in}{-0.011050in}}{\pgfqpoint{0.041667in}{0.000000in}}%
\pgfpathcurveto{\pgfqpoint{0.041667in}{0.011050in}}{\pgfqpoint{0.037276in}{0.021649in}}{\pgfqpoint{0.029463in}{0.029463in}}%
\pgfpathcurveto{\pgfqpoint{0.021649in}{0.037276in}}{\pgfqpoint{0.011050in}{0.041667in}}{\pgfqpoint{0.000000in}{0.041667in}}%
\pgfpathcurveto{\pgfqpoint{-0.011050in}{0.041667in}}{\pgfqpoint{-0.021649in}{0.037276in}}{\pgfqpoint{-0.029463in}{0.029463in}}%
\pgfpathcurveto{\pgfqpoint{-0.037276in}{0.021649in}}{\pgfqpoint{-0.041667in}{0.011050in}}{\pgfqpoint{-0.041667in}{0.000000in}}%
\pgfpathcurveto{\pgfqpoint{-0.041667in}{-0.011050in}}{\pgfqpoint{-0.037276in}{-0.021649in}}{\pgfqpoint{-0.029463in}{-0.029463in}}%
\pgfpathcurveto{\pgfqpoint{-0.021649in}{-0.037276in}}{\pgfqpoint{-0.011050in}{-0.041667in}}{\pgfqpoint{0.000000in}{-0.041667in}}%
\pgfpathlineto{\pgfqpoint{0.000000in}{-0.041667in}}%
\pgfpathclose%
\pgfusepath{stroke,fill}%
}%
\end{pgfscope}%
\begin{pgfscope}%
\pgfsetbuttcap%
\pgfsetroundjoin%
\definecolor{currentfill}{rgb}{0.000000,0.000000,0.000000}%
\pgfsetfillcolor{currentfill}%
\pgfsetlinewidth{0.803000pt}%
\definecolor{currentstroke}{rgb}{0.000000,0.000000,0.000000}%
\pgfsetstrokecolor{currentstroke}%
\pgfsetdash{}{0pt}%
\pgfsys@defobject{currentmarker}{\pgfqpoint{0.000000in}{-0.048611in}}{\pgfqpoint{0.000000in}{0.000000in}}{%
\pgfpathmoveto{\pgfqpoint{0.000000in}{0.000000in}}%
\pgfpathlineto{\pgfqpoint{0.000000in}{-0.048611in}}%
\pgfusepath{stroke,fill}%
}%
\begin{pgfscope}%
\pgfsys@transformshift{7.751654in}{7.624184in}%
\pgfsys@useobject{currentmarker}{}%
\end{pgfscope}%
\end{pgfscope}%
\begin{pgfscope}%
\pgfsetbuttcap%
\pgfsetroundjoin%
\definecolor{currentfill}{rgb}{0.000000,0.000000,0.000000}%
\pgfsetfillcolor{currentfill}%
\pgfsetlinewidth{0.803000pt}%
\definecolor{currentstroke}{rgb}{0.000000,0.000000,0.000000}%
\pgfsetstrokecolor{currentstroke}%
\pgfsetdash{}{0pt}%
\pgfsys@defobject{currentmarker}{\pgfqpoint{0.000000in}{-0.048611in}}{\pgfqpoint{0.000000in}{0.000000in}}{%
\pgfpathmoveto{\pgfqpoint{0.000000in}{0.000000in}}%
\pgfpathlineto{\pgfqpoint{0.000000in}{-0.048611in}}%
\pgfusepath{stroke,fill}%
}%
\begin{pgfscope}%
\pgfsys@transformshift{8.429193in}{7.624184in}%
\pgfsys@useobject{currentmarker}{}%
\end{pgfscope}%
\end{pgfscope}%
\begin{pgfscope}%
\pgfsetbuttcap%
\pgfsetroundjoin%
\definecolor{currentfill}{rgb}{0.000000,0.000000,0.000000}%
\pgfsetfillcolor{currentfill}%
\pgfsetlinewidth{0.803000pt}%
\definecolor{currentstroke}{rgb}{0.000000,0.000000,0.000000}%
\pgfsetstrokecolor{currentstroke}%
\pgfsetdash{}{0pt}%
\pgfsys@defobject{currentmarker}{\pgfqpoint{0.000000in}{-0.048611in}}{\pgfqpoint{0.000000in}{0.000000in}}{%
\pgfpathmoveto{\pgfqpoint{0.000000in}{0.000000in}}%
\pgfpathlineto{\pgfqpoint{0.000000in}{-0.048611in}}%
\pgfusepath{stroke,fill}%
}%
\begin{pgfscope}%
\pgfsys@transformshift{9.106731in}{7.624184in}%
\pgfsys@useobject{currentmarker}{}%
\end{pgfscope}%
\end{pgfscope}%
\begin{pgfscope}%
\pgfsetbuttcap%
\pgfsetroundjoin%
\definecolor{currentfill}{rgb}{0.000000,0.000000,0.000000}%
\pgfsetfillcolor{currentfill}%
\pgfsetlinewidth{0.803000pt}%
\definecolor{currentstroke}{rgb}{0.000000,0.000000,0.000000}%
\pgfsetstrokecolor{currentstroke}%
\pgfsetdash{}{0pt}%
\pgfsys@defobject{currentmarker}{\pgfqpoint{0.000000in}{-0.048611in}}{\pgfqpoint{0.000000in}{0.000000in}}{%
\pgfpathmoveto{\pgfqpoint{0.000000in}{0.000000in}}%
\pgfpathlineto{\pgfqpoint{0.000000in}{-0.048611in}}%
\pgfusepath{stroke,fill}%
}%
\begin{pgfscope}%
\pgfsys@transformshift{9.784270in}{7.624184in}%
\pgfsys@useobject{currentmarker}{}%
\end{pgfscope}%
\end{pgfscope}%
\begin{pgfscope}%
\pgfsetbuttcap%
\pgfsetroundjoin%
\definecolor{currentfill}{rgb}{0.000000,0.000000,0.000000}%
\pgfsetfillcolor{currentfill}%
\pgfsetlinewidth{0.803000pt}%
\definecolor{currentstroke}{rgb}{0.000000,0.000000,0.000000}%
\pgfsetstrokecolor{currentstroke}%
\pgfsetdash{}{0pt}%
\pgfsys@defobject{currentmarker}{\pgfqpoint{-0.048611in}{0.000000in}}{\pgfqpoint{-0.000000in}{0.000000in}}{%
\pgfpathmoveto{\pgfqpoint{-0.000000in}{0.000000in}}%
\pgfpathlineto{\pgfqpoint{-0.048611in}{0.000000in}}%
\pgfusepath{stroke,fill}%
}%
\begin{pgfscope}%
\pgfsys@transformshift{7.622482in}{8.113463in}%
\pgfsys@useobject{currentmarker}{}%
\end{pgfscope}%
\end{pgfscope}%
\begin{pgfscope}%
\pgfsetbuttcap%
\pgfsetroundjoin%
\definecolor{currentfill}{rgb}{0.000000,0.000000,0.000000}%
\pgfsetfillcolor{currentfill}%
\pgfsetlinewidth{0.803000pt}%
\definecolor{currentstroke}{rgb}{0.000000,0.000000,0.000000}%
\pgfsetstrokecolor{currentstroke}%
\pgfsetdash{}{0pt}%
\pgfsys@defobject{currentmarker}{\pgfqpoint{-0.048611in}{0.000000in}}{\pgfqpoint{-0.000000in}{0.000000in}}{%
\pgfpathmoveto{\pgfqpoint{-0.000000in}{0.000000in}}%
\pgfpathlineto{\pgfqpoint{-0.048611in}{0.000000in}}%
\pgfusepath{stroke,fill}%
}%
\begin{pgfscope}%
\pgfsys@transformshift{7.622482in}{8.669461in}%
\pgfsys@useobject{currentmarker}{}%
\end{pgfscope}%
\end{pgfscope}%
\begin{pgfscope}%
\pgfsetbuttcap%
\pgfsetroundjoin%
\definecolor{currentfill}{rgb}{0.000000,0.000000,0.000000}%
\pgfsetfillcolor{currentfill}%
\pgfsetlinewidth{0.803000pt}%
\definecolor{currentstroke}{rgb}{0.000000,0.000000,0.000000}%
\pgfsetstrokecolor{currentstroke}%
\pgfsetdash{}{0pt}%
\pgfsys@defobject{currentmarker}{\pgfqpoint{-0.048611in}{0.000000in}}{\pgfqpoint{-0.000000in}{0.000000in}}{%
\pgfpathmoveto{\pgfqpoint{-0.000000in}{0.000000in}}%
\pgfpathlineto{\pgfqpoint{-0.048611in}{0.000000in}}%
\pgfusepath{stroke,fill}%
}%
\begin{pgfscope}%
\pgfsys@transformshift{7.622482in}{9.225460in}%
\pgfsys@useobject{currentmarker}{}%
\end{pgfscope}%
\end{pgfscope}%
\begin{pgfscope}%
\pgfsetbuttcap%
\pgfsetroundjoin%
\definecolor{currentfill}{rgb}{0.000000,0.000000,0.000000}%
\pgfsetfillcolor{currentfill}%
\pgfsetlinewidth{0.803000pt}%
\definecolor{currentstroke}{rgb}{0.000000,0.000000,0.000000}%
\pgfsetstrokecolor{currentstroke}%
\pgfsetdash{}{0pt}%
\pgfsys@defobject{currentmarker}{\pgfqpoint{-0.048611in}{0.000000in}}{\pgfqpoint{-0.000000in}{0.000000in}}{%
\pgfpathmoveto{\pgfqpoint{-0.000000in}{0.000000in}}%
\pgfpathlineto{\pgfqpoint{-0.048611in}{0.000000in}}%
\pgfusepath{stroke,fill}%
}%
\begin{pgfscope}%
\pgfsys@transformshift{7.622482in}{9.781459in}%
\pgfsys@useobject{currentmarker}{}%
\end{pgfscope}%
\end{pgfscope}%
\begin{pgfscope}%
\pgfsetrectcap%
\pgfsetmiterjoin%
\pgfsetlinewidth{0.803000pt}%
\definecolor{currentstroke}{rgb}{0.000000,0.000000,0.000000}%
\pgfsetstrokecolor{currentstroke}%
\pgfsetdash{}{0pt}%
\pgfpathmoveto{\pgfqpoint{7.622482in}{7.624184in}}%
\pgfpathlineto{\pgfqpoint{7.622482in}{9.825939in}}%
\pgfusepath{stroke}%
\end{pgfscope}%
\begin{pgfscope}%
\pgfsetrectcap%
\pgfsetmiterjoin%
\pgfsetlinewidth{0.803000pt}%
\definecolor{currentstroke}{rgb}{0.000000,0.000000,0.000000}%
\pgfsetstrokecolor{currentstroke}%
\pgfsetdash{}{0pt}%
\pgfpathmoveto{\pgfqpoint{7.622482in}{7.624184in}}%
\pgfpathlineto{\pgfqpoint{9.799762in}{7.624184in}}%
\pgfusepath{stroke}%
\end{pgfscope}%
\begin{pgfscope}%
\pgfsetbuttcap%
\pgfsetmiterjoin%
\definecolor{currentfill}{rgb}{1.000000,1.000000,1.000000}%
\pgfsetfillcolor{currentfill}%
\pgfsetlinewidth{0.000000pt}%
\definecolor{currentstroke}{rgb}{0.000000,0.000000,0.000000}%
\pgfsetstrokecolor{currentstroke}%
\pgfsetstrokeopacity{0.000000}%
\pgfsetdash{}{0pt}%
\pgfpathmoveto{\pgfqpoint{0.633874in}{5.272501in}}%
\pgfpathlineto{\pgfqpoint{2.811154in}{5.272501in}}%
\pgfpathlineto{\pgfqpoint{2.811154in}{7.474256in}}%
\pgfpathlineto{\pgfqpoint{0.633874in}{7.474256in}}%
\pgfpathlineto{\pgfqpoint{0.633874in}{5.272501in}}%
\pgfpathclose%
\pgfusepath{fill}%
\end{pgfscope}%
\begin{pgfscope}%
\pgfpathrectangle{\pgfqpoint{0.633874in}{5.272501in}}{\pgfqpoint{2.177280in}{2.201755in}}%
\pgfusepath{clip}%
\pgfsetbuttcap%
\pgfsetroundjoin%
\definecolor{currentfill}{rgb}{0.121569,0.466667,0.705882}%
\pgfsetfillcolor{currentfill}%
\pgfsetlinewidth{0.481800pt}%
\definecolor{currentstroke}{rgb}{1.000000,1.000000,1.000000}%
\pgfsetstrokecolor{currentstroke}%
\pgfsetdash{}{0pt}%
\pgfpathmoveto{\pgfqpoint{1.245489in}{6.581911in}}%
\pgfpathcurveto{\pgfqpoint{1.256539in}{6.581911in}}{\pgfqpoint{1.267138in}{6.586302in}}{\pgfqpoint{1.274952in}{6.594115in}}%
\pgfpathcurveto{\pgfqpoint{1.282765in}{6.601929in}}{\pgfqpoint{1.287155in}{6.612528in}}{\pgfqpoint{1.287155in}{6.623578in}}%
\pgfpathcurveto{\pgfqpoint{1.287155in}{6.634628in}}{\pgfqpoint{1.282765in}{6.645227in}}{\pgfqpoint{1.274952in}{6.653041in}}%
\pgfpathcurveto{\pgfqpoint{1.267138in}{6.660854in}}{\pgfqpoint{1.256539in}{6.665245in}}{\pgfqpoint{1.245489in}{6.665245in}}%
\pgfpathcurveto{\pgfqpoint{1.234439in}{6.665245in}}{\pgfqpoint{1.223840in}{6.660854in}}{\pgfqpoint{1.216026in}{6.653041in}}%
\pgfpathcurveto{\pgfqpoint{1.208212in}{6.645227in}}{\pgfqpoint{1.203822in}{6.634628in}}{\pgfqpoint{1.203822in}{6.623578in}}%
\pgfpathcurveto{\pgfqpoint{1.203822in}{6.612528in}}{\pgfqpoint{1.208212in}{6.601929in}}{\pgfqpoint{1.216026in}{6.594115in}}%
\pgfpathcurveto{\pgfqpoint{1.223840in}{6.586302in}}{\pgfqpoint{1.234439in}{6.581911in}}{\pgfqpoint{1.245489in}{6.581911in}}%
\pgfpathlineto{\pgfqpoint{1.245489in}{6.581911in}}%
\pgfpathclose%
\pgfusepath{stroke,fill}%
\end{pgfscope}%
\begin{pgfscope}%
\pgfpathrectangle{\pgfqpoint{0.633874in}{5.272501in}}{\pgfqpoint{2.177280in}{2.201755in}}%
\pgfusepath{clip}%
\pgfsetbuttcap%
\pgfsetroundjoin%
\definecolor{currentfill}{rgb}{0.121569,0.466667,0.705882}%
\pgfsetfillcolor{currentfill}%
\pgfsetlinewidth{0.481800pt}%
\definecolor{currentstroke}{rgb}{1.000000,1.000000,1.000000}%
\pgfsetstrokecolor{currentstroke}%
\pgfsetdash{}{0pt}%
\pgfpathmoveto{\pgfqpoint{1.165611in}{6.164912in}}%
\pgfpathcurveto{\pgfqpoint{1.176662in}{6.164912in}}{\pgfqpoint{1.187261in}{6.169303in}}{\pgfqpoint{1.195074in}{6.177116in}}%
\pgfpathcurveto{\pgfqpoint{1.202888in}{6.184930in}}{\pgfqpoint{1.207278in}{6.195529in}}{\pgfqpoint{1.207278in}{6.206579in}}%
\pgfpathcurveto{\pgfqpoint{1.207278in}{6.217629in}}{\pgfqpoint{1.202888in}{6.228228in}}{\pgfqpoint{1.195074in}{6.236042in}}%
\pgfpathcurveto{\pgfqpoint{1.187261in}{6.243855in}}{\pgfqpoint{1.176662in}{6.248246in}}{\pgfqpoint{1.165611in}{6.248246in}}%
\pgfpathcurveto{\pgfqpoint{1.154561in}{6.248246in}}{\pgfqpoint{1.143962in}{6.243855in}}{\pgfqpoint{1.136149in}{6.236042in}}%
\pgfpathcurveto{\pgfqpoint{1.128335in}{6.228228in}}{\pgfqpoint{1.123945in}{6.217629in}}{\pgfqpoint{1.123945in}{6.206579in}}%
\pgfpathcurveto{\pgfqpoint{1.123945in}{6.195529in}}{\pgfqpoint{1.128335in}{6.184930in}}{\pgfqpoint{1.136149in}{6.177116in}}%
\pgfpathcurveto{\pgfqpoint{1.143962in}{6.169303in}}{\pgfqpoint{1.154561in}{6.164912in}}{\pgfqpoint{1.165611in}{6.164912in}}%
\pgfpathlineto{\pgfqpoint{1.165611in}{6.164912in}}%
\pgfpathclose%
\pgfusepath{stroke,fill}%
\end{pgfscope}%
\begin{pgfscope}%
\pgfpathrectangle{\pgfqpoint{0.633874in}{5.272501in}}{\pgfqpoint{2.177280in}{2.201755in}}%
\pgfusepath{clip}%
\pgfsetbuttcap%
\pgfsetroundjoin%
\definecolor{currentfill}{rgb}{0.121569,0.466667,0.705882}%
\pgfsetfillcolor{currentfill}%
\pgfsetlinewidth{0.481800pt}%
\definecolor{currentstroke}{rgb}{1.000000,1.000000,1.000000}%
\pgfsetstrokecolor{currentstroke}%
\pgfsetdash{}{0pt}%
\pgfpathmoveto{\pgfqpoint{1.085734in}{6.331712in}}%
\pgfpathcurveto{\pgfqpoint{1.096784in}{6.331712in}}{\pgfqpoint{1.107383in}{6.336102in}}{\pgfqpoint{1.115197in}{6.343916in}}%
\pgfpathcurveto{\pgfqpoint{1.123010in}{6.351729in}}{\pgfqpoint{1.127401in}{6.362328in}}{\pgfqpoint{1.127401in}{6.373379in}}%
\pgfpathcurveto{\pgfqpoint{1.127401in}{6.384429in}}{\pgfqpoint{1.123010in}{6.395028in}}{\pgfqpoint{1.115197in}{6.402841in}}%
\pgfpathcurveto{\pgfqpoint{1.107383in}{6.410655in}}{\pgfqpoint{1.096784in}{6.415045in}}{\pgfqpoint{1.085734in}{6.415045in}}%
\pgfpathcurveto{\pgfqpoint{1.074684in}{6.415045in}}{\pgfqpoint{1.064085in}{6.410655in}}{\pgfqpoint{1.056271in}{6.402841in}}%
\pgfpathcurveto{\pgfqpoint{1.048458in}{6.395028in}}{\pgfqpoint{1.044067in}{6.384429in}}{\pgfqpoint{1.044067in}{6.373379in}}%
\pgfpathcurveto{\pgfqpoint{1.044067in}{6.362328in}}{\pgfqpoint{1.048458in}{6.351729in}}{\pgfqpoint{1.056271in}{6.343916in}}%
\pgfpathcurveto{\pgfqpoint{1.064085in}{6.336102in}}{\pgfqpoint{1.074684in}{6.331712in}}{\pgfqpoint{1.085734in}{6.331712in}}%
\pgfpathlineto{\pgfqpoint{1.085734in}{6.331712in}}%
\pgfpathclose%
\pgfusepath{stroke,fill}%
\end{pgfscope}%
\begin{pgfscope}%
\pgfpathrectangle{\pgfqpoint{0.633874in}{5.272501in}}{\pgfqpoint{2.177280in}{2.201755in}}%
\pgfusepath{clip}%
\pgfsetbuttcap%
\pgfsetroundjoin%
\definecolor{currentfill}{rgb}{0.121569,0.466667,0.705882}%
\pgfsetfillcolor{currentfill}%
\pgfsetlinewidth{0.481800pt}%
\definecolor{currentstroke}{rgb}{1.000000,1.000000,1.000000}%
\pgfsetstrokecolor{currentstroke}%
\pgfsetdash{}{0pt}%
\pgfpathmoveto{\pgfqpoint{1.045795in}{6.248312in}}%
\pgfpathcurveto{\pgfqpoint{1.056845in}{6.248312in}}{\pgfqpoint{1.067444in}{6.252702in}}{\pgfqpoint{1.075258in}{6.260516in}}%
\pgfpathcurveto{\pgfqpoint{1.083072in}{6.268330in}}{\pgfqpoint{1.087462in}{6.278929in}}{\pgfqpoint{1.087462in}{6.289979in}}%
\pgfpathcurveto{\pgfqpoint{1.087462in}{6.301029in}}{\pgfqpoint{1.083072in}{6.311628in}}{\pgfqpoint{1.075258in}{6.319442in}}%
\pgfpathcurveto{\pgfqpoint{1.067444in}{6.327255in}}{\pgfqpoint{1.056845in}{6.331645in}}{\pgfqpoint{1.045795in}{6.331645in}}%
\pgfpathcurveto{\pgfqpoint{1.034745in}{6.331645in}}{\pgfqpoint{1.024146in}{6.327255in}}{\pgfqpoint{1.016332in}{6.319442in}}%
\pgfpathcurveto{\pgfqpoint{1.008519in}{6.311628in}}{\pgfqpoint{1.004129in}{6.301029in}}{\pgfqpoint{1.004129in}{6.289979in}}%
\pgfpathcurveto{\pgfqpoint{1.004129in}{6.278929in}}{\pgfqpoint{1.008519in}{6.268330in}}{\pgfqpoint{1.016332in}{6.260516in}}%
\pgfpathcurveto{\pgfqpoint{1.024146in}{6.252702in}}{\pgfqpoint{1.034745in}{6.248312in}}{\pgfqpoint{1.045795in}{6.248312in}}%
\pgfpathlineto{\pgfqpoint{1.045795in}{6.248312in}}%
\pgfpathclose%
\pgfusepath{stroke,fill}%
\end{pgfscope}%
\begin{pgfscope}%
\pgfpathrectangle{\pgfqpoint{0.633874in}{5.272501in}}{\pgfqpoint{2.177280in}{2.201755in}}%
\pgfusepath{clip}%
\pgfsetbuttcap%
\pgfsetroundjoin%
\definecolor{currentfill}{rgb}{0.121569,0.466667,0.705882}%
\pgfsetfillcolor{currentfill}%
\pgfsetlinewidth{0.481800pt}%
\definecolor{currentstroke}{rgb}{1.000000,1.000000,1.000000}%
\pgfsetstrokecolor{currentstroke}%
\pgfsetdash{}{0pt}%
\pgfpathmoveto{\pgfqpoint{1.205550in}{6.665311in}}%
\pgfpathcurveto{\pgfqpoint{1.216600in}{6.665311in}}{\pgfqpoint{1.227199in}{6.669701in}}{\pgfqpoint{1.235013in}{6.677515in}}%
\pgfpathcurveto{\pgfqpoint{1.242826in}{6.685329in}}{\pgfqpoint{1.247217in}{6.695928in}}{\pgfqpoint{1.247217in}{6.706978in}}%
\pgfpathcurveto{\pgfqpoint{1.247217in}{6.718028in}}{\pgfqpoint{1.242826in}{6.728627in}}{\pgfqpoint{1.235013in}{6.736441in}}%
\pgfpathcurveto{\pgfqpoint{1.227199in}{6.744254in}}{\pgfqpoint{1.216600in}{6.748644in}}{\pgfqpoint{1.205550in}{6.748644in}}%
\pgfpathcurveto{\pgfqpoint{1.194500in}{6.748644in}}{\pgfqpoint{1.183901in}{6.744254in}}{\pgfqpoint{1.176087in}{6.736441in}}%
\pgfpathcurveto{\pgfqpoint{1.168274in}{6.728627in}}{\pgfqpoint{1.163883in}{6.718028in}}{\pgfqpoint{1.163883in}{6.706978in}}%
\pgfpathcurveto{\pgfqpoint{1.163883in}{6.695928in}}{\pgfqpoint{1.168274in}{6.685329in}}{\pgfqpoint{1.176087in}{6.677515in}}%
\pgfpathcurveto{\pgfqpoint{1.183901in}{6.669701in}}{\pgfqpoint{1.194500in}{6.665311in}}{\pgfqpoint{1.205550in}{6.665311in}}%
\pgfpathlineto{\pgfqpoint{1.205550in}{6.665311in}}%
\pgfpathclose%
\pgfusepath{stroke,fill}%
\end{pgfscope}%
\begin{pgfscope}%
\pgfpathrectangle{\pgfqpoint{0.633874in}{5.272501in}}{\pgfqpoint{2.177280in}{2.201755in}}%
\pgfusepath{clip}%
\pgfsetbuttcap%
\pgfsetroundjoin%
\definecolor{currentfill}{rgb}{0.121569,0.466667,0.705882}%
\pgfsetfillcolor{currentfill}%
\pgfsetlinewidth{0.481800pt}%
\definecolor{currentstroke}{rgb}{1.000000,1.000000,1.000000}%
\pgfsetstrokecolor{currentstroke}%
\pgfsetdash{}{0pt}%
\pgfpathmoveto{\pgfqpoint{1.365305in}{6.915511in}}%
\pgfpathcurveto{\pgfqpoint{1.376355in}{6.915511in}}{\pgfqpoint{1.386954in}{6.919901in}}{\pgfqpoint{1.394768in}{6.927714in}}%
\pgfpathcurveto{\pgfqpoint{1.402581in}{6.935528in}}{\pgfqpoint{1.406972in}{6.946127in}}{\pgfqpoint{1.406972in}{6.957177in}}%
\pgfpathcurveto{\pgfqpoint{1.406972in}{6.968227in}}{\pgfqpoint{1.402581in}{6.978826in}}{\pgfqpoint{1.394768in}{6.986640in}}%
\pgfpathcurveto{\pgfqpoint{1.386954in}{6.994454in}}{\pgfqpoint{1.376355in}{6.998844in}}{\pgfqpoint{1.365305in}{6.998844in}}%
\pgfpathcurveto{\pgfqpoint{1.354255in}{6.998844in}}{\pgfqpoint{1.343656in}{6.994454in}}{\pgfqpoint{1.335842in}{6.986640in}}%
\pgfpathcurveto{\pgfqpoint{1.328029in}{6.978826in}}{\pgfqpoint{1.323638in}{6.968227in}}{\pgfqpoint{1.323638in}{6.957177in}}%
\pgfpathcurveto{\pgfqpoint{1.323638in}{6.946127in}}{\pgfqpoint{1.328029in}{6.935528in}}{\pgfqpoint{1.335842in}{6.927714in}}%
\pgfpathcurveto{\pgfqpoint{1.343656in}{6.919901in}}{\pgfqpoint{1.354255in}{6.915511in}}{\pgfqpoint{1.365305in}{6.915511in}}%
\pgfpathlineto{\pgfqpoint{1.365305in}{6.915511in}}%
\pgfpathclose%
\pgfusepath{stroke,fill}%
\end{pgfscope}%
\begin{pgfscope}%
\pgfpathrectangle{\pgfqpoint{0.633874in}{5.272501in}}{\pgfqpoint{2.177280in}{2.201755in}}%
\pgfusepath{clip}%
\pgfsetbuttcap%
\pgfsetroundjoin%
\definecolor{currentfill}{rgb}{0.121569,0.466667,0.705882}%
\pgfsetfillcolor{currentfill}%
\pgfsetlinewidth{0.481800pt}%
\definecolor{currentstroke}{rgb}{1.000000,1.000000,1.000000}%
\pgfsetstrokecolor{currentstroke}%
\pgfsetdash{}{0pt}%
\pgfpathmoveto{\pgfqpoint{1.045795in}{6.498512in}}%
\pgfpathcurveto{\pgfqpoint{1.056845in}{6.498512in}}{\pgfqpoint{1.067444in}{6.502902in}}{\pgfqpoint{1.075258in}{6.510715in}}%
\pgfpathcurveto{\pgfqpoint{1.083072in}{6.518529in}}{\pgfqpoint{1.087462in}{6.529128in}}{\pgfqpoint{1.087462in}{6.540178in}}%
\pgfpathcurveto{\pgfqpoint{1.087462in}{6.551228in}}{\pgfqpoint{1.083072in}{6.561827in}}{\pgfqpoint{1.075258in}{6.569641in}}%
\pgfpathcurveto{\pgfqpoint{1.067444in}{6.577455in}}{\pgfqpoint{1.056845in}{6.581845in}}{\pgfqpoint{1.045795in}{6.581845in}}%
\pgfpathcurveto{\pgfqpoint{1.034745in}{6.581845in}}{\pgfqpoint{1.024146in}{6.577455in}}{\pgfqpoint{1.016332in}{6.569641in}}%
\pgfpathcurveto{\pgfqpoint{1.008519in}{6.561827in}}{\pgfqpoint{1.004129in}{6.551228in}}{\pgfqpoint{1.004129in}{6.540178in}}%
\pgfpathcurveto{\pgfqpoint{1.004129in}{6.529128in}}{\pgfqpoint{1.008519in}{6.518529in}}{\pgfqpoint{1.016332in}{6.510715in}}%
\pgfpathcurveto{\pgfqpoint{1.024146in}{6.502902in}}{\pgfqpoint{1.034745in}{6.498512in}}{\pgfqpoint{1.045795in}{6.498512in}}%
\pgfpathlineto{\pgfqpoint{1.045795in}{6.498512in}}%
\pgfpathclose%
\pgfusepath{stroke,fill}%
\end{pgfscope}%
\begin{pgfscope}%
\pgfpathrectangle{\pgfqpoint{0.633874in}{5.272501in}}{\pgfqpoint{2.177280in}{2.201755in}}%
\pgfusepath{clip}%
\pgfsetbuttcap%
\pgfsetroundjoin%
\definecolor{currentfill}{rgb}{0.121569,0.466667,0.705882}%
\pgfsetfillcolor{currentfill}%
\pgfsetlinewidth{0.481800pt}%
\definecolor{currentstroke}{rgb}{1.000000,1.000000,1.000000}%
\pgfsetstrokecolor{currentstroke}%
\pgfsetdash{}{0pt}%
\pgfpathmoveto{\pgfqpoint{1.205550in}{6.498512in}}%
\pgfpathcurveto{\pgfqpoint{1.216600in}{6.498512in}}{\pgfqpoint{1.227199in}{6.502902in}}{\pgfqpoint{1.235013in}{6.510715in}}%
\pgfpathcurveto{\pgfqpoint{1.242826in}{6.518529in}}{\pgfqpoint{1.247217in}{6.529128in}}{\pgfqpoint{1.247217in}{6.540178in}}%
\pgfpathcurveto{\pgfqpoint{1.247217in}{6.551228in}}{\pgfqpoint{1.242826in}{6.561827in}}{\pgfqpoint{1.235013in}{6.569641in}}%
\pgfpathcurveto{\pgfqpoint{1.227199in}{6.577455in}}{\pgfqpoint{1.216600in}{6.581845in}}{\pgfqpoint{1.205550in}{6.581845in}}%
\pgfpathcurveto{\pgfqpoint{1.194500in}{6.581845in}}{\pgfqpoint{1.183901in}{6.577455in}}{\pgfqpoint{1.176087in}{6.569641in}}%
\pgfpathcurveto{\pgfqpoint{1.168274in}{6.561827in}}{\pgfqpoint{1.163883in}{6.551228in}}{\pgfqpoint{1.163883in}{6.540178in}}%
\pgfpathcurveto{\pgfqpoint{1.163883in}{6.529128in}}{\pgfqpoint{1.168274in}{6.518529in}}{\pgfqpoint{1.176087in}{6.510715in}}%
\pgfpathcurveto{\pgfqpoint{1.183901in}{6.502902in}}{\pgfqpoint{1.194500in}{6.498512in}}{\pgfqpoint{1.205550in}{6.498512in}}%
\pgfpathlineto{\pgfqpoint{1.205550in}{6.498512in}}%
\pgfpathclose%
\pgfusepath{stroke,fill}%
\end{pgfscope}%
\begin{pgfscope}%
\pgfpathrectangle{\pgfqpoint{0.633874in}{5.272501in}}{\pgfqpoint{2.177280in}{2.201755in}}%
\pgfusepath{clip}%
\pgfsetbuttcap%
\pgfsetroundjoin%
\definecolor{currentfill}{rgb}{0.121569,0.466667,0.705882}%
\pgfsetfillcolor{currentfill}%
\pgfsetlinewidth{0.481800pt}%
\definecolor{currentstroke}{rgb}{1.000000,1.000000,1.000000}%
\pgfsetstrokecolor{currentstroke}%
\pgfsetdash{}{0pt}%
\pgfpathmoveto{\pgfqpoint{0.965918in}{6.081512in}}%
\pgfpathcurveto{\pgfqpoint{0.976968in}{6.081512in}}{\pgfqpoint{0.987567in}{6.085903in}}{\pgfqpoint{0.995381in}{6.093716in}}%
\pgfpathcurveto{\pgfqpoint{1.003194in}{6.101530in}}{\pgfqpoint{1.007585in}{6.112129in}}{\pgfqpoint{1.007585in}{6.123179in}}%
\pgfpathcurveto{\pgfqpoint{1.007585in}{6.134229in}}{\pgfqpoint{1.003194in}{6.144828in}}{\pgfqpoint{0.995381in}{6.152642in}}%
\pgfpathcurveto{\pgfqpoint{0.987567in}{6.160456in}}{\pgfqpoint{0.976968in}{6.164846in}}{\pgfqpoint{0.965918in}{6.164846in}}%
\pgfpathcurveto{\pgfqpoint{0.954868in}{6.164846in}}{\pgfqpoint{0.944269in}{6.160456in}}{\pgfqpoint{0.936455in}{6.152642in}}%
\pgfpathcurveto{\pgfqpoint{0.928641in}{6.144828in}}{\pgfqpoint{0.924251in}{6.134229in}}{\pgfqpoint{0.924251in}{6.123179in}}%
\pgfpathcurveto{\pgfqpoint{0.924251in}{6.112129in}}{\pgfqpoint{0.928641in}{6.101530in}}{\pgfqpoint{0.936455in}{6.093716in}}%
\pgfpathcurveto{\pgfqpoint{0.944269in}{6.085903in}}{\pgfqpoint{0.954868in}{6.081512in}}{\pgfqpoint{0.965918in}{6.081512in}}%
\pgfpathlineto{\pgfqpoint{0.965918in}{6.081512in}}%
\pgfpathclose%
\pgfusepath{stroke,fill}%
\end{pgfscope}%
\begin{pgfscope}%
\pgfpathrectangle{\pgfqpoint{0.633874in}{5.272501in}}{\pgfqpoint{2.177280in}{2.201755in}}%
\pgfusepath{clip}%
\pgfsetbuttcap%
\pgfsetroundjoin%
\definecolor{currentfill}{rgb}{0.121569,0.466667,0.705882}%
\pgfsetfillcolor{currentfill}%
\pgfsetlinewidth{0.481800pt}%
\definecolor{currentstroke}{rgb}{1.000000,1.000000,1.000000}%
\pgfsetstrokecolor{currentstroke}%
\pgfsetdash{}{0pt}%
\pgfpathmoveto{\pgfqpoint{1.165611in}{6.248312in}}%
\pgfpathcurveto{\pgfqpoint{1.176662in}{6.248312in}}{\pgfqpoint{1.187261in}{6.252702in}}{\pgfqpoint{1.195074in}{6.260516in}}%
\pgfpathcurveto{\pgfqpoint{1.202888in}{6.268330in}}{\pgfqpoint{1.207278in}{6.278929in}}{\pgfqpoint{1.207278in}{6.289979in}}%
\pgfpathcurveto{\pgfqpoint{1.207278in}{6.301029in}}{\pgfqpoint{1.202888in}{6.311628in}}{\pgfqpoint{1.195074in}{6.319442in}}%
\pgfpathcurveto{\pgfqpoint{1.187261in}{6.327255in}}{\pgfqpoint{1.176662in}{6.331645in}}{\pgfqpoint{1.165611in}{6.331645in}}%
\pgfpathcurveto{\pgfqpoint{1.154561in}{6.331645in}}{\pgfqpoint{1.143962in}{6.327255in}}{\pgfqpoint{1.136149in}{6.319442in}}%
\pgfpathcurveto{\pgfqpoint{1.128335in}{6.311628in}}{\pgfqpoint{1.123945in}{6.301029in}}{\pgfqpoint{1.123945in}{6.289979in}}%
\pgfpathcurveto{\pgfqpoint{1.123945in}{6.278929in}}{\pgfqpoint{1.128335in}{6.268330in}}{\pgfqpoint{1.136149in}{6.260516in}}%
\pgfpathcurveto{\pgfqpoint{1.143962in}{6.252702in}}{\pgfqpoint{1.154561in}{6.248312in}}{\pgfqpoint{1.165611in}{6.248312in}}%
\pgfpathlineto{\pgfqpoint{1.165611in}{6.248312in}}%
\pgfpathclose%
\pgfusepath{stroke,fill}%
\end{pgfscope}%
\begin{pgfscope}%
\pgfpathrectangle{\pgfqpoint{0.633874in}{5.272501in}}{\pgfqpoint{2.177280in}{2.201755in}}%
\pgfusepath{clip}%
\pgfsetbuttcap%
\pgfsetroundjoin%
\definecolor{currentfill}{rgb}{0.121569,0.466667,0.705882}%
\pgfsetfillcolor{currentfill}%
\pgfsetlinewidth{0.481800pt}%
\definecolor{currentstroke}{rgb}{1.000000,1.000000,1.000000}%
\pgfsetstrokecolor{currentstroke}%
\pgfsetdash{}{0pt}%
\pgfpathmoveto{\pgfqpoint{1.365305in}{6.748711in}}%
\pgfpathcurveto{\pgfqpoint{1.376355in}{6.748711in}}{\pgfqpoint{1.386954in}{6.753101in}}{\pgfqpoint{1.394768in}{6.760915in}}%
\pgfpathcurveto{\pgfqpoint{1.402581in}{6.768728in}}{\pgfqpoint{1.406972in}{6.779327in}}{\pgfqpoint{1.406972in}{6.790378in}}%
\pgfpathcurveto{\pgfqpoint{1.406972in}{6.801428in}}{\pgfqpoint{1.402581in}{6.812027in}}{\pgfqpoint{1.394768in}{6.819840in}}%
\pgfpathcurveto{\pgfqpoint{1.386954in}{6.827654in}}{\pgfqpoint{1.376355in}{6.832044in}}{\pgfqpoint{1.365305in}{6.832044in}}%
\pgfpathcurveto{\pgfqpoint{1.354255in}{6.832044in}}{\pgfqpoint{1.343656in}{6.827654in}}{\pgfqpoint{1.335842in}{6.819840in}}%
\pgfpathcurveto{\pgfqpoint{1.328029in}{6.812027in}}{\pgfqpoint{1.323638in}{6.801428in}}{\pgfqpoint{1.323638in}{6.790378in}}%
\pgfpathcurveto{\pgfqpoint{1.323638in}{6.779327in}}{\pgfqpoint{1.328029in}{6.768728in}}{\pgfqpoint{1.335842in}{6.760915in}}%
\pgfpathcurveto{\pgfqpoint{1.343656in}{6.753101in}}{\pgfqpoint{1.354255in}{6.748711in}}{\pgfqpoint{1.365305in}{6.748711in}}%
\pgfpathlineto{\pgfqpoint{1.365305in}{6.748711in}}%
\pgfpathclose%
\pgfusepath{stroke,fill}%
\end{pgfscope}%
\begin{pgfscope}%
\pgfpathrectangle{\pgfqpoint{0.633874in}{5.272501in}}{\pgfqpoint{2.177280in}{2.201755in}}%
\pgfusepath{clip}%
\pgfsetbuttcap%
\pgfsetroundjoin%
\definecolor{currentfill}{rgb}{0.121569,0.466667,0.705882}%
\pgfsetfillcolor{currentfill}%
\pgfsetlinewidth{0.481800pt}%
\definecolor{currentstroke}{rgb}{1.000000,1.000000,1.000000}%
\pgfsetstrokecolor{currentstroke}%
\pgfsetdash{}{0pt}%
\pgfpathmoveto{\pgfqpoint{1.125673in}{6.498512in}}%
\pgfpathcurveto{\pgfqpoint{1.136723in}{6.498512in}}{\pgfqpoint{1.147322in}{6.502902in}}{\pgfqpoint{1.155135in}{6.510715in}}%
\pgfpathcurveto{\pgfqpoint{1.162949in}{6.518529in}}{\pgfqpoint{1.167339in}{6.529128in}}{\pgfqpoint{1.167339in}{6.540178in}}%
\pgfpathcurveto{\pgfqpoint{1.167339in}{6.551228in}}{\pgfqpoint{1.162949in}{6.561827in}}{\pgfqpoint{1.155135in}{6.569641in}}%
\pgfpathcurveto{\pgfqpoint{1.147322in}{6.577455in}}{\pgfqpoint{1.136723in}{6.581845in}}{\pgfqpoint{1.125673in}{6.581845in}}%
\pgfpathcurveto{\pgfqpoint{1.114623in}{6.581845in}}{\pgfqpoint{1.104024in}{6.577455in}}{\pgfqpoint{1.096210in}{6.569641in}}%
\pgfpathcurveto{\pgfqpoint{1.088396in}{6.561827in}}{\pgfqpoint{1.084006in}{6.551228in}}{\pgfqpoint{1.084006in}{6.540178in}}%
\pgfpathcurveto{\pgfqpoint{1.084006in}{6.529128in}}{\pgfqpoint{1.088396in}{6.518529in}}{\pgfqpoint{1.096210in}{6.510715in}}%
\pgfpathcurveto{\pgfqpoint{1.104024in}{6.502902in}}{\pgfqpoint{1.114623in}{6.498512in}}{\pgfqpoint{1.125673in}{6.498512in}}%
\pgfpathlineto{\pgfqpoint{1.125673in}{6.498512in}}%
\pgfpathclose%
\pgfusepath{stroke,fill}%
\end{pgfscope}%
\begin{pgfscope}%
\pgfpathrectangle{\pgfqpoint{0.633874in}{5.272501in}}{\pgfqpoint{2.177280in}{2.201755in}}%
\pgfusepath{clip}%
\pgfsetbuttcap%
\pgfsetroundjoin%
\definecolor{currentfill}{rgb}{0.121569,0.466667,0.705882}%
\pgfsetfillcolor{currentfill}%
\pgfsetlinewidth{0.481800pt}%
\definecolor{currentstroke}{rgb}{1.000000,1.000000,1.000000}%
\pgfsetstrokecolor{currentstroke}%
\pgfsetdash{}{0pt}%
\pgfpathmoveto{\pgfqpoint{1.125673in}{6.164912in}}%
\pgfpathcurveto{\pgfqpoint{1.136723in}{6.164912in}}{\pgfqpoint{1.147322in}{6.169303in}}{\pgfqpoint{1.155135in}{6.177116in}}%
\pgfpathcurveto{\pgfqpoint{1.162949in}{6.184930in}}{\pgfqpoint{1.167339in}{6.195529in}}{\pgfqpoint{1.167339in}{6.206579in}}%
\pgfpathcurveto{\pgfqpoint{1.167339in}{6.217629in}}{\pgfqpoint{1.162949in}{6.228228in}}{\pgfqpoint{1.155135in}{6.236042in}}%
\pgfpathcurveto{\pgfqpoint{1.147322in}{6.243855in}}{\pgfqpoint{1.136723in}{6.248246in}}{\pgfqpoint{1.125673in}{6.248246in}}%
\pgfpathcurveto{\pgfqpoint{1.114623in}{6.248246in}}{\pgfqpoint{1.104024in}{6.243855in}}{\pgfqpoint{1.096210in}{6.236042in}}%
\pgfpathcurveto{\pgfqpoint{1.088396in}{6.228228in}}{\pgfqpoint{1.084006in}{6.217629in}}{\pgfqpoint{1.084006in}{6.206579in}}%
\pgfpathcurveto{\pgfqpoint{1.084006in}{6.195529in}}{\pgfqpoint{1.088396in}{6.184930in}}{\pgfqpoint{1.096210in}{6.177116in}}%
\pgfpathcurveto{\pgfqpoint{1.104024in}{6.169303in}}{\pgfqpoint{1.114623in}{6.164912in}}{\pgfqpoint{1.125673in}{6.164912in}}%
\pgfpathlineto{\pgfqpoint{1.125673in}{6.164912in}}%
\pgfpathclose%
\pgfusepath{stroke,fill}%
\end{pgfscope}%
\begin{pgfscope}%
\pgfpathrectangle{\pgfqpoint{0.633874in}{5.272501in}}{\pgfqpoint{2.177280in}{2.201755in}}%
\pgfusepath{clip}%
\pgfsetbuttcap%
\pgfsetroundjoin%
\definecolor{currentfill}{rgb}{0.121569,0.466667,0.705882}%
\pgfsetfillcolor{currentfill}%
\pgfsetlinewidth{0.481800pt}%
\definecolor{currentstroke}{rgb}{1.000000,1.000000,1.000000}%
\pgfsetstrokecolor{currentstroke}%
\pgfsetdash{}{0pt}%
\pgfpathmoveto{\pgfqpoint{0.925979in}{6.164912in}}%
\pgfpathcurveto{\pgfqpoint{0.937029in}{6.164912in}}{\pgfqpoint{0.947628in}{6.169303in}}{\pgfqpoint{0.955442in}{6.177116in}}%
\pgfpathcurveto{\pgfqpoint{0.963256in}{6.184930in}}{\pgfqpoint{0.967646in}{6.195529in}}{\pgfqpoint{0.967646in}{6.206579in}}%
\pgfpathcurveto{\pgfqpoint{0.967646in}{6.217629in}}{\pgfqpoint{0.963256in}{6.228228in}}{\pgfqpoint{0.955442in}{6.236042in}}%
\pgfpathcurveto{\pgfqpoint{0.947628in}{6.243855in}}{\pgfqpoint{0.937029in}{6.248246in}}{\pgfqpoint{0.925979in}{6.248246in}}%
\pgfpathcurveto{\pgfqpoint{0.914929in}{6.248246in}}{\pgfqpoint{0.904330in}{6.243855in}}{\pgfqpoint{0.896516in}{6.236042in}}%
\pgfpathcurveto{\pgfqpoint{0.888703in}{6.228228in}}{\pgfqpoint{0.884313in}{6.217629in}}{\pgfqpoint{0.884313in}{6.206579in}}%
\pgfpathcurveto{\pgfqpoint{0.884313in}{6.195529in}}{\pgfqpoint{0.888703in}{6.184930in}}{\pgfqpoint{0.896516in}{6.177116in}}%
\pgfpathcurveto{\pgfqpoint{0.904330in}{6.169303in}}{\pgfqpoint{0.914929in}{6.164912in}}{\pgfqpoint{0.925979in}{6.164912in}}%
\pgfpathlineto{\pgfqpoint{0.925979in}{6.164912in}}%
\pgfpathclose%
\pgfusepath{stroke,fill}%
\end{pgfscope}%
\begin{pgfscope}%
\pgfpathrectangle{\pgfqpoint{0.633874in}{5.272501in}}{\pgfqpoint{2.177280in}{2.201755in}}%
\pgfusepath{clip}%
\pgfsetbuttcap%
\pgfsetroundjoin%
\definecolor{currentfill}{rgb}{0.121569,0.466667,0.705882}%
\pgfsetfillcolor{currentfill}%
\pgfsetlinewidth{0.481800pt}%
\definecolor{currentstroke}{rgb}{1.000000,1.000000,1.000000}%
\pgfsetstrokecolor{currentstroke}%
\pgfsetdash{}{0pt}%
\pgfpathmoveto{\pgfqpoint{1.525060in}{6.998910in}}%
\pgfpathcurveto{\pgfqpoint{1.536110in}{6.998910in}}{\pgfqpoint{1.546709in}{7.003301in}}{\pgfqpoint{1.554523in}{7.011114in}}%
\pgfpathcurveto{\pgfqpoint{1.562336in}{7.018928in}}{\pgfqpoint{1.566726in}{7.029527in}}{\pgfqpoint{1.566726in}{7.040577in}}%
\pgfpathcurveto{\pgfqpoint{1.566726in}{7.051627in}}{\pgfqpoint{1.562336in}{7.062226in}}{\pgfqpoint{1.554523in}{7.070040in}}%
\pgfpathcurveto{\pgfqpoint{1.546709in}{7.077853in}}{\pgfqpoint{1.536110in}{7.082244in}}{\pgfqpoint{1.525060in}{7.082244in}}%
\pgfpathcurveto{\pgfqpoint{1.514010in}{7.082244in}}{\pgfqpoint{1.503411in}{7.077853in}}{\pgfqpoint{1.495597in}{7.070040in}}%
\pgfpathcurveto{\pgfqpoint{1.487783in}{7.062226in}}{\pgfqpoint{1.483393in}{7.051627in}}{\pgfqpoint{1.483393in}{7.040577in}}%
\pgfpathcurveto{\pgfqpoint{1.483393in}{7.029527in}}{\pgfqpoint{1.487783in}{7.018928in}}{\pgfqpoint{1.495597in}{7.011114in}}%
\pgfpathcurveto{\pgfqpoint{1.503411in}{7.003301in}}{\pgfqpoint{1.514010in}{6.998910in}}{\pgfqpoint{1.525060in}{6.998910in}}%
\pgfpathlineto{\pgfqpoint{1.525060in}{6.998910in}}%
\pgfpathclose%
\pgfusepath{stroke,fill}%
\end{pgfscope}%
\begin{pgfscope}%
\pgfpathrectangle{\pgfqpoint{0.633874in}{5.272501in}}{\pgfqpoint{2.177280in}{2.201755in}}%
\pgfusepath{clip}%
\pgfsetbuttcap%
\pgfsetroundjoin%
\definecolor{currentfill}{rgb}{0.121569,0.466667,0.705882}%
\pgfsetfillcolor{currentfill}%
\pgfsetlinewidth{0.481800pt}%
\definecolor{currentstroke}{rgb}{1.000000,1.000000,1.000000}%
\pgfsetstrokecolor{currentstroke}%
\pgfsetdash{}{0pt}%
\pgfpathmoveto{\pgfqpoint{1.485121in}{7.332510in}}%
\pgfpathcurveto{\pgfqpoint{1.496171in}{7.332510in}}{\pgfqpoint{1.506770in}{7.336900in}}{\pgfqpoint{1.514584in}{7.344714in}}%
\pgfpathcurveto{\pgfqpoint{1.522397in}{7.352527in}}{\pgfqpoint{1.526788in}{7.363126in}}{\pgfqpoint{1.526788in}{7.374176in}}%
\pgfpathcurveto{\pgfqpoint{1.526788in}{7.385226in}}{\pgfqpoint{1.522397in}{7.395825in}}{\pgfqpoint{1.514584in}{7.403639in}}%
\pgfpathcurveto{\pgfqpoint{1.506770in}{7.411453in}}{\pgfqpoint{1.496171in}{7.415843in}}{\pgfqpoint{1.485121in}{7.415843in}}%
\pgfpathcurveto{\pgfqpoint{1.474071in}{7.415843in}}{\pgfqpoint{1.463472in}{7.411453in}}{\pgfqpoint{1.455658in}{7.403639in}}%
\pgfpathcurveto{\pgfqpoint{1.447845in}{7.395825in}}{\pgfqpoint{1.443454in}{7.385226in}}{\pgfqpoint{1.443454in}{7.374176in}}%
\pgfpathcurveto{\pgfqpoint{1.443454in}{7.363126in}}{\pgfqpoint{1.447845in}{7.352527in}}{\pgfqpoint{1.455658in}{7.344714in}}%
\pgfpathcurveto{\pgfqpoint{1.463472in}{7.336900in}}{\pgfqpoint{1.474071in}{7.332510in}}{\pgfqpoint{1.485121in}{7.332510in}}%
\pgfpathlineto{\pgfqpoint{1.485121in}{7.332510in}}%
\pgfpathclose%
\pgfusepath{stroke,fill}%
\end{pgfscope}%
\begin{pgfscope}%
\pgfpathrectangle{\pgfqpoint{0.633874in}{5.272501in}}{\pgfqpoint{2.177280in}{2.201755in}}%
\pgfusepath{clip}%
\pgfsetbuttcap%
\pgfsetroundjoin%
\definecolor{currentfill}{rgb}{0.121569,0.466667,0.705882}%
\pgfsetfillcolor{currentfill}%
\pgfsetlinewidth{0.481800pt}%
\definecolor{currentstroke}{rgb}{1.000000,1.000000,1.000000}%
\pgfsetstrokecolor{currentstroke}%
\pgfsetdash{}{0pt}%
\pgfpathmoveto{\pgfqpoint{1.365305in}{6.915511in}}%
\pgfpathcurveto{\pgfqpoint{1.376355in}{6.915511in}}{\pgfqpoint{1.386954in}{6.919901in}}{\pgfqpoint{1.394768in}{6.927714in}}%
\pgfpathcurveto{\pgfqpoint{1.402581in}{6.935528in}}{\pgfqpoint{1.406972in}{6.946127in}}{\pgfqpoint{1.406972in}{6.957177in}}%
\pgfpathcurveto{\pgfqpoint{1.406972in}{6.968227in}}{\pgfqpoint{1.402581in}{6.978826in}}{\pgfqpoint{1.394768in}{6.986640in}}%
\pgfpathcurveto{\pgfqpoint{1.386954in}{6.994454in}}{\pgfqpoint{1.376355in}{6.998844in}}{\pgfqpoint{1.365305in}{6.998844in}}%
\pgfpathcurveto{\pgfqpoint{1.354255in}{6.998844in}}{\pgfqpoint{1.343656in}{6.994454in}}{\pgfqpoint{1.335842in}{6.986640in}}%
\pgfpathcurveto{\pgfqpoint{1.328029in}{6.978826in}}{\pgfqpoint{1.323638in}{6.968227in}}{\pgfqpoint{1.323638in}{6.957177in}}%
\pgfpathcurveto{\pgfqpoint{1.323638in}{6.946127in}}{\pgfqpoint{1.328029in}{6.935528in}}{\pgfqpoint{1.335842in}{6.927714in}}%
\pgfpathcurveto{\pgfqpoint{1.343656in}{6.919901in}}{\pgfqpoint{1.354255in}{6.915511in}}{\pgfqpoint{1.365305in}{6.915511in}}%
\pgfpathlineto{\pgfqpoint{1.365305in}{6.915511in}}%
\pgfpathclose%
\pgfusepath{stroke,fill}%
\end{pgfscope}%
\begin{pgfscope}%
\pgfpathrectangle{\pgfqpoint{0.633874in}{5.272501in}}{\pgfqpoint{2.177280in}{2.201755in}}%
\pgfusepath{clip}%
\pgfsetbuttcap%
\pgfsetroundjoin%
\definecolor{currentfill}{rgb}{0.121569,0.466667,0.705882}%
\pgfsetfillcolor{currentfill}%
\pgfsetlinewidth{0.481800pt}%
\definecolor{currentstroke}{rgb}{1.000000,1.000000,1.000000}%
\pgfsetstrokecolor{currentstroke}%
\pgfsetdash{}{0pt}%
\pgfpathmoveto{\pgfqpoint{1.245489in}{6.581911in}}%
\pgfpathcurveto{\pgfqpoint{1.256539in}{6.581911in}}{\pgfqpoint{1.267138in}{6.586302in}}{\pgfqpoint{1.274952in}{6.594115in}}%
\pgfpathcurveto{\pgfqpoint{1.282765in}{6.601929in}}{\pgfqpoint{1.287155in}{6.612528in}}{\pgfqpoint{1.287155in}{6.623578in}}%
\pgfpathcurveto{\pgfqpoint{1.287155in}{6.634628in}}{\pgfqpoint{1.282765in}{6.645227in}}{\pgfqpoint{1.274952in}{6.653041in}}%
\pgfpathcurveto{\pgfqpoint{1.267138in}{6.660854in}}{\pgfqpoint{1.256539in}{6.665245in}}{\pgfqpoint{1.245489in}{6.665245in}}%
\pgfpathcurveto{\pgfqpoint{1.234439in}{6.665245in}}{\pgfqpoint{1.223840in}{6.660854in}}{\pgfqpoint{1.216026in}{6.653041in}}%
\pgfpathcurveto{\pgfqpoint{1.208212in}{6.645227in}}{\pgfqpoint{1.203822in}{6.634628in}}{\pgfqpoint{1.203822in}{6.623578in}}%
\pgfpathcurveto{\pgfqpoint{1.203822in}{6.612528in}}{\pgfqpoint{1.208212in}{6.601929in}}{\pgfqpoint{1.216026in}{6.594115in}}%
\pgfpathcurveto{\pgfqpoint{1.223840in}{6.586302in}}{\pgfqpoint{1.234439in}{6.581911in}}{\pgfqpoint{1.245489in}{6.581911in}}%
\pgfpathlineto{\pgfqpoint{1.245489in}{6.581911in}}%
\pgfpathclose%
\pgfusepath{stroke,fill}%
\end{pgfscope}%
\begin{pgfscope}%
\pgfpathrectangle{\pgfqpoint{0.633874in}{5.272501in}}{\pgfqpoint{2.177280in}{2.201755in}}%
\pgfusepath{clip}%
\pgfsetbuttcap%
\pgfsetroundjoin%
\definecolor{currentfill}{rgb}{0.121569,0.466667,0.705882}%
\pgfsetfillcolor{currentfill}%
\pgfsetlinewidth{0.481800pt}%
\definecolor{currentstroke}{rgb}{1.000000,1.000000,1.000000}%
\pgfsetstrokecolor{currentstroke}%
\pgfsetdash{}{0pt}%
\pgfpathmoveto{\pgfqpoint{1.485121in}{6.832111in}}%
\pgfpathcurveto{\pgfqpoint{1.496171in}{6.832111in}}{\pgfqpoint{1.506770in}{6.836501in}}{\pgfqpoint{1.514584in}{6.844315in}}%
\pgfpathcurveto{\pgfqpoint{1.522397in}{6.852128in}}{\pgfqpoint{1.526788in}{6.862727in}}{\pgfqpoint{1.526788in}{6.873777in}}%
\pgfpathcurveto{\pgfqpoint{1.526788in}{6.884828in}}{\pgfqpoint{1.522397in}{6.895427in}}{\pgfqpoint{1.514584in}{6.903240in}}%
\pgfpathcurveto{\pgfqpoint{1.506770in}{6.911054in}}{\pgfqpoint{1.496171in}{6.915444in}}{\pgfqpoint{1.485121in}{6.915444in}}%
\pgfpathcurveto{\pgfqpoint{1.474071in}{6.915444in}}{\pgfqpoint{1.463472in}{6.911054in}}{\pgfqpoint{1.455658in}{6.903240in}}%
\pgfpathcurveto{\pgfqpoint{1.447845in}{6.895427in}}{\pgfqpoint{1.443454in}{6.884828in}}{\pgfqpoint{1.443454in}{6.873777in}}%
\pgfpathcurveto{\pgfqpoint{1.443454in}{6.862727in}}{\pgfqpoint{1.447845in}{6.852128in}}{\pgfqpoint{1.455658in}{6.844315in}}%
\pgfpathcurveto{\pgfqpoint{1.463472in}{6.836501in}}{\pgfqpoint{1.474071in}{6.832111in}}{\pgfqpoint{1.485121in}{6.832111in}}%
\pgfpathlineto{\pgfqpoint{1.485121in}{6.832111in}}%
\pgfpathclose%
\pgfusepath{stroke,fill}%
\end{pgfscope}%
\begin{pgfscope}%
\pgfpathrectangle{\pgfqpoint{0.633874in}{5.272501in}}{\pgfqpoint{2.177280in}{2.201755in}}%
\pgfusepath{clip}%
\pgfsetbuttcap%
\pgfsetroundjoin%
\definecolor{currentfill}{rgb}{0.121569,0.466667,0.705882}%
\pgfsetfillcolor{currentfill}%
\pgfsetlinewidth{0.481800pt}%
\definecolor{currentstroke}{rgb}{1.000000,1.000000,1.000000}%
\pgfsetstrokecolor{currentstroke}%
\pgfsetdash{}{0pt}%
\pgfpathmoveto{\pgfqpoint{1.245489in}{6.832111in}}%
\pgfpathcurveto{\pgfqpoint{1.256539in}{6.832111in}}{\pgfqpoint{1.267138in}{6.836501in}}{\pgfqpoint{1.274952in}{6.844315in}}%
\pgfpathcurveto{\pgfqpoint{1.282765in}{6.852128in}}{\pgfqpoint{1.287155in}{6.862727in}}{\pgfqpoint{1.287155in}{6.873777in}}%
\pgfpathcurveto{\pgfqpoint{1.287155in}{6.884828in}}{\pgfqpoint{1.282765in}{6.895427in}}{\pgfqpoint{1.274952in}{6.903240in}}%
\pgfpathcurveto{\pgfqpoint{1.267138in}{6.911054in}}{\pgfqpoint{1.256539in}{6.915444in}}{\pgfqpoint{1.245489in}{6.915444in}}%
\pgfpathcurveto{\pgfqpoint{1.234439in}{6.915444in}}{\pgfqpoint{1.223840in}{6.911054in}}{\pgfqpoint{1.216026in}{6.903240in}}%
\pgfpathcurveto{\pgfqpoint{1.208212in}{6.895427in}}{\pgfqpoint{1.203822in}{6.884828in}}{\pgfqpoint{1.203822in}{6.873777in}}%
\pgfpathcurveto{\pgfqpoint{1.203822in}{6.862727in}}{\pgfqpoint{1.208212in}{6.852128in}}{\pgfqpoint{1.216026in}{6.844315in}}%
\pgfpathcurveto{\pgfqpoint{1.223840in}{6.836501in}}{\pgfqpoint{1.234439in}{6.832111in}}{\pgfqpoint{1.245489in}{6.832111in}}%
\pgfpathlineto{\pgfqpoint{1.245489in}{6.832111in}}%
\pgfpathclose%
\pgfusepath{stroke,fill}%
\end{pgfscope}%
\begin{pgfscope}%
\pgfpathrectangle{\pgfqpoint{0.633874in}{5.272501in}}{\pgfqpoint{2.177280in}{2.201755in}}%
\pgfusepath{clip}%
\pgfsetbuttcap%
\pgfsetroundjoin%
\definecolor{currentfill}{rgb}{0.121569,0.466667,0.705882}%
\pgfsetfillcolor{currentfill}%
\pgfsetlinewidth{0.481800pt}%
\definecolor{currentstroke}{rgb}{1.000000,1.000000,1.000000}%
\pgfsetstrokecolor{currentstroke}%
\pgfsetdash{}{0pt}%
\pgfpathmoveto{\pgfqpoint{1.365305in}{6.498512in}}%
\pgfpathcurveto{\pgfqpoint{1.376355in}{6.498512in}}{\pgfqpoint{1.386954in}{6.502902in}}{\pgfqpoint{1.394768in}{6.510715in}}%
\pgfpathcurveto{\pgfqpoint{1.402581in}{6.518529in}}{\pgfqpoint{1.406972in}{6.529128in}}{\pgfqpoint{1.406972in}{6.540178in}}%
\pgfpathcurveto{\pgfqpoint{1.406972in}{6.551228in}}{\pgfqpoint{1.402581in}{6.561827in}}{\pgfqpoint{1.394768in}{6.569641in}}%
\pgfpathcurveto{\pgfqpoint{1.386954in}{6.577455in}}{\pgfqpoint{1.376355in}{6.581845in}}{\pgfqpoint{1.365305in}{6.581845in}}%
\pgfpathcurveto{\pgfqpoint{1.354255in}{6.581845in}}{\pgfqpoint{1.343656in}{6.577455in}}{\pgfqpoint{1.335842in}{6.569641in}}%
\pgfpathcurveto{\pgfqpoint{1.328029in}{6.561827in}}{\pgfqpoint{1.323638in}{6.551228in}}{\pgfqpoint{1.323638in}{6.540178in}}%
\pgfpathcurveto{\pgfqpoint{1.323638in}{6.529128in}}{\pgfqpoint{1.328029in}{6.518529in}}{\pgfqpoint{1.335842in}{6.510715in}}%
\pgfpathcurveto{\pgfqpoint{1.343656in}{6.502902in}}{\pgfqpoint{1.354255in}{6.498512in}}{\pgfqpoint{1.365305in}{6.498512in}}%
\pgfpathlineto{\pgfqpoint{1.365305in}{6.498512in}}%
\pgfpathclose%
\pgfusepath{stroke,fill}%
\end{pgfscope}%
\begin{pgfscope}%
\pgfpathrectangle{\pgfqpoint{0.633874in}{5.272501in}}{\pgfqpoint{2.177280in}{2.201755in}}%
\pgfusepath{clip}%
\pgfsetbuttcap%
\pgfsetroundjoin%
\definecolor{currentfill}{rgb}{0.121569,0.466667,0.705882}%
\pgfsetfillcolor{currentfill}%
\pgfsetlinewidth{0.481800pt}%
\definecolor{currentstroke}{rgb}{1.000000,1.000000,1.000000}%
\pgfsetstrokecolor{currentstroke}%
\pgfsetdash{}{0pt}%
\pgfpathmoveto{\pgfqpoint{1.245489in}{6.748711in}}%
\pgfpathcurveto{\pgfqpoint{1.256539in}{6.748711in}}{\pgfqpoint{1.267138in}{6.753101in}}{\pgfqpoint{1.274952in}{6.760915in}}%
\pgfpathcurveto{\pgfqpoint{1.282765in}{6.768728in}}{\pgfqpoint{1.287155in}{6.779327in}}{\pgfqpoint{1.287155in}{6.790378in}}%
\pgfpathcurveto{\pgfqpoint{1.287155in}{6.801428in}}{\pgfqpoint{1.282765in}{6.812027in}}{\pgfqpoint{1.274952in}{6.819840in}}%
\pgfpathcurveto{\pgfqpoint{1.267138in}{6.827654in}}{\pgfqpoint{1.256539in}{6.832044in}}{\pgfqpoint{1.245489in}{6.832044in}}%
\pgfpathcurveto{\pgfqpoint{1.234439in}{6.832044in}}{\pgfqpoint{1.223840in}{6.827654in}}{\pgfqpoint{1.216026in}{6.819840in}}%
\pgfpathcurveto{\pgfqpoint{1.208212in}{6.812027in}}{\pgfqpoint{1.203822in}{6.801428in}}{\pgfqpoint{1.203822in}{6.790378in}}%
\pgfpathcurveto{\pgfqpoint{1.203822in}{6.779327in}}{\pgfqpoint{1.208212in}{6.768728in}}{\pgfqpoint{1.216026in}{6.760915in}}%
\pgfpathcurveto{\pgfqpoint{1.223840in}{6.753101in}}{\pgfqpoint{1.234439in}{6.748711in}}{\pgfqpoint{1.245489in}{6.748711in}}%
\pgfpathlineto{\pgfqpoint{1.245489in}{6.748711in}}%
\pgfpathclose%
\pgfusepath{stroke,fill}%
\end{pgfscope}%
\begin{pgfscope}%
\pgfpathrectangle{\pgfqpoint{0.633874in}{5.272501in}}{\pgfqpoint{2.177280in}{2.201755in}}%
\pgfusepath{clip}%
\pgfsetbuttcap%
\pgfsetroundjoin%
\definecolor{currentfill}{rgb}{0.121569,0.466667,0.705882}%
\pgfsetfillcolor{currentfill}%
\pgfsetlinewidth{0.481800pt}%
\definecolor{currentstroke}{rgb}{1.000000,1.000000,1.000000}%
\pgfsetstrokecolor{currentstroke}%
\pgfsetdash{}{0pt}%
\pgfpathmoveto{\pgfqpoint{1.045795in}{6.665311in}}%
\pgfpathcurveto{\pgfqpoint{1.056845in}{6.665311in}}{\pgfqpoint{1.067444in}{6.669701in}}{\pgfqpoint{1.075258in}{6.677515in}}%
\pgfpathcurveto{\pgfqpoint{1.083072in}{6.685329in}}{\pgfqpoint{1.087462in}{6.695928in}}{\pgfqpoint{1.087462in}{6.706978in}}%
\pgfpathcurveto{\pgfqpoint{1.087462in}{6.718028in}}{\pgfqpoint{1.083072in}{6.728627in}}{\pgfqpoint{1.075258in}{6.736441in}}%
\pgfpathcurveto{\pgfqpoint{1.067444in}{6.744254in}}{\pgfqpoint{1.056845in}{6.748644in}}{\pgfqpoint{1.045795in}{6.748644in}}%
\pgfpathcurveto{\pgfqpoint{1.034745in}{6.748644in}}{\pgfqpoint{1.024146in}{6.744254in}}{\pgfqpoint{1.016332in}{6.736441in}}%
\pgfpathcurveto{\pgfqpoint{1.008519in}{6.728627in}}{\pgfqpoint{1.004129in}{6.718028in}}{\pgfqpoint{1.004129in}{6.706978in}}%
\pgfpathcurveto{\pgfqpoint{1.004129in}{6.695928in}}{\pgfqpoint{1.008519in}{6.685329in}}{\pgfqpoint{1.016332in}{6.677515in}}%
\pgfpathcurveto{\pgfqpoint{1.024146in}{6.669701in}}{\pgfqpoint{1.034745in}{6.665311in}}{\pgfqpoint{1.045795in}{6.665311in}}%
\pgfpathlineto{\pgfqpoint{1.045795in}{6.665311in}}%
\pgfpathclose%
\pgfusepath{stroke,fill}%
\end{pgfscope}%
\begin{pgfscope}%
\pgfpathrectangle{\pgfqpoint{0.633874in}{5.272501in}}{\pgfqpoint{2.177280in}{2.201755in}}%
\pgfusepath{clip}%
\pgfsetbuttcap%
\pgfsetroundjoin%
\definecolor{currentfill}{rgb}{0.121569,0.466667,0.705882}%
\pgfsetfillcolor{currentfill}%
\pgfsetlinewidth{0.481800pt}%
\definecolor{currentstroke}{rgb}{1.000000,1.000000,1.000000}%
\pgfsetstrokecolor{currentstroke}%
\pgfsetdash{}{0pt}%
\pgfpathmoveto{\pgfqpoint{1.245489in}{6.415112in}}%
\pgfpathcurveto{\pgfqpoint{1.256539in}{6.415112in}}{\pgfqpoint{1.267138in}{6.419502in}}{\pgfqpoint{1.274952in}{6.427316in}}%
\pgfpathcurveto{\pgfqpoint{1.282765in}{6.435129in}}{\pgfqpoint{1.287155in}{6.445728in}}{\pgfqpoint{1.287155in}{6.456778in}}%
\pgfpathcurveto{\pgfqpoint{1.287155in}{6.467828in}}{\pgfqpoint{1.282765in}{6.478428in}}{\pgfqpoint{1.274952in}{6.486241in}}%
\pgfpathcurveto{\pgfqpoint{1.267138in}{6.494055in}}{\pgfqpoint{1.256539in}{6.498445in}}{\pgfqpoint{1.245489in}{6.498445in}}%
\pgfpathcurveto{\pgfqpoint{1.234439in}{6.498445in}}{\pgfqpoint{1.223840in}{6.494055in}}{\pgfqpoint{1.216026in}{6.486241in}}%
\pgfpathcurveto{\pgfqpoint{1.208212in}{6.478428in}}{\pgfqpoint{1.203822in}{6.467828in}}{\pgfqpoint{1.203822in}{6.456778in}}%
\pgfpathcurveto{\pgfqpoint{1.203822in}{6.445728in}}{\pgfqpoint{1.208212in}{6.435129in}}{\pgfqpoint{1.216026in}{6.427316in}}%
\pgfpathcurveto{\pgfqpoint{1.223840in}{6.419502in}}{\pgfqpoint{1.234439in}{6.415112in}}{\pgfqpoint{1.245489in}{6.415112in}}%
\pgfpathlineto{\pgfqpoint{1.245489in}{6.415112in}}%
\pgfpathclose%
\pgfusepath{stroke,fill}%
\end{pgfscope}%
\begin{pgfscope}%
\pgfpathrectangle{\pgfqpoint{0.633874in}{5.272501in}}{\pgfqpoint{2.177280in}{2.201755in}}%
\pgfusepath{clip}%
\pgfsetbuttcap%
\pgfsetroundjoin%
\definecolor{currentfill}{rgb}{0.121569,0.466667,0.705882}%
\pgfsetfillcolor{currentfill}%
\pgfsetlinewidth{0.481800pt}%
\definecolor{currentstroke}{rgb}{1.000000,1.000000,1.000000}%
\pgfsetstrokecolor{currentstroke}%
\pgfsetdash{}{0pt}%
\pgfpathmoveto{\pgfqpoint{1.125673in}{6.498512in}}%
\pgfpathcurveto{\pgfqpoint{1.136723in}{6.498512in}}{\pgfqpoint{1.147322in}{6.502902in}}{\pgfqpoint{1.155135in}{6.510715in}}%
\pgfpathcurveto{\pgfqpoint{1.162949in}{6.518529in}}{\pgfqpoint{1.167339in}{6.529128in}}{\pgfqpoint{1.167339in}{6.540178in}}%
\pgfpathcurveto{\pgfqpoint{1.167339in}{6.551228in}}{\pgfqpoint{1.162949in}{6.561827in}}{\pgfqpoint{1.155135in}{6.569641in}}%
\pgfpathcurveto{\pgfqpoint{1.147322in}{6.577455in}}{\pgfqpoint{1.136723in}{6.581845in}}{\pgfqpoint{1.125673in}{6.581845in}}%
\pgfpathcurveto{\pgfqpoint{1.114623in}{6.581845in}}{\pgfqpoint{1.104024in}{6.577455in}}{\pgfqpoint{1.096210in}{6.569641in}}%
\pgfpathcurveto{\pgfqpoint{1.088396in}{6.561827in}}{\pgfqpoint{1.084006in}{6.551228in}}{\pgfqpoint{1.084006in}{6.540178in}}%
\pgfpathcurveto{\pgfqpoint{1.084006in}{6.529128in}}{\pgfqpoint{1.088396in}{6.518529in}}{\pgfqpoint{1.096210in}{6.510715in}}%
\pgfpathcurveto{\pgfqpoint{1.104024in}{6.502902in}}{\pgfqpoint{1.114623in}{6.498512in}}{\pgfqpoint{1.125673in}{6.498512in}}%
\pgfpathlineto{\pgfqpoint{1.125673in}{6.498512in}}%
\pgfpathclose%
\pgfusepath{stroke,fill}%
\end{pgfscope}%
\begin{pgfscope}%
\pgfpathrectangle{\pgfqpoint{0.633874in}{5.272501in}}{\pgfqpoint{2.177280in}{2.201755in}}%
\pgfusepath{clip}%
\pgfsetbuttcap%
\pgfsetroundjoin%
\definecolor{currentfill}{rgb}{0.121569,0.466667,0.705882}%
\pgfsetfillcolor{currentfill}%
\pgfsetlinewidth{0.481800pt}%
\definecolor{currentstroke}{rgb}{1.000000,1.000000,1.000000}%
\pgfsetstrokecolor{currentstroke}%
\pgfsetdash{}{0pt}%
\pgfpathmoveto{\pgfqpoint{1.205550in}{6.164912in}}%
\pgfpathcurveto{\pgfqpoint{1.216600in}{6.164912in}}{\pgfqpoint{1.227199in}{6.169303in}}{\pgfqpoint{1.235013in}{6.177116in}}%
\pgfpathcurveto{\pgfqpoint{1.242826in}{6.184930in}}{\pgfqpoint{1.247217in}{6.195529in}}{\pgfqpoint{1.247217in}{6.206579in}}%
\pgfpathcurveto{\pgfqpoint{1.247217in}{6.217629in}}{\pgfqpoint{1.242826in}{6.228228in}}{\pgfqpoint{1.235013in}{6.236042in}}%
\pgfpathcurveto{\pgfqpoint{1.227199in}{6.243855in}}{\pgfqpoint{1.216600in}{6.248246in}}{\pgfqpoint{1.205550in}{6.248246in}}%
\pgfpathcurveto{\pgfqpoint{1.194500in}{6.248246in}}{\pgfqpoint{1.183901in}{6.243855in}}{\pgfqpoint{1.176087in}{6.236042in}}%
\pgfpathcurveto{\pgfqpoint{1.168274in}{6.228228in}}{\pgfqpoint{1.163883in}{6.217629in}}{\pgfqpoint{1.163883in}{6.206579in}}%
\pgfpathcurveto{\pgfqpoint{1.163883in}{6.195529in}}{\pgfqpoint{1.168274in}{6.184930in}}{\pgfqpoint{1.176087in}{6.177116in}}%
\pgfpathcurveto{\pgfqpoint{1.183901in}{6.169303in}}{\pgfqpoint{1.194500in}{6.164912in}}{\pgfqpoint{1.205550in}{6.164912in}}%
\pgfpathlineto{\pgfqpoint{1.205550in}{6.164912in}}%
\pgfpathclose%
\pgfusepath{stroke,fill}%
\end{pgfscope}%
\begin{pgfscope}%
\pgfpathrectangle{\pgfqpoint{0.633874in}{5.272501in}}{\pgfqpoint{2.177280in}{2.201755in}}%
\pgfusepath{clip}%
\pgfsetbuttcap%
\pgfsetroundjoin%
\definecolor{currentfill}{rgb}{0.121569,0.466667,0.705882}%
\pgfsetfillcolor{currentfill}%
\pgfsetlinewidth{0.481800pt}%
\definecolor{currentstroke}{rgb}{1.000000,1.000000,1.000000}%
\pgfsetstrokecolor{currentstroke}%
\pgfsetdash{}{0pt}%
\pgfpathmoveto{\pgfqpoint{1.205550in}{6.498512in}}%
\pgfpathcurveto{\pgfqpoint{1.216600in}{6.498512in}}{\pgfqpoint{1.227199in}{6.502902in}}{\pgfqpoint{1.235013in}{6.510715in}}%
\pgfpathcurveto{\pgfqpoint{1.242826in}{6.518529in}}{\pgfqpoint{1.247217in}{6.529128in}}{\pgfqpoint{1.247217in}{6.540178in}}%
\pgfpathcurveto{\pgfqpoint{1.247217in}{6.551228in}}{\pgfqpoint{1.242826in}{6.561827in}}{\pgfqpoint{1.235013in}{6.569641in}}%
\pgfpathcurveto{\pgfqpoint{1.227199in}{6.577455in}}{\pgfqpoint{1.216600in}{6.581845in}}{\pgfqpoint{1.205550in}{6.581845in}}%
\pgfpathcurveto{\pgfqpoint{1.194500in}{6.581845in}}{\pgfqpoint{1.183901in}{6.577455in}}{\pgfqpoint{1.176087in}{6.569641in}}%
\pgfpathcurveto{\pgfqpoint{1.168274in}{6.561827in}}{\pgfqpoint{1.163883in}{6.551228in}}{\pgfqpoint{1.163883in}{6.540178in}}%
\pgfpathcurveto{\pgfqpoint{1.163883in}{6.529128in}}{\pgfqpoint{1.168274in}{6.518529in}}{\pgfqpoint{1.176087in}{6.510715in}}%
\pgfpathcurveto{\pgfqpoint{1.183901in}{6.502902in}}{\pgfqpoint{1.194500in}{6.498512in}}{\pgfqpoint{1.205550in}{6.498512in}}%
\pgfpathlineto{\pgfqpoint{1.205550in}{6.498512in}}%
\pgfpathclose%
\pgfusepath{stroke,fill}%
\end{pgfscope}%
\begin{pgfscope}%
\pgfpathrectangle{\pgfqpoint{0.633874in}{5.272501in}}{\pgfqpoint{2.177280in}{2.201755in}}%
\pgfusepath{clip}%
\pgfsetbuttcap%
\pgfsetroundjoin%
\definecolor{currentfill}{rgb}{0.121569,0.466667,0.705882}%
\pgfsetfillcolor{currentfill}%
\pgfsetlinewidth{0.481800pt}%
\definecolor{currentstroke}{rgb}{1.000000,1.000000,1.000000}%
\pgfsetstrokecolor{currentstroke}%
\pgfsetdash{}{0pt}%
\pgfpathmoveto{\pgfqpoint{1.285428in}{6.581911in}}%
\pgfpathcurveto{\pgfqpoint{1.296478in}{6.581911in}}{\pgfqpoint{1.307077in}{6.586302in}}{\pgfqpoint{1.314890in}{6.594115in}}%
\pgfpathcurveto{\pgfqpoint{1.322704in}{6.601929in}}{\pgfqpoint{1.327094in}{6.612528in}}{\pgfqpoint{1.327094in}{6.623578in}}%
\pgfpathcurveto{\pgfqpoint{1.327094in}{6.634628in}}{\pgfqpoint{1.322704in}{6.645227in}}{\pgfqpoint{1.314890in}{6.653041in}}%
\pgfpathcurveto{\pgfqpoint{1.307077in}{6.660854in}}{\pgfqpoint{1.296478in}{6.665245in}}{\pgfqpoint{1.285428in}{6.665245in}}%
\pgfpathcurveto{\pgfqpoint{1.274377in}{6.665245in}}{\pgfqpoint{1.263778in}{6.660854in}}{\pgfqpoint{1.255965in}{6.653041in}}%
\pgfpathcurveto{\pgfqpoint{1.248151in}{6.645227in}}{\pgfqpoint{1.243761in}{6.634628in}}{\pgfqpoint{1.243761in}{6.623578in}}%
\pgfpathcurveto{\pgfqpoint{1.243761in}{6.612528in}}{\pgfqpoint{1.248151in}{6.601929in}}{\pgfqpoint{1.255965in}{6.594115in}}%
\pgfpathcurveto{\pgfqpoint{1.263778in}{6.586302in}}{\pgfqpoint{1.274377in}{6.581911in}}{\pgfqpoint{1.285428in}{6.581911in}}%
\pgfpathlineto{\pgfqpoint{1.285428in}{6.581911in}}%
\pgfpathclose%
\pgfusepath{stroke,fill}%
\end{pgfscope}%
\begin{pgfscope}%
\pgfpathrectangle{\pgfqpoint{0.633874in}{5.272501in}}{\pgfqpoint{2.177280in}{2.201755in}}%
\pgfusepath{clip}%
\pgfsetbuttcap%
\pgfsetroundjoin%
\definecolor{currentfill}{rgb}{0.121569,0.466667,0.705882}%
\pgfsetfillcolor{currentfill}%
\pgfsetlinewidth{0.481800pt}%
\definecolor{currentstroke}{rgb}{1.000000,1.000000,1.000000}%
\pgfsetstrokecolor{currentstroke}%
\pgfsetdash{}{0pt}%
\pgfpathmoveto{\pgfqpoint{1.285428in}{6.498512in}}%
\pgfpathcurveto{\pgfqpoint{1.296478in}{6.498512in}}{\pgfqpoint{1.307077in}{6.502902in}}{\pgfqpoint{1.314890in}{6.510715in}}%
\pgfpathcurveto{\pgfqpoint{1.322704in}{6.518529in}}{\pgfqpoint{1.327094in}{6.529128in}}{\pgfqpoint{1.327094in}{6.540178in}}%
\pgfpathcurveto{\pgfqpoint{1.327094in}{6.551228in}}{\pgfqpoint{1.322704in}{6.561827in}}{\pgfqpoint{1.314890in}{6.569641in}}%
\pgfpathcurveto{\pgfqpoint{1.307077in}{6.577455in}}{\pgfqpoint{1.296478in}{6.581845in}}{\pgfqpoint{1.285428in}{6.581845in}}%
\pgfpathcurveto{\pgfqpoint{1.274377in}{6.581845in}}{\pgfqpoint{1.263778in}{6.577455in}}{\pgfqpoint{1.255965in}{6.569641in}}%
\pgfpathcurveto{\pgfqpoint{1.248151in}{6.561827in}}{\pgfqpoint{1.243761in}{6.551228in}}{\pgfqpoint{1.243761in}{6.540178in}}%
\pgfpathcurveto{\pgfqpoint{1.243761in}{6.529128in}}{\pgfqpoint{1.248151in}{6.518529in}}{\pgfqpoint{1.255965in}{6.510715in}}%
\pgfpathcurveto{\pgfqpoint{1.263778in}{6.502902in}}{\pgfqpoint{1.274377in}{6.498512in}}{\pgfqpoint{1.285428in}{6.498512in}}%
\pgfpathlineto{\pgfqpoint{1.285428in}{6.498512in}}%
\pgfpathclose%
\pgfusepath{stroke,fill}%
\end{pgfscope}%
\begin{pgfscope}%
\pgfpathrectangle{\pgfqpoint{0.633874in}{5.272501in}}{\pgfqpoint{2.177280in}{2.201755in}}%
\pgfusepath{clip}%
\pgfsetbuttcap%
\pgfsetroundjoin%
\definecolor{currentfill}{rgb}{0.121569,0.466667,0.705882}%
\pgfsetfillcolor{currentfill}%
\pgfsetlinewidth{0.481800pt}%
\definecolor{currentstroke}{rgb}{1.000000,1.000000,1.000000}%
\pgfsetstrokecolor{currentstroke}%
\pgfsetdash{}{0pt}%
\pgfpathmoveto{\pgfqpoint{1.085734in}{6.331712in}}%
\pgfpathcurveto{\pgfqpoint{1.096784in}{6.331712in}}{\pgfqpoint{1.107383in}{6.336102in}}{\pgfqpoint{1.115197in}{6.343916in}}%
\pgfpathcurveto{\pgfqpoint{1.123010in}{6.351729in}}{\pgfqpoint{1.127401in}{6.362328in}}{\pgfqpoint{1.127401in}{6.373379in}}%
\pgfpathcurveto{\pgfqpoint{1.127401in}{6.384429in}}{\pgfqpoint{1.123010in}{6.395028in}}{\pgfqpoint{1.115197in}{6.402841in}}%
\pgfpathcurveto{\pgfqpoint{1.107383in}{6.410655in}}{\pgfqpoint{1.096784in}{6.415045in}}{\pgfqpoint{1.085734in}{6.415045in}}%
\pgfpathcurveto{\pgfqpoint{1.074684in}{6.415045in}}{\pgfqpoint{1.064085in}{6.410655in}}{\pgfqpoint{1.056271in}{6.402841in}}%
\pgfpathcurveto{\pgfqpoint{1.048458in}{6.395028in}}{\pgfqpoint{1.044067in}{6.384429in}}{\pgfqpoint{1.044067in}{6.373379in}}%
\pgfpathcurveto{\pgfqpoint{1.044067in}{6.362328in}}{\pgfqpoint{1.048458in}{6.351729in}}{\pgfqpoint{1.056271in}{6.343916in}}%
\pgfpathcurveto{\pgfqpoint{1.064085in}{6.336102in}}{\pgfqpoint{1.074684in}{6.331712in}}{\pgfqpoint{1.085734in}{6.331712in}}%
\pgfpathlineto{\pgfqpoint{1.085734in}{6.331712in}}%
\pgfpathclose%
\pgfusepath{stroke,fill}%
\end{pgfscope}%
\begin{pgfscope}%
\pgfpathrectangle{\pgfqpoint{0.633874in}{5.272501in}}{\pgfqpoint{2.177280in}{2.201755in}}%
\pgfusepath{clip}%
\pgfsetbuttcap%
\pgfsetroundjoin%
\definecolor{currentfill}{rgb}{0.121569,0.466667,0.705882}%
\pgfsetfillcolor{currentfill}%
\pgfsetlinewidth{0.481800pt}%
\definecolor{currentstroke}{rgb}{1.000000,1.000000,1.000000}%
\pgfsetstrokecolor{currentstroke}%
\pgfsetdash{}{0pt}%
\pgfpathmoveto{\pgfqpoint{1.125673in}{6.248312in}}%
\pgfpathcurveto{\pgfqpoint{1.136723in}{6.248312in}}{\pgfqpoint{1.147322in}{6.252702in}}{\pgfqpoint{1.155135in}{6.260516in}}%
\pgfpathcurveto{\pgfqpoint{1.162949in}{6.268330in}}{\pgfqpoint{1.167339in}{6.278929in}}{\pgfqpoint{1.167339in}{6.289979in}}%
\pgfpathcurveto{\pgfqpoint{1.167339in}{6.301029in}}{\pgfqpoint{1.162949in}{6.311628in}}{\pgfqpoint{1.155135in}{6.319442in}}%
\pgfpathcurveto{\pgfqpoint{1.147322in}{6.327255in}}{\pgfqpoint{1.136723in}{6.331645in}}{\pgfqpoint{1.125673in}{6.331645in}}%
\pgfpathcurveto{\pgfqpoint{1.114623in}{6.331645in}}{\pgfqpoint{1.104024in}{6.327255in}}{\pgfqpoint{1.096210in}{6.319442in}}%
\pgfpathcurveto{\pgfqpoint{1.088396in}{6.311628in}}{\pgfqpoint{1.084006in}{6.301029in}}{\pgfqpoint{1.084006in}{6.289979in}}%
\pgfpathcurveto{\pgfqpoint{1.084006in}{6.278929in}}{\pgfqpoint{1.088396in}{6.268330in}}{\pgfqpoint{1.096210in}{6.260516in}}%
\pgfpathcurveto{\pgfqpoint{1.104024in}{6.252702in}}{\pgfqpoint{1.114623in}{6.248312in}}{\pgfqpoint{1.125673in}{6.248312in}}%
\pgfpathlineto{\pgfqpoint{1.125673in}{6.248312in}}%
\pgfpathclose%
\pgfusepath{stroke,fill}%
\end{pgfscope}%
\begin{pgfscope}%
\pgfpathrectangle{\pgfqpoint{0.633874in}{5.272501in}}{\pgfqpoint{2.177280in}{2.201755in}}%
\pgfusepath{clip}%
\pgfsetbuttcap%
\pgfsetroundjoin%
\definecolor{currentfill}{rgb}{0.121569,0.466667,0.705882}%
\pgfsetfillcolor{currentfill}%
\pgfsetlinewidth{0.481800pt}%
\definecolor{currentstroke}{rgb}{1.000000,1.000000,1.000000}%
\pgfsetstrokecolor{currentstroke}%
\pgfsetdash{}{0pt}%
\pgfpathmoveto{\pgfqpoint{1.365305in}{6.498512in}}%
\pgfpathcurveto{\pgfqpoint{1.376355in}{6.498512in}}{\pgfqpoint{1.386954in}{6.502902in}}{\pgfqpoint{1.394768in}{6.510715in}}%
\pgfpathcurveto{\pgfqpoint{1.402581in}{6.518529in}}{\pgfqpoint{1.406972in}{6.529128in}}{\pgfqpoint{1.406972in}{6.540178in}}%
\pgfpathcurveto{\pgfqpoint{1.406972in}{6.551228in}}{\pgfqpoint{1.402581in}{6.561827in}}{\pgfqpoint{1.394768in}{6.569641in}}%
\pgfpathcurveto{\pgfqpoint{1.386954in}{6.577455in}}{\pgfqpoint{1.376355in}{6.581845in}}{\pgfqpoint{1.365305in}{6.581845in}}%
\pgfpathcurveto{\pgfqpoint{1.354255in}{6.581845in}}{\pgfqpoint{1.343656in}{6.577455in}}{\pgfqpoint{1.335842in}{6.569641in}}%
\pgfpathcurveto{\pgfqpoint{1.328029in}{6.561827in}}{\pgfqpoint{1.323638in}{6.551228in}}{\pgfqpoint{1.323638in}{6.540178in}}%
\pgfpathcurveto{\pgfqpoint{1.323638in}{6.529128in}}{\pgfqpoint{1.328029in}{6.518529in}}{\pgfqpoint{1.335842in}{6.510715in}}%
\pgfpathcurveto{\pgfqpoint{1.343656in}{6.502902in}}{\pgfqpoint{1.354255in}{6.498512in}}{\pgfqpoint{1.365305in}{6.498512in}}%
\pgfpathlineto{\pgfqpoint{1.365305in}{6.498512in}}%
\pgfpathclose%
\pgfusepath{stroke,fill}%
\end{pgfscope}%
\begin{pgfscope}%
\pgfpathrectangle{\pgfqpoint{0.633874in}{5.272501in}}{\pgfqpoint{2.177280in}{2.201755in}}%
\pgfusepath{clip}%
\pgfsetbuttcap%
\pgfsetroundjoin%
\definecolor{currentfill}{rgb}{0.121569,0.466667,0.705882}%
\pgfsetfillcolor{currentfill}%
\pgfsetlinewidth{0.481800pt}%
\definecolor{currentstroke}{rgb}{1.000000,1.000000,1.000000}%
\pgfsetstrokecolor{currentstroke}%
\pgfsetdash{}{0pt}%
\pgfpathmoveto{\pgfqpoint{1.285428in}{7.082310in}}%
\pgfpathcurveto{\pgfqpoint{1.296478in}{7.082310in}}{\pgfqpoint{1.307077in}{7.086700in}}{\pgfqpoint{1.314890in}{7.094514in}}%
\pgfpathcurveto{\pgfqpoint{1.322704in}{7.102328in}}{\pgfqpoint{1.327094in}{7.112927in}}{\pgfqpoint{1.327094in}{7.123977in}}%
\pgfpathcurveto{\pgfqpoint{1.327094in}{7.135027in}}{\pgfqpoint{1.322704in}{7.145626in}}{\pgfqpoint{1.314890in}{7.153440in}}%
\pgfpathcurveto{\pgfqpoint{1.307077in}{7.161253in}}{\pgfqpoint{1.296478in}{7.165644in}}{\pgfqpoint{1.285428in}{7.165644in}}%
\pgfpathcurveto{\pgfqpoint{1.274377in}{7.165644in}}{\pgfqpoint{1.263778in}{7.161253in}}{\pgfqpoint{1.255965in}{7.153440in}}%
\pgfpathcurveto{\pgfqpoint{1.248151in}{7.145626in}}{\pgfqpoint{1.243761in}{7.135027in}}{\pgfqpoint{1.243761in}{7.123977in}}%
\pgfpathcurveto{\pgfqpoint{1.243761in}{7.112927in}}{\pgfqpoint{1.248151in}{7.102328in}}{\pgfqpoint{1.255965in}{7.094514in}}%
\pgfpathcurveto{\pgfqpoint{1.263778in}{7.086700in}}{\pgfqpoint{1.274377in}{7.082310in}}{\pgfqpoint{1.285428in}{7.082310in}}%
\pgfpathlineto{\pgfqpoint{1.285428in}{7.082310in}}%
\pgfpathclose%
\pgfusepath{stroke,fill}%
\end{pgfscope}%
\begin{pgfscope}%
\pgfpathrectangle{\pgfqpoint{0.633874in}{5.272501in}}{\pgfqpoint{2.177280in}{2.201755in}}%
\pgfusepath{clip}%
\pgfsetbuttcap%
\pgfsetroundjoin%
\definecolor{currentfill}{rgb}{0.121569,0.466667,0.705882}%
\pgfsetfillcolor{currentfill}%
\pgfsetlinewidth{0.481800pt}%
\definecolor{currentstroke}{rgb}{1.000000,1.000000,1.000000}%
\pgfsetstrokecolor{currentstroke}%
\pgfsetdash{}{0pt}%
\pgfpathmoveto{\pgfqpoint{1.405244in}{7.165710in}}%
\pgfpathcurveto{\pgfqpoint{1.416294in}{7.165710in}}{\pgfqpoint{1.426893in}{7.170100in}}{\pgfqpoint{1.434706in}{7.177914in}}%
\pgfpathcurveto{\pgfqpoint{1.442520in}{7.185728in}}{\pgfqpoint{1.446910in}{7.196327in}}{\pgfqpoint{1.446910in}{7.207377in}}%
\pgfpathcurveto{\pgfqpoint{1.446910in}{7.218427in}}{\pgfqpoint{1.442520in}{7.229026in}}{\pgfqpoint{1.434706in}{7.236839in}}%
\pgfpathcurveto{\pgfqpoint{1.426893in}{7.244653in}}{\pgfqpoint{1.416294in}{7.249043in}}{\pgfqpoint{1.405244in}{7.249043in}}%
\pgfpathcurveto{\pgfqpoint{1.394193in}{7.249043in}}{\pgfqpoint{1.383594in}{7.244653in}}{\pgfqpoint{1.375781in}{7.236839in}}%
\pgfpathcurveto{\pgfqpoint{1.367967in}{7.229026in}}{\pgfqpoint{1.363577in}{7.218427in}}{\pgfqpoint{1.363577in}{7.207377in}}%
\pgfpathcurveto{\pgfqpoint{1.363577in}{7.196327in}}{\pgfqpoint{1.367967in}{7.185728in}}{\pgfqpoint{1.375781in}{7.177914in}}%
\pgfpathcurveto{\pgfqpoint{1.383594in}{7.170100in}}{\pgfqpoint{1.394193in}{7.165710in}}{\pgfqpoint{1.405244in}{7.165710in}}%
\pgfpathlineto{\pgfqpoint{1.405244in}{7.165710in}}%
\pgfpathclose%
\pgfusepath{stroke,fill}%
\end{pgfscope}%
\begin{pgfscope}%
\pgfpathrectangle{\pgfqpoint{0.633874in}{5.272501in}}{\pgfqpoint{2.177280in}{2.201755in}}%
\pgfusepath{clip}%
\pgfsetbuttcap%
\pgfsetroundjoin%
\definecolor{currentfill}{rgb}{0.121569,0.466667,0.705882}%
\pgfsetfillcolor{currentfill}%
\pgfsetlinewidth{0.481800pt}%
\definecolor{currentstroke}{rgb}{1.000000,1.000000,1.000000}%
\pgfsetstrokecolor{currentstroke}%
\pgfsetdash{}{0pt}%
\pgfpathmoveto{\pgfqpoint{1.165611in}{6.248312in}}%
\pgfpathcurveto{\pgfqpoint{1.176662in}{6.248312in}}{\pgfqpoint{1.187261in}{6.252702in}}{\pgfqpoint{1.195074in}{6.260516in}}%
\pgfpathcurveto{\pgfqpoint{1.202888in}{6.268330in}}{\pgfqpoint{1.207278in}{6.278929in}}{\pgfqpoint{1.207278in}{6.289979in}}%
\pgfpathcurveto{\pgfqpoint{1.207278in}{6.301029in}}{\pgfqpoint{1.202888in}{6.311628in}}{\pgfqpoint{1.195074in}{6.319442in}}%
\pgfpathcurveto{\pgfqpoint{1.187261in}{6.327255in}}{\pgfqpoint{1.176662in}{6.331645in}}{\pgfqpoint{1.165611in}{6.331645in}}%
\pgfpathcurveto{\pgfqpoint{1.154561in}{6.331645in}}{\pgfqpoint{1.143962in}{6.327255in}}{\pgfqpoint{1.136149in}{6.319442in}}%
\pgfpathcurveto{\pgfqpoint{1.128335in}{6.311628in}}{\pgfqpoint{1.123945in}{6.301029in}}{\pgfqpoint{1.123945in}{6.289979in}}%
\pgfpathcurveto{\pgfqpoint{1.123945in}{6.278929in}}{\pgfqpoint{1.128335in}{6.268330in}}{\pgfqpoint{1.136149in}{6.260516in}}%
\pgfpathcurveto{\pgfqpoint{1.143962in}{6.252702in}}{\pgfqpoint{1.154561in}{6.248312in}}{\pgfqpoint{1.165611in}{6.248312in}}%
\pgfpathlineto{\pgfqpoint{1.165611in}{6.248312in}}%
\pgfpathclose%
\pgfusepath{stroke,fill}%
\end{pgfscope}%
\begin{pgfscope}%
\pgfpathrectangle{\pgfqpoint{0.633874in}{5.272501in}}{\pgfqpoint{2.177280in}{2.201755in}}%
\pgfusepath{clip}%
\pgfsetbuttcap%
\pgfsetroundjoin%
\definecolor{currentfill}{rgb}{0.121569,0.466667,0.705882}%
\pgfsetfillcolor{currentfill}%
\pgfsetlinewidth{0.481800pt}%
\definecolor{currentstroke}{rgb}{1.000000,1.000000,1.000000}%
\pgfsetstrokecolor{currentstroke}%
\pgfsetdash{}{0pt}%
\pgfpathmoveto{\pgfqpoint{1.205550in}{6.331712in}}%
\pgfpathcurveto{\pgfqpoint{1.216600in}{6.331712in}}{\pgfqpoint{1.227199in}{6.336102in}}{\pgfqpoint{1.235013in}{6.343916in}}%
\pgfpathcurveto{\pgfqpoint{1.242826in}{6.351729in}}{\pgfqpoint{1.247217in}{6.362328in}}{\pgfqpoint{1.247217in}{6.373379in}}%
\pgfpathcurveto{\pgfqpoint{1.247217in}{6.384429in}}{\pgfqpoint{1.242826in}{6.395028in}}{\pgfqpoint{1.235013in}{6.402841in}}%
\pgfpathcurveto{\pgfqpoint{1.227199in}{6.410655in}}{\pgfqpoint{1.216600in}{6.415045in}}{\pgfqpoint{1.205550in}{6.415045in}}%
\pgfpathcurveto{\pgfqpoint{1.194500in}{6.415045in}}{\pgfqpoint{1.183901in}{6.410655in}}{\pgfqpoint{1.176087in}{6.402841in}}%
\pgfpathcurveto{\pgfqpoint{1.168274in}{6.395028in}}{\pgfqpoint{1.163883in}{6.384429in}}{\pgfqpoint{1.163883in}{6.373379in}}%
\pgfpathcurveto{\pgfqpoint{1.163883in}{6.362328in}}{\pgfqpoint{1.168274in}{6.351729in}}{\pgfqpoint{1.176087in}{6.343916in}}%
\pgfpathcurveto{\pgfqpoint{1.183901in}{6.336102in}}{\pgfqpoint{1.194500in}{6.331712in}}{\pgfqpoint{1.205550in}{6.331712in}}%
\pgfpathlineto{\pgfqpoint{1.205550in}{6.331712in}}%
\pgfpathclose%
\pgfusepath{stroke,fill}%
\end{pgfscope}%
\begin{pgfscope}%
\pgfpathrectangle{\pgfqpoint{0.633874in}{5.272501in}}{\pgfqpoint{2.177280in}{2.201755in}}%
\pgfusepath{clip}%
\pgfsetbuttcap%
\pgfsetroundjoin%
\definecolor{currentfill}{rgb}{0.121569,0.466667,0.705882}%
\pgfsetfillcolor{currentfill}%
\pgfsetlinewidth{0.481800pt}%
\definecolor{currentstroke}{rgb}{1.000000,1.000000,1.000000}%
\pgfsetstrokecolor{currentstroke}%
\pgfsetdash{}{0pt}%
\pgfpathmoveto{\pgfqpoint{1.405244in}{6.581911in}}%
\pgfpathcurveto{\pgfqpoint{1.416294in}{6.581911in}}{\pgfqpoint{1.426893in}{6.586302in}}{\pgfqpoint{1.434706in}{6.594115in}}%
\pgfpathcurveto{\pgfqpoint{1.442520in}{6.601929in}}{\pgfqpoint{1.446910in}{6.612528in}}{\pgfqpoint{1.446910in}{6.623578in}}%
\pgfpathcurveto{\pgfqpoint{1.446910in}{6.634628in}}{\pgfqpoint{1.442520in}{6.645227in}}{\pgfqpoint{1.434706in}{6.653041in}}%
\pgfpathcurveto{\pgfqpoint{1.426893in}{6.660854in}}{\pgfqpoint{1.416294in}{6.665245in}}{\pgfqpoint{1.405244in}{6.665245in}}%
\pgfpathcurveto{\pgfqpoint{1.394193in}{6.665245in}}{\pgfqpoint{1.383594in}{6.660854in}}{\pgfqpoint{1.375781in}{6.653041in}}%
\pgfpathcurveto{\pgfqpoint{1.367967in}{6.645227in}}{\pgfqpoint{1.363577in}{6.634628in}}{\pgfqpoint{1.363577in}{6.623578in}}%
\pgfpathcurveto{\pgfqpoint{1.363577in}{6.612528in}}{\pgfqpoint{1.367967in}{6.601929in}}{\pgfqpoint{1.375781in}{6.594115in}}%
\pgfpathcurveto{\pgfqpoint{1.383594in}{6.586302in}}{\pgfqpoint{1.394193in}{6.581911in}}{\pgfqpoint{1.405244in}{6.581911in}}%
\pgfpathlineto{\pgfqpoint{1.405244in}{6.581911in}}%
\pgfpathclose%
\pgfusepath{stroke,fill}%
\end{pgfscope}%
\begin{pgfscope}%
\pgfpathrectangle{\pgfqpoint{0.633874in}{5.272501in}}{\pgfqpoint{2.177280in}{2.201755in}}%
\pgfusepath{clip}%
\pgfsetbuttcap%
\pgfsetroundjoin%
\definecolor{currentfill}{rgb}{0.121569,0.466667,0.705882}%
\pgfsetfillcolor{currentfill}%
\pgfsetlinewidth{0.481800pt}%
\definecolor{currentstroke}{rgb}{1.000000,1.000000,1.000000}%
\pgfsetstrokecolor{currentstroke}%
\pgfsetdash{}{0pt}%
\pgfpathmoveto{\pgfqpoint{1.165611in}{6.665311in}}%
\pgfpathcurveto{\pgfqpoint{1.176662in}{6.665311in}}{\pgfqpoint{1.187261in}{6.669701in}}{\pgfqpoint{1.195074in}{6.677515in}}%
\pgfpathcurveto{\pgfqpoint{1.202888in}{6.685329in}}{\pgfqpoint{1.207278in}{6.695928in}}{\pgfqpoint{1.207278in}{6.706978in}}%
\pgfpathcurveto{\pgfqpoint{1.207278in}{6.718028in}}{\pgfqpoint{1.202888in}{6.728627in}}{\pgfqpoint{1.195074in}{6.736441in}}%
\pgfpathcurveto{\pgfqpoint{1.187261in}{6.744254in}}{\pgfqpoint{1.176662in}{6.748644in}}{\pgfqpoint{1.165611in}{6.748644in}}%
\pgfpathcurveto{\pgfqpoint{1.154561in}{6.748644in}}{\pgfqpoint{1.143962in}{6.744254in}}{\pgfqpoint{1.136149in}{6.736441in}}%
\pgfpathcurveto{\pgfqpoint{1.128335in}{6.728627in}}{\pgfqpoint{1.123945in}{6.718028in}}{\pgfqpoint{1.123945in}{6.706978in}}%
\pgfpathcurveto{\pgfqpoint{1.123945in}{6.695928in}}{\pgfqpoint{1.128335in}{6.685329in}}{\pgfqpoint{1.136149in}{6.677515in}}%
\pgfpathcurveto{\pgfqpoint{1.143962in}{6.669701in}}{\pgfqpoint{1.154561in}{6.665311in}}{\pgfqpoint{1.165611in}{6.665311in}}%
\pgfpathlineto{\pgfqpoint{1.165611in}{6.665311in}}%
\pgfpathclose%
\pgfusepath{stroke,fill}%
\end{pgfscope}%
\begin{pgfscope}%
\pgfpathrectangle{\pgfqpoint{0.633874in}{5.272501in}}{\pgfqpoint{2.177280in}{2.201755in}}%
\pgfusepath{clip}%
\pgfsetbuttcap%
\pgfsetroundjoin%
\definecolor{currentfill}{rgb}{0.121569,0.466667,0.705882}%
\pgfsetfillcolor{currentfill}%
\pgfsetlinewidth{0.481800pt}%
\definecolor{currentstroke}{rgb}{1.000000,1.000000,1.000000}%
\pgfsetstrokecolor{currentstroke}%
\pgfsetdash{}{0pt}%
\pgfpathmoveto{\pgfqpoint{0.965918in}{6.164912in}}%
\pgfpathcurveto{\pgfqpoint{0.976968in}{6.164912in}}{\pgfqpoint{0.987567in}{6.169303in}}{\pgfqpoint{0.995381in}{6.177116in}}%
\pgfpathcurveto{\pgfqpoint{1.003194in}{6.184930in}}{\pgfqpoint{1.007585in}{6.195529in}}{\pgfqpoint{1.007585in}{6.206579in}}%
\pgfpathcurveto{\pgfqpoint{1.007585in}{6.217629in}}{\pgfqpoint{1.003194in}{6.228228in}}{\pgfqpoint{0.995381in}{6.236042in}}%
\pgfpathcurveto{\pgfqpoint{0.987567in}{6.243855in}}{\pgfqpoint{0.976968in}{6.248246in}}{\pgfqpoint{0.965918in}{6.248246in}}%
\pgfpathcurveto{\pgfqpoint{0.954868in}{6.248246in}}{\pgfqpoint{0.944269in}{6.243855in}}{\pgfqpoint{0.936455in}{6.236042in}}%
\pgfpathcurveto{\pgfqpoint{0.928641in}{6.228228in}}{\pgfqpoint{0.924251in}{6.217629in}}{\pgfqpoint{0.924251in}{6.206579in}}%
\pgfpathcurveto{\pgfqpoint{0.924251in}{6.195529in}}{\pgfqpoint{0.928641in}{6.184930in}}{\pgfqpoint{0.936455in}{6.177116in}}%
\pgfpathcurveto{\pgfqpoint{0.944269in}{6.169303in}}{\pgfqpoint{0.954868in}{6.164912in}}{\pgfqpoint{0.965918in}{6.164912in}}%
\pgfpathlineto{\pgfqpoint{0.965918in}{6.164912in}}%
\pgfpathclose%
\pgfusepath{stroke,fill}%
\end{pgfscope}%
\begin{pgfscope}%
\pgfpathrectangle{\pgfqpoint{0.633874in}{5.272501in}}{\pgfqpoint{2.177280in}{2.201755in}}%
\pgfusepath{clip}%
\pgfsetbuttcap%
\pgfsetroundjoin%
\definecolor{currentfill}{rgb}{0.121569,0.466667,0.705882}%
\pgfsetfillcolor{currentfill}%
\pgfsetlinewidth{0.481800pt}%
\definecolor{currentstroke}{rgb}{1.000000,1.000000,1.000000}%
\pgfsetstrokecolor{currentstroke}%
\pgfsetdash{}{0pt}%
\pgfpathmoveto{\pgfqpoint{1.245489in}{6.498512in}}%
\pgfpathcurveto{\pgfqpoint{1.256539in}{6.498512in}}{\pgfqpoint{1.267138in}{6.502902in}}{\pgfqpoint{1.274952in}{6.510715in}}%
\pgfpathcurveto{\pgfqpoint{1.282765in}{6.518529in}}{\pgfqpoint{1.287155in}{6.529128in}}{\pgfqpoint{1.287155in}{6.540178in}}%
\pgfpathcurveto{\pgfqpoint{1.287155in}{6.551228in}}{\pgfqpoint{1.282765in}{6.561827in}}{\pgfqpoint{1.274952in}{6.569641in}}%
\pgfpathcurveto{\pgfqpoint{1.267138in}{6.577455in}}{\pgfqpoint{1.256539in}{6.581845in}}{\pgfqpoint{1.245489in}{6.581845in}}%
\pgfpathcurveto{\pgfqpoint{1.234439in}{6.581845in}}{\pgfqpoint{1.223840in}{6.577455in}}{\pgfqpoint{1.216026in}{6.569641in}}%
\pgfpathcurveto{\pgfqpoint{1.208212in}{6.561827in}}{\pgfqpoint{1.203822in}{6.551228in}}{\pgfqpoint{1.203822in}{6.540178in}}%
\pgfpathcurveto{\pgfqpoint{1.203822in}{6.529128in}}{\pgfqpoint{1.208212in}{6.518529in}}{\pgfqpoint{1.216026in}{6.510715in}}%
\pgfpathcurveto{\pgfqpoint{1.223840in}{6.502902in}}{\pgfqpoint{1.234439in}{6.498512in}}{\pgfqpoint{1.245489in}{6.498512in}}%
\pgfpathlineto{\pgfqpoint{1.245489in}{6.498512in}}%
\pgfpathclose%
\pgfusepath{stroke,fill}%
\end{pgfscope}%
\begin{pgfscope}%
\pgfpathrectangle{\pgfqpoint{0.633874in}{5.272501in}}{\pgfqpoint{2.177280in}{2.201755in}}%
\pgfusepath{clip}%
\pgfsetbuttcap%
\pgfsetroundjoin%
\definecolor{currentfill}{rgb}{0.121569,0.466667,0.705882}%
\pgfsetfillcolor{currentfill}%
\pgfsetlinewidth{0.481800pt}%
\definecolor{currentstroke}{rgb}{1.000000,1.000000,1.000000}%
\pgfsetstrokecolor{currentstroke}%
\pgfsetdash{}{0pt}%
\pgfpathmoveto{\pgfqpoint{1.205550in}{6.581911in}}%
\pgfpathcurveto{\pgfqpoint{1.216600in}{6.581911in}}{\pgfqpoint{1.227199in}{6.586302in}}{\pgfqpoint{1.235013in}{6.594115in}}%
\pgfpathcurveto{\pgfqpoint{1.242826in}{6.601929in}}{\pgfqpoint{1.247217in}{6.612528in}}{\pgfqpoint{1.247217in}{6.623578in}}%
\pgfpathcurveto{\pgfqpoint{1.247217in}{6.634628in}}{\pgfqpoint{1.242826in}{6.645227in}}{\pgfqpoint{1.235013in}{6.653041in}}%
\pgfpathcurveto{\pgfqpoint{1.227199in}{6.660854in}}{\pgfqpoint{1.216600in}{6.665245in}}{\pgfqpoint{1.205550in}{6.665245in}}%
\pgfpathcurveto{\pgfqpoint{1.194500in}{6.665245in}}{\pgfqpoint{1.183901in}{6.660854in}}{\pgfqpoint{1.176087in}{6.653041in}}%
\pgfpathcurveto{\pgfqpoint{1.168274in}{6.645227in}}{\pgfqpoint{1.163883in}{6.634628in}}{\pgfqpoint{1.163883in}{6.623578in}}%
\pgfpathcurveto{\pgfqpoint{1.163883in}{6.612528in}}{\pgfqpoint{1.168274in}{6.601929in}}{\pgfqpoint{1.176087in}{6.594115in}}%
\pgfpathcurveto{\pgfqpoint{1.183901in}{6.586302in}}{\pgfqpoint{1.194500in}{6.581911in}}{\pgfqpoint{1.205550in}{6.581911in}}%
\pgfpathlineto{\pgfqpoint{1.205550in}{6.581911in}}%
\pgfpathclose%
\pgfusepath{stroke,fill}%
\end{pgfscope}%
\begin{pgfscope}%
\pgfpathrectangle{\pgfqpoint{0.633874in}{5.272501in}}{\pgfqpoint{2.177280in}{2.201755in}}%
\pgfusepath{clip}%
\pgfsetbuttcap%
\pgfsetroundjoin%
\definecolor{currentfill}{rgb}{0.121569,0.466667,0.705882}%
\pgfsetfillcolor{currentfill}%
\pgfsetlinewidth{0.481800pt}%
\definecolor{currentstroke}{rgb}{1.000000,1.000000,1.000000}%
\pgfsetstrokecolor{currentstroke}%
\pgfsetdash{}{0pt}%
\pgfpathmoveto{\pgfqpoint{1.005857in}{5.581114in}}%
\pgfpathcurveto{\pgfqpoint{1.016907in}{5.581114in}}{\pgfqpoint{1.027506in}{5.585504in}}{\pgfqpoint{1.035319in}{5.593317in}}%
\pgfpathcurveto{\pgfqpoint{1.043133in}{5.601131in}}{\pgfqpoint{1.047523in}{5.611730in}}{\pgfqpoint{1.047523in}{5.622780in}}%
\pgfpathcurveto{\pgfqpoint{1.047523in}{5.633830in}}{\pgfqpoint{1.043133in}{5.644429in}}{\pgfqpoint{1.035319in}{5.652243in}}%
\pgfpathcurveto{\pgfqpoint{1.027506in}{5.660057in}}{\pgfqpoint{1.016907in}{5.664447in}}{\pgfqpoint{1.005857in}{5.664447in}}%
\pgfpathcurveto{\pgfqpoint{0.994806in}{5.664447in}}{\pgfqpoint{0.984207in}{5.660057in}}{\pgfqpoint{0.976394in}{5.652243in}}%
\pgfpathcurveto{\pgfqpoint{0.968580in}{5.644429in}}{\pgfqpoint{0.964190in}{5.633830in}}{\pgfqpoint{0.964190in}{5.622780in}}%
\pgfpathcurveto{\pgfqpoint{0.964190in}{5.611730in}}{\pgfqpoint{0.968580in}{5.601131in}}{\pgfqpoint{0.976394in}{5.593317in}}%
\pgfpathcurveto{\pgfqpoint{0.984207in}{5.585504in}}{\pgfqpoint{0.994806in}{5.581114in}}{\pgfqpoint{1.005857in}{5.581114in}}%
\pgfpathlineto{\pgfqpoint{1.005857in}{5.581114in}}%
\pgfpathclose%
\pgfusepath{stroke,fill}%
\end{pgfscope}%
\begin{pgfscope}%
\pgfpathrectangle{\pgfqpoint{0.633874in}{5.272501in}}{\pgfqpoint{2.177280in}{2.201755in}}%
\pgfusepath{clip}%
\pgfsetbuttcap%
\pgfsetroundjoin%
\definecolor{currentfill}{rgb}{0.121569,0.466667,0.705882}%
\pgfsetfillcolor{currentfill}%
\pgfsetlinewidth{0.481800pt}%
\definecolor{currentstroke}{rgb}{1.000000,1.000000,1.000000}%
\pgfsetstrokecolor{currentstroke}%
\pgfsetdash{}{0pt}%
\pgfpathmoveto{\pgfqpoint{0.965918in}{6.331712in}}%
\pgfpathcurveto{\pgfqpoint{0.976968in}{6.331712in}}{\pgfqpoint{0.987567in}{6.336102in}}{\pgfqpoint{0.995381in}{6.343916in}}%
\pgfpathcurveto{\pgfqpoint{1.003194in}{6.351729in}}{\pgfqpoint{1.007585in}{6.362328in}}{\pgfqpoint{1.007585in}{6.373379in}}%
\pgfpathcurveto{\pgfqpoint{1.007585in}{6.384429in}}{\pgfqpoint{1.003194in}{6.395028in}}{\pgfqpoint{0.995381in}{6.402841in}}%
\pgfpathcurveto{\pgfqpoint{0.987567in}{6.410655in}}{\pgfqpoint{0.976968in}{6.415045in}}{\pgfqpoint{0.965918in}{6.415045in}}%
\pgfpathcurveto{\pgfqpoint{0.954868in}{6.415045in}}{\pgfqpoint{0.944269in}{6.410655in}}{\pgfqpoint{0.936455in}{6.402841in}}%
\pgfpathcurveto{\pgfqpoint{0.928641in}{6.395028in}}{\pgfqpoint{0.924251in}{6.384429in}}{\pgfqpoint{0.924251in}{6.373379in}}%
\pgfpathcurveto{\pgfqpoint{0.924251in}{6.362328in}}{\pgfqpoint{0.928641in}{6.351729in}}{\pgfqpoint{0.936455in}{6.343916in}}%
\pgfpathcurveto{\pgfqpoint{0.944269in}{6.336102in}}{\pgfqpoint{0.954868in}{6.331712in}}{\pgfqpoint{0.965918in}{6.331712in}}%
\pgfpathlineto{\pgfqpoint{0.965918in}{6.331712in}}%
\pgfpathclose%
\pgfusepath{stroke,fill}%
\end{pgfscope}%
\begin{pgfscope}%
\pgfpathrectangle{\pgfqpoint{0.633874in}{5.272501in}}{\pgfqpoint{2.177280in}{2.201755in}}%
\pgfusepath{clip}%
\pgfsetbuttcap%
\pgfsetroundjoin%
\definecolor{currentfill}{rgb}{0.121569,0.466667,0.705882}%
\pgfsetfillcolor{currentfill}%
\pgfsetlinewidth{0.481800pt}%
\definecolor{currentstroke}{rgb}{1.000000,1.000000,1.000000}%
\pgfsetstrokecolor{currentstroke}%
\pgfsetdash{}{0pt}%
\pgfpathmoveto{\pgfqpoint{1.205550in}{6.581911in}}%
\pgfpathcurveto{\pgfqpoint{1.216600in}{6.581911in}}{\pgfqpoint{1.227199in}{6.586302in}}{\pgfqpoint{1.235013in}{6.594115in}}%
\pgfpathcurveto{\pgfqpoint{1.242826in}{6.601929in}}{\pgfqpoint{1.247217in}{6.612528in}}{\pgfqpoint{1.247217in}{6.623578in}}%
\pgfpathcurveto{\pgfqpoint{1.247217in}{6.634628in}}{\pgfqpoint{1.242826in}{6.645227in}}{\pgfqpoint{1.235013in}{6.653041in}}%
\pgfpathcurveto{\pgfqpoint{1.227199in}{6.660854in}}{\pgfqpoint{1.216600in}{6.665245in}}{\pgfqpoint{1.205550in}{6.665245in}}%
\pgfpathcurveto{\pgfqpoint{1.194500in}{6.665245in}}{\pgfqpoint{1.183901in}{6.660854in}}{\pgfqpoint{1.176087in}{6.653041in}}%
\pgfpathcurveto{\pgfqpoint{1.168274in}{6.645227in}}{\pgfqpoint{1.163883in}{6.634628in}}{\pgfqpoint{1.163883in}{6.623578in}}%
\pgfpathcurveto{\pgfqpoint{1.163883in}{6.612528in}}{\pgfqpoint{1.168274in}{6.601929in}}{\pgfqpoint{1.176087in}{6.594115in}}%
\pgfpathcurveto{\pgfqpoint{1.183901in}{6.586302in}}{\pgfqpoint{1.194500in}{6.581911in}}{\pgfqpoint{1.205550in}{6.581911in}}%
\pgfpathlineto{\pgfqpoint{1.205550in}{6.581911in}}%
\pgfpathclose%
\pgfusepath{stroke,fill}%
\end{pgfscope}%
\begin{pgfscope}%
\pgfpathrectangle{\pgfqpoint{0.633874in}{5.272501in}}{\pgfqpoint{2.177280in}{2.201755in}}%
\pgfusepath{clip}%
\pgfsetbuttcap%
\pgfsetroundjoin%
\definecolor{currentfill}{rgb}{0.121569,0.466667,0.705882}%
\pgfsetfillcolor{currentfill}%
\pgfsetlinewidth{0.481800pt}%
\definecolor{currentstroke}{rgb}{1.000000,1.000000,1.000000}%
\pgfsetstrokecolor{currentstroke}%
\pgfsetdash{}{0pt}%
\pgfpathmoveto{\pgfqpoint{1.245489in}{6.832111in}}%
\pgfpathcurveto{\pgfqpoint{1.256539in}{6.832111in}}{\pgfqpoint{1.267138in}{6.836501in}}{\pgfqpoint{1.274952in}{6.844315in}}%
\pgfpathcurveto{\pgfqpoint{1.282765in}{6.852128in}}{\pgfqpoint{1.287155in}{6.862727in}}{\pgfqpoint{1.287155in}{6.873777in}}%
\pgfpathcurveto{\pgfqpoint{1.287155in}{6.884828in}}{\pgfqpoint{1.282765in}{6.895427in}}{\pgfqpoint{1.274952in}{6.903240in}}%
\pgfpathcurveto{\pgfqpoint{1.267138in}{6.911054in}}{\pgfqpoint{1.256539in}{6.915444in}}{\pgfqpoint{1.245489in}{6.915444in}}%
\pgfpathcurveto{\pgfqpoint{1.234439in}{6.915444in}}{\pgfqpoint{1.223840in}{6.911054in}}{\pgfqpoint{1.216026in}{6.903240in}}%
\pgfpathcurveto{\pgfqpoint{1.208212in}{6.895427in}}{\pgfqpoint{1.203822in}{6.884828in}}{\pgfqpoint{1.203822in}{6.873777in}}%
\pgfpathcurveto{\pgfqpoint{1.203822in}{6.862727in}}{\pgfqpoint{1.208212in}{6.852128in}}{\pgfqpoint{1.216026in}{6.844315in}}%
\pgfpathcurveto{\pgfqpoint{1.223840in}{6.836501in}}{\pgfqpoint{1.234439in}{6.832111in}}{\pgfqpoint{1.245489in}{6.832111in}}%
\pgfpathlineto{\pgfqpoint{1.245489in}{6.832111in}}%
\pgfpathclose%
\pgfusepath{stroke,fill}%
\end{pgfscope}%
\begin{pgfscope}%
\pgfpathrectangle{\pgfqpoint{0.633874in}{5.272501in}}{\pgfqpoint{2.177280in}{2.201755in}}%
\pgfusepath{clip}%
\pgfsetbuttcap%
\pgfsetroundjoin%
\definecolor{currentfill}{rgb}{0.121569,0.466667,0.705882}%
\pgfsetfillcolor{currentfill}%
\pgfsetlinewidth{0.481800pt}%
\definecolor{currentstroke}{rgb}{1.000000,1.000000,1.000000}%
\pgfsetstrokecolor{currentstroke}%
\pgfsetdash{}{0pt}%
\pgfpathmoveto{\pgfqpoint{1.125673in}{6.164912in}}%
\pgfpathcurveto{\pgfqpoint{1.136723in}{6.164912in}}{\pgfqpoint{1.147322in}{6.169303in}}{\pgfqpoint{1.155135in}{6.177116in}}%
\pgfpathcurveto{\pgfqpoint{1.162949in}{6.184930in}}{\pgfqpoint{1.167339in}{6.195529in}}{\pgfqpoint{1.167339in}{6.206579in}}%
\pgfpathcurveto{\pgfqpoint{1.167339in}{6.217629in}}{\pgfqpoint{1.162949in}{6.228228in}}{\pgfqpoint{1.155135in}{6.236042in}}%
\pgfpathcurveto{\pgfqpoint{1.147322in}{6.243855in}}{\pgfqpoint{1.136723in}{6.248246in}}{\pgfqpoint{1.125673in}{6.248246in}}%
\pgfpathcurveto{\pgfqpoint{1.114623in}{6.248246in}}{\pgfqpoint{1.104024in}{6.243855in}}{\pgfqpoint{1.096210in}{6.236042in}}%
\pgfpathcurveto{\pgfqpoint{1.088396in}{6.228228in}}{\pgfqpoint{1.084006in}{6.217629in}}{\pgfqpoint{1.084006in}{6.206579in}}%
\pgfpathcurveto{\pgfqpoint{1.084006in}{6.195529in}}{\pgfqpoint{1.088396in}{6.184930in}}{\pgfqpoint{1.096210in}{6.177116in}}%
\pgfpathcurveto{\pgfqpoint{1.104024in}{6.169303in}}{\pgfqpoint{1.114623in}{6.164912in}}{\pgfqpoint{1.125673in}{6.164912in}}%
\pgfpathlineto{\pgfqpoint{1.125673in}{6.164912in}}%
\pgfpathclose%
\pgfusepath{stroke,fill}%
\end{pgfscope}%
\begin{pgfscope}%
\pgfpathrectangle{\pgfqpoint{0.633874in}{5.272501in}}{\pgfqpoint{2.177280in}{2.201755in}}%
\pgfusepath{clip}%
\pgfsetbuttcap%
\pgfsetroundjoin%
\definecolor{currentfill}{rgb}{0.121569,0.466667,0.705882}%
\pgfsetfillcolor{currentfill}%
\pgfsetlinewidth{0.481800pt}%
\definecolor{currentstroke}{rgb}{1.000000,1.000000,1.000000}%
\pgfsetstrokecolor{currentstroke}%
\pgfsetdash{}{0pt}%
\pgfpathmoveto{\pgfqpoint{1.245489in}{6.832111in}}%
\pgfpathcurveto{\pgfqpoint{1.256539in}{6.832111in}}{\pgfqpoint{1.267138in}{6.836501in}}{\pgfqpoint{1.274952in}{6.844315in}}%
\pgfpathcurveto{\pgfqpoint{1.282765in}{6.852128in}}{\pgfqpoint{1.287155in}{6.862727in}}{\pgfqpoint{1.287155in}{6.873777in}}%
\pgfpathcurveto{\pgfqpoint{1.287155in}{6.884828in}}{\pgfqpoint{1.282765in}{6.895427in}}{\pgfqpoint{1.274952in}{6.903240in}}%
\pgfpathcurveto{\pgfqpoint{1.267138in}{6.911054in}}{\pgfqpoint{1.256539in}{6.915444in}}{\pgfqpoint{1.245489in}{6.915444in}}%
\pgfpathcurveto{\pgfqpoint{1.234439in}{6.915444in}}{\pgfqpoint{1.223840in}{6.911054in}}{\pgfqpoint{1.216026in}{6.903240in}}%
\pgfpathcurveto{\pgfqpoint{1.208212in}{6.895427in}}{\pgfqpoint{1.203822in}{6.884828in}}{\pgfqpoint{1.203822in}{6.873777in}}%
\pgfpathcurveto{\pgfqpoint{1.203822in}{6.862727in}}{\pgfqpoint{1.208212in}{6.852128in}}{\pgfqpoint{1.216026in}{6.844315in}}%
\pgfpathcurveto{\pgfqpoint{1.223840in}{6.836501in}}{\pgfqpoint{1.234439in}{6.832111in}}{\pgfqpoint{1.245489in}{6.832111in}}%
\pgfpathlineto{\pgfqpoint{1.245489in}{6.832111in}}%
\pgfpathclose%
\pgfusepath{stroke,fill}%
\end{pgfscope}%
\begin{pgfscope}%
\pgfpathrectangle{\pgfqpoint{0.633874in}{5.272501in}}{\pgfqpoint{2.177280in}{2.201755in}}%
\pgfusepath{clip}%
\pgfsetbuttcap%
\pgfsetroundjoin%
\definecolor{currentfill}{rgb}{0.121569,0.466667,0.705882}%
\pgfsetfillcolor{currentfill}%
\pgfsetlinewidth{0.481800pt}%
\definecolor{currentstroke}{rgb}{1.000000,1.000000,1.000000}%
\pgfsetstrokecolor{currentstroke}%
\pgfsetdash{}{0pt}%
\pgfpathmoveto{\pgfqpoint{1.045795in}{6.331712in}}%
\pgfpathcurveto{\pgfqpoint{1.056845in}{6.331712in}}{\pgfqpoint{1.067444in}{6.336102in}}{\pgfqpoint{1.075258in}{6.343916in}}%
\pgfpathcurveto{\pgfqpoint{1.083072in}{6.351729in}}{\pgfqpoint{1.087462in}{6.362328in}}{\pgfqpoint{1.087462in}{6.373379in}}%
\pgfpathcurveto{\pgfqpoint{1.087462in}{6.384429in}}{\pgfqpoint{1.083072in}{6.395028in}}{\pgfqpoint{1.075258in}{6.402841in}}%
\pgfpathcurveto{\pgfqpoint{1.067444in}{6.410655in}}{\pgfqpoint{1.056845in}{6.415045in}}{\pgfqpoint{1.045795in}{6.415045in}}%
\pgfpathcurveto{\pgfqpoint{1.034745in}{6.415045in}}{\pgfqpoint{1.024146in}{6.410655in}}{\pgfqpoint{1.016332in}{6.402841in}}%
\pgfpathcurveto{\pgfqpoint{1.008519in}{6.395028in}}{\pgfqpoint{1.004129in}{6.384429in}}{\pgfqpoint{1.004129in}{6.373379in}}%
\pgfpathcurveto{\pgfqpoint{1.004129in}{6.362328in}}{\pgfqpoint{1.008519in}{6.351729in}}{\pgfqpoint{1.016332in}{6.343916in}}%
\pgfpathcurveto{\pgfqpoint{1.024146in}{6.336102in}}{\pgfqpoint{1.034745in}{6.331712in}}{\pgfqpoint{1.045795in}{6.331712in}}%
\pgfpathlineto{\pgfqpoint{1.045795in}{6.331712in}}%
\pgfpathclose%
\pgfusepath{stroke,fill}%
\end{pgfscope}%
\begin{pgfscope}%
\pgfpathrectangle{\pgfqpoint{0.633874in}{5.272501in}}{\pgfqpoint{2.177280in}{2.201755in}}%
\pgfusepath{clip}%
\pgfsetbuttcap%
\pgfsetroundjoin%
\definecolor{currentfill}{rgb}{0.121569,0.466667,0.705882}%
\pgfsetfillcolor{currentfill}%
\pgfsetlinewidth{0.481800pt}%
\definecolor{currentstroke}{rgb}{1.000000,1.000000,1.000000}%
\pgfsetstrokecolor{currentstroke}%
\pgfsetdash{}{0pt}%
\pgfpathmoveto{\pgfqpoint{1.325366in}{6.748711in}}%
\pgfpathcurveto{\pgfqpoint{1.336416in}{6.748711in}}{\pgfqpoint{1.347015in}{6.753101in}}{\pgfqpoint{1.354829in}{6.760915in}}%
\pgfpathcurveto{\pgfqpoint{1.362643in}{6.768728in}}{\pgfqpoint{1.367033in}{6.779327in}}{\pgfqpoint{1.367033in}{6.790378in}}%
\pgfpathcurveto{\pgfqpoint{1.367033in}{6.801428in}}{\pgfqpoint{1.362643in}{6.812027in}}{\pgfqpoint{1.354829in}{6.819840in}}%
\pgfpathcurveto{\pgfqpoint{1.347015in}{6.827654in}}{\pgfqpoint{1.336416in}{6.832044in}}{\pgfqpoint{1.325366in}{6.832044in}}%
\pgfpathcurveto{\pgfqpoint{1.314316in}{6.832044in}}{\pgfqpoint{1.303717in}{6.827654in}}{\pgfqpoint{1.295903in}{6.819840in}}%
\pgfpathcurveto{\pgfqpoint{1.288090in}{6.812027in}}{\pgfqpoint{1.283700in}{6.801428in}}{\pgfqpoint{1.283700in}{6.790378in}}%
\pgfpathcurveto{\pgfqpoint{1.283700in}{6.779327in}}{\pgfqpoint{1.288090in}{6.768728in}}{\pgfqpoint{1.295903in}{6.760915in}}%
\pgfpathcurveto{\pgfqpoint{1.303717in}{6.753101in}}{\pgfqpoint{1.314316in}{6.748711in}}{\pgfqpoint{1.325366in}{6.748711in}}%
\pgfpathlineto{\pgfqpoint{1.325366in}{6.748711in}}%
\pgfpathclose%
\pgfusepath{stroke,fill}%
\end{pgfscope}%
\begin{pgfscope}%
\pgfpathrectangle{\pgfqpoint{0.633874in}{5.272501in}}{\pgfqpoint{2.177280in}{2.201755in}}%
\pgfusepath{clip}%
\pgfsetbuttcap%
\pgfsetroundjoin%
\definecolor{currentfill}{rgb}{0.121569,0.466667,0.705882}%
\pgfsetfillcolor{currentfill}%
\pgfsetlinewidth{0.481800pt}%
\definecolor{currentstroke}{rgb}{1.000000,1.000000,1.000000}%
\pgfsetstrokecolor{currentstroke}%
\pgfsetdash{}{0pt}%
\pgfpathmoveto{\pgfqpoint{1.205550in}{6.415112in}}%
\pgfpathcurveto{\pgfqpoint{1.216600in}{6.415112in}}{\pgfqpoint{1.227199in}{6.419502in}}{\pgfqpoint{1.235013in}{6.427316in}}%
\pgfpathcurveto{\pgfqpoint{1.242826in}{6.435129in}}{\pgfqpoint{1.247217in}{6.445728in}}{\pgfqpoint{1.247217in}{6.456778in}}%
\pgfpathcurveto{\pgfqpoint{1.247217in}{6.467828in}}{\pgfqpoint{1.242826in}{6.478428in}}{\pgfqpoint{1.235013in}{6.486241in}}%
\pgfpathcurveto{\pgfqpoint{1.227199in}{6.494055in}}{\pgfqpoint{1.216600in}{6.498445in}}{\pgfqpoint{1.205550in}{6.498445in}}%
\pgfpathcurveto{\pgfqpoint{1.194500in}{6.498445in}}{\pgfqpoint{1.183901in}{6.494055in}}{\pgfqpoint{1.176087in}{6.486241in}}%
\pgfpathcurveto{\pgfqpoint{1.168274in}{6.478428in}}{\pgfqpoint{1.163883in}{6.467828in}}{\pgfqpoint{1.163883in}{6.456778in}}%
\pgfpathcurveto{\pgfqpoint{1.163883in}{6.445728in}}{\pgfqpoint{1.168274in}{6.435129in}}{\pgfqpoint{1.176087in}{6.427316in}}%
\pgfpathcurveto{\pgfqpoint{1.183901in}{6.419502in}}{\pgfqpoint{1.194500in}{6.415112in}}{\pgfqpoint{1.205550in}{6.415112in}}%
\pgfpathlineto{\pgfqpoint{1.205550in}{6.415112in}}%
\pgfpathclose%
\pgfusepath{stroke,fill}%
\end{pgfscope}%
\begin{pgfscope}%
\pgfpathrectangle{\pgfqpoint{0.633874in}{5.272501in}}{\pgfqpoint{2.177280in}{2.201755in}}%
\pgfusepath{clip}%
\pgfsetbuttcap%
\pgfsetroundjoin%
\definecolor{currentfill}{rgb}{1.000000,0.498039,0.054902}%
\pgfsetfillcolor{currentfill}%
\pgfsetlinewidth{0.481800pt}%
\definecolor{currentstroke}{rgb}{1.000000,1.000000,1.000000}%
\pgfsetstrokecolor{currentstroke}%
\pgfsetdash{}{0pt}%
\pgfpathmoveto{\pgfqpoint{2.004324in}{6.331712in}}%
\pgfpathcurveto{\pgfqpoint{2.015374in}{6.331712in}}{\pgfqpoint{2.025973in}{6.336102in}}{\pgfqpoint{2.033787in}{6.343916in}}%
\pgfpathcurveto{\pgfqpoint{2.041601in}{6.351729in}}{\pgfqpoint{2.045991in}{6.362328in}}{\pgfqpoint{2.045991in}{6.373379in}}%
\pgfpathcurveto{\pgfqpoint{2.045991in}{6.384429in}}{\pgfqpoint{2.041601in}{6.395028in}}{\pgfqpoint{2.033787in}{6.402841in}}%
\pgfpathcurveto{\pgfqpoint{2.025973in}{6.410655in}}{\pgfqpoint{2.015374in}{6.415045in}}{\pgfqpoint{2.004324in}{6.415045in}}%
\pgfpathcurveto{\pgfqpoint{1.993274in}{6.415045in}}{\pgfqpoint{1.982675in}{6.410655in}}{\pgfqpoint{1.974861in}{6.402841in}}%
\pgfpathcurveto{\pgfqpoint{1.967048in}{6.395028in}}{\pgfqpoint{1.962658in}{6.384429in}}{\pgfqpoint{1.962658in}{6.373379in}}%
\pgfpathcurveto{\pgfqpoint{1.962658in}{6.362328in}}{\pgfqpoint{1.967048in}{6.351729in}}{\pgfqpoint{1.974861in}{6.343916in}}%
\pgfpathcurveto{\pgfqpoint{1.982675in}{6.336102in}}{\pgfqpoint{1.993274in}{6.331712in}}{\pgfqpoint{2.004324in}{6.331712in}}%
\pgfpathlineto{\pgfqpoint{2.004324in}{6.331712in}}%
\pgfpathclose%
\pgfusepath{stroke,fill}%
\end{pgfscope}%
\begin{pgfscope}%
\pgfpathrectangle{\pgfqpoint{0.633874in}{5.272501in}}{\pgfqpoint{2.177280in}{2.201755in}}%
\pgfusepath{clip}%
\pgfsetbuttcap%
\pgfsetroundjoin%
\definecolor{currentfill}{rgb}{1.000000,0.498039,0.054902}%
\pgfsetfillcolor{currentfill}%
\pgfsetlinewidth{0.481800pt}%
\definecolor{currentstroke}{rgb}{1.000000,1.000000,1.000000}%
\pgfsetstrokecolor{currentstroke}%
\pgfsetdash{}{0pt}%
\pgfpathmoveto{\pgfqpoint{1.764692in}{6.331712in}}%
\pgfpathcurveto{\pgfqpoint{1.775742in}{6.331712in}}{\pgfqpoint{1.786341in}{6.336102in}}{\pgfqpoint{1.794155in}{6.343916in}}%
\pgfpathcurveto{\pgfqpoint{1.801968in}{6.351729in}}{\pgfqpoint{1.806359in}{6.362328in}}{\pgfqpoint{1.806359in}{6.373379in}}%
\pgfpathcurveto{\pgfqpoint{1.806359in}{6.384429in}}{\pgfqpoint{1.801968in}{6.395028in}}{\pgfqpoint{1.794155in}{6.402841in}}%
\pgfpathcurveto{\pgfqpoint{1.786341in}{6.410655in}}{\pgfqpoint{1.775742in}{6.415045in}}{\pgfqpoint{1.764692in}{6.415045in}}%
\pgfpathcurveto{\pgfqpoint{1.753642in}{6.415045in}}{\pgfqpoint{1.743043in}{6.410655in}}{\pgfqpoint{1.735229in}{6.402841in}}%
\pgfpathcurveto{\pgfqpoint{1.727416in}{6.395028in}}{\pgfqpoint{1.723025in}{6.384429in}}{\pgfqpoint{1.723025in}{6.373379in}}%
\pgfpathcurveto{\pgfqpoint{1.723025in}{6.362328in}}{\pgfqpoint{1.727416in}{6.351729in}}{\pgfqpoint{1.735229in}{6.343916in}}%
\pgfpathcurveto{\pgfqpoint{1.743043in}{6.336102in}}{\pgfqpoint{1.753642in}{6.331712in}}{\pgfqpoint{1.764692in}{6.331712in}}%
\pgfpathlineto{\pgfqpoint{1.764692in}{6.331712in}}%
\pgfpathclose%
\pgfusepath{stroke,fill}%
\end{pgfscope}%
\begin{pgfscope}%
\pgfpathrectangle{\pgfqpoint{0.633874in}{5.272501in}}{\pgfqpoint{2.177280in}{2.201755in}}%
\pgfusepath{clip}%
\pgfsetbuttcap%
\pgfsetroundjoin%
\definecolor{currentfill}{rgb}{1.000000,0.498039,0.054902}%
\pgfsetfillcolor{currentfill}%
\pgfsetlinewidth{0.481800pt}%
\definecolor{currentstroke}{rgb}{1.000000,1.000000,1.000000}%
\pgfsetstrokecolor{currentstroke}%
\pgfsetdash{}{0pt}%
\pgfpathmoveto{\pgfqpoint{1.964385in}{6.248312in}}%
\pgfpathcurveto{\pgfqpoint{1.975436in}{6.248312in}}{\pgfqpoint{1.986035in}{6.252702in}}{\pgfqpoint{1.993848in}{6.260516in}}%
\pgfpathcurveto{\pgfqpoint{2.001662in}{6.268330in}}{\pgfqpoint{2.006052in}{6.278929in}}{\pgfqpoint{2.006052in}{6.289979in}}%
\pgfpathcurveto{\pgfqpoint{2.006052in}{6.301029in}}{\pgfqpoint{2.001662in}{6.311628in}}{\pgfqpoint{1.993848in}{6.319442in}}%
\pgfpathcurveto{\pgfqpoint{1.986035in}{6.327255in}}{\pgfqpoint{1.975436in}{6.331645in}}{\pgfqpoint{1.964385in}{6.331645in}}%
\pgfpathcurveto{\pgfqpoint{1.953335in}{6.331645in}}{\pgfqpoint{1.942736in}{6.327255in}}{\pgfqpoint{1.934923in}{6.319442in}}%
\pgfpathcurveto{\pgfqpoint{1.927109in}{6.311628in}}{\pgfqpoint{1.922719in}{6.301029in}}{\pgfqpoint{1.922719in}{6.289979in}}%
\pgfpathcurveto{\pgfqpoint{1.922719in}{6.278929in}}{\pgfqpoint{1.927109in}{6.268330in}}{\pgfqpoint{1.934923in}{6.260516in}}%
\pgfpathcurveto{\pgfqpoint{1.942736in}{6.252702in}}{\pgfqpoint{1.953335in}{6.248312in}}{\pgfqpoint{1.964385in}{6.248312in}}%
\pgfpathlineto{\pgfqpoint{1.964385in}{6.248312in}}%
\pgfpathclose%
\pgfusepath{stroke,fill}%
\end{pgfscope}%
\begin{pgfscope}%
\pgfpathrectangle{\pgfqpoint{0.633874in}{5.272501in}}{\pgfqpoint{2.177280in}{2.201755in}}%
\pgfusepath{clip}%
\pgfsetbuttcap%
\pgfsetroundjoin%
\definecolor{currentfill}{rgb}{1.000000,0.498039,0.054902}%
\pgfsetfillcolor{currentfill}%
\pgfsetlinewidth{0.481800pt}%
\definecolor{currentstroke}{rgb}{1.000000,1.000000,1.000000}%
\pgfsetstrokecolor{currentstroke}%
\pgfsetdash{}{0pt}%
\pgfpathmoveto{\pgfqpoint{1.405244in}{5.581114in}}%
\pgfpathcurveto{\pgfqpoint{1.416294in}{5.581114in}}{\pgfqpoint{1.426893in}{5.585504in}}{\pgfqpoint{1.434706in}{5.593317in}}%
\pgfpathcurveto{\pgfqpoint{1.442520in}{5.601131in}}{\pgfqpoint{1.446910in}{5.611730in}}{\pgfqpoint{1.446910in}{5.622780in}}%
\pgfpathcurveto{\pgfqpoint{1.446910in}{5.633830in}}{\pgfqpoint{1.442520in}{5.644429in}}{\pgfqpoint{1.434706in}{5.652243in}}%
\pgfpathcurveto{\pgfqpoint{1.426893in}{5.660057in}}{\pgfqpoint{1.416294in}{5.664447in}}{\pgfqpoint{1.405244in}{5.664447in}}%
\pgfpathcurveto{\pgfqpoint{1.394193in}{5.664447in}}{\pgfqpoint{1.383594in}{5.660057in}}{\pgfqpoint{1.375781in}{5.652243in}}%
\pgfpathcurveto{\pgfqpoint{1.367967in}{5.644429in}}{\pgfqpoint{1.363577in}{5.633830in}}{\pgfqpoint{1.363577in}{5.622780in}}%
\pgfpathcurveto{\pgfqpoint{1.363577in}{5.611730in}}{\pgfqpoint{1.367967in}{5.601131in}}{\pgfqpoint{1.375781in}{5.593317in}}%
\pgfpathcurveto{\pgfqpoint{1.383594in}{5.585504in}}{\pgfqpoint{1.394193in}{5.581114in}}{\pgfqpoint{1.405244in}{5.581114in}}%
\pgfpathlineto{\pgfqpoint{1.405244in}{5.581114in}}%
\pgfpathclose%
\pgfusepath{stroke,fill}%
\end{pgfscope}%
\begin{pgfscope}%
\pgfpathrectangle{\pgfqpoint{0.633874in}{5.272501in}}{\pgfqpoint{2.177280in}{2.201755in}}%
\pgfusepath{clip}%
\pgfsetbuttcap%
\pgfsetroundjoin%
\definecolor{currentfill}{rgb}{1.000000,0.498039,0.054902}%
\pgfsetfillcolor{currentfill}%
\pgfsetlinewidth{0.481800pt}%
\definecolor{currentstroke}{rgb}{1.000000,1.000000,1.000000}%
\pgfsetstrokecolor{currentstroke}%
\pgfsetdash{}{0pt}%
\pgfpathmoveto{\pgfqpoint{1.804631in}{5.998113in}}%
\pgfpathcurveto{\pgfqpoint{1.815681in}{5.998113in}}{\pgfqpoint{1.826280in}{6.002503in}}{\pgfqpoint{1.834093in}{6.010317in}}%
\pgfpathcurveto{\pgfqpoint{1.841907in}{6.018130in}}{\pgfqpoint{1.846297in}{6.028729in}}{\pgfqpoint{1.846297in}{6.039779in}}%
\pgfpathcurveto{\pgfqpoint{1.846297in}{6.050829in}}{\pgfqpoint{1.841907in}{6.061428in}}{\pgfqpoint{1.834093in}{6.069242in}}%
\pgfpathcurveto{\pgfqpoint{1.826280in}{6.077056in}}{\pgfqpoint{1.815681in}{6.081446in}}{\pgfqpoint{1.804631in}{6.081446in}}%
\pgfpathcurveto{\pgfqpoint{1.793581in}{6.081446in}}{\pgfqpoint{1.782981in}{6.077056in}}{\pgfqpoint{1.775168in}{6.069242in}}%
\pgfpathcurveto{\pgfqpoint{1.767354in}{6.061428in}}{\pgfqpoint{1.762964in}{6.050829in}}{\pgfqpoint{1.762964in}{6.039779in}}%
\pgfpathcurveto{\pgfqpoint{1.762964in}{6.028729in}}{\pgfqpoint{1.767354in}{6.018130in}}{\pgfqpoint{1.775168in}{6.010317in}}%
\pgfpathcurveto{\pgfqpoint{1.782981in}{6.002503in}}{\pgfqpoint{1.793581in}{5.998113in}}{\pgfqpoint{1.804631in}{5.998113in}}%
\pgfpathlineto{\pgfqpoint{1.804631in}{5.998113in}}%
\pgfpathclose%
\pgfusepath{stroke,fill}%
\end{pgfscope}%
\begin{pgfscope}%
\pgfpathrectangle{\pgfqpoint{0.633874in}{5.272501in}}{\pgfqpoint{2.177280in}{2.201755in}}%
\pgfusepath{clip}%
\pgfsetbuttcap%
\pgfsetroundjoin%
\definecolor{currentfill}{rgb}{1.000000,0.498039,0.054902}%
\pgfsetfillcolor{currentfill}%
\pgfsetlinewidth{0.481800pt}%
\definecolor{currentstroke}{rgb}{1.000000,1.000000,1.000000}%
\pgfsetstrokecolor{currentstroke}%
\pgfsetdash{}{0pt}%
\pgfpathmoveto{\pgfqpoint{1.485121in}{5.998113in}}%
\pgfpathcurveto{\pgfqpoint{1.496171in}{5.998113in}}{\pgfqpoint{1.506770in}{6.002503in}}{\pgfqpoint{1.514584in}{6.010317in}}%
\pgfpathcurveto{\pgfqpoint{1.522397in}{6.018130in}}{\pgfqpoint{1.526788in}{6.028729in}}{\pgfqpoint{1.526788in}{6.039779in}}%
\pgfpathcurveto{\pgfqpoint{1.526788in}{6.050829in}}{\pgfqpoint{1.522397in}{6.061428in}}{\pgfqpoint{1.514584in}{6.069242in}}%
\pgfpathcurveto{\pgfqpoint{1.506770in}{6.077056in}}{\pgfqpoint{1.496171in}{6.081446in}}{\pgfqpoint{1.485121in}{6.081446in}}%
\pgfpathcurveto{\pgfqpoint{1.474071in}{6.081446in}}{\pgfqpoint{1.463472in}{6.077056in}}{\pgfqpoint{1.455658in}{6.069242in}}%
\pgfpathcurveto{\pgfqpoint{1.447845in}{6.061428in}}{\pgfqpoint{1.443454in}{6.050829in}}{\pgfqpoint{1.443454in}{6.039779in}}%
\pgfpathcurveto{\pgfqpoint{1.443454in}{6.028729in}}{\pgfqpoint{1.447845in}{6.018130in}}{\pgfqpoint{1.455658in}{6.010317in}}%
\pgfpathcurveto{\pgfqpoint{1.463472in}{6.002503in}}{\pgfqpoint{1.474071in}{5.998113in}}{\pgfqpoint{1.485121in}{5.998113in}}%
\pgfpathlineto{\pgfqpoint{1.485121in}{5.998113in}}%
\pgfpathclose%
\pgfusepath{stroke,fill}%
\end{pgfscope}%
\begin{pgfscope}%
\pgfpathrectangle{\pgfqpoint{0.633874in}{5.272501in}}{\pgfqpoint{2.177280in}{2.201755in}}%
\pgfusepath{clip}%
\pgfsetbuttcap%
\pgfsetroundjoin%
\definecolor{currentfill}{rgb}{1.000000,0.498039,0.054902}%
\pgfsetfillcolor{currentfill}%
\pgfsetlinewidth{0.481800pt}%
\definecolor{currentstroke}{rgb}{1.000000,1.000000,1.000000}%
\pgfsetstrokecolor{currentstroke}%
\pgfsetdash{}{0pt}%
\pgfpathmoveto{\pgfqpoint{1.724753in}{6.415112in}}%
\pgfpathcurveto{\pgfqpoint{1.735803in}{6.415112in}}{\pgfqpoint{1.746402in}{6.419502in}}{\pgfqpoint{1.754216in}{6.427316in}}%
\pgfpathcurveto{\pgfqpoint{1.762030in}{6.435129in}}{\pgfqpoint{1.766420in}{6.445728in}}{\pgfqpoint{1.766420in}{6.456778in}}%
\pgfpathcurveto{\pgfqpoint{1.766420in}{6.467828in}}{\pgfqpoint{1.762030in}{6.478428in}}{\pgfqpoint{1.754216in}{6.486241in}}%
\pgfpathcurveto{\pgfqpoint{1.746402in}{6.494055in}}{\pgfqpoint{1.735803in}{6.498445in}}{\pgfqpoint{1.724753in}{6.498445in}}%
\pgfpathcurveto{\pgfqpoint{1.713703in}{6.498445in}}{\pgfqpoint{1.703104in}{6.494055in}}{\pgfqpoint{1.695290in}{6.486241in}}%
\pgfpathcurveto{\pgfqpoint{1.687477in}{6.478428in}}{\pgfqpoint{1.683087in}{6.467828in}}{\pgfqpoint{1.683087in}{6.456778in}}%
\pgfpathcurveto{\pgfqpoint{1.683087in}{6.445728in}}{\pgfqpoint{1.687477in}{6.435129in}}{\pgfqpoint{1.695290in}{6.427316in}}%
\pgfpathcurveto{\pgfqpoint{1.703104in}{6.419502in}}{\pgfqpoint{1.713703in}{6.415112in}}{\pgfqpoint{1.724753in}{6.415112in}}%
\pgfpathlineto{\pgfqpoint{1.724753in}{6.415112in}}%
\pgfpathclose%
\pgfusepath{stroke,fill}%
\end{pgfscope}%
\begin{pgfscope}%
\pgfpathrectangle{\pgfqpoint{0.633874in}{5.272501in}}{\pgfqpoint{2.177280in}{2.201755in}}%
\pgfusepath{clip}%
\pgfsetbuttcap%
\pgfsetroundjoin%
\definecolor{currentfill}{rgb}{1.000000,0.498039,0.054902}%
\pgfsetfillcolor{currentfill}%
\pgfsetlinewidth{0.481800pt}%
\definecolor{currentstroke}{rgb}{1.000000,1.000000,1.000000}%
\pgfsetstrokecolor{currentstroke}%
\pgfsetdash{}{0pt}%
\pgfpathmoveto{\pgfqpoint{1.165611in}{5.664513in}}%
\pgfpathcurveto{\pgfqpoint{1.176662in}{5.664513in}}{\pgfqpoint{1.187261in}{5.668904in}}{\pgfqpoint{1.195074in}{5.676717in}}%
\pgfpathcurveto{\pgfqpoint{1.202888in}{5.684531in}}{\pgfqpoint{1.207278in}{5.695130in}}{\pgfqpoint{1.207278in}{5.706180in}}%
\pgfpathcurveto{\pgfqpoint{1.207278in}{5.717230in}}{\pgfqpoint{1.202888in}{5.727829in}}{\pgfqpoint{1.195074in}{5.735643in}}%
\pgfpathcurveto{\pgfqpoint{1.187261in}{5.743456in}}{\pgfqpoint{1.176662in}{5.747847in}}{\pgfqpoint{1.165611in}{5.747847in}}%
\pgfpathcurveto{\pgfqpoint{1.154561in}{5.747847in}}{\pgfqpoint{1.143962in}{5.743456in}}{\pgfqpoint{1.136149in}{5.735643in}}%
\pgfpathcurveto{\pgfqpoint{1.128335in}{5.727829in}}{\pgfqpoint{1.123945in}{5.717230in}}{\pgfqpoint{1.123945in}{5.706180in}}%
\pgfpathcurveto{\pgfqpoint{1.123945in}{5.695130in}}{\pgfqpoint{1.128335in}{5.684531in}}{\pgfqpoint{1.136149in}{5.676717in}}%
\pgfpathcurveto{\pgfqpoint{1.143962in}{5.668904in}}{\pgfqpoint{1.154561in}{5.664513in}}{\pgfqpoint{1.165611in}{5.664513in}}%
\pgfpathlineto{\pgfqpoint{1.165611in}{5.664513in}}%
\pgfpathclose%
\pgfusepath{stroke,fill}%
\end{pgfscope}%
\begin{pgfscope}%
\pgfpathrectangle{\pgfqpoint{0.633874in}{5.272501in}}{\pgfqpoint{2.177280in}{2.201755in}}%
\pgfusepath{clip}%
\pgfsetbuttcap%
\pgfsetroundjoin%
\definecolor{currentfill}{rgb}{1.000000,0.498039,0.054902}%
\pgfsetfillcolor{currentfill}%
\pgfsetlinewidth{0.481800pt}%
\definecolor{currentstroke}{rgb}{1.000000,1.000000,1.000000}%
\pgfsetstrokecolor{currentstroke}%
\pgfsetdash{}{0pt}%
\pgfpathmoveto{\pgfqpoint{1.844569in}{6.081512in}}%
\pgfpathcurveto{\pgfqpoint{1.855619in}{6.081512in}}{\pgfqpoint{1.866219in}{6.085903in}}{\pgfqpoint{1.874032in}{6.093716in}}%
\pgfpathcurveto{\pgfqpoint{1.881846in}{6.101530in}}{\pgfqpoint{1.886236in}{6.112129in}}{\pgfqpoint{1.886236in}{6.123179in}}%
\pgfpathcurveto{\pgfqpoint{1.886236in}{6.134229in}}{\pgfqpoint{1.881846in}{6.144828in}}{\pgfqpoint{1.874032in}{6.152642in}}%
\pgfpathcurveto{\pgfqpoint{1.866219in}{6.160456in}}{\pgfqpoint{1.855619in}{6.164846in}}{\pgfqpoint{1.844569in}{6.164846in}}%
\pgfpathcurveto{\pgfqpoint{1.833519in}{6.164846in}}{\pgfqpoint{1.822920in}{6.160456in}}{\pgfqpoint{1.815107in}{6.152642in}}%
\pgfpathcurveto{\pgfqpoint{1.807293in}{6.144828in}}{\pgfqpoint{1.802903in}{6.134229in}}{\pgfqpoint{1.802903in}{6.123179in}}%
\pgfpathcurveto{\pgfqpoint{1.802903in}{6.112129in}}{\pgfqpoint{1.807293in}{6.101530in}}{\pgfqpoint{1.815107in}{6.093716in}}%
\pgfpathcurveto{\pgfqpoint{1.822920in}{6.085903in}}{\pgfqpoint{1.833519in}{6.081512in}}{\pgfqpoint{1.844569in}{6.081512in}}%
\pgfpathlineto{\pgfqpoint{1.844569in}{6.081512in}}%
\pgfpathclose%
\pgfusepath{stroke,fill}%
\end{pgfscope}%
\begin{pgfscope}%
\pgfpathrectangle{\pgfqpoint{0.633874in}{5.272501in}}{\pgfqpoint{2.177280in}{2.201755in}}%
\pgfusepath{clip}%
\pgfsetbuttcap%
\pgfsetroundjoin%
\definecolor{currentfill}{rgb}{1.000000,0.498039,0.054902}%
\pgfsetfillcolor{currentfill}%
\pgfsetlinewidth{0.481800pt}%
\definecolor{currentstroke}{rgb}{1.000000,1.000000,1.000000}%
\pgfsetstrokecolor{currentstroke}%
\pgfsetdash{}{0pt}%
\pgfpathmoveto{\pgfqpoint{1.285428in}{5.914713in}}%
\pgfpathcurveto{\pgfqpoint{1.296478in}{5.914713in}}{\pgfqpoint{1.307077in}{5.919103in}}{\pgfqpoint{1.314890in}{5.926917in}}%
\pgfpathcurveto{\pgfqpoint{1.322704in}{5.934730in}}{\pgfqpoint{1.327094in}{5.945329in}}{\pgfqpoint{1.327094in}{5.956379in}}%
\pgfpathcurveto{\pgfqpoint{1.327094in}{5.967430in}}{\pgfqpoint{1.322704in}{5.978029in}}{\pgfqpoint{1.314890in}{5.985842in}}%
\pgfpathcurveto{\pgfqpoint{1.307077in}{5.993656in}}{\pgfqpoint{1.296478in}{5.998046in}}{\pgfqpoint{1.285428in}{5.998046in}}%
\pgfpathcurveto{\pgfqpoint{1.274377in}{5.998046in}}{\pgfqpoint{1.263778in}{5.993656in}}{\pgfqpoint{1.255965in}{5.985842in}}%
\pgfpathcurveto{\pgfqpoint{1.248151in}{5.978029in}}{\pgfqpoint{1.243761in}{5.967430in}}{\pgfqpoint{1.243761in}{5.956379in}}%
\pgfpathcurveto{\pgfqpoint{1.243761in}{5.945329in}}{\pgfqpoint{1.248151in}{5.934730in}}{\pgfqpoint{1.255965in}{5.926917in}}%
\pgfpathcurveto{\pgfqpoint{1.263778in}{5.919103in}}{\pgfqpoint{1.274377in}{5.914713in}}{\pgfqpoint{1.285428in}{5.914713in}}%
\pgfpathlineto{\pgfqpoint{1.285428in}{5.914713in}}%
\pgfpathclose%
\pgfusepath{stroke,fill}%
\end{pgfscope}%
\begin{pgfscope}%
\pgfpathrectangle{\pgfqpoint{0.633874in}{5.272501in}}{\pgfqpoint{2.177280in}{2.201755in}}%
\pgfusepath{clip}%
\pgfsetbuttcap%
\pgfsetroundjoin%
\definecolor{currentfill}{rgb}{1.000000,0.498039,0.054902}%
\pgfsetfillcolor{currentfill}%
\pgfsetlinewidth{0.481800pt}%
\definecolor{currentstroke}{rgb}{1.000000,1.000000,1.000000}%
\pgfsetstrokecolor{currentstroke}%
\pgfsetdash{}{0pt}%
\pgfpathmoveto{\pgfqpoint{1.205550in}{5.330914in}}%
\pgfpathcurveto{\pgfqpoint{1.216600in}{5.330914in}}{\pgfqpoint{1.227199in}{5.335304in}}{\pgfqpoint{1.235013in}{5.343118in}}%
\pgfpathcurveto{\pgfqpoint{1.242826in}{5.350932in}}{\pgfqpoint{1.247217in}{5.361531in}}{\pgfqpoint{1.247217in}{5.372581in}}%
\pgfpathcurveto{\pgfqpoint{1.247217in}{5.383631in}}{\pgfqpoint{1.242826in}{5.394230in}}{\pgfqpoint{1.235013in}{5.402044in}}%
\pgfpathcurveto{\pgfqpoint{1.227199in}{5.409857in}}{\pgfqpoint{1.216600in}{5.414247in}}{\pgfqpoint{1.205550in}{5.414247in}}%
\pgfpathcurveto{\pgfqpoint{1.194500in}{5.414247in}}{\pgfqpoint{1.183901in}{5.409857in}}{\pgfqpoint{1.176087in}{5.402044in}}%
\pgfpathcurveto{\pgfqpoint{1.168274in}{5.394230in}}{\pgfqpoint{1.163883in}{5.383631in}}{\pgfqpoint{1.163883in}{5.372581in}}%
\pgfpathcurveto{\pgfqpoint{1.163883in}{5.361531in}}{\pgfqpoint{1.168274in}{5.350932in}}{\pgfqpoint{1.176087in}{5.343118in}}%
\pgfpathcurveto{\pgfqpoint{1.183901in}{5.335304in}}{\pgfqpoint{1.194500in}{5.330914in}}{\pgfqpoint{1.205550in}{5.330914in}}%
\pgfpathlineto{\pgfqpoint{1.205550in}{5.330914in}}%
\pgfpathclose%
\pgfusepath{stroke,fill}%
\end{pgfscope}%
\begin{pgfscope}%
\pgfpathrectangle{\pgfqpoint{0.633874in}{5.272501in}}{\pgfqpoint{2.177280in}{2.201755in}}%
\pgfusepath{clip}%
\pgfsetbuttcap%
\pgfsetroundjoin%
\definecolor{currentfill}{rgb}{1.000000,0.498039,0.054902}%
\pgfsetfillcolor{currentfill}%
\pgfsetlinewidth{0.481800pt}%
\definecolor{currentstroke}{rgb}{1.000000,1.000000,1.000000}%
\pgfsetstrokecolor{currentstroke}%
\pgfsetdash{}{0pt}%
\pgfpathmoveto{\pgfqpoint{1.564998in}{6.164912in}}%
\pgfpathcurveto{\pgfqpoint{1.576049in}{6.164912in}}{\pgfqpoint{1.586648in}{6.169303in}}{\pgfqpoint{1.594461in}{6.177116in}}%
\pgfpathcurveto{\pgfqpoint{1.602275in}{6.184930in}}{\pgfqpoint{1.606665in}{6.195529in}}{\pgfqpoint{1.606665in}{6.206579in}}%
\pgfpathcurveto{\pgfqpoint{1.606665in}{6.217629in}}{\pgfqpoint{1.602275in}{6.228228in}}{\pgfqpoint{1.594461in}{6.236042in}}%
\pgfpathcurveto{\pgfqpoint{1.586648in}{6.243855in}}{\pgfqpoint{1.576049in}{6.248246in}}{\pgfqpoint{1.564998in}{6.248246in}}%
\pgfpathcurveto{\pgfqpoint{1.553948in}{6.248246in}}{\pgfqpoint{1.543349in}{6.243855in}}{\pgfqpoint{1.535536in}{6.236042in}}%
\pgfpathcurveto{\pgfqpoint{1.527722in}{6.228228in}}{\pgfqpoint{1.523332in}{6.217629in}}{\pgfqpoint{1.523332in}{6.206579in}}%
\pgfpathcurveto{\pgfqpoint{1.523332in}{6.195529in}}{\pgfqpoint{1.527722in}{6.184930in}}{\pgfqpoint{1.535536in}{6.177116in}}%
\pgfpathcurveto{\pgfqpoint{1.543349in}{6.169303in}}{\pgfqpoint{1.553948in}{6.164912in}}{\pgfqpoint{1.564998in}{6.164912in}}%
\pgfpathlineto{\pgfqpoint{1.564998in}{6.164912in}}%
\pgfpathclose%
\pgfusepath{stroke,fill}%
\end{pgfscope}%
\begin{pgfscope}%
\pgfpathrectangle{\pgfqpoint{0.633874in}{5.272501in}}{\pgfqpoint{2.177280in}{2.201755in}}%
\pgfusepath{clip}%
\pgfsetbuttcap%
\pgfsetroundjoin%
\definecolor{currentfill}{rgb}{1.000000,0.498039,0.054902}%
\pgfsetfillcolor{currentfill}%
\pgfsetlinewidth{0.481800pt}%
\definecolor{currentstroke}{rgb}{1.000000,1.000000,1.000000}%
\pgfsetstrokecolor{currentstroke}%
\pgfsetdash{}{0pt}%
\pgfpathmoveto{\pgfqpoint{1.604937in}{5.497714in}}%
\pgfpathcurveto{\pgfqpoint{1.615987in}{5.497714in}}{\pgfqpoint{1.626586in}{5.502104in}}{\pgfqpoint{1.634400in}{5.509918in}}%
\pgfpathcurveto{\pgfqpoint{1.642214in}{5.517731in}}{\pgfqpoint{1.646604in}{5.528330in}}{\pgfqpoint{1.646604in}{5.539380in}}%
\pgfpathcurveto{\pgfqpoint{1.646604in}{5.550431in}}{\pgfqpoint{1.642214in}{5.561030in}}{\pgfqpoint{1.634400in}{5.568843in}}%
\pgfpathcurveto{\pgfqpoint{1.626586in}{5.576657in}}{\pgfqpoint{1.615987in}{5.581047in}}{\pgfqpoint{1.604937in}{5.581047in}}%
\pgfpathcurveto{\pgfqpoint{1.593887in}{5.581047in}}{\pgfqpoint{1.583288in}{5.576657in}}{\pgfqpoint{1.575474in}{5.568843in}}%
\pgfpathcurveto{\pgfqpoint{1.567661in}{5.561030in}}{\pgfqpoint{1.563270in}{5.550431in}}{\pgfqpoint{1.563270in}{5.539380in}}%
\pgfpathcurveto{\pgfqpoint{1.563270in}{5.528330in}}{\pgfqpoint{1.567661in}{5.517731in}}{\pgfqpoint{1.575474in}{5.509918in}}%
\pgfpathcurveto{\pgfqpoint{1.583288in}{5.502104in}}{\pgfqpoint{1.593887in}{5.497714in}}{\pgfqpoint{1.604937in}{5.497714in}}%
\pgfpathlineto{\pgfqpoint{1.604937in}{5.497714in}}%
\pgfpathclose%
\pgfusepath{stroke,fill}%
\end{pgfscope}%
\begin{pgfscope}%
\pgfpathrectangle{\pgfqpoint{0.633874in}{5.272501in}}{\pgfqpoint{2.177280in}{2.201755in}}%
\pgfusepath{clip}%
\pgfsetbuttcap%
\pgfsetroundjoin%
\definecolor{currentfill}{rgb}{1.000000,0.498039,0.054902}%
\pgfsetfillcolor{currentfill}%
\pgfsetlinewidth{0.481800pt}%
\definecolor{currentstroke}{rgb}{1.000000,1.000000,1.000000}%
\pgfsetstrokecolor{currentstroke}%
\pgfsetdash{}{0pt}%
\pgfpathmoveto{\pgfqpoint{1.644876in}{6.081512in}}%
\pgfpathcurveto{\pgfqpoint{1.655926in}{6.081512in}}{\pgfqpoint{1.666525in}{6.085903in}}{\pgfqpoint{1.674339in}{6.093716in}}%
\pgfpathcurveto{\pgfqpoint{1.682152in}{6.101530in}}{\pgfqpoint{1.686543in}{6.112129in}}{\pgfqpoint{1.686543in}{6.123179in}}%
\pgfpathcurveto{\pgfqpoint{1.686543in}{6.134229in}}{\pgfqpoint{1.682152in}{6.144828in}}{\pgfqpoint{1.674339in}{6.152642in}}%
\pgfpathcurveto{\pgfqpoint{1.666525in}{6.160456in}}{\pgfqpoint{1.655926in}{6.164846in}}{\pgfqpoint{1.644876in}{6.164846in}}%
\pgfpathcurveto{\pgfqpoint{1.633826in}{6.164846in}}{\pgfqpoint{1.623227in}{6.160456in}}{\pgfqpoint{1.615413in}{6.152642in}}%
\pgfpathcurveto{\pgfqpoint{1.607599in}{6.144828in}}{\pgfqpoint{1.603209in}{6.134229in}}{\pgfqpoint{1.603209in}{6.123179in}}%
\pgfpathcurveto{\pgfqpoint{1.603209in}{6.112129in}}{\pgfqpoint{1.607599in}{6.101530in}}{\pgfqpoint{1.615413in}{6.093716in}}%
\pgfpathcurveto{\pgfqpoint{1.623227in}{6.085903in}}{\pgfqpoint{1.633826in}{6.081512in}}{\pgfqpoint{1.644876in}{6.081512in}}%
\pgfpathlineto{\pgfqpoint{1.644876in}{6.081512in}}%
\pgfpathclose%
\pgfusepath{stroke,fill}%
\end{pgfscope}%
\begin{pgfscope}%
\pgfpathrectangle{\pgfqpoint{0.633874in}{5.272501in}}{\pgfqpoint{2.177280in}{2.201755in}}%
\pgfusepath{clip}%
\pgfsetbuttcap%
\pgfsetroundjoin%
\definecolor{currentfill}{rgb}{1.000000,0.498039,0.054902}%
\pgfsetfillcolor{currentfill}%
\pgfsetlinewidth{0.481800pt}%
\definecolor{currentstroke}{rgb}{1.000000,1.000000,1.000000}%
\pgfsetstrokecolor{currentstroke}%
\pgfsetdash{}{0pt}%
\pgfpathmoveto{\pgfqpoint{1.445182in}{6.081512in}}%
\pgfpathcurveto{\pgfqpoint{1.456232in}{6.081512in}}{\pgfqpoint{1.466831in}{6.085903in}}{\pgfqpoint{1.474645in}{6.093716in}}%
\pgfpathcurveto{\pgfqpoint{1.482459in}{6.101530in}}{\pgfqpoint{1.486849in}{6.112129in}}{\pgfqpoint{1.486849in}{6.123179in}}%
\pgfpathcurveto{\pgfqpoint{1.486849in}{6.134229in}}{\pgfqpoint{1.482459in}{6.144828in}}{\pgfqpoint{1.474645in}{6.152642in}}%
\pgfpathcurveto{\pgfqpoint{1.466831in}{6.160456in}}{\pgfqpoint{1.456232in}{6.164846in}}{\pgfqpoint{1.445182in}{6.164846in}}%
\pgfpathcurveto{\pgfqpoint{1.434132in}{6.164846in}}{\pgfqpoint{1.423533in}{6.160456in}}{\pgfqpoint{1.415720in}{6.152642in}}%
\pgfpathcurveto{\pgfqpoint{1.407906in}{6.144828in}}{\pgfqpoint{1.403516in}{6.134229in}}{\pgfqpoint{1.403516in}{6.123179in}}%
\pgfpathcurveto{\pgfqpoint{1.403516in}{6.112129in}}{\pgfqpoint{1.407906in}{6.101530in}}{\pgfqpoint{1.415720in}{6.093716in}}%
\pgfpathcurveto{\pgfqpoint{1.423533in}{6.085903in}}{\pgfqpoint{1.434132in}{6.081512in}}{\pgfqpoint{1.445182in}{6.081512in}}%
\pgfpathlineto{\pgfqpoint{1.445182in}{6.081512in}}%
\pgfpathclose%
\pgfusepath{stroke,fill}%
\end{pgfscope}%
\begin{pgfscope}%
\pgfpathrectangle{\pgfqpoint{0.633874in}{5.272501in}}{\pgfqpoint{2.177280in}{2.201755in}}%
\pgfusepath{clip}%
\pgfsetbuttcap%
\pgfsetroundjoin%
\definecolor{currentfill}{rgb}{1.000000,0.498039,0.054902}%
\pgfsetfillcolor{currentfill}%
\pgfsetlinewidth{0.481800pt}%
\definecolor{currentstroke}{rgb}{1.000000,1.000000,1.000000}%
\pgfsetstrokecolor{currentstroke}%
\pgfsetdash{}{0pt}%
\pgfpathmoveto{\pgfqpoint{1.884508in}{6.248312in}}%
\pgfpathcurveto{\pgfqpoint{1.895558in}{6.248312in}}{\pgfqpoint{1.906157in}{6.252702in}}{\pgfqpoint{1.913971in}{6.260516in}}%
\pgfpathcurveto{\pgfqpoint{1.921784in}{6.268330in}}{\pgfqpoint{1.926175in}{6.278929in}}{\pgfqpoint{1.926175in}{6.289979in}}%
\pgfpathcurveto{\pgfqpoint{1.926175in}{6.301029in}}{\pgfqpoint{1.921784in}{6.311628in}}{\pgfqpoint{1.913971in}{6.319442in}}%
\pgfpathcurveto{\pgfqpoint{1.906157in}{6.327255in}}{\pgfqpoint{1.895558in}{6.331645in}}{\pgfqpoint{1.884508in}{6.331645in}}%
\pgfpathcurveto{\pgfqpoint{1.873458in}{6.331645in}}{\pgfqpoint{1.862859in}{6.327255in}}{\pgfqpoint{1.855045in}{6.319442in}}%
\pgfpathcurveto{\pgfqpoint{1.847232in}{6.311628in}}{\pgfqpoint{1.842841in}{6.301029in}}{\pgfqpoint{1.842841in}{6.289979in}}%
\pgfpathcurveto{\pgfqpoint{1.842841in}{6.278929in}}{\pgfqpoint{1.847232in}{6.268330in}}{\pgfqpoint{1.855045in}{6.260516in}}%
\pgfpathcurveto{\pgfqpoint{1.862859in}{6.252702in}}{\pgfqpoint{1.873458in}{6.248312in}}{\pgfqpoint{1.884508in}{6.248312in}}%
\pgfpathlineto{\pgfqpoint{1.884508in}{6.248312in}}%
\pgfpathclose%
\pgfusepath{stroke,fill}%
\end{pgfscope}%
\begin{pgfscope}%
\pgfpathrectangle{\pgfqpoint{0.633874in}{5.272501in}}{\pgfqpoint{2.177280in}{2.201755in}}%
\pgfusepath{clip}%
\pgfsetbuttcap%
\pgfsetroundjoin%
\definecolor{currentfill}{rgb}{1.000000,0.498039,0.054902}%
\pgfsetfillcolor{currentfill}%
\pgfsetlinewidth{0.481800pt}%
\definecolor{currentstroke}{rgb}{1.000000,1.000000,1.000000}%
\pgfsetstrokecolor{currentstroke}%
\pgfsetdash{}{0pt}%
\pgfpathmoveto{\pgfqpoint{1.445182in}{6.164912in}}%
\pgfpathcurveto{\pgfqpoint{1.456232in}{6.164912in}}{\pgfqpoint{1.466831in}{6.169303in}}{\pgfqpoint{1.474645in}{6.177116in}}%
\pgfpathcurveto{\pgfqpoint{1.482459in}{6.184930in}}{\pgfqpoint{1.486849in}{6.195529in}}{\pgfqpoint{1.486849in}{6.206579in}}%
\pgfpathcurveto{\pgfqpoint{1.486849in}{6.217629in}}{\pgfqpoint{1.482459in}{6.228228in}}{\pgfqpoint{1.474645in}{6.236042in}}%
\pgfpathcurveto{\pgfqpoint{1.466831in}{6.243855in}}{\pgfqpoint{1.456232in}{6.248246in}}{\pgfqpoint{1.445182in}{6.248246in}}%
\pgfpathcurveto{\pgfqpoint{1.434132in}{6.248246in}}{\pgfqpoint{1.423533in}{6.243855in}}{\pgfqpoint{1.415720in}{6.236042in}}%
\pgfpathcurveto{\pgfqpoint{1.407906in}{6.228228in}}{\pgfqpoint{1.403516in}{6.217629in}}{\pgfqpoint{1.403516in}{6.206579in}}%
\pgfpathcurveto{\pgfqpoint{1.403516in}{6.195529in}}{\pgfqpoint{1.407906in}{6.184930in}}{\pgfqpoint{1.415720in}{6.177116in}}%
\pgfpathcurveto{\pgfqpoint{1.423533in}{6.169303in}}{\pgfqpoint{1.434132in}{6.164912in}}{\pgfqpoint{1.445182in}{6.164912in}}%
\pgfpathlineto{\pgfqpoint{1.445182in}{6.164912in}}%
\pgfpathclose%
\pgfusepath{stroke,fill}%
\end{pgfscope}%
\begin{pgfscope}%
\pgfpathrectangle{\pgfqpoint{0.633874in}{5.272501in}}{\pgfqpoint{2.177280in}{2.201755in}}%
\pgfusepath{clip}%
\pgfsetbuttcap%
\pgfsetroundjoin%
\definecolor{currentfill}{rgb}{1.000000,0.498039,0.054902}%
\pgfsetfillcolor{currentfill}%
\pgfsetlinewidth{0.481800pt}%
\definecolor{currentstroke}{rgb}{1.000000,1.000000,1.000000}%
\pgfsetstrokecolor{currentstroke}%
\pgfsetdash{}{0pt}%
\pgfpathmoveto{\pgfqpoint{1.525060in}{5.914713in}}%
\pgfpathcurveto{\pgfqpoint{1.536110in}{5.914713in}}{\pgfqpoint{1.546709in}{5.919103in}}{\pgfqpoint{1.554523in}{5.926917in}}%
\pgfpathcurveto{\pgfqpoint{1.562336in}{5.934730in}}{\pgfqpoint{1.566726in}{5.945329in}}{\pgfqpoint{1.566726in}{5.956379in}}%
\pgfpathcurveto{\pgfqpoint{1.566726in}{5.967430in}}{\pgfqpoint{1.562336in}{5.978029in}}{\pgfqpoint{1.554523in}{5.985842in}}%
\pgfpathcurveto{\pgfqpoint{1.546709in}{5.993656in}}{\pgfqpoint{1.536110in}{5.998046in}}{\pgfqpoint{1.525060in}{5.998046in}}%
\pgfpathcurveto{\pgfqpoint{1.514010in}{5.998046in}}{\pgfqpoint{1.503411in}{5.993656in}}{\pgfqpoint{1.495597in}{5.985842in}}%
\pgfpathcurveto{\pgfqpoint{1.487783in}{5.978029in}}{\pgfqpoint{1.483393in}{5.967430in}}{\pgfqpoint{1.483393in}{5.956379in}}%
\pgfpathcurveto{\pgfqpoint{1.483393in}{5.945329in}}{\pgfqpoint{1.487783in}{5.934730in}}{\pgfqpoint{1.495597in}{5.926917in}}%
\pgfpathcurveto{\pgfqpoint{1.503411in}{5.919103in}}{\pgfqpoint{1.514010in}{5.914713in}}{\pgfqpoint{1.525060in}{5.914713in}}%
\pgfpathlineto{\pgfqpoint{1.525060in}{5.914713in}}%
\pgfpathclose%
\pgfusepath{stroke,fill}%
\end{pgfscope}%
\begin{pgfscope}%
\pgfpathrectangle{\pgfqpoint{0.633874in}{5.272501in}}{\pgfqpoint{2.177280in}{2.201755in}}%
\pgfusepath{clip}%
\pgfsetbuttcap%
\pgfsetroundjoin%
\definecolor{currentfill}{rgb}{1.000000,0.498039,0.054902}%
\pgfsetfillcolor{currentfill}%
\pgfsetlinewidth{0.481800pt}%
\definecolor{currentstroke}{rgb}{1.000000,1.000000,1.000000}%
\pgfsetstrokecolor{currentstroke}%
\pgfsetdash{}{0pt}%
\pgfpathmoveto{\pgfqpoint{1.684815in}{5.497714in}}%
\pgfpathcurveto{\pgfqpoint{1.695865in}{5.497714in}}{\pgfqpoint{1.706464in}{5.502104in}}{\pgfqpoint{1.714277in}{5.509918in}}%
\pgfpathcurveto{\pgfqpoint{1.722091in}{5.517731in}}{\pgfqpoint{1.726481in}{5.528330in}}{\pgfqpoint{1.726481in}{5.539380in}}%
\pgfpathcurveto{\pgfqpoint{1.726481in}{5.550431in}}{\pgfqpoint{1.722091in}{5.561030in}}{\pgfqpoint{1.714277in}{5.568843in}}%
\pgfpathcurveto{\pgfqpoint{1.706464in}{5.576657in}}{\pgfqpoint{1.695865in}{5.581047in}}{\pgfqpoint{1.684815in}{5.581047in}}%
\pgfpathcurveto{\pgfqpoint{1.673764in}{5.581047in}}{\pgfqpoint{1.663165in}{5.576657in}}{\pgfqpoint{1.655352in}{5.568843in}}%
\pgfpathcurveto{\pgfqpoint{1.647538in}{5.561030in}}{\pgfqpoint{1.643148in}{5.550431in}}{\pgfqpoint{1.643148in}{5.539380in}}%
\pgfpathcurveto{\pgfqpoint{1.643148in}{5.528330in}}{\pgfqpoint{1.647538in}{5.517731in}}{\pgfqpoint{1.655352in}{5.509918in}}%
\pgfpathcurveto{\pgfqpoint{1.663165in}{5.502104in}}{\pgfqpoint{1.673764in}{5.497714in}}{\pgfqpoint{1.684815in}{5.497714in}}%
\pgfpathlineto{\pgfqpoint{1.684815in}{5.497714in}}%
\pgfpathclose%
\pgfusepath{stroke,fill}%
\end{pgfscope}%
\begin{pgfscope}%
\pgfpathrectangle{\pgfqpoint{0.633874in}{5.272501in}}{\pgfqpoint{2.177280in}{2.201755in}}%
\pgfusepath{clip}%
\pgfsetbuttcap%
\pgfsetroundjoin%
\definecolor{currentfill}{rgb}{1.000000,0.498039,0.054902}%
\pgfsetfillcolor{currentfill}%
\pgfsetlinewidth{0.481800pt}%
\definecolor{currentstroke}{rgb}{1.000000,1.000000,1.000000}%
\pgfsetstrokecolor{currentstroke}%
\pgfsetdash{}{0pt}%
\pgfpathmoveto{\pgfqpoint{1.445182in}{5.747913in}}%
\pgfpathcurveto{\pgfqpoint{1.456232in}{5.747913in}}{\pgfqpoint{1.466831in}{5.752303in}}{\pgfqpoint{1.474645in}{5.760117in}}%
\pgfpathcurveto{\pgfqpoint{1.482459in}{5.767931in}}{\pgfqpoint{1.486849in}{5.778530in}}{\pgfqpoint{1.486849in}{5.789580in}}%
\pgfpathcurveto{\pgfqpoint{1.486849in}{5.800630in}}{\pgfqpoint{1.482459in}{5.811229in}}{\pgfqpoint{1.474645in}{5.819043in}}%
\pgfpathcurveto{\pgfqpoint{1.466831in}{5.826856in}}{\pgfqpoint{1.456232in}{5.831247in}}{\pgfqpoint{1.445182in}{5.831247in}}%
\pgfpathcurveto{\pgfqpoint{1.434132in}{5.831247in}}{\pgfqpoint{1.423533in}{5.826856in}}{\pgfqpoint{1.415720in}{5.819043in}}%
\pgfpathcurveto{\pgfqpoint{1.407906in}{5.811229in}}{\pgfqpoint{1.403516in}{5.800630in}}{\pgfqpoint{1.403516in}{5.789580in}}%
\pgfpathcurveto{\pgfqpoint{1.403516in}{5.778530in}}{\pgfqpoint{1.407906in}{5.767931in}}{\pgfqpoint{1.415720in}{5.760117in}}%
\pgfpathcurveto{\pgfqpoint{1.423533in}{5.752303in}}{\pgfqpoint{1.434132in}{5.747913in}}{\pgfqpoint{1.445182in}{5.747913in}}%
\pgfpathlineto{\pgfqpoint{1.445182in}{5.747913in}}%
\pgfpathclose%
\pgfusepath{stroke,fill}%
\end{pgfscope}%
\begin{pgfscope}%
\pgfpathrectangle{\pgfqpoint{0.633874in}{5.272501in}}{\pgfqpoint{2.177280in}{2.201755in}}%
\pgfusepath{clip}%
\pgfsetbuttcap%
\pgfsetroundjoin%
\definecolor{currentfill}{rgb}{1.000000,0.498039,0.054902}%
\pgfsetfillcolor{currentfill}%
\pgfsetlinewidth{0.481800pt}%
\definecolor{currentstroke}{rgb}{1.000000,1.000000,1.000000}%
\pgfsetstrokecolor{currentstroke}%
\pgfsetdash{}{0pt}%
\pgfpathmoveto{\pgfqpoint{1.564998in}{6.331712in}}%
\pgfpathcurveto{\pgfqpoint{1.576049in}{6.331712in}}{\pgfqpoint{1.586648in}{6.336102in}}{\pgfqpoint{1.594461in}{6.343916in}}%
\pgfpathcurveto{\pgfqpoint{1.602275in}{6.351729in}}{\pgfqpoint{1.606665in}{6.362328in}}{\pgfqpoint{1.606665in}{6.373379in}}%
\pgfpathcurveto{\pgfqpoint{1.606665in}{6.384429in}}{\pgfqpoint{1.602275in}{6.395028in}}{\pgfqpoint{1.594461in}{6.402841in}}%
\pgfpathcurveto{\pgfqpoint{1.586648in}{6.410655in}}{\pgfqpoint{1.576049in}{6.415045in}}{\pgfqpoint{1.564998in}{6.415045in}}%
\pgfpathcurveto{\pgfqpoint{1.553948in}{6.415045in}}{\pgfqpoint{1.543349in}{6.410655in}}{\pgfqpoint{1.535536in}{6.402841in}}%
\pgfpathcurveto{\pgfqpoint{1.527722in}{6.395028in}}{\pgfqpoint{1.523332in}{6.384429in}}{\pgfqpoint{1.523332in}{6.373379in}}%
\pgfpathcurveto{\pgfqpoint{1.523332in}{6.362328in}}{\pgfqpoint{1.527722in}{6.351729in}}{\pgfqpoint{1.535536in}{6.343916in}}%
\pgfpathcurveto{\pgfqpoint{1.543349in}{6.336102in}}{\pgfqpoint{1.553948in}{6.331712in}}{\pgfqpoint{1.564998in}{6.331712in}}%
\pgfpathlineto{\pgfqpoint{1.564998in}{6.331712in}}%
\pgfpathclose%
\pgfusepath{stroke,fill}%
\end{pgfscope}%
\begin{pgfscope}%
\pgfpathrectangle{\pgfqpoint{0.633874in}{5.272501in}}{\pgfqpoint{2.177280in}{2.201755in}}%
\pgfusepath{clip}%
\pgfsetbuttcap%
\pgfsetroundjoin%
\definecolor{currentfill}{rgb}{1.000000,0.498039,0.054902}%
\pgfsetfillcolor{currentfill}%
\pgfsetlinewidth{0.481800pt}%
\definecolor{currentstroke}{rgb}{1.000000,1.000000,1.000000}%
\pgfsetstrokecolor{currentstroke}%
\pgfsetdash{}{0pt}%
\pgfpathmoveto{\pgfqpoint{1.644876in}{5.998113in}}%
\pgfpathcurveto{\pgfqpoint{1.655926in}{5.998113in}}{\pgfqpoint{1.666525in}{6.002503in}}{\pgfqpoint{1.674339in}{6.010317in}}%
\pgfpathcurveto{\pgfqpoint{1.682152in}{6.018130in}}{\pgfqpoint{1.686543in}{6.028729in}}{\pgfqpoint{1.686543in}{6.039779in}}%
\pgfpathcurveto{\pgfqpoint{1.686543in}{6.050829in}}{\pgfqpoint{1.682152in}{6.061428in}}{\pgfqpoint{1.674339in}{6.069242in}}%
\pgfpathcurveto{\pgfqpoint{1.666525in}{6.077056in}}{\pgfqpoint{1.655926in}{6.081446in}}{\pgfqpoint{1.644876in}{6.081446in}}%
\pgfpathcurveto{\pgfqpoint{1.633826in}{6.081446in}}{\pgfqpoint{1.623227in}{6.077056in}}{\pgfqpoint{1.615413in}{6.069242in}}%
\pgfpathcurveto{\pgfqpoint{1.607599in}{6.061428in}}{\pgfqpoint{1.603209in}{6.050829in}}{\pgfqpoint{1.603209in}{6.039779in}}%
\pgfpathcurveto{\pgfqpoint{1.603209in}{6.028729in}}{\pgfqpoint{1.607599in}{6.018130in}}{\pgfqpoint{1.615413in}{6.010317in}}%
\pgfpathcurveto{\pgfqpoint{1.623227in}{6.002503in}}{\pgfqpoint{1.633826in}{5.998113in}}{\pgfqpoint{1.644876in}{5.998113in}}%
\pgfpathlineto{\pgfqpoint{1.644876in}{5.998113in}}%
\pgfpathclose%
\pgfusepath{stroke,fill}%
\end{pgfscope}%
\begin{pgfscope}%
\pgfpathrectangle{\pgfqpoint{0.633874in}{5.272501in}}{\pgfqpoint{2.177280in}{2.201755in}}%
\pgfusepath{clip}%
\pgfsetbuttcap%
\pgfsetroundjoin%
\definecolor{currentfill}{rgb}{1.000000,0.498039,0.054902}%
\pgfsetfillcolor{currentfill}%
\pgfsetlinewidth{0.481800pt}%
\definecolor{currentstroke}{rgb}{1.000000,1.000000,1.000000}%
\pgfsetstrokecolor{currentstroke}%
\pgfsetdash{}{0pt}%
\pgfpathmoveto{\pgfqpoint{1.724753in}{5.747913in}}%
\pgfpathcurveto{\pgfqpoint{1.735803in}{5.747913in}}{\pgfqpoint{1.746402in}{5.752303in}}{\pgfqpoint{1.754216in}{5.760117in}}%
\pgfpathcurveto{\pgfqpoint{1.762030in}{5.767931in}}{\pgfqpoint{1.766420in}{5.778530in}}{\pgfqpoint{1.766420in}{5.789580in}}%
\pgfpathcurveto{\pgfqpoint{1.766420in}{5.800630in}}{\pgfqpoint{1.762030in}{5.811229in}}{\pgfqpoint{1.754216in}{5.819043in}}%
\pgfpathcurveto{\pgfqpoint{1.746402in}{5.826856in}}{\pgfqpoint{1.735803in}{5.831247in}}{\pgfqpoint{1.724753in}{5.831247in}}%
\pgfpathcurveto{\pgfqpoint{1.713703in}{5.831247in}}{\pgfqpoint{1.703104in}{5.826856in}}{\pgfqpoint{1.695290in}{5.819043in}}%
\pgfpathcurveto{\pgfqpoint{1.687477in}{5.811229in}}{\pgfqpoint{1.683087in}{5.800630in}}{\pgfqpoint{1.683087in}{5.789580in}}%
\pgfpathcurveto{\pgfqpoint{1.683087in}{5.778530in}}{\pgfqpoint{1.687477in}{5.767931in}}{\pgfqpoint{1.695290in}{5.760117in}}%
\pgfpathcurveto{\pgfqpoint{1.703104in}{5.752303in}}{\pgfqpoint{1.713703in}{5.747913in}}{\pgfqpoint{1.724753in}{5.747913in}}%
\pgfpathlineto{\pgfqpoint{1.724753in}{5.747913in}}%
\pgfpathclose%
\pgfusepath{stroke,fill}%
\end{pgfscope}%
\begin{pgfscope}%
\pgfpathrectangle{\pgfqpoint{0.633874in}{5.272501in}}{\pgfqpoint{2.177280in}{2.201755in}}%
\pgfusepath{clip}%
\pgfsetbuttcap%
\pgfsetroundjoin%
\definecolor{currentfill}{rgb}{1.000000,0.498039,0.054902}%
\pgfsetfillcolor{currentfill}%
\pgfsetlinewidth{0.481800pt}%
\definecolor{currentstroke}{rgb}{1.000000,1.000000,1.000000}%
\pgfsetstrokecolor{currentstroke}%
\pgfsetdash{}{0pt}%
\pgfpathmoveto{\pgfqpoint{1.644876in}{5.998113in}}%
\pgfpathcurveto{\pgfqpoint{1.655926in}{5.998113in}}{\pgfqpoint{1.666525in}{6.002503in}}{\pgfqpoint{1.674339in}{6.010317in}}%
\pgfpathcurveto{\pgfqpoint{1.682152in}{6.018130in}}{\pgfqpoint{1.686543in}{6.028729in}}{\pgfqpoint{1.686543in}{6.039779in}}%
\pgfpathcurveto{\pgfqpoint{1.686543in}{6.050829in}}{\pgfqpoint{1.682152in}{6.061428in}}{\pgfqpoint{1.674339in}{6.069242in}}%
\pgfpathcurveto{\pgfqpoint{1.666525in}{6.077056in}}{\pgfqpoint{1.655926in}{6.081446in}}{\pgfqpoint{1.644876in}{6.081446in}}%
\pgfpathcurveto{\pgfqpoint{1.633826in}{6.081446in}}{\pgfqpoint{1.623227in}{6.077056in}}{\pgfqpoint{1.615413in}{6.069242in}}%
\pgfpathcurveto{\pgfqpoint{1.607599in}{6.061428in}}{\pgfqpoint{1.603209in}{6.050829in}}{\pgfqpoint{1.603209in}{6.039779in}}%
\pgfpathcurveto{\pgfqpoint{1.603209in}{6.028729in}}{\pgfqpoint{1.607599in}{6.018130in}}{\pgfqpoint{1.615413in}{6.010317in}}%
\pgfpathcurveto{\pgfqpoint{1.623227in}{6.002503in}}{\pgfqpoint{1.633826in}{5.998113in}}{\pgfqpoint{1.644876in}{5.998113in}}%
\pgfpathlineto{\pgfqpoint{1.644876in}{5.998113in}}%
\pgfpathclose%
\pgfusepath{stroke,fill}%
\end{pgfscope}%
\begin{pgfscope}%
\pgfpathrectangle{\pgfqpoint{0.633874in}{5.272501in}}{\pgfqpoint{2.177280in}{2.201755in}}%
\pgfusepath{clip}%
\pgfsetbuttcap%
\pgfsetroundjoin%
\definecolor{currentfill}{rgb}{1.000000,0.498039,0.054902}%
\pgfsetfillcolor{currentfill}%
\pgfsetlinewidth{0.481800pt}%
\definecolor{currentstroke}{rgb}{1.000000,1.000000,1.000000}%
\pgfsetstrokecolor{currentstroke}%
\pgfsetdash{}{0pt}%
\pgfpathmoveto{\pgfqpoint{1.764692in}{6.081512in}}%
\pgfpathcurveto{\pgfqpoint{1.775742in}{6.081512in}}{\pgfqpoint{1.786341in}{6.085903in}}{\pgfqpoint{1.794155in}{6.093716in}}%
\pgfpathcurveto{\pgfqpoint{1.801968in}{6.101530in}}{\pgfqpoint{1.806359in}{6.112129in}}{\pgfqpoint{1.806359in}{6.123179in}}%
\pgfpathcurveto{\pgfqpoint{1.806359in}{6.134229in}}{\pgfqpoint{1.801968in}{6.144828in}}{\pgfqpoint{1.794155in}{6.152642in}}%
\pgfpathcurveto{\pgfqpoint{1.786341in}{6.160456in}}{\pgfqpoint{1.775742in}{6.164846in}}{\pgfqpoint{1.764692in}{6.164846in}}%
\pgfpathcurveto{\pgfqpoint{1.753642in}{6.164846in}}{\pgfqpoint{1.743043in}{6.160456in}}{\pgfqpoint{1.735229in}{6.152642in}}%
\pgfpathcurveto{\pgfqpoint{1.727416in}{6.144828in}}{\pgfqpoint{1.723025in}{6.134229in}}{\pgfqpoint{1.723025in}{6.123179in}}%
\pgfpathcurveto{\pgfqpoint{1.723025in}{6.112129in}}{\pgfqpoint{1.727416in}{6.101530in}}{\pgfqpoint{1.735229in}{6.093716in}}%
\pgfpathcurveto{\pgfqpoint{1.743043in}{6.085903in}}{\pgfqpoint{1.753642in}{6.081512in}}{\pgfqpoint{1.764692in}{6.081512in}}%
\pgfpathlineto{\pgfqpoint{1.764692in}{6.081512in}}%
\pgfpathclose%
\pgfusepath{stroke,fill}%
\end{pgfscope}%
\begin{pgfscope}%
\pgfpathrectangle{\pgfqpoint{0.633874in}{5.272501in}}{\pgfqpoint{2.177280in}{2.201755in}}%
\pgfusepath{clip}%
\pgfsetbuttcap%
\pgfsetroundjoin%
\definecolor{currentfill}{rgb}{1.000000,0.498039,0.054902}%
\pgfsetfillcolor{currentfill}%
\pgfsetlinewidth{0.481800pt}%
\definecolor{currentstroke}{rgb}{1.000000,1.000000,1.000000}%
\pgfsetstrokecolor{currentstroke}%
\pgfsetdash{}{0pt}%
\pgfpathmoveto{\pgfqpoint{1.844569in}{6.164912in}}%
\pgfpathcurveto{\pgfqpoint{1.855619in}{6.164912in}}{\pgfqpoint{1.866219in}{6.169303in}}{\pgfqpoint{1.874032in}{6.177116in}}%
\pgfpathcurveto{\pgfqpoint{1.881846in}{6.184930in}}{\pgfqpoint{1.886236in}{6.195529in}}{\pgfqpoint{1.886236in}{6.206579in}}%
\pgfpathcurveto{\pgfqpoint{1.886236in}{6.217629in}}{\pgfqpoint{1.881846in}{6.228228in}}{\pgfqpoint{1.874032in}{6.236042in}}%
\pgfpathcurveto{\pgfqpoint{1.866219in}{6.243855in}}{\pgfqpoint{1.855619in}{6.248246in}}{\pgfqpoint{1.844569in}{6.248246in}}%
\pgfpathcurveto{\pgfqpoint{1.833519in}{6.248246in}}{\pgfqpoint{1.822920in}{6.243855in}}{\pgfqpoint{1.815107in}{6.236042in}}%
\pgfpathcurveto{\pgfqpoint{1.807293in}{6.228228in}}{\pgfqpoint{1.802903in}{6.217629in}}{\pgfqpoint{1.802903in}{6.206579in}}%
\pgfpathcurveto{\pgfqpoint{1.802903in}{6.195529in}}{\pgfqpoint{1.807293in}{6.184930in}}{\pgfqpoint{1.815107in}{6.177116in}}%
\pgfpathcurveto{\pgfqpoint{1.822920in}{6.169303in}}{\pgfqpoint{1.833519in}{6.164912in}}{\pgfqpoint{1.844569in}{6.164912in}}%
\pgfpathlineto{\pgfqpoint{1.844569in}{6.164912in}}%
\pgfpathclose%
\pgfusepath{stroke,fill}%
\end{pgfscope}%
\begin{pgfscope}%
\pgfpathrectangle{\pgfqpoint{0.633874in}{5.272501in}}{\pgfqpoint{2.177280in}{2.201755in}}%
\pgfusepath{clip}%
\pgfsetbuttcap%
\pgfsetroundjoin%
\definecolor{currentfill}{rgb}{1.000000,0.498039,0.054902}%
\pgfsetfillcolor{currentfill}%
\pgfsetlinewidth{0.481800pt}%
\definecolor{currentstroke}{rgb}{1.000000,1.000000,1.000000}%
\pgfsetstrokecolor{currentstroke}%
\pgfsetdash{}{0pt}%
\pgfpathmoveto{\pgfqpoint{1.924447in}{5.998113in}}%
\pgfpathcurveto{\pgfqpoint{1.935497in}{5.998113in}}{\pgfqpoint{1.946096in}{6.002503in}}{\pgfqpoint{1.953910in}{6.010317in}}%
\pgfpathcurveto{\pgfqpoint{1.961723in}{6.018130in}}{\pgfqpoint{1.966113in}{6.028729in}}{\pgfqpoint{1.966113in}{6.039779in}}%
\pgfpathcurveto{\pgfqpoint{1.966113in}{6.050829in}}{\pgfqpoint{1.961723in}{6.061428in}}{\pgfqpoint{1.953910in}{6.069242in}}%
\pgfpathcurveto{\pgfqpoint{1.946096in}{6.077056in}}{\pgfqpoint{1.935497in}{6.081446in}}{\pgfqpoint{1.924447in}{6.081446in}}%
\pgfpathcurveto{\pgfqpoint{1.913397in}{6.081446in}}{\pgfqpoint{1.902798in}{6.077056in}}{\pgfqpoint{1.894984in}{6.069242in}}%
\pgfpathcurveto{\pgfqpoint{1.887170in}{6.061428in}}{\pgfqpoint{1.882780in}{6.050829in}}{\pgfqpoint{1.882780in}{6.039779in}}%
\pgfpathcurveto{\pgfqpoint{1.882780in}{6.028729in}}{\pgfqpoint{1.887170in}{6.018130in}}{\pgfqpoint{1.894984in}{6.010317in}}%
\pgfpathcurveto{\pgfqpoint{1.902798in}{6.002503in}}{\pgfqpoint{1.913397in}{5.998113in}}{\pgfqpoint{1.924447in}{5.998113in}}%
\pgfpathlineto{\pgfqpoint{1.924447in}{5.998113in}}%
\pgfpathclose%
\pgfusepath{stroke,fill}%
\end{pgfscope}%
\begin{pgfscope}%
\pgfpathrectangle{\pgfqpoint{0.633874in}{5.272501in}}{\pgfqpoint{2.177280in}{2.201755in}}%
\pgfusepath{clip}%
\pgfsetbuttcap%
\pgfsetroundjoin%
\definecolor{currentfill}{rgb}{1.000000,0.498039,0.054902}%
\pgfsetfillcolor{currentfill}%
\pgfsetlinewidth{0.481800pt}%
\definecolor{currentstroke}{rgb}{1.000000,1.000000,1.000000}%
\pgfsetstrokecolor{currentstroke}%
\pgfsetdash{}{0pt}%
\pgfpathmoveto{\pgfqpoint{1.884508in}{6.164912in}}%
\pgfpathcurveto{\pgfqpoint{1.895558in}{6.164912in}}{\pgfqpoint{1.906157in}{6.169303in}}{\pgfqpoint{1.913971in}{6.177116in}}%
\pgfpathcurveto{\pgfqpoint{1.921784in}{6.184930in}}{\pgfqpoint{1.926175in}{6.195529in}}{\pgfqpoint{1.926175in}{6.206579in}}%
\pgfpathcurveto{\pgfqpoint{1.926175in}{6.217629in}}{\pgfqpoint{1.921784in}{6.228228in}}{\pgfqpoint{1.913971in}{6.236042in}}%
\pgfpathcurveto{\pgfqpoint{1.906157in}{6.243855in}}{\pgfqpoint{1.895558in}{6.248246in}}{\pgfqpoint{1.884508in}{6.248246in}}%
\pgfpathcurveto{\pgfqpoint{1.873458in}{6.248246in}}{\pgfqpoint{1.862859in}{6.243855in}}{\pgfqpoint{1.855045in}{6.236042in}}%
\pgfpathcurveto{\pgfqpoint{1.847232in}{6.228228in}}{\pgfqpoint{1.842841in}{6.217629in}}{\pgfqpoint{1.842841in}{6.206579in}}%
\pgfpathcurveto{\pgfqpoint{1.842841in}{6.195529in}}{\pgfqpoint{1.847232in}{6.184930in}}{\pgfqpoint{1.855045in}{6.177116in}}%
\pgfpathcurveto{\pgfqpoint{1.862859in}{6.169303in}}{\pgfqpoint{1.873458in}{6.164912in}}{\pgfqpoint{1.884508in}{6.164912in}}%
\pgfpathlineto{\pgfqpoint{1.884508in}{6.164912in}}%
\pgfpathclose%
\pgfusepath{stroke,fill}%
\end{pgfscope}%
\begin{pgfscope}%
\pgfpathrectangle{\pgfqpoint{0.633874in}{5.272501in}}{\pgfqpoint{2.177280in}{2.201755in}}%
\pgfusepath{clip}%
\pgfsetbuttcap%
\pgfsetroundjoin%
\definecolor{currentfill}{rgb}{1.000000,0.498039,0.054902}%
\pgfsetfillcolor{currentfill}%
\pgfsetlinewidth{0.481800pt}%
\definecolor{currentstroke}{rgb}{1.000000,1.000000,1.000000}%
\pgfsetstrokecolor{currentstroke}%
\pgfsetdash{}{0pt}%
\pgfpathmoveto{\pgfqpoint{1.604937in}{6.081512in}}%
\pgfpathcurveto{\pgfqpoint{1.615987in}{6.081512in}}{\pgfqpoint{1.626586in}{6.085903in}}{\pgfqpoint{1.634400in}{6.093716in}}%
\pgfpathcurveto{\pgfqpoint{1.642214in}{6.101530in}}{\pgfqpoint{1.646604in}{6.112129in}}{\pgfqpoint{1.646604in}{6.123179in}}%
\pgfpathcurveto{\pgfqpoint{1.646604in}{6.134229in}}{\pgfqpoint{1.642214in}{6.144828in}}{\pgfqpoint{1.634400in}{6.152642in}}%
\pgfpathcurveto{\pgfqpoint{1.626586in}{6.160456in}}{\pgfqpoint{1.615987in}{6.164846in}}{\pgfqpoint{1.604937in}{6.164846in}}%
\pgfpathcurveto{\pgfqpoint{1.593887in}{6.164846in}}{\pgfqpoint{1.583288in}{6.160456in}}{\pgfqpoint{1.575474in}{6.152642in}}%
\pgfpathcurveto{\pgfqpoint{1.567661in}{6.144828in}}{\pgfqpoint{1.563270in}{6.134229in}}{\pgfqpoint{1.563270in}{6.123179in}}%
\pgfpathcurveto{\pgfqpoint{1.563270in}{6.112129in}}{\pgfqpoint{1.567661in}{6.101530in}}{\pgfqpoint{1.575474in}{6.093716in}}%
\pgfpathcurveto{\pgfqpoint{1.583288in}{6.085903in}}{\pgfqpoint{1.593887in}{6.081512in}}{\pgfqpoint{1.604937in}{6.081512in}}%
\pgfpathlineto{\pgfqpoint{1.604937in}{6.081512in}}%
\pgfpathclose%
\pgfusepath{stroke,fill}%
\end{pgfscope}%
\begin{pgfscope}%
\pgfpathrectangle{\pgfqpoint{0.633874in}{5.272501in}}{\pgfqpoint{2.177280in}{2.201755in}}%
\pgfusepath{clip}%
\pgfsetbuttcap%
\pgfsetroundjoin%
\definecolor{currentfill}{rgb}{1.000000,0.498039,0.054902}%
\pgfsetfillcolor{currentfill}%
\pgfsetlinewidth{0.481800pt}%
\definecolor{currentstroke}{rgb}{1.000000,1.000000,1.000000}%
\pgfsetstrokecolor{currentstroke}%
\pgfsetdash{}{0pt}%
\pgfpathmoveto{\pgfqpoint{1.485121in}{5.831313in}}%
\pgfpathcurveto{\pgfqpoint{1.496171in}{5.831313in}}{\pgfqpoint{1.506770in}{5.835703in}}{\pgfqpoint{1.514584in}{5.843517in}}%
\pgfpathcurveto{\pgfqpoint{1.522397in}{5.851331in}}{\pgfqpoint{1.526788in}{5.861930in}}{\pgfqpoint{1.526788in}{5.872980in}}%
\pgfpathcurveto{\pgfqpoint{1.526788in}{5.884030in}}{\pgfqpoint{1.522397in}{5.894629in}}{\pgfqpoint{1.514584in}{5.902442in}}%
\pgfpathcurveto{\pgfqpoint{1.506770in}{5.910256in}}{\pgfqpoint{1.496171in}{5.914646in}}{\pgfqpoint{1.485121in}{5.914646in}}%
\pgfpathcurveto{\pgfqpoint{1.474071in}{5.914646in}}{\pgfqpoint{1.463472in}{5.910256in}}{\pgfqpoint{1.455658in}{5.902442in}}%
\pgfpathcurveto{\pgfqpoint{1.447845in}{5.894629in}}{\pgfqpoint{1.443454in}{5.884030in}}{\pgfqpoint{1.443454in}{5.872980in}}%
\pgfpathcurveto{\pgfqpoint{1.443454in}{5.861930in}}{\pgfqpoint{1.447845in}{5.851331in}}{\pgfqpoint{1.455658in}{5.843517in}}%
\pgfpathcurveto{\pgfqpoint{1.463472in}{5.835703in}}{\pgfqpoint{1.474071in}{5.831313in}}{\pgfqpoint{1.485121in}{5.831313in}}%
\pgfpathlineto{\pgfqpoint{1.485121in}{5.831313in}}%
\pgfpathclose%
\pgfusepath{stroke,fill}%
\end{pgfscope}%
\begin{pgfscope}%
\pgfpathrectangle{\pgfqpoint{0.633874in}{5.272501in}}{\pgfqpoint{2.177280in}{2.201755in}}%
\pgfusepath{clip}%
\pgfsetbuttcap%
\pgfsetroundjoin%
\definecolor{currentfill}{rgb}{1.000000,0.498039,0.054902}%
\pgfsetfillcolor{currentfill}%
\pgfsetlinewidth{0.481800pt}%
\definecolor{currentstroke}{rgb}{1.000000,1.000000,1.000000}%
\pgfsetstrokecolor{currentstroke}%
\pgfsetdash{}{0pt}%
\pgfpathmoveto{\pgfqpoint{1.405244in}{5.664513in}}%
\pgfpathcurveto{\pgfqpoint{1.416294in}{5.664513in}}{\pgfqpoint{1.426893in}{5.668904in}}{\pgfqpoint{1.434706in}{5.676717in}}%
\pgfpathcurveto{\pgfqpoint{1.442520in}{5.684531in}}{\pgfqpoint{1.446910in}{5.695130in}}{\pgfqpoint{1.446910in}{5.706180in}}%
\pgfpathcurveto{\pgfqpoint{1.446910in}{5.717230in}}{\pgfqpoint{1.442520in}{5.727829in}}{\pgfqpoint{1.434706in}{5.735643in}}%
\pgfpathcurveto{\pgfqpoint{1.426893in}{5.743456in}}{\pgfqpoint{1.416294in}{5.747847in}}{\pgfqpoint{1.405244in}{5.747847in}}%
\pgfpathcurveto{\pgfqpoint{1.394193in}{5.747847in}}{\pgfqpoint{1.383594in}{5.743456in}}{\pgfqpoint{1.375781in}{5.735643in}}%
\pgfpathcurveto{\pgfqpoint{1.367967in}{5.727829in}}{\pgfqpoint{1.363577in}{5.717230in}}{\pgfqpoint{1.363577in}{5.706180in}}%
\pgfpathcurveto{\pgfqpoint{1.363577in}{5.695130in}}{\pgfqpoint{1.367967in}{5.684531in}}{\pgfqpoint{1.375781in}{5.676717in}}%
\pgfpathcurveto{\pgfqpoint{1.383594in}{5.668904in}}{\pgfqpoint{1.394193in}{5.664513in}}{\pgfqpoint{1.405244in}{5.664513in}}%
\pgfpathlineto{\pgfqpoint{1.405244in}{5.664513in}}%
\pgfpathclose%
\pgfusepath{stroke,fill}%
\end{pgfscope}%
\begin{pgfscope}%
\pgfpathrectangle{\pgfqpoint{0.633874in}{5.272501in}}{\pgfqpoint{2.177280in}{2.201755in}}%
\pgfusepath{clip}%
\pgfsetbuttcap%
\pgfsetroundjoin%
\definecolor{currentfill}{rgb}{1.000000,0.498039,0.054902}%
\pgfsetfillcolor{currentfill}%
\pgfsetlinewidth{0.481800pt}%
\definecolor{currentstroke}{rgb}{1.000000,1.000000,1.000000}%
\pgfsetstrokecolor{currentstroke}%
\pgfsetdash{}{0pt}%
\pgfpathmoveto{\pgfqpoint{1.405244in}{5.664513in}}%
\pgfpathcurveto{\pgfqpoint{1.416294in}{5.664513in}}{\pgfqpoint{1.426893in}{5.668904in}}{\pgfqpoint{1.434706in}{5.676717in}}%
\pgfpathcurveto{\pgfqpoint{1.442520in}{5.684531in}}{\pgfqpoint{1.446910in}{5.695130in}}{\pgfqpoint{1.446910in}{5.706180in}}%
\pgfpathcurveto{\pgfqpoint{1.446910in}{5.717230in}}{\pgfqpoint{1.442520in}{5.727829in}}{\pgfqpoint{1.434706in}{5.735643in}}%
\pgfpathcurveto{\pgfqpoint{1.426893in}{5.743456in}}{\pgfqpoint{1.416294in}{5.747847in}}{\pgfqpoint{1.405244in}{5.747847in}}%
\pgfpathcurveto{\pgfqpoint{1.394193in}{5.747847in}}{\pgfqpoint{1.383594in}{5.743456in}}{\pgfqpoint{1.375781in}{5.735643in}}%
\pgfpathcurveto{\pgfqpoint{1.367967in}{5.727829in}}{\pgfqpoint{1.363577in}{5.717230in}}{\pgfqpoint{1.363577in}{5.706180in}}%
\pgfpathcurveto{\pgfqpoint{1.363577in}{5.695130in}}{\pgfqpoint{1.367967in}{5.684531in}}{\pgfqpoint{1.375781in}{5.676717in}}%
\pgfpathcurveto{\pgfqpoint{1.383594in}{5.668904in}}{\pgfqpoint{1.394193in}{5.664513in}}{\pgfqpoint{1.405244in}{5.664513in}}%
\pgfpathlineto{\pgfqpoint{1.405244in}{5.664513in}}%
\pgfpathclose%
\pgfusepath{stroke,fill}%
\end{pgfscope}%
\begin{pgfscope}%
\pgfpathrectangle{\pgfqpoint{0.633874in}{5.272501in}}{\pgfqpoint{2.177280in}{2.201755in}}%
\pgfusepath{clip}%
\pgfsetbuttcap%
\pgfsetroundjoin%
\definecolor{currentfill}{rgb}{1.000000,0.498039,0.054902}%
\pgfsetfillcolor{currentfill}%
\pgfsetlinewidth{0.481800pt}%
\definecolor{currentstroke}{rgb}{1.000000,1.000000,1.000000}%
\pgfsetstrokecolor{currentstroke}%
\pgfsetdash{}{0pt}%
\pgfpathmoveto{\pgfqpoint{1.525060in}{5.914713in}}%
\pgfpathcurveto{\pgfqpoint{1.536110in}{5.914713in}}{\pgfqpoint{1.546709in}{5.919103in}}{\pgfqpoint{1.554523in}{5.926917in}}%
\pgfpathcurveto{\pgfqpoint{1.562336in}{5.934730in}}{\pgfqpoint{1.566726in}{5.945329in}}{\pgfqpoint{1.566726in}{5.956379in}}%
\pgfpathcurveto{\pgfqpoint{1.566726in}{5.967430in}}{\pgfqpoint{1.562336in}{5.978029in}}{\pgfqpoint{1.554523in}{5.985842in}}%
\pgfpathcurveto{\pgfqpoint{1.546709in}{5.993656in}}{\pgfqpoint{1.536110in}{5.998046in}}{\pgfqpoint{1.525060in}{5.998046in}}%
\pgfpathcurveto{\pgfqpoint{1.514010in}{5.998046in}}{\pgfqpoint{1.503411in}{5.993656in}}{\pgfqpoint{1.495597in}{5.985842in}}%
\pgfpathcurveto{\pgfqpoint{1.487783in}{5.978029in}}{\pgfqpoint{1.483393in}{5.967430in}}{\pgfqpoint{1.483393in}{5.956379in}}%
\pgfpathcurveto{\pgfqpoint{1.483393in}{5.945329in}}{\pgfqpoint{1.487783in}{5.934730in}}{\pgfqpoint{1.495597in}{5.926917in}}%
\pgfpathcurveto{\pgfqpoint{1.503411in}{5.919103in}}{\pgfqpoint{1.514010in}{5.914713in}}{\pgfqpoint{1.525060in}{5.914713in}}%
\pgfpathlineto{\pgfqpoint{1.525060in}{5.914713in}}%
\pgfpathclose%
\pgfusepath{stroke,fill}%
\end{pgfscope}%
\begin{pgfscope}%
\pgfpathrectangle{\pgfqpoint{0.633874in}{5.272501in}}{\pgfqpoint{2.177280in}{2.201755in}}%
\pgfusepath{clip}%
\pgfsetbuttcap%
\pgfsetroundjoin%
\definecolor{currentfill}{rgb}{1.000000,0.498039,0.054902}%
\pgfsetfillcolor{currentfill}%
\pgfsetlinewidth{0.481800pt}%
\definecolor{currentstroke}{rgb}{1.000000,1.000000,1.000000}%
\pgfsetstrokecolor{currentstroke}%
\pgfsetdash{}{0pt}%
\pgfpathmoveto{\pgfqpoint{1.604937in}{5.914713in}}%
\pgfpathcurveto{\pgfqpoint{1.615987in}{5.914713in}}{\pgfqpoint{1.626586in}{5.919103in}}{\pgfqpoint{1.634400in}{5.926917in}}%
\pgfpathcurveto{\pgfqpoint{1.642214in}{5.934730in}}{\pgfqpoint{1.646604in}{5.945329in}}{\pgfqpoint{1.646604in}{5.956379in}}%
\pgfpathcurveto{\pgfqpoint{1.646604in}{5.967430in}}{\pgfqpoint{1.642214in}{5.978029in}}{\pgfqpoint{1.634400in}{5.985842in}}%
\pgfpathcurveto{\pgfqpoint{1.626586in}{5.993656in}}{\pgfqpoint{1.615987in}{5.998046in}}{\pgfqpoint{1.604937in}{5.998046in}}%
\pgfpathcurveto{\pgfqpoint{1.593887in}{5.998046in}}{\pgfqpoint{1.583288in}{5.993656in}}{\pgfqpoint{1.575474in}{5.985842in}}%
\pgfpathcurveto{\pgfqpoint{1.567661in}{5.978029in}}{\pgfqpoint{1.563270in}{5.967430in}}{\pgfqpoint{1.563270in}{5.956379in}}%
\pgfpathcurveto{\pgfqpoint{1.563270in}{5.945329in}}{\pgfqpoint{1.567661in}{5.934730in}}{\pgfqpoint{1.575474in}{5.926917in}}%
\pgfpathcurveto{\pgfqpoint{1.583288in}{5.919103in}}{\pgfqpoint{1.593887in}{5.914713in}}{\pgfqpoint{1.604937in}{5.914713in}}%
\pgfpathlineto{\pgfqpoint{1.604937in}{5.914713in}}%
\pgfpathclose%
\pgfusepath{stroke,fill}%
\end{pgfscope}%
\begin{pgfscope}%
\pgfpathrectangle{\pgfqpoint{0.633874in}{5.272501in}}{\pgfqpoint{2.177280in}{2.201755in}}%
\pgfusepath{clip}%
\pgfsetbuttcap%
\pgfsetroundjoin%
\definecolor{currentfill}{rgb}{1.000000,0.498039,0.054902}%
\pgfsetfillcolor{currentfill}%
\pgfsetlinewidth{0.481800pt}%
\definecolor{currentstroke}{rgb}{1.000000,1.000000,1.000000}%
\pgfsetstrokecolor{currentstroke}%
\pgfsetdash{}{0pt}%
\pgfpathmoveto{\pgfqpoint{1.365305in}{6.164912in}}%
\pgfpathcurveto{\pgfqpoint{1.376355in}{6.164912in}}{\pgfqpoint{1.386954in}{6.169303in}}{\pgfqpoint{1.394768in}{6.177116in}}%
\pgfpathcurveto{\pgfqpoint{1.402581in}{6.184930in}}{\pgfqpoint{1.406972in}{6.195529in}}{\pgfqpoint{1.406972in}{6.206579in}}%
\pgfpathcurveto{\pgfqpoint{1.406972in}{6.217629in}}{\pgfqpoint{1.402581in}{6.228228in}}{\pgfqpoint{1.394768in}{6.236042in}}%
\pgfpathcurveto{\pgfqpoint{1.386954in}{6.243855in}}{\pgfqpoint{1.376355in}{6.248246in}}{\pgfqpoint{1.365305in}{6.248246in}}%
\pgfpathcurveto{\pgfqpoint{1.354255in}{6.248246in}}{\pgfqpoint{1.343656in}{6.243855in}}{\pgfqpoint{1.335842in}{6.236042in}}%
\pgfpathcurveto{\pgfqpoint{1.328029in}{6.228228in}}{\pgfqpoint{1.323638in}{6.217629in}}{\pgfqpoint{1.323638in}{6.206579in}}%
\pgfpathcurveto{\pgfqpoint{1.323638in}{6.195529in}}{\pgfqpoint{1.328029in}{6.184930in}}{\pgfqpoint{1.335842in}{6.177116in}}%
\pgfpathcurveto{\pgfqpoint{1.343656in}{6.169303in}}{\pgfqpoint{1.354255in}{6.164912in}}{\pgfqpoint{1.365305in}{6.164912in}}%
\pgfpathlineto{\pgfqpoint{1.365305in}{6.164912in}}%
\pgfpathclose%
\pgfusepath{stroke,fill}%
\end{pgfscope}%
\begin{pgfscope}%
\pgfpathrectangle{\pgfqpoint{0.633874in}{5.272501in}}{\pgfqpoint{2.177280in}{2.201755in}}%
\pgfusepath{clip}%
\pgfsetbuttcap%
\pgfsetroundjoin%
\definecolor{currentfill}{rgb}{1.000000,0.498039,0.054902}%
\pgfsetfillcolor{currentfill}%
\pgfsetlinewidth{0.481800pt}%
\definecolor{currentstroke}{rgb}{1.000000,1.000000,1.000000}%
\pgfsetstrokecolor{currentstroke}%
\pgfsetdash{}{0pt}%
\pgfpathmoveto{\pgfqpoint{1.604937in}{6.498512in}}%
\pgfpathcurveto{\pgfqpoint{1.615987in}{6.498512in}}{\pgfqpoint{1.626586in}{6.502902in}}{\pgfqpoint{1.634400in}{6.510715in}}%
\pgfpathcurveto{\pgfqpoint{1.642214in}{6.518529in}}{\pgfqpoint{1.646604in}{6.529128in}}{\pgfqpoint{1.646604in}{6.540178in}}%
\pgfpathcurveto{\pgfqpoint{1.646604in}{6.551228in}}{\pgfqpoint{1.642214in}{6.561827in}}{\pgfqpoint{1.634400in}{6.569641in}}%
\pgfpathcurveto{\pgfqpoint{1.626586in}{6.577455in}}{\pgfqpoint{1.615987in}{6.581845in}}{\pgfqpoint{1.604937in}{6.581845in}}%
\pgfpathcurveto{\pgfqpoint{1.593887in}{6.581845in}}{\pgfqpoint{1.583288in}{6.577455in}}{\pgfqpoint{1.575474in}{6.569641in}}%
\pgfpathcurveto{\pgfqpoint{1.567661in}{6.561827in}}{\pgfqpoint{1.563270in}{6.551228in}}{\pgfqpoint{1.563270in}{6.540178in}}%
\pgfpathcurveto{\pgfqpoint{1.563270in}{6.529128in}}{\pgfqpoint{1.567661in}{6.518529in}}{\pgfqpoint{1.575474in}{6.510715in}}%
\pgfpathcurveto{\pgfqpoint{1.583288in}{6.502902in}}{\pgfqpoint{1.593887in}{6.498512in}}{\pgfqpoint{1.604937in}{6.498512in}}%
\pgfpathlineto{\pgfqpoint{1.604937in}{6.498512in}}%
\pgfpathclose%
\pgfusepath{stroke,fill}%
\end{pgfscope}%
\begin{pgfscope}%
\pgfpathrectangle{\pgfqpoint{0.633874in}{5.272501in}}{\pgfqpoint{2.177280in}{2.201755in}}%
\pgfusepath{clip}%
\pgfsetbuttcap%
\pgfsetroundjoin%
\definecolor{currentfill}{rgb}{1.000000,0.498039,0.054902}%
\pgfsetfillcolor{currentfill}%
\pgfsetlinewidth{0.481800pt}%
\definecolor{currentstroke}{rgb}{1.000000,1.000000,1.000000}%
\pgfsetstrokecolor{currentstroke}%
\pgfsetdash{}{0pt}%
\pgfpathmoveto{\pgfqpoint{1.884508in}{6.248312in}}%
\pgfpathcurveto{\pgfqpoint{1.895558in}{6.248312in}}{\pgfqpoint{1.906157in}{6.252702in}}{\pgfqpoint{1.913971in}{6.260516in}}%
\pgfpathcurveto{\pgfqpoint{1.921784in}{6.268330in}}{\pgfqpoint{1.926175in}{6.278929in}}{\pgfqpoint{1.926175in}{6.289979in}}%
\pgfpathcurveto{\pgfqpoint{1.926175in}{6.301029in}}{\pgfqpoint{1.921784in}{6.311628in}}{\pgfqpoint{1.913971in}{6.319442in}}%
\pgfpathcurveto{\pgfqpoint{1.906157in}{6.327255in}}{\pgfqpoint{1.895558in}{6.331645in}}{\pgfqpoint{1.884508in}{6.331645in}}%
\pgfpathcurveto{\pgfqpoint{1.873458in}{6.331645in}}{\pgfqpoint{1.862859in}{6.327255in}}{\pgfqpoint{1.855045in}{6.319442in}}%
\pgfpathcurveto{\pgfqpoint{1.847232in}{6.311628in}}{\pgfqpoint{1.842841in}{6.301029in}}{\pgfqpoint{1.842841in}{6.289979in}}%
\pgfpathcurveto{\pgfqpoint{1.842841in}{6.278929in}}{\pgfqpoint{1.847232in}{6.268330in}}{\pgfqpoint{1.855045in}{6.260516in}}%
\pgfpathcurveto{\pgfqpoint{1.862859in}{6.252702in}}{\pgfqpoint{1.873458in}{6.248312in}}{\pgfqpoint{1.884508in}{6.248312in}}%
\pgfpathlineto{\pgfqpoint{1.884508in}{6.248312in}}%
\pgfpathclose%
\pgfusepath{stroke,fill}%
\end{pgfscope}%
\begin{pgfscope}%
\pgfpathrectangle{\pgfqpoint{0.633874in}{5.272501in}}{\pgfqpoint{2.177280in}{2.201755in}}%
\pgfusepath{clip}%
\pgfsetbuttcap%
\pgfsetroundjoin%
\definecolor{currentfill}{rgb}{1.000000,0.498039,0.054902}%
\pgfsetfillcolor{currentfill}%
\pgfsetlinewidth{0.481800pt}%
\definecolor{currentstroke}{rgb}{1.000000,1.000000,1.000000}%
\pgfsetstrokecolor{currentstroke}%
\pgfsetdash{}{0pt}%
\pgfpathmoveto{\pgfqpoint{1.724753in}{5.581114in}}%
\pgfpathcurveto{\pgfqpoint{1.735803in}{5.581114in}}{\pgfqpoint{1.746402in}{5.585504in}}{\pgfqpoint{1.754216in}{5.593317in}}%
\pgfpathcurveto{\pgfqpoint{1.762030in}{5.601131in}}{\pgfqpoint{1.766420in}{5.611730in}}{\pgfqpoint{1.766420in}{5.622780in}}%
\pgfpathcurveto{\pgfqpoint{1.766420in}{5.633830in}}{\pgfqpoint{1.762030in}{5.644429in}}{\pgfqpoint{1.754216in}{5.652243in}}%
\pgfpathcurveto{\pgfqpoint{1.746402in}{5.660057in}}{\pgfqpoint{1.735803in}{5.664447in}}{\pgfqpoint{1.724753in}{5.664447in}}%
\pgfpathcurveto{\pgfqpoint{1.713703in}{5.664447in}}{\pgfqpoint{1.703104in}{5.660057in}}{\pgfqpoint{1.695290in}{5.652243in}}%
\pgfpathcurveto{\pgfqpoint{1.687477in}{5.644429in}}{\pgfqpoint{1.683087in}{5.633830in}}{\pgfqpoint{1.683087in}{5.622780in}}%
\pgfpathcurveto{\pgfqpoint{1.683087in}{5.611730in}}{\pgfqpoint{1.687477in}{5.601131in}}{\pgfqpoint{1.695290in}{5.593317in}}%
\pgfpathcurveto{\pgfqpoint{1.703104in}{5.585504in}}{\pgfqpoint{1.713703in}{5.581114in}}{\pgfqpoint{1.724753in}{5.581114in}}%
\pgfpathlineto{\pgfqpoint{1.724753in}{5.581114in}}%
\pgfpathclose%
\pgfusepath{stroke,fill}%
\end{pgfscope}%
\begin{pgfscope}%
\pgfpathrectangle{\pgfqpoint{0.633874in}{5.272501in}}{\pgfqpoint{2.177280in}{2.201755in}}%
\pgfusepath{clip}%
\pgfsetbuttcap%
\pgfsetroundjoin%
\definecolor{currentfill}{rgb}{1.000000,0.498039,0.054902}%
\pgfsetfillcolor{currentfill}%
\pgfsetlinewidth{0.481800pt}%
\definecolor{currentstroke}{rgb}{1.000000,1.000000,1.000000}%
\pgfsetstrokecolor{currentstroke}%
\pgfsetdash{}{0pt}%
\pgfpathmoveto{\pgfqpoint{1.445182in}{6.164912in}}%
\pgfpathcurveto{\pgfqpoint{1.456232in}{6.164912in}}{\pgfqpoint{1.466831in}{6.169303in}}{\pgfqpoint{1.474645in}{6.177116in}}%
\pgfpathcurveto{\pgfqpoint{1.482459in}{6.184930in}}{\pgfqpoint{1.486849in}{6.195529in}}{\pgfqpoint{1.486849in}{6.206579in}}%
\pgfpathcurveto{\pgfqpoint{1.486849in}{6.217629in}}{\pgfqpoint{1.482459in}{6.228228in}}{\pgfqpoint{1.474645in}{6.236042in}}%
\pgfpathcurveto{\pgfqpoint{1.466831in}{6.243855in}}{\pgfqpoint{1.456232in}{6.248246in}}{\pgfqpoint{1.445182in}{6.248246in}}%
\pgfpathcurveto{\pgfqpoint{1.434132in}{6.248246in}}{\pgfqpoint{1.423533in}{6.243855in}}{\pgfqpoint{1.415720in}{6.236042in}}%
\pgfpathcurveto{\pgfqpoint{1.407906in}{6.228228in}}{\pgfqpoint{1.403516in}{6.217629in}}{\pgfqpoint{1.403516in}{6.206579in}}%
\pgfpathcurveto{\pgfqpoint{1.403516in}{6.195529in}}{\pgfqpoint{1.407906in}{6.184930in}}{\pgfqpoint{1.415720in}{6.177116in}}%
\pgfpathcurveto{\pgfqpoint{1.423533in}{6.169303in}}{\pgfqpoint{1.434132in}{6.164912in}}{\pgfqpoint{1.445182in}{6.164912in}}%
\pgfpathlineto{\pgfqpoint{1.445182in}{6.164912in}}%
\pgfpathclose%
\pgfusepath{stroke,fill}%
\end{pgfscope}%
\begin{pgfscope}%
\pgfpathrectangle{\pgfqpoint{0.633874in}{5.272501in}}{\pgfqpoint{2.177280in}{2.201755in}}%
\pgfusepath{clip}%
\pgfsetbuttcap%
\pgfsetroundjoin%
\definecolor{currentfill}{rgb}{1.000000,0.498039,0.054902}%
\pgfsetfillcolor{currentfill}%
\pgfsetlinewidth{0.481800pt}%
\definecolor{currentstroke}{rgb}{1.000000,1.000000,1.000000}%
\pgfsetstrokecolor{currentstroke}%
\pgfsetdash{}{0pt}%
\pgfpathmoveto{\pgfqpoint{1.405244in}{5.747913in}}%
\pgfpathcurveto{\pgfqpoint{1.416294in}{5.747913in}}{\pgfqpoint{1.426893in}{5.752303in}}{\pgfqpoint{1.434706in}{5.760117in}}%
\pgfpathcurveto{\pgfqpoint{1.442520in}{5.767931in}}{\pgfqpoint{1.446910in}{5.778530in}}{\pgfqpoint{1.446910in}{5.789580in}}%
\pgfpathcurveto{\pgfqpoint{1.446910in}{5.800630in}}{\pgfqpoint{1.442520in}{5.811229in}}{\pgfqpoint{1.434706in}{5.819043in}}%
\pgfpathcurveto{\pgfqpoint{1.426893in}{5.826856in}}{\pgfqpoint{1.416294in}{5.831247in}}{\pgfqpoint{1.405244in}{5.831247in}}%
\pgfpathcurveto{\pgfqpoint{1.394193in}{5.831247in}}{\pgfqpoint{1.383594in}{5.826856in}}{\pgfqpoint{1.375781in}{5.819043in}}%
\pgfpathcurveto{\pgfqpoint{1.367967in}{5.811229in}}{\pgfqpoint{1.363577in}{5.800630in}}{\pgfqpoint{1.363577in}{5.789580in}}%
\pgfpathcurveto{\pgfqpoint{1.363577in}{5.778530in}}{\pgfqpoint{1.367967in}{5.767931in}}{\pgfqpoint{1.375781in}{5.760117in}}%
\pgfpathcurveto{\pgfqpoint{1.383594in}{5.752303in}}{\pgfqpoint{1.394193in}{5.747913in}}{\pgfqpoint{1.405244in}{5.747913in}}%
\pgfpathlineto{\pgfqpoint{1.405244in}{5.747913in}}%
\pgfpathclose%
\pgfusepath{stroke,fill}%
\end{pgfscope}%
\begin{pgfscope}%
\pgfpathrectangle{\pgfqpoint{0.633874in}{5.272501in}}{\pgfqpoint{2.177280in}{2.201755in}}%
\pgfusepath{clip}%
\pgfsetbuttcap%
\pgfsetroundjoin%
\definecolor{currentfill}{rgb}{1.000000,0.498039,0.054902}%
\pgfsetfillcolor{currentfill}%
\pgfsetlinewidth{0.481800pt}%
\definecolor{currentstroke}{rgb}{1.000000,1.000000,1.000000}%
\pgfsetstrokecolor{currentstroke}%
\pgfsetdash{}{0pt}%
\pgfpathmoveto{\pgfqpoint{1.405244in}{5.831313in}}%
\pgfpathcurveto{\pgfqpoint{1.416294in}{5.831313in}}{\pgfqpoint{1.426893in}{5.835703in}}{\pgfqpoint{1.434706in}{5.843517in}}%
\pgfpathcurveto{\pgfqpoint{1.442520in}{5.851331in}}{\pgfqpoint{1.446910in}{5.861930in}}{\pgfqpoint{1.446910in}{5.872980in}}%
\pgfpathcurveto{\pgfqpoint{1.446910in}{5.884030in}}{\pgfqpoint{1.442520in}{5.894629in}}{\pgfqpoint{1.434706in}{5.902442in}}%
\pgfpathcurveto{\pgfqpoint{1.426893in}{5.910256in}}{\pgfqpoint{1.416294in}{5.914646in}}{\pgfqpoint{1.405244in}{5.914646in}}%
\pgfpathcurveto{\pgfqpoint{1.394193in}{5.914646in}}{\pgfqpoint{1.383594in}{5.910256in}}{\pgfqpoint{1.375781in}{5.902442in}}%
\pgfpathcurveto{\pgfqpoint{1.367967in}{5.894629in}}{\pgfqpoint{1.363577in}{5.884030in}}{\pgfqpoint{1.363577in}{5.872980in}}%
\pgfpathcurveto{\pgfqpoint{1.363577in}{5.861930in}}{\pgfqpoint{1.367967in}{5.851331in}}{\pgfqpoint{1.375781in}{5.843517in}}%
\pgfpathcurveto{\pgfqpoint{1.383594in}{5.835703in}}{\pgfqpoint{1.394193in}{5.831313in}}{\pgfqpoint{1.405244in}{5.831313in}}%
\pgfpathlineto{\pgfqpoint{1.405244in}{5.831313in}}%
\pgfpathclose%
\pgfusepath{stroke,fill}%
\end{pgfscope}%
\begin{pgfscope}%
\pgfpathrectangle{\pgfqpoint{0.633874in}{5.272501in}}{\pgfqpoint{2.177280in}{2.201755in}}%
\pgfusepath{clip}%
\pgfsetbuttcap%
\pgfsetroundjoin%
\definecolor{currentfill}{rgb}{1.000000,0.498039,0.054902}%
\pgfsetfillcolor{currentfill}%
\pgfsetlinewidth{0.481800pt}%
\definecolor{currentstroke}{rgb}{1.000000,1.000000,1.000000}%
\pgfsetstrokecolor{currentstroke}%
\pgfsetdash{}{0pt}%
\pgfpathmoveto{\pgfqpoint{1.644876in}{6.164912in}}%
\pgfpathcurveto{\pgfqpoint{1.655926in}{6.164912in}}{\pgfqpoint{1.666525in}{6.169303in}}{\pgfqpoint{1.674339in}{6.177116in}}%
\pgfpathcurveto{\pgfqpoint{1.682152in}{6.184930in}}{\pgfqpoint{1.686543in}{6.195529in}}{\pgfqpoint{1.686543in}{6.206579in}}%
\pgfpathcurveto{\pgfqpoint{1.686543in}{6.217629in}}{\pgfqpoint{1.682152in}{6.228228in}}{\pgfqpoint{1.674339in}{6.236042in}}%
\pgfpathcurveto{\pgfqpoint{1.666525in}{6.243855in}}{\pgfqpoint{1.655926in}{6.248246in}}{\pgfqpoint{1.644876in}{6.248246in}}%
\pgfpathcurveto{\pgfqpoint{1.633826in}{6.248246in}}{\pgfqpoint{1.623227in}{6.243855in}}{\pgfqpoint{1.615413in}{6.236042in}}%
\pgfpathcurveto{\pgfqpoint{1.607599in}{6.228228in}}{\pgfqpoint{1.603209in}{6.217629in}}{\pgfqpoint{1.603209in}{6.206579in}}%
\pgfpathcurveto{\pgfqpoint{1.603209in}{6.195529in}}{\pgfqpoint{1.607599in}{6.184930in}}{\pgfqpoint{1.615413in}{6.177116in}}%
\pgfpathcurveto{\pgfqpoint{1.623227in}{6.169303in}}{\pgfqpoint{1.633826in}{6.164912in}}{\pgfqpoint{1.644876in}{6.164912in}}%
\pgfpathlineto{\pgfqpoint{1.644876in}{6.164912in}}%
\pgfpathclose%
\pgfusepath{stroke,fill}%
\end{pgfscope}%
\begin{pgfscope}%
\pgfpathrectangle{\pgfqpoint{0.633874in}{5.272501in}}{\pgfqpoint{2.177280in}{2.201755in}}%
\pgfusepath{clip}%
\pgfsetbuttcap%
\pgfsetroundjoin%
\definecolor{currentfill}{rgb}{1.000000,0.498039,0.054902}%
\pgfsetfillcolor{currentfill}%
\pgfsetlinewidth{0.481800pt}%
\definecolor{currentstroke}{rgb}{1.000000,1.000000,1.000000}%
\pgfsetstrokecolor{currentstroke}%
\pgfsetdash{}{0pt}%
\pgfpathmoveto{\pgfqpoint{1.525060in}{5.831313in}}%
\pgfpathcurveto{\pgfqpoint{1.536110in}{5.831313in}}{\pgfqpoint{1.546709in}{5.835703in}}{\pgfqpoint{1.554523in}{5.843517in}}%
\pgfpathcurveto{\pgfqpoint{1.562336in}{5.851331in}}{\pgfqpoint{1.566726in}{5.861930in}}{\pgfqpoint{1.566726in}{5.872980in}}%
\pgfpathcurveto{\pgfqpoint{1.566726in}{5.884030in}}{\pgfqpoint{1.562336in}{5.894629in}}{\pgfqpoint{1.554523in}{5.902442in}}%
\pgfpathcurveto{\pgfqpoint{1.546709in}{5.910256in}}{\pgfqpoint{1.536110in}{5.914646in}}{\pgfqpoint{1.525060in}{5.914646in}}%
\pgfpathcurveto{\pgfqpoint{1.514010in}{5.914646in}}{\pgfqpoint{1.503411in}{5.910256in}}{\pgfqpoint{1.495597in}{5.902442in}}%
\pgfpathcurveto{\pgfqpoint{1.487783in}{5.894629in}}{\pgfqpoint{1.483393in}{5.884030in}}{\pgfqpoint{1.483393in}{5.872980in}}%
\pgfpathcurveto{\pgfqpoint{1.483393in}{5.861930in}}{\pgfqpoint{1.487783in}{5.851331in}}{\pgfqpoint{1.495597in}{5.843517in}}%
\pgfpathcurveto{\pgfqpoint{1.503411in}{5.835703in}}{\pgfqpoint{1.514010in}{5.831313in}}{\pgfqpoint{1.525060in}{5.831313in}}%
\pgfpathlineto{\pgfqpoint{1.525060in}{5.831313in}}%
\pgfpathclose%
\pgfusepath{stroke,fill}%
\end{pgfscope}%
\begin{pgfscope}%
\pgfpathrectangle{\pgfqpoint{0.633874in}{5.272501in}}{\pgfqpoint{2.177280in}{2.201755in}}%
\pgfusepath{clip}%
\pgfsetbuttcap%
\pgfsetroundjoin%
\definecolor{currentfill}{rgb}{1.000000,0.498039,0.054902}%
\pgfsetfillcolor{currentfill}%
\pgfsetlinewidth{0.481800pt}%
\definecolor{currentstroke}{rgb}{1.000000,1.000000,1.000000}%
\pgfsetstrokecolor{currentstroke}%
\pgfsetdash{}{0pt}%
\pgfpathmoveto{\pgfqpoint{1.205550in}{5.581114in}}%
\pgfpathcurveto{\pgfqpoint{1.216600in}{5.581114in}}{\pgfqpoint{1.227199in}{5.585504in}}{\pgfqpoint{1.235013in}{5.593317in}}%
\pgfpathcurveto{\pgfqpoint{1.242826in}{5.601131in}}{\pgfqpoint{1.247217in}{5.611730in}}{\pgfqpoint{1.247217in}{5.622780in}}%
\pgfpathcurveto{\pgfqpoint{1.247217in}{5.633830in}}{\pgfqpoint{1.242826in}{5.644429in}}{\pgfqpoint{1.235013in}{5.652243in}}%
\pgfpathcurveto{\pgfqpoint{1.227199in}{5.660057in}}{\pgfqpoint{1.216600in}{5.664447in}}{\pgfqpoint{1.205550in}{5.664447in}}%
\pgfpathcurveto{\pgfqpoint{1.194500in}{5.664447in}}{\pgfqpoint{1.183901in}{5.660057in}}{\pgfqpoint{1.176087in}{5.652243in}}%
\pgfpathcurveto{\pgfqpoint{1.168274in}{5.644429in}}{\pgfqpoint{1.163883in}{5.633830in}}{\pgfqpoint{1.163883in}{5.622780in}}%
\pgfpathcurveto{\pgfqpoint{1.163883in}{5.611730in}}{\pgfqpoint{1.168274in}{5.601131in}}{\pgfqpoint{1.176087in}{5.593317in}}%
\pgfpathcurveto{\pgfqpoint{1.183901in}{5.585504in}}{\pgfqpoint{1.194500in}{5.581114in}}{\pgfqpoint{1.205550in}{5.581114in}}%
\pgfpathlineto{\pgfqpoint{1.205550in}{5.581114in}}%
\pgfpathclose%
\pgfusepath{stroke,fill}%
\end{pgfscope}%
\begin{pgfscope}%
\pgfpathrectangle{\pgfqpoint{0.633874in}{5.272501in}}{\pgfqpoint{2.177280in}{2.201755in}}%
\pgfusepath{clip}%
\pgfsetbuttcap%
\pgfsetroundjoin%
\definecolor{currentfill}{rgb}{1.000000,0.498039,0.054902}%
\pgfsetfillcolor{currentfill}%
\pgfsetlinewidth{0.481800pt}%
\definecolor{currentstroke}{rgb}{1.000000,1.000000,1.000000}%
\pgfsetstrokecolor{currentstroke}%
\pgfsetdash{}{0pt}%
\pgfpathmoveto{\pgfqpoint{1.445182in}{5.914713in}}%
\pgfpathcurveto{\pgfqpoint{1.456232in}{5.914713in}}{\pgfqpoint{1.466831in}{5.919103in}}{\pgfqpoint{1.474645in}{5.926917in}}%
\pgfpathcurveto{\pgfqpoint{1.482459in}{5.934730in}}{\pgfqpoint{1.486849in}{5.945329in}}{\pgfqpoint{1.486849in}{5.956379in}}%
\pgfpathcurveto{\pgfqpoint{1.486849in}{5.967430in}}{\pgfqpoint{1.482459in}{5.978029in}}{\pgfqpoint{1.474645in}{5.985842in}}%
\pgfpathcurveto{\pgfqpoint{1.466831in}{5.993656in}}{\pgfqpoint{1.456232in}{5.998046in}}{\pgfqpoint{1.445182in}{5.998046in}}%
\pgfpathcurveto{\pgfqpoint{1.434132in}{5.998046in}}{\pgfqpoint{1.423533in}{5.993656in}}{\pgfqpoint{1.415720in}{5.985842in}}%
\pgfpathcurveto{\pgfqpoint{1.407906in}{5.978029in}}{\pgfqpoint{1.403516in}{5.967430in}}{\pgfqpoint{1.403516in}{5.956379in}}%
\pgfpathcurveto{\pgfqpoint{1.403516in}{5.945329in}}{\pgfqpoint{1.407906in}{5.934730in}}{\pgfqpoint{1.415720in}{5.926917in}}%
\pgfpathcurveto{\pgfqpoint{1.423533in}{5.919103in}}{\pgfqpoint{1.434132in}{5.914713in}}{\pgfqpoint{1.445182in}{5.914713in}}%
\pgfpathlineto{\pgfqpoint{1.445182in}{5.914713in}}%
\pgfpathclose%
\pgfusepath{stroke,fill}%
\end{pgfscope}%
\begin{pgfscope}%
\pgfpathrectangle{\pgfqpoint{0.633874in}{5.272501in}}{\pgfqpoint{2.177280in}{2.201755in}}%
\pgfusepath{clip}%
\pgfsetbuttcap%
\pgfsetroundjoin%
\definecolor{currentfill}{rgb}{1.000000,0.498039,0.054902}%
\pgfsetfillcolor{currentfill}%
\pgfsetlinewidth{0.481800pt}%
\definecolor{currentstroke}{rgb}{1.000000,1.000000,1.000000}%
\pgfsetstrokecolor{currentstroke}%
\pgfsetdash{}{0pt}%
\pgfpathmoveto{\pgfqpoint{1.485121in}{6.164912in}}%
\pgfpathcurveto{\pgfqpoint{1.496171in}{6.164912in}}{\pgfqpoint{1.506770in}{6.169303in}}{\pgfqpoint{1.514584in}{6.177116in}}%
\pgfpathcurveto{\pgfqpoint{1.522397in}{6.184930in}}{\pgfqpoint{1.526788in}{6.195529in}}{\pgfqpoint{1.526788in}{6.206579in}}%
\pgfpathcurveto{\pgfqpoint{1.526788in}{6.217629in}}{\pgfqpoint{1.522397in}{6.228228in}}{\pgfqpoint{1.514584in}{6.236042in}}%
\pgfpathcurveto{\pgfqpoint{1.506770in}{6.243855in}}{\pgfqpoint{1.496171in}{6.248246in}}{\pgfqpoint{1.485121in}{6.248246in}}%
\pgfpathcurveto{\pgfqpoint{1.474071in}{6.248246in}}{\pgfqpoint{1.463472in}{6.243855in}}{\pgfqpoint{1.455658in}{6.236042in}}%
\pgfpathcurveto{\pgfqpoint{1.447845in}{6.228228in}}{\pgfqpoint{1.443454in}{6.217629in}}{\pgfqpoint{1.443454in}{6.206579in}}%
\pgfpathcurveto{\pgfqpoint{1.443454in}{6.195529in}}{\pgfqpoint{1.447845in}{6.184930in}}{\pgfqpoint{1.455658in}{6.177116in}}%
\pgfpathcurveto{\pgfqpoint{1.463472in}{6.169303in}}{\pgfqpoint{1.474071in}{6.164912in}}{\pgfqpoint{1.485121in}{6.164912in}}%
\pgfpathlineto{\pgfqpoint{1.485121in}{6.164912in}}%
\pgfpathclose%
\pgfusepath{stroke,fill}%
\end{pgfscope}%
\begin{pgfscope}%
\pgfpathrectangle{\pgfqpoint{0.633874in}{5.272501in}}{\pgfqpoint{2.177280in}{2.201755in}}%
\pgfusepath{clip}%
\pgfsetbuttcap%
\pgfsetroundjoin%
\definecolor{currentfill}{rgb}{1.000000,0.498039,0.054902}%
\pgfsetfillcolor{currentfill}%
\pgfsetlinewidth{0.481800pt}%
\definecolor{currentstroke}{rgb}{1.000000,1.000000,1.000000}%
\pgfsetstrokecolor{currentstroke}%
\pgfsetdash{}{0pt}%
\pgfpathmoveto{\pgfqpoint{1.485121in}{6.081512in}}%
\pgfpathcurveto{\pgfqpoint{1.496171in}{6.081512in}}{\pgfqpoint{1.506770in}{6.085903in}}{\pgfqpoint{1.514584in}{6.093716in}}%
\pgfpathcurveto{\pgfqpoint{1.522397in}{6.101530in}}{\pgfqpoint{1.526788in}{6.112129in}}{\pgfqpoint{1.526788in}{6.123179in}}%
\pgfpathcurveto{\pgfqpoint{1.526788in}{6.134229in}}{\pgfqpoint{1.522397in}{6.144828in}}{\pgfqpoint{1.514584in}{6.152642in}}%
\pgfpathcurveto{\pgfqpoint{1.506770in}{6.160456in}}{\pgfqpoint{1.496171in}{6.164846in}}{\pgfqpoint{1.485121in}{6.164846in}}%
\pgfpathcurveto{\pgfqpoint{1.474071in}{6.164846in}}{\pgfqpoint{1.463472in}{6.160456in}}{\pgfqpoint{1.455658in}{6.152642in}}%
\pgfpathcurveto{\pgfqpoint{1.447845in}{6.144828in}}{\pgfqpoint{1.443454in}{6.134229in}}{\pgfqpoint{1.443454in}{6.123179in}}%
\pgfpathcurveto{\pgfqpoint{1.443454in}{6.112129in}}{\pgfqpoint{1.447845in}{6.101530in}}{\pgfqpoint{1.455658in}{6.093716in}}%
\pgfpathcurveto{\pgfqpoint{1.463472in}{6.085903in}}{\pgfqpoint{1.474071in}{6.081512in}}{\pgfqpoint{1.485121in}{6.081512in}}%
\pgfpathlineto{\pgfqpoint{1.485121in}{6.081512in}}%
\pgfpathclose%
\pgfusepath{stroke,fill}%
\end{pgfscope}%
\begin{pgfscope}%
\pgfpathrectangle{\pgfqpoint{0.633874in}{5.272501in}}{\pgfqpoint{2.177280in}{2.201755in}}%
\pgfusepath{clip}%
\pgfsetbuttcap%
\pgfsetroundjoin%
\definecolor{currentfill}{rgb}{1.000000,0.498039,0.054902}%
\pgfsetfillcolor{currentfill}%
\pgfsetlinewidth{0.481800pt}%
\definecolor{currentstroke}{rgb}{1.000000,1.000000,1.000000}%
\pgfsetstrokecolor{currentstroke}%
\pgfsetdash{}{0pt}%
\pgfpathmoveto{\pgfqpoint{1.684815in}{6.081512in}}%
\pgfpathcurveto{\pgfqpoint{1.695865in}{6.081512in}}{\pgfqpoint{1.706464in}{6.085903in}}{\pgfqpoint{1.714277in}{6.093716in}}%
\pgfpathcurveto{\pgfqpoint{1.722091in}{6.101530in}}{\pgfqpoint{1.726481in}{6.112129in}}{\pgfqpoint{1.726481in}{6.123179in}}%
\pgfpathcurveto{\pgfqpoint{1.726481in}{6.134229in}}{\pgfqpoint{1.722091in}{6.144828in}}{\pgfqpoint{1.714277in}{6.152642in}}%
\pgfpathcurveto{\pgfqpoint{1.706464in}{6.160456in}}{\pgfqpoint{1.695865in}{6.164846in}}{\pgfqpoint{1.684815in}{6.164846in}}%
\pgfpathcurveto{\pgfqpoint{1.673764in}{6.164846in}}{\pgfqpoint{1.663165in}{6.160456in}}{\pgfqpoint{1.655352in}{6.152642in}}%
\pgfpathcurveto{\pgfqpoint{1.647538in}{6.144828in}}{\pgfqpoint{1.643148in}{6.134229in}}{\pgfqpoint{1.643148in}{6.123179in}}%
\pgfpathcurveto{\pgfqpoint{1.643148in}{6.112129in}}{\pgfqpoint{1.647538in}{6.101530in}}{\pgfqpoint{1.655352in}{6.093716in}}%
\pgfpathcurveto{\pgfqpoint{1.663165in}{6.085903in}}{\pgfqpoint{1.673764in}{6.081512in}}{\pgfqpoint{1.684815in}{6.081512in}}%
\pgfpathlineto{\pgfqpoint{1.684815in}{6.081512in}}%
\pgfpathclose%
\pgfusepath{stroke,fill}%
\end{pgfscope}%
\begin{pgfscope}%
\pgfpathrectangle{\pgfqpoint{0.633874in}{5.272501in}}{\pgfqpoint{2.177280in}{2.201755in}}%
\pgfusepath{clip}%
\pgfsetbuttcap%
\pgfsetroundjoin%
\definecolor{currentfill}{rgb}{1.000000,0.498039,0.054902}%
\pgfsetfillcolor{currentfill}%
\pgfsetlinewidth{0.481800pt}%
\definecolor{currentstroke}{rgb}{1.000000,1.000000,1.000000}%
\pgfsetstrokecolor{currentstroke}%
\pgfsetdash{}{0pt}%
\pgfpathmoveto{\pgfqpoint{1.245489in}{5.747913in}}%
\pgfpathcurveto{\pgfqpoint{1.256539in}{5.747913in}}{\pgfqpoint{1.267138in}{5.752303in}}{\pgfqpoint{1.274952in}{5.760117in}}%
\pgfpathcurveto{\pgfqpoint{1.282765in}{5.767931in}}{\pgfqpoint{1.287155in}{5.778530in}}{\pgfqpoint{1.287155in}{5.789580in}}%
\pgfpathcurveto{\pgfqpoint{1.287155in}{5.800630in}}{\pgfqpoint{1.282765in}{5.811229in}}{\pgfqpoint{1.274952in}{5.819043in}}%
\pgfpathcurveto{\pgfqpoint{1.267138in}{5.826856in}}{\pgfqpoint{1.256539in}{5.831247in}}{\pgfqpoint{1.245489in}{5.831247in}}%
\pgfpathcurveto{\pgfqpoint{1.234439in}{5.831247in}}{\pgfqpoint{1.223840in}{5.826856in}}{\pgfqpoint{1.216026in}{5.819043in}}%
\pgfpathcurveto{\pgfqpoint{1.208212in}{5.811229in}}{\pgfqpoint{1.203822in}{5.800630in}}{\pgfqpoint{1.203822in}{5.789580in}}%
\pgfpathcurveto{\pgfqpoint{1.203822in}{5.778530in}}{\pgfqpoint{1.208212in}{5.767931in}}{\pgfqpoint{1.216026in}{5.760117in}}%
\pgfpathcurveto{\pgfqpoint{1.223840in}{5.752303in}}{\pgfqpoint{1.234439in}{5.747913in}}{\pgfqpoint{1.245489in}{5.747913in}}%
\pgfpathlineto{\pgfqpoint{1.245489in}{5.747913in}}%
\pgfpathclose%
\pgfusepath{stroke,fill}%
\end{pgfscope}%
\begin{pgfscope}%
\pgfpathrectangle{\pgfqpoint{0.633874in}{5.272501in}}{\pgfqpoint{2.177280in}{2.201755in}}%
\pgfusepath{clip}%
\pgfsetbuttcap%
\pgfsetroundjoin%
\definecolor{currentfill}{rgb}{1.000000,0.498039,0.054902}%
\pgfsetfillcolor{currentfill}%
\pgfsetlinewidth{0.481800pt}%
\definecolor{currentstroke}{rgb}{1.000000,1.000000,1.000000}%
\pgfsetstrokecolor{currentstroke}%
\pgfsetdash{}{0pt}%
\pgfpathmoveto{\pgfqpoint{1.485121in}{5.998113in}}%
\pgfpathcurveto{\pgfqpoint{1.496171in}{5.998113in}}{\pgfqpoint{1.506770in}{6.002503in}}{\pgfqpoint{1.514584in}{6.010317in}}%
\pgfpathcurveto{\pgfqpoint{1.522397in}{6.018130in}}{\pgfqpoint{1.526788in}{6.028729in}}{\pgfqpoint{1.526788in}{6.039779in}}%
\pgfpathcurveto{\pgfqpoint{1.526788in}{6.050829in}}{\pgfqpoint{1.522397in}{6.061428in}}{\pgfqpoint{1.514584in}{6.069242in}}%
\pgfpathcurveto{\pgfqpoint{1.506770in}{6.077056in}}{\pgfqpoint{1.496171in}{6.081446in}}{\pgfqpoint{1.485121in}{6.081446in}}%
\pgfpathcurveto{\pgfqpoint{1.474071in}{6.081446in}}{\pgfqpoint{1.463472in}{6.077056in}}{\pgfqpoint{1.455658in}{6.069242in}}%
\pgfpathcurveto{\pgfqpoint{1.447845in}{6.061428in}}{\pgfqpoint{1.443454in}{6.050829in}}{\pgfqpoint{1.443454in}{6.039779in}}%
\pgfpathcurveto{\pgfqpoint{1.443454in}{6.028729in}}{\pgfqpoint{1.447845in}{6.018130in}}{\pgfqpoint{1.455658in}{6.010317in}}%
\pgfpathcurveto{\pgfqpoint{1.463472in}{6.002503in}}{\pgfqpoint{1.474071in}{5.998113in}}{\pgfqpoint{1.485121in}{5.998113in}}%
\pgfpathlineto{\pgfqpoint{1.485121in}{5.998113in}}%
\pgfpathclose%
\pgfusepath{stroke,fill}%
\end{pgfscope}%
\begin{pgfscope}%
\pgfpathrectangle{\pgfqpoint{0.633874in}{5.272501in}}{\pgfqpoint{2.177280in}{2.201755in}}%
\pgfusepath{clip}%
\pgfsetbuttcap%
\pgfsetroundjoin%
\definecolor{currentfill}{rgb}{0.172549,0.627451,0.172549}%
\pgfsetfillcolor{currentfill}%
\pgfsetlinewidth{0.481800pt}%
\definecolor{currentstroke}{rgb}{1.000000,1.000000,1.000000}%
\pgfsetstrokecolor{currentstroke}%
\pgfsetdash{}{0pt}%
\pgfpathmoveto{\pgfqpoint{1.724753in}{6.415112in}}%
\pgfpathcurveto{\pgfqpoint{1.735803in}{6.415112in}}{\pgfqpoint{1.746402in}{6.419502in}}{\pgfqpoint{1.754216in}{6.427316in}}%
\pgfpathcurveto{\pgfqpoint{1.762030in}{6.435129in}}{\pgfqpoint{1.766420in}{6.445728in}}{\pgfqpoint{1.766420in}{6.456778in}}%
\pgfpathcurveto{\pgfqpoint{1.766420in}{6.467828in}}{\pgfqpoint{1.762030in}{6.478428in}}{\pgfqpoint{1.754216in}{6.486241in}}%
\pgfpathcurveto{\pgfqpoint{1.746402in}{6.494055in}}{\pgfqpoint{1.735803in}{6.498445in}}{\pgfqpoint{1.724753in}{6.498445in}}%
\pgfpathcurveto{\pgfqpoint{1.713703in}{6.498445in}}{\pgfqpoint{1.703104in}{6.494055in}}{\pgfqpoint{1.695290in}{6.486241in}}%
\pgfpathcurveto{\pgfqpoint{1.687477in}{6.478428in}}{\pgfqpoint{1.683087in}{6.467828in}}{\pgfqpoint{1.683087in}{6.456778in}}%
\pgfpathcurveto{\pgfqpoint{1.683087in}{6.445728in}}{\pgfqpoint{1.687477in}{6.435129in}}{\pgfqpoint{1.695290in}{6.427316in}}%
\pgfpathcurveto{\pgfqpoint{1.703104in}{6.419502in}}{\pgfqpoint{1.713703in}{6.415112in}}{\pgfqpoint{1.724753in}{6.415112in}}%
\pgfpathlineto{\pgfqpoint{1.724753in}{6.415112in}}%
\pgfpathclose%
\pgfusepath{stroke,fill}%
\end{pgfscope}%
\begin{pgfscope}%
\pgfpathrectangle{\pgfqpoint{0.633874in}{5.272501in}}{\pgfqpoint{2.177280in}{2.201755in}}%
\pgfusepath{clip}%
\pgfsetbuttcap%
\pgfsetroundjoin%
\definecolor{currentfill}{rgb}{0.172549,0.627451,0.172549}%
\pgfsetfillcolor{currentfill}%
\pgfsetlinewidth{0.481800pt}%
\definecolor{currentstroke}{rgb}{1.000000,1.000000,1.000000}%
\pgfsetstrokecolor{currentstroke}%
\pgfsetdash{}{0pt}%
\pgfpathmoveto{\pgfqpoint{1.525060in}{5.914713in}}%
\pgfpathcurveto{\pgfqpoint{1.536110in}{5.914713in}}{\pgfqpoint{1.546709in}{5.919103in}}{\pgfqpoint{1.554523in}{5.926917in}}%
\pgfpathcurveto{\pgfqpoint{1.562336in}{5.934730in}}{\pgfqpoint{1.566726in}{5.945329in}}{\pgfqpoint{1.566726in}{5.956379in}}%
\pgfpathcurveto{\pgfqpoint{1.566726in}{5.967430in}}{\pgfqpoint{1.562336in}{5.978029in}}{\pgfqpoint{1.554523in}{5.985842in}}%
\pgfpathcurveto{\pgfqpoint{1.546709in}{5.993656in}}{\pgfqpoint{1.536110in}{5.998046in}}{\pgfqpoint{1.525060in}{5.998046in}}%
\pgfpathcurveto{\pgfqpoint{1.514010in}{5.998046in}}{\pgfqpoint{1.503411in}{5.993656in}}{\pgfqpoint{1.495597in}{5.985842in}}%
\pgfpathcurveto{\pgfqpoint{1.487783in}{5.978029in}}{\pgfqpoint{1.483393in}{5.967430in}}{\pgfqpoint{1.483393in}{5.956379in}}%
\pgfpathcurveto{\pgfqpoint{1.483393in}{5.945329in}}{\pgfqpoint{1.487783in}{5.934730in}}{\pgfqpoint{1.495597in}{5.926917in}}%
\pgfpathcurveto{\pgfqpoint{1.503411in}{5.919103in}}{\pgfqpoint{1.514010in}{5.914713in}}{\pgfqpoint{1.525060in}{5.914713in}}%
\pgfpathlineto{\pgfqpoint{1.525060in}{5.914713in}}%
\pgfpathclose%
\pgfusepath{stroke,fill}%
\end{pgfscope}%
\begin{pgfscope}%
\pgfpathrectangle{\pgfqpoint{0.633874in}{5.272501in}}{\pgfqpoint{2.177280in}{2.201755in}}%
\pgfusepath{clip}%
\pgfsetbuttcap%
\pgfsetroundjoin%
\definecolor{currentfill}{rgb}{0.172549,0.627451,0.172549}%
\pgfsetfillcolor{currentfill}%
\pgfsetlinewidth{0.481800pt}%
\definecolor{currentstroke}{rgb}{1.000000,1.000000,1.000000}%
\pgfsetstrokecolor{currentstroke}%
\pgfsetdash{}{0pt}%
\pgfpathmoveto{\pgfqpoint{2.044263in}{6.164912in}}%
\pgfpathcurveto{\pgfqpoint{2.055313in}{6.164912in}}{\pgfqpoint{2.065912in}{6.169303in}}{\pgfqpoint{2.073726in}{6.177116in}}%
\pgfpathcurveto{\pgfqpoint{2.081539in}{6.184930in}}{\pgfqpoint{2.085930in}{6.195529in}}{\pgfqpoint{2.085930in}{6.206579in}}%
\pgfpathcurveto{\pgfqpoint{2.085930in}{6.217629in}}{\pgfqpoint{2.081539in}{6.228228in}}{\pgfqpoint{2.073726in}{6.236042in}}%
\pgfpathcurveto{\pgfqpoint{2.065912in}{6.243855in}}{\pgfqpoint{2.055313in}{6.248246in}}{\pgfqpoint{2.044263in}{6.248246in}}%
\pgfpathcurveto{\pgfqpoint{2.033213in}{6.248246in}}{\pgfqpoint{2.022614in}{6.243855in}}{\pgfqpoint{2.014800in}{6.236042in}}%
\pgfpathcurveto{\pgfqpoint{2.006986in}{6.228228in}}{\pgfqpoint{2.002596in}{6.217629in}}{\pgfqpoint{2.002596in}{6.206579in}}%
\pgfpathcurveto{\pgfqpoint{2.002596in}{6.195529in}}{\pgfqpoint{2.006986in}{6.184930in}}{\pgfqpoint{2.014800in}{6.177116in}}%
\pgfpathcurveto{\pgfqpoint{2.022614in}{6.169303in}}{\pgfqpoint{2.033213in}{6.164912in}}{\pgfqpoint{2.044263in}{6.164912in}}%
\pgfpathlineto{\pgfqpoint{2.044263in}{6.164912in}}%
\pgfpathclose%
\pgfusepath{stroke,fill}%
\end{pgfscope}%
\begin{pgfscope}%
\pgfpathrectangle{\pgfqpoint{0.633874in}{5.272501in}}{\pgfqpoint{2.177280in}{2.201755in}}%
\pgfusepath{clip}%
\pgfsetbuttcap%
\pgfsetroundjoin%
\definecolor{currentfill}{rgb}{0.172549,0.627451,0.172549}%
\pgfsetfillcolor{currentfill}%
\pgfsetlinewidth{0.481800pt}%
\definecolor{currentstroke}{rgb}{1.000000,1.000000,1.000000}%
\pgfsetstrokecolor{currentstroke}%
\pgfsetdash{}{0pt}%
\pgfpathmoveto{\pgfqpoint{1.724753in}{6.081512in}}%
\pgfpathcurveto{\pgfqpoint{1.735803in}{6.081512in}}{\pgfqpoint{1.746402in}{6.085903in}}{\pgfqpoint{1.754216in}{6.093716in}}%
\pgfpathcurveto{\pgfqpoint{1.762030in}{6.101530in}}{\pgfqpoint{1.766420in}{6.112129in}}{\pgfqpoint{1.766420in}{6.123179in}}%
\pgfpathcurveto{\pgfqpoint{1.766420in}{6.134229in}}{\pgfqpoint{1.762030in}{6.144828in}}{\pgfqpoint{1.754216in}{6.152642in}}%
\pgfpathcurveto{\pgfqpoint{1.746402in}{6.160456in}}{\pgfqpoint{1.735803in}{6.164846in}}{\pgfqpoint{1.724753in}{6.164846in}}%
\pgfpathcurveto{\pgfqpoint{1.713703in}{6.164846in}}{\pgfqpoint{1.703104in}{6.160456in}}{\pgfqpoint{1.695290in}{6.152642in}}%
\pgfpathcurveto{\pgfqpoint{1.687477in}{6.144828in}}{\pgfqpoint{1.683087in}{6.134229in}}{\pgfqpoint{1.683087in}{6.123179in}}%
\pgfpathcurveto{\pgfqpoint{1.683087in}{6.112129in}}{\pgfqpoint{1.687477in}{6.101530in}}{\pgfqpoint{1.695290in}{6.093716in}}%
\pgfpathcurveto{\pgfqpoint{1.703104in}{6.085903in}}{\pgfqpoint{1.713703in}{6.081512in}}{\pgfqpoint{1.724753in}{6.081512in}}%
\pgfpathlineto{\pgfqpoint{1.724753in}{6.081512in}}%
\pgfpathclose%
\pgfusepath{stroke,fill}%
\end{pgfscope}%
\begin{pgfscope}%
\pgfpathrectangle{\pgfqpoint{0.633874in}{5.272501in}}{\pgfqpoint{2.177280in}{2.201755in}}%
\pgfusepath{clip}%
\pgfsetbuttcap%
\pgfsetroundjoin%
\definecolor{currentfill}{rgb}{0.172549,0.627451,0.172549}%
\pgfsetfillcolor{currentfill}%
\pgfsetlinewidth{0.481800pt}%
\definecolor{currentstroke}{rgb}{1.000000,1.000000,1.000000}%
\pgfsetstrokecolor{currentstroke}%
\pgfsetdash{}{0pt}%
\pgfpathmoveto{\pgfqpoint{1.804631in}{6.164912in}}%
\pgfpathcurveto{\pgfqpoint{1.815681in}{6.164912in}}{\pgfqpoint{1.826280in}{6.169303in}}{\pgfqpoint{1.834093in}{6.177116in}}%
\pgfpathcurveto{\pgfqpoint{1.841907in}{6.184930in}}{\pgfqpoint{1.846297in}{6.195529in}}{\pgfqpoint{1.846297in}{6.206579in}}%
\pgfpathcurveto{\pgfqpoint{1.846297in}{6.217629in}}{\pgfqpoint{1.841907in}{6.228228in}}{\pgfqpoint{1.834093in}{6.236042in}}%
\pgfpathcurveto{\pgfqpoint{1.826280in}{6.243855in}}{\pgfqpoint{1.815681in}{6.248246in}}{\pgfqpoint{1.804631in}{6.248246in}}%
\pgfpathcurveto{\pgfqpoint{1.793581in}{6.248246in}}{\pgfqpoint{1.782981in}{6.243855in}}{\pgfqpoint{1.775168in}{6.236042in}}%
\pgfpathcurveto{\pgfqpoint{1.767354in}{6.228228in}}{\pgfqpoint{1.762964in}{6.217629in}}{\pgfqpoint{1.762964in}{6.206579in}}%
\pgfpathcurveto{\pgfqpoint{1.762964in}{6.195529in}}{\pgfqpoint{1.767354in}{6.184930in}}{\pgfqpoint{1.775168in}{6.177116in}}%
\pgfpathcurveto{\pgfqpoint{1.782981in}{6.169303in}}{\pgfqpoint{1.793581in}{6.164912in}}{\pgfqpoint{1.804631in}{6.164912in}}%
\pgfpathlineto{\pgfqpoint{1.804631in}{6.164912in}}%
\pgfpathclose%
\pgfusepath{stroke,fill}%
\end{pgfscope}%
\begin{pgfscope}%
\pgfpathrectangle{\pgfqpoint{0.633874in}{5.272501in}}{\pgfqpoint{2.177280in}{2.201755in}}%
\pgfusepath{clip}%
\pgfsetbuttcap%
\pgfsetroundjoin%
\definecolor{currentfill}{rgb}{0.172549,0.627451,0.172549}%
\pgfsetfillcolor{currentfill}%
\pgfsetlinewidth{0.481800pt}%
\definecolor{currentstroke}{rgb}{1.000000,1.000000,1.000000}%
\pgfsetstrokecolor{currentstroke}%
\pgfsetdash{}{0pt}%
\pgfpathmoveto{\pgfqpoint{2.243956in}{6.164912in}}%
\pgfpathcurveto{\pgfqpoint{2.255007in}{6.164912in}}{\pgfqpoint{2.265606in}{6.169303in}}{\pgfqpoint{2.273419in}{6.177116in}}%
\pgfpathcurveto{\pgfqpoint{2.281233in}{6.184930in}}{\pgfqpoint{2.285623in}{6.195529in}}{\pgfqpoint{2.285623in}{6.206579in}}%
\pgfpathcurveto{\pgfqpoint{2.285623in}{6.217629in}}{\pgfqpoint{2.281233in}{6.228228in}}{\pgfqpoint{2.273419in}{6.236042in}}%
\pgfpathcurveto{\pgfqpoint{2.265606in}{6.243855in}}{\pgfqpoint{2.255007in}{6.248246in}}{\pgfqpoint{2.243956in}{6.248246in}}%
\pgfpathcurveto{\pgfqpoint{2.232906in}{6.248246in}}{\pgfqpoint{2.222307in}{6.243855in}}{\pgfqpoint{2.214494in}{6.236042in}}%
\pgfpathcurveto{\pgfqpoint{2.206680in}{6.228228in}}{\pgfqpoint{2.202290in}{6.217629in}}{\pgfqpoint{2.202290in}{6.206579in}}%
\pgfpathcurveto{\pgfqpoint{2.202290in}{6.195529in}}{\pgfqpoint{2.206680in}{6.184930in}}{\pgfqpoint{2.214494in}{6.177116in}}%
\pgfpathcurveto{\pgfqpoint{2.222307in}{6.169303in}}{\pgfqpoint{2.232906in}{6.164912in}}{\pgfqpoint{2.243956in}{6.164912in}}%
\pgfpathlineto{\pgfqpoint{2.243956in}{6.164912in}}%
\pgfpathclose%
\pgfusepath{stroke,fill}%
\end{pgfscope}%
\begin{pgfscope}%
\pgfpathrectangle{\pgfqpoint{0.633874in}{5.272501in}}{\pgfqpoint{2.177280in}{2.201755in}}%
\pgfusepath{clip}%
\pgfsetbuttcap%
\pgfsetroundjoin%
\definecolor{currentfill}{rgb}{0.172549,0.627451,0.172549}%
\pgfsetfillcolor{currentfill}%
\pgfsetlinewidth{0.481800pt}%
\definecolor{currentstroke}{rgb}{1.000000,1.000000,1.000000}%
\pgfsetstrokecolor{currentstroke}%
\pgfsetdash{}{0pt}%
\pgfpathmoveto{\pgfqpoint{1.165611in}{5.747913in}}%
\pgfpathcurveto{\pgfqpoint{1.176662in}{5.747913in}}{\pgfqpoint{1.187261in}{5.752303in}}{\pgfqpoint{1.195074in}{5.760117in}}%
\pgfpathcurveto{\pgfqpoint{1.202888in}{5.767931in}}{\pgfqpoint{1.207278in}{5.778530in}}{\pgfqpoint{1.207278in}{5.789580in}}%
\pgfpathcurveto{\pgfqpoint{1.207278in}{5.800630in}}{\pgfqpoint{1.202888in}{5.811229in}}{\pgfqpoint{1.195074in}{5.819043in}}%
\pgfpathcurveto{\pgfqpoint{1.187261in}{5.826856in}}{\pgfqpoint{1.176662in}{5.831247in}}{\pgfqpoint{1.165611in}{5.831247in}}%
\pgfpathcurveto{\pgfqpoint{1.154561in}{5.831247in}}{\pgfqpoint{1.143962in}{5.826856in}}{\pgfqpoint{1.136149in}{5.819043in}}%
\pgfpathcurveto{\pgfqpoint{1.128335in}{5.811229in}}{\pgfqpoint{1.123945in}{5.800630in}}{\pgfqpoint{1.123945in}{5.789580in}}%
\pgfpathcurveto{\pgfqpoint{1.123945in}{5.778530in}}{\pgfqpoint{1.128335in}{5.767931in}}{\pgfqpoint{1.136149in}{5.760117in}}%
\pgfpathcurveto{\pgfqpoint{1.143962in}{5.752303in}}{\pgfqpoint{1.154561in}{5.747913in}}{\pgfqpoint{1.165611in}{5.747913in}}%
\pgfpathlineto{\pgfqpoint{1.165611in}{5.747913in}}%
\pgfpathclose%
\pgfusepath{stroke,fill}%
\end{pgfscope}%
\begin{pgfscope}%
\pgfpathrectangle{\pgfqpoint{0.633874in}{5.272501in}}{\pgfqpoint{2.177280in}{2.201755in}}%
\pgfusepath{clip}%
\pgfsetbuttcap%
\pgfsetroundjoin%
\definecolor{currentfill}{rgb}{0.172549,0.627451,0.172549}%
\pgfsetfillcolor{currentfill}%
\pgfsetlinewidth{0.481800pt}%
\definecolor{currentstroke}{rgb}{1.000000,1.000000,1.000000}%
\pgfsetstrokecolor{currentstroke}%
\pgfsetdash{}{0pt}%
\pgfpathmoveto{\pgfqpoint{2.124140in}{6.081512in}}%
\pgfpathcurveto{\pgfqpoint{2.135190in}{6.081512in}}{\pgfqpoint{2.145789in}{6.085903in}}{\pgfqpoint{2.153603in}{6.093716in}}%
\pgfpathcurveto{\pgfqpoint{2.161417in}{6.101530in}}{\pgfqpoint{2.165807in}{6.112129in}}{\pgfqpoint{2.165807in}{6.123179in}}%
\pgfpathcurveto{\pgfqpoint{2.165807in}{6.134229in}}{\pgfqpoint{2.161417in}{6.144828in}}{\pgfqpoint{2.153603in}{6.152642in}}%
\pgfpathcurveto{\pgfqpoint{2.145789in}{6.160456in}}{\pgfqpoint{2.135190in}{6.164846in}}{\pgfqpoint{2.124140in}{6.164846in}}%
\pgfpathcurveto{\pgfqpoint{2.113090in}{6.164846in}}{\pgfqpoint{2.102491in}{6.160456in}}{\pgfqpoint{2.094677in}{6.152642in}}%
\pgfpathcurveto{\pgfqpoint{2.086864in}{6.144828in}}{\pgfqpoint{2.082474in}{6.134229in}}{\pgfqpoint{2.082474in}{6.123179in}}%
\pgfpathcurveto{\pgfqpoint{2.082474in}{6.112129in}}{\pgfqpoint{2.086864in}{6.101530in}}{\pgfqpoint{2.094677in}{6.093716in}}%
\pgfpathcurveto{\pgfqpoint{2.102491in}{6.085903in}}{\pgfqpoint{2.113090in}{6.081512in}}{\pgfqpoint{2.124140in}{6.081512in}}%
\pgfpathlineto{\pgfqpoint{2.124140in}{6.081512in}}%
\pgfpathclose%
\pgfusepath{stroke,fill}%
\end{pgfscope}%
\begin{pgfscope}%
\pgfpathrectangle{\pgfqpoint{0.633874in}{5.272501in}}{\pgfqpoint{2.177280in}{2.201755in}}%
\pgfusepath{clip}%
\pgfsetbuttcap%
\pgfsetroundjoin%
\definecolor{currentfill}{rgb}{0.172549,0.627451,0.172549}%
\pgfsetfillcolor{currentfill}%
\pgfsetlinewidth{0.481800pt}%
\definecolor{currentstroke}{rgb}{1.000000,1.000000,1.000000}%
\pgfsetstrokecolor{currentstroke}%
\pgfsetdash{}{0pt}%
\pgfpathmoveto{\pgfqpoint{1.884508in}{5.747913in}}%
\pgfpathcurveto{\pgfqpoint{1.895558in}{5.747913in}}{\pgfqpoint{1.906157in}{5.752303in}}{\pgfqpoint{1.913971in}{5.760117in}}%
\pgfpathcurveto{\pgfqpoint{1.921784in}{5.767931in}}{\pgfqpoint{1.926175in}{5.778530in}}{\pgfqpoint{1.926175in}{5.789580in}}%
\pgfpathcurveto{\pgfqpoint{1.926175in}{5.800630in}}{\pgfqpoint{1.921784in}{5.811229in}}{\pgfqpoint{1.913971in}{5.819043in}}%
\pgfpathcurveto{\pgfqpoint{1.906157in}{5.826856in}}{\pgfqpoint{1.895558in}{5.831247in}}{\pgfqpoint{1.884508in}{5.831247in}}%
\pgfpathcurveto{\pgfqpoint{1.873458in}{5.831247in}}{\pgfqpoint{1.862859in}{5.826856in}}{\pgfqpoint{1.855045in}{5.819043in}}%
\pgfpathcurveto{\pgfqpoint{1.847232in}{5.811229in}}{\pgfqpoint{1.842841in}{5.800630in}}{\pgfqpoint{1.842841in}{5.789580in}}%
\pgfpathcurveto{\pgfqpoint{1.842841in}{5.778530in}}{\pgfqpoint{1.847232in}{5.767931in}}{\pgfqpoint{1.855045in}{5.760117in}}%
\pgfpathcurveto{\pgfqpoint{1.862859in}{5.752303in}}{\pgfqpoint{1.873458in}{5.747913in}}{\pgfqpoint{1.884508in}{5.747913in}}%
\pgfpathlineto{\pgfqpoint{1.884508in}{5.747913in}}%
\pgfpathclose%
\pgfusepath{stroke,fill}%
\end{pgfscope}%
\begin{pgfscope}%
\pgfpathrectangle{\pgfqpoint{0.633874in}{5.272501in}}{\pgfqpoint{2.177280in}{2.201755in}}%
\pgfusepath{clip}%
\pgfsetbuttcap%
\pgfsetroundjoin%
\definecolor{currentfill}{rgb}{0.172549,0.627451,0.172549}%
\pgfsetfillcolor{currentfill}%
\pgfsetlinewidth{0.481800pt}%
\definecolor{currentstroke}{rgb}{1.000000,1.000000,1.000000}%
\pgfsetstrokecolor{currentstroke}%
\pgfsetdash{}{0pt}%
\pgfpathmoveto{\pgfqpoint{2.084202in}{6.665311in}}%
\pgfpathcurveto{\pgfqpoint{2.095252in}{6.665311in}}{\pgfqpoint{2.105851in}{6.669701in}}{\pgfqpoint{2.113664in}{6.677515in}}%
\pgfpathcurveto{\pgfqpoint{2.121478in}{6.685329in}}{\pgfqpoint{2.125868in}{6.695928in}}{\pgfqpoint{2.125868in}{6.706978in}}%
\pgfpathcurveto{\pgfqpoint{2.125868in}{6.718028in}}{\pgfqpoint{2.121478in}{6.728627in}}{\pgfqpoint{2.113664in}{6.736441in}}%
\pgfpathcurveto{\pgfqpoint{2.105851in}{6.744254in}}{\pgfqpoint{2.095252in}{6.748644in}}{\pgfqpoint{2.084202in}{6.748644in}}%
\pgfpathcurveto{\pgfqpoint{2.073151in}{6.748644in}}{\pgfqpoint{2.062552in}{6.744254in}}{\pgfqpoint{2.054739in}{6.736441in}}%
\pgfpathcurveto{\pgfqpoint{2.046925in}{6.728627in}}{\pgfqpoint{2.042535in}{6.718028in}}{\pgfqpoint{2.042535in}{6.706978in}}%
\pgfpathcurveto{\pgfqpoint{2.042535in}{6.695928in}}{\pgfqpoint{2.046925in}{6.685329in}}{\pgfqpoint{2.054739in}{6.677515in}}%
\pgfpathcurveto{\pgfqpoint{2.062552in}{6.669701in}}{\pgfqpoint{2.073151in}{6.665311in}}{\pgfqpoint{2.084202in}{6.665311in}}%
\pgfpathlineto{\pgfqpoint{2.084202in}{6.665311in}}%
\pgfpathclose%
\pgfusepath{stroke,fill}%
\end{pgfscope}%
\begin{pgfscope}%
\pgfpathrectangle{\pgfqpoint{0.633874in}{5.272501in}}{\pgfqpoint{2.177280in}{2.201755in}}%
\pgfusepath{clip}%
\pgfsetbuttcap%
\pgfsetroundjoin%
\definecolor{currentfill}{rgb}{0.172549,0.627451,0.172549}%
\pgfsetfillcolor{currentfill}%
\pgfsetlinewidth{0.481800pt}%
\definecolor{currentstroke}{rgb}{1.000000,1.000000,1.000000}%
\pgfsetstrokecolor{currentstroke}%
\pgfsetdash{}{0pt}%
\pgfpathmoveto{\pgfqpoint{1.804631in}{6.331712in}}%
\pgfpathcurveto{\pgfqpoint{1.815681in}{6.331712in}}{\pgfqpoint{1.826280in}{6.336102in}}{\pgfqpoint{1.834093in}{6.343916in}}%
\pgfpathcurveto{\pgfqpoint{1.841907in}{6.351729in}}{\pgfqpoint{1.846297in}{6.362328in}}{\pgfqpoint{1.846297in}{6.373379in}}%
\pgfpathcurveto{\pgfqpoint{1.846297in}{6.384429in}}{\pgfqpoint{1.841907in}{6.395028in}}{\pgfqpoint{1.834093in}{6.402841in}}%
\pgfpathcurveto{\pgfqpoint{1.826280in}{6.410655in}}{\pgfqpoint{1.815681in}{6.415045in}}{\pgfqpoint{1.804631in}{6.415045in}}%
\pgfpathcurveto{\pgfqpoint{1.793581in}{6.415045in}}{\pgfqpoint{1.782981in}{6.410655in}}{\pgfqpoint{1.775168in}{6.402841in}}%
\pgfpathcurveto{\pgfqpoint{1.767354in}{6.395028in}}{\pgfqpoint{1.762964in}{6.384429in}}{\pgfqpoint{1.762964in}{6.373379in}}%
\pgfpathcurveto{\pgfqpoint{1.762964in}{6.362328in}}{\pgfqpoint{1.767354in}{6.351729in}}{\pgfqpoint{1.775168in}{6.343916in}}%
\pgfpathcurveto{\pgfqpoint{1.782981in}{6.336102in}}{\pgfqpoint{1.793581in}{6.331712in}}{\pgfqpoint{1.804631in}{6.331712in}}%
\pgfpathlineto{\pgfqpoint{1.804631in}{6.331712in}}%
\pgfpathclose%
\pgfusepath{stroke,fill}%
\end{pgfscope}%
\begin{pgfscope}%
\pgfpathrectangle{\pgfqpoint{0.633874in}{5.272501in}}{\pgfqpoint{2.177280in}{2.201755in}}%
\pgfusepath{clip}%
\pgfsetbuttcap%
\pgfsetroundjoin%
\definecolor{currentfill}{rgb}{0.172549,0.627451,0.172549}%
\pgfsetfillcolor{currentfill}%
\pgfsetlinewidth{0.481800pt}%
\definecolor{currentstroke}{rgb}{1.000000,1.000000,1.000000}%
\pgfsetstrokecolor{currentstroke}%
\pgfsetdash{}{0pt}%
\pgfpathmoveto{\pgfqpoint{1.764692in}{5.914713in}}%
\pgfpathcurveto{\pgfqpoint{1.775742in}{5.914713in}}{\pgfqpoint{1.786341in}{5.919103in}}{\pgfqpoint{1.794155in}{5.926917in}}%
\pgfpathcurveto{\pgfqpoint{1.801968in}{5.934730in}}{\pgfqpoint{1.806359in}{5.945329in}}{\pgfqpoint{1.806359in}{5.956379in}}%
\pgfpathcurveto{\pgfqpoint{1.806359in}{5.967430in}}{\pgfqpoint{1.801968in}{5.978029in}}{\pgfqpoint{1.794155in}{5.985842in}}%
\pgfpathcurveto{\pgfqpoint{1.786341in}{5.993656in}}{\pgfqpoint{1.775742in}{5.998046in}}{\pgfqpoint{1.764692in}{5.998046in}}%
\pgfpathcurveto{\pgfqpoint{1.753642in}{5.998046in}}{\pgfqpoint{1.743043in}{5.993656in}}{\pgfqpoint{1.735229in}{5.985842in}}%
\pgfpathcurveto{\pgfqpoint{1.727416in}{5.978029in}}{\pgfqpoint{1.723025in}{5.967430in}}{\pgfqpoint{1.723025in}{5.956379in}}%
\pgfpathcurveto{\pgfqpoint{1.723025in}{5.945329in}}{\pgfqpoint{1.727416in}{5.934730in}}{\pgfqpoint{1.735229in}{5.926917in}}%
\pgfpathcurveto{\pgfqpoint{1.743043in}{5.919103in}}{\pgfqpoint{1.753642in}{5.914713in}}{\pgfqpoint{1.764692in}{5.914713in}}%
\pgfpathlineto{\pgfqpoint{1.764692in}{5.914713in}}%
\pgfpathclose%
\pgfusepath{stroke,fill}%
\end{pgfscope}%
\begin{pgfscope}%
\pgfpathrectangle{\pgfqpoint{0.633874in}{5.272501in}}{\pgfqpoint{2.177280in}{2.201755in}}%
\pgfusepath{clip}%
\pgfsetbuttcap%
\pgfsetroundjoin%
\definecolor{currentfill}{rgb}{0.172549,0.627451,0.172549}%
\pgfsetfillcolor{currentfill}%
\pgfsetlinewidth{0.481800pt}%
\definecolor{currentstroke}{rgb}{1.000000,1.000000,1.000000}%
\pgfsetstrokecolor{currentstroke}%
\pgfsetdash{}{0pt}%
\pgfpathmoveto{\pgfqpoint{1.924447in}{6.164912in}}%
\pgfpathcurveto{\pgfqpoint{1.935497in}{6.164912in}}{\pgfqpoint{1.946096in}{6.169303in}}{\pgfqpoint{1.953910in}{6.177116in}}%
\pgfpathcurveto{\pgfqpoint{1.961723in}{6.184930in}}{\pgfqpoint{1.966113in}{6.195529in}}{\pgfqpoint{1.966113in}{6.206579in}}%
\pgfpathcurveto{\pgfqpoint{1.966113in}{6.217629in}}{\pgfqpoint{1.961723in}{6.228228in}}{\pgfqpoint{1.953910in}{6.236042in}}%
\pgfpathcurveto{\pgfqpoint{1.946096in}{6.243855in}}{\pgfqpoint{1.935497in}{6.248246in}}{\pgfqpoint{1.924447in}{6.248246in}}%
\pgfpathcurveto{\pgfqpoint{1.913397in}{6.248246in}}{\pgfqpoint{1.902798in}{6.243855in}}{\pgfqpoint{1.894984in}{6.236042in}}%
\pgfpathcurveto{\pgfqpoint{1.887170in}{6.228228in}}{\pgfqpoint{1.882780in}{6.217629in}}{\pgfqpoint{1.882780in}{6.206579in}}%
\pgfpathcurveto{\pgfqpoint{1.882780in}{6.195529in}}{\pgfqpoint{1.887170in}{6.184930in}}{\pgfqpoint{1.894984in}{6.177116in}}%
\pgfpathcurveto{\pgfqpoint{1.902798in}{6.169303in}}{\pgfqpoint{1.913397in}{6.164912in}}{\pgfqpoint{1.924447in}{6.164912in}}%
\pgfpathlineto{\pgfqpoint{1.924447in}{6.164912in}}%
\pgfpathclose%
\pgfusepath{stroke,fill}%
\end{pgfscope}%
\begin{pgfscope}%
\pgfpathrectangle{\pgfqpoint{0.633874in}{5.272501in}}{\pgfqpoint{2.177280in}{2.201755in}}%
\pgfusepath{clip}%
\pgfsetbuttcap%
\pgfsetroundjoin%
\definecolor{currentfill}{rgb}{0.172549,0.627451,0.172549}%
\pgfsetfillcolor{currentfill}%
\pgfsetlinewidth{0.481800pt}%
\definecolor{currentstroke}{rgb}{1.000000,1.000000,1.000000}%
\pgfsetstrokecolor{currentstroke}%
\pgfsetdash{}{0pt}%
\pgfpathmoveto{\pgfqpoint{1.485121in}{5.747913in}}%
\pgfpathcurveto{\pgfqpoint{1.496171in}{5.747913in}}{\pgfqpoint{1.506770in}{5.752303in}}{\pgfqpoint{1.514584in}{5.760117in}}%
\pgfpathcurveto{\pgfqpoint{1.522397in}{5.767931in}}{\pgfqpoint{1.526788in}{5.778530in}}{\pgfqpoint{1.526788in}{5.789580in}}%
\pgfpathcurveto{\pgfqpoint{1.526788in}{5.800630in}}{\pgfqpoint{1.522397in}{5.811229in}}{\pgfqpoint{1.514584in}{5.819043in}}%
\pgfpathcurveto{\pgfqpoint{1.506770in}{5.826856in}}{\pgfqpoint{1.496171in}{5.831247in}}{\pgfqpoint{1.485121in}{5.831247in}}%
\pgfpathcurveto{\pgfqpoint{1.474071in}{5.831247in}}{\pgfqpoint{1.463472in}{5.826856in}}{\pgfqpoint{1.455658in}{5.819043in}}%
\pgfpathcurveto{\pgfqpoint{1.447845in}{5.811229in}}{\pgfqpoint{1.443454in}{5.800630in}}{\pgfqpoint{1.443454in}{5.789580in}}%
\pgfpathcurveto{\pgfqpoint{1.443454in}{5.778530in}}{\pgfqpoint{1.447845in}{5.767931in}}{\pgfqpoint{1.455658in}{5.760117in}}%
\pgfpathcurveto{\pgfqpoint{1.463472in}{5.752303in}}{\pgfqpoint{1.474071in}{5.747913in}}{\pgfqpoint{1.485121in}{5.747913in}}%
\pgfpathlineto{\pgfqpoint{1.485121in}{5.747913in}}%
\pgfpathclose%
\pgfusepath{stroke,fill}%
\end{pgfscope}%
\begin{pgfscope}%
\pgfpathrectangle{\pgfqpoint{0.633874in}{5.272501in}}{\pgfqpoint{2.177280in}{2.201755in}}%
\pgfusepath{clip}%
\pgfsetbuttcap%
\pgfsetroundjoin%
\definecolor{currentfill}{rgb}{0.172549,0.627451,0.172549}%
\pgfsetfillcolor{currentfill}%
\pgfsetlinewidth{0.481800pt}%
\definecolor{currentstroke}{rgb}{1.000000,1.000000,1.000000}%
\pgfsetstrokecolor{currentstroke}%
\pgfsetdash{}{0pt}%
\pgfpathmoveto{\pgfqpoint{1.525060in}{5.998113in}}%
\pgfpathcurveto{\pgfqpoint{1.536110in}{5.998113in}}{\pgfqpoint{1.546709in}{6.002503in}}{\pgfqpoint{1.554523in}{6.010317in}}%
\pgfpathcurveto{\pgfqpoint{1.562336in}{6.018130in}}{\pgfqpoint{1.566726in}{6.028729in}}{\pgfqpoint{1.566726in}{6.039779in}}%
\pgfpathcurveto{\pgfqpoint{1.566726in}{6.050829in}}{\pgfqpoint{1.562336in}{6.061428in}}{\pgfqpoint{1.554523in}{6.069242in}}%
\pgfpathcurveto{\pgfqpoint{1.546709in}{6.077056in}}{\pgfqpoint{1.536110in}{6.081446in}}{\pgfqpoint{1.525060in}{6.081446in}}%
\pgfpathcurveto{\pgfqpoint{1.514010in}{6.081446in}}{\pgfqpoint{1.503411in}{6.077056in}}{\pgfqpoint{1.495597in}{6.069242in}}%
\pgfpathcurveto{\pgfqpoint{1.487783in}{6.061428in}}{\pgfqpoint{1.483393in}{6.050829in}}{\pgfqpoint{1.483393in}{6.039779in}}%
\pgfpathcurveto{\pgfqpoint{1.483393in}{6.028729in}}{\pgfqpoint{1.487783in}{6.018130in}}{\pgfqpoint{1.495597in}{6.010317in}}%
\pgfpathcurveto{\pgfqpoint{1.503411in}{6.002503in}}{\pgfqpoint{1.514010in}{5.998113in}}{\pgfqpoint{1.525060in}{5.998113in}}%
\pgfpathlineto{\pgfqpoint{1.525060in}{5.998113in}}%
\pgfpathclose%
\pgfusepath{stroke,fill}%
\end{pgfscope}%
\begin{pgfscope}%
\pgfpathrectangle{\pgfqpoint{0.633874in}{5.272501in}}{\pgfqpoint{2.177280in}{2.201755in}}%
\pgfusepath{clip}%
\pgfsetbuttcap%
\pgfsetroundjoin%
\definecolor{currentfill}{rgb}{0.172549,0.627451,0.172549}%
\pgfsetfillcolor{currentfill}%
\pgfsetlinewidth{0.481800pt}%
\definecolor{currentstroke}{rgb}{1.000000,1.000000,1.000000}%
\pgfsetstrokecolor{currentstroke}%
\pgfsetdash{}{0pt}%
\pgfpathmoveto{\pgfqpoint{1.764692in}{6.331712in}}%
\pgfpathcurveto{\pgfqpoint{1.775742in}{6.331712in}}{\pgfqpoint{1.786341in}{6.336102in}}{\pgfqpoint{1.794155in}{6.343916in}}%
\pgfpathcurveto{\pgfqpoint{1.801968in}{6.351729in}}{\pgfqpoint{1.806359in}{6.362328in}}{\pgfqpoint{1.806359in}{6.373379in}}%
\pgfpathcurveto{\pgfqpoint{1.806359in}{6.384429in}}{\pgfqpoint{1.801968in}{6.395028in}}{\pgfqpoint{1.794155in}{6.402841in}}%
\pgfpathcurveto{\pgfqpoint{1.786341in}{6.410655in}}{\pgfqpoint{1.775742in}{6.415045in}}{\pgfqpoint{1.764692in}{6.415045in}}%
\pgfpathcurveto{\pgfqpoint{1.753642in}{6.415045in}}{\pgfqpoint{1.743043in}{6.410655in}}{\pgfqpoint{1.735229in}{6.402841in}}%
\pgfpathcurveto{\pgfqpoint{1.727416in}{6.395028in}}{\pgfqpoint{1.723025in}{6.384429in}}{\pgfqpoint{1.723025in}{6.373379in}}%
\pgfpathcurveto{\pgfqpoint{1.723025in}{6.362328in}}{\pgfqpoint{1.727416in}{6.351729in}}{\pgfqpoint{1.735229in}{6.343916in}}%
\pgfpathcurveto{\pgfqpoint{1.743043in}{6.336102in}}{\pgfqpoint{1.753642in}{6.331712in}}{\pgfqpoint{1.764692in}{6.331712in}}%
\pgfpathlineto{\pgfqpoint{1.764692in}{6.331712in}}%
\pgfpathclose%
\pgfusepath{stroke,fill}%
\end{pgfscope}%
\begin{pgfscope}%
\pgfpathrectangle{\pgfqpoint{0.633874in}{5.272501in}}{\pgfqpoint{2.177280in}{2.201755in}}%
\pgfusepath{clip}%
\pgfsetbuttcap%
\pgfsetroundjoin%
\definecolor{currentfill}{rgb}{0.172549,0.627451,0.172549}%
\pgfsetfillcolor{currentfill}%
\pgfsetlinewidth{0.481800pt}%
\definecolor{currentstroke}{rgb}{1.000000,1.000000,1.000000}%
\pgfsetstrokecolor{currentstroke}%
\pgfsetdash{}{0pt}%
\pgfpathmoveto{\pgfqpoint{1.804631in}{6.164912in}}%
\pgfpathcurveto{\pgfqpoint{1.815681in}{6.164912in}}{\pgfqpoint{1.826280in}{6.169303in}}{\pgfqpoint{1.834093in}{6.177116in}}%
\pgfpathcurveto{\pgfqpoint{1.841907in}{6.184930in}}{\pgfqpoint{1.846297in}{6.195529in}}{\pgfqpoint{1.846297in}{6.206579in}}%
\pgfpathcurveto{\pgfqpoint{1.846297in}{6.217629in}}{\pgfqpoint{1.841907in}{6.228228in}}{\pgfqpoint{1.834093in}{6.236042in}}%
\pgfpathcurveto{\pgfqpoint{1.826280in}{6.243855in}}{\pgfqpoint{1.815681in}{6.248246in}}{\pgfqpoint{1.804631in}{6.248246in}}%
\pgfpathcurveto{\pgfqpoint{1.793581in}{6.248246in}}{\pgfqpoint{1.782981in}{6.243855in}}{\pgfqpoint{1.775168in}{6.236042in}}%
\pgfpathcurveto{\pgfqpoint{1.767354in}{6.228228in}}{\pgfqpoint{1.762964in}{6.217629in}}{\pgfqpoint{1.762964in}{6.206579in}}%
\pgfpathcurveto{\pgfqpoint{1.762964in}{6.195529in}}{\pgfqpoint{1.767354in}{6.184930in}}{\pgfqpoint{1.775168in}{6.177116in}}%
\pgfpathcurveto{\pgfqpoint{1.782981in}{6.169303in}}{\pgfqpoint{1.793581in}{6.164912in}}{\pgfqpoint{1.804631in}{6.164912in}}%
\pgfpathlineto{\pgfqpoint{1.804631in}{6.164912in}}%
\pgfpathclose%
\pgfusepath{stroke,fill}%
\end{pgfscope}%
\begin{pgfscope}%
\pgfpathrectangle{\pgfqpoint{0.633874in}{5.272501in}}{\pgfqpoint{2.177280in}{2.201755in}}%
\pgfusepath{clip}%
\pgfsetbuttcap%
\pgfsetroundjoin%
\definecolor{currentfill}{rgb}{0.172549,0.627451,0.172549}%
\pgfsetfillcolor{currentfill}%
\pgfsetlinewidth{0.481800pt}%
\definecolor{currentstroke}{rgb}{1.000000,1.000000,1.000000}%
\pgfsetstrokecolor{currentstroke}%
\pgfsetdash{}{0pt}%
\pgfpathmoveto{\pgfqpoint{2.283895in}{6.832111in}}%
\pgfpathcurveto{\pgfqpoint{2.294945in}{6.832111in}}{\pgfqpoint{2.305544in}{6.836501in}}{\pgfqpoint{2.313358in}{6.844315in}}%
\pgfpathcurveto{\pgfqpoint{2.321171in}{6.852128in}}{\pgfqpoint{2.325562in}{6.862727in}}{\pgfqpoint{2.325562in}{6.873777in}}%
\pgfpathcurveto{\pgfqpoint{2.325562in}{6.884828in}}{\pgfqpoint{2.321171in}{6.895427in}}{\pgfqpoint{2.313358in}{6.903240in}}%
\pgfpathcurveto{\pgfqpoint{2.305544in}{6.911054in}}{\pgfqpoint{2.294945in}{6.915444in}}{\pgfqpoint{2.283895in}{6.915444in}}%
\pgfpathcurveto{\pgfqpoint{2.272845in}{6.915444in}}{\pgfqpoint{2.262246in}{6.911054in}}{\pgfqpoint{2.254432in}{6.903240in}}%
\pgfpathcurveto{\pgfqpoint{2.246619in}{6.895427in}}{\pgfqpoint{2.242228in}{6.884828in}}{\pgfqpoint{2.242228in}{6.873777in}}%
\pgfpathcurveto{\pgfqpoint{2.242228in}{6.862727in}}{\pgfqpoint{2.246619in}{6.852128in}}{\pgfqpoint{2.254432in}{6.844315in}}%
\pgfpathcurveto{\pgfqpoint{2.262246in}{6.836501in}}{\pgfqpoint{2.272845in}{6.832111in}}{\pgfqpoint{2.283895in}{6.832111in}}%
\pgfpathlineto{\pgfqpoint{2.283895in}{6.832111in}}%
\pgfpathclose%
\pgfusepath{stroke,fill}%
\end{pgfscope}%
\begin{pgfscope}%
\pgfpathrectangle{\pgfqpoint{0.633874in}{5.272501in}}{\pgfqpoint{2.177280in}{2.201755in}}%
\pgfusepath{clip}%
\pgfsetbuttcap%
\pgfsetroundjoin%
\definecolor{currentfill}{rgb}{0.172549,0.627451,0.172549}%
\pgfsetfillcolor{currentfill}%
\pgfsetlinewidth{0.481800pt}%
\definecolor{currentstroke}{rgb}{1.000000,1.000000,1.000000}%
\pgfsetstrokecolor{currentstroke}%
\pgfsetdash{}{0pt}%
\pgfpathmoveto{\pgfqpoint{2.283895in}{5.831313in}}%
\pgfpathcurveto{\pgfqpoint{2.294945in}{5.831313in}}{\pgfqpoint{2.305544in}{5.835703in}}{\pgfqpoint{2.313358in}{5.843517in}}%
\pgfpathcurveto{\pgfqpoint{2.321171in}{5.851331in}}{\pgfqpoint{2.325562in}{5.861930in}}{\pgfqpoint{2.325562in}{5.872980in}}%
\pgfpathcurveto{\pgfqpoint{2.325562in}{5.884030in}}{\pgfqpoint{2.321171in}{5.894629in}}{\pgfqpoint{2.313358in}{5.902442in}}%
\pgfpathcurveto{\pgfqpoint{2.305544in}{5.910256in}}{\pgfqpoint{2.294945in}{5.914646in}}{\pgfqpoint{2.283895in}{5.914646in}}%
\pgfpathcurveto{\pgfqpoint{2.272845in}{5.914646in}}{\pgfqpoint{2.262246in}{5.910256in}}{\pgfqpoint{2.254432in}{5.902442in}}%
\pgfpathcurveto{\pgfqpoint{2.246619in}{5.894629in}}{\pgfqpoint{2.242228in}{5.884030in}}{\pgfqpoint{2.242228in}{5.872980in}}%
\pgfpathcurveto{\pgfqpoint{2.242228in}{5.861930in}}{\pgfqpoint{2.246619in}{5.851331in}}{\pgfqpoint{2.254432in}{5.843517in}}%
\pgfpathcurveto{\pgfqpoint{2.262246in}{5.835703in}}{\pgfqpoint{2.272845in}{5.831313in}}{\pgfqpoint{2.283895in}{5.831313in}}%
\pgfpathlineto{\pgfqpoint{2.283895in}{5.831313in}}%
\pgfpathclose%
\pgfusepath{stroke,fill}%
\end{pgfscope}%
\begin{pgfscope}%
\pgfpathrectangle{\pgfqpoint{0.633874in}{5.272501in}}{\pgfqpoint{2.177280in}{2.201755in}}%
\pgfusepath{clip}%
\pgfsetbuttcap%
\pgfsetroundjoin%
\definecolor{currentfill}{rgb}{0.172549,0.627451,0.172549}%
\pgfsetfillcolor{currentfill}%
\pgfsetlinewidth{0.481800pt}%
\definecolor{currentstroke}{rgb}{1.000000,1.000000,1.000000}%
\pgfsetstrokecolor{currentstroke}%
\pgfsetdash{}{0pt}%
\pgfpathmoveto{\pgfqpoint{1.604937in}{5.497714in}}%
\pgfpathcurveto{\pgfqpoint{1.615987in}{5.497714in}}{\pgfqpoint{1.626586in}{5.502104in}}{\pgfqpoint{1.634400in}{5.509918in}}%
\pgfpathcurveto{\pgfqpoint{1.642214in}{5.517731in}}{\pgfqpoint{1.646604in}{5.528330in}}{\pgfqpoint{1.646604in}{5.539380in}}%
\pgfpathcurveto{\pgfqpoint{1.646604in}{5.550431in}}{\pgfqpoint{1.642214in}{5.561030in}}{\pgfqpoint{1.634400in}{5.568843in}}%
\pgfpathcurveto{\pgfqpoint{1.626586in}{5.576657in}}{\pgfqpoint{1.615987in}{5.581047in}}{\pgfqpoint{1.604937in}{5.581047in}}%
\pgfpathcurveto{\pgfqpoint{1.593887in}{5.581047in}}{\pgfqpoint{1.583288in}{5.576657in}}{\pgfqpoint{1.575474in}{5.568843in}}%
\pgfpathcurveto{\pgfqpoint{1.567661in}{5.561030in}}{\pgfqpoint{1.563270in}{5.550431in}}{\pgfqpoint{1.563270in}{5.539380in}}%
\pgfpathcurveto{\pgfqpoint{1.563270in}{5.528330in}}{\pgfqpoint{1.567661in}{5.517731in}}{\pgfqpoint{1.575474in}{5.509918in}}%
\pgfpathcurveto{\pgfqpoint{1.583288in}{5.502104in}}{\pgfqpoint{1.593887in}{5.497714in}}{\pgfqpoint{1.604937in}{5.497714in}}%
\pgfpathlineto{\pgfqpoint{1.604937in}{5.497714in}}%
\pgfpathclose%
\pgfusepath{stroke,fill}%
\end{pgfscope}%
\begin{pgfscope}%
\pgfpathrectangle{\pgfqpoint{0.633874in}{5.272501in}}{\pgfqpoint{2.177280in}{2.201755in}}%
\pgfusepath{clip}%
\pgfsetbuttcap%
\pgfsetroundjoin%
\definecolor{currentfill}{rgb}{0.172549,0.627451,0.172549}%
\pgfsetfillcolor{currentfill}%
\pgfsetlinewidth{0.481800pt}%
\definecolor{currentstroke}{rgb}{1.000000,1.000000,1.000000}%
\pgfsetstrokecolor{currentstroke}%
\pgfsetdash{}{0pt}%
\pgfpathmoveto{\pgfqpoint{1.964385in}{6.331712in}}%
\pgfpathcurveto{\pgfqpoint{1.975436in}{6.331712in}}{\pgfqpoint{1.986035in}{6.336102in}}{\pgfqpoint{1.993848in}{6.343916in}}%
\pgfpathcurveto{\pgfqpoint{2.001662in}{6.351729in}}{\pgfqpoint{2.006052in}{6.362328in}}{\pgfqpoint{2.006052in}{6.373379in}}%
\pgfpathcurveto{\pgfqpoint{2.006052in}{6.384429in}}{\pgfqpoint{2.001662in}{6.395028in}}{\pgfqpoint{1.993848in}{6.402841in}}%
\pgfpathcurveto{\pgfqpoint{1.986035in}{6.410655in}}{\pgfqpoint{1.975436in}{6.415045in}}{\pgfqpoint{1.964385in}{6.415045in}}%
\pgfpathcurveto{\pgfqpoint{1.953335in}{6.415045in}}{\pgfqpoint{1.942736in}{6.410655in}}{\pgfqpoint{1.934923in}{6.402841in}}%
\pgfpathcurveto{\pgfqpoint{1.927109in}{6.395028in}}{\pgfqpoint{1.922719in}{6.384429in}}{\pgfqpoint{1.922719in}{6.373379in}}%
\pgfpathcurveto{\pgfqpoint{1.922719in}{6.362328in}}{\pgfqpoint{1.927109in}{6.351729in}}{\pgfqpoint{1.934923in}{6.343916in}}%
\pgfpathcurveto{\pgfqpoint{1.942736in}{6.336102in}}{\pgfqpoint{1.953335in}{6.331712in}}{\pgfqpoint{1.964385in}{6.331712in}}%
\pgfpathlineto{\pgfqpoint{1.964385in}{6.331712in}}%
\pgfpathclose%
\pgfusepath{stroke,fill}%
\end{pgfscope}%
\begin{pgfscope}%
\pgfpathrectangle{\pgfqpoint{0.633874in}{5.272501in}}{\pgfqpoint{2.177280in}{2.201755in}}%
\pgfusepath{clip}%
\pgfsetbuttcap%
\pgfsetroundjoin%
\definecolor{currentfill}{rgb}{0.172549,0.627451,0.172549}%
\pgfsetfillcolor{currentfill}%
\pgfsetlinewidth{0.481800pt}%
\definecolor{currentstroke}{rgb}{1.000000,1.000000,1.000000}%
\pgfsetstrokecolor{currentstroke}%
\pgfsetdash{}{0pt}%
\pgfpathmoveto{\pgfqpoint{1.445182in}{5.998113in}}%
\pgfpathcurveto{\pgfqpoint{1.456232in}{5.998113in}}{\pgfqpoint{1.466831in}{6.002503in}}{\pgfqpoint{1.474645in}{6.010317in}}%
\pgfpathcurveto{\pgfqpoint{1.482459in}{6.018130in}}{\pgfqpoint{1.486849in}{6.028729in}}{\pgfqpoint{1.486849in}{6.039779in}}%
\pgfpathcurveto{\pgfqpoint{1.486849in}{6.050829in}}{\pgfqpoint{1.482459in}{6.061428in}}{\pgfqpoint{1.474645in}{6.069242in}}%
\pgfpathcurveto{\pgfqpoint{1.466831in}{6.077056in}}{\pgfqpoint{1.456232in}{6.081446in}}{\pgfqpoint{1.445182in}{6.081446in}}%
\pgfpathcurveto{\pgfqpoint{1.434132in}{6.081446in}}{\pgfqpoint{1.423533in}{6.077056in}}{\pgfqpoint{1.415720in}{6.069242in}}%
\pgfpathcurveto{\pgfqpoint{1.407906in}{6.061428in}}{\pgfqpoint{1.403516in}{6.050829in}}{\pgfqpoint{1.403516in}{6.039779in}}%
\pgfpathcurveto{\pgfqpoint{1.403516in}{6.028729in}}{\pgfqpoint{1.407906in}{6.018130in}}{\pgfqpoint{1.415720in}{6.010317in}}%
\pgfpathcurveto{\pgfqpoint{1.423533in}{6.002503in}}{\pgfqpoint{1.434132in}{5.998113in}}{\pgfqpoint{1.445182in}{5.998113in}}%
\pgfpathlineto{\pgfqpoint{1.445182in}{5.998113in}}%
\pgfpathclose%
\pgfusepath{stroke,fill}%
\end{pgfscope}%
\begin{pgfscope}%
\pgfpathrectangle{\pgfqpoint{0.633874in}{5.272501in}}{\pgfqpoint{2.177280in}{2.201755in}}%
\pgfusepath{clip}%
\pgfsetbuttcap%
\pgfsetroundjoin%
\definecolor{currentfill}{rgb}{0.172549,0.627451,0.172549}%
\pgfsetfillcolor{currentfill}%
\pgfsetlinewidth{0.481800pt}%
\definecolor{currentstroke}{rgb}{1.000000,1.000000,1.000000}%
\pgfsetstrokecolor{currentstroke}%
\pgfsetdash{}{0pt}%
\pgfpathmoveto{\pgfqpoint{2.283895in}{5.998113in}}%
\pgfpathcurveto{\pgfqpoint{2.294945in}{5.998113in}}{\pgfqpoint{2.305544in}{6.002503in}}{\pgfqpoint{2.313358in}{6.010317in}}%
\pgfpathcurveto{\pgfqpoint{2.321171in}{6.018130in}}{\pgfqpoint{2.325562in}{6.028729in}}{\pgfqpoint{2.325562in}{6.039779in}}%
\pgfpathcurveto{\pgfqpoint{2.325562in}{6.050829in}}{\pgfqpoint{2.321171in}{6.061428in}}{\pgfqpoint{2.313358in}{6.069242in}}%
\pgfpathcurveto{\pgfqpoint{2.305544in}{6.077056in}}{\pgfqpoint{2.294945in}{6.081446in}}{\pgfqpoint{2.283895in}{6.081446in}}%
\pgfpathcurveto{\pgfqpoint{2.272845in}{6.081446in}}{\pgfqpoint{2.262246in}{6.077056in}}{\pgfqpoint{2.254432in}{6.069242in}}%
\pgfpathcurveto{\pgfqpoint{2.246619in}{6.061428in}}{\pgfqpoint{2.242228in}{6.050829in}}{\pgfqpoint{2.242228in}{6.039779in}}%
\pgfpathcurveto{\pgfqpoint{2.242228in}{6.028729in}}{\pgfqpoint{2.246619in}{6.018130in}}{\pgfqpoint{2.254432in}{6.010317in}}%
\pgfpathcurveto{\pgfqpoint{2.262246in}{6.002503in}}{\pgfqpoint{2.272845in}{5.998113in}}{\pgfqpoint{2.283895in}{5.998113in}}%
\pgfpathlineto{\pgfqpoint{2.283895in}{5.998113in}}%
\pgfpathclose%
\pgfusepath{stroke,fill}%
\end{pgfscope}%
\begin{pgfscope}%
\pgfpathrectangle{\pgfqpoint{0.633874in}{5.272501in}}{\pgfqpoint{2.177280in}{2.201755in}}%
\pgfusepath{clip}%
\pgfsetbuttcap%
\pgfsetroundjoin%
\definecolor{currentfill}{rgb}{0.172549,0.627451,0.172549}%
\pgfsetfillcolor{currentfill}%
\pgfsetlinewidth{0.481800pt}%
\definecolor{currentstroke}{rgb}{1.000000,1.000000,1.000000}%
\pgfsetstrokecolor{currentstroke}%
\pgfsetdash{}{0pt}%
\pgfpathmoveto{\pgfqpoint{1.724753in}{5.914713in}}%
\pgfpathcurveto{\pgfqpoint{1.735803in}{5.914713in}}{\pgfqpoint{1.746402in}{5.919103in}}{\pgfqpoint{1.754216in}{5.926917in}}%
\pgfpathcurveto{\pgfqpoint{1.762030in}{5.934730in}}{\pgfqpoint{1.766420in}{5.945329in}}{\pgfqpoint{1.766420in}{5.956379in}}%
\pgfpathcurveto{\pgfqpoint{1.766420in}{5.967430in}}{\pgfqpoint{1.762030in}{5.978029in}}{\pgfqpoint{1.754216in}{5.985842in}}%
\pgfpathcurveto{\pgfqpoint{1.746402in}{5.993656in}}{\pgfqpoint{1.735803in}{5.998046in}}{\pgfqpoint{1.724753in}{5.998046in}}%
\pgfpathcurveto{\pgfqpoint{1.713703in}{5.998046in}}{\pgfqpoint{1.703104in}{5.993656in}}{\pgfqpoint{1.695290in}{5.985842in}}%
\pgfpathcurveto{\pgfqpoint{1.687477in}{5.978029in}}{\pgfqpoint{1.683087in}{5.967430in}}{\pgfqpoint{1.683087in}{5.956379in}}%
\pgfpathcurveto{\pgfqpoint{1.683087in}{5.945329in}}{\pgfqpoint{1.687477in}{5.934730in}}{\pgfqpoint{1.695290in}{5.926917in}}%
\pgfpathcurveto{\pgfqpoint{1.703104in}{5.919103in}}{\pgfqpoint{1.713703in}{5.914713in}}{\pgfqpoint{1.724753in}{5.914713in}}%
\pgfpathlineto{\pgfqpoint{1.724753in}{5.914713in}}%
\pgfpathclose%
\pgfusepath{stroke,fill}%
\end{pgfscope}%
\begin{pgfscope}%
\pgfpathrectangle{\pgfqpoint{0.633874in}{5.272501in}}{\pgfqpoint{2.177280in}{2.201755in}}%
\pgfusepath{clip}%
\pgfsetbuttcap%
\pgfsetroundjoin%
\definecolor{currentfill}{rgb}{0.172549,0.627451,0.172549}%
\pgfsetfillcolor{currentfill}%
\pgfsetlinewidth{0.481800pt}%
\definecolor{currentstroke}{rgb}{1.000000,1.000000,1.000000}%
\pgfsetstrokecolor{currentstroke}%
\pgfsetdash{}{0pt}%
\pgfpathmoveto{\pgfqpoint{1.884508in}{6.415112in}}%
\pgfpathcurveto{\pgfqpoint{1.895558in}{6.415112in}}{\pgfqpoint{1.906157in}{6.419502in}}{\pgfqpoint{1.913971in}{6.427316in}}%
\pgfpathcurveto{\pgfqpoint{1.921784in}{6.435129in}}{\pgfqpoint{1.926175in}{6.445728in}}{\pgfqpoint{1.926175in}{6.456778in}}%
\pgfpathcurveto{\pgfqpoint{1.926175in}{6.467828in}}{\pgfqpoint{1.921784in}{6.478428in}}{\pgfqpoint{1.913971in}{6.486241in}}%
\pgfpathcurveto{\pgfqpoint{1.906157in}{6.494055in}}{\pgfqpoint{1.895558in}{6.498445in}}{\pgfqpoint{1.884508in}{6.498445in}}%
\pgfpathcurveto{\pgfqpoint{1.873458in}{6.498445in}}{\pgfqpoint{1.862859in}{6.494055in}}{\pgfqpoint{1.855045in}{6.486241in}}%
\pgfpathcurveto{\pgfqpoint{1.847232in}{6.478428in}}{\pgfqpoint{1.842841in}{6.467828in}}{\pgfqpoint{1.842841in}{6.456778in}}%
\pgfpathcurveto{\pgfqpoint{1.842841in}{6.445728in}}{\pgfqpoint{1.847232in}{6.435129in}}{\pgfqpoint{1.855045in}{6.427316in}}%
\pgfpathcurveto{\pgfqpoint{1.862859in}{6.419502in}}{\pgfqpoint{1.873458in}{6.415112in}}{\pgfqpoint{1.884508in}{6.415112in}}%
\pgfpathlineto{\pgfqpoint{1.884508in}{6.415112in}}%
\pgfpathclose%
\pgfusepath{stroke,fill}%
\end{pgfscope}%
\begin{pgfscope}%
\pgfpathrectangle{\pgfqpoint{0.633874in}{5.272501in}}{\pgfqpoint{2.177280in}{2.201755in}}%
\pgfusepath{clip}%
\pgfsetbuttcap%
\pgfsetroundjoin%
\definecolor{currentfill}{rgb}{0.172549,0.627451,0.172549}%
\pgfsetfillcolor{currentfill}%
\pgfsetlinewidth{0.481800pt}%
\definecolor{currentstroke}{rgb}{1.000000,1.000000,1.000000}%
\pgfsetstrokecolor{currentstroke}%
\pgfsetdash{}{0pt}%
\pgfpathmoveto{\pgfqpoint{2.084202in}{6.331712in}}%
\pgfpathcurveto{\pgfqpoint{2.095252in}{6.331712in}}{\pgfqpoint{2.105851in}{6.336102in}}{\pgfqpoint{2.113664in}{6.343916in}}%
\pgfpathcurveto{\pgfqpoint{2.121478in}{6.351729in}}{\pgfqpoint{2.125868in}{6.362328in}}{\pgfqpoint{2.125868in}{6.373379in}}%
\pgfpathcurveto{\pgfqpoint{2.125868in}{6.384429in}}{\pgfqpoint{2.121478in}{6.395028in}}{\pgfqpoint{2.113664in}{6.402841in}}%
\pgfpathcurveto{\pgfqpoint{2.105851in}{6.410655in}}{\pgfqpoint{2.095252in}{6.415045in}}{\pgfqpoint{2.084202in}{6.415045in}}%
\pgfpathcurveto{\pgfqpoint{2.073151in}{6.415045in}}{\pgfqpoint{2.062552in}{6.410655in}}{\pgfqpoint{2.054739in}{6.402841in}}%
\pgfpathcurveto{\pgfqpoint{2.046925in}{6.395028in}}{\pgfqpoint{2.042535in}{6.384429in}}{\pgfqpoint{2.042535in}{6.373379in}}%
\pgfpathcurveto{\pgfqpoint{2.042535in}{6.362328in}}{\pgfqpoint{2.046925in}{6.351729in}}{\pgfqpoint{2.054739in}{6.343916in}}%
\pgfpathcurveto{\pgfqpoint{2.062552in}{6.336102in}}{\pgfqpoint{2.073151in}{6.331712in}}{\pgfqpoint{2.084202in}{6.331712in}}%
\pgfpathlineto{\pgfqpoint{2.084202in}{6.331712in}}%
\pgfpathclose%
\pgfusepath{stroke,fill}%
\end{pgfscope}%
\begin{pgfscope}%
\pgfpathrectangle{\pgfqpoint{0.633874in}{5.272501in}}{\pgfqpoint{2.177280in}{2.201755in}}%
\pgfusepath{clip}%
\pgfsetbuttcap%
\pgfsetroundjoin%
\definecolor{currentfill}{rgb}{0.172549,0.627451,0.172549}%
\pgfsetfillcolor{currentfill}%
\pgfsetlinewidth{0.481800pt}%
\definecolor{currentstroke}{rgb}{1.000000,1.000000,1.000000}%
\pgfsetstrokecolor{currentstroke}%
\pgfsetdash{}{0pt}%
\pgfpathmoveto{\pgfqpoint{1.684815in}{5.998113in}}%
\pgfpathcurveto{\pgfqpoint{1.695865in}{5.998113in}}{\pgfqpoint{1.706464in}{6.002503in}}{\pgfqpoint{1.714277in}{6.010317in}}%
\pgfpathcurveto{\pgfqpoint{1.722091in}{6.018130in}}{\pgfqpoint{1.726481in}{6.028729in}}{\pgfqpoint{1.726481in}{6.039779in}}%
\pgfpathcurveto{\pgfqpoint{1.726481in}{6.050829in}}{\pgfqpoint{1.722091in}{6.061428in}}{\pgfqpoint{1.714277in}{6.069242in}}%
\pgfpathcurveto{\pgfqpoint{1.706464in}{6.077056in}}{\pgfqpoint{1.695865in}{6.081446in}}{\pgfqpoint{1.684815in}{6.081446in}}%
\pgfpathcurveto{\pgfqpoint{1.673764in}{6.081446in}}{\pgfqpoint{1.663165in}{6.077056in}}{\pgfqpoint{1.655352in}{6.069242in}}%
\pgfpathcurveto{\pgfqpoint{1.647538in}{6.061428in}}{\pgfqpoint{1.643148in}{6.050829in}}{\pgfqpoint{1.643148in}{6.039779in}}%
\pgfpathcurveto{\pgfqpoint{1.643148in}{6.028729in}}{\pgfqpoint{1.647538in}{6.018130in}}{\pgfqpoint{1.655352in}{6.010317in}}%
\pgfpathcurveto{\pgfqpoint{1.663165in}{6.002503in}}{\pgfqpoint{1.673764in}{5.998113in}}{\pgfqpoint{1.684815in}{5.998113in}}%
\pgfpathlineto{\pgfqpoint{1.684815in}{5.998113in}}%
\pgfpathclose%
\pgfusepath{stroke,fill}%
\end{pgfscope}%
\begin{pgfscope}%
\pgfpathrectangle{\pgfqpoint{0.633874in}{5.272501in}}{\pgfqpoint{2.177280in}{2.201755in}}%
\pgfusepath{clip}%
\pgfsetbuttcap%
\pgfsetroundjoin%
\definecolor{currentfill}{rgb}{0.172549,0.627451,0.172549}%
\pgfsetfillcolor{currentfill}%
\pgfsetlinewidth{0.481800pt}%
\definecolor{currentstroke}{rgb}{1.000000,1.000000,1.000000}%
\pgfsetstrokecolor{currentstroke}%
\pgfsetdash{}{0pt}%
\pgfpathmoveto{\pgfqpoint{1.644876in}{6.164912in}}%
\pgfpathcurveto{\pgfqpoint{1.655926in}{6.164912in}}{\pgfqpoint{1.666525in}{6.169303in}}{\pgfqpoint{1.674339in}{6.177116in}}%
\pgfpathcurveto{\pgfqpoint{1.682152in}{6.184930in}}{\pgfqpoint{1.686543in}{6.195529in}}{\pgfqpoint{1.686543in}{6.206579in}}%
\pgfpathcurveto{\pgfqpoint{1.686543in}{6.217629in}}{\pgfqpoint{1.682152in}{6.228228in}}{\pgfqpoint{1.674339in}{6.236042in}}%
\pgfpathcurveto{\pgfqpoint{1.666525in}{6.243855in}}{\pgfqpoint{1.655926in}{6.248246in}}{\pgfqpoint{1.644876in}{6.248246in}}%
\pgfpathcurveto{\pgfqpoint{1.633826in}{6.248246in}}{\pgfqpoint{1.623227in}{6.243855in}}{\pgfqpoint{1.615413in}{6.236042in}}%
\pgfpathcurveto{\pgfqpoint{1.607599in}{6.228228in}}{\pgfqpoint{1.603209in}{6.217629in}}{\pgfqpoint{1.603209in}{6.206579in}}%
\pgfpathcurveto{\pgfqpoint{1.603209in}{6.195529in}}{\pgfqpoint{1.607599in}{6.184930in}}{\pgfqpoint{1.615413in}{6.177116in}}%
\pgfpathcurveto{\pgfqpoint{1.623227in}{6.169303in}}{\pgfqpoint{1.633826in}{6.164912in}}{\pgfqpoint{1.644876in}{6.164912in}}%
\pgfpathlineto{\pgfqpoint{1.644876in}{6.164912in}}%
\pgfpathclose%
\pgfusepath{stroke,fill}%
\end{pgfscope}%
\begin{pgfscope}%
\pgfpathrectangle{\pgfqpoint{0.633874in}{5.272501in}}{\pgfqpoint{2.177280in}{2.201755in}}%
\pgfusepath{clip}%
\pgfsetbuttcap%
\pgfsetroundjoin%
\definecolor{currentfill}{rgb}{0.172549,0.627451,0.172549}%
\pgfsetfillcolor{currentfill}%
\pgfsetlinewidth{0.481800pt}%
\definecolor{currentstroke}{rgb}{1.000000,1.000000,1.000000}%
\pgfsetstrokecolor{currentstroke}%
\pgfsetdash{}{0pt}%
\pgfpathmoveto{\pgfqpoint{1.764692in}{5.998113in}}%
\pgfpathcurveto{\pgfqpoint{1.775742in}{5.998113in}}{\pgfqpoint{1.786341in}{6.002503in}}{\pgfqpoint{1.794155in}{6.010317in}}%
\pgfpathcurveto{\pgfqpoint{1.801968in}{6.018130in}}{\pgfqpoint{1.806359in}{6.028729in}}{\pgfqpoint{1.806359in}{6.039779in}}%
\pgfpathcurveto{\pgfqpoint{1.806359in}{6.050829in}}{\pgfqpoint{1.801968in}{6.061428in}}{\pgfqpoint{1.794155in}{6.069242in}}%
\pgfpathcurveto{\pgfqpoint{1.786341in}{6.077056in}}{\pgfqpoint{1.775742in}{6.081446in}}{\pgfqpoint{1.764692in}{6.081446in}}%
\pgfpathcurveto{\pgfqpoint{1.753642in}{6.081446in}}{\pgfqpoint{1.743043in}{6.077056in}}{\pgfqpoint{1.735229in}{6.069242in}}%
\pgfpathcurveto{\pgfqpoint{1.727416in}{6.061428in}}{\pgfqpoint{1.723025in}{6.050829in}}{\pgfqpoint{1.723025in}{6.039779in}}%
\pgfpathcurveto{\pgfqpoint{1.723025in}{6.028729in}}{\pgfqpoint{1.727416in}{6.018130in}}{\pgfqpoint{1.735229in}{6.010317in}}%
\pgfpathcurveto{\pgfqpoint{1.743043in}{6.002503in}}{\pgfqpoint{1.753642in}{5.998113in}}{\pgfqpoint{1.764692in}{5.998113in}}%
\pgfpathlineto{\pgfqpoint{1.764692in}{5.998113in}}%
\pgfpathclose%
\pgfusepath{stroke,fill}%
\end{pgfscope}%
\begin{pgfscope}%
\pgfpathrectangle{\pgfqpoint{0.633874in}{5.272501in}}{\pgfqpoint{2.177280in}{2.201755in}}%
\pgfusepath{clip}%
\pgfsetbuttcap%
\pgfsetroundjoin%
\definecolor{currentfill}{rgb}{0.172549,0.627451,0.172549}%
\pgfsetfillcolor{currentfill}%
\pgfsetlinewidth{0.481800pt}%
\definecolor{currentstroke}{rgb}{1.000000,1.000000,1.000000}%
\pgfsetstrokecolor{currentstroke}%
\pgfsetdash{}{0pt}%
\pgfpathmoveto{\pgfqpoint{2.084202in}{6.164912in}}%
\pgfpathcurveto{\pgfqpoint{2.095252in}{6.164912in}}{\pgfqpoint{2.105851in}{6.169303in}}{\pgfqpoint{2.113664in}{6.177116in}}%
\pgfpathcurveto{\pgfqpoint{2.121478in}{6.184930in}}{\pgfqpoint{2.125868in}{6.195529in}}{\pgfqpoint{2.125868in}{6.206579in}}%
\pgfpathcurveto{\pgfqpoint{2.125868in}{6.217629in}}{\pgfqpoint{2.121478in}{6.228228in}}{\pgfqpoint{2.113664in}{6.236042in}}%
\pgfpathcurveto{\pgfqpoint{2.105851in}{6.243855in}}{\pgfqpoint{2.095252in}{6.248246in}}{\pgfqpoint{2.084202in}{6.248246in}}%
\pgfpathcurveto{\pgfqpoint{2.073151in}{6.248246in}}{\pgfqpoint{2.062552in}{6.243855in}}{\pgfqpoint{2.054739in}{6.236042in}}%
\pgfpathcurveto{\pgfqpoint{2.046925in}{6.228228in}}{\pgfqpoint{2.042535in}{6.217629in}}{\pgfqpoint{2.042535in}{6.206579in}}%
\pgfpathcurveto{\pgfqpoint{2.042535in}{6.195529in}}{\pgfqpoint{2.046925in}{6.184930in}}{\pgfqpoint{2.054739in}{6.177116in}}%
\pgfpathcurveto{\pgfqpoint{2.062552in}{6.169303in}}{\pgfqpoint{2.073151in}{6.164912in}}{\pgfqpoint{2.084202in}{6.164912in}}%
\pgfpathlineto{\pgfqpoint{2.084202in}{6.164912in}}%
\pgfpathclose%
\pgfusepath{stroke,fill}%
\end{pgfscope}%
\begin{pgfscope}%
\pgfpathrectangle{\pgfqpoint{0.633874in}{5.272501in}}{\pgfqpoint{2.177280in}{2.201755in}}%
\pgfusepath{clip}%
\pgfsetbuttcap%
\pgfsetroundjoin%
\definecolor{currentfill}{rgb}{0.172549,0.627451,0.172549}%
\pgfsetfillcolor{currentfill}%
\pgfsetlinewidth{0.481800pt}%
\definecolor{currentstroke}{rgb}{1.000000,1.000000,1.000000}%
\pgfsetstrokecolor{currentstroke}%
\pgfsetdash{}{0pt}%
\pgfpathmoveto{\pgfqpoint{2.164079in}{5.998113in}}%
\pgfpathcurveto{\pgfqpoint{2.175129in}{5.998113in}}{\pgfqpoint{2.185728in}{6.002503in}}{\pgfqpoint{2.193542in}{6.010317in}}%
\pgfpathcurveto{\pgfqpoint{2.201355in}{6.018130in}}{\pgfqpoint{2.205746in}{6.028729in}}{\pgfqpoint{2.205746in}{6.039779in}}%
\pgfpathcurveto{\pgfqpoint{2.205746in}{6.050829in}}{\pgfqpoint{2.201355in}{6.061428in}}{\pgfqpoint{2.193542in}{6.069242in}}%
\pgfpathcurveto{\pgfqpoint{2.185728in}{6.077056in}}{\pgfqpoint{2.175129in}{6.081446in}}{\pgfqpoint{2.164079in}{6.081446in}}%
\pgfpathcurveto{\pgfqpoint{2.153029in}{6.081446in}}{\pgfqpoint{2.142430in}{6.077056in}}{\pgfqpoint{2.134616in}{6.069242in}}%
\pgfpathcurveto{\pgfqpoint{2.126803in}{6.061428in}}{\pgfqpoint{2.122412in}{6.050829in}}{\pgfqpoint{2.122412in}{6.039779in}}%
\pgfpathcurveto{\pgfqpoint{2.122412in}{6.028729in}}{\pgfqpoint{2.126803in}{6.018130in}}{\pgfqpoint{2.134616in}{6.010317in}}%
\pgfpathcurveto{\pgfqpoint{2.142430in}{6.002503in}}{\pgfqpoint{2.153029in}{5.998113in}}{\pgfqpoint{2.164079in}{5.998113in}}%
\pgfpathlineto{\pgfqpoint{2.164079in}{5.998113in}}%
\pgfpathclose%
\pgfusepath{stroke,fill}%
\end{pgfscope}%
\begin{pgfscope}%
\pgfpathrectangle{\pgfqpoint{0.633874in}{5.272501in}}{\pgfqpoint{2.177280in}{2.201755in}}%
\pgfusepath{clip}%
\pgfsetbuttcap%
\pgfsetroundjoin%
\definecolor{currentfill}{rgb}{0.172549,0.627451,0.172549}%
\pgfsetfillcolor{currentfill}%
\pgfsetlinewidth{0.481800pt}%
\definecolor{currentstroke}{rgb}{1.000000,1.000000,1.000000}%
\pgfsetstrokecolor{currentstroke}%
\pgfsetdash{}{0pt}%
\pgfpathmoveto{\pgfqpoint{2.363772in}{6.832111in}}%
\pgfpathcurveto{\pgfqpoint{2.374823in}{6.832111in}}{\pgfqpoint{2.385422in}{6.836501in}}{\pgfqpoint{2.393235in}{6.844315in}}%
\pgfpathcurveto{\pgfqpoint{2.401049in}{6.852128in}}{\pgfqpoint{2.405439in}{6.862727in}}{\pgfqpoint{2.405439in}{6.873777in}}%
\pgfpathcurveto{\pgfqpoint{2.405439in}{6.884828in}}{\pgfqpoint{2.401049in}{6.895427in}}{\pgfqpoint{2.393235in}{6.903240in}}%
\pgfpathcurveto{\pgfqpoint{2.385422in}{6.911054in}}{\pgfqpoint{2.374823in}{6.915444in}}{\pgfqpoint{2.363772in}{6.915444in}}%
\pgfpathcurveto{\pgfqpoint{2.352722in}{6.915444in}}{\pgfqpoint{2.342123in}{6.911054in}}{\pgfqpoint{2.334310in}{6.903240in}}%
\pgfpathcurveto{\pgfqpoint{2.326496in}{6.895427in}}{\pgfqpoint{2.322106in}{6.884828in}}{\pgfqpoint{2.322106in}{6.873777in}}%
\pgfpathcurveto{\pgfqpoint{2.322106in}{6.862727in}}{\pgfqpoint{2.326496in}{6.852128in}}{\pgfqpoint{2.334310in}{6.844315in}}%
\pgfpathcurveto{\pgfqpoint{2.342123in}{6.836501in}}{\pgfqpoint{2.352722in}{6.832111in}}{\pgfqpoint{2.363772in}{6.832111in}}%
\pgfpathlineto{\pgfqpoint{2.363772in}{6.832111in}}%
\pgfpathclose%
\pgfusepath{stroke,fill}%
\end{pgfscope}%
\begin{pgfscope}%
\pgfpathrectangle{\pgfqpoint{0.633874in}{5.272501in}}{\pgfqpoint{2.177280in}{2.201755in}}%
\pgfusepath{clip}%
\pgfsetbuttcap%
\pgfsetroundjoin%
\definecolor{currentfill}{rgb}{0.172549,0.627451,0.172549}%
\pgfsetfillcolor{currentfill}%
\pgfsetlinewidth{0.481800pt}%
\definecolor{currentstroke}{rgb}{1.000000,1.000000,1.000000}%
\pgfsetstrokecolor{currentstroke}%
\pgfsetdash{}{0pt}%
\pgfpathmoveto{\pgfqpoint{1.764692in}{5.998113in}}%
\pgfpathcurveto{\pgfqpoint{1.775742in}{5.998113in}}{\pgfqpoint{1.786341in}{6.002503in}}{\pgfqpoint{1.794155in}{6.010317in}}%
\pgfpathcurveto{\pgfqpoint{1.801968in}{6.018130in}}{\pgfqpoint{1.806359in}{6.028729in}}{\pgfqpoint{1.806359in}{6.039779in}}%
\pgfpathcurveto{\pgfqpoint{1.806359in}{6.050829in}}{\pgfqpoint{1.801968in}{6.061428in}}{\pgfqpoint{1.794155in}{6.069242in}}%
\pgfpathcurveto{\pgfqpoint{1.786341in}{6.077056in}}{\pgfqpoint{1.775742in}{6.081446in}}{\pgfqpoint{1.764692in}{6.081446in}}%
\pgfpathcurveto{\pgfqpoint{1.753642in}{6.081446in}}{\pgfqpoint{1.743043in}{6.077056in}}{\pgfqpoint{1.735229in}{6.069242in}}%
\pgfpathcurveto{\pgfqpoint{1.727416in}{6.061428in}}{\pgfqpoint{1.723025in}{6.050829in}}{\pgfqpoint{1.723025in}{6.039779in}}%
\pgfpathcurveto{\pgfqpoint{1.723025in}{6.028729in}}{\pgfqpoint{1.727416in}{6.018130in}}{\pgfqpoint{1.735229in}{6.010317in}}%
\pgfpathcurveto{\pgfqpoint{1.743043in}{6.002503in}}{\pgfqpoint{1.753642in}{5.998113in}}{\pgfqpoint{1.764692in}{5.998113in}}%
\pgfpathlineto{\pgfqpoint{1.764692in}{5.998113in}}%
\pgfpathclose%
\pgfusepath{stroke,fill}%
\end{pgfscope}%
\begin{pgfscope}%
\pgfpathrectangle{\pgfqpoint{0.633874in}{5.272501in}}{\pgfqpoint{2.177280in}{2.201755in}}%
\pgfusepath{clip}%
\pgfsetbuttcap%
\pgfsetroundjoin%
\definecolor{currentfill}{rgb}{0.172549,0.627451,0.172549}%
\pgfsetfillcolor{currentfill}%
\pgfsetlinewidth{0.481800pt}%
\definecolor{currentstroke}{rgb}{1.000000,1.000000,1.000000}%
\pgfsetstrokecolor{currentstroke}%
\pgfsetdash{}{0pt}%
\pgfpathmoveto{\pgfqpoint{1.724753in}{5.998113in}}%
\pgfpathcurveto{\pgfqpoint{1.735803in}{5.998113in}}{\pgfqpoint{1.746402in}{6.002503in}}{\pgfqpoint{1.754216in}{6.010317in}}%
\pgfpathcurveto{\pgfqpoint{1.762030in}{6.018130in}}{\pgfqpoint{1.766420in}{6.028729in}}{\pgfqpoint{1.766420in}{6.039779in}}%
\pgfpathcurveto{\pgfqpoint{1.766420in}{6.050829in}}{\pgfqpoint{1.762030in}{6.061428in}}{\pgfqpoint{1.754216in}{6.069242in}}%
\pgfpathcurveto{\pgfqpoint{1.746402in}{6.077056in}}{\pgfqpoint{1.735803in}{6.081446in}}{\pgfqpoint{1.724753in}{6.081446in}}%
\pgfpathcurveto{\pgfqpoint{1.713703in}{6.081446in}}{\pgfqpoint{1.703104in}{6.077056in}}{\pgfqpoint{1.695290in}{6.069242in}}%
\pgfpathcurveto{\pgfqpoint{1.687477in}{6.061428in}}{\pgfqpoint{1.683087in}{6.050829in}}{\pgfqpoint{1.683087in}{6.039779in}}%
\pgfpathcurveto{\pgfqpoint{1.683087in}{6.028729in}}{\pgfqpoint{1.687477in}{6.018130in}}{\pgfqpoint{1.695290in}{6.010317in}}%
\pgfpathcurveto{\pgfqpoint{1.703104in}{6.002503in}}{\pgfqpoint{1.713703in}{5.998113in}}{\pgfqpoint{1.724753in}{5.998113in}}%
\pgfpathlineto{\pgfqpoint{1.724753in}{5.998113in}}%
\pgfpathclose%
\pgfusepath{stroke,fill}%
\end{pgfscope}%
\begin{pgfscope}%
\pgfpathrectangle{\pgfqpoint{0.633874in}{5.272501in}}{\pgfqpoint{2.177280in}{2.201755in}}%
\pgfusepath{clip}%
\pgfsetbuttcap%
\pgfsetroundjoin%
\definecolor{currentfill}{rgb}{0.172549,0.627451,0.172549}%
\pgfsetfillcolor{currentfill}%
\pgfsetlinewidth{0.481800pt}%
\definecolor{currentstroke}{rgb}{1.000000,1.000000,1.000000}%
\pgfsetstrokecolor{currentstroke}%
\pgfsetdash{}{0pt}%
\pgfpathmoveto{\pgfqpoint{1.644876in}{5.831313in}}%
\pgfpathcurveto{\pgfqpoint{1.655926in}{5.831313in}}{\pgfqpoint{1.666525in}{5.835703in}}{\pgfqpoint{1.674339in}{5.843517in}}%
\pgfpathcurveto{\pgfqpoint{1.682152in}{5.851331in}}{\pgfqpoint{1.686543in}{5.861930in}}{\pgfqpoint{1.686543in}{5.872980in}}%
\pgfpathcurveto{\pgfqpoint{1.686543in}{5.884030in}}{\pgfqpoint{1.682152in}{5.894629in}}{\pgfqpoint{1.674339in}{5.902442in}}%
\pgfpathcurveto{\pgfqpoint{1.666525in}{5.910256in}}{\pgfqpoint{1.655926in}{5.914646in}}{\pgfqpoint{1.644876in}{5.914646in}}%
\pgfpathcurveto{\pgfqpoint{1.633826in}{5.914646in}}{\pgfqpoint{1.623227in}{5.910256in}}{\pgfqpoint{1.615413in}{5.902442in}}%
\pgfpathcurveto{\pgfqpoint{1.607599in}{5.894629in}}{\pgfqpoint{1.603209in}{5.884030in}}{\pgfqpoint{1.603209in}{5.872980in}}%
\pgfpathcurveto{\pgfqpoint{1.603209in}{5.861930in}}{\pgfqpoint{1.607599in}{5.851331in}}{\pgfqpoint{1.615413in}{5.843517in}}%
\pgfpathcurveto{\pgfqpoint{1.623227in}{5.835703in}}{\pgfqpoint{1.633826in}{5.831313in}}{\pgfqpoint{1.644876in}{5.831313in}}%
\pgfpathlineto{\pgfqpoint{1.644876in}{5.831313in}}%
\pgfpathclose%
\pgfusepath{stroke,fill}%
\end{pgfscope}%
\begin{pgfscope}%
\pgfpathrectangle{\pgfqpoint{0.633874in}{5.272501in}}{\pgfqpoint{2.177280in}{2.201755in}}%
\pgfusepath{clip}%
\pgfsetbuttcap%
\pgfsetroundjoin%
\definecolor{currentfill}{rgb}{0.172549,0.627451,0.172549}%
\pgfsetfillcolor{currentfill}%
\pgfsetlinewidth{0.481800pt}%
\definecolor{currentstroke}{rgb}{1.000000,1.000000,1.000000}%
\pgfsetstrokecolor{currentstroke}%
\pgfsetdash{}{0pt}%
\pgfpathmoveto{\pgfqpoint{2.283895in}{6.164912in}}%
\pgfpathcurveto{\pgfqpoint{2.294945in}{6.164912in}}{\pgfqpoint{2.305544in}{6.169303in}}{\pgfqpoint{2.313358in}{6.177116in}}%
\pgfpathcurveto{\pgfqpoint{2.321171in}{6.184930in}}{\pgfqpoint{2.325562in}{6.195529in}}{\pgfqpoint{2.325562in}{6.206579in}}%
\pgfpathcurveto{\pgfqpoint{2.325562in}{6.217629in}}{\pgfqpoint{2.321171in}{6.228228in}}{\pgfqpoint{2.313358in}{6.236042in}}%
\pgfpathcurveto{\pgfqpoint{2.305544in}{6.243855in}}{\pgfqpoint{2.294945in}{6.248246in}}{\pgfqpoint{2.283895in}{6.248246in}}%
\pgfpathcurveto{\pgfqpoint{2.272845in}{6.248246in}}{\pgfqpoint{2.262246in}{6.243855in}}{\pgfqpoint{2.254432in}{6.236042in}}%
\pgfpathcurveto{\pgfqpoint{2.246619in}{6.228228in}}{\pgfqpoint{2.242228in}{6.217629in}}{\pgfqpoint{2.242228in}{6.206579in}}%
\pgfpathcurveto{\pgfqpoint{2.242228in}{6.195529in}}{\pgfqpoint{2.246619in}{6.184930in}}{\pgfqpoint{2.254432in}{6.177116in}}%
\pgfpathcurveto{\pgfqpoint{2.262246in}{6.169303in}}{\pgfqpoint{2.272845in}{6.164912in}}{\pgfqpoint{2.283895in}{6.164912in}}%
\pgfpathlineto{\pgfqpoint{2.283895in}{6.164912in}}%
\pgfpathclose%
\pgfusepath{stroke,fill}%
\end{pgfscope}%
\begin{pgfscope}%
\pgfpathrectangle{\pgfqpoint{0.633874in}{5.272501in}}{\pgfqpoint{2.177280in}{2.201755in}}%
\pgfusepath{clip}%
\pgfsetbuttcap%
\pgfsetroundjoin%
\definecolor{currentfill}{rgb}{0.172549,0.627451,0.172549}%
\pgfsetfillcolor{currentfill}%
\pgfsetlinewidth{0.481800pt}%
\definecolor{currentstroke}{rgb}{1.000000,1.000000,1.000000}%
\pgfsetstrokecolor{currentstroke}%
\pgfsetdash{}{0pt}%
\pgfpathmoveto{\pgfqpoint{1.724753in}{6.498512in}}%
\pgfpathcurveto{\pgfqpoint{1.735803in}{6.498512in}}{\pgfqpoint{1.746402in}{6.502902in}}{\pgfqpoint{1.754216in}{6.510715in}}%
\pgfpathcurveto{\pgfqpoint{1.762030in}{6.518529in}}{\pgfqpoint{1.766420in}{6.529128in}}{\pgfqpoint{1.766420in}{6.540178in}}%
\pgfpathcurveto{\pgfqpoint{1.766420in}{6.551228in}}{\pgfqpoint{1.762030in}{6.561827in}}{\pgfqpoint{1.754216in}{6.569641in}}%
\pgfpathcurveto{\pgfqpoint{1.746402in}{6.577455in}}{\pgfqpoint{1.735803in}{6.581845in}}{\pgfqpoint{1.724753in}{6.581845in}}%
\pgfpathcurveto{\pgfqpoint{1.713703in}{6.581845in}}{\pgfqpoint{1.703104in}{6.577455in}}{\pgfqpoint{1.695290in}{6.569641in}}%
\pgfpathcurveto{\pgfqpoint{1.687477in}{6.561827in}}{\pgfqpoint{1.683087in}{6.551228in}}{\pgfqpoint{1.683087in}{6.540178in}}%
\pgfpathcurveto{\pgfqpoint{1.683087in}{6.529128in}}{\pgfqpoint{1.687477in}{6.518529in}}{\pgfqpoint{1.695290in}{6.510715in}}%
\pgfpathcurveto{\pgfqpoint{1.703104in}{6.502902in}}{\pgfqpoint{1.713703in}{6.498512in}}{\pgfqpoint{1.724753in}{6.498512in}}%
\pgfpathlineto{\pgfqpoint{1.724753in}{6.498512in}}%
\pgfpathclose%
\pgfusepath{stroke,fill}%
\end{pgfscope}%
\begin{pgfscope}%
\pgfpathrectangle{\pgfqpoint{0.633874in}{5.272501in}}{\pgfqpoint{2.177280in}{2.201755in}}%
\pgfusepath{clip}%
\pgfsetbuttcap%
\pgfsetroundjoin%
\definecolor{currentfill}{rgb}{0.172549,0.627451,0.172549}%
\pgfsetfillcolor{currentfill}%
\pgfsetlinewidth{0.481800pt}%
\definecolor{currentstroke}{rgb}{1.000000,1.000000,1.000000}%
\pgfsetstrokecolor{currentstroke}%
\pgfsetdash{}{0pt}%
\pgfpathmoveto{\pgfqpoint{1.764692in}{6.248312in}}%
\pgfpathcurveto{\pgfqpoint{1.775742in}{6.248312in}}{\pgfqpoint{1.786341in}{6.252702in}}{\pgfqpoint{1.794155in}{6.260516in}}%
\pgfpathcurveto{\pgfqpoint{1.801968in}{6.268330in}}{\pgfqpoint{1.806359in}{6.278929in}}{\pgfqpoint{1.806359in}{6.289979in}}%
\pgfpathcurveto{\pgfqpoint{1.806359in}{6.301029in}}{\pgfqpoint{1.801968in}{6.311628in}}{\pgfqpoint{1.794155in}{6.319442in}}%
\pgfpathcurveto{\pgfqpoint{1.786341in}{6.327255in}}{\pgfqpoint{1.775742in}{6.331645in}}{\pgfqpoint{1.764692in}{6.331645in}}%
\pgfpathcurveto{\pgfqpoint{1.753642in}{6.331645in}}{\pgfqpoint{1.743043in}{6.327255in}}{\pgfqpoint{1.735229in}{6.319442in}}%
\pgfpathcurveto{\pgfqpoint{1.727416in}{6.311628in}}{\pgfqpoint{1.723025in}{6.301029in}}{\pgfqpoint{1.723025in}{6.289979in}}%
\pgfpathcurveto{\pgfqpoint{1.723025in}{6.278929in}}{\pgfqpoint{1.727416in}{6.268330in}}{\pgfqpoint{1.735229in}{6.260516in}}%
\pgfpathcurveto{\pgfqpoint{1.743043in}{6.252702in}}{\pgfqpoint{1.753642in}{6.248312in}}{\pgfqpoint{1.764692in}{6.248312in}}%
\pgfpathlineto{\pgfqpoint{1.764692in}{6.248312in}}%
\pgfpathclose%
\pgfusepath{stroke,fill}%
\end{pgfscope}%
\begin{pgfscope}%
\pgfpathrectangle{\pgfqpoint{0.633874in}{5.272501in}}{\pgfqpoint{2.177280in}{2.201755in}}%
\pgfusepath{clip}%
\pgfsetbuttcap%
\pgfsetroundjoin%
\definecolor{currentfill}{rgb}{0.172549,0.627451,0.172549}%
\pgfsetfillcolor{currentfill}%
\pgfsetlinewidth{0.481800pt}%
\definecolor{currentstroke}{rgb}{1.000000,1.000000,1.000000}%
\pgfsetstrokecolor{currentstroke}%
\pgfsetdash{}{0pt}%
\pgfpathmoveto{\pgfqpoint{1.604937in}{6.164912in}}%
\pgfpathcurveto{\pgfqpoint{1.615987in}{6.164912in}}{\pgfqpoint{1.626586in}{6.169303in}}{\pgfqpoint{1.634400in}{6.177116in}}%
\pgfpathcurveto{\pgfqpoint{1.642214in}{6.184930in}}{\pgfqpoint{1.646604in}{6.195529in}}{\pgfqpoint{1.646604in}{6.206579in}}%
\pgfpathcurveto{\pgfqpoint{1.646604in}{6.217629in}}{\pgfqpoint{1.642214in}{6.228228in}}{\pgfqpoint{1.634400in}{6.236042in}}%
\pgfpathcurveto{\pgfqpoint{1.626586in}{6.243855in}}{\pgfqpoint{1.615987in}{6.248246in}}{\pgfqpoint{1.604937in}{6.248246in}}%
\pgfpathcurveto{\pgfqpoint{1.593887in}{6.248246in}}{\pgfqpoint{1.583288in}{6.243855in}}{\pgfqpoint{1.575474in}{6.236042in}}%
\pgfpathcurveto{\pgfqpoint{1.567661in}{6.228228in}}{\pgfqpoint{1.563270in}{6.217629in}}{\pgfqpoint{1.563270in}{6.206579in}}%
\pgfpathcurveto{\pgfqpoint{1.563270in}{6.195529in}}{\pgfqpoint{1.567661in}{6.184930in}}{\pgfqpoint{1.575474in}{6.177116in}}%
\pgfpathcurveto{\pgfqpoint{1.583288in}{6.169303in}}{\pgfqpoint{1.593887in}{6.164912in}}{\pgfqpoint{1.604937in}{6.164912in}}%
\pgfpathlineto{\pgfqpoint{1.604937in}{6.164912in}}%
\pgfpathclose%
\pgfusepath{stroke,fill}%
\end{pgfscope}%
\begin{pgfscope}%
\pgfpathrectangle{\pgfqpoint{0.633874in}{5.272501in}}{\pgfqpoint{2.177280in}{2.201755in}}%
\pgfusepath{clip}%
\pgfsetbuttcap%
\pgfsetroundjoin%
\definecolor{currentfill}{rgb}{0.172549,0.627451,0.172549}%
\pgfsetfillcolor{currentfill}%
\pgfsetlinewidth{0.481800pt}%
\definecolor{currentstroke}{rgb}{1.000000,1.000000,1.000000}%
\pgfsetstrokecolor{currentstroke}%
\pgfsetdash{}{0pt}%
\pgfpathmoveto{\pgfqpoint{1.964385in}{6.248312in}}%
\pgfpathcurveto{\pgfqpoint{1.975436in}{6.248312in}}{\pgfqpoint{1.986035in}{6.252702in}}{\pgfqpoint{1.993848in}{6.260516in}}%
\pgfpathcurveto{\pgfqpoint{2.001662in}{6.268330in}}{\pgfqpoint{2.006052in}{6.278929in}}{\pgfqpoint{2.006052in}{6.289979in}}%
\pgfpathcurveto{\pgfqpoint{2.006052in}{6.301029in}}{\pgfqpoint{2.001662in}{6.311628in}}{\pgfqpoint{1.993848in}{6.319442in}}%
\pgfpathcurveto{\pgfqpoint{1.986035in}{6.327255in}}{\pgfqpoint{1.975436in}{6.331645in}}{\pgfqpoint{1.964385in}{6.331645in}}%
\pgfpathcurveto{\pgfqpoint{1.953335in}{6.331645in}}{\pgfqpoint{1.942736in}{6.327255in}}{\pgfqpoint{1.934923in}{6.319442in}}%
\pgfpathcurveto{\pgfqpoint{1.927109in}{6.311628in}}{\pgfqpoint{1.922719in}{6.301029in}}{\pgfqpoint{1.922719in}{6.289979in}}%
\pgfpathcurveto{\pgfqpoint{1.922719in}{6.278929in}}{\pgfqpoint{1.927109in}{6.268330in}}{\pgfqpoint{1.934923in}{6.260516in}}%
\pgfpathcurveto{\pgfqpoint{1.942736in}{6.252702in}}{\pgfqpoint{1.953335in}{6.248312in}}{\pgfqpoint{1.964385in}{6.248312in}}%
\pgfpathlineto{\pgfqpoint{1.964385in}{6.248312in}}%
\pgfpathclose%
\pgfusepath{stroke,fill}%
\end{pgfscope}%
\begin{pgfscope}%
\pgfpathrectangle{\pgfqpoint{0.633874in}{5.272501in}}{\pgfqpoint{2.177280in}{2.201755in}}%
\pgfusepath{clip}%
\pgfsetbuttcap%
\pgfsetroundjoin%
\definecolor{currentfill}{rgb}{0.172549,0.627451,0.172549}%
\pgfsetfillcolor{currentfill}%
\pgfsetlinewidth{0.481800pt}%
\definecolor{currentstroke}{rgb}{1.000000,1.000000,1.000000}%
\pgfsetstrokecolor{currentstroke}%
\pgfsetdash{}{0pt}%
\pgfpathmoveto{\pgfqpoint{1.884508in}{6.248312in}}%
\pgfpathcurveto{\pgfqpoint{1.895558in}{6.248312in}}{\pgfqpoint{1.906157in}{6.252702in}}{\pgfqpoint{1.913971in}{6.260516in}}%
\pgfpathcurveto{\pgfqpoint{1.921784in}{6.268330in}}{\pgfqpoint{1.926175in}{6.278929in}}{\pgfqpoint{1.926175in}{6.289979in}}%
\pgfpathcurveto{\pgfqpoint{1.926175in}{6.301029in}}{\pgfqpoint{1.921784in}{6.311628in}}{\pgfqpoint{1.913971in}{6.319442in}}%
\pgfpathcurveto{\pgfqpoint{1.906157in}{6.327255in}}{\pgfqpoint{1.895558in}{6.331645in}}{\pgfqpoint{1.884508in}{6.331645in}}%
\pgfpathcurveto{\pgfqpoint{1.873458in}{6.331645in}}{\pgfqpoint{1.862859in}{6.327255in}}{\pgfqpoint{1.855045in}{6.319442in}}%
\pgfpathcurveto{\pgfqpoint{1.847232in}{6.311628in}}{\pgfqpoint{1.842841in}{6.301029in}}{\pgfqpoint{1.842841in}{6.289979in}}%
\pgfpathcurveto{\pgfqpoint{1.842841in}{6.278929in}}{\pgfqpoint{1.847232in}{6.268330in}}{\pgfqpoint{1.855045in}{6.260516in}}%
\pgfpathcurveto{\pgfqpoint{1.862859in}{6.252702in}}{\pgfqpoint{1.873458in}{6.248312in}}{\pgfqpoint{1.884508in}{6.248312in}}%
\pgfpathlineto{\pgfqpoint{1.884508in}{6.248312in}}%
\pgfpathclose%
\pgfusepath{stroke,fill}%
\end{pgfscope}%
\begin{pgfscope}%
\pgfpathrectangle{\pgfqpoint{0.633874in}{5.272501in}}{\pgfqpoint{2.177280in}{2.201755in}}%
\pgfusepath{clip}%
\pgfsetbuttcap%
\pgfsetroundjoin%
\definecolor{currentfill}{rgb}{0.172549,0.627451,0.172549}%
\pgfsetfillcolor{currentfill}%
\pgfsetlinewidth{0.481800pt}%
\definecolor{currentstroke}{rgb}{1.000000,1.000000,1.000000}%
\pgfsetstrokecolor{currentstroke}%
\pgfsetdash{}{0pt}%
\pgfpathmoveto{\pgfqpoint{1.964385in}{6.248312in}}%
\pgfpathcurveto{\pgfqpoint{1.975436in}{6.248312in}}{\pgfqpoint{1.986035in}{6.252702in}}{\pgfqpoint{1.993848in}{6.260516in}}%
\pgfpathcurveto{\pgfqpoint{2.001662in}{6.268330in}}{\pgfqpoint{2.006052in}{6.278929in}}{\pgfqpoint{2.006052in}{6.289979in}}%
\pgfpathcurveto{\pgfqpoint{2.006052in}{6.301029in}}{\pgfqpoint{2.001662in}{6.311628in}}{\pgfqpoint{1.993848in}{6.319442in}}%
\pgfpathcurveto{\pgfqpoint{1.986035in}{6.327255in}}{\pgfqpoint{1.975436in}{6.331645in}}{\pgfqpoint{1.964385in}{6.331645in}}%
\pgfpathcurveto{\pgfqpoint{1.953335in}{6.331645in}}{\pgfqpoint{1.942736in}{6.327255in}}{\pgfqpoint{1.934923in}{6.319442in}}%
\pgfpathcurveto{\pgfqpoint{1.927109in}{6.311628in}}{\pgfqpoint{1.922719in}{6.301029in}}{\pgfqpoint{1.922719in}{6.289979in}}%
\pgfpathcurveto{\pgfqpoint{1.922719in}{6.278929in}}{\pgfqpoint{1.927109in}{6.268330in}}{\pgfqpoint{1.934923in}{6.260516in}}%
\pgfpathcurveto{\pgfqpoint{1.942736in}{6.252702in}}{\pgfqpoint{1.953335in}{6.248312in}}{\pgfqpoint{1.964385in}{6.248312in}}%
\pgfpathlineto{\pgfqpoint{1.964385in}{6.248312in}}%
\pgfpathclose%
\pgfusepath{stroke,fill}%
\end{pgfscope}%
\begin{pgfscope}%
\pgfpathrectangle{\pgfqpoint{0.633874in}{5.272501in}}{\pgfqpoint{2.177280in}{2.201755in}}%
\pgfusepath{clip}%
\pgfsetbuttcap%
\pgfsetroundjoin%
\definecolor{currentfill}{rgb}{0.172549,0.627451,0.172549}%
\pgfsetfillcolor{currentfill}%
\pgfsetlinewidth{0.481800pt}%
\definecolor{currentstroke}{rgb}{1.000000,1.000000,1.000000}%
\pgfsetstrokecolor{currentstroke}%
\pgfsetdash{}{0pt}%
\pgfpathmoveto{\pgfqpoint{1.525060in}{5.914713in}}%
\pgfpathcurveto{\pgfqpoint{1.536110in}{5.914713in}}{\pgfqpoint{1.546709in}{5.919103in}}{\pgfqpoint{1.554523in}{5.926917in}}%
\pgfpathcurveto{\pgfqpoint{1.562336in}{5.934730in}}{\pgfqpoint{1.566726in}{5.945329in}}{\pgfqpoint{1.566726in}{5.956379in}}%
\pgfpathcurveto{\pgfqpoint{1.566726in}{5.967430in}}{\pgfqpoint{1.562336in}{5.978029in}}{\pgfqpoint{1.554523in}{5.985842in}}%
\pgfpathcurveto{\pgfqpoint{1.546709in}{5.993656in}}{\pgfqpoint{1.536110in}{5.998046in}}{\pgfqpoint{1.525060in}{5.998046in}}%
\pgfpathcurveto{\pgfqpoint{1.514010in}{5.998046in}}{\pgfqpoint{1.503411in}{5.993656in}}{\pgfqpoint{1.495597in}{5.985842in}}%
\pgfpathcurveto{\pgfqpoint{1.487783in}{5.978029in}}{\pgfqpoint{1.483393in}{5.967430in}}{\pgfqpoint{1.483393in}{5.956379in}}%
\pgfpathcurveto{\pgfqpoint{1.483393in}{5.945329in}}{\pgfqpoint{1.487783in}{5.934730in}}{\pgfqpoint{1.495597in}{5.926917in}}%
\pgfpathcurveto{\pgfqpoint{1.503411in}{5.919103in}}{\pgfqpoint{1.514010in}{5.914713in}}{\pgfqpoint{1.525060in}{5.914713in}}%
\pgfpathlineto{\pgfqpoint{1.525060in}{5.914713in}}%
\pgfpathclose%
\pgfusepath{stroke,fill}%
\end{pgfscope}%
\begin{pgfscope}%
\pgfpathrectangle{\pgfqpoint{0.633874in}{5.272501in}}{\pgfqpoint{2.177280in}{2.201755in}}%
\pgfusepath{clip}%
\pgfsetbuttcap%
\pgfsetroundjoin%
\definecolor{currentfill}{rgb}{0.172549,0.627451,0.172549}%
\pgfsetfillcolor{currentfill}%
\pgfsetlinewidth{0.481800pt}%
\definecolor{currentstroke}{rgb}{1.000000,1.000000,1.000000}%
\pgfsetstrokecolor{currentstroke}%
\pgfsetdash{}{0pt}%
\pgfpathmoveto{\pgfqpoint{1.924447in}{6.331712in}}%
\pgfpathcurveto{\pgfqpoint{1.935497in}{6.331712in}}{\pgfqpoint{1.946096in}{6.336102in}}{\pgfqpoint{1.953910in}{6.343916in}}%
\pgfpathcurveto{\pgfqpoint{1.961723in}{6.351729in}}{\pgfqpoint{1.966113in}{6.362328in}}{\pgfqpoint{1.966113in}{6.373379in}}%
\pgfpathcurveto{\pgfqpoint{1.966113in}{6.384429in}}{\pgfqpoint{1.961723in}{6.395028in}}{\pgfqpoint{1.953910in}{6.402841in}}%
\pgfpathcurveto{\pgfqpoint{1.946096in}{6.410655in}}{\pgfqpoint{1.935497in}{6.415045in}}{\pgfqpoint{1.924447in}{6.415045in}}%
\pgfpathcurveto{\pgfqpoint{1.913397in}{6.415045in}}{\pgfqpoint{1.902798in}{6.410655in}}{\pgfqpoint{1.894984in}{6.402841in}}%
\pgfpathcurveto{\pgfqpoint{1.887170in}{6.395028in}}{\pgfqpoint{1.882780in}{6.384429in}}{\pgfqpoint{1.882780in}{6.373379in}}%
\pgfpathcurveto{\pgfqpoint{1.882780in}{6.362328in}}{\pgfqpoint{1.887170in}{6.351729in}}{\pgfqpoint{1.894984in}{6.343916in}}%
\pgfpathcurveto{\pgfqpoint{1.902798in}{6.336102in}}{\pgfqpoint{1.913397in}{6.331712in}}{\pgfqpoint{1.924447in}{6.331712in}}%
\pgfpathlineto{\pgfqpoint{1.924447in}{6.331712in}}%
\pgfpathclose%
\pgfusepath{stroke,fill}%
\end{pgfscope}%
\begin{pgfscope}%
\pgfpathrectangle{\pgfqpoint{0.633874in}{5.272501in}}{\pgfqpoint{2.177280in}{2.201755in}}%
\pgfusepath{clip}%
\pgfsetbuttcap%
\pgfsetroundjoin%
\definecolor{currentfill}{rgb}{0.172549,0.627451,0.172549}%
\pgfsetfillcolor{currentfill}%
\pgfsetlinewidth{0.481800pt}%
\definecolor{currentstroke}{rgb}{1.000000,1.000000,1.000000}%
\pgfsetstrokecolor{currentstroke}%
\pgfsetdash{}{0pt}%
\pgfpathmoveto{\pgfqpoint{1.884508in}{6.415112in}}%
\pgfpathcurveto{\pgfqpoint{1.895558in}{6.415112in}}{\pgfqpoint{1.906157in}{6.419502in}}{\pgfqpoint{1.913971in}{6.427316in}}%
\pgfpathcurveto{\pgfqpoint{1.921784in}{6.435129in}}{\pgfqpoint{1.926175in}{6.445728in}}{\pgfqpoint{1.926175in}{6.456778in}}%
\pgfpathcurveto{\pgfqpoint{1.926175in}{6.467828in}}{\pgfqpoint{1.921784in}{6.478428in}}{\pgfqpoint{1.913971in}{6.486241in}}%
\pgfpathcurveto{\pgfqpoint{1.906157in}{6.494055in}}{\pgfqpoint{1.895558in}{6.498445in}}{\pgfqpoint{1.884508in}{6.498445in}}%
\pgfpathcurveto{\pgfqpoint{1.873458in}{6.498445in}}{\pgfqpoint{1.862859in}{6.494055in}}{\pgfqpoint{1.855045in}{6.486241in}}%
\pgfpathcurveto{\pgfqpoint{1.847232in}{6.478428in}}{\pgfqpoint{1.842841in}{6.467828in}}{\pgfqpoint{1.842841in}{6.456778in}}%
\pgfpathcurveto{\pgfqpoint{1.842841in}{6.445728in}}{\pgfqpoint{1.847232in}{6.435129in}}{\pgfqpoint{1.855045in}{6.427316in}}%
\pgfpathcurveto{\pgfqpoint{1.862859in}{6.419502in}}{\pgfqpoint{1.873458in}{6.415112in}}{\pgfqpoint{1.884508in}{6.415112in}}%
\pgfpathlineto{\pgfqpoint{1.884508in}{6.415112in}}%
\pgfpathclose%
\pgfusepath{stroke,fill}%
\end{pgfscope}%
\begin{pgfscope}%
\pgfpathrectangle{\pgfqpoint{0.633874in}{5.272501in}}{\pgfqpoint{2.177280in}{2.201755in}}%
\pgfusepath{clip}%
\pgfsetbuttcap%
\pgfsetroundjoin%
\definecolor{currentfill}{rgb}{0.172549,0.627451,0.172549}%
\pgfsetfillcolor{currentfill}%
\pgfsetlinewidth{0.481800pt}%
\definecolor{currentstroke}{rgb}{1.000000,1.000000,1.000000}%
\pgfsetstrokecolor{currentstroke}%
\pgfsetdash{}{0pt}%
\pgfpathmoveto{\pgfqpoint{1.884508in}{6.164912in}}%
\pgfpathcurveto{\pgfqpoint{1.895558in}{6.164912in}}{\pgfqpoint{1.906157in}{6.169303in}}{\pgfqpoint{1.913971in}{6.177116in}}%
\pgfpathcurveto{\pgfqpoint{1.921784in}{6.184930in}}{\pgfqpoint{1.926175in}{6.195529in}}{\pgfqpoint{1.926175in}{6.206579in}}%
\pgfpathcurveto{\pgfqpoint{1.926175in}{6.217629in}}{\pgfqpoint{1.921784in}{6.228228in}}{\pgfqpoint{1.913971in}{6.236042in}}%
\pgfpathcurveto{\pgfqpoint{1.906157in}{6.243855in}}{\pgfqpoint{1.895558in}{6.248246in}}{\pgfqpoint{1.884508in}{6.248246in}}%
\pgfpathcurveto{\pgfqpoint{1.873458in}{6.248246in}}{\pgfqpoint{1.862859in}{6.243855in}}{\pgfqpoint{1.855045in}{6.236042in}}%
\pgfpathcurveto{\pgfqpoint{1.847232in}{6.228228in}}{\pgfqpoint{1.842841in}{6.217629in}}{\pgfqpoint{1.842841in}{6.206579in}}%
\pgfpathcurveto{\pgfqpoint{1.842841in}{6.195529in}}{\pgfqpoint{1.847232in}{6.184930in}}{\pgfqpoint{1.855045in}{6.177116in}}%
\pgfpathcurveto{\pgfqpoint{1.862859in}{6.169303in}}{\pgfqpoint{1.873458in}{6.164912in}}{\pgfqpoint{1.884508in}{6.164912in}}%
\pgfpathlineto{\pgfqpoint{1.884508in}{6.164912in}}%
\pgfpathclose%
\pgfusepath{stroke,fill}%
\end{pgfscope}%
\begin{pgfscope}%
\pgfpathrectangle{\pgfqpoint{0.633874in}{5.272501in}}{\pgfqpoint{2.177280in}{2.201755in}}%
\pgfusepath{clip}%
\pgfsetbuttcap%
\pgfsetroundjoin%
\definecolor{currentfill}{rgb}{0.172549,0.627451,0.172549}%
\pgfsetfillcolor{currentfill}%
\pgfsetlinewidth{0.481800pt}%
\definecolor{currentstroke}{rgb}{1.000000,1.000000,1.000000}%
\pgfsetstrokecolor{currentstroke}%
\pgfsetdash{}{0pt}%
\pgfpathmoveto{\pgfqpoint{1.724753in}{5.747913in}}%
\pgfpathcurveto{\pgfqpoint{1.735803in}{5.747913in}}{\pgfqpoint{1.746402in}{5.752303in}}{\pgfqpoint{1.754216in}{5.760117in}}%
\pgfpathcurveto{\pgfqpoint{1.762030in}{5.767931in}}{\pgfqpoint{1.766420in}{5.778530in}}{\pgfqpoint{1.766420in}{5.789580in}}%
\pgfpathcurveto{\pgfqpoint{1.766420in}{5.800630in}}{\pgfqpoint{1.762030in}{5.811229in}}{\pgfqpoint{1.754216in}{5.819043in}}%
\pgfpathcurveto{\pgfqpoint{1.746402in}{5.826856in}}{\pgfqpoint{1.735803in}{5.831247in}}{\pgfqpoint{1.724753in}{5.831247in}}%
\pgfpathcurveto{\pgfqpoint{1.713703in}{5.831247in}}{\pgfqpoint{1.703104in}{5.826856in}}{\pgfqpoint{1.695290in}{5.819043in}}%
\pgfpathcurveto{\pgfqpoint{1.687477in}{5.811229in}}{\pgfqpoint{1.683087in}{5.800630in}}{\pgfqpoint{1.683087in}{5.789580in}}%
\pgfpathcurveto{\pgfqpoint{1.683087in}{5.778530in}}{\pgfqpoint{1.687477in}{5.767931in}}{\pgfqpoint{1.695290in}{5.760117in}}%
\pgfpathcurveto{\pgfqpoint{1.703104in}{5.752303in}}{\pgfqpoint{1.713703in}{5.747913in}}{\pgfqpoint{1.724753in}{5.747913in}}%
\pgfpathlineto{\pgfqpoint{1.724753in}{5.747913in}}%
\pgfpathclose%
\pgfusepath{stroke,fill}%
\end{pgfscope}%
\begin{pgfscope}%
\pgfpathrectangle{\pgfqpoint{0.633874in}{5.272501in}}{\pgfqpoint{2.177280in}{2.201755in}}%
\pgfusepath{clip}%
\pgfsetbuttcap%
\pgfsetroundjoin%
\definecolor{currentfill}{rgb}{0.172549,0.627451,0.172549}%
\pgfsetfillcolor{currentfill}%
\pgfsetlinewidth{0.481800pt}%
\definecolor{currentstroke}{rgb}{1.000000,1.000000,1.000000}%
\pgfsetstrokecolor{currentstroke}%
\pgfsetdash{}{0pt}%
\pgfpathmoveto{\pgfqpoint{1.804631in}{6.164912in}}%
\pgfpathcurveto{\pgfqpoint{1.815681in}{6.164912in}}{\pgfqpoint{1.826280in}{6.169303in}}{\pgfqpoint{1.834093in}{6.177116in}}%
\pgfpathcurveto{\pgfqpoint{1.841907in}{6.184930in}}{\pgfqpoint{1.846297in}{6.195529in}}{\pgfqpoint{1.846297in}{6.206579in}}%
\pgfpathcurveto{\pgfqpoint{1.846297in}{6.217629in}}{\pgfqpoint{1.841907in}{6.228228in}}{\pgfqpoint{1.834093in}{6.236042in}}%
\pgfpathcurveto{\pgfqpoint{1.826280in}{6.243855in}}{\pgfqpoint{1.815681in}{6.248246in}}{\pgfqpoint{1.804631in}{6.248246in}}%
\pgfpathcurveto{\pgfqpoint{1.793581in}{6.248246in}}{\pgfqpoint{1.782981in}{6.243855in}}{\pgfqpoint{1.775168in}{6.236042in}}%
\pgfpathcurveto{\pgfqpoint{1.767354in}{6.228228in}}{\pgfqpoint{1.762964in}{6.217629in}}{\pgfqpoint{1.762964in}{6.206579in}}%
\pgfpathcurveto{\pgfqpoint{1.762964in}{6.195529in}}{\pgfqpoint{1.767354in}{6.184930in}}{\pgfqpoint{1.775168in}{6.177116in}}%
\pgfpathcurveto{\pgfqpoint{1.782981in}{6.169303in}}{\pgfqpoint{1.793581in}{6.164912in}}{\pgfqpoint{1.804631in}{6.164912in}}%
\pgfpathlineto{\pgfqpoint{1.804631in}{6.164912in}}%
\pgfpathclose%
\pgfusepath{stroke,fill}%
\end{pgfscope}%
\begin{pgfscope}%
\pgfpathrectangle{\pgfqpoint{0.633874in}{5.272501in}}{\pgfqpoint{2.177280in}{2.201755in}}%
\pgfusepath{clip}%
\pgfsetbuttcap%
\pgfsetroundjoin%
\definecolor{currentfill}{rgb}{0.172549,0.627451,0.172549}%
\pgfsetfillcolor{currentfill}%
\pgfsetlinewidth{0.481800pt}%
\definecolor{currentstroke}{rgb}{1.000000,1.000000,1.000000}%
\pgfsetstrokecolor{currentstroke}%
\pgfsetdash{}{0pt}%
\pgfpathmoveto{\pgfqpoint{1.684815in}{6.498512in}}%
\pgfpathcurveto{\pgfqpoint{1.695865in}{6.498512in}}{\pgfqpoint{1.706464in}{6.502902in}}{\pgfqpoint{1.714277in}{6.510715in}}%
\pgfpathcurveto{\pgfqpoint{1.722091in}{6.518529in}}{\pgfqpoint{1.726481in}{6.529128in}}{\pgfqpoint{1.726481in}{6.540178in}}%
\pgfpathcurveto{\pgfqpoint{1.726481in}{6.551228in}}{\pgfqpoint{1.722091in}{6.561827in}}{\pgfqpoint{1.714277in}{6.569641in}}%
\pgfpathcurveto{\pgfqpoint{1.706464in}{6.577455in}}{\pgfqpoint{1.695865in}{6.581845in}}{\pgfqpoint{1.684815in}{6.581845in}}%
\pgfpathcurveto{\pgfqpoint{1.673764in}{6.581845in}}{\pgfqpoint{1.663165in}{6.577455in}}{\pgfqpoint{1.655352in}{6.569641in}}%
\pgfpathcurveto{\pgfqpoint{1.647538in}{6.561827in}}{\pgfqpoint{1.643148in}{6.551228in}}{\pgfqpoint{1.643148in}{6.540178in}}%
\pgfpathcurveto{\pgfqpoint{1.643148in}{6.529128in}}{\pgfqpoint{1.647538in}{6.518529in}}{\pgfqpoint{1.655352in}{6.510715in}}%
\pgfpathcurveto{\pgfqpoint{1.663165in}{6.502902in}}{\pgfqpoint{1.673764in}{6.498512in}}{\pgfqpoint{1.684815in}{6.498512in}}%
\pgfpathlineto{\pgfqpoint{1.684815in}{6.498512in}}%
\pgfpathclose%
\pgfusepath{stroke,fill}%
\end{pgfscope}%
\begin{pgfscope}%
\pgfpathrectangle{\pgfqpoint{0.633874in}{5.272501in}}{\pgfqpoint{2.177280in}{2.201755in}}%
\pgfusepath{clip}%
\pgfsetbuttcap%
\pgfsetroundjoin%
\definecolor{currentfill}{rgb}{0.172549,0.627451,0.172549}%
\pgfsetfillcolor{currentfill}%
\pgfsetlinewidth{0.481800pt}%
\definecolor{currentstroke}{rgb}{1.000000,1.000000,1.000000}%
\pgfsetstrokecolor{currentstroke}%
\pgfsetdash{}{0pt}%
\pgfpathmoveto{\pgfqpoint{1.564998in}{6.164912in}}%
\pgfpathcurveto{\pgfqpoint{1.576049in}{6.164912in}}{\pgfqpoint{1.586648in}{6.169303in}}{\pgfqpoint{1.594461in}{6.177116in}}%
\pgfpathcurveto{\pgfqpoint{1.602275in}{6.184930in}}{\pgfqpoint{1.606665in}{6.195529in}}{\pgfqpoint{1.606665in}{6.206579in}}%
\pgfpathcurveto{\pgfqpoint{1.606665in}{6.217629in}}{\pgfqpoint{1.602275in}{6.228228in}}{\pgfqpoint{1.594461in}{6.236042in}}%
\pgfpathcurveto{\pgfqpoint{1.586648in}{6.243855in}}{\pgfqpoint{1.576049in}{6.248246in}}{\pgfqpoint{1.564998in}{6.248246in}}%
\pgfpathcurveto{\pgfqpoint{1.553948in}{6.248246in}}{\pgfqpoint{1.543349in}{6.243855in}}{\pgfqpoint{1.535536in}{6.236042in}}%
\pgfpathcurveto{\pgfqpoint{1.527722in}{6.228228in}}{\pgfqpoint{1.523332in}{6.217629in}}{\pgfqpoint{1.523332in}{6.206579in}}%
\pgfpathcurveto{\pgfqpoint{1.523332in}{6.195529in}}{\pgfqpoint{1.527722in}{6.184930in}}{\pgfqpoint{1.535536in}{6.177116in}}%
\pgfpathcurveto{\pgfqpoint{1.543349in}{6.169303in}}{\pgfqpoint{1.553948in}{6.164912in}}{\pgfqpoint{1.564998in}{6.164912in}}%
\pgfpathlineto{\pgfqpoint{1.564998in}{6.164912in}}%
\pgfpathclose%
\pgfusepath{stroke,fill}%
\end{pgfscope}%
\begin{pgfscope}%
\pgfpathrectangle{\pgfqpoint{0.633874in}{5.272501in}}{\pgfqpoint{2.177280in}{2.201755in}}%
\pgfusepath{clip}%
\pgfsetbuttcap%
\pgfsetroundjoin%
\definecolor{currentfill}{rgb}{0.121569,0.466667,0.705882}%
\pgfsetfillcolor{currentfill}%
\pgfsetlinewidth{1.003750pt}%
\definecolor{currentstroke}{rgb}{0.121569,0.466667,0.705882}%
\pgfsetstrokecolor{currentstroke}%
\pgfsetdash{}{0pt}%
\pgfsys@defobject{currentmarker}{\pgfqpoint{-0.041667in}{-0.041667in}}{\pgfqpoint{0.041667in}{0.041667in}}{%
\pgfpathmoveto{\pgfqpoint{0.000000in}{-0.041667in}}%
\pgfpathcurveto{\pgfqpoint{0.011050in}{-0.041667in}}{\pgfqpoint{0.021649in}{-0.037276in}}{\pgfqpoint{0.029463in}{-0.029463in}}%
\pgfpathcurveto{\pgfqpoint{0.037276in}{-0.021649in}}{\pgfqpoint{0.041667in}{-0.011050in}}{\pgfqpoint{0.041667in}{0.000000in}}%
\pgfpathcurveto{\pgfqpoint{0.041667in}{0.011050in}}{\pgfqpoint{0.037276in}{0.021649in}}{\pgfqpoint{0.029463in}{0.029463in}}%
\pgfpathcurveto{\pgfqpoint{0.021649in}{0.037276in}}{\pgfqpoint{0.011050in}{0.041667in}}{\pgfqpoint{0.000000in}{0.041667in}}%
\pgfpathcurveto{\pgfqpoint{-0.011050in}{0.041667in}}{\pgfqpoint{-0.021649in}{0.037276in}}{\pgfqpoint{-0.029463in}{0.029463in}}%
\pgfpathcurveto{\pgfqpoint{-0.037276in}{0.021649in}}{\pgfqpoint{-0.041667in}{0.011050in}}{\pgfqpoint{-0.041667in}{0.000000in}}%
\pgfpathcurveto{\pgfqpoint{-0.041667in}{-0.011050in}}{\pgfqpoint{-0.037276in}{-0.021649in}}{\pgfqpoint{-0.029463in}{-0.029463in}}%
\pgfpathcurveto{\pgfqpoint{-0.021649in}{-0.037276in}}{\pgfqpoint{-0.011050in}{-0.041667in}}{\pgfqpoint{0.000000in}{-0.041667in}}%
\pgfpathlineto{\pgfqpoint{0.000000in}{-0.041667in}}%
\pgfpathclose%
\pgfusepath{stroke,fill}%
}%
\end{pgfscope}%
\begin{pgfscope}%
\pgfpathrectangle{\pgfqpoint{0.633874in}{5.272501in}}{\pgfqpoint{2.177280in}{2.201755in}}%
\pgfusepath{clip}%
\pgfsetbuttcap%
\pgfsetroundjoin%
\definecolor{currentfill}{rgb}{1.000000,0.498039,0.054902}%
\pgfsetfillcolor{currentfill}%
\pgfsetlinewidth{1.003750pt}%
\definecolor{currentstroke}{rgb}{1.000000,0.498039,0.054902}%
\pgfsetstrokecolor{currentstroke}%
\pgfsetdash{}{0pt}%
\pgfsys@defobject{currentmarker}{\pgfqpoint{-0.041667in}{-0.041667in}}{\pgfqpoint{0.041667in}{0.041667in}}{%
\pgfpathmoveto{\pgfqpoint{0.000000in}{-0.041667in}}%
\pgfpathcurveto{\pgfqpoint{0.011050in}{-0.041667in}}{\pgfqpoint{0.021649in}{-0.037276in}}{\pgfqpoint{0.029463in}{-0.029463in}}%
\pgfpathcurveto{\pgfqpoint{0.037276in}{-0.021649in}}{\pgfqpoint{0.041667in}{-0.011050in}}{\pgfqpoint{0.041667in}{0.000000in}}%
\pgfpathcurveto{\pgfqpoint{0.041667in}{0.011050in}}{\pgfqpoint{0.037276in}{0.021649in}}{\pgfqpoint{0.029463in}{0.029463in}}%
\pgfpathcurveto{\pgfqpoint{0.021649in}{0.037276in}}{\pgfqpoint{0.011050in}{0.041667in}}{\pgfqpoint{0.000000in}{0.041667in}}%
\pgfpathcurveto{\pgfqpoint{-0.011050in}{0.041667in}}{\pgfqpoint{-0.021649in}{0.037276in}}{\pgfqpoint{-0.029463in}{0.029463in}}%
\pgfpathcurveto{\pgfqpoint{-0.037276in}{0.021649in}}{\pgfqpoint{-0.041667in}{0.011050in}}{\pgfqpoint{-0.041667in}{0.000000in}}%
\pgfpathcurveto{\pgfqpoint{-0.041667in}{-0.011050in}}{\pgfqpoint{-0.037276in}{-0.021649in}}{\pgfqpoint{-0.029463in}{-0.029463in}}%
\pgfpathcurveto{\pgfqpoint{-0.021649in}{-0.037276in}}{\pgfqpoint{-0.011050in}{-0.041667in}}{\pgfqpoint{0.000000in}{-0.041667in}}%
\pgfpathlineto{\pgfqpoint{0.000000in}{-0.041667in}}%
\pgfpathclose%
\pgfusepath{stroke,fill}%
}%
\end{pgfscope}%
\begin{pgfscope}%
\pgfpathrectangle{\pgfqpoint{0.633874in}{5.272501in}}{\pgfqpoint{2.177280in}{2.201755in}}%
\pgfusepath{clip}%
\pgfsetbuttcap%
\pgfsetroundjoin%
\definecolor{currentfill}{rgb}{0.172549,0.627451,0.172549}%
\pgfsetfillcolor{currentfill}%
\pgfsetlinewidth{1.003750pt}%
\definecolor{currentstroke}{rgb}{0.172549,0.627451,0.172549}%
\pgfsetstrokecolor{currentstroke}%
\pgfsetdash{}{0pt}%
\pgfsys@defobject{currentmarker}{\pgfqpoint{-0.041667in}{-0.041667in}}{\pgfqpoint{0.041667in}{0.041667in}}{%
\pgfpathmoveto{\pgfqpoint{0.000000in}{-0.041667in}}%
\pgfpathcurveto{\pgfqpoint{0.011050in}{-0.041667in}}{\pgfqpoint{0.021649in}{-0.037276in}}{\pgfqpoint{0.029463in}{-0.029463in}}%
\pgfpathcurveto{\pgfqpoint{0.037276in}{-0.021649in}}{\pgfqpoint{0.041667in}{-0.011050in}}{\pgfqpoint{0.041667in}{0.000000in}}%
\pgfpathcurveto{\pgfqpoint{0.041667in}{0.011050in}}{\pgfqpoint{0.037276in}{0.021649in}}{\pgfqpoint{0.029463in}{0.029463in}}%
\pgfpathcurveto{\pgfqpoint{0.021649in}{0.037276in}}{\pgfqpoint{0.011050in}{0.041667in}}{\pgfqpoint{0.000000in}{0.041667in}}%
\pgfpathcurveto{\pgfqpoint{-0.011050in}{0.041667in}}{\pgfqpoint{-0.021649in}{0.037276in}}{\pgfqpoint{-0.029463in}{0.029463in}}%
\pgfpathcurveto{\pgfqpoint{-0.037276in}{0.021649in}}{\pgfqpoint{-0.041667in}{0.011050in}}{\pgfqpoint{-0.041667in}{0.000000in}}%
\pgfpathcurveto{\pgfqpoint{-0.041667in}{-0.011050in}}{\pgfqpoint{-0.037276in}{-0.021649in}}{\pgfqpoint{-0.029463in}{-0.029463in}}%
\pgfpathcurveto{\pgfqpoint{-0.021649in}{-0.037276in}}{\pgfqpoint{-0.011050in}{-0.041667in}}{\pgfqpoint{0.000000in}{-0.041667in}}%
\pgfpathlineto{\pgfqpoint{0.000000in}{-0.041667in}}%
\pgfpathclose%
\pgfusepath{stroke,fill}%
}%
\end{pgfscope}%
\begin{pgfscope}%
\pgfsetbuttcap%
\pgfsetroundjoin%
\definecolor{currentfill}{rgb}{0.000000,0.000000,0.000000}%
\pgfsetfillcolor{currentfill}%
\pgfsetlinewidth{0.803000pt}%
\definecolor{currentstroke}{rgb}{0.000000,0.000000,0.000000}%
\pgfsetstrokecolor{currentstroke}%
\pgfsetdash{}{0pt}%
\pgfsys@defobject{currentmarker}{\pgfqpoint{0.000000in}{-0.048611in}}{\pgfqpoint{0.000000in}{0.000000in}}{%
\pgfpathmoveto{\pgfqpoint{0.000000in}{0.000000in}}%
\pgfpathlineto{\pgfqpoint{0.000000in}{-0.048611in}}%
\pgfusepath{stroke,fill}%
}%
\begin{pgfscope}%
\pgfsys@transformshift{0.806163in}{5.272501in}%
\pgfsys@useobject{currentmarker}{}%
\end{pgfscope}%
\end{pgfscope}%
\begin{pgfscope}%
\pgfsetbuttcap%
\pgfsetroundjoin%
\definecolor{currentfill}{rgb}{0.000000,0.000000,0.000000}%
\pgfsetfillcolor{currentfill}%
\pgfsetlinewidth{0.803000pt}%
\definecolor{currentstroke}{rgb}{0.000000,0.000000,0.000000}%
\pgfsetstrokecolor{currentstroke}%
\pgfsetdash{}{0pt}%
\pgfsys@defobject{currentmarker}{\pgfqpoint{0.000000in}{-0.048611in}}{\pgfqpoint{0.000000in}{0.000000in}}{%
\pgfpathmoveto{\pgfqpoint{0.000000in}{0.000000in}}%
\pgfpathlineto{\pgfqpoint{0.000000in}{-0.048611in}}%
\pgfusepath{stroke,fill}%
}%
\begin{pgfscope}%
\pgfsys@transformshift{1.604937in}{5.272501in}%
\pgfsys@useobject{currentmarker}{}%
\end{pgfscope}%
\end{pgfscope}%
\begin{pgfscope}%
\pgfsetbuttcap%
\pgfsetroundjoin%
\definecolor{currentfill}{rgb}{0.000000,0.000000,0.000000}%
\pgfsetfillcolor{currentfill}%
\pgfsetlinewidth{0.803000pt}%
\definecolor{currentstroke}{rgb}{0.000000,0.000000,0.000000}%
\pgfsetstrokecolor{currentstroke}%
\pgfsetdash{}{0pt}%
\pgfsys@defobject{currentmarker}{\pgfqpoint{0.000000in}{-0.048611in}}{\pgfqpoint{0.000000in}{0.000000in}}{%
\pgfpathmoveto{\pgfqpoint{0.000000in}{0.000000in}}%
\pgfpathlineto{\pgfqpoint{0.000000in}{-0.048611in}}%
\pgfusepath{stroke,fill}%
}%
\begin{pgfscope}%
\pgfsys@transformshift{2.403711in}{5.272501in}%
\pgfsys@useobject{currentmarker}{}%
\end{pgfscope}%
\end{pgfscope}%
\begin{pgfscope}%
\pgfsetbuttcap%
\pgfsetroundjoin%
\definecolor{currentfill}{rgb}{0.000000,0.000000,0.000000}%
\pgfsetfillcolor{currentfill}%
\pgfsetlinewidth{0.803000pt}%
\definecolor{currentstroke}{rgb}{0.000000,0.000000,0.000000}%
\pgfsetstrokecolor{currentstroke}%
\pgfsetdash{}{0pt}%
\pgfsys@defobject{currentmarker}{\pgfqpoint{-0.048611in}{0.000000in}}{\pgfqpoint{-0.000000in}{0.000000in}}{%
\pgfpathmoveto{\pgfqpoint{-0.000000in}{0.000000in}}%
\pgfpathlineto{\pgfqpoint{-0.048611in}{0.000000in}}%
\pgfusepath{stroke,fill}%
}%
\begin{pgfscope}%
\pgfsys@transformshift{0.633874in}{5.372581in}%
\pgfsys@useobject{currentmarker}{}%
\end{pgfscope}%
\end{pgfscope}%
\begin{pgfscope}%
\definecolor{textcolor}{rgb}{0.000000,0.000000,0.000000}%
\pgfsetstrokecolor{textcolor}%
\pgfsetfillcolor{textcolor}%
\pgftext[x=0.359182in, y=5.324356in, left, base]{\color{textcolor}\rmfamily\fontsize{10.000000}{12.000000}\selectfont \(\displaystyle {2.0}\)}%
\end{pgfscope}%
\begin{pgfscope}%
\pgfsetbuttcap%
\pgfsetroundjoin%
\definecolor{currentfill}{rgb}{0.000000,0.000000,0.000000}%
\pgfsetfillcolor{currentfill}%
\pgfsetlinewidth{0.803000pt}%
\definecolor{currentstroke}{rgb}{0.000000,0.000000,0.000000}%
\pgfsetstrokecolor{currentstroke}%
\pgfsetdash{}{0pt}%
\pgfsys@defobject{currentmarker}{\pgfqpoint{-0.048611in}{0.000000in}}{\pgfqpoint{-0.000000in}{0.000000in}}{%
\pgfpathmoveto{\pgfqpoint{-0.000000in}{0.000000in}}%
\pgfpathlineto{\pgfqpoint{-0.048611in}{0.000000in}}%
\pgfusepath{stroke,fill}%
}%
\begin{pgfscope}%
\pgfsys@transformshift{0.633874in}{5.789580in}%
\pgfsys@useobject{currentmarker}{}%
\end{pgfscope}%
\end{pgfscope}%
\begin{pgfscope}%
\definecolor{textcolor}{rgb}{0.000000,0.000000,0.000000}%
\pgfsetstrokecolor{textcolor}%
\pgfsetfillcolor{textcolor}%
\pgftext[x=0.359182in, y=5.741355in, left, base]{\color{textcolor}\rmfamily\fontsize{10.000000}{12.000000}\selectfont \(\displaystyle {2.5}\)}%
\end{pgfscope}%
\begin{pgfscope}%
\pgfsetbuttcap%
\pgfsetroundjoin%
\definecolor{currentfill}{rgb}{0.000000,0.000000,0.000000}%
\pgfsetfillcolor{currentfill}%
\pgfsetlinewidth{0.803000pt}%
\definecolor{currentstroke}{rgb}{0.000000,0.000000,0.000000}%
\pgfsetstrokecolor{currentstroke}%
\pgfsetdash{}{0pt}%
\pgfsys@defobject{currentmarker}{\pgfqpoint{-0.048611in}{0.000000in}}{\pgfqpoint{-0.000000in}{0.000000in}}{%
\pgfpathmoveto{\pgfqpoint{-0.000000in}{0.000000in}}%
\pgfpathlineto{\pgfqpoint{-0.048611in}{0.000000in}}%
\pgfusepath{stroke,fill}%
}%
\begin{pgfscope}%
\pgfsys@transformshift{0.633874in}{6.206579in}%
\pgfsys@useobject{currentmarker}{}%
\end{pgfscope}%
\end{pgfscope}%
\begin{pgfscope}%
\definecolor{textcolor}{rgb}{0.000000,0.000000,0.000000}%
\pgfsetstrokecolor{textcolor}%
\pgfsetfillcolor{textcolor}%
\pgftext[x=0.359182in, y=6.158354in, left, base]{\color{textcolor}\rmfamily\fontsize{10.000000}{12.000000}\selectfont \(\displaystyle {3.0}\)}%
\end{pgfscope}%
\begin{pgfscope}%
\pgfsetbuttcap%
\pgfsetroundjoin%
\definecolor{currentfill}{rgb}{0.000000,0.000000,0.000000}%
\pgfsetfillcolor{currentfill}%
\pgfsetlinewidth{0.803000pt}%
\definecolor{currentstroke}{rgb}{0.000000,0.000000,0.000000}%
\pgfsetstrokecolor{currentstroke}%
\pgfsetdash{}{0pt}%
\pgfsys@defobject{currentmarker}{\pgfqpoint{-0.048611in}{0.000000in}}{\pgfqpoint{-0.000000in}{0.000000in}}{%
\pgfpathmoveto{\pgfqpoint{-0.000000in}{0.000000in}}%
\pgfpathlineto{\pgfqpoint{-0.048611in}{0.000000in}}%
\pgfusepath{stroke,fill}%
}%
\begin{pgfscope}%
\pgfsys@transformshift{0.633874in}{6.623578in}%
\pgfsys@useobject{currentmarker}{}%
\end{pgfscope}%
\end{pgfscope}%
\begin{pgfscope}%
\definecolor{textcolor}{rgb}{0.000000,0.000000,0.000000}%
\pgfsetstrokecolor{textcolor}%
\pgfsetfillcolor{textcolor}%
\pgftext[x=0.359182in, y=6.575353in, left, base]{\color{textcolor}\rmfamily\fontsize{10.000000}{12.000000}\selectfont \(\displaystyle {3.5}\)}%
\end{pgfscope}%
\begin{pgfscope}%
\pgfsetbuttcap%
\pgfsetroundjoin%
\definecolor{currentfill}{rgb}{0.000000,0.000000,0.000000}%
\pgfsetfillcolor{currentfill}%
\pgfsetlinewidth{0.803000pt}%
\definecolor{currentstroke}{rgb}{0.000000,0.000000,0.000000}%
\pgfsetstrokecolor{currentstroke}%
\pgfsetdash{}{0pt}%
\pgfsys@defobject{currentmarker}{\pgfqpoint{-0.048611in}{0.000000in}}{\pgfqpoint{-0.000000in}{0.000000in}}{%
\pgfpathmoveto{\pgfqpoint{-0.000000in}{0.000000in}}%
\pgfpathlineto{\pgfqpoint{-0.048611in}{0.000000in}}%
\pgfusepath{stroke,fill}%
}%
\begin{pgfscope}%
\pgfsys@transformshift{0.633874in}{7.040577in}%
\pgfsys@useobject{currentmarker}{}%
\end{pgfscope}%
\end{pgfscope}%
\begin{pgfscope}%
\definecolor{textcolor}{rgb}{0.000000,0.000000,0.000000}%
\pgfsetstrokecolor{textcolor}%
\pgfsetfillcolor{textcolor}%
\pgftext[x=0.359182in, y=6.992352in, left, base]{\color{textcolor}\rmfamily\fontsize{10.000000}{12.000000}\selectfont \(\displaystyle {4.0}\)}%
\end{pgfscope}%
\begin{pgfscope}%
\pgfsetbuttcap%
\pgfsetroundjoin%
\definecolor{currentfill}{rgb}{0.000000,0.000000,0.000000}%
\pgfsetfillcolor{currentfill}%
\pgfsetlinewidth{0.803000pt}%
\definecolor{currentstroke}{rgb}{0.000000,0.000000,0.000000}%
\pgfsetstrokecolor{currentstroke}%
\pgfsetdash{}{0pt}%
\pgfsys@defobject{currentmarker}{\pgfqpoint{-0.048611in}{0.000000in}}{\pgfqpoint{-0.000000in}{0.000000in}}{%
\pgfpathmoveto{\pgfqpoint{-0.000000in}{0.000000in}}%
\pgfpathlineto{\pgfqpoint{-0.048611in}{0.000000in}}%
\pgfusepath{stroke,fill}%
}%
\begin{pgfscope}%
\pgfsys@transformshift{0.633874in}{7.457576in}%
\pgfsys@useobject{currentmarker}{}%
\end{pgfscope}%
\end{pgfscope}%
\begin{pgfscope}%
\definecolor{textcolor}{rgb}{0.000000,0.000000,0.000000}%
\pgfsetstrokecolor{textcolor}%
\pgfsetfillcolor{textcolor}%
\pgftext[x=0.359182in, y=7.409351in, left, base]{\color{textcolor}\rmfamily\fontsize{10.000000}{12.000000}\selectfont \(\displaystyle {4.5}\)}%
\end{pgfscope}%
\begin{pgfscope}%
\definecolor{textcolor}{rgb}{0.000000,0.000000,0.000000}%
\pgfsetstrokecolor{textcolor}%
\pgfsetfillcolor{textcolor}%
\pgftext[x=0.303626in,y=6.373379in,,bottom,rotate=90.000000]{\color{textcolor}\rmfamily\fontsize{10.000000}{12.000000}\selectfont sepal\_width}%
\end{pgfscope}%
\begin{pgfscope}%
\pgfsetrectcap%
\pgfsetmiterjoin%
\pgfsetlinewidth{0.803000pt}%
\definecolor{currentstroke}{rgb}{0.000000,0.000000,0.000000}%
\pgfsetstrokecolor{currentstroke}%
\pgfsetdash{}{0pt}%
\pgfpathmoveto{\pgfqpoint{0.633874in}{5.272501in}}%
\pgfpathlineto{\pgfqpoint{0.633874in}{7.474256in}}%
\pgfusepath{stroke}%
\end{pgfscope}%
\begin{pgfscope}%
\pgfsetrectcap%
\pgfsetmiterjoin%
\pgfsetlinewidth{0.803000pt}%
\definecolor{currentstroke}{rgb}{0.000000,0.000000,0.000000}%
\pgfsetstrokecolor{currentstroke}%
\pgfsetdash{}{0pt}%
\pgfpathmoveto{\pgfqpoint{0.633874in}{5.272501in}}%
\pgfpathlineto{\pgfqpoint{2.811154in}{5.272501in}}%
\pgfusepath{stroke}%
\end{pgfscope}%
\begin{pgfscope}%
\pgfsetbuttcap%
\pgfsetmiterjoin%
\definecolor{currentfill}{rgb}{1.000000,1.000000,1.000000}%
\pgfsetfillcolor{currentfill}%
\pgfsetlinewidth{0.000000pt}%
\definecolor{currentstroke}{rgb}{0.000000,0.000000,0.000000}%
\pgfsetstrokecolor{currentstroke}%
\pgfsetstrokeopacity{0.000000}%
\pgfsetdash{}{0pt}%
\pgfpathmoveto{\pgfqpoint{2.963410in}{5.272501in}}%
\pgfpathlineto{\pgfqpoint{5.140690in}{5.272501in}}%
\pgfpathlineto{\pgfqpoint{5.140690in}{7.474256in}}%
\pgfpathlineto{\pgfqpoint{2.963410in}{7.474256in}}%
\pgfpathlineto{\pgfqpoint{2.963410in}{5.272501in}}%
\pgfpathclose%
\pgfusepath{fill}%
\end{pgfscope}%
\begin{pgfscope}%
\pgfsetbuttcap%
\pgfsetroundjoin%
\definecolor{currentfill}{rgb}{0.000000,0.000000,0.000000}%
\pgfsetfillcolor{currentfill}%
\pgfsetlinewidth{0.803000pt}%
\definecolor{currentstroke}{rgb}{0.000000,0.000000,0.000000}%
\pgfsetstrokecolor{currentstroke}%
\pgfsetdash{}{0pt}%
\pgfsys@defobject{currentmarker}{\pgfqpoint{0.000000in}{-0.048611in}}{\pgfqpoint{0.000000in}{0.000000in}}{%
\pgfpathmoveto{\pgfqpoint{0.000000in}{0.000000in}}%
\pgfpathlineto{\pgfqpoint{0.000000in}{-0.048611in}}%
\pgfusepath{stroke,fill}%
}%
\begin{pgfscope}%
\pgfsys@transformshift{3.316699in}{5.272501in}%
\pgfsys@useobject{currentmarker}{}%
\end{pgfscope}%
\end{pgfscope}%
\begin{pgfscope}%
\pgfsetbuttcap%
\pgfsetroundjoin%
\definecolor{currentfill}{rgb}{0.000000,0.000000,0.000000}%
\pgfsetfillcolor{currentfill}%
\pgfsetlinewidth{0.803000pt}%
\definecolor{currentstroke}{rgb}{0.000000,0.000000,0.000000}%
\pgfsetstrokecolor{currentstroke}%
\pgfsetdash{}{0pt}%
\pgfsys@defobject{currentmarker}{\pgfqpoint{0.000000in}{-0.048611in}}{\pgfqpoint{0.000000in}{0.000000in}}{%
\pgfpathmoveto{\pgfqpoint{0.000000in}{0.000000in}}%
\pgfpathlineto{\pgfqpoint{0.000000in}{-0.048611in}}%
\pgfusepath{stroke,fill}%
}%
\begin{pgfscope}%
\pgfsys@transformshift{3.907452in}{5.272501in}%
\pgfsys@useobject{currentmarker}{}%
\end{pgfscope}%
\end{pgfscope}%
\begin{pgfscope}%
\pgfsetbuttcap%
\pgfsetroundjoin%
\definecolor{currentfill}{rgb}{0.000000,0.000000,0.000000}%
\pgfsetfillcolor{currentfill}%
\pgfsetlinewidth{0.803000pt}%
\definecolor{currentstroke}{rgb}{0.000000,0.000000,0.000000}%
\pgfsetstrokecolor{currentstroke}%
\pgfsetdash{}{0pt}%
\pgfsys@defobject{currentmarker}{\pgfqpoint{0.000000in}{-0.048611in}}{\pgfqpoint{0.000000in}{0.000000in}}{%
\pgfpathmoveto{\pgfqpoint{0.000000in}{0.000000in}}%
\pgfpathlineto{\pgfqpoint{0.000000in}{-0.048611in}}%
\pgfusepath{stroke,fill}%
}%
\begin{pgfscope}%
\pgfsys@transformshift{4.498204in}{5.272501in}%
\pgfsys@useobject{currentmarker}{}%
\end{pgfscope}%
\end{pgfscope}%
\begin{pgfscope}%
\pgfsetbuttcap%
\pgfsetroundjoin%
\definecolor{currentfill}{rgb}{0.000000,0.000000,0.000000}%
\pgfsetfillcolor{currentfill}%
\pgfsetlinewidth{0.803000pt}%
\definecolor{currentstroke}{rgb}{0.000000,0.000000,0.000000}%
\pgfsetstrokecolor{currentstroke}%
\pgfsetdash{}{0pt}%
\pgfsys@defobject{currentmarker}{\pgfqpoint{0.000000in}{-0.048611in}}{\pgfqpoint{0.000000in}{0.000000in}}{%
\pgfpathmoveto{\pgfqpoint{0.000000in}{0.000000in}}%
\pgfpathlineto{\pgfqpoint{0.000000in}{-0.048611in}}%
\pgfusepath{stroke,fill}%
}%
\begin{pgfscope}%
\pgfsys@transformshift{5.088957in}{5.272501in}%
\pgfsys@useobject{currentmarker}{}%
\end{pgfscope}%
\end{pgfscope}%
\begin{pgfscope}%
\pgfsetbuttcap%
\pgfsetroundjoin%
\definecolor{currentfill}{rgb}{0.000000,0.000000,0.000000}%
\pgfsetfillcolor{currentfill}%
\pgfsetlinewidth{0.803000pt}%
\definecolor{currentstroke}{rgb}{0.000000,0.000000,0.000000}%
\pgfsetstrokecolor{currentstroke}%
\pgfsetdash{}{0pt}%
\pgfsys@defobject{currentmarker}{\pgfqpoint{-0.048611in}{0.000000in}}{\pgfqpoint{-0.000000in}{0.000000in}}{%
\pgfpathmoveto{\pgfqpoint{-0.000000in}{0.000000in}}%
\pgfpathlineto{\pgfqpoint{-0.048611in}{0.000000in}}%
\pgfusepath{stroke,fill}%
}%
\begin{pgfscope}%
\pgfsys@transformshift{2.963410in}{5.372581in}%
\pgfsys@useobject{currentmarker}{}%
\end{pgfscope}%
\end{pgfscope}%
\begin{pgfscope}%
\pgfsetbuttcap%
\pgfsetroundjoin%
\definecolor{currentfill}{rgb}{0.000000,0.000000,0.000000}%
\pgfsetfillcolor{currentfill}%
\pgfsetlinewidth{0.803000pt}%
\definecolor{currentstroke}{rgb}{0.000000,0.000000,0.000000}%
\pgfsetstrokecolor{currentstroke}%
\pgfsetdash{}{0pt}%
\pgfsys@defobject{currentmarker}{\pgfqpoint{-0.048611in}{0.000000in}}{\pgfqpoint{-0.000000in}{0.000000in}}{%
\pgfpathmoveto{\pgfqpoint{-0.000000in}{0.000000in}}%
\pgfpathlineto{\pgfqpoint{-0.048611in}{0.000000in}}%
\pgfusepath{stroke,fill}%
}%
\begin{pgfscope}%
\pgfsys@transformshift{2.963410in}{5.789580in}%
\pgfsys@useobject{currentmarker}{}%
\end{pgfscope}%
\end{pgfscope}%
\begin{pgfscope}%
\pgfsetbuttcap%
\pgfsetroundjoin%
\definecolor{currentfill}{rgb}{0.000000,0.000000,0.000000}%
\pgfsetfillcolor{currentfill}%
\pgfsetlinewidth{0.803000pt}%
\definecolor{currentstroke}{rgb}{0.000000,0.000000,0.000000}%
\pgfsetstrokecolor{currentstroke}%
\pgfsetdash{}{0pt}%
\pgfsys@defobject{currentmarker}{\pgfqpoint{-0.048611in}{0.000000in}}{\pgfqpoint{-0.000000in}{0.000000in}}{%
\pgfpathmoveto{\pgfqpoint{-0.000000in}{0.000000in}}%
\pgfpathlineto{\pgfqpoint{-0.048611in}{0.000000in}}%
\pgfusepath{stroke,fill}%
}%
\begin{pgfscope}%
\pgfsys@transformshift{2.963410in}{6.206579in}%
\pgfsys@useobject{currentmarker}{}%
\end{pgfscope}%
\end{pgfscope}%
\begin{pgfscope}%
\pgfsetbuttcap%
\pgfsetroundjoin%
\definecolor{currentfill}{rgb}{0.000000,0.000000,0.000000}%
\pgfsetfillcolor{currentfill}%
\pgfsetlinewidth{0.803000pt}%
\definecolor{currentstroke}{rgb}{0.000000,0.000000,0.000000}%
\pgfsetstrokecolor{currentstroke}%
\pgfsetdash{}{0pt}%
\pgfsys@defobject{currentmarker}{\pgfqpoint{-0.048611in}{0.000000in}}{\pgfqpoint{-0.000000in}{0.000000in}}{%
\pgfpathmoveto{\pgfqpoint{-0.000000in}{0.000000in}}%
\pgfpathlineto{\pgfqpoint{-0.048611in}{0.000000in}}%
\pgfusepath{stroke,fill}%
}%
\begin{pgfscope}%
\pgfsys@transformshift{2.963410in}{6.623578in}%
\pgfsys@useobject{currentmarker}{}%
\end{pgfscope}%
\end{pgfscope}%
\begin{pgfscope}%
\pgfsetbuttcap%
\pgfsetroundjoin%
\definecolor{currentfill}{rgb}{0.000000,0.000000,0.000000}%
\pgfsetfillcolor{currentfill}%
\pgfsetlinewidth{0.803000pt}%
\definecolor{currentstroke}{rgb}{0.000000,0.000000,0.000000}%
\pgfsetstrokecolor{currentstroke}%
\pgfsetdash{}{0pt}%
\pgfsys@defobject{currentmarker}{\pgfqpoint{-0.048611in}{0.000000in}}{\pgfqpoint{-0.000000in}{0.000000in}}{%
\pgfpathmoveto{\pgfqpoint{-0.000000in}{0.000000in}}%
\pgfpathlineto{\pgfqpoint{-0.048611in}{0.000000in}}%
\pgfusepath{stroke,fill}%
}%
\begin{pgfscope}%
\pgfsys@transformshift{2.963410in}{7.040577in}%
\pgfsys@useobject{currentmarker}{}%
\end{pgfscope}%
\end{pgfscope}%
\begin{pgfscope}%
\pgfsetbuttcap%
\pgfsetroundjoin%
\definecolor{currentfill}{rgb}{0.000000,0.000000,0.000000}%
\pgfsetfillcolor{currentfill}%
\pgfsetlinewidth{0.803000pt}%
\definecolor{currentstroke}{rgb}{0.000000,0.000000,0.000000}%
\pgfsetstrokecolor{currentstroke}%
\pgfsetdash{}{0pt}%
\pgfsys@defobject{currentmarker}{\pgfqpoint{-0.048611in}{0.000000in}}{\pgfqpoint{-0.000000in}{0.000000in}}{%
\pgfpathmoveto{\pgfqpoint{-0.000000in}{0.000000in}}%
\pgfpathlineto{\pgfqpoint{-0.048611in}{0.000000in}}%
\pgfusepath{stroke,fill}%
}%
\begin{pgfscope}%
\pgfsys@transformshift{2.963410in}{7.457576in}%
\pgfsys@useobject{currentmarker}{}%
\end{pgfscope}%
\end{pgfscope}%
\begin{pgfscope}%
\pgfsetrectcap%
\pgfsetmiterjoin%
\pgfsetlinewidth{0.803000pt}%
\definecolor{currentstroke}{rgb}{0.000000,0.000000,0.000000}%
\pgfsetstrokecolor{currentstroke}%
\pgfsetdash{}{0pt}%
\pgfpathmoveto{\pgfqpoint{2.963410in}{5.272501in}}%
\pgfpathlineto{\pgfqpoint{2.963410in}{7.474256in}}%
\pgfusepath{stroke}%
\end{pgfscope}%
\begin{pgfscope}%
\pgfsetrectcap%
\pgfsetmiterjoin%
\pgfsetlinewidth{0.803000pt}%
\definecolor{currentstroke}{rgb}{0.000000,0.000000,0.000000}%
\pgfsetstrokecolor{currentstroke}%
\pgfsetdash{}{0pt}%
\pgfpathmoveto{\pgfqpoint{2.963410in}{5.272501in}}%
\pgfpathlineto{\pgfqpoint{5.140690in}{5.272501in}}%
\pgfusepath{stroke}%
\end{pgfscope}%
\begin{pgfscope}%
\pgfsetbuttcap%
\pgfsetmiterjoin%
\definecolor{currentfill}{rgb}{1.000000,1.000000,1.000000}%
\pgfsetfillcolor{currentfill}%
\pgfsetlinewidth{0.000000pt}%
\definecolor{currentstroke}{rgb}{0.000000,0.000000,0.000000}%
\pgfsetstrokecolor{currentstroke}%
\pgfsetstrokeopacity{0.000000}%
\pgfsetdash{}{0pt}%
\pgfpathmoveto{\pgfqpoint{5.292946in}{5.272501in}}%
\pgfpathlineto{\pgfqpoint{7.470226in}{5.272501in}}%
\pgfpathlineto{\pgfqpoint{7.470226in}{7.474256in}}%
\pgfpathlineto{\pgfqpoint{5.292946in}{7.474256in}}%
\pgfpathlineto{\pgfqpoint{5.292946in}{5.272501in}}%
\pgfpathclose%
\pgfusepath{fill}%
\end{pgfscope}%
\begin{pgfscope}%
\pgfpathrectangle{\pgfqpoint{5.292946in}{5.272501in}}{\pgfqpoint{2.177280in}{2.201755in}}%
\pgfusepath{clip}%
\pgfsetbuttcap%
\pgfsetroundjoin%
\definecolor{currentfill}{rgb}{0.121569,0.466667,0.705882}%
\pgfsetfillcolor{currentfill}%
\pgfsetlinewidth{0.481800pt}%
\definecolor{currentstroke}{rgb}{1.000000,1.000000,1.000000}%
\pgfsetstrokecolor{currentstroke}%
\pgfsetdash{}{0pt}%
\pgfpathmoveto{\pgfqpoint{5.575125in}{6.581911in}}%
\pgfpathcurveto{\pgfqpoint{5.586176in}{6.581911in}}{\pgfqpoint{5.596775in}{6.586302in}}{\pgfqpoint{5.604588in}{6.594115in}}%
\pgfpathcurveto{\pgfqpoint{5.612402in}{6.601929in}}{\pgfqpoint{5.616792in}{6.612528in}}{\pgfqpoint{5.616792in}{6.623578in}}%
\pgfpathcurveto{\pgfqpoint{5.616792in}{6.634628in}}{\pgfqpoint{5.612402in}{6.645227in}}{\pgfqpoint{5.604588in}{6.653041in}}%
\pgfpathcurveto{\pgfqpoint{5.596775in}{6.660854in}}{\pgfqpoint{5.586176in}{6.665245in}}{\pgfqpoint{5.575125in}{6.665245in}}%
\pgfpathcurveto{\pgfqpoint{5.564075in}{6.665245in}}{\pgfqpoint{5.553476in}{6.660854in}}{\pgfqpoint{5.545663in}{6.653041in}}%
\pgfpathcurveto{\pgfqpoint{5.537849in}{6.645227in}}{\pgfqpoint{5.533459in}{6.634628in}}{\pgfqpoint{5.533459in}{6.623578in}}%
\pgfpathcurveto{\pgfqpoint{5.533459in}{6.612528in}}{\pgfqpoint{5.537849in}{6.601929in}}{\pgfqpoint{5.545663in}{6.594115in}}%
\pgfpathcurveto{\pgfqpoint{5.553476in}{6.586302in}}{\pgfqpoint{5.564075in}{6.581911in}}{\pgfqpoint{5.575125in}{6.581911in}}%
\pgfpathlineto{\pgfqpoint{5.575125in}{6.581911in}}%
\pgfpathclose%
\pgfusepath{stroke,fill}%
\end{pgfscope}%
\begin{pgfscope}%
\pgfpathrectangle{\pgfqpoint{5.292946in}{5.272501in}}{\pgfqpoint{2.177280in}{2.201755in}}%
\pgfusepath{clip}%
\pgfsetbuttcap%
\pgfsetroundjoin%
\definecolor{currentfill}{rgb}{0.121569,0.466667,0.705882}%
\pgfsetfillcolor{currentfill}%
\pgfsetlinewidth{0.481800pt}%
\definecolor{currentstroke}{rgb}{1.000000,1.000000,1.000000}%
\pgfsetstrokecolor{currentstroke}%
\pgfsetdash{}{0pt}%
\pgfpathmoveto{\pgfqpoint{5.575125in}{6.164912in}}%
\pgfpathcurveto{\pgfqpoint{5.586176in}{6.164912in}}{\pgfqpoint{5.596775in}{6.169303in}}{\pgfqpoint{5.604588in}{6.177116in}}%
\pgfpathcurveto{\pgfqpoint{5.612402in}{6.184930in}}{\pgfqpoint{5.616792in}{6.195529in}}{\pgfqpoint{5.616792in}{6.206579in}}%
\pgfpathcurveto{\pgfqpoint{5.616792in}{6.217629in}}{\pgfqpoint{5.612402in}{6.228228in}}{\pgfqpoint{5.604588in}{6.236042in}}%
\pgfpathcurveto{\pgfqpoint{5.596775in}{6.243855in}}{\pgfqpoint{5.586176in}{6.248246in}}{\pgfqpoint{5.575125in}{6.248246in}}%
\pgfpathcurveto{\pgfqpoint{5.564075in}{6.248246in}}{\pgfqpoint{5.553476in}{6.243855in}}{\pgfqpoint{5.545663in}{6.236042in}}%
\pgfpathcurveto{\pgfqpoint{5.537849in}{6.228228in}}{\pgfqpoint{5.533459in}{6.217629in}}{\pgfqpoint{5.533459in}{6.206579in}}%
\pgfpathcurveto{\pgfqpoint{5.533459in}{6.195529in}}{\pgfqpoint{5.537849in}{6.184930in}}{\pgfqpoint{5.545663in}{6.177116in}}%
\pgfpathcurveto{\pgfqpoint{5.553476in}{6.169303in}}{\pgfqpoint{5.564075in}{6.164912in}}{\pgfqpoint{5.575125in}{6.164912in}}%
\pgfpathlineto{\pgfqpoint{5.575125in}{6.164912in}}%
\pgfpathclose%
\pgfusepath{stroke,fill}%
\end{pgfscope}%
\begin{pgfscope}%
\pgfpathrectangle{\pgfqpoint{5.292946in}{5.272501in}}{\pgfqpoint{2.177280in}{2.201755in}}%
\pgfusepath{clip}%
\pgfsetbuttcap%
\pgfsetroundjoin%
\definecolor{currentfill}{rgb}{0.121569,0.466667,0.705882}%
\pgfsetfillcolor{currentfill}%
\pgfsetlinewidth{0.481800pt}%
\definecolor{currentstroke}{rgb}{1.000000,1.000000,1.000000}%
\pgfsetstrokecolor{currentstroke}%
\pgfsetdash{}{0pt}%
\pgfpathmoveto{\pgfqpoint{5.546420in}{6.331712in}}%
\pgfpathcurveto{\pgfqpoint{5.557470in}{6.331712in}}{\pgfqpoint{5.568069in}{6.336102in}}{\pgfqpoint{5.575883in}{6.343916in}}%
\pgfpathcurveto{\pgfqpoint{5.583697in}{6.351729in}}{\pgfqpoint{5.588087in}{6.362328in}}{\pgfqpoint{5.588087in}{6.373379in}}%
\pgfpathcurveto{\pgfqpoint{5.588087in}{6.384429in}}{\pgfqpoint{5.583697in}{6.395028in}}{\pgfqpoint{5.575883in}{6.402841in}}%
\pgfpathcurveto{\pgfqpoint{5.568069in}{6.410655in}}{\pgfqpoint{5.557470in}{6.415045in}}{\pgfqpoint{5.546420in}{6.415045in}}%
\pgfpathcurveto{\pgfqpoint{5.535370in}{6.415045in}}{\pgfqpoint{5.524771in}{6.410655in}}{\pgfqpoint{5.516957in}{6.402841in}}%
\pgfpathcurveto{\pgfqpoint{5.509144in}{6.395028in}}{\pgfqpoint{5.504753in}{6.384429in}}{\pgfqpoint{5.504753in}{6.373379in}}%
\pgfpathcurveto{\pgfqpoint{5.504753in}{6.362328in}}{\pgfqpoint{5.509144in}{6.351729in}}{\pgfqpoint{5.516957in}{6.343916in}}%
\pgfpathcurveto{\pgfqpoint{5.524771in}{6.336102in}}{\pgfqpoint{5.535370in}{6.331712in}}{\pgfqpoint{5.546420in}{6.331712in}}%
\pgfpathlineto{\pgfqpoint{5.546420in}{6.331712in}}%
\pgfpathclose%
\pgfusepath{stroke,fill}%
\end{pgfscope}%
\begin{pgfscope}%
\pgfpathrectangle{\pgfqpoint{5.292946in}{5.272501in}}{\pgfqpoint{2.177280in}{2.201755in}}%
\pgfusepath{clip}%
\pgfsetbuttcap%
\pgfsetroundjoin%
\definecolor{currentfill}{rgb}{0.121569,0.466667,0.705882}%
\pgfsetfillcolor{currentfill}%
\pgfsetlinewidth{0.481800pt}%
\definecolor{currentstroke}{rgb}{1.000000,1.000000,1.000000}%
\pgfsetstrokecolor{currentstroke}%
\pgfsetdash{}{0pt}%
\pgfpathmoveto{\pgfqpoint{5.603831in}{6.248312in}}%
\pgfpathcurveto{\pgfqpoint{5.614881in}{6.248312in}}{\pgfqpoint{5.625480in}{6.252702in}}{\pgfqpoint{5.633294in}{6.260516in}}%
\pgfpathcurveto{\pgfqpoint{5.641107in}{6.268330in}}{\pgfqpoint{5.645497in}{6.278929in}}{\pgfqpoint{5.645497in}{6.289979in}}%
\pgfpathcurveto{\pgfqpoint{5.645497in}{6.301029in}}{\pgfqpoint{5.641107in}{6.311628in}}{\pgfqpoint{5.633294in}{6.319442in}}%
\pgfpathcurveto{\pgfqpoint{5.625480in}{6.327255in}}{\pgfqpoint{5.614881in}{6.331645in}}{\pgfqpoint{5.603831in}{6.331645in}}%
\pgfpathcurveto{\pgfqpoint{5.592781in}{6.331645in}}{\pgfqpoint{5.582182in}{6.327255in}}{\pgfqpoint{5.574368in}{6.319442in}}%
\pgfpathcurveto{\pgfqpoint{5.566554in}{6.311628in}}{\pgfqpoint{5.562164in}{6.301029in}}{\pgfqpoint{5.562164in}{6.289979in}}%
\pgfpathcurveto{\pgfqpoint{5.562164in}{6.278929in}}{\pgfqpoint{5.566554in}{6.268330in}}{\pgfqpoint{5.574368in}{6.260516in}}%
\pgfpathcurveto{\pgfqpoint{5.582182in}{6.252702in}}{\pgfqpoint{5.592781in}{6.248312in}}{\pgfqpoint{5.603831in}{6.248312in}}%
\pgfpathlineto{\pgfqpoint{5.603831in}{6.248312in}}%
\pgfpathclose%
\pgfusepath{stroke,fill}%
\end{pgfscope}%
\begin{pgfscope}%
\pgfpathrectangle{\pgfqpoint{5.292946in}{5.272501in}}{\pgfqpoint{2.177280in}{2.201755in}}%
\pgfusepath{clip}%
\pgfsetbuttcap%
\pgfsetroundjoin%
\definecolor{currentfill}{rgb}{0.121569,0.466667,0.705882}%
\pgfsetfillcolor{currentfill}%
\pgfsetlinewidth{0.481800pt}%
\definecolor{currentstroke}{rgb}{1.000000,1.000000,1.000000}%
\pgfsetstrokecolor{currentstroke}%
\pgfsetdash{}{0pt}%
\pgfpathmoveto{\pgfqpoint{5.575125in}{6.665311in}}%
\pgfpathcurveto{\pgfqpoint{5.586176in}{6.665311in}}{\pgfqpoint{5.596775in}{6.669701in}}{\pgfqpoint{5.604588in}{6.677515in}}%
\pgfpathcurveto{\pgfqpoint{5.612402in}{6.685329in}}{\pgfqpoint{5.616792in}{6.695928in}}{\pgfqpoint{5.616792in}{6.706978in}}%
\pgfpathcurveto{\pgfqpoint{5.616792in}{6.718028in}}{\pgfqpoint{5.612402in}{6.728627in}}{\pgfqpoint{5.604588in}{6.736441in}}%
\pgfpathcurveto{\pgfqpoint{5.596775in}{6.744254in}}{\pgfqpoint{5.586176in}{6.748644in}}{\pgfqpoint{5.575125in}{6.748644in}}%
\pgfpathcurveto{\pgfqpoint{5.564075in}{6.748644in}}{\pgfqpoint{5.553476in}{6.744254in}}{\pgfqpoint{5.545663in}{6.736441in}}%
\pgfpathcurveto{\pgfqpoint{5.537849in}{6.728627in}}{\pgfqpoint{5.533459in}{6.718028in}}{\pgfqpoint{5.533459in}{6.706978in}}%
\pgfpathcurveto{\pgfqpoint{5.533459in}{6.695928in}}{\pgfqpoint{5.537849in}{6.685329in}}{\pgfqpoint{5.545663in}{6.677515in}}%
\pgfpathcurveto{\pgfqpoint{5.553476in}{6.669701in}}{\pgfqpoint{5.564075in}{6.665311in}}{\pgfqpoint{5.575125in}{6.665311in}}%
\pgfpathlineto{\pgfqpoint{5.575125in}{6.665311in}}%
\pgfpathclose%
\pgfusepath{stroke,fill}%
\end{pgfscope}%
\begin{pgfscope}%
\pgfpathrectangle{\pgfqpoint{5.292946in}{5.272501in}}{\pgfqpoint{2.177280in}{2.201755in}}%
\pgfusepath{clip}%
\pgfsetbuttcap%
\pgfsetroundjoin%
\definecolor{currentfill}{rgb}{0.121569,0.466667,0.705882}%
\pgfsetfillcolor{currentfill}%
\pgfsetlinewidth{0.481800pt}%
\definecolor{currentstroke}{rgb}{1.000000,1.000000,1.000000}%
\pgfsetstrokecolor{currentstroke}%
\pgfsetdash{}{0pt}%
\pgfpathmoveto{\pgfqpoint{5.661241in}{6.915511in}}%
\pgfpathcurveto{\pgfqpoint{5.672291in}{6.915511in}}{\pgfqpoint{5.682890in}{6.919901in}}{\pgfqpoint{5.690704in}{6.927714in}}%
\pgfpathcurveto{\pgfqpoint{5.698518in}{6.935528in}}{\pgfqpoint{5.702908in}{6.946127in}}{\pgfqpoint{5.702908in}{6.957177in}}%
\pgfpathcurveto{\pgfqpoint{5.702908in}{6.968227in}}{\pgfqpoint{5.698518in}{6.978826in}}{\pgfqpoint{5.690704in}{6.986640in}}%
\pgfpathcurveto{\pgfqpoint{5.682890in}{6.994454in}}{\pgfqpoint{5.672291in}{6.998844in}}{\pgfqpoint{5.661241in}{6.998844in}}%
\pgfpathcurveto{\pgfqpoint{5.650191in}{6.998844in}}{\pgfqpoint{5.639592in}{6.994454in}}{\pgfqpoint{5.631779in}{6.986640in}}%
\pgfpathcurveto{\pgfqpoint{5.623965in}{6.978826in}}{\pgfqpoint{5.619575in}{6.968227in}}{\pgfqpoint{5.619575in}{6.957177in}}%
\pgfpathcurveto{\pgfqpoint{5.619575in}{6.946127in}}{\pgfqpoint{5.623965in}{6.935528in}}{\pgfqpoint{5.631779in}{6.927714in}}%
\pgfpathcurveto{\pgfqpoint{5.639592in}{6.919901in}}{\pgfqpoint{5.650191in}{6.915511in}}{\pgfqpoint{5.661241in}{6.915511in}}%
\pgfpathlineto{\pgfqpoint{5.661241in}{6.915511in}}%
\pgfpathclose%
\pgfusepath{stroke,fill}%
\end{pgfscope}%
\begin{pgfscope}%
\pgfpathrectangle{\pgfqpoint{5.292946in}{5.272501in}}{\pgfqpoint{2.177280in}{2.201755in}}%
\pgfusepath{clip}%
\pgfsetbuttcap%
\pgfsetroundjoin%
\definecolor{currentfill}{rgb}{0.121569,0.466667,0.705882}%
\pgfsetfillcolor{currentfill}%
\pgfsetlinewidth{0.481800pt}%
\definecolor{currentstroke}{rgb}{1.000000,1.000000,1.000000}%
\pgfsetstrokecolor{currentstroke}%
\pgfsetdash{}{0pt}%
\pgfpathmoveto{\pgfqpoint{5.575125in}{6.498512in}}%
\pgfpathcurveto{\pgfqpoint{5.586176in}{6.498512in}}{\pgfqpoint{5.596775in}{6.502902in}}{\pgfqpoint{5.604588in}{6.510715in}}%
\pgfpathcurveto{\pgfqpoint{5.612402in}{6.518529in}}{\pgfqpoint{5.616792in}{6.529128in}}{\pgfqpoint{5.616792in}{6.540178in}}%
\pgfpathcurveto{\pgfqpoint{5.616792in}{6.551228in}}{\pgfqpoint{5.612402in}{6.561827in}}{\pgfqpoint{5.604588in}{6.569641in}}%
\pgfpathcurveto{\pgfqpoint{5.596775in}{6.577455in}}{\pgfqpoint{5.586176in}{6.581845in}}{\pgfqpoint{5.575125in}{6.581845in}}%
\pgfpathcurveto{\pgfqpoint{5.564075in}{6.581845in}}{\pgfqpoint{5.553476in}{6.577455in}}{\pgfqpoint{5.545663in}{6.569641in}}%
\pgfpathcurveto{\pgfqpoint{5.537849in}{6.561827in}}{\pgfqpoint{5.533459in}{6.551228in}}{\pgfqpoint{5.533459in}{6.540178in}}%
\pgfpathcurveto{\pgfqpoint{5.533459in}{6.529128in}}{\pgfqpoint{5.537849in}{6.518529in}}{\pgfqpoint{5.545663in}{6.510715in}}%
\pgfpathcurveto{\pgfqpoint{5.553476in}{6.502902in}}{\pgfqpoint{5.564075in}{6.498512in}}{\pgfqpoint{5.575125in}{6.498512in}}%
\pgfpathlineto{\pgfqpoint{5.575125in}{6.498512in}}%
\pgfpathclose%
\pgfusepath{stroke,fill}%
\end{pgfscope}%
\begin{pgfscope}%
\pgfpathrectangle{\pgfqpoint{5.292946in}{5.272501in}}{\pgfqpoint{2.177280in}{2.201755in}}%
\pgfusepath{clip}%
\pgfsetbuttcap%
\pgfsetroundjoin%
\definecolor{currentfill}{rgb}{0.121569,0.466667,0.705882}%
\pgfsetfillcolor{currentfill}%
\pgfsetlinewidth{0.481800pt}%
\definecolor{currentstroke}{rgb}{1.000000,1.000000,1.000000}%
\pgfsetstrokecolor{currentstroke}%
\pgfsetdash{}{0pt}%
\pgfpathmoveto{\pgfqpoint{5.603831in}{6.498512in}}%
\pgfpathcurveto{\pgfqpoint{5.614881in}{6.498512in}}{\pgfqpoint{5.625480in}{6.502902in}}{\pgfqpoint{5.633294in}{6.510715in}}%
\pgfpathcurveto{\pgfqpoint{5.641107in}{6.518529in}}{\pgfqpoint{5.645497in}{6.529128in}}{\pgfqpoint{5.645497in}{6.540178in}}%
\pgfpathcurveto{\pgfqpoint{5.645497in}{6.551228in}}{\pgfqpoint{5.641107in}{6.561827in}}{\pgfqpoint{5.633294in}{6.569641in}}%
\pgfpathcurveto{\pgfqpoint{5.625480in}{6.577455in}}{\pgfqpoint{5.614881in}{6.581845in}}{\pgfqpoint{5.603831in}{6.581845in}}%
\pgfpathcurveto{\pgfqpoint{5.592781in}{6.581845in}}{\pgfqpoint{5.582182in}{6.577455in}}{\pgfqpoint{5.574368in}{6.569641in}}%
\pgfpathcurveto{\pgfqpoint{5.566554in}{6.561827in}}{\pgfqpoint{5.562164in}{6.551228in}}{\pgfqpoint{5.562164in}{6.540178in}}%
\pgfpathcurveto{\pgfqpoint{5.562164in}{6.529128in}}{\pgfqpoint{5.566554in}{6.518529in}}{\pgfqpoint{5.574368in}{6.510715in}}%
\pgfpathcurveto{\pgfqpoint{5.582182in}{6.502902in}}{\pgfqpoint{5.592781in}{6.498512in}}{\pgfqpoint{5.603831in}{6.498512in}}%
\pgfpathlineto{\pgfqpoint{5.603831in}{6.498512in}}%
\pgfpathclose%
\pgfusepath{stroke,fill}%
\end{pgfscope}%
\begin{pgfscope}%
\pgfpathrectangle{\pgfqpoint{5.292946in}{5.272501in}}{\pgfqpoint{2.177280in}{2.201755in}}%
\pgfusepath{clip}%
\pgfsetbuttcap%
\pgfsetroundjoin%
\definecolor{currentfill}{rgb}{0.121569,0.466667,0.705882}%
\pgfsetfillcolor{currentfill}%
\pgfsetlinewidth{0.481800pt}%
\definecolor{currentstroke}{rgb}{1.000000,1.000000,1.000000}%
\pgfsetstrokecolor{currentstroke}%
\pgfsetdash{}{0pt}%
\pgfpathmoveto{\pgfqpoint{5.575125in}{6.081512in}}%
\pgfpathcurveto{\pgfqpoint{5.586176in}{6.081512in}}{\pgfqpoint{5.596775in}{6.085903in}}{\pgfqpoint{5.604588in}{6.093716in}}%
\pgfpathcurveto{\pgfqpoint{5.612402in}{6.101530in}}{\pgfqpoint{5.616792in}{6.112129in}}{\pgfqpoint{5.616792in}{6.123179in}}%
\pgfpathcurveto{\pgfqpoint{5.616792in}{6.134229in}}{\pgfqpoint{5.612402in}{6.144828in}}{\pgfqpoint{5.604588in}{6.152642in}}%
\pgfpathcurveto{\pgfqpoint{5.596775in}{6.160456in}}{\pgfqpoint{5.586176in}{6.164846in}}{\pgfqpoint{5.575125in}{6.164846in}}%
\pgfpathcurveto{\pgfqpoint{5.564075in}{6.164846in}}{\pgfqpoint{5.553476in}{6.160456in}}{\pgfqpoint{5.545663in}{6.152642in}}%
\pgfpathcurveto{\pgfqpoint{5.537849in}{6.144828in}}{\pgfqpoint{5.533459in}{6.134229in}}{\pgfqpoint{5.533459in}{6.123179in}}%
\pgfpathcurveto{\pgfqpoint{5.533459in}{6.112129in}}{\pgfqpoint{5.537849in}{6.101530in}}{\pgfqpoint{5.545663in}{6.093716in}}%
\pgfpathcurveto{\pgfqpoint{5.553476in}{6.085903in}}{\pgfqpoint{5.564075in}{6.081512in}}{\pgfqpoint{5.575125in}{6.081512in}}%
\pgfpathlineto{\pgfqpoint{5.575125in}{6.081512in}}%
\pgfpathclose%
\pgfusepath{stroke,fill}%
\end{pgfscope}%
\begin{pgfscope}%
\pgfpathrectangle{\pgfqpoint{5.292946in}{5.272501in}}{\pgfqpoint{2.177280in}{2.201755in}}%
\pgfusepath{clip}%
\pgfsetbuttcap%
\pgfsetroundjoin%
\definecolor{currentfill}{rgb}{0.121569,0.466667,0.705882}%
\pgfsetfillcolor{currentfill}%
\pgfsetlinewidth{0.481800pt}%
\definecolor{currentstroke}{rgb}{1.000000,1.000000,1.000000}%
\pgfsetstrokecolor{currentstroke}%
\pgfsetdash{}{0pt}%
\pgfpathmoveto{\pgfqpoint{5.603831in}{6.248312in}}%
\pgfpathcurveto{\pgfqpoint{5.614881in}{6.248312in}}{\pgfqpoint{5.625480in}{6.252702in}}{\pgfqpoint{5.633294in}{6.260516in}}%
\pgfpathcurveto{\pgfqpoint{5.641107in}{6.268330in}}{\pgfqpoint{5.645497in}{6.278929in}}{\pgfqpoint{5.645497in}{6.289979in}}%
\pgfpathcurveto{\pgfqpoint{5.645497in}{6.301029in}}{\pgfqpoint{5.641107in}{6.311628in}}{\pgfqpoint{5.633294in}{6.319442in}}%
\pgfpathcurveto{\pgfqpoint{5.625480in}{6.327255in}}{\pgfqpoint{5.614881in}{6.331645in}}{\pgfqpoint{5.603831in}{6.331645in}}%
\pgfpathcurveto{\pgfqpoint{5.592781in}{6.331645in}}{\pgfqpoint{5.582182in}{6.327255in}}{\pgfqpoint{5.574368in}{6.319442in}}%
\pgfpathcurveto{\pgfqpoint{5.566554in}{6.311628in}}{\pgfqpoint{5.562164in}{6.301029in}}{\pgfqpoint{5.562164in}{6.289979in}}%
\pgfpathcurveto{\pgfqpoint{5.562164in}{6.278929in}}{\pgfqpoint{5.566554in}{6.268330in}}{\pgfqpoint{5.574368in}{6.260516in}}%
\pgfpathcurveto{\pgfqpoint{5.582182in}{6.252702in}}{\pgfqpoint{5.592781in}{6.248312in}}{\pgfqpoint{5.603831in}{6.248312in}}%
\pgfpathlineto{\pgfqpoint{5.603831in}{6.248312in}}%
\pgfpathclose%
\pgfusepath{stroke,fill}%
\end{pgfscope}%
\begin{pgfscope}%
\pgfpathrectangle{\pgfqpoint{5.292946in}{5.272501in}}{\pgfqpoint{2.177280in}{2.201755in}}%
\pgfusepath{clip}%
\pgfsetbuttcap%
\pgfsetroundjoin%
\definecolor{currentfill}{rgb}{0.121569,0.466667,0.705882}%
\pgfsetfillcolor{currentfill}%
\pgfsetlinewidth{0.481800pt}%
\definecolor{currentstroke}{rgb}{1.000000,1.000000,1.000000}%
\pgfsetstrokecolor{currentstroke}%
\pgfsetdash{}{0pt}%
\pgfpathmoveto{\pgfqpoint{5.603831in}{6.748711in}}%
\pgfpathcurveto{\pgfqpoint{5.614881in}{6.748711in}}{\pgfqpoint{5.625480in}{6.753101in}}{\pgfqpoint{5.633294in}{6.760915in}}%
\pgfpathcurveto{\pgfqpoint{5.641107in}{6.768728in}}{\pgfqpoint{5.645497in}{6.779327in}}{\pgfqpoint{5.645497in}{6.790378in}}%
\pgfpathcurveto{\pgfqpoint{5.645497in}{6.801428in}}{\pgfqpoint{5.641107in}{6.812027in}}{\pgfqpoint{5.633294in}{6.819840in}}%
\pgfpathcurveto{\pgfqpoint{5.625480in}{6.827654in}}{\pgfqpoint{5.614881in}{6.832044in}}{\pgfqpoint{5.603831in}{6.832044in}}%
\pgfpathcurveto{\pgfqpoint{5.592781in}{6.832044in}}{\pgfqpoint{5.582182in}{6.827654in}}{\pgfqpoint{5.574368in}{6.819840in}}%
\pgfpathcurveto{\pgfqpoint{5.566554in}{6.812027in}}{\pgfqpoint{5.562164in}{6.801428in}}{\pgfqpoint{5.562164in}{6.790378in}}%
\pgfpathcurveto{\pgfqpoint{5.562164in}{6.779327in}}{\pgfqpoint{5.566554in}{6.768728in}}{\pgfqpoint{5.574368in}{6.760915in}}%
\pgfpathcurveto{\pgfqpoint{5.582182in}{6.753101in}}{\pgfqpoint{5.592781in}{6.748711in}}{\pgfqpoint{5.603831in}{6.748711in}}%
\pgfpathlineto{\pgfqpoint{5.603831in}{6.748711in}}%
\pgfpathclose%
\pgfusepath{stroke,fill}%
\end{pgfscope}%
\begin{pgfscope}%
\pgfpathrectangle{\pgfqpoint{5.292946in}{5.272501in}}{\pgfqpoint{2.177280in}{2.201755in}}%
\pgfusepath{clip}%
\pgfsetbuttcap%
\pgfsetroundjoin%
\definecolor{currentfill}{rgb}{0.121569,0.466667,0.705882}%
\pgfsetfillcolor{currentfill}%
\pgfsetlinewidth{0.481800pt}%
\definecolor{currentstroke}{rgb}{1.000000,1.000000,1.000000}%
\pgfsetstrokecolor{currentstroke}%
\pgfsetdash{}{0pt}%
\pgfpathmoveto{\pgfqpoint{5.632536in}{6.498512in}}%
\pgfpathcurveto{\pgfqpoint{5.643586in}{6.498512in}}{\pgfqpoint{5.654185in}{6.502902in}}{\pgfqpoint{5.661999in}{6.510715in}}%
\pgfpathcurveto{\pgfqpoint{5.669812in}{6.518529in}}{\pgfqpoint{5.674203in}{6.529128in}}{\pgfqpoint{5.674203in}{6.540178in}}%
\pgfpathcurveto{\pgfqpoint{5.674203in}{6.551228in}}{\pgfqpoint{5.669812in}{6.561827in}}{\pgfqpoint{5.661999in}{6.569641in}}%
\pgfpathcurveto{\pgfqpoint{5.654185in}{6.577455in}}{\pgfqpoint{5.643586in}{6.581845in}}{\pgfqpoint{5.632536in}{6.581845in}}%
\pgfpathcurveto{\pgfqpoint{5.621486in}{6.581845in}}{\pgfqpoint{5.610887in}{6.577455in}}{\pgfqpoint{5.603073in}{6.569641in}}%
\pgfpathcurveto{\pgfqpoint{5.595260in}{6.561827in}}{\pgfqpoint{5.590869in}{6.551228in}}{\pgfqpoint{5.590869in}{6.540178in}}%
\pgfpathcurveto{\pgfqpoint{5.590869in}{6.529128in}}{\pgfqpoint{5.595260in}{6.518529in}}{\pgfqpoint{5.603073in}{6.510715in}}%
\pgfpathcurveto{\pgfqpoint{5.610887in}{6.502902in}}{\pgfqpoint{5.621486in}{6.498512in}}{\pgfqpoint{5.632536in}{6.498512in}}%
\pgfpathlineto{\pgfqpoint{5.632536in}{6.498512in}}%
\pgfpathclose%
\pgfusepath{stroke,fill}%
\end{pgfscope}%
\begin{pgfscope}%
\pgfpathrectangle{\pgfqpoint{5.292946in}{5.272501in}}{\pgfqpoint{2.177280in}{2.201755in}}%
\pgfusepath{clip}%
\pgfsetbuttcap%
\pgfsetroundjoin%
\definecolor{currentfill}{rgb}{0.121569,0.466667,0.705882}%
\pgfsetfillcolor{currentfill}%
\pgfsetlinewidth{0.481800pt}%
\definecolor{currentstroke}{rgb}{1.000000,1.000000,1.000000}%
\pgfsetstrokecolor{currentstroke}%
\pgfsetdash{}{0pt}%
\pgfpathmoveto{\pgfqpoint{5.575125in}{6.164912in}}%
\pgfpathcurveto{\pgfqpoint{5.586176in}{6.164912in}}{\pgfqpoint{5.596775in}{6.169303in}}{\pgfqpoint{5.604588in}{6.177116in}}%
\pgfpathcurveto{\pgfqpoint{5.612402in}{6.184930in}}{\pgfqpoint{5.616792in}{6.195529in}}{\pgfqpoint{5.616792in}{6.206579in}}%
\pgfpathcurveto{\pgfqpoint{5.616792in}{6.217629in}}{\pgfqpoint{5.612402in}{6.228228in}}{\pgfqpoint{5.604588in}{6.236042in}}%
\pgfpathcurveto{\pgfqpoint{5.596775in}{6.243855in}}{\pgfqpoint{5.586176in}{6.248246in}}{\pgfqpoint{5.575125in}{6.248246in}}%
\pgfpathcurveto{\pgfqpoint{5.564075in}{6.248246in}}{\pgfqpoint{5.553476in}{6.243855in}}{\pgfqpoint{5.545663in}{6.236042in}}%
\pgfpathcurveto{\pgfqpoint{5.537849in}{6.228228in}}{\pgfqpoint{5.533459in}{6.217629in}}{\pgfqpoint{5.533459in}{6.206579in}}%
\pgfpathcurveto{\pgfqpoint{5.533459in}{6.195529in}}{\pgfqpoint{5.537849in}{6.184930in}}{\pgfqpoint{5.545663in}{6.177116in}}%
\pgfpathcurveto{\pgfqpoint{5.553476in}{6.169303in}}{\pgfqpoint{5.564075in}{6.164912in}}{\pgfqpoint{5.575125in}{6.164912in}}%
\pgfpathlineto{\pgfqpoint{5.575125in}{6.164912in}}%
\pgfpathclose%
\pgfusepath{stroke,fill}%
\end{pgfscope}%
\begin{pgfscope}%
\pgfpathrectangle{\pgfqpoint{5.292946in}{5.272501in}}{\pgfqpoint{2.177280in}{2.201755in}}%
\pgfusepath{clip}%
\pgfsetbuttcap%
\pgfsetroundjoin%
\definecolor{currentfill}{rgb}{0.121569,0.466667,0.705882}%
\pgfsetfillcolor{currentfill}%
\pgfsetlinewidth{0.481800pt}%
\definecolor{currentstroke}{rgb}{1.000000,1.000000,1.000000}%
\pgfsetstrokecolor{currentstroke}%
\pgfsetdash{}{0pt}%
\pgfpathmoveto{\pgfqpoint{5.489010in}{6.164912in}}%
\pgfpathcurveto{\pgfqpoint{5.500060in}{6.164912in}}{\pgfqpoint{5.510659in}{6.169303in}}{\pgfqpoint{5.518472in}{6.177116in}}%
\pgfpathcurveto{\pgfqpoint{5.526286in}{6.184930in}}{\pgfqpoint{5.530676in}{6.195529in}}{\pgfqpoint{5.530676in}{6.206579in}}%
\pgfpathcurveto{\pgfqpoint{5.530676in}{6.217629in}}{\pgfqpoint{5.526286in}{6.228228in}}{\pgfqpoint{5.518472in}{6.236042in}}%
\pgfpathcurveto{\pgfqpoint{5.510659in}{6.243855in}}{\pgfqpoint{5.500060in}{6.248246in}}{\pgfqpoint{5.489010in}{6.248246in}}%
\pgfpathcurveto{\pgfqpoint{5.477959in}{6.248246in}}{\pgfqpoint{5.467360in}{6.243855in}}{\pgfqpoint{5.459547in}{6.236042in}}%
\pgfpathcurveto{\pgfqpoint{5.451733in}{6.228228in}}{\pgfqpoint{5.447343in}{6.217629in}}{\pgfqpoint{5.447343in}{6.206579in}}%
\pgfpathcurveto{\pgfqpoint{5.447343in}{6.195529in}}{\pgfqpoint{5.451733in}{6.184930in}}{\pgfqpoint{5.459547in}{6.177116in}}%
\pgfpathcurveto{\pgfqpoint{5.467360in}{6.169303in}}{\pgfqpoint{5.477959in}{6.164912in}}{\pgfqpoint{5.489010in}{6.164912in}}%
\pgfpathlineto{\pgfqpoint{5.489010in}{6.164912in}}%
\pgfpathclose%
\pgfusepath{stroke,fill}%
\end{pgfscope}%
\begin{pgfscope}%
\pgfpathrectangle{\pgfqpoint{5.292946in}{5.272501in}}{\pgfqpoint{2.177280in}{2.201755in}}%
\pgfusepath{clip}%
\pgfsetbuttcap%
\pgfsetroundjoin%
\definecolor{currentfill}{rgb}{0.121569,0.466667,0.705882}%
\pgfsetfillcolor{currentfill}%
\pgfsetlinewidth{0.481800pt}%
\definecolor{currentstroke}{rgb}{1.000000,1.000000,1.000000}%
\pgfsetstrokecolor{currentstroke}%
\pgfsetdash{}{0pt}%
\pgfpathmoveto{\pgfqpoint{5.517715in}{6.998910in}}%
\pgfpathcurveto{\pgfqpoint{5.528765in}{6.998910in}}{\pgfqpoint{5.539364in}{7.003301in}}{\pgfqpoint{5.547178in}{7.011114in}}%
\pgfpathcurveto{\pgfqpoint{5.554991in}{7.018928in}}{\pgfqpoint{5.559382in}{7.029527in}}{\pgfqpoint{5.559382in}{7.040577in}}%
\pgfpathcurveto{\pgfqpoint{5.559382in}{7.051627in}}{\pgfqpoint{5.554991in}{7.062226in}}{\pgfqpoint{5.547178in}{7.070040in}}%
\pgfpathcurveto{\pgfqpoint{5.539364in}{7.077853in}}{\pgfqpoint{5.528765in}{7.082244in}}{\pgfqpoint{5.517715in}{7.082244in}}%
\pgfpathcurveto{\pgfqpoint{5.506665in}{7.082244in}}{\pgfqpoint{5.496066in}{7.077853in}}{\pgfqpoint{5.488252in}{7.070040in}}%
\pgfpathcurveto{\pgfqpoint{5.480438in}{7.062226in}}{\pgfqpoint{5.476048in}{7.051627in}}{\pgfqpoint{5.476048in}{7.040577in}}%
\pgfpathcurveto{\pgfqpoint{5.476048in}{7.029527in}}{\pgfqpoint{5.480438in}{7.018928in}}{\pgfqpoint{5.488252in}{7.011114in}}%
\pgfpathcurveto{\pgfqpoint{5.496066in}{7.003301in}}{\pgfqpoint{5.506665in}{6.998910in}}{\pgfqpoint{5.517715in}{6.998910in}}%
\pgfpathlineto{\pgfqpoint{5.517715in}{6.998910in}}%
\pgfpathclose%
\pgfusepath{stroke,fill}%
\end{pgfscope}%
\begin{pgfscope}%
\pgfpathrectangle{\pgfqpoint{5.292946in}{5.272501in}}{\pgfqpoint{2.177280in}{2.201755in}}%
\pgfusepath{clip}%
\pgfsetbuttcap%
\pgfsetroundjoin%
\definecolor{currentfill}{rgb}{0.121569,0.466667,0.705882}%
\pgfsetfillcolor{currentfill}%
\pgfsetlinewidth{0.481800pt}%
\definecolor{currentstroke}{rgb}{1.000000,1.000000,1.000000}%
\pgfsetstrokecolor{currentstroke}%
\pgfsetdash{}{0pt}%
\pgfpathmoveto{\pgfqpoint{5.603831in}{7.332510in}}%
\pgfpathcurveto{\pgfqpoint{5.614881in}{7.332510in}}{\pgfqpoint{5.625480in}{7.336900in}}{\pgfqpoint{5.633294in}{7.344714in}}%
\pgfpathcurveto{\pgfqpoint{5.641107in}{7.352527in}}{\pgfqpoint{5.645497in}{7.363126in}}{\pgfqpoint{5.645497in}{7.374176in}}%
\pgfpathcurveto{\pgfqpoint{5.645497in}{7.385226in}}{\pgfqpoint{5.641107in}{7.395825in}}{\pgfqpoint{5.633294in}{7.403639in}}%
\pgfpathcurveto{\pgfqpoint{5.625480in}{7.411453in}}{\pgfqpoint{5.614881in}{7.415843in}}{\pgfqpoint{5.603831in}{7.415843in}}%
\pgfpathcurveto{\pgfqpoint{5.592781in}{7.415843in}}{\pgfqpoint{5.582182in}{7.411453in}}{\pgfqpoint{5.574368in}{7.403639in}}%
\pgfpathcurveto{\pgfqpoint{5.566554in}{7.395825in}}{\pgfqpoint{5.562164in}{7.385226in}}{\pgfqpoint{5.562164in}{7.374176in}}%
\pgfpathcurveto{\pgfqpoint{5.562164in}{7.363126in}}{\pgfqpoint{5.566554in}{7.352527in}}{\pgfqpoint{5.574368in}{7.344714in}}%
\pgfpathcurveto{\pgfqpoint{5.582182in}{7.336900in}}{\pgfqpoint{5.592781in}{7.332510in}}{\pgfqpoint{5.603831in}{7.332510in}}%
\pgfpathlineto{\pgfqpoint{5.603831in}{7.332510in}}%
\pgfpathclose%
\pgfusepath{stroke,fill}%
\end{pgfscope}%
\begin{pgfscope}%
\pgfpathrectangle{\pgfqpoint{5.292946in}{5.272501in}}{\pgfqpoint{2.177280in}{2.201755in}}%
\pgfusepath{clip}%
\pgfsetbuttcap%
\pgfsetroundjoin%
\definecolor{currentfill}{rgb}{0.121569,0.466667,0.705882}%
\pgfsetfillcolor{currentfill}%
\pgfsetlinewidth{0.481800pt}%
\definecolor{currentstroke}{rgb}{1.000000,1.000000,1.000000}%
\pgfsetstrokecolor{currentstroke}%
\pgfsetdash{}{0pt}%
\pgfpathmoveto{\pgfqpoint{5.546420in}{6.915511in}}%
\pgfpathcurveto{\pgfqpoint{5.557470in}{6.915511in}}{\pgfqpoint{5.568069in}{6.919901in}}{\pgfqpoint{5.575883in}{6.927714in}}%
\pgfpathcurveto{\pgfqpoint{5.583697in}{6.935528in}}{\pgfqpoint{5.588087in}{6.946127in}}{\pgfqpoint{5.588087in}{6.957177in}}%
\pgfpathcurveto{\pgfqpoint{5.588087in}{6.968227in}}{\pgfqpoint{5.583697in}{6.978826in}}{\pgfqpoint{5.575883in}{6.986640in}}%
\pgfpathcurveto{\pgfqpoint{5.568069in}{6.994454in}}{\pgfqpoint{5.557470in}{6.998844in}}{\pgfqpoint{5.546420in}{6.998844in}}%
\pgfpathcurveto{\pgfqpoint{5.535370in}{6.998844in}}{\pgfqpoint{5.524771in}{6.994454in}}{\pgfqpoint{5.516957in}{6.986640in}}%
\pgfpathcurveto{\pgfqpoint{5.509144in}{6.978826in}}{\pgfqpoint{5.504753in}{6.968227in}}{\pgfqpoint{5.504753in}{6.957177in}}%
\pgfpathcurveto{\pgfqpoint{5.504753in}{6.946127in}}{\pgfqpoint{5.509144in}{6.935528in}}{\pgfqpoint{5.516957in}{6.927714in}}%
\pgfpathcurveto{\pgfqpoint{5.524771in}{6.919901in}}{\pgfqpoint{5.535370in}{6.915511in}}{\pgfqpoint{5.546420in}{6.915511in}}%
\pgfpathlineto{\pgfqpoint{5.546420in}{6.915511in}}%
\pgfpathclose%
\pgfusepath{stroke,fill}%
\end{pgfscope}%
\begin{pgfscope}%
\pgfpathrectangle{\pgfqpoint{5.292946in}{5.272501in}}{\pgfqpoint{2.177280in}{2.201755in}}%
\pgfusepath{clip}%
\pgfsetbuttcap%
\pgfsetroundjoin%
\definecolor{currentfill}{rgb}{0.121569,0.466667,0.705882}%
\pgfsetfillcolor{currentfill}%
\pgfsetlinewidth{0.481800pt}%
\definecolor{currentstroke}{rgb}{1.000000,1.000000,1.000000}%
\pgfsetstrokecolor{currentstroke}%
\pgfsetdash{}{0pt}%
\pgfpathmoveto{\pgfqpoint{5.575125in}{6.581911in}}%
\pgfpathcurveto{\pgfqpoint{5.586176in}{6.581911in}}{\pgfqpoint{5.596775in}{6.586302in}}{\pgfqpoint{5.604588in}{6.594115in}}%
\pgfpathcurveto{\pgfqpoint{5.612402in}{6.601929in}}{\pgfqpoint{5.616792in}{6.612528in}}{\pgfqpoint{5.616792in}{6.623578in}}%
\pgfpathcurveto{\pgfqpoint{5.616792in}{6.634628in}}{\pgfqpoint{5.612402in}{6.645227in}}{\pgfqpoint{5.604588in}{6.653041in}}%
\pgfpathcurveto{\pgfqpoint{5.596775in}{6.660854in}}{\pgfqpoint{5.586176in}{6.665245in}}{\pgfqpoint{5.575125in}{6.665245in}}%
\pgfpathcurveto{\pgfqpoint{5.564075in}{6.665245in}}{\pgfqpoint{5.553476in}{6.660854in}}{\pgfqpoint{5.545663in}{6.653041in}}%
\pgfpathcurveto{\pgfqpoint{5.537849in}{6.645227in}}{\pgfqpoint{5.533459in}{6.634628in}}{\pgfqpoint{5.533459in}{6.623578in}}%
\pgfpathcurveto{\pgfqpoint{5.533459in}{6.612528in}}{\pgfqpoint{5.537849in}{6.601929in}}{\pgfqpoint{5.545663in}{6.594115in}}%
\pgfpathcurveto{\pgfqpoint{5.553476in}{6.586302in}}{\pgfqpoint{5.564075in}{6.581911in}}{\pgfqpoint{5.575125in}{6.581911in}}%
\pgfpathlineto{\pgfqpoint{5.575125in}{6.581911in}}%
\pgfpathclose%
\pgfusepath{stroke,fill}%
\end{pgfscope}%
\begin{pgfscope}%
\pgfpathrectangle{\pgfqpoint{5.292946in}{5.272501in}}{\pgfqpoint{2.177280in}{2.201755in}}%
\pgfusepath{clip}%
\pgfsetbuttcap%
\pgfsetroundjoin%
\definecolor{currentfill}{rgb}{0.121569,0.466667,0.705882}%
\pgfsetfillcolor{currentfill}%
\pgfsetlinewidth{0.481800pt}%
\definecolor{currentstroke}{rgb}{1.000000,1.000000,1.000000}%
\pgfsetstrokecolor{currentstroke}%
\pgfsetdash{}{0pt}%
\pgfpathmoveto{\pgfqpoint{5.661241in}{6.832111in}}%
\pgfpathcurveto{\pgfqpoint{5.672291in}{6.832111in}}{\pgfqpoint{5.682890in}{6.836501in}}{\pgfqpoint{5.690704in}{6.844315in}}%
\pgfpathcurveto{\pgfqpoint{5.698518in}{6.852128in}}{\pgfqpoint{5.702908in}{6.862727in}}{\pgfqpoint{5.702908in}{6.873777in}}%
\pgfpathcurveto{\pgfqpoint{5.702908in}{6.884828in}}{\pgfqpoint{5.698518in}{6.895427in}}{\pgfqpoint{5.690704in}{6.903240in}}%
\pgfpathcurveto{\pgfqpoint{5.682890in}{6.911054in}}{\pgfqpoint{5.672291in}{6.915444in}}{\pgfqpoint{5.661241in}{6.915444in}}%
\pgfpathcurveto{\pgfqpoint{5.650191in}{6.915444in}}{\pgfqpoint{5.639592in}{6.911054in}}{\pgfqpoint{5.631779in}{6.903240in}}%
\pgfpathcurveto{\pgfqpoint{5.623965in}{6.895427in}}{\pgfqpoint{5.619575in}{6.884828in}}{\pgfqpoint{5.619575in}{6.873777in}}%
\pgfpathcurveto{\pgfqpoint{5.619575in}{6.862727in}}{\pgfqpoint{5.623965in}{6.852128in}}{\pgfqpoint{5.631779in}{6.844315in}}%
\pgfpathcurveto{\pgfqpoint{5.639592in}{6.836501in}}{\pgfqpoint{5.650191in}{6.832111in}}{\pgfqpoint{5.661241in}{6.832111in}}%
\pgfpathlineto{\pgfqpoint{5.661241in}{6.832111in}}%
\pgfpathclose%
\pgfusepath{stroke,fill}%
\end{pgfscope}%
\begin{pgfscope}%
\pgfpathrectangle{\pgfqpoint{5.292946in}{5.272501in}}{\pgfqpoint{2.177280in}{2.201755in}}%
\pgfusepath{clip}%
\pgfsetbuttcap%
\pgfsetroundjoin%
\definecolor{currentfill}{rgb}{0.121569,0.466667,0.705882}%
\pgfsetfillcolor{currentfill}%
\pgfsetlinewidth{0.481800pt}%
\definecolor{currentstroke}{rgb}{1.000000,1.000000,1.000000}%
\pgfsetstrokecolor{currentstroke}%
\pgfsetdash{}{0pt}%
\pgfpathmoveto{\pgfqpoint{5.603831in}{6.832111in}}%
\pgfpathcurveto{\pgfqpoint{5.614881in}{6.832111in}}{\pgfqpoint{5.625480in}{6.836501in}}{\pgfqpoint{5.633294in}{6.844315in}}%
\pgfpathcurveto{\pgfqpoint{5.641107in}{6.852128in}}{\pgfqpoint{5.645497in}{6.862727in}}{\pgfqpoint{5.645497in}{6.873777in}}%
\pgfpathcurveto{\pgfqpoint{5.645497in}{6.884828in}}{\pgfqpoint{5.641107in}{6.895427in}}{\pgfqpoint{5.633294in}{6.903240in}}%
\pgfpathcurveto{\pgfqpoint{5.625480in}{6.911054in}}{\pgfqpoint{5.614881in}{6.915444in}}{\pgfqpoint{5.603831in}{6.915444in}}%
\pgfpathcurveto{\pgfqpoint{5.592781in}{6.915444in}}{\pgfqpoint{5.582182in}{6.911054in}}{\pgfqpoint{5.574368in}{6.903240in}}%
\pgfpathcurveto{\pgfqpoint{5.566554in}{6.895427in}}{\pgfqpoint{5.562164in}{6.884828in}}{\pgfqpoint{5.562164in}{6.873777in}}%
\pgfpathcurveto{\pgfqpoint{5.562164in}{6.862727in}}{\pgfqpoint{5.566554in}{6.852128in}}{\pgfqpoint{5.574368in}{6.844315in}}%
\pgfpathcurveto{\pgfqpoint{5.582182in}{6.836501in}}{\pgfqpoint{5.592781in}{6.832111in}}{\pgfqpoint{5.603831in}{6.832111in}}%
\pgfpathlineto{\pgfqpoint{5.603831in}{6.832111in}}%
\pgfpathclose%
\pgfusepath{stroke,fill}%
\end{pgfscope}%
\begin{pgfscope}%
\pgfpathrectangle{\pgfqpoint{5.292946in}{5.272501in}}{\pgfqpoint{2.177280in}{2.201755in}}%
\pgfusepath{clip}%
\pgfsetbuttcap%
\pgfsetroundjoin%
\definecolor{currentfill}{rgb}{0.121569,0.466667,0.705882}%
\pgfsetfillcolor{currentfill}%
\pgfsetlinewidth{0.481800pt}%
\definecolor{currentstroke}{rgb}{1.000000,1.000000,1.000000}%
\pgfsetstrokecolor{currentstroke}%
\pgfsetdash{}{0pt}%
\pgfpathmoveto{\pgfqpoint{5.661241in}{6.498512in}}%
\pgfpathcurveto{\pgfqpoint{5.672291in}{6.498512in}}{\pgfqpoint{5.682890in}{6.502902in}}{\pgfqpoint{5.690704in}{6.510715in}}%
\pgfpathcurveto{\pgfqpoint{5.698518in}{6.518529in}}{\pgfqpoint{5.702908in}{6.529128in}}{\pgfqpoint{5.702908in}{6.540178in}}%
\pgfpathcurveto{\pgfqpoint{5.702908in}{6.551228in}}{\pgfqpoint{5.698518in}{6.561827in}}{\pgfqpoint{5.690704in}{6.569641in}}%
\pgfpathcurveto{\pgfqpoint{5.682890in}{6.577455in}}{\pgfqpoint{5.672291in}{6.581845in}}{\pgfqpoint{5.661241in}{6.581845in}}%
\pgfpathcurveto{\pgfqpoint{5.650191in}{6.581845in}}{\pgfqpoint{5.639592in}{6.577455in}}{\pgfqpoint{5.631779in}{6.569641in}}%
\pgfpathcurveto{\pgfqpoint{5.623965in}{6.561827in}}{\pgfqpoint{5.619575in}{6.551228in}}{\pgfqpoint{5.619575in}{6.540178in}}%
\pgfpathcurveto{\pgfqpoint{5.619575in}{6.529128in}}{\pgfqpoint{5.623965in}{6.518529in}}{\pgfqpoint{5.631779in}{6.510715in}}%
\pgfpathcurveto{\pgfqpoint{5.639592in}{6.502902in}}{\pgfqpoint{5.650191in}{6.498512in}}{\pgfqpoint{5.661241in}{6.498512in}}%
\pgfpathlineto{\pgfqpoint{5.661241in}{6.498512in}}%
\pgfpathclose%
\pgfusepath{stroke,fill}%
\end{pgfscope}%
\begin{pgfscope}%
\pgfpathrectangle{\pgfqpoint{5.292946in}{5.272501in}}{\pgfqpoint{2.177280in}{2.201755in}}%
\pgfusepath{clip}%
\pgfsetbuttcap%
\pgfsetroundjoin%
\definecolor{currentfill}{rgb}{0.121569,0.466667,0.705882}%
\pgfsetfillcolor{currentfill}%
\pgfsetlinewidth{0.481800pt}%
\definecolor{currentstroke}{rgb}{1.000000,1.000000,1.000000}%
\pgfsetstrokecolor{currentstroke}%
\pgfsetdash{}{0pt}%
\pgfpathmoveto{\pgfqpoint{5.603831in}{6.748711in}}%
\pgfpathcurveto{\pgfqpoint{5.614881in}{6.748711in}}{\pgfqpoint{5.625480in}{6.753101in}}{\pgfqpoint{5.633294in}{6.760915in}}%
\pgfpathcurveto{\pgfqpoint{5.641107in}{6.768728in}}{\pgfqpoint{5.645497in}{6.779327in}}{\pgfqpoint{5.645497in}{6.790378in}}%
\pgfpathcurveto{\pgfqpoint{5.645497in}{6.801428in}}{\pgfqpoint{5.641107in}{6.812027in}}{\pgfqpoint{5.633294in}{6.819840in}}%
\pgfpathcurveto{\pgfqpoint{5.625480in}{6.827654in}}{\pgfqpoint{5.614881in}{6.832044in}}{\pgfqpoint{5.603831in}{6.832044in}}%
\pgfpathcurveto{\pgfqpoint{5.592781in}{6.832044in}}{\pgfqpoint{5.582182in}{6.827654in}}{\pgfqpoint{5.574368in}{6.819840in}}%
\pgfpathcurveto{\pgfqpoint{5.566554in}{6.812027in}}{\pgfqpoint{5.562164in}{6.801428in}}{\pgfqpoint{5.562164in}{6.790378in}}%
\pgfpathcurveto{\pgfqpoint{5.562164in}{6.779327in}}{\pgfqpoint{5.566554in}{6.768728in}}{\pgfqpoint{5.574368in}{6.760915in}}%
\pgfpathcurveto{\pgfqpoint{5.582182in}{6.753101in}}{\pgfqpoint{5.592781in}{6.748711in}}{\pgfqpoint{5.603831in}{6.748711in}}%
\pgfpathlineto{\pgfqpoint{5.603831in}{6.748711in}}%
\pgfpathclose%
\pgfusepath{stroke,fill}%
\end{pgfscope}%
\begin{pgfscope}%
\pgfpathrectangle{\pgfqpoint{5.292946in}{5.272501in}}{\pgfqpoint{2.177280in}{2.201755in}}%
\pgfusepath{clip}%
\pgfsetbuttcap%
\pgfsetroundjoin%
\definecolor{currentfill}{rgb}{0.121569,0.466667,0.705882}%
\pgfsetfillcolor{currentfill}%
\pgfsetlinewidth{0.481800pt}%
\definecolor{currentstroke}{rgb}{1.000000,1.000000,1.000000}%
\pgfsetstrokecolor{currentstroke}%
\pgfsetdash{}{0pt}%
\pgfpathmoveto{\pgfqpoint{5.460304in}{6.665311in}}%
\pgfpathcurveto{\pgfqpoint{5.471354in}{6.665311in}}{\pgfqpoint{5.481953in}{6.669701in}}{\pgfqpoint{5.489767in}{6.677515in}}%
\pgfpathcurveto{\pgfqpoint{5.497581in}{6.685329in}}{\pgfqpoint{5.501971in}{6.695928in}}{\pgfqpoint{5.501971in}{6.706978in}}%
\pgfpathcurveto{\pgfqpoint{5.501971in}{6.718028in}}{\pgfqpoint{5.497581in}{6.728627in}}{\pgfqpoint{5.489767in}{6.736441in}}%
\pgfpathcurveto{\pgfqpoint{5.481953in}{6.744254in}}{\pgfqpoint{5.471354in}{6.748644in}}{\pgfqpoint{5.460304in}{6.748644in}}%
\pgfpathcurveto{\pgfqpoint{5.449254in}{6.748644in}}{\pgfqpoint{5.438655in}{6.744254in}}{\pgfqpoint{5.430842in}{6.736441in}}%
\pgfpathcurveto{\pgfqpoint{5.423028in}{6.728627in}}{\pgfqpoint{5.418638in}{6.718028in}}{\pgfqpoint{5.418638in}{6.706978in}}%
\pgfpathcurveto{\pgfqpoint{5.418638in}{6.695928in}}{\pgfqpoint{5.423028in}{6.685329in}}{\pgfqpoint{5.430842in}{6.677515in}}%
\pgfpathcurveto{\pgfqpoint{5.438655in}{6.669701in}}{\pgfqpoint{5.449254in}{6.665311in}}{\pgfqpoint{5.460304in}{6.665311in}}%
\pgfpathlineto{\pgfqpoint{5.460304in}{6.665311in}}%
\pgfpathclose%
\pgfusepath{stroke,fill}%
\end{pgfscope}%
\begin{pgfscope}%
\pgfpathrectangle{\pgfqpoint{5.292946in}{5.272501in}}{\pgfqpoint{2.177280in}{2.201755in}}%
\pgfusepath{clip}%
\pgfsetbuttcap%
\pgfsetroundjoin%
\definecolor{currentfill}{rgb}{0.121569,0.466667,0.705882}%
\pgfsetfillcolor{currentfill}%
\pgfsetlinewidth{0.481800pt}%
\definecolor{currentstroke}{rgb}{1.000000,1.000000,1.000000}%
\pgfsetstrokecolor{currentstroke}%
\pgfsetdash{}{0pt}%
\pgfpathmoveto{\pgfqpoint{5.661241in}{6.415112in}}%
\pgfpathcurveto{\pgfqpoint{5.672291in}{6.415112in}}{\pgfqpoint{5.682890in}{6.419502in}}{\pgfqpoint{5.690704in}{6.427316in}}%
\pgfpathcurveto{\pgfqpoint{5.698518in}{6.435129in}}{\pgfqpoint{5.702908in}{6.445728in}}{\pgfqpoint{5.702908in}{6.456778in}}%
\pgfpathcurveto{\pgfqpoint{5.702908in}{6.467828in}}{\pgfqpoint{5.698518in}{6.478428in}}{\pgfqpoint{5.690704in}{6.486241in}}%
\pgfpathcurveto{\pgfqpoint{5.682890in}{6.494055in}}{\pgfqpoint{5.672291in}{6.498445in}}{\pgfqpoint{5.661241in}{6.498445in}}%
\pgfpathcurveto{\pgfqpoint{5.650191in}{6.498445in}}{\pgfqpoint{5.639592in}{6.494055in}}{\pgfqpoint{5.631779in}{6.486241in}}%
\pgfpathcurveto{\pgfqpoint{5.623965in}{6.478428in}}{\pgfqpoint{5.619575in}{6.467828in}}{\pgfqpoint{5.619575in}{6.456778in}}%
\pgfpathcurveto{\pgfqpoint{5.619575in}{6.445728in}}{\pgfqpoint{5.623965in}{6.435129in}}{\pgfqpoint{5.631779in}{6.427316in}}%
\pgfpathcurveto{\pgfqpoint{5.639592in}{6.419502in}}{\pgfqpoint{5.650191in}{6.415112in}}{\pgfqpoint{5.661241in}{6.415112in}}%
\pgfpathlineto{\pgfqpoint{5.661241in}{6.415112in}}%
\pgfpathclose%
\pgfusepath{stroke,fill}%
\end{pgfscope}%
\begin{pgfscope}%
\pgfpathrectangle{\pgfqpoint{5.292946in}{5.272501in}}{\pgfqpoint{2.177280in}{2.201755in}}%
\pgfusepath{clip}%
\pgfsetbuttcap%
\pgfsetroundjoin%
\definecolor{currentfill}{rgb}{0.121569,0.466667,0.705882}%
\pgfsetfillcolor{currentfill}%
\pgfsetlinewidth{0.481800pt}%
\definecolor{currentstroke}{rgb}{1.000000,1.000000,1.000000}%
\pgfsetstrokecolor{currentstroke}%
\pgfsetdash{}{0pt}%
\pgfpathmoveto{\pgfqpoint{5.718652in}{6.498512in}}%
\pgfpathcurveto{\pgfqpoint{5.729702in}{6.498512in}}{\pgfqpoint{5.740301in}{6.502902in}}{\pgfqpoint{5.748115in}{6.510715in}}%
\pgfpathcurveto{\pgfqpoint{5.755928in}{6.518529in}}{\pgfqpoint{5.760319in}{6.529128in}}{\pgfqpoint{5.760319in}{6.540178in}}%
\pgfpathcurveto{\pgfqpoint{5.760319in}{6.551228in}}{\pgfqpoint{5.755928in}{6.561827in}}{\pgfqpoint{5.748115in}{6.569641in}}%
\pgfpathcurveto{\pgfqpoint{5.740301in}{6.577455in}}{\pgfqpoint{5.729702in}{6.581845in}}{\pgfqpoint{5.718652in}{6.581845in}}%
\pgfpathcurveto{\pgfqpoint{5.707602in}{6.581845in}}{\pgfqpoint{5.697003in}{6.577455in}}{\pgfqpoint{5.689189in}{6.569641in}}%
\pgfpathcurveto{\pgfqpoint{5.681375in}{6.561827in}}{\pgfqpoint{5.676985in}{6.551228in}}{\pgfqpoint{5.676985in}{6.540178in}}%
\pgfpathcurveto{\pgfqpoint{5.676985in}{6.529128in}}{\pgfqpoint{5.681375in}{6.518529in}}{\pgfqpoint{5.689189in}{6.510715in}}%
\pgfpathcurveto{\pgfqpoint{5.697003in}{6.502902in}}{\pgfqpoint{5.707602in}{6.498512in}}{\pgfqpoint{5.718652in}{6.498512in}}%
\pgfpathlineto{\pgfqpoint{5.718652in}{6.498512in}}%
\pgfpathclose%
\pgfusepath{stroke,fill}%
\end{pgfscope}%
\begin{pgfscope}%
\pgfpathrectangle{\pgfqpoint{5.292946in}{5.272501in}}{\pgfqpoint{2.177280in}{2.201755in}}%
\pgfusepath{clip}%
\pgfsetbuttcap%
\pgfsetroundjoin%
\definecolor{currentfill}{rgb}{0.121569,0.466667,0.705882}%
\pgfsetfillcolor{currentfill}%
\pgfsetlinewidth{0.481800pt}%
\definecolor{currentstroke}{rgb}{1.000000,1.000000,1.000000}%
\pgfsetstrokecolor{currentstroke}%
\pgfsetdash{}{0pt}%
\pgfpathmoveto{\pgfqpoint{5.632536in}{6.164912in}}%
\pgfpathcurveto{\pgfqpoint{5.643586in}{6.164912in}}{\pgfqpoint{5.654185in}{6.169303in}}{\pgfqpoint{5.661999in}{6.177116in}}%
\pgfpathcurveto{\pgfqpoint{5.669812in}{6.184930in}}{\pgfqpoint{5.674203in}{6.195529in}}{\pgfqpoint{5.674203in}{6.206579in}}%
\pgfpathcurveto{\pgfqpoint{5.674203in}{6.217629in}}{\pgfqpoint{5.669812in}{6.228228in}}{\pgfqpoint{5.661999in}{6.236042in}}%
\pgfpathcurveto{\pgfqpoint{5.654185in}{6.243855in}}{\pgfqpoint{5.643586in}{6.248246in}}{\pgfqpoint{5.632536in}{6.248246in}}%
\pgfpathcurveto{\pgfqpoint{5.621486in}{6.248246in}}{\pgfqpoint{5.610887in}{6.243855in}}{\pgfqpoint{5.603073in}{6.236042in}}%
\pgfpathcurveto{\pgfqpoint{5.595260in}{6.228228in}}{\pgfqpoint{5.590869in}{6.217629in}}{\pgfqpoint{5.590869in}{6.206579in}}%
\pgfpathcurveto{\pgfqpoint{5.590869in}{6.195529in}}{\pgfqpoint{5.595260in}{6.184930in}}{\pgfqpoint{5.603073in}{6.177116in}}%
\pgfpathcurveto{\pgfqpoint{5.610887in}{6.169303in}}{\pgfqpoint{5.621486in}{6.164912in}}{\pgfqpoint{5.632536in}{6.164912in}}%
\pgfpathlineto{\pgfqpoint{5.632536in}{6.164912in}}%
\pgfpathclose%
\pgfusepath{stroke,fill}%
\end{pgfscope}%
\begin{pgfscope}%
\pgfpathrectangle{\pgfqpoint{5.292946in}{5.272501in}}{\pgfqpoint{2.177280in}{2.201755in}}%
\pgfusepath{clip}%
\pgfsetbuttcap%
\pgfsetroundjoin%
\definecolor{currentfill}{rgb}{0.121569,0.466667,0.705882}%
\pgfsetfillcolor{currentfill}%
\pgfsetlinewidth{0.481800pt}%
\definecolor{currentstroke}{rgb}{1.000000,1.000000,1.000000}%
\pgfsetstrokecolor{currentstroke}%
\pgfsetdash{}{0pt}%
\pgfpathmoveto{\pgfqpoint{5.632536in}{6.498512in}}%
\pgfpathcurveto{\pgfqpoint{5.643586in}{6.498512in}}{\pgfqpoint{5.654185in}{6.502902in}}{\pgfqpoint{5.661999in}{6.510715in}}%
\pgfpathcurveto{\pgfqpoint{5.669812in}{6.518529in}}{\pgfqpoint{5.674203in}{6.529128in}}{\pgfqpoint{5.674203in}{6.540178in}}%
\pgfpathcurveto{\pgfqpoint{5.674203in}{6.551228in}}{\pgfqpoint{5.669812in}{6.561827in}}{\pgfqpoint{5.661999in}{6.569641in}}%
\pgfpathcurveto{\pgfqpoint{5.654185in}{6.577455in}}{\pgfqpoint{5.643586in}{6.581845in}}{\pgfqpoint{5.632536in}{6.581845in}}%
\pgfpathcurveto{\pgfqpoint{5.621486in}{6.581845in}}{\pgfqpoint{5.610887in}{6.577455in}}{\pgfqpoint{5.603073in}{6.569641in}}%
\pgfpathcurveto{\pgfqpoint{5.595260in}{6.561827in}}{\pgfqpoint{5.590869in}{6.551228in}}{\pgfqpoint{5.590869in}{6.540178in}}%
\pgfpathcurveto{\pgfqpoint{5.590869in}{6.529128in}}{\pgfqpoint{5.595260in}{6.518529in}}{\pgfqpoint{5.603073in}{6.510715in}}%
\pgfpathcurveto{\pgfqpoint{5.610887in}{6.502902in}}{\pgfqpoint{5.621486in}{6.498512in}}{\pgfqpoint{5.632536in}{6.498512in}}%
\pgfpathlineto{\pgfqpoint{5.632536in}{6.498512in}}%
\pgfpathclose%
\pgfusepath{stroke,fill}%
\end{pgfscope}%
\begin{pgfscope}%
\pgfpathrectangle{\pgfqpoint{5.292946in}{5.272501in}}{\pgfqpoint{2.177280in}{2.201755in}}%
\pgfusepath{clip}%
\pgfsetbuttcap%
\pgfsetroundjoin%
\definecolor{currentfill}{rgb}{0.121569,0.466667,0.705882}%
\pgfsetfillcolor{currentfill}%
\pgfsetlinewidth{0.481800pt}%
\definecolor{currentstroke}{rgb}{1.000000,1.000000,1.000000}%
\pgfsetstrokecolor{currentstroke}%
\pgfsetdash{}{0pt}%
\pgfpathmoveto{\pgfqpoint{5.603831in}{6.581911in}}%
\pgfpathcurveto{\pgfqpoint{5.614881in}{6.581911in}}{\pgfqpoint{5.625480in}{6.586302in}}{\pgfqpoint{5.633294in}{6.594115in}}%
\pgfpathcurveto{\pgfqpoint{5.641107in}{6.601929in}}{\pgfqpoint{5.645497in}{6.612528in}}{\pgfqpoint{5.645497in}{6.623578in}}%
\pgfpathcurveto{\pgfqpoint{5.645497in}{6.634628in}}{\pgfqpoint{5.641107in}{6.645227in}}{\pgfqpoint{5.633294in}{6.653041in}}%
\pgfpathcurveto{\pgfqpoint{5.625480in}{6.660854in}}{\pgfqpoint{5.614881in}{6.665245in}}{\pgfqpoint{5.603831in}{6.665245in}}%
\pgfpathcurveto{\pgfqpoint{5.592781in}{6.665245in}}{\pgfqpoint{5.582182in}{6.660854in}}{\pgfqpoint{5.574368in}{6.653041in}}%
\pgfpathcurveto{\pgfqpoint{5.566554in}{6.645227in}}{\pgfqpoint{5.562164in}{6.634628in}}{\pgfqpoint{5.562164in}{6.623578in}}%
\pgfpathcurveto{\pgfqpoint{5.562164in}{6.612528in}}{\pgfqpoint{5.566554in}{6.601929in}}{\pgfqpoint{5.574368in}{6.594115in}}%
\pgfpathcurveto{\pgfqpoint{5.582182in}{6.586302in}}{\pgfqpoint{5.592781in}{6.581911in}}{\pgfqpoint{5.603831in}{6.581911in}}%
\pgfpathlineto{\pgfqpoint{5.603831in}{6.581911in}}%
\pgfpathclose%
\pgfusepath{stroke,fill}%
\end{pgfscope}%
\begin{pgfscope}%
\pgfpathrectangle{\pgfqpoint{5.292946in}{5.272501in}}{\pgfqpoint{2.177280in}{2.201755in}}%
\pgfusepath{clip}%
\pgfsetbuttcap%
\pgfsetroundjoin%
\definecolor{currentfill}{rgb}{0.121569,0.466667,0.705882}%
\pgfsetfillcolor{currentfill}%
\pgfsetlinewidth{0.481800pt}%
\definecolor{currentstroke}{rgb}{1.000000,1.000000,1.000000}%
\pgfsetstrokecolor{currentstroke}%
\pgfsetdash{}{0pt}%
\pgfpathmoveto{\pgfqpoint{5.575125in}{6.498512in}}%
\pgfpathcurveto{\pgfqpoint{5.586176in}{6.498512in}}{\pgfqpoint{5.596775in}{6.502902in}}{\pgfqpoint{5.604588in}{6.510715in}}%
\pgfpathcurveto{\pgfqpoint{5.612402in}{6.518529in}}{\pgfqpoint{5.616792in}{6.529128in}}{\pgfqpoint{5.616792in}{6.540178in}}%
\pgfpathcurveto{\pgfqpoint{5.616792in}{6.551228in}}{\pgfqpoint{5.612402in}{6.561827in}}{\pgfqpoint{5.604588in}{6.569641in}}%
\pgfpathcurveto{\pgfqpoint{5.596775in}{6.577455in}}{\pgfqpoint{5.586176in}{6.581845in}}{\pgfqpoint{5.575125in}{6.581845in}}%
\pgfpathcurveto{\pgfqpoint{5.564075in}{6.581845in}}{\pgfqpoint{5.553476in}{6.577455in}}{\pgfqpoint{5.545663in}{6.569641in}}%
\pgfpathcurveto{\pgfqpoint{5.537849in}{6.561827in}}{\pgfqpoint{5.533459in}{6.551228in}}{\pgfqpoint{5.533459in}{6.540178in}}%
\pgfpathcurveto{\pgfqpoint{5.533459in}{6.529128in}}{\pgfqpoint{5.537849in}{6.518529in}}{\pgfqpoint{5.545663in}{6.510715in}}%
\pgfpathcurveto{\pgfqpoint{5.553476in}{6.502902in}}{\pgfqpoint{5.564075in}{6.498512in}}{\pgfqpoint{5.575125in}{6.498512in}}%
\pgfpathlineto{\pgfqpoint{5.575125in}{6.498512in}}%
\pgfpathclose%
\pgfusepath{stroke,fill}%
\end{pgfscope}%
\begin{pgfscope}%
\pgfpathrectangle{\pgfqpoint{5.292946in}{5.272501in}}{\pgfqpoint{2.177280in}{2.201755in}}%
\pgfusepath{clip}%
\pgfsetbuttcap%
\pgfsetroundjoin%
\definecolor{currentfill}{rgb}{0.121569,0.466667,0.705882}%
\pgfsetfillcolor{currentfill}%
\pgfsetlinewidth{0.481800pt}%
\definecolor{currentstroke}{rgb}{1.000000,1.000000,1.000000}%
\pgfsetstrokecolor{currentstroke}%
\pgfsetdash{}{0pt}%
\pgfpathmoveto{\pgfqpoint{5.632536in}{6.331712in}}%
\pgfpathcurveto{\pgfqpoint{5.643586in}{6.331712in}}{\pgfqpoint{5.654185in}{6.336102in}}{\pgfqpoint{5.661999in}{6.343916in}}%
\pgfpathcurveto{\pgfqpoint{5.669812in}{6.351729in}}{\pgfqpoint{5.674203in}{6.362328in}}{\pgfqpoint{5.674203in}{6.373379in}}%
\pgfpathcurveto{\pgfqpoint{5.674203in}{6.384429in}}{\pgfqpoint{5.669812in}{6.395028in}}{\pgfqpoint{5.661999in}{6.402841in}}%
\pgfpathcurveto{\pgfqpoint{5.654185in}{6.410655in}}{\pgfqpoint{5.643586in}{6.415045in}}{\pgfqpoint{5.632536in}{6.415045in}}%
\pgfpathcurveto{\pgfqpoint{5.621486in}{6.415045in}}{\pgfqpoint{5.610887in}{6.410655in}}{\pgfqpoint{5.603073in}{6.402841in}}%
\pgfpathcurveto{\pgfqpoint{5.595260in}{6.395028in}}{\pgfqpoint{5.590869in}{6.384429in}}{\pgfqpoint{5.590869in}{6.373379in}}%
\pgfpathcurveto{\pgfqpoint{5.590869in}{6.362328in}}{\pgfqpoint{5.595260in}{6.351729in}}{\pgfqpoint{5.603073in}{6.343916in}}%
\pgfpathcurveto{\pgfqpoint{5.610887in}{6.336102in}}{\pgfqpoint{5.621486in}{6.331712in}}{\pgfqpoint{5.632536in}{6.331712in}}%
\pgfpathlineto{\pgfqpoint{5.632536in}{6.331712in}}%
\pgfpathclose%
\pgfusepath{stroke,fill}%
\end{pgfscope}%
\begin{pgfscope}%
\pgfpathrectangle{\pgfqpoint{5.292946in}{5.272501in}}{\pgfqpoint{2.177280in}{2.201755in}}%
\pgfusepath{clip}%
\pgfsetbuttcap%
\pgfsetroundjoin%
\definecolor{currentfill}{rgb}{0.121569,0.466667,0.705882}%
\pgfsetfillcolor{currentfill}%
\pgfsetlinewidth{0.481800pt}%
\definecolor{currentstroke}{rgb}{1.000000,1.000000,1.000000}%
\pgfsetstrokecolor{currentstroke}%
\pgfsetdash{}{0pt}%
\pgfpathmoveto{\pgfqpoint{5.632536in}{6.248312in}}%
\pgfpathcurveto{\pgfqpoint{5.643586in}{6.248312in}}{\pgfqpoint{5.654185in}{6.252702in}}{\pgfqpoint{5.661999in}{6.260516in}}%
\pgfpathcurveto{\pgfqpoint{5.669812in}{6.268330in}}{\pgfqpoint{5.674203in}{6.278929in}}{\pgfqpoint{5.674203in}{6.289979in}}%
\pgfpathcurveto{\pgfqpoint{5.674203in}{6.301029in}}{\pgfqpoint{5.669812in}{6.311628in}}{\pgfqpoint{5.661999in}{6.319442in}}%
\pgfpathcurveto{\pgfqpoint{5.654185in}{6.327255in}}{\pgfqpoint{5.643586in}{6.331645in}}{\pgfqpoint{5.632536in}{6.331645in}}%
\pgfpathcurveto{\pgfqpoint{5.621486in}{6.331645in}}{\pgfqpoint{5.610887in}{6.327255in}}{\pgfqpoint{5.603073in}{6.319442in}}%
\pgfpathcurveto{\pgfqpoint{5.595260in}{6.311628in}}{\pgfqpoint{5.590869in}{6.301029in}}{\pgfqpoint{5.590869in}{6.289979in}}%
\pgfpathcurveto{\pgfqpoint{5.590869in}{6.278929in}}{\pgfqpoint{5.595260in}{6.268330in}}{\pgfqpoint{5.603073in}{6.260516in}}%
\pgfpathcurveto{\pgfqpoint{5.610887in}{6.252702in}}{\pgfqpoint{5.621486in}{6.248312in}}{\pgfqpoint{5.632536in}{6.248312in}}%
\pgfpathlineto{\pgfqpoint{5.632536in}{6.248312in}}%
\pgfpathclose%
\pgfusepath{stroke,fill}%
\end{pgfscope}%
\begin{pgfscope}%
\pgfpathrectangle{\pgfqpoint{5.292946in}{5.272501in}}{\pgfqpoint{2.177280in}{2.201755in}}%
\pgfusepath{clip}%
\pgfsetbuttcap%
\pgfsetroundjoin%
\definecolor{currentfill}{rgb}{0.121569,0.466667,0.705882}%
\pgfsetfillcolor{currentfill}%
\pgfsetlinewidth{0.481800pt}%
\definecolor{currentstroke}{rgb}{1.000000,1.000000,1.000000}%
\pgfsetstrokecolor{currentstroke}%
\pgfsetdash{}{0pt}%
\pgfpathmoveto{\pgfqpoint{5.603831in}{6.498512in}}%
\pgfpathcurveto{\pgfqpoint{5.614881in}{6.498512in}}{\pgfqpoint{5.625480in}{6.502902in}}{\pgfqpoint{5.633294in}{6.510715in}}%
\pgfpathcurveto{\pgfqpoint{5.641107in}{6.518529in}}{\pgfqpoint{5.645497in}{6.529128in}}{\pgfqpoint{5.645497in}{6.540178in}}%
\pgfpathcurveto{\pgfqpoint{5.645497in}{6.551228in}}{\pgfqpoint{5.641107in}{6.561827in}}{\pgfqpoint{5.633294in}{6.569641in}}%
\pgfpathcurveto{\pgfqpoint{5.625480in}{6.577455in}}{\pgfqpoint{5.614881in}{6.581845in}}{\pgfqpoint{5.603831in}{6.581845in}}%
\pgfpathcurveto{\pgfqpoint{5.592781in}{6.581845in}}{\pgfqpoint{5.582182in}{6.577455in}}{\pgfqpoint{5.574368in}{6.569641in}}%
\pgfpathcurveto{\pgfqpoint{5.566554in}{6.561827in}}{\pgfqpoint{5.562164in}{6.551228in}}{\pgfqpoint{5.562164in}{6.540178in}}%
\pgfpathcurveto{\pgfqpoint{5.562164in}{6.529128in}}{\pgfqpoint{5.566554in}{6.518529in}}{\pgfqpoint{5.574368in}{6.510715in}}%
\pgfpathcurveto{\pgfqpoint{5.582182in}{6.502902in}}{\pgfqpoint{5.592781in}{6.498512in}}{\pgfqpoint{5.603831in}{6.498512in}}%
\pgfpathlineto{\pgfqpoint{5.603831in}{6.498512in}}%
\pgfpathclose%
\pgfusepath{stroke,fill}%
\end{pgfscope}%
\begin{pgfscope}%
\pgfpathrectangle{\pgfqpoint{5.292946in}{5.272501in}}{\pgfqpoint{2.177280in}{2.201755in}}%
\pgfusepath{clip}%
\pgfsetbuttcap%
\pgfsetroundjoin%
\definecolor{currentfill}{rgb}{0.121569,0.466667,0.705882}%
\pgfsetfillcolor{currentfill}%
\pgfsetlinewidth{0.481800pt}%
\definecolor{currentstroke}{rgb}{1.000000,1.000000,1.000000}%
\pgfsetstrokecolor{currentstroke}%
\pgfsetdash{}{0pt}%
\pgfpathmoveto{\pgfqpoint{5.603831in}{7.082310in}}%
\pgfpathcurveto{\pgfqpoint{5.614881in}{7.082310in}}{\pgfqpoint{5.625480in}{7.086700in}}{\pgfqpoint{5.633294in}{7.094514in}}%
\pgfpathcurveto{\pgfqpoint{5.641107in}{7.102328in}}{\pgfqpoint{5.645497in}{7.112927in}}{\pgfqpoint{5.645497in}{7.123977in}}%
\pgfpathcurveto{\pgfqpoint{5.645497in}{7.135027in}}{\pgfqpoint{5.641107in}{7.145626in}}{\pgfqpoint{5.633294in}{7.153440in}}%
\pgfpathcurveto{\pgfqpoint{5.625480in}{7.161253in}}{\pgfqpoint{5.614881in}{7.165644in}}{\pgfqpoint{5.603831in}{7.165644in}}%
\pgfpathcurveto{\pgfqpoint{5.592781in}{7.165644in}}{\pgfqpoint{5.582182in}{7.161253in}}{\pgfqpoint{5.574368in}{7.153440in}}%
\pgfpathcurveto{\pgfqpoint{5.566554in}{7.145626in}}{\pgfqpoint{5.562164in}{7.135027in}}{\pgfqpoint{5.562164in}{7.123977in}}%
\pgfpathcurveto{\pgfqpoint{5.562164in}{7.112927in}}{\pgfqpoint{5.566554in}{7.102328in}}{\pgfqpoint{5.574368in}{7.094514in}}%
\pgfpathcurveto{\pgfqpoint{5.582182in}{7.086700in}}{\pgfqpoint{5.592781in}{7.082310in}}{\pgfqpoint{5.603831in}{7.082310in}}%
\pgfpathlineto{\pgfqpoint{5.603831in}{7.082310in}}%
\pgfpathclose%
\pgfusepath{stroke,fill}%
\end{pgfscope}%
\begin{pgfscope}%
\pgfpathrectangle{\pgfqpoint{5.292946in}{5.272501in}}{\pgfqpoint{2.177280in}{2.201755in}}%
\pgfusepath{clip}%
\pgfsetbuttcap%
\pgfsetroundjoin%
\definecolor{currentfill}{rgb}{0.121569,0.466667,0.705882}%
\pgfsetfillcolor{currentfill}%
\pgfsetlinewidth{0.481800pt}%
\definecolor{currentstroke}{rgb}{1.000000,1.000000,1.000000}%
\pgfsetstrokecolor{currentstroke}%
\pgfsetdash{}{0pt}%
\pgfpathmoveto{\pgfqpoint{5.575125in}{7.165710in}}%
\pgfpathcurveto{\pgfqpoint{5.586176in}{7.165710in}}{\pgfqpoint{5.596775in}{7.170100in}}{\pgfqpoint{5.604588in}{7.177914in}}%
\pgfpathcurveto{\pgfqpoint{5.612402in}{7.185728in}}{\pgfqpoint{5.616792in}{7.196327in}}{\pgfqpoint{5.616792in}{7.207377in}}%
\pgfpathcurveto{\pgfqpoint{5.616792in}{7.218427in}}{\pgfqpoint{5.612402in}{7.229026in}}{\pgfqpoint{5.604588in}{7.236839in}}%
\pgfpathcurveto{\pgfqpoint{5.596775in}{7.244653in}}{\pgfqpoint{5.586176in}{7.249043in}}{\pgfqpoint{5.575125in}{7.249043in}}%
\pgfpathcurveto{\pgfqpoint{5.564075in}{7.249043in}}{\pgfqpoint{5.553476in}{7.244653in}}{\pgfqpoint{5.545663in}{7.236839in}}%
\pgfpathcurveto{\pgfqpoint{5.537849in}{7.229026in}}{\pgfqpoint{5.533459in}{7.218427in}}{\pgfqpoint{5.533459in}{7.207377in}}%
\pgfpathcurveto{\pgfqpoint{5.533459in}{7.196327in}}{\pgfqpoint{5.537849in}{7.185728in}}{\pgfqpoint{5.545663in}{7.177914in}}%
\pgfpathcurveto{\pgfqpoint{5.553476in}{7.170100in}}{\pgfqpoint{5.564075in}{7.165710in}}{\pgfqpoint{5.575125in}{7.165710in}}%
\pgfpathlineto{\pgfqpoint{5.575125in}{7.165710in}}%
\pgfpathclose%
\pgfusepath{stroke,fill}%
\end{pgfscope}%
\begin{pgfscope}%
\pgfpathrectangle{\pgfqpoint{5.292946in}{5.272501in}}{\pgfqpoint{2.177280in}{2.201755in}}%
\pgfusepath{clip}%
\pgfsetbuttcap%
\pgfsetroundjoin%
\definecolor{currentfill}{rgb}{0.121569,0.466667,0.705882}%
\pgfsetfillcolor{currentfill}%
\pgfsetlinewidth{0.481800pt}%
\definecolor{currentstroke}{rgb}{1.000000,1.000000,1.000000}%
\pgfsetstrokecolor{currentstroke}%
\pgfsetdash{}{0pt}%
\pgfpathmoveto{\pgfqpoint{5.603831in}{6.248312in}}%
\pgfpathcurveto{\pgfqpoint{5.614881in}{6.248312in}}{\pgfqpoint{5.625480in}{6.252702in}}{\pgfqpoint{5.633294in}{6.260516in}}%
\pgfpathcurveto{\pgfqpoint{5.641107in}{6.268330in}}{\pgfqpoint{5.645497in}{6.278929in}}{\pgfqpoint{5.645497in}{6.289979in}}%
\pgfpathcurveto{\pgfqpoint{5.645497in}{6.301029in}}{\pgfqpoint{5.641107in}{6.311628in}}{\pgfqpoint{5.633294in}{6.319442in}}%
\pgfpathcurveto{\pgfqpoint{5.625480in}{6.327255in}}{\pgfqpoint{5.614881in}{6.331645in}}{\pgfqpoint{5.603831in}{6.331645in}}%
\pgfpathcurveto{\pgfqpoint{5.592781in}{6.331645in}}{\pgfqpoint{5.582182in}{6.327255in}}{\pgfqpoint{5.574368in}{6.319442in}}%
\pgfpathcurveto{\pgfqpoint{5.566554in}{6.311628in}}{\pgfqpoint{5.562164in}{6.301029in}}{\pgfqpoint{5.562164in}{6.289979in}}%
\pgfpathcurveto{\pgfqpoint{5.562164in}{6.278929in}}{\pgfqpoint{5.566554in}{6.268330in}}{\pgfqpoint{5.574368in}{6.260516in}}%
\pgfpathcurveto{\pgfqpoint{5.582182in}{6.252702in}}{\pgfqpoint{5.592781in}{6.248312in}}{\pgfqpoint{5.603831in}{6.248312in}}%
\pgfpathlineto{\pgfqpoint{5.603831in}{6.248312in}}%
\pgfpathclose%
\pgfusepath{stroke,fill}%
\end{pgfscope}%
\begin{pgfscope}%
\pgfpathrectangle{\pgfqpoint{5.292946in}{5.272501in}}{\pgfqpoint{2.177280in}{2.201755in}}%
\pgfusepath{clip}%
\pgfsetbuttcap%
\pgfsetroundjoin%
\definecolor{currentfill}{rgb}{0.121569,0.466667,0.705882}%
\pgfsetfillcolor{currentfill}%
\pgfsetlinewidth{0.481800pt}%
\definecolor{currentstroke}{rgb}{1.000000,1.000000,1.000000}%
\pgfsetstrokecolor{currentstroke}%
\pgfsetdash{}{0pt}%
\pgfpathmoveto{\pgfqpoint{5.517715in}{6.331712in}}%
\pgfpathcurveto{\pgfqpoint{5.528765in}{6.331712in}}{\pgfqpoint{5.539364in}{6.336102in}}{\pgfqpoint{5.547178in}{6.343916in}}%
\pgfpathcurveto{\pgfqpoint{5.554991in}{6.351729in}}{\pgfqpoint{5.559382in}{6.362328in}}{\pgfqpoint{5.559382in}{6.373379in}}%
\pgfpathcurveto{\pgfqpoint{5.559382in}{6.384429in}}{\pgfqpoint{5.554991in}{6.395028in}}{\pgfqpoint{5.547178in}{6.402841in}}%
\pgfpathcurveto{\pgfqpoint{5.539364in}{6.410655in}}{\pgfqpoint{5.528765in}{6.415045in}}{\pgfqpoint{5.517715in}{6.415045in}}%
\pgfpathcurveto{\pgfqpoint{5.506665in}{6.415045in}}{\pgfqpoint{5.496066in}{6.410655in}}{\pgfqpoint{5.488252in}{6.402841in}}%
\pgfpathcurveto{\pgfqpoint{5.480438in}{6.395028in}}{\pgfqpoint{5.476048in}{6.384429in}}{\pgfqpoint{5.476048in}{6.373379in}}%
\pgfpathcurveto{\pgfqpoint{5.476048in}{6.362328in}}{\pgfqpoint{5.480438in}{6.351729in}}{\pgfqpoint{5.488252in}{6.343916in}}%
\pgfpathcurveto{\pgfqpoint{5.496066in}{6.336102in}}{\pgfqpoint{5.506665in}{6.331712in}}{\pgfqpoint{5.517715in}{6.331712in}}%
\pgfpathlineto{\pgfqpoint{5.517715in}{6.331712in}}%
\pgfpathclose%
\pgfusepath{stroke,fill}%
\end{pgfscope}%
\begin{pgfscope}%
\pgfpathrectangle{\pgfqpoint{5.292946in}{5.272501in}}{\pgfqpoint{2.177280in}{2.201755in}}%
\pgfusepath{clip}%
\pgfsetbuttcap%
\pgfsetroundjoin%
\definecolor{currentfill}{rgb}{0.121569,0.466667,0.705882}%
\pgfsetfillcolor{currentfill}%
\pgfsetlinewidth{0.481800pt}%
\definecolor{currentstroke}{rgb}{1.000000,1.000000,1.000000}%
\pgfsetstrokecolor{currentstroke}%
\pgfsetdash{}{0pt}%
\pgfpathmoveto{\pgfqpoint{5.546420in}{6.581911in}}%
\pgfpathcurveto{\pgfqpoint{5.557470in}{6.581911in}}{\pgfqpoint{5.568069in}{6.586302in}}{\pgfqpoint{5.575883in}{6.594115in}}%
\pgfpathcurveto{\pgfqpoint{5.583697in}{6.601929in}}{\pgfqpoint{5.588087in}{6.612528in}}{\pgfqpoint{5.588087in}{6.623578in}}%
\pgfpathcurveto{\pgfqpoint{5.588087in}{6.634628in}}{\pgfqpoint{5.583697in}{6.645227in}}{\pgfqpoint{5.575883in}{6.653041in}}%
\pgfpathcurveto{\pgfqpoint{5.568069in}{6.660854in}}{\pgfqpoint{5.557470in}{6.665245in}}{\pgfqpoint{5.546420in}{6.665245in}}%
\pgfpathcurveto{\pgfqpoint{5.535370in}{6.665245in}}{\pgfqpoint{5.524771in}{6.660854in}}{\pgfqpoint{5.516957in}{6.653041in}}%
\pgfpathcurveto{\pgfqpoint{5.509144in}{6.645227in}}{\pgfqpoint{5.504753in}{6.634628in}}{\pgfqpoint{5.504753in}{6.623578in}}%
\pgfpathcurveto{\pgfqpoint{5.504753in}{6.612528in}}{\pgfqpoint{5.509144in}{6.601929in}}{\pgfqpoint{5.516957in}{6.594115in}}%
\pgfpathcurveto{\pgfqpoint{5.524771in}{6.586302in}}{\pgfqpoint{5.535370in}{6.581911in}}{\pgfqpoint{5.546420in}{6.581911in}}%
\pgfpathlineto{\pgfqpoint{5.546420in}{6.581911in}}%
\pgfpathclose%
\pgfusepath{stroke,fill}%
\end{pgfscope}%
\begin{pgfscope}%
\pgfpathrectangle{\pgfqpoint{5.292946in}{5.272501in}}{\pgfqpoint{2.177280in}{2.201755in}}%
\pgfusepath{clip}%
\pgfsetbuttcap%
\pgfsetroundjoin%
\definecolor{currentfill}{rgb}{0.121569,0.466667,0.705882}%
\pgfsetfillcolor{currentfill}%
\pgfsetlinewidth{0.481800pt}%
\definecolor{currentstroke}{rgb}{1.000000,1.000000,1.000000}%
\pgfsetstrokecolor{currentstroke}%
\pgfsetdash{}{0pt}%
\pgfpathmoveto{\pgfqpoint{5.575125in}{6.665311in}}%
\pgfpathcurveto{\pgfqpoint{5.586176in}{6.665311in}}{\pgfqpoint{5.596775in}{6.669701in}}{\pgfqpoint{5.604588in}{6.677515in}}%
\pgfpathcurveto{\pgfqpoint{5.612402in}{6.685329in}}{\pgfqpoint{5.616792in}{6.695928in}}{\pgfqpoint{5.616792in}{6.706978in}}%
\pgfpathcurveto{\pgfqpoint{5.616792in}{6.718028in}}{\pgfqpoint{5.612402in}{6.728627in}}{\pgfqpoint{5.604588in}{6.736441in}}%
\pgfpathcurveto{\pgfqpoint{5.596775in}{6.744254in}}{\pgfqpoint{5.586176in}{6.748644in}}{\pgfqpoint{5.575125in}{6.748644in}}%
\pgfpathcurveto{\pgfqpoint{5.564075in}{6.748644in}}{\pgfqpoint{5.553476in}{6.744254in}}{\pgfqpoint{5.545663in}{6.736441in}}%
\pgfpathcurveto{\pgfqpoint{5.537849in}{6.728627in}}{\pgfqpoint{5.533459in}{6.718028in}}{\pgfqpoint{5.533459in}{6.706978in}}%
\pgfpathcurveto{\pgfqpoint{5.533459in}{6.695928in}}{\pgfqpoint{5.537849in}{6.685329in}}{\pgfqpoint{5.545663in}{6.677515in}}%
\pgfpathcurveto{\pgfqpoint{5.553476in}{6.669701in}}{\pgfqpoint{5.564075in}{6.665311in}}{\pgfqpoint{5.575125in}{6.665311in}}%
\pgfpathlineto{\pgfqpoint{5.575125in}{6.665311in}}%
\pgfpathclose%
\pgfusepath{stroke,fill}%
\end{pgfscope}%
\begin{pgfscope}%
\pgfpathrectangle{\pgfqpoint{5.292946in}{5.272501in}}{\pgfqpoint{2.177280in}{2.201755in}}%
\pgfusepath{clip}%
\pgfsetbuttcap%
\pgfsetroundjoin%
\definecolor{currentfill}{rgb}{0.121569,0.466667,0.705882}%
\pgfsetfillcolor{currentfill}%
\pgfsetlinewidth{0.481800pt}%
\definecolor{currentstroke}{rgb}{1.000000,1.000000,1.000000}%
\pgfsetstrokecolor{currentstroke}%
\pgfsetdash{}{0pt}%
\pgfpathmoveto{\pgfqpoint{5.546420in}{6.164912in}}%
\pgfpathcurveto{\pgfqpoint{5.557470in}{6.164912in}}{\pgfqpoint{5.568069in}{6.169303in}}{\pgfqpoint{5.575883in}{6.177116in}}%
\pgfpathcurveto{\pgfqpoint{5.583697in}{6.184930in}}{\pgfqpoint{5.588087in}{6.195529in}}{\pgfqpoint{5.588087in}{6.206579in}}%
\pgfpathcurveto{\pgfqpoint{5.588087in}{6.217629in}}{\pgfqpoint{5.583697in}{6.228228in}}{\pgfqpoint{5.575883in}{6.236042in}}%
\pgfpathcurveto{\pgfqpoint{5.568069in}{6.243855in}}{\pgfqpoint{5.557470in}{6.248246in}}{\pgfqpoint{5.546420in}{6.248246in}}%
\pgfpathcurveto{\pgfqpoint{5.535370in}{6.248246in}}{\pgfqpoint{5.524771in}{6.243855in}}{\pgfqpoint{5.516957in}{6.236042in}}%
\pgfpathcurveto{\pgfqpoint{5.509144in}{6.228228in}}{\pgfqpoint{5.504753in}{6.217629in}}{\pgfqpoint{5.504753in}{6.206579in}}%
\pgfpathcurveto{\pgfqpoint{5.504753in}{6.195529in}}{\pgfqpoint{5.509144in}{6.184930in}}{\pgfqpoint{5.516957in}{6.177116in}}%
\pgfpathcurveto{\pgfqpoint{5.524771in}{6.169303in}}{\pgfqpoint{5.535370in}{6.164912in}}{\pgfqpoint{5.546420in}{6.164912in}}%
\pgfpathlineto{\pgfqpoint{5.546420in}{6.164912in}}%
\pgfpathclose%
\pgfusepath{stroke,fill}%
\end{pgfscope}%
\begin{pgfscope}%
\pgfpathrectangle{\pgfqpoint{5.292946in}{5.272501in}}{\pgfqpoint{2.177280in}{2.201755in}}%
\pgfusepath{clip}%
\pgfsetbuttcap%
\pgfsetroundjoin%
\definecolor{currentfill}{rgb}{0.121569,0.466667,0.705882}%
\pgfsetfillcolor{currentfill}%
\pgfsetlinewidth{0.481800pt}%
\definecolor{currentstroke}{rgb}{1.000000,1.000000,1.000000}%
\pgfsetstrokecolor{currentstroke}%
\pgfsetdash{}{0pt}%
\pgfpathmoveto{\pgfqpoint{5.603831in}{6.498512in}}%
\pgfpathcurveto{\pgfqpoint{5.614881in}{6.498512in}}{\pgfqpoint{5.625480in}{6.502902in}}{\pgfqpoint{5.633294in}{6.510715in}}%
\pgfpathcurveto{\pgfqpoint{5.641107in}{6.518529in}}{\pgfqpoint{5.645497in}{6.529128in}}{\pgfqpoint{5.645497in}{6.540178in}}%
\pgfpathcurveto{\pgfqpoint{5.645497in}{6.551228in}}{\pgfqpoint{5.641107in}{6.561827in}}{\pgfqpoint{5.633294in}{6.569641in}}%
\pgfpathcurveto{\pgfqpoint{5.625480in}{6.577455in}}{\pgfqpoint{5.614881in}{6.581845in}}{\pgfqpoint{5.603831in}{6.581845in}}%
\pgfpathcurveto{\pgfqpoint{5.592781in}{6.581845in}}{\pgfqpoint{5.582182in}{6.577455in}}{\pgfqpoint{5.574368in}{6.569641in}}%
\pgfpathcurveto{\pgfqpoint{5.566554in}{6.561827in}}{\pgfqpoint{5.562164in}{6.551228in}}{\pgfqpoint{5.562164in}{6.540178in}}%
\pgfpathcurveto{\pgfqpoint{5.562164in}{6.529128in}}{\pgfqpoint{5.566554in}{6.518529in}}{\pgfqpoint{5.574368in}{6.510715in}}%
\pgfpathcurveto{\pgfqpoint{5.582182in}{6.502902in}}{\pgfqpoint{5.592781in}{6.498512in}}{\pgfqpoint{5.603831in}{6.498512in}}%
\pgfpathlineto{\pgfqpoint{5.603831in}{6.498512in}}%
\pgfpathclose%
\pgfusepath{stroke,fill}%
\end{pgfscope}%
\begin{pgfscope}%
\pgfpathrectangle{\pgfqpoint{5.292946in}{5.272501in}}{\pgfqpoint{2.177280in}{2.201755in}}%
\pgfusepath{clip}%
\pgfsetbuttcap%
\pgfsetroundjoin%
\definecolor{currentfill}{rgb}{0.121569,0.466667,0.705882}%
\pgfsetfillcolor{currentfill}%
\pgfsetlinewidth{0.481800pt}%
\definecolor{currentstroke}{rgb}{1.000000,1.000000,1.000000}%
\pgfsetstrokecolor{currentstroke}%
\pgfsetdash{}{0pt}%
\pgfpathmoveto{\pgfqpoint{5.546420in}{6.581911in}}%
\pgfpathcurveto{\pgfqpoint{5.557470in}{6.581911in}}{\pgfqpoint{5.568069in}{6.586302in}}{\pgfqpoint{5.575883in}{6.594115in}}%
\pgfpathcurveto{\pgfqpoint{5.583697in}{6.601929in}}{\pgfqpoint{5.588087in}{6.612528in}}{\pgfqpoint{5.588087in}{6.623578in}}%
\pgfpathcurveto{\pgfqpoint{5.588087in}{6.634628in}}{\pgfqpoint{5.583697in}{6.645227in}}{\pgfqpoint{5.575883in}{6.653041in}}%
\pgfpathcurveto{\pgfqpoint{5.568069in}{6.660854in}}{\pgfqpoint{5.557470in}{6.665245in}}{\pgfqpoint{5.546420in}{6.665245in}}%
\pgfpathcurveto{\pgfqpoint{5.535370in}{6.665245in}}{\pgfqpoint{5.524771in}{6.660854in}}{\pgfqpoint{5.516957in}{6.653041in}}%
\pgfpathcurveto{\pgfqpoint{5.509144in}{6.645227in}}{\pgfqpoint{5.504753in}{6.634628in}}{\pgfqpoint{5.504753in}{6.623578in}}%
\pgfpathcurveto{\pgfqpoint{5.504753in}{6.612528in}}{\pgfqpoint{5.509144in}{6.601929in}}{\pgfqpoint{5.516957in}{6.594115in}}%
\pgfpathcurveto{\pgfqpoint{5.524771in}{6.586302in}}{\pgfqpoint{5.535370in}{6.581911in}}{\pgfqpoint{5.546420in}{6.581911in}}%
\pgfpathlineto{\pgfqpoint{5.546420in}{6.581911in}}%
\pgfpathclose%
\pgfusepath{stroke,fill}%
\end{pgfscope}%
\begin{pgfscope}%
\pgfpathrectangle{\pgfqpoint{5.292946in}{5.272501in}}{\pgfqpoint{2.177280in}{2.201755in}}%
\pgfusepath{clip}%
\pgfsetbuttcap%
\pgfsetroundjoin%
\definecolor{currentfill}{rgb}{0.121569,0.466667,0.705882}%
\pgfsetfillcolor{currentfill}%
\pgfsetlinewidth{0.481800pt}%
\definecolor{currentstroke}{rgb}{1.000000,1.000000,1.000000}%
\pgfsetstrokecolor{currentstroke}%
\pgfsetdash{}{0pt}%
\pgfpathmoveto{\pgfqpoint{5.546420in}{5.581114in}}%
\pgfpathcurveto{\pgfqpoint{5.557470in}{5.581114in}}{\pgfqpoint{5.568069in}{5.585504in}}{\pgfqpoint{5.575883in}{5.593317in}}%
\pgfpathcurveto{\pgfqpoint{5.583697in}{5.601131in}}{\pgfqpoint{5.588087in}{5.611730in}}{\pgfqpoint{5.588087in}{5.622780in}}%
\pgfpathcurveto{\pgfqpoint{5.588087in}{5.633830in}}{\pgfqpoint{5.583697in}{5.644429in}}{\pgfqpoint{5.575883in}{5.652243in}}%
\pgfpathcurveto{\pgfqpoint{5.568069in}{5.660057in}}{\pgfqpoint{5.557470in}{5.664447in}}{\pgfqpoint{5.546420in}{5.664447in}}%
\pgfpathcurveto{\pgfqpoint{5.535370in}{5.664447in}}{\pgfqpoint{5.524771in}{5.660057in}}{\pgfqpoint{5.516957in}{5.652243in}}%
\pgfpathcurveto{\pgfqpoint{5.509144in}{5.644429in}}{\pgfqpoint{5.504753in}{5.633830in}}{\pgfqpoint{5.504753in}{5.622780in}}%
\pgfpathcurveto{\pgfqpoint{5.504753in}{5.611730in}}{\pgfqpoint{5.509144in}{5.601131in}}{\pgfqpoint{5.516957in}{5.593317in}}%
\pgfpathcurveto{\pgfqpoint{5.524771in}{5.585504in}}{\pgfqpoint{5.535370in}{5.581114in}}{\pgfqpoint{5.546420in}{5.581114in}}%
\pgfpathlineto{\pgfqpoint{5.546420in}{5.581114in}}%
\pgfpathclose%
\pgfusepath{stroke,fill}%
\end{pgfscope}%
\begin{pgfscope}%
\pgfpathrectangle{\pgfqpoint{5.292946in}{5.272501in}}{\pgfqpoint{2.177280in}{2.201755in}}%
\pgfusepath{clip}%
\pgfsetbuttcap%
\pgfsetroundjoin%
\definecolor{currentfill}{rgb}{0.121569,0.466667,0.705882}%
\pgfsetfillcolor{currentfill}%
\pgfsetlinewidth{0.481800pt}%
\definecolor{currentstroke}{rgb}{1.000000,1.000000,1.000000}%
\pgfsetstrokecolor{currentstroke}%
\pgfsetdash{}{0pt}%
\pgfpathmoveto{\pgfqpoint{5.546420in}{6.331712in}}%
\pgfpathcurveto{\pgfqpoint{5.557470in}{6.331712in}}{\pgfqpoint{5.568069in}{6.336102in}}{\pgfqpoint{5.575883in}{6.343916in}}%
\pgfpathcurveto{\pgfqpoint{5.583697in}{6.351729in}}{\pgfqpoint{5.588087in}{6.362328in}}{\pgfqpoint{5.588087in}{6.373379in}}%
\pgfpathcurveto{\pgfqpoint{5.588087in}{6.384429in}}{\pgfqpoint{5.583697in}{6.395028in}}{\pgfqpoint{5.575883in}{6.402841in}}%
\pgfpathcurveto{\pgfqpoint{5.568069in}{6.410655in}}{\pgfqpoint{5.557470in}{6.415045in}}{\pgfqpoint{5.546420in}{6.415045in}}%
\pgfpathcurveto{\pgfqpoint{5.535370in}{6.415045in}}{\pgfqpoint{5.524771in}{6.410655in}}{\pgfqpoint{5.516957in}{6.402841in}}%
\pgfpathcurveto{\pgfqpoint{5.509144in}{6.395028in}}{\pgfqpoint{5.504753in}{6.384429in}}{\pgfqpoint{5.504753in}{6.373379in}}%
\pgfpathcurveto{\pgfqpoint{5.504753in}{6.362328in}}{\pgfqpoint{5.509144in}{6.351729in}}{\pgfqpoint{5.516957in}{6.343916in}}%
\pgfpathcurveto{\pgfqpoint{5.524771in}{6.336102in}}{\pgfqpoint{5.535370in}{6.331712in}}{\pgfqpoint{5.546420in}{6.331712in}}%
\pgfpathlineto{\pgfqpoint{5.546420in}{6.331712in}}%
\pgfpathclose%
\pgfusepath{stroke,fill}%
\end{pgfscope}%
\begin{pgfscope}%
\pgfpathrectangle{\pgfqpoint{5.292946in}{5.272501in}}{\pgfqpoint{2.177280in}{2.201755in}}%
\pgfusepath{clip}%
\pgfsetbuttcap%
\pgfsetroundjoin%
\definecolor{currentfill}{rgb}{0.121569,0.466667,0.705882}%
\pgfsetfillcolor{currentfill}%
\pgfsetlinewidth{0.481800pt}%
\definecolor{currentstroke}{rgb}{1.000000,1.000000,1.000000}%
\pgfsetstrokecolor{currentstroke}%
\pgfsetdash{}{0pt}%
\pgfpathmoveto{\pgfqpoint{5.632536in}{6.581911in}}%
\pgfpathcurveto{\pgfqpoint{5.643586in}{6.581911in}}{\pgfqpoint{5.654185in}{6.586302in}}{\pgfqpoint{5.661999in}{6.594115in}}%
\pgfpathcurveto{\pgfqpoint{5.669812in}{6.601929in}}{\pgfqpoint{5.674203in}{6.612528in}}{\pgfqpoint{5.674203in}{6.623578in}}%
\pgfpathcurveto{\pgfqpoint{5.674203in}{6.634628in}}{\pgfqpoint{5.669812in}{6.645227in}}{\pgfqpoint{5.661999in}{6.653041in}}%
\pgfpathcurveto{\pgfqpoint{5.654185in}{6.660854in}}{\pgfqpoint{5.643586in}{6.665245in}}{\pgfqpoint{5.632536in}{6.665245in}}%
\pgfpathcurveto{\pgfqpoint{5.621486in}{6.665245in}}{\pgfqpoint{5.610887in}{6.660854in}}{\pgfqpoint{5.603073in}{6.653041in}}%
\pgfpathcurveto{\pgfqpoint{5.595260in}{6.645227in}}{\pgfqpoint{5.590869in}{6.634628in}}{\pgfqpoint{5.590869in}{6.623578in}}%
\pgfpathcurveto{\pgfqpoint{5.590869in}{6.612528in}}{\pgfqpoint{5.595260in}{6.601929in}}{\pgfqpoint{5.603073in}{6.594115in}}%
\pgfpathcurveto{\pgfqpoint{5.610887in}{6.586302in}}{\pgfqpoint{5.621486in}{6.581911in}}{\pgfqpoint{5.632536in}{6.581911in}}%
\pgfpathlineto{\pgfqpoint{5.632536in}{6.581911in}}%
\pgfpathclose%
\pgfusepath{stroke,fill}%
\end{pgfscope}%
\begin{pgfscope}%
\pgfpathrectangle{\pgfqpoint{5.292946in}{5.272501in}}{\pgfqpoint{2.177280in}{2.201755in}}%
\pgfusepath{clip}%
\pgfsetbuttcap%
\pgfsetroundjoin%
\definecolor{currentfill}{rgb}{0.121569,0.466667,0.705882}%
\pgfsetfillcolor{currentfill}%
\pgfsetlinewidth{0.481800pt}%
\definecolor{currentstroke}{rgb}{1.000000,1.000000,1.000000}%
\pgfsetstrokecolor{currentstroke}%
\pgfsetdash{}{0pt}%
\pgfpathmoveto{\pgfqpoint{5.718652in}{6.832111in}}%
\pgfpathcurveto{\pgfqpoint{5.729702in}{6.832111in}}{\pgfqpoint{5.740301in}{6.836501in}}{\pgfqpoint{5.748115in}{6.844315in}}%
\pgfpathcurveto{\pgfqpoint{5.755928in}{6.852128in}}{\pgfqpoint{5.760319in}{6.862727in}}{\pgfqpoint{5.760319in}{6.873777in}}%
\pgfpathcurveto{\pgfqpoint{5.760319in}{6.884828in}}{\pgfqpoint{5.755928in}{6.895427in}}{\pgfqpoint{5.748115in}{6.903240in}}%
\pgfpathcurveto{\pgfqpoint{5.740301in}{6.911054in}}{\pgfqpoint{5.729702in}{6.915444in}}{\pgfqpoint{5.718652in}{6.915444in}}%
\pgfpathcurveto{\pgfqpoint{5.707602in}{6.915444in}}{\pgfqpoint{5.697003in}{6.911054in}}{\pgfqpoint{5.689189in}{6.903240in}}%
\pgfpathcurveto{\pgfqpoint{5.681375in}{6.895427in}}{\pgfqpoint{5.676985in}{6.884828in}}{\pgfqpoint{5.676985in}{6.873777in}}%
\pgfpathcurveto{\pgfqpoint{5.676985in}{6.862727in}}{\pgfqpoint{5.681375in}{6.852128in}}{\pgfqpoint{5.689189in}{6.844315in}}%
\pgfpathcurveto{\pgfqpoint{5.697003in}{6.836501in}}{\pgfqpoint{5.707602in}{6.832111in}}{\pgfqpoint{5.718652in}{6.832111in}}%
\pgfpathlineto{\pgfqpoint{5.718652in}{6.832111in}}%
\pgfpathclose%
\pgfusepath{stroke,fill}%
\end{pgfscope}%
\begin{pgfscope}%
\pgfpathrectangle{\pgfqpoint{5.292946in}{5.272501in}}{\pgfqpoint{2.177280in}{2.201755in}}%
\pgfusepath{clip}%
\pgfsetbuttcap%
\pgfsetroundjoin%
\definecolor{currentfill}{rgb}{0.121569,0.466667,0.705882}%
\pgfsetfillcolor{currentfill}%
\pgfsetlinewidth{0.481800pt}%
\definecolor{currentstroke}{rgb}{1.000000,1.000000,1.000000}%
\pgfsetstrokecolor{currentstroke}%
\pgfsetdash{}{0pt}%
\pgfpathmoveto{\pgfqpoint{5.575125in}{6.164912in}}%
\pgfpathcurveto{\pgfqpoint{5.586176in}{6.164912in}}{\pgfqpoint{5.596775in}{6.169303in}}{\pgfqpoint{5.604588in}{6.177116in}}%
\pgfpathcurveto{\pgfqpoint{5.612402in}{6.184930in}}{\pgfqpoint{5.616792in}{6.195529in}}{\pgfqpoint{5.616792in}{6.206579in}}%
\pgfpathcurveto{\pgfqpoint{5.616792in}{6.217629in}}{\pgfqpoint{5.612402in}{6.228228in}}{\pgfqpoint{5.604588in}{6.236042in}}%
\pgfpathcurveto{\pgfqpoint{5.596775in}{6.243855in}}{\pgfqpoint{5.586176in}{6.248246in}}{\pgfqpoint{5.575125in}{6.248246in}}%
\pgfpathcurveto{\pgfqpoint{5.564075in}{6.248246in}}{\pgfqpoint{5.553476in}{6.243855in}}{\pgfqpoint{5.545663in}{6.236042in}}%
\pgfpathcurveto{\pgfqpoint{5.537849in}{6.228228in}}{\pgfqpoint{5.533459in}{6.217629in}}{\pgfqpoint{5.533459in}{6.206579in}}%
\pgfpathcurveto{\pgfqpoint{5.533459in}{6.195529in}}{\pgfqpoint{5.537849in}{6.184930in}}{\pgfqpoint{5.545663in}{6.177116in}}%
\pgfpathcurveto{\pgfqpoint{5.553476in}{6.169303in}}{\pgfqpoint{5.564075in}{6.164912in}}{\pgfqpoint{5.575125in}{6.164912in}}%
\pgfpathlineto{\pgfqpoint{5.575125in}{6.164912in}}%
\pgfpathclose%
\pgfusepath{stroke,fill}%
\end{pgfscope}%
\begin{pgfscope}%
\pgfpathrectangle{\pgfqpoint{5.292946in}{5.272501in}}{\pgfqpoint{2.177280in}{2.201755in}}%
\pgfusepath{clip}%
\pgfsetbuttcap%
\pgfsetroundjoin%
\definecolor{currentfill}{rgb}{0.121569,0.466667,0.705882}%
\pgfsetfillcolor{currentfill}%
\pgfsetlinewidth{0.481800pt}%
\definecolor{currentstroke}{rgb}{1.000000,1.000000,1.000000}%
\pgfsetstrokecolor{currentstroke}%
\pgfsetdash{}{0pt}%
\pgfpathmoveto{\pgfqpoint{5.632536in}{6.832111in}}%
\pgfpathcurveto{\pgfqpoint{5.643586in}{6.832111in}}{\pgfqpoint{5.654185in}{6.836501in}}{\pgfqpoint{5.661999in}{6.844315in}}%
\pgfpathcurveto{\pgfqpoint{5.669812in}{6.852128in}}{\pgfqpoint{5.674203in}{6.862727in}}{\pgfqpoint{5.674203in}{6.873777in}}%
\pgfpathcurveto{\pgfqpoint{5.674203in}{6.884828in}}{\pgfqpoint{5.669812in}{6.895427in}}{\pgfqpoint{5.661999in}{6.903240in}}%
\pgfpathcurveto{\pgfqpoint{5.654185in}{6.911054in}}{\pgfqpoint{5.643586in}{6.915444in}}{\pgfqpoint{5.632536in}{6.915444in}}%
\pgfpathcurveto{\pgfqpoint{5.621486in}{6.915444in}}{\pgfqpoint{5.610887in}{6.911054in}}{\pgfqpoint{5.603073in}{6.903240in}}%
\pgfpathcurveto{\pgfqpoint{5.595260in}{6.895427in}}{\pgfqpoint{5.590869in}{6.884828in}}{\pgfqpoint{5.590869in}{6.873777in}}%
\pgfpathcurveto{\pgfqpoint{5.590869in}{6.862727in}}{\pgfqpoint{5.595260in}{6.852128in}}{\pgfqpoint{5.603073in}{6.844315in}}%
\pgfpathcurveto{\pgfqpoint{5.610887in}{6.836501in}}{\pgfqpoint{5.621486in}{6.832111in}}{\pgfqpoint{5.632536in}{6.832111in}}%
\pgfpathlineto{\pgfqpoint{5.632536in}{6.832111in}}%
\pgfpathclose%
\pgfusepath{stroke,fill}%
\end{pgfscope}%
\begin{pgfscope}%
\pgfpathrectangle{\pgfqpoint{5.292946in}{5.272501in}}{\pgfqpoint{2.177280in}{2.201755in}}%
\pgfusepath{clip}%
\pgfsetbuttcap%
\pgfsetroundjoin%
\definecolor{currentfill}{rgb}{0.121569,0.466667,0.705882}%
\pgfsetfillcolor{currentfill}%
\pgfsetlinewidth{0.481800pt}%
\definecolor{currentstroke}{rgb}{1.000000,1.000000,1.000000}%
\pgfsetstrokecolor{currentstroke}%
\pgfsetdash{}{0pt}%
\pgfpathmoveto{\pgfqpoint{5.575125in}{6.331712in}}%
\pgfpathcurveto{\pgfqpoint{5.586176in}{6.331712in}}{\pgfqpoint{5.596775in}{6.336102in}}{\pgfqpoint{5.604588in}{6.343916in}}%
\pgfpathcurveto{\pgfqpoint{5.612402in}{6.351729in}}{\pgfqpoint{5.616792in}{6.362328in}}{\pgfqpoint{5.616792in}{6.373379in}}%
\pgfpathcurveto{\pgfqpoint{5.616792in}{6.384429in}}{\pgfqpoint{5.612402in}{6.395028in}}{\pgfqpoint{5.604588in}{6.402841in}}%
\pgfpathcurveto{\pgfqpoint{5.596775in}{6.410655in}}{\pgfqpoint{5.586176in}{6.415045in}}{\pgfqpoint{5.575125in}{6.415045in}}%
\pgfpathcurveto{\pgfqpoint{5.564075in}{6.415045in}}{\pgfqpoint{5.553476in}{6.410655in}}{\pgfqpoint{5.545663in}{6.402841in}}%
\pgfpathcurveto{\pgfqpoint{5.537849in}{6.395028in}}{\pgfqpoint{5.533459in}{6.384429in}}{\pgfqpoint{5.533459in}{6.373379in}}%
\pgfpathcurveto{\pgfqpoint{5.533459in}{6.362328in}}{\pgfqpoint{5.537849in}{6.351729in}}{\pgfqpoint{5.545663in}{6.343916in}}%
\pgfpathcurveto{\pgfqpoint{5.553476in}{6.336102in}}{\pgfqpoint{5.564075in}{6.331712in}}{\pgfqpoint{5.575125in}{6.331712in}}%
\pgfpathlineto{\pgfqpoint{5.575125in}{6.331712in}}%
\pgfpathclose%
\pgfusepath{stroke,fill}%
\end{pgfscope}%
\begin{pgfscope}%
\pgfpathrectangle{\pgfqpoint{5.292946in}{5.272501in}}{\pgfqpoint{2.177280in}{2.201755in}}%
\pgfusepath{clip}%
\pgfsetbuttcap%
\pgfsetroundjoin%
\definecolor{currentfill}{rgb}{0.121569,0.466667,0.705882}%
\pgfsetfillcolor{currentfill}%
\pgfsetlinewidth{0.481800pt}%
\definecolor{currentstroke}{rgb}{1.000000,1.000000,1.000000}%
\pgfsetstrokecolor{currentstroke}%
\pgfsetdash{}{0pt}%
\pgfpathmoveto{\pgfqpoint{5.603831in}{6.748711in}}%
\pgfpathcurveto{\pgfqpoint{5.614881in}{6.748711in}}{\pgfqpoint{5.625480in}{6.753101in}}{\pgfqpoint{5.633294in}{6.760915in}}%
\pgfpathcurveto{\pgfqpoint{5.641107in}{6.768728in}}{\pgfqpoint{5.645497in}{6.779327in}}{\pgfqpoint{5.645497in}{6.790378in}}%
\pgfpathcurveto{\pgfqpoint{5.645497in}{6.801428in}}{\pgfqpoint{5.641107in}{6.812027in}}{\pgfqpoint{5.633294in}{6.819840in}}%
\pgfpathcurveto{\pgfqpoint{5.625480in}{6.827654in}}{\pgfqpoint{5.614881in}{6.832044in}}{\pgfqpoint{5.603831in}{6.832044in}}%
\pgfpathcurveto{\pgfqpoint{5.592781in}{6.832044in}}{\pgfqpoint{5.582182in}{6.827654in}}{\pgfqpoint{5.574368in}{6.819840in}}%
\pgfpathcurveto{\pgfqpoint{5.566554in}{6.812027in}}{\pgfqpoint{5.562164in}{6.801428in}}{\pgfqpoint{5.562164in}{6.790378in}}%
\pgfpathcurveto{\pgfqpoint{5.562164in}{6.779327in}}{\pgfqpoint{5.566554in}{6.768728in}}{\pgfqpoint{5.574368in}{6.760915in}}%
\pgfpathcurveto{\pgfqpoint{5.582182in}{6.753101in}}{\pgfqpoint{5.592781in}{6.748711in}}{\pgfqpoint{5.603831in}{6.748711in}}%
\pgfpathlineto{\pgfqpoint{5.603831in}{6.748711in}}%
\pgfpathclose%
\pgfusepath{stroke,fill}%
\end{pgfscope}%
\begin{pgfscope}%
\pgfpathrectangle{\pgfqpoint{5.292946in}{5.272501in}}{\pgfqpoint{2.177280in}{2.201755in}}%
\pgfusepath{clip}%
\pgfsetbuttcap%
\pgfsetroundjoin%
\definecolor{currentfill}{rgb}{0.121569,0.466667,0.705882}%
\pgfsetfillcolor{currentfill}%
\pgfsetlinewidth{0.481800pt}%
\definecolor{currentstroke}{rgb}{1.000000,1.000000,1.000000}%
\pgfsetstrokecolor{currentstroke}%
\pgfsetdash{}{0pt}%
\pgfpathmoveto{\pgfqpoint{5.575125in}{6.415112in}}%
\pgfpathcurveto{\pgfqpoint{5.586176in}{6.415112in}}{\pgfqpoint{5.596775in}{6.419502in}}{\pgfqpoint{5.604588in}{6.427316in}}%
\pgfpathcurveto{\pgfqpoint{5.612402in}{6.435129in}}{\pgfqpoint{5.616792in}{6.445728in}}{\pgfqpoint{5.616792in}{6.456778in}}%
\pgfpathcurveto{\pgfqpoint{5.616792in}{6.467828in}}{\pgfqpoint{5.612402in}{6.478428in}}{\pgfqpoint{5.604588in}{6.486241in}}%
\pgfpathcurveto{\pgfqpoint{5.596775in}{6.494055in}}{\pgfqpoint{5.586176in}{6.498445in}}{\pgfqpoint{5.575125in}{6.498445in}}%
\pgfpathcurveto{\pgfqpoint{5.564075in}{6.498445in}}{\pgfqpoint{5.553476in}{6.494055in}}{\pgfqpoint{5.545663in}{6.486241in}}%
\pgfpathcurveto{\pgfqpoint{5.537849in}{6.478428in}}{\pgfqpoint{5.533459in}{6.467828in}}{\pgfqpoint{5.533459in}{6.456778in}}%
\pgfpathcurveto{\pgfqpoint{5.533459in}{6.445728in}}{\pgfqpoint{5.537849in}{6.435129in}}{\pgfqpoint{5.545663in}{6.427316in}}%
\pgfpathcurveto{\pgfqpoint{5.553476in}{6.419502in}}{\pgfqpoint{5.564075in}{6.415112in}}{\pgfqpoint{5.575125in}{6.415112in}}%
\pgfpathlineto{\pgfqpoint{5.575125in}{6.415112in}}%
\pgfpathclose%
\pgfusepath{stroke,fill}%
\end{pgfscope}%
\begin{pgfscope}%
\pgfpathrectangle{\pgfqpoint{5.292946in}{5.272501in}}{\pgfqpoint{2.177280in}{2.201755in}}%
\pgfusepath{clip}%
\pgfsetbuttcap%
\pgfsetroundjoin%
\definecolor{currentfill}{rgb}{1.000000,0.498039,0.054902}%
\pgfsetfillcolor{currentfill}%
\pgfsetlinewidth{0.481800pt}%
\definecolor{currentstroke}{rgb}{1.000000,1.000000,1.000000}%
\pgfsetstrokecolor{currentstroke}%
\pgfsetdash{}{0pt}%
\pgfpathmoveto{\pgfqpoint{6.522400in}{6.331712in}}%
\pgfpathcurveto{\pgfqpoint{6.533450in}{6.331712in}}{\pgfqpoint{6.544049in}{6.336102in}}{\pgfqpoint{6.551863in}{6.343916in}}%
\pgfpathcurveto{\pgfqpoint{6.559676in}{6.351729in}}{\pgfqpoint{6.564067in}{6.362328in}}{\pgfqpoint{6.564067in}{6.373379in}}%
\pgfpathcurveto{\pgfqpoint{6.564067in}{6.384429in}}{\pgfqpoint{6.559676in}{6.395028in}}{\pgfqpoint{6.551863in}{6.402841in}}%
\pgfpathcurveto{\pgfqpoint{6.544049in}{6.410655in}}{\pgfqpoint{6.533450in}{6.415045in}}{\pgfqpoint{6.522400in}{6.415045in}}%
\pgfpathcurveto{\pgfqpoint{6.511350in}{6.415045in}}{\pgfqpoint{6.500751in}{6.410655in}}{\pgfqpoint{6.492937in}{6.402841in}}%
\pgfpathcurveto{\pgfqpoint{6.485123in}{6.395028in}}{\pgfqpoint{6.480733in}{6.384429in}}{\pgfqpoint{6.480733in}{6.373379in}}%
\pgfpathcurveto{\pgfqpoint{6.480733in}{6.362328in}}{\pgfqpoint{6.485123in}{6.351729in}}{\pgfqpoint{6.492937in}{6.343916in}}%
\pgfpathcurveto{\pgfqpoint{6.500751in}{6.336102in}}{\pgfqpoint{6.511350in}{6.331712in}}{\pgfqpoint{6.522400in}{6.331712in}}%
\pgfpathlineto{\pgfqpoint{6.522400in}{6.331712in}}%
\pgfpathclose%
\pgfusepath{stroke,fill}%
\end{pgfscope}%
\begin{pgfscope}%
\pgfpathrectangle{\pgfqpoint{5.292946in}{5.272501in}}{\pgfqpoint{2.177280in}{2.201755in}}%
\pgfusepath{clip}%
\pgfsetbuttcap%
\pgfsetroundjoin%
\definecolor{currentfill}{rgb}{1.000000,0.498039,0.054902}%
\pgfsetfillcolor{currentfill}%
\pgfsetlinewidth{0.481800pt}%
\definecolor{currentstroke}{rgb}{1.000000,1.000000,1.000000}%
\pgfsetstrokecolor{currentstroke}%
\pgfsetdash{}{0pt}%
\pgfpathmoveto{\pgfqpoint{6.464989in}{6.331712in}}%
\pgfpathcurveto{\pgfqpoint{6.476039in}{6.331712in}}{\pgfqpoint{6.486638in}{6.336102in}}{\pgfqpoint{6.494452in}{6.343916in}}%
\pgfpathcurveto{\pgfqpoint{6.502266in}{6.351729in}}{\pgfqpoint{6.506656in}{6.362328in}}{\pgfqpoint{6.506656in}{6.373379in}}%
\pgfpathcurveto{\pgfqpoint{6.506656in}{6.384429in}}{\pgfqpoint{6.502266in}{6.395028in}}{\pgfqpoint{6.494452in}{6.402841in}}%
\pgfpathcurveto{\pgfqpoint{6.486638in}{6.410655in}}{\pgfqpoint{6.476039in}{6.415045in}}{\pgfqpoint{6.464989in}{6.415045in}}%
\pgfpathcurveto{\pgfqpoint{6.453939in}{6.415045in}}{\pgfqpoint{6.443340in}{6.410655in}}{\pgfqpoint{6.435527in}{6.402841in}}%
\pgfpathcurveto{\pgfqpoint{6.427713in}{6.395028in}}{\pgfqpoint{6.423323in}{6.384429in}}{\pgfqpoint{6.423323in}{6.373379in}}%
\pgfpathcurveto{\pgfqpoint{6.423323in}{6.362328in}}{\pgfqpoint{6.427713in}{6.351729in}}{\pgfqpoint{6.435527in}{6.343916in}}%
\pgfpathcurveto{\pgfqpoint{6.443340in}{6.336102in}}{\pgfqpoint{6.453939in}{6.331712in}}{\pgfqpoint{6.464989in}{6.331712in}}%
\pgfpathlineto{\pgfqpoint{6.464989in}{6.331712in}}%
\pgfpathclose%
\pgfusepath{stroke,fill}%
\end{pgfscope}%
\begin{pgfscope}%
\pgfpathrectangle{\pgfqpoint{5.292946in}{5.272501in}}{\pgfqpoint{2.177280in}{2.201755in}}%
\pgfusepath{clip}%
\pgfsetbuttcap%
\pgfsetroundjoin%
\definecolor{currentfill}{rgb}{1.000000,0.498039,0.054902}%
\pgfsetfillcolor{currentfill}%
\pgfsetlinewidth{0.481800pt}%
\definecolor{currentstroke}{rgb}{1.000000,1.000000,1.000000}%
\pgfsetstrokecolor{currentstroke}%
\pgfsetdash{}{0pt}%
\pgfpathmoveto{\pgfqpoint{6.579810in}{6.248312in}}%
\pgfpathcurveto{\pgfqpoint{6.590861in}{6.248312in}}{\pgfqpoint{6.601460in}{6.252702in}}{\pgfqpoint{6.609273in}{6.260516in}}%
\pgfpathcurveto{\pgfqpoint{6.617087in}{6.268330in}}{\pgfqpoint{6.621477in}{6.278929in}}{\pgfqpoint{6.621477in}{6.289979in}}%
\pgfpathcurveto{\pgfqpoint{6.621477in}{6.301029in}}{\pgfqpoint{6.617087in}{6.311628in}}{\pgfqpoint{6.609273in}{6.319442in}}%
\pgfpathcurveto{\pgfqpoint{6.601460in}{6.327255in}}{\pgfqpoint{6.590861in}{6.331645in}}{\pgfqpoint{6.579810in}{6.331645in}}%
\pgfpathcurveto{\pgfqpoint{6.568760in}{6.331645in}}{\pgfqpoint{6.558161in}{6.327255in}}{\pgfqpoint{6.550348in}{6.319442in}}%
\pgfpathcurveto{\pgfqpoint{6.542534in}{6.311628in}}{\pgfqpoint{6.538144in}{6.301029in}}{\pgfqpoint{6.538144in}{6.289979in}}%
\pgfpathcurveto{\pgfqpoint{6.538144in}{6.278929in}}{\pgfqpoint{6.542534in}{6.268330in}}{\pgfqpoint{6.550348in}{6.260516in}}%
\pgfpathcurveto{\pgfqpoint{6.558161in}{6.252702in}}{\pgfqpoint{6.568760in}{6.248312in}}{\pgfqpoint{6.579810in}{6.248312in}}%
\pgfpathlineto{\pgfqpoint{6.579810in}{6.248312in}}%
\pgfpathclose%
\pgfusepath{stroke,fill}%
\end{pgfscope}%
\begin{pgfscope}%
\pgfpathrectangle{\pgfqpoint{5.292946in}{5.272501in}}{\pgfqpoint{2.177280in}{2.201755in}}%
\pgfusepath{clip}%
\pgfsetbuttcap%
\pgfsetroundjoin%
\definecolor{currentfill}{rgb}{1.000000,0.498039,0.054902}%
\pgfsetfillcolor{currentfill}%
\pgfsetlinewidth{0.481800pt}%
\definecolor{currentstroke}{rgb}{1.000000,1.000000,1.000000}%
\pgfsetstrokecolor{currentstroke}%
\pgfsetdash{}{0pt}%
\pgfpathmoveto{\pgfqpoint{6.321463in}{5.581114in}}%
\pgfpathcurveto{\pgfqpoint{6.332513in}{5.581114in}}{\pgfqpoint{6.343112in}{5.585504in}}{\pgfqpoint{6.350926in}{5.593317in}}%
\pgfpathcurveto{\pgfqpoint{6.358739in}{5.601131in}}{\pgfqpoint{6.363130in}{5.611730in}}{\pgfqpoint{6.363130in}{5.622780in}}%
\pgfpathcurveto{\pgfqpoint{6.363130in}{5.633830in}}{\pgfqpoint{6.358739in}{5.644429in}}{\pgfqpoint{6.350926in}{5.652243in}}%
\pgfpathcurveto{\pgfqpoint{6.343112in}{5.660057in}}{\pgfqpoint{6.332513in}{5.664447in}}{\pgfqpoint{6.321463in}{5.664447in}}%
\pgfpathcurveto{\pgfqpoint{6.310413in}{5.664447in}}{\pgfqpoint{6.299814in}{5.660057in}}{\pgfqpoint{6.292000in}{5.652243in}}%
\pgfpathcurveto{\pgfqpoint{6.284186in}{5.644429in}}{\pgfqpoint{6.279796in}{5.633830in}}{\pgfqpoint{6.279796in}{5.622780in}}%
\pgfpathcurveto{\pgfqpoint{6.279796in}{5.611730in}}{\pgfqpoint{6.284186in}{5.601131in}}{\pgfqpoint{6.292000in}{5.593317in}}%
\pgfpathcurveto{\pgfqpoint{6.299814in}{5.585504in}}{\pgfqpoint{6.310413in}{5.581114in}}{\pgfqpoint{6.321463in}{5.581114in}}%
\pgfpathlineto{\pgfqpoint{6.321463in}{5.581114in}}%
\pgfpathclose%
\pgfusepath{stroke,fill}%
\end{pgfscope}%
\begin{pgfscope}%
\pgfpathrectangle{\pgfqpoint{5.292946in}{5.272501in}}{\pgfqpoint{2.177280in}{2.201755in}}%
\pgfusepath{clip}%
\pgfsetbuttcap%
\pgfsetroundjoin%
\definecolor{currentfill}{rgb}{1.000000,0.498039,0.054902}%
\pgfsetfillcolor{currentfill}%
\pgfsetlinewidth{0.481800pt}%
\definecolor{currentstroke}{rgb}{1.000000,1.000000,1.000000}%
\pgfsetstrokecolor{currentstroke}%
\pgfsetdash{}{0pt}%
\pgfpathmoveto{\pgfqpoint{6.493695in}{5.998113in}}%
\pgfpathcurveto{\pgfqpoint{6.504745in}{5.998113in}}{\pgfqpoint{6.515344in}{6.002503in}}{\pgfqpoint{6.523157in}{6.010317in}}%
\pgfpathcurveto{\pgfqpoint{6.530971in}{6.018130in}}{\pgfqpoint{6.535361in}{6.028729in}}{\pgfqpoint{6.535361in}{6.039779in}}%
\pgfpathcurveto{\pgfqpoint{6.535361in}{6.050829in}}{\pgfqpoint{6.530971in}{6.061428in}}{\pgfqpoint{6.523157in}{6.069242in}}%
\pgfpathcurveto{\pgfqpoint{6.515344in}{6.077056in}}{\pgfqpoint{6.504745in}{6.081446in}}{\pgfqpoint{6.493695in}{6.081446in}}%
\pgfpathcurveto{\pgfqpoint{6.482644in}{6.081446in}}{\pgfqpoint{6.472045in}{6.077056in}}{\pgfqpoint{6.464232in}{6.069242in}}%
\pgfpathcurveto{\pgfqpoint{6.456418in}{6.061428in}}{\pgfqpoint{6.452028in}{6.050829in}}{\pgfqpoint{6.452028in}{6.039779in}}%
\pgfpathcurveto{\pgfqpoint{6.452028in}{6.028729in}}{\pgfqpoint{6.456418in}{6.018130in}}{\pgfqpoint{6.464232in}{6.010317in}}%
\pgfpathcurveto{\pgfqpoint{6.472045in}{6.002503in}}{\pgfqpoint{6.482644in}{5.998113in}}{\pgfqpoint{6.493695in}{5.998113in}}%
\pgfpathlineto{\pgfqpoint{6.493695in}{5.998113in}}%
\pgfpathclose%
\pgfusepath{stroke,fill}%
\end{pgfscope}%
\begin{pgfscope}%
\pgfpathrectangle{\pgfqpoint{5.292946in}{5.272501in}}{\pgfqpoint{2.177280in}{2.201755in}}%
\pgfusepath{clip}%
\pgfsetbuttcap%
\pgfsetroundjoin%
\definecolor{currentfill}{rgb}{1.000000,0.498039,0.054902}%
\pgfsetfillcolor{currentfill}%
\pgfsetlinewidth{0.481800pt}%
\definecolor{currentstroke}{rgb}{1.000000,1.000000,1.000000}%
\pgfsetstrokecolor{currentstroke}%
\pgfsetdash{}{0pt}%
\pgfpathmoveto{\pgfqpoint{6.464989in}{5.998113in}}%
\pgfpathcurveto{\pgfqpoint{6.476039in}{5.998113in}}{\pgfqpoint{6.486638in}{6.002503in}}{\pgfqpoint{6.494452in}{6.010317in}}%
\pgfpathcurveto{\pgfqpoint{6.502266in}{6.018130in}}{\pgfqpoint{6.506656in}{6.028729in}}{\pgfqpoint{6.506656in}{6.039779in}}%
\pgfpathcurveto{\pgfqpoint{6.506656in}{6.050829in}}{\pgfqpoint{6.502266in}{6.061428in}}{\pgfqpoint{6.494452in}{6.069242in}}%
\pgfpathcurveto{\pgfqpoint{6.486638in}{6.077056in}}{\pgfqpoint{6.476039in}{6.081446in}}{\pgfqpoint{6.464989in}{6.081446in}}%
\pgfpathcurveto{\pgfqpoint{6.453939in}{6.081446in}}{\pgfqpoint{6.443340in}{6.077056in}}{\pgfqpoint{6.435527in}{6.069242in}}%
\pgfpathcurveto{\pgfqpoint{6.427713in}{6.061428in}}{\pgfqpoint{6.423323in}{6.050829in}}{\pgfqpoint{6.423323in}{6.039779in}}%
\pgfpathcurveto{\pgfqpoint{6.423323in}{6.028729in}}{\pgfqpoint{6.427713in}{6.018130in}}{\pgfqpoint{6.435527in}{6.010317in}}%
\pgfpathcurveto{\pgfqpoint{6.443340in}{6.002503in}}{\pgfqpoint{6.453939in}{5.998113in}}{\pgfqpoint{6.464989in}{5.998113in}}%
\pgfpathlineto{\pgfqpoint{6.464989in}{5.998113in}}%
\pgfpathclose%
\pgfusepath{stroke,fill}%
\end{pgfscope}%
\begin{pgfscope}%
\pgfpathrectangle{\pgfqpoint{5.292946in}{5.272501in}}{\pgfqpoint{2.177280in}{2.201755in}}%
\pgfusepath{clip}%
\pgfsetbuttcap%
\pgfsetroundjoin%
\definecolor{currentfill}{rgb}{1.000000,0.498039,0.054902}%
\pgfsetfillcolor{currentfill}%
\pgfsetlinewidth{0.481800pt}%
\definecolor{currentstroke}{rgb}{1.000000,1.000000,1.000000}%
\pgfsetstrokecolor{currentstroke}%
\pgfsetdash{}{0pt}%
\pgfpathmoveto{\pgfqpoint{6.522400in}{6.415112in}}%
\pgfpathcurveto{\pgfqpoint{6.533450in}{6.415112in}}{\pgfqpoint{6.544049in}{6.419502in}}{\pgfqpoint{6.551863in}{6.427316in}}%
\pgfpathcurveto{\pgfqpoint{6.559676in}{6.435129in}}{\pgfqpoint{6.564067in}{6.445728in}}{\pgfqpoint{6.564067in}{6.456778in}}%
\pgfpathcurveto{\pgfqpoint{6.564067in}{6.467828in}}{\pgfqpoint{6.559676in}{6.478428in}}{\pgfqpoint{6.551863in}{6.486241in}}%
\pgfpathcurveto{\pgfqpoint{6.544049in}{6.494055in}}{\pgfqpoint{6.533450in}{6.498445in}}{\pgfqpoint{6.522400in}{6.498445in}}%
\pgfpathcurveto{\pgfqpoint{6.511350in}{6.498445in}}{\pgfqpoint{6.500751in}{6.494055in}}{\pgfqpoint{6.492937in}{6.486241in}}%
\pgfpathcurveto{\pgfqpoint{6.485123in}{6.478428in}}{\pgfqpoint{6.480733in}{6.467828in}}{\pgfqpoint{6.480733in}{6.456778in}}%
\pgfpathcurveto{\pgfqpoint{6.480733in}{6.445728in}}{\pgfqpoint{6.485123in}{6.435129in}}{\pgfqpoint{6.492937in}{6.427316in}}%
\pgfpathcurveto{\pgfqpoint{6.500751in}{6.419502in}}{\pgfqpoint{6.511350in}{6.415112in}}{\pgfqpoint{6.522400in}{6.415112in}}%
\pgfpathlineto{\pgfqpoint{6.522400in}{6.415112in}}%
\pgfpathclose%
\pgfusepath{stroke,fill}%
\end{pgfscope}%
\begin{pgfscope}%
\pgfpathrectangle{\pgfqpoint{5.292946in}{5.272501in}}{\pgfqpoint{2.177280in}{2.201755in}}%
\pgfusepath{clip}%
\pgfsetbuttcap%
\pgfsetroundjoin%
\definecolor{currentfill}{rgb}{1.000000,0.498039,0.054902}%
\pgfsetfillcolor{currentfill}%
\pgfsetlinewidth{0.481800pt}%
\definecolor{currentstroke}{rgb}{1.000000,1.000000,1.000000}%
\pgfsetstrokecolor{currentstroke}%
\pgfsetdash{}{0pt}%
\pgfpathmoveto{\pgfqpoint{6.120526in}{5.664513in}}%
\pgfpathcurveto{\pgfqpoint{6.131576in}{5.664513in}}{\pgfqpoint{6.142175in}{5.668904in}}{\pgfqpoint{6.149989in}{5.676717in}}%
\pgfpathcurveto{\pgfqpoint{6.157802in}{5.684531in}}{\pgfqpoint{6.162193in}{5.695130in}}{\pgfqpoint{6.162193in}{5.706180in}}%
\pgfpathcurveto{\pgfqpoint{6.162193in}{5.717230in}}{\pgfqpoint{6.157802in}{5.727829in}}{\pgfqpoint{6.149989in}{5.735643in}}%
\pgfpathcurveto{\pgfqpoint{6.142175in}{5.743456in}}{\pgfqpoint{6.131576in}{5.747847in}}{\pgfqpoint{6.120526in}{5.747847in}}%
\pgfpathcurveto{\pgfqpoint{6.109476in}{5.747847in}}{\pgfqpoint{6.098877in}{5.743456in}}{\pgfqpoint{6.091063in}{5.735643in}}%
\pgfpathcurveto{\pgfqpoint{6.083249in}{5.727829in}}{\pgfqpoint{6.078859in}{5.717230in}}{\pgfqpoint{6.078859in}{5.706180in}}%
\pgfpathcurveto{\pgfqpoint{6.078859in}{5.695130in}}{\pgfqpoint{6.083249in}{5.684531in}}{\pgfqpoint{6.091063in}{5.676717in}}%
\pgfpathcurveto{\pgfqpoint{6.098877in}{5.668904in}}{\pgfqpoint{6.109476in}{5.664513in}}{\pgfqpoint{6.120526in}{5.664513in}}%
\pgfpathlineto{\pgfqpoint{6.120526in}{5.664513in}}%
\pgfpathclose%
\pgfusepath{stroke,fill}%
\end{pgfscope}%
\begin{pgfscope}%
\pgfpathrectangle{\pgfqpoint{5.292946in}{5.272501in}}{\pgfqpoint{2.177280in}{2.201755in}}%
\pgfusepath{clip}%
\pgfsetbuttcap%
\pgfsetroundjoin%
\definecolor{currentfill}{rgb}{1.000000,0.498039,0.054902}%
\pgfsetfillcolor{currentfill}%
\pgfsetlinewidth{0.481800pt}%
\definecolor{currentstroke}{rgb}{1.000000,1.000000,1.000000}%
\pgfsetstrokecolor{currentstroke}%
\pgfsetdash{}{0pt}%
\pgfpathmoveto{\pgfqpoint{6.493695in}{6.081512in}}%
\pgfpathcurveto{\pgfqpoint{6.504745in}{6.081512in}}{\pgfqpoint{6.515344in}{6.085903in}}{\pgfqpoint{6.523157in}{6.093716in}}%
\pgfpathcurveto{\pgfqpoint{6.530971in}{6.101530in}}{\pgfqpoint{6.535361in}{6.112129in}}{\pgfqpoint{6.535361in}{6.123179in}}%
\pgfpathcurveto{\pgfqpoint{6.535361in}{6.134229in}}{\pgfqpoint{6.530971in}{6.144828in}}{\pgfqpoint{6.523157in}{6.152642in}}%
\pgfpathcurveto{\pgfqpoint{6.515344in}{6.160456in}}{\pgfqpoint{6.504745in}{6.164846in}}{\pgfqpoint{6.493695in}{6.164846in}}%
\pgfpathcurveto{\pgfqpoint{6.482644in}{6.164846in}}{\pgfqpoint{6.472045in}{6.160456in}}{\pgfqpoint{6.464232in}{6.152642in}}%
\pgfpathcurveto{\pgfqpoint{6.456418in}{6.144828in}}{\pgfqpoint{6.452028in}{6.134229in}}{\pgfqpoint{6.452028in}{6.123179in}}%
\pgfpathcurveto{\pgfqpoint{6.452028in}{6.112129in}}{\pgfqpoint{6.456418in}{6.101530in}}{\pgfqpoint{6.464232in}{6.093716in}}%
\pgfpathcurveto{\pgfqpoint{6.472045in}{6.085903in}}{\pgfqpoint{6.482644in}{6.081512in}}{\pgfqpoint{6.493695in}{6.081512in}}%
\pgfpathlineto{\pgfqpoint{6.493695in}{6.081512in}}%
\pgfpathclose%
\pgfusepath{stroke,fill}%
\end{pgfscope}%
\begin{pgfscope}%
\pgfpathrectangle{\pgfqpoint{5.292946in}{5.272501in}}{\pgfqpoint{2.177280in}{2.201755in}}%
\pgfusepath{clip}%
\pgfsetbuttcap%
\pgfsetroundjoin%
\definecolor{currentfill}{rgb}{1.000000,0.498039,0.054902}%
\pgfsetfillcolor{currentfill}%
\pgfsetlinewidth{0.481800pt}%
\definecolor{currentstroke}{rgb}{1.000000,1.000000,1.000000}%
\pgfsetstrokecolor{currentstroke}%
\pgfsetdash{}{0pt}%
\pgfpathmoveto{\pgfqpoint{6.292758in}{5.914713in}}%
\pgfpathcurveto{\pgfqpoint{6.303808in}{5.914713in}}{\pgfqpoint{6.314407in}{5.919103in}}{\pgfqpoint{6.322220in}{5.926917in}}%
\pgfpathcurveto{\pgfqpoint{6.330034in}{5.934730in}}{\pgfqpoint{6.334424in}{5.945329in}}{\pgfqpoint{6.334424in}{5.956379in}}%
\pgfpathcurveto{\pgfqpoint{6.334424in}{5.967430in}}{\pgfqpoint{6.330034in}{5.978029in}}{\pgfqpoint{6.322220in}{5.985842in}}%
\pgfpathcurveto{\pgfqpoint{6.314407in}{5.993656in}}{\pgfqpoint{6.303808in}{5.998046in}}{\pgfqpoint{6.292758in}{5.998046in}}%
\pgfpathcurveto{\pgfqpoint{6.281707in}{5.998046in}}{\pgfqpoint{6.271108in}{5.993656in}}{\pgfqpoint{6.263295in}{5.985842in}}%
\pgfpathcurveto{\pgfqpoint{6.255481in}{5.978029in}}{\pgfqpoint{6.251091in}{5.967430in}}{\pgfqpoint{6.251091in}{5.956379in}}%
\pgfpathcurveto{\pgfqpoint{6.251091in}{5.945329in}}{\pgfqpoint{6.255481in}{5.934730in}}{\pgfqpoint{6.263295in}{5.926917in}}%
\pgfpathcurveto{\pgfqpoint{6.271108in}{5.919103in}}{\pgfqpoint{6.281707in}{5.914713in}}{\pgfqpoint{6.292758in}{5.914713in}}%
\pgfpathlineto{\pgfqpoint{6.292758in}{5.914713in}}%
\pgfpathclose%
\pgfusepath{stroke,fill}%
\end{pgfscope}%
\begin{pgfscope}%
\pgfpathrectangle{\pgfqpoint{5.292946in}{5.272501in}}{\pgfqpoint{2.177280in}{2.201755in}}%
\pgfusepath{clip}%
\pgfsetbuttcap%
\pgfsetroundjoin%
\definecolor{currentfill}{rgb}{1.000000,0.498039,0.054902}%
\pgfsetfillcolor{currentfill}%
\pgfsetlinewidth{0.481800pt}%
\definecolor{currentstroke}{rgb}{1.000000,1.000000,1.000000}%
\pgfsetstrokecolor{currentstroke}%
\pgfsetdash{}{0pt}%
\pgfpathmoveto{\pgfqpoint{6.177936in}{5.330914in}}%
\pgfpathcurveto{\pgfqpoint{6.188987in}{5.330914in}}{\pgfqpoint{6.199586in}{5.335304in}}{\pgfqpoint{6.207399in}{5.343118in}}%
\pgfpathcurveto{\pgfqpoint{6.215213in}{5.350932in}}{\pgfqpoint{6.219603in}{5.361531in}}{\pgfqpoint{6.219603in}{5.372581in}}%
\pgfpathcurveto{\pgfqpoint{6.219603in}{5.383631in}}{\pgfqpoint{6.215213in}{5.394230in}}{\pgfqpoint{6.207399in}{5.402044in}}%
\pgfpathcurveto{\pgfqpoint{6.199586in}{5.409857in}}{\pgfqpoint{6.188987in}{5.414247in}}{\pgfqpoint{6.177936in}{5.414247in}}%
\pgfpathcurveto{\pgfqpoint{6.166886in}{5.414247in}}{\pgfqpoint{6.156287in}{5.409857in}}{\pgfqpoint{6.148474in}{5.402044in}}%
\pgfpathcurveto{\pgfqpoint{6.140660in}{5.394230in}}{\pgfqpoint{6.136270in}{5.383631in}}{\pgfqpoint{6.136270in}{5.372581in}}%
\pgfpathcurveto{\pgfqpoint{6.136270in}{5.361531in}}{\pgfqpoint{6.140660in}{5.350932in}}{\pgfqpoint{6.148474in}{5.343118in}}%
\pgfpathcurveto{\pgfqpoint{6.156287in}{5.335304in}}{\pgfqpoint{6.166886in}{5.330914in}}{\pgfqpoint{6.177936in}{5.330914in}}%
\pgfpathlineto{\pgfqpoint{6.177936in}{5.330914in}}%
\pgfpathclose%
\pgfusepath{stroke,fill}%
\end{pgfscope}%
\begin{pgfscope}%
\pgfpathrectangle{\pgfqpoint{5.292946in}{5.272501in}}{\pgfqpoint{2.177280in}{2.201755in}}%
\pgfusepath{clip}%
\pgfsetbuttcap%
\pgfsetroundjoin%
\definecolor{currentfill}{rgb}{1.000000,0.498039,0.054902}%
\pgfsetfillcolor{currentfill}%
\pgfsetlinewidth{0.481800pt}%
\definecolor{currentstroke}{rgb}{1.000000,1.000000,1.000000}%
\pgfsetstrokecolor{currentstroke}%
\pgfsetdash{}{0pt}%
\pgfpathmoveto{\pgfqpoint{6.378873in}{6.164912in}}%
\pgfpathcurveto{\pgfqpoint{6.389924in}{6.164912in}}{\pgfqpoint{6.400523in}{6.169303in}}{\pgfqpoint{6.408336in}{6.177116in}}%
\pgfpathcurveto{\pgfqpoint{6.416150in}{6.184930in}}{\pgfqpoint{6.420540in}{6.195529in}}{\pgfqpoint{6.420540in}{6.206579in}}%
\pgfpathcurveto{\pgfqpoint{6.420540in}{6.217629in}}{\pgfqpoint{6.416150in}{6.228228in}}{\pgfqpoint{6.408336in}{6.236042in}}%
\pgfpathcurveto{\pgfqpoint{6.400523in}{6.243855in}}{\pgfqpoint{6.389924in}{6.248246in}}{\pgfqpoint{6.378873in}{6.248246in}}%
\pgfpathcurveto{\pgfqpoint{6.367823in}{6.248246in}}{\pgfqpoint{6.357224in}{6.243855in}}{\pgfqpoint{6.349411in}{6.236042in}}%
\pgfpathcurveto{\pgfqpoint{6.341597in}{6.228228in}}{\pgfqpoint{6.337207in}{6.217629in}}{\pgfqpoint{6.337207in}{6.206579in}}%
\pgfpathcurveto{\pgfqpoint{6.337207in}{6.195529in}}{\pgfqpoint{6.341597in}{6.184930in}}{\pgfqpoint{6.349411in}{6.177116in}}%
\pgfpathcurveto{\pgfqpoint{6.357224in}{6.169303in}}{\pgfqpoint{6.367823in}{6.164912in}}{\pgfqpoint{6.378873in}{6.164912in}}%
\pgfpathlineto{\pgfqpoint{6.378873in}{6.164912in}}%
\pgfpathclose%
\pgfusepath{stroke,fill}%
\end{pgfscope}%
\begin{pgfscope}%
\pgfpathrectangle{\pgfqpoint{5.292946in}{5.272501in}}{\pgfqpoint{2.177280in}{2.201755in}}%
\pgfusepath{clip}%
\pgfsetbuttcap%
\pgfsetroundjoin%
\definecolor{currentfill}{rgb}{1.000000,0.498039,0.054902}%
\pgfsetfillcolor{currentfill}%
\pgfsetlinewidth{0.481800pt}%
\definecolor{currentstroke}{rgb}{1.000000,1.000000,1.000000}%
\pgfsetstrokecolor{currentstroke}%
\pgfsetdash{}{0pt}%
\pgfpathmoveto{\pgfqpoint{6.321463in}{5.497714in}}%
\pgfpathcurveto{\pgfqpoint{6.332513in}{5.497714in}}{\pgfqpoint{6.343112in}{5.502104in}}{\pgfqpoint{6.350926in}{5.509918in}}%
\pgfpathcurveto{\pgfqpoint{6.358739in}{5.517731in}}{\pgfqpoint{6.363130in}{5.528330in}}{\pgfqpoint{6.363130in}{5.539380in}}%
\pgfpathcurveto{\pgfqpoint{6.363130in}{5.550431in}}{\pgfqpoint{6.358739in}{5.561030in}}{\pgfqpoint{6.350926in}{5.568843in}}%
\pgfpathcurveto{\pgfqpoint{6.343112in}{5.576657in}}{\pgfqpoint{6.332513in}{5.581047in}}{\pgfqpoint{6.321463in}{5.581047in}}%
\pgfpathcurveto{\pgfqpoint{6.310413in}{5.581047in}}{\pgfqpoint{6.299814in}{5.576657in}}{\pgfqpoint{6.292000in}{5.568843in}}%
\pgfpathcurveto{\pgfqpoint{6.284186in}{5.561030in}}{\pgfqpoint{6.279796in}{5.550431in}}{\pgfqpoint{6.279796in}{5.539380in}}%
\pgfpathcurveto{\pgfqpoint{6.279796in}{5.528330in}}{\pgfqpoint{6.284186in}{5.517731in}}{\pgfqpoint{6.292000in}{5.509918in}}%
\pgfpathcurveto{\pgfqpoint{6.299814in}{5.502104in}}{\pgfqpoint{6.310413in}{5.497714in}}{\pgfqpoint{6.321463in}{5.497714in}}%
\pgfpathlineto{\pgfqpoint{6.321463in}{5.497714in}}%
\pgfpathclose%
\pgfusepath{stroke,fill}%
\end{pgfscope}%
\begin{pgfscope}%
\pgfpathrectangle{\pgfqpoint{5.292946in}{5.272501in}}{\pgfqpoint{2.177280in}{2.201755in}}%
\pgfusepath{clip}%
\pgfsetbuttcap%
\pgfsetroundjoin%
\definecolor{currentfill}{rgb}{1.000000,0.498039,0.054902}%
\pgfsetfillcolor{currentfill}%
\pgfsetlinewidth{0.481800pt}%
\definecolor{currentstroke}{rgb}{1.000000,1.000000,1.000000}%
\pgfsetstrokecolor{currentstroke}%
\pgfsetdash{}{0pt}%
\pgfpathmoveto{\pgfqpoint{6.522400in}{6.081512in}}%
\pgfpathcurveto{\pgfqpoint{6.533450in}{6.081512in}}{\pgfqpoint{6.544049in}{6.085903in}}{\pgfqpoint{6.551863in}{6.093716in}}%
\pgfpathcurveto{\pgfqpoint{6.559676in}{6.101530in}}{\pgfqpoint{6.564067in}{6.112129in}}{\pgfqpoint{6.564067in}{6.123179in}}%
\pgfpathcurveto{\pgfqpoint{6.564067in}{6.134229in}}{\pgfqpoint{6.559676in}{6.144828in}}{\pgfqpoint{6.551863in}{6.152642in}}%
\pgfpathcurveto{\pgfqpoint{6.544049in}{6.160456in}}{\pgfqpoint{6.533450in}{6.164846in}}{\pgfqpoint{6.522400in}{6.164846in}}%
\pgfpathcurveto{\pgfqpoint{6.511350in}{6.164846in}}{\pgfqpoint{6.500751in}{6.160456in}}{\pgfqpoint{6.492937in}{6.152642in}}%
\pgfpathcurveto{\pgfqpoint{6.485123in}{6.144828in}}{\pgfqpoint{6.480733in}{6.134229in}}{\pgfqpoint{6.480733in}{6.123179in}}%
\pgfpathcurveto{\pgfqpoint{6.480733in}{6.112129in}}{\pgfqpoint{6.485123in}{6.101530in}}{\pgfqpoint{6.492937in}{6.093716in}}%
\pgfpathcurveto{\pgfqpoint{6.500751in}{6.085903in}}{\pgfqpoint{6.511350in}{6.081512in}}{\pgfqpoint{6.522400in}{6.081512in}}%
\pgfpathlineto{\pgfqpoint{6.522400in}{6.081512in}}%
\pgfpathclose%
\pgfusepath{stroke,fill}%
\end{pgfscope}%
\begin{pgfscope}%
\pgfpathrectangle{\pgfqpoint{5.292946in}{5.272501in}}{\pgfqpoint{2.177280in}{2.201755in}}%
\pgfusepath{clip}%
\pgfsetbuttcap%
\pgfsetroundjoin%
\definecolor{currentfill}{rgb}{1.000000,0.498039,0.054902}%
\pgfsetfillcolor{currentfill}%
\pgfsetlinewidth{0.481800pt}%
\definecolor{currentstroke}{rgb}{1.000000,1.000000,1.000000}%
\pgfsetstrokecolor{currentstroke}%
\pgfsetdash{}{0pt}%
\pgfpathmoveto{\pgfqpoint{6.206642in}{6.081512in}}%
\pgfpathcurveto{\pgfqpoint{6.217692in}{6.081512in}}{\pgfqpoint{6.228291in}{6.085903in}}{\pgfqpoint{6.236104in}{6.093716in}}%
\pgfpathcurveto{\pgfqpoint{6.243918in}{6.101530in}}{\pgfqpoint{6.248308in}{6.112129in}}{\pgfqpoint{6.248308in}{6.123179in}}%
\pgfpathcurveto{\pgfqpoint{6.248308in}{6.134229in}}{\pgfqpoint{6.243918in}{6.144828in}}{\pgfqpoint{6.236104in}{6.152642in}}%
\pgfpathcurveto{\pgfqpoint{6.228291in}{6.160456in}}{\pgfqpoint{6.217692in}{6.164846in}}{\pgfqpoint{6.206642in}{6.164846in}}%
\pgfpathcurveto{\pgfqpoint{6.195592in}{6.164846in}}{\pgfqpoint{6.184993in}{6.160456in}}{\pgfqpoint{6.177179in}{6.152642in}}%
\pgfpathcurveto{\pgfqpoint{6.169365in}{6.144828in}}{\pgfqpoint{6.164975in}{6.134229in}}{\pgfqpoint{6.164975in}{6.123179in}}%
\pgfpathcurveto{\pgfqpoint{6.164975in}{6.112129in}}{\pgfqpoint{6.169365in}{6.101530in}}{\pgfqpoint{6.177179in}{6.093716in}}%
\pgfpathcurveto{\pgfqpoint{6.184993in}{6.085903in}}{\pgfqpoint{6.195592in}{6.081512in}}{\pgfqpoint{6.206642in}{6.081512in}}%
\pgfpathlineto{\pgfqpoint{6.206642in}{6.081512in}}%
\pgfpathclose%
\pgfusepath{stroke,fill}%
\end{pgfscope}%
\begin{pgfscope}%
\pgfpathrectangle{\pgfqpoint{5.292946in}{5.272501in}}{\pgfqpoint{2.177280in}{2.201755in}}%
\pgfusepath{clip}%
\pgfsetbuttcap%
\pgfsetroundjoin%
\definecolor{currentfill}{rgb}{1.000000,0.498039,0.054902}%
\pgfsetfillcolor{currentfill}%
\pgfsetlinewidth{0.481800pt}%
\definecolor{currentstroke}{rgb}{1.000000,1.000000,1.000000}%
\pgfsetstrokecolor{currentstroke}%
\pgfsetdash{}{0pt}%
\pgfpathmoveto{\pgfqpoint{6.436284in}{6.248312in}}%
\pgfpathcurveto{\pgfqpoint{6.447334in}{6.248312in}}{\pgfqpoint{6.457933in}{6.252702in}}{\pgfqpoint{6.465747in}{6.260516in}}%
\pgfpathcurveto{\pgfqpoint{6.473560in}{6.268330in}}{\pgfqpoint{6.477951in}{6.278929in}}{\pgfqpoint{6.477951in}{6.289979in}}%
\pgfpathcurveto{\pgfqpoint{6.477951in}{6.301029in}}{\pgfqpoint{6.473560in}{6.311628in}}{\pgfqpoint{6.465747in}{6.319442in}}%
\pgfpathcurveto{\pgfqpoint{6.457933in}{6.327255in}}{\pgfqpoint{6.447334in}{6.331645in}}{\pgfqpoint{6.436284in}{6.331645in}}%
\pgfpathcurveto{\pgfqpoint{6.425234in}{6.331645in}}{\pgfqpoint{6.414635in}{6.327255in}}{\pgfqpoint{6.406821in}{6.319442in}}%
\pgfpathcurveto{\pgfqpoint{6.399008in}{6.311628in}}{\pgfqpoint{6.394617in}{6.301029in}}{\pgfqpoint{6.394617in}{6.289979in}}%
\pgfpathcurveto{\pgfqpoint{6.394617in}{6.278929in}}{\pgfqpoint{6.399008in}{6.268330in}}{\pgfqpoint{6.406821in}{6.260516in}}%
\pgfpathcurveto{\pgfqpoint{6.414635in}{6.252702in}}{\pgfqpoint{6.425234in}{6.248312in}}{\pgfqpoint{6.436284in}{6.248312in}}%
\pgfpathlineto{\pgfqpoint{6.436284in}{6.248312in}}%
\pgfpathclose%
\pgfusepath{stroke,fill}%
\end{pgfscope}%
\begin{pgfscope}%
\pgfpathrectangle{\pgfqpoint{5.292946in}{5.272501in}}{\pgfqpoint{2.177280in}{2.201755in}}%
\pgfusepath{clip}%
\pgfsetbuttcap%
\pgfsetroundjoin%
\definecolor{currentfill}{rgb}{1.000000,0.498039,0.054902}%
\pgfsetfillcolor{currentfill}%
\pgfsetlinewidth{0.481800pt}%
\definecolor{currentstroke}{rgb}{1.000000,1.000000,1.000000}%
\pgfsetstrokecolor{currentstroke}%
\pgfsetdash{}{0pt}%
\pgfpathmoveto{\pgfqpoint{6.464989in}{6.164912in}}%
\pgfpathcurveto{\pgfqpoint{6.476039in}{6.164912in}}{\pgfqpoint{6.486638in}{6.169303in}}{\pgfqpoint{6.494452in}{6.177116in}}%
\pgfpathcurveto{\pgfqpoint{6.502266in}{6.184930in}}{\pgfqpoint{6.506656in}{6.195529in}}{\pgfqpoint{6.506656in}{6.206579in}}%
\pgfpathcurveto{\pgfqpoint{6.506656in}{6.217629in}}{\pgfqpoint{6.502266in}{6.228228in}}{\pgfqpoint{6.494452in}{6.236042in}}%
\pgfpathcurveto{\pgfqpoint{6.486638in}{6.243855in}}{\pgfqpoint{6.476039in}{6.248246in}}{\pgfqpoint{6.464989in}{6.248246in}}%
\pgfpathcurveto{\pgfqpoint{6.453939in}{6.248246in}}{\pgfqpoint{6.443340in}{6.243855in}}{\pgfqpoint{6.435527in}{6.236042in}}%
\pgfpathcurveto{\pgfqpoint{6.427713in}{6.228228in}}{\pgfqpoint{6.423323in}{6.217629in}}{\pgfqpoint{6.423323in}{6.206579in}}%
\pgfpathcurveto{\pgfqpoint{6.423323in}{6.195529in}}{\pgfqpoint{6.427713in}{6.184930in}}{\pgfqpoint{6.435527in}{6.177116in}}%
\pgfpathcurveto{\pgfqpoint{6.443340in}{6.169303in}}{\pgfqpoint{6.453939in}{6.164912in}}{\pgfqpoint{6.464989in}{6.164912in}}%
\pgfpathlineto{\pgfqpoint{6.464989in}{6.164912in}}%
\pgfpathclose%
\pgfusepath{stroke,fill}%
\end{pgfscope}%
\begin{pgfscope}%
\pgfpathrectangle{\pgfqpoint{5.292946in}{5.272501in}}{\pgfqpoint{2.177280in}{2.201755in}}%
\pgfusepath{clip}%
\pgfsetbuttcap%
\pgfsetroundjoin%
\definecolor{currentfill}{rgb}{1.000000,0.498039,0.054902}%
\pgfsetfillcolor{currentfill}%
\pgfsetlinewidth{0.481800pt}%
\definecolor{currentstroke}{rgb}{1.000000,1.000000,1.000000}%
\pgfsetstrokecolor{currentstroke}%
\pgfsetdash{}{0pt}%
\pgfpathmoveto{\pgfqpoint{6.350168in}{5.914713in}}%
\pgfpathcurveto{\pgfqpoint{6.361218in}{5.914713in}}{\pgfqpoint{6.371817in}{5.919103in}}{\pgfqpoint{6.379631in}{5.926917in}}%
\pgfpathcurveto{\pgfqpoint{6.387445in}{5.934730in}}{\pgfqpoint{6.391835in}{5.945329in}}{\pgfqpoint{6.391835in}{5.956379in}}%
\pgfpathcurveto{\pgfqpoint{6.391835in}{5.967430in}}{\pgfqpoint{6.387445in}{5.978029in}}{\pgfqpoint{6.379631in}{5.985842in}}%
\pgfpathcurveto{\pgfqpoint{6.371817in}{5.993656in}}{\pgfqpoint{6.361218in}{5.998046in}}{\pgfqpoint{6.350168in}{5.998046in}}%
\pgfpathcurveto{\pgfqpoint{6.339118in}{5.998046in}}{\pgfqpoint{6.328519in}{5.993656in}}{\pgfqpoint{6.320705in}{5.985842in}}%
\pgfpathcurveto{\pgfqpoint{6.312892in}{5.978029in}}{\pgfqpoint{6.308501in}{5.967430in}}{\pgfqpoint{6.308501in}{5.956379in}}%
\pgfpathcurveto{\pgfqpoint{6.308501in}{5.945329in}}{\pgfqpoint{6.312892in}{5.934730in}}{\pgfqpoint{6.320705in}{5.926917in}}%
\pgfpathcurveto{\pgfqpoint{6.328519in}{5.919103in}}{\pgfqpoint{6.339118in}{5.914713in}}{\pgfqpoint{6.350168in}{5.914713in}}%
\pgfpathlineto{\pgfqpoint{6.350168in}{5.914713in}}%
\pgfpathclose%
\pgfusepath{stroke,fill}%
\end{pgfscope}%
\begin{pgfscope}%
\pgfpathrectangle{\pgfqpoint{5.292946in}{5.272501in}}{\pgfqpoint{2.177280in}{2.201755in}}%
\pgfusepath{clip}%
\pgfsetbuttcap%
\pgfsetroundjoin%
\definecolor{currentfill}{rgb}{1.000000,0.498039,0.054902}%
\pgfsetfillcolor{currentfill}%
\pgfsetlinewidth{0.481800pt}%
\definecolor{currentstroke}{rgb}{1.000000,1.000000,1.000000}%
\pgfsetstrokecolor{currentstroke}%
\pgfsetdash{}{0pt}%
\pgfpathmoveto{\pgfqpoint{6.464989in}{5.497714in}}%
\pgfpathcurveto{\pgfqpoint{6.476039in}{5.497714in}}{\pgfqpoint{6.486638in}{5.502104in}}{\pgfqpoint{6.494452in}{5.509918in}}%
\pgfpathcurveto{\pgfqpoint{6.502266in}{5.517731in}}{\pgfqpoint{6.506656in}{5.528330in}}{\pgfqpoint{6.506656in}{5.539380in}}%
\pgfpathcurveto{\pgfqpoint{6.506656in}{5.550431in}}{\pgfqpoint{6.502266in}{5.561030in}}{\pgfqpoint{6.494452in}{5.568843in}}%
\pgfpathcurveto{\pgfqpoint{6.486638in}{5.576657in}}{\pgfqpoint{6.476039in}{5.581047in}}{\pgfqpoint{6.464989in}{5.581047in}}%
\pgfpathcurveto{\pgfqpoint{6.453939in}{5.581047in}}{\pgfqpoint{6.443340in}{5.576657in}}{\pgfqpoint{6.435527in}{5.568843in}}%
\pgfpathcurveto{\pgfqpoint{6.427713in}{5.561030in}}{\pgfqpoint{6.423323in}{5.550431in}}{\pgfqpoint{6.423323in}{5.539380in}}%
\pgfpathcurveto{\pgfqpoint{6.423323in}{5.528330in}}{\pgfqpoint{6.427713in}{5.517731in}}{\pgfqpoint{6.435527in}{5.509918in}}%
\pgfpathcurveto{\pgfqpoint{6.443340in}{5.502104in}}{\pgfqpoint{6.453939in}{5.497714in}}{\pgfqpoint{6.464989in}{5.497714in}}%
\pgfpathlineto{\pgfqpoint{6.464989in}{5.497714in}}%
\pgfpathclose%
\pgfusepath{stroke,fill}%
\end{pgfscope}%
\begin{pgfscope}%
\pgfpathrectangle{\pgfqpoint{5.292946in}{5.272501in}}{\pgfqpoint{2.177280in}{2.201755in}}%
\pgfusepath{clip}%
\pgfsetbuttcap%
\pgfsetroundjoin%
\definecolor{currentfill}{rgb}{1.000000,0.498039,0.054902}%
\pgfsetfillcolor{currentfill}%
\pgfsetlinewidth{0.481800pt}%
\definecolor{currentstroke}{rgb}{1.000000,1.000000,1.000000}%
\pgfsetstrokecolor{currentstroke}%
\pgfsetdash{}{0pt}%
\pgfpathmoveto{\pgfqpoint{6.292758in}{5.747913in}}%
\pgfpathcurveto{\pgfqpoint{6.303808in}{5.747913in}}{\pgfqpoint{6.314407in}{5.752303in}}{\pgfqpoint{6.322220in}{5.760117in}}%
\pgfpathcurveto{\pgfqpoint{6.330034in}{5.767931in}}{\pgfqpoint{6.334424in}{5.778530in}}{\pgfqpoint{6.334424in}{5.789580in}}%
\pgfpathcurveto{\pgfqpoint{6.334424in}{5.800630in}}{\pgfqpoint{6.330034in}{5.811229in}}{\pgfqpoint{6.322220in}{5.819043in}}%
\pgfpathcurveto{\pgfqpoint{6.314407in}{5.826856in}}{\pgfqpoint{6.303808in}{5.831247in}}{\pgfqpoint{6.292758in}{5.831247in}}%
\pgfpathcurveto{\pgfqpoint{6.281707in}{5.831247in}}{\pgfqpoint{6.271108in}{5.826856in}}{\pgfqpoint{6.263295in}{5.819043in}}%
\pgfpathcurveto{\pgfqpoint{6.255481in}{5.811229in}}{\pgfqpoint{6.251091in}{5.800630in}}{\pgfqpoint{6.251091in}{5.789580in}}%
\pgfpathcurveto{\pgfqpoint{6.251091in}{5.778530in}}{\pgfqpoint{6.255481in}{5.767931in}}{\pgfqpoint{6.263295in}{5.760117in}}%
\pgfpathcurveto{\pgfqpoint{6.271108in}{5.752303in}}{\pgfqpoint{6.281707in}{5.747913in}}{\pgfqpoint{6.292758in}{5.747913in}}%
\pgfpathlineto{\pgfqpoint{6.292758in}{5.747913in}}%
\pgfpathclose%
\pgfusepath{stroke,fill}%
\end{pgfscope}%
\begin{pgfscope}%
\pgfpathrectangle{\pgfqpoint{5.292946in}{5.272501in}}{\pgfqpoint{2.177280in}{2.201755in}}%
\pgfusepath{clip}%
\pgfsetbuttcap%
\pgfsetroundjoin%
\definecolor{currentfill}{rgb}{1.000000,0.498039,0.054902}%
\pgfsetfillcolor{currentfill}%
\pgfsetlinewidth{0.481800pt}%
\definecolor{currentstroke}{rgb}{1.000000,1.000000,1.000000}%
\pgfsetstrokecolor{currentstroke}%
\pgfsetdash{}{0pt}%
\pgfpathmoveto{\pgfqpoint{6.551105in}{6.331712in}}%
\pgfpathcurveto{\pgfqpoint{6.562155in}{6.331712in}}{\pgfqpoint{6.572754in}{6.336102in}}{\pgfqpoint{6.580568in}{6.343916in}}%
\pgfpathcurveto{\pgfqpoint{6.588382in}{6.351729in}}{\pgfqpoint{6.592772in}{6.362328in}}{\pgfqpoint{6.592772in}{6.373379in}}%
\pgfpathcurveto{\pgfqpoint{6.592772in}{6.384429in}}{\pgfqpoint{6.588382in}{6.395028in}}{\pgfqpoint{6.580568in}{6.402841in}}%
\pgfpathcurveto{\pgfqpoint{6.572754in}{6.410655in}}{\pgfqpoint{6.562155in}{6.415045in}}{\pgfqpoint{6.551105in}{6.415045in}}%
\pgfpathcurveto{\pgfqpoint{6.540055in}{6.415045in}}{\pgfqpoint{6.529456in}{6.410655in}}{\pgfqpoint{6.521642in}{6.402841in}}%
\pgfpathcurveto{\pgfqpoint{6.513829in}{6.395028in}}{\pgfqpoint{6.509438in}{6.384429in}}{\pgfqpoint{6.509438in}{6.373379in}}%
\pgfpathcurveto{\pgfqpoint{6.509438in}{6.362328in}}{\pgfqpoint{6.513829in}{6.351729in}}{\pgfqpoint{6.521642in}{6.343916in}}%
\pgfpathcurveto{\pgfqpoint{6.529456in}{6.336102in}}{\pgfqpoint{6.540055in}{6.331712in}}{\pgfqpoint{6.551105in}{6.331712in}}%
\pgfpathlineto{\pgfqpoint{6.551105in}{6.331712in}}%
\pgfpathclose%
\pgfusepath{stroke,fill}%
\end{pgfscope}%
\begin{pgfscope}%
\pgfpathrectangle{\pgfqpoint{5.292946in}{5.272501in}}{\pgfqpoint{2.177280in}{2.201755in}}%
\pgfusepath{clip}%
\pgfsetbuttcap%
\pgfsetroundjoin%
\definecolor{currentfill}{rgb}{1.000000,0.498039,0.054902}%
\pgfsetfillcolor{currentfill}%
\pgfsetlinewidth{0.481800pt}%
\definecolor{currentstroke}{rgb}{1.000000,1.000000,1.000000}%
\pgfsetstrokecolor{currentstroke}%
\pgfsetdash{}{0pt}%
\pgfpathmoveto{\pgfqpoint{6.321463in}{5.998113in}}%
\pgfpathcurveto{\pgfqpoint{6.332513in}{5.998113in}}{\pgfqpoint{6.343112in}{6.002503in}}{\pgfqpoint{6.350926in}{6.010317in}}%
\pgfpathcurveto{\pgfqpoint{6.358739in}{6.018130in}}{\pgfqpoint{6.363130in}{6.028729in}}{\pgfqpoint{6.363130in}{6.039779in}}%
\pgfpathcurveto{\pgfqpoint{6.363130in}{6.050829in}}{\pgfqpoint{6.358739in}{6.061428in}}{\pgfqpoint{6.350926in}{6.069242in}}%
\pgfpathcurveto{\pgfqpoint{6.343112in}{6.077056in}}{\pgfqpoint{6.332513in}{6.081446in}}{\pgfqpoint{6.321463in}{6.081446in}}%
\pgfpathcurveto{\pgfqpoint{6.310413in}{6.081446in}}{\pgfqpoint{6.299814in}{6.077056in}}{\pgfqpoint{6.292000in}{6.069242in}}%
\pgfpathcurveto{\pgfqpoint{6.284186in}{6.061428in}}{\pgfqpoint{6.279796in}{6.050829in}}{\pgfqpoint{6.279796in}{6.039779in}}%
\pgfpathcurveto{\pgfqpoint{6.279796in}{6.028729in}}{\pgfqpoint{6.284186in}{6.018130in}}{\pgfqpoint{6.292000in}{6.010317in}}%
\pgfpathcurveto{\pgfqpoint{6.299814in}{6.002503in}}{\pgfqpoint{6.310413in}{5.998113in}}{\pgfqpoint{6.321463in}{5.998113in}}%
\pgfpathlineto{\pgfqpoint{6.321463in}{5.998113in}}%
\pgfpathclose%
\pgfusepath{stroke,fill}%
\end{pgfscope}%
\begin{pgfscope}%
\pgfpathrectangle{\pgfqpoint{5.292946in}{5.272501in}}{\pgfqpoint{2.177280in}{2.201755in}}%
\pgfusepath{clip}%
\pgfsetbuttcap%
\pgfsetroundjoin%
\definecolor{currentfill}{rgb}{1.000000,0.498039,0.054902}%
\pgfsetfillcolor{currentfill}%
\pgfsetlinewidth{0.481800pt}%
\definecolor{currentstroke}{rgb}{1.000000,1.000000,1.000000}%
\pgfsetstrokecolor{currentstroke}%
\pgfsetdash{}{0pt}%
\pgfpathmoveto{\pgfqpoint{6.579810in}{5.747913in}}%
\pgfpathcurveto{\pgfqpoint{6.590861in}{5.747913in}}{\pgfqpoint{6.601460in}{5.752303in}}{\pgfqpoint{6.609273in}{5.760117in}}%
\pgfpathcurveto{\pgfqpoint{6.617087in}{5.767931in}}{\pgfqpoint{6.621477in}{5.778530in}}{\pgfqpoint{6.621477in}{5.789580in}}%
\pgfpathcurveto{\pgfqpoint{6.621477in}{5.800630in}}{\pgfqpoint{6.617087in}{5.811229in}}{\pgfqpoint{6.609273in}{5.819043in}}%
\pgfpathcurveto{\pgfqpoint{6.601460in}{5.826856in}}{\pgfqpoint{6.590861in}{5.831247in}}{\pgfqpoint{6.579810in}{5.831247in}}%
\pgfpathcurveto{\pgfqpoint{6.568760in}{5.831247in}}{\pgfqpoint{6.558161in}{5.826856in}}{\pgfqpoint{6.550348in}{5.819043in}}%
\pgfpathcurveto{\pgfqpoint{6.542534in}{5.811229in}}{\pgfqpoint{6.538144in}{5.800630in}}{\pgfqpoint{6.538144in}{5.789580in}}%
\pgfpathcurveto{\pgfqpoint{6.538144in}{5.778530in}}{\pgfqpoint{6.542534in}{5.767931in}}{\pgfqpoint{6.550348in}{5.760117in}}%
\pgfpathcurveto{\pgfqpoint{6.558161in}{5.752303in}}{\pgfqpoint{6.568760in}{5.747913in}}{\pgfqpoint{6.579810in}{5.747913in}}%
\pgfpathlineto{\pgfqpoint{6.579810in}{5.747913in}}%
\pgfpathclose%
\pgfusepath{stroke,fill}%
\end{pgfscope}%
\begin{pgfscope}%
\pgfpathrectangle{\pgfqpoint{5.292946in}{5.272501in}}{\pgfqpoint{2.177280in}{2.201755in}}%
\pgfusepath{clip}%
\pgfsetbuttcap%
\pgfsetroundjoin%
\definecolor{currentfill}{rgb}{1.000000,0.498039,0.054902}%
\pgfsetfillcolor{currentfill}%
\pgfsetlinewidth{0.481800pt}%
\definecolor{currentstroke}{rgb}{1.000000,1.000000,1.000000}%
\pgfsetstrokecolor{currentstroke}%
\pgfsetdash{}{0pt}%
\pgfpathmoveto{\pgfqpoint{6.522400in}{5.998113in}}%
\pgfpathcurveto{\pgfqpoint{6.533450in}{5.998113in}}{\pgfqpoint{6.544049in}{6.002503in}}{\pgfqpoint{6.551863in}{6.010317in}}%
\pgfpathcurveto{\pgfqpoint{6.559676in}{6.018130in}}{\pgfqpoint{6.564067in}{6.028729in}}{\pgfqpoint{6.564067in}{6.039779in}}%
\pgfpathcurveto{\pgfqpoint{6.564067in}{6.050829in}}{\pgfqpoint{6.559676in}{6.061428in}}{\pgfqpoint{6.551863in}{6.069242in}}%
\pgfpathcurveto{\pgfqpoint{6.544049in}{6.077056in}}{\pgfqpoint{6.533450in}{6.081446in}}{\pgfqpoint{6.522400in}{6.081446in}}%
\pgfpathcurveto{\pgfqpoint{6.511350in}{6.081446in}}{\pgfqpoint{6.500751in}{6.077056in}}{\pgfqpoint{6.492937in}{6.069242in}}%
\pgfpathcurveto{\pgfqpoint{6.485123in}{6.061428in}}{\pgfqpoint{6.480733in}{6.050829in}}{\pgfqpoint{6.480733in}{6.039779in}}%
\pgfpathcurveto{\pgfqpoint{6.480733in}{6.028729in}}{\pgfqpoint{6.485123in}{6.018130in}}{\pgfqpoint{6.492937in}{6.010317in}}%
\pgfpathcurveto{\pgfqpoint{6.500751in}{6.002503in}}{\pgfqpoint{6.511350in}{5.998113in}}{\pgfqpoint{6.522400in}{5.998113in}}%
\pgfpathlineto{\pgfqpoint{6.522400in}{5.998113in}}%
\pgfpathclose%
\pgfusepath{stroke,fill}%
\end{pgfscope}%
\begin{pgfscope}%
\pgfpathrectangle{\pgfqpoint{5.292946in}{5.272501in}}{\pgfqpoint{2.177280in}{2.201755in}}%
\pgfusepath{clip}%
\pgfsetbuttcap%
\pgfsetroundjoin%
\definecolor{currentfill}{rgb}{1.000000,0.498039,0.054902}%
\pgfsetfillcolor{currentfill}%
\pgfsetlinewidth{0.481800pt}%
\definecolor{currentstroke}{rgb}{1.000000,1.000000,1.000000}%
\pgfsetstrokecolor{currentstroke}%
\pgfsetdash{}{0pt}%
\pgfpathmoveto{\pgfqpoint{6.407579in}{6.081512in}}%
\pgfpathcurveto{\pgfqpoint{6.418629in}{6.081512in}}{\pgfqpoint{6.429228in}{6.085903in}}{\pgfqpoint{6.437041in}{6.093716in}}%
\pgfpathcurveto{\pgfqpoint{6.444855in}{6.101530in}}{\pgfqpoint{6.449245in}{6.112129in}}{\pgfqpoint{6.449245in}{6.123179in}}%
\pgfpathcurveto{\pgfqpoint{6.449245in}{6.134229in}}{\pgfqpoint{6.444855in}{6.144828in}}{\pgfqpoint{6.437041in}{6.152642in}}%
\pgfpathcurveto{\pgfqpoint{6.429228in}{6.160456in}}{\pgfqpoint{6.418629in}{6.164846in}}{\pgfqpoint{6.407579in}{6.164846in}}%
\pgfpathcurveto{\pgfqpoint{6.396529in}{6.164846in}}{\pgfqpoint{6.385930in}{6.160456in}}{\pgfqpoint{6.378116in}{6.152642in}}%
\pgfpathcurveto{\pgfqpoint{6.370302in}{6.144828in}}{\pgfqpoint{6.365912in}{6.134229in}}{\pgfqpoint{6.365912in}{6.123179in}}%
\pgfpathcurveto{\pgfqpoint{6.365912in}{6.112129in}}{\pgfqpoint{6.370302in}{6.101530in}}{\pgfqpoint{6.378116in}{6.093716in}}%
\pgfpathcurveto{\pgfqpoint{6.385930in}{6.085903in}}{\pgfqpoint{6.396529in}{6.081512in}}{\pgfqpoint{6.407579in}{6.081512in}}%
\pgfpathlineto{\pgfqpoint{6.407579in}{6.081512in}}%
\pgfpathclose%
\pgfusepath{stroke,fill}%
\end{pgfscope}%
\begin{pgfscope}%
\pgfpathrectangle{\pgfqpoint{5.292946in}{5.272501in}}{\pgfqpoint{2.177280in}{2.201755in}}%
\pgfusepath{clip}%
\pgfsetbuttcap%
\pgfsetroundjoin%
\definecolor{currentfill}{rgb}{1.000000,0.498039,0.054902}%
\pgfsetfillcolor{currentfill}%
\pgfsetlinewidth{0.481800pt}%
\definecolor{currentstroke}{rgb}{1.000000,1.000000,1.000000}%
\pgfsetstrokecolor{currentstroke}%
\pgfsetdash{}{0pt}%
\pgfpathmoveto{\pgfqpoint{6.436284in}{6.164912in}}%
\pgfpathcurveto{\pgfqpoint{6.447334in}{6.164912in}}{\pgfqpoint{6.457933in}{6.169303in}}{\pgfqpoint{6.465747in}{6.177116in}}%
\pgfpathcurveto{\pgfqpoint{6.473560in}{6.184930in}}{\pgfqpoint{6.477951in}{6.195529in}}{\pgfqpoint{6.477951in}{6.206579in}}%
\pgfpathcurveto{\pgfqpoint{6.477951in}{6.217629in}}{\pgfqpoint{6.473560in}{6.228228in}}{\pgfqpoint{6.465747in}{6.236042in}}%
\pgfpathcurveto{\pgfqpoint{6.457933in}{6.243855in}}{\pgfqpoint{6.447334in}{6.248246in}}{\pgfqpoint{6.436284in}{6.248246in}}%
\pgfpathcurveto{\pgfqpoint{6.425234in}{6.248246in}}{\pgfqpoint{6.414635in}{6.243855in}}{\pgfqpoint{6.406821in}{6.236042in}}%
\pgfpathcurveto{\pgfqpoint{6.399008in}{6.228228in}}{\pgfqpoint{6.394617in}{6.217629in}}{\pgfqpoint{6.394617in}{6.206579in}}%
\pgfpathcurveto{\pgfqpoint{6.394617in}{6.195529in}}{\pgfqpoint{6.399008in}{6.184930in}}{\pgfqpoint{6.406821in}{6.177116in}}%
\pgfpathcurveto{\pgfqpoint{6.414635in}{6.169303in}}{\pgfqpoint{6.425234in}{6.164912in}}{\pgfqpoint{6.436284in}{6.164912in}}%
\pgfpathlineto{\pgfqpoint{6.436284in}{6.164912in}}%
\pgfpathclose%
\pgfusepath{stroke,fill}%
\end{pgfscope}%
\begin{pgfscope}%
\pgfpathrectangle{\pgfqpoint{5.292946in}{5.272501in}}{\pgfqpoint{2.177280in}{2.201755in}}%
\pgfusepath{clip}%
\pgfsetbuttcap%
\pgfsetroundjoin%
\definecolor{currentfill}{rgb}{1.000000,0.498039,0.054902}%
\pgfsetfillcolor{currentfill}%
\pgfsetlinewidth{0.481800pt}%
\definecolor{currentstroke}{rgb}{1.000000,1.000000,1.000000}%
\pgfsetstrokecolor{currentstroke}%
\pgfsetdash{}{0pt}%
\pgfpathmoveto{\pgfqpoint{6.551105in}{5.998113in}}%
\pgfpathcurveto{\pgfqpoint{6.562155in}{5.998113in}}{\pgfqpoint{6.572754in}{6.002503in}}{\pgfqpoint{6.580568in}{6.010317in}}%
\pgfpathcurveto{\pgfqpoint{6.588382in}{6.018130in}}{\pgfqpoint{6.592772in}{6.028729in}}{\pgfqpoint{6.592772in}{6.039779in}}%
\pgfpathcurveto{\pgfqpoint{6.592772in}{6.050829in}}{\pgfqpoint{6.588382in}{6.061428in}}{\pgfqpoint{6.580568in}{6.069242in}}%
\pgfpathcurveto{\pgfqpoint{6.572754in}{6.077056in}}{\pgfqpoint{6.562155in}{6.081446in}}{\pgfqpoint{6.551105in}{6.081446in}}%
\pgfpathcurveto{\pgfqpoint{6.540055in}{6.081446in}}{\pgfqpoint{6.529456in}{6.077056in}}{\pgfqpoint{6.521642in}{6.069242in}}%
\pgfpathcurveto{\pgfqpoint{6.513829in}{6.061428in}}{\pgfqpoint{6.509438in}{6.050829in}}{\pgfqpoint{6.509438in}{6.039779in}}%
\pgfpathcurveto{\pgfqpoint{6.509438in}{6.028729in}}{\pgfqpoint{6.513829in}{6.018130in}}{\pgfqpoint{6.521642in}{6.010317in}}%
\pgfpathcurveto{\pgfqpoint{6.529456in}{6.002503in}}{\pgfqpoint{6.540055in}{5.998113in}}{\pgfqpoint{6.551105in}{5.998113in}}%
\pgfpathlineto{\pgfqpoint{6.551105in}{5.998113in}}%
\pgfpathclose%
\pgfusepath{stroke,fill}%
\end{pgfscope}%
\begin{pgfscope}%
\pgfpathrectangle{\pgfqpoint{5.292946in}{5.272501in}}{\pgfqpoint{2.177280in}{2.201755in}}%
\pgfusepath{clip}%
\pgfsetbuttcap%
\pgfsetroundjoin%
\definecolor{currentfill}{rgb}{1.000000,0.498039,0.054902}%
\pgfsetfillcolor{currentfill}%
\pgfsetlinewidth{0.481800pt}%
\definecolor{currentstroke}{rgb}{1.000000,1.000000,1.000000}%
\pgfsetstrokecolor{currentstroke}%
\pgfsetdash{}{0pt}%
\pgfpathmoveto{\pgfqpoint{6.608516in}{6.164912in}}%
\pgfpathcurveto{\pgfqpoint{6.619566in}{6.164912in}}{\pgfqpoint{6.630165in}{6.169303in}}{\pgfqpoint{6.637978in}{6.177116in}}%
\pgfpathcurveto{\pgfqpoint{6.645792in}{6.184930in}}{\pgfqpoint{6.650182in}{6.195529in}}{\pgfqpoint{6.650182in}{6.206579in}}%
\pgfpathcurveto{\pgfqpoint{6.650182in}{6.217629in}}{\pgfqpoint{6.645792in}{6.228228in}}{\pgfqpoint{6.637978in}{6.236042in}}%
\pgfpathcurveto{\pgfqpoint{6.630165in}{6.243855in}}{\pgfqpoint{6.619566in}{6.248246in}}{\pgfqpoint{6.608516in}{6.248246in}}%
\pgfpathcurveto{\pgfqpoint{6.597466in}{6.248246in}}{\pgfqpoint{6.586867in}{6.243855in}}{\pgfqpoint{6.579053in}{6.236042in}}%
\pgfpathcurveto{\pgfqpoint{6.571239in}{6.228228in}}{\pgfqpoint{6.566849in}{6.217629in}}{\pgfqpoint{6.566849in}{6.206579in}}%
\pgfpathcurveto{\pgfqpoint{6.566849in}{6.195529in}}{\pgfqpoint{6.571239in}{6.184930in}}{\pgfqpoint{6.579053in}{6.177116in}}%
\pgfpathcurveto{\pgfqpoint{6.586867in}{6.169303in}}{\pgfqpoint{6.597466in}{6.164912in}}{\pgfqpoint{6.608516in}{6.164912in}}%
\pgfpathlineto{\pgfqpoint{6.608516in}{6.164912in}}%
\pgfpathclose%
\pgfusepath{stroke,fill}%
\end{pgfscope}%
\begin{pgfscope}%
\pgfpathrectangle{\pgfqpoint{5.292946in}{5.272501in}}{\pgfqpoint{2.177280in}{2.201755in}}%
\pgfusepath{clip}%
\pgfsetbuttcap%
\pgfsetroundjoin%
\definecolor{currentfill}{rgb}{1.000000,0.498039,0.054902}%
\pgfsetfillcolor{currentfill}%
\pgfsetlinewidth{0.481800pt}%
\definecolor{currentstroke}{rgb}{1.000000,1.000000,1.000000}%
\pgfsetstrokecolor{currentstroke}%
\pgfsetdash{}{0pt}%
\pgfpathmoveto{\pgfqpoint{6.464989in}{6.081512in}}%
\pgfpathcurveto{\pgfqpoint{6.476039in}{6.081512in}}{\pgfqpoint{6.486638in}{6.085903in}}{\pgfqpoint{6.494452in}{6.093716in}}%
\pgfpathcurveto{\pgfqpoint{6.502266in}{6.101530in}}{\pgfqpoint{6.506656in}{6.112129in}}{\pgfqpoint{6.506656in}{6.123179in}}%
\pgfpathcurveto{\pgfqpoint{6.506656in}{6.134229in}}{\pgfqpoint{6.502266in}{6.144828in}}{\pgfqpoint{6.494452in}{6.152642in}}%
\pgfpathcurveto{\pgfqpoint{6.486638in}{6.160456in}}{\pgfqpoint{6.476039in}{6.164846in}}{\pgfqpoint{6.464989in}{6.164846in}}%
\pgfpathcurveto{\pgfqpoint{6.453939in}{6.164846in}}{\pgfqpoint{6.443340in}{6.160456in}}{\pgfqpoint{6.435527in}{6.152642in}}%
\pgfpathcurveto{\pgfqpoint{6.427713in}{6.144828in}}{\pgfqpoint{6.423323in}{6.134229in}}{\pgfqpoint{6.423323in}{6.123179in}}%
\pgfpathcurveto{\pgfqpoint{6.423323in}{6.112129in}}{\pgfqpoint{6.427713in}{6.101530in}}{\pgfqpoint{6.435527in}{6.093716in}}%
\pgfpathcurveto{\pgfqpoint{6.443340in}{6.085903in}}{\pgfqpoint{6.453939in}{6.081512in}}{\pgfqpoint{6.464989in}{6.081512in}}%
\pgfpathlineto{\pgfqpoint{6.464989in}{6.081512in}}%
\pgfpathclose%
\pgfusepath{stroke,fill}%
\end{pgfscope}%
\begin{pgfscope}%
\pgfpathrectangle{\pgfqpoint{5.292946in}{5.272501in}}{\pgfqpoint{2.177280in}{2.201755in}}%
\pgfusepath{clip}%
\pgfsetbuttcap%
\pgfsetroundjoin%
\definecolor{currentfill}{rgb}{1.000000,0.498039,0.054902}%
\pgfsetfillcolor{currentfill}%
\pgfsetlinewidth{0.481800pt}%
\definecolor{currentstroke}{rgb}{1.000000,1.000000,1.000000}%
\pgfsetstrokecolor{currentstroke}%
\pgfsetdash{}{0pt}%
\pgfpathmoveto{\pgfqpoint{6.177936in}{5.831313in}}%
\pgfpathcurveto{\pgfqpoint{6.188987in}{5.831313in}}{\pgfqpoint{6.199586in}{5.835703in}}{\pgfqpoint{6.207399in}{5.843517in}}%
\pgfpathcurveto{\pgfqpoint{6.215213in}{5.851331in}}{\pgfqpoint{6.219603in}{5.861930in}}{\pgfqpoint{6.219603in}{5.872980in}}%
\pgfpathcurveto{\pgfqpoint{6.219603in}{5.884030in}}{\pgfqpoint{6.215213in}{5.894629in}}{\pgfqpoint{6.207399in}{5.902442in}}%
\pgfpathcurveto{\pgfqpoint{6.199586in}{5.910256in}}{\pgfqpoint{6.188987in}{5.914646in}}{\pgfqpoint{6.177936in}{5.914646in}}%
\pgfpathcurveto{\pgfqpoint{6.166886in}{5.914646in}}{\pgfqpoint{6.156287in}{5.910256in}}{\pgfqpoint{6.148474in}{5.902442in}}%
\pgfpathcurveto{\pgfqpoint{6.140660in}{5.894629in}}{\pgfqpoint{6.136270in}{5.884030in}}{\pgfqpoint{6.136270in}{5.872980in}}%
\pgfpathcurveto{\pgfqpoint{6.136270in}{5.861930in}}{\pgfqpoint{6.140660in}{5.851331in}}{\pgfqpoint{6.148474in}{5.843517in}}%
\pgfpathcurveto{\pgfqpoint{6.156287in}{5.835703in}}{\pgfqpoint{6.166886in}{5.831313in}}{\pgfqpoint{6.177936in}{5.831313in}}%
\pgfpathlineto{\pgfqpoint{6.177936in}{5.831313in}}%
\pgfpathclose%
\pgfusepath{stroke,fill}%
\end{pgfscope}%
\begin{pgfscope}%
\pgfpathrectangle{\pgfqpoint{5.292946in}{5.272501in}}{\pgfqpoint{2.177280in}{2.201755in}}%
\pgfusepath{clip}%
\pgfsetbuttcap%
\pgfsetroundjoin%
\definecolor{currentfill}{rgb}{1.000000,0.498039,0.054902}%
\pgfsetfillcolor{currentfill}%
\pgfsetlinewidth{0.481800pt}%
\definecolor{currentstroke}{rgb}{1.000000,1.000000,1.000000}%
\pgfsetstrokecolor{currentstroke}%
\pgfsetdash{}{0pt}%
\pgfpathmoveto{\pgfqpoint{6.264052in}{5.664513in}}%
\pgfpathcurveto{\pgfqpoint{6.275102in}{5.664513in}}{\pgfqpoint{6.285701in}{5.668904in}}{\pgfqpoint{6.293515in}{5.676717in}}%
\pgfpathcurveto{\pgfqpoint{6.301329in}{5.684531in}}{\pgfqpoint{6.305719in}{5.695130in}}{\pgfqpoint{6.305719in}{5.706180in}}%
\pgfpathcurveto{\pgfqpoint{6.305719in}{5.717230in}}{\pgfqpoint{6.301329in}{5.727829in}}{\pgfqpoint{6.293515in}{5.735643in}}%
\pgfpathcurveto{\pgfqpoint{6.285701in}{5.743456in}}{\pgfqpoint{6.275102in}{5.747847in}}{\pgfqpoint{6.264052in}{5.747847in}}%
\pgfpathcurveto{\pgfqpoint{6.253002in}{5.747847in}}{\pgfqpoint{6.242403in}{5.743456in}}{\pgfqpoint{6.234590in}{5.735643in}}%
\pgfpathcurveto{\pgfqpoint{6.226776in}{5.727829in}}{\pgfqpoint{6.222386in}{5.717230in}}{\pgfqpoint{6.222386in}{5.706180in}}%
\pgfpathcurveto{\pgfqpoint{6.222386in}{5.695130in}}{\pgfqpoint{6.226776in}{5.684531in}}{\pgfqpoint{6.234590in}{5.676717in}}%
\pgfpathcurveto{\pgfqpoint{6.242403in}{5.668904in}}{\pgfqpoint{6.253002in}{5.664513in}}{\pgfqpoint{6.264052in}{5.664513in}}%
\pgfpathlineto{\pgfqpoint{6.264052in}{5.664513in}}%
\pgfpathclose%
\pgfusepath{stroke,fill}%
\end{pgfscope}%
\begin{pgfscope}%
\pgfpathrectangle{\pgfqpoint{5.292946in}{5.272501in}}{\pgfqpoint{2.177280in}{2.201755in}}%
\pgfusepath{clip}%
\pgfsetbuttcap%
\pgfsetroundjoin%
\definecolor{currentfill}{rgb}{1.000000,0.498039,0.054902}%
\pgfsetfillcolor{currentfill}%
\pgfsetlinewidth{0.481800pt}%
\definecolor{currentstroke}{rgb}{1.000000,1.000000,1.000000}%
\pgfsetstrokecolor{currentstroke}%
\pgfsetdash{}{0pt}%
\pgfpathmoveto{\pgfqpoint{6.235347in}{5.664513in}}%
\pgfpathcurveto{\pgfqpoint{6.246397in}{5.664513in}}{\pgfqpoint{6.256996in}{5.668904in}}{\pgfqpoint{6.264810in}{5.676717in}}%
\pgfpathcurveto{\pgfqpoint{6.272623in}{5.684531in}}{\pgfqpoint{6.277014in}{5.695130in}}{\pgfqpoint{6.277014in}{5.706180in}}%
\pgfpathcurveto{\pgfqpoint{6.277014in}{5.717230in}}{\pgfqpoint{6.272623in}{5.727829in}}{\pgfqpoint{6.264810in}{5.735643in}}%
\pgfpathcurveto{\pgfqpoint{6.256996in}{5.743456in}}{\pgfqpoint{6.246397in}{5.747847in}}{\pgfqpoint{6.235347in}{5.747847in}}%
\pgfpathcurveto{\pgfqpoint{6.224297in}{5.747847in}}{\pgfqpoint{6.213698in}{5.743456in}}{\pgfqpoint{6.205884in}{5.735643in}}%
\pgfpathcurveto{\pgfqpoint{6.198071in}{5.727829in}}{\pgfqpoint{6.193680in}{5.717230in}}{\pgfqpoint{6.193680in}{5.706180in}}%
\pgfpathcurveto{\pgfqpoint{6.193680in}{5.695130in}}{\pgfqpoint{6.198071in}{5.684531in}}{\pgfqpoint{6.205884in}{5.676717in}}%
\pgfpathcurveto{\pgfqpoint{6.213698in}{5.668904in}}{\pgfqpoint{6.224297in}{5.664513in}}{\pgfqpoint{6.235347in}{5.664513in}}%
\pgfpathlineto{\pgfqpoint{6.235347in}{5.664513in}}%
\pgfpathclose%
\pgfusepath{stroke,fill}%
\end{pgfscope}%
\begin{pgfscope}%
\pgfpathrectangle{\pgfqpoint{5.292946in}{5.272501in}}{\pgfqpoint{2.177280in}{2.201755in}}%
\pgfusepath{clip}%
\pgfsetbuttcap%
\pgfsetroundjoin%
\definecolor{currentfill}{rgb}{1.000000,0.498039,0.054902}%
\pgfsetfillcolor{currentfill}%
\pgfsetlinewidth{0.481800pt}%
\definecolor{currentstroke}{rgb}{1.000000,1.000000,1.000000}%
\pgfsetstrokecolor{currentstroke}%
\pgfsetdash{}{0pt}%
\pgfpathmoveto{\pgfqpoint{6.292758in}{5.914713in}}%
\pgfpathcurveto{\pgfqpoint{6.303808in}{5.914713in}}{\pgfqpoint{6.314407in}{5.919103in}}{\pgfqpoint{6.322220in}{5.926917in}}%
\pgfpathcurveto{\pgfqpoint{6.330034in}{5.934730in}}{\pgfqpoint{6.334424in}{5.945329in}}{\pgfqpoint{6.334424in}{5.956379in}}%
\pgfpathcurveto{\pgfqpoint{6.334424in}{5.967430in}}{\pgfqpoint{6.330034in}{5.978029in}}{\pgfqpoint{6.322220in}{5.985842in}}%
\pgfpathcurveto{\pgfqpoint{6.314407in}{5.993656in}}{\pgfqpoint{6.303808in}{5.998046in}}{\pgfqpoint{6.292758in}{5.998046in}}%
\pgfpathcurveto{\pgfqpoint{6.281707in}{5.998046in}}{\pgfqpoint{6.271108in}{5.993656in}}{\pgfqpoint{6.263295in}{5.985842in}}%
\pgfpathcurveto{\pgfqpoint{6.255481in}{5.978029in}}{\pgfqpoint{6.251091in}{5.967430in}}{\pgfqpoint{6.251091in}{5.956379in}}%
\pgfpathcurveto{\pgfqpoint{6.251091in}{5.945329in}}{\pgfqpoint{6.255481in}{5.934730in}}{\pgfqpoint{6.263295in}{5.926917in}}%
\pgfpathcurveto{\pgfqpoint{6.271108in}{5.919103in}}{\pgfqpoint{6.281707in}{5.914713in}}{\pgfqpoint{6.292758in}{5.914713in}}%
\pgfpathlineto{\pgfqpoint{6.292758in}{5.914713in}}%
\pgfpathclose%
\pgfusepath{stroke,fill}%
\end{pgfscope}%
\begin{pgfscope}%
\pgfpathrectangle{\pgfqpoint{5.292946in}{5.272501in}}{\pgfqpoint{2.177280in}{2.201755in}}%
\pgfusepath{clip}%
\pgfsetbuttcap%
\pgfsetroundjoin%
\definecolor{currentfill}{rgb}{1.000000,0.498039,0.054902}%
\pgfsetfillcolor{currentfill}%
\pgfsetlinewidth{0.481800pt}%
\definecolor{currentstroke}{rgb}{1.000000,1.000000,1.000000}%
\pgfsetstrokecolor{currentstroke}%
\pgfsetdash{}{0pt}%
\pgfpathmoveto{\pgfqpoint{6.637221in}{5.914713in}}%
\pgfpathcurveto{\pgfqpoint{6.648271in}{5.914713in}}{\pgfqpoint{6.658870in}{5.919103in}}{\pgfqpoint{6.666684in}{5.926917in}}%
\pgfpathcurveto{\pgfqpoint{6.674497in}{5.934730in}}{\pgfqpoint{6.678888in}{5.945329in}}{\pgfqpoint{6.678888in}{5.956379in}}%
\pgfpathcurveto{\pgfqpoint{6.678888in}{5.967430in}}{\pgfqpoint{6.674497in}{5.978029in}}{\pgfqpoint{6.666684in}{5.985842in}}%
\pgfpathcurveto{\pgfqpoint{6.658870in}{5.993656in}}{\pgfqpoint{6.648271in}{5.998046in}}{\pgfqpoint{6.637221in}{5.998046in}}%
\pgfpathcurveto{\pgfqpoint{6.626171in}{5.998046in}}{\pgfqpoint{6.615572in}{5.993656in}}{\pgfqpoint{6.607758in}{5.985842in}}%
\pgfpathcurveto{\pgfqpoint{6.599945in}{5.978029in}}{\pgfqpoint{6.595554in}{5.967430in}}{\pgfqpoint{6.595554in}{5.956379in}}%
\pgfpathcurveto{\pgfqpoint{6.595554in}{5.945329in}}{\pgfqpoint{6.599945in}{5.934730in}}{\pgfqpoint{6.607758in}{5.926917in}}%
\pgfpathcurveto{\pgfqpoint{6.615572in}{5.919103in}}{\pgfqpoint{6.626171in}{5.914713in}}{\pgfqpoint{6.637221in}{5.914713in}}%
\pgfpathlineto{\pgfqpoint{6.637221in}{5.914713in}}%
\pgfpathclose%
\pgfusepath{stroke,fill}%
\end{pgfscope}%
\begin{pgfscope}%
\pgfpathrectangle{\pgfqpoint{5.292946in}{5.272501in}}{\pgfqpoint{2.177280in}{2.201755in}}%
\pgfusepath{clip}%
\pgfsetbuttcap%
\pgfsetroundjoin%
\definecolor{currentfill}{rgb}{1.000000,0.498039,0.054902}%
\pgfsetfillcolor{currentfill}%
\pgfsetlinewidth{0.481800pt}%
\definecolor{currentstroke}{rgb}{1.000000,1.000000,1.000000}%
\pgfsetstrokecolor{currentstroke}%
\pgfsetdash{}{0pt}%
\pgfpathmoveto{\pgfqpoint{6.464989in}{6.164912in}}%
\pgfpathcurveto{\pgfqpoint{6.476039in}{6.164912in}}{\pgfqpoint{6.486638in}{6.169303in}}{\pgfqpoint{6.494452in}{6.177116in}}%
\pgfpathcurveto{\pgfqpoint{6.502266in}{6.184930in}}{\pgfqpoint{6.506656in}{6.195529in}}{\pgfqpoint{6.506656in}{6.206579in}}%
\pgfpathcurveto{\pgfqpoint{6.506656in}{6.217629in}}{\pgfqpoint{6.502266in}{6.228228in}}{\pgfqpoint{6.494452in}{6.236042in}}%
\pgfpathcurveto{\pgfqpoint{6.486638in}{6.243855in}}{\pgfqpoint{6.476039in}{6.248246in}}{\pgfqpoint{6.464989in}{6.248246in}}%
\pgfpathcurveto{\pgfqpoint{6.453939in}{6.248246in}}{\pgfqpoint{6.443340in}{6.243855in}}{\pgfqpoint{6.435527in}{6.236042in}}%
\pgfpathcurveto{\pgfqpoint{6.427713in}{6.228228in}}{\pgfqpoint{6.423323in}{6.217629in}}{\pgfqpoint{6.423323in}{6.206579in}}%
\pgfpathcurveto{\pgfqpoint{6.423323in}{6.195529in}}{\pgfqpoint{6.427713in}{6.184930in}}{\pgfqpoint{6.435527in}{6.177116in}}%
\pgfpathcurveto{\pgfqpoint{6.443340in}{6.169303in}}{\pgfqpoint{6.453939in}{6.164912in}}{\pgfqpoint{6.464989in}{6.164912in}}%
\pgfpathlineto{\pgfqpoint{6.464989in}{6.164912in}}%
\pgfpathclose%
\pgfusepath{stroke,fill}%
\end{pgfscope}%
\begin{pgfscope}%
\pgfpathrectangle{\pgfqpoint{5.292946in}{5.272501in}}{\pgfqpoint{2.177280in}{2.201755in}}%
\pgfusepath{clip}%
\pgfsetbuttcap%
\pgfsetroundjoin%
\definecolor{currentfill}{rgb}{1.000000,0.498039,0.054902}%
\pgfsetfillcolor{currentfill}%
\pgfsetlinewidth{0.481800pt}%
\definecolor{currentstroke}{rgb}{1.000000,1.000000,1.000000}%
\pgfsetstrokecolor{currentstroke}%
\pgfsetdash{}{0pt}%
\pgfpathmoveto{\pgfqpoint{6.464989in}{6.498512in}}%
\pgfpathcurveto{\pgfqpoint{6.476039in}{6.498512in}}{\pgfqpoint{6.486638in}{6.502902in}}{\pgfqpoint{6.494452in}{6.510715in}}%
\pgfpathcurveto{\pgfqpoint{6.502266in}{6.518529in}}{\pgfqpoint{6.506656in}{6.529128in}}{\pgfqpoint{6.506656in}{6.540178in}}%
\pgfpathcurveto{\pgfqpoint{6.506656in}{6.551228in}}{\pgfqpoint{6.502266in}{6.561827in}}{\pgfqpoint{6.494452in}{6.569641in}}%
\pgfpathcurveto{\pgfqpoint{6.486638in}{6.577455in}}{\pgfqpoint{6.476039in}{6.581845in}}{\pgfqpoint{6.464989in}{6.581845in}}%
\pgfpathcurveto{\pgfqpoint{6.453939in}{6.581845in}}{\pgfqpoint{6.443340in}{6.577455in}}{\pgfqpoint{6.435527in}{6.569641in}}%
\pgfpathcurveto{\pgfqpoint{6.427713in}{6.561827in}}{\pgfqpoint{6.423323in}{6.551228in}}{\pgfqpoint{6.423323in}{6.540178in}}%
\pgfpathcurveto{\pgfqpoint{6.423323in}{6.529128in}}{\pgfqpoint{6.427713in}{6.518529in}}{\pgfqpoint{6.435527in}{6.510715in}}%
\pgfpathcurveto{\pgfqpoint{6.443340in}{6.502902in}}{\pgfqpoint{6.453939in}{6.498512in}}{\pgfqpoint{6.464989in}{6.498512in}}%
\pgfpathlineto{\pgfqpoint{6.464989in}{6.498512in}}%
\pgfpathclose%
\pgfusepath{stroke,fill}%
\end{pgfscope}%
\begin{pgfscope}%
\pgfpathrectangle{\pgfqpoint{5.292946in}{5.272501in}}{\pgfqpoint{2.177280in}{2.201755in}}%
\pgfusepath{clip}%
\pgfsetbuttcap%
\pgfsetroundjoin%
\definecolor{currentfill}{rgb}{1.000000,0.498039,0.054902}%
\pgfsetfillcolor{currentfill}%
\pgfsetlinewidth{0.481800pt}%
\definecolor{currentstroke}{rgb}{1.000000,1.000000,1.000000}%
\pgfsetstrokecolor{currentstroke}%
\pgfsetdash{}{0pt}%
\pgfpathmoveto{\pgfqpoint{6.522400in}{6.248312in}}%
\pgfpathcurveto{\pgfqpoint{6.533450in}{6.248312in}}{\pgfqpoint{6.544049in}{6.252702in}}{\pgfqpoint{6.551863in}{6.260516in}}%
\pgfpathcurveto{\pgfqpoint{6.559676in}{6.268330in}}{\pgfqpoint{6.564067in}{6.278929in}}{\pgfqpoint{6.564067in}{6.289979in}}%
\pgfpathcurveto{\pgfqpoint{6.564067in}{6.301029in}}{\pgfqpoint{6.559676in}{6.311628in}}{\pgfqpoint{6.551863in}{6.319442in}}%
\pgfpathcurveto{\pgfqpoint{6.544049in}{6.327255in}}{\pgfqpoint{6.533450in}{6.331645in}}{\pgfqpoint{6.522400in}{6.331645in}}%
\pgfpathcurveto{\pgfqpoint{6.511350in}{6.331645in}}{\pgfqpoint{6.500751in}{6.327255in}}{\pgfqpoint{6.492937in}{6.319442in}}%
\pgfpathcurveto{\pgfqpoint{6.485123in}{6.311628in}}{\pgfqpoint{6.480733in}{6.301029in}}{\pgfqpoint{6.480733in}{6.289979in}}%
\pgfpathcurveto{\pgfqpoint{6.480733in}{6.278929in}}{\pgfqpoint{6.485123in}{6.268330in}}{\pgfqpoint{6.492937in}{6.260516in}}%
\pgfpathcurveto{\pgfqpoint{6.500751in}{6.252702in}}{\pgfqpoint{6.511350in}{6.248312in}}{\pgfqpoint{6.522400in}{6.248312in}}%
\pgfpathlineto{\pgfqpoint{6.522400in}{6.248312in}}%
\pgfpathclose%
\pgfusepath{stroke,fill}%
\end{pgfscope}%
\begin{pgfscope}%
\pgfpathrectangle{\pgfqpoint{5.292946in}{5.272501in}}{\pgfqpoint{2.177280in}{2.201755in}}%
\pgfusepath{clip}%
\pgfsetbuttcap%
\pgfsetroundjoin%
\definecolor{currentfill}{rgb}{1.000000,0.498039,0.054902}%
\pgfsetfillcolor{currentfill}%
\pgfsetlinewidth{0.481800pt}%
\definecolor{currentstroke}{rgb}{1.000000,1.000000,1.000000}%
\pgfsetstrokecolor{currentstroke}%
\pgfsetdash{}{0pt}%
\pgfpathmoveto{\pgfqpoint{6.436284in}{5.581114in}}%
\pgfpathcurveto{\pgfqpoint{6.447334in}{5.581114in}}{\pgfqpoint{6.457933in}{5.585504in}}{\pgfqpoint{6.465747in}{5.593317in}}%
\pgfpathcurveto{\pgfqpoint{6.473560in}{5.601131in}}{\pgfqpoint{6.477951in}{5.611730in}}{\pgfqpoint{6.477951in}{5.622780in}}%
\pgfpathcurveto{\pgfqpoint{6.477951in}{5.633830in}}{\pgfqpoint{6.473560in}{5.644429in}}{\pgfqpoint{6.465747in}{5.652243in}}%
\pgfpathcurveto{\pgfqpoint{6.457933in}{5.660057in}}{\pgfqpoint{6.447334in}{5.664447in}}{\pgfqpoint{6.436284in}{5.664447in}}%
\pgfpathcurveto{\pgfqpoint{6.425234in}{5.664447in}}{\pgfqpoint{6.414635in}{5.660057in}}{\pgfqpoint{6.406821in}{5.652243in}}%
\pgfpathcurveto{\pgfqpoint{6.399008in}{5.644429in}}{\pgfqpoint{6.394617in}{5.633830in}}{\pgfqpoint{6.394617in}{5.622780in}}%
\pgfpathcurveto{\pgfqpoint{6.394617in}{5.611730in}}{\pgfqpoint{6.399008in}{5.601131in}}{\pgfqpoint{6.406821in}{5.593317in}}%
\pgfpathcurveto{\pgfqpoint{6.414635in}{5.585504in}}{\pgfqpoint{6.425234in}{5.581114in}}{\pgfqpoint{6.436284in}{5.581114in}}%
\pgfpathlineto{\pgfqpoint{6.436284in}{5.581114in}}%
\pgfpathclose%
\pgfusepath{stroke,fill}%
\end{pgfscope}%
\begin{pgfscope}%
\pgfpathrectangle{\pgfqpoint{5.292946in}{5.272501in}}{\pgfqpoint{2.177280in}{2.201755in}}%
\pgfusepath{clip}%
\pgfsetbuttcap%
\pgfsetroundjoin%
\definecolor{currentfill}{rgb}{1.000000,0.498039,0.054902}%
\pgfsetfillcolor{currentfill}%
\pgfsetlinewidth{0.481800pt}%
\definecolor{currentstroke}{rgb}{1.000000,1.000000,1.000000}%
\pgfsetstrokecolor{currentstroke}%
\pgfsetdash{}{0pt}%
\pgfpathmoveto{\pgfqpoint{6.350168in}{6.164912in}}%
\pgfpathcurveto{\pgfqpoint{6.361218in}{6.164912in}}{\pgfqpoint{6.371817in}{6.169303in}}{\pgfqpoint{6.379631in}{6.177116in}}%
\pgfpathcurveto{\pgfqpoint{6.387445in}{6.184930in}}{\pgfqpoint{6.391835in}{6.195529in}}{\pgfqpoint{6.391835in}{6.206579in}}%
\pgfpathcurveto{\pgfqpoint{6.391835in}{6.217629in}}{\pgfqpoint{6.387445in}{6.228228in}}{\pgfqpoint{6.379631in}{6.236042in}}%
\pgfpathcurveto{\pgfqpoint{6.371817in}{6.243855in}}{\pgfqpoint{6.361218in}{6.248246in}}{\pgfqpoint{6.350168in}{6.248246in}}%
\pgfpathcurveto{\pgfqpoint{6.339118in}{6.248246in}}{\pgfqpoint{6.328519in}{6.243855in}}{\pgfqpoint{6.320705in}{6.236042in}}%
\pgfpathcurveto{\pgfqpoint{6.312892in}{6.228228in}}{\pgfqpoint{6.308501in}{6.217629in}}{\pgfqpoint{6.308501in}{6.206579in}}%
\pgfpathcurveto{\pgfqpoint{6.308501in}{6.195529in}}{\pgfqpoint{6.312892in}{6.184930in}}{\pgfqpoint{6.320705in}{6.177116in}}%
\pgfpathcurveto{\pgfqpoint{6.328519in}{6.169303in}}{\pgfqpoint{6.339118in}{6.164912in}}{\pgfqpoint{6.350168in}{6.164912in}}%
\pgfpathlineto{\pgfqpoint{6.350168in}{6.164912in}}%
\pgfpathclose%
\pgfusepath{stroke,fill}%
\end{pgfscope}%
\begin{pgfscope}%
\pgfpathrectangle{\pgfqpoint{5.292946in}{5.272501in}}{\pgfqpoint{2.177280in}{2.201755in}}%
\pgfusepath{clip}%
\pgfsetbuttcap%
\pgfsetroundjoin%
\definecolor{currentfill}{rgb}{1.000000,0.498039,0.054902}%
\pgfsetfillcolor{currentfill}%
\pgfsetlinewidth{0.481800pt}%
\definecolor{currentstroke}{rgb}{1.000000,1.000000,1.000000}%
\pgfsetstrokecolor{currentstroke}%
\pgfsetdash{}{0pt}%
\pgfpathmoveto{\pgfqpoint{6.321463in}{5.747913in}}%
\pgfpathcurveto{\pgfqpoint{6.332513in}{5.747913in}}{\pgfqpoint{6.343112in}{5.752303in}}{\pgfqpoint{6.350926in}{5.760117in}}%
\pgfpathcurveto{\pgfqpoint{6.358739in}{5.767931in}}{\pgfqpoint{6.363130in}{5.778530in}}{\pgfqpoint{6.363130in}{5.789580in}}%
\pgfpathcurveto{\pgfqpoint{6.363130in}{5.800630in}}{\pgfqpoint{6.358739in}{5.811229in}}{\pgfqpoint{6.350926in}{5.819043in}}%
\pgfpathcurveto{\pgfqpoint{6.343112in}{5.826856in}}{\pgfqpoint{6.332513in}{5.831247in}}{\pgfqpoint{6.321463in}{5.831247in}}%
\pgfpathcurveto{\pgfqpoint{6.310413in}{5.831247in}}{\pgfqpoint{6.299814in}{5.826856in}}{\pgfqpoint{6.292000in}{5.819043in}}%
\pgfpathcurveto{\pgfqpoint{6.284186in}{5.811229in}}{\pgfqpoint{6.279796in}{5.800630in}}{\pgfqpoint{6.279796in}{5.789580in}}%
\pgfpathcurveto{\pgfqpoint{6.279796in}{5.778530in}}{\pgfqpoint{6.284186in}{5.767931in}}{\pgfqpoint{6.292000in}{5.760117in}}%
\pgfpathcurveto{\pgfqpoint{6.299814in}{5.752303in}}{\pgfqpoint{6.310413in}{5.747913in}}{\pgfqpoint{6.321463in}{5.747913in}}%
\pgfpathlineto{\pgfqpoint{6.321463in}{5.747913in}}%
\pgfpathclose%
\pgfusepath{stroke,fill}%
\end{pgfscope}%
\begin{pgfscope}%
\pgfpathrectangle{\pgfqpoint{5.292946in}{5.272501in}}{\pgfqpoint{2.177280in}{2.201755in}}%
\pgfusepath{clip}%
\pgfsetbuttcap%
\pgfsetroundjoin%
\definecolor{currentfill}{rgb}{1.000000,0.498039,0.054902}%
\pgfsetfillcolor{currentfill}%
\pgfsetlinewidth{0.481800pt}%
\definecolor{currentstroke}{rgb}{1.000000,1.000000,1.000000}%
\pgfsetstrokecolor{currentstroke}%
\pgfsetdash{}{0pt}%
\pgfpathmoveto{\pgfqpoint{6.436284in}{5.831313in}}%
\pgfpathcurveto{\pgfqpoint{6.447334in}{5.831313in}}{\pgfqpoint{6.457933in}{5.835703in}}{\pgfqpoint{6.465747in}{5.843517in}}%
\pgfpathcurveto{\pgfqpoint{6.473560in}{5.851331in}}{\pgfqpoint{6.477951in}{5.861930in}}{\pgfqpoint{6.477951in}{5.872980in}}%
\pgfpathcurveto{\pgfqpoint{6.477951in}{5.884030in}}{\pgfqpoint{6.473560in}{5.894629in}}{\pgfqpoint{6.465747in}{5.902442in}}%
\pgfpathcurveto{\pgfqpoint{6.457933in}{5.910256in}}{\pgfqpoint{6.447334in}{5.914646in}}{\pgfqpoint{6.436284in}{5.914646in}}%
\pgfpathcurveto{\pgfqpoint{6.425234in}{5.914646in}}{\pgfqpoint{6.414635in}{5.910256in}}{\pgfqpoint{6.406821in}{5.902442in}}%
\pgfpathcurveto{\pgfqpoint{6.399008in}{5.894629in}}{\pgfqpoint{6.394617in}{5.884030in}}{\pgfqpoint{6.394617in}{5.872980in}}%
\pgfpathcurveto{\pgfqpoint{6.394617in}{5.861930in}}{\pgfqpoint{6.399008in}{5.851331in}}{\pgfqpoint{6.406821in}{5.843517in}}%
\pgfpathcurveto{\pgfqpoint{6.414635in}{5.835703in}}{\pgfqpoint{6.425234in}{5.831313in}}{\pgfqpoint{6.436284in}{5.831313in}}%
\pgfpathlineto{\pgfqpoint{6.436284in}{5.831313in}}%
\pgfpathclose%
\pgfusepath{stroke,fill}%
\end{pgfscope}%
\begin{pgfscope}%
\pgfpathrectangle{\pgfqpoint{5.292946in}{5.272501in}}{\pgfqpoint{2.177280in}{2.201755in}}%
\pgfusepath{clip}%
\pgfsetbuttcap%
\pgfsetroundjoin%
\definecolor{currentfill}{rgb}{1.000000,0.498039,0.054902}%
\pgfsetfillcolor{currentfill}%
\pgfsetlinewidth{0.481800pt}%
\definecolor{currentstroke}{rgb}{1.000000,1.000000,1.000000}%
\pgfsetstrokecolor{currentstroke}%
\pgfsetdash{}{0pt}%
\pgfpathmoveto{\pgfqpoint{6.493695in}{6.164912in}}%
\pgfpathcurveto{\pgfqpoint{6.504745in}{6.164912in}}{\pgfqpoint{6.515344in}{6.169303in}}{\pgfqpoint{6.523157in}{6.177116in}}%
\pgfpathcurveto{\pgfqpoint{6.530971in}{6.184930in}}{\pgfqpoint{6.535361in}{6.195529in}}{\pgfqpoint{6.535361in}{6.206579in}}%
\pgfpathcurveto{\pgfqpoint{6.535361in}{6.217629in}}{\pgfqpoint{6.530971in}{6.228228in}}{\pgfqpoint{6.523157in}{6.236042in}}%
\pgfpathcurveto{\pgfqpoint{6.515344in}{6.243855in}}{\pgfqpoint{6.504745in}{6.248246in}}{\pgfqpoint{6.493695in}{6.248246in}}%
\pgfpathcurveto{\pgfqpoint{6.482644in}{6.248246in}}{\pgfqpoint{6.472045in}{6.243855in}}{\pgfqpoint{6.464232in}{6.236042in}}%
\pgfpathcurveto{\pgfqpoint{6.456418in}{6.228228in}}{\pgfqpoint{6.452028in}{6.217629in}}{\pgfqpoint{6.452028in}{6.206579in}}%
\pgfpathcurveto{\pgfqpoint{6.452028in}{6.195529in}}{\pgfqpoint{6.456418in}{6.184930in}}{\pgfqpoint{6.464232in}{6.177116in}}%
\pgfpathcurveto{\pgfqpoint{6.472045in}{6.169303in}}{\pgfqpoint{6.482644in}{6.164912in}}{\pgfqpoint{6.493695in}{6.164912in}}%
\pgfpathlineto{\pgfqpoint{6.493695in}{6.164912in}}%
\pgfpathclose%
\pgfusepath{stroke,fill}%
\end{pgfscope}%
\begin{pgfscope}%
\pgfpathrectangle{\pgfqpoint{5.292946in}{5.272501in}}{\pgfqpoint{2.177280in}{2.201755in}}%
\pgfusepath{clip}%
\pgfsetbuttcap%
\pgfsetroundjoin%
\definecolor{currentfill}{rgb}{1.000000,0.498039,0.054902}%
\pgfsetfillcolor{currentfill}%
\pgfsetlinewidth{0.481800pt}%
\definecolor{currentstroke}{rgb}{1.000000,1.000000,1.000000}%
\pgfsetstrokecolor{currentstroke}%
\pgfsetdash{}{0pt}%
\pgfpathmoveto{\pgfqpoint{6.321463in}{5.831313in}}%
\pgfpathcurveto{\pgfqpoint{6.332513in}{5.831313in}}{\pgfqpoint{6.343112in}{5.835703in}}{\pgfqpoint{6.350926in}{5.843517in}}%
\pgfpathcurveto{\pgfqpoint{6.358739in}{5.851331in}}{\pgfqpoint{6.363130in}{5.861930in}}{\pgfqpoint{6.363130in}{5.872980in}}%
\pgfpathcurveto{\pgfqpoint{6.363130in}{5.884030in}}{\pgfqpoint{6.358739in}{5.894629in}}{\pgfqpoint{6.350926in}{5.902442in}}%
\pgfpathcurveto{\pgfqpoint{6.343112in}{5.910256in}}{\pgfqpoint{6.332513in}{5.914646in}}{\pgfqpoint{6.321463in}{5.914646in}}%
\pgfpathcurveto{\pgfqpoint{6.310413in}{5.914646in}}{\pgfqpoint{6.299814in}{5.910256in}}{\pgfqpoint{6.292000in}{5.902442in}}%
\pgfpathcurveto{\pgfqpoint{6.284186in}{5.894629in}}{\pgfqpoint{6.279796in}{5.884030in}}{\pgfqpoint{6.279796in}{5.872980in}}%
\pgfpathcurveto{\pgfqpoint{6.279796in}{5.861930in}}{\pgfqpoint{6.284186in}{5.851331in}}{\pgfqpoint{6.292000in}{5.843517in}}%
\pgfpathcurveto{\pgfqpoint{6.299814in}{5.835703in}}{\pgfqpoint{6.310413in}{5.831313in}}{\pgfqpoint{6.321463in}{5.831313in}}%
\pgfpathlineto{\pgfqpoint{6.321463in}{5.831313in}}%
\pgfpathclose%
\pgfusepath{stroke,fill}%
\end{pgfscope}%
\begin{pgfscope}%
\pgfpathrectangle{\pgfqpoint{5.292946in}{5.272501in}}{\pgfqpoint{2.177280in}{2.201755in}}%
\pgfusepath{clip}%
\pgfsetbuttcap%
\pgfsetroundjoin%
\definecolor{currentfill}{rgb}{1.000000,0.498039,0.054902}%
\pgfsetfillcolor{currentfill}%
\pgfsetlinewidth{0.481800pt}%
\definecolor{currentstroke}{rgb}{1.000000,1.000000,1.000000}%
\pgfsetstrokecolor{currentstroke}%
\pgfsetdash{}{0pt}%
\pgfpathmoveto{\pgfqpoint{6.120526in}{5.581114in}}%
\pgfpathcurveto{\pgfqpoint{6.131576in}{5.581114in}}{\pgfqpoint{6.142175in}{5.585504in}}{\pgfqpoint{6.149989in}{5.593317in}}%
\pgfpathcurveto{\pgfqpoint{6.157802in}{5.601131in}}{\pgfqpoint{6.162193in}{5.611730in}}{\pgfqpoint{6.162193in}{5.622780in}}%
\pgfpathcurveto{\pgfqpoint{6.162193in}{5.633830in}}{\pgfqpoint{6.157802in}{5.644429in}}{\pgfqpoint{6.149989in}{5.652243in}}%
\pgfpathcurveto{\pgfqpoint{6.142175in}{5.660057in}}{\pgfqpoint{6.131576in}{5.664447in}}{\pgfqpoint{6.120526in}{5.664447in}}%
\pgfpathcurveto{\pgfqpoint{6.109476in}{5.664447in}}{\pgfqpoint{6.098877in}{5.660057in}}{\pgfqpoint{6.091063in}{5.652243in}}%
\pgfpathcurveto{\pgfqpoint{6.083249in}{5.644429in}}{\pgfqpoint{6.078859in}{5.633830in}}{\pgfqpoint{6.078859in}{5.622780in}}%
\pgfpathcurveto{\pgfqpoint{6.078859in}{5.611730in}}{\pgfqpoint{6.083249in}{5.601131in}}{\pgfqpoint{6.091063in}{5.593317in}}%
\pgfpathcurveto{\pgfqpoint{6.098877in}{5.585504in}}{\pgfqpoint{6.109476in}{5.581114in}}{\pgfqpoint{6.120526in}{5.581114in}}%
\pgfpathlineto{\pgfqpoint{6.120526in}{5.581114in}}%
\pgfpathclose%
\pgfusepath{stroke,fill}%
\end{pgfscope}%
\begin{pgfscope}%
\pgfpathrectangle{\pgfqpoint{5.292946in}{5.272501in}}{\pgfqpoint{2.177280in}{2.201755in}}%
\pgfusepath{clip}%
\pgfsetbuttcap%
\pgfsetroundjoin%
\definecolor{currentfill}{rgb}{1.000000,0.498039,0.054902}%
\pgfsetfillcolor{currentfill}%
\pgfsetlinewidth{0.481800pt}%
\definecolor{currentstroke}{rgb}{1.000000,1.000000,1.000000}%
\pgfsetstrokecolor{currentstroke}%
\pgfsetdash{}{0pt}%
\pgfpathmoveto{\pgfqpoint{6.378873in}{5.914713in}}%
\pgfpathcurveto{\pgfqpoint{6.389924in}{5.914713in}}{\pgfqpoint{6.400523in}{5.919103in}}{\pgfqpoint{6.408336in}{5.926917in}}%
\pgfpathcurveto{\pgfqpoint{6.416150in}{5.934730in}}{\pgfqpoint{6.420540in}{5.945329in}}{\pgfqpoint{6.420540in}{5.956379in}}%
\pgfpathcurveto{\pgfqpoint{6.420540in}{5.967430in}}{\pgfqpoint{6.416150in}{5.978029in}}{\pgfqpoint{6.408336in}{5.985842in}}%
\pgfpathcurveto{\pgfqpoint{6.400523in}{5.993656in}}{\pgfqpoint{6.389924in}{5.998046in}}{\pgfqpoint{6.378873in}{5.998046in}}%
\pgfpathcurveto{\pgfqpoint{6.367823in}{5.998046in}}{\pgfqpoint{6.357224in}{5.993656in}}{\pgfqpoint{6.349411in}{5.985842in}}%
\pgfpathcurveto{\pgfqpoint{6.341597in}{5.978029in}}{\pgfqpoint{6.337207in}{5.967430in}}{\pgfqpoint{6.337207in}{5.956379in}}%
\pgfpathcurveto{\pgfqpoint{6.337207in}{5.945329in}}{\pgfqpoint{6.341597in}{5.934730in}}{\pgfqpoint{6.349411in}{5.926917in}}%
\pgfpathcurveto{\pgfqpoint{6.357224in}{5.919103in}}{\pgfqpoint{6.367823in}{5.914713in}}{\pgfqpoint{6.378873in}{5.914713in}}%
\pgfpathlineto{\pgfqpoint{6.378873in}{5.914713in}}%
\pgfpathclose%
\pgfusepath{stroke,fill}%
\end{pgfscope}%
\begin{pgfscope}%
\pgfpathrectangle{\pgfqpoint{5.292946in}{5.272501in}}{\pgfqpoint{2.177280in}{2.201755in}}%
\pgfusepath{clip}%
\pgfsetbuttcap%
\pgfsetroundjoin%
\definecolor{currentfill}{rgb}{1.000000,0.498039,0.054902}%
\pgfsetfillcolor{currentfill}%
\pgfsetlinewidth{0.481800pt}%
\definecolor{currentstroke}{rgb}{1.000000,1.000000,1.000000}%
\pgfsetstrokecolor{currentstroke}%
\pgfsetdash{}{0pt}%
\pgfpathmoveto{\pgfqpoint{6.378873in}{6.164912in}}%
\pgfpathcurveto{\pgfqpoint{6.389924in}{6.164912in}}{\pgfqpoint{6.400523in}{6.169303in}}{\pgfqpoint{6.408336in}{6.177116in}}%
\pgfpathcurveto{\pgfqpoint{6.416150in}{6.184930in}}{\pgfqpoint{6.420540in}{6.195529in}}{\pgfqpoint{6.420540in}{6.206579in}}%
\pgfpathcurveto{\pgfqpoint{6.420540in}{6.217629in}}{\pgfqpoint{6.416150in}{6.228228in}}{\pgfqpoint{6.408336in}{6.236042in}}%
\pgfpathcurveto{\pgfqpoint{6.400523in}{6.243855in}}{\pgfqpoint{6.389924in}{6.248246in}}{\pgfqpoint{6.378873in}{6.248246in}}%
\pgfpathcurveto{\pgfqpoint{6.367823in}{6.248246in}}{\pgfqpoint{6.357224in}{6.243855in}}{\pgfqpoint{6.349411in}{6.236042in}}%
\pgfpathcurveto{\pgfqpoint{6.341597in}{6.228228in}}{\pgfqpoint{6.337207in}{6.217629in}}{\pgfqpoint{6.337207in}{6.206579in}}%
\pgfpathcurveto{\pgfqpoint{6.337207in}{6.195529in}}{\pgfqpoint{6.341597in}{6.184930in}}{\pgfqpoint{6.349411in}{6.177116in}}%
\pgfpathcurveto{\pgfqpoint{6.357224in}{6.169303in}}{\pgfqpoint{6.367823in}{6.164912in}}{\pgfqpoint{6.378873in}{6.164912in}}%
\pgfpathlineto{\pgfqpoint{6.378873in}{6.164912in}}%
\pgfpathclose%
\pgfusepath{stroke,fill}%
\end{pgfscope}%
\begin{pgfscope}%
\pgfpathrectangle{\pgfqpoint{5.292946in}{5.272501in}}{\pgfqpoint{2.177280in}{2.201755in}}%
\pgfusepath{clip}%
\pgfsetbuttcap%
\pgfsetroundjoin%
\definecolor{currentfill}{rgb}{1.000000,0.498039,0.054902}%
\pgfsetfillcolor{currentfill}%
\pgfsetlinewidth{0.481800pt}%
\definecolor{currentstroke}{rgb}{1.000000,1.000000,1.000000}%
\pgfsetstrokecolor{currentstroke}%
\pgfsetdash{}{0pt}%
\pgfpathmoveto{\pgfqpoint{6.378873in}{6.081512in}}%
\pgfpathcurveto{\pgfqpoint{6.389924in}{6.081512in}}{\pgfqpoint{6.400523in}{6.085903in}}{\pgfqpoint{6.408336in}{6.093716in}}%
\pgfpathcurveto{\pgfqpoint{6.416150in}{6.101530in}}{\pgfqpoint{6.420540in}{6.112129in}}{\pgfqpoint{6.420540in}{6.123179in}}%
\pgfpathcurveto{\pgfqpoint{6.420540in}{6.134229in}}{\pgfqpoint{6.416150in}{6.144828in}}{\pgfqpoint{6.408336in}{6.152642in}}%
\pgfpathcurveto{\pgfqpoint{6.400523in}{6.160456in}}{\pgfqpoint{6.389924in}{6.164846in}}{\pgfqpoint{6.378873in}{6.164846in}}%
\pgfpathcurveto{\pgfqpoint{6.367823in}{6.164846in}}{\pgfqpoint{6.357224in}{6.160456in}}{\pgfqpoint{6.349411in}{6.152642in}}%
\pgfpathcurveto{\pgfqpoint{6.341597in}{6.144828in}}{\pgfqpoint{6.337207in}{6.134229in}}{\pgfqpoint{6.337207in}{6.123179in}}%
\pgfpathcurveto{\pgfqpoint{6.337207in}{6.112129in}}{\pgfqpoint{6.341597in}{6.101530in}}{\pgfqpoint{6.349411in}{6.093716in}}%
\pgfpathcurveto{\pgfqpoint{6.357224in}{6.085903in}}{\pgfqpoint{6.367823in}{6.081512in}}{\pgfqpoint{6.378873in}{6.081512in}}%
\pgfpathlineto{\pgfqpoint{6.378873in}{6.081512in}}%
\pgfpathclose%
\pgfusepath{stroke,fill}%
\end{pgfscope}%
\begin{pgfscope}%
\pgfpathrectangle{\pgfqpoint{5.292946in}{5.272501in}}{\pgfqpoint{2.177280in}{2.201755in}}%
\pgfusepath{clip}%
\pgfsetbuttcap%
\pgfsetroundjoin%
\definecolor{currentfill}{rgb}{1.000000,0.498039,0.054902}%
\pgfsetfillcolor{currentfill}%
\pgfsetlinewidth{0.481800pt}%
\definecolor{currentstroke}{rgb}{1.000000,1.000000,1.000000}%
\pgfsetstrokecolor{currentstroke}%
\pgfsetdash{}{0pt}%
\pgfpathmoveto{\pgfqpoint{6.407579in}{6.081512in}}%
\pgfpathcurveto{\pgfqpoint{6.418629in}{6.081512in}}{\pgfqpoint{6.429228in}{6.085903in}}{\pgfqpoint{6.437041in}{6.093716in}}%
\pgfpathcurveto{\pgfqpoint{6.444855in}{6.101530in}}{\pgfqpoint{6.449245in}{6.112129in}}{\pgfqpoint{6.449245in}{6.123179in}}%
\pgfpathcurveto{\pgfqpoint{6.449245in}{6.134229in}}{\pgfqpoint{6.444855in}{6.144828in}}{\pgfqpoint{6.437041in}{6.152642in}}%
\pgfpathcurveto{\pgfqpoint{6.429228in}{6.160456in}}{\pgfqpoint{6.418629in}{6.164846in}}{\pgfqpoint{6.407579in}{6.164846in}}%
\pgfpathcurveto{\pgfqpoint{6.396529in}{6.164846in}}{\pgfqpoint{6.385930in}{6.160456in}}{\pgfqpoint{6.378116in}{6.152642in}}%
\pgfpathcurveto{\pgfqpoint{6.370302in}{6.144828in}}{\pgfqpoint{6.365912in}{6.134229in}}{\pgfqpoint{6.365912in}{6.123179in}}%
\pgfpathcurveto{\pgfqpoint{6.365912in}{6.112129in}}{\pgfqpoint{6.370302in}{6.101530in}}{\pgfqpoint{6.378116in}{6.093716in}}%
\pgfpathcurveto{\pgfqpoint{6.385930in}{6.085903in}}{\pgfqpoint{6.396529in}{6.081512in}}{\pgfqpoint{6.407579in}{6.081512in}}%
\pgfpathlineto{\pgfqpoint{6.407579in}{6.081512in}}%
\pgfpathclose%
\pgfusepath{stroke,fill}%
\end{pgfscope}%
\begin{pgfscope}%
\pgfpathrectangle{\pgfqpoint{5.292946in}{5.272501in}}{\pgfqpoint{2.177280in}{2.201755in}}%
\pgfusepath{clip}%
\pgfsetbuttcap%
\pgfsetroundjoin%
\definecolor{currentfill}{rgb}{1.000000,0.498039,0.054902}%
\pgfsetfillcolor{currentfill}%
\pgfsetlinewidth{0.481800pt}%
\definecolor{currentstroke}{rgb}{1.000000,1.000000,1.000000}%
\pgfsetstrokecolor{currentstroke}%
\pgfsetdash{}{0pt}%
\pgfpathmoveto{\pgfqpoint{6.034410in}{5.747913in}}%
\pgfpathcurveto{\pgfqpoint{6.045460in}{5.747913in}}{\pgfqpoint{6.056059in}{5.752303in}}{\pgfqpoint{6.063873in}{5.760117in}}%
\pgfpathcurveto{\pgfqpoint{6.071686in}{5.767931in}}{\pgfqpoint{6.076077in}{5.778530in}}{\pgfqpoint{6.076077in}{5.789580in}}%
\pgfpathcurveto{\pgfqpoint{6.076077in}{5.800630in}}{\pgfqpoint{6.071686in}{5.811229in}}{\pgfqpoint{6.063873in}{5.819043in}}%
\pgfpathcurveto{\pgfqpoint{6.056059in}{5.826856in}}{\pgfqpoint{6.045460in}{5.831247in}}{\pgfqpoint{6.034410in}{5.831247in}}%
\pgfpathcurveto{\pgfqpoint{6.023360in}{5.831247in}}{\pgfqpoint{6.012761in}{5.826856in}}{\pgfqpoint{6.004947in}{5.819043in}}%
\pgfpathcurveto{\pgfqpoint{5.997134in}{5.811229in}}{\pgfqpoint{5.992743in}{5.800630in}}{\pgfqpoint{5.992743in}{5.789580in}}%
\pgfpathcurveto{\pgfqpoint{5.992743in}{5.778530in}}{\pgfqpoint{5.997134in}{5.767931in}}{\pgfqpoint{6.004947in}{5.760117in}}%
\pgfpathcurveto{\pgfqpoint{6.012761in}{5.752303in}}{\pgfqpoint{6.023360in}{5.747913in}}{\pgfqpoint{6.034410in}{5.747913in}}%
\pgfpathlineto{\pgfqpoint{6.034410in}{5.747913in}}%
\pgfpathclose%
\pgfusepath{stroke,fill}%
\end{pgfscope}%
\begin{pgfscope}%
\pgfpathrectangle{\pgfqpoint{5.292946in}{5.272501in}}{\pgfqpoint{2.177280in}{2.201755in}}%
\pgfusepath{clip}%
\pgfsetbuttcap%
\pgfsetroundjoin%
\definecolor{currentfill}{rgb}{1.000000,0.498039,0.054902}%
\pgfsetfillcolor{currentfill}%
\pgfsetlinewidth{0.481800pt}%
\definecolor{currentstroke}{rgb}{1.000000,1.000000,1.000000}%
\pgfsetstrokecolor{currentstroke}%
\pgfsetdash{}{0pt}%
\pgfpathmoveto{\pgfqpoint{6.350168in}{5.998113in}}%
\pgfpathcurveto{\pgfqpoint{6.361218in}{5.998113in}}{\pgfqpoint{6.371817in}{6.002503in}}{\pgfqpoint{6.379631in}{6.010317in}}%
\pgfpathcurveto{\pgfqpoint{6.387445in}{6.018130in}}{\pgfqpoint{6.391835in}{6.028729in}}{\pgfqpoint{6.391835in}{6.039779in}}%
\pgfpathcurveto{\pgfqpoint{6.391835in}{6.050829in}}{\pgfqpoint{6.387445in}{6.061428in}}{\pgfqpoint{6.379631in}{6.069242in}}%
\pgfpathcurveto{\pgfqpoint{6.371817in}{6.077056in}}{\pgfqpoint{6.361218in}{6.081446in}}{\pgfqpoint{6.350168in}{6.081446in}}%
\pgfpathcurveto{\pgfqpoint{6.339118in}{6.081446in}}{\pgfqpoint{6.328519in}{6.077056in}}{\pgfqpoint{6.320705in}{6.069242in}}%
\pgfpathcurveto{\pgfqpoint{6.312892in}{6.061428in}}{\pgfqpoint{6.308501in}{6.050829in}}{\pgfqpoint{6.308501in}{6.039779in}}%
\pgfpathcurveto{\pgfqpoint{6.308501in}{6.028729in}}{\pgfqpoint{6.312892in}{6.018130in}}{\pgfqpoint{6.320705in}{6.010317in}}%
\pgfpathcurveto{\pgfqpoint{6.328519in}{6.002503in}}{\pgfqpoint{6.339118in}{5.998113in}}{\pgfqpoint{6.350168in}{5.998113in}}%
\pgfpathlineto{\pgfqpoint{6.350168in}{5.998113in}}%
\pgfpathclose%
\pgfusepath{stroke,fill}%
\end{pgfscope}%
\begin{pgfscope}%
\pgfpathrectangle{\pgfqpoint{5.292946in}{5.272501in}}{\pgfqpoint{2.177280in}{2.201755in}}%
\pgfusepath{clip}%
\pgfsetbuttcap%
\pgfsetroundjoin%
\definecolor{currentfill}{rgb}{0.172549,0.627451,0.172549}%
\pgfsetfillcolor{currentfill}%
\pgfsetlinewidth{0.481800pt}%
\definecolor{currentstroke}{rgb}{1.000000,1.000000,1.000000}%
\pgfsetstrokecolor{currentstroke}%
\pgfsetdash{}{0pt}%
\pgfpathmoveto{\pgfqpoint{6.895569in}{6.415112in}}%
\pgfpathcurveto{\pgfqpoint{6.906619in}{6.415112in}}{\pgfqpoint{6.917218in}{6.419502in}}{\pgfqpoint{6.925031in}{6.427316in}}%
\pgfpathcurveto{\pgfqpoint{6.932845in}{6.435129in}}{\pgfqpoint{6.937235in}{6.445728in}}{\pgfqpoint{6.937235in}{6.456778in}}%
\pgfpathcurveto{\pgfqpoint{6.937235in}{6.467828in}}{\pgfqpoint{6.932845in}{6.478428in}}{\pgfqpoint{6.925031in}{6.486241in}}%
\pgfpathcurveto{\pgfqpoint{6.917218in}{6.494055in}}{\pgfqpoint{6.906619in}{6.498445in}}{\pgfqpoint{6.895569in}{6.498445in}}%
\pgfpathcurveto{\pgfqpoint{6.884518in}{6.498445in}}{\pgfqpoint{6.873919in}{6.494055in}}{\pgfqpoint{6.866106in}{6.486241in}}%
\pgfpathcurveto{\pgfqpoint{6.858292in}{6.478428in}}{\pgfqpoint{6.853902in}{6.467828in}}{\pgfqpoint{6.853902in}{6.456778in}}%
\pgfpathcurveto{\pgfqpoint{6.853902in}{6.445728in}}{\pgfqpoint{6.858292in}{6.435129in}}{\pgfqpoint{6.866106in}{6.427316in}}%
\pgfpathcurveto{\pgfqpoint{6.873919in}{6.419502in}}{\pgfqpoint{6.884518in}{6.415112in}}{\pgfqpoint{6.895569in}{6.415112in}}%
\pgfpathlineto{\pgfqpoint{6.895569in}{6.415112in}}%
\pgfpathclose%
\pgfusepath{stroke,fill}%
\end{pgfscope}%
\begin{pgfscope}%
\pgfpathrectangle{\pgfqpoint{5.292946in}{5.272501in}}{\pgfqpoint{2.177280in}{2.201755in}}%
\pgfusepath{clip}%
\pgfsetbuttcap%
\pgfsetroundjoin%
\definecolor{currentfill}{rgb}{0.172549,0.627451,0.172549}%
\pgfsetfillcolor{currentfill}%
\pgfsetlinewidth{0.481800pt}%
\definecolor{currentstroke}{rgb}{1.000000,1.000000,1.000000}%
\pgfsetstrokecolor{currentstroke}%
\pgfsetdash{}{0pt}%
\pgfpathmoveto{\pgfqpoint{6.637221in}{5.914713in}}%
\pgfpathcurveto{\pgfqpoint{6.648271in}{5.914713in}}{\pgfqpoint{6.658870in}{5.919103in}}{\pgfqpoint{6.666684in}{5.926917in}}%
\pgfpathcurveto{\pgfqpoint{6.674497in}{5.934730in}}{\pgfqpoint{6.678888in}{5.945329in}}{\pgfqpoint{6.678888in}{5.956379in}}%
\pgfpathcurveto{\pgfqpoint{6.678888in}{5.967430in}}{\pgfqpoint{6.674497in}{5.978029in}}{\pgfqpoint{6.666684in}{5.985842in}}%
\pgfpathcurveto{\pgfqpoint{6.658870in}{5.993656in}}{\pgfqpoint{6.648271in}{5.998046in}}{\pgfqpoint{6.637221in}{5.998046in}}%
\pgfpathcurveto{\pgfqpoint{6.626171in}{5.998046in}}{\pgfqpoint{6.615572in}{5.993656in}}{\pgfqpoint{6.607758in}{5.985842in}}%
\pgfpathcurveto{\pgfqpoint{6.599945in}{5.978029in}}{\pgfqpoint{6.595554in}{5.967430in}}{\pgfqpoint{6.595554in}{5.956379in}}%
\pgfpathcurveto{\pgfqpoint{6.595554in}{5.945329in}}{\pgfqpoint{6.599945in}{5.934730in}}{\pgfqpoint{6.607758in}{5.926917in}}%
\pgfpathcurveto{\pgfqpoint{6.615572in}{5.919103in}}{\pgfqpoint{6.626171in}{5.914713in}}{\pgfqpoint{6.637221in}{5.914713in}}%
\pgfpathlineto{\pgfqpoint{6.637221in}{5.914713in}}%
\pgfpathclose%
\pgfusepath{stroke,fill}%
\end{pgfscope}%
\begin{pgfscope}%
\pgfpathrectangle{\pgfqpoint{5.292946in}{5.272501in}}{\pgfqpoint{2.177280in}{2.201755in}}%
\pgfusepath{clip}%
\pgfsetbuttcap%
\pgfsetroundjoin%
\definecolor{currentfill}{rgb}{0.172549,0.627451,0.172549}%
\pgfsetfillcolor{currentfill}%
\pgfsetlinewidth{0.481800pt}%
\definecolor{currentstroke}{rgb}{1.000000,1.000000,1.000000}%
\pgfsetstrokecolor{currentstroke}%
\pgfsetdash{}{0pt}%
\pgfpathmoveto{\pgfqpoint{6.866863in}{6.164912in}}%
\pgfpathcurveto{\pgfqpoint{6.877913in}{6.164912in}}{\pgfqpoint{6.888512in}{6.169303in}}{\pgfqpoint{6.896326in}{6.177116in}}%
\pgfpathcurveto{\pgfqpoint{6.904140in}{6.184930in}}{\pgfqpoint{6.908530in}{6.195529in}}{\pgfqpoint{6.908530in}{6.206579in}}%
\pgfpathcurveto{\pgfqpoint{6.908530in}{6.217629in}}{\pgfqpoint{6.904140in}{6.228228in}}{\pgfqpoint{6.896326in}{6.236042in}}%
\pgfpathcurveto{\pgfqpoint{6.888512in}{6.243855in}}{\pgfqpoint{6.877913in}{6.248246in}}{\pgfqpoint{6.866863in}{6.248246in}}%
\pgfpathcurveto{\pgfqpoint{6.855813in}{6.248246in}}{\pgfqpoint{6.845214in}{6.243855in}}{\pgfqpoint{6.837401in}{6.236042in}}%
\pgfpathcurveto{\pgfqpoint{6.829587in}{6.228228in}}{\pgfqpoint{6.825197in}{6.217629in}}{\pgfqpoint{6.825197in}{6.206579in}}%
\pgfpathcurveto{\pgfqpoint{6.825197in}{6.195529in}}{\pgfqpoint{6.829587in}{6.184930in}}{\pgfqpoint{6.837401in}{6.177116in}}%
\pgfpathcurveto{\pgfqpoint{6.845214in}{6.169303in}}{\pgfqpoint{6.855813in}{6.164912in}}{\pgfqpoint{6.866863in}{6.164912in}}%
\pgfpathlineto{\pgfqpoint{6.866863in}{6.164912in}}%
\pgfpathclose%
\pgfusepath{stroke,fill}%
\end{pgfscope}%
\begin{pgfscope}%
\pgfpathrectangle{\pgfqpoint{5.292946in}{5.272501in}}{\pgfqpoint{2.177280in}{2.201755in}}%
\pgfusepath{clip}%
\pgfsetbuttcap%
\pgfsetroundjoin%
\definecolor{currentfill}{rgb}{0.172549,0.627451,0.172549}%
\pgfsetfillcolor{currentfill}%
\pgfsetlinewidth{0.481800pt}%
\definecolor{currentstroke}{rgb}{1.000000,1.000000,1.000000}%
\pgfsetstrokecolor{currentstroke}%
\pgfsetdash{}{0pt}%
\pgfpathmoveto{\pgfqpoint{6.780747in}{6.081512in}}%
\pgfpathcurveto{\pgfqpoint{6.791798in}{6.081512in}}{\pgfqpoint{6.802397in}{6.085903in}}{\pgfqpoint{6.810210in}{6.093716in}}%
\pgfpathcurveto{\pgfqpoint{6.818024in}{6.101530in}}{\pgfqpoint{6.822414in}{6.112129in}}{\pgfqpoint{6.822414in}{6.123179in}}%
\pgfpathcurveto{\pgfqpoint{6.822414in}{6.134229in}}{\pgfqpoint{6.818024in}{6.144828in}}{\pgfqpoint{6.810210in}{6.152642in}}%
\pgfpathcurveto{\pgfqpoint{6.802397in}{6.160456in}}{\pgfqpoint{6.791798in}{6.164846in}}{\pgfqpoint{6.780747in}{6.164846in}}%
\pgfpathcurveto{\pgfqpoint{6.769697in}{6.164846in}}{\pgfqpoint{6.759098in}{6.160456in}}{\pgfqpoint{6.751285in}{6.152642in}}%
\pgfpathcurveto{\pgfqpoint{6.743471in}{6.144828in}}{\pgfqpoint{6.739081in}{6.134229in}}{\pgfqpoint{6.739081in}{6.123179in}}%
\pgfpathcurveto{\pgfqpoint{6.739081in}{6.112129in}}{\pgfqpoint{6.743471in}{6.101530in}}{\pgfqpoint{6.751285in}{6.093716in}}%
\pgfpathcurveto{\pgfqpoint{6.759098in}{6.085903in}}{\pgfqpoint{6.769697in}{6.081512in}}{\pgfqpoint{6.780747in}{6.081512in}}%
\pgfpathlineto{\pgfqpoint{6.780747in}{6.081512in}}%
\pgfpathclose%
\pgfusepath{stroke,fill}%
\end{pgfscope}%
\begin{pgfscope}%
\pgfpathrectangle{\pgfqpoint{5.292946in}{5.272501in}}{\pgfqpoint{2.177280in}{2.201755in}}%
\pgfusepath{clip}%
\pgfsetbuttcap%
\pgfsetroundjoin%
\definecolor{currentfill}{rgb}{0.172549,0.627451,0.172549}%
\pgfsetfillcolor{currentfill}%
\pgfsetlinewidth{0.481800pt}%
\definecolor{currentstroke}{rgb}{1.000000,1.000000,1.000000}%
\pgfsetstrokecolor{currentstroke}%
\pgfsetdash{}{0pt}%
\pgfpathmoveto{\pgfqpoint{6.838158in}{6.164912in}}%
\pgfpathcurveto{\pgfqpoint{6.849208in}{6.164912in}}{\pgfqpoint{6.859807in}{6.169303in}}{\pgfqpoint{6.867621in}{6.177116in}}%
\pgfpathcurveto{\pgfqpoint{6.875434in}{6.184930in}}{\pgfqpoint{6.879825in}{6.195529in}}{\pgfqpoint{6.879825in}{6.206579in}}%
\pgfpathcurveto{\pgfqpoint{6.879825in}{6.217629in}}{\pgfqpoint{6.875434in}{6.228228in}}{\pgfqpoint{6.867621in}{6.236042in}}%
\pgfpathcurveto{\pgfqpoint{6.859807in}{6.243855in}}{\pgfqpoint{6.849208in}{6.248246in}}{\pgfqpoint{6.838158in}{6.248246in}}%
\pgfpathcurveto{\pgfqpoint{6.827108in}{6.248246in}}{\pgfqpoint{6.816509in}{6.243855in}}{\pgfqpoint{6.808695in}{6.236042in}}%
\pgfpathcurveto{\pgfqpoint{6.800882in}{6.228228in}}{\pgfqpoint{6.796491in}{6.217629in}}{\pgfqpoint{6.796491in}{6.206579in}}%
\pgfpathcurveto{\pgfqpoint{6.796491in}{6.195529in}}{\pgfqpoint{6.800882in}{6.184930in}}{\pgfqpoint{6.808695in}{6.177116in}}%
\pgfpathcurveto{\pgfqpoint{6.816509in}{6.169303in}}{\pgfqpoint{6.827108in}{6.164912in}}{\pgfqpoint{6.838158in}{6.164912in}}%
\pgfpathlineto{\pgfqpoint{6.838158in}{6.164912in}}%
\pgfpathclose%
\pgfusepath{stroke,fill}%
\end{pgfscope}%
\begin{pgfscope}%
\pgfpathrectangle{\pgfqpoint{5.292946in}{5.272501in}}{\pgfqpoint{2.177280in}{2.201755in}}%
\pgfusepath{clip}%
\pgfsetbuttcap%
\pgfsetroundjoin%
\definecolor{currentfill}{rgb}{0.172549,0.627451,0.172549}%
\pgfsetfillcolor{currentfill}%
\pgfsetlinewidth{0.481800pt}%
\definecolor{currentstroke}{rgb}{1.000000,1.000000,1.000000}%
\pgfsetstrokecolor{currentstroke}%
\pgfsetdash{}{0pt}%
\pgfpathmoveto{\pgfqpoint{7.067800in}{6.164912in}}%
\pgfpathcurveto{\pgfqpoint{7.078850in}{6.164912in}}{\pgfqpoint{7.089449in}{6.169303in}}{\pgfqpoint{7.097263in}{6.177116in}}%
\pgfpathcurveto{\pgfqpoint{7.105077in}{6.184930in}}{\pgfqpoint{7.109467in}{6.195529in}}{\pgfqpoint{7.109467in}{6.206579in}}%
\pgfpathcurveto{\pgfqpoint{7.109467in}{6.217629in}}{\pgfqpoint{7.105077in}{6.228228in}}{\pgfqpoint{7.097263in}{6.236042in}}%
\pgfpathcurveto{\pgfqpoint{7.089449in}{6.243855in}}{\pgfqpoint{7.078850in}{6.248246in}}{\pgfqpoint{7.067800in}{6.248246in}}%
\pgfpathcurveto{\pgfqpoint{7.056750in}{6.248246in}}{\pgfqpoint{7.046151in}{6.243855in}}{\pgfqpoint{7.038338in}{6.236042in}}%
\pgfpathcurveto{\pgfqpoint{7.030524in}{6.228228in}}{\pgfqpoint{7.026134in}{6.217629in}}{\pgfqpoint{7.026134in}{6.206579in}}%
\pgfpathcurveto{\pgfqpoint{7.026134in}{6.195529in}}{\pgfqpoint{7.030524in}{6.184930in}}{\pgfqpoint{7.038338in}{6.177116in}}%
\pgfpathcurveto{\pgfqpoint{7.046151in}{6.169303in}}{\pgfqpoint{7.056750in}{6.164912in}}{\pgfqpoint{7.067800in}{6.164912in}}%
\pgfpathlineto{\pgfqpoint{7.067800in}{6.164912in}}%
\pgfpathclose%
\pgfusepath{stroke,fill}%
\end{pgfscope}%
\begin{pgfscope}%
\pgfpathrectangle{\pgfqpoint{5.292946in}{5.272501in}}{\pgfqpoint{2.177280in}{2.201755in}}%
\pgfusepath{clip}%
\pgfsetbuttcap%
\pgfsetroundjoin%
\definecolor{currentfill}{rgb}{0.172549,0.627451,0.172549}%
\pgfsetfillcolor{currentfill}%
\pgfsetlinewidth{0.481800pt}%
\definecolor{currentstroke}{rgb}{1.000000,1.000000,1.000000}%
\pgfsetstrokecolor{currentstroke}%
\pgfsetdash{}{0pt}%
\pgfpathmoveto{\pgfqpoint{6.464989in}{5.747913in}}%
\pgfpathcurveto{\pgfqpoint{6.476039in}{5.747913in}}{\pgfqpoint{6.486638in}{5.752303in}}{\pgfqpoint{6.494452in}{5.760117in}}%
\pgfpathcurveto{\pgfqpoint{6.502266in}{5.767931in}}{\pgfqpoint{6.506656in}{5.778530in}}{\pgfqpoint{6.506656in}{5.789580in}}%
\pgfpathcurveto{\pgfqpoint{6.506656in}{5.800630in}}{\pgfqpoint{6.502266in}{5.811229in}}{\pgfqpoint{6.494452in}{5.819043in}}%
\pgfpathcurveto{\pgfqpoint{6.486638in}{5.826856in}}{\pgfqpoint{6.476039in}{5.831247in}}{\pgfqpoint{6.464989in}{5.831247in}}%
\pgfpathcurveto{\pgfqpoint{6.453939in}{5.831247in}}{\pgfqpoint{6.443340in}{5.826856in}}{\pgfqpoint{6.435527in}{5.819043in}}%
\pgfpathcurveto{\pgfqpoint{6.427713in}{5.811229in}}{\pgfqpoint{6.423323in}{5.800630in}}{\pgfqpoint{6.423323in}{5.789580in}}%
\pgfpathcurveto{\pgfqpoint{6.423323in}{5.778530in}}{\pgfqpoint{6.427713in}{5.767931in}}{\pgfqpoint{6.435527in}{5.760117in}}%
\pgfpathcurveto{\pgfqpoint{6.443340in}{5.752303in}}{\pgfqpoint{6.453939in}{5.747913in}}{\pgfqpoint{6.464989in}{5.747913in}}%
\pgfpathlineto{\pgfqpoint{6.464989in}{5.747913in}}%
\pgfpathclose%
\pgfusepath{stroke,fill}%
\end{pgfscope}%
\begin{pgfscope}%
\pgfpathrectangle{\pgfqpoint{5.292946in}{5.272501in}}{\pgfqpoint{2.177280in}{2.201755in}}%
\pgfusepath{clip}%
\pgfsetbuttcap%
\pgfsetroundjoin%
\definecolor{currentfill}{rgb}{0.172549,0.627451,0.172549}%
\pgfsetfillcolor{currentfill}%
\pgfsetlinewidth{0.481800pt}%
\definecolor{currentstroke}{rgb}{1.000000,1.000000,1.000000}%
\pgfsetstrokecolor{currentstroke}%
\pgfsetdash{}{0pt}%
\pgfpathmoveto{\pgfqpoint{6.981684in}{6.081512in}}%
\pgfpathcurveto{\pgfqpoint{6.992735in}{6.081512in}}{\pgfqpoint{7.003334in}{6.085903in}}{\pgfqpoint{7.011147in}{6.093716in}}%
\pgfpathcurveto{\pgfqpoint{7.018961in}{6.101530in}}{\pgfqpoint{7.023351in}{6.112129in}}{\pgfqpoint{7.023351in}{6.123179in}}%
\pgfpathcurveto{\pgfqpoint{7.023351in}{6.134229in}}{\pgfqpoint{7.018961in}{6.144828in}}{\pgfqpoint{7.011147in}{6.152642in}}%
\pgfpathcurveto{\pgfqpoint{7.003334in}{6.160456in}}{\pgfqpoint{6.992735in}{6.164846in}}{\pgfqpoint{6.981684in}{6.164846in}}%
\pgfpathcurveto{\pgfqpoint{6.970634in}{6.164846in}}{\pgfqpoint{6.960035in}{6.160456in}}{\pgfqpoint{6.952222in}{6.152642in}}%
\pgfpathcurveto{\pgfqpoint{6.944408in}{6.144828in}}{\pgfqpoint{6.940018in}{6.134229in}}{\pgfqpoint{6.940018in}{6.123179in}}%
\pgfpathcurveto{\pgfqpoint{6.940018in}{6.112129in}}{\pgfqpoint{6.944408in}{6.101530in}}{\pgfqpoint{6.952222in}{6.093716in}}%
\pgfpathcurveto{\pgfqpoint{6.960035in}{6.085903in}}{\pgfqpoint{6.970634in}{6.081512in}}{\pgfqpoint{6.981684in}{6.081512in}}%
\pgfpathlineto{\pgfqpoint{6.981684in}{6.081512in}}%
\pgfpathclose%
\pgfusepath{stroke,fill}%
\end{pgfscope}%
\begin{pgfscope}%
\pgfpathrectangle{\pgfqpoint{5.292946in}{5.272501in}}{\pgfqpoint{2.177280in}{2.201755in}}%
\pgfusepath{clip}%
\pgfsetbuttcap%
\pgfsetroundjoin%
\definecolor{currentfill}{rgb}{0.172549,0.627451,0.172549}%
\pgfsetfillcolor{currentfill}%
\pgfsetlinewidth{0.481800pt}%
\definecolor{currentstroke}{rgb}{1.000000,1.000000,1.000000}%
\pgfsetstrokecolor{currentstroke}%
\pgfsetdash{}{0pt}%
\pgfpathmoveto{\pgfqpoint{6.838158in}{5.747913in}}%
\pgfpathcurveto{\pgfqpoint{6.849208in}{5.747913in}}{\pgfqpoint{6.859807in}{5.752303in}}{\pgfqpoint{6.867621in}{5.760117in}}%
\pgfpathcurveto{\pgfqpoint{6.875434in}{5.767931in}}{\pgfqpoint{6.879825in}{5.778530in}}{\pgfqpoint{6.879825in}{5.789580in}}%
\pgfpathcurveto{\pgfqpoint{6.879825in}{5.800630in}}{\pgfqpoint{6.875434in}{5.811229in}}{\pgfqpoint{6.867621in}{5.819043in}}%
\pgfpathcurveto{\pgfqpoint{6.859807in}{5.826856in}}{\pgfqpoint{6.849208in}{5.831247in}}{\pgfqpoint{6.838158in}{5.831247in}}%
\pgfpathcurveto{\pgfqpoint{6.827108in}{5.831247in}}{\pgfqpoint{6.816509in}{5.826856in}}{\pgfqpoint{6.808695in}{5.819043in}}%
\pgfpathcurveto{\pgfqpoint{6.800882in}{5.811229in}}{\pgfqpoint{6.796491in}{5.800630in}}{\pgfqpoint{6.796491in}{5.789580in}}%
\pgfpathcurveto{\pgfqpoint{6.796491in}{5.778530in}}{\pgfqpoint{6.800882in}{5.767931in}}{\pgfqpoint{6.808695in}{5.760117in}}%
\pgfpathcurveto{\pgfqpoint{6.816509in}{5.752303in}}{\pgfqpoint{6.827108in}{5.747913in}}{\pgfqpoint{6.838158in}{5.747913in}}%
\pgfpathlineto{\pgfqpoint{6.838158in}{5.747913in}}%
\pgfpathclose%
\pgfusepath{stroke,fill}%
\end{pgfscope}%
\begin{pgfscope}%
\pgfpathrectangle{\pgfqpoint{5.292946in}{5.272501in}}{\pgfqpoint{2.177280in}{2.201755in}}%
\pgfusepath{clip}%
\pgfsetbuttcap%
\pgfsetroundjoin%
\definecolor{currentfill}{rgb}{0.172549,0.627451,0.172549}%
\pgfsetfillcolor{currentfill}%
\pgfsetlinewidth{0.481800pt}%
\definecolor{currentstroke}{rgb}{1.000000,1.000000,1.000000}%
\pgfsetstrokecolor{currentstroke}%
\pgfsetdash{}{0pt}%
\pgfpathmoveto{\pgfqpoint{6.924274in}{6.665311in}}%
\pgfpathcurveto{\pgfqpoint{6.935324in}{6.665311in}}{\pgfqpoint{6.945923in}{6.669701in}}{\pgfqpoint{6.953737in}{6.677515in}}%
\pgfpathcurveto{\pgfqpoint{6.961550in}{6.685329in}}{\pgfqpoint{6.965941in}{6.695928in}}{\pgfqpoint{6.965941in}{6.706978in}}%
\pgfpathcurveto{\pgfqpoint{6.965941in}{6.718028in}}{\pgfqpoint{6.961550in}{6.728627in}}{\pgfqpoint{6.953737in}{6.736441in}}%
\pgfpathcurveto{\pgfqpoint{6.945923in}{6.744254in}}{\pgfqpoint{6.935324in}{6.748644in}}{\pgfqpoint{6.924274in}{6.748644in}}%
\pgfpathcurveto{\pgfqpoint{6.913224in}{6.748644in}}{\pgfqpoint{6.902625in}{6.744254in}}{\pgfqpoint{6.894811in}{6.736441in}}%
\pgfpathcurveto{\pgfqpoint{6.886997in}{6.728627in}}{\pgfqpoint{6.882607in}{6.718028in}}{\pgfqpoint{6.882607in}{6.706978in}}%
\pgfpathcurveto{\pgfqpoint{6.882607in}{6.695928in}}{\pgfqpoint{6.886997in}{6.685329in}}{\pgfqpoint{6.894811in}{6.677515in}}%
\pgfpathcurveto{\pgfqpoint{6.902625in}{6.669701in}}{\pgfqpoint{6.913224in}{6.665311in}}{\pgfqpoint{6.924274in}{6.665311in}}%
\pgfpathlineto{\pgfqpoint{6.924274in}{6.665311in}}%
\pgfpathclose%
\pgfusepath{stroke,fill}%
\end{pgfscope}%
\begin{pgfscope}%
\pgfpathrectangle{\pgfqpoint{5.292946in}{5.272501in}}{\pgfqpoint{2.177280in}{2.201755in}}%
\pgfusepath{clip}%
\pgfsetbuttcap%
\pgfsetroundjoin%
\definecolor{currentfill}{rgb}{0.172549,0.627451,0.172549}%
\pgfsetfillcolor{currentfill}%
\pgfsetlinewidth{0.481800pt}%
\definecolor{currentstroke}{rgb}{1.000000,1.000000,1.000000}%
\pgfsetstrokecolor{currentstroke}%
\pgfsetdash{}{0pt}%
\pgfpathmoveto{\pgfqpoint{6.637221in}{6.331712in}}%
\pgfpathcurveto{\pgfqpoint{6.648271in}{6.331712in}}{\pgfqpoint{6.658870in}{6.336102in}}{\pgfqpoint{6.666684in}{6.343916in}}%
\pgfpathcurveto{\pgfqpoint{6.674497in}{6.351729in}}{\pgfqpoint{6.678888in}{6.362328in}}{\pgfqpoint{6.678888in}{6.373379in}}%
\pgfpathcurveto{\pgfqpoint{6.678888in}{6.384429in}}{\pgfqpoint{6.674497in}{6.395028in}}{\pgfqpoint{6.666684in}{6.402841in}}%
\pgfpathcurveto{\pgfqpoint{6.658870in}{6.410655in}}{\pgfqpoint{6.648271in}{6.415045in}}{\pgfqpoint{6.637221in}{6.415045in}}%
\pgfpathcurveto{\pgfqpoint{6.626171in}{6.415045in}}{\pgfqpoint{6.615572in}{6.410655in}}{\pgfqpoint{6.607758in}{6.402841in}}%
\pgfpathcurveto{\pgfqpoint{6.599945in}{6.395028in}}{\pgfqpoint{6.595554in}{6.384429in}}{\pgfqpoint{6.595554in}{6.373379in}}%
\pgfpathcurveto{\pgfqpoint{6.595554in}{6.362328in}}{\pgfqpoint{6.599945in}{6.351729in}}{\pgfqpoint{6.607758in}{6.343916in}}%
\pgfpathcurveto{\pgfqpoint{6.615572in}{6.336102in}}{\pgfqpoint{6.626171in}{6.331712in}}{\pgfqpoint{6.637221in}{6.331712in}}%
\pgfpathlineto{\pgfqpoint{6.637221in}{6.331712in}}%
\pgfpathclose%
\pgfusepath{stroke,fill}%
\end{pgfscope}%
\begin{pgfscope}%
\pgfpathrectangle{\pgfqpoint{5.292946in}{5.272501in}}{\pgfqpoint{2.177280in}{2.201755in}}%
\pgfusepath{clip}%
\pgfsetbuttcap%
\pgfsetroundjoin%
\definecolor{currentfill}{rgb}{0.172549,0.627451,0.172549}%
\pgfsetfillcolor{currentfill}%
\pgfsetlinewidth{0.481800pt}%
\definecolor{currentstroke}{rgb}{1.000000,1.000000,1.000000}%
\pgfsetstrokecolor{currentstroke}%
\pgfsetdash{}{0pt}%
\pgfpathmoveto{\pgfqpoint{6.694632in}{5.914713in}}%
\pgfpathcurveto{\pgfqpoint{6.705682in}{5.914713in}}{\pgfqpoint{6.716281in}{5.919103in}}{\pgfqpoint{6.724094in}{5.926917in}}%
\pgfpathcurveto{\pgfqpoint{6.731908in}{5.934730in}}{\pgfqpoint{6.736298in}{5.945329in}}{\pgfqpoint{6.736298in}{5.956379in}}%
\pgfpathcurveto{\pgfqpoint{6.736298in}{5.967430in}}{\pgfqpoint{6.731908in}{5.978029in}}{\pgfqpoint{6.724094in}{5.985842in}}%
\pgfpathcurveto{\pgfqpoint{6.716281in}{5.993656in}}{\pgfqpoint{6.705682in}{5.998046in}}{\pgfqpoint{6.694632in}{5.998046in}}%
\pgfpathcurveto{\pgfqpoint{6.683581in}{5.998046in}}{\pgfqpoint{6.672982in}{5.993656in}}{\pgfqpoint{6.665169in}{5.985842in}}%
\pgfpathcurveto{\pgfqpoint{6.657355in}{5.978029in}}{\pgfqpoint{6.652965in}{5.967430in}}{\pgfqpoint{6.652965in}{5.956379in}}%
\pgfpathcurveto{\pgfqpoint{6.652965in}{5.945329in}}{\pgfqpoint{6.657355in}{5.934730in}}{\pgfqpoint{6.665169in}{5.926917in}}%
\pgfpathcurveto{\pgfqpoint{6.672982in}{5.919103in}}{\pgfqpoint{6.683581in}{5.914713in}}{\pgfqpoint{6.694632in}{5.914713in}}%
\pgfpathlineto{\pgfqpoint{6.694632in}{5.914713in}}%
\pgfpathclose%
\pgfusepath{stroke,fill}%
\end{pgfscope}%
\begin{pgfscope}%
\pgfpathrectangle{\pgfqpoint{5.292946in}{5.272501in}}{\pgfqpoint{2.177280in}{2.201755in}}%
\pgfusepath{clip}%
\pgfsetbuttcap%
\pgfsetroundjoin%
\definecolor{currentfill}{rgb}{0.172549,0.627451,0.172549}%
\pgfsetfillcolor{currentfill}%
\pgfsetlinewidth{0.481800pt}%
\definecolor{currentstroke}{rgb}{1.000000,1.000000,1.000000}%
\pgfsetstrokecolor{currentstroke}%
\pgfsetdash{}{0pt}%
\pgfpathmoveto{\pgfqpoint{6.752042in}{6.164912in}}%
\pgfpathcurveto{\pgfqpoint{6.763092in}{6.164912in}}{\pgfqpoint{6.773691in}{6.169303in}}{\pgfqpoint{6.781505in}{6.177116in}}%
\pgfpathcurveto{\pgfqpoint{6.789319in}{6.184930in}}{\pgfqpoint{6.793709in}{6.195529in}}{\pgfqpoint{6.793709in}{6.206579in}}%
\pgfpathcurveto{\pgfqpoint{6.793709in}{6.217629in}}{\pgfqpoint{6.789319in}{6.228228in}}{\pgfqpoint{6.781505in}{6.236042in}}%
\pgfpathcurveto{\pgfqpoint{6.773691in}{6.243855in}}{\pgfqpoint{6.763092in}{6.248246in}}{\pgfqpoint{6.752042in}{6.248246in}}%
\pgfpathcurveto{\pgfqpoint{6.740992in}{6.248246in}}{\pgfqpoint{6.730393in}{6.243855in}}{\pgfqpoint{6.722579in}{6.236042in}}%
\pgfpathcurveto{\pgfqpoint{6.714766in}{6.228228in}}{\pgfqpoint{6.710375in}{6.217629in}}{\pgfqpoint{6.710375in}{6.206579in}}%
\pgfpathcurveto{\pgfqpoint{6.710375in}{6.195529in}}{\pgfqpoint{6.714766in}{6.184930in}}{\pgfqpoint{6.722579in}{6.177116in}}%
\pgfpathcurveto{\pgfqpoint{6.730393in}{6.169303in}}{\pgfqpoint{6.740992in}{6.164912in}}{\pgfqpoint{6.752042in}{6.164912in}}%
\pgfpathlineto{\pgfqpoint{6.752042in}{6.164912in}}%
\pgfpathclose%
\pgfusepath{stroke,fill}%
\end{pgfscope}%
\begin{pgfscope}%
\pgfpathrectangle{\pgfqpoint{5.292946in}{5.272501in}}{\pgfqpoint{2.177280in}{2.201755in}}%
\pgfusepath{clip}%
\pgfsetbuttcap%
\pgfsetroundjoin%
\definecolor{currentfill}{rgb}{0.172549,0.627451,0.172549}%
\pgfsetfillcolor{currentfill}%
\pgfsetlinewidth{0.481800pt}%
\definecolor{currentstroke}{rgb}{1.000000,1.000000,1.000000}%
\pgfsetstrokecolor{currentstroke}%
\pgfsetdash{}{0pt}%
\pgfpathmoveto{\pgfqpoint{6.608516in}{5.747913in}}%
\pgfpathcurveto{\pgfqpoint{6.619566in}{5.747913in}}{\pgfqpoint{6.630165in}{5.752303in}}{\pgfqpoint{6.637978in}{5.760117in}}%
\pgfpathcurveto{\pgfqpoint{6.645792in}{5.767931in}}{\pgfqpoint{6.650182in}{5.778530in}}{\pgfqpoint{6.650182in}{5.789580in}}%
\pgfpathcurveto{\pgfqpoint{6.650182in}{5.800630in}}{\pgfqpoint{6.645792in}{5.811229in}}{\pgfqpoint{6.637978in}{5.819043in}}%
\pgfpathcurveto{\pgfqpoint{6.630165in}{5.826856in}}{\pgfqpoint{6.619566in}{5.831247in}}{\pgfqpoint{6.608516in}{5.831247in}}%
\pgfpathcurveto{\pgfqpoint{6.597466in}{5.831247in}}{\pgfqpoint{6.586867in}{5.826856in}}{\pgfqpoint{6.579053in}{5.819043in}}%
\pgfpathcurveto{\pgfqpoint{6.571239in}{5.811229in}}{\pgfqpoint{6.566849in}{5.800630in}}{\pgfqpoint{6.566849in}{5.789580in}}%
\pgfpathcurveto{\pgfqpoint{6.566849in}{5.778530in}}{\pgfqpoint{6.571239in}{5.767931in}}{\pgfqpoint{6.579053in}{5.760117in}}%
\pgfpathcurveto{\pgfqpoint{6.586867in}{5.752303in}}{\pgfqpoint{6.597466in}{5.747913in}}{\pgfqpoint{6.608516in}{5.747913in}}%
\pgfpathlineto{\pgfqpoint{6.608516in}{5.747913in}}%
\pgfpathclose%
\pgfusepath{stroke,fill}%
\end{pgfscope}%
\begin{pgfscope}%
\pgfpathrectangle{\pgfqpoint{5.292946in}{5.272501in}}{\pgfqpoint{2.177280in}{2.201755in}}%
\pgfusepath{clip}%
\pgfsetbuttcap%
\pgfsetroundjoin%
\definecolor{currentfill}{rgb}{0.172549,0.627451,0.172549}%
\pgfsetfillcolor{currentfill}%
\pgfsetlinewidth{0.481800pt}%
\definecolor{currentstroke}{rgb}{1.000000,1.000000,1.000000}%
\pgfsetstrokecolor{currentstroke}%
\pgfsetdash{}{0pt}%
\pgfpathmoveto{\pgfqpoint{6.637221in}{5.998113in}}%
\pgfpathcurveto{\pgfqpoint{6.648271in}{5.998113in}}{\pgfqpoint{6.658870in}{6.002503in}}{\pgfqpoint{6.666684in}{6.010317in}}%
\pgfpathcurveto{\pgfqpoint{6.674497in}{6.018130in}}{\pgfqpoint{6.678888in}{6.028729in}}{\pgfqpoint{6.678888in}{6.039779in}}%
\pgfpathcurveto{\pgfqpoint{6.678888in}{6.050829in}}{\pgfqpoint{6.674497in}{6.061428in}}{\pgfqpoint{6.666684in}{6.069242in}}%
\pgfpathcurveto{\pgfqpoint{6.658870in}{6.077056in}}{\pgfqpoint{6.648271in}{6.081446in}}{\pgfqpoint{6.637221in}{6.081446in}}%
\pgfpathcurveto{\pgfqpoint{6.626171in}{6.081446in}}{\pgfqpoint{6.615572in}{6.077056in}}{\pgfqpoint{6.607758in}{6.069242in}}%
\pgfpathcurveto{\pgfqpoint{6.599945in}{6.061428in}}{\pgfqpoint{6.595554in}{6.050829in}}{\pgfqpoint{6.595554in}{6.039779in}}%
\pgfpathcurveto{\pgfqpoint{6.595554in}{6.028729in}}{\pgfqpoint{6.599945in}{6.018130in}}{\pgfqpoint{6.607758in}{6.010317in}}%
\pgfpathcurveto{\pgfqpoint{6.615572in}{6.002503in}}{\pgfqpoint{6.626171in}{5.998113in}}{\pgfqpoint{6.637221in}{5.998113in}}%
\pgfpathlineto{\pgfqpoint{6.637221in}{5.998113in}}%
\pgfpathclose%
\pgfusepath{stroke,fill}%
\end{pgfscope}%
\begin{pgfscope}%
\pgfpathrectangle{\pgfqpoint{5.292946in}{5.272501in}}{\pgfqpoint{2.177280in}{2.201755in}}%
\pgfusepath{clip}%
\pgfsetbuttcap%
\pgfsetroundjoin%
\definecolor{currentfill}{rgb}{0.172549,0.627451,0.172549}%
\pgfsetfillcolor{currentfill}%
\pgfsetlinewidth{0.481800pt}%
\definecolor{currentstroke}{rgb}{1.000000,1.000000,1.000000}%
\pgfsetstrokecolor{currentstroke}%
\pgfsetdash{}{0pt}%
\pgfpathmoveto{\pgfqpoint{6.694632in}{6.331712in}}%
\pgfpathcurveto{\pgfqpoint{6.705682in}{6.331712in}}{\pgfqpoint{6.716281in}{6.336102in}}{\pgfqpoint{6.724094in}{6.343916in}}%
\pgfpathcurveto{\pgfqpoint{6.731908in}{6.351729in}}{\pgfqpoint{6.736298in}{6.362328in}}{\pgfqpoint{6.736298in}{6.373379in}}%
\pgfpathcurveto{\pgfqpoint{6.736298in}{6.384429in}}{\pgfqpoint{6.731908in}{6.395028in}}{\pgfqpoint{6.724094in}{6.402841in}}%
\pgfpathcurveto{\pgfqpoint{6.716281in}{6.410655in}}{\pgfqpoint{6.705682in}{6.415045in}}{\pgfqpoint{6.694632in}{6.415045in}}%
\pgfpathcurveto{\pgfqpoint{6.683581in}{6.415045in}}{\pgfqpoint{6.672982in}{6.410655in}}{\pgfqpoint{6.665169in}{6.402841in}}%
\pgfpathcurveto{\pgfqpoint{6.657355in}{6.395028in}}{\pgfqpoint{6.652965in}{6.384429in}}{\pgfqpoint{6.652965in}{6.373379in}}%
\pgfpathcurveto{\pgfqpoint{6.652965in}{6.362328in}}{\pgfqpoint{6.657355in}{6.351729in}}{\pgfqpoint{6.665169in}{6.343916in}}%
\pgfpathcurveto{\pgfqpoint{6.672982in}{6.336102in}}{\pgfqpoint{6.683581in}{6.331712in}}{\pgfqpoint{6.694632in}{6.331712in}}%
\pgfpathlineto{\pgfqpoint{6.694632in}{6.331712in}}%
\pgfpathclose%
\pgfusepath{stroke,fill}%
\end{pgfscope}%
\begin{pgfscope}%
\pgfpathrectangle{\pgfqpoint{5.292946in}{5.272501in}}{\pgfqpoint{2.177280in}{2.201755in}}%
\pgfusepath{clip}%
\pgfsetbuttcap%
\pgfsetroundjoin%
\definecolor{currentfill}{rgb}{0.172549,0.627451,0.172549}%
\pgfsetfillcolor{currentfill}%
\pgfsetlinewidth{0.481800pt}%
\definecolor{currentstroke}{rgb}{1.000000,1.000000,1.000000}%
\pgfsetstrokecolor{currentstroke}%
\pgfsetdash{}{0pt}%
\pgfpathmoveto{\pgfqpoint{6.752042in}{6.164912in}}%
\pgfpathcurveto{\pgfqpoint{6.763092in}{6.164912in}}{\pgfqpoint{6.773691in}{6.169303in}}{\pgfqpoint{6.781505in}{6.177116in}}%
\pgfpathcurveto{\pgfqpoint{6.789319in}{6.184930in}}{\pgfqpoint{6.793709in}{6.195529in}}{\pgfqpoint{6.793709in}{6.206579in}}%
\pgfpathcurveto{\pgfqpoint{6.793709in}{6.217629in}}{\pgfqpoint{6.789319in}{6.228228in}}{\pgfqpoint{6.781505in}{6.236042in}}%
\pgfpathcurveto{\pgfqpoint{6.773691in}{6.243855in}}{\pgfqpoint{6.763092in}{6.248246in}}{\pgfqpoint{6.752042in}{6.248246in}}%
\pgfpathcurveto{\pgfqpoint{6.740992in}{6.248246in}}{\pgfqpoint{6.730393in}{6.243855in}}{\pgfqpoint{6.722579in}{6.236042in}}%
\pgfpathcurveto{\pgfqpoint{6.714766in}{6.228228in}}{\pgfqpoint{6.710375in}{6.217629in}}{\pgfqpoint{6.710375in}{6.206579in}}%
\pgfpathcurveto{\pgfqpoint{6.710375in}{6.195529in}}{\pgfqpoint{6.714766in}{6.184930in}}{\pgfqpoint{6.722579in}{6.177116in}}%
\pgfpathcurveto{\pgfqpoint{6.730393in}{6.169303in}}{\pgfqpoint{6.740992in}{6.164912in}}{\pgfqpoint{6.752042in}{6.164912in}}%
\pgfpathlineto{\pgfqpoint{6.752042in}{6.164912in}}%
\pgfpathclose%
\pgfusepath{stroke,fill}%
\end{pgfscope}%
\begin{pgfscope}%
\pgfpathrectangle{\pgfqpoint{5.292946in}{5.272501in}}{\pgfqpoint{2.177280in}{2.201755in}}%
\pgfusepath{clip}%
\pgfsetbuttcap%
\pgfsetroundjoin%
\definecolor{currentfill}{rgb}{0.172549,0.627451,0.172549}%
\pgfsetfillcolor{currentfill}%
\pgfsetlinewidth{0.481800pt}%
\definecolor{currentstroke}{rgb}{1.000000,1.000000,1.000000}%
\pgfsetstrokecolor{currentstroke}%
\pgfsetdash{}{0pt}%
\pgfpathmoveto{\pgfqpoint{7.096506in}{6.832111in}}%
\pgfpathcurveto{\pgfqpoint{7.107556in}{6.832111in}}{\pgfqpoint{7.118155in}{6.836501in}}{\pgfqpoint{7.125968in}{6.844315in}}%
\pgfpathcurveto{\pgfqpoint{7.133782in}{6.852128in}}{\pgfqpoint{7.138172in}{6.862727in}}{\pgfqpoint{7.138172in}{6.873777in}}%
\pgfpathcurveto{\pgfqpoint{7.138172in}{6.884828in}}{\pgfqpoint{7.133782in}{6.895427in}}{\pgfqpoint{7.125968in}{6.903240in}}%
\pgfpathcurveto{\pgfqpoint{7.118155in}{6.911054in}}{\pgfqpoint{7.107556in}{6.915444in}}{\pgfqpoint{7.096506in}{6.915444in}}%
\pgfpathcurveto{\pgfqpoint{7.085455in}{6.915444in}}{\pgfqpoint{7.074856in}{6.911054in}}{\pgfqpoint{7.067043in}{6.903240in}}%
\pgfpathcurveto{\pgfqpoint{7.059229in}{6.895427in}}{\pgfqpoint{7.054839in}{6.884828in}}{\pgfqpoint{7.054839in}{6.873777in}}%
\pgfpathcurveto{\pgfqpoint{7.054839in}{6.862727in}}{\pgfqpoint{7.059229in}{6.852128in}}{\pgfqpoint{7.067043in}{6.844315in}}%
\pgfpathcurveto{\pgfqpoint{7.074856in}{6.836501in}}{\pgfqpoint{7.085455in}{6.832111in}}{\pgfqpoint{7.096506in}{6.832111in}}%
\pgfpathlineto{\pgfqpoint{7.096506in}{6.832111in}}%
\pgfpathclose%
\pgfusepath{stroke,fill}%
\end{pgfscope}%
\begin{pgfscope}%
\pgfpathrectangle{\pgfqpoint{5.292946in}{5.272501in}}{\pgfqpoint{2.177280in}{2.201755in}}%
\pgfusepath{clip}%
\pgfsetbuttcap%
\pgfsetroundjoin%
\definecolor{currentfill}{rgb}{0.172549,0.627451,0.172549}%
\pgfsetfillcolor{currentfill}%
\pgfsetlinewidth{0.481800pt}%
\definecolor{currentstroke}{rgb}{1.000000,1.000000,1.000000}%
\pgfsetstrokecolor{currentstroke}%
\pgfsetdash{}{0pt}%
\pgfpathmoveto{\pgfqpoint{7.153916in}{5.831313in}}%
\pgfpathcurveto{\pgfqpoint{7.164966in}{5.831313in}}{\pgfqpoint{7.175565in}{5.835703in}}{\pgfqpoint{7.183379in}{5.843517in}}%
\pgfpathcurveto{\pgfqpoint{7.191193in}{5.851331in}}{\pgfqpoint{7.195583in}{5.861930in}}{\pgfqpoint{7.195583in}{5.872980in}}%
\pgfpathcurveto{\pgfqpoint{7.195583in}{5.884030in}}{\pgfqpoint{7.191193in}{5.894629in}}{\pgfqpoint{7.183379in}{5.902442in}}%
\pgfpathcurveto{\pgfqpoint{7.175565in}{5.910256in}}{\pgfqpoint{7.164966in}{5.914646in}}{\pgfqpoint{7.153916in}{5.914646in}}%
\pgfpathcurveto{\pgfqpoint{7.142866in}{5.914646in}}{\pgfqpoint{7.132267in}{5.910256in}}{\pgfqpoint{7.124453in}{5.902442in}}%
\pgfpathcurveto{\pgfqpoint{7.116640in}{5.894629in}}{\pgfqpoint{7.112249in}{5.884030in}}{\pgfqpoint{7.112249in}{5.872980in}}%
\pgfpathcurveto{\pgfqpoint{7.112249in}{5.861930in}}{\pgfqpoint{7.116640in}{5.851331in}}{\pgfqpoint{7.124453in}{5.843517in}}%
\pgfpathcurveto{\pgfqpoint{7.132267in}{5.835703in}}{\pgfqpoint{7.142866in}{5.831313in}}{\pgfqpoint{7.153916in}{5.831313in}}%
\pgfpathlineto{\pgfqpoint{7.153916in}{5.831313in}}%
\pgfpathclose%
\pgfusepath{stroke,fill}%
\end{pgfscope}%
\begin{pgfscope}%
\pgfpathrectangle{\pgfqpoint{5.292946in}{5.272501in}}{\pgfqpoint{2.177280in}{2.201755in}}%
\pgfusepath{clip}%
\pgfsetbuttcap%
\pgfsetroundjoin%
\definecolor{currentfill}{rgb}{0.172549,0.627451,0.172549}%
\pgfsetfillcolor{currentfill}%
\pgfsetlinewidth{0.481800pt}%
\definecolor{currentstroke}{rgb}{1.000000,1.000000,1.000000}%
\pgfsetstrokecolor{currentstroke}%
\pgfsetdash{}{0pt}%
\pgfpathmoveto{\pgfqpoint{6.608516in}{5.497714in}}%
\pgfpathcurveto{\pgfqpoint{6.619566in}{5.497714in}}{\pgfqpoint{6.630165in}{5.502104in}}{\pgfqpoint{6.637978in}{5.509918in}}%
\pgfpathcurveto{\pgfqpoint{6.645792in}{5.517731in}}{\pgfqpoint{6.650182in}{5.528330in}}{\pgfqpoint{6.650182in}{5.539380in}}%
\pgfpathcurveto{\pgfqpoint{6.650182in}{5.550431in}}{\pgfqpoint{6.645792in}{5.561030in}}{\pgfqpoint{6.637978in}{5.568843in}}%
\pgfpathcurveto{\pgfqpoint{6.630165in}{5.576657in}}{\pgfqpoint{6.619566in}{5.581047in}}{\pgfqpoint{6.608516in}{5.581047in}}%
\pgfpathcurveto{\pgfqpoint{6.597466in}{5.581047in}}{\pgfqpoint{6.586867in}{5.576657in}}{\pgfqpoint{6.579053in}{5.568843in}}%
\pgfpathcurveto{\pgfqpoint{6.571239in}{5.561030in}}{\pgfqpoint{6.566849in}{5.550431in}}{\pgfqpoint{6.566849in}{5.539380in}}%
\pgfpathcurveto{\pgfqpoint{6.566849in}{5.528330in}}{\pgfqpoint{6.571239in}{5.517731in}}{\pgfqpoint{6.579053in}{5.509918in}}%
\pgfpathcurveto{\pgfqpoint{6.586867in}{5.502104in}}{\pgfqpoint{6.597466in}{5.497714in}}{\pgfqpoint{6.608516in}{5.497714in}}%
\pgfpathlineto{\pgfqpoint{6.608516in}{5.497714in}}%
\pgfpathclose%
\pgfusepath{stroke,fill}%
\end{pgfscope}%
\begin{pgfscope}%
\pgfpathrectangle{\pgfqpoint{5.292946in}{5.272501in}}{\pgfqpoint{2.177280in}{2.201755in}}%
\pgfusepath{clip}%
\pgfsetbuttcap%
\pgfsetroundjoin%
\definecolor{currentfill}{rgb}{0.172549,0.627451,0.172549}%
\pgfsetfillcolor{currentfill}%
\pgfsetlinewidth{0.481800pt}%
\definecolor{currentstroke}{rgb}{1.000000,1.000000,1.000000}%
\pgfsetstrokecolor{currentstroke}%
\pgfsetdash{}{0pt}%
\pgfpathmoveto{\pgfqpoint{6.809453in}{6.331712in}}%
\pgfpathcurveto{\pgfqpoint{6.820503in}{6.331712in}}{\pgfqpoint{6.831102in}{6.336102in}}{\pgfqpoint{6.838915in}{6.343916in}}%
\pgfpathcurveto{\pgfqpoint{6.846729in}{6.351729in}}{\pgfqpoint{6.851119in}{6.362328in}}{\pgfqpoint{6.851119in}{6.373379in}}%
\pgfpathcurveto{\pgfqpoint{6.851119in}{6.384429in}}{\pgfqpoint{6.846729in}{6.395028in}}{\pgfqpoint{6.838915in}{6.402841in}}%
\pgfpathcurveto{\pgfqpoint{6.831102in}{6.410655in}}{\pgfqpoint{6.820503in}{6.415045in}}{\pgfqpoint{6.809453in}{6.415045in}}%
\pgfpathcurveto{\pgfqpoint{6.798403in}{6.415045in}}{\pgfqpoint{6.787804in}{6.410655in}}{\pgfqpoint{6.779990in}{6.402841in}}%
\pgfpathcurveto{\pgfqpoint{6.772176in}{6.395028in}}{\pgfqpoint{6.767786in}{6.384429in}}{\pgfqpoint{6.767786in}{6.373379in}}%
\pgfpathcurveto{\pgfqpoint{6.767786in}{6.362328in}}{\pgfqpoint{6.772176in}{6.351729in}}{\pgfqpoint{6.779990in}{6.343916in}}%
\pgfpathcurveto{\pgfqpoint{6.787804in}{6.336102in}}{\pgfqpoint{6.798403in}{6.331712in}}{\pgfqpoint{6.809453in}{6.331712in}}%
\pgfpathlineto{\pgfqpoint{6.809453in}{6.331712in}}%
\pgfpathclose%
\pgfusepath{stroke,fill}%
\end{pgfscope}%
\begin{pgfscope}%
\pgfpathrectangle{\pgfqpoint{5.292946in}{5.272501in}}{\pgfqpoint{2.177280in}{2.201755in}}%
\pgfusepath{clip}%
\pgfsetbuttcap%
\pgfsetroundjoin%
\definecolor{currentfill}{rgb}{0.172549,0.627451,0.172549}%
\pgfsetfillcolor{currentfill}%
\pgfsetlinewidth{0.481800pt}%
\definecolor{currentstroke}{rgb}{1.000000,1.000000,1.000000}%
\pgfsetstrokecolor{currentstroke}%
\pgfsetdash{}{0pt}%
\pgfpathmoveto{\pgfqpoint{6.579810in}{5.998113in}}%
\pgfpathcurveto{\pgfqpoint{6.590861in}{5.998113in}}{\pgfqpoint{6.601460in}{6.002503in}}{\pgfqpoint{6.609273in}{6.010317in}}%
\pgfpathcurveto{\pgfqpoint{6.617087in}{6.018130in}}{\pgfqpoint{6.621477in}{6.028729in}}{\pgfqpoint{6.621477in}{6.039779in}}%
\pgfpathcurveto{\pgfqpoint{6.621477in}{6.050829in}}{\pgfqpoint{6.617087in}{6.061428in}}{\pgfqpoint{6.609273in}{6.069242in}}%
\pgfpathcurveto{\pgfqpoint{6.601460in}{6.077056in}}{\pgfqpoint{6.590861in}{6.081446in}}{\pgfqpoint{6.579810in}{6.081446in}}%
\pgfpathcurveto{\pgfqpoint{6.568760in}{6.081446in}}{\pgfqpoint{6.558161in}{6.077056in}}{\pgfqpoint{6.550348in}{6.069242in}}%
\pgfpathcurveto{\pgfqpoint{6.542534in}{6.061428in}}{\pgfqpoint{6.538144in}{6.050829in}}{\pgfqpoint{6.538144in}{6.039779in}}%
\pgfpathcurveto{\pgfqpoint{6.538144in}{6.028729in}}{\pgfqpoint{6.542534in}{6.018130in}}{\pgfqpoint{6.550348in}{6.010317in}}%
\pgfpathcurveto{\pgfqpoint{6.558161in}{6.002503in}}{\pgfqpoint{6.568760in}{5.998113in}}{\pgfqpoint{6.579810in}{5.998113in}}%
\pgfpathlineto{\pgfqpoint{6.579810in}{5.998113in}}%
\pgfpathclose%
\pgfusepath{stroke,fill}%
\end{pgfscope}%
\begin{pgfscope}%
\pgfpathrectangle{\pgfqpoint{5.292946in}{5.272501in}}{\pgfqpoint{2.177280in}{2.201755in}}%
\pgfusepath{clip}%
\pgfsetbuttcap%
\pgfsetroundjoin%
\definecolor{currentfill}{rgb}{0.172549,0.627451,0.172549}%
\pgfsetfillcolor{currentfill}%
\pgfsetlinewidth{0.481800pt}%
\definecolor{currentstroke}{rgb}{1.000000,1.000000,1.000000}%
\pgfsetstrokecolor{currentstroke}%
\pgfsetdash{}{0pt}%
\pgfpathmoveto{\pgfqpoint{7.096506in}{5.998113in}}%
\pgfpathcurveto{\pgfqpoint{7.107556in}{5.998113in}}{\pgfqpoint{7.118155in}{6.002503in}}{\pgfqpoint{7.125968in}{6.010317in}}%
\pgfpathcurveto{\pgfqpoint{7.133782in}{6.018130in}}{\pgfqpoint{7.138172in}{6.028729in}}{\pgfqpoint{7.138172in}{6.039779in}}%
\pgfpathcurveto{\pgfqpoint{7.138172in}{6.050829in}}{\pgfqpoint{7.133782in}{6.061428in}}{\pgfqpoint{7.125968in}{6.069242in}}%
\pgfpathcurveto{\pgfqpoint{7.118155in}{6.077056in}}{\pgfqpoint{7.107556in}{6.081446in}}{\pgfqpoint{7.096506in}{6.081446in}}%
\pgfpathcurveto{\pgfqpoint{7.085455in}{6.081446in}}{\pgfqpoint{7.074856in}{6.077056in}}{\pgfqpoint{7.067043in}{6.069242in}}%
\pgfpathcurveto{\pgfqpoint{7.059229in}{6.061428in}}{\pgfqpoint{7.054839in}{6.050829in}}{\pgfqpoint{7.054839in}{6.039779in}}%
\pgfpathcurveto{\pgfqpoint{7.054839in}{6.028729in}}{\pgfqpoint{7.059229in}{6.018130in}}{\pgfqpoint{7.067043in}{6.010317in}}%
\pgfpathcurveto{\pgfqpoint{7.074856in}{6.002503in}}{\pgfqpoint{7.085455in}{5.998113in}}{\pgfqpoint{7.096506in}{5.998113in}}%
\pgfpathlineto{\pgfqpoint{7.096506in}{5.998113in}}%
\pgfpathclose%
\pgfusepath{stroke,fill}%
\end{pgfscope}%
\begin{pgfscope}%
\pgfpathrectangle{\pgfqpoint{5.292946in}{5.272501in}}{\pgfqpoint{2.177280in}{2.201755in}}%
\pgfusepath{clip}%
\pgfsetbuttcap%
\pgfsetroundjoin%
\definecolor{currentfill}{rgb}{0.172549,0.627451,0.172549}%
\pgfsetfillcolor{currentfill}%
\pgfsetlinewidth{0.481800pt}%
\definecolor{currentstroke}{rgb}{1.000000,1.000000,1.000000}%
\pgfsetstrokecolor{currentstroke}%
\pgfsetdash{}{0pt}%
\pgfpathmoveto{\pgfqpoint{6.579810in}{5.914713in}}%
\pgfpathcurveto{\pgfqpoint{6.590861in}{5.914713in}}{\pgfqpoint{6.601460in}{5.919103in}}{\pgfqpoint{6.609273in}{5.926917in}}%
\pgfpathcurveto{\pgfqpoint{6.617087in}{5.934730in}}{\pgfqpoint{6.621477in}{5.945329in}}{\pgfqpoint{6.621477in}{5.956379in}}%
\pgfpathcurveto{\pgfqpoint{6.621477in}{5.967430in}}{\pgfqpoint{6.617087in}{5.978029in}}{\pgfqpoint{6.609273in}{5.985842in}}%
\pgfpathcurveto{\pgfqpoint{6.601460in}{5.993656in}}{\pgfqpoint{6.590861in}{5.998046in}}{\pgfqpoint{6.579810in}{5.998046in}}%
\pgfpathcurveto{\pgfqpoint{6.568760in}{5.998046in}}{\pgfqpoint{6.558161in}{5.993656in}}{\pgfqpoint{6.550348in}{5.985842in}}%
\pgfpathcurveto{\pgfqpoint{6.542534in}{5.978029in}}{\pgfqpoint{6.538144in}{5.967430in}}{\pgfqpoint{6.538144in}{5.956379in}}%
\pgfpathcurveto{\pgfqpoint{6.538144in}{5.945329in}}{\pgfqpoint{6.542534in}{5.934730in}}{\pgfqpoint{6.550348in}{5.926917in}}%
\pgfpathcurveto{\pgfqpoint{6.558161in}{5.919103in}}{\pgfqpoint{6.568760in}{5.914713in}}{\pgfqpoint{6.579810in}{5.914713in}}%
\pgfpathlineto{\pgfqpoint{6.579810in}{5.914713in}}%
\pgfpathclose%
\pgfusepath{stroke,fill}%
\end{pgfscope}%
\begin{pgfscope}%
\pgfpathrectangle{\pgfqpoint{5.292946in}{5.272501in}}{\pgfqpoint{2.177280in}{2.201755in}}%
\pgfusepath{clip}%
\pgfsetbuttcap%
\pgfsetroundjoin%
\definecolor{currentfill}{rgb}{0.172549,0.627451,0.172549}%
\pgfsetfillcolor{currentfill}%
\pgfsetlinewidth{0.481800pt}%
\definecolor{currentstroke}{rgb}{1.000000,1.000000,1.000000}%
\pgfsetstrokecolor{currentstroke}%
\pgfsetdash{}{0pt}%
\pgfpathmoveto{\pgfqpoint{6.809453in}{6.415112in}}%
\pgfpathcurveto{\pgfqpoint{6.820503in}{6.415112in}}{\pgfqpoint{6.831102in}{6.419502in}}{\pgfqpoint{6.838915in}{6.427316in}}%
\pgfpathcurveto{\pgfqpoint{6.846729in}{6.435129in}}{\pgfqpoint{6.851119in}{6.445728in}}{\pgfqpoint{6.851119in}{6.456778in}}%
\pgfpathcurveto{\pgfqpoint{6.851119in}{6.467828in}}{\pgfqpoint{6.846729in}{6.478428in}}{\pgfqpoint{6.838915in}{6.486241in}}%
\pgfpathcurveto{\pgfqpoint{6.831102in}{6.494055in}}{\pgfqpoint{6.820503in}{6.498445in}}{\pgfqpoint{6.809453in}{6.498445in}}%
\pgfpathcurveto{\pgfqpoint{6.798403in}{6.498445in}}{\pgfqpoint{6.787804in}{6.494055in}}{\pgfqpoint{6.779990in}{6.486241in}}%
\pgfpathcurveto{\pgfqpoint{6.772176in}{6.478428in}}{\pgfqpoint{6.767786in}{6.467828in}}{\pgfqpoint{6.767786in}{6.456778in}}%
\pgfpathcurveto{\pgfqpoint{6.767786in}{6.445728in}}{\pgfqpoint{6.772176in}{6.435129in}}{\pgfqpoint{6.779990in}{6.427316in}}%
\pgfpathcurveto{\pgfqpoint{6.787804in}{6.419502in}}{\pgfqpoint{6.798403in}{6.415112in}}{\pgfqpoint{6.809453in}{6.415112in}}%
\pgfpathlineto{\pgfqpoint{6.809453in}{6.415112in}}%
\pgfpathclose%
\pgfusepath{stroke,fill}%
\end{pgfscope}%
\begin{pgfscope}%
\pgfpathrectangle{\pgfqpoint{5.292946in}{5.272501in}}{\pgfqpoint{2.177280in}{2.201755in}}%
\pgfusepath{clip}%
\pgfsetbuttcap%
\pgfsetroundjoin%
\definecolor{currentfill}{rgb}{0.172549,0.627451,0.172549}%
\pgfsetfillcolor{currentfill}%
\pgfsetlinewidth{0.481800pt}%
\definecolor{currentstroke}{rgb}{1.000000,1.000000,1.000000}%
\pgfsetstrokecolor{currentstroke}%
\pgfsetdash{}{0pt}%
\pgfpathmoveto{\pgfqpoint{6.895569in}{6.331712in}}%
\pgfpathcurveto{\pgfqpoint{6.906619in}{6.331712in}}{\pgfqpoint{6.917218in}{6.336102in}}{\pgfqpoint{6.925031in}{6.343916in}}%
\pgfpathcurveto{\pgfqpoint{6.932845in}{6.351729in}}{\pgfqpoint{6.937235in}{6.362328in}}{\pgfqpoint{6.937235in}{6.373379in}}%
\pgfpathcurveto{\pgfqpoint{6.937235in}{6.384429in}}{\pgfqpoint{6.932845in}{6.395028in}}{\pgfqpoint{6.925031in}{6.402841in}}%
\pgfpathcurveto{\pgfqpoint{6.917218in}{6.410655in}}{\pgfqpoint{6.906619in}{6.415045in}}{\pgfqpoint{6.895569in}{6.415045in}}%
\pgfpathcurveto{\pgfqpoint{6.884518in}{6.415045in}}{\pgfqpoint{6.873919in}{6.410655in}}{\pgfqpoint{6.866106in}{6.402841in}}%
\pgfpathcurveto{\pgfqpoint{6.858292in}{6.395028in}}{\pgfqpoint{6.853902in}{6.384429in}}{\pgfqpoint{6.853902in}{6.373379in}}%
\pgfpathcurveto{\pgfqpoint{6.853902in}{6.362328in}}{\pgfqpoint{6.858292in}{6.351729in}}{\pgfqpoint{6.866106in}{6.343916in}}%
\pgfpathcurveto{\pgfqpoint{6.873919in}{6.336102in}}{\pgfqpoint{6.884518in}{6.331712in}}{\pgfqpoint{6.895569in}{6.331712in}}%
\pgfpathlineto{\pgfqpoint{6.895569in}{6.331712in}}%
\pgfpathclose%
\pgfusepath{stroke,fill}%
\end{pgfscope}%
\begin{pgfscope}%
\pgfpathrectangle{\pgfqpoint{5.292946in}{5.272501in}}{\pgfqpoint{2.177280in}{2.201755in}}%
\pgfusepath{clip}%
\pgfsetbuttcap%
\pgfsetroundjoin%
\definecolor{currentfill}{rgb}{0.172549,0.627451,0.172549}%
\pgfsetfillcolor{currentfill}%
\pgfsetlinewidth{0.481800pt}%
\definecolor{currentstroke}{rgb}{1.000000,1.000000,1.000000}%
\pgfsetstrokecolor{currentstroke}%
\pgfsetdash{}{0pt}%
\pgfpathmoveto{\pgfqpoint{6.551105in}{5.998113in}}%
\pgfpathcurveto{\pgfqpoint{6.562155in}{5.998113in}}{\pgfqpoint{6.572754in}{6.002503in}}{\pgfqpoint{6.580568in}{6.010317in}}%
\pgfpathcurveto{\pgfqpoint{6.588382in}{6.018130in}}{\pgfqpoint{6.592772in}{6.028729in}}{\pgfqpoint{6.592772in}{6.039779in}}%
\pgfpathcurveto{\pgfqpoint{6.592772in}{6.050829in}}{\pgfqpoint{6.588382in}{6.061428in}}{\pgfqpoint{6.580568in}{6.069242in}}%
\pgfpathcurveto{\pgfqpoint{6.572754in}{6.077056in}}{\pgfqpoint{6.562155in}{6.081446in}}{\pgfqpoint{6.551105in}{6.081446in}}%
\pgfpathcurveto{\pgfqpoint{6.540055in}{6.081446in}}{\pgfqpoint{6.529456in}{6.077056in}}{\pgfqpoint{6.521642in}{6.069242in}}%
\pgfpathcurveto{\pgfqpoint{6.513829in}{6.061428in}}{\pgfqpoint{6.509438in}{6.050829in}}{\pgfqpoint{6.509438in}{6.039779in}}%
\pgfpathcurveto{\pgfqpoint{6.509438in}{6.028729in}}{\pgfqpoint{6.513829in}{6.018130in}}{\pgfqpoint{6.521642in}{6.010317in}}%
\pgfpathcurveto{\pgfqpoint{6.529456in}{6.002503in}}{\pgfqpoint{6.540055in}{5.998113in}}{\pgfqpoint{6.551105in}{5.998113in}}%
\pgfpathlineto{\pgfqpoint{6.551105in}{5.998113in}}%
\pgfpathclose%
\pgfusepath{stroke,fill}%
\end{pgfscope}%
\begin{pgfscope}%
\pgfpathrectangle{\pgfqpoint{5.292946in}{5.272501in}}{\pgfqpoint{2.177280in}{2.201755in}}%
\pgfusepath{clip}%
\pgfsetbuttcap%
\pgfsetroundjoin%
\definecolor{currentfill}{rgb}{0.172549,0.627451,0.172549}%
\pgfsetfillcolor{currentfill}%
\pgfsetlinewidth{0.481800pt}%
\definecolor{currentstroke}{rgb}{1.000000,1.000000,1.000000}%
\pgfsetstrokecolor{currentstroke}%
\pgfsetdash{}{0pt}%
\pgfpathmoveto{\pgfqpoint{6.579810in}{6.164912in}}%
\pgfpathcurveto{\pgfqpoint{6.590861in}{6.164912in}}{\pgfqpoint{6.601460in}{6.169303in}}{\pgfqpoint{6.609273in}{6.177116in}}%
\pgfpathcurveto{\pgfqpoint{6.617087in}{6.184930in}}{\pgfqpoint{6.621477in}{6.195529in}}{\pgfqpoint{6.621477in}{6.206579in}}%
\pgfpathcurveto{\pgfqpoint{6.621477in}{6.217629in}}{\pgfqpoint{6.617087in}{6.228228in}}{\pgfqpoint{6.609273in}{6.236042in}}%
\pgfpathcurveto{\pgfqpoint{6.601460in}{6.243855in}}{\pgfqpoint{6.590861in}{6.248246in}}{\pgfqpoint{6.579810in}{6.248246in}}%
\pgfpathcurveto{\pgfqpoint{6.568760in}{6.248246in}}{\pgfqpoint{6.558161in}{6.243855in}}{\pgfqpoint{6.550348in}{6.236042in}}%
\pgfpathcurveto{\pgfqpoint{6.542534in}{6.228228in}}{\pgfqpoint{6.538144in}{6.217629in}}{\pgfqpoint{6.538144in}{6.206579in}}%
\pgfpathcurveto{\pgfqpoint{6.538144in}{6.195529in}}{\pgfqpoint{6.542534in}{6.184930in}}{\pgfqpoint{6.550348in}{6.177116in}}%
\pgfpathcurveto{\pgfqpoint{6.558161in}{6.169303in}}{\pgfqpoint{6.568760in}{6.164912in}}{\pgfqpoint{6.579810in}{6.164912in}}%
\pgfpathlineto{\pgfqpoint{6.579810in}{6.164912in}}%
\pgfpathclose%
\pgfusepath{stroke,fill}%
\end{pgfscope}%
\begin{pgfscope}%
\pgfpathrectangle{\pgfqpoint{5.292946in}{5.272501in}}{\pgfqpoint{2.177280in}{2.201755in}}%
\pgfusepath{clip}%
\pgfsetbuttcap%
\pgfsetroundjoin%
\definecolor{currentfill}{rgb}{0.172549,0.627451,0.172549}%
\pgfsetfillcolor{currentfill}%
\pgfsetlinewidth{0.481800pt}%
\definecolor{currentstroke}{rgb}{1.000000,1.000000,1.000000}%
\pgfsetstrokecolor{currentstroke}%
\pgfsetdash{}{0pt}%
\pgfpathmoveto{\pgfqpoint{6.780747in}{5.998113in}}%
\pgfpathcurveto{\pgfqpoint{6.791798in}{5.998113in}}{\pgfqpoint{6.802397in}{6.002503in}}{\pgfqpoint{6.810210in}{6.010317in}}%
\pgfpathcurveto{\pgfqpoint{6.818024in}{6.018130in}}{\pgfqpoint{6.822414in}{6.028729in}}{\pgfqpoint{6.822414in}{6.039779in}}%
\pgfpathcurveto{\pgfqpoint{6.822414in}{6.050829in}}{\pgfqpoint{6.818024in}{6.061428in}}{\pgfqpoint{6.810210in}{6.069242in}}%
\pgfpathcurveto{\pgfqpoint{6.802397in}{6.077056in}}{\pgfqpoint{6.791798in}{6.081446in}}{\pgfqpoint{6.780747in}{6.081446in}}%
\pgfpathcurveto{\pgfqpoint{6.769697in}{6.081446in}}{\pgfqpoint{6.759098in}{6.077056in}}{\pgfqpoint{6.751285in}{6.069242in}}%
\pgfpathcurveto{\pgfqpoint{6.743471in}{6.061428in}}{\pgfqpoint{6.739081in}{6.050829in}}{\pgfqpoint{6.739081in}{6.039779in}}%
\pgfpathcurveto{\pgfqpoint{6.739081in}{6.028729in}}{\pgfqpoint{6.743471in}{6.018130in}}{\pgfqpoint{6.751285in}{6.010317in}}%
\pgfpathcurveto{\pgfqpoint{6.759098in}{6.002503in}}{\pgfqpoint{6.769697in}{5.998113in}}{\pgfqpoint{6.780747in}{5.998113in}}%
\pgfpathlineto{\pgfqpoint{6.780747in}{5.998113in}}%
\pgfpathclose%
\pgfusepath{stroke,fill}%
\end{pgfscope}%
\begin{pgfscope}%
\pgfpathrectangle{\pgfqpoint{5.292946in}{5.272501in}}{\pgfqpoint{2.177280in}{2.201755in}}%
\pgfusepath{clip}%
\pgfsetbuttcap%
\pgfsetroundjoin%
\definecolor{currentfill}{rgb}{0.172549,0.627451,0.172549}%
\pgfsetfillcolor{currentfill}%
\pgfsetlinewidth{0.481800pt}%
\definecolor{currentstroke}{rgb}{1.000000,1.000000,1.000000}%
\pgfsetstrokecolor{currentstroke}%
\pgfsetdash{}{0pt}%
\pgfpathmoveto{\pgfqpoint{6.838158in}{6.164912in}}%
\pgfpathcurveto{\pgfqpoint{6.849208in}{6.164912in}}{\pgfqpoint{6.859807in}{6.169303in}}{\pgfqpoint{6.867621in}{6.177116in}}%
\pgfpathcurveto{\pgfqpoint{6.875434in}{6.184930in}}{\pgfqpoint{6.879825in}{6.195529in}}{\pgfqpoint{6.879825in}{6.206579in}}%
\pgfpathcurveto{\pgfqpoint{6.879825in}{6.217629in}}{\pgfqpoint{6.875434in}{6.228228in}}{\pgfqpoint{6.867621in}{6.236042in}}%
\pgfpathcurveto{\pgfqpoint{6.859807in}{6.243855in}}{\pgfqpoint{6.849208in}{6.248246in}}{\pgfqpoint{6.838158in}{6.248246in}}%
\pgfpathcurveto{\pgfqpoint{6.827108in}{6.248246in}}{\pgfqpoint{6.816509in}{6.243855in}}{\pgfqpoint{6.808695in}{6.236042in}}%
\pgfpathcurveto{\pgfqpoint{6.800882in}{6.228228in}}{\pgfqpoint{6.796491in}{6.217629in}}{\pgfqpoint{6.796491in}{6.206579in}}%
\pgfpathcurveto{\pgfqpoint{6.796491in}{6.195529in}}{\pgfqpoint{6.800882in}{6.184930in}}{\pgfqpoint{6.808695in}{6.177116in}}%
\pgfpathcurveto{\pgfqpoint{6.816509in}{6.169303in}}{\pgfqpoint{6.827108in}{6.164912in}}{\pgfqpoint{6.838158in}{6.164912in}}%
\pgfpathlineto{\pgfqpoint{6.838158in}{6.164912in}}%
\pgfpathclose%
\pgfusepath{stroke,fill}%
\end{pgfscope}%
\begin{pgfscope}%
\pgfpathrectangle{\pgfqpoint{5.292946in}{5.272501in}}{\pgfqpoint{2.177280in}{2.201755in}}%
\pgfusepath{clip}%
\pgfsetbuttcap%
\pgfsetroundjoin%
\definecolor{currentfill}{rgb}{0.172549,0.627451,0.172549}%
\pgfsetfillcolor{currentfill}%
\pgfsetlinewidth{0.481800pt}%
\definecolor{currentstroke}{rgb}{1.000000,1.000000,1.000000}%
\pgfsetstrokecolor{currentstroke}%
\pgfsetdash{}{0pt}%
\pgfpathmoveto{\pgfqpoint{6.924274in}{5.998113in}}%
\pgfpathcurveto{\pgfqpoint{6.935324in}{5.998113in}}{\pgfqpoint{6.945923in}{6.002503in}}{\pgfqpoint{6.953737in}{6.010317in}}%
\pgfpathcurveto{\pgfqpoint{6.961550in}{6.018130in}}{\pgfqpoint{6.965941in}{6.028729in}}{\pgfqpoint{6.965941in}{6.039779in}}%
\pgfpathcurveto{\pgfqpoint{6.965941in}{6.050829in}}{\pgfqpoint{6.961550in}{6.061428in}}{\pgfqpoint{6.953737in}{6.069242in}}%
\pgfpathcurveto{\pgfqpoint{6.945923in}{6.077056in}}{\pgfqpoint{6.935324in}{6.081446in}}{\pgfqpoint{6.924274in}{6.081446in}}%
\pgfpathcurveto{\pgfqpoint{6.913224in}{6.081446in}}{\pgfqpoint{6.902625in}{6.077056in}}{\pgfqpoint{6.894811in}{6.069242in}}%
\pgfpathcurveto{\pgfqpoint{6.886997in}{6.061428in}}{\pgfqpoint{6.882607in}{6.050829in}}{\pgfqpoint{6.882607in}{6.039779in}}%
\pgfpathcurveto{\pgfqpoint{6.882607in}{6.028729in}}{\pgfqpoint{6.886997in}{6.018130in}}{\pgfqpoint{6.894811in}{6.010317in}}%
\pgfpathcurveto{\pgfqpoint{6.902625in}{6.002503in}}{\pgfqpoint{6.913224in}{5.998113in}}{\pgfqpoint{6.924274in}{5.998113in}}%
\pgfpathlineto{\pgfqpoint{6.924274in}{5.998113in}}%
\pgfpathclose%
\pgfusepath{stroke,fill}%
\end{pgfscope}%
\begin{pgfscope}%
\pgfpathrectangle{\pgfqpoint{5.292946in}{5.272501in}}{\pgfqpoint{2.177280in}{2.201755in}}%
\pgfusepath{clip}%
\pgfsetbuttcap%
\pgfsetroundjoin%
\definecolor{currentfill}{rgb}{0.172549,0.627451,0.172549}%
\pgfsetfillcolor{currentfill}%
\pgfsetlinewidth{0.481800pt}%
\definecolor{currentstroke}{rgb}{1.000000,1.000000,1.000000}%
\pgfsetstrokecolor{currentstroke}%
\pgfsetdash{}{0pt}%
\pgfpathmoveto{\pgfqpoint{7.010390in}{6.832111in}}%
\pgfpathcurveto{\pgfqpoint{7.021440in}{6.832111in}}{\pgfqpoint{7.032039in}{6.836501in}}{\pgfqpoint{7.039852in}{6.844315in}}%
\pgfpathcurveto{\pgfqpoint{7.047666in}{6.852128in}}{\pgfqpoint{7.052056in}{6.862727in}}{\pgfqpoint{7.052056in}{6.873777in}}%
\pgfpathcurveto{\pgfqpoint{7.052056in}{6.884828in}}{\pgfqpoint{7.047666in}{6.895427in}}{\pgfqpoint{7.039852in}{6.903240in}}%
\pgfpathcurveto{\pgfqpoint{7.032039in}{6.911054in}}{\pgfqpoint{7.021440in}{6.915444in}}{\pgfqpoint{7.010390in}{6.915444in}}%
\pgfpathcurveto{\pgfqpoint{6.999340in}{6.915444in}}{\pgfqpoint{6.988741in}{6.911054in}}{\pgfqpoint{6.980927in}{6.903240in}}%
\pgfpathcurveto{\pgfqpoint{6.973113in}{6.895427in}}{\pgfqpoint{6.968723in}{6.884828in}}{\pgfqpoint{6.968723in}{6.873777in}}%
\pgfpathcurveto{\pgfqpoint{6.968723in}{6.862727in}}{\pgfqpoint{6.973113in}{6.852128in}}{\pgfqpoint{6.980927in}{6.844315in}}%
\pgfpathcurveto{\pgfqpoint{6.988741in}{6.836501in}}{\pgfqpoint{6.999340in}{6.832111in}}{\pgfqpoint{7.010390in}{6.832111in}}%
\pgfpathlineto{\pgfqpoint{7.010390in}{6.832111in}}%
\pgfpathclose%
\pgfusepath{stroke,fill}%
\end{pgfscope}%
\begin{pgfscope}%
\pgfpathrectangle{\pgfqpoint{5.292946in}{5.272501in}}{\pgfqpoint{2.177280in}{2.201755in}}%
\pgfusepath{clip}%
\pgfsetbuttcap%
\pgfsetroundjoin%
\definecolor{currentfill}{rgb}{0.172549,0.627451,0.172549}%
\pgfsetfillcolor{currentfill}%
\pgfsetlinewidth{0.481800pt}%
\definecolor{currentstroke}{rgb}{1.000000,1.000000,1.000000}%
\pgfsetstrokecolor{currentstroke}%
\pgfsetdash{}{0pt}%
\pgfpathmoveto{\pgfqpoint{6.780747in}{5.998113in}}%
\pgfpathcurveto{\pgfqpoint{6.791798in}{5.998113in}}{\pgfqpoint{6.802397in}{6.002503in}}{\pgfqpoint{6.810210in}{6.010317in}}%
\pgfpathcurveto{\pgfqpoint{6.818024in}{6.018130in}}{\pgfqpoint{6.822414in}{6.028729in}}{\pgfqpoint{6.822414in}{6.039779in}}%
\pgfpathcurveto{\pgfqpoint{6.822414in}{6.050829in}}{\pgfqpoint{6.818024in}{6.061428in}}{\pgfqpoint{6.810210in}{6.069242in}}%
\pgfpathcurveto{\pgfqpoint{6.802397in}{6.077056in}}{\pgfqpoint{6.791798in}{6.081446in}}{\pgfqpoint{6.780747in}{6.081446in}}%
\pgfpathcurveto{\pgfqpoint{6.769697in}{6.081446in}}{\pgfqpoint{6.759098in}{6.077056in}}{\pgfqpoint{6.751285in}{6.069242in}}%
\pgfpathcurveto{\pgfqpoint{6.743471in}{6.061428in}}{\pgfqpoint{6.739081in}{6.050829in}}{\pgfqpoint{6.739081in}{6.039779in}}%
\pgfpathcurveto{\pgfqpoint{6.739081in}{6.028729in}}{\pgfqpoint{6.743471in}{6.018130in}}{\pgfqpoint{6.751285in}{6.010317in}}%
\pgfpathcurveto{\pgfqpoint{6.759098in}{6.002503in}}{\pgfqpoint{6.769697in}{5.998113in}}{\pgfqpoint{6.780747in}{5.998113in}}%
\pgfpathlineto{\pgfqpoint{6.780747in}{5.998113in}}%
\pgfpathclose%
\pgfusepath{stroke,fill}%
\end{pgfscope}%
\begin{pgfscope}%
\pgfpathrectangle{\pgfqpoint{5.292946in}{5.272501in}}{\pgfqpoint{2.177280in}{2.201755in}}%
\pgfusepath{clip}%
\pgfsetbuttcap%
\pgfsetroundjoin%
\definecolor{currentfill}{rgb}{0.172549,0.627451,0.172549}%
\pgfsetfillcolor{currentfill}%
\pgfsetlinewidth{0.481800pt}%
\definecolor{currentstroke}{rgb}{1.000000,1.000000,1.000000}%
\pgfsetstrokecolor{currentstroke}%
\pgfsetdash{}{0pt}%
\pgfpathmoveto{\pgfqpoint{6.637221in}{5.998113in}}%
\pgfpathcurveto{\pgfqpoint{6.648271in}{5.998113in}}{\pgfqpoint{6.658870in}{6.002503in}}{\pgfqpoint{6.666684in}{6.010317in}}%
\pgfpathcurveto{\pgfqpoint{6.674497in}{6.018130in}}{\pgfqpoint{6.678888in}{6.028729in}}{\pgfqpoint{6.678888in}{6.039779in}}%
\pgfpathcurveto{\pgfqpoint{6.678888in}{6.050829in}}{\pgfqpoint{6.674497in}{6.061428in}}{\pgfqpoint{6.666684in}{6.069242in}}%
\pgfpathcurveto{\pgfqpoint{6.658870in}{6.077056in}}{\pgfqpoint{6.648271in}{6.081446in}}{\pgfqpoint{6.637221in}{6.081446in}}%
\pgfpathcurveto{\pgfqpoint{6.626171in}{6.081446in}}{\pgfqpoint{6.615572in}{6.077056in}}{\pgfqpoint{6.607758in}{6.069242in}}%
\pgfpathcurveto{\pgfqpoint{6.599945in}{6.061428in}}{\pgfqpoint{6.595554in}{6.050829in}}{\pgfqpoint{6.595554in}{6.039779in}}%
\pgfpathcurveto{\pgfqpoint{6.595554in}{6.028729in}}{\pgfqpoint{6.599945in}{6.018130in}}{\pgfqpoint{6.607758in}{6.010317in}}%
\pgfpathcurveto{\pgfqpoint{6.615572in}{6.002503in}}{\pgfqpoint{6.626171in}{5.998113in}}{\pgfqpoint{6.637221in}{5.998113in}}%
\pgfpathlineto{\pgfqpoint{6.637221in}{5.998113in}}%
\pgfpathclose%
\pgfusepath{stroke,fill}%
\end{pgfscope}%
\begin{pgfscope}%
\pgfpathrectangle{\pgfqpoint{5.292946in}{5.272501in}}{\pgfqpoint{2.177280in}{2.201755in}}%
\pgfusepath{clip}%
\pgfsetbuttcap%
\pgfsetroundjoin%
\definecolor{currentfill}{rgb}{0.172549,0.627451,0.172549}%
\pgfsetfillcolor{currentfill}%
\pgfsetlinewidth{0.481800pt}%
\definecolor{currentstroke}{rgb}{1.000000,1.000000,1.000000}%
\pgfsetstrokecolor{currentstroke}%
\pgfsetdash{}{0pt}%
\pgfpathmoveto{\pgfqpoint{6.780747in}{5.831313in}}%
\pgfpathcurveto{\pgfqpoint{6.791798in}{5.831313in}}{\pgfqpoint{6.802397in}{5.835703in}}{\pgfqpoint{6.810210in}{5.843517in}}%
\pgfpathcurveto{\pgfqpoint{6.818024in}{5.851331in}}{\pgfqpoint{6.822414in}{5.861930in}}{\pgfqpoint{6.822414in}{5.872980in}}%
\pgfpathcurveto{\pgfqpoint{6.822414in}{5.884030in}}{\pgfqpoint{6.818024in}{5.894629in}}{\pgfqpoint{6.810210in}{5.902442in}}%
\pgfpathcurveto{\pgfqpoint{6.802397in}{5.910256in}}{\pgfqpoint{6.791798in}{5.914646in}}{\pgfqpoint{6.780747in}{5.914646in}}%
\pgfpathcurveto{\pgfqpoint{6.769697in}{5.914646in}}{\pgfqpoint{6.759098in}{5.910256in}}{\pgfqpoint{6.751285in}{5.902442in}}%
\pgfpathcurveto{\pgfqpoint{6.743471in}{5.894629in}}{\pgfqpoint{6.739081in}{5.884030in}}{\pgfqpoint{6.739081in}{5.872980in}}%
\pgfpathcurveto{\pgfqpoint{6.739081in}{5.861930in}}{\pgfqpoint{6.743471in}{5.851331in}}{\pgfqpoint{6.751285in}{5.843517in}}%
\pgfpathcurveto{\pgfqpoint{6.759098in}{5.835703in}}{\pgfqpoint{6.769697in}{5.831313in}}{\pgfqpoint{6.780747in}{5.831313in}}%
\pgfpathlineto{\pgfqpoint{6.780747in}{5.831313in}}%
\pgfpathclose%
\pgfusepath{stroke,fill}%
\end{pgfscope}%
\begin{pgfscope}%
\pgfpathrectangle{\pgfqpoint{5.292946in}{5.272501in}}{\pgfqpoint{2.177280in}{2.201755in}}%
\pgfusepath{clip}%
\pgfsetbuttcap%
\pgfsetroundjoin%
\definecolor{currentfill}{rgb}{0.172549,0.627451,0.172549}%
\pgfsetfillcolor{currentfill}%
\pgfsetlinewidth{0.481800pt}%
\definecolor{currentstroke}{rgb}{1.000000,1.000000,1.000000}%
\pgfsetstrokecolor{currentstroke}%
\pgfsetdash{}{0pt}%
\pgfpathmoveto{\pgfqpoint{6.924274in}{6.164912in}}%
\pgfpathcurveto{\pgfqpoint{6.935324in}{6.164912in}}{\pgfqpoint{6.945923in}{6.169303in}}{\pgfqpoint{6.953737in}{6.177116in}}%
\pgfpathcurveto{\pgfqpoint{6.961550in}{6.184930in}}{\pgfqpoint{6.965941in}{6.195529in}}{\pgfqpoint{6.965941in}{6.206579in}}%
\pgfpathcurveto{\pgfqpoint{6.965941in}{6.217629in}}{\pgfqpoint{6.961550in}{6.228228in}}{\pgfqpoint{6.953737in}{6.236042in}}%
\pgfpathcurveto{\pgfqpoint{6.945923in}{6.243855in}}{\pgfqpoint{6.935324in}{6.248246in}}{\pgfqpoint{6.924274in}{6.248246in}}%
\pgfpathcurveto{\pgfqpoint{6.913224in}{6.248246in}}{\pgfqpoint{6.902625in}{6.243855in}}{\pgfqpoint{6.894811in}{6.236042in}}%
\pgfpathcurveto{\pgfqpoint{6.886997in}{6.228228in}}{\pgfqpoint{6.882607in}{6.217629in}}{\pgfqpoint{6.882607in}{6.206579in}}%
\pgfpathcurveto{\pgfqpoint{6.882607in}{6.195529in}}{\pgfqpoint{6.886997in}{6.184930in}}{\pgfqpoint{6.894811in}{6.177116in}}%
\pgfpathcurveto{\pgfqpoint{6.902625in}{6.169303in}}{\pgfqpoint{6.913224in}{6.164912in}}{\pgfqpoint{6.924274in}{6.164912in}}%
\pgfpathlineto{\pgfqpoint{6.924274in}{6.164912in}}%
\pgfpathclose%
\pgfusepath{stroke,fill}%
\end{pgfscope}%
\begin{pgfscope}%
\pgfpathrectangle{\pgfqpoint{5.292946in}{5.272501in}}{\pgfqpoint{2.177280in}{2.201755in}}%
\pgfusepath{clip}%
\pgfsetbuttcap%
\pgfsetroundjoin%
\definecolor{currentfill}{rgb}{0.172549,0.627451,0.172549}%
\pgfsetfillcolor{currentfill}%
\pgfsetlinewidth{0.481800pt}%
\definecolor{currentstroke}{rgb}{1.000000,1.000000,1.000000}%
\pgfsetstrokecolor{currentstroke}%
\pgfsetdash{}{0pt}%
\pgfpathmoveto{\pgfqpoint{6.780747in}{6.498512in}}%
\pgfpathcurveto{\pgfqpoint{6.791798in}{6.498512in}}{\pgfqpoint{6.802397in}{6.502902in}}{\pgfqpoint{6.810210in}{6.510715in}}%
\pgfpathcurveto{\pgfqpoint{6.818024in}{6.518529in}}{\pgfqpoint{6.822414in}{6.529128in}}{\pgfqpoint{6.822414in}{6.540178in}}%
\pgfpathcurveto{\pgfqpoint{6.822414in}{6.551228in}}{\pgfqpoint{6.818024in}{6.561827in}}{\pgfqpoint{6.810210in}{6.569641in}}%
\pgfpathcurveto{\pgfqpoint{6.802397in}{6.577455in}}{\pgfqpoint{6.791798in}{6.581845in}}{\pgfqpoint{6.780747in}{6.581845in}}%
\pgfpathcurveto{\pgfqpoint{6.769697in}{6.581845in}}{\pgfqpoint{6.759098in}{6.577455in}}{\pgfqpoint{6.751285in}{6.569641in}}%
\pgfpathcurveto{\pgfqpoint{6.743471in}{6.561827in}}{\pgfqpoint{6.739081in}{6.551228in}}{\pgfqpoint{6.739081in}{6.540178in}}%
\pgfpathcurveto{\pgfqpoint{6.739081in}{6.529128in}}{\pgfqpoint{6.743471in}{6.518529in}}{\pgfqpoint{6.751285in}{6.510715in}}%
\pgfpathcurveto{\pgfqpoint{6.759098in}{6.502902in}}{\pgfqpoint{6.769697in}{6.498512in}}{\pgfqpoint{6.780747in}{6.498512in}}%
\pgfpathlineto{\pgfqpoint{6.780747in}{6.498512in}}%
\pgfpathclose%
\pgfusepath{stroke,fill}%
\end{pgfscope}%
\begin{pgfscope}%
\pgfpathrectangle{\pgfqpoint{5.292946in}{5.272501in}}{\pgfqpoint{2.177280in}{2.201755in}}%
\pgfusepath{clip}%
\pgfsetbuttcap%
\pgfsetroundjoin%
\definecolor{currentfill}{rgb}{0.172549,0.627451,0.172549}%
\pgfsetfillcolor{currentfill}%
\pgfsetlinewidth{0.481800pt}%
\definecolor{currentstroke}{rgb}{1.000000,1.000000,1.000000}%
\pgfsetstrokecolor{currentstroke}%
\pgfsetdash{}{0pt}%
\pgfpathmoveto{\pgfqpoint{6.752042in}{6.248312in}}%
\pgfpathcurveto{\pgfqpoint{6.763092in}{6.248312in}}{\pgfqpoint{6.773691in}{6.252702in}}{\pgfqpoint{6.781505in}{6.260516in}}%
\pgfpathcurveto{\pgfqpoint{6.789319in}{6.268330in}}{\pgfqpoint{6.793709in}{6.278929in}}{\pgfqpoint{6.793709in}{6.289979in}}%
\pgfpathcurveto{\pgfqpoint{6.793709in}{6.301029in}}{\pgfqpoint{6.789319in}{6.311628in}}{\pgfqpoint{6.781505in}{6.319442in}}%
\pgfpathcurveto{\pgfqpoint{6.773691in}{6.327255in}}{\pgfqpoint{6.763092in}{6.331645in}}{\pgfqpoint{6.752042in}{6.331645in}}%
\pgfpathcurveto{\pgfqpoint{6.740992in}{6.331645in}}{\pgfqpoint{6.730393in}{6.327255in}}{\pgfqpoint{6.722579in}{6.319442in}}%
\pgfpathcurveto{\pgfqpoint{6.714766in}{6.311628in}}{\pgfqpoint{6.710375in}{6.301029in}}{\pgfqpoint{6.710375in}{6.289979in}}%
\pgfpathcurveto{\pgfqpoint{6.710375in}{6.278929in}}{\pgfqpoint{6.714766in}{6.268330in}}{\pgfqpoint{6.722579in}{6.260516in}}%
\pgfpathcurveto{\pgfqpoint{6.730393in}{6.252702in}}{\pgfqpoint{6.740992in}{6.248312in}}{\pgfqpoint{6.752042in}{6.248312in}}%
\pgfpathlineto{\pgfqpoint{6.752042in}{6.248312in}}%
\pgfpathclose%
\pgfusepath{stroke,fill}%
\end{pgfscope}%
\begin{pgfscope}%
\pgfpathrectangle{\pgfqpoint{5.292946in}{5.272501in}}{\pgfqpoint{2.177280in}{2.201755in}}%
\pgfusepath{clip}%
\pgfsetbuttcap%
\pgfsetroundjoin%
\definecolor{currentfill}{rgb}{0.172549,0.627451,0.172549}%
\pgfsetfillcolor{currentfill}%
\pgfsetlinewidth{0.481800pt}%
\definecolor{currentstroke}{rgb}{1.000000,1.000000,1.000000}%
\pgfsetstrokecolor{currentstroke}%
\pgfsetdash{}{0pt}%
\pgfpathmoveto{\pgfqpoint{6.551105in}{6.164912in}}%
\pgfpathcurveto{\pgfqpoint{6.562155in}{6.164912in}}{\pgfqpoint{6.572754in}{6.169303in}}{\pgfqpoint{6.580568in}{6.177116in}}%
\pgfpathcurveto{\pgfqpoint{6.588382in}{6.184930in}}{\pgfqpoint{6.592772in}{6.195529in}}{\pgfqpoint{6.592772in}{6.206579in}}%
\pgfpathcurveto{\pgfqpoint{6.592772in}{6.217629in}}{\pgfqpoint{6.588382in}{6.228228in}}{\pgfqpoint{6.580568in}{6.236042in}}%
\pgfpathcurveto{\pgfqpoint{6.572754in}{6.243855in}}{\pgfqpoint{6.562155in}{6.248246in}}{\pgfqpoint{6.551105in}{6.248246in}}%
\pgfpathcurveto{\pgfqpoint{6.540055in}{6.248246in}}{\pgfqpoint{6.529456in}{6.243855in}}{\pgfqpoint{6.521642in}{6.236042in}}%
\pgfpathcurveto{\pgfqpoint{6.513829in}{6.228228in}}{\pgfqpoint{6.509438in}{6.217629in}}{\pgfqpoint{6.509438in}{6.206579in}}%
\pgfpathcurveto{\pgfqpoint{6.509438in}{6.195529in}}{\pgfqpoint{6.513829in}{6.184930in}}{\pgfqpoint{6.521642in}{6.177116in}}%
\pgfpathcurveto{\pgfqpoint{6.529456in}{6.169303in}}{\pgfqpoint{6.540055in}{6.164912in}}{\pgfqpoint{6.551105in}{6.164912in}}%
\pgfpathlineto{\pgfqpoint{6.551105in}{6.164912in}}%
\pgfpathclose%
\pgfusepath{stroke,fill}%
\end{pgfscope}%
\begin{pgfscope}%
\pgfpathrectangle{\pgfqpoint{5.292946in}{5.272501in}}{\pgfqpoint{2.177280in}{2.201755in}}%
\pgfusepath{clip}%
\pgfsetbuttcap%
\pgfsetroundjoin%
\definecolor{currentfill}{rgb}{0.172549,0.627451,0.172549}%
\pgfsetfillcolor{currentfill}%
\pgfsetlinewidth{0.481800pt}%
\definecolor{currentstroke}{rgb}{1.000000,1.000000,1.000000}%
\pgfsetstrokecolor{currentstroke}%
\pgfsetdash{}{0pt}%
\pgfpathmoveto{\pgfqpoint{6.723337in}{6.248312in}}%
\pgfpathcurveto{\pgfqpoint{6.734387in}{6.248312in}}{\pgfqpoint{6.744986in}{6.252702in}}{\pgfqpoint{6.752800in}{6.260516in}}%
\pgfpathcurveto{\pgfqpoint{6.760613in}{6.268330in}}{\pgfqpoint{6.765004in}{6.278929in}}{\pgfqpoint{6.765004in}{6.289979in}}%
\pgfpathcurveto{\pgfqpoint{6.765004in}{6.301029in}}{\pgfqpoint{6.760613in}{6.311628in}}{\pgfqpoint{6.752800in}{6.319442in}}%
\pgfpathcurveto{\pgfqpoint{6.744986in}{6.327255in}}{\pgfqpoint{6.734387in}{6.331645in}}{\pgfqpoint{6.723337in}{6.331645in}}%
\pgfpathcurveto{\pgfqpoint{6.712287in}{6.331645in}}{\pgfqpoint{6.701688in}{6.327255in}}{\pgfqpoint{6.693874in}{6.319442in}}%
\pgfpathcurveto{\pgfqpoint{6.686060in}{6.311628in}}{\pgfqpoint{6.681670in}{6.301029in}}{\pgfqpoint{6.681670in}{6.289979in}}%
\pgfpathcurveto{\pgfqpoint{6.681670in}{6.278929in}}{\pgfqpoint{6.686060in}{6.268330in}}{\pgfqpoint{6.693874in}{6.260516in}}%
\pgfpathcurveto{\pgfqpoint{6.701688in}{6.252702in}}{\pgfqpoint{6.712287in}{6.248312in}}{\pgfqpoint{6.723337in}{6.248312in}}%
\pgfpathlineto{\pgfqpoint{6.723337in}{6.248312in}}%
\pgfpathclose%
\pgfusepath{stroke,fill}%
\end{pgfscope}%
\begin{pgfscope}%
\pgfpathrectangle{\pgfqpoint{5.292946in}{5.272501in}}{\pgfqpoint{2.177280in}{2.201755in}}%
\pgfusepath{clip}%
\pgfsetbuttcap%
\pgfsetroundjoin%
\definecolor{currentfill}{rgb}{0.172549,0.627451,0.172549}%
\pgfsetfillcolor{currentfill}%
\pgfsetlinewidth{0.481800pt}%
\definecolor{currentstroke}{rgb}{1.000000,1.000000,1.000000}%
\pgfsetstrokecolor{currentstroke}%
\pgfsetdash{}{0pt}%
\pgfpathmoveto{\pgfqpoint{6.780747in}{6.248312in}}%
\pgfpathcurveto{\pgfqpoint{6.791798in}{6.248312in}}{\pgfqpoint{6.802397in}{6.252702in}}{\pgfqpoint{6.810210in}{6.260516in}}%
\pgfpathcurveto{\pgfqpoint{6.818024in}{6.268330in}}{\pgfqpoint{6.822414in}{6.278929in}}{\pgfqpoint{6.822414in}{6.289979in}}%
\pgfpathcurveto{\pgfqpoint{6.822414in}{6.301029in}}{\pgfqpoint{6.818024in}{6.311628in}}{\pgfqpoint{6.810210in}{6.319442in}}%
\pgfpathcurveto{\pgfqpoint{6.802397in}{6.327255in}}{\pgfqpoint{6.791798in}{6.331645in}}{\pgfqpoint{6.780747in}{6.331645in}}%
\pgfpathcurveto{\pgfqpoint{6.769697in}{6.331645in}}{\pgfqpoint{6.759098in}{6.327255in}}{\pgfqpoint{6.751285in}{6.319442in}}%
\pgfpathcurveto{\pgfqpoint{6.743471in}{6.311628in}}{\pgfqpoint{6.739081in}{6.301029in}}{\pgfqpoint{6.739081in}{6.289979in}}%
\pgfpathcurveto{\pgfqpoint{6.739081in}{6.278929in}}{\pgfqpoint{6.743471in}{6.268330in}}{\pgfqpoint{6.751285in}{6.260516in}}%
\pgfpathcurveto{\pgfqpoint{6.759098in}{6.252702in}}{\pgfqpoint{6.769697in}{6.248312in}}{\pgfqpoint{6.780747in}{6.248312in}}%
\pgfpathlineto{\pgfqpoint{6.780747in}{6.248312in}}%
\pgfpathclose%
\pgfusepath{stroke,fill}%
\end{pgfscope}%
\begin{pgfscope}%
\pgfpathrectangle{\pgfqpoint{5.292946in}{5.272501in}}{\pgfqpoint{2.177280in}{2.201755in}}%
\pgfusepath{clip}%
\pgfsetbuttcap%
\pgfsetroundjoin%
\definecolor{currentfill}{rgb}{0.172549,0.627451,0.172549}%
\pgfsetfillcolor{currentfill}%
\pgfsetlinewidth{0.481800pt}%
\definecolor{currentstroke}{rgb}{1.000000,1.000000,1.000000}%
\pgfsetstrokecolor{currentstroke}%
\pgfsetdash{}{0pt}%
\pgfpathmoveto{\pgfqpoint{6.637221in}{6.248312in}}%
\pgfpathcurveto{\pgfqpoint{6.648271in}{6.248312in}}{\pgfqpoint{6.658870in}{6.252702in}}{\pgfqpoint{6.666684in}{6.260516in}}%
\pgfpathcurveto{\pgfqpoint{6.674497in}{6.268330in}}{\pgfqpoint{6.678888in}{6.278929in}}{\pgfqpoint{6.678888in}{6.289979in}}%
\pgfpathcurveto{\pgfqpoint{6.678888in}{6.301029in}}{\pgfqpoint{6.674497in}{6.311628in}}{\pgfqpoint{6.666684in}{6.319442in}}%
\pgfpathcurveto{\pgfqpoint{6.658870in}{6.327255in}}{\pgfqpoint{6.648271in}{6.331645in}}{\pgfqpoint{6.637221in}{6.331645in}}%
\pgfpathcurveto{\pgfqpoint{6.626171in}{6.331645in}}{\pgfqpoint{6.615572in}{6.327255in}}{\pgfqpoint{6.607758in}{6.319442in}}%
\pgfpathcurveto{\pgfqpoint{6.599945in}{6.311628in}}{\pgfqpoint{6.595554in}{6.301029in}}{\pgfqpoint{6.595554in}{6.289979in}}%
\pgfpathcurveto{\pgfqpoint{6.595554in}{6.278929in}}{\pgfqpoint{6.599945in}{6.268330in}}{\pgfqpoint{6.607758in}{6.260516in}}%
\pgfpathcurveto{\pgfqpoint{6.615572in}{6.252702in}}{\pgfqpoint{6.626171in}{6.248312in}}{\pgfqpoint{6.637221in}{6.248312in}}%
\pgfpathlineto{\pgfqpoint{6.637221in}{6.248312in}}%
\pgfpathclose%
\pgfusepath{stroke,fill}%
\end{pgfscope}%
\begin{pgfscope}%
\pgfpathrectangle{\pgfqpoint{5.292946in}{5.272501in}}{\pgfqpoint{2.177280in}{2.201755in}}%
\pgfusepath{clip}%
\pgfsetbuttcap%
\pgfsetroundjoin%
\definecolor{currentfill}{rgb}{0.172549,0.627451,0.172549}%
\pgfsetfillcolor{currentfill}%
\pgfsetlinewidth{0.481800pt}%
\definecolor{currentstroke}{rgb}{1.000000,1.000000,1.000000}%
\pgfsetstrokecolor{currentstroke}%
\pgfsetdash{}{0pt}%
\pgfpathmoveto{\pgfqpoint{6.637221in}{5.914713in}}%
\pgfpathcurveto{\pgfqpoint{6.648271in}{5.914713in}}{\pgfqpoint{6.658870in}{5.919103in}}{\pgfqpoint{6.666684in}{5.926917in}}%
\pgfpathcurveto{\pgfqpoint{6.674497in}{5.934730in}}{\pgfqpoint{6.678888in}{5.945329in}}{\pgfqpoint{6.678888in}{5.956379in}}%
\pgfpathcurveto{\pgfqpoint{6.678888in}{5.967430in}}{\pgfqpoint{6.674497in}{5.978029in}}{\pgfqpoint{6.666684in}{5.985842in}}%
\pgfpathcurveto{\pgfqpoint{6.658870in}{5.993656in}}{\pgfqpoint{6.648271in}{5.998046in}}{\pgfqpoint{6.637221in}{5.998046in}}%
\pgfpathcurveto{\pgfqpoint{6.626171in}{5.998046in}}{\pgfqpoint{6.615572in}{5.993656in}}{\pgfqpoint{6.607758in}{5.985842in}}%
\pgfpathcurveto{\pgfqpoint{6.599945in}{5.978029in}}{\pgfqpoint{6.595554in}{5.967430in}}{\pgfqpoint{6.595554in}{5.956379in}}%
\pgfpathcurveto{\pgfqpoint{6.595554in}{5.945329in}}{\pgfqpoint{6.599945in}{5.934730in}}{\pgfqpoint{6.607758in}{5.926917in}}%
\pgfpathcurveto{\pgfqpoint{6.615572in}{5.919103in}}{\pgfqpoint{6.626171in}{5.914713in}}{\pgfqpoint{6.637221in}{5.914713in}}%
\pgfpathlineto{\pgfqpoint{6.637221in}{5.914713in}}%
\pgfpathclose%
\pgfusepath{stroke,fill}%
\end{pgfscope}%
\begin{pgfscope}%
\pgfpathrectangle{\pgfqpoint{5.292946in}{5.272501in}}{\pgfqpoint{2.177280in}{2.201755in}}%
\pgfusepath{clip}%
\pgfsetbuttcap%
\pgfsetroundjoin%
\definecolor{currentfill}{rgb}{0.172549,0.627451,0.172549}%
\pgfsetfillcolor{currentfill}%
\pgfsetlinewidth{0.481800pt}%
\definecolor{currentstroke}{rgb}{1.000000,1.000000,1.000000}%
\pgfsetstrokecolor{currentstroke}%
\pgfsetdash{}{0pt}%
\pgfpathmoveto{\pgfqpoint{6.866863in}{6.331712in}}%
\pgfpathcurveto{\pgfqpoint{6.877913in}{6.331712in}}{\pgfqpoint{6.888512in}{6.336102in}}{\pgfqpoint{6.896326in}{6.343916in}}%
\pgfpathcurveto{\pgfqpoint{6.904140in}{6.351729in}}{\pgfqpoint{6.908530in}{6.362328in}}{\pgfqpoint{6.908530in}{6.373379in}}%
\pgfpathcurveto{\pgfqpoint{6.908530in}{6.384429in}}{\pgfqpoint{6.904140in}{6.395028in}}{\pgfqpoint{6.896326in}{6.402841in}}%
\pgfpathcurveto{\pgfqpoint{6.888512in}{6.410655in}}{\pgfqpoint{6.877913in}{6.415045in}}{\pgfqpoint{6.866863in}{6.415045in}}%
\pgfpathcurveto{\pgfqpoint{6.855813in}{6.415045in}}{\pgfqpoint{6.845214in}{6.410655in}}{\pgfqpoint{6.837401in}{6.402841in}}%
\pgfpathcurveto{\pgfqpoint{6.829587in}{6.395028in}}{\pgfqpoint{6.825197in}{6.384429in}}{\pgfqpoint{6.825197in}{6.373379in}}%
\pgfpathcurveto{\pgfqpoint{6.825197in}{6.362328in}}{\pgfqpoint{6.829587in}{6.351729in}}{\pgfqpoint{6.837401in}{6.343916in}}%
\pgfpathcurveto{\pgfqpoint{6.845214in}{6.336102in}}{\pgfqpoint{6.855813in}{6.331712in}}{\pgfqpoint{6.866863in}{6.331712in}}%
\pgfpathlineto{\pgfqpoint{6.866863in}{6.331712in}}%
\pgfpathclose%
\pgfusepath{stroke,fill}%
\end{pgfscope}%
\begin{pgfscope}%
\pgfpathrectangle{\pgfqpoint{5.292946in}{5.272501in}}{\pgfqpoint{2.177280in}{2.201755in}}%
\pgfusepath{clip}%
\pgfsetbuttcap%
\pgfsetroundjoin%
\definecolor{currentfill}{rgb}{0.172549,0.627451,0.172549}%
\pgfsetfillcolor{currentfill}%
\pgfsetlinewidth{0.481800pt}%
\definecolor{currentstroke}{rgb}{1.000000,1.000000,1.000000}%
\pgfsetstrokecolor{currentstroke}%
\pgfsetdash{}{0pt}%
\pgfpathmoveto{\pgfqpoint{6.809453in}{6.415112in}}%
\pgfpathcurveto{\pgfqpoint{6.820503in}{6.415112in}}{\pgfqpoint{6.831102in}{6.419502in}}{\pgfqpoint{6.838915in}{6.427316in}}%
\pgfpathcurveto{\pgfqpoint{6.846729in}{6.435129in}}{\pgfqpoint{6.851119in}{6.445728in}}{\pgfqpoint{6.851119in}{6.456778in}}%
\pgfpathcurveto{\pgfqpoint{6.851119in}{6.467828in}}{\pgfqpoint{6.846729in}{6.478428in}}{\pgfqpoint{6.838915in}{6.486241in}}%
\pgfpathcurveto{\pgfqpoint{6.831102in}{6.494055in}}{\pgfqpoint{6.820503in}{6.498445in}}{\pgfqpoint{6.809453in}{6.498445in}}%
\pgfpathcurveto{\pgfqpoint{6.798403in}{6.498445in}}{\pgfqpoint{6.787804in}{6.494055in}}{\pgfqpoint{6.779990in}{6.486241in}}%
\pgfpathcurveto{\pgfqpoint{6.772176in}{6.478428in}}{\pgfqpoint{6.767786in}{6.467828in}}{\pgfqpoint{6.767786in}{6.456778in}}%
\pgfpathcurveto{\pgfqpoint{6.767786in}{6.445728in}}{\pgfqpoint{6.772176in}{6.435129in}}{\pgfqpoint{6.779990in}{6.427316in}}%
\pgfpathcurveto{\pgfqpoint{6.787804in}{6.419502in}}{\pgfqpoint{6.798403in}{6.415112in}}{\pgfqpoint{6.809453in}{6.415112in}}%
\pgfpathlineto{\pgfqpoint{6.809453in}{6.415112in}}%
\pgfpathclose%
\pgfusepath{stroke,fill}%
\end{pgfscope}%
\begin{pgfscope}%
\pgfpathrectangle{\pgfqpoint{5.292946in}{5.272501in}}{\pgfqpoint{2.177280in}{2.201755in}}%
\pgfusepath{clip}%
\pgfsetbuttcap%
\pgfsetroundjoin%
\definecolor{currentfill}{rgb}{0.172549,0.627451,0.172549}%
\pgfsetfillcolor{currentfill}%
\pgfsetlinewidth{0.481800pt}%
\definecolor{currentstroke}{rgb}{1.000000,1.000000,1.000000}%
\pgfsetstrokecolor{currentstroke}%
\pgfsetdash{}{0pt}%
\pgfpathmoveto{\pgfqpoint{6.665926in}{6.164912in}}%
\pgfpathcurveto{\pgfqpoint{6.676976in}{6.164912in}}{\pgfqpoint{6.687575in}{6.169303in}}{\pgfqpoint{6.695389in}{6.177116in}}%
\pgfpathcurveto{\pgfqpoint{6.703203in}{6.184930in}}{\pgfqpoint{6.707593in}{6.195529in}}{\pgfqpoint{6.707593in}{6.206579in}}%
\pgfpathcurveto{\pgfqpoint{6.707593in}{6.217629in}}{\pgfqpoint{6.703203in}{6.228228in}}{\pgfqpoint{6.695389in}{6.236042in}}%
\pgfpathcurveto{\pgfqpoint{6.687575in}{6.243855in}}{\pgfqpoint{6.676976in}{6.248246in}}{\pgfqpoint{6.665926in}{6.248246in}}%
\pgfpathcurveto{\pgfqpoint{6.654876in}{6.248246in}}{\pgfqpoint{6.644277in}{6.243855in}}{\pgfqpoint{6.636464in}{6.236042in}}%
\pgfpathcurveto{\pgfqpoint{6.628650in}{6.228228in}}{\pgfqpoint{6.624260in}{6.217629in}}{\pgfqpoint{6.624260in}{6.206579in}}%
\pgfpathcurveto{\pgfqpoint{6.624260in}{6.195529in}}{\pgfqpoint{6.628650in}{6.184930in}}{\pgfqpoint{6.636464in}{6.177116in}}%
\pgfpathcurveto{\pgfqpoint{6.644277in}{6.169303in}}{\pgfqpoint{6.654876in}{6.164912in}}{\pgfqpoint{6.665926in}{6.164912in}}%
\pgfpathlineto{\pgfqpoint{6.665926in}{6.164912in}}%
\pgfpathclose%
\pgfusepath{stroke,fill}%
\end{pgfscope}%
\begin{pgfscope}%
\pgfpathrectangle{\pgfqpoint{5.292946in}{5.272501in}}{\pgfqpoint{2.177280in}{2.201755in}}%
\pgfusepath{clip}%
\pgfsetbuttcap%
\pgfsetroundjoin%
\definecolor{currentfill}{rgb}{0.172549,0.627451,0.172549}%
\pgfsetfillcolor{currentfill}%
\pgfsetlinewidth{0.481800pt}%
\definecolor{currentstroke}{rgb}{1.000000,1.000000,1.000000}%
\pgfsetstrokecolor{currentstroke}%
\pgfsetdash{}{0pt}%
\pgfpathmoveto{\pgfqpoint{6.608516in}{5.747913in}}%
\pgfpathcurveto{\pgfqpoint{6.619566in}{5.747913in}}{\pgfqpoint{6.630165in}{5.752303in}}{\pgfqpoint{6.637978in}{5.760117in}}%
\pgfpathcurveto{\pgfqpoint{6.645792in}{5.767931in}}{\pgfqpoint{6.650182in}{5.778530in}}{\pgfqpoint{6.650182in}{5.789580in}}%
\pgfpathcurveto{\pgfqpoint{6.650182in}{5.800630in}}{\pgfqpoint{6.645792in}{5.811229in}}{\pgfqpoint{6.637978in}{5.819043in}}%
\pgfpathcurveto{\pgfqpoint{6.630165in}{5.826856in}}{\pgfqpoint{6.619566in}{5.831247in}}{\pgfqpoint{6.608516in}{5.831247in}}%
\pgfpathcurveto{\pgfqpoint{6.597466in}{5.831247in}}{\pgfqpoint{6.586867in}{5.826856in}}{\pgfqpoint{6.579053in}{5.819043in}}%
\pgfpathcurveto{\pgfqpoint{6.571239in}{5.811229in}}{\pgfqpoint{6.566849in}{5.800630in}}{\pgfqpoint{6.566849in}{5.789580in}}%
\pgfpathcurveto{\pgfqpoint{6.566849in}{5.778530in}}{\pgfqpoint{6.571239in}{5.767931in}}{\pgfqpoint{6.579053in}{5.760117in}}%
\pgfpathcurveto{\pgfqpoint{6.586867in}{5.752303in}}{\pgfqpoint{6.597466in}{5.747913in}}{\pgfqpoint{6.608516in}{5.747913in}}%
\pgfpathlineto{\pgfqpoint{6.608516in}{5.747913in}}%
\pgfpathclose%
\pgfusepath{stroke,fill}%
\end{pgfscope}%
\begin{pgfscope}%
\pgfpathrectangle{\pgfqpoint{5.292946in}{5.272501in}}{\pgfqpoint{2.177280in}{2.201755in}}%
\pgfusepath{clip}%
\pgfsetbuttcap%
\pgfsetroundjoin%
\definecolor{currentfill}{rgb}{0.172549,0.627451,0.172549}%
\pgfsetfillcolor{currentfill}%
\pgfsetlinewidth{0.481800pt}%
\definecolor{currentstroke}{rgb}{1.000000,1.000000,1.000000}%
\pgfsetstrokecolor{currentstroke}%
\pgfsetdash{}{0pt}%
\pgfpathmoveto{\pgfqpoint{6.665926in}{6.164912in}}%
\pgfpathcurveto{\pgfqpoint{6.676976in}{6.164912in}}{\pgfqpoint{6.687575in}{6.169303in}}{\pgfqpoint{6.695389in}{6.177116in}}%
\pgfpathcurveto{\pgfqpoint{6.703203in}{6.184930in}}{\pgfqpoint{6.707593in}{6.195529in}}{\pgfqpoint{6.707593in}{6.206579in}}%
\pgfpathcurveto{\pgfqpoint{6.707593in}{6.217629in}}{\pgfqpoint{6.703203in}{6.228228in}}{\pgfqpoint{6.695389in}{6.236042in}}%
\pgfpathcurveto{\pgfqpoint{6.687575in}{6.243855in}}{\pgfqpoint{6.676976in}{6.248246in}}{\pgfqpoint{6.665926in}{6.248246in}}%
\pgfpathcurveto{\pgfqpoint{6.654876in}{6.248246in}}{\pgfqpoint{6.644277in}{6.243855in}}{\pgfqpoint{6.636464in}{6.236042in}}%
\pgfpathcurveto{\pgfqpoint{6.628650in}{6.228228in}}{\pgfqpoint{6.624260in}{6.217629in}}{\pgfqpoint{6.624260in}{6.206579in}}%
\pgfpathcurveto{\pgfqpoint{6.624260in}{6.195529in}}{\pgfqpoint{6.628650in}{6.184930in}}{\pgfqpoint{6.636464in}{6.177116in}}%
\pgfpathcurveto{\pgfqpoint{6.644277in}{6.169303in}}{\pgfqpoint{6.654876in}{6.164912in}}{\pgfqpoint{6.665926in}{6.164912in}}%
\pgfpathlineto{\pgfqpoint{6.665926in}{6.164912in}}%
\pgfpathclose%
\pgfusepath{stroke,fill}%
\end{pgfscope}%
\begin{pgfscope}%
\pgfpathrectangle{\pgfqpoint{5.292946in}{5.272501in}}{\pgfqpoint{2.177280in}{2.201755in}}%
\pgfusepath{clip}%
\pgfsetbuttcap%
\pgfsetroundjoin%
\definecolor{currentfill}{rgb}{0.172549,0.627451,0.172549}%
\pgfsetfillcolor{currentfill}%
\pgfsetlinewidth{0.481800pt}%
\definecolor{currentstroke}{rgb}{1.000000,1.000000,1.000000}%
\pgfsetstrokecolor{currentstroke}%
\pgfsetdash{}{0pt}%
\pgfpathmoveto{\pgfqpoint{6.723337in}{6.498512in}}%
\pgfpathcurveto{\pgfqpoint{6.734387in}{6.498512in}}{\pgfqpoint{6.744986in}{6.502902in}}{\pgfqpoint{6.752800in}{6.510715in}}%
\pgfpathcurveto{\pgfqpoint{6.760613in}{6.518529in}}{\pgfqpoint{6.765004in}{6.529128in}}{\pgfqpoint{6.765004in}{6.540178in}}%
\pgfpathcurveto{\pgfqpoint{6.765004in}{6.551228in}}{\pgfqpoint{6.760613in}{6.561827in}}{\pgfqpoint{6.752800in}{6.569641in}}%
\pgfpathcurveto{\pgfqpoint{6.744986in}{6.577455in}}{\pgfqpoint{6.734387in}{6.581845in}}{\pgfqpoint{6.723337in}{6.581845in}}%
\pgfpathcurveto{\pgfqpoint{6.712287in}{6.581845in}}{\pgfqpoint{6.701688in}{6.577455in}}{\pgfqpoint{6.693874in}{6.569641in}}%
\pgfpathcurveto{\pgfqpoint{6.686060in}{6.561827in}}{\pgfqpoint{6.681670in}{6.551228in}}{\pgfqpoint{6.681670in}{6.540178in}}%
\pgfpathcurveto{\pgfqpoint{6.681670in}{6.529128in}}{\pgfqpoint{6.686060in}{6.518529in}}{\pgfqpoint{6.693874in}{6.510715in}}%
\pgfpathcurveto{\pgfqpoint{6.701688in}{6.502902in}}{\pgfqpoint{6.712287in}{6.498512in}}{\pgfqpoint{6.723337in}{6.498512in}}%
\pgfpathlineto{\pgfqpoint{6.723337in}{6.498512in}}%
\pgfpathclose%
\pgfusepath{stroke,fill}%
\end{pgfscope}%
\begin{pgfscope}%
\pgfpathrectangle{\pgfqpoint{5.292946in}{5.272501in}}{\pgfqpoint{2.177280in}{2.201755in}}%
\pgfusepath{clip}%
\pgfsetbuttcap%
\pgfsetroundjoin%
\definecolor{currentfill}{rgb}{0.172549,0.627451,0.172549}%
\pgfsetfillcolor{currentfill}%
\pgfsetlinewidth{0.481800pt}%
\definecolor{currentstroke}{rgb}{1.000000,1.000000,1.000000}%
\pgfsetstrokecolor{currentstroke}%
\pgfsetdash{}{0pt}%
\pgfpathmoveto{\pgfqpoint{6.637221in}{6.164912in}}%
\pgfpathcurveto{\pgfqpoint{6.648271in}{6.164912in}}{\pgfqpoint{6.658870in}{6.169303in}}{\pgfqpoint{6.666684in}{6.177116in}}%
\pgfpathcurveto{\pgfqpoint{6.674497in}{6.184930in}}{\pgfqpoint{6.678888in}{6.195529in}}{\pgfqpoint{6.678888in}{6.206579in}}%
\pgfpathcurveto{\pgfqpoint{6.678888in}{6.217629in}}{\pgfqpoint{6.674497in}{6.228228in}}{\pgfqpoint{6.666684in}{6.236042in}}%
\pgfpathcurveto{\pgfqpoint{6.658870in}{6.243855in}}{\pgfqpoint{6.648271in}{6.248246in}}{\pgfqpoint{6.637221in}{6.248246in}}%
\pgfpathcurveto{\pgfqpoint{6.626171in}{6.248246in}}{\pgfqpoint{6.615572in}{6.243855in}}{\pgfqpoint{6.607758in}{6.236042in}}%
\pgfpathcurveto{\pgfqpoint{6.599945in}{6.228228in}}{\pgfqpoint{6.595554in}{6.217629in}}{\pgfqpoint{6.595554in}{6.206579in}}%
\pgfpathcurveto{\pgfqpoint{6.595554in}{6.195529in}}{\pgfqpoint{6.599945in}{6.184930in}}{\pgfqpoint{6.607758in}{6.177116in}}%
\pgfpathcurveto{\pgfqpoint{6.615572in}{6.169303in}}{\pgfqpoint{6.626171in}{6.164912in}}{\pgfqpoint{6.637221in}{6.164912in}}%
\pgfpathlineto{\pgfqpoint{6.637221in}{6.164912in}}%
\pgfpathclose%
\pgfusepath{stroke,fill}%
\end{pgfscope}%
\begin{pgfscope}%
\pgfpathrectangle{\pgfqpoint{5.292946in}{5.272501in}}{\pgfqpoint{2.177280in}{2.201755in}}%
\pgfusepath{clip}%
\pgfsetbuttcap%
\pgfsetroundjoin%
\definecolor{currentfill}{rgb}{0.121569,0.466667,0.705882}%
\pgfsetfillcolor{currentfill}%
\pgfsetlinewidth{1.003750pt}%
\definecolor{currentstroke}{rgb}{0.121569,0.466667,0.705882}%
\pgfsetstrokecolor{currentstroke}%
\pgfsetdash{}{0pt}%
\pgfsys@defobject{currentmarker}{\pgfqpoint{-0.041667in}{-0.041667in}}{\pgfqpoint{0.041667in}{0.041667in}}{%
\pgfpathmoveto{\pgfqpoint{0.000000in}{-0.041667in}}%
\pgfpathcurveto{\pgfqpoint{0.011050in}{-0.041667in}}{\pgfqpoint{0.021649in}{-0.037276in}}{\pgfqpoint{0.029463in}{-0.029463in}}%
\pgfpathcurveto{\pgfqpoint{0.037276in}{-0.021649in}}{\pgfqpoint{0.041667in}{-0.011050in}}{\pgfqpoint{0.041667in}{0.000000in}}%
\pgfpathcurveto{\pgfqpoint{0.041667in}{0.011050in}}{\pgfqpoint{0.037276in}{0.021649in}}{\pgfqpoint{0.029463in}{0.029463in}}%
\pgfpathcurveto{\pgfqpoint{0.021649in}{0.037276in}}{\pgfqpoint{0.011050in}{0.041667in}}{\pgfqpoint{0.000000in}{0.041667in}}%
\pgfpathcurveto{\pgfqpoint{-0.011050in}{0.041667in}}{\pgfqpoint{-0.021649in}{0.037276in}}{\pgfqpoint{-0.029463in}{0.029463in}}%
\pgfpathcurveto{\pgfqpoint{-0.037276in}{0.021649in}}{\pgfqpoint{-0.041667in}{0.011050in}}{\pgfqpoint{-0.041667in}{0.000000in}}%
\pgfpathcurveto{\pgfqpoint{-0.041667in}{-0.011050in}}{\pgfqpoint{-0.037276in}{-0.021649in}}{\pgfqpoint{-0.029463in}{-0.029463in}}%
\pgfpathcurveto{\pgfqpoint{-0.021649in}{-0.037276in}}{\pgfqpoint{-0.011050in}{-0.041667in}}{\pgfqpoint{0.000000in}{-0.041667in}}%
\pgfpathlineto{\pgfqpoint{0.000000in}{-0.041667in}}%
\pgfpathclose%
\pgfusepath{stroke,fill}%
}%
\end{pgfscope}%
\begin{pgfscope}%
\pgfpathrectangle{\pgfqpoint{5.292946in}{5.272501in}}{\pgfqpoint{2.177280in}{2.201755in}}%
\pgfusepath{clip}%
\pgfsetbuttcap%
\pgfsetroundjoin%
\definecolor{currentfill}{rgb}{1.000000,0.498039,0.054902}%
\pgfsetfillcolor{currentfill}%
\pgfsetlinewidth{1.003750pt}%
\definecolor{currentstroke}{rgb}{1.000000,0.498039,0.054902}%
\pgfsetstrokecolor{currentstroke}%
\pgfsetdash{}{0pt}%
\pgfsys@defobject{currentmarker}{\pgfqpoint{-0.041667in}{-0.041667in}}{\pgfqpoint{0.041667in}{0.041667in}}{%
\pgfpathmoveto{\pgfqpoint{0.000000in}{-0.041667in}}%
\pgfpathcurveto{\pgfqpoint{0.011050in}{-0.041667in}}{\pgfqpoint{0.021649in}{-0.037276in}}{\pgfqpoint{0.029463in}{-0.029463in}}%
\pgfpathcurveto{\pgfqpoint{0.037276in}{-0.021649in}}{\pgfqpoint{0.041667in}{-0.011050in}}{\pgfqpoint{0.041667in}{0.000000in}}%
\pgfpathcurveto{\pgfqpoint{0.041667in}{0.011050in}}{\pgfqpoint{0.037276in}{0.021649in}}{\pgfqpoint{0.029463in}{0.029463in}}%
\pgfpathcurveto{\pgfqpoint{0.021649in}{0.037276in}}{\pgfqpoint{0.011050in}{0.041667in}}{\pgfqpoint{0.000000in}{0.041667in}}%
\pgfpathcurveto{\pgfqpoint{-0.011050in}{0.041667in}}{\pgfqpoint{-0.021649in}{0.037276in}}{\pgfqpoint{-0.029463in}{0.029463in}}%
\pgfpathcurveto{\pgfqpoint{-0.037276in}{0.021649in}}{\pgfqpoint{-0.041667in}{0.011050in}}{\pgfqpoint{-0.041667in}{0.000000in}}%
\pgfpathcurveto{\pgfqpoint{-0.041667in}{-0.011050in}}{\pgfqpoint{-0.037276in}{-0.021649in}}{\pgfqpoint{-0.029463in}{-0.029463in}}%
\pgfpathcurveto{\pgfqpoint{-0.021649in}{-0.037276in}}{\pgfqpoint{-0.011050in}{-0.041667in}}{\pgfqpoint{0.000000in}{-0.041667in}}%
\pgfpathlineto{\pgfqpoint{0.000000in}{-0.041667in}}%
\pgfpathclose%
\pgfusepath{stroke,fill}%
}%
\end{pgfscope}%
\begin{pgfscope}%
\pgfpathrectangle{\pgfqpoint{5.292946in}{5.272501in}}{\pgfqpoint{2.177280in}{2.201755in}}%
\pgfusepath{clip}%
\pgfsetbuttcap%
\pgfsetroundjoin%
\definecolor{currentfill}{rgb}{0.172549,0.627451,0.172549}%
\pgfsetfillcolor{currentfill}%
\pgfsetlinewidth{1.003750pt}%
\definecolor{currentstroke}{rgb}{0.172549,0.627451,0.172549}%
\pgfsetstrokecolor{currentstroke}%
\pgfsetdash{}{0pt}%
\pgfsys@defobject{currentmarker}{\pgfqpoint{-0.041667in}{-0.041667in}}{\pgfqpoint{0.041667in}{0.041667in}}{%
\pgfpathmoveto{\pgfqpoint{0.000000in}{-0.041667in}}%
\pgfpathcurveto{\pgfqpoint{0.011050in}{-0.041667in}}{\pgfqpoint{0.021649in}{-0.037276in}}{\pgfqpoint{0.029463in}{-0.029463in}}%
\pgfpathcurveto{\pgfqpoint{0.037276in}{-0.021649in}}{\pgfqpoint{0.041667in}{-0.011050in}}{\pgfqpoint{0.041667in}{0.000000in}}%
\pgfpathcurveto{\pgfqpoint{0.041667in}{0.011050in}}{\pgfqpoint{0.037276in}{0.021649in}}{\pgfqpoint{0.029463in}{0.029463in}}%
\pgfpathcurveto{\pgfqpoint{0.021649in}{0.037276in}}{\pgfqpoint{0.011050in}{0.041667in}}{\pgfqpoint{0.000000in}{0.041667in}}%
\pgfpathcurveto{\pgfqpoint{-0.011050in}{0.041667in}}{\pgfqpoint{-0.021649in}{0.037276in}}{\pgfqpoint{-0.029463in}{0.029463in}}%
\pgfpathcurveto{\pgfqpoint{-0.037276in}{0.021649in}}{\pgfqpoint{-0.041667in}{0.011050in}}{\pgfqpoint{-0.041667in}{0.000000in}}%
\pgfpathcurveto{\pgfqpoint{-0.041667in}{-0.011050in}}{\pgfqpoint{-0.037276in}{-0.021649in}}{\pgfqpoint{-0.029463in}{-0.029463in}}%
\pgfpathcurveto{\pgfqpoint{-0.021649in}{-0.037276in}}{\pgfqpoint{-0.011050in}{-0.041667in}}{\pgfqpoint{0.000000in}{-0.041667in}}%
\pgfpathlineto{\pgfqpoint{0.000000in}{-0.041667in}}%
\pgfpathclose%
\pgfusepath{stroke,fill}%
}%
\end{pgfscope}%
\begin{pgfscope}%
\pgfsetbuttcap%
\pgfsetroundjoin%
\definecolor{currentfill}{rgb}{0.000000,0.000000,0.000000}%
\pgfsetfillcolor{currentfill}%
\pgfsetlinewidth{0.803000pt}%
\definecolor{currentstroke}{rgb}{0.000000,0.000000,0.000000}%
\pgfsetstrokecolor{currentstroke}%
\pgfsetdash{}{0pt}%
\pgfsys@defobject{currentmarker}{\pgfqpoint{0.000000in}{-0.048611in}}{\pgfqpoint{0.000000in}{0.000000in}}{%
\pgfpathmoveto{\pgfqpoint{0.000000in}{0.000000in}}%
\pgfpathlineto{\pgfqpoint{0.000000in}{-0.048611in}}%
\pgfusepath{stroke,fill}%
}%
\begin{pgfscope}%
\pgfsys@transformshift{5.747357in}{5.272501in}%
\pgfsys@useobject{currentmarker}{}%
\end{pgfscope}%
\end{pgfscope}%
\begin{pgfscope}%
\pgfsetbuttcap%
\pgfsetroundjoin%
\definecolor{currentfill}{rgb}{0.000000,0.000000,0.000000}%
\pgfsetfillcolor{currentfill}%
\pgfsetlinewidth{0.803000pt}%
\definecolor{currentstroke}{rgb}{0.000000,0.000000,0.000000}%
\pgfsetstrokecolor{currentstroke}%
\pgfsetdash{}{0pt}%
\pgfsys@defobject{currentmarker}{\pgfqpoint{0.000000in}{-0.048611in}}{\pgfqpoint{0.000000in}{0.000000in}}{%
\pgfpathmoveto{\pgfqpoint{0.000000in}{0.000000in}}%
\pgfpathlineto{\pgfqpoint{0.000000in}{-0.048611in}}%
\pgfusepath{stroke,fill}%
}%
\begin{pgfscope}%
\pgfsys@transformshift{6.321463in}{5.272501in}%
\pgfsys@useobject{currentmarker}{}%
\end{pgfscope}%
\end{pgfscope}%
\begin{pgfscope}%
\pgfsetbuttcap%
\pgfsetroundjoin%
\definecolor{currentfill}{rgb}{0.000000,0.000000,0.000000}%
\pgfsetfillcolor{currentfill}%
\pgfsetlinewidth{0.803000pt}%
\definecolor{currentstroke}{rgb}{0.000000,0.000000,0.000000}%
\pgfsetstrokecolor{currentstroke}%
\pgfsetdash{}{0pt}%
\pgfsys@defobject{currentmarker}{\pgfqpoint{0.000000in}{-0.048611in}}{\pgfqpoint{0.000000in}{0.000000in}}{%
\pgfpathmoveto{\pgfqpoint{0.000000in}{0.000000in}}%
\pgfpathlineto{\pgfqpoint{0.000000in}{-0.048611in}}%
\pgfusepath{stroke,fill}%
}%
\begin{pgfscope}%
\pgfsys@transformshift{6.895569in}{5.272501in}%
\pgfsys@useobject{currentmarker}{}%
\end{pgfscope}%
\end{pgfscope}%
\begin{pgfscope}%
\pgfsetbuttcap%
\pgfsetroundjoin%
\definecolor{currentfill}{rgb}{0.000000,0.000000,0.000000}%
\pgfsetfillcolor{currentfill}%
\pgfsetlinewidth{0.803000pt}%
\definecolor{currentstroke}{rgb}{0.000000,0.000000,0.000000}%
\pgfsetstrokecolor{currentstroke}%
\pgfsetdash{}{0pt}%
\pgfsys@defobject{currentmarker}{\pgfqpoint{0.000000in}{-0.048611in}}{\pgfqpoint{0.000000in}{0.000000in}}{%
\pgfpathmoveto{\pgfqpoint{0.000000in}{0.000000in}}%
\pgfpathlineto{\pgfqpoint{0.000000in}{-0.048611in}}%
\pgfusepath{stroke,fill}%
}%
\begin{pgfscope}%
\pgfsys@transformshift{7.469674in}{5.272501in}%
\pgfsys@useobject{currentmarker}{}%
\end{pgfscope}%
\end{pgfscope}%
\begin{pgfscope}%
\pgfsetbuttcap%
\pgfsetroundjoin%
\definecolor{currentfill}{rgb}{0.000000,0.000000,0.000000}%
\pgfsetfillcolor{currentfill}%
\pgfsetlinewidth{0.803000pt}%
\definecolor{currentstroke}{rgb}{0.000000,0.000000,0.000000}%
\pgfsetstrokecolor{currentstroke}%
\pgfsetdash{}{0pt}%
\pgfsys@defobject{currentmarker}{\pgfqpoint{-0.048611in}{0.000000in}}{\pgfqpoint{-0.000000in}{0.000000in}}{%
\pgfpathmoveto{\pgfqpoint{-0.000000in}{0.000000in}}%
\pgfpathlineto{\pgfqpoint{-0.048611in}{0.000000in}}%
\pgfusepath{stroke,fill}%
}%
\begin{pgfscope}%
\pgfsys@transformshift{5.292946in}{5.372581in}%
\pgfsys@useobject{currentmarker}{}%
\end{pgfscope}%
\end{pgfscope}%
\begin{pgfscope}%
\pgfsetbuttcap%
\pgfsetroundjoin%
\definecolor{currentfill}{rgb}{0.000000,0.000000,0.000000}%
\pgfsetfillcolor{currentfill}%
\pgfsetlinewidth{0.803000pt}%
\definecolor{currentstroke}{rgb}{0.000000,0.000000,0.000000}%
\pgfsetstrokecolor{currentstroke}%
\pgfsetdash{}{0pt}%
\pgfsys@defobject{currentmarker}{\pgfqpoint{-0.048611in}{0.000000in}}{\pgfqpoint{-0.000000in}{0.000000in}}{%
\pgfpathmoveto{\pgfqpoint{-0.000000in}{0.000000in}}%
\pgfpathlineto{\pgfqpoint{-0.048611in}{0.000000in}}%
\pgfusepath{stroke,fill}%
}%
\begin{pgfscope}%
\pgfsys@transformshift{5.292946in}{5.789580in}%
\pgfsys@useobject{currentmarker}{}%
\end{pgfscope}%
\end{pgfscope}%
\begin{pgfscope}%
\pgfsetbuttcap%
\pgfsetroundjoin%
\definecolor{currentfill}{rgb}{0.000000,0.000000,0.000000}%
\pgfsetfillcolor{currentfill}%
\pgfsetlinewidth{0.803000pt}%
\definecolor{currentstroke}{rgb}{0.000000,0.000000,0.000000}%
\pgfsetstrokecolor{currentstroke}%
\pgfsetdash{}{0pt}%
\pgfsys@defobject{currentmarker}{\pgfqpoint{-0.048611in}{0.000000in}}{\pgfqpoint{-0.000000in}{0.000000in}}{%
\pgfpathmoveto{\pgfqpoint{-0.000000in}{0.000000in}}%
\pgfpathlineto{\pgfqpoint{-0.048611in}{0.000000in}}%
\pgfusepath{stroke,fill}%
}%
\begin{pgfscope}%
\pgfsys@transformshift{5.292946in}{6.206579in}%
\pgfsys@useobject{currentmarker}{}%
\end{pgfscope}%
\end{pgfscope}%
\begin{pgfscope}%
\pgfsetbuttcap%
\pgfsetroundjoin%
\definecolor{currentfill}{rgb}{0.000000,0.000000,0.000000}%
\pgfsetfillcolor{currentfill}%
\pgfsetlinewidth{0.803000pt}%
\definecolor{currentstroke}{rgb}{0.000000,0.000000,0.000000}%
\pgfsetstrokecolor{currentstroke}%
\pgfsetdash{}{0pt}%
\pgfsys@defobject{currentmarker}{\pgfqpoint{-0.048611in}{0.000000in}}{\pgfqpoint{-0.000000in}{0.000000in}}{%
\pgfpathmoveto{\pgfqpoint{-0.000000in}{0.000000in}}%
\pgfpathlineto{\pgfqpoint{-0.048611in}{0.000000in}}%
\pgfusepath{stroke,fill}%
}%
\begin{pgfscope}%
\pgfsys@transformshift{5.292946in}{6.623578in}%
\pgfsys@useobject{currentmarker}{}%
\end{pgfscope}%
\end{pgfscope}%
\begin{pgfscope}%
\pgfsetbuttcap%
\pgfsetroundjoin%
\definecolor{currentfill}{rgb}{0.000000,0.000000,0.000000}%
\pgfsetfillcolor{currentfill}%
\pgfsetlinewidth{0.803000pt}%
\definecolor{currentstroke}{rgb}{0.000000,0.000000,0.000000}%
\pgfsetstrokecolor{currentstroke}%
\pgfsetdash{}{0pt}%
\pgfsys@defobject{currentmarker}{\pgfqpoint{-0.048611in}{0.000000in}}{\pgfqpoint{-0.000000in}{0.000000in}}{%
\pgfpathmoveto{\pgfqpoint{-0.000000in}{0.000000in}}%
\pgfpathlineto{\pgfqpoint{-0.048611in}{0.000000in}}%
\pgfusepath{stroke,fill}%
}%
\begin{pgfscope}%
\pgfsys@transformshift{5.292946in}{7.040577in}%
\pgfsys@useobject{currentmarker}{}%
\end{pgfscope}%
\end{pgfscope}%
\begin{pgfscope}%
\pgfsetbuttcap%
\pgfsetroundjoin%
\definecolor{currentfill}{rgb}{0.000000,0.000000,0.000000}%
\pgfsetfillcolor{currentfill}%
\pgfsetlinewidth{0.803000pt}%
\definecolor{currentstroke}{rgb}{0.000000,0.000000,0.000000}%
\pgfsetstrokecolor{currentstroke}%
\pgfsetdash{}{0pt}%
\pgfsys@defobject{currentmarker}{\pgfqpoint{-0.048611in}{0.000000in}}{\pgfqpoint{-0.000000in}{0.000000in}}{%
\pgfpathmoveto{\pgfqpoint{-0.000000in}{0.000000in}}%
\pgfpathlineto{\pgfqpoint{-0.048611in}{0.000000in}}%
\pgfusepath{stroke,fill}%
}%
\begin{pgfscope}%
\pgfsys@transformshift{5.292946in}{7.457576in}%
\pgfsys@useobject{currentmarker}{}%
\end{pgfscope}%
\end{pgfscope}%
\begin{pgfscope}%
\pgfsetrectcap%
\pgfsetmiterjoin%
\pgfsetlinewidth{0.803000pt}%
\definecolor{currentstroke}{rgb}{0.000000,0.000000,0.000000}%
\pgfsetstrokecolor{currentstroke}%
\pgfsetdash{}{0pt}%
\pgfpathmoveto{\pgfqpoint{5.292946in}{5.272501in}}%
\pgfpathlineto{\pgfqpoint{5.292946in}{7.474256in}}%
\pgfusepath{stroke}%
\end{pgfscope}%
\begin{pgfscope}%
\pgfsetrectcap%
\pgfsetmiterjoin%
\pgfsetlinewidth{0.803000pt}%
\definecolor{currentstroke}{rgb}{0.000000,0.000000,0.000000}%
\pgfsetstrokecolor{currentstroke}%
\pgfsetdash{}{0pt}%
\pgfpathmoveto{\pgfqpoint{5.292946in}{5.272501in}}%
\pgfpathlineto{\pgfqpoint{7.470226in}{5.272501in}}%
\pgfusepath{stroke}%
\end{pgfscope}%
\begin{pgfscope}%
\pgfsetbuttcap%
\pgfsetmiterjoin%
\definecolor{currentfill}{rgb}{1.000000,1.000000,1.000000}%
\pgfsetfillcolor{currentfill}%
\pgfsetlinewidth{0.000000pt}%
\definecolor{currentstroke}{rgb}{0.000000,0.000000,0.000000}%
\pgfsetstrokecolor{currentstroke}%
\pgfsetstrokeopacity{0.000000}%
\pgfsetdash{}{0pt}%
\pgfpathmoveto{\pgfqpoint{7.622482in}{5.272501in}}%
\pgfpathlineto{\pgfqpoint{9.799762in}{5.272501in}}%
\pgfpathlineto{\pgfqpoint{9.799762in}{7.474256in}}%
\pgfpathlineto{\pgfqpoint{7.622482in}{7.474256in}}%
\pgfpathlineto{\pgfqpoint{7.622482in}{5.272501in}}%
\pgfpathclose%
\pgfusepath{fill}%
\end{pgfscope}%
\begin{pgfscope}%
\pgfpathrectangle{\pgfqpoint{7.622482in}{5.272501in}}{\pgfqpoint{2.177280in}{2.201755in}}%
\pgfusepath{clip}%
\pgfsetbuttcap%
\pgfsetroundjoin%
\definecolor{currentfill}{rgb}{0.121569,0.466667,0.705882}%
\pgfsetfillcolor{currentfill}%
\pgfsetlinewidth{0.481800pt}%
\definecolor{currentstroke}{rgb}{1.000000,1.000000,1.000000}%
\pgfsetstrokecolor{currentstroke}%
\pgfsetdash{}{0pt}%
\pgfpathmoveto{\pgfqpoint{7.887162in}{6.581911in}}%
\pgfpathcurveto{\pgfqpoint{7.898212in}{6.581911in}}{\pgfqpoint{7.908811in}{6.586302in}}{\pgfqpoint{7.916625in}{6.594115in}}%
\pgfpathcurveto{\pgfqpoint{7.924438in}{6.601929in}}{\pgfqpoint{7.928828in}{6.612528in}}{\pgfqpoint{7.928828in}{6.623578in}}%
\pgfpathcurveto{\pgfqpoint{7.928828in}{6.634628in}}{\pgfqpoint{7.924438in}{6.645227in}}{\pgfqpoint{7.916625in}{6.653041in}}%
\pgfpathcurveto{\pgfqpoint{7.908811in}{6.660854in}}{\pgfqpoint{7.898212in}{6.665245in}}{\pgfqpoint{7.887162in}{6.665245in}}%
\pgfpathcurveto{\pgfqpoint{7.876112in}{6.665245in}}{\pgfqpoint{7.865513in}{6.660854in}}{\pgfqpoint{7.857699in}{6.653041in}}%
\pgfpathcurveto{\pgfqpoint{7.849885in}{6.645227in}}{\pgfqpoint{7.845495in}{6.634628in}}{\pgfqpoint{7.845495in}{6.623578in}}%
\pgfpathcurveto{\pgfqpoint{7.845495in}{6.612528in}}{\pgfqpoint{7.849885in}{6.601929in}}{\pgfqpoint{7.857699in}{6.594115in}}%
\pgfpathcurveto{\pgfqpoint{7.865513in}{6.586302in}}{\pgfqpoint{7.876112in}{6.581911in}}{\pgfqpoint{7.887162in}{6.581911in}}%
\pgfpathlineto{\pgfqpoint{7.887162in}{6.581911in}}%
\pgfpathclose%
\pgfusepath{stroke,fill}%
\end{pgfscope}%
\begin{pgfscope}%
\pgfpathrectangle{\pgfqpoint{7.622482in}{5.272501in}}{\pgfqpoint{2.177280in}{2.201755in}}%
\pgfusepath{clip}%
\pgfsetbuttcap%
\pgfsetroundjoin%
\definecolor{currentfill}{rgb}{0.121569,0.466667,0.705882}%
\pgfsetfillcolor{currentfill}%
\pgfsetlinewidth{0.481800pt}%
\definecolor{currentstroke}{rgb}{1.000000,1.000000,1.000000}%
\pgfsetstrokecolor{currentstroke}%
\pgfsetdash{}{0pt}%
\pgfpathmoveto{\pgfqpoint{7.887162in}{6.164912in}}%
\pgfpathcurveto{\pgfqpoint{7.898212in}{6.164912in}}{\pgfqpoint{7.908811in}{6.169303in}}{\pgfqpoint{7.916625in}{6.177116in}}%
\pgfpathcurveto{\pgfqpoint{7.924438in}{6.184930in}}{\pgfqpoint{7.928828in}{6.195529in}}{\pgfqpoint{7.928828in}{6.206579in}}%
\pgfpathcurveto{\pgfqpoint{7.928828in}{6.217629in}}{\pgfqpoint{7.924438in}{6.228228in}}{\pgfqpoint{7.916625in}{6.236042in}}%
\pgfpathcurveto{\pgfqpoint{7.908811in}{6.243855in}}{\pgfqpoint{7.898212in}{6.248246in}}{\pgfqpoint{7.887162in}{6.248246in}}%
\pgfpathcurveto{\pgfqpoint{7.876112in}{6.248246in}}{\pgfqpoint{7.865513in}{6.243855in}}{\pgfqpoint{7.857699in}{6.236042in}}%
\pgfpathcurveto{\pgfqpoint{7.849885in}{6.228228in}}{\pgfqpoint{7.845495in}{6.217629in}}{\pgfqpoint{7.845495in}{6.206579in}}%
\pgfpathcurveto{\pgfqpoint{7.845495in}{6.195529in}}{\pgfqpoint{7.849885in}{6.184930in}}{\pgfqpoint{7.857699in}{6.177116in}}%
\pgfpathcurveto{\pgfqpoint{7.865513in}{6.169303in}}{\pgfqpoint{7.876112in}{6.164912in}}{\pgfqpoint{7.887162in}{6.164912in}}%
\pgfpathlineto{\pgfqpoint{7.887162in}{6.164912in}}%
\pgfpathclose%
\pgfusepath{stroke,fill}%
\end{pgfscope}%
\begin{pgfscope}%
\pgfpathrectangle{\pgfqpoint{7.622482in}{5.272501in}}{\pgfqpoint{2.177280in}{2.201755in}}%
\pgfusepath{clip}%
\pgfsetbuttcap%
\pgfsetroundjoin%
\definecolor{currentfill}{rgb}{0.121569,0.466667,0.705882}%
\pgfsetfillcolor{currentfill}%
\pgfsetlinewidth{0.481800pt}%
\definecolor{currentstroke}{rgb}{1.000000,1.000000,1.000000}%
\pgfsetstrokecolor{currentstroke}%
\pgfsetdash{}{0pt}%
\pgfpathmoveto{\pgfqpoint{7.887162in}{6.331712in}}%
\pgfpathcurveto{\pgfqpoint{7.898212in}{6.331712in}}{\pgfqpoint{7.908811in}{6.336102in}}{\pgfqpoint{7.916625in}{6.343916in}}%
\pgfpathcurveto{\pgfqpoint{7.924438in}{6.351729in}}{\pgfqpoint{7.928828in}{6.362328in}}{\pgfqpoint{7.928828in}{6.373379in}}%
\pgfpathcurveto{\pgfqpoint{7.928828in}{6.384429in}}{\pgfqpoint{7.924438in}{6.395028in}}{\pgfqpoint{7.916625in}{6.402841in}}%
\pgfpathcurveto{\pgfqpoint{7.908811in}{6.410655in}}{\pgfqpoint{7.898212in}{6.415045in}}{\pgfqpoint{7.887162in}{6.415045in}}%
\pgfpathcurveto{\pgfqpoint{7.876112in}{6.415045in}}{\pgfqpoint{7.865513in}{6.410655in}}{\pgfqpoint{7.857699in}{6.402841in}}%
\pgfpathcurveto{\pgfqpoint{7.849885in}{6.395028in}}{\pgfqpoint{7.845495in}{6.384429in}}{\pgfqpoint{7.845495in}{6.373379in}}%
\pgfpathcurveto{\pgfqpoint{7.845495in}{6.362328in}}{\pgfqpoint{7.849885in}{6.351729in}}{\pgfqpoint{7.857699in}{6.343916in}}%
\pgfpathcurveto{\pgfqpoint{7.865513in}{6.336102in}}{\pgfqpoint{7.876112in}{6.331712in}}{\pgfqpoint{7.887162in}{6.331712in}}%
\pgfpathlineto{\pgfqpoint{7.887162in}{6.331712in}}%
\pgfpathclose%
\pgfusepath{stroke,fill}%
\end{pgfscope}%
\begin{pgfscope}%
\pgfpathrectangle{\pgfqpoint{7.622482in}{5.272501in}}{\pgfqpoint{2.177280in}{2.201755in}}%
\pgfusepath{clip}%
\pgfsetbuttcap%
\pgfsetroundjoin%
\definecolor{currentfill}{rgb}{0.121569,0.466667,0.705882}%
\pgfsetfillcolor{currentfill}%
\pgfsetlinewidth{0.481800pt}%
\definecolor{currentstroke}{rgb}{1.000000,1.000000,1.000000}%
\pgfsetstrokecolor{currentstroke}%
\pgfsetdash{}{0pt}%
\pgfpathmoveto{\pgfqpoint{7.887162in}{6.248312in}}%
\pgfpathcurveto{\pgfqpoint{7.898212in}{6.248312in}}{\pgfqpoint{7.908811in}{6.252702in}}{\pgfqpoint{7.916625in}{6.260516in}}%
\pgfpathcurveto{\pgfqpoint{7.924438in}{6.268330in}}{\pgfqpoint{7.928828in}{6.278929in}}{\pgfqpoint{7.928828in}{6.289979in}}%
\pgfpathcurveto{\pgfqpoint{7.928828in}{6.301029in}}{\pgfqpoint{7.924438in}{6.311628in}}{\pgfqpoint{7.916625in}{6.319442in}}%
\pgfpathcurveto{\pgfqpoint{7.908811in}{6.327255in}}{\pgfqpoint{7.898212in}{6.331645in}}{\pgfqpoint{7.887162in}{6.331645in}}%
\pgfpathcurveto{\pgfqpoint{7.876112in}{6.331645in}}{\pgfqpoint{7.865513in}{6.327255in}}{\pgfqpoint{7.857699in}{6.319442in}}%
\pgfpathcurveto{\pgfqpoint{7.849885in}{6.311628in}}{\pgfqpoint{7.845495in}{6.301029in}}{\pgfqpoint{7.845495in}{6.289979in}}%
\pgfpathcurveto{\pgfqpoint{7.845495in}{6.278929in}}{\pgfqpoint{7.849885in}{6.268330in}}{\pgfqpoint{7.857699in}{6.260516in}}%
\pgfpathcurveto{\pgfqpoint{7.865513in}{6.252702in}}{\pgfqpoint{7.876112in}{6.248312in}}{\pgfqpoint{7.887162in}{6.248312in}}%
\pgfpathlineto{\pgfqpoint{7.887162in}{6.248312in}}%
\pgfpathclose%
\pgfusepath{stroke,fill}%
\end{pgfscope}%
\begin{pgfscope}%
\pgfpathrectangle{\pgfqpoint{7.622482in}{5.272501in}}{\pgfqpoint{2.177280in}{2.201755in}}%
\pgfusepath{clip}%
\pgfsetbuttcap%
\pgfsetroundjoin%
\definecolor{currentfill}{rgb}{0.121569,0.466667,0.705882}%
\pgfsetfillcolor{currentfill}%
\pgfsetlinewidth{0.481800pt}%
\definecolor{currentstroke}{rgb}{1.000000,1.000000,1.000000}%
\pgfsetstrokecolor{currentstroke}%
\pgfsetdash{}{0pt}%
\pgfpathmoveto{\pgfqpoint{7.887162in}{6.665311in}}%
\pgfpathcurveto{\pgfqpoint{7.898212in}{6.665311in}}{\pgfqpoint{7.908811in}{6.669701in}}{\pgfqpoint{7.916625in}{6.677515in}}%
\pgfpathcurveto{\pgfqpoint{7.924438in}{6.685329in}}{\pgfqpoint{7.928828in}{6.695928in}}{\pgfqpoint{7.928828in}{6.706978in}}%
\pgfpathcurveto{\pgfqpoint{7.928828in}{6.718028in}}{\pgfqpoint{7.924438in}{6.728627in}}{\pgfqpoint{7.916625in}{6.736441in}}%
\pgfpathcurveto{\pgfqpoint{7.908811in}{6.744254in}}{\pgfqpoint{7.898212in}{6.748644in}}{\pgfqpoint{7.887162in}{6.748644in}}%
\pgfpathcurveto{\pgfqpoint{7.876112in}{6.748644in}}{\pgfqpoint{7.865513in}{6.744254in}}{\pgfqpoint{7.857699in}{6.736441in}}%
\pgfpathcurveto{\pgfqpoint{7.849885in}{6.728627in}}{\pgfqpoint{7.845495in}{6.718028in}}{\pgfqpoint{7.845495in}{6.706978in}}%
\pgfpathcurveto{\pgfqpoint{7.845495in}{6.695928in}}{\pgfqpoint{7.849885in}{6.685329in}}{\pgfqpoint{7.857699in}{6.677515in}}%
\pgfpathcurveto{\pgfqpoint{7.865513in}{6.669701in}}{\pgfqpoint{7.876112in}{6.665311in}}{\pgfqpoint{7.887162in}{6.665311in}}%
\pgfpathlineto{\pgfqpoint{7.887162in}{6.665311in}}%
\pgfpathclose%
\pgfusepath{stroke,fill}%
\end{pgfscope}%
\begin{pgfscope}%
\pgfpathrectangle{\pgfqpoint{7.622482in}{5.272501in}}{\pgfqpoint{2.177280in}{2.201755in}}%
\pgfusepath{clip}%
\pgfsetbuttcap%
\pgfsetroundjoin%
\definecolor{currentfill}{rgb}{0.121569,0.466667,0.705882}%
\pgfsetfillcolor{currentfill}%
\pgfsetlinewidth{0.481800pt}%
\definecolor{currentstroke}{rgb}{1.000000,1.000000,1.000000}%
\pgfsetstrokecolor{currentstroke}%
\pgfsetdash{}{0pt}%
\pgfpathmoveto{\pgfqpoint{8.022670in}{6.915511in}}%
\pgfpathcurveto{\pgfqpoint{8.033720in}{6.915511in}}{\pgfqpoint{8.044319in}{6.919901in}}{\pgfqpoint{8.052132in}{6.927714in}}%
\pgfpathcurveto{\pgfqpoint{8.059946in}{6.935528in}}{\pgfqpoint{8.064336in}{6.946127in}}{\pgfqpoint{8.064336in}{6.957177in}}%
\pgfpathcurveto{\pgfqpoint{8.064336in}{6.968227in}}{\pgfqpoint{8.059946in}{6.978826in}}{\pgfqpoint{8.052132in}{6.986640in}}%
\pgfpathcurveto{\pgfqpoint{8.044319in}{6.994454in}}{\pgfqpoint{8.033720in}{6.998844in}}{\pgfqpoint{8.022670in}{6.998844in}}%
\pgfpathcurveto{\pgfqpoint{8.011619in}{6.998844in}}{\pgfqpoint{8.001020in}{6.994454in}}{\pgfqpoint{7.993207in}{6.986640in}}%
\pgfpathcurveto{\pgfqpoint{7.985393in}{6.978826in}}{\pgfqpoint{7.981003in}{6.968227in}}{\pgfqpoint{7.981003in}{6.957177in}}%
\pgfpathcurveto{\pgfqpoint{7.981003in}{6.946127in}}{\pgfqpoint{7.985393in}{6.935528in}}{\pgfqpoint{7.993207in}{6.927714in}}%
\pgfpathcurveto{\pgfqpoint{8.001020in}{6.919901in}}{\pgfqpoint{8.011619in}{6.915511in}}{\pgfqpoint{8.022670in}{6.915511in}}%
\pgfpathlineto{\pgfqpoint{8.022670in}{6.915511in}}%
\pgfpathclose%
\pgfusepath{stroke,fill}%
\end{pgfscope}%
\begin{pgfscope}%
\pgfpathrectangle{\pgfqpoint{7.622482in}{5.272501in}}{\pgfqpoint{2.177280in}{2.201755in}}%
\pgfusepath{clip}%
\pgfsetbuttcap%
\pgfsetroundjoin%
\definecolor{currentfill}{rgb}{0.121569,0.466667,0.705882}%
\pgfsetfillcolor{currentfill}%
\pgfsetlinewidth{0.481800pt}%
\definecolor{currentstroke}{rgb}{1.000000,1.000000,1.000000}%
\pgfsetstrokecolor{currentstroke}%
\pgfsetdash{}{0pt}%
\pgfpathmoveto{\pgfqpoint{7.954916in}{6.498512in}}%
\pgfpathcurveto{\pgfqpoint{7.965966in}{6.498512in}}{\pgfqpoint{7.976565in}{6.502902in}}{\pgfqpoint{7.984378in}{6.510715in}}%
\pgfpathcurveto{\pgfqpoint{7.992192in}{6.518529in}}{\pgfqpoint{7.996582in}{6.529128in}}{\pgfqpoint{7.996582in}{6.540178in}}%
\pgfpathcurveto{\pgfqpoint{7.996582in}{6.551228in}}{\pgfqpoint{7.992192in}{6.561827in}}{\pgfqpoint{7.984378in}{6.569641in}}%
\pgfpathcurveto{\pgfqpoint{7.976565in}{6.577455in}}{\pgfqpoint{7.965966in}{6.581845in}}{\pgfqpoint{7.954916in}{6.581845in}}%
\pgfpathcurveto{\pgfqpoint{7.943866in}{6.581845in}}{\pgfqpoint{7.933267in}{6.577455in}}{\pgfqpoint{7.925453in}{6.569641in}}%
\pgfpathcurveto{\pgfqpoint{7.917639in}{6.561827in}}{\pgfqpoint{7.913249in}{6.551228in}}{\pgfqpoint{7.913249in}{6.540178in}}%
\pgfpathcurveto{\pgfqpoint{7.913249in}{6.529128in}}{\pgfqpoint{7.917639in}{6.518529in}}{\pgfqpoint{7.925453in}{6.510715in}}%
\pgfpathcurveto{\pgfqpoint{7.933267in}{6.502902in}}{\pgfqpoint{7.943866in}{6.498512in}}{\pgfqpoint{7.954916in}{6.498512in}}%
\pgfpathlineto{\pgfqpoint{7.954916in}{6.498512in}}%
\pgfpathclose%
\pgfusepath{stroke,fill}%
\end{pgfscope}%
\begin{pgfscope}%
\pgfpathrectangle{\pgfqpoint{7.622482in}{5.272501in}}{\pgfqpoint{2.177280in}{2.201755in}}%
\pgfusepath{clip}%
\pgfsetbuttcap%
\pgfsetroundjoin%
\definecolor{currentfill}{rgb}{0.121569,0.466667,0.705882}%
\pgfsetfillcolor{currentfill}%
\pgfsetlinewidth{0.481800pt}%
\definecolor{currentstroke}{rgb}{1.000000,1.000000,1.000000}%
\pgfsetstrokecolor{currentstroke}%
\pgfsetdash{}{0pt}%
\pgfpathmoveto{\pgfqpoint{7.887162in}{6.498512in}}%
\pgfpathcurveto{\pgfqpoint{7.898212in}{6.498512in}}{\pgfqpoint{7.908811in}{6.502902in}}{\pgfqpoint{7.916625in}{6.510715in}}%
\pgfpathcurveto{\pgfqpoint{7.924438in}{6.518529in}}{\pgfqpoint{7.928828in}{6.529128in}}{\pgfqpoint{7.928828in}{6.540178in}}%
\pgfpathcurveto{\pgfqpoint{7.928828in}{6.551228in}}{\pgfqpoint{7.924438in}{6.561827in}}{\pgfqpoint{7.916625in}{6.569641in}}%
\pgfpathcurveto{\pgfqpoint{7.908811in}{6.577455in}}{\pgfqpoint{7.898212in}{6.581845in}}{\pgfqpoint{7.887162in}{6.581845in}}%
\pgfpathcurveto{\pgfqpoint{7.876112in}{6.581845in}}{\pgfqpoint{7.865513in}{6.577455in}}{\pgfqpoint{7.857699in}{6.569641in}}%
\pgfpathcurveto{\pgfqpoint{7.849885in}{6.561827in}}{\pgfqpoint{7.845495in}{6.551228in}}{\pgfqpoint{7.845495in}{6.540178in}}%
\pgfpathcurveto{\pgfqpoint{7.845495in}{6.529128in}}{\pgfqpoint{7.849885in}{6.518529in}}{\pgfqpoint{7.857699in}{6.510715in}}%
\pgfpathcurveto{\pgfqpoint{7.865513in}{6.502902in}}{\pgfqpoint{7.876112in}{6.498512in}}{\pgfqpoint{7.887162in}{6.498512in}}%
\pgfpathlineto{\pgfqpoint{7.887162in}{6.498512in}}%
\pgfpathclose%
\pgfusepath{stroke,fill}%
\end{pgfscope}%
\begin{pgfscope}%
\pgfpathrectangle{\pgfqpoint{7.622482in}{5.272501in}}{\pgfqpoint{2.177280in}{2.201755in}}%
\pgfusepath{clip}%
\pgfsetbuttcap%
\pgfsetroundjoin%
\definecolor{currentfill}{rgb}{0.121569,0.466667,0.705882}%
\pgfsetfillcolor{currentfill}%
\pgfsetlinewidth{0.481800pt}%
\definecolor{currentstroke}{rgb}{1.000000,1.000000,1.000000}%
\pgfsetstrokecolor{currentstroke}%
\pgfsetdash{}{0pt}%
\pgfpathmoveto{\pgfqpoint{7.887162in}{6.081512in}}%
\pgfpathcurveto{\pgfqpoint{7.898212in}{6.081512in}}{\pgfqpoint{7.908811in}{6.085903in}}{\pgfqpoint{7.916625in}{6.093716in}}%
\pgfpathcurveto{\pgfqpoint{7.924438in}{6.101530in}}{\pgfqpoint{7.928828in}{6.112129in}}{\pgfqpoint{7.928828in}{6.123179in}}%
\pgfpathcurveto{\pgfqpoint{7.928828in}{6.134229in}}{\pgfqpoint{7.924438in}{6.144828in}}{\pgfqpoint{7.916625in}{6.152642in}}%
\pgfpathcurveto{\pgfqpoint{7.908811in}{6.160456in}}{\pgfqpoint{7.898212in}{6.164846in}}{\pgfqpoint{7.887162in}{6.164846in}}%
\pgfpathcurveto{\pgfqpoint{7.876112in}{6.164846in}}{\pgfqpoint{7.865513in}{6.160456in}}{\pgfqpoint{7.857699in}{6.152642in}}%
\pgfpathcurveto{\pgfqpoint{7.849885in}{6.144828in}}{\pgfqpoint{7.845495in}{6.134229in}}{\pgfqpoint{7.845495in}{6.123179in}}%
\pgfpathcurveto{\pgfqpoint{7.845495in}{6.112129in}}{\pgfqpoint{7.849885in}{6.101530in}}{\pgfqpoint{7.857699in}{6.093716in}}%
\pgfpathcurveto{\pgfqpoint{7.865513in}{6.085903in}}{\pgfqpoint{7.876112in}{6.081512in}}{\pgfqpoint{7.887162in}{6.081512in}}%
\pgfpathlineto{\pgfqpoint{7.887162in}{6.081512in}}%
\pgfpathclose%
\pgfusepath{stroke,fill}%
\end{pgfscope}%
\begin{pgfscope}%
\pgfpathrectangle{\pgfqpoint{7.622482in}{5.272501in}}{\pgfqpoint{2.177280in}{2.201755in}}%
\pgfusepath{clip}%
\pgfsetbuttcap%
\pgfsetroundjoin%
\definecolor{currentfill}{rgb}{0.121569,0.466667,0.705882}%
\pgfsetfillcolor{currentfill}%
\pgfsetlinewidth{0.481800pt}%
\definecolor{currentstroke}{rgb}{1.000000,1.000000,1.000000}%
\pgfsetstrokecolor{currentstroke}%
\pgfsetdash{}{0pt}%
\pgfpathmoveto{\pgfqpoint{7.819408in}{6.248312in}}%
\pgfpathcurveto{\pgfqpoint{7.830458in}{6.248312in}}{\pgfqpoint{7.841057in}{6.252702in}}{\pgfqpoint{7.848871in}{6.260516in}}%
\pgfpathcurveto{\pgfqpoint{7.856684in}{6.268330in}}{\pgfqpoint{7.861075in}{6.278929in}}{\pgfqpoint{7.861075in}{6.289979in}}%
\pgfpathcurveto{\pgfqpoint{7.861075in}{6.301029in}}{\pgfqpoint{7.856684in}{6.311628in}}{\pgfqpoint{7.848871in}{6.319442in}}%
\pgfpathcurveto{\pgfqpoint{7.841057in}{6.327255in}}{\pgfqpoint{7.830458in}{6.331645in}}{\pgfqpoint{7.819408in}{6.331645in}}%
\pgfpathcurveto{\pgfqpoint{7.808358in}{6.331645in}}{\pgfqpoint{7.797759in}{6.327255in}}{\pgfqpoint{7.789945in}{6.319442in}}%
\pgfpathcurveto{\pgfqpoint{7.782132in}{6.311628in}}{\pgfqpoint{7.777741in}{6.301029in}}{\pgfqpoint{7.777741in}{6.289979in}}%
\pgfpathcurveto{\pgfqpoint{7.777741in}{6.278929in}}{\pgfqpoint{7.782132in}{6.268330in}}{\pgfqpoint{7.789945in}{6.260516in}}%
\pgfpathcurveto{\pgfqpoint{7.797759in}{6.252702in}}{\pgfqpoint{7.808358in}{6.248312in}}{\pgfqpoint{7.819408in}{6.248312in}}%
\pgfpathlineto{\pgfqpoint{7.819408in}{6.248312in}}%
\pgfpathclose%
\pgfusepath{stroke,fill}%
\end{pgfscope}%
\begin{pgfscope}%
\pgfpathrectangle{\pgfqpoint{7.622482in}{5.272501in}}{\pgfqpoint{2.177280in}{2.201755in}}%
\pgfusepath{clip}%
\pgfsetbuttcap%
\pgfsetroundjoin%
\definecolor{currentfill}{rgb}{0.121569,0.466667,0.705882}%
\pgfsetfillcolor{currentfill}%
\pgfsetlinewidth{0.481800pt}%
\definecolor{currentstroke}{rgb}{1.000000,1.000000,1.000000}%
\pgfsetstrokecolor{currentstroke}%
\pgfsetdash{}{0pt}%
\pgfpathmoveto{\pgfqpoint{7.887162in}{6.748711in}}%
\pgfpathcurveto{\pgfqpoint{7.898212in}{6.748711in}}{\pgfqpoint{7.908811in}{6.753101in}}{\pgfqpoint{7.916625in}{6.760915in}}%
\pgfpathcurveto{\pgfqpoint{7.924438in}{6.768728in}}{\pgfqpoint{7.928828in}{6.779327in}}{\pgfqpoint{7.928828in}{6.790378in}}%
\pgfpathcurveto{\pgfqpoint{7.928828in}{6.801428in}}{\pgfqpoint{7.924438in}{6.812027in}}{\pgfqpoint{7.916625in}{6.819840in}}%
\pgfpathcurveto{\pgfqpoint{7.908811in}{6.827654in}}{\pgfqpoint{7.898212in}{6.832044in}}{\pgfqpoint{7.887162in}{6.832044in}}%
\pgfpathcurveto{\pgfqpoint{7.876112in}{6.832044in}}{\pgfqpoint{7.865513in}{6.827654in}}{\pgfqpoint{7.857699in}{6.819840in}}%
\pgfpathcurveto{\pgfqpoint{7.849885in}{6.812027in}}{\pgfqpoint{7.845495in}{6.801428in}}{\pgfqpoint{7.845495in}{6.790378in}}%
\pgfpathcurveto{\pgfqpoint{7.845495in}{6.779327in}}{\pgfqpoint{7.849885in}{6.768728in}}{\pgfqpoint{7.857699in}{6.760915in}}%
\pgfpathcurveto{\pgfqpoint{7.865513in}{6.753101in}}{\pgfqpoint{7.876112in}{6.748711in}}{\pgfqpoint{7.887162in}{6.748711in}}%
\pgfpathlineto{\pgfqpoint{7.887162in}{6.748711in}}%
\pgfpathclose%
\pgfusepath{stroke,fill}%
\end{pgfscope}%
\begin{pgfscope}%
\pgfpathrectangle{\pgfqpoint{7.622482in}{5.272501in}}{\pgfqpoint{2.177280in}{2.201755in}}%
\pgfusepath{clip}%
\pgfsetbuttcap%
\pgfsetroundjoin%
\definecolor{currentfill}{rgb}{0.121569,0.466667,0.705882}%
\pgfsetfillcolor{currentfill}%
\pgfsetlinewidth{0.481800pt}%
\definecolor{currentstroke}{rgb}{1.000000,1.000000,1.000000}%
\pgfsetstrokecolor{currentstroke}%
\pgfsetdash{}{0pt}%
\pgfpathmoveto{\pgfqpoint{7.887162in}{6.498512in}}%
\pgfpathcurveto{\pgfqpoint{7.898212in}{6.498512in}}{\pgfqpoint{7.908811in}{6.502902in}}{\pgfqpoint{7.916625in}{6.510715in}}%
\pgfpathcurveto{\pgfqpoint{7.924438in}{6.518529in}}{\pgfqpoint{7.928828in}{6.529128in}}{\pgfqpoint{7.928828in}{6.540178in}}%
\pgfpathcurveto{\pgfqpoint{7.928828in}{6.551228in}}{\pgfqpoint{7.924438in}{6.561827in}}{\pgfqpoint{7.916625in}{6.569641in}}%
\pgfpathcurveto{\pgfqpoint{7.908811in}{6.577455in}}{\pgfqpoint{7.898212in}{6.581845in}}{\pgfqpoint{7.887162in}{6.581845in}}%
\pgfpathcurveto{\pgfqpoint{7.876112in}{6.581845in}}{\pgfqpoint{7.865513in}{6.577455in}}{\pgfqpoint{7.857699in}{6.569641in}}%
\pgfpathcurveto{\pgfqpoint{7.849885in}{6.561827in}}{\pgfqpoint{7.845495in}{6.551228in}}{\pgfqpoint{7.845495in}{6.540178in}}%
\pgfpathcurveto{\pgfqpoint{7.845495in}{6.529128in}}{\pgfqpoint{7.849885in}{6.518529in}}{\pgfqpoint{7.857699in}{6.510715in}}%
\pgfpathcurveto{\pgfqpoint{7.865513in}{6.502902in}}{\pgfqpoint{7.876112in}{6.498512in}}{\pgfqpoint{7.887162in}{6.498512in}}%
\pgfpathlineto{\pgfqpoint{7.887162in}{6.498512in}}%
\pgfpathclose%
\pgfusepath{stroke,fill}%
\end{pgfscope}%
\begin{pgfscope}%
\pgfpathrectangle{\pgfqpoint{7.622482in}{5.272501in}}{\pgfqpoint{2.177280in}{2.201755in}}%
\pgfusepath{clip}%
\pgfsetbuttcap%
\pgfsetroundjoin%
\definecolor{currentfill}{rgb}{0.121569,0.466667,0.705882}%
\pgfsetfillcolor{currentfill}%
\pgfsetlinewidth{0.481800pt}%
\definecolor{currentstroke}{rgb}{1.000000,1.000000,1.000000}%
\pgfsetstrokecolor{currentstroke}%
\pgfsetdash{}{0pt}%
\pgfpathmoveto{\pgfqpoint{7.819408in}{6.164912in}}%
\pgfpathcurveto{\pgfqpoint{7.830458in}{6.164912in}}{\pgfqpoint{7.841057in}{6.169303in}}{\pgfqpoint{7.848871in}{6.177116in}}%
\pgfpathcurveto{\pgfqpoint{7.856684in}{6.184930in}}{\pgfqpoint{7.861075in}{6.195529in}}{\pgfqpoint{7.861075in}{6.206579in}}%
\pgfpathcurveto{\pgfqpoint{7.861075in}{6.217629in}}{\pgfqpoint{7.856684in}{6.228228in}}{\pgfqpoint{7.848871in}{6.236042in}}%
\pgfpathcurveto{\pgfqpoint{7.841057in}{6.243855in}}{\pgfqpoint{7.830458in}{6.248246in}}{\pgfqpoint{7.819408in}{6.248246in}}%
\pgfpathcurveto{\pgfqpoint{7.808358in}{6.248246in}}{\pgfqpoint{7.797759in}{6.243855in}}{\pgfqpoint{7.789945in}{6.236042in}}%
\pgfpathcurveto{\pgfqpoint{7.782132in}{6.228228in}}{\pgfqpoint{7.777741in}{6.217629in}}{\pgfqpoint{7.777741in}{6.206579in}}%
\pgfpathcurveto{\pgfqpoint{7.777741in}{6.195529in}}{\pgfqpoint{7.782132in}{6.184930in}}{\pgfqpoint{7.789945in}{6.177116in}}%
\pgfpathcurveto{\pgfqpoint{7.797759in}{6.169303in}}{\pgfqpoint{7.808358in}{6.164912in}}{\pgfqpoint{7.819408in}{6.164912in}}%
\pgfpathlineto{\pgfqpoint{7.819408in}{6.164912in}}%
\pgfpathclose%
\pgfusepath{stroke,fill}%
\end{pgfscope}%
\begin{pgfscope}%
\pgfpathrectangle{\pgfqpoint{7.622482in}{5.272501in}}{\pgfqpoint{2.177280in}{2.201755in}}%
\pgfusepath{clip}%
\pgfsetbuttcap%
\pgfsetroundjoin%
\definecolor{currentfill}{rgb}{0.121569,0.466667,0.705882}%
\pgfsetfillcolor{currentfill}%
\pgfsetlinewidth{0.481800pt}%
\definecolor{currentstroke}{rgb}{1.000000,1.000000,1.000000}%
\pgfsetstrokecolor{currentstroke}%
\pgfsetdash{}{0pt}%
\pgfpathmoveto{\pgfqpoint{7.819408in}{6.164912in}}%
\pgfpathcurveto{\pgfqpoint{7.830458in}{6.164912in}}{\pgfqpoint{7.841057in}{6.169303in}}{\pgfqpoint{7.848871in}{6.177116in}}%
\pgfpathcurveto{\pgfqpoint{7.856684in}{6.184930in}}{\pgfqpoint{7.861075in}{6.195529in}}{\pgfqpoint{7.861075in}{6.206579in}}%
\pgfpathcurveto{\pgfqpoint{7.861075in}{6.217629in}}{\pgfqpoint{7.856684in}{6.228228in}}{\pgfqpoint{7.848871in}{6.236042in}}%
\pgfpathcurveto{\pgfqpoint{7.841057in}{6.243855in}}{\pgfqpoint{7.830458in}{6.248246in}}{\pgfqpoint{7.819408in}{6.248246in}}%
\pgfpathcurveto{\pgfqpoint{7.808358in}{6.248246in}}{\pgfqpoint{7.797759in}{6.243855in}}{\pgfqpoint{7.789945in}{6.236042in}}%
\pgfpathcurveto{\pgfqpoint{7.782132in}{6.228228in}}{\pgfqpoint{7.777741in}{6.217629in}}{\pgfqpoint{7.777741in}{6.206579in}}%
\pgfpathcurveto{\pgfqpoint{7.777741in}{6.195529in}}{\pgfqpoint{7.782132in}{6.184930in}}{\pgfqpoint{7.789945in}{6.177116in}}%
\pgfpathcurveto{\pgfqpoint{7.797759in}{6.169303in}}{\pgfqpoint{7.808358in}{6.164912in}}{\pgfqpoint{7.819408in}{6.164912in}}%
\pgfpathlineto{\pgfqpoint{7.819408in}{6.164912in}}%
\pgfpathclose%
\pgfusepath{stroke,fill}%
\end{pgfscope}%
\begin{pgfscope}%
\pgfpathrectangle{\pgfqpoint{7.622482in}{5.272501in}}{\pgfqpoint{2.177280in}{2.201755in}}%
\pgfusepath{clip}%
\pgfsetbuttcap%
\pgfsetroundjoin%
\definecolor{currentfill}{rgb}{0.121569,0.466667,0.705882}%
\pgfsetfillcolor{currentfill}%
\pgfsetlinewidth{0.481800pt}%
\definecolor{currentstroke}{rgb}{1.000000,1.000000,1.000000}%
\pgfsetstrokecolor{currentstroke}%
\pgfsetdash{}{0pt}%
\pgfpathmoveto{\pgfqpoint{7.887162in}{6.998910in}}%
\pgfpathcurveto{\pgfqpoint{7.898212in}{6.998910in}}{\pgfqpoint{7.908811in}{7.003301in}}{\pgfqpoint{7.916625in}{7.011114in}}%
\pgfpathcurveto{\pgfqpoint{7.924438in}{7.018928in}}{\pgfqpoint{7.928828in}{7.029527in}}{\pgfqpoint{7.928828in}{7.040577in}}%
\pgfpathcurveto{\pgfqpoint{7.928828in}{7.051627in}}{\pgfqpoint{7.924438in}{7.062226in}}{\pgfqpoint{7.916625in}{7.070040in}}%
\pgfpathcurveto{\pgfqpoint{7.908811in}{7.077853in}}{\pgfqpoint{7.898212in}{7.082244in}}{\pgfqpoint{7.887162in}{7.082244in}}%
\pgfpathcurveto{\pgfqpoint{7.876112in}{7.082244in}}{\pgfqpoint{7.865513in}{7.077853in}}{\pgfqpoint{7.857699in}{7.070040in}}%
\pgfpathcurveto{\pgfqpoint{7.849885in}{7.062226in}}{\pgfqpoint{7.845495in}{7.051627in}}{\pgfqpoint{7.845495in}{7.040577in}}%
\pgfpathcurveto{\pgfqpoint{7.845495in}{7.029527in}}{\pgfqpoint{7.849885in}{7.018928in}}{\pgfqpoint{7.857699in}{7.011114in}}%
\pgfpathcurveto{\pgfqpoint{7.865513in}{7.003301in}}{\pgfqpoint{7.876112in}{6.998910in}}{\pgfqpoint{7.887162in}{6.998910in}}%
\pgfpathlineto{\pgfqpoint{7.887162in}{6.998910in}}%
\pgfpathclose%
\pgfusepath{stroke,fill}%
\end{pgfscope}%
\begin{pgfscope}%
\pgfpathrectangle{\pgfqpoint{7.622482in}{5.272501in}}{\pgfqpoint{2.177280in}{2.201755in}}%
\pgfusepath{clip}%
\pgfsetbuttcap%
\pgfsetroundjoin%
\definecolor{currentfill}{rgb}{0.121569,0.466667,0.705882}%
\pgfsetfillcolor{currentfill}%
\pgfsetlinewidth{0.481800pt}%
\definecolor{currentstroke}{rgb}{1.000000,1.000000,1.000000}%
\pgfsetstrokecolor{currentstroke}%
\pgfsetdash{}{0pt}%
\pgfpathmoveto{\pgfqpoint{8.022670in}{7.332510in}}%
\pgfpathcurveto{\pgfqpoint{8.033720in}{7.332510in}}{\pgfqpoint{8.044319in}{7.336900in}}{\pgfqpoint{8.052132in}{7.344714in}}%
\pgfpathcurveto{\pgfqpoint{8.059946in}{7.352527in}}{\pgfqpoint{8.064336in}{7.363126in}}{\pgfqpoint{8.064336in}{7.374176in}}%
\pgfpathcurveto{\pgfqpoint{8.064336in}{7.385226in}}{\pgfqpoint{8.059946in}{7.395825in}}{\pgfqpoint{8.052132in}{7.403639in}}%
\pgfpathcurveto{\pgfqpoint{8.044319in}{7.411453in}}{\pgfqpoint{8.033720in}{7.415843in}}{\pgfqpoint{8.022670in}{7.415843in}}%
\pgfpathcurveto{\pgfqpoint{8.011619in}{7.415843in}}{\pgfqpoint{8.001020in}{7.411453in}}{\pgfqpoint{7.993207in}{7.403639in}}%
\pgfpathcurveto{\pgfqpoint{7.985393in}{7.395825in}}{\pgfqpoint{7.981003in}{7.385226in}}{\pgfqpoint{7.981003in}{7.374176in}}%
\pgfpathcurveto{\pgfqpoint{7.981003in}{7.363126in}}{\pgfqpoint{7.985393in}{7.352527in}}{\pgfqpoint{7.993207in}{7.344714in}}%
\pgfpathcurveto{\pgfqpoint{8.001020in}{7.336900in}}{\pgfqpoint{8.011619in}{7.332510in}}{\pgfqpoint{8.022670in}{7.332510in}}%
\pgfpathlineto{\pgfqpoint{8.022670in}{7.332510in}}%
\pgfpathclose%
\pgfusepath{stroke,fill}%
\end{pgfscope}%
\begin{pgfscope}%
\pgfpathrectangle{\pgfqpoint{7.622482in}{5.272501in}}{\pgfqpoint{2.177280in}{2.201755in}}%
\pgfusepath{clip}%
\pgfsetbuttcap%
\pgfsetroundjoin%
\definecolor{currentfill}{rgb}{0.121569,0.466667,0.705882}%
\pgfsetfillcolor{currentfill}%
\pgfsetlinewidth{0.481800pt}%
\definecolor{currentstroke}{rgb}{1.000000,1.000000,1.000000}%
\pgfsetstrokecolor{currentstroke}%
\pgfsetdash{}{0pt}%
\pgfpathmoveto{\pgfqpoint{8.022670in}{6.915511in}}%
\pgfpathcurveto{\pgfqpoint{8.033720in}{6.915511in}}{\pgfqpoint{8.044319in}{6.919901in}}{\pgfqpoint{8.052132in}{6.927714in}}%
\pgfpathcurveto{\pgfqpoint{8.059946in}{6.935528in}}{\pgfqpoint{8.064336in}{6.946127in}}{\pgfqpoint{8.064336in}{6.957177in}}%
\pgfpathcurveto{\pgfqpoint{8.064336in}{6.968227in}}{\pgfqpoint{8.059946in}{6.978826in}}{\pgfqpoint{8.052132in}{6.986640in}}%
\pgfpathcurveto{\pgfqpoint{8.044319in}{6.994454in}}{\pgfqpoint{8.033720in}{6.998844in}}{\pgfqpoint{8.022670in}{6.998844in}}%
\pgfpathcurveto{\pgfqpoint{8.011619in}{6.998844in}}{\pgfqpoint{8.001020in}{6.994454in}}{\pgfqpoint{7.993207in}{6.986640in}}%
\pgfpathcurveto{\pgfqpoint{7.985393in}{6.978826in}}{\pgfqpoint{7.981003in}{6.968227in}}{\pgfqpoint{7.981003in}{6.957177in}}%
\pgfpathcurveto{\pgfqpoint{7.981003in}{6.946127in}}{\pgfqpoint{7.985393in}{6.935528in}}{\pgfqpoint{7.993207in}{6.927714in}}%
\pgfpathcurveto{\pgfqpoint{8.001020in}{6.919901in}}{\pgfqpoint{8.011619in}{6.915511in}}{\pgfqpoint{8.022670in}{6.915511in}}%
\pgfpathlineto{\pgfqpoint{8.022670in}{6.915511in}}%
\pgfpathclose%
\pgfusepath{stroke,fill}%
\end{pgfscope}%
\begin{pgfscope}%
\pgfpathrectangle{\pgfqpoint{7.622482in}{5.272501in}}{\pgfqpoint{2.177280in}{2.201755in}}%
\pgfusepath{clip}%
\pgfsetbuttcap%
\pgfsetroundjoin%
\definecolor{currentfill}{rgb}{0.121569,0.466667,0.705882}%
\pgfsetfillcolor{currentfill}%
\pgfsetlinewidth{0.481800pt}%
\definecolor{currentstroke}{rgb}{1.000000,1.000000,1.000000}%
\pgfsetstrokecolor{currentstroke}%
\pgfsetdash{}{0pt}%
\pgfpathmoveto{\pgfqpoint{7.954916in}{6.581911in}}%
\pgfpathcurveto{\pgfqpoint{7.965966in}{6.581911in}}{\pgfqpoint{7.976565in}{6.586302in}}{\pgfqpoint{7.984378in}{6.594115in}}%
\pgfpathcurveto{\pgfqpoint{7.992192in}{6.601929in}}{\pgfqpoint{7.996582in}{6.612528in}}{\pgfqpoint{7.996582in}{6.623578in}}%
\pgfpathcurveto{\pgfqpoint{7.996582in}{6.634628in}}{\pgfqpoint{7.992192in}{6.645227in}}{\pgfqpoint{7.984378in}{6.653041in}}%
\pgfpathcurveto{\pgfqpoint{7.976565in}{6.660854in}}{\pgfqpoint{7.965966in}{6.665245in}}{\pgfqpoint{7.954916in}{6.665245in}}%
\pgfpathcurveto{\pgfqpoint{7.943866in}{6.665245in}}{\pgfqpoint{7.933267in}{6.660854in}}{\pgfqpoint{7.925453in}{6.653041in}}%
\pgfpathcurveto{\pgfqpoint{7.917639in}{6.645227in}}{\pgfqpoint{7.913249in}{6.634628in}}{\pgfqpoint{7.913249in}{6.623578in}}%
\pgfpathcurveto{\pgfqpoint{7.913249in}{6.612528in}}{\pgfqpoint{7.917639in}{6.601929in}}{\pgfqpoint{7.925453in}{6.594115in}}%
\pgfpathcurveto{\pgfqpoint{7.933267in}{6.586302in}}{\pgfqpoint{7.943866in}{6.581911in}}{\pgfqpoint{7.954916in}{6.581911in}}%
\pgfpathlineto{\pgfqpoint{7.954916in}{6.581911in}}%
\pgfpathclose%
\pgfusepath{stroke,fill}%
\end{pgfscope}%
\begin{pgfscope}%
\pgfpathrectangle{\pgfqpoint{7.622482in}{5.272501in}}{\pgfqpoint{2.177280in}{2.201755in}}%
\pgfusepath{clip}%
\pgfsetbuttcap%
\pgfsetroundjoin%
\definecolor{currentfill}{rgb}{0.121569,0.466667,0.705882}%
\pgfsetfillcolor{currentfill}%
\pgfsetlinewidth{0.481800pt}%
\definecolor{currentstroke}{rgb}{1.000000,1.000000,1.000000}%
\pgfsetstrokecolor{currentstroke}%
\pgfsetdash{}{0pt}%
\pgfpathmoveto{\pgfqpoint{7.954916in}{6.832111in}}%
\pgfpathcurveto{\pgfqpoint{7.965966in}{6.832111in}}{\pgfqpoint{7.976565in}{6.836501in}}{\pgfqpoint{7.984378in}{6.844315in}}%
\pgfpathcurveto{\pgfqpoint{7.992192in}{6.852128in}}{\pgfqpoint{7.996582in}{6.862727in}}{\pgfqpoint{7.996582in}{6.873777in}}%
\pgfpathcurveto{\pgfqpoint{7.996582in}{6.884828in}}{\pgfqpoint{7.992192in}{6.895427in}}{\pgfqpoint{7.984378in}{6.903240in}}%
\pgfpathcurveto{\pgfqpoint{7.976565in}{6.911054in}}{\pgfqpoint{7.965966in}{6.915444in}}{\pgfqpoint{7.954916in}{6.915444in}}%
\pgfpathcurveto{\pgfqpoint{7.943866in}{6.915444in}}{\pgfqpoint{7.933267in}{6.911054in}}{\pgfqpoint{7.925453in}{6.903240in}}%
\pgfpathcurveto{\pgfqpoint{7.917639in}{6.895427in}}{\pgfqpoint{7.913249in}{6.884828in}}{\pgfqpoint{7.913249in}{6.873777in}}%
\pgfpathcurveto{\pgfqpoint{7.913249in}{6.862727in}}{\pgfqpoint{7.917639in}{6.852128in}}{\pgfqpoint{7.925453in}{6.844315in}}%
\pgfpathcurveto{\pgfqpoint{7.933267in}{6.836501in}}{\pgfqpoint{7.943866in}{6.832111in}}{\pgfqpoint{7.954916in}{6.832111in}}%
\pgfpathlineto{\pgfqpoint{7.954916in}{6.832111in}}%
\pgfpathclose%
\pgfusepath{stroke,fill}%
\end{pgfscope}%
\begin{pgfscope}%
\pgfpathrectangle{\pgfqpoint{7.622482in}{5.272501in}}{\pgfqpoint{2.177280in}{2.201755in}}%
\pgfusepath{clip}%
\pgfsetbuttcap%
\pgfsetroundjoin%
\definecolor{currentfill}{rgb}{0.121569,0.466667,0.705882}%
\pgfsetfillcolor{currentfill}%
\pgfsetlinewidth{0.481800pt}%
\definecolor{currentstroke}{rgb}{1.000000,1.000000,1.000000}%
\pgfsetstrokecolor{currentstroke}%
\pgfsetdash{}{0pt}%
\pgfpathmoveto{\pgfqpoint{7.954916in}{6.832111in}}%
\pgfpathcurveto{\pgfqpoint{7.965966in}{6.832111in}}{\pgfqpoint{7.976565in}{6.836501in}}{\pgfqpoint{7.984378in}{6.844315in}}%
\pgfpathcurveto{\pgfqpoint{7.992192in}{6.852128in}}{\pgfqpoint{7.996582in}{6.862727in}}{\pgfqpoint{7.996582in}{6.873777in}}%
\pgfpathcurveto{\pgfqpoint{7.996582in}{6.884828in}}{\pgfqpoint{7.992192in}{6.895427in}}{\pgfqpoint{7.984378in}{6.903240in}}%
\pgfpathcurveto{\pgfqpoint{7.976565in}{6.911054in}}{\pgfqpoint{7.965966in}{6.915444in}}{\pgfqpoint{7.954916in}{6.915444in}}%
\pgfpathcurveto{\pgfqpoint{7.943866in}{6.915444in}}{\pgfqpoint{7.933267in}{6.911054in}}{\pgfqpoint{7.925453in}{6.903240in}}%
\pgfpathcurveto{\pgfqpoint{7.917639in}{6.895427in}}{\pgfqpoint{7.913249in}{6.884828in}}{\pgfqpoint{7.913249in}{6.873777in}}%
\pgfpathcurveto{\pgfqpoint{7.913249in}{6.862727in}}{\pgfqpoint{7.917639in}{6.852128in}}{\pgfqpoint{7.925453in}{6.844315in}}%
\pgfpathcurveto{\pgfqpoint{7.933267in}{6.836501in}}{\pgfqpoint{7.943866in}{6.832111in}}{\pgfqpoint{7.954916in}{6.832111in}}%
\pgfpathlineto{\pgfqpoint{7.954916in}{6.832111in}}%
\pgfpathclose%
\pgfusepath{stroke,fill}%
\end{pgfscope}%
\begin{pgfscope}%
\pgfpathrectangle{\pgfqpoint{7.622482in}{5.272501in}}{\pgfqpoint{2.177280in}{2.201755in}}%
\pgfusepath{clip}%
\pgfsetbuttcap%
\pgfsetroundjoin%
\definecolor{currentfill}{rgb}{0.121569,0.466667,0.705882}%
\pgfsetfillcolor{currentfill}%
\pgfsetlinewidth{0.481800pt}%
\definecolor{currentstroke}{rgb}{1.000000,1.000000,1.000000}%
\pgfsetstrokecolor{currentstroke}%
\pgfsetdash{}{0pt}%
\pgfpathmoveto{\pgfqpoint{7.887162in}{6.498512in}}%
\pgfpathcurveto{\pgfqpoint{7.898212in}{6.498512in}}{\pgfqpoint{7.908811in}{6.502902in}}{\pgfqpoint{7.916625in}{6.510715in}}%
\pgfpathcurveto{\pgfqpoint{7.924438in}{6.518529in}}{\pgfqpoint{7.928828in}{6.529128in}}{\pgfqpoint{7.928828in}{6.540178in}}%
\pgfpathcurveto{\pgfqpoint{7.928828in}{6.551228in}}{\pgfqpoint{7.924438in}{6.561827in}}{\pgfqpoint{7.916625in}{6.569641in}}%
\pgfpathcurveto{\pgfqpoint{7.908811in}{6.577455in}}{\pgfqpoint{7.898212in}{6.581845in}}{\pgfqpoint{7.887162in}{6.581845in}}%
\pgfpathcurveto{\pgfqpoint{7.876112in}{6.581845in}}{\pgfqpoint{7.865513in}{6.577455in}}{\pgfqpoint{7.857699in}{6.569641in}}%
\pgfpathcurveto{\pgfqpoint{7.849885in}{6.561827in}}{\pgfqpoint{7.845495in}{6.551228in}}{\pgfqpoint{7.845495in}{6.540178in}}%
\pgfpathcurveto{\pgfqpoint{7.845495in}{6.529128in}}{\pgfqpoint{7.849885in}{6.518529in}}{\pgfqpoint{7.857699in}{6.510715in}}%
\pgfpathcurveto{\pgfqpoint{7.865513in}{6.502902in}}{\pgfqpoint{7.876112in}{6.498512in}}{\pgfqpoint{7.887162in}{6.498512in}}%
\pgfpathlineto{\pgfqpoint{7.887162in}{6.498512in}}%
\pgfpathclose%
\pgfusepath{stroke,fill}%
\end{pgfscope}%
\begin{pgfscope}%
\pgfpathrectangle{\pgfqpoint{7.622482in}{5.272501in}}{\pgfqpoint{2.177280in}{2.201755in}}%
\pgfusepath{clip}%
\pgfsetbuttcap%
\pgfsetroundjoin%
\definecolor{currentfill}{rgb}{0.121569,0.466667,0.705882}%
\pgfsetfillcolor{currentfill}%
\pgfsetlinewidth{0.481800pt}%
\definecolor{currentstroke}{rgb}{1.000000,1.000000,1.000000}%
\pgfsetstrokecolor{currentstroke}%
\pgfsetdash{}{0pt}%
\pgfpathmoveto{\pgfqpoint{8.022670in}{6.748711in}}%
\pgfpathcurveto{\pgfqpoint{8.033720in}{6.748711in}}{\pgfqpoint{8.044319in}{6.753101in}}{\pgfqpoint{8.052132in}{6.760915in}}%
\pgfpathcurveto{\pgfqpoint{8.059946in}{6.768728in}}{\pgfqpoint{8.064336in}{6.779327in}}{\pgfqpoint{8.064336in}{6.790378in}}%
\pgfpathcurveto{\pgfqpoint{8.064336in}{6.801428in}}{\pgfqpoint{8.059946in}{6.812027in}}{\pgfqpoint{8.052132in}{6.819840in}}%
\pgfpathcurveto{\pgfqpoint{8.044319in}{6.827654in}}{\pgfqpoint{8.033720in}{6.832044in}}{\pgfqpoint{8.022670in}{6.832044in}}%
\pgfpathcurveto{\pgfqpoint{8.011619in}{6.832044in}}{\pgfqpoint{8.001020in}{6.827654in}}{\pgfqpoint{7.993207in}{6.819840in}}%
\pgfpathcurveto{\pgfqpoint{7.985393in}{6.812027in}}{\pgfqpoint{7.981003in}{6.801428in}}{\pgfqpoint{7.981003in}{6.790378in}}%
\pgfpathcurveto{\pgfqpoint{7.981003in}{6.779327in}}{\pgfqpoint{7.985393in}{6.768728in}}{\pgfqpoint{7.993207in}{6.760915in}}%
\pgfpathcurveto{\pgfqpoint{8.001020in}{6.753101in}}{\pgfqpoint{8.011619in}{6.748711in}}{\pgfqpoint{8.022670in}{6.748711in}}%
\pgfpathlineto{\pgfqpoint{8.022670in}{6.748711in}}%
\pgfpathclose%
\pgfusepath{stroke,fill}%
\end{pgfscope}%
\begin{pgfscope}%
\pgfpathrectangle{\pgfqpoint{7.622482in}{5.272501in}}{\pgfqpoint{2.177280in}{2.201755in}}%
\pgfusepath{clip}%
\pgfsetbuttcap%
\pgfsetroundjoin%
\definecolor{currentfill}{rgb}{0.121569,0.466667,0.705882}%
\pgfsetfillcolor{currentfill}%
\pgfsetlinewidth{0.481800pt}%
\definecolor{currentstroke}{rgb}{1.000000,1.000000,1.000000}%
\pgfsetstrokecolor{currentstroke}%
\pgfsetdash{}{0pt}%
\pgfpathmoveto{\pgfqpoint{7.887162in}{6.665311in}}%
\pgfpathcurveto{\pgfqpoint{7.898212in}{6.665311in}}{\pgfqpoint{7.908811in}{6.669701in}}{\pgfqpoint{7.916625in}{6.677515in}}%
\pgfpathcurveto{\pgfqpoint{7.924438in}{6.685329in}}{\pgfqpoint{7.928828in}{6.695928in}}{\pgfqpoint{7.928828in}{6.706978in}}%
\pgfpathcurveto{\pgfqpoint{7.928828in}{6.718028in}}{\pgfqpoint{7.924438in}{6.728627in}}{\pgfqpoint{7.916625in}{6.736441in}}%
\pgfpathcurveto{\pgfqpoint{7.908811in}{6.744254in}}{\pgfqpoint{7.898212in}{6.748644in}}{\pgfqpoint{7.887162in}{6.748644in}}%
\pgfpathcurveto{\pgfqpoint{7.876112in}{6.748644in}}{\pgfqpoint{7.865513in}{6.744254in}}{\pgfqpoint{7.857699in}{6.736441in}}%
\pgfpathcurveto{\pgfqpoint{7.849885in}{6.728627in}}{\pgfqpoint{7.845495in}{6.718028in}}{\pgfqpoint{7.845495in}{6.706978in}}%
\pgfpathcurveto{\pgfqpoint{7.845495in}{6.695928in}}{\pgfqpoint{7.849885in}{6.685329in}}{\pgfqpoint{7.857699in}{6.677515in}}%
\pgfpathcurveto{\pgfqpoint{7.865513in}{6.669701in}}{\pgfqpoint{7.876112in}{6.665311in}}{\pgfqpoint{7.887162in}{6.665311in}}%
\pgfpathlineto{\pgfqpoint{7.887162in}{6.665311in}}%
\pgfpathclose%
\pgfusepath{stroke,fill}%
\end{pgfscope}%
\begin{pgfscope}%
\pgfpathrectangle{\pgfqpoint{7.622482in}{5.272501in}}{\pgfqpoint{2.177280in}{2.201755in}}%
\pgfusepath{clip}%
\pgfsetbuttcap%
\pgfsetroundjoin%
\definecolor{currentfill}{rgb}{0.121569,0.466667,0.705882}%
\pgfsetfillcolor{currentfill}%
\pgfsetlinewidth{0.481800pt}%
\definecolor{currentstroke}{rgb}{1.000000,1.000000,1.000000}%
\pgfsetstrokecolor{currentstroke}%
\pgfsetdash{}{0pt}%
\pgfpathmoveto{\pgfqpoint{8.090423in}{6.415112in}}%
\pgfpathcurveto{\pgfqpoint{8.101474in}{6.415112in}}{\pgfqpoint{8.112073in}{6.419502in}}{\pgfqpoint{8.119886in}{6.427316in}}%
\pgfpathcurveto{\pgfqpoint{8.127700in}{6.435129in}}{\pgfqpoint{8.132090in}{6.445728in}}{\pgfqpoint{8.132090in}{6.456778in}}%
\pgfpathcurveto{\pgfqpoint{8.132090in}{6.467828in}}{\pgfqpoint{8.127700in}{6.478428in}}{\pgfqpoint{8.119886in}{6.486241in}}%
\pgfpathcurveto{\pgfqpoint{8.112073in}{6.494055in}}{\pgfqpoint{8.101474in}{6.498445in}}{\pgfqpoint{8.090423in}{6.498445in}}%
\pgfpathcurveto{\pgfqpoint{8.079373in}{6.498445in}}{\pgfqpoint{8.068774in}{6.494055in}}{\pgfqpoint{8.060961in}{6.486241in}}%
\pgfpathcurveto{\pgfqpoint{8.053147in}{6.478428in}}{\pgfqpoint{8.048757in}{6.467828in}}{\pgfqpoint{8.048757in}{6.456778in}}%
\pgfpathcurveto{\pgfqpoint{8.048757in}{6.445728in}}{\pgfqpoint{8.053147in}{6.435129in}}{\pgfqpoint{8.060961in}{6.427316in}}%
\pgfpathcurveto{\pgfqpoint{8.068774in}{6.419502in}}{\pgfqpoint{8.079373in}{6.415112in}}{\pgfqpoint{8.090423in}{6.415112in}}%
\pgfpathlineto{\pgfqpoint{8.090423in}{6.415112in}}%
\pgfpathclose%
\pgfusepath{stroke,fill}%
\end{pgfscope}%
\begin{pgfscope}%
\pgfpathrectangle{\pgfqpoint{7.622482in}{5.272501in}}{\pgfqpoint{2.177280in}{2.201755in}}%
\pgfusepath{clip}%
\pgfsetbuttcap%
\pgfsetroundjoin%
\definecolor{currentfill}{rgb}{0.121569,0.466667,0.705882}%
\pgfsetfillcolor{currentfill}%
\pgfsetlinewidth{0.481800pt}%
\definecolor{currentstroke}{rgb}{1.000000,1.000000,1.000000}%
\pgfsetstrokecolor{currentstroke}%
\pgfsetdash{}{0pt}%
\pgfpathmoveto{\pgfqpoint{7.887162in}{6.498512in}}%
\pgfpathcurveto{\pgfqpoint{7.898212in}{6.498512in}}{\pgfqpoint{7.908811in}{6.502902in}}{\pgfqpoint{7.916625in}{6.510715in}}%
\pgfpathcurveto{\pgfqpoint{7.924438in}{6.518529in}}{\pgfqpoint{7.928828in}{6.529128in}}{\pgfqpoint{7.928828in}{6.540178in}}%
\pgfpathcurveto{\pgfqpoint{7.928828in}{6.551228in}}{\pgfqpoint{7.924438in}{6.561827in}}{\pgfqpoint{7.916625in}{6.569641in}}%
\pgfpathcurveto{\pgfqpoint{7.908811in}{6.577455in}}{\pgfqpoint{7.898212in}{6.581845in}}{\pgfqpoint{7.887162in}{6.581845in}}%
\pgfpathcurveto{\pgfqpoint{7.876112in}{6.581845in}}{\pgfqpoint{7.865513in}{6.577455in}}{\pgfqpoint{7.857699in}{6.569641in}}%
\pgfpathcurveto{\pgfqpoint{7.849885in}{6.561827in}}{\pgfqpoint{7.845495in}{6.551228in}}{\pgfqpoint{7.845495in}{6.540178in}}%
\pgfpathcurveto{\pgfqpoint{7.845495in}{6.529128in}}{\pgfqpoint{7.849885in}{6.518529in}}{\pgfqpoint{7.857699in}{6.510715in}}%
\pgfpathcurveto{\pgfqpoint{7.865513in}{6.502902in}}{\pgfqpoint{7.876112in}{6.498512in}}{\pgfqpoint{7.887162in}{6.498512in}}%
\pgfpathlineto{\pgfqpoint{7.887162in}{6.498512in}}%
\pgfpathclose%
\pgfusepath{stroke,fill}%
\end{pgfscope}%
\begin{pgfscope}%
\pgfpathrectangle{\pgfqpoint{7.622482in}{5.272501in}}{\pgfqpoint{2.177280in}{2.201755in}}%
\pgfusepath{clip}%
\pgfsetbuttcap%
\pgfsetroundjoin%
\definecolor{currentfill}{rgb}{0.121569,0.466667,0.705882}%
\pgfsetfillcolor{currentfill}%
\pgfsetlinewidth{0.481800pt}%
\definecolor{currentstroke}{rgb}{1.000000,1.000000,1.000000}%
\pgfsetstrokecolor{currentstroke}%
\pgfsetdash{}{0pt}%
\pgfpathmoveto{\pgfqpoint{7.887162in}{6.164912in}}%
\pgfpathcurveto{\pgfqpoint{7.898212in}{6.164912in}}{\pgfqpoint{7.908811in}{6.169303in}}{\pgfqpoint{7.916625in}{6.177116in}}%
\pgfpathcurveto{\pgfqpoint{7.924438in}{6.184930in}}{\pgfqpoint{7.928828in}{6.195529in}}{\pgfqpoint{7.928828in}{6.206579in}}%
\pgfpathcurveto{\pgfqpoint{7.928828in}{6.217629in}}{\pgfqpoint{7.924438in}{6.228228in}}{\pgfqpoint{7.916625in}{6.236042in}}%
\pgfpathcurveto{\pgfqpoint{7.908811in}{6.243855in}}{\pgfqpoint{7.898212in}{6.248246in}}{\pgfqpoint{7.887162in}{6.248246in}}%
\pgfpathcurveto{\pgfqpoint{7.876112in}{6.248246in}}{\pgfqpoint{7.865513in}{6.243855in}}{\pgfqpoint{7.857699in}{6.236042in}}%
\pgfpathcurveto{\pgfqpoint{7.849885in}{6.228228in}}{\pgfqpoint{7.845495in}{6.217629in}}{\pgfqpoint{7.845495in}{6.206579in}}%
\pgfpathcurveto{\pgfqpoint{7.845495in}{6.195529in}}{\pgfqpoint{7.849885in}{6.184930in}}{\pgfqpoint{7.857699in}{6.177116in}}%
\pgfpathcurveto{\pgfqpoint{7.865513in}{6.169303in}}{\pgfqpoint{7.876112in}{6.164912in}}{\pgfqpoint{7.887162in}{6.164912in}}%
\pgfpathlineto{\pgfqpoint{7.887162in}{6.164912in}}%
\pgfpathclose%
\pgfusepath{stroke,fill}%
\end{pgfscope}%
\begin{pgfscope}%
\pgfpathrectangle{\pgfqpoint{7.622482in}{5.272501in}}{\pgfqpoint{2.177280in}{2.201755in}}%
\pgfusepath{clip}%
\pgfsetbuttcap%
\pgfsetroundjoin%
\definecolor{currentfill}{rgb}{0.121569,0.466667,0.705882}%
\pgfsetfillcolor{currentfill}%
\pgfsetlinewidth{0.481800pt}%
\definecolor{currentstroke}{rgb}{1.000000,1.000000,1.000000}%
\pgfsetstrokecolor{currentstroke}%
\pgfsetdash{}{0pt}%
\pgfpathmoveto{\pgfqpoint{8.022670in}{6.498512in}}%
\pgfpathcurveto{\pgfqpoint{8.033720in}{6.498512in}}{\pgfqpoint{8.044319in}{6.502902in}}{\pgfqpoint{8.052132in}{6.510715in}}%
\pgfpathcurveto{\pgfqpoint{8.059946in}{6.518529in}}{\pgfqpoint{8.064336in}{6.529128in}}{\pgfqpoint{8.064336in}{6.540178in}}%
\pgfpathcurveto{\pgfqpoint{8.064336in}{6.551228in}}{\pgfqpoint{8.059946in}{6.561827in}}{\pgfqpoint{8.052132in}{6.569641in}}%
\pgfpathcurveto{\pgfqpoint{8.044319in}{6.577455in}}{\pgfqpoint{8.033720in}{6.581845in}}{\pgfqpoint{8.022670in}{6.581845in}}%
\pgfpathcurveto{\pgfqpoint{8.011619in}{6.581845in}}{\pgfqpoint{8.001020in}{6.577455in}}{\pgfqpoint{7.993207in}{6.569641in}}%
\pgfpathcurveto{\pgfqpoint{7.985393in}{6.561827in}}{\pgfqpoint{7.981003in}{6.551228in}}{\pgfqpoint{7.981003in}{6.540178in}}%
\pgfpathcurveto{\pgfqpoint{7.981003in}{6.529128in}}{\pgfqpoint{7.985393in}{6.518529in}}{\pgfqpoint{7.993207in}{6.510715in}}%
\pgfpathcurveto{\pgfqpoint{8.001020in}{6.502902in}}{\pgfqpoint{8.011619in}{6.498512in}}{\pgfqpoint{8.022670in}{6.498512in}}%
\pgfpathlineto{\pgfqpoint{8.022670in}{6.498512in}}%
\pgfpathclose%
\pgfusepath{stroke,fill}%
\end{pgfscope}%
\begin{pgfscope}%
\pgfpathrectangle{\pgfqpoint{7.622482in}{5.272501in}}{\pgfqpoint{2.177280in}{2.201755in}}%
\pgfusepath{clip}%
\pgfsetbuttcap%
\pgfsetroundjoin%
\definecolor{currentfill}{rgb}{0.121569,0.466667,0.705882}%
\pgfsetfillcolor{currentfill}%
\pgfsetlinewidth{0.481800pt}%
\definecolor{currentstroke}{rgb}{1.000000,1.000000,1.000000}%
\pgfsetstrokecolor{currentstroke}%
\pgfsetdash{}{0pt}%
\pgfpathmoveto{\pgfqpoint{7.887162in}{6.581911in}}%
\pgfpathcurveto{\pgfqpoint{7.898212in}{6.581911in}}{\pgfqpoint{7.908811in}{6.586302in}}{\pgfqpoint{7.916625in}{6.594115in}}%
\pgfpathcurveto{\pgfqpoint{7.924438in}{6.601929in}}{\pgfqpoint{7.928828in}{6.612528in}}{\pgfqpoint{7.928828in}{6.623578in}}%
\pgfpathcurveto{\pgfqpoint{7.928828in}{6.634628in}}{\pgfqpoint{7.924438in}{6.645227in}}{\pgfqpoint{7.916625in}{6.653041in}}%
\pgfpathcurveto{\pgfqpoint{7.908811in}{6.660854in}}{\pgfqpoint{7.898212in}{6.665245in}}{\pgfqpoint{7.887162in}{6.665245in}}%
\pgfpathcurveto{\pgfqpoint{7.876112in}{6.665245in}}{\pgfqpoint{7.865513in}{6.660854in}}{\pgfqpoint{7.857699in}{6.653041in}}%
\pgfpathcurveto{\pgfqpoint{7.849885in}{6.645227in}}{\pgfqpoint{7.845495in}{6.634628in}}{\pgfqpoint{7.845495in}{6.623578in}}%
\pgfpathcurveto{\pgfqpoint{7.845495in}{6.612528in}}{\pgfqpoint{7.849885in}{6.601929in}}{\pgfqpoint{7.857699in}{6.594115in}}%
\pgfpathcurveto{\pgfqpoint{7.865513in}{6.586302in}}{\pgfqpoint{7.876112in}{6.581911in}}{\pgfqpoint{7.887162in}{6.581911in}}%
\pgfpathlineto{\pgfqpoint{7.887162in}{6.581911in}}%
\pgfpathclose%
\pgfusepath{stroke,fill}%
\end{pgfscope}%
\begin{pgfscope}%
\pgfpathrectangle{\pgfqpoint{7.622482in}{5.272501in}}{\pgfqpoint{2.177280in}{2.201755in}}%
\pgfusepath{clip}%
\pgfsetbuttcap%
\pgfsetroundjoin%
\definecolor{currentfill}{rgb}{0.121569,0.466667,0.705882}%
\pgfsetfillcolor{currentfill}%
\pgfsetlinewidth{0.481800pt}%
\definecolor{currentstroke}{rgb}{1.000000,1.000000,1.000000}%
\pgfsetstrokecolor{currentstroke}%
\pgfsetdash{}{0pt}%
\pgfpathmoveto{\pgfqpoint{7.887162in}{6.498512in}}%
\pgfpathcurveto{\pgfqpoint{7.898212in}{6.498512in}}{\pgfqpoint{7.908811in}{6.502902in}}{\pgfqpoint{7.916625in}{6.510715in}}%
\pgfpathcurveto{\pgfqpoint{7.924438in}{6.518529in}}{\pgfqpoint{7.928828in}{6.529128in}}{\pgfqpoint{7.928828in}{6.540178in}}%
\pgfpathcurveto{\pgfqpoint{7.928828in}{6.551228in}}{\pgfqpoint{7.924438in}{6.561827in}}{\pgfqpoint{7.916625in}{6.569641in}}%
\pgfpathcurveto{\pgfqpoint{7.908811in}{6.577455in}}{\pgfqpoint{7.898212in}{6.581845in}}{\pgfqpoint{7.887162in}{6.581845in}}%
\pgfpathcurveto{\pgfqpoint{7.876112in}{6.581845in}}{\pgfqpoint{7.865513in}{6.577455in}}{\pgfqpoint{7.857699in}{6.569641in}}%
\pgfpathcurveto{\pgfqpoint{7.849885in}{6.561827in}}{\pgfqpoint{7.845495in}{6.551228in}}{\pgfqpoint{7.845495in}{6.540178in}}%
\pgfpathcurveto{\pgfqpoint{7.845495in}{6.529128in}}{\pgfqpoint{7.849885in}{6.518529in}}{\pgfqpoint{7.857699in}{6.510715in}}%
\pgfpathcurveto{\pgfqpoint{7.865513in}{6.502902in}}{\pgfqpoint{7.876112in}{6.498512in}}{\pgfqpoint{7.887162in}{6.498512in}}%
\pgfpathlineto{\pgfqpoint{7.887162in}{6.498512in}}%
\pgfpathclose%
\pgfusepath{stroke,fill}%
\end{pgfscope}%
\begin{pgfscope}%
\pgfpathrectangle{\pgfqpoint{7.622482in}{5.272501in}}{\pgfqpoint{2.177280in}{2.201755in}}%
\pgfusepath{clip}%
\pgfsetbuttcap%
\pgfsetroundjoin%
\definecolor{currentfill}{rgb}{0.121569,0.466667,0.705882}%
\pgfsetfillcolor{currentfill}%
\pgfsetlinewidth{0.481800pt}%
\definecolor{currentstroke}{rgb}{1.000000,1.000000,1.000000}%
\pgfsetstrokecolor{currentstroke}%
\pgfsetdash{}{0pt}%
\pgfpathmoveto{\pgfqpoint{7.887162in}{6.331712in}}%
\pgfpathcurveto{\pgfqpoint{7.898212in}{6.331712in}}{\pgfqpoint{7.908811in}{6.336102in}}{\pgfqpoint{7.916625in}{6.343916in}}%
\pgfpathcurveto{\pgfqpoint{7.924438in}{6.351729in}}{\pgfqpoint{7.928828in}{6.362328in}}{\pgfqpoint{7.928828in}{6.373379in}}%
\pgfpathcurveto{\pgfqpoint{7.928828in}{6.384429in}}{\pgfqpoint{7.924438in}{6.395028in}}{\pgfqpoint{7.916625in}{6.402841in}}%
\pgfpathcurveto{\pgfqpoint{7.908811in}{6.410655in}}{\pgfqpoint{7.898212in}{6.415045in}}{\pgfqpoint{7.887162in}{6.415045in}}%
\pgfpathcurveto{\pgfqpoint{7.876112in}{6.415045in}}{\pgfqpoint{7.865513in}{6.410655in}}{\pgfqpoint{7.857699in}{6.402841in}}%
\pgfpathcurveto{\pgfqpoint{7.849885in}{6.395028in}}{\pgfqpoint{7.845495in}{6.384429in}}{\pgfqpoint{7.845495in}{6.373379in}}%
\pgfpathcurveto{\pgfqpoint{7.845495in}{6.362328in}}{\pgfqpoint{7.849885in}{6.351729in}}{\pgfqpoint{7.857699in}{6.343916in}}%
\pgfpathcurveto{\pgfqpoint{7.865513in}{6.336102in}}{\pgfqpoint{7.876112in}{6.331712in}}{\pgfqpoint{7.887162in}{6.331712in}}%
\pgfpathlineto{\pgfqpoint{7.887162in}{6.331712in}}%
\pgfpathclose%
\pgfusepath{stroke,fill}%
\end{pgfscope}%
\begin{pgfscope}%
\pgfpathrectangle{\pgfqpoint{7.622482in}{5.272501in}}{\pgfqpoint{2.177280in}{2.201755in}}%
\pgfusepath{clip}%
\pgfsetbuttcap%
\pgfsetroundjoin%
\definecolor{currentfill}{rgb}{0.121569,0.466667,0.705882}%
\pgfsetfillcolor{currentfill}%
\pgfsetlinewidth{0.481800pt}%
\definecolor{currentstroke}{rgb}{1.000000,1.000000,1.000000}%
\pgfsetstrokecolor{currentstroke}%
\pgfsetdash{}{0pt}%
\pgfpathmoveto{\pgfqpoint{7.887162in}{6.248312in}}%
\pgfpathcurveto{\pgfqpoint{7.898212in}{6.248312in}}{\pgfqpoint{7.908811in}{6.252702in}}{\pgfqpoint{7.916625in}{6.260516in}}%
\pgfpathcurveto{\pgfqpoint{7.924438in}{6.268330in}}{\pgfqpoint{7.928828in}{6.278929in}}{\pgfqpoint{7.928828in}{6.289979in}}%
\pgfpathcurveto{\pgfqpoint{7.928828in}{6.301029in}}{\pgfqpoint{7.924438in}{6.311628in}}{\pgfqpoint{7.916625in}{6.319442in}}%
\pgfpathcurveto{\pgfqpoint{7.908811in}{6.327255in}}{\pgfqpoint{7.898212in}{6.331645in}}{\pgfqpoint{7.887162in}{6.331645in}}%
\pgfpathcurveto{\pgfqpoint{7.876112in}{6.331645in}}{\pgfqpoint{7.865513in}{6.327255in}}{\pgfqpoint{7.857699in}{6.319442in}}%
\pgfpathcurveto{\pgfqpoint{7.849885in}{6.311628in}}{\pgfqpoint{7.845495in}{6.301029in}}{\pgfqpoint{7.845495in}{6.289979in}}%
\pgfpathcurveto{\pgfqpoint{7.845495in}{6.278929in}}{\pgfqpoint{7.849885in}{6.268330in}}{\pgfqpoint{7.857699in}{6.260516in}}%
\pgfpathcurveto{\pgfqpoint{7.865513in}{6.252702in}}{\pgfqpoint{7.876112in}{6.248312in}}{\pgfqpoint{7.887162in}{6.248312in}}%
\pgfpathlineto{\pgfqpoint{7.887162in}{6.248312in}}%
\pgfpathclose%
\pgfusepath{stroke,fill}%
\end{pgfscope}%
\begin{pgfscope}%
\pgfpathrectangle{\pgfqpoint{7.622482in}{5.272501in}}{\pgfqpoint{2.177280in}{2.201755in}}%
\pgfusepath{clip}%
\pgfsetbuttcap%
\pgfsetroundjoin%
\definecolor{currentfill}{rgb}{0.121569,0.466667,0.705882}%
\pgfsetfillcolor{currentfill}%
\pgfsetlinewidth{0.481800pt}%
\definecolor{currentstroke}{rgb}{1.000000,1.000000,1.000000}%
\pgfsetstrokecolor{currentstroke}%
\pgfsetdash{}{0pt}%
\pgfpathmoveto{\pgfqpoint{8.022670in}{6.498512in}}%
\pgfpathcurveto{\pgfqpoint{8.033720in}{6.498512in}}{\pgfqpoint{8.044319in}{6.502902in}}{\pgfqpoint{8.052132in}{6.510715in}}%
\pgfpathcurveto{\pgfqpoint{8.059946in}{6.518529in}}{\pgfqpoint{8.064336in}{6.529128in}}{\pgfqpoint{8.064336in}{6.540178in}}%
\pgfpathcurveto{\pgfqpoint{8.064336in}{6.551228in}}{\pgfqpoint{8.059946in}{6.561827in}}{\pgfqpoint{8.052132in}{6.569641in}}%
\pgfpathcurveto{\pgfqpoint{8.044319in}{6.577455in}}{\pgfqpoint{8.033720in}{6.581845in}}{\pgfqpoint{8.022670in}{6.581845in}}%
\pgfpathcurveto{\pgfqpoint{8.011619in}{6.581845in}}{\pgfqpoint{8.001020in}{6.577455in}}{\pgfqpoint{7.993207in}{6.569641in}}%
\pgfpathcurveto{\pgfqpoint{7.985393in}{6.561827in}}{\pgfqpoint{7.981003in}{6.551228in}}{\pgfqpoint{7.981003in}{6.540178in}}%
\pgfpathcurveto{\pgfqpoint{7.981003in}{6.529128in}}{\pgfqpoint{7.985393in}{6.518529in}}{\pgfqpoint{7.993207in}{6.510715in}}%
\pgfpathcurveto{\pgfqpoint{8.001020in}{6.502902in}}{\pgfqpoint{8.011619in}{6.498512in}}{\pgfqpoint{8.022670in}{6.498512in}}%
\pgfpathlineto{\pgfqpoint{8.022670in}{6.498512in}}%
\pgfpathclose%
\pgfusepath{stroke,fill}%
\end{pgfscope}%
\begin{pgfscope}%
\pgfpathrectangle{\pgfqpoint{7.622482in}{5.272501in}}{\pgfqpoint{2.177280in}{2.201755in}}%
\pgfusepath{clip}%
\pgfsetbuttcap%
\pgfsetroundjoin%
\definecolor{currentfill}{rgb}{0.121569,0.466667,0.705882}%
\pgfsetfillcolor{currentfill}%
\pgfsetlinewidth{0.481800pt}%
\definecolor{currentstroke}{rgb}{1.000000,1.000000,1.000000}%
\pgfsetstrokecolor{currentstroke}%
\pgfsetdash{}{0pt}%
\pgfpathmoveto{\pgfqpoint{7.819408in}{7.082310in}}%
\pgfpathcurveto{\pgfqpoint{7.830458in}{7.082310in}}{\pgfqpoint{7.841057in}{7.086700in}}{\pgfqpoint{7.848871in}{7.094514in}}%
\pgfpathcurveto{\pgfqpoint{7.856684in}{7.102328in}}{\pgfqpoint{7.861075in}{7.112927in}}{\pgfqpoint{7.861075in}{7.123977in}}%
\pgfpathcurveto{\pgfqpoint{7.861075in}{7.135027in}}{\pgfqpoint{7.856684in}{7.145626in}}{\pgfqpoint{7.848871in}{7.153440in}}%
\pgfpathcurveto{\pgfqpoint{7.841057in}{7.161253in}}{\pgfqpoint{7.830458in}{7.165644in}}{\pgfqpoint{7.819408in}{7.165644in}}%
\pgfpathcurveto{\pgfqpoint{7.808358in}{7.165644in}}{\pgfqpoint{7.797759in}{7.161253in}}{\pgfqpoint{7.789945in}{7.153440in}}%
\pgfpathcurveto{\pgfqpoint{7.782132in}{7.145626in}}{\pgfqpoint{7.777741in}{7.135027in}}{\pgfqpoint{7.777741in}{7.123977in}}%
\pgfpathcurveto{\pgfqpoint{7.777741in}{7.112927in}}{\pgfqpoint{7.782132in}{7.102328in}}{\pgfqpoint{7.789945in}{7.094514in}}%
\pgfpathcurveto{\pgfqpoint{7.797759in}{7.086700in}}{\pgfqpoint{7.808358in}{7.082310in}}{\pgfqpoint{7.819408in}{7.082310in}}%
\pgfpathlineto{\pgfqpoint{7.819408in}{7.082310in}}%
\pgfpathclose%
\pgfusepath{stroke,fill}%
\end{pgfscope}%
\begin{pgfscope}%
\pgfpathrectangle{\pgfqpoint{7.622482in}{5.272501in}}{\pgfqpoint{2.177280in}{2.201755in}}%
\pgfusepath{clip}%
\pgfsetbuttcap%
\pgfsetroundjoin%
\definecolor{currentfill}{rgb}{0.121569,0.466667,0.705882}%
\pgfsetfillcolor{currentfill}%
\pgfsetlinewidth{0.481800pt}%
\definecolor{currentstroke}{rgb}{1.000000,1.000000,1.000000}%
\pgfsetstrokecolor{currentstroke}%
\pgfsetdash{}{0pt}%
\pgfpathmoveto{\pgfqpoint{7.887162in}{7.165710in}}%
\pgfpathcurveto{\pgfqpoint{7.898212in}{7.165710in}}{\pgfqpoint{7.908811in}{7.170100in}}{\pgfqpoint{7.916625in}{7.177914in}}%
\pgfpathcurveto{\pgfqpoint{7.924438in}{7.185728in}}{\pgfqpoint{7.928828in}{7.196327in}}{\pgfqpoint{7.928828in}{7.207377in}}%
\pgfpathcurveto{\pgfqpoint{7.928828in}{7.218427in}}{\pgfqpoint{7.924438in}{7.229026in}}{\pgfqpoint{7.916625in}{7.236839in}}%
\pgfpathcurveto{\pgfqpoint{7.908811in}{7.244653in}}{\pgfqpoint{7.898212in}{7.249043in}}{\pgfqpoint{7.887162in}{7.249043in}}%
\pgfpathcurveto{\pgfqpoint{7.876112in}{7.249043in}}{\pgfqpoint{7.865513in}{7.244653in}}{\pgfqpoint{7.857699in}{7.236839in}}%
\pgfpathcurveto{\pgfqpoint{7.849885in}{7.229026in}}{\pgfqpoint{7.845495in}{7.218427in}}{\pgfqpoint{7.845495in}{7.207377in}}%
\pgfpathcurveto{\pgfqpoint{7.845495in}{7.196327in}}{\pgfqpoint{7.849885in}{7.185728in}}{\pgfqpoint{7.857699in}{7.177914in}}%
\pgfpathcurveto{\pgfqpoint{7.865513in}{7.170100in}}{\pgfqpoint{7.876112in}{7.165710in}}{\pgfqpoint{7.887162in}{7.165710in}}%
\pgfpathlineto{\pgfqpoint{7.887162in}{7.165710in}}%
\pgfpathclose%
\pgfusepath{stroke,fill}%
\end{pgfscope}%
\begin{pgfscope}%
\pgfpathrectangle{\pgfqpoint{7.622482in}{5.272501in}}{\pgfqpoint{2.177280in}{2.201755in}}%
\pgfusepath{clip}%
\pgfsetbuttcap%
\pgfsetroundjoin%
\definecolor{currentfill}{rgb}{0.121569,0.466667,0.705882}%
\pgfsetfillcolor{currentfill}%
\pgfsetlinewidth{0.481800pt}%
\definecolor{currentstroke}{rgb}{1.000000,1.000000,1.000000}%
\pgfsetstrokecolor{currentstroke}%
\pgfsetdash{}{0pt}%
\pgfpathmoveto{\pgfqpoint{7.887162in}{6.248312in}}%
\pgfpathcurveto{\pgfqpoint{7.898212in}{6.248312in}}{\pgfqpoint{7.908811in}{6.252702in}}{\pgfqpoint{7.916625in}{6.260516in}}%
\pgfpathcurveto{\pgfqpoint{7.924438in}{6.268330in}}{\pgfqpoint{7.928828in}{6.278929in}}{\pgfqpoint{7.928828in}{6.289979in}}%
\pgfpathcurveto{\pgfqpoint{7.928828in}{6.301029in}}{\pgfqpoint{7.924438in}{6.311628in}}{\pgfqpoint{7.916625in}{6.319442in}}%
\pgfpathcurveto{\pgfqpoint{7.908811in}{6.327255in}}{\pgfqpoint{7.898212in}{6.331645in}}{\pgfqpoint{7.887162in}{6.331645in}}%
\pgfpathcurveto{\pgfqpoint{7.876112in}{6.331645in}}{\pgfqpoint{7.865513in}{6.327255in}}{\pgfqpoint{7.857699in}{6.319442in}}%
\pgfpathcurveto{\pgfqpoint{7.849885in}{6.311628in}}{\pgfqpoint{7.845495in}{6.301029in}}{\pgfqpoint{7.845495in}{6.289979in}}%
\pgfpathcurveto{\pgfqpoint{7.845495in}{6.278929in}}{\pgfqpoint{7.849885in}{6.268330in}}{\pgfqpoint{7.857699in}{6.260516in}}%
\pgfpathcurveto{\pgfqpoint{7.865513in}{6.252702in}}{\pgfqpoint{7.876112in}{6.248312in}}{\pgfqpoint{7.887162in}{6.248312in}}%
\pgfpathlineto{\pgfqpoint{7.887162in}{6.248312in}}%
\pgfpathclose%
\pgfusepath{stroke,fill}%
\end{pgfscope}%
\begin{pgfscope}%
\pgfpathrectangle{\pgfqpoint{7.622482in}{5.272501in}}{\pgfqpoint{2.177280in}{2.201755in}}%
\pgfusepath{clip}%
\pgfsetbuttcap%
\pgfsetroundjoin%
\definecolor{currentfill}{rgb}{0.121569,0.466667,0.705882}%
\pgfsetfillcolor{currentfill}%
\pgfsetlinewidth{0.481800pt}%
\definecolor{currentstroke}{rgb}{1.000000,1.000000,1.000000}%
\pgfsetstrokecolor{currentstroke}%
\pgfsetdash{}{0pt}%
\pgfpathmoveto{\pgfqpoint{7.887162in}{6.331712in}}%
\pgfpathcurveto{\pgfqpoint{7.898212in}{6.331712in}}{\pgfqpoint{7.908811in}{6.336102in}}{\pgfqpoint{7.916625in}{6.343916in}}%
\pgfpathcurveto{\pgfqpoint{7.924438in}{6.351729in}}{\pgfqpoint{7.928828in}{6.362328in}}{\pgfqpoint{7.928828in}{6.373379in}}%
\pgfpathcurveto{\pgfqpoint{7.928828in}{6.384429in}}{\pgfqpoint{7.924438in}{6.395028in}}{\pgfqpoint{7.916625in}{6.402841in}}%
\pgfpathcurveto{\pgfqpoint{7.908811in}{6.410655in}}{\pgfqpoint{7.898212in}{6.415045in}}{\pgfqpoint{7.887162in}{6.415045in}}%
\pgfpathcurveto{\pgfqpoint{7.876112in}{6.415045in}}{\pgfqpoint{7.865513in}{6.410655in}}{\pgfqpoint{7.857699in}{6.402841in}}%
\pgfpathcurveto{\pgfqpoint{7.849885in}{6.395028in}}{\pgfqpoint{7.845495in}{6.384429in}}{\pgfqpoint{7.845495in}{6.373379in}}%
\pgfpathcurveto{\pgfqpoint{7.845495in}{6.362328in}}{\pgfqpoint{7.849885in}{6.351729in}}{\pgfqpoint{7.857699in}{6.343916in}}%
\pgfpathcurveto{\pgfqpoint{7.865513in}{6.336102in}}{\pgfqpoint{7.876112in}{6.331712in}}{\pgfqpoint{7.887162in}{6.331712in}}%
\pgfpathlineto{\pgfqpoint{7.887162in}{6.331712in}}%
\pgfpathclose%
\pgfusepath{stroke,fill}%
\end{pgfscope}%
\begin{pgfscope}%
\pgfpathrectangle{\pgfqpoint{7.622482in}{5.272501in}}{\pgfqpoint{2.177280in}{2.201755in}}%
\pgfusepath{clip}%
\pgfsetbuttcap%
\pgfsetroundjoin%
\definecolor{currentfill}{rgb}{0.121569,0.466667,0.705882}%
\pgfsetfillcolor{currentfill}%
\pgfsetlinewidth{0.481800pt}%
\definecolor{currentstroke}{rgb}{1.000000,1.000000,1.000000}%
\pgfsetstrokecolor{currentstroke}%
\pgfsetdash{}{0pt}%
\pgfpathmoveto{\pgfqpoint{7.887162in}{6.581911in}}%
\pgfpathcurveto{\pgfqpoint{7.898212in}{6.581911in}}{\pgfqpoint{7.908811in}{6.586302in}}{\pgfqpoint{7.916625in}{6.594115in}}%
\pgfpathcurveto{\pgfqpoint{7.924438in}{6.601929in}}{\pgfqpoint{7.928828in}{6.612528in}}{\pgfqpoint{7.928828in}{6.623578in}}%
\pgfpathcurveto{\pgfqpoint{7.928828in}{6.634628in}}{\pgfqpoint{7.924438in}{6.645227in}}{\pgfqpoint{7.916625in}{6.653041in}}%
\pgfpathcurveto{\pgfqpoint{7.908811in}{6.660854in}}{\pgfqpoint{7.898212in}{6.665245in}}{\pgfqpoint{7.887162in}{6.665245in}}%
\pgfpathcurveto{\pgfqpoint{7.876112in}{6.665245in}}{\pgfqpoint{7.865513in}{6.660854in}}{\pgfqpoint{7.857699in}{6.653041in}}%
\pgfpathcurveto{\pgfqpoint{7.849885in}{6.645227in}}{\pgfqpoint{7.845495in}{6.634628in}}{\pgfqpoint{7.845495in}{6.623578in}}%
\pgfpathcurveto{\pgfqpoint{7.845495in}{6.612528in}}{\pgfqpoint{7.849885in}{6.601929in}}{\pgfqpoint{7.857699in}{6.594115in}}%
\pgfpathcurveto{\pgfqpoint{7.865513in}{6.586302in}}{\pgfqpoint{7.876112in}{6.581911in}}{\pgfqpoint{7.887162in}{6.581911in}}%
\pgfpathlineto{\pgfqpoint{7.887162in}{6.581911in}}%
\pgfpathclose%
\pgfusepath{stroke,fill}%
\end{pgfscope}%
\begin{pgfscope}%
\pgfpathrectangle{\pgfqpoint{7.622482in}{5.272501in}}{\pgfqpoint{2.177280in}{2.201755in}}%
\pgfusepath{clip}%
\pgfsetbuttcap%
\pgfsetroundjoin%
\definecolor{currentfill}{rgb}{0.121569,0.466667,0.705882}%
\pgfsetfillcolor{currentfill}%
\pgfsetlinewidth{0.481800pt}%
\definecolor{currentstroke}{rgb}{1.000000,1.000000,1.000000}%
\pgfsetstrokecolor{currentstroke}%
\pgfsetdash{}{0pt}%
\pgfpathmoveto{\pgfqpoint{7.819408in}{6.665311in}}%
\pgfpathcurveto{\pgfqpoint{7.830458in}{6.665311in}}{\pgfqpoint{7.841057in}{6.669701in}}{\pgfqpoint{7.848871in}{6.677515in}}%
\pgfpathcurveto{\pgfqpoint{7.856684in}{6.685329in}}{\pgfqpoint{7.861075in}{6.695928in}}{\pgfqpoint{7.861075in}{6.706978in}}%
\pgfpathcurveto{\pgfqpoint{7.861075in}{6.718028in}}{\pgfqpoint{7.856684in}{6.728627in}}{\pgfqpoint{7.848871in}{6.736441in}}%
\pgfpathcurveto{\pgfqpoint{7.841057in}{6.744254in}}{\pgfqpoint{7.830458in}{6.748644in}}{\pgfqpoint{7.819408in}{6.748644in}}%
\pgfpathcurveto{\pgfqpoint{7.808358in}{6.748644in}}{\pgfqpoint{7.797759in}{6.744254in}}{\pgfqpoint{7.789945in}{6.736441in}}%
\pgfpathcurveto{\pgfqpoint{7.782132in}{6.728627in}}{\pgfqpoint{7.777741in}{6.718028in}}{\pgfqpoint{7.777741in}{6.706978in}}%
\pgfpathcurveto{\pgfqpoint{7.777741in}{6.695928in}}{\pgfqpoint{7.782132in}{6.685329in}}{\pgfqpoint{7.789945in}{6.677515in}}%
\pgfpathcurveto{\pgfqpoint{7.797759in}{6.669701in}}{\pgfqpoint{7.808358in}{6.665311in}}{\pgfqpoint{7.819408in}{6.665311in}}%
\pgfpathlineto{\pgfqpoint{7.819408in}{6.665311in}}%
\pgfpathclose%
\pgfusepath{stroke,fill}%
\end{pgfscope}%
\begin{pgfscope}%
\pgfpathrectangle{\pgfqpoint{7.622482in}{5.272501in}}{\pgfqpoint{2.177280in}{2.201755in}}%
\pgfusepath{clip}%
\pgfsetbuttcap%
\pgfsetroundjoin%
\definecolor{currentfill}{rgb}{0.121569,0.466667,0.705882}%
\pgfsetfillcolor{currentfill}%
\pgfsetlinewidth{0.481800pt}%
\definecolor{currentstroke}{rgb}{1.000000,1.000000,1.000000}%
\pgfsetstrokecolor{currentstroke}%
\pgfsetdash{}{0pt}%
\pgfpathmoveto{\pgfqpoint{7.887162in}{6.164912in}}%
\pgfpathcurveto{\pgfqpoint{7.898212in}{6.164912in}}{\pgfqpoint{7.908811in}{6.169303in}}{\pgfqpoint{7.916625in}{6.177116in}}%
\pgfpathcurveto{\pgfqpoint{7.924438in}{6.184930in}}{\pgfqpoint{7.928828in}{6.195529in}}{\pgfqpoint{7.928828in}{6.206579in}}%
\pgfpathcurveto{\pgfqpoint{7.928828in}{6.217629in}}{\pgfqpoint{7.924438in}{6.228228in}}{\pgfqpoint{7.916625in}{6.236042in}}%
\pgfpathcurveto{\pgfqpoint{7.908811in}{6.243855in}}{\pgfqpoint{7.898212in}{6.248246in}}{\pgfqpoint{7.887162in}{6.248246in}}%
\pgfpathcurveto{\pgfqpoint{7.876112in}{6.248246in}}{\pgfqpoint{7.865513in}{6.243855in}}{\pgfqpoint{7.857699in}{6.236042in}}%
\pgfpathcurveto{\pgfqpoint{7.849885in}{6.228228in}}{\pgfqpoint{7.845495in}{6.217629in}}{\pgfqpoint{7.845495in}{6.206579in}}%
\pgfpathcurveto{\pgfqpoint{7.845495in}{6.195529in}}{\pgfqpoint{7.849885in}{6.184930in}}{\pgfqpoint{7.857699in}{6.177116in}}%
\pgfpathcurveto{\pgfqpoint{7.865513in}{6.169303in}}{\pgfqpoint{7.876112in}{6.164912in}}{\pgfqpoint{7.887162in}{6.164912in}}%
\pgfpathlineto{\pgfqpoint{7.887162in}{6.164912in}}%
\pgfpathclose%
\pgfusepath{stroke,fill}%
\end{pgfscope}%
\begin{pgfscope}%
\pgfpathrectangle{\pgfqpoint{7.622482in}{5.272501in}}{\pgfqpoint{2.177280in}{2.201755in}}%
\pgfusepath{clip}%
\pgfsetbuttcap%
\pgfsetroundjoin%
\definecolor{currentfill}{rgb}{0.121569,0.466667,0.705882}%
\pgfsetfillcolor{currentfill}%
\pgfsetlinewidth{0.481800pt}%
\definecolor{currentstroke}{rgb}{1.000000,1.000000,1.000000}%
\pgfsetstrokecolor{currentstroke}%
\pgfsetdash{}{0pt}%
\pgfpathmoveto{\pgfqpoint{7.887162in}{6.498512in}}%
\pgfpathcurveto{\pgfqpoint{7.898212in}{6.498512in}}{\pgfqpoint{7.908811in}{6.502902in}}{\pgfqpoint{7.916625in}{6.510715in}}%
\pgfpathcurveto{\pgfqpoint{7.924438in}{6.518529in}}{\pgfqpoint{7.928828in}{6.529128in}}{\pgfqpoint{7.928828in}{6.540178in}}%
\pgfpathcurveto{\pgfqpoint{7.928828in}{6.551228in}}{\pgfqpoint{7.924438in}{6.561827in}}{\pgfqpoint{7.916625in}{6.569641in}}%
\pgfpathcurveto{\pgfqpoint{7.908811in}{6.577455in}}{\pgfqpoint{7.898212in}{6.581845in}}{\pgfqpoint{7.887162in}{6.581845in}}%
\pgfpathcurveto{\pgfqpoint{7.876112in}{6.581845in}}{\pgfqpoint{7.865513in}{6.577455in}}{\pgfqpoint{7.857699in}{6.569641in}}%
\pgfpathcurveto{\pgfqpoint{7.849885in}{6.561827in}}{\pgfqpoint{7.845495in}{6.551228in}}{\pgfqpoint{7.845495in}{6.540178in}}%
\pgfpathcurveto{\pgfqpoint{7.845495in}{6.529128in}}{\pgfqpoint{7.849885in}{6.518529in}}{\pgfqpoint{7.857699in}{6.510715in}}%
\pgfpathcurveto{\pgfqpoint{7.865513in}{6.502902in}}{\pgfqpoint{7.876112in}{6.498512in}}{\pgfqpoint{7.887162in}{6.498512in}}%
\pgfpathlineto{\pgfqpoint{7.887162in}{6.498512in}}%
\pgfpathclose%
\pgfusepath{stroke,fill}%
\end{pgfscope}%
\begin{pgfscope}%
\pgfpathrectangle{\pgfqpoint{7.622482in}{5.272501in}}{\pgfqpoint{2.177280in}{2.201755in}}%
\pgfusepath{clip}%
\pgfsetbuttcap%
\pgfsetroundjoin%
\definecolor{currentfill}{rgb}{0.121569,0.466667,0.705882}%
\pgfsetfillcolor{currentfill}%
\pgfsetlinewidth{0.481800pt}%
\definecolor{currentstroke}{rgb}{1.000000,1.000000,1.000000}%
\pgfsetstrokecolor{currentstroke}%
\pgfsetdash{}{0pt}%
\pgfpathmoveto{\pgfqpoint{7.954916in}{6.581911in}}%
\pgfpathcurveto{\pgfqpoint{7.965966in}{6.581911in}}{\pgfqpoint{7.976565in}{6.586302in}}{\pgfqpoint{7.984378in}{6.594115in}}%
\pgfpathcurveto{\pgfqpoint{7.992192in}{6.601929in}}{\pgfqpoint{7.996582in}{6.612528in}}{\pgfqpoint{7.996582in}{6.623578in}}%
\pgfpathcurveto{\pgfqpoint{7.996582in}{6.634628in}}{\pgfqpoint{7.992192in}{6.645227in}}{\pgfqpoint{7.984378in}{6.653041in}}%
\pgfpathcurveto{\pgfqpoint{7.976565in}{6.660854in}}{\pgfqpoint{7.965966in}{6.665245in}}{\pgfqpoint{7.954916in}{6.665245in}}%
\pgfpathcurveto{\pgfqpoint{7.943866in}{6.665245in}}{\pgfqpoint{7.933267in}{6.660854in}}{\pgfqpoint{7.925453in}{6.653041in}}%
\pgfpathcurveto{\pgfqpoint{7.917639in}{6.645227in}}{\pgfqpoint{7.913249in}{6.634628in}}{\pgfqpoint{7.913249in}{6.623578in}}%
\pgfpathcurveto{\pgfqpoint{7.913249in}{6.612528in}}{\pgfqpoint{7.917639in}{6.601929in}}{\pgfqpoint{7.925453in}{6.594115in}}%
\pgfpathcurveto{\pgfqpoint{7.933267in}{6.586302in}}{\pgfqpoint{7.943866in}{6.581911in}}{\pgfqpoint{7.954916in}{6.581911in}}%
\pgfpathlineto{\pgfqpoint{7.954916in}{6.581911in}}%
\pgfpathclose%
\pgfusepath{stroke,fill}%
\end{pgfscope}%
\begin{pgfscope}%
\pgfpathrectangle{\pgfqpoint{7.622482in}{5.272501in}}{\pgfqpoint{2.177280in}{2.201755in}}%
\pgfusepath{clip}%
\pgfsetbuttcap%
\pgfsetroundjoin%
\definecolor{currentfill}{rgb}{0.121569,0.466667,0.705882}%
\pgfsetfillcolor{currentfill}%
\pgfsetlinewidth{0.481800pt}%
\definecolor{currentstroke}{rgb}{1.000000,1.000000,1.000000}%
\pgfsetstrokecolor{currentstroke}%
\pgfsetdash{}{0pt}%
\pgfpathmoveto{\pgfqpoint{7.954916in}{5.581114in}}%
\pgfpathcurveto{\pgfqpoint{7.965966in}{5.581114in}}{\pgfqpoint{7.976565in}{5.585504in}}{\pgfqpoint{7.984378in}{5.593317in}}%
\pgfpathcurveto{\pgfqpoint{7.992192in}{5.601131in}}{\pgfqpoint{7.996582in}{5.611730in}}{\pgfqpoint{7.996582in}{5.622780in}}%
\pgfpathcurveto{\pgfqpoint{7.996582in}{5.633830in}}{\pgfqpoint{7.992192in}{5.644429in}}{\pgfqpoint{7.984378in}{5.652243in}}%
\pgfpathcurveto{\pgfqpoint{7.976565in}{5.660057in}}{\pgfqpoint{7.965966in}{5.664447in}}{\pgfqpoint{7.954916in}{5.664447in}}%
\pgfpathcurveto{\pgfqpoint{7.943866in}{5.664447in}}{\pgfqpoint{7.933267in}{5.660057in}}{\pgfqpoint{7.925453in}{5.652243in}}%
\pgfpathcurveto{\pgfqpoint{7.917639in}{5.644429in}}{\pgfqpoint{7.913249in}{5.633830in}}{\pgfqpoint{7.913249in}{5.622780in}}%
\pgfpathcurveto{\pgfqpoint{7.913249in}{5.611730in}}{\pgfqpoint{7.917639in}{5.601131in}}{\pgfqpoint{7.925453in}{5.593317in}}%
\pgfpathcurveto{\pgfqpoint{7.933267in}{5.585504in}}{\pgfqpoint{7.943866in}{5.581114in}}{\pgfqpoint{7.954916in}{5.581114in}}%
\pgfpathlineto{\pgfqpoint{7.954916in}{5.581114in}}%
\pgfpathclose%
\pgfusepath{stroke,fill}%
\end{pgfscope}%
\begin{pgfscope}%
\pgfpathrectangle{\pgfqpoint{7.622482in}{5.272501in}}{\pgfqpoint{2.177280in}{2.201755in}}%
\pgfusepath{clip}%
\pgfsetbuttcap%
\pgfsetroundjoin%
\definecolor{currentfill}{rgb}{0.121569,0.466667,0.705882}%
\pgfsetfillcolor{currentfill}%
\pgfsetlinewidth{0.481800pt}%
\definecolor{currentstroke}{rgb}{1.000000,1.000000,1.000000}%
\pgfsetstrokecolor{currentstroke}%
\pgfsetdash{}{0pt}%
\pgfpathmoveto{\pgfqpoint{7.887162in}{6.331712in}}%
\pgfpathcurveto{\pgfqpoint{7.898212in}{6.331712in}}{\pgfqpoint{7.908811in}{6.336102in}}{\pgfqpoint{7.916625in}{6.343916in}}%
\pgfpathcurveto{\pgfqpoint{7.924438in}{6.351729in}}{\pgfqpoint{7.928828in}{6.362328in}}{\pgfqpoint{7.928828in}{6.373379in}}%
\pgfpathcurveto{\pgfqpoint{7.928828in}{6.384429in}}{\pgfqpoint{7.924438in}{6.395028in}}{\pgfqpoint{7.916625in}{6.402841in}}%
\pgfpathcurveto{\pgfqpoint{7.908811in}{6.410655in}}{\pgfqpoint{7.898212in}{6.415045in}}{\pgfqpoint{7.887162in}{6.415045in}}%
\pgfpathcurveto{\pgfqpoint{7.876112in}{6.415045in}}{\pgfqpoint{7.865513in}{6.410655in}}{\pgfqpoint{7.857699in}{6.402841in}}%
\pgfpathcurveto{\pgfqpoint{7.849885in}{6.395028in}}{\pgfqpoint{7.845495in}{6.384429in}}{\pgfqpoint{7.845495in}{6.373379in}}%
\pgfpathcurveto{\pgfqpoint{7.845495in}{6.362328in}}{\pgfqpoint{7.849885in}{6.351729in}}{\pgfqpoint{7.857699in}{6.343916in}}%
\pgfpathcurveto{\pgfqpoint{7.865513in}{6.336102in}}{\pgfqpoint{7.876112in}{6.331712in}}{\pgfqpoint{7.887162in}{6.331712in}}%
\pgfpathlineto{\pgfqpoint{7.887162in}{6.331712in}}%
\pgfpathclose%
\pgfusepath{stroke,fill}%
\end{pgfscope}%
\begin{pgfscope}%
\pgfpathrectangle{\pgfqpoint{7.622482in}{5.272501in}}{\pgfqpoint{2.177280in}{2.201755in}}%
\pgfusepath{clip}%
\pgfsetbuttcap%
\pgfsetroundjoin%
\definecolor{currentfill}{rgb}{0.121569,0.466667,0.705882}%
\pgfsetfillcolor{currentfill}%
\pgfsetlinewidth{0.481800pt}%
\definecolor{currentstroke}{rgb}{1.000000,1.000000,1.000000}%
\pgfsetstrokecolor{currentstroke}%
\pgfsetdash{}{0pt}%
\pgfpathmoveto{\pgfqpoint{8.158177in}{6.581911in}}%
\pgfpathcurveto{\pgfqpoint{8.169227in}{6.581911in}}{\pgfqpoint{8.179826in}{6.586302in}}{\pgfqpoint{8.187640in}{6.594115in}}%
\pgfpathcurveto{\pgfqpoint{8.195454in}{6.601929in}}{\pgfqpoint{8.199844in}{6.612528in}}{\pgfqpoint{8.199844in}{6.623578in}}%
\pgfpathcurveto{\pgfqpoint{8.199844in}{6.634628in}}{\pgfqpoint{8.195454in}{6.645227in}}{\pgfqpoint{8.187640in}{6.653041in}}%
\pgfpathcurveto{\pgfqpoint{8.179826in}{6.660854in}}{\pgfqpoint{8.169227in}{6.665245in}}{\pgfqpoint{8.158177in}{6.665245in}}%
\pgfpathcurveto{\pgfqpoint{8.147127in}{6.665245in}}{\pgfqpoint{8.136528in}{6.660854in}}{\pgfqpoint{8.128714in}{6.653041in}}%
\pgfpathcurveto{\pgfqpoint{8.120901in}{6.645227in}}{\pgfqpoint{8.116511in}{6.634628in}}{\pgfqpoint{8.116511in}{6.623578in}}%
\pgfpathcurveto{\pgfqpoint{8.116511in}{6.612528in}}{\pgfqpoint{8.120901in}{6.601929in}}{\pgfqpoint{8.128714in}{6.594115in}}%
\pgfpathcurveto{\pgfqpoint{8.136528in}{6.586302in}}{\pgfqpoint{8.147127in}{6.581911in}}{\pgfqpoint{8.158177in}{6.581911in}}%
\pgfpathlineto{\pgfqpoint{8.158177in}{6.581911in}}%
\pgfpathclose%
\pgfusepath{stroke,fill}%
\end{pgfscope}%
\begin{pgfscope}%
\pgfpathrectangle{\pgfqpoint{7.622482in}{5.272501in}}{\pgfqpoint{2.177280in}{2.201755in}}%
\pgfusepath{clip}%
\pgfsetbuttcap%
\pgfsetroundjoin%
\definecolor{currentfill}{rgb}{0.121569,0.466667,0.705882}%
\pgfsetfillcolor{currentfill}%
\pgfsetlinewidth{0.481800pt}%
\definecolor{currentstroke}{rgb}{1.000000,1.000000,1.000000}%
\pgfsetstrokecolor{currentstroke}%
\pgfsetdash{}{0pt}%
\pgfpathmoveto{\pgfqpoint{8.022670in}{6.832111in}}%
\pgfpathcurveto{\pgfqpoint{8.033720in}{6.832111in}}{\pgfqpoint{8.044319in}{6.836501in}}{\pgfqpoint{8.052132in}{6.844315in}}%
\pgfpathcurveto{\pgfqpoint{8.059946in}{6.852128in}}{\pgfqpoint{8.064336in}{6.862727in}}{\pgfqpoint{8.064336in}{6.873777in}}%
\pgfpathcurveto{\pgfqpoint{8.064336in}{6.884828in}}{\pgfqpoint{8.059946in}{6.895427in}}{\pgfqpoint{8.052132in}{6.903240in}}%
\pgfpathcurveto{\pgfqpoint{8.044319in}{6.911054in}}{\pgfqpoint{8.033720in}{6.915444in}}{\pgfqpoint{8.022670in}{6.915444in}}%
\pgfpathcurveto{\pgfqpoint{8.011619in}{6.915444in}}{\pgfqpoint{8.001020in}{6.911054in}}{\pgfqpoint{7.993207in}{6.903240in}}%
\pgfpathcurveto{\pgfqpoint{7.985393in}{6.895427in}}{\pgfqpoint{7.981003in}{6.884828in}}{\pgfqpoint{7.981003in}{6.873777in}}%
\pgfpathcurveto{\pgfqpoint{7.981003in}{6.862727in}}{\pgfqpoint{7.985393in}{6.852128in}}{\pgfqpoint{7.993207in}{6.844315in}}%
\pgfpathcurveto{\pgfqpoint{8.001020in}{6.836501in}}{\pgfqpoint{8.011619in}{6.832111in}}{\pgfqpoint{8.022670in}{6.832111in}}%
\pgfpathlineto{\pgfqpoint{8.022670in}{6.832111in}}%
\pgfpathclose%
\pgfusepath{stroke,fill}%
\end{pgfscope}%
\begin{pgfscope}%
\pgfpathrectangle{\pgfqpoint{7.622482in}{5.272501in}}{\pgfqpoint{2.177280in}{2.201755in}}%
\pgfusepath{clip}%
\pgfsetbuttcap%
\pgfsetroundjoin%
\definecolor{currentfill}{rgb}{0.121569,0.466667,0.705882}%
\pgfsetfillcolor{currentfill}%
\pgfsetlinewidth{0.481800pt}%
\definecolor{currentstroke}{rgb}{1.000000,1.000000,1.000000}%
\pgfsetstrokecolor{currentstroke}%
\pgfsetdash{}{0pt}%
\pgfpathmoveto{\pgfqpoint{7.954916in}{6.164912in}}%
\pgfpathcurveto{\pgfqpoint{7.965966in}{6.164912in}}{\pgfqpoint{7.976565in}{6.169303in}}{\pgfqpoint{7.984378in}{6.177116in}}%
\pgfpathcurveto{\pgfqpoint{7.992192in}{6.184930in}}{\pgfqpoint{7.996582in}{6.195529in}}{\pgfqpoint{7.996582in}{6.206579in}}%
\pgfpathcurveto{\pgfqpoint{7.996582in}{6.217629in}}{\pgfqpoint{7.992192in}{6.228228in}}{\pgfqpoint{7.984378in}{6.236042in}}%
\pgfpathcurveto{\pgfqpoint{7.976565in}{6.243855in}}{\pgfqpoint{7.965966in}{6.248246in}}{\pgfqpoint{7.954916in}{6.248246in}}%
\pgfpathcurveto{\pgfqpoint{7.943866in}{6.248246in}}{\pgfqpoint{7.933267in}{6.243855in}}{\pgfqpoint{7.925453in}{6.236042in}}%
\pgfpathcurveto{\pgfqpoint{7.917639in}{6.228228in}}{\pgfqpoint{7.913249in}{6.217629in}}{\pgfqpoint{7.913249in}{6.206579in}}%
\pgfpathcurveto{\pgfqpoint{7.913249in}{6.195529in}}{\pgfqpoint{7.917639in}{6.184930in}}{\pgfqpoint{7.925453in}{6.177116in}}%
\pgfpathcurveto{\pgfqpoint{7.933267in}{6.169303in}}{\pgfqpoint{7.943866in}{6.164912in}}{\pgfqpoint{7.954916in}{6.164912in}}%
\pgfpathlineto{\pgfqpoint{7.954916in}{6.164912in}}%
\pgfpathclose%
\pgfusepath{stroke,fill}%
\end{pgfscope}%
\begin{pgfscope}%
\pgfpathrectangle{\pgfqpoint{7.622482in}{5.272501in}}{\pgfqpoint{2.177280in}{2.201755in}}%
\pgfusepath{clip}%
\pgfsetbuttcap%
\pgfsetroundjoin%
\definecolor{currentfill}{rgb}{0.121569,0.466667,0.705882}%
\pgfsetfillcolor{currentfill}%
\pgfsetlinewidth{0.481800pt}%
\definecolor{currentstroke}{rgb}{1.000000,1.000000,1.000000}%
\pgfsetstrokecolor{currentstroke}%
\pgfsetdash{}{0pt}%
\pgfpathmoveto{\pgfqpoint{7.887162in}{6.832111in}}%
\pgfpathcurveto{\pgfqpoint{7.898212in}{6.832111in}}{\pgfqpoint{7.908811in}{6.836501in}}{\pgfqpoint{7.916625in}{6.844315in}}%
\pgfpathcurveto{\pgfqpoint{7.924438in}{6.852128in}}{\pgfqpoint{7.928828in}{6.862727in}}{\pgfqpoint{7.928828in}{6.873777in}}%
\pgfpathcurveto{\pgfqpoint{7.928828in}{6.884828in}}{\pgfqpoint{7.924438in}{6.895427in}}{\pgfqpoint{7.916625in}{6.903240in}}%
\pgfpathcurveto{\pgfqpoint{7.908811in}{6.911054in}}{\pgfqpoint{7.898212in}{6.915444in}}{\pgfqpoint{7.887162in}{6.915444in}}%
\pgfpathcurveto{\pgfqpoint{7.876112in}{6.915444in}}{\pgfqpoint{7.865513in}{6.911054in}}{\pgfqpoint{7.857699in}{6.903240in}}%
\pgfpathcurveto{\pgfqpoint{7.849885in}{6.895427in}}{\pgfqpoint{7.845495in}{6.884828in}}{\pgfqpoint{7.845495in}{6.873777in}}%
\pgfpathcurveto{\pgfqpoint{7.845495in}{6.862727in}}{\pgfqpoint{7.849885in}{6.852128in}}{\pgfqpoint{7.857699in}{6.844315in}}%
\pgfpathcurveto{\pgfqpoint{7.865513in}{6.836501in}}{\pgfqpoint{7.876112in}{6.832111in}}{\pgfqpoint{7.887162in}{6.832111in}}%
\pgfpathlineto{\pgfqpoint{7.887162in}{6.832111in}}%
\pgfpathclose%
\pgfusepath{stroke,fill}%
\end{pgfscope}%
\begin{pgfscope}%
\pgfpathrectangle{\pgfqpoint{7.622482in}{5.272501in}}{\pgfqpoint{2.177280in}{2.201755in}}%
\pgfusepath{clip}%
\pgfsetbuttcap%
\pgfsetroundjoin%
\definecolor{currentfill}{rgb}{0.121569,0.466667,0.705882}%
\pgfsetfillcolor{currentfill}%
\pgfsetlinewidth{0.481800pt}%
\definecolor{currentstroke}{rgb}{1.000000,1.000000,1.000000}%
\pgfsetstrokecolor{currentstroke}%
\pgfsetdash{}{0pt}%
\pgfpathmoveto{\pgfqpoint{7.887162in}{6.331712in}}%
\pgfpathcurveto{\pgfqpoint{7.898212in}{6.331712in}}{\pgfqpoint{7.908811in}{6.336102in}}{\pgfqpoint{7.916625in}{6.343916in}}%
\pgfpathcurveto{\pgfqpoint{7.924438in}{6.351729in}}{\pgfqpoint{7.928828in}{6.362328in}}{\pgfqpoint{7.928828in}{6.373379in}}%
\pgfpathcurveto{\pgfqpoint{7.928828in}{6.384429in}}{\pgfqpoint{7.924438in}{6.395028in}}{\pgfqpoint{7.916625in}{6.402841in}}%
\pgfpathcurveto{\pgfqpoint{7.908811in}{6.410655in}}{\pgfqpoint{7.898212in}{6.415045in}}{\pgfqpoint{7.887162in}{6.415045in}}%
\pgfpathcurveto{\pgfqpoint{7.876112in}{6.415045in}}{\pgfqpoint{7.865513in}{6.410655in}}{\pgfqpoint{7.857699in}{6.402841in}}%
\pgfpathcurveto{\pgfqpoint{7.849885in}{6.395028in}}{\pgfqpoint{7.845495in}{6.384429in}}{\pgfqpoint{7.845495in}{6.373379in}}%
\pgfpathcurveto{\pgfqpoint{7.845495in}{6.362328in}}{\pgfqpoint{7.849885in}{6.351729in}}{\pgfqpoint{7.857699in}{6.343916in}}%
\pgfpathcurveto{\pgfqpoint{7.865513in}{6.336102in}}{\pgfqpoint{7.876112in}{6.331712in}}{\pgfqpoint{7.887162in}{6.331712in}}%
\pgfpathlineto{\pgfqpoint{7.887162in}{6.331712in}}%
\pgfpathclose%
\pgfusepath{stroke,fill}%
\end{pgfscope}%
\begin{pgfscope}%
\pgfpathrectangle{\pgfqpoint{7.622482in}{5.272501in}}{\pgfqpoint{2.177280in}{2.201755in}}%
\pgfusepath{clip}%
\pgfsetbuttcap%
\pgfsetroundjoin%
\definecolor{currentfill}{rgb}{0.121569,0.466667,0.705882}%
\pgfsetfillcolor{currentfill}%
\pgfsetlinewidth{0.481800pt}%
\definecolor{currentstroke}{rgb}{1.000000,1.000000,1.000000}%
\pgfsetstrokecolor{currentstroke}%
\pgfsetdash{}{0pt}%
\pgfpathmoveto{\pgfqpoint{7.887162in}{6.748711in}}%
\pgfpathcurveto{\pgfqpoint{7.898212in}{6.748711in}}{\pgfqpoint{7.908811in}{6.753101in}}{\pgfqpoint{7.916625in}{6.760915in}}%
\pgfpathcurveto{\pgfqpoint{7.924438in}{6.768728in}}{\pgfqpoint{7.928828in}{6.779327in}}{\pgfqpoint{7.928828in}{6.790378in}}%
\pgfpathcurveto{\pgfqpoint{7.928828in}{6.801428in}}{\pgfqpoint{7.924438in}{6.812027in}}{\pgfqpoint{7.916625in}{6.819840in}}%
\pgfpathcurveto{\pgfqpoint{7.908811in}{6.827654in}}{\pgfqpoint{7.898212in}{6.832044in}}{\pgfqpoint{7.887162in}{6.832044in}}%
\pgfpathcurveto{\pgfqpoint{7.876112in}{6.832044in}}{\pgfqpoint{7.865513in}{6.827654in}}{\pgfqpoint{7.857699in}{6.819840in}}%
\pgfpathcurveto{\pgfqpoint{7.849885in}{6.812027in}}{\pgfqpoint{7.845495in}{6.801428in}}{\pgfqpoint{7.845495in}{6.790378in}}%
\pgfpathcurveto{\pgfqpoint{7.845495in}{6.779327in}}{\pgfqpoint{7.849885in}{6.768728in}}{\pgfqpoint{7.857699in}{6.760915in}}%
\pgfpathcurveto{\pgfqpoint{7.865513in}{6.753101in}}{\pgfqpoint{7.876112in}{6.748711in}}{\pgfqpoint{7.887162in}{6.748711in}}%
\pgfpathlineto{\pgfqpoint{7.887162in}{6.748711in}}%
\pgfpathclose%
\pgfusepath{stroke,fill}%
\end{pgfscope}%
\begin{pgfscope}%
\pgfpathrectangle{\pgfqpoint{7.622482in}{5.272501in}}{\pgfqpoint{2.177280in}{2.201755in}}%
\pgfusepath{clip}%
\pgfsetbuttcap%
\pgfsetroundjoin%
\definecolor{currentfill}{rgb}{0.121569,0.466667,0.705882}%
\pgfsetfillcolor{currentfill}%
\pgfsetlinewidth{0.481800pt}%
\definecolor{currentstroke}{rgb}{1.000000,1.000000,1.000000}%
\pgfsetstrokecolor{currentstroke}%
\pgfsetdash{}{0pt}%
\pgfpathmoveto{\pgfqpoint{7.887162in}{6.415112in}}%
\pgfpathcurveto{\pgfqpoint{7.898212in}{6.415112in}}{\pgfqpoint{7.908811in}{6.419502in}}{\pgfqpoint{7.916625in}{6.427316in}}%
\pgfpathcurveto{\pgfqpoint{7.924438in}{6.435129in}}{\pgfqpoint{7.928828in}{6.445728in}}{\pgfqpoint{7.928828in}{6.456778in}}%
\pgfpathcurveto{\pgfqpoint{7.928828in}{6.467828in}}{\pgfqpoint{7.924438in}{6.478428in}}{\pgfqpoint{7.916625in}{6.486241in}}%
\pgfpathcurveto{\pgfqpoint{7.908811in}{6.494055in}}{\pgfqpoint{7.898212in}{6.498445in}}{\pgfqpoint{7.887162in}{6.498445in}}%
\pgfpathcurveto{\pgfqpoint{7.876112in}{6.498445in}}{\pgfqpoint{7.865513in}{6.494055in}}{\pgfqpoint{7.857699in}{6.486241in}}%
\pgfpathcurveto{\pgfqpoint{7.849885in}{6.478428in}}{\pgfqpoint{7.845495in}{6.467828in}}{\pgfqpoint{7.845495in}{6.456778in}}%
\pgfpathcurveto{\pgfqpoint{7.845495in}{6.445728in}}{\pgfqpoint{7.849885in}{6.435129in}}{\pgfqpoint{7.857699in}{6.427316in}}%
\pgfpathcurveto{\pgfqpoint{7.865513in}{6.419502in}}{\pgfqpoint{7.876112in}{6.415112in}}{\pgfqpoint{7.887162in}{6.415112in}}%
\pgfpathlineto{\pgfqpoint{7.887162in}{6.415112in}}%
\pgfpathclose%
\pgfusepath{stroke,fill}%
\end{pgfscope}%
\begin{pgfscope}%
\pgfpathrectangle{\pgfqpoint{7.622482in}{5.272501in}}{\pgfqpoint{2.177280in}{2.201755in}}%
\pgfusepath{clip}%
\pgfsetbuttcap%
\pgfsetroundjoin%
\definecolor{currentfill}{rgb}{1.000000,0.498039,0.054902}%
\pgfsetfillcolor{currentfill}%
\pgfsetlinewidth{0.481800pt}%
\definecolor{currentstroke}{rgb}{1.000000,1.000000,1.000000}%
\pgfsetstrokecolor{currentstroke}%
\pgfsetdash{}{0pt}%
\pgfpathmoveto{\pgfqpoint{8.700208in}{6.331712in}}%
\pgfpathcurveto{\pgfqpoint{8.711258in}{6.331712in}}{\pgfqpoint{8.721857in}{6.336102in}}{\pgfqpoint{8.729671in}{6.343916in}}%
\pgfpathcurveto{\pgfqpoint{8.737485in}{6.351729in}}{\pgfqpoint{8.741875in}{6.362328in}}{\pgfqpoint{8.741875in}{6.373379in}}%
\pgfpathcurveto{\pgfqpoint{8.741875in}{6.384429in}}{\pgfqpoint{8.737485in}{6.395028in}}{\pgfqpoint{8.729671in}{6.402841in}}%
\pgfpathcurveto{\pgfqpoint{8.721857in}{6.410655in}}{\pgfqpoint{8.711258in}{6.415045in}}{\pgfqpoint{8.700208in}{6.415045in}}%
\pgfpathcurveto{\pgfqpoint{8.689158in}{6.415045in}}{\pgfqpoint{8.678559in}{6.410655in}}{\pgfqpoint{8.670745in}{6.402841in}}%
\pgfpathcurveto{\pgfqpoint{8.662932in}{6.395028in}}{\pgfqpoint{8.658542in}{6.384429in}}{\pgfqpoint{8.658542in}{6.373379in}}%
\pgfpathcurveto{\pgfqpoint{8.658542in}{6.362328in}}{\pgfqpoint{8.662932in}{6.351729in}}{\pgfqpoint{8.670745in}{6.343916in}}%
\pgfpathcurveto{\pgfqpoint{8.678559in}{6.336102in}}{\pgfqpoint{8.689158in}{6.331712in}}{\pgfqpoint{8.700208in}{6.331712in}}%
\pgfpathlineto{\pgfqpoint{8.700208in}{6.331712in}}%
\pgfpathclose%
\pgfusepath{stroke,fill}%
\end{pgfscope}%
\begin{pgfscope}%
\pgfpathrectangle{\pgfqpoint{7.622482in}{5.272501in}}{\pgfqpoint{2.177280in}{2.201755in}}%
\pgfusepath{clip}%
\pgfsetbuttcap%
\pgfsetroundjoin%
\definecolor{currentfill}{rgb}{1.000000,0.498039,0.054902}%
\pgfsetfillcolor{currentfill}%
\pgfsetlinewidth{0.481800pt}%
\definecolor{currentstroke}{rgb}{1.000000,1.000000,1.000000}%
\pgfsetstrokecolor{currentstroke}%
\pgfsetdash{}{0pt}%
\pgfpathmoveto{\pgfqpoint{8.767962in}{6.331712in}}%
\pgfpathcurveto{\pgfqpoint{8.779012in}{6.331712in}}{\pgfqpoint{8.789611in}{6.336102in}}{\pgfqpoint{8.797425in}{6.343916in}}%
\pgfpathcurveto{\pgfqpoint{8.805238in}{6.351729in}}{\pgfqpoint{8.809629in}{6.362328in}}{\pgfqpoint{8.809629in}{6.373379in}}%
\pgfpathcurveto{\pgfqpoint{8.809629in}{6.384429in}}{\pgfqpoint{8.805238in}{6.395028in}}{\pgfqpoint{8.797425in}{6.402841in}}%
\pgfpathcurveto{\pgfqpoint{8.789611in}{6.410655in}}{\pgfqpoint{8.779012in}{6.415045in}}{\pgfqpoint{8.767962in}{6.415045in}}%
\pgfpathcurveto{\pgfqpoint{8.756912in}{6.415045in}}{\pgfqpoint{8.746313in}{6.410655in}}{\pgfqpoint{8.738499in}{6.402841in}}%
\pgfpathcurveto{\pgfqpoint{8.730686in}{6.395028in}}{\pgfqpoint{8.726295in}{6.384429in}}{\pgfqpoint{8.726295in}{6.373379in}}%
\pgfpathcurveto{\pgfqpoint{8.726295in}{6.362328in}}{\pgfqpoint{8.730686in}{6.351729in}}{\pgfqpoint{8.738499in}{6.343916in}}%
\pgfpathcurveto{\pgfqpoint{8.746313in}{6.336102in}}{\pgfqpoint{8.756912in}{6.331712in}}{\pgfqpoint{8.767962in}{6.331712in}}%
\pgfpathlineto{\pgfqpoint{8.767962in}{6.331712in}}%
\pgfpathclose%
\pgfusepath{stroke,fill}%
\end{pgfscope}%
\begin{pgfscope}%
\pgfpathrectangle{\pgfqpoint{7.622482in}{5.272501in}}{\pgfqpoint{2.177280in}{2.201755in}}%
\pgfusepath{clip}%
\pgfsetbuttcap%
\pgfsetroundjoin%
\definecolor{currentfill}{rgb}{1.000000,0.498039,0.054902}%
\pgfsetfillcolor{currentfill}%
\pgfsetlinewidth{0.481800pt}%
\definecolor{currentstroke}{rgb}{1.000000,1.000000,1.000000}%
\pgfsetstrokecolor{currentstroke}%
\pgfsetdash{}{0pt}%
\pgfpathmoveto{\pgfqpoint{8.767962in}{6.248312in}}%
\pgfpathcurveto{\pgfqpoint{8.779012in}{6.248312in}}{\pgfqpoint{8.789611in}{6.252702in}}{\pgfqpoint{8.797425in}{6.260516in}}%
\pgfpathcurveto{\pgfqpoint{8.805238in}{6.268330in}}{\pgfqpoint{8.809629in}{6.278929in}}{\pgfqpoint{8.809629in}{6.289979in}}%
\pgfpathcurveto{\pgfqpoint{8.809629in}{6.301029in}}{\pgfqpoint{8.805238in}{6.311628in}}{\pgfqpoint{8.797425in}{6.319442in}}%
\pgfpathcurveto{\pgfqpoint{8.789611in}{6.327255in}}{\pgfqpoint{8.779012in}{6.331645in}}{\pgfqpoint{8.767962in}{6.331645in}}%
\pgfpathcurveto{\pgfqpoint{8.756912in}{6.331645in}}{\pgfqpoint{8.746313in}{6.327255in}}{\pgfqpoint{8.738499in}{6.319442in}}%
\pgfpathcurveto{\pgfqpoint{8.730686in}{6.311628in}}{\pgfqpoint{8.726295in}{6.301029in}}{\pgfqpoint{8.726295in}{6.289979in}}%
\pgfpathcurveto{\pgfqpoint{8.726295in}{6.278929in}}{\pgfqpoint{8.730686in}{6.268330in}}{\pgfqpoint{8.738499in}{6.260516in}}%
\pgfpathcurveto{\pgfqpoint{8.746313in}{6.252702in}}{\pgfqpoint{8.756912in}{6.248312in}}{\pgfqpoint{8.767962in}{6.248312in}}%
\pgfpathlineto{\pgfqpoint{8.767962in}{6.248312in}}%
\pgfpathclose%
\pgfusepath{stroke,fill}%
\end{pgfscope}%
\begin{pgfscope}%
\pgfpathrectangle{\pgfqpoint{7.622482in}{5.272501in}}{\pgfqpoint{2.177280in}{2.201755in}}%
\pgfusepath{clip}%
\pgfsetbuttcap%
\pgfsetroundjoin%
\definecolor{currentfill}{rgb}{1.000000,0.498039,0.054902}%
\pgfsetfillcolor{currentfill}%
\pgfsetlinewidth{0.481800pt}%
\definecolor{currentstroke}{rgb}{1.000000,1.000000,1.000000}%
\pgfsetstrokecolor{currentstroke}%
\pgfsetdash{}{0pt}%
\pgfpathmoveto{\pgfqpoint{8.632454in}{5.581114in}}%
\pgfpathcurveto{\pgfqpoint{8.643504in}{5.581114in}}{\pgfqpoint{8.654104in}{5.585504in}}{\pgfqpoint{8.661917in}{5.593317in}}%
\pgfpathcurveto{\pgfqpoint{8.669731in}{5.601131in}}{\pgfqpoint{8.674121in}{5.611730in}}{\pgfqpoint{8.674121in}{5.622780in}}%
\pgfpathcurveto{\pgfqpoint{8.674121in}{5.633830in}}{\pgfqpoint{8.669731in}{5.644429in}}{\pgfqpoint{8.661917in}{5.652243in}}%
\pgfpathcurveto{\pgfqpoint{8.654104in}{5.660057in}}{\pgfqpoint{8.643504in}{5.664447in}}{\pgfqpoint{8.632454in}{5.664447in}}%
\pgfpathcurveto{\pgfqpoint{8.621404in}{5.664447in}}{\pgfqpoint{8.610805in}{5.660057in}}{\pgfqpoint{8.602992in}{5.652243in}}%
\pgfpathcurveto{\pgfqpoint{8.595178in}{5.644429in}}{\pgfqpoint{8.590788in}{5.633830in}}{\pgfqpoint{8.590788in}{5.622780in}}%
\pgfpathcurveto{\pgfqpoint{8.590788in}{5.611730in}}{\pgfqpoint{8.595178in}{5.601131in}}{\pgfqpoint{8.602992in}{5.593317in}}%
\pgfpathcurveto{\pgfqpoint{8.610805in}{5.585504in}}{\pgfqpoint{8.621404in}{5.581114in}}{\pgfqpoint{8.632454in}{5.581114in}}%
\pgfpathlineto{\pgfqpoint{8.632454in}{5.581114in}}%
\pgfpathclose%
\pgfusepath{stroke,fill}%
\end{pgfscope}%
\begin{pgfscope}%
\pgfpathrectangle{\pgfqpoint{7.622482in}{5.272501in}}{\pgfqpoint{2.177280in}{2.201755in}}%
\pgfusepath{clip}%
\pgfsetbuttcap%
\pgfsetroundjoin%
\definecolor{currentfill}{rgb}{1.000000,0.498039,0.054902}%
\pgfsetfillcolor{currentfill}%
\pgfsetlinewidth{0.481800pt}%
\definecolor{currentstroke}{rgb}{1.000000,1.000000,1.000000}%
\pgfsetstrokecolor{currentstroke}%
\pgfsetdash{}{0pt}%
\pgfpathmoveto{\pgfqpoint{8.767962in}{5.998113in}}%
\pgfpathcurveto{\pgfqpoint{8.779012in}{5.998113in}}{\pgfqpoint{8.789611in}{6.002503in}}{\pgfqpoint{8.797425in}{6.010317in}}%
\pgfpathcurveto{\pgfqpoint{8.805238in}{6.018130in}}{\pgfqpoint{8.809629in}{6.028729in}}{\pgfqpoint{8.809629in}{6.039779in}}%
\pgfpathcurveto{\pgfqpoint{8.809629in}{6.050829in}}{\pgfqpoint{8.805238in}{6.061428in}}{\pgfqpoint{8.797425in}{6.069242in}}%
\pgfpathcurveto{\pgfqpoint{8.789611in}{6.077056in}}{\pgfqpoint{8.779012in}{6.081446in}}{\pgfqpoint{8.767962in}{6.081446in}}%
\pgfpathcurveto{\pgfqpoint{8.756912in}{6.081446in}}{\pgfqpoint{8.746313in}{6.077056in}}{\pgfqpoint{8.738499in}{6.069242in}}%
\pgfpathcurveto{\pgfqpoint{8.730686in}{6.061428in}}{\pgfqpoint{8.726295in}{6.050829in}}{\pgfqpoint{8.726295in}{6.039779in}}%
\pgfpathcurveto{\pgfqpoint{8.726295in}{6.028729in}}{\pgfqpoint{8.730686in}{6.018130in}}{\pgfqpoint{8.738499in}{6.010317in}}%
\pgfpathcurveto{\pgfqpoint{8.746313in}{6.002503in}}{\pgfqpoint{8.756912in}{5.998113in}}{\pgfqpoint{8.767962in}{5.998113in}}%
\pgfpathlineto{\pgfqpoint{8.767962in}{5.998113in}}%
\pgfpathclose%
\pgfusepath{stroke,fill}%
\end{pgfscope}%
\begin{pgfscope}%
\pgfpathrectangle{\pgfqpoint{7.622482in}{5.272501in}}{\pgfqpoint{2.177280in}{2.201755in}}%
\pgfusepath{clip}%
\pgfsetbuttcap%
\pgfsetroundjoin%
\definecolor{currentfill}{rgb}{1.000000,0.498039,0.054902}%
\pgfsetfillcolor{currentfill}%
\pgfsetlinewidth{0.481800pt}%
\definecolor{currentstroke}{rgb}{1.000000,1.000000,1.000000}%
\pgfsetstrokecolor{currentstroke}%
\pgfsetdash{}{0pt}%
\pgfpathmoveto{\pgfqpoint{8.632454in}{5.998113in}}%
\pgfpathcurveto{\pgfqpoint{8.643504in}{5.998113in}}{\pgfqpoint{8.654104in}{6.002503in}}{\pgfqpoint{8.661917in}{6.010317in}}%
\pgfpathcurveto{\pgfqpoint{8.669731in}{6.018130in}}{\pgfqpoint{8.674121in}{6.028729in}}{\pgfqpoint{8.674121in}{6.039779in}}%
\pgfpathcurveto{\pgfqpoint{8.674121in}{6.050829in}}{\pgfqpoint{8.669731in}{6.061428in}}{\pgfqpoint{8.661917in}{6.069242in}}%
\pgfpathcurveto{\pgfqpoint{8.654104in}{6.077056in}}{\pgfqpoint{8.643504in}{6.081446in}}{\pgfqpoint{8.632454in}{6.081446in}}%
\pgfpathcurveto{\pgfqpoint{8.621404in}{6.081446in}}{\pgfqpoint{8.610805in}{6.077056in}}{\pgfqpoint{8.602992in}{6.069242in}}%
\pgfpathcurveto{\pgfqpoint{8.595178in}{6.061428in}}{\pgfqpoint{8.590788in}{6.050829in}}{\pgfqpoint{8.590788in}{6.039779in}}%
\pgfpathcurveto{\pgfqpoint{8.590788in}{6.028729in}}{\pgfqpoint{8.595178in}{6.018130in}}{\pgfqpoint{8.602992in}{6.010317in}}%
\pgfpathcurveto{\pgfqpoint{8.610805in}{6.002503in}}{\pgfqpoint{8.621404in}{5.998113in}}{\pgfqpoint{8.632454in}{5.998113in}}%
\pgfpathlineto{\pgfqpoint{8.632454in}{5.998113in}}%
\pgfpathclose%
\pgfusepath{stroke,fill}%
\end{pgfscope}%
\begin{pgfscope}%
\pgfpathrectangle{\pgfqpoint{7.622482in}{5.272501in}}{\pgfqpoint{2.177280in}{2.201755in}}%
\pgfusepath{clip}%
\pgfsetbuttcap%
\pgfsetroundjoin%
\definecolor{currentfill}{rgb}{1.000000,0.498039,0.054902}%
\pgfsetfillcolor{currentfill}%
\pgfsetlinewidth{0.481800pt}%
\definecolor{currentstroke}{rgb}{1.000000,1.000000,1.000000}%
\pgfsetstrokecolor{currentstroke}%
\pgfsetdash{}{0pt}%
\pgfpathmoveto{\pgfqpoint{8.835716in}{6.415112in}}%
\pgfpathcurveto{\pgfqpoint{8.846766in}{6.415112in}}{\pgfqpoint{8.857365in}{6.419502in}}{\pgfqpoint{8.865179in}{6.427316in}}%
\pgfpathcurveto{\pgfqpoint{8.872992in}{6.435129in}}{\pgfqpoint{8.877383in}{6.445728in}}{\pgfqpoint{8.877383in}{6.456778in}}%
\pgfpathcurveto{\pgfqpoint{8.877383in}{6.467828in}}{\pgfqpoint{8.872992in}{6.478428in}}{\pgfqpoint{8.865179in}{6.486241in}}%
\pgfpathcurveto{\pgfqpoint{8.857365in}{6.494055in}}{\pgfqpoint{8.846766in}{6.498445in}}{\pgfqpoint{8.835716in}{6.498445in}}%
\pgfpathcurveto{\pgfqpoint{8.824666in}{6.498445in}}{\pgfqpoint{8.814067in}{6.494055in}}{\pgfqpoint{8.806253in}{6.486241in}}%
\pgfpathcurveto{\pgfqpoint{8.798440in}{6.478428in}}{\pgfqpoint{8.794049in}{6.467828in}}{\pgfqpoint{8.794049in}{6.456778in}}%
\pgfpathcurveto{\pgfqpoint{8.794049in}{6.445728in}}{\pgfqpoint{8.798440in}{6.435129in}}{\pgfqpoint{8.806253in}{6.427316in}}%
\pgfpathcurveto{\pgfqpoint{8.814067in}{6.419502in}}{\pgfqpoint{8.824666in}{6.415112in}}{\pgfqpoint{8.835716in}{6.415112in}}%
\pgfpathlineto{\pgfqpoint{8.835716in}{6.415112in}}%
\pgfpathclose%
\pgfusepath{stroke,fill}%
\end{pgfscope}%
\begin{pgfscope}%
\pgfpathrectangle{\pgfqpoint{7.622482in}{5.272501in}}{\pgfqpoint{2.177280in}{2.201755in}}%
\pgfusepath{clip}%
\pgfsetbuttcap%
\pgfsetroundjoin%
\definecolor{currentfill}{rgb}{1.000000,0.498039,0.054902}%
\pgfsetfillcolor{currentfill}%
\pgfsetlinewidth{0.481800pt}%
\definecolor{currentstroke}{rgb}{1.000000,1.000000,1.000000}%
\pgfsetstrokecolor{currentstroke}%
\pgfsetdash{}{0pt}%
\pgfpathmoveto{\pgfqpoint{8.429193in}{5.664513in}}%
\pgfpathcurveto{\pgfqpoint{8.440243in}{5.664513in}}{\pgfqpoint{8.450842in}{5.668904in}}{\pgfqpoint{8.458656in}{5.676717in}}%
\pgfpathcurveto{\pgfqpoint{8.466469in}{5.684531in}}{\pgfqpoint{8.470859in}{5.695130in}}{\pgfqpoint{8.470859in}{5.706180in}}%
\pgfpathcurveto{\pgfqpoint{8.470859in}{5.717230in}}{\pgfqpoint{8.466469in}{5.727829in}}{\pgfqpoint{8.458656in}{5.735643in}}%
\pgfpathcurveto{\pgfqpoint{8.450842in}{5.743456in}}{\pgfqpoint{8.440243in}{5.747847in}}{\pgfqpoint{8.429193in}{5.747847in}}%
\pgfpathcurveto{\pgfqpoint{8.418143in}{5.747847in}}{\pgfqpoint{8.407544in}{5.743456in}}{\pgfqpoint{8.399730in}{5.735643in}}%
\pgfpathcurveto{\pgfqpoint{8.391916in}{5.727829in}}{\pgfqpoint{8.387526in}{5.717230in}}{\pgfqpoint{8.387526in}{5.706180in}}%
\pgfpathcurveto{\pgfqpoint{8.387526in}{5.695130in}}{\pgfqpoint{8.391916in}{5.684531in}}{\pgfqpoint{8.399730in}{5.676717in}}%
\pgfpathcurveto{\pgfqpoint{8.407544in}{5.668904in}}{\pgfqpoint{8.418143in}{5.664513in}}{\pgfqpoint{8.429193in}{5.664513in}}%
\pgfpathlineto{\pgfqpoint{8.429193in}{5.664513in}}%
\pgfpathclose%
\pgfusepath{stroke,fill}%
\end{pgfscope}%
\begin{pgfscope}%
\pgfpathrectangle{\pgfqpoint{7.622482in}{5.272501in}}{\pgfqpoint{2.177280in}{2.201755in}}%
\pgfusepath{clip}%
\pgfsetbuttcap%
\pgfsetroundjoin%
\definecolor{currentfill}{rgb}{1.000000,0.498039,0.054902}%
\pgfsetfillcolor{currentfill}%
\pgfsetlinewidth{0.481800pt}%
\definecolor{currentstroke}{rgb}{1.000000,1.000000,1.000000}%
\pgfsetstrokecolor{currentstroke}%
\pgfsetdash{}{0pt}%
\pgfpathmoveto{\pgfqpoint{8.632454in}{6.081512in}}%
\pgfpathcurveto{\pgfqpoint{8.643504in}{6.081512in}}{\pgfqpoint{8.654104in}{6.085903in}}{\pgfqpoint{8.661917in}{6.093716in}}%
\pgfpathcurveto{\pgfqpoint{8.669731in}{6.101530in}}{\pgfqpoint{8.674121in}{6.112129in}}{\pgfqpoint{8.674121in}{6.123179in}}%
\pgfpathcurveto{\pgfqpoint{8.674121in}{6.134229in}}{\pgfqpoint{8.669731in}{6.144828in}}{\pgfqpoint{8.661917in}{6.152642in}}%
\pgfpathcurveto{\pgfqpoint{8.654104in}{6.160456in}}{\pgfqpoint{8.643504in}{6.164846in}}{\pgfqpoint{8.632454in}{6.164846in}}%
\pgfpathcurveto{\pgfqpoint{8.621404in}{6.164846in}}{\pgfqpoint{8.610805in}{6.160456in}}{\pgfqpoint{8.602992in}{6.152642in}}%
\pgfpathcurveto{\pgfqpoint{8.595178in}{6.144828in}}{\pgfqpoint{8.590788in}{6.134229in}}{\pgfqpoint{8.590788in}{6.123179in}}%
\pgfpathcurveto{\pgfqpoint{8.590788in}{6.112129in}}{\pgfqpoint{8.595178in}{6.101530in}}{\pgfqpoint{8.602992in}{6.093716in}}%
\pgfpathcurveto{\pgfqpoint{8.610805in}{6.085903in}}{\pgfqpoint{8.621404in}{6.081512in}}{\pgfqpoint{8.632454in}{6.081512in}}%
\pgfpathlineto{\pgfqpoint{8.632454in}{6.081512in}}%
\pgfpathclose%
\pgfusepath{stroke,fill}%
\end{pgfscope}%
\begin{pgfscope}%
\pgfpathrectangle{\pgfqpoint{7.622482in}{5.272501in}}{\pgfqpoint{2.177280in}{2.201755in}}%
\pgfusepath{clip}%
\pgfsetbuttcap%
\pgfsetroundjoin%
\definecolor{currentfill}{rgb}{1.000000,0.498039,0.054902}%
\pgfsetfillcolor{currentfill}%
\pgfsetlinewidth{0.481800pt}%
\definecolor{currentstroke}{rgb}{1.000000,1.000000,1.000000}%
\pgfsetstrokecolor{currentstroke}%
\pgfsetdash{}{0pt}%
\pgfpathmoveto{\pgfqpoint{8.700208in}{5.914713in}}%
\pgfpathcurveto{\pgfqpoint{8.711258in}{5.914713in}}{\pgfqpoint{8.721857in}{5.919103in}}{\pgfqpoint{8.729671in}{5.926917in}}%
\pgfpathcurveto{\pgfqpoint{8.737485in}{5.934730in}}{\pgfqpoint{8.741875in}{5.945329in}}{\pgfqpoint{8.741875in}{5.956379in}}%
\pgfpathcurveto{\pgfqpoint{8.741875in}{5.967430in}}{\pgfqpoint{8.737485in}{5.978029in}}{\pgfqpoint{8.729671in}{5.985842in}}%
\pgfpathcurveto{\pgfqpoint{8.721857in}{5.993656in}}{\pgfqpoint{8.711258in}{5.998046in}}{\pgfqpoint{8.700208in}{5.998046in}}%
\pgfpathcurveto{\pgfqpoint{8.689158in}{5.998046in}}{\pgfqpoint{8.678559in}{5.993656in}}{\pgfqpoint{8.670745in}{5.985842in}}%
\pgfpathcurveto{\pgfqpoint{8.662932in}{5.978029in}}{\pgfqpoint{8.658542in}{5.967430in}}{\pgfqpoint{8.658542in}{5.956379in}}%
\pgfpathcurveto{\pgfqpoint{8.658542in}{5.945329in}}{\pgfqpoint{8.662932in}{5.934730in}}{\pgfqpoint{8.670745in}{5.926917in}}%
\pgfpathcurveto{\pgfqpoint{8.678559in}{5.919103in}}{\pgfqpoint{8.689158in}{5.914713in}}{\pgfqpoint{8.700208in}{5.914713in}}%
\pgfpathlineto{\pgfqpoint{8.700208in}{5.914713in}}%
\pgfpathclose%
\pgfusepath{stroke,fill}%
\end{pgfscope}%
\begin{pgfscope}%
\pgfpathrectangle{\pgfqpoint{7.622482in}{5.272501in}}{\pgfqpoint{2.177280in}{2.201755in}}%
\pgfusepath{clip}%
\pgfsetbuttcap%
\pgfsetroundjoin%
\definecolor{currentfill}{rgb}{1.000000,0.498039,0.054902}%
\pgfsetfillcolor{currentfill}%
\pgfsetlinewidth{0.481800pt}%
\definecolor{currentstroke}{rgb}{1.000000,1.000000,1.000000}%
\pgfsetstrokecolor{currentstroke}%
\pgfsetdash{}{0pt}%
\pgfpathmoveto{\pgfqpoint{8.429193in}{5.330914in}}%
\pgfpathcurveto{\pgfqpoint{8.440243in}{5.330914in}}{\pgfqpoint{8.450842in}{5.335304in}}{\pgfqpoint{8.458656in}{5.343118in}}%
\pgfpathcurveto{\pgfqpoint{8.466469in}{5.350932in}}{\pgfqpoint{8.470859in}{5.361531in}}{\pgfqpoint{8.470859in}{5.372581in}}%
\pgfpathcurveto{\pgfqpoint{8.470859in}{5.383631in}}{\pgfqpoint{8.466469in}{5.394230in}}{\pgfqpoint{8.458656in}{5.402044in}}%
\pgfpathcurveto{\pgfqpoint{8.450842in}{5.409857in}}{\pgfqpoint{8.440243in}{5.414247in}}{\pgfqpoint{8.429193in}{5.414247in}}%
\pgfpathcurveto{\pgfqpoint{8.418143in}{5.414247in}}{\pgfqpoint{8.407544in}{5.409857in}}{\pgfqpoint{8.399730in}{5.402044in}}%
\pgfpathcurveto{\pgfqpoint{8.391916in}{5.394230in}}{\pgfqpoint{8.387526in}{5.383631in}}{\pgfqpoint{8.387526in}{5.372581in}}%
\pgfpathcurveto{\pgfqpoint{8.387526in}{5.361531in}}{\pgfqpoint{8.391916in}{5.350932in}}{\pgfqpoint{8.399730in}{5.343118in}}%
\pgfpathcurveto{\pgfqpoint{8.407544in}{5.335304in}}{\pgfqpoint{8.418143in}{5.330914in}}{\pgfqpoint{8.429193in}{5.330914in}}%
\pgfpathlineto{\pgfqpoint{8.429193in}{5.330914in}}%
\pgfpathclose%
\pgfusepath{stroke,fill}%
\end{pgfscope}%
\begin{pgfscope}%
\pgfpathrectangle{\pgfqpoint{7.622482in}{5.272501in}}{\pgfqpoint{2.177280in}{2.201755in}}%
\pgfusepath{clip}%
\pgfsetbuttcap%
\pgfsetroundjoin%
\definecolor{currentfill}{rgb}{1.000000,0.498039,0.054902}%
\pgfsetfillcolor{currentfill}%
\pgfsetlinewidth{0.481800pt}%
\definecolor{currentstroke}{rgb}{1.000000,1.000000,1.000000}%
\pgfsetstrokecolor{currentstroke}%
\pgfsetdash{}{0pt}%
\pgfpathmoveto{\pgfqpoint{8.767962in}{6.164912in}}%
\pgfpathcurveto{\pgfqpoint{8.779012in}{6.164912in}}{\pgfqpoint{8.789611in}{6.169303in}}{\pgfqpoint{8.797425in}{6.177116in}}%
\pgfpathcurveto{\pgfqpoint{8.805238in}{6.184930in}}{\pgfqpoint{8.809629in}{6.195529in}}{\pgfqpoint{8.809629in}{6.206579in}}%
\pgfpathcurveto{\pgfqpoint{8.809629in}{6.217629in}}{\pgfqpoint{8.805238in}{6.228228in}}{\pgfqpoint{8.797425in}{6.236042in}}%
\pgfpathcurveto{\pgfqpoint{8.789611in}{6.243855in}}{\pgfqpoint{8.779012in}{6.248246in}}{\pgfqpoint{8.767962in}{6.248246in}}%
\pgfpathcurveto{\pgfqpoint{8.756912in}{6.248246in}}{\pgfqpoint{8.746313in}{6.243855in}}{\pgfqpoint{8.738499in}{6.236042in}}%
\pgfpathcurveto{\pgfqpoint{8.730686in}{6.228228in}}{\pgfqpoint{8.726295in}{6.217629in}}{\pgfqpoint{8.726295in}{6.206579in}}%
\pgfpathcurveto{\pgfqpoint{8.726295in}{6.195529in}}{\pgfqpoint{8.730686in}{6.184930in}}{\pgfqpoint{8.738499in}{6.177116in}}%
\pgfpathcurveto{\pgfqpoint{8.746313in}{6.169303in}}{\pgfqpoint{8.756912in}{6.164912in}}{\pgfqpoint{8.767962in}{6.164912in}}%
\pgfpathlineto{\pgfqpoint{8.767962in}{6.164912in}}%
\pgfpathclose%
\pgfusepath{stroke,fill}%
\end{pgfscope}%
\begin{pgfscope}%
\pgfpathrectangle{\pgfqpoint{7.622482in}{5.272501in}}{\pgfqpoint{2.177280in}{2.201755in}}%
\pgfusepath{clip}%
\pgfsetbuttcap%
\pgfsetroundjoin%
\definecolor{currentfill}{rgb}{1.000000,0.498039,0.054902}%
\pgfsetfillcolor{currentfill}%
\pgfsetlinewidth{0.481800pt}%
\definecolor{currentstroke}{rgb}{1.000000,1.000000,1.000000}%
\pgfsetstrokecolor{currentstroke}%
\pgfsetdash{}{0pt}%
\pgfpathmoveto{\pgfqpoint{8.429193in}{5.497714in}}%
\pgfpathcurveto{\pgfqpoint{8.440243in}{5.497714in}}{\pgfqpoint{8.450842in}{5.502104in}}{\pgfqpoint{8.458656in}{5.509918in}}%
\pgfpathcurveto{\pgfqpoint{8.466469in}{5.517731in}}{\pgfqpoint{8.470859in}{5.528330in}}{\pgfqpoint{8.470859in}{5.539380in}}%
\pgfpathcurveto{\pgfqpoint{8.470859in}{5.550431in}}{\pgfqpoint{8.466469in}{5.561030in}}{\pgfqpoint{8.458656in}{5.568843in}}%
\pgfpathcurveto{\pgfqpoint{8.450842in}{5.576657in}}{\pgfqpoint{8.440243in}{5.581047in}}{\pgfqpoint{8.429193in}{5.581047in}}%
\pgfpathcurveto{\pgfqpoint{8.418143in}{5.581047in}}{\pgfqpoint{8.407544in}{5.576657in}}{\pgfqpoint{8.399730in}{5.568843in}}%
\pgfpathcurveto{\pgfqpoint{8.391916in}{5.561030in}}{\pgfqpoint{8.387526in}{5.550431in}}{\pgfqpoint{8.387526in}{5.539380in}}%
\pgfpathcurveto{\pgfqpoint{8.387526in}{5.528330in}}{\pgfqpoint{8.391916in}{5.517731in}}{\pgfqpoint{8.399730in}{5.509918in}}%
\pgfpathcurveto{\pgfqpoint{8.407544in}{5.502104in}}{\pgfqpoint{8.418143in}{5.497714in}}{\pgfqpoint{8.429193in}{5.497714in}}%
\pgfpathlineto{\pgfqpoint{8.429193in}{5.497714in}}%
\pgfpathclose%
\pgfusepath{stroke,fill}%
\end{pgfscope}%
\begin{pgfscope}%
\pgfpathrectangle{\pgfqpoint{7.622482in}{5.272501in}}{\pgfqpoint{2.177280in}{2.201755in}}%
\pgfusepath{clip}%
\pgfsetbuttcap%
\pgfsetroundjoin%
\definecolor{currentfill}{rgb}{1.000000,0.498039,0.054902}%
\pgfsetfillcolor{currentfill}%
\pgfsetlinewidth{0.481800pt}%
\definecolor{currentstroke}{rgb}{1.000000,1.000000,1.000000}%
\pgfsetstrokecolor{currentstroke}%
\pgfsetdash{}{0pt}%
\pgfpathmoveto{\pgfqpoint{8.700208in}{6.081512in}}%
\pgfpathcurveto{\pgfqpoint{8.711258in}{6.081512in}}{\pgfqpoint{8.721857in}{6.085903in}}{\pgfqpoint{8.729671in}{6.093716in}}%
\pgfpathcurveto{\pgfqpoint{8.737485in}{6.101530in}}{\pgfqpoint{8.741875in}{6.112129in}}{\pgfqpoint{8.741875in}{6.123179in}}%
\pgfpathcurveto{\pgfqpoint{8.741875in}{6.134229in}}{\pgfqpoint{8.737485in}{6.144828in}}{\pgfqpoint{8.729671in}{6.152642in}}%
\pgfpathcurveto{\pgfqpoint{8.721857in}{6.160456in}}{\pgfqpoint{8.711258in}{6.164846in}}{\pgfqpoint{8.700208in}{6.164846in}}%
\pgfpathcurveto{\pgfqpoint{8.689158in}{6.164846in}}{\pgfqpoint{8.678559in}{6.160456in}}{\pgfqpoint{8.670745in}{6.152642in}}%
\pgfpathcurveto{\pgfqpoint{8.662932in}{6.144828in}}{\pgfqpoint{8.658542in}{6.134229in}}{\pgfqpoint{8.658542in}{6.123179in}}%
\pgfpathcurveto{\pgfqpoint{8.658542in}{6.112129in}}{\pgfqpoint{8.662932in}{6.101530in}}{\pgfqpoint{8.670745in}{6.093716in}}%
\pgfpathcurveto{\pgfqpoint{8.678559in}{6.085903in}}{\pgfqpoint{8.689158in}{6.081512in}}{\pgfqpoint{8.700208in}{6.081512in}}%
\pgfpathlineto{\pgfqpoint{8.700208in}{6.081512in}}%
\pgfpathclose%
\pgfusepath{stroke,fill}%
\end{pgfscope}%
\begin{pgfscope}%
\pgfpathrectangle{\pgfqpoint{7.622482in}{5.272501in}}{\pgfqpoint{2.177280in}{2.201755in}}%
\pgfusepath{clip}%
\pgfsetbuttcap%
\pgfsetroundjoin%
\definecolor{currentfill}{rgb}{1.000000,0.498039,0.054902}%
\pgfsetfillcolor{currentfill}%
\pgfsetlinewidth{0.481800pt}%
\definecolor{currentstroke}{rgb}{1.000000,1.000000,1.000000}%
\pgfsetstrokecolor{currentstroke}%
\pgfsetdash{}{0pt}%
\pgfpathmoveto{\pgfqpoint{8.632454in}{6.081512in}}%
\pgfpathcurveto{\pgfqpoint{8.643504in}{6.081512in}}{\pgfqpoint{8.654104in}{6.085903in}}{\pgfqpoint{8.661917in}{6.093716in}}%
\pgfpathcurveto{\pgfqpoint{8.669731in}{6.101530in}}{\pgfqpoint{8.674121in}{6.112129in}}{\pgfqpoint{8.674121in}{6.123179in}}%
\pgfpathcurveto{\pgfqpoint{8.674121in}{6.134229in}}{\pgfqpoint{8.669731in}{6.144828in}}{\pgfqpoint{8.661917in}{6.152642in}}%
\pgfpathcurveto{\pgfqpoint{8.654104in}{6.160456in}}{\pgfqpoint{8.643504in}{6.164846in}}{\pgfqpoint{8.632454in}{6.164846in}}%
\pgfpathcurveto{\pgfqpoint{8.621404in}{6.164846in}}{\pgfqpoint{8.610805in}{6.160456in}}{\pgfqpoint{8.602992in}{6.152642in}}%
\pgfpathcurveto{\pgfqpoint{8.595178in}{6.144828in}}{\pgfqpoint{8.590788in}{6.134229in}}{\pgfqpoint{8.590788in}{6.123179in}}%
\pgfpathcurveto{\pgfqpoint{8.590788in}{6.112129in}}{\pgfqpoint{8.595178in}{6.101530in}}{\pgfqpoint{8.602992in}{6.093716in}}%
\pgfpathcurveto{\pgfqpoint{8.610805in}{6.085903in}}{\pgfqpoint{8.621404in}{6.081512in}}{\pgfqpoint{8.632454in}{6.081512in}}%
\pgfpathlineto{\pgfqpoint{8.632454in}{6.081512in}}%
\pgfpathclose%
\pgfusepath{stroke,fill}%
\end{pgfscope}%
\begin{pgfscope}%
\pgfpathrectangle{\pgfqpoint{7.622482in}{5.272501in}}{\pgfqpoint{2.177280in}{2.201755in}}%
\pgfusepath{clip}%
\pgfsetbuttcap%
\pgfsetroundjoin%
\definecolor{currentfill}{rgb}{1.000000,0.498039,0.054902}%
\pgfsetfillcolor{currentfill}%
\pgfsetlinewidth{0.481800pt}%
\definecolor{currentstroke}{rgb}{1.000000,1.000000,1.000000}%
\pgfsetstrokecolor{currentstroke}%
\pgfsetdash{}{0pt}%
\pgfpathmoveto{\pgfqpoint{8.700208in}{6.248312in}}%
\pgfpathcurveto{\pgfqpoint{8.711258in}{6.248312in}}{\pgfqpoint{8.721857in}{6.252702in}}{\pgfqpoint{8.729671in}{6.260516in}}%
\pgfpathcurveto{\pgfqpoint{8.737485in}{6.268330in}}{\pgfqpoint{8.741875in}{6.278929in}}{\pgfqpoint{8.741875in}{6.289979in}}%
\pgfpathcurveto{\pgfqpoint{8.741875in}{6.301029in}}{\pgfqpoint{8.737485in}{6.311628in}}{\pgfqpoint{8.729671in}{6.319442in}}%
\pgfpathcurveto{\pgfqpoint{8.721857in}{6.327255in}}{\pgfqpoint{8.711258in}{6.331645in}}{\pgfqpoint{8.700208in}{6.331645in}}%
\pgfpathcurveto{\pgfqpoint{8.689158in}{6.331645in}}{\pgfqpoint{8.678559in}{6.327255in}}{\pgfqpoint{8.670745in}{6.319442in}}%
\pgfpathcurveto{\pgfqpoint{8.662932in}{6.311628in}}{\pgfqpoint{8.658542in}{6.301029in}}{\pgfqpoint{8.658542in}{6.289979in}}%
\pgfpathcurveto{\pgfqpoint{8.658542in}{6.278929in}}{\pgfqpoint{8.662932in}{6.268330in}}{\pgfqpoint{8.670745in}{6.260516in}}%
\pgfpathcurveto{\pgfqpoint{8.678559in}{6.252702in}}{\pgfqpoint{8.689158in}{6.248312in}}{\pgfqpoint{8.700208in}{6.248312in}}%
\pgfpathlineto{\pgfqpoint{8.700208in}{6.248312in}}%
\pgfpathclose%
\pgfusepath{stroke,fill}%
\end{pgfscope}%
\begin{pgfscope}%
\pgfpathrectangle{\pgfqpoint{7.622482in}{5.272501in}}{\pgfqpoint{2.177280in}{2.201755in}}%
\pgfusepath{clip}%
\pgfsetbuttcap%
\pgfsetroundjoin%
\definecolor{currentfill}{rgb}{1.000000,0.498039,0.054902}%
\pgfsetfillcolor{currentfill}%
\pgfsetlinewidth{0.481800pt}%
\definecolor{currentstroke}{rgb}{1.000000,1.000000,1.000000}%
\pgfsetstrokecolor{currentstroke}%
\pgfsetdash{}{0pt}%
\pgfpathmoveto{\pgfqpoint{8.767962in}{6.164912in}}%
\pgfpathcurveto{\pgfqpoint{8.779012in}{6.164912in}}{\pgfqpoint{8.789611in}{6.169303in}}{\pgfqpoint{8.797425in}{6.177116in}}%
\pgfpathcurveto{\pgfqpoint{8.805238in}{6.184930in}}{\pgfqpoint{8.809629in}{6.195529in}}{\pgfqpoint{8.809629in}{6.206579in}}%
\pgfpathcurveto{\pgfqpoint{8.809629in}{6.217629in}}{\pgfqpoint{8.805238in}{6.228228in}}{\pgfqpoint{8.797425in}{6.236042in}}%
\pgfpathcurveto{\pgfqpoint{8.789611in}{6.243855in}}{\pgfqpoint{8.779012in}{6.248246in}}{\pgfqpoint{8.767962in}{6.248246in}}%
\pgfpathcurveto{\pgfqpoint{8.756912in}{6.248246in}}{\pgfqpoint{8.746313in}{6.243855in}}{\pgfqpoint{8.738499in}{6.236042in}}%
\pgfpathcurveto{\pgfqpoint{8.730686in}{6.228228in}}{\pgfqpoint{8.726295in}{6.217629in}}{\pgfqpoint{8.726295in}{6.206579in}}%
\pgfpathcurveto{\pgfqpoint{8.726295in}{6.195529in}}{\pgfqpoint{8.730686in}{6.184930in}}{\pgfqpoint{8.738499in}{6.177116in}}%
\pgfpathcurveto{\pgfqpoint{8.746313in}{6.169303in}}{\pgfqpoint{8.756912in}{6.164912in}}{\pgfqpoint{8.767962in}{6.164912in}}%
\pgfpathlineto{\pgfqpoint{8.767962in}{6.164912in}}%
\pgfpathclose%
\pgfusepath{stroke,fill}%
\end{pgfscope}%
\begin{pgfscope}%
\pgfpathrectangle{\pgfqpoint{7.622482in}{5.272501in}}{\pgfqpoint{2.177280in}{2.201755in}}%
\pgfusepath{clip}%
\pgfsetbuttcap%
\pgfsetroundjoin%
\definecolor{currentfill}{rgb}{1.000000,0.498039,0.054902}%
\pgfsetfillcolor{currentfill}%
\pgfsetlinewidth{0.481800pt}%
\definecolor{currentstroke}{rgb}{1.000000,1.000000,1.000000}%
\pgfsetstrokecolor{currentstroke}%
\pgfsetdash{}{0pt}%
\pgfpathmoveto{\pgfqpoint{8.429193in}{5.914713in}}%
\pgfpathcurveto{\pgfqpoint{8.440243in}{5.914713in}}{\pgfqpoint{8.450842in}{5.919103in}}{\pgfqpoint{8.458656in}{5.926917in}}%
\pgfpathcurveto{\pgfqpoint{8.466469in}{5.934730in}}{\pgfqpoint{8.470859in}{5.945329in}}{\pgfqpoint{8.470859in}{5.956379in}}%
\pgfpathcurveto{\pgfqpoint{8.470859in}{5.967430in}}{\pgfqpoint{8.466469in}{5.978029in}}{\pgfqpoint{8.458656in}{5.985842in}}%
\pgfpathcurveto{\pgfqpoint{8.450842in}{5.993656in}}{\pgfqpoint{8.440243in}{5.998046in}}{\pgfqpoint{8.429193in}{5.998046in}}%
\pgfpathcurveto{\pgfqpoint{8.418143in}{5.998046in}}{\pgfqpoint{8.407544in}{5.993656in}}{\pgfqpoint{8.399730in}{5.985842in}}%
\pgfpathcurveto{\pgfqpoint{8.391916in}{5.978029in}}{\pgfqpoint{8.387526in}{5.967430in}}{\pgfqpoint{8.387526in}{5.956379in}}%
\pgfpathcurveto{\pgfqpoint{8.387526in}{5.945329in}}{\pgfqpoint{8.391916in}{5.934730in}}{\pgfqpoint{8.399730in}{5.926917in}}%
\pgfpathcurveto{\pgfqpoint{8.407544in}{5.919103in}}{\pgfqpoint{8.418143in}{5.914713in}}{\pgfqpoint{8.429193in}{5.914713in}}%
\pgfpathlineto{\pgfqpoint{8.429193in}{5.914713in}}%
\pgfpathclose%
\pgfusepath{stroke,fill}%
\end{pgfscope}%
\begin{pgfscope}%
\pgfpathrectangle{\pgfqpoint{7.622482in}{5.272501in}}{\pgfqpoint{2.177280in}{2.201755in}}%
\pgfusepath{clip}%
\pgfsetbuttcap%
\pgfsetroundjoin%
\definecolor{currentfill}{rgb}{1.000000,0.498039,0.054902}%
\pgfsetfillcolor{currentfill}%
\pgfsetlinewidth{0.481800pt}%
\definecolor{currentstroke}{rgb}{1.000000,1.000000,1.000000}%
\pgfsetstrokecolor{currentstroke}%
\pgfsetdash{}{0pt}%
\pgfpathmoveto{\pgfqpoint{8.767962in}{5.497714in}}%
\pgfpathcurveto{\pgfqpoint{8.779012in}{5.497714in}}{\pgfqpoint{8.789611in}{5.502104in}}{\pgfqpoint{8.797425in}{5.509918in}}%
\pgfpathcurveto{\pgfqpoint{8.805238in}{5.517731in}}{\pgfqpoint{8.809629in}{5.528330in}}{\pgfqpoint{8.809629in}{5.539380in}}%
\pgfpathcurveto{\pgfqpoint{8.809629in}{5.550431in}}{\pgfqpoint{8.805238in}{5.561030in}}{\pgfqpoint{8.797425in}{5.568843in}}%
\pgfpathcurveto{\pgfqpoint{8.789611in}{5.576657in}}{\pgfqpoint{8.779012in}{5.581047in}}{\pgfqpoint{8.767962in}{5.581047in}}%
\pgfpathcurveto{\pgfqpoint{8.756912in}{5.581047in}}{\pgfqpoint{8.746313in}{5.576657in}}{\pgfqpoint{8.738499in}{5.568843in}}%
\pgfpathcurveto{\pgfqpoint{8.730686in}{5.561030in}}{\pgfqpoint{8.726295in}{5.550431in}}{\pgfqpoint{8.726295in}{5.539380in}}%
\pgfpathcurveto{\pgfqpoint{8.726295in}{5.528330in}}{\pgfqpoint{8.730686in}{5.517731in}}{\pgfqpoint{8.738499in}{5.509918in}}%
\pgfpathcurveto{\pgfqpoint{8.746313in}{5.502104in}}{\pgfqpoint{8.756912in}{5.497714in}}{\pgfqpoint{8.767962in}{5.497714in}}%
\pgfpathlineto{\pgfqpoint{8.767962in}{5.497714in}}%
\pgfpathclose%
\pgfusepath{stroke,fill}%
\end{pgfscope}%
\begin{pgfscope}%
\pgfpathrectangle{\pgfqpoint{7.622482in}{5.272501in}}{\pgfqpoint{2.177280in}{2.201755in}}%
\pgfusepath{clip}%
\pgfsetbuttcap%
\pgfsetroundjoin%
\definecolor{currentfill}{rgb}{1.000000,0.498039,0.054902}%
\pgfsetfillcolor{currentfill}%
\pgfsetlinewidth{0.481800pt}%
\definecolor{currentstroke}{rgb}{1.000000,1.000000,1.000000}%
\pgfsetstrokecolor{currentstroke}%
\pgfsetdash{}{0pt}%
\pgfpathmoveto{\pgfqpoint{8.496947in}{5.747913in}}%
\pgfpathcurveto{\pgfqpoint{8.507997in}{5.747913in}}{\pgfqpoint{8.518596in}{5.752303in}}{\pgfqpoint{8.526409in}{5.760117in}}%
\pgfpathcurveto{\pgfqpoint{8.534223in}{5.767931in}}{\pgfqpoint{8.538613in}{5.778530in}}{\pgfqpoint{8.538613in}{5.789580in}}%
\pgfpathcurveto{\pgfqpoint{8.538613in}{5.800630in}}{\pgfqpoint{8.534223in}{5.811229in}}{\pgfqpoint{8.526409in}{5.819043in}}%
\pgfpathcurveto{\pgfqpoint{8.518596in}{5.826856in}}{\pgfqpoint{8.507997in}{5.831247in}}{\pgfqpoint{8.496947in}{5.831247in}}%
\pgfpathcurveto{\pgfqpoint{8.485896in}{5.831247in}}{\pgfqpoint{8.475297in}{5.826856in}}{\pgfqpoint{8.467484in}{5.819043in}}%
\pgfpathcurveto{\pgfqpoint{8.459670in}{5.811229in}}{\pgfqpoint{8.455280in}{5.800630in}}{\pgfqpoint{8.455280in}{5.789580in}}%
\pgfpathcurveto{\pgfqpoint{8.455280in}{5.778530in}}{\pgfqpoint{8.459670in}{5.767931in}}{\pgfqpoint{8.467484in}{5.760117in}}%
\pgfpathcurveto{\pgfqpoint{8.475297in}{5.752303in}}{\pgfqpoint{8.485896in}{5.747913in}}{\pgfqpoint{8.496947in}{5.747913in}}%
\pgfpathlineto{\pgfqpoint{8.496947in}{5.747913in}}%
\pgfpathclose%
\pgfusepath{stroke,fill}%
\end{pgfscope}%
\begin{pgfscope}%
\pgfpathrectangle{\pgfqpoint{7.622482in}{5.272501in}}{\pgfqpoint{2.177280in}{2.201755in}}%
\pgfusepath{clip}%
\pgfsetbuttcap%
\pgfsetroundjoin%
\definecolor{currentfill}{rgb}{1.000000,0.498039,0.054902}%
\pgfsetfillcolor{currentfill}%
\pgfsetlinewidth{0.481800pt}%
\definecolor{currentstroke}{rgb}{1.000000,1.000000,1.000000}%
\pgfsetstrokecolor{currentstroke}%
\pgfsetdash{}{0pt}%
\pgfpathmoveto{\pgfqpoint{8.971224in}{6.331712in}}%
\pgfpathcurveto{\pgfqpoint{8.982274in}{6.331712in}}{\pgfqpoint{8.992873in}{6.336102in}}{\pgfqpoint{9.000686in}{6.343916in}}%
\pgfpathcurveto{\pgfqpoint{9.008500in}{6.351729in}}{\pgfqpoint{9.012890in}{6.362328in}}{\pgfqpoint{9.012890in}{6.373379in}}%
\pgfpathcurveto{\pgfqpoint{9.012890in}{6.384429in}}{\pgfqpoint{9.008500in}{6.395028in}}{\pgfqpoint{9.000686in}{6.402841in}}%
\pgfpathcurveto{\pgfqpoint{8.992873in}{6.410655in}}{\pgfqpoint{8.982274in}{6.415045in}}{\pgfqpoint{8.971224in}{6.415045in}}%
\pgfpathcurveto{\pgfqpoint{8.960174in}{6.415045in}}{\pgfqpoint{8.949575in}{6.410655in}}{\pgfqpoint{8.941761in}{6.402841in}}%
\pgfpathcurveto{\pgfqpoint{8.933947in}{6.395028in}}{\pgfqpoint{8.929557in}{6.384429in}}{\pgfqpoint{8.929557in}{6.373379in}}%
\pgfpathcurveto{\pgfqpoint{8.929557in}{6.362328in}}{\pgfqpoint{8.933947in}{6.351729in}}{\pgfqpoint{8.941761in}{6.343916in}}%
\pgfpathcurveto{\pgfqpoint{8.949575in}{6.336102in}}{\pgfqpoint{8.960174in}{6.331712in}}{\pgfqpoint{8.971224in}{6.331712in}}%
\pgfpathlineto{\pgfqpoint{8.971224in}{6.331712in}}%
\pgfpathclose%
\pgfusepath{stroke,fill}%
\end{pgfscope}%
\begin{pgfscope}%
\pgfpathrectangle{\pgfqpoint{7.622482in}{5.272501in}}{\pgfqpoint{2.177280in}{2.201755in}}%
\pgfusepath{clip}%
\pgfsetbuttcap%
\pgfsetroundjoin%
\definecolor{currentfill}{rgb}{1.000000,0.498039,0.054902}%
\pgfsetfillcolor{currentfill}%
\pgfsetlinewidth{0.481800pt}%
\definecolor{currentstroke}{rgb}{1.000000,1.000000,1.000000}%
\pgfsetstrokecolor{currentstroke}%
\pgfsetdash{}{0pt}%
\pgfpathmoveto{\pgfqpoint{8.632454in}{5.998113in}}%
\pgfpathcurveto{\pgfqpoint{8.643504in}{5.998113in}}{\pgfqpoint{8.654104in}{6.002503in}}{\pgfqpoint{8.661917in}{6.010317in}}%
\pgfpathcurveto{\pgfqpoint{8.669731in}{6.018130in}}{\pgfqpoint{8.674121in}{6.028729in}}{\pgfqpoint{8.674121in}{6.039779in}}%
\pgfpathcurveto{\pgfqpoint{8.674121in}{6.050829in}}{\pgfqpoint{8.669731in}{6.061428in}}{\pgfqpoint{8.661917in}{6.069242in}}%
\pgfpathcurveto{\pgfqpoint{8.654104in}{6.077056in}}{\pgfqpoint{8.643504in}{6.081446in}}{\pgfqpoint{8.632454in}{6.081446in}}%
\pgfpathcurveto{\pgfqpoint{8.621404in}{6.081446in}}{\pgfqpoint{8.610805in}{6.077056in}}{\pgfqpoint{8.602992in}{6.069242in}}%
\pgfpathcurveto{\pgfqpoint{8.595178in}{6.061428in}}{\pgfqpoint{8.590788in}{6.050829in}}{\pgfqpoint{8.590788in}{6.039779in}}%
\pgfpathcurveto{\pgfqpoint{8.590788in}{6.028729in}}{\pgfqpoint{8.595178in}{6.018130in}}{\pgfqpoint{8.602992in}{6.010317in}}%
\pgfpathcurveto{\pgfqpoint{8.610805in}{6.002503in}}{\pgfqpoint{8.621404in}{5.998113in}}{\pgfqpoint{8.632454in}{5.998113in}}%
\pgfpathlineto{\pgfqpoint{8.632454in}{5.998113in}}%
\pgfpathclose%
\pgfusepath{stroke,fill}%
\end{pgfscope}%
\begin{pgfscope}%
\pgfpathrectangle{\pgfqpoint{7.622482in}{5.272501in}}{\pgfqpoint{2.177280in}{2.201755in}}%
\pgfusepath{clip}%
\pgfsetbuttcap%
\pgfsetroundjoin%
\definecolor{currentfill}{rgb}{1.000000,0.498039,0.054902}%
\pgfsetfillcolor{currentfill}%
\pgfsetlinewidth{0.481800pt}%
\definecolor{currentstroke}{rgb}{1.000000,1.000000,1.000000}%
\pgfsetstrokecolor{currentstroke}%
\pgfsetdash{}{0pt}%
\pgfpathmoveto{\pgfqpoint{8.767962in}{5.747913in}}%
\pgfpathcurveto{\pgfqpoint{8.779012in}{5.747913in}}{\pgfqpoint{8.789611in}{5.752303in}}{\pgfqpoint{8.797425in}{5.760117in}}%
\pgfpathcurveto{\pgfqpoint{8.805238in}{5.767931in}}{\pgfqpoint{8.809629in}{5.778530in}}{\pgfqpoint{8.809629in}{5.789580in}}%
\pgfpathcurveto{\pgfqpoint{8.809629in}{5.800630in}}{\pgfqpoint{8.805238in}{5.811229in}}{\pgfqpoint{8.797425in}{5.819043in}}%
\pgfpathcurveto{\pgfqpoint{8.789611in}{5.826856in}}{\pgfqpoint{8.779012in}{5.831247in}}{\pgfqpoint{8.767962in}{5.831247in}}%
\pgfpathcurveto{\pgfqpoint{8.756912in}{5.831247in}}{\pgfqpoint{8.746313in}{5.826856in}}{\pgfqpoint{8.738499in}{5.819043in}}%
\pgfpathcurveto{\pgfqpoint{8.730686in}{5.811229in}}{\pgfqpoint{8.726295in}{5.800630in}}{\pgfqpoint{8.726295in}{5.789580in}}%
\pgfpathcurveto{\pgfqpoint{8.726295in}{5.778530in}}{\pgfqpoint{8.730686in}{5.767931in}}{\pgfqpoint{8.738499in}{5.760117in}}%
\pgfpathcurveto{\pgfqpoint{8.746313in}{5.752303in}}{\pgfqpoint{8.756912in}{5.747913in}}{\pgfqpoint{8.767962in}{5.747913in}}%
\pgfpathlineto{\pgfqpoint{8.767962in}{5.747913in}}%
\pgfpathclose%
\pgfusepath{stroke,fill}%
\end{pgfscope}%
\begin{pgfscope}%
\pgfpathrectangle{\pgfqpoint{7.622482in}{5.272501in}}{\pgfqpoint{2.177280in}{2.201755in}}%
\pgfusepath{clip}%
\pgfsetbuttcap%
\pgfsetroundjoin%
\definecolor{currentfill}{rgb}{1.000000,0.498039,0.054902}%
\pgfsetfillcolor{currentfill}%
\pgfsetlinewidth{0.481800pt}%
\definecolor{currentstroke}{rgb}{1.000000,1.000000,1.000000}%
\pgfsetstrokecolor{currentstroke}%
\pgfsetdash{}{0pt}%
\pgfpathmoveto{\pgfqpoint{8.564700in}{5.998113in}}%
\pgfpathcurveto{\pgfqpoint{8.575751in}{5.998113in}}{\pgfqpoint{8.586350in}{6.002503in}}{\pgfqpoint{8.594163in}{6.010317in}}%
\pgfpathcurveto{\pgfqpoint{8.601977in}{6.018130in}}{\pgfqpoint{8.606367in}{6.028729in}}{\pgfqpoint{8.606367in}{6.039779in}}%
\pgfpathcurveto{\pgfqpoint{8.606367in}{6.050829in}}{\pgfqpoint{8.601977in}{6.061428in}}{\pgfqpoint{8.594163in}{6.069242in}}%
\pgfpathcurveto{\pgfqpoint{8.586350in}{6.077056in}}{\pgfqpoint{8.575751in}{6.081446in}}{\pgfqpoint{8.564700in}{6.081446in}}%
\pgfpathcurveto{\pgfqpoint{8.553650in}{6.081446in}}{\pgfqpoint{8.543051in}{6.077056in}}{\pgfqpoint{8.535238in}{6.069242in}}%
\pgfpathcurveto{\pgfqpoint{8.527424in}{6.061428in}}{\pgfqpoint{8.523034in}{6.050829in}}{\pgfqpoint{8.523034in}{6.039779in}}%
\pgfpathcurveto{\pgfqpoint{8.523034in}{6.028729in}}{\pgfqpoint{8.527424in}{6.018130in}}{\pgfqpoint{8.535238in}{6.010317in}}%
\pgfpathcurveto{\pgfqpoint{8.543051in}{6.002503in}}{\pgfqpoint{8.553650in}{5.998113in}}{\pgfqpoint{8.564700in}{5.998113in}}%
\pgfpathlineto{\pgfqpoint{8.564700in}{5.998113in}}%
\pgfpathclose%
\pgfusepath{stroke,fill}%
\end{pgfscope}%
\begin{pgfscope}%
\pgfpathrectangle{\pgfqpoint{7.622482in}{5.272501in}}{\pgfqpoint{2.177280in}{2.201755in}}%
\pgfusepath{clip}%
\pgfsetbuttcap%
\pgfsetroundjoin%
\definecolor{currentfill}{rgb}{1.000000,0.498039,0.054902}%
\pgfsetfillcolor{currentfill}%
\pgfsetlinewidth{0.481800pt}%
\definecolor{currentstroke}{rgb}{1.000000,1.000000,1.000000}%
\pgfsetstrokecolor{currentstroke}%
\pgfsetdash{}{0pt}%
\pgfpathmoveto{\pgfqpoint{8.632454in}{6.081512in}}%
\pgfpathcurveto{\pgfqpoint{8.643504in}{6.081512in}}{\pgfqpoint{8.654104in}{6.085903in}}{\pgfqpoint{8.661917in}{6.093716in}}%
\pgfpathcurveto{\pgfqpoint{8.669731in}{6.101530in}}{\pgfqpoint{8.674121in}{6.112129in}}{\pgfqpoint{8.674121in}{6.123179in}}%
\pgfpathcurveto{\pgfqpoint{8.674121in}{6.134229in}}{\pgfqpoint{8.669731in}{6.144828in}}{\pgfqpoint{8.661917in}{6.152642in}}%
\pgfpathcurveto{\pgfqpoint{8.654104in}{6.160456in}}{\pgfqpoint{8.643504in}{6.164846in}}{\pgfqpoint{8.632454in}{6.164846in}}%
\pgfpathcurveto{\pgfqpoint{8.621404in}{6.164846in}}{\pgfqpoint{8.610805in}{6.160456in}}{\pgfqpoint{8.602992in}{6.152642in}}%
\pgfpathcurveto{\pgfqpoint{8.595178in}{6.144828in}}{\pgfqpoint{8.590788in}{6.134229in}}{\pgfqpoint{8.590788in}{6.123179in}}%
\pgfpathcurveto{\pgfqpoint{8.590788in}{6.112129in}}{\pgfqpoint{8.595178in}{6.101530in}}{\pgfqpoint{8.602992in}{6.093716in}}%
\pgfpathcurveto{\pgfqpoint{8.610805in}{6.085903in}}{\pgfqpoint{8.621404in}{6.081512in}}{\pgfqpoint{8.632454in}{6.081512in}}%
\pgfpathlineto{\pgfqpoint{8.632454in}{6.081512in}}%
\pgfpathclose%
\pgfusepath{stroke,fill}%
\end{pgfscope}%
\begin{pgfscope}%
\pgfpathrectangle{\pgfqpoint{7.622482in}{5.272501in}}{\pgfqpoint{2.177280in}{2.201755in}}%
\pgfusepath{clip}%
\pgfsetbuttcap%
\pgfsetroundjoin%
\definecolor{currentfill}{rgb}{1.000000,0.498039,0.054902}%
\pgfsetfillcolor{currentfill}%
\pgfsetlinewidth{0.481800pt}%
\definecolor{currentstroke}{rgb}{1.000000,1.000000,1.000000}%
\pgfsetstrokecolor{currentstroke}%
\pgfsetdash{}{0pt}%
\pgfpathmoveto{\pgfqpoint{8.700208in}{6.164912in}}%
\pgfpathcurveto{\pgfqpoint{8.711258in}{6.164912in}}{\pgfqpoint{8.721857in}{6.169303in}}{\pgfqpoint{8.729671in}{6.177116in}}%
\pgfpathcurveto{\pgfqpoint{8.737485in}{6.184930in}}{\pgfqpoint{8.741875in}{6.195529in}}{\pgfqpoint{8.741875in}{6.206579in}}%
\pgfpathcurveto{\pgfqpoint{8.741875in}{6.217629in}}{\pgfqpoint{8.737485in}{6.228228in}}{\pgfqpoint{8.729671in}{6.236042in}}%
\pgfpathcurveto{\pgfqpoint{8.721857in}{6.243855in}}{\pgfqpoint{8.711258in}{6.248246in}}{\pgfqpoint{8.700208in}{6.248246in}}%
\pgfpathcurveto{\pgfqpoint{8.689158in}{6.248246in}}{\pgfqpoint{8.678559in}{6.243855in}}{\pgfqpoint{8.670745in}{6.236042in}}%
\pgfpathcurveto{\pgfqpoint{8.662932in}{6.228228in}}{\pgfqpoint{8.658542in}{6.217629in}}{\pgfqpoint{8.658542in}{6.206579in}}%
\pgfpathcurveto{\pgfqpoint{8.658542in}{6.195529in}}{\pgfqpoint{8.662932in}{6.184930in}}{\pgfqpoint{8.670745in}{6.177116in}}%
\pgfpathcurveto{\pgfqpoint{8.678559in}{6.169303in}}{\pgfqpoint{8.689158in}{6.164912in}}{\pgfqpoint{8.700208in}{6.164912in}}%
\pgfpathlineto{\pgfqpoint{8.700208in}{6.164912in}}%
\pgfpathclose%
\pgfusepath{stroke,fill}%
\end{pgfscope}%
\begin{pgfscope}%
\pgfpathrectangle{\pgfqpoint{7.622482in}{5.272501in}}{\pgfqpoint{2.177280in}{2.201755in}}%
\pgfusepath{clip}%
\pgfsetbuttcap%
\pgfsetroundjoin%
\definecolor{currentfill}{rgb}{1.000000,0.498039,0.054902}%
\pgfsetfillcolor{currentfill}%
\pgfsetlinewidth{0.481800pt}%
\definecolor{currentstroke}{rgb}{1.000000,1.000000,1.000000}%
\pgfsetstrokecolor{currentstroke}%
\pgfsetdash{}{0pt}%
\pgfpathmoveto{\pgfqpoint{8.700208in}{5.998113in}}%
\pgfpathcurveto{\pgfqpoint{8.711258in}{5.998113in}}{\pgfqpoint{8.721857in}{6.002503in}}{\pgfqpoint{8.729671in}{6.010317in}}%
\pgfpathcurveto{\pgfqpoint{8.737485in}{6.018130in}}{\pgfqpoint{8.741875in}{6.028729in}}{\pgfqpoint{8.741875in}{6.039779in}}%
\pgfpathcurveto{\pgfqpoint{8.741875in}{6.050829in}}{\pgfqpoint{8.737485in}{6.061428in}}{\pgfqpoint{8.729671in}{6.069242in}}%
\pgfpathcurveto{\pgfqpoint{8.721857in}{6.077056in}}{\pgfqpoint{8.711258in}{6.081446in}}{\pgfqpoint{8.700208in}{6.081446in}}%
\pgfpathcurveto{\pgfqpoint{8.689158in}{6.081446in}}{\pgfqpoint{8.678559in}{6.077056in}}{\pgfqpoint{8.670745in}{6.069242in}}%
\pgfpathcurveto{\pgfqpoint{8.662932in}{6.061428in}}{\pgfqpoint{8.658542in}{6.050829in}}{\pgfqpoint{8.658542in}{6.039779in}}%
\pgfpathcurveto{\pgfqpoint{8.658542in}{6.028729in}}{\pgfqpoint{8.662932in}{6.018130in}}{\pgfqpoint{8.670745in}{6.010317in}}%
\pgfpathcurveto{\pgfqpoint{8.678559in}{6.002503in}}{\pgfqpoint{8.689158in}{5.998113in}}{\pgfqpoint{8.700208in}{5.998113in}}%
\pgfpathlineto{\pgfqpoint{8.700208in}{5.998113in}}%
\pgfpathclose%
\pgfusepath{stroke,fill}%
\end{pgfscope}%
\begin{pgfscope}%
\pgfpathrectangle{\pgfqpoint{7.622482in}{5.272501in}}{\pgfqpoint{2.177280in}{2.201755in}}%
\pgfusepath{clip}%
\pgfsetbuttcap%
\pgfsetroundjoin%
\definecolor{currentfill}{rgb}{1.000000,0.498039,0.054902}%
\pgfsetfillcolor{currentfill}%
\pgfsetlinewidth{0.481800pt}%
\definecolor{currentstroke}{rgb}{1.000000,1.000000,1.000000}%
\pgfsetstrokecolor{currentstroke}%
\pgfsetdash{}{0pt}%
\pgfpathmoveto{\pgfqpoint{8.903470in}{6.164912in}}%
\pgfpathcurveto{\pgfqpoint{8.914520in}{6.164912in}}{\pgfqpoint{8.925119in}{6.169303in}}{\pgfqpoint{8.932933in}{6.177116in}}%
\pgfpathcurveto{\pgfqpoint{8.940746in}{6.184930in}}{\pgfqpoint{8.945136in}{6.195529in}}{\pgfqpoint{8.945136in}{6.206579in}}%
\pgfpathcurveto{\pgfqpoint{8.945136in}{6.217629in}}{\pgfqpoint{8.940746in}{6.228228in}}{\pgfqpoint{8.932933in}{6.236042in}}%
\pgfpathcurveto{\pgfqpoint{8.925119in}{6.243855in}}{\pgfqpoint{8.914520in}{6.248246in}}{\pgfqpoint{8.903470in}{6.248246in}}%
\pgfpathcurveto{\pgfqpoint{8.892420in}{6.248246in}}{\pgfqpoint{8.881821in}{6.243855in}}{\pgfqpoint{8.874007in}{6.236042in}}%
\pgfpathcurveto{\pgfqpoint{8.866193in}{6.228228in}}{\pgfqpoint{8.861803in}{6.217629in}}{\pgfqpoint{8.861803in}{6.206579in}}%
\pgfpathcurveto{\pgfqpoint{8.861803in}{6.195529in}}{\pgfqpoint{8.866193in}{6.184930in}}{\pgfqpoint{8.874007in}{6.177116in}}%
\pgfpathcurveto{\pgfqpoint{8.881821in}{6.169303in}}{\pgfqpoint{8.892420in}{6.164912in}}{\pgfqpoint{8.903470in}{6.164912in}}%
\pgfpathlineto{\pgfqpoint{8.903470in}{6.164912in}}%
\pgfpathclose%
\pgfusepath{stroke,fill}%
\end{pgfscope}%
\begin{pgfscope}%
\pgfpathrectangle{\pgfqpoint{7.622482in}{5.272501in}}{\pgfqpoint{2.177280in}{2.201755in}}%
\pgfusepath{clip}%
\pgfsetbuttcap%
\pgfsetroundjoin%
\definecolor{currentfill}{rgb}{1.000000,0.498039,0.054902}%
\pgfsetfillcolor{currentfill}%
\pgfsetlinewidth{0.481800pt}%
\definecolor{currentstroke}{rgb}{1.000000,1.000000,1.000000}%
\pgfsetstrokecolor{currentstroke}%
\pgfsetdash{}{0pt}%
\pgfpathmoveto{\pgfqpoint{8.767962in}{6.081512in}}%
\pgfpathcurveto{\pgfqpoint{8.779012in}{6.081512in}}{\pgfqpoint{8.789611in}{6.085903in}}{\pgfqpoint{8.797425in}{6.093716in}}%
\pgfpathcurveto{\pgfqpoint{8.805238in}{6.101530in}}{\pgfqpoint{8.809629in}{6.112129in}}{\pgfqpoint{8.809629in}{6.123179in}}%
\pgfpathcurveto{\pgfqpoint{8.809629in}{6.134229in}}{\pgfqpoint{8.805238in}{6.144828in}}{\pgfqpoint{8.797425in}{6.152642in}}%
\pgfpathcurveto{\pgfqpoint{8.789611in}{6.160456in}}{\pgfqpoint{8.779012in}{6.164846in}}{\pgfqpoint{8.767962in}{6.164846in}}%
\pgfpathcurveto{\pgfqpoint{8.756912in}{6.164846in}}{\pgfqpoint{8.746313in}{6.160456in}}{\pgfqpoint{8.738499in}{6.152642in}}%
\pgfpathcurveto{\pgfqpoint{8.730686in}{6.144828in}}{\pgfqpoint{8.726295in}{6.134229in}}{\pgfqpoint{8.726295in}{6.123179in}}%
\pgfpathcurveto{\pgfqpoint{8.726295in}{6.112129in}}{\pgfqpoint{8.730686in}{6.101530in}}{\pgfqpoint{8.738499in}{6.093716in}}%
\pgfpathcurveto{\pgfqpoint{8.746313in}{6.085903in}}{\pgfqpoint{8.756912in}{6.081512in}}{\pgfqpoint{8.767962in}{6.081512in}}%
\pgfpathlineto{\pgfqpoint{8.767962in}{6.081512in}}%
\pgfpathclose%
\pgfusepath{stroke,fill}%
\end{pgfscope}%
\begin{pgfscope}%
\pgfpathrectangle{\pgfqpoint{7.622482in}{5.272501in}}{\pgfqpoint{2.177280in}{2.201755in}}%
\pgfusepath{clip}%
\pgfsetbuttcap%
\pgfsetroundjoin%
\definecolor{currentfill}{rgb}{1.000000,0.498039,0.054902}%
\pgfsetfillcolor{currentfill}%
\pgfsetlinewidth{0.481800pt}%
\definecolor{currentstroke}{rgb}{1.000000,1.000000,1.000000}%
\pgfsetstrokecolor{currentstroke}%
\pgfsetdash{}{0pt}%
\pgfpathmoveto{\pgfqpoint{8.429193in}{5.831313in}}%
\pgfpathcurveto{\pgfqpoint{8.440243in}{5.831313in}}{\pgfqpoint{8.450842in}{5.835703in}}{\pgfqpoint{8.458656in}{5.843517in}}%
\pgfpathcurveto{\pgfqpoint{8.466469in}{5.851331in}}{\pgfqpoint{8.470859in}{5.861930in}}{\pgfqpoint{8.470859in}{5.872980in}}%
\pgfpathcurveto{\pgfqpoint{8.470859in}{5.884030in}}{\pgfqpoint{8.466469in}{5.894629in}}{\pgfqpoint{8.458656in}{5.902442in}}%
\pgfpathcurveto{\pgfqpoint{8.450842in}{5.910256in}}{\pgfqpoint{8.440243in}{5.914646in}}{\pgfqpoint{8.429193in}{5.914646in}}%
\pgfpathcurveto{\pgfqpoint{8.418143in}{5.914646in}}{\pgfqpoint{8.407544in}{5.910256in}}{\pgfqpoint{8.399730in}{5.902442in}}%
\pgfpathcurveto{\pgfqpoint{8.391916in}{5.894629in}}{\pgfqpoint{8.387526in}{5.884030in}}{\pgfqpoint{8.387526in}{5.872980in}}%
\pgfpathcurveto{\pgfqpoint{8.387526in}{5.861930in}}{\pgfqpoint{8.391916in}{5.851331in}}{\pgfqpoint{8.399730in}{5.843517in}}%
\pgfpathcurveto{\pgfqpoint{8.407544in}{5.835703in}}{\pgfqpoint{8.418143in}{5.831313in}}{\pgfqpoint{8.429193in}{5.831313in}}%
\pgfpathlineto{\pgfqpoint{8.429193in}{5.831313in}}%
\pgfpathclose%
\pgfusepath{stroke,fill}%
\end{pgfscope}%
\begin{pgfscope}%
\pgfpathrectangle{\pgfqpoint{7.622482in}{5.272501in}}{\pgfqpoint{2.177280in}{2.201755in}}%
\pgfusepath{clip}%
\pgfsetbuttcap%
\pgfsetroundjoin%
\definecolor{currentfill}{rgb}{1.000000,0.498039,0.054902}%
\pgfsetfillcolor{currentfill}%
\pgfsetlinewidth{0.481800pt}%
\definecolor{currentstroke}{rgb}{1.000000,1.000000,1.000000}%
\pgfsetstrokecolor{currentstroke}%
\pgfsetdash{}{0pt}%
\pgfpathmoveto{\pgfqpoint{8.496947in}{5.664513in}}%
\pgfpathcurveto{\pgfqpoint{8.507997in}{5.664513in}}{\pgfqpoint{8.518596in}{5.668904in}}{\pgfqpoint{8.526409in}{5.676717in}}%
\pgfpathcurveto{\pgfqpoint{8.534223in}{5.684531in}}{\pgfqpoint{8.538613in}{5.695130in}}{\pgfqpoint{8.538613in}{5.706180in}}%
\pgfpathcurveto{\pgfqpoint{8.538613in}{5.717230in}}{\pgfqpoint{8.534223in}{5.727829in}}{\pgfqpoint{8.526409in}{5.735643in}}%
\pgfpathcurveto{\pgfqpoint{8.518596in}{5.743456in}}{\pgfqpoint{8.507997in}{5.747847in}}{\pgfqpoint{8.496947in}{5.747847in}}%
\pgfpathcurveto{\pgfqpoint{8.485896in}{5.747847in}}{\pgfqpoint{8.475297in}{5.743456in}}{\pgfqpoint{8.467484in}{5.735643in}}%
\pgfpathcurveto{\pgfqpoint{8.459670in}{5.727829in}}{\pgfqpoint{8.455280in}{5.717230in}}{\pgfqpoint{8.455280in}{5.706180in}}%
\pgfpathcurveto{\pgfqpoint{8.455280in}{5.695130in}}{\pgfqpoint{8.459670in}{5.684531in}}{\pgfqpoint{8.467484in}{5.676717in}}%
\pgfpathcurveto{\pgfqpoint{8.475297in}{5.668904in}}{\pgfqpoint{8.485896in}{5.664513in}}{\pgfqpoint{8.496947in}{5.664513in}}%
\pgfpathlineto{\pgfqpoint{8.496947in}{5.664513in}}%
\pgfpathclose%
\pgfusepath{stroke,fill}%
\end{pgfscope}%
\begin{pgfscope}%
\pgfpathrectangle{\pgfqpoint{7.622482in}{5.272501in}}{\pgfqpoint{2.177280in}{2.201755in}}%
\pgfusepath{clip}%
\pgfsetbuttcap%
\pgfsetroundjoin%
\definecolor{currentfill}{rgb}{1.000000,0.498039,0.054902}%
\pgfsetfillcolor{currentfill}%
\pgfsetlinewidth{0.481800pt}%
\definecolor{currentstroke}{rgb}{1.000000,1.000000,1.000000}%
\pgfsetstrokecolor{currentstroke}%
\pgfsetdash{}{0pt}%
\pgfpathmoveto{\pgfqpoint{8.429193in}{5.664513in}}%
\pgfpathcurveto{\pgfqpoint{8.440243in}{5.664513in}}{\pgfqpoint{8.450842in}{5.668904in}}{\pgfqpoint{8.458656in}{5.676717in}}%
\pgfpathcurveto{\pgfqpoint{8.466469in}{5.684531in}}{\pgfqpoint{8.470859in}{5.695130in}}{\pgfqpoint{8.470859in}{5.706180in}}%
\pgfpathcurveto{\pgfqpoint{8.470859in}{5.717230in}}{\pgfqpoint{8.466469in}{5.727829in}}{\pgfqpoint{8.458656in}{5.735643in}}%
\pgfpathcurveto{\pgfqpoint{8.450842in}{5.743456in}}{\pgfqpoint{8.440243in}{5.747847in}}{\pgfqpoint{8.429193in}{5.747847in}}%
\pgfpathcurveto{\pgfqpoint{8.418143in}{5.747847in}}{\pgfqpoint{8.407544in}{5.743456in}}{\pgfqpoint{8.399730in}{5.735643in}}%
\pgfpathcurveto{\pgfqpoint{8.391916in}{5.727829in}}{\pgfqpoint{8.387526in}{5.717230in}}{\pgfqpoint{8.387526in}{5.706180in}}%
\pgfpathcurveto{\pgfqpoint{8.387526in}{5.695130in}}{\pgfqpoint{8.391916in}{5.684531in}}{\pgfqpoint{8.399730in}{5.676717in}}%
\pgfpathcurveto{\pgfqpoint{8.407544in}{5.668904in}}{\pgfqpoint{8.418143in}{5.664513in}}{\pgfqpoint{8.429193in}{5.664513in}}%
\pgfpathlineto{\pgfqpoint{8.429193in}{5.664513in}}%
\pgfpathclose%
\pgfusepath{stroke,fill}%
\end{pgfscope}%
\begin{pgfscope}%
\pgfpathrectangle{\pgfqpoint{7.622482in}{5.272501in}}{\pgfqpoint{2.177280in}{2.201755in}}%
\pgfusepath{clip}%
\pgfsetbuttcap%
\pgfsetroundjoin%
\definecolor{currentfill}{rgb}{1.000000,0.498039,0.054902}%
\pgfsetfillcolor{currentfill}%
\pgfsetlinewidth{0.481800pt}%
\definecolor{currentstroke}{rgb}{1.000000,1.000000,1.000000}%
\pgfsetstrokecolor{currentstroke}%
\pgfsetdash{}{0pt}%
\pgfpathmoveto{\pgfqpoint{8.564700in}{5.914713in}}%
\pgfpathcurveto{\pgfqpoint{8.575751in}{5.914713in}}{\pgfqpoint{8.586350in}{5.919103in}}{\pgfqpoint{8.594163in}{5.926917in}}%
\pgfpathcurveto{\pgfqpoint{8.601977in}{5.934730in}}{\pgfqpoint{8.606367in}{5.945329in}}{\pgfqpoint{8.606367in}{5.956379in}}%
\pgfpathcurveto{\pgfqpoint{8.606367in}{5.967430in}}{\pgfqpoint{8.601977in}{5.978029in}}{\pgfqpoint{8.594163in}{5.985842in}}%
\pgfpathcurveto{\pgfqpoint{8.586350in}{5.993656in}}{\pgfqpoint{8.575751in}{5.998046in}}{\pgfqpoint{8.564700in}{5.998046in}}%
\pgfpathcurveto{\pgfqpoint{8.553650in}{5.998046in}}{\pgfqpoint{8.543051in}{5.993656in}}{\pgfqpoint{8.535238in}{5.985842in}}%
\pgfpathcurveto{\pgfqpoint{8.527424in}{5.978029in}}{\pgfqpoint{8.523034in}{5.967430in}}{\pgfqpoint{8.523034in}{5.956379in}}%
\pgfpathcurveto{\pgfqpoint{8.523034in}{5.945329in}}{\pgfqpoint{8.527424in}{5.934730in}}{\pgfqpoint{8.535238in}{5.926917in}}%
\pgfpathcurveto{\pgfqpoint{8.543051in}{5.919103in}}{\pgfqpoint{8.553650in}{5.914713in}}{\pgfqpoint{8.564700in}{5.914713in}}%
\pgfpathlineto{\pgfqpoint{8.564700in}{5.914713in}}%
\pgfpathclose%
\pgfusepath{stroke,fill}%
\end{pgfscope}%
\begin{pgfscope}%
\pgfpathrectangle{\pgfqpoint{7.622482in}{5.272501in}}{\pgfqpoint{2.177280in}{2.201755in}}%
\pgfusepath{clip}%
\pgfsetbuttcap%
\pgfsetroundjoin%
\definecolor{currentfill}{rgb}{1.000000,0.498039,0.054902}%
\pgfsetfillcolor{currentfill}%
\pgfsetlinewidth{0.481800pt}%
\definecolor{currentstroke}{rgb}{1.000000,1.000000,1.000000}%
\pgfsetstrokecolor{currentstroke}%
\pgfsetdash{}{0pt}%
\pgfpathmoveto{\pgfqpoint{8.835716in}{5.914713in}}%
\pgfpathcurveto{\pgfqpoint{8.846766in}{5.914713in}}{\pgfqpoint{8.857365in}{5.919103in}}{\pgfqpoint{8.865179in}{5.926917in}}%
\pgfpathcurveto{\pgfqpoint{8.872992in}{5.934730in}}{\pgfqpoint{8.877383in}{5.945329in}}{\pgfqpoint{8.877383in}{5.956379in}}%
\pgfpathcurveto{\pgfqpoint{8.877383in}{5.967430in}}{\pgfqpoint{8.872992in}{5.978029in}}{\pgfqpoint{8.865179in}{5.985842in}}%
\pgfpathcurveto{\pgfqpoint{8.857365in}{5.993656in}}{\pgfqpoint{8.846766in}{5.998046in}}{\pgfqpoint{8.835716in}{5.998046in}}%
\pgfpathcurveto{\pgfqpoint{8.824666in}{5.998046in}}{\pgfqpoint{8.814067in}{5.993656in}}{\pgfqpoint{8.806253in}{5.985842in}}%
\pgfpathcurveto{\pgfqpoint{8.798440in}{5.978029in}}{\pgfqpoint{8.794049in}{5.967430in}}{\pgfqpoint{8.794049in}{5.956379in}}%
\pgfpathcurveto{\pgfqpoint{8.794049in}{5.945329in}}{\pgfqpoint{8.798440in}{5.934730in}}{\pgfqpoint{8.806253in}{5.926917in}}%
\pgfpathcurveto{\pgfqpoint{8.814067in}{5.919103in}}{\pgfqpoint{8.824666in}{5.914713in}}{\pgfqpoint{8.835716in}{5.914713in}}%
\pgfpathlineto{\pgfqpoint{8.835716in}{5.914713in}}%
\pgfpathclose%
\pgfusepath{stroke,fill}%
\end{pgfscope}%
\begin{pgfscope}%
\pgfpathrectangle{\pgfqpoint{7.622482in}{5.272501in}}{\pgfqpoint{2.177280in}{2.201755in}}%
\pgfusepath{clip}%
\pgfsetbuttcap%
\pgfsetroundjoin%
\definecolor{currentfill}{rgb}{1.000000,0.498039,0.054902}%
\pgfsetfillcolor{currentfill}%
\pgfsetlinewidth{0.481800pt}%
\definecolor{currentstroke}{rgb}{1.000000,1.000000,1.000000}%
\pgfsetstrokecolor{currentstroke}%
\pgfsetdash{}{0pt}%
\pgfpathmoveto{\pgfqpoint{8.767962in}{6.164912in}}%
\pgfpathcurveto{\pgfqpoint{8.779012in}{6.164912in}}{\pgfqpoint{8.789611in}{6.169303in}}{\pgfqpoint{8.797425in}{6.177116in}}%
\pgfpathcurveto{\pgfqpoint{8.805238in}{6.184930in}}{\pgfqpoint{8.809629in}{6.195529in}}{\pgfqpoint{8.809629in}{6.206579in}}%
\pgfpathcurveto{\pgfqpoint{8.809629in}{6.217629in}}{\pgfqpoint{8.805238in}{6.228228in}}{\pgfqpoint{8.797425in}{6.236042in}}%
\pgfpathcurveto{\pgfqpoint{8.789611in}{6.243855in}}{\pgfqpoint{8.779012in}{6.248246in}}{\pgfqpoint{8.767962in}{6.248246in}}%
\pgfpathcurveto{\pgfqpoint{8.756912in}{6.248246in}}{\pgfqpoint{8.746313in}{6.243855in}}{\pgfqpoint{8.738499in}{6.236042in}}%
\pgfpathcurveto{\pgfqpoint{8.730686in}{6.228228in}}{\pgfqpoint{8.726295in}{6.217629in}}{\pgfqpoint{8.726295in}{6.206579in}}%
\pgfpathcurveto{\pgfqpoint{8.726295in}{6.195529in}}{\pgfqpoint{8.730686in}{6.184930in}}{\pgfqpoint{8.738499in}{6.177116in}}%
\pgfpathcurveto{\pgfqpoint{8.746313in}{6.169303in}}{\pgfqpoint{8.756912in}{6.164912in}}{\pgfqpoint{8.767962in}{6.164912in}}%
\pgfpathlineto{\pgfqpoint{8.767962in}{6.164912in}}%
\pgfpathclose%
\pgfusepath{stroke,fill}%
\end{pgfscope}%
\begin{pgfscope}%
\pgfpathrectangle{\pgfqpoint{7.622482in}{5.272501in}}{\pgfqpoint{2.177280in}{2.201755in}}%
\pgfusepath{clip}%
\pgfsetbuttcap%
\pgfsetroundjoin%
\definecolor{currentfill}{rgb}{1.000000,0.498039,0.054902}%
\pgfsetfillcolor{currentfill}%
\pgfsetlinewidth{0.481800pt}%
\definecolor{currentstroke}{rgb}{1.000000,1.000000,1.000000}%
\pgfsetstrokecolor{currentstroke}%
\pgfsetdash{}{0pt}%
\pgfpathmoveto{\pgfqpoint{8.835716in}{6.498512in}}%
\pgfpathcurveto{\pgfqpoint{8.846766in}{6.498512in}}{\pgfqpoint{8.857365in}{6.502902in}}{\pgfqpoint{8.865179in}{6.510715in}}%
\pgfpathcurveto{\pgfqpoint{8.872992in}{6.518529in}}{\pgfqpoint{8.877383in}{6.529128in}}{\pgfqpoint{8.877383in}{6.540178in}}%
\pgfpathcurveto{\pgfqpoint{8.877383in}{6.551228in}}{\pgfqpoint{8.872992in}{6.561827in}}{\pgfqpoint{8.865179in}{6.569641in}}%
\pgfpathcurveto{\pgfqpoint{8.857365in}{6.577455in}}{\pgfqpoint{8.846766in}{6.581845in}}{\pgfqpoint{8.835716in}{6.581845in}}%
\pgfpathcurveto{\pgfqpoint{8.824666in}{6.581845in}}{\pgfqpoint{8.814067in}{6.577455in}}{\pgfqpoint{8.806253in}{6.569641in}}%
\pgfpathcurveto{\pgfqpoint{8.798440in}{6.561827in}}{\pgfqpoint{8.794049in}{6.551228in}}{\pgfqpoint{8.794049in}{6.540178in}}%
\pgfpathcurveto{\pgfqpoint{8.794049in}{6.529128in}}{\pgfqpoint{8.798440in}{6.518529in}}{\pgfqpoint{8.806253in}{6.510715in}}%
\pgfpathcurveto{\pgfqpoint{8.814067in}{6.502902in}}{\pgfqpoint{8.824666in}{6.498512in}}{\pgfqpoint{8.835716in}{6.498512in}}%
\pgfpathlineto{\pgfqpoint{8.835716in}{6.498512in}}%
\pgfpathclose%
\pgfusepath{stroke,fill}%
\end{pgfscope}%
\begin{pgfscope}%
\pgfpathrectangle{\pgfqpoint{7.622482in}{5.272501in}}{\pgfqpoint{2.177280in}{2.201755in}}%
\pgfusepath{clip}%
\pgfsetbuttcap%
\pgfsetroundjoin%
\definecolor{currentfill}{rgb}{1.000000,0.498039,0.054902}%
\pgfsetfillcolor{currentfill}%
\pgfsetlinewidth{0.481800pt}%
\definecolor{currentstroke}{rgb}{1.000000,1.000000,1.000000}%
\pgfsetstrokecolor{currentstroke}%
\pgfsetdash{}{0pt}%
\pgfpathmoveto{\pgfqpoint{8.767962in}{6.248312in}}%
\pgfpathcurveto{\pgfqpoint{8.779012in}{6.248312in}}{\pgfqpoint{8.789611in}{6.252702in}}{\pgfqpoint{8.797425in}{6.260516in}}%
\pgfpathcurveto{\pgfqpoint{8.805238in}{6.268330in}}{\pgfqpoint{8.809629in}{6.278929in}}{\pgfqpoint{8.809629in}{6.289979in}}%
\pgfpathcurveto{\pgfqpoint{8.809629in}{6.301029in}}{\pgfqpoint{8.805238in}{6.311628in}}{\pgfqpoint{8.797425in}{6.319442in}}%
\pgfpathcurveto{\pgfqpoint{8.789611in}{6.327255in}}{\pgfqpoint{8.779012in}{6.331645in}}{\pgfqpoint{8.767962in}{6.331645in}}%
\pgfpathcurveto{\pgfqpoint{8.756912in}{6.331645in}}{\pgfqpoint{8.746313in}{6.327255in}}{\pgfqpoint{8.738499in}{6.319442in}}%
\pgfpathcurveto{\pgfqpoint{8.730686in}{6.311628in}}{\pgfqpoint{8.726295in}{6.301029in}}{\pgfqpoint{8.726295in}{6.289979in}}%
\pgfpathcurveto{\pgfqpoint{8.726295in}{6.278929in}}{\pgfqpoint{8.730686in}{6.268330in}}{\pgfqpoint{8.738499in}{6.260516in}}%
\pgfpathcurveto{\pgfqpoint{8.746313in}{6.252702in}}{\pgfqpoint{8.756912in}{6.248312in}}{\pgfqpoint{8.767962in}{6.248312in}}%
\pgfpathlineto{\pgfqpoint{8.767962in}{6.248312in}}%
\pgfpathclose%
\pgfusepath{stroke,fill}%
\end{pgfscope}%
\begin{pgfscope}%
\pgfpathrectangle{\pgfqpoint{7.622482in}{5.272501in}}{\pgfqpoint{2.177280in}{2.201755in}}%
\pgfusepath{clip}%
\pgfsetbuttcap%
\pgfsetroundjoin%
\definecolor{currentfill}{rgb}{1.000000,0.498039,0.054902}%
\pgfsetfillcolor{currentfill}%
\pgfsetlinewidth{0.481800pt}%
\definecolor{currentstroke}{rgb}{1.000000,1.000000,1.000000}%
\pgfsetstrokecolor{currentstroke}%
\pgfsetdash{}{0pt}%
\pgfpathmoveto{\pgfqpoint{8.632454in}{5.581114in}}%
\pgfpathcurveto{\pgfqpoint{8.643504in}{5.581114in}}{\pgfqpoint{8.654104in}{5.585504in}}{\pgfqpoint{8.661917in}{5.593317in}}%
\pgfpathcurveto{\pgfqpoint{8.669731in}{5.601131in}}{\pgfqpoint{8.674121in}{5.611730in}}{\pgfqpoint{8.674121in}{5.622780in}}%
\pgfpathcurveto{\pgfqpoint{8.674121in}{5.633830in}}{\pgfqpoint{8.669731in}{5.644429in}}{\pgfqpoint{8.661917in}{5.652243in}}%
\pgfpathcurveto{\pgfqpoint{8.654104in}{5.660057in}}{\pgfqpoint{8.643504in}{5.664447in}}{\pgfqpoint{8.632454in}{5.664447in}}%
\pgfpathcurveto{\pgfqpoint{8.621404in}{5.664447in}}{\pgfqpoint{8.610805in}{5.660057in}}{\pgfqpoint{8.602992in}{5.652243in}}%
\pgfpathcurveto{\pgfqpoint{8.595178in}{5.644429in}}{\pgfqpoint{8.590788in}{5.633830in}}{\pgfqpoint{8.590788in}{5.622780in}}%
\pgfpathcurveto{\pgfqpoint{8.590788in}{5.611730in}}{\pgfqpoint{8.595178in}{5.601131in}}{\pgfqpoint{8.602992in}{5.593317in}}%
\pgfpathcurveto{\pgfqpoint{8.610805in}{5.585504in}}{\pgfqpoint{8.621404in}{5.581114in}}{\pgfqpoint{8.632454in}{5.581114in}}%
\pgfpathlineto{\pgfqpoint{8.632454in}{5.581114in}}%
\pgfpathclose%
\pgfusepath{stroke,fill}%
\end{pgfscope}%
\begin{pgfscope}%
\pgfpathrectangle{\pgfqpoint{7.622482in}{5.272501in}}{\pgfqpoint{2.177280in}{2.201755in}}%
\pgfusepath{clip}%
\pgfsetbuttcap%
\pgfsetroundjoin%
\definecolor{currentfill}{rgb}{1.000000,0.498039,0.054902}%
\pgfsetfillcolor{currentfill}%
\pgfsetlinewidth{0.481800pt}%
\definecolor{currentstroke}{rgb}{1.000000,1.000000,1.000000}%
\pgfsetstrokecolor{currentstroke}%
\pgfsetdash{}{0pt}%
\pgfpathmoveto{\pgfqpoint{8.632454in}{6.164912in}}%
\pgfpathcurveto{\pgfqpoint{8.643504in}{6.164912in}}{\pgfqpoint{8.654104in}{6.169303in}}{\pgfqpoint{8.661917in}{6.177116in}}%
\pgfpathcurveto{\pgfqpoint{8.669731in}{6.184930in}}{\pgfqpoint{8.674121in}{6.195529in}}{\pgfqpoint{8.674121in}{6.206579in}}%
\pgfpathcurveto{\pgfqpoint{8.674121in}{6.217629in}}{\pgfqpoint{8.669731in}{6.228228in}}{\pgfqpoint{8.661917in}{6.236042in}}%
\pgfpathcurveto{\pgfqpoint{8.654104in}{6.243855in}}{\pgfqpoint{8.643504in}{6.248246in}}{\pgfqpoint{8.632454in}{6.248246in}}%
\pgfpathcurveto{\pgfqpoint{8.621404in}{6.248246in}}{\pgfqpoint{8.610805in}{6.243855in}}{\pgfqpoint{8.602992in}{6.236042in}}%
\pgfpathcurveto{\pgfqpoint{8.595178in}{6.228228in}}{\pgfqpoint{8.590788in}{6.217629in}}{\pgfqpoint{8.590788in}{6.206579in}}%
\pgfpathcurveto{\pgfqpoint{8.590788in}{6.195529in}}{\pgfqpoint{8.595178in}{6.184930in}}{\pgfqpoint{8.602992in}{6.177116in}}%
\pgfpathcurveto{\pgfqpoint{8.610805in}{6.169303in}}{\pgfqpoint{8.621404in}{6.164912in}}{\pgfqpoint{8.632454in}{6.164912in}}%
\pgfpathlineto{\pgfqpoint{8.632454in}{6.164912in}}%
\pgfpathclose%
\pgfusepath{stroke,fill}%
\end{pgfscope}%
\begin{pgfscope}%
\pgfpathrectangle{\pgfqpoint{7.622482in}{5.272501in}}{\pgfqpoint{2.177280in}{2.201755in}}%
\pgfusepath{clip}%
\pgfsetbuttcap%
\pgfsetroundjoin%
\definecolor{currentfill}{rgb}{1.000000,0.498039,0.054902}%
\pgfsetfillcolor{currentfill}%
\pgfsetlinewidth{0.481800pt}%
\definecolor{currentstroke}{rgb}{1.000000,1.000000,1.000000}%
\pgfsetstrokecolor{currentstroke}%
\pgfsetdash{}{0pt}%
\pgfpathmoveto{\pgfqpoint{8.632454in}{5.747913in}}%
\pgfpathcurveto{\pgfqpoint{8.643504in}{5.747913in}}{\pgfqpoint{8.654104in}{5.752303in}}{\pgfqpoint{8.661917in}{5.760117in}}%
\pgfpathcurveto{\pgfqpoint{8.669731in}{5.767931in}}{\pgfqpoint{8.674121in}{5.778530in}}{\pgfqpoint{8.674121in}{5.789580in}}%
\pgfpathcurveto{\pgfqpoint{8.674121in}{5.800630in}}{\pgfqpoint{8.669731in}{5.811229in}}{\pgfqpoint{8.661917in}{5.819043in}}%
\pgfpathcurveto{\pgfqpoint{8.654104in}{5.826856in}}{\pgfqpoint{8.643504in}{5.831247in}}{\pgfqpoint{8.632454in}{5.831247in}}%
\pgfpathcurveto{\pgfqpoint{8.621404in}{5.831247in}}{\pgfqpoint{8.610805in}{5.826856in}}{\pgfqpoint{8.602992in}{5.819043in}}%
\pgfpathcurveto{\pgfqpoint{8.595178in}{5.811229in}}{\pgfqpoint{8.590788in}{5.800630in}}{\pgfqpoint{8.590788in}{5.789580in}}%
\pgfpathcurveto{\pgfqpoint{8.590788in}{5.778530in}}{\pgfqpoint{8.595178in}{5.767931in}}{\pgfqpoint{8.602992in}{5.760117in}}%
\pgfpathcurveto{\pgfqpoint{8.610805in}{5.752303in}}{\pgfqpoint{8.621404in}{5.747913in}}{\pgfqpoint{8.632454in}{5.747913in}}%
\pgfpathlineto{\pgfqpoint{8.632454in}{5.747913in}}%
\pgfpathclose%
\pgfusepath{stroke,fill}%
\end{pgfscope}%
\begin{pgfscope}%
\pgfpathrectangle{\pgfqpoint{7.622482in}{5.272501in}}{\pgfqpoint{2.177280in}{2.201755in}}%
\pgfusepath{clip}%
\pgfsetbuttcap%
\pgfsetroundjoin%
\definecolor{currentfill}{rgb}{1.000000,0.498039,0.054902}%
\pgfsetfillcolor{currentfill}%
\pgfsetlinewidth{0.481800pt}%
\definecolor{currentstroke}{rgb}{1.000000,1.000000,1.000000}%
\pgfsetstrokecolor{currentstroke}%
\pgfsetdash{}{0pt}%
\pgfpathmoveto{\pgfqpoint{8.564700in}{5.831313in}}%
\pgfpathcurveto{\pgfqpoint{8.575751in}{5.831313in}}{\pgfqpoint{8.586350in}{5.835703in}}{\pgfqpoint{8.594163in}{5.843517in}}%
\pgfpathcurveto{\pgfqpoint{8.601977in}{5.851331in}}{\pgfqpoint{8.606367in}{5.861930in}}{\pgfqpoint{8.606367in}{5.872980in}}%
\pgfpathcurveto{\pgfqpoint{8.606367in}{5.884030in}}{\pgfqpoint{8.601977in}{5.894629in}}{\pgfqpoint{8.594163in}{5.902442in}}%
\pgfpathcurveto{\pgfqpoint{8.586350in}{5.910256in}}{\pgfqpoint{8.575751in}{5.914646in}}{\pgfqpoint{8.564700in}{5.914646in}}%
\pgfpathcurveto{\pgfqpoint{8.553650in}{5.914646in}}{\pgfqpoint{8.543051in}{5.910256in}}{\pgfqpoint{8.535238in}{5.902442in}}%
\pgfpathcurveto{\pgfqpoint{8.527424in}{5.894629in}}{\pgfqpoint{8.523034in}{5.884030in}}{\pgfqpoint{8.523034in}{5.872980in}}%
\pgfpathcurveto{\pgfqpoint{8.523034in}{5.861930in}}{\pgfqpoint{8.527424in}{5.851331in}}{\pgfqpoint{8.535238in}{5.843517in}}%
\pgfpathcurveto{\pgfqpoint{8.543051in}{5.835703in}}{\pgfqpoint{8.553650in}{5.831313in}}{\pgfqpoint{8.564700in}{5.831313in}}%
\pgfpathlineto{\pgfqpoint{8.564700in}{5.831313in}}%
\pgfpathclose%
\pgfusepath{stroke,fill}%
\end{pgfscope}%
\begin{pgfscope}%
\pgfpathrectangle{\pgfqpoint{7.622482in}{5.272501in}}{\pgfqpoint{2.177280in}{2.201755in}}%
\pgfusepath{clip}%
\pgfsetbuttcap%
\pgfsetroundjoin%
\definecolor{currentfill}{rgb}{1.000000,0.498039,0.054902}%
\pgfsetfillcolor{currentfill}%
\pgfsetlinewidth{0.481800pt}%
\definecolor{currentstroke}{rgb}{1.000000,1.000000,1.000000}%
\pgfsetstrokecolor{currentstroke}%
\pgfsetdash{}{0pt}%
\pgfpathmoveto{\pgfqpoint{8.700208in}{6.164912in}}%
\pgfpathcurveto{\pgfqpoint{8.711258in}{6.164912in}}{\pgfqpoint{8.721857in}{6.169303in}}{\pgfqpoint{8.729671in}{6.177116in}}%
\pgfpathcurveto{\pgfqpoint{8.737485in}{6.184930in}}{\pgfqpoint{8.741875in}{6.195529in}}{\pgfqpoint{8.741875in}{6.206579in}}%
\pgfpathcurveto{\pgfqpoint{8.741875in}{6.217629in}}{\pgfqpoint{8.737485in}{6.228228in}}{\pgfqpoint{8.729671in}{6.236042in}}%
\pgfpathcurveto{\pgfqpoint{8.721857in}{6.243855in}}{\pgfqpoint{8.711258in}{6.248246in}}{\pgfqpoint{8.700208in}{6.248246in}}%
\pgfpathcurveto{\pgfqpoint{8.689158in}{6.248246in}}{\pgfqpoint{8.678559in}{6.243855in}}{\pgfqpoint{8.670745in}{6.236042in}}%
\pgfpathcurveto{\pgfqpoint{8.662932in}{6.228228in}}{\pgfqpoint{8.658542in}{6.217629in}}{\pgfqpoint{8.658542in}{6.206579in}}%
\pgfpathcurveto{\pgfqpoint{8.658542in}{6.195529in}}{\pgfqpoint{8.662932in}{6.184930in}}{\pgfqpoint{8.670745in}{6.177116in}}%
\pgfpathcurveto{\pgfqpoint{8.678559in}{6.169303in}}{\pgfqpoint{8.689158in}{6.164912in}}{\pgfqpoint{8.700208in}{6.164912in}}%
\pgfpathlineto{\pgfqpoint{8.700208in}{6.164912in}}%
\pgfpathclose%
\pgfusepath{stroke,fill}%
\end{pgfscope}%
\begin{pgfscope}%
\pgfpathrectangle{\pgfqpoint{7.622482in}{5.272501in}}{\pgfqpoint{2.177280in}{2.201755in}}%
\pgfusepath{clip}%
\pgfsetbuttcap%
\pgfsetroundjoin%
\definecolor{currentfill}{rgb}{1.000000,0.498039,0.054902}%
\pgfsetfillcolor{currentfill}%
\pgfsetlinewidth{0.481800pt}%
\definecolor{currentstroke}{rgb}{1.000000,1.000000,1.000000}%
\pgfsetstrokecolor{currentstroke}%
\pgfsetdash{}{0pt}%
\pgfpathmoveto{\pgfqpoint{8.564700in}{5.831313in}}%
\pgfpathcurveto{\pgfqpoint{8.575751in}{5.831313in}}{\pgfqpoint{8.586350in}{5.835703in}}{\pgfqpoint{8.594163in}{5.843517in}}%
\pgfpathcurveto{\pgfqpoint{8.601977in}{5.851331in}}{\pgfqpoint{8.606367in}{5.861930in}}{\pgfqpoint{8.606367in}{5.872980in}}%
\pgfpathcurveto{\pgfqpoint{8.606367in}{5.884030in}}{\pgfqpoint{8.601977in}{5.894629in}}{\pgfqpoint{8.594163in}{5.902442in}}%
\pgfpathcurveto{\pgfqpoint{8.586350in}{5.910256in}}{\pgfqpoint{8.575751in}{5.914646in}}{\pgfqpoint{8.564700in}{5.914646in}}%
\pgfpathcurveto{\pgfqpoint{8.553650in}{5.914646in}}{\pgfqpoint{8.543051in}{5.910256in}}{\pgfqpoint{8.535238in}{5.902442in}}%
\pgfpathcurveto{\pgfqpoint{8.527424in}{5.894629in}}{\pgfqpoint{8.523034in}{5.884030in}}{\pgfqpoint{8.523034in}{5.872980in}}%
\pgfpathcurveto{\pgfqpoint{8.523034in}{5.861930in}}{\pgfqpoint{8.527424in}{5.851331in}}{\pgfqpoint{8.535238in}{5.843517in}}%
\pgfpathcurveto{\pgfqpoint{8.543051in}{5.835703in}}{\pgfqpoint{8.553650in}{5.831313in}}{\pgfqpoint{8.564700in}{5.831313in}}%
\pgfpathlineto{\pgfqpoint{8.564700in}{5.831313in}}%
\pgfpathclose%
\pgfusepath{stroke,fill}%
\end{pgfscope}%
\begin{pgfscope}%
\pgfpathrectangle{\pgfqpoint{7.622482in}{5.272501in}}{\pgfqpoint{2.177280in}{2.201755in}}%
\pgfusepath{clip}%
\pgfsetbuttcap%
\pgfsetroundjoin%
\definecolor{currentfill}{rgb}{1.000000,0.498039,0.054902}%
\pgfsetfillcolor{currentfill}%
\pgfsetlinewidth{0.481800pt}%
\definecolor{currentstroke}{rgb}{1.000000,1.000000,1.000000}%
\pgfsetstrokecolor{currentstroke}%
\pgfsetdash{}{0pt}%
\pgfpathmoveto{\pgfqpoint{8.429193in}{5.581114in}}%
\pgfpathcurveto{\pgfqpoint{8.440243in}{5.581114in}}{\pgfqpoint{8.450842in}{5.585504in}}{\pgfqpoint{8.458656in}{5.593317in}}%
\pgfpathcurveto{\pgfqpoint{8.466469in}{5.601131in}}{\pgfqpoint{8.470859in}{5.611730in}}{\pgfqpoint{8.470859in}{5.622780in}}%
\pgfpathcurveto{\pgfqpoint{8.470859in}{5.633830in}}{\pgfqpoint{8.466469in}{5.644429in}}{\pgfqpoint{8.458656in}{5.652243in}}%
\pgfpathcurveto{\pgfqpoint{8.450842in}{5.660057in}}{\pgfqpoint{8.440243in}{5.664447in}}{\pgfqpoint{8.429193in}{5.664447in}}%
\pgfpathcurveto{\pgfqpoint{8.418143in}{5.664447in}}{\pgfqpoint{8.407544in}{5.660057in}}{\pgfqpoint{8.399730in}{5.652243in}}%
\pgfpathcurveto{\pgfqpoint{8.391916in}{5.644429in}}{\pgfqpoint{8.387526in}{5.633830in}}{\pgfqpoint{8.387526in}{5.622780in}}%
\pgfpathcurveto{\pgfqpoint{8.387526in}{5.611730in}}{\pgfqpoint{8.391916in}{5.601131in}}{\pgfqpoint{8.399730in}{5.593317in}}%
\pgfpathcurveto{\pgfqpoint{8.407544in}{5.585504in}}{\pgfqpoint{8.418143in}{5.581114in}}{\pgfqpoint{8.429193in}{5.581114in}}%
\pgfpathlineto{\pgfqpoint{8.429193in}{5.581114in}}%
\pgfpathclose%
\pgfusepath{stroke,fill}%
\end{pgfscope}%
\begin{pgfscope}%
\pgfpathrectangle{\pgfqpoint{7.622482in}{5.272501in}}{\pgfqpoint{2.177280in}{2.201755in}}%
\pgfusepath{clip}%
\pgfsetbuttcap%
\pgfsetroundjoin%
\definecolor{currentfill}{rgb}{1.000000,0.498039,0.054902}%
\pgfsetfillcolor{currentfill}%
\pgfsetlinewidth{0.481800pt}%
\definecolor{currentstroke}{rgb}{1.000000,1.000000,1.000000}%
\pgfsetstrokecolor{currentstroke}%
\pgfsetdash{}{0pt}%
\pgfpathmoveto{\pgfqpoint{8.632454in}{5.914713in}}%
\pgfpathcurveto{\pgfqpoint{8.643504in}{5.914713in}}{\pgfqpoint{8.654104in}{5.919103in}}{\pgfqpoint{8.661917in}{5.926917in}}%
\pgfpathcurveto{\pgfqpoint{8.669731in}{5.934730in}}{\pgfqpoint{8.674121in}{5.945329in}}{\pgfqpoint{8.674121in}{5.956379in}}%
\pgfpathcurveto{\pgfqpoint{8.674121in}{5.967430in}}{\pgfqpoint{8.669731in}{5.978029in}}{\pgfqpoint{8.661917in}{5.985842in}}%
\pgfpathcurveto{\pgfqpoint{8.654104in}{5.993656in}}{\pgfqpoint{8.643504in}{5.998046in}}{\pgfqpoint{8.632454in}{5.998046in}}%
\pgfpathcurveto{\pgfqpoint{8.621404in}{5.998046in}}{\pgfqpoint{8.610805in}{5.993656in}}{\pgfqpoint{8.602992in}{5.985842in}}%
\pgfpathcurveto{\pgfqpoint{8.595178in}{5.978029in}}{\pgfqpoint{8.590788in}{5.967430in}}{\pgfqpoint{8.590788in}{5.956379in}}%
\pgfpathcurveto{\pgfqpoint{8.590788in}{5.945329in}}{\pgfqpoint{8.595178in}{5.934730in}}{\pgfqpoint{8.602992in}{5.926917in}}%
\pgfpathcurveto{\pgfqpoint{8.610805in}{5.919103in}}{\pgfqpoint{8.621404in}{5.914713in}}{\pgfqpoint{8.632454in}{5.914713in}}%
\pgfpathlineto{\pgfqpoint{8.632454in}{5.914713in}}%
\pgfpathclose%
\pgfusepath{stroke,fill}%
\end{pgfscope}%
\begin{pgfscope}%
\pgfpathrectangle{\pgfqpoint{7.622482in}{5.272501in}}{\pgfqpoint{2.177280in}{2.201755in}}%
\pgfusepath{clip}%
\pgfsetbuttcap%
\pgfsetroundjoin%
\definecolor{currentfill}{rgb}{1.000000,0.498039,0.054902}%
\pgfsetfillcolor{currentfill}%
\pgfsetlinewidth{0.481800pt}%
\definecolor{currentstroke}{rgb}{1.000000,1.000000,1.000000}%
\pgfsetstrokecolor{currentstroke}%
\pgfsetdash{}{0pt}%
\pgfpathmoveto{\pgfqpoint{8.564700in}{6.164912in}}%
\pgfpathcurveto{\pgfqpoint{8.575751in}{6.164912in}}{\pgfqpoint{8.586350in}{6.169303in}}{\pgfqpoint{8.594163in}{6.177116in}}%
\pgfpathcurveto{\pgfqpoint{8.601977in}{6.184930in}}{\pgfqpoint{8.606367in}{6.195529in}}{\pgfqpoint{8.606367in}{6.206579in}}%
\pgfpathcurveto{\pgfqpoint{8.606367in}{6.217629in}}{\pgfqpoint{8.601977in}{6.228228in}}{\pgfqpoint{8.594163in}{6.236042in}}%
\pgfpathcurveto{\pgfqpoint{8.586350in}{6.243855in}}{\pgfqpoint{8.575751in}{6.248246in}}{\pgfqpoint{8.564700in}{6.248246in}}%
\pgfpathcurveto{\pgfqpoint{8.553650in}{6.248246in}}{\pgfqpoint{8.543051in}{6.243855in}}{\pgfqpoint{8.535238in}{6.236042in}}%
\pgfpathcurveto{\pgfqpoint{8.527424in}{6.228228in}}{\pgfqpoint{8.523034in}{6.217629in}}{\pgfqpoint{8.523034in}{6.206579in}}%
\pgfpathcurveto{\pgfqpoint{8.523034in}{6.195529in}}{\pgfqpoint{8.527424in}{6.184930in}}{\pgfqpoint{8.535238in}{6.177116in}}%
\pgfpathcurveto{\pgfqpoint{8.543051in}{6.169303in}}{\pgfqpoint{8.553650in}{6.164912in}}{\pgfqpoint{8.564700in}{6.164912in}}%
\pgfpathlineto{\pgfqpoint{8.564700in}{6.164912in}}%
\pgfpathclose%
\pgfusepath{stroke,fill}%
\end{pgfscope}%
\begin{pgfscope}%
\pgfpathrectangle{\pgfqpoint{7.622482in}{5.272501in}}{\pgfqpoint{2.177280in}{2.201755in}}%
\pgfusepath{clip}%
\pgfsetbuttcap%
\pgfsetroundjoin%
\definecolor{currentfill}{rgb}{1.000000,0.498039,0.054902}%
\pgfsetfillcolor{currentfill}%
\pgfsetlinewidth{0.481800pt}%
\definecolor{currentstroke}{rgb}{1.000000,1.000000,1.000000}%
\pgfsetstrokecolor{currentstroke}%
\pgfsetdash{}{0pt}%
\pgfpathmoveto{\pgfqpoint{8.632454in}{6.081512in}}%
\pgfpathcurveto{\pgfqpoint{8.643504in}{6.081512in}}{\pgfqpoint{8.654104in}{6.085903in}}{\pgfqpoint{8.661917in}{6.093716in}}%
\pgfpathcurveto{\pgfqpoint{8.669731in}{6.101530in}}{\pgfqpoint{8.674121in}{6.112129in}}{\pgfqpoint{8.674121in}{6.123179in}}%
\pgfpathcurveto{\pgfqpoint{8.674121in}{6.134229in}}{\pgfqpoint{8.669731in}{6.144828in}}{\pgfqpoint{8.661917in}{6.152642in}}%
\pgfpathcurveto{\pgfqpoint{8.654104in}{6.160456in}}{\pgfqpoint{8.643504in}{6.164846in}}{\pgfqpoint{8.632454in}{6.164846in}}%
\pgfpathcurveto{\pgfqpoint{8.621404in}{6.164846in}}{\pgfqpoint{8.610805in}{6.160456in}}{\pgfqpoint{8.602992in}{6.152642in}}%
\pgfpathcurveto{\pgfqpoint{8.595178in}{6.144828in}}{\pgfqpoint{8.590788in}{6.134229in}}{\pgfqpoint{8.590788in}{6.123179in}}%
\pgfpathcurveto{\pgfqpoint{8.590788in}{6.112129in}}{\pgfqpoint{8.595178in}{6.101530in}}{\pgfqpoint{8.602992in}{6.093716in}}%
\pgfpathcurveto{\pgfqpoint{8.610805in}{6.085903in}}{\pgfqpoint{8.621404in}{6.081512in}}{\pgfqpoint{8.632454in}{6.081512in}}%
\pgfpathlineto{\pgfqpoint{8.632454in}{6.081512in}}%
\pgfpathclose%
\pgfusepath{stroke,fill}%
\end{pgfscope}%
\begin{pgfscope}%
\pgfpathrectangle{\pgfqpoint{7.622482in}{5.272501in}}{\pgfqpoint{2.177280in}{2.201755in}}%
\pgfusepath{clip}%
\pgfsetbuttcap%
\pgfsetroundjoin%
\definecolor{currentfill}{rgb}{1.000000,0.498039,0.054902}%
\pgfsetfillcolor{currentfill}%
\pgfsetlinewidth{0.481800pt}%
\definecolor{currentstroke}{rgb}{1.000000,1.000000,1.000000}%
\pgfsetstrokecolor{currentstroke}%
\pgfsetdash{}{0pt}%
\pgfpathmoveto{\pgfqpoint{8.632454in}{6.081512in}}%
\pgfpathcurveto{\pgfqpoint{8.643504in}{6.081512in}}{\pgfqpoint{8.654104in}{6.085903in}}{\pgfqpoint{8.661917in}{6.093716in}}%
\pgfpathcurveto{\pgfqpoint{8.669731in}{6.101530in}}{\pgfqpoint{8.674121in}{6.112129in}}{\pgfqpoint{8.674121in}{6.123179in}}%
\pgfpathcurveto{\pgfqpoint{8.674121in}{6.134229in}}{\pgfqpoint{8.669731in}{6.144828in}}{\pgfqpoint{8.661917in}{6.152642in}}%
\pgfpathcurveto{\pgfqpoint{8.654104in}{6.160456in}}{\pgfqpoint{8.643504in}{6.164846in}}{\pgfqpoint{8.632454in}{6.164846in}}%
\pgfpathcurveto{\pgfqpoint{8.621404in}{6.164846in}}{\pgfqpoint{8.610805in}{6.160456in}}{\pgfqpoint{8.602992in}{6.152642in}}%
\pgfpathcurveto{\pgfqpoint{8.595178in}{6.144828in}}{\pgfqpoint{8.590788in}{6.134229in}}{\pgfqpoint{8.590788in}{6.123179in}}%
\pgfpathcurveto{\pgfqpoint{8.590788in}{6.112129in}}{\pgfqpoint{8.595178in}{6.101530in}}{\pgfqpoint{8.602992in}{6.093716in}}%
\pgfpathcurveto{\pgfqpoint{8.610805in}{6.085903in}}{\pgfqpoint{8.621404in}{6.081512in}}{\pgfqpoint{8.632454in}{6.081512in}}%
\pgfpathlineto{\pgfqpoint{8.632454in}{6.081512in}}%
\pgfpathclose%
\pgfusepath{stroke,fill}%
\end{pgfscope}%
\begin{pgfscope}%
\pgfpathrectangle{\pgfqpoint{7.622482in}{5.272501in}}{\pgfqpoint{2.177280in}{2.201755in}}%
\pgfusepath{clip}%
\pgfsetbuttcap%
\pgfsetroundjoin%
\definecolor{currentfill}{rgb}{1.000000,0.498039,0.054902}%
\pgfsetfillcolor{currentfill}%
\pgfsetlinewidth{0.481800pt}%
\definecolor{currentstroke}{rgb}{1.000000,1.000000,1.000000}%
\pgfsetstrokecolor{currentstroke}%
\pgfsetdash{}{0pt}%
\pgfpathmoveto{\pgfqpoint{8.496947in}{5.747913in}}%
\pgfpathcurveto{\pgfqpoint{8.507997in}{5.747913in}}{\pgfqpoint{8.518596in}{5.752303in}}{\pgfqpoint{8.526409in}{5.760117in}}%
\pgfpathcurveto{\pgfqpoint{8.534223in}{5.767931in}}{\pgfqpoint{8.538613in}{5.778530in}}{\pgfqpoint{8.538613in}{5.789580in}}%
\pgfpathcurveto{\pgfqpoint{8.538613in}{5.800630in}}{\pgfqpoint{8.534223in}{5.811229in}}{\pgfqpoint{8.526409in}{5.819043in}}%
\pgfpathcurveto{\pgfqpoint{8.518596in}{5.826856in}}{\pgfqpoint{8.507997in}{5.831247in}}{\pgfqpoint{8.496947in}{5.831247in}}%
\pgfpathcurveto{\pgfqpoint{8.485896in}{5.831247in}}{\pgfqpoint{8.475297in}{5.826856in}}{\pgfqpoint{8.467484in}{5.819043in}}%
\pgfpathcurveto{\pgfqpoint{8.459670in}{5.811229in}}{\pgfqpoint{8.455280in}{5.800630in}}{\pgfqpoint{8.455280in}{5.789580in}}%
\pgfpathcurveto{\pgfqpoint{8.455280in}{5.778530in}}{\pgfqpoint{8.459670in}{5.767931in}}{\pgfqpoint{8.467484in}{5.760117in}}%
\pgfpathcurveto{\pgfqpoint{8.475297in}{5.752303in}}{\pgfqpoint{8.485896in}{5.747913in}}{\pgfqpoint{8.496947in}{5.747913in}}%
\pgfpathlineto{\pgfqpoint{8.496947in}{5.747913in}}%
\pgfpathclose%
\pgfusepath{stroke,fill}%
\end{pgfscope}%
\begin{pgfscope}%
\pgfpathrectangle{\pgfqpoint{7.622482in}{5.272501in}}{\pgfqpoint{2.177280in}{2.201755in}}%
\pgfusepath{clip}%
\pgfsetbuttcap%
\pgfsetroundjoin%
\definecolor{currentfill}{rgb}{1.000000,0.498039,0.054902}%
\pgfsetfillcolor{currentfill}%
\pgfsetlinewidth{0.481800pt}%
\definecolor{currentstroke}{rgb}{1.000000,1.000000,1.000000}%
\pgfsetstrokecolor{currentstroke}%
\pgfsetdash{}{0pt}%
\pgfpathmoveto{\pgfqpoint{8.632454in}{5.998113in}}%
\pgfpathcurveto{\pgfqpoint{8.643504in}{5.998113in}}{\pgfqpoint{8.654104in}{6.002503in}}{\pgfqpoint{8.661917in}{6.010317in}}%
\pgfpathcurveto{\pgfqpoint{8.669731in}{6.018130in}}{\pgfqpoint{8.674121in}{6.028729in}}{\pgfqpoint{8.674121in}{6.039779in}}%
\pgfpathcurveto{\pgfqpoint{8.674121in}{6.050829in}}{\pgfqpoint{8.669731in}{6.061428in}}{\pgfqpoint{8.661917in}{6.069242in}}%
\pgfpathcurveto{\pgfqpoint{8.654104in}{6.077056in}}{\pgfqpoint{8.643504in}{6.081446in}}{\pgfqpoint{8.632454in}{6.081446in}}%
\pgfpathcurveto{\pgfqpoint{8.621404in}{6.081446in}}{\pgfqpoint{8.610805in}{6.077056in}}{\pgfqpoint{8.602992in}{6.069242in}}%
\pgfpathcurveto{\pgfqpoint{8.595178in}{6.061428in}}{\pgfqpoint{8.590788in}{6.050829in}}{\pgfqpoint{8.590788in}{6.039779in}}%
\pgfpathcurveto{\pgfqpoint{8.590788in}{6.028729in}}{\pgfqpoint{8.595178in}{6.018130in}}{\pgfqpoint{8.602992in}{6.010317in}}%
\pgfpathcurveto{\pgfqpoint{8.610805in}{6.002503in}}{\pgfqpoint{8.621404in}{5.998113in}}{\pgfqpoint{8.632454in}{5.998113in}}%
\pgfpathlineto{\pgfqpoint{8.632454in}{5.998113in}}%
\pgfpathclose%
\pgfusepath{stroke,fill}%
\end{pgfscope}%
\begin{pgfscope}%
\pgfpathrectangle{\pgfqpoint{7.622482in}{5.272501in}}{\pgfqpoint{2.177280in}{2.201755in}}%
\pgfusepath{clip}%
\pgfsetbuttcap%
\pgfsetroundjoin%
\definecolor{currentfill}{rgb}{0.172549,0.627451,0.172549}%
\pgfsetfillcolor{currentfill}%
\pgfsetlinewidth{0.481800pt}%
\definecolor{currentstroke}{rgb}{1.000000,1.000000,1.000000}%
\pgfsetstrokecolor{currentstroke}%
\pgfsetdash{}{0pt}%
\pgfpathmoveto{\pgfqpoint{9.445501in}{6.415112in}}%
\pgfpathcurveto{\pgfqpoint{9.456551in}{6.415112in}}{\pgfqpoint{9.467150in}{6.419502in}}{\pgfqpoint{9.474964in}{6.427316in}}%
\pgfpathcurveto{\pgfqpoint{9.482777in}{6.435129in}}{\pgfqpoint{9.487167in}{6.445728in}}{\pgfqpoint{9.487167in}{6.456778in}}%
\pgfpathcurveto{\pgfqpoint{9.487167in}{6.467828in}}{\pgfqpoint{9.482777in}{6.478428in}}{\pgfqpoint{9.474964in}{6.486241in}}%
\pgfpathcurveto{\pgfqpoint{9.467150in}{6.494055in}}{\pgfqpoint{9.456551in}{6.498445in}}{\pgfqpoint{9.445501in}{6.498445in}}%
\pgfpathcurveto{\pgfqpoint{9.434451in}{6.498445in}}{\pgfqpoint{9.423852in}{6.494055in}}{\pgfqpoint{9.416038in}{6.486241in}}%
\pgfpathcurveto{\pgfqpoint{9.408224in}{6.478428in}}{\pgfqpoint{9.403834in}{6.467828in}}{\pgfqpoint{9.403834in}{6.456778in}}%
\pgfpathcurveto{\pgfqpoint{9.403834in}{6.445728in}}{\pgfqpoint{9.408224in}{6.435129in}}{\pgfqpoint{9.416038in}{6.427316in}}%
\pgfpathcurveto{\pgfqpoint{9.423852in}{6.419502in}}{\pgfqpoint{9.434451in}{6.415112in}}{\pgfqpoint{9.445501in}{6.415112in}}%
\pgfpathlineto{\pgfqpoint{9.445501in}{6.415112in}}%
\pgfpathclose%
\pgfusepath{stroke,fill}%
\end{pgfscope}%
\begin{pgfscope}%
\pgfpathrectangle{\pgfqpoint{7.622482in}{5.272501in}}{\pgfqpoint{2.177280in}{2.201755in}}%
\pgfusepath{clip}%
\pgfsetbuttcap%
\pgfsetroundjoin%
\definecolor{currentfill}{rgb}{0.172549,0.627451,0.172549}%
\pgfsetfillcolor{currentfill}%
\pgfsetlinewidth{0.481800pt}%
\definecolor{currentstroke}{rgb}{1.000000,1.000000,1.000000}%
\pgfsetstrokecolor{currentstroke}%
\pgfsetdash{}{0pt}%
\pgfpathmoveto{\pgfqpoint{9.038978in}{5.914713in}}%
\pgfpathcurveto{\pgfqpoint{9.050028in}{5.914713in}}{\pgfqpoint{9.060627in}{5.919103in}}{\pgfqpoint{9.068440in}{5.926917in}}%
\pgfpathcurveto{\pgfqpoint{9.076254in}{5.934730in}}{\pgfqpoint{9.080644in}{5.945329in}}{\pgfqpoint{9.080644in}{5.956379in}}%
\pgfpathcurveto{\pgfqpoint{9.080644in}{5.967430in}}{\pgfqpoint{9.076254in}{5.978029in}}{\pgfqpoint{9.068440in}{5.985842in}}%
\pgfpathcurveto{\pgfqpoint{9.060627in}{5.993656in}}{\pgfqpoint{9.050028in}{5.998046in}}{\pgfqpoint{9.038978in}{5.998046in}}%
\pgfpathcurveto{\pgfqpoint{9.027927in}{5.998046in}}{\pgfqpoint{9.017328in}{5.993656in}}{\pgfqpoint{9.009515in}{5.985842in}}%
\pgfpathcurveto{\pgfqpoint{9.001701in}{5.978029in}}{\pgfqpoint{8.997311in}{5.967430in}}{\pgfqpoint{8.997311in}{5.956379in}}%
\pgfpathcurveto{\pgfqpoint{8.997311in}{5.945329in}}{\pgfqpoint{9.001701in}{5.934730in}}{\pgfqpoint{9.009515in}{5.926917in}}%
\pgfpathcurveto{\pgfqpoint{9.017328in}{5.919103in}}{\pgfqpoint{9.027927in}{5.914713in}}{\pgfqpoint{9.038978in}{5.914713in}}%
\pgfpathlineto{\pgfqpoint{9.038978in}{5.914713in}}%
\pgfpathclose%
\pgfusepath{stroke,fill}%
\end{pgfscope}%
\begin{pgfscope}%
\pgfpathrectangle{\pgfqpoint{7.622482in}{5.272501in}}{\pgfqpoint{2.177280in}{2.201755in}}%
\pgfusepath{clip}%
\pgfsetbuttcap%
\pgfsetroundjoin%
\definecolor{currentfill}{rgb}{0.172549,0.627451,0.172549}%
\pgfsetfillcolor{currentfill}%
\pgfsetlinewidth{0.481800pt}%
\definecolor{currentstroke}{rgb}{1.000000,1.000000,1.000000}%
\pgfsetstrokecolor{currentstroke}%
\pgfsetdash{}{0pt}%
\pgfpathmoveto{\pgfqpoint{9.174485in}{6.164912in}}%
\pgfpathcurveto{\pgfqpoint{9.185535in}{6.164912in}}{\pgfqpoint{9.196134in}{6.169303in}}{\pgfqpoint{9.203948in}{6.177116in}}%
\pgfpathcurveto{\pgfqpoint{9.211762in}{6.184930in}}{\pgfqpoint{9.216152in}{6.195529in}}{\pgfqpoint{9.216152in}{6.206579in}}%
\pgfpathcurveto{\pgfqpoint{9.216152in}{6.217629in}}{\pgfqpoint{9.211762in}{6.228228in}}{\pgfqpoint{9.203948in}{6.236042in}}%
\pgfpathcurveto{\pgfqpoint{9.196134in}{6.243855in}}{\pgfqpoint{9.185535in}{6.248246in}}{\pgfqpoint{9.174485in}{6.248246in}}%
\pgfpathcurveto{\pgfqpoint{9.163435in}{6.248246in}}{\pgfqpoint{9.152836in}{6.243855in}}{\pgfqpoint{9.145023in}{6.236042in}}%
\pgfpathcurveto{\pgfqpoint{9.137209in}{6.228228in}}{\pgfqpoint{9.132819in}{6.217629in}}{\pgfqpoint{9.132819in}{6.206579in}}%
\pgfpathcurveto{\pgfqpoint{9.132819in}{6.195529in}}{\pgfqpoint{9.137209in}{6.184930in}}{\pgfqpoint{9.145023in}{6.177116in}}%
\pgfpathcurveto{\pgfqpoint{9.152836in}{6.169303in}}{\pgfqpoint{9.163435in}{6.164912in}}{\pgfqpoint{9.174485in}{6.164912in}}%
\pgfpathlineto{\pgfqpoint{9.174485in}{6.164912in}}%
\pgfpathclose%
\pgfusepath{stroke,fill}%
\end{pgfscope}%
\begin{pgfscope}%
\pgfpathrectangle{\pgfqpoint{7.622482in}{5.272501in}}{\pgfqpoint{2.177280in}{2.201755in}}%
\pgfusepath{clip}%
\pgfsetbuttcap%
\pgfsetroundjoin%
\definecolor{currentfill}{rgb}{0.172549,0.627451,0.172549}%
\pgfsetfillcolor{currentfill}%
\pgfsetlinewidth{0.481800pt}%
\definecolor{currentstroke}{rgb}{1.000000,1.000000,1.000000}%
\pgfsetstrokecolor{currentstroke}%
\pgfsetdash{}{0pt}%
\pgfpathmoveto{\pgfqpoint{8.971224in}{6.081512in}}%
\pgfpathcurveto{\pgfqpoint{8.982274in}{6.081512in}}{\pgfqpoint{8.992873in}{6.085903in}}{\pgfqpoint{9.000686in}{6.093716in}}%
\pgfpathcurveto{\pgfqpoint{9.008500in}{6.101530in}}{\pgfqpoint{9.012890in}{6.112129in}}{\pgfqpoint{9.012890in}{6.123179in}}%
\pgfpathcurveto{\pgfqpoint{9.012890in}{6.134229in}}{\pgfqpoint{9.008500in}{6.144828in}}{\pgfqpoint{9.000686in}{6.152642in}}%
\pgfpathcurveto{\pgfqpoint{8.992873in}{6.160456in}}{\pgfqpoint{8.982274in}{6.164846in}}{\pgfqpoint{8.971224in}{6.164846in}}%
\pgfpathcurveto{\pgfqpoint{8.960174in}{6.164846in}}{\pgfqpoint{8.949575in}{6.160456in}}{\pgfqpoint{8.941761in}{6.152642in}}%
\pgfpathcurveto{\pgfqpoint{8.933947in}{6.144828in}}{\pgfqpoint{8.929557in}{6.134229in}}{\pgfqpoint{8.929557in}{6.123179in}}%
\pgfpathcurveto{\pgfqpoint{8.929557in}{6.112129in}}{\pgfqpoint{8.933947in}{6.101530in}}{\pgfqpoint{8.941761in}{6.093716in}}%
\pgfpathcurveto{\pgfqpoint{8.949575in}{6.085903in}}{\pgfqpoint{8.960174in}{6.081512in}}{\pgfqpoint{8.971224in}{6.081512in}}%
\pgfpathlineto{\pgfqpoint{8.971224in}{6.081512in}}%
\pgfpathclose%
\pgfusepath{stroke,fill}%
\end{pgfscope}%
\begin{pgfscope}%
\pgfpathrectangle{\pgfqpoint{7.622482in}{5.272501in}}{\pgfqpoint{2.177280in}{2.201755in}}%
\pgfusepath{clip}%
\pgfsetbuttcap%
\pgfsetroundjoin%
\definecolor{currentfill}{rgb}{0.172549,0.627451,0.172549}%
\pgfsetfillcolor{currentfill}%
\pgfsetlinewidth{0.481800pt}%
\definecolor{currentstroke}{rgb}{1.000000,1.000000,1.000000}%
\pgfsetstrokecolor{currentstroke}%
\pgfsetdash{}{0pt}%
\pgfpathmoveto{\pgfqpoint{9.242239in}{6.164912in}}%
\pgfpathcurveto{\pgfqpoint{9.253289in}{6.164912in}}{\pgfqpoint{9.263888in}{6.169303in}}{\pgfqpoint{9.271702in}{6.177116in}}%
\pgfpathcurveto{\pgfqpoint{9.279516in}{6.184930in}}{\pgfqpoint{9.283906in}{6.195529in}}{\pgfqpoint{9.283906in}{6.206579in}}%
\pgfpathcurveto{\pgfqpoint{9.283906in}{6.217629in}}{\pgfqpoint{9.279516in}{6.228228in}}{\pgfqpoint{9.271702in}{6.236042in}}%
\pgfpathcurveto{\pgfqpoint{9.263888in}{6.243855in}}{\pgfqpoint{9.253289in}{6.248246in}}{\pgfqpoint{9.242239in}{6.248246in}}%
\pgfpathcurveto{\pgfqpoint{9.231189in}{6.248246in}}{\pgfqpoint{9.220590in}{6.243855in}}{\pgfqpoint{9.212776in}{6.236042in}}%
\pgfpathcurveto{\pgfqpoint{9.204963in}{6.228228in}}{\pgfqpoint{9.200573in}{6.217629in}}{\pgfqpoint{9.200573in}{6.206579in}}%
\pgfpathcurveto{\pgfqpoint{9.200573in}{6.195529in}}{\pgfqpoint{9.204963in}{6.184930in}}{\pgfqpoint{9.212776in}{6.177116in}}%
\pgfpathcurveto{\pgfqpoint{9.220590in}{6.169303in}}{\pgfqpoint{9.231189in}{6.164912in}}{\pgfqpoint{9.242239in}{6.164912in}}%
\pgfpathlineto{\pgfqpoint{9.242239in}{6.164912in}}%
\pgfpathclose%
\pgfusepath{stroke,fill}%
\end{pgfscope}%
\begin{pgfscope}%
\pgfpathrectangle{\pgfqpoint{7.622482in}{5.272501in}}{\pgfqpoint{2.177280in}{2.201755in}}%
\pgfusepath{clip}%
\pgfsetbuttcap%
\pgfsetroundjoin%
\definecolor{currentfill}{rgb}{0.172549,0.627451,0.172549}%
\pgfsetfillcolor{currentfill}%
\pgfsetlinewidth{0.481800pt}%
\definecolor{currentstroke}{rgb}{1.000000,1.000000,1.000000}%
\pgfsetstrokecolor{currentstroke}%
\pgfsetdash{}{0pt}%
\pgfpathmoveto{\pgfqpoint{9.174485in}{6.164912in}}%
\pgfpathcurveto{\pgfqpoint{9.185535in}{6.164912in}}{\pgfqpoint{9.196134in}{6.169303in}}{\pgfqpoint{9.203948in}{6.177116in}}%
\pgfpathcurveto{\pgfqpoint{9.211762in}{6.184930in}}{\pgfqpoint{9.216152in}{6.195529in}}{\pgfqpoint{9.216152in}{6.206579in}}%
\pgfpathcurveto{\pgfqpoint{9.216152in}{6.217629in}}{\pgfqpoint{9.211762in}{6.228228in}}{\pgfqpoint{9.203948in}{6.236042in}}%
\pgfpathcurveto{\pgfqpoint{9.196134in}{6.243855in}}{\pgfqpoint{9.185535in}{6.248246in}}{\pgfqpoint{9.174485in}{6.248246in}}%
\pgfpathcurveto{\pgfqpoint{9.163435in}{6.248246in}}{\pgfqpoint{9.152836in}{6.243855in}}{\pgfqpoint{9.145023in}{6.236042in}}%
\pgfpathcurveto{\pgfqpoint{9.137209in}{6.228228in}}{\pgfqpoint{9.132819in}{6.217629in}}{\pgfqpoint{9.132819in}{6.206579in}}%
\pgfpathcurveto{\pgfqpoint{9.132819in}{6.195529in}}{\pgfqpoint{9.137209in}{6.184930in}}{\pgfqpoint{9.145023in}{6.177116in}}%
\pgfpathcurveto{\pgfqpoint{9.152836in}{6.169303in}}{\pgfqpoint{9.163435in}{6.164912in}}{\pgfqpoint{9.174485in}{6.164912in}}%
\pgfpathlineto{\pgfqpoint{9.174485in}{6.164912in}}%
\pgfpathclose%
\pgfusepath{stroke,fill}%
\end{pgfscope}%
\begin{pgfscope}%
\pgfpathrectangle{\pgfqpoint{7.622482in}{5.272501in}}{\pgfqpoint{2.177280in}{2.201755in}}%
\pgfusepath{clip}%
\pgfsetbuttcap%
\pgfsetroundjoin%
\definecolor{currentfill}{rgb}{0.172549,0.627451,0.172549}%
\pgfsetfillcolor{currentfill}%
\pgfsetlinewidth{0.481800pt}%
\definecolor{currentstroke}{rgb}{1.000000,1.000000,1.000000}%
\pgfsetstrokecolor{currentstroke}%
\pgfsetdash{}{0pt}%
\pgfpathmoveto{\pgfqpoint{8.903470in}{5.747913in}}%
\pgfpathcurveto{\pgfqpoint{8.914520in}{5.747913in}}{\pgfqpoint{8.925119in}{5.752303in}}{\pgfqpoint{8.932933in}{5.760117in}}%
\pgfpathcurveto{\pgfqpoint{8.940746in}{5.767931in}}{\pgfqpoint{8.945136in}{5.778530in}}{\pgfqpoint{8.945136in}{5.789580in}}%
\pgfpathcurveto{\pgfqpoint{8.945136in}{5.800630in}}{\pgfqpoint{8.940746in}{5.811229in}}{\pgfqpoint{8.932933in}{5.819043in}}%
\pgfpathcurveto{\pgfqpoint{8.925119in}{5.826856in}}{\pgfqpoint{8.914520in}{5.831247in}}{\pgfqpoint{8.903470in}{5.831247in}}%
\pgfpathcurveto{\pgfqpoint{8.892420in}{5.831247in}}{\pgfqpoint{8.881821in}{5.826856in}}{\pgfqpoint{8.874007in}{5.819043in}}%
\pgfpathcurveto{\pgfqpoint{8.866193in}{5.811229in}}{\pgfqpoint{8.861803in}{5.800630in}}{\pgfqpoint{8.861803in}{5.789580in}}%
\pgfpathcurveto{\pgfqpoint{8.861803in}{5.778530in}}{\pgfqpoint{8.866193in}{5.767931in}}{\pgfqpoint{8.874007in}{5.760117in}}%
\pgfpathcurveto{\pgfqpoint{8.881821in}{5.752303in}}{\pgfqpoint{8.892420in}{5.747913in}}{\pgfqpoint{8.903470in}{5.747913in}}%
\pgfpathlineto{\pgfqpoint{8.903470in}{5.747913in}}%
\pgfpathclose%
\pgfusepath{stroke,fill}%
\end{pgfscope}%
\begin{pgfscope}%
\pgfpathrectangle{\pgfqpoint{7.622482in}{5.272501in}}{\pgfqpoint{2.177280in}{2.201755in}}%
\pgfusepath{clip}%
\pgfsetbuttcap%
\pgfsetroundjoin%
\definecolor{currentfill}{rgb}{0.172549,0.627451,0.172549}%
\pgfsetfillcolor{currentfill}%
\pgfsetlinewidth{0.481800pt}%
\definecolor{currentstroke}{rgb}{1.000000,1.000000,1.000000}%
\pgfsetstrokecolor{currentstroke}%
\pgfsetdash{}{0pt}%
\pgfpathmoveto{\pgfqpoint{8.971224in}{6.081512in}}%
\pgfpathcurveto{\pgfqpoint{8.982274in}{6.081512in}}{\pgfqpoint{8.992873in}{6.085903in}}{\pgfqpoint{9.000686in}{6.093716in}}%
\pgfpathcurveto{\pgfqpoint{9.008500in}{6.101530in}}{\pgfqpoint{9.012890in}{6.112129in}}{\pgfqpoint{9.012890in}{6.123179in}}%
\pgfpathcurveto{\pgfqpoint{9.012890in}{6.134229in}}{\pgfqpoint{9.008500in}{6.144828in}}{\pgfqpoint{9.000686in}{6.152642in}}%
\pgfpathcurveto{\pgfqpoint{8.992873in}{6.160456in}}{\pgfqpoint{8.982274in}{6.164846in}}{\pgfqpoint{8.971224in}{6.164846in}}%
\pgfpathcurveto{\pgfqpoint{8.960174in}{6.164846in}}{\pgfqpoint{8.949575in}{6.160456in}}{\pgfqpoint{8.941761in}{6.152642in}}%
\pgfpathcurveto{\pgfqpoint{8.933947in}{6.144828in}}{\pgfqpoint{8.929557in}{6.134229in}}{\pgfqpoint{8.929557in}{6.123179in}}%
\pgfpathcurveto{\pgfqpoint{8.929557in}{6.112129in}}{\pgfqpoint{8.933947in}{6.101530in}}{\pgfqpoint{8.941761in}{6.093716in}}%
\pgfpathcurveto{\pgfqpoint{8.949575in}{6.085903in}}{\pgfqpoint{8.960174in}{6.081512in}}{\pgfqpoint{8.971224in}{6.081512in}}%
\pgfpathlineto{\pgfqpoint{8.971224in}{6.081512in}}%
\pgfpathclose%
\pgfusepath{stroke,fill}%
\end{pgfscope}%
\begin{pgfscope}%
\pgfpathrectangle{\pgfqpoint{7.622482in}{5.272501in}}{\pgfqpoint{2.177280in}{2.201755in}}%
\pgfusepath{clip}%
\pgfsetbuttcap%
\pgfsetroundjoin%
\definecolor{currentfill}{rgb}{0.172549,0.627451,0.172549}%
\pgfsetfillcolor{currentfill}%
\pgfsetlinewidth{0.481800pt}%
\definecolor{currentstroke}{rgb}{1.000000,1.000000,1.000000}%
\pgfsetstrokecolor{currentstroke}%
\pgfsetdash{}{0pt}%
\pgfpathmoveto{\pgfqpoint{8.971224in}{5.747913in}}%
\pgfpathcurveto{\pgfqpoint{8.982274in}{5.747913in}}{\pgfqpoint{8.992873in}{5.752303in}}{\pgfqpoint{9.000686in}{5.760117in}}%
\pgfpathcurveto{\pgfqpoint{9.008500in}{5.767931in}}{\pgfqpoint{9.012890in}{5.778530in}}{\pgfqpoint{9.012890in}{5.789580in}}%
\pgfpathcurveto{\pgfqpoint{9.012890in}{5.800630in}}{\pgfqpoint{9.008500in}{5.811229in}}{\pgfqpoint{9.000686in}{5.819043in}}%
\pgfpathcurveto{\pgfqpoint{8.992873in}{5.826856in}}{\pgfqpoint{8.982274in}{5.831247in}}{\pgfqpoint{8.971224in}{5.831247in}}%
\pgfpathcurveto{\pgfqpoint{8.960174in}{5.831247in}}{\pgfqpoint{8.949575in}{5.826856in}}{\pgfqpoint{8.941761in}{5.819043in}}%
\pgfpathcurveto{\pgfqpoint{8.933947in}{5.811229in}}{\pgfqpoint{8.929557in}{5.800630in}}{\pgfqpoint{8.929557in}{5.789580in}}%
\pgfpathcurveto{\pgfqpoint{8.929557in}{5.778530in}}{\pgfqpoint{8.933947in}{5.767931in}}{\pgfqpoint{8.941761in}{5.760117in}}%
\pgfpathcurveto{\pgfqpoint{8.949575in}{5.752303in}}{\pgfqpoint{8.960174in}{5.747913in}}{\pgfqpoint{8.971224in}{5.747913in}}%
\pgfpathlineto{\pgfqpoint{8.971224in}{5.747913in}}%
\pgfpathclose%
\pgfusepath{stroke,fill}%
\end{pgfscope}%
\begin{pgfscope}%
\pgfpathrectangle{\pgfqpoint{7.622482in}{5.272501in}}{\pgfqpoint{2.177280in}{2.201755in}}%
\pgfusepath{clip}%
\pgfsetbuttcap%
\pgfsetroundjoin%
\definecolor{currentfill}{rgb}{0.172549,0.627451,0.172549}%
\pgfsetfillcolor{currentfill}%
\pgfsetlinewidth{0.481800pt}%
\definecolor{currentstroke}{rgb}{1.000000,1.000000,1.000000}%
\pgfsetstrokecolor{currentstroke}%
\pgfsetdash{}{0pt}%
\pgfpathmoveto{\pgfqpoint{9.445501in}{6.665311in}}%
\pgfpathcurveto{\pgfqpoint{9.456551in}{6.665311in}}{\pgfqpoint{9.467150in}{6.669701in}}{\pgfqpoint{9.474964in}{6.677515in}}%
\pgfpathcurveto{\pgfqpoint{9.482777in}{6.685329in}}{\pgfqpoint{9.487167in}{6.695928in}}{\pgfqpoint{9.487167in}{6.706978in}}%
\pgfpathcurveto{\pgfqpoint{9.487167in}{6.718028in}}{\pgfqpoint{9.482777in}{6.728627in}}{\pgfqpoint{9.474964in}{6.736441in}}%
\pgfpathcurveto{\pgfqpoint{9.467150in}{6.744254in}}{\pgfqpoint{9.456551in}{6.748644in}}{\pgfqpoint{9.445501in}{6.748644in}}%
\pgfpathcurveto{\pgfqpoint{9.434451in}{6.748644in}}{\pgfqpoint{9.423852in}{6.744254in}}{\pgfqpoint{9.416038in}{6.736441in}}%
\pgfpathcurveto{\pgfqpoint{9.408224in}{6.728627in}}{\pgfqpoint{9.403834in}{6.718028in}}{\pgfqpoint{9.403834in}{6.706978in}}%
\pgfpathcurveto{\pgfqpoint{9.403834in}{6.695928in}}{\pgfqpoint{9.408224in}{6.685329in}}{\pgfqpoint{9.416038in}{6.677515in}}%
\pgfpathcurveto{\pgfqpoint{9.423852in}{6.669701in}}{\pgfqpoint{9.434451in}{6.665311in}}{\pgfqpoint{9.445501in}{6.665311in}}%
\pgfpathlineto{\pgfqpoint{9.445501in}{6.665311in}}%
\pgfpathclose%
\pgfusepath{stroke,fill}%
\end{pgfscope}%
\begin{pgfscope}%
\pgfpathrectangle{\pgfqpoint{7.622482in}{5.272501in}}{\pgfqpoint{2.177280in}{2.201755in}}%
\pgfusepath{clip}%
\pgfsetbuttcap%
\pgfsetroundjoin%
\definecolor{currentfill}{rgb}{0.172549,0.627451,0.172549}%
\pgfsetfillcolor{currentfill}%
\pgfsetlinewidth{0.481800pt}%
\definecolor{currentstroke}{rgb}{1.000000,1.000000,1.000000}%
\pgfsetstrokecolor{currentstroke}%
\pgfsetdash{}{0pt}%
\pgfpathmoveto{\pgfqpoint{9.106731in}{6.331712in}}%
\pgfpathcurveto{\pgfqpoint{9.117782in}{6.331712in}}{\pgfqpoint{9.128381in}{6.336102in}}{\pgfqpoint{9.136194in}{6.343916in}}%
\pgfpathcurveto{\pgfqpoint{9.144008in}{6.351729in}}{\pgfqpoint{9.148398in}{6.362328in}}{\pgfqpoint{9.148398in}{6.373379in}}%
\pgfpathcurveto{\pgfqpoint{9.148398in}{6.384429in}}{\pgfqpoint{9.144008in}{6.395028in}}{\pgfqpoint{9.136194in}{6.402841in}}%
\pgfpathcurveto{\pgfqpoint{9.128381in}{6.410655in}}{\pgfqpoint{9.117782in}{6.415045in}}{\pgfqpoint{9.106731in}{6.415045in}}%
\pgfpathcurveto{\pgfqpoint{9.095681in}{6.415045in}}{\pgfqpoint{9.085082in}{6.410655in}}{\pgfqpoint{9.077269in}{6.402841in}}%
\pgfpathcurveto{\pgfqpoint{9.069455in}{6.395028in}}{\pgfqpoint{9.065065in}{6.384429in}}{\pgfqpoint{9.065065in}{6.373379in}}%
\pgfpathcurveto{\pgfqpoint{9.065065in}{6.362328in}}{\pgfqpoint{9.069455in}{6.351729in}}{\pgfqpoint{9.077269in}{6.343916in}}%
\pgfpathcurveto{\pgfqpoint{9.085082in}{6.336102in}}{\pgfqpoint{9.095681in}{6.331712in}}{\pgfqpoint{9.106731in}{6.331712in}}%
\pgfpathlineto{\pgfqpoint{9.106731in}{6.331712in}}%
\pgfpathclose%
\pgfusepath{stroke,fill}%
\end{pgfscope}%
\begin{pgfscope}%
\pgfpathrectangle{\pgfqpoint{7.622482in}{5.272501in}}{\pgfqpoint{2.177280in}{2.201755in}}%
\pgfusepath{clip}%
\pgfsetbuttcap%
\pgfsetroundjoin%
\definecolor{currentfill}{rgb}{0.172549,0.627451,0.172549}%
\pgfsetfillcolor{currentfill}%
\pgfsetlinewidth{0.481800pt}%
\definecolor{currentstroke}{rgb}{1.000000,1.000000,1.000000}%
\pgfsetstrokecolor{currentstroke}%
\pgfsetdash{}{0pt}%
\pgfpathmoveto{\pgfqpoint{9.038978in}{5.914713in}}%
\pgfpathcurveto{\pgfqpoint{9.050028in}{5.914713in}}{\pgfqpoint{9.060627in}{5.919103in}}{\pgfqpoint{9.068440in}{5.926917in}}%
\pgfpathcurveto{\pgfqpoint{9.076254in}{5.934730in}}{\pgfqpoint{9.080644in}{5.945329in}}{\pgfqpoint{9.080644in}{5.956379in}}%
\pgfpathcurveto{\pgfqpoint{9.080644in}{5.967430in}}{\pgfqpoint{9.076254in}{5.978029in}}{\pgfqpoint{9.068440in}{5.985842in}}%
\pgfpathcurveto{\pgfqpoint{9.060627in}{5.993656in}}{\pgfqpoint{9.050028in}{5.998046in}}{\pgfqpoint{9.038978in}{5.998046in}}%
\pgfpathcurveto{\pgfqpoint{9.027927in}{5.998046in}}{\pgfqpoint{9.017328in}{5.993656in}}{\pgfqpoint{9.009515in}{5.985842in}}%
\pgfpathcurveto{\pgfqpoint{9.001701in}{5.978029in}}{\pgfqpoint{8.997311in}{5.967430in}}{\pgfqpoint{8.997311in}{5.956379in}}%
\pgfpathcurveto{\pgfqpoint{8.997311in}{5.945329in}}{\pgfqpoint{9.001701in}{5.934730in}}{\pgfqpoint{9.009515in}{5.926917in}}%
\pgfpathcurveto{\pgfqpoint{9.017328in}{5.919103in}}{\pgfqpoint{9.027927in}{5.914713in}}{\pgfqpoint{9.038978in}{5.914713in}}%
\pgfpathlineto{\pgfqpoint{9.038978in}{5.914713in}}%
\pgfpathclose%
\pgfusepath{stroke,fill}%
\end{pgfscope}%
\begin{pgfscope}%
\pgfpathrectangle{\pgfqpoint{7.622482in}{5.272501in}}{\pgfqpoint{2.177280in}{2.201755in}}%
\pgfusepath{clip}%
\pgfsetbuttcap%
\pgfsetroundjoin%
\definecolor{currentfill}{rgb}{0.172549,0.627451,0.172549}%
\pgfsetfillcolor{currentfill}%
\pgfsetlinewidth{0.481800pt}%
\definecolor{currentstroke}{rgb}{1.000000,1.000000,1.000000}%
\pgfsetstrokecolor{currentstroke}%
\pgfsetdash{}{0pt}%
\pgfpathmoveto{\pgfqpoint{9.174485in}{6.164912in}}%
\pgfpathcurveto{\pgfqpoint{9.185535in}{6.164912in}}{\pgfqpoint{9.196134in}{6.169303in}}{\pgfqpoint{9.203948in}{6.177116in}}%
\pgfpathcurveto{\pgfqpoint{9.211762in}{6.184930in}}{\pgfqpoint{9.216152in}{6.195529in}}{\pgfqpoint{9.216152in}{6.206579in}}%
\pgfpathcurveto{\pgfqpoint{9.216152in}{6.217629in}}{\pgfqpoint{9.211762in}{6.228228in}}{\pgfqpoint{9.203948in}{6.236042in}}%
\pgfpathcurveto{\pgfqpoint{9.196134in}{6.243855in}}{\pgfqpoint{9.185535in}{6.248246in}}{\pgfqpoint{9.174485in}{6.248246in}}%
\pgfpathcurveto{\pgfqpoint{9.163435in}{6.248246in}}{\pgfqpoint{9.152836in}{6.243855in}}{\pgfqpoint{9.145023in}{6.236042in}}%
\pgfpathcurveto{\pgfqpoint{9.137209in}{6.228228in}}{\pgfqpoint{9.132819in}{6.217629in}}{\pgfqpoint{9.132819in}{6.206579in}}%
\pgfpathcurveto{\pgfqpoint{9.132819in}{6.195529in}}{\pgfqpoint{9.137209in}{6.184930in}}{\pgfqpoint{9.145023in}{6.177116in}}%
\pgfpathcurveto{\pgfqpoint{9.152836in}{6.169303in}}{\pgfqpoint{9.163435in}{6.164912in}}{\pgfqpoint{9.174485in}{6.164912in}}%
\pgfpathlineto{\pgfqpoint{9.174485in}{6.164912in}}%
\pgfpathclose%
\pgfusepath{stroke,fill}%
\end{pgfscope}%
\begin{pgfscope}%
\pgfpathrectangle{\pgfqpoint{7.622482in}{5.272501in}}{\pgfqpoint{2.177280in}{2.201755in}}%
\pgfusepath{clip}%
\pgfsetbuttcap%
\pgfsetroundjoin%
\definecolor{currentfill}{rgb}{0.172549,0.627451,0.172549}%
\pgfsetfillcolor{currentfill}%
\pgfsetlinewidth{0.481800pt}%
\definecolor{currentstroke}{rgb}{1.000000,1.000000,1.000000}%
\pgfsetstrokecolor{currentstroke}%
\pgfsetdash{}{0pt}%
\pgfpathmoveto{\pgfqpoint{9.106731in}{5.747913in}}%
\pgfpathcurveto{\pgfqpoint{9.117782in}{5.747913in}}{\pgfqpoint{9.128381in}{5.752303in}}{\pgfqpoint{9.136194in}{5.760117in}}%
\pgfpathcurveto{\pgfqpoint{9.144008in}{5.767931in}}{\pgfqpoint{9.148398in}{5.778530in}}{\pgfqpoint{9.148398in}{5.789580in}}%
\pgfpathcurveto{\pgfqpoint{9.148398in}{5.800630in}}{\pgfqpoint{9.144008in}{5.811229in}}{\pgfqpoint{9.136194in}{5.819043in}}%
\pgfpathcurveto{\pgfqpoint{9.128381in}{5.826856in}}{\pgfqpoint{9.117782in}{5.831247in}}{\pgfqpoint{9.106731in}{5.831247in}}%
\pgfpathcurveto{\pgfqpoint{9.095681in}{5.831247in}}{\pgfqpoint{9.085082in}{5.826856in}}{\pgfqpoint{9.077269in}{5.819043in}}%
\pgfpathcurveto{\pgfqpoint{9.069455in}{5.811229in}}{\pgfqpoint{9.065065in}{5.800630in}}{\pgfqpoint{9.065065in}{5.789580in}}%
\pgfpathcurveto{\pgfqpoint{9.065065in}{5.778530in}}{\pgfqpoint{9.069455in}{5.767931in}}{\pgfqpoint{9.077269in}{5.760117in}}%
\pgfpathcurveto{\pgfqpoint{9.085082in}{5.752303in}}{\pgfqpoint{9.095681in}{5.747913in}}{\pgfqpoint{9.106731in}{5.747913in}}%
\pgfpathlineto{\pgfqpoint{9.106731in}{5.747913in}}%
\pgfpathclose%
\pgfusepath{stroke,fill}%
\end{pgfscope}%
\begin{pgfscope}%
\pgfpathrectangle{\pgfqpoint{7.622482in}{5.272501in}}{\pgfqpoint{2.177280in}{2.201755in}}%
\pgfusepath{clip}%
\pgfsetbuttcap%
\pgfsetroundjoin%
\definecolor{currentfill}{rgb}{0.172549,0.627451,0.172549}%
\pgfsetfillcolor{currentfill}%
\pgfsetlinewidth{0.481800pt}%
\definecolor{currentstroke}{rgb}{1.000000,1.000000,1.000000}%
\pgfsetstrokecolor{currentstroke}%
\pgfsetdash{}{0pt}%
\pgfpathmoveto{\pgfqpoint{9.377747in}{5.998113in}}%
\pgfpathcurveto{\pgfqpoint{9.388797in}{5.998113in}}{\pgfqpoint{9.399396in}{6.002503in}}{\pgfqpoint{9.407210in}{6.010317in}}%
\pgfpathcurveto{\pgfqpoint{9.415023in}{6.018130in}}{\pgfqpoint{9.419414in}{6.028729in}}{\pgfqpoint{9.419414in}{6.039779in}}%
\pgfpathcurveto{\pgfqpoint{9.419414in}{6.050829in}}{\pgfqpoint{9.415023in}{6.061428in}}{\pgfqpoint{9.407210in}{6.069242in}}%
\pgfpathcurveto{\pgfqpoint{9.399396in}{6.077056in}}{\pgfqpoint{9.388797in}{6.081446in}}{\pgfqpoint{9.377747in}{6.081446in}}%
\pgfpathcurveto{\pgfqpoint{9.366697in}{6.081446in}}{\pgfqpoint{9.356098in}{6.077056in}}{\pgfqpoint{9.348284in}{6.069242in}}%
\pgfpathcurveto{\pgfqpoint{9.340471in}{6.061428in}}{\pgfqpoint{9.336080in}{6.050829in}}{\pgfqpoint{9.336080in}{6.039779in}}%
\pgfpathcurveto{\pgfqpoint{9.336080in}{6.028729in}}{\pgfqpoint{9.340471in}{6.018130in}}{\pgfqpoint{9.348284in}{6.010317in}}%
\pgfpathcurveto{\pgfqpoint{9.356098in}{6.002503in}}{\pgfqpoint{9.366697in}{5.998113in}}{\pgfqpoint{9.377747in}{5.998113in}}%
\pgfpathlineto{\pgfqpoint{9.377747in}{5.998113in}}%
\pgfpathclose%
\pgfusepath{stroke,fill}%
\end{pgfscope}%
\begin{pgfscope}%
\pgfpathrectangle{\pgfqpoint{7.622482in}{5.272501in}}{\pgfqpoint{2.177280in}{2.201755in}}%
\pgfusepath{clip}%
\pgfsetbuttcap%
\pgfsetroundjoin%
\definecolor{currentfill}{rgb}{0.172549,0.627451,0.172549}%
\pgfsetfillcolor{currentfill}%
\pgfsetlinewidth{0.481800pt}%
\definecolor{currentstroke}{rgb}{1.000000,1.000000,1.000000}%
\pgfsetstrokecolor{currentstroke}%
\pgfsetdash{}{0pt}%
\pgfpathmoveto{\pgfqpoint{9.309993in}{6.331712in}}%
\pgfpathcurveto{\pgfqpoint{9.321043in}{6.331712in}}{\pgfqpoint{9.331642in}{6.336102in}}{\pgfqpoint{9.339456in}{6.343916in}}%
\pgfpathcurveto{\pgfqpoint{9.347269in}{6.351729in}}{\pgfqpoint{9.351660in}{6.362328in}}{\pgfqpoint{9.351660in}{6.373379in}}%
\pgfpathcurveto{\pgfqpoint{9.351660in}{6.384429in}}{\pgfqpoint{9.347269in}{6.395028in}}{\pgfqpoint{9.339456in}{6.402841in}}%
\pgfpathcurveto{\pgfqpoint{9.331642in}{6.410655in}}{\pgfqpoint{9.321043in}{6.415045in}}{\pgfqpoint{9.309993in}{6.415045in}}%
\pgfpathcurveto{\pgfqpoint{9.298943in}{6.415045in}}{\pgfqpoint{9.288344in}{6.410655in}}{\pgfqpoint{9.280530in}{6.402841in}}%
\pgfpathcurveto{\pgfqpoint{9.272717in}{6.395028in}}{\pgfqpoint{9.268326in}{6.384429in}}{\pgfqpoint{9.268326in}{6.373379in}}%
\pgfpathcurveto{\pgfqpoint{9.268326in}{6.362328in}}{\pgfqpoint{9.272717in}{6.351729in}}{\pgfqpoint{9.280530in}{6.343916in}}%
\pgfpathcurveto{\pgfqpoint{9.288344in}{6.336102in}}{\pgfqpoint{9.298943in}{6.331712in}}{\pgfqpoint{9.309993in}{6.331712in}}%
\pgfpathlineto{\pgfqpoint{9.309993in}{6.331712in}}%
\pgfpathclose%
\pgfusepath{stroke,fill}%
\end{pgfscope}%
\begin{pgfscope}%
\pgfpathrectangle{\pgfqpoint{7.622482in}{5.272501in}}{\pgfqpoint{2.177280in}{2.201755in}}%
\pgfusepath{clip}%
\pgfsetbuttcap%
\pgfsetroundjoin%
\definecolor{currentfill}{rgb}{0.172549,0.627451,0.172549}%
\pgfsetfillcolor{currentfill}%
\pgfsetlinewidth{0.481800pt}%
\definecolor{currentstroke}{rgb}{1.000000,1.000000,1.000000}%
\pgfsetstrokecolor{currentstroke}%
\pgfsetdash{}{0pt}%
\pgfpathmoveto{\pgfqpoint{8.971224in}{6.164912in}}%
\pgfpathcurveto{\pgfqpoint{8.982274in}{6.164912in}}{\pgfqpoint{8.992873in}{6.169303in}}{\pgfqpoint{9.000686in}{6.177116in}}%
\pgfpathcurveto{\pgfqpoint{9.008500in}{6.184930in}}{\pgfqpoint{9.012890in}{6.195529in}}{\pgfqpoint{9.012890in}{6.206579in}}%
\pgfpathcurveto{\pgfqpoint{9.012890in}{6.217629in}}{\pgfqpoint{9.008500in}{6.228228in}}{\pgfqpoint{9.000686in}{6.236042in}}%
\pgfpathcurveto{\pgfqpoint{8.992873in}{6.243855in}}{\pgfqpoint{8.982274in}{6.248246in}}{\pgfqpoint{8.971224in}{6.248246in}}%
\pgfpathcurveto{\pgfqpoint{8.960174in}{6.248246in}}{\pgfqpoint{8.949575in}{6.243855in}}{\pgfqpoint{8.941761in}{6.236042in}}%
\pgfpathcurveto{\pgfqpoint{8.933947in}{6.228228in}}{\pgfqpoint{8.929557in}{6.217629in}}{\pgfqpoint{8.929557in}{6.206579in}}%
\pgfpathcurveto{\pgfqpoint{8.929557in}{6.195529in}}{\pgfqpoint{8.933947in}{6.184930in}}{\pgfqpoint{8.941761in}{6.177116in}}%
\pgfpathcurveto{\pgfqpoint{8.949575in}{6.169303in}}{\pgfqpoint{8.960174in}{6.164912in}}{\pgfqpoint{8.971224in}{6.164912in}}%
\pgfpathlineto{\pgfqpoint{8.971224in}{6.164912in}}%
\pgfpathclose%
\pgfusepath{stroke,fill}%
\end{pgfscope}%
\begin{pgfscope}%
\pgfpathrectangle{\pgfqpoint{7.622482in}{5.272501in}}{\pgfqpoint{2.177280in}{2.201755in}}%
\pgfusepath{clip}%
\pgfsetbuttcap%
\pgfsetroundjoin%
\definecolor{currentfill}{rgb}{0.172549,0.627451,0.172549}%
\pgfsetfillcolor{currentfill}%
\pgfsetlinewidth{0.481800pt}%
\definecolor{currentstroke}{rgb}{1.000000,1.000000,1.000000}%
\pgfsetstrokecolor{currentstroke}%
\pgfsetdash{}{0pt}%
\pgfpathmoveto{\pgfqpoint{9.242239in}{6.832111in}}%
\pgfpathcurveto{\pgfqpoint{9.253289in}{6.832111in}}{\pgfqpoint{9.263888in}{6.836501in}}{\pgfqpoint{9.271702in}{6.844315in}}%
\pgfpathcurveto{\pgfqpoint{9.279516in}{6.852128in}}{\pgfqpoint{9.283906in}{6.862727in}}{\pgfqpoint{9.283906in}{6.873777in}}%
\pgfpathcurveto{\pgfqpoint{9.283906in}{6.884828in}}{\pgfqpoint{9.279516in}{6.895427in}}{\pgfqpoint{9.271702in}{6.903240in}}%
\pgfpathcurveto{\pgfqpoint{9.263888in}{6.911054in}}{\pgfqpoint{9.253289in}{6.915444in}}{\pgfqpoint{9.242239in}{6.915444in}}%
\pgfpathcurveto{\pgfqpoint{9.231189in}{6.915444in}}{\pgfqpoint{9.220590in}{6.911054in}}{\pgfqpoint{9.212776in}{6.903240in}}%
\pgfpathcurveto{\pgfqpoint{9.204963in}{6.895427in}}{\pgfqpoint{9.200573in}{6.884828in}}{\pgfqpoint{9.200573in}{6.873777in}}%
\pgfpathcurveto{\pgfqpoint{9.200573in}{6.862727in}}{\pgfqpoint{9.204963in}{6.852128in}}{\pgfqpoint{9.212776in}{6.844315in}}%
\pgfpathcurveto{\pgfqpoint{9.220590in}{6.836501in}}{\pgfqpoint{9.231189in}{6.832111in}}{\pgfqpoint{9.242239in}{6.832111in}}%
\pgfpathlineto{\pgfqpoint{9.242239in}{6.832111in}}%
\pgfpathclose%
\pgfusepath{stroke,fill}%
\end{pgfscope}%
\begin{pgfscope}%
\pgfpathrectangle{\pgfqpoint{7.622482in}{5.272501in}}{\pgfqpoint{2.177280in}{2.201755in}}%
\pgfusepath{clip}%
\pgfsetbuttcap%
\pgfsetroundjoin%
\definecolor{currentfill}{rgb}{0.172549,0.627451,0.172549}%
\pgfsetfillcolor{currentfill}%
\pgfsetlinewidth{0.481800pt}%
\definecolor{currentstroke}{rgb}{1.000000,1.000000,1.000000}%
\pgfsetstrokecolor{currentstroke}%
\pgfsetdash{}{0pt}%
\pgfpathmoveto{\pgfqpoint{9.309993in}{5.831313in}}%
\pgfpathcurveto{\pgfqpoint{9.321043in}{5.831313in}}{\pgfqpoint{9.331642in}{5.835703in}}{\pgfqpoint{9.339456in}{5.843517in}}%
\pgfpathcurveto{\pgfqpoint{9.347269in}{5.851331in}}{\pgfqpoint{9.351660in}{5.861930in}}{\pgfqpoint{9.351660in}{5.872980in}}%
\pgfpathcurveto{\pgfqpoint{9.351660in}{5.884030in}}{\pgfqpoint{9.347269in}{5.894629in}}{\pgfqpoint{9.339456in}{5.902442in}}%
\pgfpathcurveto{\pgfqpoint{9.331642in}{5.910256in}}{\pgfqpoint{9.321043in}{5.914646in}}{\pgfqpoint{9.309993in}{5.914646in}}%
\pgfpathcurveto{\pgfqpoint{9.298943in}{5.914646in}}{\pgfqpoint{9.288344in}{5.910256in}}{\pgfqpoint{9.280530in}{5.902442in}}%
\pgfpathcurveto{\pgfqpoint{9.272717in}{5.894629in}}{\pgfqpoint{9.268326in}{5.884030in}}{\pgfqpoint{9.268326in}{5.872980in}}%
\pgfpathcurveto{\pgfqpoint{9.268326in}{5.861930in}}{\pgfqpoint{9.272717in}{5.851331in}}{\pgfqpoint{9.280530in}{5.843517in}}%
\pgfpathcurveto{\pgfqpoint{9.288344in}{5.835703in}}{\pgfqpoint{9.298943in}{5.831313in}}{\pgfqpoint{9.309993in}{5.831313in}}%
\pgfpathlineto{\pgfqpoint{9.309993in}{5.831313in}}%
\pgfpathclose%
\pgfusepath{stroke,fill}%
\end{pgfscope}%
\begin{pgfscope}%
\pgfpathrectangle{\pgfqpoint{7.622482in}{5.272501in}}{\pgfqpoint{2.177280in}{2.201755in}}%
\pgfusepath{clip}%
\pgfsetbuttcap%
\pgfsetroundjoin%
\definecolor{currentfill}{rgb}{0.172549,0.627451,0.172549}%
\pgfsetfillcolor{currentfill}%
\pgfsetlinewidth{0.481800pt}%
\definecolor{currentstroke}{rgb}{1.000000,1.000000,1.000000}%
\pgfsetstrokecolor{currentstroke}%
\pgfsetdash{}{0pt}%
\pgfpathmoveto{\pgfqpoint{8.767962in}{5.497714in}}%
\pgfpathcurveto{\pgfqpoint{8.779012in}{5.497714in}}{\pgfqpoint{8.789611in}{5.502104in}}{\pgfqpoint{8.797425in}{5.509918in}}%
\pgfpathcurveto{\pgfqpoint{8.805238in}{5.517731in}}{\pgfqpoint{8.809629in}{5.528330in}}{\pgfqpoint{8.809629in}{5.539380in}}%
\pgfpathcurveto{\pgfqpoint{8.809629in}{5.550431in}}{\pgfqpoint{8.805238in}{5.561030in}}{\pgfqpoint{8.797425in}{5.568843in}}%
\pgfpathcurveto{\pgfqpoint{8.789611in}{5.576657in}}{\pgfqpoint{8.779012in}{5.581047in}}{\pgfqpoint{8.767962in}{5.581047in}}%
\pgfpathcurveto{\pgfqpoint{8.756912in}{5.581047in}}{\pgfqpoint{8.746313in}{5.576657in}}{\pgfqpoint{8.738499in}{5.568843in}}%
\pgfpathcurveto{\pgfqpoint{8.730686in}{5.561030in}}{\pgfqpoint{8.726295in}{5.550431in}}{\pgfqpoint{8.726295in}{5.539380in}}%
\pgfpathcurveto{\pgfqpoint{8.726295in}{5.528330in}}{\pgfqpoint{8.730686in}{5.517731in}}{\pgfqpoint{8.738499in}{5.509918in}}%
\pgfpathcurveto{\pgfqpoint{8.746313in}{5.502104in}}{\pgfqpoint{8.756912in}{5.497714in}}{\pgfqpoint{8.767962in}{5.497714in}}%
\pgfpathlineto{\pgfqpoint{8.767962in}{5.497714in}}%
\pgfpathclose%
\pgfusepath{stroke,fill}%
\end{pgfscope}%
\begin{pgfscope}%
\pgfpathrectangle{\pgfqpoint{7.622482in}{5.272501in}}{\pgfqpoint{2.177280in}{2.201755in}}%
\pgfusepath{clip}%
\pgfsetbuttcap%
\pgfsetroundjoin%
\definecolor{currentfill}{rgb}{0.172549,0.627451,0.172549}%
\pgfsetfillcolor{currentfill}%
\pgfsetlinewidth{0.481800pt}%
\definecolor{currentstroke}{rgb}{1.000000,1.000000,1.000000}%
\pgfsetstrokecolor{currentstroke}%
\pgfsetdash{}{0pt}%
\pgfpathmoveto{\pgfqpoint{9.309993in}{6.331712in}}%
\pgfpathcurveto{\pgfqpoint{9.321043in}{6.331712in}}{\pgfqpoint{9.331642in}{6.336102in}}{\pgfqpoint{9.339456in}{6.343916in}}%
\pgfpathcurveto{\pgfqpoint{9.347269in}{6.351729in}}{\pgfqpoint{9.351660in}{6.362328in}}{\pgfqpoint{9.351660in}{6.373379in}}%
\pgfpathcurveto{\pgfqpoint{9.351660in}{6.384429in}}{\pgfqpoint{9.347269in}{6.395028in}}{\pgfqpoint{9.339456in}{6.402841in}}%
\pgfpathcurveto{\pgfqpoint{9.331642in}{6.410655in}}{\pgfqpoint{9.321043in}{6.415045in}}{\pgfqpoint{9.309993in}{6.415045in}}%
\pgfpathcurveto{\pgfqpoint{9.298943in}{6.415045in}}{\pgfqpoint{9.288344in}{6.410655in}}{\pgfqpoint{9.280530in}{6.402841in}}%
\pgfpathcurveto{\pgfqpoint{9.272717in}{6.395028in}}{\pgfqpoint{9.268326in}{6.384429in}}{\pgfqpoint{9.268326in}{6.373379in}}%
\pgfpathcurveto{\pgfqpoint{9.268326in}{6.362328in}}{\pgfqpoint{9.272717in}{6.351729in}}{\pgfqpoint{9.280530in}{6.343916in}}%
\pgfpathcurveto{\pgfqpoint{9.288344in}{6.336102in}}{\pgfqpoint{9.298943in}{6.331712in}}{\pgfqpoint{9.309993in}{6.331712in}}%
\pgfpathlineto{\pgfqpoint{9.309993in}{6.331712in}}%
\pgfpathclose%
\pgfusepath{stroke,fill}%
\end{pgfscope}%
\begin{pgfscope}%
\pgfpathrectangle{\pgfqpoint{7.622482in}{5.272501in}}{\pgfqpoint{2.177280in}{2.201755in}}%
\pgfusepath{clip}%
\pgfsetbuttcap%
\pgfsetroundjoin%
\definecolor{currentfill}{rgb}{0.172549,0.627451,0.172549}%
\pgfsetfillcolor{currentfill}%
\pgfsetlinewidth{0.481800pt}%
\definecolor{currentstroke}{rgb}{1.000000,1.000000,1.000000}%
\pgfsetstrokecolor{currentstroke}%
\pgfsetdash{}{0pt}%
\pgfpathmoveto{\pgfqpoint{9.106731in}{5.998113in}}%
\pgfpathcurveto{\pgfqpoint{9.117782in}{5.998113in}}{\pgfqpoint{9.128381in}{6.002503in}}{\pgfqpoint{9.136194in}{6.010317in}}%
\pgfpathcurveto{\pgfqpoint{9.144008in}{6.018130in}}{\pgfqpoint{9.148398in}{6.028729in}}{\pgfqpoint{9.148398in}{6.039779in}}%
\pgfpathcurveto{\pgfqpoint{9.148398in}{6.050829in}}{\pgfqpoint{9.144008in}{6.061428in}}{\pgfqpoint{9.136194in}{6.069242in}}%
\pgfpathcurveto{\pgfqpoint{9.128381in}{6.077056in}}{\pgfqpoint{9.117782in}{6.081446in}}{\pgfqpoint{9.106731in}{6.081446in}}%
\pgfpathcurveto{\pgfqpoint{9.095681in}{6.081446in}}{\pgfqpoint{9.085082in}{6.077056in}}{\pgfqpoint{9.077269in}{6.069242in}}%
\pgfpathcurveto{\pgfqpoint{9.069455in}{6.061428in}}{\pgfqpoint{9.065065in}{6.050829in}}{\pgfqpoint{9.065065in}{6.039779in}}%
\pgfpathcurveto{\pgfqpoint{9.065065in}{6.028729in}}{\pgfqpoint{9.069455in}{6.018130in}}{\pgfqpoint{9.077269in}{6.010317in}}%
\pgfpathcurveto{\pgfqpoint{9.085082in}{6.002503in}}{\pgfqpoint{9.095681in}{5.998113in}}{\pgfqpoint{9.106731in}{5.998113in}}%
\pgfpathlineto{\pgfqpoint{9.106731in}{5.998113in}}%
\pgfpathclose%
\pgfusepath{stroke,fill}%
\end{pgfscope}%
\begin{pgfscope}%
\pgfpathrectangle{\pgfqpoint{7.622482in}{5.272501in}}{\pgfqpoint{2.177280in}{2.201755in}}%
\pgfusepath{clip}%
\pgfsetbuttcap%
\pgfsetroundjoin%
\definecolor{currentfill}{rgb}{0.172549,0.627451,0.172549}%
\pgfsetfillcolor{currentfill}%
\pgfsetlinewidth{0.481800pt}%
\definecolor{currentstroke}{rgb}{1.000000,1.000000,1.000000}%
\pgfsetstrokecolor{currentstroke}%
\pgfsetdash{}{0pt}%
\pgfpathmoveto{\pgfqpoint{9.106731in}{5.998113in}}%
\pgfpathcurveto{\pgfqpoint{9.117782in}{5.998113in}}{\pgfqpoint{9.128381in}{6.002503in}}{\pgfqpoint{9.136194in}{6.010317in}}%
\pgfpathcurveto{\pgfqpoint{9.144008in}{6.018130in}}{\pgfqpoint{9.148398in}{6.028729in}}{\pgfqpoint{9.148398in}{6.039779in}}%
\pgfpathcurveto{\pgfqpoint{9.148398in}{6.050829in}}{\pgfqpoint{9.144008in}{6.061428in}}{\pgfqpoint{9.136194in}{6.069242in}}%
\pgfpathcurveto{\pgfqpoint{9.128381in}{6.077056in}}{\pgfqpoint{9.117782in}{6.081446in}}{\pgfqpoint{9.106731in}{6.081446in}}%
\pgfpathcurveto{\pgfqpoint{9.095681in}{6.081446in}}{\pgfqpoint{9.085082in}{6.077056in}}{\pgfqpoint{9.077269in}{6.069242in}}%
\pgfpathcurveto{\pgfqpoint{9.069455in}{6.061428in}}{\pgfqpoint{9.065065in}{6.050829in}}{\pgfqpoint{9.065065in}{6.039779in}}%
\pgfpathcurveto{\pgfqpoint{9.065065in}{6.028729in}}{\pgfqpoint{9.069455in}{6.018130in}}{\pgfqpoint{9.077269in}{6.010317in}}%
\pgfpathcurveto{\pgfqpoint{9.085082in}{6.002503in}}{\pgfqpoint{9.095681in}{5.998113in}}{\pgfqpoint{9.106731in}{5.998113in}}%
\pgfpathlineto{\pgfqpoint{9.106731in}{5.998113in}}%
\pgfpathclose%
\pgfusepath{stroke,fill}%
\end{pgfscope}%
\begin{pgfscope}%
\pgfpathrectangle{\pgfqpoint{7.622482in}{5.272501in}}{\pgfqpoint{2.177280in}{2.201755in}}%
\pgfusepath{clip}%
\pgfsetbuttcap%
\pgfsetroundjoin%
\definecolor{currentfill}{rgb}{0.172549,0.627451,0.172549}%
\pgfsetfillcolor{currentfill}%
\pgfsetlinewidth{0.481800pt}%
\definecolor{currentstroke}{rgb}{1.000000,1.000000,1.000000}%
\pgfsetstrokecolor{currentstroke}%
\pgfsetdash{}{0pt}%
\pgfpathmoveto{\pgfqpoint{8.971224in}{5.914713in}}%
\pgfpathcurveto{\pgfqpoint{8.982274in}{5.914713in}}{\pgfqpoint{8.992873in}{5.919103in}}{\pgfqpoint{9.000686in}{5.926917in}}%
\pgfpathcurveto{\pgfqpoint{9.008500in}{5.934730in}}{\pgfqpoint{9.012890in}{5.945329in}}{\pgfqpoint{9.012890in}{5.956379in}}%
\pgfpathcurveto{\pgfqpoint{9.012890in}{5.967430in}}{\pgfqpoint{9.008500in}{5.978029in}}{\pgfqpoint{9.000686in}{5.985842in}}%
\pgfpathcurveto{\pgfqpoint{8.992873in}{5.993656in}}{\pgfqpoint{8.982274in}{5.998046in}}{\pgfqpoint{8.971224in}{5.998046in}}%
\pgfpathcurveto{\pgfqpoint{8.960174in}{5.998046in}}{\pgfqpoint{8.949575in}{5.993656in}}{\pgfqpoint{8.941761in}{5.985842in}}%
\pgfpathcurveto{\pgfqpoint{8.933947in}{5.978029in}}{\pgfqpoint{8.929557in}{5.967430in}}{\pgfqpoint{8.929557in}{5.956379in}}%
\pgfpathcurveto{\pgfqpoint{8.929557in}{5.945329in}}{\pgfqpoint{8.933947in}{5.934730in}}{\pgfqpoint{8.941761in}{5.926917in}}%
\pgfpathcurveto{\pgfqpoint{8.949575in}{5.919103in}}{\pgfqpoint{8.960174in}{5.914713in}}{\pgfqpoint{8.971224in}{5.914713in}}%
\pgfpathlineto{\pgfqpoint{8.971224in}{5.914713in}}%
\pgfpathclose%
\pgfusepath{stroke,fill}%
\end{pgfscope}%
\begin{pgfscope}%
\pgfpathrectangle{\pgfqpoint{7.622482in}{5.272501in}}{\pgfqpoint{2.177280in}{2.201755in}}%
\pgfusepath{clip}%
\pgfsetbuttcap%
\pgfsetroundjoin%
\definecolor{currentfill}{rgb}{0.172549,0.627451,0.172549}%
\pgfsetfillcolor{currentfill}%
\pgfsetlinewidth{0.481800pt}%
\definecolor{currentstroke}{rgb}{1.000000,1.000000,1.000000}%
\pgfsetstrokecolor{currentstroke}%
\pgfsetdash{}{0pt}%
\pgfpathmoveto{\pgfqpoint{9.174485in}{6.415112in}}%
\pgfpathcurveto{\pgfqpoint{9.185535in}{6.415112in}}{\pgfqpoint{9.196134in}{6.419502in}}{\pgfqpoint{9.203948in}{6.427316in}}%
\pgfpathcurveto{\pgfqpoint{9.211762in}{6.435129in}}{\pgfqpoint{9.216152in}{6.445728in}}{\pgfqpoint{9.216152in}{6.456778in}}%
\pgfpathcurveto{\pgfqpoint{9.216152in}{6.467828in}}{\pgfqpoint{9.211762in}{6.478428in}}{\pgfqpoint{9.203948in}{6.486241in}}%
\pgfpathcurveto{\pgfqpoint{9.196134in}{6.494055in}}{\pgfqpoint{9.185535in}{6.498445in}}{\pgfqpoint{9.174485in}{6.498445in}}%
\pgfpathcurveto{\pgfqpoint{9.163435in}{6.498445in}}{\pgfqpoint{9.152836in}{6.494055in}}{\pgfqpoint{9.145023in}{6.486241in}}%
\pgfpathcurveto{\pgfqpoint{9.137209in}{6.478428in}}{\pgfqpoint{9.132819in}{6.467828in}}{\pgfqpoint{9.132819in}{6.456778in}}%
\pgfpathcurveto{\pgfqpoint{9.132819in}{6.445728in}}{\pgfqpoint{9.137209in}{6.435129in}}{\pgfqpoint{9.145023in}{6.427316in}}%
\pgfpathcurveto{\pgfqpoint{9.152836in}{6.419502in}}{\pgfqpoint{9.163435in}{6.415112in}}{\pgfqpoint{9.174485in}{6.415112in}}%
\pgfpathlineto{\pgfqpoint{9.174485in}{6.415112in}}%
\pgfpathclose%
\pgfusepath{stroke,fill}%
\end{pgfscope}%
\begin{pgfscope}%
\pgfpathrectangle{\pgfqpoint{7.622482in}{5.272501in}}{\pgfqpoint{2.177280in}{2.201755in}}%
\pgfusepath{clip}%
\pgfsetbuttcap%
\pgfsetroundjoin%
\definecolor{currentfill}{rgb}{0.172549,0.627451,0.172549}%
\pgfsetfillcolor{currentfill}%
\pgfsetlinewidth{0.481800pt}%
\definecolor{currentstroke}{rgb}{1.000000,1.000000,1.000000}%
\pgfsetstrokecolor{currentstroke}%
\pgfsetdash{}{0pt}%
\pgfpathmoveto{\pgfqpoint{8.971224in}{6.331712in}}%
\pgfpathcurveto{\pgfqpoint{8.982274in}{6.331712in}}{\pgfqpoint{8.992873in}{6.336102in}}{\pgfqpoint{9.000686in}{6.343916in}}%
\pgfpathcurveto{\pgfqpoint{9.008500in}{6.351729in}}{\pgfqpoint{9.012890in}{6.362328in}}{\pgfqpoint{9.012890in}{6.373379in}}%
\pgfpathcurveto{\pgfqpoint{9.012890in}{6.384429in}}{\pgfqpoint{9.008500in}{6.395028in}}{\pgfqpoint{9.000686in}{6.402841in}}%
\pgfpathcurveto{\pgfqpoint{8.992873in}{6.410655in}}{\pgfqpoint{8.982274in}{6.415045in}}{\pgfqpoint{8.971224in}{6.415045in}}%
\pgfpathcurveto{\pgfqpoint{8.960174in}{6.415045in}}{\pgfqpoint{8.949575in}{6.410655in}}{\pgfqpoint{8.941761in}{6.402841in}}%
\pgfpathcurveto{\pgfqpoint{8.933947in}{6.395028in}}{\pgfqpoint{8.929557in}{6.384429in}}{\pgfqpoint{8.929557in}{6.373379in}}%
\pgfpathcurveto{\pgfqpoint{8.929557in}{6.362328in}}{\pgfqpoint{8.933947in}{6.351729in}}{\pgfqpoint{8.941761in}{6.343916in}}%
\pgfpathcurveto{\pgfqpoint{8.949575in}{6.336102in}}{\pgfqpoint{8.960174in}{6.331712in}}{\pgfqpoint{8.971224in}{6.331712in}}%
\pgfpathlineto{\pgfqpoint{8.971224in}{6.331712in}}%
\pgfpathclose%
\pgfusepath{stroke,fill}%
\end{pgfscope}%
\begin{pgfscope}%
\pgfpathrectangle{\pgfqpoint{7.622482in}{5.272501in}}{\pgfqpoint{2.177280in}{2.201755in}}%
\pgfusepath{clip}%
\pgfsetbuttcap%
\pgfsetroundjoin%
\definecolor{currentfill}{rgb}{0.172549,0.627451,0.172549}%
\pgfsetfillcolor{currentfill}%
\pgfsetlinewidth{0.481800pt}%
\definecolor{currentstroke}{rgb}{1.000000,1.000000,1.000000}%
\pgfsetstrokecolor{currentstroke}%
\pgfsetdash{}{0pt}%
\pgfpathmoveto{\pgfqpoint{8.971224in}{5.998113in}}%
\pgfpathcurveto{\pgfqpoint{8.982274in}{5.998113in}}{\pgfqpoint{8.992873in}{6.002503in}}{\pgfqpoint{9.000686in}{6.010317in}}%
\pgfpathcurveto{\pgfqpoint{9.008500in}{6.018130in}}{\pgfqpoint{9.012890in}{6.028729in}}{\pgfqpoint{9.012890in}{6.039779in}}%
\pgfpathcurveto{\pgfqpoint{9.012890in}{6.050829in}}{\pgfqpoint{9.008500in}{6.061428in}}{\pgfqpoint{9.000686in}{6.069242in}}%
\pgfpathcurveto{\pgfqpoint{8.992873in}{6.077056in}}{\pgfqpoint{8.982274in}{6.081446in}}{\pgfqpoint{8.971224in}{6.081446in}}%
\pgfpathcurveto{\pgfqpoint{8.960174in}{6.081446in}}{\pgfqpoint{8.949575in}{6.077056in}}{\pgfqpoint{8.941761in}{6.069242in}}%
\pgfpathcurveto{\pgfqpoint{8.933947in}{6.061428in}}{\pgfqpoint{8.929557in}{6.050829in}}{\pgfqpoint{8.929557in}{6.039779in}}%
\pgfpathcurveto{\pgfqpoint{8.929557in}{6.028729in}}{\pgfqpoint{8.933947in}{6.018130in}}{\pgfqpoint{8.941761in}{6.010317in}}%
\pgfpathcurveto{\pgfqpoint{8.949575in}{6.002503in}}{\pgfqpoint{8.960174in}{5.998113in}}{\pgfqpoint{8.971224in}{5.998113in}}%
\pgfpathlineto{\pgfqpoint{8.971224in}{5.998113in}}%
\pgfpathclose%
\pgfusepath{stroke,fill}%
\end{pgfscope}%
\begin{pgfscope}%
\pgfpathrectangle{\pgfqpoint{7.622482in}{5.272501in}}{\pgfqpoint{2.177280in}{2.201755in}}%
\pgfusepath{clip}%
\pgfsetbuttcap%
\pgfsetroundjoin%
\definecolor{currentfill}{rgb}{0.172549,0.627451,0.172549}%
\pgfsetfillcolor{currentfill}%
\pgfsetlinewidth{0.481800pt}%
\definecolor{currentstroke}{rgb}{1.000000,1.000000,1.000000}%
\pgfsetstrokecolor{currentstroke}%
\pgfsetdash{}{0pt}%
\pgfpathmoveto{\pgfqpoint{8.971224in}{6.164912in}}%
\pgfpathcurveto{\pgfqpoint{8.982274in}{6.164912in}}{\pgfqpoint{8.992873in}{6.169303in}}{\pgfqpoint{9.000686in}{6.177116in}}%
\pgfpathcurveto{\pgfqpoint{9.008500in}{6.184930in}}{\pgfqpoint{9.012890in}{6.195529in}}{\pgfqpoint{9.012890in}{6.206579in}}%
\pgfpathcurveto{\pgfqpoint{9.012890in}{6.217629in}}{\pgfqpoint{9.008500in}{6.228228in}}{\pgfqpoint{9.000686in}{6.236042in}}%
\pgfpathcurveto{\pgfqpoint{8.992873in}{6.243855in}}{\pgfqpoint{8.982274in}{6.248246in}}{\pgfqpoint{8.971224in}{6.248246in}}%
\pgfpathcurveto{\pgfqpoint{8.960174in}{6.248246in}}{\pgfqpoint{8.949575in}{6.243855in}}{\pgfqpoint{8.941761in}{6.236042in}}%
\pgfpathcurveto{\pgfqpoint{8.933947in}{6.228228in}}{\pgfqpoint{8.929557in}{6.217629in}}{\pgfqpoint{8.929557in}{6.206579in}}%
\pgfpathcurveto{\pgfqpoint{8.929557in}{6.195529in}}{\pgfqpoint{8.933947in}{6.184930in}}{\pgfqpoint{8.941761in}{6.177116in}}%
\pgfpathcurveto{\pgfqpoint{8.949575in}{6.169303in}}{\pgfqpoint{8.960174in}{6.164912in}}{\pgfqpoint{8.971224in}{6.164912in}}%
\pgfpathlineto{\pgfqpoint{8.971224in}{6.164912in}}%
\pgfpathclose%
\pgfusepath{stroke,fill}%
\end{pgfscope}%
\begin{pgfscope}%
\pgfpathrectangle{\pgfqpoint{7.622482in}{5.272501in}}{\pgfqpoint{2.177280in}{2.201755in}}%
\pgfusepath{clip}%
\pgfsetbuttcap%
\pgfsetroundjoin%
\definecolor{currentfill}{rgb}{0.172549,0.627451,0.172549}%
\pgfsetfillcolor{currentfill}%
\pgfsetlinewidth{0.481800pt}%
\definecolor{currentstroke}{rgb}{1.000000,1.000000,1.000000}%
\pgfsetstrokecolor{currentstroke}%
\pgfsetdash{}{0pt}%
\pgfpathmoveto{\pgfqpoint{9.174485in}{5.998113in}}%
\pgfpathcurveto{\pgfqpoint{9.185535in}{5.998113in}}{\pgfqpoint{9.196134in}{6.002503in}}{\pgfqpoint{9.203948in}{6.010317in}}%
\pgfpathcurveto{\pgfqpoint{9.211762in}{6.018130in}}{\pgfqpoint{9.216152in}{6.028729in}}{\pgfqpoint{9.216152in}{6.039779in}}%
\pgfpathcurveto{\pgfqpoint{9.216152in}{6.050829in}}{\pgfqpoint{9.211762in}{6.061428in}}{\pgfqpoint{9.203948in}{6.069242in}}%
\pgfpathcurveto{\pgfqpoint{9.196134in}{6.077056in}}{\pgfqpoint{9.185535in}{6.081446in}}{\pgfqpoint{9.174485in}{6.081446in}}%
\pgfpathcurveto{\pgfqpoint{9.163435in}{6.081446in}}{\pgfqpoint{9.152836in}{6.077056in}}{\pgfqpoint{9.145023in}{6.069242in}}%
\pgfpathcurveto{\pgfqpoint{9.137209in}{6.061428in}}{\pgfqpoint{9.132819in}{6.050829in}}{\pgfqpoint{9.132819in}{6.039779in}}%
\pgfpathcurveto{\pgfqpoint{9.132819in}{6.028729in}}{\pgfqpoint{9.137209in}{6.018130in}}{\pgfqpoint{9.145023in}{6.010317in}}%
\pgfpathcurveto{\pgfqpoint{9.152836in}{6.002503in}}{\pgfqpoint{9.163435in}{5.998113in}}{\pgfqpoint{9.174485in}{5.998113in}}%
\pgfpathlineto{\pgfqpoint{9.174485in}{5.998113in}}%
\pgfpathclose%
\pgfusepath{stroke,fill}%
\end{pgfscope}%
\begin{pgfscope}%
\pgfpathrectangle{\pgfqpoint{7.622482in}{5.272501in}}{\pgfqpoint{2.177280in}{2.201755in}}%
\pgfusepath{clip}%
\pgfsetbuttcap%
\pgfsetroundjoin%
\definecolor{currentfill}{rgb}{0.172549,0.627451,0.172549}%
\pgfsetfillcolor{currentfill}%
\pgfsetlinewidth{0.481800pt}%
\definecolor{currentstroke}{rgb}{1.000000,1.000000,1.000000}%
\pgfsetstrokecolor{currentstroke}%
\pgfsetdash{}{0pt}%
\pgfpathmoveto{\pgfqpoint{8.835716in}{6.164912in}}%
\pgfpathcurveto{\pgfqpoint{8.846766in}{6.164912in}}{\pgfqpoint{8.857365in}{6.169303in}}{\pgfqpoint{8.865179in}{6.177116in}}%
\pgfpathcurveto{\pgfqpoint{8.872992in}{6.184930in}}{\pgfqpoint{8.877383in}{6.195529in}}{\pgfqpoint{8.877383in}{6.206579in}}%
\pgfpathcurveto{\pgfqpoint{8.877383in}{6.217629in}}{\pgfqpoint{8.872992in}{6.228228in}}{\pgfqpoint{8.865179in}{6.236042in}}%
\pgfpathcurveto{\pgfqpoint{8.857365in}{6.243855in}}{\pgfqpoint{8.846766in}{6.248246in}}{\pgfqpoint{8.835716in}{6.248246in}}%
\pgfpathcurveto{\pgfqpoint{8.824666in}{6.248246in}}{\pgfqpoint{8.814067in}{6.243855in}}{\pgfqpoint{8.806253in}{6.236042in}}%
\pgfpathcurveto{\pgfqpoint{8.798440in}{6.228228in}}{\pgfqpoint{8.794049in}{6.217629in}}{\pgfqpoint{8.794049in}{6.206579in}}%
\pgfpathcurveto{\pgfqpoint{8.794049in}{6.195529in}}{\pgfqpoint{8.798440in}{6.184930in}}{\pgfqpoint{8.806253in}{6.177116in}}%
\pgfpathcurveto{\pgfqpoint{8.814067in}{6.169303in}}{\pgfqpoint{8.824666in}{6.164912in}}{\pgfqpoint{8.835716in}{6.164912in}}%
\pgfpathlineto{\pgfqpoint{8.835716in}{6.164912in}}%
\pgfpathclose%
\pgfusepath{stroke,fill}%
\end{pgfscope}%
\begin{pgfscope}%
\pgfpathrectangle{\pgfqpoint{7.622482in}{5.272501in}}{\pgfqpoint{2.177280in}{2.201755in}}%
\pgfusepath{clip}%
\pgfsetbuttcap%
\pgfsetroundjoin%
\definecolor{currentfill}{rgb}{0.172549,0.627451,0.172549}%
\pgfsetfillcolor{currentfill}%
\pgfsetlinewidth{0.481800pt}%
\definecolor{currentstroke}{rgb}{1.000000,1.000000,1.000000}%
\pgfsetstrokecolor{currentstroke}%
\pgfsetdash{}{0pt}%
\pgfpathmoveto{\pgfqpoint{9.038978in}{5.998113in}}%
\pgfpathcurveto{\pgfqpoint{9.050028in}{5.998113in}}{\pgfqpoint{9.060627in}{6.002503in}}{\pgfqpoint{9.068440in}{6.010317in}}%
\pgfpathcurveto{\pgfqpoint{9.076254in}{6.018130in}}{\pgfqpoint{9.080644in}{6.028729in}}{\pgfqpoint{9.080644in}{6.039779in}}%
\pgfpathcurveto{\pgfqpoint{9.080644in}{6.050829in}}{\pgfqpoint{9.076254in}{6.061428in}}{\pgfqpoint{9.068440in}{6.069242in}}%
\pgfpathcurveto{\pgfqpoint{9.060627in}{6.077056in}}{\pgfqpoint{9.050028in}{6.081446in}}{\pgfqpoint{9.038978in}{6.081446in}}%
\pgfpathcurveto{\pgfqpoint{9.027927in}{6.081446in}}{\pgfqpoint{9.017328in}{6.077056in}}{\pgfqpoint{9.009515in}{6.069242in}}%
\pgfpathcurveto{\pgfqpoint{9.001701in}{6.061428in}}{\pgfqpoint{8.997311in}{6.050829in}}{\pgfqpoint{8.997311in}{6.039779in}}%
\pgfpathcurveto{\pgfqpoint{8.997311in}{6.028729in}}{\pgfqpoint{9.001701in}{6.018130in}}{\pgfqpoint{9.009515in}{6.010317in}}%
\pgfpathcurveto{\pgfqpoint{9.017328in}{6.002503in}}{\pgfqpoint{9.027927in}{5.998113in}}{\pgfqpoint{9.038978in}{5.998113in}}%
\pgfpathlineto{\pgfqpoint{9.038978in}{5.998113in}}%
\pgfpathclose%
\pgfusepath{stroke,fill}%
\end{pgfscope}%
\begin{pgfscope}%
\pgfpathrectangle{\pgfqpoint{7.622482in}{5.272501in}}{\pgfqpoint{2.177280in}{2.201755in}}%
\pgfusepath{clip}%
\pgfsetbuttcap%
\pgfsetroundjoin%
\definecolor{currentfill}{rgb}{0.172549,0.627451,0.172549}%
\pgfsetfillcolor{currentfill}%
\pgfsetlinewidth{0.481800pt}%
\definecolor{currentstroke}{rgb}{1.000000,1.000000,1.000000}%
\pgfsetstrokecolor{currentstroke}%
\pgfsetdash{}{0pt}%
\pgfpathmoveto{\pgfqpoint{9.106731in}{6.832111in}}%
\pgfpathcurveto{\pgfqpoint{9.117782in}{6.832111in}}{\pgfqpoint{9.128381in}{6.836501in}}{\pgfqpoint{9.136194in}{6.844315in}}%
\pgfpathcurveto{\pgfqpoint{9.144008in}{6.852128in}}{\pgfqpoint{9.148398in}{6.862727in}}{\pgfqpoint{9.148398in}{6.873777in}}%
\pgfpathcurveto{\pgfqpoint{9.148398in}{6.884828in}}{\pgfqpoint{9.144008in}{6.895427in}}{\pgfqpoint{9.136194in}{6.903240in}}%
\pgfpathcurveto{\pgfqpoint{9.128381in}{6.911054in}}{\pgfqpoint{9.117782in}{6.915444in}}{\pgfqpoint{9.106731in}{6.915444in}}%
\pgfpathcurveto{\pgfqpoint{9.095681in}{6.915444in}}{\pgfqpoint{9.085082in}{6.911054in}}{\pgfqpoint{9.077269in}{6.903240in}}%
\pgfpathcurveto{\pgfqpoint{9.069455in}{6.895427in}}{\pgfqpoint{9.065065in}{6.884828in}}{\pgfqpoint{9.065065in}{6.873777in}}%
\pgfpathcurveto{\pgfqpoint{9.065065in}{6.862727in}}{\pgfqpoint{9.069455in}{6.852128in}}{\pgfqpoint{9.077269in}{6.844315in}}%
\pgfpathcurveto{\pgfqpoint{9.085082in}{6.836501in}}{\pgfqpoint{9.095681in}{6.832111in}}{\pgfqpoint{9.106731in}{6.832111in}}%
\pgfpathlineto{\pgfqpoint{9.106731in}{6.832111in}}%
\pgfpathclose%
\pgfusepath{stroke,fill}%
\end{pgfscope}%
\begin{pgfscope}%
\pgfpathrectangle{\pgfqpoint{7.622482in}{5.272501in}}{\pgfqpoint{2.177280in}{2.201755in}}%
\pgfusepath{clip}%
\pgfsetbuttcap%
\pgfsetroundjoin%
\definecolor{currentfill}{rgb}{0.172549,0.627451,0.172549}%
\pgfsetfillcolor{currentfill}%
\pgfsetlinewidth{0.481800pt}%
\definecolor{currentstroke}{rgb}{1.000000,1.000000,1.000000}%
\pgfsetstrokecolor{currentstroke}%
\pgfsetdash{}{0pt}%
\pgfpathmoveto{\pgfqpoint{9.242239in}{5.998113in}}%
\pgfpathcurveto{\pgfqpoint{9.253289in}{5.998113in}}{\pgfqpoint{9.263888in}{6.002503in}}{\pgfqpoint{9.271702in}{6.010317in}}%
\pgfpathcurveto{\pgfqpoint{9.279516in}{6.018130in}}{\pgfqpoint{9.283906in}{6.028729in}}{\pgfqpoint{9.283906in}{6.039779in}}%
\pgfpathcurveto{\pgfqpoint{9.283906in}{6.050829in}}{\pgfqpoint{9.279516in}{6.061428in}}{\pgfqpoint{9.271702in}{6.069242in}}%
\pgfpathcurveto{\pgfqpoint{9.263888in}{6.077056in}}{\pgfqpoint{9.253289in}{6.081446in}}{\pgfqpoint{9.242239in}{6.081446in}}%
\pgfpathcurveto{\pgfqpoint{9.231189in}{6.081446in}}{\pgfqpoint{9.220590in}{6.077056in}}{\pgfqpoint{9.212776in}{6.069242in}}%
\pgfpathcurveto{\pgfqpoint{9.204963in}{6.061428in}}{\pgfqpoint{9.200573in}{6.050829in}}{\pgfqpoint{9.200573in}{6.039779in}}%
\pgfpathcurveto{\pgfqpoint{9.200573in}{6.028729in}}{\pgfqpoint{9.204963in}{6.018130in}}{\pgfqpoint{9.212776in}{6.010317in}}%
\pgfpathcurveto{\pgfqpoint{9.220590in}{6.002503in}}{\pgfqpoint{9.231189in}{5.998113in}}{\pgfqpoint{9.242239in}{5.998113in}}%
\pgfpathlineto{\pgfqpoint{9.242239in}{5.998113in}}%
\pgfpathclose%
\pgfusepath{stroke,fill}%
\end{pgfscope}%
\begin{pgfscope}%
\pgfpathrectangle{\pgfqpoint{7.622482in}{5.272501in}}{\pgfqpoint{2.177280in}{2.201755in}}%
\pgfusepath{clip}%
\pgfsetbuttcap%
\pgfsetroundjoin%
\definecolor{currentfill}{rgb}{0.172549,0.627451,0.172549}%
\pgfsetfillcolor{currentfill}%
\pgfsetlinewidth{0.481800pt}%
\definecolor{currentstroke}{rgb}{1.000000,1.000000,1.000000}%
\pgfsetstrokecolor{currentstroke}%
\pgfsetdash{}{0pt}%
\pgfpathmoveto{\pgfqpoint{8.767962in}{5.998113in}}%
\pgfpathcurveto{\pgfqpoint{8.779012in}{5.998113in}}{\pgfqpoint{8.789611in}{6.002503in}}{\pgfqpoint{8.797425in}{6.010317in}}%
\pgfpathcurveto{\pgfqpoint{8.805238in}{6.018130in}}{\pgfqpoint{8.809629in}{6.028729in}}{\pgfqpoint{8.809629in}{6.039779in}}%
\pgfpathcurveto{\pgfqpoint{8.809629in}{6.050829in}}{\pgfqpoint{8.805238in}{6.061428in}}{\pgfqpoint{8.797425in}{6.069242in}}%
\pgfpathcurveto{\pgfqpoint{8.789611in}{6.077056in}}{\pgfqpoint{8.779012in}{6.081446in}}{\pgfqpoint{8.767962in}{6.081446in}}%
\pgfpathcurveto{\pgfqpoint{8.756912in}{6.081446in}}{\pgfqpoint{8.746313in}{6.077056in}}{\pgfqpoint{8.738499in}{6.069242in}}%
\pgfpathcurveto{\pgfqpoint{8.730686in}{6.061428in}}{\pgfqpoint{8.726295in}{6.050829in}}{\pgfqpoint{8.726295in}{6.039779in}}%
\pgfpathcurveto{\pgfqpoint{8.726295in}{6.028729in}}{\pgfqpoint{8.730686in}{6.018130in}}{\pgfqpoint{8.738499in}{6.010317in}}%
\pgfpathcurveto{\pgfqpoint{8.746313in}{6.002503in}}{\pgfqpoint{8.756912in}{5.998113in}}{\pgfqpoint{8.767962in}{5.998113in}}%
\pgfpathlineto{\pgfqpoint{8.767962in}{5.998113in}}%
\pgfpathclose%
\pgfusepath{stroke,fill}%
\end{pgfscope}%
\begin{pgfscope}%
\pgfpathrectangle{\pgfqpoint{7.622482in}{5.272501in}}{\pgfqpoint{2.177280in}{2.201755in}}%
\pgfusepath{clip}%
\pgfsetbuttcap%
\pgfsetroundjoin%
\definecolor{currentfill}{rgb}{0.172549,0.627451,0.172549}%
\pgfsetfillcolor{currentfill}%
\pgfsetlinewidth{0.481800pt}%
\definecolor{currentstroke}{rgb}{1.000000,1.000000,1.000000}%
\pgfsetstrokecolor{currentstroke}%
\pgfsetdash{}{0pt}%
\pgfpathmoveto{\pgfqpoint{8.700208in}{5.831313in}}%
\pgfpathcurveto{\pgfqpoint{8.711258in}{5.831313in}}{\pgfqpoint{8.721857in}{5.835703in}}{\pgfqpoint{8.729671in}{5.843517in}}%
\pgfpathcurveto{\pgfqpoint{8.737485in}{5.851331in}}{\pgfqpoint{8.741875in}{5.861930in}}{\pgfqpoint{8.741875in}{5.872980in}}%
\pgfpathcurveto{\pgfqpoint{8.741875in}{5.884030in}}{\pgfqpoint{8.737485in}{5.894629in}}{\pgfqpoint{8.729671in}{5.902442in}}%
\pgfpathcurveto{\pgfqpoint{8.721857in}{5.910256in}}{\pgfqpoint{8.711258in}{5.914646in}}{\pgfqpoint{8.700208in}{5.914646in}}%
\pgfpathcurveto{\pgfqpoint{8.689158in}{5.914646in}}{\pgfqpoint{8.678559in}{5.910256in}}{\pgfqpoint{8.670745in}{5.902442in}}%
\pgfpathcurveto{\pgfqpoint{8.662932in}{5.894629in}}{\pgfqpoint{8.658542in}{5.884030in}}{\pgfqpoint{8.658542in}{5.872980in}}%
\pgfpathcurveto{\pgfqpoint{8.658542in}{5.861930in}}{\pgfqpoint{8.662932in}{5.851331in}}{\pgfqpoint{8.670745in}{5.843517in}}%
\pgfpathcurveto{\pgfqpoint{8.678559in}{5.835703in}}{\pgfqpoint{8.689158in}{5.831313in}}{\pgfqpoint{8.700208in}{5.831313in}}%
\pgfpathlineto{\pgfqpoint{8.700208in}{5.831313in}}%
\pgfpathclose%
\pgfusepath{stroke,fill}%
\end{pgfscope}%
\begin{pgfscope}%
\pgfpathrectangle{\pgfqpoint{7.622482in}{5.272501in}}{\pgfqpoint{2.177280in}{2.201755in}}%
\pgfusepath{clip}%
\pgfsetbuttcap%
\pgfsetroundjoin%
\definecolor{currentfill}{rgb}{0.172549,0.627451,0.172549}%
\pgfsetfillcolor{currentfill}%
\pgfsetlinewidth{0.481800pt}%
\definecolor{currentstroke}{rgb}{1.000000,1.000000,1.000000}%
\pgfsetstrokecolor{currentstroke}%
\pgfsetdash{}{0pt}%
\pgfpathmoveto{\pgfqpoint{9.309993in}{6.164912in}}%
\pgfpathcurveto{\pgfqpoint{9.321043in}{6.164912in}}{\pgfqpoint{9.331642in}{6.169303in}}{\pgfqpoint{9.339456in}{6.177116in}}%
\pgfpathcurveto{\pgfqpoint{9.347269in}{6.184930in}}{\pgfqpoint{9.351660in}{6.195529in}}{\pgfqpoint{9.351660in}{6.206579in}}%
\pgfpathcurveto{\pgfqpoint{9.351660in}{6.217629in}}{\pgfqpoint{9.347269in}{6.228228in}}{\pgfqpoint{9.339456in}{6.236042in}}%
\pgfpathcurveto{\pgfqpoint{9.331642in}{6.243855in}}{\pgfqpoint{9.321043in}{6.248246in}}{\pgfqpoint{9.309993in}{6.248246in}}%
\pgfpathcurveto{\pgfqpoint{9.298943in}{6.248246in}}{\pgfqpoint{9.288344in}{6.243855in}}{\pgfqpoint{9.280530in}{6.236042in}}%
\pgfpathcurveto{\pgfqpoint{9.272717in}{6.228228in}}{\pgfqpoint{9.268326in}{6.217629in}}{\pgfqpoint{9.268326in}{6.206579in}}%
\pgfpathcurveto{\pgfqpoint{9.268326in}{6.195529in}}{\pgfqpoint{9.272717in}{6.184930in}}{\pgfqpoint{9.280530in}{6.177116in}}%
\pgfpathcurveto{\pgfqpoint{9.288344in}{6.169303in}}{\pgfqpoint{9.298943in}{6.164912in}}{\pgfqpoint{9.309993in}{6.164912in}}%
\pgfpathlineto{\pgfqpoint{9.309993in}{6.164912in}}%
\pgfpathclose%
\pgfusepath{stroke,fill}%
\end{pgfscope}%
\begin{pgfscope}%
\pgfpathrectangle{\pgfqpoint{7.622482in}{5.272501in}}{\pgfqpoint{2.177280in}{2.201755in}}%
\pgfusepath{clip}%
\pgfsetbuttcap%
\pgfsetroundjoin%
\definecolor{currentfill}{rgb}{0.172549,0.627451,0.172549}%
\pgfsetfillcolor{currentfill}%
\pgfsetlinewidth{0.481800pt}%
\definecolor{currentstroke}{rgb}{1.000000,1.000000,1.000000}%
\pgfsetstrokecolor{currentstroke}%
\pgfsetdash{}{0pt}%
\pgfpathmoveto{\pgfqpoint{9.377747in}{6.498512in}}%
\pgfpathcurveto{\pgfqpoint{9.388797in}{6.498512in}}{\pgfqpoint{9.399396in}{6.502902in}}{\pgfqpoint{9.407210in}{6.510715in}}%
\pgfpathcurveto{\pgfqpoint{9.415023in}{6.518529in}}{\pgfqpoint{9.419414in}{6.529128in}}{\pgfqpoint{9.419414in}{6.540178in}}%
\pgfpathcurveto{\pgfqpoint{9.419414in}{6.551228in}}{\pgfqpoint{9.415023in}{6.561827in}}{\pgfqpoint{9.407210in}{6.569641in}}%
\pgfpathcurveto{\pgfqpoint{9.399396in}{6.577455in}}{\pgfqpoint{9.388797in}{6.581845in}}{\pgfqpoint{9.377747in}{6.581845in}}%
\pgfpathcurveto{\pgfqpoint{9.366697in}{6.581845in}}{\pgfqpoint{9.356098in}{6.577455in}}{\pgfqpoint{9.348284in}{6.569641in}}%
\pgfpathcurveto{\pgfqpoint{9.340471in}{6.561827in}}{\pgfqpoint{9.336080in}{6.551228in}}{\pgfqpoint{9.336080in}{6.540178in}}%
\pgfpathcurveto{\pgfqpoint{9.336080in}{6.529128in}}{\pgfqpoint{9.340471in}{6.518529in}}{\pgfqpoint{9.348284in}{6.510715in}}%
\pgfpathcurveto{\pgfqpoint{9.356098in}{6.502902in}}{\pgfqpoint{9.366697in}{6.498512in}}{\pgfqpoint{9.377747in}{6.498512in}}%
\pgfpathlineto{\pgfqpoint{9.377747in}{6.498512in}}%
\pgfpathclose%
\pgfusepath{stroke,fill}%
\end{pgfscope}%
\begin{pgfscope}%
\pgfpathrectangle{\pgfqpoint{7.622482in}{5.272501in}}{\pgfqpoint{2.177280in}{2.201755in}}%
\pgfusepath{clip}%
\pgfsetbuttcap%
\pgfsetroundjoin%
\definecolor{currentfill}{rgb}{0.172549,0.627451,0.172549}%
\pgfsetfillcolor{currentfill}%
\pgfsetlinewidth{0.481800pt}%
\definecolor{currentstroke}{rgb}{1.000000,1.000000,1.000000}%
\pgfsetstrokecolor{currentstroke}%
\pgfsetdash{}{0pt}%
\pgfpathmoveto{\pgfqpoint{8.971224in}{6.248312in}}%
\pgfpathcurveto{\pgfqpoint{8.982274in}{6.248312in}}{\pgfqpoint{8.992873in}{6.252702in}}{\pgfqpoint{9.000686in}{6.260516in}}%
\pgfpathcurveto{\pgfqpoint{9.008500in}{6.268330in}}{\pgfqpoint{9.012890in}{6.278929in}}{\pgfqpoint{9.012890in}{6.289979in}}%
\pgfpathcurveto{\pgfqpoint{9.012890in}{6.301029in}}{\pgfqpoint{9.008500in}{6.311628in}}{\pgfqpoint{9.000686in}{6.319442in}}%
\pgfpathcurveto{\pgfqpoint{8.992873in}{6.327255in}}{\pgfqpoint{8.982274in}{6.331645in}}{\pgfqpoint{8.971224in}{6.331645in}}%
\pgfpathcurveto{\pgfqpoint{8.960174in}{6.331645in}}{\pgfqpoint{8.949575in}{6.327255in}}{\pgfqpoint{8.941761in}{6.319442in}}%
\pgfpathcurveto{\pgfqpoint{8.933947in}{6.311628in}}{\pgfqpoint{8.929557in}{6.301029in}}{\pgfqpoint{8.929557in}{6.289979in}}%
\pgfpathcurveto{\pgfqpoint{8.929557in}{6.278929in}}{\pgfqpoint{8.933947in}{6.268330in}}{\pgfqpoint{8.941761in}{6.260516in}}%
\pgfpathcurveto{\pgfqpoint{8.949575in}{6.252702in}}{\pgfqpoint{8.960174in}{6.248312in}}{\pgfqpoint{8.971224in}{6.248312in}}%
\pgfpathlineto{\pgfqpoint{8.971224in}{6.248312in}}%
\pgfpathclose%
\pgfusepath{stroke,fill}%
\end{pgfscope}%
\begin{pgfscope}%
\pgfpathrectangle{\pgfqpoint{7.622482in}{5.272501in}}{\pgfqpoint{2.177280in}{2.201755in}}%
\pgfusepath{clip}%
\pgfsetbuttcap%
\pgfsetroundjoin%
\definecolor{currentfill}{rgb}{0.172549,0.627451,0.172549}%
\pgfsetfillcolor{currentfill}%
\pgfsetlinewidth{0.481800pt}%
\definecolor{currentstroke}{rgb}{1.000000,1.000000,1.000000}%
\pgfsetstrokecolor{currentstroke}%
\pgfsetdash{}{0pt}%
\pgfpathmoveto{\pgfqpoint{8.971224in}{6.164912in}}%
\pgfpathcurveto{\pgfqpoint{8.982274in}{6.164912in}}{\pgfqpoint{8.992873in}{6.169303in}}{\pgfqpoint{9.000686in}{6.177116in}}%
\pgfpathcurveto{\pgfqpoint{9.008500in}{6.184930in}}{\pgfqpoint{9.012890in}{6.195529in}}{\pgfqpoint{9.012890in}{6.206579in}}%
\pgfpathcurveto{\pgfqpoint{9.012890in}{6.217629in}}{\pgfqpoint{9.008500in}{6.228228in}}{\pgfqpoint{9.000686in}{6.236042in}}%
\pgfpathcurveto{\pgfqpoint{8.992873in}{6.243855in}}{\pgfqpoint{8.982274in}{6.248246in}}{\pgfqpoint{8.971224in}{6.248246in}}%
\pgfpathcurveto{\pgfqpoint{8.960174in}{6.248246in}}{\pgfqpoint{8.949575in}{6.243855in}}{\pgfqpoint{8.941761in}{6.236042in}}%
\pgfpathcurveto{\pgfqpoint{8.933947in}{6.228228in}}{\pgfqpoint{8.929557in}{6.217629in}}{\pgfqpoint{8.929557in}{6.206579in}}%
\pgfpathcurveto{\pgfqpoint{8.929557in}{6.195529in}}{\pgfqpoint{8.933947in}{6.184930in}}{\pgfqpoint{8.941761in}{6.177116in}}%
\pgfpathcurveto{\pgfqpoint{8.949575in}{6.169303in}}{\pgfqpoint{8.960174in}{6.164912in}}{\pgfqpoint{8.971224in}{6.164912in}}%
\pgfpathlineto{\pgfqpoint{8.971224in}{6.164912in}}%
\pgfpathclose%
\pgfusepath{stroke,fill}%
\end{pgfscope}%
\begin{pgfscope}%
\pgfpathrectangle{\pgfqpoint{7.622482in}{5.272501in}}{\pgfqpoint{2.177280in}{2.201755in}}%
\pgfusepath{clip}%
\pgfsetbuttcap%
\pgfsetroundjoin%
\definecolor{currentfill}{rgb}{0.172549,0.627451,0.172549}%
\pgfsetfillcolor{currentfill}%
\pgfsetlinewidth{0.481800pt}%
\definecolor{currentstroke}{rgb}{1.000000,1.000000,1.000000}%
\pgfsetstrokecolor{currentstroke}%
\pgfsetdash{}{0pt}%
\pgfpathmoveto{\pgfqpoint{9.174485in}{6.248312in}}%
\pgfpathcurveto{\pgfqpoint{9.185535in}{6.248312in}}{\pgfqpoint{9.196134in}{6.252702in}}{\pgfqpoint{9.203948in}{6.260516in}}%
\pgfpathcurveto{\pgfqpoint{9.211762in}{6.268330in}}{\pgfqpoint{9.216152in}{6.278929in}}{\pgfqpoint{9.216152in}{6.289979in}}%
\pgfpathcurveto{\pgfqpoint{9.216152in}{6.301029in}}{\pgfqpoint{9.211762in}{6.311628in}}{\pgfqpoint{9.203948in}{6.319442in}}%
\pgfpathcurveto{\pgfqpoint{9.196134in}{6.327255in}}{\pgfqpoint{9.185535in}{6.331645in}}{\pgfqpoint{9.174485in}{6.331645in}}%
\pgfpathcurveto{\pgfqpoint{9.163435in}{6.331645in}}{\pgfqpoint{9.152836in}{6.327255in}}{\pgfqpoint{9.145023in}{6.319442in}}%
\pgfpathcurveto{\pgfqpoint{9.137209in}{6.311628in}}{\pgfqpoint{9.132819in}{6.301029in}}{\pgfqpoint{9.132819in}{6.289979in}}%
\pgfpathcurveto{\pgfqpoint{9.132819in}{6.278929in}}{\pgfqpoint{9.137209in}{6.268330in}}{\pgfqpoint{9.145023in}{6.260516in}}%
\pgfpathcurveto{\pgfqpoint{9.152836in}{6.252702in}}{\pgfqpoint{9.163435in}{6.248312in}}{\pgfqpoint{9.174485in}{6.248312in}}%
\pgfpathlineto{\pgfqpoint{9.174485in}{6.248312in}}%
\pgfpathclose%
\pgfusepath{stroke,fill}%
\end{pgfscope}%
\begin{pgfscope}%
\pgfpathrectangle{\pgfqpoint{7.622482in}{5.272501in}}{\pgfqpoint{2.177280in}{2.201755in}}%
\pgfusepath{clip}%
\pgfsetbuttcap%
\pgfsetroundjoin%
\definecolor{currentfill}{rgb}{0.172549,0.627451,0.172549}%
\pgfsetfillcolor{currentfill}%
\pgfsetlinewidth{0.481800pt}%
\definecolor{currentstroke}{rgb}{1.000000,1.000000,1.000000}%
\pgfsetstrokecolor{currentstroke}%
\pgfsetdash{}{0pt}%
\pgfpathmoveto{\pgfqpoint{9.377747in}{6.248312in}}%
\pgfpathcurveto{\pgfqpoint{9.388797in}{6.248312in}}{\pgfqpoint{9.399396in}{6.252702in}}{\pgfqpoint{9.407210in}{6.260516in}}%
\pgfpathcurveto{\pgfqpoint{9.415023in}{6.268330in}}{\pgfqpoint{9.419414in}{6.278929in}}{\pgfqpoint{9.419414in}{6.289979in}}%
\pgfpathcurveto{\pgfqpoint{9.419414in}{6.301029in}}{\pgfqpoint{9.415023in}{6.311628in}}{\pgfqpoint{9.407210in}{6.319442in}}%
\pgfpathcurveto{\pgfqpoint{9.399396in}{6.327255in}}{\pgfqpoint{9.388797in}{6.331645in}}{\pgfqpoint{9.377747in}{6.331645in}}%
\pgfpathcurveto{\pgfqpoint{9.366697in}{6.331645in}}{\pgfqpoint{9.356098in}{6.327255in}}{\pgfqpoint{9.348284in}{6.319442in}}%
\pgfpathcurveto{\pgfqpoint{9.340471in}{6.311628in}}{\pgfqpoint{9.336080in}{6.301029in}}{\pgfqpoint{9.336080in}{6.289979in}}%
\pgfpathcurveto{\pgfqpoint{9.336080in}{6.278929in}}{\pgfqpoint{9.340471in}{6.268330in}}{\pgfqpoint{9.348284in}{6.260516in}}%
\pgfpathcurveto{\pgfqpoint{9.356098in}{6.252702in}}{\pgfqpoint{9.366697in}{6.248312in}}{\pgfqpoint{9.377747in}{6.248312in}}%
\pgfpathlineto{\pgfqpoint{9.377747in}{6.248312in}}%
\pgfpathclose%
\pgfusepath{stroke,fill}%
\end{pgfscope}%
\begin{pgfscope}%
\pgfpathrectangle{\pgfqpoint{7.622482in}{5.272501in}}{\pgfqpoint{2.177280in}{2.201755in}}%
\pgfusepath{clip}%
\pgfsetbuttcap%
\pgfsetroundjoin%
\definecolor{currentfill}{rgb}{0.172549,0.627451,0.172549}%
\pgfsetfillcolor{currentfill}%
\pgfsetlinewidth{0.481800pt}%
\definecolor{currentstroke}{rgb}{1.000000,1.000000,1.000000}%
\pgfsetstrokecolor{currentstroke}%
\pgfsetdash{}{0pt}%
\pgfpathmoveto{\pgfqpoint{9.309993in}{6.248312in}}%
\pgfpathcurveto{\pgfqpoint{9.321043in}{6.248312in}}{\pgfqpoint{9.331642in}{6.252702in}}{\pgfqpoint{9.339456in}{6.260516in}}%
\pgfpathcurveto{\pgfqpoint{9.347269in}{6.268330in}}{\pgfqpoint{9.351660in}{6.278929in}}{\pgfqpoint{9.351660in}{6.289979in}}%
\pgfpathcurveto{\pgfqpoint{9.351660in}{6.301029in}}{\pgfqpoint{9.347269in}{6.311628in}}{\pgfqpoint{9.339456in}{6.319442in}}%
\pgfpathcurveto{\pgfqpoint{9.331642in}{6.327255in}}{\pgfqpoint{9.321043in}{6.331645in}}{\pgfqpoint{9.309993in}{6.331645in}}%
\pgfpathcurveto{\pgfqpoint{9.298943in}{6.331645in}}{\pgfqpoint{9.288344in}{6.327255in}}{\pgfqpoint{9.280530in}{6.319442in}}%
\pgfpathcurveto{\pgfqpoint{9.272717in}{6.311628in}}{\pgfqpoint{9.268326in}{6.301029in}}{\pgfqpoint{9.268326in}{6.289979in}}%
\pgfpathcurveto{\pgfqpoint{9.268326in}{6.278929in}}{\pgfqpoint{9.272717in}{6.268330in}}{\pgfqpoint{9.280530in}{6.260516in}}%
\pgfpathcurveto{\pgfqpoint{9.288344in}{6.252702in}}{\pgfqpoint{9.298943in}{6.248312in}}{\pgfqpoint{9.309993in}{6.248312in}}%
\pgfpathlineto{\pgfqpoint{9.309993in}{6.248312in}}%
\pgfpathclose%
\pgfusepath{stroke,fill}%
\end{pgfscope}%
\begin{pgfscope}%
\pgfpathrectangle{\pgfqpoint{7.622482in}{5.272501in}}{\pgfqpoint{2.177280in}{2.201755in}}%
\pgfusepath{clip}%
\pgfsetbuttcap%
\pgfsetroundjoin%
\definecolor{currentfill}{rgb}{0.172549,0.627451,0.172549}%
\pgfsetfillcolor{currentfill}%
\pgfsetlinewidth{0.481800pt}%
\definecolor{currentstroke}{rgb}{1.000000,1.000000,1.000000}%
\pgfsetstrokecolor{currentstroke}%
\pgfsetdash{}{0pt}%
\pgfpathmoveto{\pgfqpoint{9.038978in}{5.914713in}}%
\pgfpathcurveto{\pgfqpoint{9.050028in}{5.914713in}}{\pgfqpoint{9.060627in}{5.919103in}}{\pgfqpoint{9.068440in}{5.926917in}}%
\pgfpathcurveto{\pgfqpoint{9.076254in}{5.934730in}}{\pgfqpoint{9.080644in}{5.945329in}}{\pgfqpoint{9.080644in}{5.956379in}}%
\pgfpathcurveto{\pgfqpoint{9.080644in}{5.967430in}}{\pgfqpoint{9.076254in}{5.978029in}}{\pgfqpoint{9.068440in}{5.985842in}}%
\pgfpathcurveto{\pgfqpoint{9.060627in}{5.993656in}}{\pgfqpoint{9.050028in}{5.998046in}}{\pgfqpoint{9.038978in}{5.998046in}}%
\pgfpathcurveto{\pgfqpoint{9.027927in}{5.998046in}}{\pgfqpoint{9.017328in}{5.993656in}}{\pgfqpoint{9.009515in}{5.985842in}}%
\pgfpathcurveto{\pgfqpoint{9.001701in}{5.978029in}}{\pgfqpoint{8.997311in}{5.967430in}}{\pgfqpoint{8.997311in}{5.956379in}}%
\pgfpathcurveto{\pgfqpoint{8.997311in}{5.945329in}}{\pgfqpoint{9.001701in}{5.934730in}}{\pgfqpoint{9.009515in}{5.926917in}}%
\pgfpathcurveto{\pgfqpoint{9.017328in}{5.919103in}}{\pgfqpoint{9.027927in}{5.914713in}}{\pgfqpoint{9.038978in}{5.914713in}}%
\pgfpathlineto{\pgfqpoint{9.038978in}{5.914713in}}%
\pgfpathclose%
\pgfusepath{stroke,fill}%
\end{pgfscope}%
\begin{pgfscope}%
\pgfpathrectangle{\pgfqpoint{7.622482in}{5.272501in}}{\pgfqpoint{2.177280in}{2.201755in}}%
\pgfusepath{clip}%
\pgfsetbuttcap%
\pgfsetroundjoin%
\definecolor{currentfill}{rgb}{0.172549,0.627451,0.172549}%
\pgfsetfillcolor{currentfill}%
\pgfsetlinewidth{0.481800pt}%
\definecolor{currentstroke}{rgb}{1.000000,1.000000,1.000000}%
\pgfsetstrokecolor{currentstroke}%
\pgfsetdash{}{0pt}%
\pgfpathmoveto{\pgfqpoint{9.309993in}{6.331712in}}%
\pgfpathcurveto{\pgfqpoint{9.321043in}{6.331712in}}{\pgfqpoint{9.331642in}{6.336102in}}{\pgfqpoint{9.339456in}{6.343916in}}%
\pgfpathcurveto{\pgfqpoint{9.347269in}{6.351729in}}{\pgfqpoint{9.351660in}{6.362328in}}{\pgfqpoint{9.351660in}{6.373379in}}%
\pgfpathcurveto{\pgfqpoint{9.351660in}{6.384429in}}{\pgfqpoint{9.347269in}{6.395028in}}{\pgfqpoint{9.339456in}{6.402841in}}%
\pgfpathcurveto{\pgfqpoint{9.331642in}{6.410655in}}{\pgfqpoint{9.321043in}{6.415045in}}{\pgfqpoint{9.309993in}{6.415045in}}%
\pgfpathcurveto{\pgfqpoint{9.298943in}{6.415045in}}{\pgfqpoint{9.288344in}{6.410655in}}{\pgfqpoint{9.280530in}{6.402841in}}%
\pgfpathcurveto{\pgfqpoint{9.272717in}{6.395028in}}{\pgfqpoint{9.268326in}{6.384429in}}{\pgfqpoint{9.268326in}{6.373379in}}%
\pgfpathcurveto{\pgfqpoint{9.268326in}{6.362328in}}{\pgfqpoint{9.272717in}{6.351729in}}{\pgfqpoint{9.280530in}{6.343916in}}%
\pgfpathcurveto{\pgfqpoint{9.288344in}{6.336102in}}{\pgfqpoint{9.298943in}{6.331712in}}{\pgfqpoint{9.309993in}{6.331712in}}%
\pgfpathlineto{\pgfqpoint{9.309993in}{6.331712in}}%
\pgfpathclose%
\pgfusepath{stroke,fill}%
\end{pgfscope}%
\begin{pgfscope}%
\pgfpathrectangle{\pgfqpoint{7.622482in}{5.272501in}}{\pgfqpoint{2.177280in}{2.201755in}}%
\pgfusepath{clip}%
\pgfsetbuttcap%
\pgfsetroundjoin%
\definecolor{currentfill}{rgb}{0.172549,0.627451,0.172549}%
\pgfsetfillcolor{currentfill}%
\pgfsetlinewidth{0.481800pt}%
\definecolor{currentstroke}{rgb}{1.000000,1.000000,1.000000}%
\pgfsetstrokecolor{currentstroke}%
\pgfsetdash{}{0pt}%
\pgfpathmoveto{\pgfqpoint{9.445501in}{6.415112in}}%
\pgfpathcurveto{\pgfqpoint{9.456551in}{6.415112in}}{\pgfqpoint{9.467150in}{6.419502in}}{\pgfqpoint{9.474964in}{6.427316in}}%
\pgfpathcurveto{\pgfqpoint{9.482777in}{6.435129in}}{\pgfqpoint{9.487167in}{6.445728in}}{\pgfqpoint{9.487167in}{6.456778in}}%
\pgfpathcurveto{\pgfqpoint{9.487167in}{6.467828in}}{\pgfqpoint{9.482777in}{6.478428in}}{\pgfqpoint{9.474964in}{6.486241in}}%
\pgfpathcurveto{\pgfqpoint{9.467150in}{6.494055in}}{\pgfqpoint{9.456551in}{6.498445in}}{\pgfqpoint{9.445501in}{6.498445in}}%
\pgfpathcurveto{\pgfqpoint{9.434451in}{6.498445in}}{\pgfqpoint{9.423852in}{6.494055in}}{\pgfqpoint{9.416038in}{6.486241in}}%
\pgfpathcurveto{\pgfqpoint{9.408224in}{6.478428in}}{\pgfqpoint{9.403834in}{6.467828in}}{\pgfqpoint{9.403834in}{6.456778in}}%
\pgfpathcurveto{\pgfqpoint{9.403834in}{6.445728in}}{\pgfqpoint{9.408224in}{6.435129in}}{\pgfqpoint{9.416038in}{6.427316in}}%
\pgfpathcurveto{\pgfqpoint{9.423852in}{6.419502in}}{\pgfqpoint{9.434451in}{6.415112in}}{\pgfqpoint{9.445501in}{6.415112in}}%
\pgfpathlineto{\pgfqpoint{9.445501in}{6.415112in}}%
\pgfpathclose%
\pgfusepath{stroke,fill}%
\end{pgfscope}%
\begin{pgfscope}%
\pgfpathrectangle{\pgfqpoint{7.622482in}{5.272501in}}{\pgfqpoint{2.177280in}{2.201755in}}%
\pgfusepath{clip}%
\pgfsetbuttcap%
\pgfsetroundjoin%
\definecolor{currentfill}{rgb}{0.172549,0.627451,0.172549}%
\pgfsetfillcolor{currentfill}%
\pgfsetlinewidth{0.481800pt}%
\definecolor{currentstroke}{rgb}{1.000000,1.000000,1.000000}%
\pgfsetstrokecolor{currentstroke}%
\pgfsetdash{}{0pt}%
\pgfpathmoveto{\pgfqpoint{9.309993in}{6.164912in}}%
\pgfpathcurveto{\pgfqpoint{9.321043in}{6.164912in}}{\pgfqpoint{9.331642in}{6.169303in}}{\pgfqpoint{9.339456in}{6.177116in}}%
\pgfpathcurveto{\pgfqpoint{9.347269in}{6.184930in}}{\pgfqpoint{9.351660in}{6.195529in}}{\pgfqpoint{9.351660in}{6.206579in}}%
\pgfpathcurveto{\pgfqpoint{9.351660in}{6.217629in}}{\pgfqpoint{9.347269in}{6.228228in}}{\pgfqpoint{9.339456in}{6.236042in}}%
\pgfpathcurveto{\pgfqpoint{9.331642in}{6.243855in}}{\pgfqpoint{9.321043in}{6.248246in}}{\pgfqpoint{9.309993in}{6.248246in}}%
\pgfpathcurveto{\pgfqpoint{9.298943in}{6.248246in}}{\pgfqpoint{9.288344in}{6.243855in}}{\pgfqpoint{9.280530in}{6.236042in}}%
\pgfpathcurveto{\pgfqpoint{9.272717in}{6.228228in}}{\pgfqpoint{9.268326in}{6.217629in}}{\pgfqpoint{9.268326in}{6.206579in}}%
\pgfpathcurveto{\pgfqpoint{9.268326in}{6.195529in}}{\pgfqpoint{9.272717in}{6.184930in}}{\pgfqpoint{9.280530in}{6.177116in}}%
\pgfpathcurveto{\pgfqpoint{9.288344in}{6.169303in}}{\pgfqpoint{9.298943in}{6.164912in}}{\pgfqpoint{9.309993in}{6.164912in}}%
\pgfpathlineto{\pgfqpoint{9.309993in}{6.164912in}}%
\pgfpathclose%
\pgfusepath{stroke,fill}%
\end{pgfscope}%
\begin{pgfscope}%
\pgfpathrectangle{\pgfqpoint{7.622482in}{5.272501in}}{\pgfqpoint{2.177280in}{2.201755in}}%
\pgfusepath{clip}%
\pgfsetbuttcap%
\pgfsetroundjoin%
\definecolor{currentfill}{rgb}{0.172549,0.627451,0.172549}%
\pgfsetfillcolor{currentfill}%
\pgfsetlinewidth{0.481800pt}%
\definecolor{currentstroke}{rgb}{1.000000,1.000000,1.000000}%
\pgfsetstrokecolor{currentstroke}%
\pgfsetdash{}{0pt}%
\pgfpathmoveto{\pgfqpoint{9.038978in}{5.747913in}}%
\pgfpathcurveto{\pgfqpoint{9.050028in}{5.747913in}}{\pgfqpoint{9.060627in}{5.752303in}}{\pgfqpoint{9.068440in}{5.760117in}}%
\pgfpathcurveto{\pgfqpoint{9.076254in}{5.767931in}}{\pgfqpoint{9.080644in}{5.778530in}}{\pgfqpoint{9.080644in}{5.789580in}}%
\pgfpathcurveto{\pgfqpoint{9.080644in}{5.800630in}}{\pgfqpoint{9.076254in}{5.811229in}}{\pgfqpoint{9.068440in}{5.819043in}}%
\pgfpathcurveto{\pgfqpoint{9.060627in}{5.826856in}}{\pgfqpoint{9.050028in}{5.831247in}}{\pgfqpoint{9.038978in}{5.831247in}}%
\pgfpathcurveto{\pgfqpoint{9.027927in}{5.831247in}}{\pgfqpoint{9.017328in}{5.826856in}}{\pgfqpoint{9.009515in}{5.819043in}}%
\pgfpathcurveto{\pgfqpoint{9.001701in}{5.811229in}}{\pgfqpoint{8.997311in}{5.800630in}}{\pgfqpoint{8.997311in}{5.789580in}}%
\pgfpathcurveto{\pgfqpoint{8.997311in}{5.778530in}}{\pgfqpoint{9.001701in}{5.767931in}}{\pgfqpoint{9.009515in}{5.760117in}}%
\pgfpathcurveto{\pgfqpoint{9.017328in}{5.752303in}}{\pgfqpoint{9.027927in}{5.747913in}}{\pgfqpoint{9.038978in}{5.747913in}}%
\pgfpathlineto{\pgfqpoint{9.038978in}{5.747913in}}%
\pgfpathclose%
\pgfusepath{stroke,fill}%
\end{pgfscope}%
\begin{pgfscope}%
\pgfpathrectangle{\pgfqpoint{7.622482in}{5.272501in}}{\pgfqpoint{2.177280in}{2.201755in}}%
\pgfusepath{clip}%
\pgfsetbuttcap%
\pgfsetroundjoin%
\definecolor{currentfill}{rgb}{0.172549,0.627451,0.172549}%
\pgfsetfillcolor{currentfill}%
\pgfsetlinewidth{0.481800pt}%
\definecolor{currentstroke}{rgb}{1.000000,1.000000,1.000000}%
\pgfsetstrokecolor{currentstroke}%
\pgfsetdash{}{0pt}%
\pgfpathmoveto{\pgfqpoint{9.106731in}{6.164912in}}%
\pgfpathcurveto{\pgfqpoint{9.117782in}{6.164912in}}{\pgfqpoint{9.128381in}{6.169303in}}{\pgfqpoint{9.136194in}{6.177116in}}%
\pgfpathcurveto{\pgfqpoint{9.144008in}{6.184930in}}{\pgfqpoint{9.148398in}{6.195529in}}{\pgfqpoint{9.148398in}{6.206579in}}%
\pgfpathcurveto{\pgfqpoint{9.148398in}{6.217629in}}{\pgfqpoint{9.144008in}{6.228228in}}{\pgfqpoint{9.136194in}{6.236042in}}%
\pgfpathcurveto{\pgfqpoint{9.128381in}{6.243855in}}{\pgfqpoint{9.117782in}{6.248246in}}{\pgfqpoint{9.106731in}{6.248246in}}%
\pgfpathcurveto{\pgfqpoint{9.095681in}{6.248246in}}{\pgfqpoint{9.085082in}{6.243855in}}{\pgfqpoint{9.077269in}{6.236042in}}%
\pgfpathcurveto{\pgfqpoint{9.069455in}{6.228228in}}{\pgfqpoint{9.065065in}{6.217629in}}{\pgfqpoint{9.065065in}{6.206579in}}%
\pgfpathcurveto{\pgfqpoint{9.065065in}{6.195529in}}{\pgfqpoint{9.069455in}{6.184930in}}{\pgfqpoint{9.077269in}{6.177116in}}%
\pgfpathcurveto{\pgfqpoint{9.085082in}{6.169303in}}{\pgfqpoint{9.095681in}{6.164912in}}{\pgfqpoint{9.106731in}{6.164912in}}%
\pgfpathlineto{\pgfqpoint{9.106731in}{6.164912in}}%
\pgfpathclose%
\pgfusepath{stroke,fill}%
\end{pgfscope}%
\begin{pgfscope}%
\pgfpathrectangle{\pgfqpoint{7.622482in}{5.272501in}}{\pgfqpoint{2.177280in}{2.201755in}}%
\pgfusepath{clip}%
\pgfsetbuttcap%
\pgfsetroundjoin%
\definecolor{currentfill}{rgb}{0.172549,0.627451,0.172549}%
\pgfsetfillcolor{currentfill}%
\pgfsetlinewidth{0.481800pt}%
\definecolor{currentstroke}{rgb}{1.000000,1.000000,1.000000}%
\pgfsetstrokecolor{currentstroke}%
\pgfsetdash{}{0pt}%
\pgfpathmoveto{\pgfqpoint{9.309993in}{6.498512in}}%
\pgfpathcurveto{\pgfqpoint{9.321043in}{6.498512in}}{\pgfqpoint{9.331642in}{6.502902in}}{\pgfqpoint{9.339456in}{6.510715in}}%
\pgfpathcurveto{\pgfqpoint{9.347269in}{6.518529in}}{\pgfqpoint{9.351660in}{6.529128in}}{\pgfqpoint{9.351660in}{6.540178in}}%
\pgfpathcurveto{\pgfqpoint{9.351660in}{6.551228in}}{\pgfqpoint{9.347269in}{6.561827in}}{\pgfqpoint{9.339456in}{6.569641in}}%
\pgfpathcurveto{\pgfqpoint{9.331642in}{6.577455in}}{\pgfqpoint{9.321043in}{6.581845in}}{\pgfqpoint{9.309993in}{6.581845in}}%
\pgfpathcurveto{\pgfqpoint{9.298943in}{6.581845in}}{\pgfqpoint{9.288344in}{6.577455in}}{\pgfqpoint{9.280530in}{6.569641in}}%
\pgfpathcurveto{\pgfqpoint{9.272717in}{6.561827in}}{\pgfqpoint{9.268326in}{6.551228in}}{\pgfqpoint{9.268326in}{6.540178in}}%
\pgfpathcurveto{\pgfqpoint{9.268326in}{6.529128in}}{\pgfqpoint{9.272717in}{6.518529in}}{\pgfqpoint{9.280530in}{6.510715in}}%
\pgfpathcurveto{\pgfqpoint{9.288344in}{6.502902in}}{\pgfqpoint{9.298943in}{6.498512in}}{\pgfqpoint{9.309993in}{6.498512in}}%
\pgfpathlineto{\pgfqpoint{9.309993in}{6.498512in}}%
\pgfpathclose%
\pgfusepath{stroke,fill}%
\end{pgfscope}%
\begin{pgfscope}%
\pgfpathrectangle{\pgfqpoint{7.622482in}{5.272501in}}{\pgfqpoint{2.177280in}{2.201755in}}%
\pgfusepath{clip}%
\pgfsetbuttcap%
\pgfsetroundjoin%
\definecolor{currentfill}{rgb}{0.172549,0.627451,0.172549}%
\pgfsetfillcolor{currentfill}%
\pgfsetlinewidth{0.481800pt}%
\definecolor{currentstroke}{rgb}{1.000000,1.000000,1.000000}%
\pgfsetstrokecolor{currentstroke}%
\pgfsetdash{}{0pt}%
\pgfpathmoveto{\pgfqpoint{8.971224in}{6.164912in}}%
\pgfpathcurveto{\pgfqpoint{8.982274in}{6.164912in}}{\pgfqpoint{8.992873in}{6.169303in}}{\pgfqpoint{9.000686in}{6.177116in}}%
\pgfpathcurveto{\pgfqpoint{9.008500in}{6.184930in}}{\pgfqpoint{9.012890in}{6.195529in}}{\pgfqpoint{9.012890in}{6.206579in}}%
\pgfpathcurveto{\pgfqpoint{9.012890in}{6.217629in}}{\pgfqpoint{9.008500in}{6.228228in}}{\pgfqpoint{9.000686in}{6.236042in}}%
\pgfpathcurveto{\pgfqpoint{8.992873in}{6.243855in}}{\pgfqpoint{8.982274in}{6.248246in}}{\pgfqpoint{8.971224in}{6.248246in}}%
\pgfpathcurveto{\pgfqpoint{8.960174in}{6.248246in}}{\pgfqpoint{8.949575in}{6.243855in}}{\pgfqpoint{8.941761in}{6.236042in}}%
\pgfpathcurveto{\pgfqpoint{8.933947in}{6.228228in}}{\pgfqpoint{8.929557in}{6.217629in}}{\pgfqpoint{8.929557in}{6.206579in}}%
\pgfpathcurveto{\pgfqpoint{8.929557in}{6.195529in}}{\pgfqpoint{8.933947in}{6.184930in}}{\pgfqpoint{8.941761in}{6.177116in}}%
\pgfpathcurveto{\pgfqpoint{8.949575in}{6.169303in}}{\pgfqpoint{8.960174in}{6.164912in}}{\pgfqpoint{8.971224in}{6.164912in}}%
\pgfpathlineto{\pgfqpoint{8.971224in}{6.164912in}}%
\pgfpathclose%
\pgfusepath{stroke,fill}%
\end{pgfscope}%
\begin{pgfscope}%
\pgfpathrectangle{\pgfqpoint{7.622482in}{5.272501in}}{\pgfqpoint{2.177280in}{2.201755in}}%
\pgfusepath{clip}%
\pgfsetbuttcap%
\pgfsetroundjoin%
\definecolor{currentfill}{rgb}{0.121569,0.466667,0.705882}%
\pgfsetfillcolor{currentfill}%
\pgfsetlinewidth{1.003750pt}%
\definecolor{currentstroke}{rgb}{0.121569,0.466667,0.705882}%
\pgfsetstrokecolor{currentstroke}%
\pgfsetdash{}{0pt}%
\pgfsys@defobject{currentmarker}{\pgfqpoint{-0.041667in}{-0.041667in}}{\pgfqpoint{0.041667in}{0.041667in}}{%
\pgfpathmoveto{\pgfqpoint{0.000000in}{-0.041667in}}%
\pgfpathcurveto{\pgfqpoint{0.011050in}{-0.041667in}}{\pgfqpoint{0.021649in}{-0.037276in}}{\pgfqpoint{0.029463in}{-0.029463in}}%
\pgfpathcurveto{\pgfqpoint{0.037276in}{-0.021649in}}{\pgfqpoint{0.041667in}{-0.011050in}}{\pgfqpoint{0.041667in}{0.000000in}}%
\pgfpathcurveto{\pgfqpoint{0.041667in}{0.011050in}}{\pgfqpoint{0.037276in}{0.021649in}}{\pgfqpoint{0.029463in}{0.029463in}}%
\pgfpathcurveto{\pgfqpoint{0.021649in}{0.037276in}}{\pgfqpoint{0.011050in}{0.041667in}}{\pgfqpoint{0.000000in}{0.041667in}}%
\pgfpathcurveto{\pgfqpoint{-0.011050in}{0.041667in}}{\pgfqpoint{-0.021649in}{0.037276in}}{\pgfqpoint{-0.029463in}{0.029463in}}%
\pgfpathcurveto{\pgfqpoint{-0.037276in}{0.021649in}}{\pgfqpoint{-0.041667in}{0.011050in}}{\pgfqpoint{-0.041667in}{0.000000in}}%
\pgfpathcurveto{\pgfqpoint{-0.041667in}{-0.011050in}}{\pgfqpoint{-0.037276in}{-0.021649in}}{\pgfqpoint{-0.029463in}{-0.029463in}}%
\pgfpathcurveto{\pgfqpoint{-0.021649in}{-0.037276in}}{\pgfqpoint{-0.011050in}{-0.041667in}}{\pgfqpoint{0.000000in}{-0.041667in}}%
\pgfpathlineto{\pgfqpoint{0.000000in}{-0.041667in}}%
\pgfpathclose%
\pgfusepath{stroke,fill}%
}%
\end{pgfscope}%
\begin{pgfscope}%
\pgfpathrectangle{\pgfqpoint{7.622482in}{5.272501in}}{\pgfqpoint{2.177280in}{2.201755in}}%
\pgfusepath{clip}%
\pgfsetbuttcap%
\pgfsetroundjoin%
\definecolor{currentfill}{rgb}{1.000000,0.498039,0.054902}%
\pgfsetfillcolor{currentfill}%
\pgfsetlinewidth{1.003750pt}%
\definecolor{currentstroke}{rgb}{1.000000,0.498039,0.054902}%
\pgfsetstrokecolor{currentstroke}%
\pgfsetdash{}{0pt}%
\pgfsys@defobject{currentmarker}{\pgfqpoint{-0.041667in}{-0.041667in}}{\pgfqpoint{0.041667in}{0.041667in}}{%
\pgfpathmoveto{\pgfqpoint{0.000000in}{-0.041667in}}%
\pgfpathcurveto{\pgfqpoint{0.011050in}{-0.041667in}}{\pgfqpoint{0.021649in}{-0.037276in}}{\pgfqpoint{0.029463in}{-0.029463in}}%
\pgfpathcurveto{\pgfqpoint{0.037276in}{-0.021649in}}{\pgfqpoint{0.041667in}{-0.011050in}}{\pgfqpoint{0.041667in}{0.000000in}}%
\pgfpathcurveto{\pgfqpoint{0.041667in}{0.011050in}}{\pgfqpoint{0.037276in}{0.021649in}}{\pgfqpoint{0.029463in}{0.029463in}}%
\pgfpathcurveto{\pgfqpoint{0.021649in}{0.037276in}}{\pgfqpoint{0.011050in}{0.041667in}}{\pgfqpoint{0.000000in}{0.041667in}}%
\pgfpathcurveto{\pgfqpoint{-0.011050in}{0.041667in}}{\pgfqpoint{-0.021649in}{0.037276in}}{\pgfqpoint{-0.029463in}{0.029463in}}%
\pgfpathcurveto{\pgfqpoint{-0.037276in}{0.021649in}}{\pgfqpoint{-0.041667in}{0.011050in}}{\pgfqpoint{-0.041667in}{0.000000in}}%
\pgfpathcurveto{\pgfqpoint{-0.041667in}{-0.011050in}}{\pgfqpoint{-0.037276in}{-0.021649in}}{\pgfqpoint{-0.029463in}{-0.029463in}}%
\pgfpathcurveto{\pgfqpoint{-0.021649in}{-0.037276in}}{\pgfqpoint{-0.011050in}{-0.041667in}}{\pgfqpoint{0.000000in}{-0.041667in}}%
\pgfpathlineto{\pgfqpoint{0.000000in}{-0.041667in}}%
\pgfpathclose%
\pgfusepath{stroke,fill}%
}%
\end{pgfscope}%
\begin{pgfscope}%
\pgfpathrectangle{\pgfqpoint{7.622482in}{5.272501in}}{\pgfqpoint{2.177280in}{2.201755in}}%
\pgfusepath{clip}%
\pgfsetbuttcap%
\pgfsetroundjoin%
\definecolor{currentfill}{rgb}{0.172549,0.627451,0.172549}%
\pgfsetfillcolor{currentfill}%
\pgfsetlinewidth{1.003750pt}%
\definecolor{currentstroke}{rgb}{0.172549,0.627451,0.172549}%
\pgfsetstrokecolor{currentstroke}%
\pgfsetdash{}{0pt}%
\pgfsys@defobject{currentmarker}{\pgfqpoint{-0.041667in}{-0.041667in}}{\pgfqpoint{0.041667in}{0.041667in}}{%
\pgfpathmoveto{\pgfqpoint{0.000000in}{-0.041667in}}%
\pgfpathcurveto{\pgfqpoint{0.011050in}{-0.041667in}}{\pgfqpoint{0.021649in}{-0.037276in}}{\pgfqpoint{0.029463in}{-0.029463in}}%
\pgfpathcurveto{\pgfqpoint{0.037276in}{-0.021649in}}{\pgfqpoint{0.041667in}{-0.011050in}}{\pgfqpoint{0.041667in}{0.000000in}}%
\pgfpathcurveto{\pgfqpoint{0.041667in}{0.011050in}}{\pgfqpoint{0.037276in}{0.021649in}}{\pgfqpoint{0.029463in}{0.029463in}}%
\pgfpathcurveto{\pgfqpoint{0.021649in}{0.037276in}}{\pgfqpoint{0.011050in}{0.041667in}}{\pgfqpoint{0.000000in}{0.041667in}}%
\pgfpathcurveto{\pgfqpoint{-0.011050in}{0.041667in}}{\pgfqpoint{-0.021649in}{0.037276in}}{\pgfqpoint{-0.029463in}{0.029463in}}%
\pgfpathcurveto{\pgfqpoint{-0.037276in}{0.021649in}}{\pgfqpoint{-0.041667in}{0.011050in}}{\pgfqpoint{-0.041667in}{0.000000in}}%
\pgfpathcurveto{\pgfqpoint{-0.041667in}{-0.011050in}}{\pgfqpoint{-0.037276in}{-0.021649in}}{\pgfqpoint{-0.029463in}{-0.029463in}}%
\pgfpathcurveto{\pgfqpoint{-0.021649in}{-0.037276in}}{\pgfqpoint{-0.011050in}{-0.041667in}}{\pgfqpoint{0.000000in}{-0.041667in}}%
\pgfpathlineto{\pgfqpoint{0.000000in}{-0.041667in}}%
\pgfpathclose%
\pgfusepath{stroke,fill}%
}%
\end{pgfscope}%
\begin{pgfscope}%
\pgfsetbuttcap%
\pgfsetroundjoin%
\definecolor{currentfill}{rgb}{0.000000,0.000000,0.000000}%
\pgfsetfillcolor{currentfill}%
\pgfsetlinewidth{0.803000pt}%
\definecolor{currentstroke}{rgb}{0.000000,0.000000,0.000000}%
\pgfsetstrokecolor{currentstroke}%
\pgfsetdash{}{0pt}%
\pgfsys@defobject{currentmarker}{\pgfqpoint{0.000000in}{-0.048611in}}{\pgfqpoint{0.000000in}{0.000000in}}{%
\pgfpathmoveto{\pgfqpoint{0.000000in}{0.000000in}}%
\pgfpathlineto{\pgfqpoint{0.000000in}{-0.048611in}}%
\pgfusepath{stroke,fill}%
}%
\begin{pgfscope}%
\pgfsys@transformshift{7.751654in}{5.272501in}%
\pgfsys@useobject{currentmarker}{}%
\end{pgfscope}%
\end{pgfscope}%
\begin{pgfscope}%
\pgfsetbuttcap%
\pgfsetroundjoin%
\definecolor{currentfill}{rgb}{0.000000,0.000000,0.000000}%
\pgfsetfillcolor{currentfill}%
\pgfsetlinewidth{0.803000pt}%
\definecolor{currentstroke}{rgb}{0.000000,0.000000,0.000000}%
\pgfsetstrokecolor{currentstroke}%
\pgfsetdash{}{0pt}%
\pgfsys@defobject{currentmarker}{\pgfqpoint{0.000000in}{-0.048611in}}{\pgfqpoint{0.000000in}{0.000000in}}{%
\pgfpathmoveto{\pgfqpoint{0.000000in}{0.000000in}}%
\pgfpathlineto{\pgfqpoint{0.000000in}{-0.048611in}}%
\pgfusepath{stroke,fill}%
}%
\begin{pgfscope}%
\pgfsys@transformshift{8.429193in}{5.272501in}%
\pgfsys@useobject{currentmarker}{}%
\end{pgfscope}%
\end{pgfscope}%
\begin{pgfscope}%
\pgfsetbuttcap%
\pgfsetroundjoin%
\definecolor{currentfill}{rgb}{0.000000,0.000000,0.000000}%
\pgfsetfillcolor{currentfill}%
\pgfsetlinewidth{0.803000pt}%
\definecolor{currentstroke}{rgb}{0.000000,0.000000,0.000000}%
\pgfsetstrokecolor{currentstroke}%
\pgfsetdash{}{0pt}%
\pgfsys@defobject{currentmarker}{\pgfqpoint{0.000000in}{-0.048611in}}{\pgfqpoint{0.000000in}{0.000000in}}{%
\pgfpathmoveto{\pgfqpoint{0.000000in}{0.000000in}}%
\pgfpathlineto{\pgfqpoint{0.000000in}{-0.048611in}}%
\pgfusepath{stroke,fill}%
}%
\begin{pgfscope}%
\pgfsys@transformshift{9.106731in}{5.272501in}%
\pgfsys@useobject{currentmarker}{}%
\end{pgfscope}%
\end{pgfscope}%
\begin{pgfscope}%
\pgfsetbuttcap%
\pgfsetroundjoin%
\definecolor{currentfill}{rgb}{0.000000,0.000000,0.000000}%
\pgfsetfillcolor{currentfill}%
\pgfsetlinewidth{0.803000pt}%
\definecolor{currentstroke}{rgb}{0.000000,0.000000,0.000000}%
\pgfsetstrokecolor{currentstroke}%
\pgfsetdash{}{0pt}%
\pgfsys@defobject{currentmarker}{\pgfqpoint{0.000000in}{-0.048611in}}{\pgfqpoint{0.000000in}{0.000000in}}{%
\pgfpathmoveto{\pgfqpoint{0.000000in}{0.000000in}}%
\pgfpathlineto{\pgfqpoint{0.000000in}{-0.048611in}}%
\pgfusepath{stroke,fill}%
}%
\begin{pgfscope}%
\pgfsys@transformshift{9.784270in}{5.272501in}%
\pgfsys@useobject{currentmarker}{}%
\end{pgfscope}%
\end{pgfscope}%
\begin{pgfscope}%
\pgfsetbuttcap%
\pgfsetroundjoin%
\definecolor{currentfill}{rgb}{0.000000,0.000000,0.000000}%
\pgfsetfillcolor{currentfill}%
\pgfsetlinewidth{0.803000pt}%
\definecolor{currentstroke}{rgb}{0.000000,0.000000,0.000000}%
\pgfsetstrokecolor{currentstroke}%
\pgfsetdash{}{0pt}%
\pgfsys@defobject{currentmarker}{\pgfqpoint{-0.048611in}{0.000000in}}{\pgfqpoint{-0.000000in}{0.000000in}}{%
\pgfpathmoveto{\pgfqpoint{-0.000000in}{0.000000in}}%
\pgfpathlineto{\pgfqpoint{-0.048611in}{0.000000in}}%
\pgfusepath{stroke,fill}%
}%
\begin{pgfscope}%
\pgfsys@transformshift{7.622482in}{5.372581in}%
\pgfsys@useobject{currentmarker}{}%
\end{pgfscope}%
\end{pgfscope}%
\begin{pgfscope}%
\pgfsetbuttcap%
\pgfsetroundjoin%
\definecolor{currentfill}{rgb}{0.000000,0.000000,0.000000}%
\pgfsetfillcolor{currentfill}%
\pgfsetlinewidth{0.803000pt}%
\definecolor{currentstroke}{rgb}{0.000000,0.000000,0.000000}%
\pgfsetstrokecolor{currentstroke}%
\pgfsetdash{}{0pt}%
\pgfsys@defobject{currentmarker}{\pgfqpoint{-0.048611in}{0.000000in}}{\pgfqpoint{-0.000000in}{0.000000in}}{%
\pgfpathmoveto{\pgfqpoint{-0.000000in}{0.000000in}}%
\pgfpathlineto{\pgfqpoint{-0.048611in}{0.000000in}}%
\pgfusepath{stroke,fill}%
}%
\begin{pgfscope}%
\pgfsys@transformshift{7.622482in}{5.789580in}%
\pgfsys@useobject{currentmarker}{}%
\end{pgfscope}%
\end{pgfscope}%
\begin{pgfscope}%
\pgfsetbuttcap%
\pgfsetroundjoin%
\definecolor{currentfill}{rgb}{0.000000,0.000000,0.000000}%
\pgfsetfillcolor{currentfill}%
\pgfsetlinewidth{0.803000pt}%
\definecolor{currentstroke}{rgb}{0.000000,0.000000,0.000000}%
\pgfsetstrokecolor{currentstroke}%
\pgfsetdash{}{0pt}%
\pgfsys@defobject{currentmarker}{\pgfqpoint{-0.048611in}{0.000000in}}{\pgfqpoint{-0.000000in}{0.000000in}}{%
\pgfpathmoveto{\pgfqpoint{-0.000000in}{0.000000in}}%
\pgfpathlineto{\pgfqpoint{-0.048611in}{0.000000in}}%
\pgfusepath{stroke,fill}%
}%
\begin{pgfscope}%
\pgfsys@transformshift{7.622482in}{6.206579in}%
\pgfsys@useobject{currentmarker}{}%
\end{pgfscope}%
\end{pgfscope}%
\begin{pgfscope}%
\pgfsetbuttcap%
\pgfsetroundjoin%
\definecolor{currentfill}{rgb}{0.000000,0.000000,0.000000}%
\pgfsetfillcolor{currentfill}%
\pgfsetlinewidth{0.803000pt}%
\definecolor{currentstroke}{rgb}{0.000000,0.000000,0.000000}%
\pgfsetstrokecolor{currentstroke}%
\pgfsetdash{}{0pt}%
\pgfsys@defobject{currentmarker}{\pgfqpoint{-0.048611in}{0.000000in}}{\pgfqpoint{-0.000000in}{0.000000in}}{%
\pgfpathmoveto{\pgfqpoint{-0.000000in}{0.000000in}}%
\pgfpathlineto{\pgfqpoint{-0.048611in}{0.000000in}}%
\pgfusepath{stroke,fill}%
}%
\begin{pgfscope}%
\pgfsys@transformshift{7.622482in}{6.623578in}%
\pgfsys@useobject{currentmarker}{}%
\end{pgfscope}%
\end{pgfscope}%
\begin{pgfscope}%
\pgfsetbuttcap%
\pgfsetroundjoin%
\definecolor{currentfill}{rgb}{0.000000,0.000000,0.000000}%
\pgfsetfillcolor{currentfill}%
\pgfsetlinewidth{0.803000pt}%
\definecolor{currentstroke}{rgb}{0.000000,0.000000,0.000000}%
\pgfsetstrokecolor{currentstroke}%
\pgfsetdash{}{0pt}%
\pgfsys@defobject{currentmarker}{\pgfqpoint{-0.048611in}{0.000000in}}{\pgfqpoint{-0.000000in}{0.000000in}}{%
\pgfpathmoveto{\pgfqpoint{-0.000000in}{0.000000in}}%
\pgfpathlineto{\pgfqpoint{-0.048611in}{0.000000in}}%
\pgfusepath{stroke,fill}%
}%
\begin{pgfscope}%
\pgfsys@transformshift{7.622482in}{7.040577in}%
\pgfsys@useobject{currentmarker}{}%
\end{pgfscope}%
\end{pgfscope}%
\begin{pgfscope}%
\pgfsetbuttcap%
\pgfsetroundjoin%
\definecolor{currentfill}{rgb}{0.000000,0.000000,0.000000}%
\pgfsetfillcolor{currentfill}%
\pgfsetlinewidth{0.803000pt}%
\definecolor{currentstroke}{rgb}{0.000000,0.000000,0.000000}%
\pgfsetstrokecolor{currentstroke}%
\pgfsetdash{}{0pt}%
\pgfsys@defobject{currentmarker}{\pgfqpoint{-0.048611in}{0.000000in}}{\pgfqpoint{-0.000000in}{0.000000in}}{%
\pgfpathmoveto{\pgfqpoint{-0.000000in}{0.000000in}}%
\pgfpathlineto{\pgfqpoint{-0.048611in}{0.000000in}}%
\pgfusepath{stroke,fill}%
}%
\begin{pgfscope}%
\pgfsys@transformshift{7.622482in}{7.457576in}%
\pgfsys@useobject{currentmarker}{}%
\end{pgfscope}%
\end{pgfscope}%
\begin{pgfscope}%
\pgfsetrectcap%
\pgfsetmiterjoin%
\pgfsetlinewidth{0.803000pt}%
\definecolor{currentstroke}{rgb}{0.000000,0.000000,0.000000}%
\pgfsetstrokecolor{currentstroke}%
\pgfsetdash{}{0pt}%
\pgfpathmoveto{\pgfqpoint{7.622482in}{5.272501in}}%
\pgfpathlineto{\pgfqpoint{7.622482in}{7.474256in}}%
\pgfusepath{stroke}%
\end{pgfscope}%
\begin{pgfscope}%
\pgfsetrectcap%
\pgfsetmiterjoin%
\pgfsetlinewidth{0.803000pt}%
\definecolor{currentstroke}{rgb}{0.000000,0.000000,0.000000}%
\pgfsetstrokecolor{currentstroke}%
\pgfsetdash{}{0pt}%
\pgfpathmoveto{\pgfqpoint{7.622482in}{5.272501in}}%
\pgfpathlineto{\pgfqpoint{9.799762in}{5.272501in}}%
\pgfusepath{stroke}%
\end{pgfscope}%
\begin{pgfscope}%
\pgfsetbuttcap%
\pgfsetmiterjoin%
\definecolor{currentfill}{rgb}{1.000000,1.000000,1.000000}%
\pgfsetfillcolor{currentfill}%
\pgfsetlinewidth{0.000000pt}%
\definecolor{currentstroke}{rgb}{0.000000,0.000000,0.000000}%
\pgfsetstrokecolor{currentstroke}%
\pgfsetstrokeopacity{0.000000}%
\pgfsetdash{}{0pt}%
\pgfpathmoveto{\pgfqpoint{0.633874in}{2.920818in}}%
\pgfpathlineto{\pgfqpoint{2.811154in}{2.920818in}}%
\pgfpathlineto{\pgfqpoint{2.811154in}{5.122573in}}%
\pgfpathlineto{\pgfqpoint{0.633874in}{5.122573in}}%
\pgfpathlineto{\pgfqpoint{0.633874in}{2.920818in}}%
\pgfpathclose%
\pgfusepath{fill}%
\end{pgfscope}%
\begin{pgfscope}%
\pgfpathrectangle{\pgfqpoint{0.633874in}{2.920818in}}{\pgfqpoint{2.177280in}{2.201755in}}%
\pgfusepath{clip}%
\pgfsetbuttcap%
\pgfsetroundjoin%
\definecolor{currentfill}{rgb}{0.121569,0.466667,0.705882}%
\pgfsetfillcolor{currentfill}%
\pgfsetlinewidth{0.481800pt}%
\definecolor{currentstroke}{rgb}{1.000000,1.000000,1.000000}%
\pgfsetstrokecolor{currentstroke}%
\pgfsetdash{}{0pt}%
\pgfpathmoveto{\pgfqpoint{1.245489in}{3.114933in}}%
\pgfpathcurveto{\pgfqpoint{1.256539in}{3.114933in}}{\pgfqpoint{1.267138in}{3.119323in}}{\pgfqpoint{1.274952in}{3.127137in}}%
\pgfpathcurveto{\pgfqpoint{1.282765in}{3.134950in}}{\pgfqpoint{1.287155in}{3.145549in}}{\pgfqpoint{1.287155in}{3.156599in}}%
\pgfpathcurveto{\pgfqpoint{1.287155in}{3.167650in}}{\pgfqpoint{1.282765in}{3.178249in}}{\pgfqpoint{1.274952in}{3.186062in}}%
\pgfpathcurveto{\pgfqpoint{1.267138in}{3.193876in}}{\pgfqpoint{1.256539in}{3.198266in}}{\pgfqpoint{1.245489in}{3.198266in}}%
\pgfpathcurveto{\pgfqpoint{1.234439in}{3.198266in}}{\pgfqpoint{1.223840in}{3.193876in}}{\pgfqpoint{1.216026in}{3.186062in}}%
\pgfpathcurveto{\pgfqpoint{1.208212in}{3.178249in}}{\pgfqpoint{1.203822in}{3.167650in}}{\pgfqpoint{1.203822in}{3.156599in}}%
\pgfpathcurveto{\pgfqpoint{1.203822in}{3.145549in}}{\pgfqpoint{1.208212in}{3.134950in}}{\pgfqpoint{1.216026in}{3.127137in}}%
\pgfpathcurveto{\pgfqpoint{1.223840in}{3.119323in}}{\pgfqpoint{1.234439in}{3.114933in}}{\pgfqpoint{1.245489in}{3.114933in}}%
\pgfpathlineto{\pgfqpoint{1.245489in}{3.114933in}}%
\pgfpathclose%
\pgfusepath{stroke,fill}%
\end{pgfscope}%
\begin{pgfscope}%
\pgfpathrectangle{\pgfqpoint{0.633874in}{2.920818in}}{\pgfqpoint{2.177280in}{2.201755in}}%
\pgfusepath{clip}%
\pgfsetbuttcap%
\pgfsetroundjoin%
\definecolor{currentfill}{rgb}{0.121569,0.466667,0.705882}%
\pgfsetfillcolor{currentfill}%
\pgfsetlinewidth{0.481800pt}%
\definecolor{currentstroke}{rgb}{1.000000,1.000000,1.000000}%
\pgfsetstrokecolor{currentstroke}%
\pgfsetdash{}{0pt}%
\pgfpathmoveto{\pgfqpoint{1.165611in}{3.114933in}}%
\pgfpathcurveto{\pgfqpoint{1.176662in}{3.114933in}}{\pgfqpoint{1.187261in}{3.119323in}}{\pgfqpoint{1.195074in}{3.127137in}}%
\pgfpathcurveto{\pgfqpoint{1.202888in}{3.134950in}}{\pgfqpoint{1.207278in}{3.145549in}}{\pgfqpoint{1.207278in}{3.156599in}}%
\pgfpathcurveto{\pgfqpoint{1.207278in}{3.167650in}}{\pgfqpoint{1.202888in}{3.178249in}}{\pgfqpoint{1.195074in}{3.186062in}}%
\pgfpathcurveto{\pgfqpoint{1.187261in}{3.193876in}}{\pgfqpoint{1.176662in}{3.198266in}}{\pgfqpoint{1.165611in}{3.198266in}}%
\pgfpathcurveto{\pgfqpoint{1.154561in}{3.198266in}}{\pgfqpoint{1.143962in}{3.193876in}}{\pgfqpoint{1.136149in}{3.186062in}}%
\pgfpathcurveto{\pgfqpoint{1.128335in}{3.178249in}}{\pgfqpoint{1.123945in}{3.167650in}}{\pgfqpoint{1.123945in}{3.156599in}}%
\pgfpathcurveto{\pgfqpoint{1.123945in}{3.145549in}}{\pgfqpoint{1.128335in}{3.134950in}}{\pgfqpoint{1.136149in}{3.127137in}}%
\pgfpathcurveto{\pgfqpoint{1.143962in}{3.119323in}}{\pgfqpoint{1.154561in}{3.114933in}}{\pgfqpoint{1.165611in}{3.114933in}}%
\pgfpathlineto{\pgfqpoint{1.165611in}{3.114933in}}%
\pgfpathclose%
\pgfusepath{stroke,fill}%
\end{pgfscope}%
\begin{pgfscope}%
\pgfpathrectangle{\pgfqpoint{0.633874in}{2.920818in}}{\pgfqpoint{2.177280in}{2.201755in}}%
\pgfusepath{clip}%
\pgfsetbuttcap%
\pgfsetroundjoin%
\definecolor{currentfill}{rgb}{0.121569,0.466667,0.705882}%
\pgfsetfillcolor{currentfill}%
\pgfsetlinewidth{0.481800pt}%
\definecolor{currentstroke}{rgb}{1.000000,1.000000,1.000000}%
\pgfsetstrokecolor{currentstroke}%
\pgfsetdash{}{0pt}%
\pgfpathmoveto{\pgfqpoint{1.085734in}{3.081007in}}%
\pgfpathcurveto{\pgfqpoint{1.096784in}{3.081007in}}{\pgfqpoint{1.107383in}{3.085398in}}{\pgfqpoint{1.115197in}{3.093211in}}%
\pgfpathcurveto{\pgfqpoint{1.123010in}{3.101025in}}{\pgfqpoint{1.127401in}{3.111624in}}{\pgfqpoint{1.127401in}{3.122674in}}%
\pgfpathcurveto{\pgfqpoint{1.127401in}{3.133724in}}{\pgfqpoint{1.123010in}{3.144323in}}{\pgfqpoint{1.115197in}{3.152137in}}%
\pgfpathcurveto{\pgfqpoint{1.107383in}{3.159951in}}{\pgfqpoint{1.096784in}{3.164341in}}{\pgfqpoint{1.085734in}{3.164341in}}%
\pgfpathcurveto{\pgfqpoint{1.074684in}{3.164341in}}{\pgfqpoint{1.064085in}{3.159951in}}{\pgfqpoint{1.056271in}{3.152137in}}%
\pgfpathcurveto{\pgfqpoint{1.048458in}{3.144323in}}{\pgfqpoint{1.044067in}{3.133724in}}{\pgfqpoint{1.044067in}{3.122674in}}%
\pgfpathcurveto{\pgfqpoint{1.044067in}{3.111624in}}{\pgfqpoint{1.048458in}{3.101025in}}{\pgfqpoint{1.056271in}{3.093211in}}%
\pgfpathcurveto{\pgfqpoint{1.064085in}{3.085398in}}{\pgfqpoint{1.074684in}{3.081007in}}{\pgfqpoint{1.085734in}{3.081007in}}%
\pgfpathlineto{\pgfqpoint{1.085734in}{3.081007in}}%
\pgfpathclose%
\pgfusepath{stroke,fill}%
\end{pgfscope}%
\begin{pgfscope}%
\pgfpathrectangle{\pgfqpoint{0.633874in}{2.920818in}}{\pgfqpoint{2.177280in}{2.201755in}}%
\pgfusepath{clip}%
\pgfsetbuttcap%
\pgfsetroundjoin%
\definecolor{currentfill}{rgb}{0.121569,0.466667,0.705882}%
\pgfsetfillcolor{currentfill}%
\pgfsetlinewidth{0.481800pt}%
\definecolor{currentstroke}{rgb}{1.000000,1.000000,1.000000}%
\pgfsetstrokecolor{currentstroke}%
\pgfsetdash{}{0pt}%
\pgfpathmoveto{\pgfqpoint{1.045795in}{3.148858in}}%
\pgfpathcurveto{\pgfqpoint{1.056845in}{3.148858in}}{\pgfqpoint{1.067444in}{3.153248in}}{\pgfqpoint{1.075258in}{3.161062in}}%
\pgfpathcurveto{\pgfqpoint{1.083072in}{3.168876in}}{\pgfqpoint{1.087462in}{3.179475in}}{\pgfqpoint{1.087462in}{3.190525in}}%
\pgfpathcurveto{\pgfqpoint{1.087462in}{3.201575in}}{\pgfqpoint{1.083072in}{3.212174in}}{\pgfqpoint{1.075258in}{3.219988in}}%
\pgfpathcurveto{\pgfqpoint{1.067444in}{3.227801in}}{\pgfqpoint{1.056845in}{3.232191in}}{\pgfqpoint{1.045795in}{3.232191in}}%
\pgfpathcurveto{\pgfqpoint{1.034745in}{3.232191in}}{\pgfqpoint{1.024146in}{3.227801in}}{\pgfqpoint{1.016332in}{3.219988in}}%
\pgfpathcurveto{\pgfqpoint{1.008519in}{3.212174in}}{\pgfqpoint{1.004129in}{3.201575in}}{\pgfqpoint{1.004129in}{3.190525in}}%
\pgfpathcurveto{\pgfqpoint{1.004129in}{3.179475in}}{\pgfqpoint{1.008519in}{3.168876in}}{\pgfqpoint{1.016332in}{3.161062in}}%
\pgfpathcurveto{\pgfqpoint{1.024146in}{3.153248in}}{\pgfqpoint{1.034745in}{3.148858in}}{\pgfqpoint{1.045795in}{3.148858in}}%
\pgfpathlineto{\pgfqpoint{1.045795in}{3.148858in}}%
\pgfpathclose%
\pgfusepath{stroke,fill}%
\end{pgfscope}%
\begin{pgfscope}%
\pgfpathrectangle{\pgfqpoint{0.633874in}{2.920818in}}{\pgfqpoint{2.177280in}{2.201755in}}%
\pgfusepath{clip}%
\pgfsetbuttcap%
\pgfsetroundjoin%
\definecolor{currentfill}{rgb}{0.121569,0.466667,0.705882}%
\pgfsetfillcolor{currentfill}%
\pgfsetlinewidth{0.481800pt}%
\definecolor{currentstroke}{rgb}{1.000000,1.000000,1.000000}%
\pgfsetstrokecolor{currentstroke}%
\pgfsetdash{}{0pt}%
\pgfpathmoveto{\pgfqpoint{1.205550in}{3.114933in}}%
\pgfpathcurveto{\pgfqpoint{1.216600in}{3.114933in}}{\pgfqpoint{1.227199in}{3.119323in}}{\pgfqpoint{1.235013in}{3.127137in}}%
\pgfpathcurveto{\pgfqpoint{1.242826in}{3.134950in}}{\pgfqpoint{1.247217in}{3.145549in}}{\pgfqpoint{1.247217in}{3.156599in}}%
\pgfpathcurveto{\pgfqpoint{1.247217in}{3.167650in}}{\pgfqpoint{1.242826in}{3.178249in}}{\pgfqpoint{1.235013in}{3.186062in}}%
\pgfpathcurveto{\pgfqpoint{1.227199in}{3.193876in}}{\pgfqpoint{1.216600in}{3.198266in}}{\pgfqpoint{1.205550in}{3.198266in}}%
\pgfpathcurveto{\pgfqpoint{1.194500in}{3.198266in}}{\pgfqpoint{1.183901in}{3.193876in}}{\pgfqpoint{1.176087in}{3.186062in}}%
\pgfpathcurveto{\pgfqpoint{1.168274in}{3.178249in}}{\pgfqpoint{1.163883in}{3.167650in}}{\pgfqpoint{1.163883in}{3.156599in}}%
\pgfpathcurveto{\pgfqpoint{1.163883in}{3.145549in}}{\pgfqpoint{1.168274in}{3.134950in}}{\pgfqpoint{1.176087in}{3.127137in}}%
\pgfpathcurveto{\pgfqpoint{1.183901in}{3.119323in}}{\pgfqpoint{1.194500in}{3.114933in}}{\pgfqpoint{1.205550in}{3.114933in}}%
\pgfpathlineto{\pgfqpoint{1.205550in}{3.114933in}}%
\pgfpathclose%
\pgfusepath{stroke,fill}%
\end{pgfscope}%
\begin{pgfscope}%
\pgfpathrectangle{\pgfqpoint{0.633874in}{2.920818in}}{\pgfqpoint{2.177280in}{2.201755in}}%
\pgfusepath{clip}%
\pgfsetbuttcap%
\pgfsetroundjoin%
\definecolor{currentfill}{rgb}{0.121569,0.466667,0.705882}%
\pgfsetfillcolor{currentfill}%
\pgfsetlinewidth{0.481800pt}%
\definecolor{currentstroke}{rgb}{1.000000,1.000000,1.000000}%
\pgfsetstrokecolor{currentstroke}%
\pgfsetdash{}{0pt}%
\pgfpathmoveto{\pgfqpoint{1.365305in}{3.216709in}}%
\pgfpathcurveto{\pgfqpoint{1.376355in}{3.216709in}}{\pgfqpoint{1.386954in}{3.221099in}}{\pgfqpoint{1.394768in}{3.228913in}}%
\pgfpathcurveto{\pgfqpoint{1.402581in}{3.236726in}}{\pgfqpoint{1.406972in}{3.247325in}}{\pgfqpoint{1.406972in}{3.258375in}}%
\pgfpathcurveto{\pgfqpoint{1.406972in}{3.269426in}}{\pgfqpoint{1.402581in}{3.280025in}}{\pgfqpoint{1.394768in}{3.287838in}}%
\pgfpathcurveto{\pgfqpoint{1.386954in}{3.295652in}}{\pgfqpoint{1.376355in}{3.300042in}}{\pgfqpoint{1.365305in}{3.300042in}}%
\pgfpathcurveto{\pgfqpoint{1.354255in}{3.300042in}}{\pgfqpoint{1.343656in}{3.295652in}}{\pgfqpoint{1.335842in}{3.287838in}}%
\pgfpathcurveto{\pgfqpoint{1.328029in}{3.280025in}}{\pgfqpoint{1.323638in}{3.269426in}}{\pgfqpoint{1.323638in}{3.258375in}}%
\pgfpathcurveto{\pgfqpoint{1.323638in}{3.247325in}}{\pgfqpoint{1.328029in}{3.236726in}}{\pgfqpoint{1.335842in}{3.228913in}}%
\pgfpathcurveto{\pgfqpoint{1.343656in}{3.221099in}}{\pgfqpoint{1.354255in}{3.216709in}}{\pgfqpoint{1.365305in}{3.216709in}}%
\pgfpathlineto{\pgfqpoint{1.365305in}{3.216709in}}%
\pgfpathclose%
\pgfusepath{stroke,fill}%
\end{pgfscope}%
\begin{pgfscope}%
\pgfpathrectangle{\pgfqpoint{0.633874in}{2.920818in}}{\pgfqpoint{2.177280in}{2.201755in}}%
\pgfusepath{clip}%
\pgfsetbuttcap%
\pgfsetroundjoin%
\definecolor{currentfill}{rgb}{0.121569,0.466667,0.705882}%
\pgfsetfillcolor{currentfill}%
\pgfsetlinewidth{0.481800pt}%
\definecolor{currentstroke}{rgb}{1.000000,1.000000,1.000000}%
\pgfsetstrokecolor{currentstroke}%
\pgfsetdash{}{0pt}%
\pgfpathmoveto{\pgfqpoint{1.045795in}{3.114933in}}%
\pgfpathcurveto{\pgfqpoint{1.056845in}{3.114933in}}{\pgfqpoint{1.067444in}{3.119323in}}{\pgfqpoint{1.075258in}{3.127137in}}%
\pgfpathcurveto{\pgfqpoint{1.083072in}{3.134950in}}{\pgfqpoint{1.087462in}{3.145549in}}{\pgfqpoint{1.087462in}{3.156599in}}%
\pgfpathcurveto{\pgfqpoint{1.087462in}{3.167650in}}{\pgfqpoint{1.083072in}{3.178249in}}{\pgfqpoint{1.075258in}{3.186062in}}%
\pgfpathcurveto{\pgfqpoint{1.067444in}{3.193876in}}{\pgfqpoint{1.056845in}{3.198266in}}{\pgfqpoint{1.045795in}{3.198266in}}%
\pgfpathcurveto{\pgfqpoint{1.034745in}{3.198266in}}{\pgfqpoint{1.024146in}{3.193876in}}{\pgfqpoint{1.016332in}{3.186062in}}%
\pgfpathcurveto{\pgfqpoint{1.008519in}{3.178249in}}{\pgfqpoint{1.004129in}{3.167650in}}{\pgfqpoint{1.004129in}{3.156599in}}%
\pgfpathcurveto{\pgfqpoint{1.004129in}{3.145549in}}{\pgfqpoint{1.008519in}{3.134950in}}{\pgfqpoint{1.016332in}{3.127137in}}%
\pgfpathcurveto{\pgfqpoint{1.024146in}{3.119323in}}{\pgfqpoint{1.034745in}{3.114933in}}{\pgfqpoint{1.045795in}{3.114933in}}%
\pgfpathlineto{\pgfqpoint{1.045795in}{3.114933in}}%
\pgfpathclose%
\pgfusepath{stroke,fill}%
\end{pgfscope}%
\begin{pgfscope}%
\pgfpathrectangle{\pgfqpoint{0.633874in}{2.920818in}}{\pgfqpoint{2.177280in}{2.201755in}}%
\pgfusepath{clip}%
\pgfsetbuttcap%
\pgfsetroundjoin%
\definecolor{currentfill}{rgb}{0.121569,0.466667,0.705882}%
\pgfsetfillcolor{currentfill}%
\pgfsetlinewidth{0.481800pt}%
\definecolor{currentstroke}{rgb}{1.000000,1.000000,1.000000}%
\pgfsetstrokecolor{currentstroke}%
\pgfsetdash{}{0pt}%
\pgfpathmoveto{\pgfqpoint{1.205550in}{3.148858in}}%
\pgfpathcurveto{\pgfqpoint{1.216600in}{3.148858in}}{\pgfqpoint{1.227199in}{3.153248in}}{\pgfqpoint{1.235013in}{3.161062in}}%
\pgfpathcurveto{\pgfqpoint{1.242826in}{3.168876in}}{\pgfqpoint{1.247217in}{3.179475in}}{\pgfqpoint{1.247217in}{3.190525in}}%
\pgfpathcurveto{\pgfqpoint{1.247217in}{3.201575in}}{\pgfqpoint{1.242826in}{3.212174in}}{\pgfqpoint{1.235013in}{3.219988in}}%
\pgfpathcurveto{\pgfqpoint{1.227199in}{3.227801in}}{\pgfqpoint{1.216600in}{3.232191in}}{\pgfqpoint{1.205550in}{3.232191in}}%
\pgfpathcurveto{\pgfqpoint{1.194500in}{3.232191in}}{\pgfqpoint{1.183901in}{3.227801in}}{\pgfqpoint{1.176087in}{3.219988in}}%
\pgfpathcurveto{\pgfqpoint{1.168274in}{3.212174in}}{\pgfqpoint{1.163883in}{3.201575in}}{\pgfqpoint{1.163883in}{3.190525in}}%
\pgfpathcurveto{\pgfqpoint{1.163883in}{3.179475in}}{\pgfqpoint{1.168274in}{3.168876in}}{\pgfqpoint{1.176087in}{3.161062in}}%
\pgfpathcurveto{\pgfqpoint{1.183901in}{3.153248in}}{\pgfqpoint{1.194500in}{3.148858in}}{\pgfqpoint{1.205550in}{3.148858in}}%
\pgfpathlineto{\pgfqpoint{1.205550in}{3.148858in}}%
\pgfpathclose%
\pgfusepath{stroke,fill}%
\end{pgfscope}%
\begin{pgfscope}%
\pgfpathrectangle{\pgfqpoint{0.633874in}{2.920818in}}{\pgfqpoint{2.177280in}{2.201755in}}%
\pgfusepath{clip}%
\pgfsetbuttcap%
\pgfsetroundjoin%
\definecolor{currentfill}{rgb}{0.121569,0.466667,0.705882}%
\pgfsetfillcolor{currentfill}%
\pgfsetlinewidth{0.481800pt}%
\definecolor{currentstroke}{rgb}{1.000000,1.000000,1.000000}%
\pgfsetstrokecolor{currentstroke}%
\pgfsetdash{}{0pt}%
\pgfpathmoveto{\pgfqpoint{0.965918in}{3.114933in}}%
\pgfpathcurveto{\pgfqpoint{0.976968in}{3.114933in}}{\pgfqpoint{0.987567in}{3.119323in}}{\pgfqpoint{0.995381in}{3.127137in}}%
\pgfpathcurveto{\pgfqpoint{1.003194in}{3.134950in}}{\pgfqpoint{1.007585in}{3.145549in}}{\pgfqpoint{1.007585in}{3.156599in}}%
\pgfpathcurveto{\pgfqpoint{1.007585in}{3.167650in}}{\pgfqpoint{1.003194in}{3.178249in}}{\pgfqpoint{0.995381in}{3.186062in}}%
\pgfpathcurveto{\pgfqpoint{0.987567in}{3.193876in}}{\pgfqpoint{0.976968in}{3.198266in}}{\pgfqpoint{0.965918in}{3.198266in}}%
\pgfpathcurveto{\pgfqpoint{0.954868in}{3.198266in}}{\pgfqpoint{0.944269in}{3.193876in}}{\pgfqpoint{0.936455in}{3.186062in}}%
\pgfpathcurveto{\pgfqpoint{0.928641in}{3.178249in}}{\pgfqpoint{0.924251in}{3.167650in}}{\pgfqpoint{0.924251in}{3.156599in}}%
\pgfpathcurveto{\pgfqpoint{0.924251in}{3.145549in}}{\pgfqpoint{0.928641in}{3.134950in}}{\pgfqpoint{0.936455in}{3.127137in}}%
\pgfpathcurveto{\pgfqpoint{0.944269in}{3.119323in}}{\pgfqpoint{0.954868in}{3.114933in}}{\pgfqpoint{0.965918in}{3.114933in}}%
\pgfpathlineto{\pgfqpoint{0.965918in}{3.114933in}}%
\pgfpathclose%
\pgfusepath{stroke,fill}%
\end{pgfscope}%
\begin{pgfscope}%
\pgfpathrectangle{\pgfqpoint{0.633874in}{2.920818in}}{\pgfqpoint{2.177280in}{2.201755in}}%
\pgfusepath{clip}%
\pgfsetbuttcap%
\pgfsetroundjoin%
\definecolor{currentfill}{rgb}{0.121569,0.466667,0.705882}%
\pgfsetfillcolor{currentfill}%
\pgfsetlinewidth{0.481800pt}%
\definecolor{currentstroke}{rgb}{1.000000,1.000000,1.000000}%
\pgfsetstrokecolor{currentstroke}%
\pgfsetdash{}{0pt}%
\pgfpathmoveto{\pgfqpoint{1.165611in}{3.148858in}}%
\pgfpathcurveto{\pgfqpoint{1.176662in}{3.148858in}}{\pgfqpoint{1.187261in}{3.153248in}}{\pgfqpoint{1.195074in}{3.161062in}}%
\pgfpathcurveto{\pgfqpoint{1.202888in}{3.168876in}}{\pgfqpoint{1.207278in}{3.179475in}}{\pgfqpoint{1.207278in}{3.190525in}}%
\pgfpathcurveto{\pgfqpoint{1.207278in}{3.201575in}}{\pgfqpoint{1.202888in}{3.212174in}}{\pgfqpoint{1.195074in}{3.219988in}}%
\pgfpathcurveto{\pgfqpoint{1.187261in}{3.227801in}}{\pgfqpoint{1.176662in}{3.232191in}}{\pgfqpoint{1.165611in}{3.232191in}}%
\pgfpathcurveto{\pgfqpoint{1.154561in}{3.232191in}}{\pgfqpoint{1.143962in}{3.227801in}}{\pgfqpoint{1.136149in}{3.219988in}}%
\pgfpathcurveto{\pgfqpoint{1.128335in}{3.212174in}}{\pgfqpoint{1.123945in}{3.201575in}}{\pgfqpoint{1.123945in}{3.190525in}}%
\pgfpathcurveto{\pgfqpoint{1.123945in}{3.179475in}}{\pgfqpoint{1.128335in}{3.168876in}}{\pgfqpoint{1.136149in}{3.161062in}}%
\pgfpathcurveto{\pgfqpoint{1.143962in}{3.153248in}}{\pgfqpoint{1.154561in}{3.148858in}}{\pgfqpoint{1.165611in}{3.148858in}}%
\pgfpathlineto{\pgfqpoint{1.165611in}{3.148858in}}%
\pgfpathclose%
\pgfusepath{stroke,fill}%
\end{pgfscope}%
\begin{pgfscope}%
\pgfpathrectangle{\pgfqpoint{0.633874in}{2.920818in}}{\pgfqpoint{2.177280in}{2.201755in}}%
\pgfusepath{clip}%
\pgfsetbuttcap%
\pgfsetroundjoin%
\definecolor{currentfill}{rgb}{0.121569,0.466667,0.705882}%
\pgfsetfillcolor{currentfill}%
\pgfsetlinewidth{0.481800pt}%
\definecolor{currentstroke}{rgb}{1.000000,1.000000,1.000000}%
\pgfsetstrokecolor{currentstroke}%
\pgfsetdash{}{0pt}%
\pgfpathmoveto{\pgfqpoint{1.365305in}{3.148858in}}%
\pgfpathcurveto{\pgfqpoint{1.376355in}{3.148858in}}{\pgfqpoint{1.386954in}{3.153248in}}{\pgfqpoint{1.394768in}{3.161062in}}%
\pgfpathcurveto{\pgfqpoint{1.402581in}{3.168876in}}{\pgfqpoint{1.406972in}{3.179475in}}{\pgfqpoint{1.406972in}{3.190525in}}%
\pgfpathcurveto{\pgfqpoint{1.406972in}{3.201575in}}{\pgfqpoint{1.402581in}{3.212174in}}{\pgfqpoint{1.394768in}{3.219988in}}%
\pgfpathcurveto{\pgfqpoint{1.386954in}{3.227801in}}{\pgfqpoint{1.376355in}{3.232191in}}{\pgfqpoint{1.365305in}{3.232191in}}%
\pgfpathcurveto{\pgfqpoint{1.354255in}{3.232191in}}{\pgfqpoint{1.343656in}{3.227801in}}{\pgfqpoint{1.335842in}{3.219988in}}%
\pgfpathcurveto{\pgfqpoint{1.328029in}{3.212174in}}{\pgfqpoint{1.323638in}{3.201575in}}{\pgfqpoint{1.323638in}{3.190525in}}%
\pgfpathcurveto{\pgfqpoint{1.323638in}{3.179475in}}{\pgfqpoint{1.328029in}{3.168876in}}{\pgfqpoint{1.335842in}{3.161062in}}%
\pgfpathcurveto{\pgfqpoint{1.343656in}{3.153248in}}{\pgfqpoint{1.354255in}{3.148858in}}{\pgfqpoint{1.365305in}{3.148858in}}%
\pgfpathlineto{\pgfqpoint{1.365305in}{3.148858in}}%
\pgfpathclose%
\pgfusepath{stroke,fill}%
\end{pgfscope}%
\begin{pgfscope}%
\pgfpathrectangle{\pgfqpoint{0.633874in}{2.920818in}}{\pgfqpoint{2.177280in}{2.201755in}}%
\pgfusepath{clip}%
\pgfsetbuttcap%
\pgfsetroundjoin%
\definecolor{currentfill}{rgb}{0.121569,0.466667,0.705882}%
\pgfsetfillcolor{currentfill}%
\pgfsetlinewidth{0.481800pt}%
\definecolor{currentstroke}{rgb}{1.000000,1.000000,1.000000}%
\pgfsetstrokecolor{currentstroke}%
\pgfsetdash{}{0pt}%
\pgfpathmoveto{\pgfqpoint{1.125673in}{3.182783in}}%
\pgfpathcurveto{\pgfqpoint{1.136723in}{3.182783in}}{\pgfqpoint{1.147322in}{3.187174in}}{\pgfqpoint{1.155135in}{3.194987in}}%
\pgfpathcurveto{\pgfqpoint{1.162949in}{3.202801in}}{\pgfqpoint{1.167339in}{3.213400in}}{\pgfqpoint{1.167339in}{3.224450in}}%
\pgfpathcurveto{\pgfqpoint{1.167339in}{3.235500in}}{\pgfqpoint{1.162949in}{3.246099in}}{\pgfqpoint{1.155135in}{3.253913in}}%
\pgfpathcurveto{\pgfqpoint{1.147322in}{3.261727in}}{\pgfqpoint{1.136723in}{3.266117in}}{\pgfqpoint{1.125673in}{3.266117in}}%
\pgfpathcurveto{\pgfqpoint{1.114623in}{3.266117in}}{\pgfqpoint{1.104024in}{3.261727in}}{\pgfqpoint{1.096210in}{3.253913in}}%
\pgfpathcurveto{\pgfqpoint{1.088396in}{3.246099in}}{\pgfqpoint{1.084006in}{3.235500in}}{\pgfqpoint{1.084006in}{3.224450in}}%
\pgfpathcurveto{\pgfqpoint{1.084006in}{3.213400in}}{\pgfqpoint{1.088396in}{3.202801in}}{\pgfqpoint{1.096210in}{3.194987in}}%
\pgfpathcurveto{\pgfqpoint{1.104024in}{3.187174in}}{\pgfqpoint{1.114623in}{3.182783in}}{\pgfqpoint{1.125673in}{3.182783in}}%
\pgfpathlineto{\pgfqpoint{1.125673in}{3.182783in}}%
\pgfpathclose%
\pgfusepath{stroke,fill}%
\end{pgfscope}%
\begin{pgfscope}%
\pgfpathrectangle{\pgfqpoint{0.633874in}{2.920818in}}{\pgfqpoint{2.177280in}{2.201755in}}%
\pgfusepath{clip}%
\pgfsetbuttcap%
\pgfsetroundjoin%
\definecolor{currentfill}{rgb}{0.121569,0.466667,0.705882}%
\pgfsetfillcolor{currentfill}%
\pgfsetlinewidth{0.481800pt}%
\definecolor{currentstroke}{rgb}{1.000000,1.000000,1.000000}%
\pgfsetstrokecolor{currentstroke}%
\pgfsetdash{}{0pt}%
\pgfpathmoveto{\pgfqpoint{1.125673in}{3.114933in}}%
\pgfpathcurveto{\pgfqpoint{1.136723in}{3.114933in}}{\pgfqpoint{1.147322in}{3.119323in}}{\pgfqpoint{1.155135in}{3.127137in}}%
\pgfpathcurveto{\pgfqpoint{1.162949in}{3.134950in}}{\pgfqpoint{1.167339in}{3.145549in}}{\pgfqpoint{1.167339in}{3.156599in}}%
\pgfpathcurveto{\pgfqpoint{1.167339in}{3.167650in}}{\pgfqpoint{1.162949in}{3.178249in}}{\pgfqpoint{1.155135in}{3.186062in}}%
\pgfpathcurveto{\pgfqpoint{1.147322in}{3.193876in}}{\pgfqpoint{1.136723in}{3.198266in}}{\pgfqpoint{1.125673in}{3.198266in}}%
\pgfpathcurveto{\pgfqpoint{1.114623in}{3.198266in}}{\pgfqpoint{1.104024in}{3.193876in}}{\pgfqpoint{1.096210in}{3.186062in}}%
\pgfpathcurveto{\pgfqpoint{1.088396in}{3.178249in}}{\pgfqpoint{1.084006in}{3.167650in}}{\pgfqpoint{1.084006in}{3.156599in}}%
\pgfpathcurveto{\pgfqpoint{1.084006in}{3.145549in}}{\pgfqpoint{1.088396in}{3.134950in}}{\pgfqpoint{1.096210in}{3.127137in}}%
\pgfpathcurveto{\pgfqpoint{1.104024in}{3.119323in}}{\pgfqpoint{1.114623in}{3.114933in}}{\pgfqpoint{1.125673in}{3.114933in}}%
\pgfpathlineto{\pgfqpoint{1.125673in}{3.114933in}}%
\pgfpathclose%
\pgfusepath{stroke,fill}%
\end{pgfscope}%
\begin{pgfscope}%
\pgfpathrectangle{\pgfqpoint{0.633874in}{2.920818in}}{\pgfqpoint{2.177280in}{2.201755in}}%
\pgfusepath{clip}%
\pgfsetbuttcap%
\pgfsetroundjoin%
\definecolor{currentfill}{rgb}{0.121569,0.466667,0.705882}%
\pgfsetfillcolor{currentfill}%
\pgfsetlinewidth{0.481800pt}%
\definecolor{currentstroke}{rgb}{1.000000,1.000000,1.000000}%
\pgfsetstrokecolor{currentstroke}%
\pgfsetdash{}{0pt}%
\pgfpathmoveto{\pgfqpoint{0.925979in}{3.013157in}}%
\pgfpathcurveto{\pgfqpoint{0.937029in}{3.013157in}}{\pgfqpoint{0.947628in}{3.017547in}}{\pgfqpoint{0.955442in}{3.025361in}}%
\pgfpathcurveto{\pgfqpoint{0.963256in}{3.033174in}}{\pgfqpoint{0.967646in}{3.043773in}}{\pgfqpoint{0.967646in}{3.054823in}}%
\pgfpathcurveto{\pgfqpoint{0.967646in}{3.065874in}}{\pgfqpoint{0.963256in}{3.076473in}}{\pgfqpoint{0.955442in}{3.084286in}}%
\pgfpathcurveto{\pgfqpoint{0.947628in}{3.092100in}}{\pgfqpoint{0.937029in}{3.096490in}}{\pgfqpoint{0.925979in}{3.096490in}}%
\pgfpathcurveto{\pgfqpoint{0.914929in}{3.096490in}}{\pgfqpoint{0.904330in}{3.092100in}}{\pgfqpoint{0.896516in}{3.084286in}}%
\pgfpathcurveto{\pgfqpoint{0.888703in}{3.076473in}}{\pgfqpoint{0.884313in}{3.065874in}}{\pgfqpoint{0.884313in}{3.054823in}}%
\pgfpathcurveto{\pgfqpoint{0.884313in}{3.043773in}}{\pgfqpoint{0.888703in}{3.033174in}}{\pgfqpoint{0.896516in}{3.025361in}}%
\pgfpathcurveto{\pgfqpoint{0.904330in}{3.017547in}}{\pgfqpoint{0.914929in}{3.013157in}}{\pgfqpoint{0.925979in}{3.013157in}}%
\pgfpathlineto{\pgfqpoint{0.925979in}{3.013157in}}%
\pgfpathclose%
\pgfusepath{stroke,fill}%
\end{pgfscope}%
\begin{pgfscope}%
\pgfpathrectangle{\pgfqpoint{0.633874in}{2.920818in}}{\pgfqpoint{2.177280in}{2.201755in}}%
\pgfusepath{clip}%
\pgfsetbuttcap%
\pgfsetroundjoin%
\definecolor{currentfill}{rgb}{0.121569,0.466667,0.705882}%
\pgfsetfillcolor{currentfill}%
\pgfsetlinewidth{0.481800pt}%
\definecolor{currentstroke}{rgb}{1.000000,1.000000,1.000000}%
\pgfsetstrokecolor{currentstroke}%
\pgfsetdash{}{0pt}%
\pgfpathmoveto{\pgfqpoint{1.525060in}{3.047082in}}%
\pgfpathcurveto{\pgfqpoint{1.536110in}{3.047082in}}{\pgfqpoint{1.546709in}{3.051472in}}{\pgfqpoint{1.554523in}{3.059286in}}%
\pgfpathcurveto{\pgfqpoint{1.562336in}{3.067100in}}{\pgfqpoint{1.566726in}{3.077699in}}{\pgfqpoint{1.566726in}{3.088749in}}%
\pgfpathcurveto{\pgfqpoint{1.566726in}{3.099799in}}{\pgfqpoint{1.562336in}{3.110398in}}{\pgfqpoint{1.554523in}{3.118212in}}%
\pgfpathcurveto{\pgfqpoint{1.546709in}{3.126025in}}{\pgfqpoint{1.536110in}{3.130415in}}{\pgfqpoint{1.525060in}{3.130415in}}%
\pgfpathcurveto{\pgfqpoint{1.514010in}{3.130415in}}{\pgfqpoint{1.503411in}{3.126025in}}{\pgfqpoint{1.495597in}{3.118212in}}%
\pgfpathcurveto{\pgfqpoint{1.487783in}{3.110398in}}{\pgfqpoint{1.483393in}{3.099799in}}{\pgfqpoint{1.483393in}{3.088749in}}%
\pgfpathcurveto{\pgfqpoint{1.483393in}{3.077699in}}{\pgfqpoint{1.487783in}{3.067100in}}{\pgfqpoint{1.495597in}{3.059286in}}%
\pgfpathcurveto{\pgfqpoint{1.503411in}{3.051472in}}{\pgfqpoint{1.514010in}{3.047082in}}{\pgfqpoint{1.525060in}{3.047082in}}%
\pgfpathlineto{\pgfqpoint{1.525060in}{3.047082in}}%
\pgfpathclose%
\pgfusepath{stroke,fill}%
\end{pgfscope}%
\begin{pgfscope}%
\pgfpathrectangle{\pgfqpoint{0.633874in}{2.920818in}}{\pgfqpoint{2.177280in}{2.201755in}}%
\pgfusepath{clip}%
\pgfsetbuttcap%
\pgfsetroundjoin%
\definecolor{currentfill}{rgb}{0.121569,0.466667,0.705882}%
\pgfsetfillcolor{currentfill}%
\pgfsetlinewidth{0.481800pt}%
\definecolor{currentstroke}{rgb}{1.000000,1.000000,1.000000}%
\pgfsetstrokecolor{currentstroke}%
\pgfsetdash{}{0pt}%
\pgfpathmoveto{\pgfqpoint{1.485121in}{3.148858in}}%
\pgfpathcurveto{\pgfqpoint{1.496171in}{3.148858in}}{\pgfqpoint{1.506770in}{3.153248in}}{\pgfqpoint{1.514584in}{3.161062in}}%
\pgfpathcurveto{\pgfqpoint{1.522397in}{3.168876in}}{\pgfqpoint{1.526788in}{3.179475in}}{\pgfqpoint{1.526788in}{3.190525in}}%
\pgfpathcurveto{\pgfqpoint{1.526788in}{3.201575in}}{\pgfqpoint{1.522397in}{3.212174in}}{\pgfqpoint{1.514584in}{3.219988in}}%
\pgfpathcurveto{\pgfqpoint{1.506770in}{3.227801in}}{\pgfqpoint{1.496171in}{3.232191in}}{\pgfqpoint{1.485121in}{3.232191in}}%
\pgfpathcurveto{\pgfqpoint{1.474071in}{3.232191in}}{\pgfqpoint{1.463472in}{3.227801in}}{\pgfqpoint{1.455658in}{3.219988in}}%
\pgfpathcurveto{\pgfqpoint{1.447845in}{3.212174in}}{\pgfqpoint{1.443454in}{3.201575in}}{\pgfqpoint{1.443454in}{3.190525in}}%
\pgfpathcurveto{\pgfqpoint{1.443454in}{3.179475in}}{\pgfqpoint{1.447845in}{3.168876in}}{\pgfqpoint{1.455658in}{3.161062in}}%
\pgfpathcurveto{\pgfqpoint{1.463472in}{3.153248in}}{\pgfqpoint{1.474071in}{3.148858in}}{\pgfqpoint{1.485121in}{3.148858in}}%
\pgfpathlineto{\pgfqpoint{1.485121in}{3.148858in}}%
\pgfpathclose%
\pgfusepath{stroke,fill}%
\end{pgfscope}%
\begin{pgfscope}%
\pgfpathrectangle{\pgfqpoint{0.633874in}{2.920818in}}{\pgfqpoint{2.177280in}{2.201755in}}%
\pgfusepath{clip}%
\pgfsetbuttcap%
\pgfsetroundjoin%
\definecolor{currentfill}{rgb}{0.121569,0.466667,0.705882}%
\pgfsetfillcolor{currentfill}%
\pgfsetlinewidth{0.481800pt}%
\definecolor{currentstroke}{rgb}{1.000000,1.000000,1.000000}%
\pgfsetstrokecolor{currentstroke}%
\pgfsetdash{}{0pt}%
\pgfpathmoveto{\pgfqpoint{1.365305in}{3.081007in}}%
\pgfpathcurveto{\pgfqpoint{1.376355in}{3.081007in}}{\pgfqpoint{1.386954in}{3.085398in}}{\pgfqpoint{1.394768in}{3.093211in}}%
\pgfpathcurveto{\pgfqpoint{1.402581in}{3.101025in}}{\pgfqpoint{1.406972in}{3.111624in}}{\pgfqpoint{1.406972in}{3.122674in}}%
\pgfpathcurveto{\pgfqpoint{1.406972in}{3.133724in}}{\pgfqpoint{1.402581in}{3.144323in}}{\pgfqpoint{1.394768in}{3.152137in}}%
\pgfpathcurveto{\pgfqpoint{1.386954in}{3.159951in}}{\pgfqpoint{1.376355in}{3.164341in}}{\pgfqpoint{1.365305in}{3.164341in}}%
\pgfpathcurveto{\pgfqpoint{1.354255in}{3.164341in}}{\pgfqpoint{1.343656in}{3.159951in}}{\pgfqpoint{1.335842in}{3.152137in}}%
\pgfpathcurveto{\pgfqpoint{1.328029in}{3.144323in}}{\pgfqpoint{1.323638in}{3.133724in}}{\pgfqpoint{1.323638in}{3.122674in}}%
\pgfpathcurveto{\pgfqpoint{1.323638in}{3.111624in}}{\pgfqpoint{1.328029in}{3.101025in}}{\pgfqpoint{1.335842in}{3.093211in}}%
\pgfpathcurveto{\pgfqpoint{1.343656in}{3.085398in}}{\pgfqpoint{1.354255in}{3.081007in}}{\pgfqpoint{1.365305in}{3.081007in}}%
\pgfpathlineto{\pgfqpoint{1.365305in}{3.081007in}}%
\pgfpathclose%
\pgfusepath{stroke,fill}%
\end{pgfscope}%
\begin{pgfscope}%
\pgfpathrectangle{\pgfqpoint{0.633874in}{2.920818in}}{\pgfqpoint{2.177280in}{2.201755in}}%
\pgfusepath{clip}%
\pgfsetbuttcap%
\pgfsetroundjoin%
\definecolor{currentfill}{rgb}{0.121569,0.466667,0.705882}%
\pgfsetfillcolor{currentfill}%
\pgfsetlinewidth{0.481800pt}%
\definecolor{currentstroke}{rgb}{1.000000,1.000000,1.000000}%
\pgfsetstrokecolor{currentstroke}%
\pgfsetdash{}{0pt}%
\pgfpathmoveto{\pgfqpoint{1.245489in}{3.114933in}}%
\pgfpathcurveto{\pgfqpoint{1.256539in}{3.114933in}}{\pgfqpoint{1.267138in}{3.119323in}}{\pgfqpoint{1.274952in}{3.127137in}}%
\pgfpathcurveto{\pgfqpoint{1.282765in}{3.134950in}}{\pgfqpoint{1.287155in}{3.145549in}}{\pgfqpoint{1.287155in}{3.156599in}}%
\pgfpathcurveto{\pgfqpoint{1.287155in}{3.167650in}}{\pgfqpoint{1.282765in}{3.178249in}}{\pgfqpoint{1.274952in}{3.186062in}}%
\pgfpathcurveto{\pgfqpoint{1.267138in}{3.193876in}}{\pgfqpoint{1.256539in}{3.198266in}}{\pgfqpoint{1.245489in}{3.198266in}}%
\pgfpathcurveto{\pgfqpoint{1.234439in}{3.198266in}}{\pgfqpoint{1.223840in}{3.193876in}}{\pgfqpoint{1.216026in}{3.186062in}}%
\pgfpathcurveto{\pgfqpoint{1.208212in}{3.178249in}}{\pgfqpoint{1.203822in}{3.167650in}}{\pgfqpoint{1.203822in}{3.156599in}}%
\pgfpathcurveto{\pgfqpoint{1.203822in}{3.145549in}}{\pgfqpoint{1.208212in}{3.134950in}}{\pgfqpoint{1.216026in}{3.127137in}}%
\pgfpathcurveto{\pgfqpoint{1.223840in}{3.119323in}}{\pgfqpoint{1.234439in}{3.114933in}}{\pgfqpoint{1.245489in}{3.114933in}}%
\pgfpathlineto{\pgfqpoint{1.245489in}{3.114933in}}%
\pgfpathclose%
\pgfusepath{stroke,fill}%
\end{pgfscope}%
\begin{pgfscope}%
\pgfpathrectangle{\pgfqpoint{0.633874in}{2.920818in}}{\pgfqpoint{2.177280in}{2.201755in}}%
\pgfusepath{clip}%
\pgfsetbuttcap%
\pgfsetroundjoin%
\definecolor{currentfill}{rgb}{0.121569,0.466667,0.705882}%
\pgfsetfillcolor{currentfill}%
\pgfsetlinewidth{0.481800pt}%
\definecolor{currentstroke}{rgb}{1.000000,1.000000,1.000000}%
\pgfsetstrokecolor{currentstroke}%
\pgfsetdash{}{0pt}%
\pgfpathmoveto{\pgfqpoint{1.485121in}{3.216709in}}%
\pgfpathcurveto{\pgfqpoint{1.496171in}{3.216709in}}{\pgfqpoint{1.506770in}{3.221099in}}{\pgfqpoint{1.514584in}{3.228913in}}%
\pgfpathcurveto{\pgfqpoint{1.522397in}{3.236726in}}{\pgfqpoint{1.526788in}{3.247325in}}{\pgfqpoint{1.526788in}{3.258375in}}%
\pgfpathcurveto{\pgfqpoint{1.526788in}{3.269426in}}{\pgfqpoint{1.522397in}{3.280025in}}{\pgfqpoint{1.514584in}{3.287838in}}%
\pgfpathcurveto{\pgfqpoint{1.506770in}{3.295652in}}{\pgfqpoint{1.496171in}{3.300042in}}{\pgfqpoint{1.485121in}{3.300042in}}%
\pgfpathcurveto{\pgfqpoint{1.474071in}{3.300042in}}{\pgfqpoint{1.463472in}{3.295652in}}{\pgfqpoint{1.455658in}{3.287838in}}%
\pgfpathcurveto{\pgfqpoint{1.447845in}{3.280025in}}{\pgfqpoint{1.443454in}{3.269426in}}{\pgfqpoint{1.443454in}{3.258375in}}%
\pgfpathcurveto{\pgfqpoint{1.443454in}{3.247325in}}{\pgfqpoint{1.447845in}{3.236726in}}{\pgfqpoint{1.455658in}{3.228913in}}%
\pgfpathcurveto{\pgfqpoint{1.463472in}{3.221099in}}{\pgfqpoint{1.474071in}{3.216709in}}{\pgfqpoint{1.485121in}{3.216709in}}%
\pgfpathlineto{\pgfqpoint{1.485121in}{3.216709in}}%
\pgfpathclose%
\pgfusepath{stroke,fill}%
\end{pgfscope}%
\begin{pgfscope}%
\pgfpathrectangle{\pgfqpoint{0.633874in}{2.920818in}}{\pgfqpoint{2.177280in}{2.201755in}}%
\pgfusepath{clip}%
\pgfsetbuttcap%
\pgfsetroundjoin%
\definecolor{currentfill}{rgb}{0.121569,0.466667,0.705882}%
\pgfsetfillcolor{currentfill}%
\pgfsetlinewidth{0.481800pt}%
\definecolor{currentstroke}{rgb}{1.000000,1.000000,1.000000}%
\pgfsetstrokecolor{currentstroke}%
\pgfsetdash{}{0pt}%
\pgfpathmoveto{\pgfqpoint{1.245489in}{3.148858in}}%
\pgfpathcurveto{\pgfqpoint{1.256539in}{3.148858in}}{\pgfqpoint{1.267138in}{3.153248in}}{\pgfqpoint{1.274952in}{3.161062in}}%
\pgfpathcurveto{\pgfqpoint{1.282765in}{3.168876in}}{\pgfqpoint{1.287155in}{3.179475in}}{\pgfqpoint{1.287155in}{3.190525in}}%
\pgfpathcurveto{\pgfqpoint{1.287155in}{3.201575in}}{\pgfqpoint{1.282765in}{3.212174in}}{\pgfqpoint{1.274952in}{3.219988in}}%
\pgfpathcurveto{\pgfqpoint{1.267138in}{3.227801in}}{\pgfqpoint{1.256539in}{3.232191in}}{\pgfqpoint{1.245489in}{3.232191in}}%
\pgfpathcurveto{\pgfqpoint{1.234439in}{3.232191in}}{\pgfqpoint{1.223840in}{3.227801in}}{\pgfqpoint{1.216026in}{3.219988in}}%
\pgfpathcurveto{\pgfqpoint{1.208212in}{3.212174in}}{\pgfqpoint{1.203822in}{3.201575in}}{\pgfqpoint{1.203822in}{3.190525in}}%
\pgfpathcurveto{\pgfqpoint{1.203822in}{3.179475in}}{\pgfqpoint{1.208212in}{3.168876in}}{\pgfqpoint{1.216026in}{3.161062in}}%
\pgfpathcurveto{\pgfqpoint{1.223840in}{3.153248in}}{\pgfqpoint{1.234439in}{3.148858in}}{\pgfqpoint{1.245489in}{3.148858in}}%
\pgfpathlineto{\pgfqpoint{1.245489in}{3.148858in}}%
\pgfpathclose%
\pgfusepath{stroke,fill}%
\end{pgfscope}%
\begin{pgfscope}%
\pgfpathrectangle{\pgfqpoint{0.633874in}{2.920818in}}{\pgfqpoint{2.177280in}{2.201755in}}%
\pgfusepath{clip}%
\pgfsetbuttcap%
\pgfsetroundjoin%
\definecolor{currentfill}{rgb}{0.121569,0.466667,0.705882}%
\pgfsetfillcolor{currentfill}%
\pgfsetlinewidth{0.481800pt}%
\definecolor{currentstroke}{rgb}{1.000000,1.000000,1.000000}%
\pgfsetstrokecolor{currentstroke}%
\pgfsetdash{}{0pt}%
\pgfpathmoveto{\pgfqpoint{1.365305in}{3.216709in}}%
\pgfpathcurveto{\pgfqpoint{1.376355in}{3.216709in}}{\pgfqpoint{1.386954in}{3.221099in}}{\pgfqpoint{1.394768in}{3.228913in}}%
\pgfpathcurveto{\pgfqpoint{1.402581in}{3.236726in}}{\pgfqpoint{1.406972in}{3.247325in}}{\pgfqpoint{1.406972in}{3.258375in}}%
\pgfpathcurveto{\pgfqpoint{1.406972in}{3.269426in}}{\pgfqpoint{1.402581in}{3.280025in}}{\pgfqpoint{1.394768in}{3.287838in}}%
\pgfpathcurveto{\pgfqpoint{1.386954in}{3.295652in}}{\pgfqpoint{1.376355in}{3.300042in}}{\pgfqpoint{1.365305in}{3.300042in}}%
\pgfpathcurveto{\pgfqpoint{1.354255in}{3.300042in}}{\pgfqpoint{1.343656in}{3.295652in}}{\pgfqpoint{1.335842in}{3.287838in}}%
\pgfpathcurveto{\pgfqpoint{1.328029in}{3.280025in}}{\pgfqpoint{1.323638in}{3.269426in}}{\pgfqpoint{1.323638in}{3.258375in}}%
\pgfpathcurveto{\pgfqpoint{1.323638in}{3.247325in}}{\pgfqpoint{1.328029in}{3.236726in}}{\pgfqpoint{1.335842in}{3.228913in}}%
\pgfpathcurveto{\pgfqpoint{1.343656in}{3.221099in}}{\pgfqpoint{1.354255in}{3.216709in}}{\pgfqpoint{1.365305in}{3.216709in}}%
\pgfpathlineto{\pgfqpoint{1.365305in}{3.216709in}}%
\pgfpathclose%
\pgfusepath{stroke,fill}%
\end{pgfscope}%
\begin{pgfscope}%
\pgfpathrectangle{\pgfqpoint{0.633874in}{2.920818in}}{\pgfqpoint{2.177280in}{2.201755in}}%
\pgfusepath{clip}%
\pgfsetbuttcap%
\pgfsetroundjoin%
\definecolor{currentfill}{rgb}{0.121569,0.466667,0.705882}%
\pgfsetfillcolor{currentfill}%
\pgfsetlinewidth{0.481800pt}%
\definecolor{currentstroke}{rgb}{1.000000,1.000000,1.000000}%
\pgfsetstrokecolor{currentstroke}%
\pgfsetdash{}{0pt}%
\pgfpathmoveto{\pgfqpoint{1.245489in}{3.148858in}}%
\pgfpathcurveto{\pgfqpoint{1.256539in}{3.148858in}}{\pgfqpoint{1.267138in}{3.153248in}}{\pgfqpoint{1.274952in}{3.161062in}}%
\pgfpathcurveto{\pgfqpoint{1.282765in}{3.168876in}}{\pgfqpoint{1.287155in}{3.179475in}}{\pgfqpoint{1.287155in}{3.190525in}}%
\pgfpathcurveto{\pgfqpoint{1.287155in}{3.201575in}}{\pgfqpoint{1.282765in}{3.212174in}}{\pgfqpoint{1.274952in}{3.219988in}}%
\pgfpathcurveto{\pgfqpoint{1.267138in}{3.227801in}}{\pgfqpoint{1.256539in}{3.232191in}}{\pgfqpoint{1.245489in}{3.232191in}}%
\pgfpathcurveto{\pgfqpoint{1.234439in}{3.232191in}}{\pgfqpoint{1.223840in}{3.227801in}}{\pgfqpoint{1.216026in}{3.219988in}}%
\pgfpathcurveto{\pgfqpoint{1.208212in}{3.212174in}}{\pgfqpoint{1.203822in}{3.201575in}}{\pgfqpoint{1.203822in}{3.190525in}}%
\pgfpathcurveto{\pgfqpoint{1.203822in}{3.179475in}}{\pgfqpoint{1.208212in}{3.168876in}}{\pgfqpoint{1.216026in}{3.161062in}}%
\pgfpathcurveto{\pgfqpoint{1.223840in}{3.153248in}}{\pgfqpoint{1.234439in}{3.148858in}}{\pgfqpoint{1.245489in}{3.148858in}}%
\pgfpathlineto{\pgfqpoint{1.245489in}{3.148858in}}%
\pgfpathclose%
\pgfusepath{stroke,fill}%
\end{pgfscope}%
\begin{pgfscope}%
\pgfpathrectangle{\pgfqpoint{0.633874in}{2.920818in}}{\pgfqpoint{2.177280in}{2.201755in}}%
\pgfusepath{clip}%
\pgfsetbuttcap%
\pgfsetroundjoin%
\definecolor{currentfill}{rgb}{0.121569,0.466667,0.705882}%
\pgfsetfillcolor{currentfill}%
\pgfsetlinewidth{0.481800pt}%
\definecolor{currentstroke}{rgb}{1.000000,1.000000,1.000000}%
\pgfsetstrokecolor{currentstroke}%
\pgfsetdash{}{0pt}%
\pgfpathmoveto{\pgfqpoint{1.045795in}{2.979231in}}%
\pgfpathcurveto{\pgfqpoint{1.056845in}{2.979231in}}{\pgfqpoint{1.067444in}{2.983622in}}{\pgfqpoint{1.075258in}{2.991435in}}%
\pgfpathcurveto{\pgfqpoint{1.083072in}{2.999249in}}{\pgfqpoint{1.087462in}{3.009848in}}{\pgfqpoint{1.087462in}{3.020898in}}%
\pgfpathcurveto{\pgfqpoint{1.087462in}{3.031948in}}{\pgfqpoint{1.083072in}{3.042547in}}{\pgfqpoint{1.075258in}{3.050361in}}%
\pgfpathcurveto{\pgfqpoint{1.067444in}{3.058174in}}{\pgfqpoint{1.056845in}{3.062565in}}{\pgfqpoint{1.045795in}{3.062565in}}%
\pgfpathcurveto{\pgfqpoint{1.034745in}{3.062565in}}{\pgfqpoint{1.024146in}{3.058174in}}{\pgfqpoint{1.016332in}{3.050361in}}%
\pgfpathcurveto{\pgfqpoint{1.008519in}{3.042547in}}{\pgfqpoint{1.004129in}{3.031948in}}{\pgfqpoint{1.004129in}{3.020898in}}%
\pgfpathcurveto{\pgfqpoint{1.004129in}{3.009848in}}{\pgfqpoint{1.008519in}{2.999249in}}{\pgfqpoint{1.016332in}{2.991435in}}%
\pgfpathcurveto{\pgfqpoint{1.024146in}{2.983622in}}{\pgfqpoint{1.034745in}{2.979231in}}{\pgfqpoint{1.045795in}{2.979231in}}%
\pgfpathlineto{\pgfqpoint{1.045795in}{2.979231in}}%
\pgfpathclose%
\pgfusepath{stroke,fill}%
\end{pgfscope}%
\begin{pgfscope}%
\pgfpathrectangle{\pgfqpoint{0.633874in}{2.920818in}}{\pgfqpoint{2.177280in}{2.201755in}}%
\pgfusepath{clip}%
\pgfsetbuttcap%
\pgfsetroundjoin%
\definecolor{currentfill}{rgb}{0.121569,0.466667,0.705882}%
\pgfsetfillcolor{currentfill}%
\pgfsetlinewidth{0.481800pt}%
\definecolor{currentstroke}{rgb}{1.000000,1.000000,1.000000}%
\pgfsetstrokecolor{currentstroke}%
\pgfsetdash{}{0pt}%
\pgfpathmoveto{\pgfqpoint{1.245489in}{3.216709in}}%
\pgfpathcurveto{\pgfqpoint{1.256539in}{3.216709in}}{\pgfqpoint{1.267138in}{3.221099in}}{\pgfqpoint{1.274952in}{3.228913in}}%
\pgfpathcurveto{\pgfqpoint{1.282765in}{3.236726in}}{\pgfqpoint{1.287155in}{3.247325in}}{\pgfqpoint{1.287155in}{3.258375in}}%
\pgfpathcurveto{\pgfqpoint{1.287155in}{3.269426in}}{\pgfqpoint{1.282765in}{3.280025in}}{\pgfqpoint{1.274952in}{3.287838in}}%
\pgfpathcurveto{\pgfqpoint{1.267138in}{3.295652in}}{\pgfqpoint{1.256539in}{3.300042in}}{\pgfqpoint{1.245489in}{3.300042in}}%
\pgfpathcurveto{\pgfqpoint{1.234439in}{3.300042in}}{\pgfqpoint{1.223840in}{3.295652in}}{\pgfqpoint{1.216026in}{3.287838in}}%
\pgfpathcurveto{\pgfqpoint{1.208212in}{3.280025in}}{\pgfqpoint{1.203822in}{3.269426in}}{\pgfqpoint{1.203822in}{3.258375in}}%
\pgfpathcurveto{\pgfqpoint{1.203822in}{3.247325in}}{\pgfqpoint{1.208212in}{3.236726in}}{\pgfqpoint{1.216026in}{3.228913in}}%
\pgfpathcurveto{\pgfqpoint{1.223840in}{3.221099in}}{\pgfqpoint{1.234439in}{3.216709in}}{\pgfqpoint{1.245489in}{3.216709in}}%
\pgfpathlineto{\pgfqpoint{1.245489in}{3.216709in}}%
\pgfpathclose%
\pgfusepath{stroke,fill}%
\end{pgfscope}%
\begin{pgfscope}%
\pgfpathrectangle{\pgfqpoint{0.633874in}{2.920818in}}{\pgfqpoint{2.177280in}{2.201755in}}%
\pgfusepath{clip}%
\pgfsetbuttcap%
\pgfsetroundjoin%
\definecolor{currentfill}{rgb}{0.121569,0.466667,0.705882}%
\pgfsetfillcolor{currentfill}%
\pgfsetlinewidth{0.481800pt}%
\definecolor{currentstroke}{rgb}{1.000000,1.000000,1.000000}%
\pgfsetstrokecolor{currentstroke}%
\pgfsetdash{}{0pt}%
\pgfpathmoveto{\pgfqpoint{1.125673in}{3.284560in}}%
\pgfpathcurveto{\pgfqpoint{1.136723in}{3.284560in}}{\pgfqpoint{1.147322in}{3.288950in}}{\pgfqpoint{1.155135in}{3.296763in}}%
\pgfpathcurveto{\pgfqpoint{1.162949in}{3.304577in}}{\pgfqpoint{1.167339in}{3.315176in}}{\pgfqpoint{1.167339in}{3.326226in}}%
\pgfpathcurveto{\pgfqpoint{1.167339in}{3.337276in}}{\pgfqpoint{1.162949in}{3.347875in}}{\pgfqpoint{1.155135in}{3.355689in}}%
\pgfpathcurveto{\pgfqpoint{1.147322in}{3.363503in}}{\pgfqpoint{1.136723in}{3.367893in}}{\pgfqpoint{1.125673in}{3.367893in}}%
\pgfpathcurveto{\pgfqpoint{1.114623in}{3.367893in}}{\pgfqpoint{1.104024in}{3.363503in}}{\pgfqpoint{1.096210in}{3.355689in}}%
\pgfpathcurveto{\pgfqpoint{1.088396in}{3.347875in}}{\pgfqpoint{1.084006in}{3.337276in}}{\pgfqpoint{1.084006in}{3.326226in}}%
\pgfpathcurveto{\pgfqpoint{1.084006in}{3.315176in}}{\pgfqpoint{1.088396in}{3.304577in}}{\pgfqpoint{1.096210in}{3.296763in}}%
\pgfpathcurveto{\pgfqpoint{1.104024in}{3.288950in}}{\pgfqpoint{1.114623in}{3.284560in}}{\pgfqpoint{1.125673in}{3.284560in}}%
\pgfpathlineto{\pgfqpoint{1.125673in}{3.284560in}}%
\pgfpathclose%
\pgfusepath{stroke,fill}%
\end{pgfscope}%
\begin{pgfscope}%
\pgfpathrectangle{\pgfqpoint{0.633874in}{2.920818in}}{\pgfqpoint{2.177280in}{2.201755in}}%
\pgfusepath{clip}%
\pgfsetbuttcap%
\pgfsetroundjoin%
\definecolor{currentfill}{rgb}{0.121569,0.466667,0.705882}%
\pgfsetfillcolor{currentfill}%
\pgfsetlinewidth{0.481800pt}%
\definecolor{currentstroke}{rgb}{1.000000,1.000000,1.000000}%
\pgfsetstrokecolor{currentstroke}%
\pgfsetdash{}{0pt}%
\pgfpathmoveto{\pgfqpoint{1.205550in}{3.182783in}}%
\pgfpathcurveto{\pgfqpoint{1.216600in}{3.182783in}}{\pgfqpoint{1.227199in}{3.187174in}}{\pgfqpoint{1.235013in}{3.194987in}}%
\pgfpathcurveto{\pgfqpoint{1.242826in}{3.202801in}}{\pgfqpoint{1.247217in}{3.213400in}}{\pgfqpoint{1.247217in}{3.224450in}}%
\pgfpathcurveto{\pgfqpoint{1.247217in}{3.235500in}}{\pgfqpoint{1.242826in}{3.246099in}}{\pgfqpoint{1.235013in}{3.253913in}}%
\pgfpathcurveto{\pgfqpoint{1.227199in}{3.261727in}}{\pgfqpoint{1.216600in}{3.266117in}}{\pgfqpoint{1.205550in}{3.266117in}}%
\pgfpathcurveto{\pgfqpoint{1.194500in}{3.266117in}}{\pgfqpoint{1.183901in}{3.261727in}}{\pgfqpoint{1.176087in}{3.253913in}}%
\pgfpathcurveto{\pgfqpoint{1.168274in}{3.246099in}}{\pgfqpoint{1.163883in}{3.235500in}}{\pgfqpoint{1.163883in}{3.224450in}}%
\pgfpathcurveto{\pgfqpoint{1.163883in}{3.213400in}}{\pgfqpoint{1.168274in}{3.202801in}}{\pgfqpoint{1.176087in}{3.194987in}}%
\pgfpathcurveto{\pgfqpoint{1.183901in}{3.187174in}}{\pgfqpoint{1.194500in}{3.182783in}}{\pgfqpoint{1.205550in}{3.182783in}}%
\pgfpathlineto{\pgfqpoint{1.205550in}{3.182783in}}%
\pgfpathclose%
\pgfusepath{stroke,fill}%
\end{pgfscope}%
\begin{pgfscope}%
\pgfpathrectangle{\pgfqpoint{0.633874in}{2.920818in}}{\pgfqpoint{2.177280in}{2.201755in}}%
\pgfusepath{clip}%
\pgfsetbuttcap%
\pgfsetroundjoin%
\definecolor{currentfill}{rgb}{0.121569,0.466667,0.705882}%
\pgfsetfillcolor{currentfill}%
\pgfsetlinewidth{0.481800pt}%
\definecolor{currentstroke}{rgb}{1.000000,1.000000,1.000000}%
\pgfsetstrokecolor{currentstroke}%
\pgfsetdash{}{0pt}%
\pgfpathmoveto{\pgfqpoint{1.205550in}{3.182783in}}%
\pgfpathcurveto{\pgfqpoint{1.216600in}{3.182783in}}{\pgfqpoint{1.227199in}{3.187174in}}{\pgfqpoint{1.235013in}{3.194987in}}%
\pgfpathcurveto{\pgfqpoint{1.242826in}{3.202801in}}{\pgfqpoint{1.247217in}{3.213400in}}{\pgfqpoint{1.247217in}{3.224450in}}%
\pgfpathcurveto{\pgfqpoint{1.247217in}{3.235500in}}{\pgfqpoint{1.242826in}{3.246099in}}{\pgfqpoint{1.235013in}{3.253913in}}%
\pgfpathcurveto{\pgfqpoint{1.227199in}{3.261727in}}{\pgfqpoint{1.216600in}{3.266117in}}{\pgfqpoint{1.205550in}{3.266117in}}%
\pgfpathcurveto{\pgfqpoint{1.194500in}{3.266117in}}{\pgfqpoint{1.183901in}{3.261727in}}{\pgfqpoint{1.176087in}{3.253913in}}%
\pgfpathcurveto{\pgfqpoint{1.168274in}{3.246099in}}{\pgfqpoint{1.163883in}{3.235500in}}{\pgfqpoint{1.163883in}{3.224450in}}%
\pgfpathcurveto{\pgfqpoint{1.163883in}{3.213400in}}{\pgfqpoint{1.168274in}{3.202801in}}{\pgfqpoint{1.176087in}{3.194987in}}%
\pgfpathcurveto{\pgfqpoint{1.183901in}{3.187174in}}{\pgfqpoint{1.194500in}{3.182783in}}{\pgfqpoint{1.205550in}{3.182783in}}%
\pgfpathlineto{\pgfqpoint{1.205550in}{3.182783in}}%
\pgfpathclose%
\pgfusepath{stroke,fill}%
\end{pgfscope}%
\begin{pgfscope}%
\pgfpathrectangle{\pgfqpoint{0.633874in}{2.920818in}}{\pgfqpoint{2.177280in}{2.201755in}}%
\pgfusepath{clip}%
\pgfsetbuttcap%
\pgfsetroundjoin%
\definecolor{currentfill}{rgb}{0.121569,0.466667,0.705882}%
\pgfsetfillcolor{currentfill}%
\pgfsetlinewidth{0.481800pt}%
\definecolor{currentstroke}{rgb}{1.000000,1.000000,1.000000}%
\pgfsetstrokecolor{currentstroke}%
\pgfsetdash{}{0pt}%
\pgfpathmoveto{\pgfqpoint{1.285428in}{3.148858in}}%
\pgfpathcurveto{\pgfqpoint{1.296478in}{3.148858in}}{\pgfqpoint{1.307077in}{3.153248in}}{\pgfqpoint{1.314890in}{3.161062in}}%
\pgfpathcurveto{\pgfqpoint{1.322704in}{3.168876in}}{\pgfqpoint{1.327094in}{3.179475in}}{\pgfqpoint{1.327094in}{3.190525in}}%
\pgfpathcurveto{\pgfqpoint{1.327094in}{3.201575in}}{\pgfqpoint{1.322704in}{3.212174in}}{\pgfqpoint{1.314890in}{3.219988in}}%
\pgfpathcurveto{\pgfqpoint{1.307077in}{3.227801in}}{\pgfqpoint{1.296478in}{3.232191in}}{\pgfqpoint{1.285428in}{3.232191in}}%
\pgfpathcurveto{\pgfqpoint{1.274377in}{3.232191in}}{\pgfqpoint{1.263778in}{3.227801in}}{\pgfqpoint{1.255965in}{3.219988in}}%
\pgfpathcurveto{\pgfqpoint{1.248151in}{3.212174in}}{\pgfqpoint{1.243761in}{3.201575in}}{\pgfqpoint{1.243761in}{3.190525in}}%
\pgfpathcurveto{\pgfqpoint{1.243761in}{3.179475in}}{\pgfqpoint{1.248151in}{3.168876in}}{\pgfqpoint{1.255965in}{3.161062in}}%
\pgfpathcurveto{\pgfqpoint{1.263778in}{3.153248in}}{\pgfqpoint{1.274377in}{3.148858in}}{\pgfqpoint{1.285428in}{3.148858in}}%
\pgfpathlineto{\pgfqpoint{1.285428in}{3.148858in}}%
\pgfpathclose%
\pgfusepath{stroke,fill}%
\end{pgfscope}%
\begin{pgfscope}%
\pgfpathrectangle{\pgfqpoint{0.633874in}{2.920818in}}{\pgfqpoint{2.177280in}{2.201755in}}%
\pgfusepath{clip}%
\pgfsetbuttcap%
\pgfsetroundjoin%
\definecolor{currentfill}{rgb}{0.121569,0.466667,0.705882}%
\pgfsetfillcolor{currentfill}%
\pgfsetlinewidth{0.481800pt}%
\definecolor{currentstroke}{rgb}{1.000000,1.000000,1.000000}%
\pgfsetstrokecolor{currentstroke}%
\pgfsetdash{}{0pt}%
\pgfpathmoveto{\pgfqpoint{1.285428in}{3.114933in}}%
\pgfpathcurveto{\pgfqpoint{1.296478in}{3.114933in}}{\pgfqpoint{1.307077in}{3.119323in}}{\pgfqpoint{1.314890in}{3.127137in}}%
\pgfpathcurveto{\pgfqpoint{1.322704in}{3.134950in}}{\pgfqpoint{1.327094in}{3.145549in}}{\pgfqpoint{1.327094in}{3.156599in}}%
\pgfpathcurveto{\pgfqpoint{1.327094in}{3.167650in}}{\pgfqpoint{1.322704in}{3.178249in}}{\pgfqpoint{1.314890in}{3.186062in}}%
\pgfpathcurveto{\pgfqpoint{1.307077in}{3.193876in}}{\pgfqpoint{1.296478in}{3.198266in}}{\pgfqpoint{1.285428in}{3.198266in}}%
\pgfpathcurveto{\pgfqpoint{1.274377in}{3.198266in}}{\pgfqpoint{1.263778in}{3.193876in}}{\pgfqpoint{1.255965in}{3.186062in}}%
\pgfpathcurveto{\pgfqpoint{1.248151in}{3.178249in}}{\pgfqpoint{1.243761in}{3.167650in}}{\pgfqpoint{1.243761in}{3.156599in}}%
\pgfpathcurveto{\pgfqpoint{1.243761in}{3.145549in}}{\pgfqpoint{1.248151in}{3.134950in}}{\pgfqpoint{1.255965in}{3.127137in}}%
\pgfpathcurveto{\pgfqpoint{1.263778in}{3.119323in}}{\pgfqpoint{1.274377in}{3.114933in}}{\pgfqpoint{1.285428in}{3.114933in}}%
\pgfpathlineto{\pgfqpoint{1.285428in}{3.114933in}}%
\pgfpathclose%
\pgfusepath{stroke,fill}%
\end{pgfscope}%
\begin{pgfscope}%
\pgfpathrectangle{\pgfqpoint{0.633874in}{2.920818in}}{\pgfqpoint{2.177280in}{2.201755in}}%
\pgfusepath{clip}%
\pgfsetbuttcap%
\pgfsetroundjoin%
\definecolor{currentfill}{rgb}{0.121569,0.466667,0.705882}%
\pgfsetfillcolor{currentfill}%
\pgfsetlinewidth{0.481800pt}%
\definecolor{currentstroke}{rgb}{1.000000,1.000000,1.000000}%
\pgfsetstrokecolor{currentstroke}%
\pgfsetdash{}{0pt}%
\pgfpathmoveto{\pgfqpoint{1.085734in}{3.182783in}}%
\pgfpathcurveto{\pgfqpoint{1.096784in}{3.182783in}}{\pgfqpoint{1.107383in}{3.187174in}}{\pgfqpoint{1.115197in}{3.194987in}}%
\pgfpathcurveto{\pgfqpoint{1.123010in}{3.202801in}}{\pgfqpoint{1.127401in}{3.213400in}}{\pgfqpoint{1.127401in}{3.224450in}}%
\pgfpathcurveto{\pgfqpoint{1.127401in}{3.235500in}}{\pgfqpoint{1.123010in}{3.246099in}}{\pgfqpoint{1.115197in}{3.253913in}}%
\pgfpathcurveto{\pgfqpoint{1.107383in}{3.261727in}}{\pgfqpoint{1.096784in}{3.266117in}}{\pgfqpoint{1.085734in}{3.266117in}}%
\pgfpathcurveto{\pgfqpoint{1.074684in}{3.266117in}}{\pgfqpoint{1.064085in}{3.261727in}}{\pgfqpoint{1.056271in}{3.253913in}}%
\pgfpathcurveto{\pgfqpoint{1.048458in}{3.246099in}}{\pgfqpoint{1.044067in}{3.235500in}}{\pgfqpoint{1.044067in}{3.224450in}}%
\pgfpathcurveto{\pgfqpoint{1.044067in}{3.213400in}}{\pgfqpoint{1.048458in}{3.202801in}}{\pgfqpoint{1.056271in}{3.194987in}}%
\pgfpathcurveto{\pgfqpoint{1.064085in}{3.187174in}}{\pgfqpoint{1.074684in}{3.182783in}}{\pgfqpoint{1.085734in}{3.182783in}}%
\pgfpathlineto{\pgfqpoint{1.085734in}{3.182783in}}%
\pgfpathclose%
\pgfusepath{stroke,fill}%
\end{pgfscope}%
\begin{pgfscope}%
\pgfpathrectangle{\pgfqpoint{0.633874in}{2.920818in}}{\pgfqpoint{2.177280in}{2.201755in}}%
\pgfusepath{clip}%
\pgfsetbuttcap%
\pgfsetroundjoin%
\definecolor{currentfill}{rgb}{0.121569,0.466667,0.705882}%
\pgfsetfillcolor{currentfill}%
\pgfsetlinewidth{0.481800pt}%
\definecolor{currentstroke}{rgb}{1.000000,1.000000,1.000000}%
\pgfsetstrokecolor{currentstroke}%
\pgfsetdash{}{0pt}%
\pgfpathmoveto{\pgfqpoint{1.125673in}{3.182783in}}%
\pgfpathcurveto{\pgfqpoint{1.136723in}{3.182783in}}{\pgfqpoint{1.147322in}{3.187174in}}{\pgfqpoint{1.155135in}{3.194987in}}%
\pgfpathcurveto{\pgfqpoint{1.162949in}{3.202801in}}{\pgfqpoint{1.167339in}{3.213400in}}{\pgfqpoint{1.167339in}{3.224450in}}%
\pgfpathcurveto{\pgfqpoint{1.167339in}{3.235500in}}{\pgfqpoint{1.162949in}{3.246099in}}{\pgfqpoint{1.155135in}{3.253913in}}%
\pgfpathcurveto{\pgfqpoint{1.147322in}{3.261727in}}{\pgfqpoint{1.136723in}{3.266117in}}{\pgfqpoint{1.125673in}{3.266117in}}%
\pgfpathcurveto{\pgfqpoint{1.114623in}{3.266117in}}{\pgfqpoint{1.104024in}{3.261727in}}{\pgfqpoint{1.096210in}{3.253913in}}%
\pgfpathcurveto{\pgfqpoint{1.088396in}{3.246099in}}{\pgfqpoint{1.084006in}{3.235500in}}{\pgfqpoint{1.084006in}{3.224450in}}%
\pgfpathcurveto{\pgfqpoint{1.084006in}{3.213400in}}{\pgfqpoint{1.088396in}{3.202801in}}{\pgfqpoint{1.096210in}{3.194987in}}%
\pgfpathcurveto{\pgfqpoint{1.104024in}{3.187174in}}{\pgfqpoint{1.114623in}{3.182783in}}{\pgfqpoint{1.125673in}{3.182783in}}%
\pgfpathlineto{\pgfqpoint{1.125673in}{3.182783in}}%
\pgfpathclose%
\pgfusepath{stroke,fill}%
\end{pgfscope}%
\begin{pgfscope}%
\pgfpathrectangle{\pgfqpoint{0.633874in}{2.920818in}}{\pgfqpoint{2.177280in}{2.201755in}}%
\pgfusepath{clip}%
\pgfsetbuttcap%
\pgfsetroundjoin%
\definecolor{currentfill}{rgb}{0.121569,0.466667,0.705882}%
\pgfsetfillcolor{currentfill}%
\pgfsetlinewidth{0.481800pt}%
\definecolor{currentstroke}{rgb}{1.000000,1.000000,1.000000}%
\pgfsetstrokecolor{currentstroke}%
\pgfsetdash{}{0pt}%
\pgfpathmoveto{\pgfqpoint{1.365305in}{3.148858in}}%
\pgfpathcurveto{\pgfqpoint{1.376355in}{3.148858in}}{\pgfqpoint{1.386954in}{3.153248in}}{\pgfqpoint{1.394768in}{3.161062in}}%
\pgfpathcurveto{\pgfqpoint{1.402581in}{3.168876in}}{\pgfqpoint{1.406972in}{3.179475in}}{\pgfqpoint{1.406972in}{3.190525in}}%
\pgfpathcurveto{\pgfqpoint{1.406972in}{3.201575in}}{\pgfqpoint{1.402581in}{3.212174in}}{\pgfqpoint{1.394768in}{3.219988in}}%
\pgfpathcurveto{\pgfqpoint{1.386954in}{3.227801in}}{\pgfqpoint{1.376355in}{3.232191in}}{\pgfqpoint{1.365305in}{3.232191in}}%
\pgfpathcurveto{\pgfqpoint{1.354255in}{3.232191in}}{\pgfqpoint{1.343656in}{3.227801in}}{\pgfqpoint{1.335842in}{3.219988in}}%
\pgfpathcurveto{\pgfqpoint{1.328029in}{3.212174in}}{\pgfqpoint{1.323638in}{3.201575in}}{\pgfqpoint{1.323638in}{3.190525in}}%
\pgfpathcurveto{\pgfqpoint{1.323638in}{3.179475in}}{\pgfqpoint{1.328029in}{3.168876in}}{\pgfqpoint{1.335842in}{3.161062in}}%
\pgfpathcurveto{\pgfqpoint{1.343656in}{3.153248in}}{\pgfqpoint{1.354255in}{3.148858in}}{\pgfqpoint{1.365305in}{3.148858in}}%
\pgfpathlineto{\pgfqpoint{1.365305in}{3.148858in}}%
\pgfpathclose%
\pgfusepath{stroke,fill}%
\end{pgfscope}%
\begin{pgfscope}%
\pgfpathrectangle{\pgfqpoint{0.633874in}{2.920818in}}{\pgfqpoint{2.177280in}{2.201755in}}%
\pgfusepath{clip}%
\pgfsetbuttcap%
\pgfsetroundjoin%
\definecolor{currentfill}{rgb}{0.121569,0.466667,0.705882}%
\pgfsetfillcolor{currentfill}%
\pgfsetlinewidth{0.481800pt}%
\definecolor{currentstroke}{rgb}{1.000000,1.000000,1.000000}%
\pgfsetstrokecolor{currentstroke}%
\pgfsetdash{}{0pt}%
\pgfpathmoveto{\pgfqpoint{1.285428in}{3.148858in}}%
\pgfpathcurveto{\pgfqpoint{1.296478in}{3.148858in}}{\pgfqpoint{1.307077in}{3.153248in}}{\pgfqpoint{1.314890in}{3.161062in}}%
\pgfpathcurveto{\pgfqpoint{1.322704in}{3.168876in}}{\pgfqpoint{1.327094in}{3.179475in}}{\pgfqpoint{1.327094in}{3.190525in}}%
\pgfpathcurveto{\pgfqpoint{1.327094in}{3.201575in}}{\pgfqpoint{1.322704in}{3.212174in}}{\pgfqpoint{1.314890in}{3.219988in}}%
\pgfpathcurveto{\pgfqpoint{1.307077in}{3.227801in}}{\pgfqpoint{1.296478in}{3.232191in}}{\pgfqpoint{1.285428in}{3.232191in}}%
\pgfpathcurveto{\pgfqpoint{1.274377in}{3.232191in}}{\pgfqpoint{1.263778in}{3.227801in}}{\pgfqpoint{1.255965in}{3.219988in}}%
\pgfpathcurveto{\pgfqpoint{1.248151in}{3.212174in}}{\pgfqpoint{1.243761in}{3.201575in}}{\pgfqpoint{1.243761in}{3.190525in}}%
\pgfpathcurveto{\pgfqpoint{1.243761in}{3.179475in}}{\pgfqpoint{1.248151in}{3.168876in}}{\pgfqpoint{1.255965in}{3.161062in}}%
\pgfpathcurveto{\pgfqpoint{1.263778in}{3.153248in}}{\pgfqpoint{1.274377in}{3.148858in}}{\pgfqpoint{1.285428in}{3.148858in}}%
\pgfpathlineto{\pgfqpoint{1.285428in}{3.148858in}}%
\pgfpathclose%
\pgfusepath{stroke,fill}%
\end{pgfscope}%
\begin{pgfscope}%
\pgfpathrectangle{\pgfqpoint{0.633874in}{2.920818in}}{\pgfqpoint{2.177280in}{2.201755in}}%
\pgfusepath{clip}%
\pgfsetbuttcap%
\pgfsetroundjoin%
\definecolor{currentfill}{rgb}{0.121569,0.466667,0.705882}%
\pgfsetfillcolor{currentfill}%
\pgfsetlinewidth{0.481800pt}%
\definecolor{currentstroke}{rgb}{1.000000,1.000000,1.000000}%
\pgfsetstrokecolor{currentstroke}%
\pgfsetdash{}{0pt}%
\pgfpathmoveto{\pgfqpoint{1.405244in}{3.114933in}}%
\pgfpathcurveto{\pgfqpoint{1.416294in}{3.114933in}}{\pgfqpoint{1.426893in}{3.119323in}}{\pgfqpoint{1.434706in}{3.127137in}}%
\pgfpathcurveto{\pgfqpoint{1.442520in}{3.134950in}}{\pgfqpoint{1.446910in}{3.145549in}}{\pgfqpoint{1.446910in}{3.156599in}}%
\pgfpathcurveto{\pgfqpoint{1.446910in}{3.167650in}}{\pgfqpoint{1.442520in}{3.178249in}}{\pgfqpoint{1.434706in}{3.186062in}}%
\pgfpathcurveto{\pgfqpoint{1.426893in}{3.193876in}}{\pgfqpoint{1.416294in}{3.198266in}}{\pgfqpoint{1.405244in}{3.198266in}}%
\pgfpathcurveto{\pgfqpoint{1.394193in}{3.198266in}}{\pgfqpoint{1.383594in}{3.193876in}}{\pgfqpoint{1.375781in}{3.186062in}}%
\pgfpathcurveto{\pgfqpoint{1.367967in}{3.178249in}}{\pgfqpoint{1.363577in}{3.167650in}}{\pgfqpoint{1.363577in}{3.156599in}}%
\pgfpathcurveto{\pgfqpoint{1.363577in}{3.145549in}}{\pgfqpoint{1.367967in}{3.134950in}}{\pgfqpoint{1.375781in}{3.127137in}}%
\pgfpathcurveto{\pgfqpoint{1.383594in}{3.119323in}}{\pgfqpoint{1.394193in}{3.114933in}}{\pgfqpoint{1.405244in}{3.114933in}}%
\pgfpathlineto{\pgfqpoint{1.405244in}{3.114933in}}%
\pgfpathclose%
\pgfusepath{stroke,fill}%
\end{pgfscope}%
\begin{pgfscope}%
\pgfpathrectangle{\pgfqpoint{0.633874in}{2.920818in}}{\pgfqpoint{2.177280in}{2.201755in}}%
\pgfusepath{clip}%
\pgfsetbuttcap%
\pgfsetroundjoin%
\definecolor{currentfill}{rgb}{0.121569,0.466667,0.705882}%
\pgfsetfillcolor{currentfill}%
\pgfsetlinewidth{0.481800pt}%
\definecolor{currentstroke}{rgb}{1.000000,1.000000,1.000000}%
\pgfsetstrokecolor{currentstroke}%
\pgfsetdash{}{0pt}%
\pgfpathmoveto{\pgfqpoint{1.165611in}{3.148858in}}%
\pgfpathcurveto{\pgfqpoint{1.176662in}{3.148858in}}{\pgfqpoint{1.187261in}{3.153248in}}{\pgfqpoint{1.195074in}{3.161062in}}%
\pgfpathcurveto{\pgfqpoint{1.202888in}{3.168876in}}{\pgfqpoint{1.207278in}{3.179475in}}{\pgfqpoint{1.207278in}{3.190525in}}%
\pgfpathcurveto{\pgfqpoint{1.207278in}{3.201575in}}{\pgfqpoint{1.202888in}{3.212174in}}{\pgfqpoint{1.195074in}{3.219988in}}%
\pgfpathcurveto{\pgfqpoint{1.187261in}{3.227801in}}{\pgfqpoint{1.176662in}{3.232191in}}{\pgfqpoint{1.165611in}{3.232191in}}%
\pgfpathcurveto{\pgfqpoint{1.154561in}{3.232191in}}{\pgfqpoint{1.143962in}{3.227801in}}{\pgfqpoint{1.136149in}{3.219988in}}%
\pgfpathcurveto{\pgfqpoint{1.128335in}{3.212174in}}{\pgfqpoint{1.123945in}{3.201575in}}{\pgfqpoint{1.123945in}{3.190525in}}%
\pgfpathcurveto{\pgfqpoint{1.123945in}{3.179475in}}{\pgfqpoint{1.128335in}{3.168876in}}{\pgfqpoint{1.136149in}{3.161062in}}%
\pgfpathcurveto{\pgfqpoint{1.143962in}{3.153248in}}{\pgfqpoint{1.154561in}{3.148858in}}{\pgfqpoint{1.165611in}{3.148858in}}%
\pgfpathlineto{\pgfqpoint{1.165611in}{3.148858in}}%
\pgfpathclose%
\pgfusepath{stroke,fill}%
\end{pgfscope}%
\begin{pgfscope}%
\pgfpathrectangle{\pgfqpoint{0.633874in}{2.920818in}}{\pgfqpoint{2.177280in}{2.201755in}}%
\pgfusepath{clip}%
\pgfsetbuttcap%
\pgfsetroundjoin%
\definecolor{currentfill}{rgb}{0.121569,0.466667,0.705882}%
\pgfsetfillcolor{currentfill}%
\pgfsetlinewidth{0.481800pt}%
\definecolor{currentstroke}{rgb}{1.000000,1.000000,1.000000}%
\pgfsetstrokecolor{currentstroke}%
\pgfsetdash{}{0pt}%
\pgfpathmoveto{\pgfqpoint{1.205550in}{3.047082in}}%
\pgfpathcurveto{\pgfqpoint{1.216600in}{3.047082in}}{\pgfqpoint{1.227199in}{3.051472in}}{\pgfqpoint{1.235013in}{3.059286in}}%
\pgfpathcurveto{\pgfqpoint{1.242826in}{3.067100in}}{\pgfqpoint{1.247217in}{3.077699in}}{\pgfqpoint{1.247217in}{3.088749in}}%
\pgfpathcurveto{\pgfqpoint{1.247217in}{3.099799in}}{\pgfqpoint{1.242826in}{3.110398in}}{\pgfqpoint{1.235013in}{3.118212in}}%
\pgfpathcurveto{\pgfqpoint{1.227199in}{3.126025in}}{\pgfqpoint{1.216600in}{3.130415in}}{\pgfqpoint{1.205550in}{3.130415in}}%
\pgfpathcurveto{\pgfqpoint{1.194500in}{3.130415in}}{\pgfqpoint{1.183901in}{3.126025in}}{\pgfqpoint{1.176087in}{3.118212in}}%
\pgfpathcurveto{\pgfqpoint{1.168274in}{3.110398in}}{\pgfqpoint{1.163883in}{3.099799in}}{\pgfqpoint{1.163883in}{3.088749in}}%
\pgfpathcurveto{\pgfqpoint{1.163883in}{3.077699in}}{\pgfqpoint{1.168274in}{3.067100in}}{\pgfqpoint{1.176087in}{3.059286in}}%
\pgfpathcurveto{\pgfqpoint{1.183901in}{3.051472in}}{\pgfqpoint{1.194500in}{3.047082in}}{\pgfqpoint{1.205550in}{3.047082in}}%
\pgfpathlineto{\pgfqpoint{1.205550in}{3.047082in}}%
\pgfpathclose%
\pgfusepath{stroke,fill}%
\end{pgfscope}%
\begin{pgfscope}%
\pgfpathrectangle{\pgfqpoint{0.633874in}{2.920818in}}{\pgfqpoint{2.177280in}{2.201755in}}%
\pgfusepath{clip}%
\pgfsetbuttcap%
\pgfsetroundjoin%
\definecolor{currentfill}{rgb}{0.121569,0.466667,0.705882}%
\pgfsetfillcolor{currentfill}%
\pgfsetlinewidth{0.481800pt}%
\definecolor{currentstroke}{rgb}{1.000000,1.000000,1.000000}%
\pgfsetstrokecolor{currentstroke}%
\pgfsetdash{}{0pt}%
\pgfpathmoveto{\pgfqpoint{1.405244in}{3.081007in}}%
\pgfpathcurveto{\pgfqpoint{1.416294in}{3.081007in}}{\pgfqpoint{1.426893in}{3.085398in}}{\pgfqpoint{1.434706in}{3.093211in}}%
\pgfpathcurveto{\pgfqpoint{1.442520in}{3.101025in}}{\pgfqpoint{1.446910in}{3.111624in}}{\pgfqpoint{1.446910in}{3.122674in}}%
\pgfpathcurveto{\pgfqpoint{1.446910in}{3.133724in}}{\pgfqpoint{1.442520in}{3.144323in}}{\pgfqpoint{1.434706in}{3.152137in}}%
\pgfpathcurveto{\pgfqpoint{1.426893in}{3.159951in}}{\pgfqpoint{1.416294in}{3.164341in}}{\pgfqpoint{1.405244in}{3.164341in}}%
\pgfpathcurveto{\pgfqpoint{1.394193in}{3.164341in}}{\pgfqpoint{1.383594in}{3.159951in}}{\pgfqpoint{1.375781in}{3.152137in}}%
\pgfpathcurveto{\pgfqpoint{1.367967in}{3.144323in}}{\pgfqpoint{1.363577in}{3.133724in}}{\pgfqpoint{1.363577in}{3.122674in}}%
\pgfpathcurveto{\pgfqpoint{1.363577in}{3.111624in}}{\pgfqpoint{1.367967in}{3.101025in}}{\pgfqpoint{1.375781in}{3.093211in}}%
\pgfpathcurveto{\pgfqpoint{1.383594in}{3.085398in}}{\pgfqpoint{1.394193in}{3.081007in}}{\pgfqpoint{1.405244in}{3.081007in}}%
\pgfpathlineto{\pgfqpoint{1.405244in}{3.081007in}}%
\pgfpathclose%
\pgfusepath{stroke,fill}%
\end{pgfscope}%
\begin{pgfscope}%
\pgfpathrectangle{\pgfqpoint{0.633874in}{2.920818in}}{\pgfqpoint{2.177280in}{2.201755in}}%
\pgfusepath{clip}%
\pgfsetbuttcap%
\pgfsetroundjoin%
\definecolor{currentfill}{rgb}{0.121569,0.466667,0.705882}%
\pgfsetfillcolor{currentfill}%
\pgfsetlinewidth{0.481800pt}%
\definecolor{currentstroke}{rgb}{1.000000,1.000000,1.000000}%
\pgfsetstrokecolor{currentstroke}%
\pgfsetdash{}{0pt}%
\pgfpathmoveto{\pgfqpoint{1.165611in}{3.114933in}}%
\pgfpathcurveto{\pgfqpoint{1.176662in}{3.114933in}}{\pgfqpoint{1.187261in}{3.119323in}}{\pgfqpoint{1.195074in}{3.127137in}}%
\pgfpathcurveto{\pgfqpoint{1.202888in}{3.134950in}}{\pgfqpoint{1.207278in}{3.145549in}}{\pgfqpoint{1.207278in}{3.156599in}}%
\pgfpathcurveto{\pgfqpoint{1.207278in}{3.167650in}}{\pgfqpoint{1.202888in}{3.178249in}}{\pgfqpoint{1.195074in}{3.186062in}}%
\pgfpathcurveto{\pgfqpoint{1.187261in}{3.193876in}}{\pgfqpoint{1.176662in}{3.198266in}}{\pgfqpoint{1.165611in}{3.198266in}}%
\pgfpathcurveto{\pgfqpoint{1.154561in}{3.198266in}}{\pgfqpoint{1.143962in}{3.193876in}}{\pgfqpoint{1.136149in}{3.186062in}}%
\pgfpathcurveto{\pgfqpoint{1.128335in}{3.178249in}}{\pgfqpoint{1.123945in}{3.167650in}}{\pgfqpoint{1.123945in}{3.156599in}}%
\pgfpathcurveto{\pgfqpoint{1.123945in}{3.145549in}}{\pgfqpoint{1.128335in}{3.134950in}}{\pgfqpoint{1.136149in}{3.127137in}}%
\pgfpathcurveto{\pgfqpoint{1.143962in}{3.119323in}}{\pgfqpoint{1.154561in}{3.114933in}}{\pgfqpoint{1.165611in}{3.114933in}}%
\pgfpathlineto{\pgfqpoint{1.165611in}{3.114933in}}%
\pgfpathclose%
\pgfusepath{stroke,fill}%
\end{pgfscope}%
\begin{pgfscope}%
\pgfpathrectangle{\pgfqpoint{0.633874in}{2.920818in}}{\pgfqpoint{2.177280in}{2.201755in}}%
\pgfusepath{clip}%
\pgfsetbuttcap%
\pgfsetroundjoin%
\definecolor{currentfill}{rgb}{0.121569,0.466667,0.705882}%
\pgfsetfillcolor{currentfill}%
\pgfsetlinewidth{0.481800pt}%
\definecolor{currentstroke}{rgb}{1.000000,1.000000,1.000000}%
\pgfsetstrokecolor{currentstroke}%
\pgfsetdash{}{0pt}%
\pgfpathmoveto{\pgfqpoint{0.965918in}{3.081007in}}%
\pgfpathcurveto{\pgfqpoint{0.976968in}{3.081007in}}{\pgfqpoint{0.987567in}{3.085398in}}{\pgfqpoint{0.995381in}{3.093211in}}%
\pgfpathcurveto{\pgfqpoint{1.003194in}{3.101025in}}{\pgfqpoint{1.007585in}{3.111624in}}{\pgfqpoint{1.007585in}{3.122674in}}%
\pgfpathcurveto{\pgfqpoint{1.007585in}{3.133724in}}{\pgfqpoint{1.003194in}{3.144323in}}{\pgfqpoint{0.995381in}{3.152137in}}%
\pgfpathcurveto{\pgfqpoint{0.987567in}{3.159951in}}{\pgfqpoint{0.976968in}{3.164341in}}{\pgfqpoint{0.965918in}{3.164341in}}%
\pgfpathcurveto{\pgfqpoint{0.954868in}{3.164341in}}{\pgfqpoint{0.944269in}{3.159951in}}{\pgfqpoint{0.936455in}{3.152137in}}%
\pgfpathcurveto{\pgfqpoint{0.928641in}{3.144323in}}{\pgfqpoint{0.924251in}{3.133724in}}{\pgfqpoint{0.924251in}{3.122674in}}%
\pgfpathcurveto{\pgfqpoint{0.924251in}{3.111624in}}{\pgfqpoint{0.928641in}{3.101025in}}{\pgfqpoint{0.936455in}{3.093211in}}%
\pgfpathcurveto{\pgfqpoint{0.944269in}{3.085398in}}{\pgfqpoint{0.954868in}{3.081007in}}{\pgfqpoint{0.965918in}{3.081007in}}%
\pgfpathlineto{\pgfqpoint{0.965918in}{3.081007in}}%
\pgfpathclose%
\pgfusepath{stroke,fill}%
\end{pgfscope}%
\begin{pgfscope}%
\pgfpathrectangle{\pgfqpoint{0.633874in}{2.920818in}}{\pgfqpoint{2.177280in}{2.201755in}}%
\pgfusepath{clip}%
\pgfsetbuttcap%
\pgfsetroundjoin%
\definecolor{currentfill}{rgb}{0.121569,0.466667,0.705882}%
\pgfsetfillcolor{currentfill}%
\pgfsetlinewidth{0.481800pt}%
\definecolor{currentstroke}{rgb}{1.000000,1.000000,1.000000}%
\pgfsetstrokecolor{currentstroke}%
\pgfsetdash{}{0pt}%
\pgfpathmoveto{\pgfqpoint{1.245489in}{3.148858in}}%
\pgfpathcurveto{\pgfqpoint{1.256539in}{3.148858in}}{\pgfqpoint{1.267138in}{3.153248in}}{\pgfqpoint{1.274952in}{3.161062in}}%
\pgfpathcurveto{\pgfqpoint{1.282765in}{3.168876in}}{\pgfqpoint{1.287155in}{3.179475in}}{\pgfqpoint{1.287155in}{3.190525in}}%
\pgfpathcurveto{\pgfqpoint{1.287155in}{3.201575in}}{\pgfqpoint{1.282765in}{3.212174in}}{\pgfqpoint{1.274952in}{3.219988in}}%
\pgfpathcurveto{\pgfqpoint{1.267138in}{3.227801in}}{\pgfqpoint{1.256539in}{3.232191in}}{\pgfqpoint{1.245489in}{3.232191in}}%
\pgfpathcurveto{\pgfqpoint{1.234439in}{3.232191in}}{\pgfqpoint{1.223840in}{3.227801in}}{\pgfqpoint{1.216026in}{3.219988in}}%
\pgfpathcurveto{\pgfqpoint{1.208212in}{3.212174in}}{\pgfqpoint{1.203822in}{3.201575in}}{\pgfqpoint{1.203822in}{3.190525in}}%
\pgfpathcurveto{\pgfqpoint{1.203822in}{3.179475in}}{\pgfqpoint{1.208212in}{3.168876in}}{\pgfqpoint{1.216026in}{3.161062in}}%
\pgfpathcurveto{\pgfqpoint{1.223840in}{3.153248in}}{\pgfqpoint{1.234439in}{3.148858in}}{\pgfqpoint{1.245489in}{3.148858in}}%
\pgfpathlineto{\pgfqpoint{1.245489in}{3.148858in}}%
\pgfpathclose%
\pgfusepath{stroke,fill}%
\end{pgfscope}%
\begin{pgfscope}%
\pgfpathrectangle{\pgfqpoint{0.633874in}{2.920818in}}{\pgfqpoint{2.177280in}{2.201755in}}%
\pgfusepath{clip}%
\pgfsetbuttcap%
\pgfsetroundjoin%
\definecolor{currentfill}{rgb}{0.121569,0.466667,0.705882}%
\pgfsetfillcolor{currentfill}%
\pgfsetlinewidth{0.481800pt}%
\definecolor{currentstroke}{rgb}{1.000000,1.000000,1.000000}%
\pgfsetstrokecolor{currentstroke}%
\pgfsetdash{}{0pt}%
\pgfpathmoveto{\pgfqpoint{1.205550in}{3.081007in}}%
\pgfpathcurveto{\pgfqpoint{1.216600in}{3.081007in}}{\pgfqpoint{1.227199in}{3.085398in}}{\pgfqpoint{1.235013in}{3.093211in}}%
\pgfpathcurveto{\pgfqpoint{1.242826in}{3.101025in}}{\pgfqpoint{1.247217in}{3.111624in}}{\pgfqpoint{1.247217in}{3.122674in}}%
\pgfpathcurveto{\pgfqpoint{1.247217in}{3.133724in}}{\pgfqpoint{1.242826in}{3.144323in}}{\pgfqpoint{1.235013in}{3.152137in}}%
\pgfpathcurveto{\pgfqpoint{1.227199in}{3.159951in}}{\pgfqpoint{1.216600in}{3.164341in}}{\pgfqpoint{1.205550in}{3.164341in}}%
\pgfpathcurveto{\pgfqpoint{1.194500in}{3.164341in}}{\pgfqpoint{1.183901in}{3.159951in}}{\pgfqpoint{1.176087in}{3.152137in}}%
\pgfpathcurveto{\pgfqpoint{1.168274in}{3.144323in}}{\pgfqpoint{1.163883in}{3.133724in}}{\pgfqpoint{1.163883in}{3.122674in}}%
\pgfpathcurveto{\pgfqpoint{1.163883in}{3.111624in}}{\pgfqpoint{1.168274in}{3.101025in}}{\pgfqpoint{1.176087in}{3.093211in}}%
\pgfpathcurveto{\pgfqpoint{1.183901in}{3.085398in}}{\pgfqpoint{1.194500in}{3.081007in}}{\pgfqpoint{1.205550in}{3.081007in}}%
\pgfpathlineto{\pgfqpoint{1.205550in}{3.081007in}}%
\pgfpathclose%
\pgfusepath{stroke,fill}%
\end{pgfscope}%
\begin{pgfscope}%
\pgfpathrectangle{\pgfqpoint{0.633874in}{2.920818in}}{\pgfqpoint{2.177280in}{2.201755in}}%
\pgfusepath{clip}%
\pgfsetbuttcap%
\pgfsetroundjoin%
\definecolor{currentfill}{rgb}{0.121569,0.466667,0.705882}%
\pgfsetfillcolor{currentfill}%
\pgfsetlinewidth{0.481800pt}%
\definecolor{currentstroke}{rgb}{1.000000,1.000000,1.000000}%
\pgfsetstrokecolor{currentstroke}%
\pgfsetdash{}{0pt}%
\pgfpathmoveto{\pgfqpoint{1.005857in}{3.081007in}}%
\pgfpathcurveto{\pgfqpoint{1.016907in}{3.081007in}}{\pgfqpoint{1.027506in}{3.085398in}}{\pgfqpoint{1.035319in}{3.093211in}}%
\pgfpathcurveto{\pgfqpoint{1.043133in}{3.101025in}}{\pgfqpoint{1.047523in}{3.111624in}}{\pgfqpoint{1.047523in}{3.122674in}}%
\pgfpathcurveto{\pgfqpoint{1.047523in}{3.133724in}}{\pgfqpoint{1.043133in}{3.144323in}}{\pgfqpoint{1.035319in}{3.152137in}}%
\pgfpathcurveto{\pgfqpoint{1.027506in}{3.159951in}}{\pgfqpoint{1.016907in}{3.164341in}}{\pgfqpoint{1.005857in}{3.164341in}}%
\pgfpathcurveto{\pgfqpoint{0.994806in}{3.164341in}}{\pgfqpoint{0.984207in}{3.159951in}}{\pgfqpoint{0.976394in}{3.152137in}}%
\pgfpathcurveto{\pgfqpoint{0.968580in}{3.144323in}}{\pgfqpoint{0.964190in}{3.133724in}}{\pgfqpoint{0.964190in}{3.122674in}}%
\pgfpathcurveto{\pgfqpoint{0.964190in}{3.111624in}}{\pgfqpoint{0.968580in}{3.101025in}}{\pgfqpoint{0.976394in}{3.093211in}}%
\pgfpathcurveto{\pgfqpoint{0.984207in}{3.085398in}}{\pgfqpoint{0.994806in}{3.081007in}}{\pgfqpoint{1.005857in}{3.081007in}}%
\pgfpathlineto{\pgfqpoint{1.005857in}{3.081007in}}%
\pgfpathclose%
\pgfusepath{stroke,fill}%
\end{pgfscope}%
\begin{pgfscope}%
\pgfpathrectangle{\pgfqpoint{0.633874in}{2.920818in}}{\pgfqpoint{2.177280in}{2.201755in}}%
\pgfusepath{clip}%
\pgfsetbuttcap%
\pgfsetroundjoin%
\definecolor{currentfill}{rgb}{0.121569,0.466667,0.705882}%
\pgfsetfillcolor{currentfill}%
\pgfsetlinewidth{0.481800pt}%
\definecolor{currentstroke}{rgb}{1.000000,1.000000,1.000000}%
\pgfsetstrokecolor{currentstroke}%
\pgfsetdash{}{0pt}%
\pgfpathmoveto{\pgfqpoint{0.965918in}{3.081007in}}%
\pgfpathcurveto{\pgfqpoint{0.976968in}{3.081007in}}{\pgfqpoint{0.987567in}{3.085398in}}{\pgfqpoint{0.995381in}{3.093211in}}%
\pgfpathcurveto{\pgfqpoint{1.003194in}{3.101025in}}{\pgfqpoint{1.007585in}{3.111624in}}{\pgfqpoint{1.007585in}{3.122674in}}%
\pgfpathcurveto{\pgfqpoint{1.007585in}{3.133724in}}{\pgfqpoint{1.003194in}{3.144323in}}{\pgfqpoint{0.995381in}{3.152137in}}%
\pgfpathcurveto{\pgfqpoint{0.987567in}{3.159951in}}{\pgfqpoint{0.976968in}{3.164341in}}{\pgfqpoint{0.965918in}{3.164341in}}%
\pgfpathcurveto{\pgfqpoint{0.954868in}{3.164341in}}{\pgfqpoint{0.944269in}{3.159951in}}{\pgfqpoint{0.936455in}{3.152137in}}%
\pgfpathcurveto{\pgfqpoint{0.928641in}{3.144323in}}{\pgfqpoint{0.924251in}{3.133724in}}{\pgfqpoint{0.924251in}{3.122674in}}%
\pgfpathcurveto{\pgfqpoint{0.924251in}{3.111624in}}{\pgfqpoint{0.928641in}{3.101025in}}{\pgfqpoint{0.936455in}{3.093211in}}%
\pgfpathcurveto{\pgfqpoint{0.944269in}{3.085398in}}{\pgfqpoint{0.954868in}{3.081007in}}{\pgfqpoint{0.965918in}{3.081007in}}%
\pgfpathlineto{\pgfqpoint{0.965918in}{3.081007in}}%
\pgfpathclose%
\pgfusepath{stroke,fill}%
\end{pgfscope}%
\begin{pgfscope}%
\pgfpathrectangle{\pgfqpoint{0.633874in}{2.920818in}}{\pgfqpoint{2.177280in}{2.201755in}}%
\pgfusepath{clip}%
\pgfsetbuttcap%
\pgfsetroundjoin%
\definecolor{currentfill}{rgb}{0.121569,0.466667,0.705882}%
\pgfsetfillcolor{currentfill}%
\pgfsetlinewidth{0.481800pt}%
\definecolor{currentstroke}{rgb}{1.000000,1.000000,1.000000}%
\pgfsetstrokecolor{currentstroke}%
\pgfsetdash{}{0pt}%
\pgfpathmoveto{\pgfqpoint{1.205550in}{3.182783in}}%
\pgfpathcurveto{\pgfqpoint{1.216600in}{3.182783in}}{\pgfqpoint{1.227199in}{3.187174in}}{\pgfqpoint{1.235013in}{3.194987in}}%
\pgfpathcurveto{\pgfqpoint{1.242826in}{3.202801in}}{\pgfqpoint{1.247217in}{3.213400in}}{\pgfqpoint{1.247217in}{3.224450in}}%
\pgfpathcurveto{\pgfqpoint{1.247217in}{3.235500in}}{\pgfqpoint{1.242826in}{3.246099in}}{\pgfqpoint{1.235013in}{3.253913in}}%
\pgfpathcurveto{\pgfqpoint{1.227199in}{3.261727in}}{\pgfqpoint{1.216600in}{3.266117in}}{\pgfqpoint{1.205550in}{3.266117in}}%
\pgfpathcurveto{\pgfqpoint{1.194500in}{3.266117in}}{\pgfqpoint{1.183901in}{3.261727in}}{\pgfqpoint{1.176087in}{3.253913in}}%
\pgfpathcurveto{\pgfqpoint{1.168274in}{3.246099in}}{\pgfqpoint{1.163883in}{3.235500in}}{\pgfqpoint{1.163883in}{3.224450in}}%
\pgfpathcurveto{\pgfqpoint{1.163883in}{3.213400in}}{\pgfqpoint{1.168274in}{3.202801in}}{\pgfqpoint{1.176087in}{3.194987in}}%
\pgfpathcurveto{\pgfqpoint{1.183901in}{3.187174in}}{\pgfqpoint{1.194500in}{3.182783in}}{\pgfqpoint{1.205550in}{3.182783in}}%
\pgfpathlineto{\pgfqpoint{1.205550in}{3.182783in}}%
\pgfpathclose%
\pgfusepath{stroke,fill}%
\end{pgfscope}%
\begin{pgfscope}%
\pgfpathrectangle{\pgfqpoint{0.633874in}{2.920818in}}{\pgfqpoint{2.177280in}{2.201755in}}%
\pgfusepath{clip}%
\pgfsetbuttcap%
\pgfsetroundjoin%
\definecolor{currentfill}{rgb}{0.121569,0.466667,0.705882}%
\pgfsetfillcolor{currentfill}%
\pgfsetlinewidth{0.481800pt}%
\definecolor{currentstroke}{rgb}{1.000000,1.000000,1.000000}%
\pgfsetstrokecolor{currentstroke}%
\pgfsetdash{}{0pt}%
\pgfpathmoveto{\pgfqpoint{1.245489in}{3.284560in}}%
\pgfpathcurveto{\pgfqpoint{1.256539in}{3.284560in}}{\pgfqpoint{1.267138in}{3.288950in}}{\pgfqpoint{1.274952in}{3.296763in}}%
\pgfpathcurveto{\pgfqpoint{1.282765in}{3.304577in}}{\pgfqpoint{1.287155in}{3.315176in}}{\pgfqpoint{1.287155in}{3.326226in}}%
\pgfpathcurveto{\pgfqpoint{1.287155in}{3.337276in}}{\pgfqpoint{1.282765in}{3.347875in}}{\pgfqpoint{1.274952in}{3.355689in}}%
\pgfpathcurveto{\pgfqpoint{1.267138in}{3.363503in}}{\pgfqpoint{1.256539in}{3.367893in}}{\pgfqpoint{1.245489in}{3.367893in}}%
\pgfpathcurveto{\pgfqpoint{1.234439in}{3.367893in}}{\pgfqpoint{1.223840in}{3.363503in}}{\pgfqpoint{1.216026in}{3.355689in}}%
\pgfpathcurveto{\pgfqpoint{1.208212in}{3.347875in}}{\pgfqpoint{1.203822in}{3.337276in}}{\pgfqpoint{1.203822in}{3.326226in}}%
\pgfpathcurveto{\pgfqpoint{1.203822in}{3.315176in}}{\pgfqpoint{1.208212in}{3.304577in}}{\pgfqpoint{1.216026in}{3.296763in}}%
\pgfpathcurveto{\pgfqpoint{1.223840in}{3.288950in}}{\pgfqpoint{1.234439in}{3.284560in}}{\pgfqpoint{1.245489in}{3.284560in}}%
\pgfpathlineto{\pgfqpoint{1.245489in}{3.284560in}}%
\pgfpathclose%
\pgfusepath{stroke,fill}%
\end{pgfscope}%
\begin{pgfscope}%
\pgfpathrectangle{\pgfqpoint{0.633874in}{2.920818in}}{\pgfqpoint{2.177280in}{2.201755in}}%
\pgfusepath{clip}%
\pgfsetbuttcap%
\pgfsetroundjoin%
\definecolor{currentfill}{rgb}{0.121569,0.466667,0.705882}%
\pgfsetfillcolor{currentfill}%
\pgfsetlinewidth{0.481800pt}%
\definecolor{currentstroke}{rgb}{1.000000,1.000000,1.000000}%
\pgfsetstrokecolor{currentstroke}%
\pgfsetdash{}{0pt}%
\pgfpathmoveto{\pgfqpoint{1.125673in}{3.114933in}}%
\pgfpathcurveto{\pgfqpoint{1.136723in}{3.114933in}}{\pgfqpoint{1.147322in}{3.119323in}}{\pgfqpoint{1.155135in}{3.127137in}}%
\pgfpathcurveto{\pgfqpoint{1.162949in}{3.134950in}}{\pgfqpoint{1.167339in}{3.145549in}}{\pgfqpoint{1.167339in}{3.156599in}}%
\pgfpathcurveto{\pgfqpoint{1.167339in}{3.167650in}}{\pgfqpoint{1.162949in}{3.178249in}}{\pgfqpoint{1.155135in}{3.186062in}}%
\pgfpathcurveto{\pgfqpoint{1.147322in}{3.193876in}}{\pgfqpoint{1.136723in}{3.198266in}}{\pgfqpoint{1.125673in}{3.198266in}}%
\pgfpathcurveto{\pgfqpoint{1.114623in}{3.198266in}}{\pgfqpoint{1.104024in}{3.193876in}}{\pgfqpoint{1.096210in}{3.186062in}}%
\pgfpathcurveto{\pgfqpoint{1.088396in}{3.178249in}}{\pgfqpoint{1.084006in}{3.167650in}}{\pgfqpoint{1.084006in}{3.156599in}}%
\pgfpathcurveto{\pgfqpoint{1.084006in}{3.145549in}}{\pgfqpoint{1.088396in}{3.134950in}}{\pgfqpoint{1.096210in}{3.127137in}}%
\pgfpathcurveto{\pgfqpoint{1.104024in}{3.119323in}}{\pgfqpoint{1.114623in}{3.114933in}}{\pgfqpoint{1.125673in}{3.114933in}}%
\pgfpathlineto{\pgfqpoint{1.125673in}{3.114933in}}%
\pgfpathclose%
\pgfusepath{stroke,fill}%
\end{pgfscope}%
\begin{pgfscope}%
\pgfpathrectangle{\pgfqpoint{0.633874in}{2.920818in}}{\pgfqpoint{2.177280in}{2.201755in}}%
\pgfusepath{clip}%
\pgfsetbuttcap%
\pgfsetroundjoin%
\definecolor{currentfill}{rgb}{0.121569,0.466667,0.705882}%
\pgfsetfillcolor{currentfill}%
\pgfsetlinewidth{0.481800pt}%
\definecolor{currentstroke}{rgb}{1.000000,1.000000,1.000000}%
\pgfsetstrokecolor{currentstroke}%
\pgfsetdash{}{0pt}%
\pgfpathmoveto{\pgfqpoint{1.245489in}{3.182783in}}%
\pgfpathcurveto{\pgfqpoint{1.256539in}{3.182783in}}{\pgfqpoint{1.267138in}{3.187174in}}{\pgfqpoint{1.274952in}{3.194987in}}%
\pgfpathcurveto{\pgfqpoint{1.282765in}{3.202801in}}{\pgfqpoint{1.287155in}{3.213400in}}{\pgfqpoint{1.287155in}{3.224450in}}%
\pgfpathcurveto{\pgfqpoint{1.287155in}{3.235500in}}{\pgfqpoint{1.282765in}{3.246099in}}{\pgfqpoint{1.274952in}{3.253913in}}%
\pgfpathcurveto{\pgfqpoint{1.267138in}{3.261727in}}{\pgfqpoint{1.256539in}{3.266117in}}{\pgfqpoint{1.245489in}{3.266117in}}%
\pgfpathcurveto{\pgfqpoint{1.234439in}{3.266117in}}{\pgfqpoint{1.223840in}{3.261727in}}{\pgfqpoint{1.216026in}{3.253913in}}%
\pgfpathcurveto{\pgfqpoint{1.208212in}{3.246099in}}{\pgfqpoint{1.203822in}{3.235500in}}{\pgfqpoint{1.203822in}{3.224450in}}%
\pgfpathcurveto{\pgfqpoint{1.203822in}{3.213400in}}{\pgfqpoint{1.208212in}{3.202801in}}{\pgfqpoint{1.216026in}{3.194987in}}%
\pgfpathcurveto{\pgfqpoint{1.223840in}{3.187174in}}{\pgfqpoint{1.234439in}{3.182783in}}{\pgfqpoint{1.245489in}{3.182783in}}%
\pgfpathlineto{\pgfqpoint{1.245489in}{3.182783in}}%
\pgfpathclose%
\pgfusepath{stroke,fill}%
\end{pgfscope}%
\begin{pgfscope}%
\pgfpathrectangle{\pgfqpoint{0.633874in}{2.920818in}}{\pgfqpoint{2.177280in}{2.201755in}}%
\pgfusepath{clip}%
\pgfsetbuttcap%
\pgfsetroundjoin%
\definecolor{currentfill}{rgb}{0.121569,0.466667,0.705882}%
\pgfsetfillcolor{currentfill}%
\pgfsetlinewidth{0.481800pt}%
\definecolor{currentstroke}{rgb}{1.000000,1.000000,1.000000}%
\pgfsetstrokecolor{currentstroke}%
\pgfsetdash{}{0pt}%
\pgfpathmoveto{\pgfqpoint{1.045795in}{3.114933in}}%
\pgfpathcurveto{\pgfqpoint{1.056845in}{3.114933in}}{\pgfqpoint{1.067444in}{3.119323in}}{\pgfqpoint{1.075258in}{3.127137in}}%
\pgfpathcurveto{\pgfqpoint{1.083072in}{3.134950in}}{\pgfqpoint{1.087462in}{3.145549in}}{\pgfqpoint{1.087462in}{3.156599in}}%
\pgfpathcurveto{\pgfqpoint{1.087462in}{3.167650in}}{\pgfqpoint{1.083072in}{3.178249in}}{\pgfqpoint{1.075258in}{3.186062in}}%
\pgfpathcurveto{\pgfqpoint{1.067444in}{3.193876in}}{\pgfqpoint{1.056845in}{3.198266in}}{\pgfqpoint{1.045795in}{3.198266in}}%
\pgfpathcurveto{\pgfqpoint{1.034745in}{3.198266in}}{\pgfqpoint{1.024146in}{3.193876in}}{\pgfqpoint{1.016332in}{3.186062in}}%
\pgfpathcurveto{\pgfqpoint{1.008519in}{3.178249in}}{\pgfqpoint{1.004129in}{3.167650in}}{\pgfqpoint{1.004129in}{3.156599in}}%
\pgfpathcurveto{\pgfqpoint{1.004129in}{3.145549in}}{\pgfqpoint{1.008519in}{3.134950in}}{\pgfqpoint{1.016332in}{3.127137in}}%
\pgfpathcurveto{\pgfqpoint{1.024146in}{3.119323in}}{\pgfqpoint{1.034745in}{3.114933in}}{\pgfqpoint{1.045795in}{3.114933in}}%
\pgfpathlineto{\pgfqpoint{1.045795in}{3.114933in}}%
\pgfpathclose%
\pgfusepath{stroke,fill}%
\end{pgfscope}%
\begin{pgfscope}%
\pgfpathrectangle{\pgfqpoint{0.633874in}{2.920818in}}{\pgfqpoint{2.177280in}{2.201755in}}%
\pgfusepath{clip}%
\pgfsetbuttcap%
\pgfsetroundjoin%
\definecolor{currentfill}{rgb}{0.121569,0.466667,0.705882}%
\pgfsetfillcolor{currentfill}%
\pgfsetlinewidth{0.481800pt}%
\definecolor{currentstroke}{rgb}{1.000000,1.000000,1.000000}%
\pgfsetstrokecolor{currentstroke}%
\pgfsetdash{}{0pt}%
\pgfpathmoveto{\pgfqpoint{1.325366in}{3.148858in}}%
\pgfpathcurveto{\pgfqpoint{1.336416in}{3.148858in}}{\pgfqpoint{1.347015in}{3.153248in}}{\pgfqpoint{1.354829in}{3.161062in}}%
\pgfpathcurveto{\pgfqpoint{1.362643in}{3.168876in}}{\pgfqpoint{1.367033in}{3.179475in}}{\pgfqpoint{1.367033in}{3.190525in}}%
\pgfpathcurveto{\pgfqpoint{1.367033in}{3.201575in}}{\pgfqpoint{1.362643in}{3.212174in}}{\pgfqpoint{1.354829in}{3.219988in}}%
\pgfpathcurveto{\pgfqpoint{1.347015in}{3.227801in}}{\pgfqpoint{1.336416in}{3.232191in}}{\pgfqpoint{1.325366in}{3.232191in}}%
\pgfpathcurveto{\pgfqpoint{1.314316in}{3.232191in}}{\pgfqpoint{1.303717in}{3.227801in}}{\pgfqpoint{1.295903in}{3.219988in}}%
\pgfpathcurveto{\pgfqpoint{1.288090in}{3.212174in}}{\pgfqpoint{1.283700in}{3.201575in}}{\pgfqpoint{1.283700in}{3.190525in}}%
\pgfpathcurveto{\pgfqpoint{1.283700in}{3.179475in}}{\pgfqpoint{1.288090in}{3.168876in}}{\pgfqpoint{1.295903in}{3.161062in}}%
\pgfpathcurveto{\pgfqpoint{1.303717in}{3.153248in}}{\pgfqpoint{1.314316in}{3.148858in}}{\pgfqpoint{1.325366in}{3.148858in}}%
\pgfpathlineto{\pgfqpoint{1.325366in}{3.148858in}}%
\pgfpathclose%
\pgfusepath{stroke,fill}%
\end{pgfscope}%
\begin{pgfscope}%
\pgfpathrectangle{\pgfqpoint{0.633874in}{2.920818in}}{\pgfqpoint{2.177280in}{2.201755in}}%
\pgfusepath{clip}%
\pgfsetbuttcap%
\pgfsetroundjoin%
\definecolor{currentfill}{rgb}{0.121569,0.466667,0.705882}%
\pgfsetfillcolor{currentfill}%
\pgfsetlinewidth{0.481800pt}%
\definecolor{currentstroke}{rgb}{1.000000,1.000000,1.000000}%
\pgfsetstrokecolor{currentstroke}%
\pgfsetdash{}{0pt}%
\pgfpathmoveto{\pgfqpoint{1.205550in}{3.114933in}}%
\pgfpathcurveto{\pgfqpoint{1.216600in}{3.114933in}}{\pgfqpoint{1.227199in}{3.119323in}}{\pgfqpoint{1.235013in}{3.127137in}}%
\pgfpathcurveto{\pgfqpoint{1.242826in}{3.134950in}}{\pgfqpoint{1.247217in}{3.145549in}}{\pgfqpoint{1.247217in}{3.156599in}}%
\pgfpathcurveto{\pgfqpoint{1.247217in}{3.167650in}}{\pgfqpoint{1.242826in}{3.178249in}}{\pgfqpoint{1.235013in}{3.186062in}}%
\pgfpathcurveto{\pgfqpoint{1.227199in}{3.193876in}}{\pgfqpoint{1.216600in}{3.198266in}}{\pgfqpoint{1.205550in}{3.198266in}}%
\pgfpathcurveto{\pgfqpoint{1.194500in}{3.198266in}}{\pgfqpoint{1.183901in}{3.193876in}}{\pgfqpoint{1.176087in}{3.186062in}}%
\pgfpathcurveto{\pgfqpoint{1.168274in}{3.178249in}}{\pgfqpoint{1.163883in}{3.167650in}}{\pgfqpoint{1.163883in}{3.156599in}}%
\pgfpathcurveto{\pgfqpoint{1.163883in}{3.145549in}}{\pgfqpoint{1.168274in}{3.134950in}}{\pgfqpoint{1.176087in}{3.127137in}}%
\pgfpathcurveto{\pgfqpoint{1.183901in}{3.119323in}}{\pgfqpoint{1.194500in}{3.114933in}}{\pgfqpoint{1.205550in}{3.114933in}}%
\pgfpathlineto{\pgfqpoint{1.205550in}{3.114933in}}%
\pgfpathclose%
\pgfusepath{stroke,fill}%
\end{pgfscope}%
\begin{pgfscope}%
\pgfpathrectangle{\pgfqpoint{0.633874in}{2.920818in}}{\pgfqpoint{2.177280in}{2.201755in}}%
\pgfusepath{clip}%
\pgfsetbuttcap%
\pgfsetroundjoin%
\definecolor{currentfill}{rgb}{1.000000,0.498039,0.054902}%
\pgfsetfillcolor{currentfill}%
\pgfsetlinewidth{0.481800pt}%
\definecolor{currentstroke}{rgb}{1.000000,1.000000,1.000000}%
\pgfsetstrokecolor{currentstroke}%
\pgfsetdash{}{0pt}%
\pgfpathmoveto{\pgfqpoint{2.004324in}{4.234469in}}%
\pgfpathcurveto{\pgfqpoint{2.015374in}{4.234469in}}{\pgfqpoint{2.025973in}{4.238860in}}{\pgfqpoint{2.033787in}{4.246673in}}%
\pgfpathcurveto{\pgfqpoint{2.041601in}{4.254487in}}{\pgfqpoint{2.045991in}{4.265086in}}{\pgfqpoint{2.045991in}{4.276136in}}%
\pgfpathcurveto{\pgfqpoint{2.045991in}{4.287186in}}{\pgfqpoint{2.041601in}{4.297785in}}{\pgfqpoint{2.033787in}{4.305599in}}%
\pgfpathcurveto{\pgfqpoint{2.025973in}{4.313412in}}{\pgfqpoint{2.015374in}{4.317803in}}{\pgfqpoint{2.004324in}{4.317803in}}%
\pgfpathcurveto{\pgfqpoint{1.993274in}{4.317803in}}{\pgfqpoint{1.982675in}{4.313412in}}{\pgfqpoint{1.974861in}{4.305599in}}%
\pgfpathcurveto{\pgfqpoint{1.967048in}{4.297785in}}{\pgfqpoint{1.962658in}{4.287186in}}{\pgfqpoint{1.962658in}{4.276136in}}%
\pgfpathcurveto{\pgfqpoint{1.962658in}{4.265086in}}{\pgfqpoint{1.967048in}{4.254487in}}{\pgfqpoint{1.974861in}{4.246673in}}%
\pgfpathcurveto{\pgfqpoint{1.982675in}{4.238860in}}{\pgfqpoint{1.993274in}{4.234469in}}{\pgfqpoint{2.004324in}{4.234469in}}%
\pgfpathlineto{\pgfqpoint{2.004324in}{4.234469in}}%
\pgfpathclose%
\pgfusepath{stroke,fill}%
\end{pgfscope}%
\begin{pgfscope}%
\pgfpathrectangle{\pgfqpoint{0.633874in}{2.920818in}}{\pgfqpoint{2.177280in}{2.201755in}}%
\pgfusepath{clip}%
\pgfsetbuttcap%
\pgfsetroundjoin%
\definecolor{currentfill}{rgb}{1.000000,0.498039,0.054902}%
\pgfsetfillcolor{currentfill}%
\pgfsetlinewidth{0.481800pt}%
\definecolor{currentstroke}{rgb}{1.000000,1.000000,1.000000}%
\pgfsetstrokecolor{currentstroke}%
\pgfsetdash{}{0pt}%
\pgfpathmoveto{\pgfqpoint{1.764692in}{4.166619in}}%
\pgfpathcurveto{\pgfqpoint{1.775742in}{4.166619in}}{\pgfqpoint{1.786341in}{4.171009in}}{\pgfqpoint{1.794155in}{4.178822in}}%
\pgfpathcurveto{\pgfqpoint{1.801968in}{4.186636in}}{\pgfqpoint{1.806359in}{4.197235in}}{\pgfqpoint{1.806359in}{4.208285in}}%
\pgfpathcurveto{\pgfqpoint{1.806359in}{4.219335in}}{\pgfqpoint{1.801968in}{4.229934in}}{\pgfqpoint{1.794155in}{4.237748in}}%
\pgfpathcurveto{\pgfqpoint{1.786341in}{4.245562in}}{\pgfqpoint{1.775742in}{4.249952in}}{\pgfqpoint{1.764692in}{4.249952in}}%
\pgfpathcurveto{\pgfqpoint{1.753642in}{4.249952in}}{\pgfqpoint{1.743043in}{4.245562in}}{\pgfqpoint{1.735229in}{4.237748in}}%
\pgfpathcurveto{\pgfqpoint{1.727416in}{4.229934in}}{\pgfqpoint{1.723025in}{4.219335in}}{\pgfqpoint{1.723025in}{4.208285in}}%
\pgfpathcurveto{\pgfqpoint{1.723025in}{4.197235in}}{\pgfqpoint{1.727416in}{4.186636in}}{\pgfqpoint{1.735229in}{4.178822in}}%
\pgfpathcurveto{\pgfqpoint{1.743043in}{4.171009in}}{\pgfqpoint{1.753642in}{4.166619in}}{\pgfqpoint{1.764692in}{4.166619in}}%
\pgfpathlineto{\pgfqpoint{1.764692in}{4.166619in}}%
\pgfpathclose%
\pgfusepath{stroke,fill}%
\end{pgfscope}%
\begin{pgfscope}%
\pgfpathrectangle{\pgfqpoint{0.633874in}{2.920818in}}{\pgfqpoint{2.177280in}{2.201755in}}%
\pgfusepath{clip}%
\pgfsetbuttcap%
\pgfsetroundjoin%
\definecolor{currentfill}{rgb}{1.000000,0.498039,0.054902}%
\pgfsetfillcolor{currentfill}%
\pgfsetlinewidth{0.481800pt}%
\definecolor{currentstroke}{rgb}{1.000000,1.000000,1.000000}%
\pgfsetstrokecolor{currentstroke}%
\pgfsetdash{}{0pt}%
\pgfpathmoveto{\pgfqpoint{1.964385in}{4.302320in}}%
\pgfpathcurveto{\pgfqpoint{1.975436in}{4.302320in}}{\pgfqpoint{1.986035in}{4.306710in}}{\pgfqpoint{1.993848in}{4.314524in}}%
\pgfpathcurveto{\pgfqpoint{2.001662in}{4.322337in}}{\pgfqpoint{2.006052in}{4.332937in}}{\pgfqpoint{2.006052in}{4.343987in}}%
\pgfpathcurveto{\pgfqpoint{2.006052in}{4.355037in}}{\pgfqpoint{2.001662in}{4.365636in}}{\pgfqpoint{1.993848in}{4.373449in}}%
\pgfpathcurveto{\pgfqpoint{1.986035in}{4.381263in}}{\pgfqpoint{1.975436in}{4.385653in}}{\pgfqpoint{1.964385in}{4.385653in}}%
\pgfpathcurveto{\pgfqpoint{1.953335in}{4.385653in}}{\pgfqpoint{1.942736in}{4.381263in}}{\pgfqpoint{1.934923in}{4.373449in}}%
\pgfpathcurveto{\pgfqpoint{1.927109in}{4.365636in}}{\pgfqpoint{1.922719in}{4.355037in}}{\pgfqpoint{1.922719in}{4.343987in}}%
\pgfpathcurveto{\pgfqpoint{1.922719in}{4.332937in}}{\pgfqpoint{1.927109in}{4.322337in}}{\pgfqpoint{1.934923in}{4.314524in}}%
\pgfpathcurveto{\pgfqpoint{1.942736in}{4.306710in}}{\pgfqpoint{1.953335in}{4.302320in}}{\pgfqpoint{1.964385in}{4.302320in}}%
\pgfpathlineto{\pgfqpoint{1.964385in}{4.302320in}}%
\pgfpathclose%
\pgfusepath{stroke,fill}%
\end{pgfscope}%
\begin{pgfscope}%
\pgfpathrectangle{\pgfqpoint{0.633874in}{2.920818in}}{\pgfqpoint{2.177280in}{2.201755in}}%
\pgfusepath{clip}%
\pgfsetbuttcap%
\pgfsetroundjoin%
\definecolor{currentfill}{rgb}{1.000000,0.498039,0.054902}%
\pgfsetfillcolor{currentfill}%
\pgfsetlinewidth{0.481800pt}%
\definecolor{currentstroke}{rgb}{1.000000,1.000000,1.000000}%
\pgfsetstrokecolor{currentstroke}%
\pgfsetdash{}{0pt}%
\pgfpathmoveto{\pgfqpoint{1.405244in}{3.996992in}}%
\pgfpathcurveto{\pgfqpoint{1.416294in}{3.996992in}}{\pgfqpoint{1.426893in}{4.001382in}}{\pgfqpoint{1.434706in}{4.009196in}}%
\pgfpathcurveto{\pgfqpoint{1.442520in}{4.017009in}}{\pgfqpoint{1.446910in}{4.027608in}}{\pgfqpoint{1.446910in}{4.038659in}}%
\pgfpathcurveto{\pgfqpoint{1.446910in}{4.049709in}}{\pgfqpoint{1.442520in}{4.060308in}}{\pgfqpoint{1.434706in}{4.068121in}}%
\pgfpathcurveto{\pgfqpoint{1.426893in}{4.075935in}}{\pgfqpoint{1.416294in}{4.080325in}}{\pgfqpoint{1.405244in}{4.080325in}}%
\pgfpathcurveto{\pgfqpoint{1.394193in}{4.080325in}}{\pgfqpoint{1.383594in}{4.075935in}}{\pgfqpoint{1.375781in}{4.068121in}}%
\pgfpathcurveto{\pgfqpoint{1.367967in}{4.060308in}}{\pgfqpoint{1.363577in}{4.049709in}}{\pgfqpoint{1.363577in}{4.038659in}}%
\pgfpathcurveto{\pgfqpoint{1.363577in}{4.027608in}}{\pgfqpoint{1.367967in}{4.017009in}}{\pgfqpoint{1.375781in}{4.009196in}}%
\pgfpathcurveto{\pgfqpoint{1.383594in}{4.001382in}}{\pgfqpoint{1.394193in}{3.996992in}}{\pgfqpoint{1.405244in}{3.996992in}}%
\pgfpathlineto{\pgfqpoint{1.405244in}{3.996992in}}%
\pgfpathclose%
\pgfusepath{stroke,fill}%
\end{pgfscope}%
\begin{pgfscope}%
\pgfpathrectangle{\pgfqpoint{0.633874in}{2.920818in}}{\pgfqpoint{2.177280in}{2.201755in}}%
\pgfusepath{clip}%
\pgfsetbuttcap%
\pgfsetroundjoin%
\definecolor{currentfill}{rgb}{1.000000,0.498039,0.054902}%
\pgfsetfillcolor{currentfill}%
\pgfsetlinewidth{0.481800pt}%
\definecolor{currentstroke}{rgb}{1.000000,1.000000,1.000000}%
\pgfsetstrokecolor{currentstroke}%
\pgfsetdash{}{0pt}%
\pgfpathmoveto{\pgfqpoint{1.804631in}{4.200544in}}%
\pgfpathcurveto{\pgfqpoint{1.815681in}{4.200544in}}{\pgfqpoint{1.826280in}{4.204934in}}{\pgfqpoint{1.834093in}{4.212748in}}%
\pgfpathcurveto{\pgfqpoint{1.841907in}{4.220561in}}{\pgfqpoint{1.846297in}{4.231160in}}{\pgfqpoint{1.846297in}{4.242211in}}%
\pgfpathcurveto{\pgfqpoint{1.846297in}{4.253261in}}{\pgfqpoint{1.841907in}{4.263860in}}{\pgfqpoint{1.834093in}{4.271673in}}%
\pgfpathcurveto{\pgfqpoint{1.826280in}{4.279487in}}{\pgfqpoint{1.815681in}{4.283877in}}{\pgfqpoint{1.804631in}{4.283877in}}%
\pgfpathcurveto{\pgfqpoint{1.793581in}{4.283877in}}{\pgfqpoint{1.782981in}{4.279487in}}{\pgfqpoint{1.775168in}{4.271673in}}%
\pgfpathcurveto{\pgfqpoint{1.767354in}{4.263860in}}{\pgfqpoint{1.762964in}{4.253261in}}{\pgfqpoint{1.762964in}{4.242211in}}%
\pgfpathcurveto{\pgfqpoint{1.762964in}{4.231160in}}{\pgfqpoint{1.767354in}{4.220561in}}{\pgfqpoint{1.775168in}{4.212748in}}%
\pgfpathcurveto{\pgfqpoint{1.782981in}{4.204934in}}{\pgfqpoint{1.793581in}{4.200544in}}{\pgfqpoint{1.804631in}{4.200544in}}%
\pgfpathlineto{\pgfqpoint{1.804631in}{4.200544in}}%
\pgfpathclose%
\pgfusepath{stroke,fill}%
\end{pgfscope}%
\begin{pgfscope}%
\pgfpathrectangle{\pgfqpoint{0.633874in}{2.920818in}}{\pgfqpoint{2.177280in}{2.201755in}}%
\pgfusepath{clip}%
\pgfsetbuttcap%
\pgfsetroundjoin%
\definecolor{currentfill}{rgb}{1.000000,0.498039,0.054902}%
\pgfsetfillcolor{currentfill}%
\pgfsetlinewidth{0.481800pt}%
\definecolor{currentstroke}{rgb}{1.000000,1.000000,1.000000}%
\pgfsetstrokecolor{currentstroke}%
\pgfsetdash{}{0pt}%
\pgfpathmoveto{\pgfqpoint{1.485121in}{4.166619in}}%
\pgfpathcurveto{\pgfqpoint{1.496171in}{4.166619in}}{\pgfqpoint{1.506770in}{4.171009in}}{\pgfqpoint{1.514584in}{4.178822in}}%
\pgfpathcurveto{\pgfqpoint{1.522397in}{4.186636in}}{\pgfqpoint{1.526788in}{4.197235in}}{\pgfqpoint{1.526788in}{4.208285in}}%
\pgfpathcurveto{\pgfqpoint{1.526788in}{4.219335in}}{\pgfqpoint{1.522397in}{4.229934in}}{\pgfqpoint{1.514584in}{4.237748in}}%
\pgfpathcurveto{\pgfqpoint{1.506770in}{4.245562in}}{\pgfqpoint{1.496171in}{4.249952in}}{\pgfqpoint{1.485121in}{4.249952in}}%
\pgfpathcurveto{\pgfqpoint{1.474071in}{4.249952in}}{\pgfqpoint{1.463472in}{4.245562in}}{\pgfqpoint{1.455658in}{4.237748in}}%
\pgfpathcurveto{\pgfqpoint{1.447845in}{4.229934in}}{\pgfqpoint{1.443454in}{4.219335in}}{\pgfqpoint{1.443454in}{4.208285in}}%
\pgfpathcurveto{\pgfqpoint{1.443454in}{4.197235in}}{\pgfqpoint{1.447845in}{4.186636in}}{\pgfqpoint{1.455658in}{4.178822in}}%
\pgfpathcurveto{\pgfqpoint{1.463472in}{4.171009in}}{\pgfqpoint{1.474071in}{4.166619in}}{\pgfqpoint{1.485121in}{4.166619in}}%
\pgfpathlineto{\pgfqpoint{1.485121in}{4.166619in}}%
\pgfpathclose%
\pgfusepath{stroke,fill}%
\end{pgfscope}%
\begin{pgfscope}%
\pgfpathrectangle{\pgfqpoint{0.633874in}{2.920818in}}{\pgfqpoint{2.177280in}{2.201755in}}%
\pgfusepath{clip}%
\pgfsetbuttcap%
\pgfsetroundjoin%
\definecolor{currentfill}{rgb}{1.000000,0.498039,0.054902}%
\pgfsetfillcolor{currentfill}%
\pgfsetlinewidth{0.481800pt}%
\definecolor{currentstroke}{rgb}{1.000000,1.000000,1.000000}%
\pgfsetstrokecolor{currentstroke}%
\pgfsetdash{}{0pt}%
\pgfpathmoveto{\pgfqpoint{1.724753in}{4.234469in}}%
\pgfpathcurveto{\pgfqpoint{1.735803in}{4.234469in}}{\pgfqpoint{1.746402in}{4.238860in}}{\pgfqpoint{1.754216in}{4.246673in}}%
\pgfpathcurveto{\pgfqpoint{1.762030in}{4.254487in}}{\pgfqpoint{1.766420in}{4.265086in}}{\pgfqpoint{1.766420in}{4.276136in}}%
\pgfpathcurveto{\pgfqpoint{1.766420in}{4.287186in}}{\pgfqpoint{1.762030in}{4.297785in}}{\pgfqpoint{1.754216in}{4.305599in}}%
\pgfpathcurveto{\pgfqpoint{1.746402in}{4.313412in}}{\pgfqpoint{1.735803in}{4.317803in}}{\pgfqpoint{1.724753in}{4.317803in}}%
\pgfpathcurveto{\pgfqpoint{1.713703in}{4.317803in}}{\pgfqpoint{1.703104in}{4.313412in}}{\pgfqpoint{1.695290in}{4.305599in}}%
\pgfpathcurveto{\pgfqpoint{1.687477in}{4.297785in}}{\pgfqpoint{1.683087in}{4.287186in}}{\pgfqpoint{1.683087in}{4.276136in}}%
\pgfpathcurveto{\pgfqpoint{1.683087in}{4.265086in}}{\pgfqpoint{1.687477in}{4.254487in}}{\pgfqpoint{1.695290in}{4.246673in}}%
\pgfpathcurveto{\pgfqpoint{1.703104in}{4.238860in}}{\pgfqpoint{1.713703in}{4.234469in}}{\pgfqpoint{1.724753in}{4.234469in}}%
\pgfpathlineto{\pgfqpoint{1.724753in}{4.234469in}}%
\pgfpathclose%
\pgfusepath{stroke,fill}%
\end{pgfscope}%
\begin{pgfscope}%
\pgfpathrectangle{\pgfqpoint{0.633874in}{2.920818in}}{\pgfqpoint{2.177280in}{2.201755in}}%
\pgfusepath{clip}%
\pgfsetbuttcap%
\pgfsetroundjoin%
\definecolor{currentfill}{rgb}{1.000000,0.498039,0.054902}%
\pgfsetfillcolor{currentfill}%
\pgfsetlinewidth{0.481800pt}%
\definecolor{currentstroke}{rgb}{1.000000,1.000000,1.000000}%
\pgfsetstrokecolor{currentstroke}%
\pgfsetdash{}{0pt}%
\pgfpathmoveto{\pgfqpoint{1.165611in}{3.759514in}}%
\pgfpathcurveto{\pgfqpoint{1.176662in}{3.759514in}}{\pgfqpoint{1.187261in}{3.763905in}}{\pgfqpoint{1.195074in}{3.771718in}}%
\pgfpathcurveto{\pgfqpoint{1.202888in}{3.779532in}}{\pgfqpoint{1.207278in}{3.790131in}}{\pgfqpoint{1.207278in}{3.801181in}}%
\pgfpathcurveto{\pgfqpoint{1.207278in}{3.812231in}}{\pgfqpoint{1.202888in}{3.822830in}}{\pgfqpoint{1.195074in}{3.830644in}}%
\pgfpathcurveto{\pgfqpoint{1.187261in}{3.838457in}}{\pgfqpoint{1.176662in}{3.842848in}}{\pgfqpoint{1.165611in}{3.842848in}}%
\pgfpathcurveto{\pgfqpoint{1.154561in}{3.842848in}}{\pgfqpoint{1.143962in}{3.838457in}}{\pgfqpoint{1.136149in}{3.830644in}}%
\pgfpathcurveto{\pgfqpoint{1.128335in}{3.822830in}}{\pgfqpoint{1.123945in}{3.812231in}}{\pgfqpoint{1.123945in}{3.801181in}}%
\pgfpathcurveto{\pgfqpoint{1.123945in}{3.790131in}}{\pgfqpoint{1.128335in}{3.779532in}}{\pgfqpoint{1.136149in}{3.771718in}}%
\pgfpathcurveto{\pgfqpoint{1.143962in}{3.763905in}}{\pgfqpoint{1.154561in}{3.759514in}}{\pgfqpoint{1.165611in}{3.759514in}}%
\pgfpathlineto{\pgfqpoint{1.165611in}{3.759514in}}%
\pgfpathclose%
\pgfusepath{stroke,fill}%
\end{pgfscope}%
\begin{pgfscope}%
\pgfpathrectangle{\pgfqpoint{0.633874in}{2.920818in}}{\pgfqpoint{2.177280in}{2.201755in}}%
\pgfusepath{clip}%
\pgfsetbuttcap%
\pgfsetroundjoin%
\definecolor{currentfill}{rgb}{1.000000,0.498039,0.054902}%
\pgfsetfillcolor{currentfill}%
\pgfsetlinewidth{0.481800pt}%
\definecolor{currentstroke}{rgb}{1.000000,1.000000,1.000000}%
\pgfsetstrokecolor{currentstroke}%
\pgfsetdash{}{0pt}%
\pgfpathmoveto{\pgfqpoint{1.844569in}{4.200544in}}%
\pgfpathcurveto{\pgfqpoint{1.855619in}{4.200544in}}{\pgfqpoint{1.866219in}{4.204934in}}{\pgfqpoint{1.874032in}{4.212748in}}%
\pgfpathcurveto{\pgfqpoint{1.881846in}{4.220561in}}{\pgfqpoint{1.886236in}{4.231160in}}{\pgfqpoint{1.886236in}{4.242211in}}%
\pgfpathcurveto{\pgfqpoint{1.886236in}{4.253261in}}{\pgfqpoint{1.881846in}{4.263860in}}{\pgfqpoint{1.874032in}{4.271673in}}%
\pgfpathcurveto{\pgfqpoint{1.866219in}{4.279487in}}{\pgfqpoint{1.855619in}{4.283877in}}{\pgfqpoint{1.844569in}{4.283877in}}%
\pgfpathcurveto{\pgfqpoint{1.833519in}{4.283877in}}{\pgfqpoint{1.822920in}{4.279487in}}{\pgfqpoint{1.815107in}{4.271673in}}%
\pgfpathcurveto{\pgfqpoint{1.807293in}{4.263860in}}{\pgfqpoint{1.802903in}{4.253261in}}{\pgfqpoint{1.802903in}{4.242211in}}%
\pgfpathcurveto{\pgfqpoint{1.802903in}{4.231160in}}{\pgfqpoint{1.807293in}{4.220561in}}{\pgfqpoint{1.815107in}{4.212748in}}%
\pgfpathcurveto{\pgfqpoint{1.822920in}{4.204934in}}{\pgfqpoint{1.833519in}{4.200544in}}{\pgfqpoint{1.844569in}{4.200544in}}%
\pgfpathlineto{\pgfqpoint{1.844569in}{4.200544in}}%
\pgfpathclose%
\pgfusepath{stroke,fill}%
\end{pgfscope}%
\begin{pgfscope}%
\pgfpathrectangle{\pgfqpoint{0.633874in}{2.920818in}}{\pgfqpoint{2.177280in}{2.201755in}}%
\pgfusepath{clip}%
\pgfsetbuttcap%
\pgfsetroundjoin%
\definecolor{currentfill}{rgb}{1.000000,0.498039,0.054902}%
\pgfsetfillcolor{currentfill}%
\pgfsetlinewidth{0.481800pt}%
\definecolor{currentstroke}{rgb}{1.000000,1.000000,1.000000}%
\pgfsetstrokecolor{currentstroke}%
\pgfsetdash{}{0pt}%
\pgfpathmoveto{\pgfqpoint{1.285428in}{3.963066in}}%
\pgfpathcurveto{\pgfqpoint{1.296478in}{3.963066in}}{\pgfqpoint{1.307077in}{3.967457in}}{\pgfqpoint{1.314890in}{3.975270in}}%
\pgfpathcurveto{\pgfqpoint{1.322704in}{3.983084in}}{\pgfqpoint{1.327094in}{3.993683in}}{\pgfqpoint{1.327094in}{4.004733in}}%
\pgfpathcurveto{\pgfqpoint{1.327094in}{4.015783in}}{\pgfqpoint{1.322704in}{4.026382in}}{\pgfqpoint{1.314890in}{4.034196in}}%
\pgfpathcurveto{\pgfqpoint{1.307077in}{4.042010in}}{\pgfqpoint{1.296478in}{4.046400in}}{\pgfqpoint{1.285428in}{4.046400in}}%
\pgfpathcurveto{\pgfqpoint{1.274377in}{4.046400in}}{\pgfqpoint{1.263778in}{4.042010in}}{\pgfqpoint{1.255965in}{4.034196in}}%
\pgfpathcurveto{\pgfqpoint{1.248151in}{4.026382in}}{\pgfqpoint{1.243761in}{4.015783in}}{\pgfqpoint{1.243761in}{4.004733in}}%
\pgfpathcurveto{\pgfqpoint{1.243761in}{3.993683in}}{\pgfqpoint{1.248151in}{3.983084in}}{\pgfqpoint{1.255965in}{3.975270in}}%
\pgfpathcurveto{\pgfqpoint{1.263778in}{3.967457in}}{\pgfqpoint{1.274377in}{3.963066in}}{\pgfqpoint{1.285428in}{3.963066in}}%
\pgfpathlineto{\pgfqpoint{1.285428in}{3.963066in}}%
\pgfpathclose%
\pgfusepath{stroke,fill}%
\end{pgfscope}%
\begin{pgfscope}%
\pgfpathrectangle{\pgfqpoint{0.633874in}{2.920818in}}{\pgfqpoint{2.177280in}{2.201755in}}%
\pgfusepath{clip}%
\pgfsetbuttcap%
\pgfsetroundjoin%
\definecolor{currentfill}{rgb}{1.000000,0.498039,0.054902}%
\pgfsetfillcolor{currentfill}%
\pgfsetlinewidth{0.481800pt}%
\definecolor{currentstroke}{rgb}{1.000000,1.000000,1.000000}%
\pgfsetstrokecolor{currentstroke}%
\pgfsetdash{}{0pt}%
\pgfpathmoveto{\pgfqpoint{1.205550in}{3.827365in}}%
\pgfpathcurveto{\pgfqpoint{1.216600in}{3.827365in}}{\pgfqpoint{1.227199in}{3.831755in}}{\pgfqpoint{1.235013in}{3.839569in}}%
\pgfpathcurveto{\pgfqpoint{1.242826in}{3.847383in}}{\pgfqpoint{1.247217in}{3.857982in}}{\pgfqpoint{1.247217in}{3.869032in}}%
\pgfpathcurveto{\pgfqpoint{1.247217in}{3.880082in}}{\pgfqpoint{1.242826in}{3.890681in}}{\pgfqpoint{1.235013in}{3.898495in}}%
\pgfpathcurveto{\pgfqpoint{1.227199in}{3.906308in}}{\pgfqpoint{1.216600in}{3.910698in}}{\pgfqpoint{1.205550in}{3.910698in}}%
\pgfpathcurveto{\pgfqpoint{1.194500in}{3.910698in}}{\pgfqpoint{1.183901in}{3.906308in}}{\pgfqpoint{1.176087in}{3.898495in}}%
\pgfpathcurveto{\pgfqpoint{1.168274in}{3.890681in}}{\pgfqpoint{1.163883in}{3.880082in}}{\pgfqpoint{1.163883in}{3.869032in}}%
\pgfpathcurveto{\pgfqpoint{1.163883in}{3.857982in}}{\pgfqpoint{1.168274in}{3.847383in}}{\pgfqpoint{1.176087in}{3.839569in}}%
\pgfpathcurveto{\pgfqpoint{1.183901in}{3.831755in}}{\pgfqpoint{1.194500in}{3.827365in}}{\pgfqpoint{1.205550in}{3.827365in}}%
\pgfpathlineto{\pgfqpoint{1.205550in}{3.827365in}}%
\pgfpathclose%
\pgfusepath{stroke,fill}%
\end{pgfscope}%
\begin{pgfscope}%
\pgfpathrectangle{\pgfqpoint{0.633874in}{2.920818in}}{\pgfqpoint{2.177280in}{2.201755in}}%
\pgfusepath{clip}%
\pgfsetbuttcap%
\pgfsetroundjoin%
\definecolor{currentfill}{rgb}{1.000000,0.498039,0.054902}%
\pgfsetfillcolor{currentfill}%
\pgfsetlinewidth{0.481800pt}%
\definecolor{currentstroke}{rgb}{1.000000,1.000000,1.000000}%
\pgfsetstrokecolor{currentstroke}%
\pgfsetdash{}{0pt}%
\pgfpathmoveto{\pgfqpoint{1.564998in}{4.064843in}}%
\pgfpathcurveto{\pgfqpoint{1.576049in}{4.064843in}}{\pgfqpoint{1.586648in}{4.069233in}}{\pgfqpoint{1.594461in}{4.077046in}}%
\pgfpathcurveto{\pgfqpoint{1.602275in}{4.084860in}}{\pgfqpoint{1.606665in}{4.095459in}}{\pgfqpoint{1.606665in}{4.106509in}}%
\pgfpathcurveto{\pgfqpoint{1.606665in}{4.117559in}}{\pgfqpoint{1.602275in}{4.128158in}}{\pgfqpoint{1.594461in}{4.135972in}}%
\pgfpathcurveto{\pgfqpoint{1.586648in}{4.143786in}}{\pgfqpoint{1.576049in}{4.148176in}}{\pgfqpoint{1.564998in}{4.148176in}}%
\pgfpathcurveto{\pgfqpoint{1.553948in}{4.148176in}}{\pgfqpoint{1.543349in}{4.143786in}}{\pgfqpoint{1.535536in}{4.135972in}}%
\pgfpathcurveto{\pgfqpoint{1.527722in}{4.128158in}}{\pgfqpoint{1.523332in}{4.117559in}}{\pgfqpoint{1.523332in}{4.106509in}}%
\pgfpathcurveto{\pgfqpoint{1.523332in}{4.095459in}}{\pgfqpoint{1.527722in}{4.084860in}}{\pgfqpoint{1.535536in}{4.077046in}}%
\pgfpathcurveto{\pgfqpoint{1.543349in}{4.069233in}}{\pgfqpoint{1.553948in}{4.064843in}}{\pgfqpoint{1.564998in}{4.064843in}}%
\pgfpathlineto{\pgfqpoint{1.564998in}{4.064843in}}%
\pgfpathclose%
\pgfusepath{stroke,fill}%
\end{pgfscope}%
\begin{pgfscope}%
\pgfpathrectangle{\pgfqpoint{0.633874in}{2.920818in}}{\pgfqpoint{2.177280in}{2.201755in}}%
\pgfusepath{clip}%
\pgfsetbuttcap%
\pgfsetroundjoin%
\definecolor{currentfill}{rgb}{1.000000,0.498039,0.054902}%
\pgfsetfillcolor{currentfill}%
\pgfsetlinewidth{0.481800pt}%
\definecolor{currentstroke}{rgb}{1.000000,1.000000,1.000000}%
\pgfsetstrokecolor{currentstroke}%
\pgfsetdash{}{0pt}%
\pgfpathmoveto{\pgfqpoint{1.604937in}{3.996992in}}%
\pgfpathcurveto{\pgfqpoint{1.615987in}{3.996992in}}{\pgfqpoint{1.626586in}{4.001382in}}{\pgfqpoint{1.634400in}{4.009196in}}%
\pgfpathcurveto{\pgfqpoint{1.642214in}{4.017009in}}{\pgfqpoint{1.646604in}{4.027608in}}{\pgfqpoint{1.646604in}{4.038659in}}%
\pgfpathcurveto{\pgfqpoint{1.646604in}{4.049709in}}{\pgfqpoint{1.642214in}{4.060308in}}{\pgfqpoint{1.634400in}{4.068121in}}%
\pgfpathcurveto{\pgfqpoint{1.626586in}{4.075935in}}{\pgfqpoint{1.615987in}{4.080325in}}{\pgfqpoint{1.604937in}{4.080325in}}%
\pgfpathcurveto{\pgfqpoint{1.593887in}{4.080325in}}{\pgfqpoint{1.583288in}{4.075935in}}{\pgfqpoint{1.575474in}{4.068121in}}%
\pgfpathcurveto{\pgfqpoint{1.567661in}{4.060308in}}{\pgfqpoint{1.563270in}{4.049709in}}{\pgfqpoint{1.563270in}{4.038659in}}%
\pgfpathcurveto{\pgfqpoint{1.563270in}{4.027608in}}{\pgfqpoint{1.567661in}{4.017009in}}{\pgfqpoint{1.575474in}{4.009196in}}%
\pgfpathcurveto{\pgfqpoint{1.583288in}{4.001382in}}{\pgfqpoint{1.593887in}{3.996992in}}{\pgfqpoint{1.604937in}{3.996992in}}%
\pgfpathlineto{\pgfqpoint{1.604937in}{3.996992in}}%
\pgfpathclose%
\pgfusepath{stroke,fill}%
\end{pgfscope}%
\begin{pgfscope}%
\pgfpathrectangle{\pgfqpoint{0.633874in}{2.920818in}}{\pgfqpoint{2.177280in}{2.201755in}}%
\pgfusepath{clip}%
\pgfsetbuttcap%
\pgfsetroundjoin%
\definecolor{currentfill}{rgb}{1.000000,0.498039,0.054902}%
\pgfsetfillcolor{currentfill}%
\pgfsetlinewidth{0.481800pt}%
\definecolor{currentstroke}{rgb}{1.000000,1.000000,1.000000}%
\pgfsetstrokecolor{currentstroke}%
\pgfsetdash{}{0pt}%
\pgfpathmoveto{\pgfqpoint{1.644876in}{4.234469in}}%
\pgfpathcurveto{\pgfqpoint{1.655926in}{4.234469in}}{\pgfqpoint{1.666525in}{4.238860in}}{\pgfqpoint{1.674339in}{4.246673in}}%
\pgfpathcurveto{\pgfqpoint{1.682152in}{4.254487in}}{\pgfqpoint{1.686543in}{4.265086in}}{\pgfqpoint{1.686543in}{4.276136in}}%
\pgfpathcurveto{\pgfqpoint{1.686543in}{4.287186in}}{\pgfqpoint{1.682152in}{4.297785in}}{\pgfqpoint{1.674339in}{4.305599in}}%
\pgfpathcurveto{\pgfqpoint{1.666525in}{4.313412in}}{\pgfqpoint{1.655926in}{4.317803in}}{\pgfqpoint{1.644876in}{4.317803in}}%
\pgfpathcurveto{\pgfqpoint{1.633826in}{4.317803in}}{\pgfqpoint{1.623227in}{4.313412in}}{\pgfqpoint{1.615413in}{4.305599in}}%
\pgfpathcurveto{\pgfqpoint{1.607599in}{4.297785in}}{\pgfqpoint{1.603209in}{4.287186in}}{\pgfqpoint{1.603209in}{4.276136in}}%
\pgfpathcurveto{\pgfqpoint{1.603209in}{4.265086in}}{\pgfqpoint{1.607599in}{4.254487in}}{\pgfqpoint{1.615413in}{4.246673in}}%
\pgfpathcurveto{\pgfqpoint{1.623227in}{4.238860in}}{\pgfqpoint{1.633826in}{4.234469in}}{\pgfqpoint{1.644876in}{4.234469in}}%
\pgfpathlineto{\pgfqpoint{1.644876in}{4.234469in}}%
\pgfpathclose%
\pgfusepath{stroke,fill}%
\end{pgfscope}%
\begin{pgfscope}%
\pgfpathrectangle{\pgfqpoint{0.633874in}{2.920818in}}{\pgfqpoint{2.177280in}{2.201755in}}%
\pgfusepath{clip}%
\pgfsetbuttcap%
\pgfsetroundjoin%
\definecolor{currentfill}{rgb}{1.000000,0.498039,0.054902}%
\pgfsetfillcolor{currentfill}%
\pgfsetlinewidth{0.481800pt}%
\definecolor{currentstroke}{rgb}{1.000000,1.000000,1.000000}%
\pgfsetstrokecolor{currentstroke}%
\pgfsetdash{}{0pt}%
\pgfpathmoveto{\pgfqpoint{1.445182in}{3.861290in}}%
\pgfpathcurveto{\pgfqpoint{1.456232in}{3.861290in}}{\pgfqpoint{1.466831in}{3.865681in}}{\pgfqpoint{1.474645in}{3.873494in}}%
\pgfpathcurveto{\pgfqpoint{1.482459in}{3.881308in}}{\pgfqpoint{1.486849in}{3.891907in}}{\pgfqpoint{1.486849in}{3.902957in}}%
\pgfpathcurveto{\pgfqpoint{1.486849in}{3.914007in}}{\pgfqpoint{1.482459in}{3.924606in}}{\pgfqpoint{1.474645in}{3.932420in}}%
\pgfpathcurveto{\pgfqpoint{1.466831in}{3.940234in}}{\pgfqpoint{1.456232in}{3.944624in}}{\pgfqpoint{1.445182in}{3.944624in}}%
\pgfpathcurveto{\pgfqpoint{1.434132in}{3.944624in}}{\pgfqpoint{1.423533in}{3.940234in}}{\pgfqpoint{1.415720in}{3.932420in}}%
\pgfpathcurveto{\pgfqpoint{1.407906in}{3.924606in}}{\pgfqpoint{1.403516in}{3.914007in}}{\pgfqpoint{1.403516in}{3.902957in}}%
\pgfpathcurveto{\pgfqpoint{1.403516in}{3.891907in}}{\pgfqpoint{1.407906in}{3.881308in}}{\pgfqpoint{1.415720in}{3.873494in}}%
\pgfpathcurveto{\pgfqpoint{1.423533in}{3.865681in}}{\pgfqpoint{1.434132in}{3.861290in}}{\pgfqpoint{1.445182in}{3.861290in}}%
\pgfpathlineto{\pgfqpoint{1.445182in}{3.861290in}}%
\pgfpathclose%
\pgfusepath{stroke,fill}%
\end{pgfscope}%
\begin{pgfscope}%
\pgfpathrectangle{\pgfqpoint{0.633874in}{2.920818in}}{\pgfqpoint{2.177280in}{2.201755in}}%
\pgfusepath{clip}%
\pgfsetbuttcap%
\pgfsetroundjoin%
\definecolor{currentfill}{rgb}{1.000000,0.498039,0.054902}%
\pgfsetfillcolor{currentfill}%
\pgfsetlinewidth{0.481800pt}%
\definecolor{currentstroke}{rgb}{1.000000,1.000000,1.000000}%
\pgfsetstrokecolor{currentstroke}%
\pgfsetdash{}{0pt}%
\pgfpathmoveto{\pgfqpoint{1.884508in}{4.132693in}}%
\pgfpathcurveto{\pgfqpoint{1.895558in}{4.132693in}}{\pgfqpoint{1.906157in}{4.137083in}}{\pgfqpoint{1.913971in}{4.144897in}}%
\pgfpathcurveto{\pgfqpoint{1.921784in}{4.152711in}}{\pgfqpoint{1.926175in}{4.163310in}}{\pgfqpoint{1.926175in}{4.174360in}}%
\pgfpathcurveto{\pgfqpoint{1.926175in}{4.185410in}}{\pgfqpoint{1.921784in}{4.196009in}}{\pgfqpoint{1.913971in}{4.203823in}}%
\pgfpathcurveto{\pgfqpoint{1.906157in}{4.211636in}}{\pgfqpoint{1.895558in}{4.216027in}}{\pgfqpoint{1.884508in}{4.216027in}}%
\pgfpathcurveto{\pgfqpoint{1.873458in}{4.216027in}}{\pgfqpoint{1.862859in}{4.211636in}}{\pgfqpoint{1.855045in}{4.203823in}}%
\pgfpathcurveto{\pgfqpoint{1.847232in}{4.196009in}}{\pgfqpoint{1.842841in}{4.185410in}}{\pgfqpoint{1.842841in}{4.174360in}}%
\pgfpathcurveto{\pgfqpoint{1.842841in}{4.163310in}}{\pgfqpoint{1.847232in}{4.152711in}}{\pgfqpoint{1.855045in}{4.144897in}}%
\pgfpathcurveto{\pgfqpoint{1.862859in}{4.137083in}}{\pgfqpoint{1.873458in}{4.132693in}}{\pgfqpoint{1.884508in}{4.132693in}}%
\pgfpathlineto{\pgfqpoint{1.884508in}{4.132693in}}%
\pgfpathclose%
\pgfusepath{stroke,fill}%
\end{pgfscope}%
\begin{pgfscope}%
\pgfpathrectangle{\pgfqpoint{0.633874in}{2.920818in}}{\pgfqpoint{2.177280in}{2.201755in}}%
\pgfusepath{clip}%
\pgfsetbuttcap%
\pgfsetroundjoin%
\definecolor{currentfill}{rgb}{1.000000,0.498039,0.054902}%
\pgfsetfillcolor{currentfill}%
\pgfsetlinewidth{0.481800pt}%
\definecolor{currentstroke}{rgb}{1.000000,1.000000,1.000000}%
\pgfsetstrokecolor{currentstroke}%
\pgfsetdash{}{0pt}%
\pgfpathmoveto{\pgfqpoint{1.445182in}{4.166619in}}%
\pgfpathcurveto{\pgfqpoint{1.456232in}{4.166619in}}{\pgfqpoint{1.466831in}{4.171009in}}{\pgfqpoint{1.474645in}{4.178822in}}%
\pgfpathcurveto{\pgfqpoint{1.482459in}{4.186636in}}{\pgfqpoint{1.486849in}{4.197235in}}{\pgfqpoint{1.486849in}{4.208285in}}%
\pgfpathcurveto{\pgfqpoint{1.486849in}{4.219335in}}{\pgfqpoint{1.482459in}{4.229934in}}{\pgfqpoint{1.474645in}{4.237748in}}%
\pgfpathcurveto{\pgfqpoint{1.466831in}{4.245562in}}{\pgfqpoint{1.456232in}{4.249952in}}{\pgfqpoint{1.445182in}{4.249952in}}%
\pgfpathcurveto{\pgfqpoint{1.434132in}{4.249952in}}{\pgfqpoint{1.423533in}{4.245562in}}{\pgfqpoint{1.415720in}{4.237748in}}%
\pgfpathcurveto{\pgfqpoint{1.407906in}{4.229934in}}{\pgfqpoint{1.403516in}{4.219335in}}{\pgfqpoint{1.403516in}{4.208285in}}%
\pgfpathcurveto{\pgfqpoint{1.403516in}{4.197235in}}{\pgfqpoint{1.407906in}{4.186636in}}{\pgfqpoint{1.415720in}{4.178822in}}%
\pgfpathcurveto{\pgfqpoint{1.423533in}{4.171009in}}{\pgfqpoint{1.434132in}{4.166619in}}{\pgfqpoint{1.445182in}{4.166619in}}%
\pgfpathlineto{\pgfqpoint{1.445182in}{4.166619in}}%
\pgfpathclose%
\pgfusepath{stroke,fill}%
\end{pgfscope}%
\begin{pgfscope}%
\pgfpathrectangle{\pgfqpoint{0.633874in}{2.920818in}}{\pgfqpoint{2.177280in}{2.201755in}}%
\pgfusepath{clip}%
\pgfsetbuttcap%
\pgfsetroundjoin%
\definecolor{currentfill}{rgb}{1.000000,0.498039,0.054902}%
\pgfsetfillcolor{currentfill}%
\pgfsetlinewidth{0.481800pt}%
\definecolor{currentstroke}{rgb}{1.000000,1.000000,1.000000}%
\pgfsetstrokecolor{currentstroke}%
\pgfsetdash{}{0pt}%
\pgfpathmoveto{\pgfqpoint{1.525060in}{4.030917in}}%
\pgfpathcurveto{\pgfqpoint{1.536110in}{4.030917in}}{\pgfqpoint{1.546709in}{4.035307in}}{\pgfqpoint{1.554523in}{4.043121in}}%
\pgfpathcurveto{\pgfqpoint{1.562336in}{4.050935in}}{\pgfqpoint{1.566726in}{4.061534in}}{\pgfqpoint{1.566726in}{4.072584in}}%
\pgfpathcurveto{\pgfqpoint{1.566726in}{4.083634in}}{\pgfqpoint{1.562336in}{4.094233in}}{\pgfqpoint{1.554523in}{4.102047in}}%
\pgfpathcurveto{\pgfqpoint{1.546709in}{4.109860in}}{\pgfqpoint{1.536110in}{4.114251in}}{\pgfqpoint{1.525060in}{4.114251in}}%
\pgfpathcurveto{\pgfqpoint{1.514010in}{4.114251in}}{\pgfqpoint{1.503411in}{4.109860in}}{\pgfqpoint{1.495597in}{4.102047in}}%
\pgfpathcurveto{\pgfqpoint{1.487783in}{4.094233in}}{\pgfqpoint{1.483393in}{4.083634in}}{\pgfqpoint{1.483393in}{4.072584in}}%
\pgfpathcurveto{\pgfqpoint{1.483393in}{4.061534in}}{\pgfqpoint{1.487783in}{4.050935in}}{\pgfqpoint{1.495597in}{4.043121in}}%
\pgfpathcurveto{\pgfqpoint{1.503411in}{4.035307in}}{\pgfqpoint{1.514010in}{4.030917in}}{\pgfqpoint{1.525060in}{4.030917in}}%
\pgfpathlineto{\pgfqpoint{1.525060in}{4.030917in}}%
\pgfpathclose%
\pgfusepath{stroke,fill}%
\end{pgfscope}%
\begin{pgfscope}%
\pgfpathrectangle{\pgfqpoint{0.633874in}{2.920818in}}{\pgfqpoint{2.177280in}{2.201755in}}%
\pgfusepath{clip}%
\pgfsetbuttcap%
\pgfsetroundjoin%
\definecolor{currentfill}{rgb}{1.000000,0.498039,0.054902}%
\pgfsetfillcolor{currentfill}%
\pgfsetlinewidth{0.481800pt}%
\definecolor{currentstroke}{rgb}{1.000000,1.000000,1.000000}%
\pgfsetstrokecolor{currentstroke}%
\pgfsetdash{}{0pt}%
\pgfpathmoveto{\pgfqpoint{1.684815in}{4.166619in}}%
\pgfpathcurveto{\pgfqpoint{1.695865in}{4.166619in}}{\pgfqpoint{1.706464in}{4.171009in}}{\pgfqpoint{1.714277in}{4.178822in}}%
\pgfpathcurveto{\pgfqpoint{1.722091in}{4.186636in}}{\pgfqpoint{1.726481in}{4.197235in}}{\pgfqpoint{1.726481in}{4.208285in}}%
\pgfpathcurveto{\pgfqpoint{1.726481in}{4.219335in}}{\pgfqpoint{1.722091in}{4.229934in}}{\pgfqpoint{1.714277in}{4.237748in}}%
\pgfpathcurveto{\pgfqpoint{1.706464in}{4.245562in}}{\pgfqpoint{1.695865in}{4.249952in}}{\pgfqpoint{1.684815in}{4.249952in}}%
\pgfpathcurveto{\pgfqpoint{1.673764in}{4.249952in}}{\pgfqpoint{1.663165in}{4.245562in}}{\pgfqpoint{1.655352in}{4.237748in}}%
\pgfpathcurveto{\pgfqpoint{1.647538in}{4.229934in}}{\pgfqpoint{1.643148in}{4.219335in}}{\pgfqpoint{1.643148in}{4.208285in}}%
\pgfpathcurveto{\pgfqpoint{1.643148in}{4.197235in}}{\pgfqpoint{1.647538in}{4.186636in}}{\pgfqpoint{1.655352in}{4.178822in}}%
\pgfpathcurveto{\pgfqpoint{1.663165in}{4.171009in}}{\pgfqpoint{1.673764in}{4.166619in}}{\pgfqpoint{1.684815in}{4.166619in}}%
\pgfpathlineto{\pgfqpoint{1.684815in}{4.166619in}}%
\pgfpathclose%
\pgfusepath{stroke,fill}%
\end{pgfscope}%
\begin{pgfscope}%
\pgfpathrectangle{\pgfqpoint{0.633874in}{2.920818in}}{\pgfqpoint{2.177280in}{2.201755in}}%
\pgfusepath{clip}%
\pgfsetbuttcap%
\pgfsetroundjoin%
\definecolor{currentfill}{rgb}{1.000000,0.498039,0.054902}%
\pgfsetfillcolor{currentfill}%
\pgfsetlinewidth{0.481800pt}%
\definecolor{currentstroke}{rgb}{1.000000,1.000000,1.000000}%
\pgfsetstrokecolor{currentstroke}%
\pgfsetdash{}{0pt}%
\pgfpathmoveto{\pgfqpoint{1.445182in}{3.963066in}}%
\pgfpathcurveto{\pgfqpoint{1.456232in}{3.963066in}}{\pgfqpoint{1.466831in}{3.967457in}}{\pgfqpoint{1.474645in}{3.975270in}}%
\pgfpathcurveto{\pgfqpoint{1.482459in}{3.983084in}}{\pgfqpoint{1.486849in}{3.993683in}}{\pgfqpoint{1.486849in}{4.004733in}}%
\pgfpathcurveto{\pgfqpoint{1.486849in}{4.015783in}}{\pgfqpoint{1.482459in}{4.026382in}}{\pgfqpoint{1.474645in}{4.034196in}}%
\pgfpathcurveto{\pgfqpoint{1.466831in}{4.042010in}}{\pgfqpoint{1.456232in}{4.046400in}}{\pgfqpoint{1.445182in}{4.046400in}}%
\pgfpathcurveto{\pgfqpoint{1.434132in}{4.046400in}}{\pgfqpoint{1.423533in}{4.042010in}}{\pgfqpoint{1.415720in}{4.034196in}}%
\pgfpathcurveto{\pgfqpoint{1.407906in}{4.026382in}}{\pgfqpoint{1.403516in}{4.015783in}}{\pgfqpoint{1.403516in}{4.004733in}}%
\pgfpathcurveto{\pgfqpoint{1.403516in}{3.993683in}}{\pgfqpoint{1.407906in}{3.983084in}}{\pgfqpoint{1.415720in}{3.975270in}}%
\pgfpathcurveto{\pgfqpoint{1.423533in}{3.967457in}}{\pgfqpoint{1.434132in}{3.963066in}}{\pgfqpoint{1.445182in}{3.963066in}}%
\pgfpathlineto{\pgfqpoint{1.445182in}{3.963066in}}%
\pgfpathclose%
\pgfusepath{stroke,fill}%
\end{pgfscope}%
\begin{pgfscope}%
\pgfpathrectangle{\pgfqpoint{0.633874in}{2.920818in}}{\pgfqpoint{2.177280in}{2.201755in}}%
\pgfusepath{clip}%
\pgfsetbuttcap%
\pgfsetroundjoin%
\definecolor{currentfill}{rgb}{1.000000,0.498039,0.054902}%
\pgfsetfillcolor{currentfill}%
\pgfsetlinewidth{0.481800pt}%
\definecolor{currentstroke}{rgb}{1.000000,1.000000,1.000000}%
\pgfsetstrokecolor{currentstroke}%
\pgfsetdash{}{0pt}%
\pgfpathmoveto{\pgfqpoint{1.564998in}{4.268395in}}%
\pgfpathcurveto{\pgfqpoint{1.576049in}{4.268395in}}{\pgfqpoint{1.586648in}{4.272785in}}{\pgfqpoint{1.594461in}{4.280599in}}%
\pgfpathcurveto{\pgfqpoint{1.602275in}{4.288412in}}{\pgfqpoint{1.606665in}{4.299011in}}{\pgfqpoint{1.606665in}{4.310061in}}%
\pgfpathcurveto{\pgfqpoint{1.606665in}{4.321111in}}{\pgfqpoint{1.602275in}{4.331710in}}{\pgfqpoint{1.594461in}{4.339524in}}%
\pgfpathcurveto{\pgfqpoint{1.586648in}{4.347338in}}{\pgfqpoint{1.576049in}{4.351728in}}{\pgfqpoint{1.564998in}{4.351728in}}%
\pgfpathcurveto{\pgfqpoint{1.553948in}{4.351728in}}{\pgfqpoint{1.543349in}{4.347338in}}{\pgfqpoint{1.535536in}{4.339524in}}%
\pgfpathcurveto{\pgfqpoint{1.527722in}{4.331710in}}{\pgfqpoint{1.523332in}{4.321111in}}{\pgfqpoint{1.523332in}{4.310061in}}%
\pgfpathcurveto{\pgfqpoint{1.523332in}{4.299011in}}{\pgfqpoint{1.527722in}{4.288412in}}{\pgfqpoint{1.535536in}{4.280599in}}%
\pgfpathcurveto{\pgfqpoint{1.543349in}{4.272785in}}{\pgfqpoint{1.553948in}{4.268395in}}{\pgfqpoint{1.564998in}{4.268395in}}%
\pgfpathlineto{\pgfqpoint{1.564998in}{4.268395in}}%
\pgfpathclose%
\pgfusepath{stroke,fill}%
\end{pgfscope}%
\begin{pgfscope}%
\pgfpathrectangle{\pgfqpoint{0.633874in}{2.920818in}}{\pgfqpoint{2.177280in}{2.201755in}}%
\pgfusepath{clip}%
\pgfsetbuttcap%
\pgfsetroundjoin%
\definecolor{currentfill}{rgb}{1.000000,0.498039,0.054902}%
\pgfsetfillcolor{currentfill}%
\pgfsetlinewidth{0.481800pt}%
\definecolor{currentstroke}{rgb}{1.000000,1.000000,1.000000}%
\pgfsetstrokecolor{currentstroke}%
\pgfsetdash{}{0pt}%
\pgfpathmoveto{\pgfqpoint{1.644876in}{3.996992in}}%
\pgfpathcurveto{\pgfqpoint{1.655926in}{3.996992in}}{\pgfqpoint{1.666525in}{4.001382in}}{\pgfqpoint{1.674339in}{4.009196in}}%
\pgfpathcurveto{\pgfqpoint{1.682152in}{4.017009in}}{\pgfqpoint{1.686543in}{4.027608in}}{\pgfqpoint{1.686543in}{4.038659in}}%
\pgfpathcurveto{\pgfqpoint{1.686543in}{4.049709in}}{\pgfqpoint{1.682152in}{4.060308in}}{\pgfqpoint{1.674339in}{4.068121in}}%
\pgfpathcurveto{\pgfqpoint{1.666525in}{4.075935in}}{\pgfqpoint{1.655926in}{4.080325in}}{\pgfqpoint{1.644876in}{4.080325in}}%
\pgfpathcurveto{\pgfqpoint{1.633826in}{4.080325in}}{\pgfqpoint{1.623227in}{4.075935in}}{\pgfqpoint{1.615413in}{4.068121in}}%
\pgfpathcurveto{\pgfqpoint{1.607599in}{4.060308in}}{\pgfqpoint{1.603209in}{4.049709in}}{\pgfqpoint{1.603209in}{4.038659in}}%
\pgfpathcurveto{\pgfqpoint{1.603209in}{4.027608in}}{\pgfqpoint{1.607599in}{4.017009in}}{\pgfqpoint{1.615413in}{4.009196in}}%
\pgfpathcurveto{\pgfqpoint{1.623227in}{4.001382in}}{\pgfqpoint{1.633826in}{3.996992in}}{\pgfqpoint{1.644876in}{3.996992in}}%
\pgfpathlineto{\pgfqpoint{1.644876in}{3.996992in}}%
\pgfpathclose%
\pgfusepath{stroke,fill}%
\end{pgfscope}%
\begin{pgfscope}%
\pgfpathrectangle{\pgfqpoint{0.633874in}{2.920818in}}{\pgfqpoint{2.177280in}{2.201755in}}%
\pgfusepath{clip}%
\pgfsetbuttcap%
\pgfsetroundjoin%
\definecolor{currentfill}{rgb}{1.000000,0.498039,0.054902}%
\pgfsetfillcolor{currentfill}%
\pgfsetlinewidth{0.481800pt}%
\definecolor{currentstroke}{rgb}{1.000000,1.000000,1.000000}%
\pgfsetstrokecolor{currentstroke}%
\pgfsetdash{}{0pt}%
\pgfpathmoveto{\pgfqpoint{1.724753in}{4.302320in}}%
\pgfpathcurveto{\pgfqpoint{1.735803in}{4.302320in}}{\pgfqpoint{1.746402in}{4.306710in}}{\pgfqpoint{1.754216in}{4.314524in}}%
\pgfpathcurveto{\pgfqpoint{1.762030in}{4.322337in}}{\pgfqpoint{1.766420in}{4.332937in}}{\pgfqpoint{1.766420in}{4.343987in}}%
\pgfpathcurveto{\pgfqpoint{1.766420in}{4.355037in}}{\pgfqpoint{1.762030in}{4.365636in}}{\pgfqpoint{1.754216in}{4.373449in}}%
\pgfpathcurveto{\pgfqpoint{1.746402in}{4.381263in}}{\pgfqpoint{1.735803in}{4.385653in}}{\pgfqpoint{1.724753in}{4.385653in}}%
\pgfpathcurveto{\pgfqpoint{1.713703in}{4.385653in}}{\pgfqpoint{1.703104in}{4.381263in}}{\pgfqpoint{1.695290in}{4.373449in}}%
\pgfpathcurveto{\pgfqpoint{1.687477in}{4.365636in}}{\pgfqpoint{1.683087in}{4.355037in}}{\pgfqpoint{1.683087in}{4.343987in}}%
\pgfpathcurveto{\pgfqpoint{1.683087in}{4.332937in}}{\pgfqpoint{1.687477in}{4.322337in}}{\pgfqpoint{1.695290in}{4.314524in}}%
\pgfpathcurveto{\pgfqpoint{1.703104in}{4.306710in}}{\pgfqpoint{1.713703in}{4.302320in}}{\pgfqpoint{1.724753in}{4.302320in}}%
\pgfpathlineto{\pgfqpoint{1.724753in}{4.302320in}}%
\pgfpathclose%
\pgfusepath{stroke,fill}%
\end{pgfscope}%
\begin{pgfscope}%
\pgfpathrectangle{\pgfqpoint{0.633874in}{2.920818in}}{\pgfqpoint{2.177280in}{2.201755in}}%
\pgfusepath{clip}%
\pgfsetbuttcap%
\pgfsetroundjoin%
\definecolor{currentfill}{rgb}{1.000000,0.498039,0.054902}%
\pgfsetfillcolor{currentfill}%
\pgfsetlinewidth{0.481800pt}%
\definecolor{currentstroke}{rgb}{1.000000,1.000000,1.000000}%
\pgfsetstrokecolor{currentstroke}%
\pgfsetdash{}{0pt}%
\pgfpathmoveto{\pgfqpoint{1.644876in}{4.234469in}}%
\pgfpathcurveto{\pgfqpoint{1.655926in}{4.234469in}}{\pgfqpoint{1.666525in}{4.238860in}}{\pgfqpoint{1.674339in}{4.246673in}}%
\pgfpathcurveto{\pgfqpoint{1.682152in}{4.254487in}}{\pgfqpoint{1.686543in}{4.265086in}}{\pgfqpoint{1.686543in}{4.276136in}}%
\pgfpathcurveto{\pgfqpoint{1.686543in}{4.287186in}}{\pgfqpoint{1.682152in}{4.297785in}}{\pgfqpoint{1.674339in}{4.305599in}}%
\pgfpathcurveto{\pgfqpoint{1.666525in}{4.313412in}}{\pgfqpoint{1.655926in}{4.317803in}}{\pgfqpoint{1.644876in}{4.317803in}}%
\pgfpathcurveto{\pgfqpoint{1.633826in}{4.317803in}}{\pgfqpoint{1.623227in}{4.313412in}}{\pgfqpoint{1.615413in}{4.305599in}}%
\pgfpathcurveto{\pgfqpoint{1.607599in}{4.297785in}}{\pgfqpoint{1.603209in}{4.287186in}}{\pgfqpoint{1.603209in}{4.276136in}}%
\pgfpathcurveto{\pgfqpoint{1.603209in}{4.265086in}}{\pgfqpoint{1.607599in}{4.254487in}}{\pgfqpoint{1.615413in}{4.246673in}}%
\pgfpathcurveto{\pgfqpoint{1.623227in}{4.238860in}}{\pgfqpoint{1.633826in}{4.234469in}}{\pgfqpoint{1.644876in}{4.234469in}}%
\pgfpathlineto{\pgfqpoint{1.644876in}{4.234469in}}%
\pgfpathclose%
\pgfusepath{stroke,fill}%
\end{pgfscope}%
\begin{pgfscope}%
\pgfpathrectangle{\pgfqpoint{0.633874in}{2.920818in}}{\pgfqpoint{2.177280in}{2.201755in}}%
\pgfusepath{clip}%
\pgfsetbuttcap%
\pgfsetroundjoin%
\definecolor{currentfill}{rgb}{1.000000,0.498039,0.054902}%
\pgfsetfillcolor{currentfill}%
\pgfsetlinewidth{0.481800pt}%
\definecolor{currentstroke}{rgb}{1.000000,1.000000,1.000000}%
\pgfsetstrokecolor{currentstroke}%
\pgfsetdash{}{0pt}%
\pgfpathmoveto{\pgfqpoint{1.764692in}{4.098768in}}%
\pgfpathcurveto{\pgfqpoint{1.775742in}{4.098768in}}{\pgfqpoint{1.786341in}{4.103158in}}{\pgfqpoint{1.794155in}{4.110972in}}%
\pgfpathcurveto{\pgfqpoint{1.801968in}{4.118785in}}{\pgfqpoint{1.806359in}{4.129384in}}{\pgfqpoint{1.806359in}{4.140435in}}%
\pgfpathcurveto{\pgfqpoint{1.806359in}{4.151485in}}{\pgfqpoint{1.801968in}{4.162084in}}{\pgfqpoint{1.794155in}{4.169897in}}%
\pgfpathcurveto{\pgfqpoint{1.786341in}{4.177711in}}{\pgfqpoint{1.775742in}{4.182101in}}{\pgfqpoint{1.764692in}{4.182101in}}%
\pgfpathcurveto{\pgfqpoint{1.753642in}{4.182101in}}{\pgfqpoint{1.743043in}{4.177711in}}{\pgfqpoint{1.735229in}{4.169897in}}%
\pgfpathcurveto{\pgfqpoint{1.727416in}{4.162084in}}{\pgfqpoint{1.723025in}{4.151485in}}{\pgfqpoint{1.723025in}{4.140435in}}%
\pgfpathcurveto{\pgfqpoint{1.723025in}{4.129384in}}{\pgfqpoint{1.727416in}{4.118785in}}{\pgfqpoint{1.735229in}{4.110972in}}%
\pgfpathcurveto{\pgfqpoint{1.743043in}{4.103158in}}{\pgfqpoint{1.753642in}{4.098768in}}{\pgfqpoint{1.764692in}{4.098768in}}%
\pgfpathlineto{\pgfqpoint{1.764692in}{4.098768in}}%
\pgfpathclose%
\pgfusepath{stroke,fill}%
\end{pgfscope}%
\begin{pgfscope}%
\pgfpathrectangle{\pgfqpoint{0.633874in}{2.920818in}}{\pgfqpoint{2.177280in}{2.201755in}}%
\pgfusepath{clip}%
\pgfsetbuttcap%
\pgfsetroundjoin%
\definecolor{currentfill}{rgb}{1.000000,0.498039,0.054902}%
\pgfsetfillcolor{currentfill}%
\pgfsetlinewidth{0.481800pt}%
\definecolor{currentstroke}{rgb}{1.000000,1.000000,1.000000}%
\pgfsetstrokecolor{currentstroke}%
\pgfsetdash{}{0pt}%
\pgfpathmoveto{\pgfqpoint{1.844569in}{4.132693in}}%
\pgfpathcurveto{\pgfqpoint{1.855619in}{4.132693in}}{\pgfqpoint{1.866219in}{4.137083in}}{\pgfqpoint{1.874032in}{4.144897in}}%
\pgfpathcurveto{\pgfqpoint{1.881846in}{4.152711in}}{\pgfqpoint{1.886236in}{4.163310in}}{\pgfqpoint{1.886236in}{4.174360in}}%
\pgfpathcurveto{\pgfqpoint{1.886236in}{4.185410in}}{\pgfqpoint{1.881846in}{4.196009in}}{\pgfqpoint{1.874032in}{4.203823in}}%
\pgfpathcurveto{\pgfqpoint{1.866219in}{4.211636in}}{\pgfqpoint{1.855619in}{4.216027in}}{\pgfqpoint{1.844569in}{4.216027in}}%
\pgfpathcurveto{\pgfqpoint{1.833519in}{4.216027in}}{\pgfqpoint{1.822920in}{4.211636in}}{\pgfqpoint{1.815107in}{4.203823in}}%
\pgfpathcurveto{\pgfqpoint{1.807293in}{4.196009in}}{\pgfqpoint{1.802903in}{4.185410in}}{\pgfqpoint{1.802903in}{4.174360in}}%
\pgfpathcurveto{\pgfqpoint{1.802903in}{4.163310in}}{\pgfqpoint{1.807293in}{4.152711in}}{\pgfqpoint{1.815107in}{4.144897in}}%
\pgfpathcurveto{\pgfqpoint{1.822920in}{4.137083in}}{\pgfqpoint{1.833519in}{4.132693in}}{\pgfqpoint{1.844569in}{4.132693in}}%
\pgfpathlineto{\pgfqpoint{1.844569in}{4.132693in}}%
\pgfpathclose%
\pgfusepath{stroke,fill}%
\end{pgfscope}%
\begin{pgfscope}%
\pgfpathrectangle{\pgfqpoint{0.633874in}{2.920818in}}{\pgfqpoint{2.177280in}{2.201755in}}%
\pgfusepath{clip}%
\pgfsetbuttcap%
\pgfsetroundjoin%
\definecolor{currentfill}{rgb}{1.000000,0.498039,0.054902}%
\pgfsetfillcolor{currentfill}%
\pgfsetlinewidth{0.481800pt}%
\definecolor{currentstroke}{rgb}{1.000000,1.000000,1.000000}%
\pgfsetstrokecolor{currentstroke}%
\pgfsetdash{}{0pt}%
\pgfpathmoveto{\pgfqpoint{1.924447in}{4.268395in}}%
\pgfpathcurveto{\pgfqpoint{1.935497in}{4.268395in}}{\pgfqpoint{1.946096in}{4.272785in}}{\pgfqpoint{1.953910in}{4.280599in}}%
\pgfpathcurveto{\pgfqpoint{1.961723in}{4.288412in}}{\pgfqpoint{1.966113in}{4.299011in}}{\pgfqpoint{1.966113in}{4.310061in}}%
\pgfpathcurveto{\pgfqpoint{1.966113in}{4.321111in}}{\pgfqpoint{1.961723in}{4.331710in}}{\pgfqpoint{1.953910in}{4.339524in}}%
\pgfpathcurveto{\pgfqpoint{1.946096in}{4.347338in}}{\pgfqpoint{1.935497in}{4.351728in}}{\pgfqpoint{1.924447in}{4.351728in}}%
\pgfpathcurveto{\pgfqpoint{1.913397in}{4.351728in}}{\pgfqpoint{1.902798in}{4.347338in}}{\pgfqpoint{1.894984in}{4.339524in}}%
\pgfpathcurveto{\pgfqpoint{1.887170in}{4.331710in}}{\pgfqpoint{1.882780in}{4.321111in}}{\pgfqpoint{1.882780in}{4.310061in}}%
\pgfpathcurveto{\pgfqpoint{1.882780in}{4.299011in}}{\pgfqpoint{1.887170in}{4.288412in}}{\pgfqpoint{1.894984in}{4.280599in}}%
\pgfpathcurveto{\pgfqpoint{1.902798in}{4.272785in}}{\pgfqpoint{1.913397in}{4.268395in}}{\pgfqpoint{1.924447in}{4.268395in}}%
\pgfpathlineto{\pgfqpoint{1.924447in}{4.268395in}}%
\pgfpathclose%
\pgfusepath{stroke,fill}%
\end{pgfscope}%
\begin{pgfscope}%
\pgfpathrectangle{\pgfqpoint{0.633874in}{2.920818in}}{\pgfqpoint{2.177280in}{2.201755in}}%
\pgfusepath{clip}%
\pgfsetbuttcap%
\pgfsetroundjoin%
\definecolor{currentfill}{rgb}{1.000000,0.498039,0.054902}%
\pgfsetfillcolor{currentfill}%
\pgfsetlinewidth{0.481800pt}%
\definecolor{currentstroke}{rgb}{1.000000,1.000000,1.000000}%
\pgfsetstrokecolor{currentstroke}%
\pgfsetdash{}{0pt}%
\pgfpathmoveto{\pgfqpoint{1.884508in}{4.336245in}}%
\pgfpathcurveto{\pgfqpoint{1.895558in}{4.336245in}}{\pgfqpoint{1.906157in}{4.340636in}}{\pgfqpoint{1.913971in}{4.348449in}}%
\pgfpathcurveto{\pgfqpoint{1.921784in}{4.356263in}}{\pgfqpoint{1.926175in}{4.366862in}}{\pgfqpoint{1.926175in}{4.377912in}}%
\pgfpathcurveto{\pgfqpoint{1.926175in}{4.388962in}}{\pgfqpoint{1.921784in}{4.399561in}}{\pgfqpoint{1.913971in}{4.407375in}}%
\pgfpathcurveto{\pgfqpoint{1.906157in}{4.415188in}}{\pgfqpoint{1.895558in}{4.419579in}}{\pgfqpoint{1.884508in}{4.419579in}}%
\pgfpathcurveto{\pgfqpoint{1.873458in}{4.419579in}}{\pgfqpoint{1.862859in}{4.415188in}}{\pgfqpoint{1.855045in}{4.407375in}}%
\pgfpathcurveto{\pgfqpoint{1.847232in}{4.399561in}}{\pgfqpoint{1.842841in}{4.388962in}}{\pgfqpoint{1.842841in}{4.377912in}}%
\pgfpathcurveto{\pgfqpoint{1.842841in}{4.366862in}}{\pgfqpoint{1.847232in}{4.356263in}}{\pgfqpoint{1.855045in}{4.348449in}}%
\pgfpathcurveto{\pgfqpoint{1.862859in}{4.340636in}}{\pgfqpoint{1.873458in}{4.336245in}}{\pgfqpoint{1.884508in}{4.336245in}}%
\pgfpathlineto{\pgfqpoint{1.884508in}{4.336245in}}%
\pgfpathclose%
\pgfusepath{stroke,fill}%
\end{pgfscope}%
\begin{pgfscope}%
\pgfpathrectangle{\pgfqpoint{0.633874in}{2.920818in}}{\pgfqpoint{2.177280in}{2.201755in}}%
\pgfusepath{clip}%
\pgfsetbuttcap%
\pgfsetroundjoin%
\definecolor{currentfill}{rgb}{1.000000,0.498039,0.054902}%
\pgfsetfillcolor{currentfill}%
\pgfsetlinewidth{0.481800pt}%
\definecolor{currentstroke}{rgb}{1.000000,1.000000,1.000000}%
\pgfsetstrokecolor{currentstroke}%
\pgfsetdash{}{0pt}%
\pgfpathmoveto{\pgfqpoint{1.604937in}{4.166619in}}%
\pgfpathcurveto{\pgfqpoint{1.615987in}{4.166619in}}{\pgfqpoint{1.626586in}{4.171009in}}{\pgfqpoint{1.634400in}{4.178822in}}%
\pgfpathcurveto{\pgfqpoint{1.642214in}{4.186636in}}{\pgfqpoint{1.646604in}{4.197235in}}{\pgfqpoint{1.646604in}{4.208285in}}%
\pgfpathcurveto{\pgfqpoint{1.646604in}{4.219335in}}{\pgfqpoint{1.642214in}{4.229934in}}{\pgfqpoint{1.634400in}{4.237748in}}%
\pgfpathcurveto{\pgfqpoint{1.626586in}{4.245562in}}{\pgfqpoint{1.615987in}{4.249952in}}{\pgfqpoint{1.604937in}{4.249952in}}%
\pgfpathcurveto{\pgfqpoint{1.593887in}{4.249952in}}{\pgfqpoint{1.583288in}{4.245562in}}{\pgfqpoint{1.575474in}{4.237748in}}%
\pgfpathcurveto{\pgfqpoint{1.567661in}{4.229934in}}{\pgfqpoint{1.563270in}{4.219335in}}{\pgfqpoint{1.563270in}{4.208285in}}%
\pgfpathcurveto{\pgfqpoint{1.563270in}{4.197235in}}{\pgfqpoint{1.567661in}{4.186636in}}{\pgfqpoint{1.575474in}{4.178822in}}%
\pgfpathcurveto{\pgfqpoint{1.583288in}{4.171009in}}{\pgfqpoint{1.593887in}{4.166619in}}{\pgfqpoint{1.604937in}{4.166619in}}%
\pgfpathlineto{\pgfqpoint{1.604937in}{4.166619in}}%
\pgfpathclose%
\pgfusepath{stroke,fill}%
\end{pgfscope}%
\begin{pgfscope}%
\pgfpathrectangle{\pgfqpoint{0.633874in}{2.920818in}}{\pgfqpoint{2.177280in}{2.201755in}}%
\pgfusepath{clip}%
\pgfsetbuttcap%
\pgfsetroundjoin%
\definecolor{currentfill}{rgb}{1.000000,0.498039,0.054902}%
\pgfsetfillcolor{currentfill}%
\pgfsetlinewidth{0.481800pt}%
\definecolor{currentstroke}{rgb}{1.000000,1.000000,1.000000}%
\pgfsetstrokecolor{currentstroke}%
\pgfsetdash{}{0pt}%
\pgfpathmoveto{\pgfqpoint{1.485121in}{3.827365in}}%
\pgfpathcurveto{\pgfqpoint{1.496171in}{3.827365in}}{\pgfqpoint{1.506770in}{3.831755in}}{\pgfqpoint{1.514584in}{3.839569in}}%
\pgfpathcurveto{\pgfqpoint{1.522397in}{3.847383in}}{\pgfqpoint{1.526788in}{3.857982in}}{\pgfqpoint{1.526788in}{3.869032in}}%
\pgfpathcurveto{\pgfqpoint{1.526788in}{3.880082in}}{\pgfqpoint{1.522397in}{3.890681in}}{\pgfqpoint{1.514584in}{3.898495in}}%
\pgfpathcurveto{\pgfqpoint{1.506770in}{3.906308in}}{\pgfqpoint{1.496171in}{3.910698in}}{\pgfqpoint{1.485121in}{3.910698in}}%
\pgfpathcurveto{\pgfqpoint{1.474071in}{3.910698in}}{\pgfqpoint{1.463472in}{3.906308in}}{\pgfqpoint{1.455658in}{3.898495in}}%
\pgfpathcurveto{\pgfqpoint{1.447845in}{3.890681in}}{\pgfqpoint{1.443454in}{3.880082in}}{\pgfqpoint{1.443454in}{3.869032in}}%
\pgfpathcurveto{\pgfqpoint{1.443454in}{3.857982in}}{\pgfqpoint{1.447845in}{3.847383in}}{\pgfqpoint{1.455658in}{3.839569in}}%
\pgfpathcurveto{\pgfqpoint{1.463472in}{3.831755in}}{\pgfqpoint{1.474071in}{3.827365in}}{\pgfqpoint{1.485121in}{3.827365in}}%
\pgfpathlineto{\pgfqpoint{1.485121in}{3.827365in}}%
\pgfpathclose%
\pgfusepath{stroke,fill}%
\end{pgfscope}%
\begin{pgfscope}%
\pgfpathrectangle{\pgfqpoint{0.633874in}{2.920818in}}{\pgfqpoint{2.177280in}{2.201755in}}%
\pgfusepath{clip}%
\pgfsetbuttcap%
\pgfsetroundjoin%
\definecolor{currentfill}{rgb}{1.000000,0.498039,0.054902}%
\pgfsetfillcolor{currentfill}%
\pgfsetlinewidth{0.481800pt}%
\definecolor{currentstroke}{rgb}{1.000000,1.000000,1.000000}%
\pgfsetstrokecolor{currentstroke}%
\pgfsetdash{}{0pt}%
\pgfpathmoveto{\pgfqpoint{1.405244in}{3.929141in}}%
\pgfpathcurveto{\pgfqpoint{1.416294in}{3.929141in}}{\pgfqpoint{1.426893in}{3.933531in}}{\pgfqpoint{1.434706in}{3.941345in}}%
\pgfpathcurveto{\pgfqpoint{1.442520in}{3.949159in}}{\pgfqpoint{1.446910in}{3.959758in}}{\pgfqpoint{1.446910in}{3.970808in}}%
\pgfpathcurveto{\pgfqpoint{1.446910in}{3.981858in}}{\pgfqpoint{1.442520in}{3.992457in}}{\pgfqpoint{1.434706in}{4.000271in}}%
\pgfpathcurveto{\pgfqpoint{1.426893in}{4.008084in}}{\pgfqpoint{1.416294in}{4.012474in}}{\pgfqpoint{1.405244in}{4.012474in}}%
\pgfpathcurveto{\pgfqpoint{1.394193in}{4.012474in}}{\pgfqpoint{1.383594in}{4.008084in}}{\pgfqpoint{1.375781in}{4.000271in}}%
\pgfpathcurveto{\pgfqpoint{1.367967in}{3.992457in}}{\pgfqpoint{1.363577in}{3.981858in}}{\pgfqpoint{1.363577in}{3.970808in}}%
\pgfpathcurveto{\pgfqpoint{1.363577in}{3.959758in}}{\pgfqpoint{1.367967in}{3.949159in}}{\pgfqpoint{1.375781in}{3.941345in}}%
\pgfpathcurveto{\pgfqpoint{1.383594in}{3.933531in}}{\pgfqpoint{1.394193in}{3.929141in}}{\pgfqpoint{1.405244in}{3.929141in}}%
\pgfpathlineto{\pgfqpoint{1.405244in}{3.929141in}}%
\pgfpathclose%
\pgfusepath{stroke,fill}%
\end{pgfscope}%
\begin{pgfscope}%
\pgfpathrectangle{\pgfqpoint{0.633874in}{2.920818in}}{\pgfqpoint{2.177280in}{2.201755in}}%
\pgfusepath{clip}%
\pgfsetbuttcap%
\pgfsetroundjoin%
\definecolor{currentfill}{rgb}{1.000000,0.498039,0.054902}%
\pgfsetfillcolor{currentfill}%
\pgfsetlinewidth{0.481800pt}%
\definecolor{currentstroke}{rgb}{1.000000,1.000000,1.000000}%
\pgfsetstrokecolor{currentstroke}%
\pgfsetdash{}{0pt}%
\pgfpathmoveto{\pgfqpoint{1.405244in}{3.895216in}}%
\pgfpathcurveto{\pgfqpoint{1.416294in}{3.895216in}}{\pgfqpoint{1.426893in}{3.899606in}}{\pgfqpoint{1.434706in}{3.907420in}}%
\pgfpathcurveto{\pgfqpoint{1.442520in}{3.915233in}}{\pgfqpoint{1.446910in}{3.925832in}}{\pgfqpoint{1.446910in}{3.936882in}}%
\pgfpathcurveto{\pgfqpoint{1.446910in}{3.947933in}}{\pgfqpoint{1.442520in}{3.958532in}}{\pgfqpoint{1.434706in}{3.966345in}}%
\pgfpathcurveto{\pgfqpoint{1.426893in}{3.974159in}}{\pgfqpoint{1.416294in}{3.978549in}}{\pgfqpoint{1.405244in}{3.978549in}}%
\pgfpathcurveto{\pgfqpoint{1.394193in}{3.978549in}}{\pgfqpoint{1.383594in}{3.974159in}}{\pgfqpoint{1.375781in}{3.966345in}}%
\pgfpathcurveto{\pgfqpoint{1.367967in}{3.958532in}}{\pgfqpoint{1.363577in}{3.947933in}}{\pgfqpoint{1.363577in}{3.936882in}}%
\pgfpathcurveto{\pgfqpoint{1.363577in}{3.925832in}}{\pgfqpoint{1.367967in}{3.915233in}}{\pgfqpoint{1.375781in}{3.907420in}}%
\pgfpathcurveto{\pgfqpoint{1.383594in}{3.899606in}}{\pgfqpoint{1.394193in}{3.895216in}}{\pgfqpoint{1.405244in}{3.895216in}}%
\pgfpathlineto{\pgfqpoint{1.405244in}{3.895216in}}%
\pgfpathclose%
\pgfusepath{stroke,fill}%
\end{pgfscope}%
\begin{pgfscope}%
\pgfpathrectangle{\pgfqpoint{0.633874in}{2.920818in}}{\pgfqpoint{2.177280in}{2.201755in}}%
\pgfusepath{clip}%
\pgfsetbuttcap%
\pgfsetroundjoin%
\definecolor{currentfill}{rgb}{1.000000,0.498039,0.054902}%
\pgfsetfillcolor{currentfill}%
\pgfsetlinewidth{0.481800pt}%
\definecolor{currentstroke}{rgb}{1.000000,1.000000,1.000000}%
\pgfsetstrokecolor{currentstroke}%
\pgfsetdash{}{0pt}%
\pgfpathmoveto{\pgfqpoint{1.525060in}{3.963066in}}%
\pgfpathcurveto{\pgfqpoint{1.536110in}{3.963066in}}{\pgfqpoint{1.546709in}{3.967457in}}{\pgfqpoint{1.554523in}{3.975270in}}%
\pgfpathcurveto{\pgfqpoint{1.562336in}{3.983084in}}{\pgfqpoint{1.566726in}{3.993683in}}{\pgfqpoint{1.566726in}{4.004733in}}%
\pgfpathcurveto{\pgfqpoint{1.566726in}{4.015783in}}{\pgfqpoint{1.562336in}{4.026382in}}{\pgfqpoint{1.554523in}{4.034196in}}%
\pgfpathcurveto{\pgfqpoint{1.546709in}{4.042010in}}{\pgfqpoint{1.536110in}{4.046400in}}{\pgfqpoint{1.525060in}{4.046400in}}%
\pgfpathcurveto{\pgfqpoint{1.514010in}{4.046400in}}{\pgfqpoint{1.503411in}{4.042010in}}{\pgfqpoint{1.495597in}{4.034196in}}%
\pgfpathcurveto{\pgfqpoint{1.487783in}{4.026382in}}{\pgfqpoint{1.483393in}{4.015783in}}{\pgfqpoint{1.483393in}{4.004733in}}%
\pgfpathcurveto{\pgfqpoint{1.483393in}{3.993683in}}{\pgfqpoint{1.487783in}{3.983084in}}{\pgfqpoint{1.495597in}{3.975270in}}%
\pgfpathcurveto{\pgfqpoint{1.503411in}{3.967457in}}{\pgfqpoint{1.514010in}{3.963066in}}{\pgfqpoint{1.525060in}{3.963066in}}%
\pgfpathlineto{\pgfqpoint{1.525060in}{3.963066in}}%
\pgfpathclose%
\pgfusepath{stroke,fill}%
\end{pgfscope}%
\begin{pgfscope}%
\pgfpathrectangle{\pgfqpoint{0.633874in}{2.920818in}}{\pgfqpoint{2.177280in}{2.201755in}}%
\pgfusepath{clip}%
\pgfsetbuttcap%
\pgfsetroundjoin%
\definecolor{currentfill}{rgb}{1.000000,0.498039,0.054902}%
\pgfsetfillcolor{currentfill}%
\pgfsetlinewidth{0.481800pt}%
\definecolor{currentstroke}{rgb}{1.000000,1.000000,1.000000}%
\pgfsetstrokecolor{currentstroke}%
\pgfsetdash{}{0pt}%
\pgfpathmoveto{\pgfqpoint{1.604937in}{4.370171in}}%
\pgfpathcurveto{\pgfqpoint{1.615987in}{4.370171in}}{\pgfqpoint{1.626586in}{4.374561in}}{\pgfqpoint{1.634400in}{4.382375in}}%
\pgfpathcurveto{\pgfqpoint{1.642214in}{4.390188in}}{\pgfqpoint{1.646604in}{4.400787in}}{\pgfqpoint{1.646604in}{4.411837in}}%
\pgfpathcurveto{\pgfqpoint{1.646604in}{4.422887in}}{\pgfqpoint{1.642214in}{4.433486in}}{\pgfqpoint{1.634400in}{4.441300in}}%
\pgfpathcurveto{\pgfqpoint{1.626586in}{4.449114in}}{\pgfqpoint{1.615987in}{4.453504in}}{\pgfqpoint{1.604937in}{4.453504in}}%
\pgfpathcurveto{\pgfqpoint{1.593887in}{4.453504in}}{\pgfqpoint{1.583288in}{4.449114in}}{\pgfqpoint{1.575474in}{4.441300in}}%
\pgfpathcurveto{\pgfqpoint{1.567661in}{4.433486in}}{\pgfqpoint{1.563270in}{4.422887in}}{\pgfqpoint{1.563270in}{4.411837in}}%
\pgfpathcurveto{\pgfqpoint{1.563270in}{4.400787in}}{\pgfqpoint{1.567661in}{4.390188in}}{\pgfqpoint{1.575474in}{4.382375in}}%
\pgfpathcurveto{\pgfqpoint{1.583288in}{4.374561in}}{\pgfqpoint{1.593887in}{4.370171in}}{\pgfqpoint{1.604937in}{4.370171in}}%
\pgfpathlineto{\pgfqpoint{1.604937in}{4.370171in}}%
\pgfpathclose%
\pgfusepath{stroke,fill}%
\end{pgfscope}%
\begin{pgfscope}%
\pgfpathrectangle{\pgfqpoint{0.633874in}{2.920818in}}{\pgfqpoint{2.177280in}{2.201755in}}%
\pgfusepath{clip}%
\pgfsetbuttcap%
\pgfsetroundjoin%
\definecolor{currentfill}{rgb}{1.000000,0.498039,0.054902}%
\pgfsetfillcolor{currentfill}%
\pgfsetlinewidth{0.481800pt}%
\definecolor{currentstroke}{rgb}{1.000000,1.000000,1.000000}%
\pgfsetstrokecolor{currentstroke}%
\pgfsetdash{}{0pt}%
\pgfpathmoveto{\pgfqpoint{1.365305in}{4.166619in}}%
\pgfpathcurveto{\pgfqpoint{1.376355in}{4.166619in}}{\pgfqpoint{1.386954in}{4.171009in}}{\pgfqpoint{1.394768in}{4.178822in}}%
\pgfpathcurveto{\pgfqpoint{1.402581in}{4.186636in}}{\pgfqpoint{1.406972in}{4.197235in}}{\pgfqpoint{1.406972in}{4.208285in}}%
\pgfpathcurveto{\pgfqpoint{1.406972in}{4.219335in}}{\pgfqpoint{1.402581in}{4.229934in}}{\pgfqpoint{1.394768in}{4.237748in}}%
\pgfpathcurveto{\pgfqpoint{1.386954in}{4.245562in}}{\pgfqpoint{1.376355in}{4.249952in}}{\pgfqpoint{1.365305in}{4.249952in}}%
\pgfpathcurveto{\pgfqpoint{1.354255in}{4.249952in}}{\pgfqpoint{1.343656in}{4.245562in}}{\pgfqpoint{1.335842in}{4.237748in}}%
\pgfpathcurveto{\pgfqpoint{1.328029in}{4.229934in}}{\pgfqpoint{1.323638in}{4.219335in}}{\pgfqpoint{1.323638in}{4.208285in}}%
\pgfpathcurveto{\pgfqpoint{1.323638in}{4.197235in}}{\pgfqpoint{1.328029in}{4.186636in}}{\pgfqpoint{1.335842in}{4.178822in}}%
\pgfpathcurveto{\pgfqpoint{1.343656in}{4.171009in}}{\pgfqpoint{1.354255in}{4.166619in}}{\pgfqpoint{1.365305in}{4.166619in}}%
\pgfpathlineto{\pgfqpoint{1.365305in}{4.166619in}}%
\pgfpathclose%
\pgfusepath{stroke,fill}%
\end{pgfscope}%
\begin{pgfscope}%
\pgfpathrectangle{\pgfqpoint{0.633874in}{2.920818in}}{\pgfqpoint{2.177280in}{2.201755in}}%
\pgfusepath{clip}%
\pgfsetbuttcap%
\pgfsetroundjoin%
\definecolor{currentfill}{rgb}{1.000000,0.498039,0.054902}%
\pgfsetfillcolor{currentfill}%
\pgfsetlinewidth{0.481800pt}%
\definecolor{currentstroke}{rgb}{1.000000,1.000000,1.000000}%
\pgfsetstrokecolor{currentstroke}%
\pgfsetdash{}{0pt}%
\pgfpathmoveto{\pgfqpoint{1.604937in}{4.166619in}}%
\pgfpathcurveto{\pgfqpoint{1.615987in}{4.166619in}}{\pgfqpoint{1.626586in}{4.171009in}}{\pgfqpoint{1.634400in}{4.178822in}}%
\pgfpathcurveto{\pgfqpoint{1.642214in}{4.186636in}}{\pgfqpoint{1.646604in}{4.197235in}}{\pgfqpoint{1.646604in}{4.208285in}}%
\pgfpathcurveto{\pgfqpoint{1.646604in}{4.219335in}}{\pgfqpoint{1.642214in}{4.229934in}}{\pgfqpoint{1.634400in}{4.237748in}}%
\pgfpathcurveto{\pgfqpoint{1.626586in}{4.245562in}}{\pgfqpoint{1.615987in}{4.249952in}}{\pgfqpoint{1.604937in}{4.249952in}}%
\pgfpathcurveto{\pgfqpoint{1.593887in}{4.249952in}}{\pgfqpoint{1.583288in}{4.245562in}}{\pgfqpoint{1.575474in}{4.237748in}}%
\pgfpathcurveto{\pgfqpoint{1.567661in}{4.229934in}}{\pgfqpoint{1.563270in}{4.219335in}}{\pgfqpoint{1.563270in}{4.208285in}}%
\pgfpathcurveto{\pgfqpoint{1.563270in}{4.197235in}}{\pgfqpoint{1.567661in}{4.186636in}}{\pgfqpoint{1.575474in}{4.178822in}}%
\pgfpathcurveto{\pgfqpoint{1.583288in}{4.171009in}}{\pgfqpoint{1.593887in}{4.166619in}}{\pgfqpoint{1.604937in}{4.166619in}}%
\pgfpathlineto{\pgfqpoint{1.604937in}{4.166619in}}%
\pgfpathclose%
\pgfusepath{stroke,fill}%
\end{pgfscope}%
\begin{pgfscope}%
\pgfpathrectangle{\pgfqpoint{0.633874in}{2.920818in}}{\pgfqpoint{2.177280in}{2.201755in}}%
\pgfusepath{clip}%
\pgfsetbuttcap%
\pgfsetroundjoin%
\definecolor{currentfill}{rgb}{1.000000,0.498039,0.054902}%
\pgfsetfillcolor{currentfill}%
\pgfsetlinewidth{0.481800pt}%
\definecolor{currentstroke}{rgb}{1.000000,1.000000,1.000000}%
\pgfsetstrokecolor{currentstroke}%
\pgfsetdash{}{0pt}%
\pgfpathmoveto{\pgfqpoint{1.884508in}{4.234469in}}%
\pgfpathcurveto{\pgfqpoint{1.895558in}{4.234469in}}{\pgfqpoint{1.906157in}{4.238860in}}{\pgfqpoint{1.913971in}{4.246673in}}%
\pgfpathcurveto{\pgfqpoint{1.921784in}{4.254487in}}{\pgfqpoint{1.926175in}{4.265086in}}{\pgfqpoint{1.926175in}{4.276136in}}%
\pgfpathcurveto{\pgfqpoint{1.926175in}{4.287186in}}{\pgfqpoint{1.921784in}{4.297785in}}{\pgfqpoint{1.913971in}{4.305599in}}%
\pgfpathcurveto{\pgfqpoint{1.906157in}{4.313412in}}{\pgfqpoint{1.895558in}{4.317803in}}{\pgfqpoint{1.884508in}{4.317803in}}%
\pgfpathcurveto{\pgfqpoint{1.873458in}{4.317803in}}{\pgfqpoint{1.862859in}{4.313412in}}{\pgfqpoint{1.855045in}{4.305599in}}%
\pgfpathcurveto{\pgfqpoint{1.847232in}{4.297785in}}{\pgfqpoint{1.842841in}{4.287186in}}{\pgfqpoint{1.842841in}{4.276136in}}%
\pgfpathcurveto{\pgfqpoint{1.842841in}{4.265086in}}{\pgfqpoint{1.847232in}{4.254487in}}{\pgfqpoint{1.855045in}{4.246673in}}%
\pgfpathcurveto{\pgfqpoint{1.862859in}{4.238860in}}{\pgfqpoint{1.873458in}{4.234469in}}{\pgfqpoint{1.884508in}{4.234469in}}%
\pgfpathlineto{\pgfqpoint{1.884508in}{4.234469in}}%
\pgfpathclose%
\pgfusepath{stroke,fill}%
\end{pgfscope}%
\begin{pgfscope}%
\pgfpathrectangle{\pgfqpoint{0.633874in}{2.920818in}}{\pgfqpoint{2.177280in}{2.201755in}}%
\pgfusepath{clip}%
\pgfsetbuttcap%
\pgfsetroundjoin%
\definecolor{currentfill}{rgb}{1.000000,0.498039,0.054902}%
\pgfsetfillcolor{currentfill}%
\pgfsetlinewidth{0.481800pt}%
\definecolor{currentstroke}{rgb}{1.000000,1.000000,1.000000}%
\pgfsetstrokecolor{currentstroke}%
\pgfsetdash{}{0pt}%
\pgfpathmoveto{\pgfqpoint{1.724753in}{4.132693in}}%
\pgfpathcurveto{\pgfqpoint{1.735803in}{4.132693in}}{\pgfqpoint{1.746402in}{4.137083in}}{\pgfqpoint{1.754216in}{4.144897in}}%
\pgfpathcurveto{\pgfqpoint{1.762030in}{4.152711in}}{\pgfqpoint{1.766420in}{4.163310in}}{\pgfqpoint{1.766420in}{4.174360in}}%
\pgfpathcurveto{\pgfqpoint{1.766420in}{4.185410in}}{\pgfqpoint{1.762030in}{4.196009in}}{\pgfqpoint{1.754216in}{4.203823in}}%
\pgfpathcurveto{\pgfqpoint{1.746402in}{4.211636in}}{\pgfqpoint{1.735803in}{4.216027in}}{\pgfqpoint{1.724753in}{4.216027in}}%
\pgfpathcurveto{\pgfqpoint{1.713703in}{4.216027in}}{\pgfqpoint{1.703104in}{4.211636in}}{\pgfqpoint{1.695290in}{4.203823in}}%
\pgfpathcurveto{\pgfqpoint{1.687477in}{4.196009in}}{\pgfqpoint{1.683087in}{4.185410in}}{\pgfqpoint{1.683087in}{4.174360in}}%
\pgfpathcurveto{\pgfqpoint{1.683087in}{4.163310in}}{\pgfqpoint{1.687477in}{4.152711in}}{\pgfqpoint{1.695290in}{4.144897in}}%
\pgfpathcurveto{\pgfqpoint{1.703104in}{4.137083in}}{\pgfqpoint{1.713703in}{4.132693in}}{\pgfqpoint{1.724753in}{4.132693in}}%
\pgfpathlineto{\pgfqpoint{1.724753in}{4.132693in}}%
\pgfpathclose%
\pgfusepath{stroke,fill}%
\end{pgfscope}%
\begin{pgfscope}%
\pgfpathrectangle{\pgfqpoint{0.633874in}{2.920818in}}{\pgfqpoint{2.177280in}{2.201755in}}%
\pgfusepath{clip}%
\pgfsetbuttcap%
\pgfsetroundjoin%
\definecolor{currentfill}{rgb}{1.000000,0.498039,0.054902}%
\pgfsetfillcolor{currentfill}%
\pgfsetlinewidth{0.481800pt}%
\definecolor{currentstroke}{rgb}{1.000000,1.000000,1.000000}%
\pgfsetstrokecolor{currentstroke}%
\pgfsetdash{}{0pt}%
\pgfpathmoveto{\pgfqpoint{1.445182in}{4.030917in}}%
\pgfpathcurveto{\pgfqpoint{1.456232in}{4.030917in}}{\pgfqpoint{1.466831in}{4.035307in}}{\pgfqpoint{1.474645in}{4.043121in}}%
\pgfpathcurveto{\pgfqpoint{1.482459in}{4.050935in}}{\pgfqpoint{1.486849in}{4.061534in}}{\pgfqpoint{1.486849in}{4.072584in}}%
\pgfpathcurveto{\pgfqpoint{1.486849in}{4.083634in}}{\pgfqpoint{1.482459in}{4.094233in}}{\pgfqpoint{1.474645in}{4.102047in}}%
\pgfpathcurveto{\pgfqpoint{1.466831in}{4.109860in}}{\pgfqpoint{1.456232in}{4.114251in}}{\pgfqpoint{1.445182in}{4.114251in}}%
\pgfpathcurveto{\pgfqpoint{1.434132in}{4.114251in}}{\pgfqpoint{1.423533in}{4.109860in}}{\pgfqpoint{1.415720in}{4.102047in}}%
\pgfpathcurveto{\pgfqpoint{1.407906in}{4.094233in}}{\pgfqpoint{1.403516in}{4.083634in}}{\pgfqpoint{1.403516in}{4.072584in}}%
\pgfpathcurveto{\pgfqpoint{1.403516in}{4.061534in}}{\pgfqpoint{1.407906in}{4.050935in}}{\pgfqpoint{1.415720in}{4.043121in}}%
\pgfpathcurveto{\pgfqpoint{1.423533in}{4.035307in}}{\pgfqpoint{1.434132in}{4.030917in}}{\pgfqpoint{1.445182in}{4.030917in}}%
\pgfpathlineto{\pgfqpoint{1.445182in}{4.030917in}}%
\pgfpathclose%
\pgfusepath{stroke,fill}%
\end{pgfscope}%
\begin{pgfscope}%
\pgfpathrectangle{\pgfqpoint{0.633874in}{2.920818in}}{\pgfqpoint{2.177280in}{2.201755in}}%
\pgfusepath{clip}%
\pgfsetbuttcap%
\pgfsetroundjoin%
\definecolor{currentfill}{rgb}{1.000000,0.498039,0.054902}%
\pgfsetfillcolor{currentfill}%
\pgfsetlinewidth{0.481800pt}%
\definecolor{currentstroke}{rgb}{1.000000,1.000000,1.000000}%
\pgfsetstrokecolor{currentstroke}%
\pgfsetdash{}{0pt}%
\pgfpathmoveto{\pgfqpoint{1.405244in}{3.996992in}}%
\pgfpathcurveto{\pgfqpoint{1.416294in}{3.996992in}}{\pgfqpoint{1.426893in}{4.001382in}}{\pgfqpoint{1.434706in}{4.009196in}}%
\pgfpathcurveto{\pgfqpoint{1.442520in}{4.017009in}}{\pgfqpoint{1.446910in}{4.027608in}}{\pgfqpoint{1.446910in}{4.038659in}}%
\pgfpathcurveto{\pgfqpoint{1.446910in}{4.049709in}}{\pgfqpoint{1.442520in}{4.060308in}}{\pgfqpoint{1.434706in}{4.068121in}}%
\pgfpathcurveto{\pgfqpoint{1.426893in}{4.075935in}}{\pgfqpoint{1.416294in}{4.080325in}}{\pgfqpoint{1.405244in}{4.080325in}}%
\pgfpathcurveto{\pgfqpoint{1.394193in}{4.080325in}}{\pgfqpoint{1.383594in}{4.075935in}}{\pgfqpoint{1.375781in}{4.068121in}}%
\pgfpathcurveto{\pgfqpoint{1.367967in}{4.060308in}}{\pgfqpoint{1.363577in}{4.049709in}}{\pgfqpoint{1.363577in}{4.038659in}}%
\pgfpathcurveto{\pgfqpoint{1.363577in}{4.027608in}}{\pgfqpoint{1.367967in}{4.017009in}}{\pgfqpoint{1.375781in}{4.009196in}}%
\pgfpathcurveto{\pgfqpoint{1.383594in}{4.001382in}}{\pgfqpoint{1.394193in}{3.996992in}}{\pgfqpoint{1.405244in}{3.996992in}}%
\pgfpathlineto{\pgfqpoint{1.405244in}{3.996992in}}%
\pgfpathclose%
\pgfusepath{stroke,fill}%
\end{pgfscope}%
\begin{pgfscope}%
\pgfpathrectangle{\pgfqpoint{0.633874in}{2.920818in}}{\pgfqpoint{2.177280in}{2.201755in}}%
\pgfusepath{clip}%
\pgfsetbuttcap%
\pgfsetroundjoin%
\definecolor{currentfill}{rgb}{1.000000,0.498039,0.054902}%
\pgfsetfillcolor{currentfill}%
\pgfsetlinewidth{0.481800pt}%
\definecolor{currentstroke}{rgb}{1.000000,1.000000,1.000000}%
\pgfsetstrokecolor{currentstroke}%
\pgfsetdash{}{0pt}%
\pgfpathmoveto{\pgfqpoint{1.405244in}{4.132693in}}%
\pgfpathcurveto{\pgfqpoint{1.416294in}{4.132693in}}{\pgfqpoint{1.426893in}{4.137083in}}{\pgfqpoint{1.434706in}{4.144897in}}%
\pgfpathcurveto{\pgfqpoint{1.442520in}{4.152711in}}{\pgfqpoint{1.446910in}{4.163310in}}{\pgfqpoint{1.446910in}{4.174360in}}%
\pgfpathcurveto{\pgfqpoint{1.446910in}{4.185410in}}{\pgfqpoint{1.442520in}{4.196009in}}{\pgfqpoint{1.434706in}{4.203823in}}%
\pgfpathcurveto{\pgfqpoint{1.426893in}{4.211636in}}{\pgfqpoint{1.416294in}{4.216027in}}{\pgfqpoint{1.405244in}{4.216027in}}%
\pgfpathcurveto{\pgfqpoint{1.394193in}{4.216027in}}{\pgfqpoint{1.383594in}{4.211636in}}{\pgfqpoint{1.375781in}{4.203823in}}%
\pgfpathcurveto{\pgfqpoint{1.367967in}{4.196009in}}{\pgfqpoint{1.363577in}{4.185410in}}{\pgfqpoint{1.363577in}{4.174360in}}%
\pgfpathcurveto{\pgfqpoint{1.363577in}{4.163310in}}{\pgfqpoint{1.367967in}{4.152711in}}{\pgfqpoint{1.375781in}{4.144897in}}%
\pgfpathcurveto{\pgfqpoint{1.383594in}{4.137083in}}{\pgfqpoint{1.394193in}{4.132693in}}{\pgfqpoint{1.405244in}{4.132693in}}%
\pgfpathlineto{\pgfqpoint{1.405244in}{4.132693in}}%
\pgfpathclose%
\pgfusepath{stroke,fill}%
\end{pgfscope}%
\begin{pgfscope}%
\pgfpathrectangle{\pgfqpoint{0.633874in}{2.920818in}}{\pgfqpoint{2.177280in}{2.201755in}}%
\pgfusepath{clip}%
\pgfsetbuttcap%
\pgfsetroundjoin%
\definecolor{currentfill}{rgb}{1.000000,0.498039,0.054902}%
\pgfsetfillcolor{currentfill}%
\pgfsetlinewidth{0.481800pt}%
\definecolor{currentstroke}{rgb}{1.000000,1.000000,1.000000}%
\pgfsetstrokecolor{currentstroke}%
\pgfsetdash{}{0pt}%
\pgfpathmoveto{\pgfqpoint{1.644876in}{4.200544in}}%
\pgfpathcurveto{\pgfqpoint{1.655926in}{4.200544in}}{\pgfqpoint{1.666525in}{4.204934in}}{\pgfqpoint{1.674339in}{4.212748in}}%
\pgfpathcurveto{\pgfqpoint{1.682152in}{4.220561in}}{\pgfqpoint{1.686543in}{4.231160in}}{\pgfqpoint{1.686543in}{4.242211in}}%
\pgfpathcurveto{\pgfqpoint{1.686543in}{4.253261in}}{\pgfqpoint{1.682152in}{4.263860in}}{\pgfqpoint{1.674339in}{4.271673in}}%
\pgfpathcurveto{\pgfqpoint{1.666525in}{4.279487in}}{\pgfqpoint{1.655926in}{4.283877in}}{\pgfqpoint{1.644876in}{4.283877in}}%
\pgfpathcurveto{\pgfqpoint{1.633826in}{4.283877in}}{\pgfqpoint{1.623227in}{4.279487in}}{\pgfqpoint{1.615413in}{4.271673in}}%
\pgfpathcurveto{\pgfqpoint{1.607599in}{4.263860in}}{\pgfqpoint{1.603209in}{4.253261in}}{\pgfqpoint{1.603209in}{4.242211in}}%
\pgfpathcurveto{\pgfqpoint{1.603209in}{4.231160in}}{\pgfqpoint{1.607599in}{4.220561in}}{\pgfqpoint{1.615413in}{4.212748in}}%
\pgfpathcurveto{\pgfqpoint{1.623227in}{4.204934in}}{\pgfqpoint{1.633826in}{4.200544in}}{\pgfqpoint{1.644876in}{4.200544in}}%
\pgfpathlineto{\pgfqpoint{1.644876in}{4.200544in}}%
\pgfpathclose%
\pgfusepath{stroke,fill}%
\end{pgfscope}%
\begin{pgfscope}%
\pgfpathrectangle{\pgfqpoint{0.633874in}{2.920818in}}{\pgfqpoint{2.177280in}{2.201755in}}%
\pgfusepath{clip}%
\pgfsetbuttcap%
\pgfsetroundjoin%
\definecolor{currentfill}{rgb}{1.000000,0.498039,0.054902}%
\pgfsetfillcolor{currentfill}%
\pgfsetlinewidth{0.481800pt}%
\definecolor{currentstroke}{rgb}{1.000000,1.000000,1.000000}%
\pgfsetstrokecolor{currentstroke}%
\pgfsetdash{}{0pt}%
\pgfpathmoveto{\pgfqpoint{1.525060in}{3.996992in}}%
\pgfpathcurveto{\pgfqpoint{1.536110in}{3.996992in}}{\pgfqpoint{1.546709in}{4.001382in}}{\pgfqpoint{1.554523in}{4.009196in}}%
\pgfpathcurveto{\pgfqpoint{1.562336in}{4.017009in}}{\pgfqpoint{1.566726in}{4.027608in}}{\pgfqpoint{1.566726in}{4.038659in}}%
\pgfpathcurveto{\pgfqpoint{1.566726in}{4.049709in}}{\pgfqpoint{1.562336in}{4.060308in}}{\pgfqpoint{1.554523in}{4.068121in}}%
\pgfpathcurveto{\pgfqpoint{1.546709in}{4.075935in}}{\pgfqpoint{1.536110in}{4.080325in}}{\pgfqpoint{1.525060in}{4.080325in}}%
\pgfpathcurveto{\pgfqpoint{1.514010in}{4.080325in}}{\pgfqpoint{1.503411in}{4.075935in}}{\pgfqpoint{1.495597in}{4.068121in}}%
\pgfpathcurveto{\pgfqpoint{1.487783in}{4.060308in}}{\pgfqpoint{1.483393in}{4.049709in}}{\pgfqpoint{1.483393in}{4.038659in}}%
\pgfpathcurveto{\pgfqpoint{1.483393in}{4.027608in}}{\pgfqpoint{1.487783in}{4.017009in}}{\pgfqpoint{1.495597in}{4.009196in}}%
\pgfpathcurveto{\pgfqpoint{1.503411in}{4.001382in}}{\pgfqpoint{1.514010in}{3.996992in}}{\pgfqpoint{1.525060in}{3.996992in}}%
\pgfpathlineto{\pgfqpoint{1.525060in}{3.996992in}}%
\pgfpathclose%
\pgfusepath{stroke,fill}%
\end{pgfscope}%
\begin{pgfscope}%
\pgfpathrectangle{\pgfqpoint{0.633874in}{2.920818in}}{\pgfqpoint{2.177280in}{2.201755in}}%
\pgfusepath{clip}%
\pgfsetbuttcap%
\pgfsetroundjoin%
\definecolor{currentfill}{rgb}{1.000000,0.498039,0.054902}%
\pgfsetfillcolor{currentfill}%
\pgfsetlinewidth{0.481800pt}%
\definecolor{currentstroke}{rgb}{1.000000,1.000000,1.000000}%
\pgfsetstrokecolor{currentstroke}%
\pgfsetdash{}{0pt}%
\pgfpathmoveto{\pgfqpoint{1.205550in}{3.759514in}}%
\pgfpathcurveto{\pgfqpoint{1.216600in}{3.759514in}}{\pgfqpoint{1.227199in}{3.763905in}}{\pgfqpoint{1.235013in}{3.771718in}}%
\pgfpathcurveto{\pgfqpoint{1.242826in}{3.779532in}}{\pgfqpoint{1.247217in}{3.790131in}}{\pgfqpoint{1.247217in}{3.801181in}}%
\pgfpathcurveto{\pgfqpoint{1.247217in}{3.812231in}}{\pgfqpoint{1.242826in}{3.822830in}}{\pgfqpoint{1.235013in}{3.830644in}}%
\pgfpathcurveto{\pgfqpoint{1.227199in}{3.838457in}}{\pgfqpoint{1.216600in}{3.842848in}}{\pgfqpoint{1.205550in}{3.842848in}}%
\pgfpathcurveto{\pgfqpoint{1.194500in}{3.842848in}}{\pgfqpoint{1.183901in}{3.838457in}}{\pgfqpoint{1.176087in}{3.830644in}}%
\pgfpathcurveto{\pgfqpoint{1.168274in}{3.822830in}}{\pgfqpoint{1.163883in}{3.812231in}}{\pgfqpoint{1.163883in}{3.801181in}}%
\pgfpathcurveto{\pgfqpoint{1.163883in}{3.790131in}}{\pgfqpoint{1.168274in}{3.779532in}}{\pgfqpoint{1.176087in}{3.771718in}}%
\pgfpathcurveto{\pgfqpoint{1.183901in}{3.763905in}}{\pgfqpoint{1.194500in}{3.759514in}}{\pgfqpoint{1.205550in}{3.759514in}}%
\pgfpathlineto{\pgfqpoint{1.205550in}{3.759514in}}%
\pgfpathclose%
\pgfusepath{stroke,fill}%
\end{pgfscope}%
\begin{pgfscope}%
\pgfpathrectangle{\pgfqpoint{0.633874in}{2.920818in}}{\pgfqpoint{2.177280in}{2.201755in}}%
\pgfusepath{clip}%
\pgfsetbuttcap%
\pgfsetroundjoin%
\definecolor{currentfill}{rgb}{1.000000,0.498039,0.054902}%
\pgfsetfillcolor{currentfill}%
\pgfsetlinewidth{0.481800pt}%
\definecolor{currentstroke}{rgb}{1.000000,1.000000,1.000000}%
\pgfsetstrokecolor{currentstroke}%
\pgfsetdash{}{0pt}%
\pgfpathmoveto{\pgfqpoint{1.445182in}{4.064843in}}%
\pgfpathcurveto{\pgfqpoint{1.456232in}{4.064843in}}{\pgfqpoint{1.466831in}{4.069233in}}{\pgfqpoint{1.474645in}{4.077046in}}%
\pgfpathcurveto{\pgfqpoint{1.482459in}{4.084860in}}{\pgfqpoint{1.486849in}{4.095459in}}{\pgfqpoint{1.486849in}{4.106509in}}%
\pgfpathcurveto{\pgfqpoint{1.486849in}{4.117559in}}{\pgfqpoint{1.482459in}{4.128158in}}{\pgfqpoint{1.474645in}{4.135972in}}%
\pgfpathcurveto{\pgfqpoint{1.466831in}{4.143786in}}{\pgfqpoint{1.456232in}{4.148176in}}{\pgfqpoint{1.445182in}{4.148176in}}%
\pgfpathcurveto{\pgfqpoint{1.434132in}{4.148176in}}{\pgfqpoint{1.423533in}{4.143786in}}{\pgfqpoint{1.415720in}{4.135972in}}%
\pgfpathcurveto{\pgfqpoint{1.407906in}{4.128158in}}{\pgfqpoint{1.403516in}{4.117559in}}{\pgfqpoint{1.403516in}{4.106509in}}%
\pgfpathcurveto{\pgfqpoint{1.403516in}{4.095459in}}{\pgfqpoint{1.407906in}{4.084860in}}{\pgfqpoint{1.415720in}{4.077046in}}%
\pgfpathcurveto{\pgfqpoint{1.423533in}{4.069233in}}{\pgfqpoint{1.434132in}{4.064843in}}{\pgfqpoint{1.445182in}{4.064843in}}%
\pgfpathlineto{\pgfqpoint{1.445182in}{4.064843in}}%
\pgfpathclose%
\pgfusepath{stroke,fill}%
\end{pgfscope}%
\begin{pgfscope}%
\pgfpathrectangle{\pgfqpoint{0.633874in}{2.920818in}}{\pgfqpoint{2.177280in}{2.201755in}}%
\pgfusepath{clip}%
\pgfsetbuttcap%
\pgfsetroundjoin%
\definecolor{currentfill}{rgb}{1.000000,0.498039,0.054902}%
\pgfsetfillcolor{currentfill}%
\pgfsetlinewidth{0.481800pt}%
\definecolor{currentstroke}{rgb}{1.000000,1.000000,1.000000}%
\pgfsetstrokecolor{currentstroke}%
\pgfsetdash{}{0pt}%
\pgfpathmoveto{\pgfqpoint{1.485121in}{4.064843in}}%
\pgfpathcurveto{\pgfqpoint{1.496171in}{4.064843in}}{\pgfqpoint{1.506770in}{4.069233in}}{\pgfqpoint{1.514584in}{4.077046in}}%
\pgfpathcurveto{\pgfqpoint{1.522397in}{4.084860in}}{\pgfqpoint{1.526788in}{4.095459in}}{\pgfqpoint{1.526788in}{4.106509in}}%
\pgfpathcurveto{\pgfqpoint{1.526788in}{4.117559in}}{\pgfqpoint{1.522397in}{4.128158in}}{\pgfqpoint{1.514584in}{4.135972in}}%
\pgfpathcurveto{\pgfqpoint{1.506770in}{4.143786in}}{\pgfqpoint{1.496171in}{4.148176in}}{\pgfqpoint{1.485121in}{4.148176in}}%
\pgfpathcurveto{\pgfqpoint{1.474071in}{4.148176in}}{\pgfqpoint{1.463472in}{4.143786in}}{\pgfqpoint{1.455658in}{4.135972in}}%
\pgfpathcurveto{\pgfqpoint{1.447845in}{4.128158in}}{\pgfqpoint{1.443454in}{4.117559in}}{\pgfqpoint{1.443454in}{4.106509in}}%
\pgfpathcurveto{\pgfqpoint{1.443454in}{4.095459in}}{\pgfqpoint{1.447845in}{4.084860in}}{\pgfqpoint{1.455658in}{4.077046in}}%
\pgfpathcurveto{\pgfqpoint{1.463472in}{4.069233in}}{\pgfqpoint{1.474071in}{4.064843in}}{\pgfqpoint{1.485121in}{4.064843in}}%
\pgfpathlineto{\pgfqpoint{1.485121in}{4.064843in}}%
\pgfpathclose%
\pgfusepath{stroke,fill}%
\end{pgfscope}%
\begin{pgfscope}%
\pgfpathrectangle{\pgfqpoint{0.633874in}{2.920818in}}{\pgfqpoint{2.177280in}{2.201755in}}%
\pgfusepath{clip}%
\pgfsetbuttcap%
\pgfsetroundjoin%
\definecolor{currentfill}{rgb}{1.000000,0.498039,0.054902}%
\pgfsetfillcolor{currentfill}%
\pgfsetlinewidth{0.481800pt}%
\definecolor{currentstroke}{rgb}{1.000000,1.000000,1.000000}%
\pgfsetstrokecolor{currentstroke}%
\pgfsetdash{}{0pt}%
\pgfpathmoveto{\pgfqpoint{1.485121in}{4.064843in}}%
\pgfpathcurveto{\pgfqpoint{1.496171in}{4.064843in}}{\pgfqpoint{1.506770in}{4.069233in}}{\pgfqpoint{1.514584in}{4.077046in}}%
\pgfpathcurveto{\pgfqpoint{1.522397in}{4.084860in}}{\pgfqpoint{1.526788in}{4.095459in}}{\pgfqpoint{1.526788in}{4.106509in}}%
\pgfpathcurveto{\pgfqpoint{1.526788in}{4.117559in}}{\pgfqpoint{1.522397in}{4.128158in}}{\pgfqpoint{1.514584in}{4.135972in}}%
\pgfpathcurveto{\pgfqpoint{1.506770in}{4.143786in}}{\pgfqpoint{1.496171in}{4.148176in}}{\pgfqpoint{1.485121in}{4.148176in}}%
\pgfpathcurveto{\pgfqpoint{1.474071in}{4.148176in}}{\pgfqpoint{1.463472in}{4.143786in}}{\pgfqpoint{1.455658in}{4.135972in}}%
\pgfpathcurveto{\pgfqpoint{1.447845in}{4.128158in}}{\pgfqpoint{1.443454in}{4.117559in}}{\pgfqpoint{1.443454in}{4.106509in}}%
\pgfpathcurveto{\pgfqpoint{1.443454in}{4.095459in}}{\pgfqpoint{1.447845in}{4.084860in}}{\pgfqpoint{1.455658in}{4.077046in}}%
\pgfpathcurveto{\pgfqpoint{1.463472in}{4.069233in}}{\pgfqpoint{1.474071in}{4.064843in}}{\pgfqpoint{1.485121in}{4.064843in}}%
\pgfpathlineto{\pgfqpoint{1.485121in}{4.064843in}}%
\pgfpathclose%
\pgfusepath{stroke,fill}%
\end{pgfscope}%
\begin{pgfscope}%
\pgfpathrectangle{\pgfqpoint{0.633874in}{2.920818in}}{\pgfqpoint{2.177280in}{2.201755in}}%
\pgfusepath{clip}%
\pgfsetbuttcap%
\pgfsetroundjoin%
\definecolor{currentfill}{rgb}{1.000000,0.498039,0.054902}%
\pgfsetfillcolor{currentfill}%
\pgfsetlinewidth{0.481800pt}%
\definecolor{currentstroke}{rgb}{1.000000,1.000000,1.000000}%
\pgfsetstrokecolor{currentstroke}%
\pgfsetdash{}{0pt}%
\pgfpathmoveto{\pgfqpoint{1.684815in}{4.098768in}}%
\pgfpathcurveto{\pgfqpoint{1.695865in}{4.098768in}}{\pgfqpoint{1.706464in}{4.103158in}}{\pgfqpoint{1.714277in}{4.110972in}}%
\pgfpathcurveto{\pgfqpoint{1.722091in}{4.118785in}}{\pgfqpoint{1.726481in}{4.129384in}}{\pgfqpoint{1.726481in}{4.140435in}}%
\pgfpathcurveto{\pgfqpoint{1.726481in}{4.151485in}}{\pgfqpoint{1.722091in}{4.162084in}}{\pgfqpoint{1.714277in}{4.169897in}}%
\pgfpathcurveto{\pgfqpoint{1.706464in}{4.177711in}}{\pgfqpoint{1.695865in}{4.182101in}}{\pgfqpoint{1.684815in}{4.182101in}}%
\pgfpathcurveto{\pgfqpoint{1.673764in}{4.182101in}}{\pgfqpoint{1.663165in}{4.177711in}}{\pgfqpoint{1.655352in}{4.169897in}}%
\pgfpathcurveto{\pgfqpoint{1.647538in}{4.162084in}}{\pgfqpoint{1.643148in}{4.151485in}}{\pgfqpoint{1.643148in}{4.140435in}}%
\pgfpathcurveto{\pgfqpoint{1.643148in}{4.129384in}}{\pgfqpoint{1.647538in}{4.118785in}}{\pgfqpoint{1.655352in}{4.110972in}}%
\pgfpathcurveto{\pgfqpoint{1.663165in}{4.103158in}}{\pgfqpoint{1.673764in}{4.098768in}}{\pgfqpoint{1.684815in}{4.098768in}}%
\pgfpathlineto{\pgfqpoint{1.684815in}{4.098768in}}%
\pgfpathclose%
\pgfusepath{stroke,fill}%
\end{pgfscope}%
\begin{pgfscope}%
\pgfpathrectangle{\pgfqpoint{0.633874in}{2.920818in}}{\pgfqpoint{2.177280in}{2.201755in}}%
\pgfusepath{clip}%
\pgfsetbuttcap%
\pgfsetroundjoin%
\definecolor{currentfill}{rgb}{1.000000,0.498039,0.054902}%
\pgfsetfillcolor{currentfill}%
\pgfsetlinewidth{0.481800pt}%
\definecolor{currentstroke}{rgb}{1.000000,1.000000,1.000000}%
\pgfsetstrokecolor{currentstroke}%
\pgfsetdash{}{0pt}%
\pgfpathmoveto{\pgfqpoint{1.245489in}{3.657738in}}%
\pgfpathcurveto{\pgfqpoint{1.256539in}{3.657738in}}{\pgfqpoint{1.267138in}{3.662129in}}{\pgfqpoint{1.274952in}{3.669942in}}%
\pgfpathcurveto{\pgfqpoint{1.282765in}{3.677756in}}{\pgfqpoint{1.287155in}{3.688355in}}{\pgfqpoint{1.287155in}{3.699405in}}%
\pgfpathcurveto{\pgfqpoint{1.287155in}{3.710455in}}{\pgfqpoint{1.282765in}{3.721054in}}{\pgfqpoint{1.274952in}{3.728868in}}%
\pgfpathcurveto{\pgfqpoint{1.267138in}{3.736681in}}{\pgfqpoint{1.256539in}{3.741072in}}{\pgfqpoint{1.245489in}{3.741072in}}%
\pgfpathcurveto{\pgfqpoint{1.234439in}{3.741072in}}{\pgfqpoint{1.223840in}{3.736681in}}{\pgfqpoint{1.216026in}{3.728868in}}%
\pgfpathcurveto{\pgfqpoint{1.208212in}{3.721054in}}{\pgfqpoint{1.203822in}{3.710455in}}{\pgfqpoint{1.203822in}{3.699405in}}%
\pgfpathcurveto{\pgfqpoint{1.203822in}{3.688355in}}{\pgfqpoint{1.208212in}{3.677756in}}{\pgfqpoint{1.216026in}{3.669942in}}%
\pgfpathcurveto{\pgfqpoint{1.223840in}{3.662129in}}{\pgfqpoint{1.234439in}{3.657738in}}{\pgfqpoint{1.245489in}{3.657738in}}%
\pgfpathlineto{\pgfqpoint{1.245489in}{3.657738in}}%
\pgfpathclose%
\pgfusepath{stroke,fill}%
\end{pgfscope}%
\begin{pgfscope}%
\pgfpathrectangle{\pgfqpoint{0.633874in}{2.920818in}}{\pgfqpoint{2.177280in}{2.201755in}}%
\pgfusepath{clip}%
\pgfsetbuttcap%
\pgfsetroundjoin%
\definecolor{currentfill}{rgb}{1.000000,0.498039,0.054902}%
\pgfsetfillcolor{currentfill}%
\pgfsetlinewidth{0.481800pt}%
\definecolor{currentstroke}{rgb}{1.000000,1.000000,1.000000}%
\pgfsetstrokecolor{currentstroke}%
\pgfsetdash{}{0pt}%
\pgfpathmoveto{\pgfqpoint{1.485121in}{4.030917in}}%
\pgfpathcurveto{\pgfqpoint{1.496171in}{4.030917in}}{\pgfqpoint{1.506770in}{4.035307in}}{\pgfqpoint{1.514584in}{4.043121in}}%
\pgfpathcurveto{\pgfqpoint{1.522397in}{4.050935in}}{\pgfqpoint{1.526788in}{4.061534in}}{\pgfqpoint{1.526788in}{4.072584in}}%
\pgfpathcurveto{\pgfqpoint{1.526788in}{4.083634in}}{\pgfqpoint{1.522397in}{4.094233in}}{\pgfqpoint{1.514584in}{4.102047in}}%
\pgfpathcurveto{\pgfqpoint{1.506770in}{4.109860in}}{\pgfqpoint{1.496171in}{4.114251in}}{\pgfqpoint{1.485121in}{4.114251in}}%
\pgfpathcurveto{\pgfqpoint{1.474071in}{4.114251in}}{\pgfqpoint{1.463472in}{4.109860in}}{\pgfqpoint{1.455658in}{4.102047in}}%
\pgfpathcurveto{\pgfqpoint{1.447845in}{4.094233in}}{\pgfqpoint{1.443454in}{4.083634in}}{\pgfqpoint{1.443454in}{4.072584in}}%
\pgfpathcurveto{\pgfqpoint{1.443454in}{4.061534in}}{\pgfqpoint{1.447845in}{4.050935in}}{\pgfqpoint{1.455658in}{4.043121in}}%
\pgfpathcurveto{\pgfqpoint{1.463472in}{4.035307in}}{\pgfqpoint{1.474071in}{4.030917in}}{\pgfqpoint{1.485121in}{4.030917in}}%
\pgfpathlineto{\pgfqpoint{1.485121in}{4.030917in}}%
\pgfpathclose%
\pgfusepath{stroke,fill}%
\end{pgfscope}%
\begin{pgfscope}%
\pgfpathrectangle{\pgfqpoint{0.633874in}{2.920818in}}{\pgfqpoint{2.177280in}{2.201755in}}%
\pgfusepath{clip}%
\pgfsetbuttcap%
\pgfsetroundjoin%
\definecolor{currentfill}{rgb}{0.172549,0.627451,0.172549}%
\pgfsetfillcolor{currentfill}%
\pgfsetlinewidth{0.481800pt}%
\definecolor{currentstroke}{rgb}{1.000000,1.000000,1.000000}%
\pgfsetstrokecolor{currentstroke}%
\pgfsetdash{}{0pt}%
\pgfpathmoveto{\pgfqpoint{1.724753in}{4.675499in}}%
\pgfpathcurveto{\pgfqpoint{1.735803in}{4.675499in}}{\pgfqpoint{1.746402in}{4.679889in}}{\pgfqpoint{1.754216in}{4.687703in}}%
\pgfpathcurveto{\pgfqpoint{1.762030in}{4.695516in}}{\pgfqpoint{1.766420in}{4.706115in}}{\pgfqpoint{1.766420in}{4.717165in}}%
\pgfpathcurveto{\pgfqpoint{1.766420in}{4.728216in}}{\pgfqpoint{1.762030in}{4.738815in}}{\pgfqpoint{1.754216in}{4.746628in}}%
\pgfpathcurveto{\pgfqpoint{1.746402in}{4.754442in}}{\pgfqpoint{1.735803in}{4.758832in}}{\pgfqpoint{1.724753in}{4.758832in}}%
\pgfpathcurveto{\pgfqpoint{1.713703in}{4.758832in}}{\pgfqpoint{1.703104in}{4.754442in}}{\pgfqpoint{1.695290in}{4.746628in}}%
\pgfpathcurveto{\pgfqpoint{1.687477in}{4.738815in}}{\pgfqpoint{1.683087in}{4.728216in}}{\pgfqpoint{1.683087in}{4.717165in}}%
\pgfpathcurveto{\pgfqpoint{1.683087in}{4.706115in}}{\pgfqpoint{1.687477in}{4.695516in}}{\pgfqpoint{1.695290in}{4.687703in}}%
\pgfpathcurveto{\pgfqpoint{1.703104in}{4.679889in}}{\pgfqpoint{1.713703in}{4.675499in}}{\pgfqpoint{1.724753in}{4.675499in}}%
\pgfpathlineto{\pgfqpoint{1.724753in}{4.675499in}}%
\pgfpathclose%
\pgfusepath{stroke,fill}%
\end{pgfscope}%
\begin{pgfscope}%
\pgfpathrectangle{\pgfqpoint{0.633874in}{2.920818in}}{\pgfqpoint{2.177280in}{2.201755in}}%
\pgfusepath{clip}%
\pgfsetbuttcap%
\pgfsetroundjoin%
\definecolor{currentfill}{rgb}{0.172549,0.627451,0.172549}%
\pgfsetfillcolor{currentfill}%
\pgfsetlinewidth{0.481800pt}%
\definecolor{currentstroke}{rgb}{1.000000,1.000000,1.000000}%
\pgfsetstrokecolor{currentstroke}%
\pgfsetdash{}{0pt}%
\pgfpathmoveto{\pgfqpoint{1.525060in}{4.370171in}}%
\pgfpathcurveto{\pgfqpoint{1.536110in}{4.370171in}}{\pgfqpoint{1.546709in}{4.374561in}}{\pgfqpoint{1.554523in}{4.382375in}}%
\pgfpathcurveto{\pgfqpoint{1.562336in}{4.390188in}}{\pgfqpoint{1.566726in}{4.400787in}}{\pgfqpoint{1.566726in}{4.411837in}}%
\pgfpathcurveto{\pgfqpoint{1.566726in}{4.422887in}}{\pgfqpoint{1.562336in}{4.433486in}}{\pgfqpoint{1.554523in}{4.441300in}}%
\pgfpathcurveto{\pgfqpoint{1.546709in}{4.449114in}}{\pgfqpoint{1.536110in}{4.453504in}}{\pgfqpoint{1.525060in}{4.453504in}}%
\pgfpathcurveto{\pgfqpoint{1.514010in}{4.453504in}}{\pgfqpoint{1.503411in}{4.449114in}}{\pgfqpoint{1.495597in}{4.441300in}}%
\pgfpathcurveto{\pgfqpoint{1.487783in}{4.433486in}}{\pgfqpoint{1.483393in}{4.422887in}}{\pgfqpoint{1.483393in}{4.411837in}}%
\pgfpathcurveto{\pgfqpoint{1.483393in}{4.400787in}}{\pgfqpoint{1.487783in}{4.390188in}}{\pgfqpoint{1.495597in}{4.382375in}}%
\pgfpathcurveto{\pgfqpoint{1.503411in}{4.374561in}}{\pgfqpoint{1.514010in}{4.370171in}}{\pgfqpoint{1.525060in}{4.370171in}}%
\pgfpathlineto{\pgfqpoint{1.525060in}{4.370171in}}%
\pgfpathclose%
\pgfusepath{stroke,fill}%
\end{pgfscope}%
\begin{pgfscope}%
\pgfpathrectangle{\pgfqpoint{0.633874in}{2.920818in}}{\pgfqpoint{2.177280in}{2.201755in}}%
\pgfusepath{clip}%
\pgfsetbuttcap%
\pgfsetroundjoin%
\definecolor{currentfill}{rgb}{0.172549,0.627451,0.172549}%
\pgfsetfillcolor{currentfill}%
\pgfsetlinewidth{0.481800pt}%
\definecolor{currentstroke}{rgb}{1.000000,1.000000,1.000000}%
\pgfsetstrokecolor{currentstroke}%
\pgfsetdash{}{0pt}%
\pgfpathmoveto{\pgfqpoint{2.044263in}{4.641573in}}%
\pgfpathcurveto{\pgfqpoint{2.055313in}{4.641573in}}{\pgfqpoint{2.065912in}{4.645964in}}{\pgfqpoint{2.073726in}{4.653777in}}%
\pgfpathcurveto{\pgfqpoint{2.081539in}{4.661591in}}{\pgfqpoint{2.085930in}{4.672190in}}{\pgfqpoint{2.085930in}{4.683240in}}%
\pgfpathcurveto{\pgfqpoint{2.085930in}{4.694290in}}{\pgfqpoint{2.081539in}{4.704889in}}{\pgfqpoint{2.073726in}{4.712703in}}%
\pgfpathcurveto{\pgfqpoint{2.065912in}{4.720517in}}{\pgfqpoint{2.055313in}{4.724907in}}{\pgfqpoint{2.044263in}{4.724907in}}%
\pgfpathcurveto{\pgfqpoint{2.033213in}{4.724907in}}{\pgfqpoint{2.022614in}{4.720517in}}{\pgfqpoint{2.014800in}{4.712703in}}%
\pgfpathcurveto{\pgfqpoint{2.006986in}{4.704889in}}{\pgfqpoint{2.002596in}{4.694290in}}{\pgfqpoint{2.002596in}{4.683240in}}%
\pgfpathcurveto{\pgfqpoint{2.002596in}{4.672190in}}{\pgfqpoint{2.006986in}{4.661591in}}{\pgfqpoint{2.014800in}{4.653777in}}%
\pgfpathcurveto{\pgfqpoint{2.022614in}{4.645964in}}{\pgfqpoint{2.033213in}{4.641573in}}{\pgfqpoint{2.044263in}{4.641573in}}%
\pgfpathlineto{\pgfqpoint{2.044263in}{4.641573in}}%
\pgfpathclose%
\pgfusepath{stroke,fill}%
\end{pgfscope}%
\begin{pgfscope}%
\pgfpathrectangle{\pgfqpoint{0.633874in}{2.920818in}}{\pgfqpoint{2.177280in}{2.201755in}}%
\pgfusepath{clip}%
\pgfsetbuttcap%
\pgfsetroundjoin%
\definecolor{currentfill}{rgb}{0.172549,0.627451,0.172549}%
\pgfsetfillcolor{currentfill}%
\pgfsetlinewidth{0.481800pt}%
\definecolor{currentstroke}{rgb}{1.000000,1.000000,1.000000}%
\pgfsetstrokecolor{currentstroke}%
\pgfsetdash{}{0pt}%
\pgfpathmoveto{\pgfqpoint{1.724753in}{4.539797in}}%
\pgfpathcurveto{\pgfqpoint{1.735803in}{4.539797in}}{\pgfqpoint{1.746402in}{4.544188in}}{\pgfqpoint{1.754216in}{4.552001in}}%
\pgfpathcurveto{\pgfqpoint{1.762030in}{4.559815in}}{\pgfqpoint{1.766420in}{4.570414in}}{\pgfqpoint{1.766420in}{4.581464in}}%
\pgfpathcurveto{\pgfqpoint{1.766420in}{4.592514in}}{\pgfqpoint{1.762030in}{4.603113in}}{\pgfqpoint{1.754216in}{4.610927in}}%
\pgfpathcurveto{\pgfqpoint{1.746402in}{4.618740in}}{\pgfqpoint{1.735803in}{4.623131in}}{\pgfqpoint{1.724753in}{4.623131in}}%
\pgfpathcurveto{\pgfqpoint{1.713703in}{4.623131in}}{\pgfqpoint{1.703104in}{4.618740in}}{\pgfqpoint{1.695290in}{4.610927in}}%
\pgfpathcurveto{\pgfqpoint{1.687477in}{4.603113in}}{\pgfqpoint{1.683087in}{4.592514in}}{\pgfqpoint{1.683087in}{4.581464in}}%
\pgfpathcurveto{\pgfqpoint{1.683087in}{4.570414in}}{\pgfqpoint{1.687477in}{4.559815in}}{\pgfqpoint{1.695290in}{4.552001in}}%
\pgfpathcurveto{\pgfqpoint{1.703104in}{4.544188in}}{\pgfqpoint{1.713703in}{4.539797in}}{\pgfqpoint{1.724753in}{4.539797in}}%
\pgfpathlineto{\pgfqpoint{1.724753in}{4.539797in}}%
\pgfpathclose%
\pgfusepath{stroke,fill}%
\end{pgfscope}%
\begin{pgfscope}%
\pgfpathrectangle{\pgfqpoint{0.633874in}{2.920818in}}{\pgfqpoint{2.177280in}{2.201755in}}%
\pgfusepath{clip}%
\pgfsetbuttcap%
\pgfsetroundjoin%
\definecolor{currentfill}{rgb}{0.172549,0.627451,0.172549}%
\pgfsetfillcolor{currentfill}%
\pgfsetlinewidth{0.481800pt}%
\definecolor{currentstroke}{rgb}{1.000000,1.000000,1.000000}%
\pgfsetstrokecolor{currentstroke}%
\pgfsetdash{}{0pt}%
\pgfpathmoveto{\pgfqpoint{1.804631in}{4.607648in}}%
\pgfpathcurveto{\pgfqpoint{1.815681in}{4.607648in}}{\pgfqpoint{1.826280in}{4.612038in}}{\pgfqpoint{1.834093in}{4.619852in}}%
\pgfpathcurveto{\pgfqpoint{1.841907in}{4.627666in}}{\pgfqpoint{1.846297in}{4.638265in}}{\pgfqpoint{1.846297in}{4.649315in}}%
\pgfpathcurveto{\pgfqpoint{1.846297in}{4.660365in}}{\pgfqpoint{1.841907in}{4.670964in}}{\pgfqpoint{1.834093in}{4.678778in}}%
\pgfpathcurveto{\pgfqpoint{1.826280in}{4.686591in}}{\pgfqpoint{1.815681in}{4.690981in}}{\pgfqpoint{1.804631in}{4.690981in}}%
\pgfpathcurveto{\pgfqpoint{1.793581in}{4.690981in}}{\pgfqpoint{1.782981in}{4.686591in}}{\pgfqpoint{1.775168in}{4.678778in}}%
\pgfpathcurveto{\pgfqpoint{1.767354in}{4.670964in}}{\pgfqpoint{1.762964in}{4.660365in}}{\pgfqpoint{1.762964in}{4.649315in}}%
\pgfpathcurveto{\pgfqpoint{1.762964in}{4.638265in}}{\pgfqpoint{1.767354in}{4.627666in}}{\pgfqpoint{1.775168in}{4.619852in}}%
\pgfpathcurveto{\pgfqpoint{1.782981in}{4.612038in}}{\pgfqpoint{1.793581in}{4.607648in}}{\pgfqpoint{1.804631in}{4.607648in}}%
\pgfpathlineto{\pgfqpoint{1.804631in}{4.607648in}}%
\pgfpathclose%
\pgfusepath{stroke,fill}%
\end{pgfscope}%
\begin{pgfscope}%
\pgfpathrectangle{\pgfqpoint{0.633874in}{2.920818in}}{\pgfqpoint{2.177280in}{2.201755in}}%
\pgfusepath{clip}%
\pgfsetbuttcap%
\pgfsetroundjoin%
\definecolor{currentfill}{rgb}{0.172549,0.627451,0.172549}%
\pgfsetfillcolor{currentfill}%
\pgfsetlinewidth{0.481800pt}%
\definecolor{currentstroke}{rgb}{1.000000,1.000000,1.000000}%
\pgfsetstrokecolor{currentstroke}%
\pgfsetdash{}{0pt}%
\pgfpathmoveto{\pgfqpoint{2.243956in}{4.879051in}}%
\pgfpathcurveto{\pgfqpoint{2.255007in}{4.879051in}}{\pgfqpoint{2.265606in}{4.883441in}}{\pgfqpoint{2.273419in}{4.891255in}}%
\pgfpathcurveto{\pgfqpoint{2.281233in}{4.899068in}}{\pgfqpoint{2.285623in}{4.909667in}}{\pgfqpoint{2.285623in}{4.920718in}}%
\pgfpathcurveto{\pgfqpoint{2.285623in}{4.931768in}}{\pgfqpoint{2.281233in}{4.942367in}}{\pgfqpoint{2.273419in}{4.950180in}}%
\pgfpathcurveto{\pgfqpoint{2.265606in}{4.957994in}}{\pgfqpoint{2.255007in}{4.962384in}}{\pgfqpoint{2.243956in}{4.962384in}}%
\pgfpathcurveto{\pgfqpoint{2.232906in}{4.962384in}}{\pgfqpoint{2.222307in}{4.957994in}}{\pgfqpoint{2.214494in}{4.950180in}}%
\pgfpathcurveto{\pgfqpoint{2.206680in}{4.942367in}}{\pgfqpoint{2.202290in}{4.931768in}}{\pgfqpoint{2.202290in}{4.920718in}}%
\pgfpathcurveto{\pgfqpoint{2.202290in}{4.909667in}}{\pgfqpoint{2.206680in}{4.899068in}}{\pgfqpoint{2.214494in}{4.891255in}}%
\pgfpathcurveto{\pgfqpoint{2.222307in}{4.883441in}}{\pgfqpoint{2.232906in}{4.879051in}}{\pgfqpoint{2.243956in}{4.879051in}}%
\pgfpathlineto{\pgfqpoint{2.243956in}{4.879051in}}%
\pgfpathclose%
\pgfusepath{stroke,fill}%
\end{pgfscope}%
\begin{pgfscope}%
\pgfpathrectangle{\pgfqpoint{0.633874in}{2.920818in}}{\pgfqpoint{2.177280in}{2.201755in}}%
\pgfusepath{clip}%
\pgfsetbuttcap%
\pgfsetroundjoin%
\definecolor{currentfill}{rgb}{0.172549,0.627451,0.172549}%
\pgfsetfillcolor{currentfill}%
\pgfsetlinewidth{0.481800pt}%
\definecolor{currentstroke}{rgb}{1.000000,1.000000,1.000000}%
\pgfsetstrokecolor{currentstroke}%
\pgfsetdash{}{0pt}%
\pgfpathmoveto{\pgfqpoint{1.165611in}{4.166619in}}%
\pgfpathcurveto{\pgfqpoint{1.176662in}{4.166619in}}{\pgfqpoint{1.187261in}{4.171009in}}{\pgfqpoint{1.195074in}{4.178822in}}%
\pgfpathcurveto{\pgfqpoint{1.202888in}{4.186636in}}{\pgfqpoint{1.207278in}{4.197235in}}{\pgfqpoint{1.207278in}{4.208285in}}%
\pgfpathcurveto{\pgfqpoint{1.207278in}{4.219335in}}{\pgfqpoint{1.202888in}{4.229934in}}{\pgfqpoint{1.195074in}{4.237748in}}%
\pgfpathcurveto{\pgfqpoint{1.187261in}{4.245562in}}{\pgfqpoint{1.176662in}{4.249952in}}{\pgfqpoint{1.165611in}{4.249952in}}%
\pgfpathcurveto{\pgfqpoint{1.154561in}{4.249952in}}{\pgfqpoint{1.143962in}{4.245562in}}{\pgfqpoint{1.136149in}{4.237748in}}%
\pgfpathcurveto{\pgfqpoint{1.128335in}{4.229934in}}{\pgfqpoint{1.123945in}{4.219335in}}{\pgfqpoint{1.123945in}{4.208285in}}%
\pgfpathcurveto{\pgfqpoint{1.123945in}{4.197235in}}{\pgfqpoint{1.128335in}{4.186636in}}{\pgfqpoint{1.136149in}{4.178822in}}%
\pgfpathcurveto{\pgfqpoint{1.143962in}{4.171009in}}{\pgfqpoint{1.154561in}{4.166619in}}{\pgfqpoint{1.165611in}{4.166619in}}%
\pgfpathlineto{\pgfqpoint{1.165611in}{4.166619in}}%
\pgfpathclose%
\pgfusepath{stroke,fill}%
\end{pgfscope}%
\begin{pgfscope}%
\pgfpathrectangle{\pgfqpoint{0.633874in}{2.920818in}}{\pgfqpoint{2.177280in}{2.201755in}}%
\pgfusepath{clip}%
\pgfsetbuttcap%
\pgfsetroundjoin%
\definecolor{currentfill}{rgb}{0.172549,0.627451,0.172549}%
\pgfsetfillcolor{currentfill}%
\pgfsetlinewidth{0.481800pt}%
\definecolor{currentstroke}{rgb}{1.000000,1.000000,1.000000}%
\pgfsetstrokecolor{currentstroke}%
\pgfsetdash{}{0pt}%
\pgfpathmoveto{\pgfqpoint{2.124140in}{4.777275in}}%
\pgfpathcurveto{\pgfqpoint{2.135190in}{4.777275in}}{\pgfqpoint{2.145789in}{4.781665in}}{\pgfqpoint{2.153603in}{4.789479in}}%
\pgfpathcurveto{\pgfqpoint{2.161417in}{4.797292in}}{\pgfqpoint{2.165807in}{4.807891in}}{\pgfqpoint{2.165807in}{4.818942in}}%
\pgfpathcurveto{\pgfqpoint{2.165807in}{4.829992in}}{\pgfqpoint{2.161417in}{4.840591in}}{\pgfqpoint{2.153603in}{4.848404in}}%
\pgfpathcurveto{\pgfqpoint{2.145789in}{4.856218in}}{\pgfqpoint{2.135190in}{4.860608in}}{\pgfqpoint{2.124140in}{4.860608in}}%
\pgfpathcurveto{\pgfqpoint{2.113090in}{4.860608in}}{\pgfqpoint{2.102491in}{4.856218in}}{\pgfqpoint{2.094677in}{4.848404in}}%
\pgfpathcurveto{\pgfqpoint{2.086864in}{4.840591in}}{\pgfqpoint{2.082474in}{4.829992in}}{\pgfqpoint{2.082474in}{4.818942in}}%
\pgfpathcurveto{\pgfqpoint{2.082474in}{4.807891in}}{\pgfqpoint{2.086864in}{4.797292in}}{\pgfqpoint{2.094677in}{4.789479in}}%
\pgfpathcurveto{\pgfqpoint{2.102491in}{4.781665in}}{\pgfqpoint{2.113090in}{4.777275in}}{\pgfqpoint{2.124140in}{4.777275in}}%
\pgfpathlineto{\pgfqpoint{2.124140in}{4.777275in}}%
\pgfpathclose%
\pgfusepath{stroke,fill}%
\end{pgfscope}%
\begin{pgfscope}%
\pgfpathrectangle{\pgfqpoint{0.633874in}{2.920818in}}{\pgfqpoint{2.177280in}{2.201755in}}%
\pgfusepath{clip}%
\pgfsetbuttcap%
\pgfsetroundjoin%
\definecolor{currentfill}{rgb}{0.172549,0.627451,0.172549}%
\pgfsetfillcolor{currentfill}%
\pgfsetlinewidth{0.481800pt}%
\definecolor{currentstroke}{rgb}{1.000000,1.000000,1.000000}%
\pgfsetstrokecolor{currentstroke}%
\pgfsetdash{}{0pt}%
\pgfpathmoveto{\pgfqpoint{1.884508in}{4.607648in}}%
\pgfpathcurveto{\pgfqpoint{1.895558in}{4.607648in}}{\pgfqpoint{1.906157in}{4.612038in}}{\pgfqpoint{1.913971in}{4.619852in}}%
\pgfpathcurveto{\pgfqpoint{1.921784in}{4.627666in}}{\pgfqpoint{1.926175in}{4.638265in}}{\pgfqpoint{1.926175in}{4.649315in}}%
\pgfpathcurveto{\pgfqpoint{1.926175in}{4.660365in}}{\pgfqpoint{1.921784in}{4.670964in}}{\pgfqpoint{1.913971in}{4.678778in}}%
\pgfpathcurveto{\pgfqpoint{1.906157in}{4.686591in}}{\pgfqpoint{1.895558in}{4.690981in}}{\pgfqpoint{1.884508in}{4.690981in}}%
\pgfpathcurveto{\pgfqpoint{1.873458in}{4.690981in}}{\pgfqpoint{1.862859in}{4.686591in}}{\pgfqpoint{1.855045in}{4.678778in}}%
\pgfpathcurveto{\pgfqpoint{1.847232in}{4.670964in}}{\pgfqpoint{1.842841in}{4.660365in}}{\pgfqpoint{1.842841in}{4.649315in}}%
\pgfpathcurveto{\pgfqpoint{1.842841in}{4.638265in}}{\pgfqpoint{1.847232in}{4.627666in}}{\pgfqpoint{1.855045in}{4.619852in}}%
\pgfpathcurveto{\pgfqpoint{1.862859in}{4.612038in}}{\pgfqpoint{1.873458in}{4.607648in}}{\pgfqpoint{1.884508in}{4.607648in}}%
\pgfpathlineto{\pgfqpoint{1.884508in}{4.607648in}}%
\pgfpathclose%
\pgfusepath{stroke,fill}%
\end{pgfscope}%
\begin{pgfscope}%
\pgfpathrectangle{\pgfqpoint{0.633874in}{2.920818in}}{\pgfqpoint{2.177280in}{2.201755in}}%
\pgfusepath{clip}%
\pgfsetbuttcap%
\pgfsetroundjoin%
\definecolor{currentfill}{rgb}{0.172549,0.627451,0.172549}%
\pgfsetfillcolor{currentfill}%
\pgfsetlinewidth{0.481800pt}%
\definecolor{currentstroke}{rgb}{1.000000,1.000000,1.000000}%
\pgfsetstrokecolor{currentstroke}%
\pgfsetdash{}{0pt}%
\pgfpathmoveto{\pgfqpoint{2.084202in}{4.709424in}}%
\pgfpathcurveto{\pgfqpoint{2.095252in}{4.709424in}}{\pgfqpoint{2.105851in}{4.713814in}}{\pgfqpoint{2.113664in}{4.721628in}}%
\pgfpathcurveto{\pgfqpoint{2.121478in}{4.729442in}}{\pgfqpoint{2.125868in}{4.740041in}}{\pgfqpoint{2.125868in}{4.751091in}}%
\pgfpathcurveto{\pgfqpoint{2.125868in}{4.762141in}}{\pgfqpoint{2.121478in}{4.772740in}}{\pgfqpoint{2.113664in}{4.780554in}}%
\pgfpathcurveto{\pgfqpoint{2.105851in}{4.788367in}}{\pgfqpoint{2.095252in}{4.792757in}}{\pgfqpoint{2.084202in}{4.792757in}}%
\pgfpathcurveto{\pgfqpoint{2.073151in}{4.792757in}}{\pgfqpoint{2.062552in}{4.788367in}}{\pgfqpoint{2.054739in}{4.780554in}}%
\pgfpathcurveto{\pgfqpoint{2.046925in}{4.772740in}}{\pgfqpoint{2.042535in}{4.762141in}}{\pgfqpoint{2.042535in}{4.751091in}}%
\pgfpathcurveto{\pgfqpoint{2.042535in}{4.740041in}}{\pgfqpoint{2.046925in}{4.729442in}}{\pgfqpoint{2.054739in}{4.721628in}}%
\pgfpathcurveto{\pgfqpoint{2.062552in}{4.713814in}}{\pgfqpoint{2.073151in}{4.709424in}}{\pgfqpoint{2.084202in}{4.709424in}}%
\pgfpathlineto{\pgfqpoint{2.084202in}{4.709424in}}%
\pgfpathclose%
\pgfusepath{stroke,fill}%
\end{pgfscope}%
\begin{pgfscope}%
\pgfpathrectangle{\pgfqpoint{0.633874in}{2.920818in}}{\pgfqpoint{2.177280in}{2.201755in}}%
\pgfusepath{clip}%
\pgfsetbuttcap%
\pgfsetroundjoin%
\definecolor{currentfill}{rgb}{0.172549,0.627451,0.172549}%
\pgfsetfillcolor{currentfill}%
\pgfsetlinewidth{0.481800pt}%
\definecolor{currentstroke}{rgb}{1.000000,1.000000,1.000000}%
\pgfsetstrokecolor{currentstroke}%
\pgfsetdash{}{0pt}%
\pgfpathmoveto{\pgfqpoint{1.804631in}{4.370171in}}%
\pgfpathcurveto{\pgfqpoint{1.815681in}{4.370171in}}{\pgfqpoint{1.826280in}{4.374561in}}{\pgfqpoint{1.834093in}{4.382375in}}%
\pgfpathcurveto{\pgfqpoint{1.841907in}{4.390188in}}{\pgfqpoint{1.846297in}{4.400787in}}{\pgfqpoint{1.846297in}{4.411837in}}%
\pgfpathcurveto{\pgfqpoint{1.846297in}{4.422887in}}{\pgfqpoint{1.841907in}{4.433486in}}{\pgfqpoint{1.834093in}{4.441300in}}%
\pgfpathcurveto{\pgfqpoint{1.826280in}{4.449114in}}{\pgfqpoint{1.815681in}{4.453504in}}{\pgfqpoint{1.804631in}{4.453504in}}%
\pgfpathcurveto{\pgfqpoint{1.793581in}{4.453504in}}{\pgfqpoint{1.782981in}{4.449114in}}{\pgfqpoint{1.775168in}{4.441300in}}%
\pgfpathcurveto{\pgfqpoint{1.767354in}{4.433486in}}{\pgfqpoint{1.762964in}{4.422887in}}{\pgfqpoint{1.762964in}{4.411837in}}%
\pgfpathcurveto{\pgfqpoint{1.762964in}{4.400787in}}{\pgfqpoint{1.767354in}{4.390188in}}{\pgfqpoint{1.775168in}{4.382375in}}%
\pgfpathcurveto{\pgfqpoint{1.782981in}{4.374561in}}{\pgfqpoint{1.793581in}{4.370171in}}{\pgfqpoint{1.804631in}{4.370171in}}%
\pgfpathlineto{\pgfqpoint{1.804631in}{4.370171in}}%
\pgfpathclose%
\pgfusepath{stroke,fill}%
\end{pgfscope}%
\begin{pgfscope}%
\pgfpathrectangle{\pgfqpoint{0.633874in}{2.920818in}}{\pgfqpoint{2.177280in}{2.201755in}}%
\pgfusepath{clip}%
\pgfsetbuttcap%
\pgfsetroundjoin%
\definecolor{currentfill}{rgb}{0.172549,0.627451,0.172549}%
\pgfsetfillcolor{currentfill}%
\pgfsetlinewidth{0.481800pt}%
\definecolor{currentstroke}{rgb}{1.000000,1.000000,1.000000}%
\pgfsetstrokecolor{currentstroke}%
\pgfsetdash{}{0pt}%
\pgfpathmoveto{\pgfqpoint{1.764692in}{4.438021in}}%
\pgfpathcurveto{\pgfqpoint{1.775742in}{4.438021in}}{\pgfqpoint{1.786341in}{4.442412in}}{\pgfqpoint{1.794155in}{4.450225in}}%
\pgfpathcurveto{\pgfqpoint{1.801968in}{4.458039in}}{\pgfqpoint{1.806359in}{4.468638in}}{\pgfqpoint{1.806359in}{4.479688in}}%
\pgfpathcurveto{\pgfqpoint{1.806359in}{4.490738in}}{\pgfqpoint{1.801968in}{4.501337in}}{\pgfqpoint{1.794155in}{4.509151in}}%
\pgfpathcurveto{\pgfqpoint{1.786341in}{4.516964in}}{\pgfqpoint{1.775742in}{4.521355in}}{\pgfqpoint{1.764692in}{4.521355in}}%
\pgfpathcurveto{\pgfqpoint{1.753642in}{4.521355in}}{\pgfqpoint{1.743043in}{4.516964in}}{\pgfqpoint{1.735229in}{4.509151in}}%
\pgfpathcurveto{\pgfqpoint{1.727416in}{4.501337in}}{\pgfqpoint{1.723025in}{4.490738in}}{\pgfqpoint{1.723025in}{4.479688in}}%
\pgfpathcurveto{\pgfqpoint{1.723025in}{4.468638in}}{\pgfqpoint{1.727416in}{4.458039in}}{\pgfqpoint{1.735229in}{4.450225in}}%
\pgfpathcurveto{\pgfqpoint{1.743043in}{4.442412in}}{\pgfqpoint{1.753642in}{4.438021in}}{\pgfqpoint{1.764692in}{4.438021in}}%
\pgfpathlineto{\pgfqpoint{1.764692in}{4.438021in}}%
\pgfpathclose%
\pgfusepath{stroke,fill}%
\end{pgfscope}%
\begin{pgfscope}%
\pgfpathrectangle{\pgfqpoint{0.633874in}{2.920818in}}{\pgfqpoint{2.177280in}{2.201755in}}%
\pgfusepath{clip}%
\pgfsetbuttcap%
\pgfsetroundjoin%
\definecolor{currentfill}{rgb}{0.172549,0.627451,0.172549}%
\pgfsetfillcolor{currentfill}%
\pgfsetlinewidth{0.481800pt}%
\definecolor{currentstroke}{rgb}{1.000000,1.000000,1.000000}%
\pgfsetstrokecolor{currentstroke}%
\pgfsetdash{}{0pt}%
\pgfpathmoveto{\pgfqpoint{1.924447in}{4.505872in}}%
\pgfpathcurveto{\pgfqpoint{1.935497in}{4.505872in}}{\pgfqpoint{1.946096in}{4.510262in}}{\pgfqpoint{1.953910in}{4.518076in}}%
\pgfpathcurveto{\pgfqpoint{1.961723in}{4.525890in}}{\pgfqpoint{1.966113in}{4.536489in}}{\pgfqpoint{1.966113in}{4.547539in}}%
\pgfpathcurveto{\pgfqpoint{1.966113in}{4.558589in}}{\pgfqpoint{1.961723in}{4.569188in}}{\pgfqpoint{1.953910in}{4.577002in}}%
\pgfpathcurveto{\pgfqpoint{1.946096in}{4.584815in}}{\pgfqpoint{1.935497in}{4.589205in}}{\pgfqpoint{1.924447in}{4.589205in}}%
\pgfpathcurveto{\pgfqpoint{1.913397in}{4.589205in}}{\pgfqpoint{1.902798in}{4.584815in}}{\pgfqpoint{1.894984in}{4.577002in}}%
\pgfpathcurveto{\pgfqpoint{1.887170in}{4.569188in}}{\pgfqpoint{1.882780in}{4.558589in}}{\pgfqpoint{1.882780in}{4.547539in}}%
\pgfpathcurveto{\pgfqpoint{1.882780in}{4.536489in}}{\pgfqpoint{1.887170in}{4.525890in}}{\pgfqpoint{1.894984in}{4.518076in}}%
\pgfpathcurveto{\pgfqpoint{1.902798in}{4.510262in}}{\pgfqpoint{1.913397in}{4.505872in}}{\pgfqpoint{1.924447in}{4.505872in}}%
\pgfpathlineto{\pgfqpoint{1.924447in}{4.505872in}}%
\pgfpathclose%
\pgfusepath{stroke,fill}%
\end{pgfscope}%
\begin{pgfscope}%
\pgfpathrectangle{\pgfqpoint{0.633874in}{2.920818in}}{\pgfqpoint{2.177280in}{2.201755in}}%
\pgfusepath{clip}%
\pgfsetbuttcap%
\pgfsetroundjoin%
\definecolor{currentfill}{rgb}{0.172549,0.627451,0.172549}%
\pgfsetfillcolor{currentfill}%
\pgfsetlinewidth{0.481800pt}%
\definecolor{currentstroke}{rgb}{1.000000,1.000000,1.000000}%
\pgfsetstrokecolor{currentstroke}%
\pgfsetdash{}{0pt}%
\pgfpathmoveto{\pgfqpoint{1.485121in}{4.336245in}}%
\pgfpathcurveto{\pgfqpoint{1.496171in}{4.336245in}}{\pgfqpoint{1.506770in}{4.340636in}}{\pgfqpoint{1.514584in}{4.348449in}}%
\pgfpathcurveto{\pgfqpoint{1.522397in}{4.356263in}}{\pgfqpoint{1.526788in}{4.366862in}}{\pgfqpoint{1.526788in}{4.377912in}}%
\pgfpathcurveto{\pgfqpoint{1.526788in}{4.388962in}}{\pgfqpoint{1.522397in}{4.399561in}}{\pgfqpoint{1.514584in}{4.407375in}}%
\pgfpathcurveto{\pgfqpoint{1.506770in}{4.415188in}}{\pgfqpoint{1.496171in}{4.419579in}}{\pgfqpoint{1.485121in}{4.419579in}}%
\pgfpathcurveto{\pgfqpoint{1.474071in}{4.419579in}}{\pgfqpoint{1.463472in}{4.415188in}}{\pgfqpoint{1.455658in}{4.407375in}}%
\pgfpathcurveto{\pgfqpoint{1.447845in}{4.399561in}}{\pgfqpoint{1.443454in}{4.388962in}}{\pgfqpoint{1.443454in}{4.377912in}}%
\pgfpathcurveto{\pgfqpoint{1.443454in}{4.366862in}}{\pgfqpoint{1.447845in}{4.356263in}}{\pgfqpoint{1.455658in}{4.348449in}}%
\pgfpathcurveto{\pgfqpoint{1.463472in}{4.340636in}}{\pgfqpoint{1.474071in}{4.336245in}}{\pgfqpoint{1.485121in}{4.336245in}}%
\pgfpathlineto{\pgfqpoint{1.485121in}{4.336245in}}%
\pgfpathclose%
\pgfusepath{stroke,fill}%
\end{pgfscope}%
\begin{pgfscope}%
\pgfpathrectangle{\pgfqpoint{0.633874in}{2.920818in}}{\pgfqpoint{2.177280in}{2.201755in}}%
\pgfusepath{clip}%
\pgfsetbuttcap%
\pgfsetroundjoin%
\definecolor{currentfill}{rgb}{0.172549,0.627451,0.172549}%
\pgfsetfillcolor{currentfill}%
\pgfsetlinewidth{0.481800pt}%
\definecolor{currentstroke}{rgb}{1.000000,1.000000,1.000000}%
\pgfsetstrokecolor{currentstroke}%
\pgfsetdash{}{0pt}%
\pgfpathmoveto{\pgfqpoint{1.525060in}{4.370171in}}%
\pgfpathcurveto{\pgfqpoint{1.536110in}{4.370171in}}{\pgfqpoint{1.546709in}{4.374561in}}{\pgfqpoint{1.554523in}{4.382375in}}%
\pgfpathcurveto{\pgfqpoint{1.562336in}{4.390188in}}{\pgfqpoint{1.566726in}{4.400787in}}{\pgfqpoint{1.566726in}{4.411837in}}%
\pgfpathcurveto{\pgfqpoint{1.566726in}{4.422887in}}{\pgfqpoint{1.562336in}{4.433486in}}{\pgfqpoint{1.554523in}{4.441300in}}%
\pgfpathcurveto{\pgfqpoint{1.546709in}{4.449114in}}{\pgfqpoint{1.536110in}{4.453504in}}{\pgfqpoint{1.525060in}{4.453504in}}%
\pgfpathcurveto{\pgfqpoint{1.514010in}{4.453504in}}{\pgfqpoint{1.503411in}{4.449114in}}{\pgfqpoint{1.495597in}{4.441300in}}%
\pgfpathcurveto{\pgfqpoint{1.487783in}{4.433486in}}{\pgfqpoint{1.483393in}{4.422887in}}{\pgfqpoint{1.483393in}{4.411837in}}%
\pgfpathcurveto{\pgfqpoint{1.483393in}{4.400787in}}{\pgfqpoint{1.487783in}{4.390188in}}{\pgfqpoint{1.495597in}{4.382375in}}%
\pgfpathcurveto{\pgfqpoint{1.503411in}{4.374561in}}{\pgfqpoint{1.514010in}{4.370171in}}{\pgfqpoint{1.525060in}{4.370171in}}%
\pgfpathlineto{\pgfqpoint{1.525060in}{4.370171in}}%
\pgfpathclose%
\pgfusepath{stroke,fill}%
\end{pgfscope}%
\begin{pgfscope}%
\pgfpathrectangle{\pgfqpoint{0.633874in}{2.920818in}}{\pgfqpoint{2.177280in}{2.201755in}}%
\pgfusepath{clip}%
\pgfsetbuttcap%
\pgfsetroundjoin%
\definecolor{currentfill}{rgb}{0.172549,0.627451,0.172549}%
\pgfsetfillcolor{currentfill}%
\pgfsetlinewidth{0.481800pt}%
\definecolor{currentstroke}{rgb}{1.000000,1.000000,1.000000}%
\pgfsetstrokecolor{currentstroke}%
\pgfsetdash{}{0pt}%
\pgfpathmoveto{\pgfqpoint{1.764692in}{4.438021in}}%
\pgfpathcurveto{\pgfqpoint{1.775742in}{4.438021in}}{\pgfqpoint{1.786341in}{4.442412in}}{\pgfqpoint{1.794155in}{4.450225in}}%
\pgfpathcurveto{\pgfqpoint{1.801968in}{4.458039in}}{\pgfqpoint{1.806359in}{4.468638in}}{\pgfqpoint{1.806359in}{4.479688in}}%
\pgfpathcurveto{\pgfqpoint{1.806359in}{4.490738in}}{\pgfqpoint{1.801968in}{4.501337in}}{\pgfqpoint{1.794155in}{4.509151in}}%
\pgfpathcurveto{\pgfqpoint{1.786341in}{4.516964in}}{\pgfqpoint{1.775742in}{4.521355in}}{\pgfqpoint{1.764692in}{4.521355in}}%
\pgfpathcurveto{\pgfqpoint{1.753642in}{4.521355in}}{\pgfqpoint{1.743043in}{4.516964in}}{\pgfqpoint{1.735229in}{4.509151in}}%
\pgfpathcurveto{\pgfqpoint{1.727416in}{4.501337in}}{\pgfqpoint{1.723025in}{4.490738in}}{\pgfqpoint{1.723025in}{4.479688in}}%
\pgfpathcurveto{\pgfqpoint{1.723025in}{4.468638in}}{\pgfqpoint{1.727416in}{4.458039in}}{\pgfqpoint{1.735229in}{4.450225in}}%
\pgfpathcurveto{\pgfqpoint{1.743043in}{4.442412in}}{\pgfqpoint{1.753642in}{4.438021in}}{\pgfqpoint{1.764692in}{4.438021in}}%
\pgfpathlineto{\pgfqpoint{1.764692in}{4.438021in}}%
\pgfpathclose%
\pgfusepath{stroke,fill}%
\end{pgfscope}%
\begin{pgfscope}%
\pgfpathrectangle{\pgfqpoint{0.633874in}{2.920818in}}{\pgfqpoint{2.177280in}{2.201755in}}%
\pgfusepath{clip}%
\pgfsetbuttcap%
\pgfsetroundjoin%
\definecolor{currentfill}{rgb}{0.172549,0.627451,0.172549}%
\pgfsetfillcolor{currentfill}%
\pgfsetlinewidth{0.481800pt}%
\definecolor{currentstroke}{rgb}{1.000000,1.000000,1.000000}%
\pgfsetstrokecolor{currentstroke}%
\pgfsetdash{}{0pt}%
\pgfpathmoveto{\pgfqpoint{1.804631in}{4.505872in}}%
\pgfpathcurveto{\pgfqpoint{1.815681in}{4.505872in}}{\pgfqpoint{1.826280in}{4.510262in}}{\pgfqpoint{1.834093in}{4.518076in}}%
\pgfpathcurveto{\pgfqpoint{1.841907in}{4.525890in}}{\pgfqpoint{1.846297in}{4.536489in}}{\pgfqpoint{1.846297in}{4.547539in}}%
\pgfpathcurveto{\pgfqpoint{1.846297in}{4.558589in}}{\pgfqpoint{1.841907in}{4.569188in}}{\pgfqpoint{1.834093in}{4.577002in}}%
\pgfpathcurveto{\pgfqpoint{1.826280in}{4.584815in}}{\pgfqpoint{1.815681in}{4.589205in}}{\pgfqpoint{1.804631in}{4.589205in}}%
\pgfpathcurveto{\pgfqpoint{1.793581in}{4.589205in}}{\pgfqpoint{1.782981in}{4.584815in}}{\pgfqpoint{1.775168in}{4.577002in}}%
\pgfpathcurveto{\pgfqpoint{1.767354in}{4.569188in}}{\pgfqpoint{1.762964in}{4.558589in}}{\pgfqpoint{1.762964in}{4.547539in}}%
\pgfpathcurveto{\pgfqpoint{1.762964in}{4.536489in}}{\pgfqpoint{1.767354in}{4.525890in}}{\pgfqpoint{1.775168in}{4.518076in}}%
\pgfpathcurveto{\pgfqpoint{1.782981in}{4.510262in}}{\pgfqpoint{1.793581in}{4.505872in}}{\pgfqpoint{1.804631in}{4.505872in}}%
\pgfpathlineto{\pgfqpoint{1.804631in}{4.505872in}}%
\pgfpathclose%
\pgfusepath{stroke,fill}%
\end{pgfscope}%
\begin{pgfscope}%
\pgfpathrectangle{\pgfqpoint{0.633874in}{2.920818in}}{\pgfqpoint{2.177280in}{2.201755in}}%
\pgfusepath{clip}%
\pgfsetbuttcap%
\pgfsetroundjoin%
\definecolor{currentfill}{rgb}{0.172549,0.627451,0.172549}%
\pgfsetfillcolor{currentfill}%
\pgfsetlinewidth{0.481800pt}%
\definecolor{currentstroke}{rgb}{1.000000,1.000000,1.000000}%
\pgfsetstrokecolor{currentstroke}%
\pgfsetdash{}{0pt}%
\pgfpathmoveto{\pgfqpoint{2.283895in}{4.912976in}}%
\pgfpathcurveto{\pgfqpoint{2.294945in}{4.912976in}}{\pgfqpoint{2.305544in}{4.917366in}}{\pgfqpoint{2.313358in}{4.925180in}}%
\pgfpathcurveto{\pgfqpoint{2.321171in}{4.932994in}}{\pgfqpoint{2.325562in}{4.943593in}}{\pgfqpoint{2.325562in}{4.954643in}}%
\pgfpathcurveto{\pgfqpoint{2.325562in}{4.965693in}}{\pgfqpoint{2.321171in}{4.976292in}}{\pgfqpoint{2.313358in}{4.984106in}}%
\pgfpathcurveto{\pgfqpoint{2.305544in}{4.991919in}}{\pgfqpoint{2.294945in}{4.996310in}}{\pgfqpoint{2.283895in}{4.996310in}}%
\pgfpathcurveto{\pgfqpoint{2.272845in}{4.996310in}}{\pgfqpoint{2.262246in}{4.991919in}}{\pgfqpoint{2.254432in}{4.984106in}}%
\pgfpathcurveto{\pgfqpoint{2.246619in}{4.976292in}}{\pgfqpoint{2.242228in}{4.965693in}}{\pgfqpoint{2.242228in}{4.954643in}}%
\pgfpathcurveto{\pgfqpoint{2.242228in}{4.943593in}}{\pgfqpoint{2.246619in}{4.932994in}}{\pgfqpoint{2.254432in}{4.925180in}}%
\pgfpathcurveto{\pgfqpoint{2.262246in}{4.917366in}}{\pgfqpoint{2.272845in}{4.912976in}}{\pgfqpoint{2.283895in}{4.912976in}}%
\pgfpathlineto{\pgfqpoint{2.283895in}{4.912976in}}%
\pgfpathclose%
\pgfusepath{stroke,fill}%
\end{pgfscope}%
\begin{pgfscope}%
\pgfpathrectangle{\pgfqpoint{0.633874in}{2.920818in}}{\pgfqpoint{2.177280in}{2.201755in}}%
\pgfusepath{clip}%
\pgfsetbuttcap%
\pgfsetroundjoin%
\definecolor{currentfill}{rgb}{0.172549,0.627451,0.172549}%
\pgfsetfillcolor{currentfill}%
\pgfsetlinewidth{0.481800pt}%
\definecolor{currentstroke}{rgb}{1.000000,1.000000,1.000000}%
\pgfsetstrokecolor{currentstroke}%
\pgfsetdash{}{0pt}%
\pgfpathmoveto{\pgfqpoint{2.283895in}{4.980827in}}%
\pgfpathcurveto{\pgfqpoint{2.294945in}{4.980827in}}{\pgfqpoint{2.305544in}{4.985217in}}{\pgfqpoint{2.313358in}{4.993031in}}%
\pgfpathcurveto{\pgfqpoint{2.321171in}{5.000844in}}{\pgfqpoint{2.325562in}{5.011443in}}{\pgfqpoint{2.325562in}{5.022494in}}%
\pgfpathcurveto{\pgfqpoint{2.325562in}{5.033544in}}{\pgfqpoint{2.321171in}{5.044143in}}{\pgfqpoint{2.313358in}{5.051956in}}%
\pgfpathcurveto{\pgfqpoint{2.305544in}{5.059770in}}{\pgfqpoint{2.294945in}{5.064160in}}{\pgfqpoint{2.283895in}{5.064160in}}%
\pgfpathcurveto{\pgfqpoint{2.272845in}{5.064160in}}{\pgfqpoint{2.262246in}{5.059770in}}{\pgfqpoint{2.254432in}{5.051956in}}%
\pgfpathcurveto{\pgfqpoint{2.246619in}{5.044143in}}{\pgfqpoint{2.242228in}{5.033544in}}{\pgfqpoint{2.242228in}{5.022494in}}%
\pgfpathcurveto{\pgfqpoint{2.242228in}{5.011443in}}{\pgfqpoint{2.246619in}{5.000844in}}{\pgfqpoint{2.254432in}{4.993031in}}%
\pgfpathcurveto{\pgfqpoint{2.262246in}{4.985217in}}{\pgfqpoint{2.272845in}{4.980827in}}{\pgfqpoint{2.283895in}{4.980827in}}%
\pgfpathlineto{\pgfqpoint{2.283895in}{4.980827in}}%
\pgfpathclose%
\pgfusepath{stroke,fill}%
\end{pgfscope}%
\begin{pgfscope}%
\pgfpathrectangle{\pgfqpoint{0.633874in}{2.920818in}}{\pgfqpoint{2.177280in}{2.201755in}}%
\pgfusepath{clip}%
\pgfsetbuttcap%
\pgfsetroundjoin%
\definecolor{currentfill}{rgb}{0.172549,0.627451,0.172549}%
\pgfsetfillcolor{currentfill}%
\pgfsetlinewidth{0.481800pt}%
\definecolor{currentstroke}{rgb}{1.000000,1.000000,1.000000}%
\pgfsetstrokecolor{currentstroke}%
\pgfsetdash{}{0pt}%
\pgfpathmoveto{\pgfqpoint{1.604937in}{4.336245in}}%
\pgfpathcurveto{\pgfqpoint{1.615987in}{4.336245in}}{\pgfqpoint{1.626586in}{4.340636in}}{\pgfqpoint{1.634400in}{4.348449in}}%
\pgfpathcurveto{\pgfqpoint{1.642214in}{4.356263in}}{\pgfqpoint{1.646604in}{4.366862in}}{\pgfqpoint{1.646604in}{4.377912in}}%
\pgfpathcurveto{\pgfqpoint{1.646604in}{4.388962in}}{\pgfqpoint{1.642214in}{4.399561in}}{\pgfqpoint{1.634400in}{4.407375in}}%
\pgfpathcurveto{\pgfqpoint{1.626586in}{4.415188in}}{\pgfqpoint{1.615987in}{4.419579in}}{\pgfqpoint{1.604937in}{4.419579in}}%
\pgfpathcurveto{\pgfqpoint{1.593887in}{4.419579in}}{\pgfqpoint{1.583288in}{4.415188in}}{\pgfqpoint{1.575474in}{4.407375in}}%
\pgfpathcurveto{\pgfqpoint{1.567661in}{4.399561in}}{\pgfqpoint{1.563270in}{4.388962in}}{\pgfqpoint{1.563270in}{4.377912in}}%
\pgfpathcurveto{\pgfqpoint{1.563270in}{4.366862in}}{\pgfqpoint{1.567661in}{4.356263in}}{\pgfqpoint{1.575474in}{4.348449in}}%
\pgfpathcurveto{\pgfqpoint{1.583288in}{4.340636in}}{\pgfqpoint{1.593887in}{4.336245in}}{\pgfqpoint{1.604937in}{4.336245in}}%
\pgfpathlineto{\pgfqpoint{1.604937in}{4.336245in}}%
\pgfpathclose%
\pgfusepath{stroke,fill}%
\end{pgfscope}%
\begin{pgfscope}%
\pgfpathrectangle{\pgfqpoint{0.633874in}{2.920818in}}{\pgfqpoint{2.177280in}{2.201755in}}%
\pgfusepath{clip}%
\pgfsetbuttcap%
\pgfsetroundjoin%
\definecolor{currentfill}{rgb}{0.172549,0.627451,0.172549}%
\pgfsetfillcolor{currentfill}%
\pgfsetlinewidth{0.481800pt}%
\definecolor{currentstroke}{rgb}{1.000000,1.000000,1.000000}%
\pgfsetstrokecolor{currentstroke}%
\pgfsetdash{}{0pt}%
\pgfpathmoveto{\pgfqpoint{1.964385in}{4.573723in}}%
\pgfpathcurveto{\pgfqpoint{1.975436in}{4.573723in}}{\pgfqpoint{1.986035in}{4.578113in}}{\pgfqpoint{1.993848in}{4.585927in}}%
\pgfpathcurveto{\pgfqpoint{2.001662in}{4.593740in}}{\pgfqpoint{2.006052in}{4.604339in}}{\pgfqpoint{2.006052in}{4.615389in}}%
\pgfpathcurveto{\pgfqpoint{2.006052in}{4.626440in}}{\pgfqpoint{2.001662in}{4.637039in}}{\pgfqpoint{1.993848in}{4.644852in}}%
\pgfpathcurveto{\pgfqpoint{1.986035in}{4.652666in}}{\pgfqpoint{1.975436in}{4.657056in}}{\pgfqpoint{1.964385in}{4.657056in}}%
\pgfpathcurveto{\pgfqpoint{1.953335in}{4.657056in}}{\pgfqpoint{1.942736in}{4.652666in}}{\pgfqpoint{1.934923in}{4.644852in}}%
\pgfpathcurveto{\pgfqpoint{1.927109in}{4.637039in}}{\pgfqpoint{1.922719in}{4.626440in}}{\pgfqpoint{1.922719in}{4.615389in}}%
\pgfpathcurveto{\pgfqpoint{1.922719in}{4.604339in}}{\pgfqpoint{1.927109in}{4.593740in}}{\pgfqpoint{1.934923in}{4.585927in}}%
\pgfpathcurveto{\pgfqpoint{1.942736in}{4.578113in}}{\pgfqpoint{1.953335in}{4.573723in}}{\pgfqpoint{1.964385in}{4.573723in}}%
\pgfpathlineto{\pgfqpoint{1.964385in}{4.573723in}}%
\pgfpathclose%
\pgfusepath{stroke,fill}%
\end{pgfscope}%
\begin{pgfscope}%
\pgfpathrectangle{\pgfqpoint{0.633874in}{2.920818in}}{\pgfqpoint{2.177280in}{2.201755in}}%
\pgfusepath{clip}%
\pgfsetbuttcap%
\pgfsetroundjoin%
\definecolor{currentfill}{rgb}{0.172549,0.627451,0.172549}%
\pgfsetfillcolor{currentfill}%
\pgfsetlinewidth{0.481800pt}%
\definecolor{currentstroke}{rgb}{1.000000,1.000000,1.000000}%
\pgfsetstrokecolor{currentstroke}%
\pgfsetdash{}{0pt}%
\pgfpathmoveto{\pgfqpoint{1.445182in}{4.302320in}}%
\pgfpathcurveto{\pgfqpoint{1.456232in}{4.302320in}}{\pgfqpoint{1.466831in}{4.306710in}}{\pgfqpoint{1.474645in}{4.314524in}}%
\pgfpathcurveto{\pgfqpoint{1.482459in}{4.322337in}}{\pgfqpoint{1.486849in}{4.332937in}}{\pgfqpoint{1.486849in}{4.343987in}}%
\pgfpathcurveto{\pgfqpoint{1.486849in}{4.355037in}}{\pgfqpoint{1.482459in}{4.365636in}}{\pgfqpoint{1.474645in}{4.373449in}}%
\pgfpathcurveto{\pgfqpoint{1.466831in}{4.381263in}}{\pgfqpoint{1.456232in}{4.385653in}}{\pgfqpoint{1.445182in}{4.385653in}}%
\pgfpathcurveto{\pgfqpoint{1.434132in}{4.385653in}}{\pgfqpoint{1.423533in}{4.381263in}}{\pgfqpoint{1.415720in}{4.373449in}}%
\pgfpathcurveto{\pgfqpoint{1.407906in}{4.365636in}}{\pgfqpoint{1.403516in}{4.355037in}}{\pgfqpoint{1.403516in}{4.343987in}}%
\pgfpathcurveto{\pgfqpoint{1.403516in}{4.332937in}}{\pgfqpoint{1.407906in}{4.322337in}}{\pgfqpoint{1.415720in}{4.314524in}}%
\pgfpathcurveto{\pgfqpoint{1.423533in}{4.306710in}}{\pgfqpoint{1.434132in}{4.302320in}}{\pgfqpoint{1.445182in}{4.302320in}}%
\pgfpathlineto{\pgfqpoint{1.445182in}{4.302320in}}%
\pgfpathclose%
\pgfusepath{stroke,fill}%
\end{pgfscope}%
\begin{pgfscope}%
\pgfpathrectangle{\pgfqpoint{0.633874in}{2.920818in}}{\pgfqpoint{2.177280in}{2.201755in}}%
\pgfusepath{clip}%
\pgfsetbuttcap%
\pgfsetroundjoin%
\definecolor{currentfill}{rgb}{0.172549,0.627451,0.172549}%
\pgfsetfillcolor{currentfill}%
\pgfsetlinewidth{0.481800pt}%
\definecolor{currentstroke}{rgb}{1.000000,1.000000,1.000000}%
\pgfsetstrokecolor{currentstroke}%
\pgfsetdash{}{0pt}%
\pgfpathmoveto{\pgfqpoint{2.283895in}{4.912976in}}%
\pgfpathcurveto{\pgfqpoint{2.294945in}{4.912976in}}{\pgfqpoint{2.305544in}{4.917366in}}{\pgfqpoint{2.313358in}{4.925180in}}%
\pgfpathcurveto{\pgfqpoint{2.321171in}{4.932994in}}{\pgfqpoint{2.325562in}{4.943593in}}{\pgfqpoint{2.325562in}{4.954643in}}%
\pgfpathcurveto{\pgfqpoint{2.325562in}{4.965693in}}{\pgfqpoint{2.321171in}{4.976292in}}{\pgfqpoint{2.313358in}{4.984106in}}%
\pgfpathcurveto{\pgfqpoint{2.305544in}{4.991919in}}{\pgfqpoint{2.294945in}{4.996310in}}{\pgfqpoint{2.283895in}{4.996310in}}%
\pgfpathcurveto{\pgfqpoint{2.272845in}{4.996310in}}{\pgfqpoint{2.262246in}{4.991919in}}{\pgfqpoint{2.254432in}{4.984106in}}%
\pgfpathcurveto{\pgfqpoint{2.246619in}{4.976292in}}{\pgfqpoint{2.242228in}{4.965693in}}{\pgfqpoint{2.242228in}{4.954643in}}%
\pgfpathcurveto{\pgfqpoint{2.242228in}{4.943593in}}{\pgfqpoint{2.246619in}{4.932994in}}{\pgfqpoint{2.254432in}{4.925180in}}%
\pgfpathcurveto{\pgfqpoint{2.262246in}{4.917366in}}{\pgfqpoint{2.272845in}{4.912976in}}{\pgfqpoint{2.283895in}{4.912976in}}%
\pgfpathlineto{\pgfqpoint{2.283895in}{4.912976in}}%
\pgfpathclose%
\pgfusepath{stroke,fill}%
\end{pgfscope}%
\begin{pgfscope}%
\pgfpathrectangle{\pgfqpoint{0.633874in}{2.920818in}}{\pgfqpoint{2.177280in}{2.201755in}}%
\pgfusepath{clip}%
\pgfsetbuttcap%
\pgfsetroundjoin%
\definecolor{currentfill}{rgb}{0.172549,0.627451,0.172549}%
\pgfsetfillcolor{currentfill}%
\pgfsetlinewidth{0.481800pt}%
\definecolor{currentstroke}{rgb}{1.000000,1.000000,1.000000}%
\pgfsetstrokecolor{currentstroke}%
\pgfsetdash{}{0pt}%
\pgfpathmoveto{\pgfqpoint{1.724753in}{4.302320in}}%
\pgfpathcurveto{\pgfqpoint{1.735803in}{4.302320in}}{\pgfqpoint{1.746402in}{4.306710in}}{\pgfqpoint{1.754216in}{4.314524in}}%
\pgfpathcurveto{\pgfqpoint{1.762030in}{4.322337in}}{\pgfqpoint{1.766420in}{4.332937in}}{\pgfqpoint{1.766420in}{4.343987in}}%
\pgfpathcurveto{\pgfqpoint{1.766420in}{4.355037in}}{\pgfqpoint{1.762030in}{4.365636in}}{\pgfqpoint{1.754216in}{4.373449in}}%
\pgfpathcurveto{\pgfqpoint{1.746402in}{4.381263in}}{\pgfqpoint{1.735803in}{4.385653in}}{\pgfqpoint{1.724753in}{4.385653in}}%
\pgfpathcurveto{\pgfqpoint{1.713703in}{4.385653in}}{\pgfqpoint{1.703104in}{4.381263in}}{\pgfqpoint{1.695290in}{4.373449in}}%
\pgfpathcurveto{\pgfqpoint{1.687477in}{4.365636in}}{\pgfqpoint{1.683087in}{4.355037in}}{\pgfqpoint{1.683087in}{4.343987in}}%
\pgfpathcurveto{\pgfqpoint{1.683087in}{4.332937in}}{\pgfqpoint{1.687477in}{4.322337in}}{\pgfqpoint{1.695290in}{4.314524in}}%
\pgfpathcurveto{\pgfqpoint{1.703104in}{4.306710in}}{\pgfqpoint{1.713703in}{4.302320in}}{\pgfqpoint{1.724753in}{4.302320in}}%
\pgfpathlineto{\pgfqpoint{1.724753in}{4.302320in}}%
\pgfpathclose%
\pgfusepath{stroke,fill}%
\end{pgfscope}%
\begin{pgfscope}%
\pgfpathrectangle{\pgfqpoint{0.633874in}{2.920818in}}{\pgfqpoint{2.177280in}{2.201755in}}%
\pgfusepath{clip}%
\pgfsetbuttcap%
\pgfsetroundjoin%
\definecolor{currentfill}{rgb}{0.172549,0.627451,0.172549}%
\pgfsetfillcolor{currentfill}%
\pgfsetlinewidth{0.481800pt}%
\definecolor{currentstroke}{rgb}{1.000000,1.000000,1.000000}%
\pgfsetstrokecolor{currentstroke}%
\pgfsetdash{}{0pt}%
\pgfpathmoveto{\pgfqpoint{1.884508in}{4.573723in}}%
\pgfpathcurveto{\pgfqpoint{1.895558in}{4.573723in}}{\pgfqpoint{1.906157in}{4.578113in}}{\pgfqpoint{1.913971in}{4.585927in}}%
\pgfpathcurveto{\pgfqpoint{1.921784in}{4.593740in}}{\pgfqpoint{1.926175in}{4.604339in}}{\pgfqpoint{1.926175in}{4.615389in}}%
\pgfpathcurveto{\pgfqpoint{1.926175in}{4.626440in}}{\pgfqpoint{1.921784in}{4.637039in}}{\pgfqpoint{1.913971in}{4.644852in}}%
\pgfpathcurveto{\pgfqpoint{1.906157in}{4.652666in}}{\pgfqpoint{1.895558in}{4.657056in}}{\pgfqpoint{1.884508in}{4.657056in}}%
\pgfpathcurveto{\pgfqpoint{1.873458in}{4.657056in}}{\pgfqpoint{1.862859in}{4.652666in}}{\pgfqpoint{1.855045in}{4.644852in}}%
\pgfpathcurveto{\pgfqpoint{1.847232in}{4.637039in}}{\pgfqpoint{1.842841in}{4.626440in}}{\pgfqpoint{1.842841in}{4.615389in}}%
\pgfpathcurveto{\pgfqpoint{1.842841in}{4.604339in}}{\pgfqpoint{1.847232in}{4.593740in}}{\pgfqpoint{1.855045in}{4.585927in}}%
\pgfpathcurveto{\pgfqpoint{1.862859in}{4.578113in}}{\pgfqpoint{1.873458in}{4.573723in}}{\pgfqpoint{1.884508in}{4.573723in}}%
\pgfpathlineto{\pgfqpoint{1.884508in}{4.573723in}}%
\pgfpathclose%
\pgfusepath{stroke,fill}%
\end{pgfscope}%
\begin{pgfscope}%
\pgfpathrectangle{\pgfqpoint{0.633874in}{2.920818in}}{\pgfqpoint{2.177280in}{2.201755in}}%
\pgfusepath{clip}%
\pgfsetbuttcap%
\pgfsetroundjoin%
\definecolor{currentfill}{rgb}{0.172549,0.627451,0.172549}%
\pgfsetfillcolor{currentfill}%
\pgfsetlinewidth{0.481800pt}%
\definecolor{currentstroke}{rgb}{1.000000,1.000000,1.000000}%
\pgfsetstrokecolor{currentstroke}%
\pgfsetdash{}{0pt}%
\pgfpathmoveto{\pgfqpoint{2.084202in}{4.675499in}}%
\pgfpathcurveto{\pgfqpoint{2.095252in}{4.675499in}}{\pgfqpoint{2.105851in}{4.679889in}}{\pgfqpoint{2.113664in}{4.687703in}}%
\pgfpathcurveto{\pgfqpoint{2.121478in}{4.695516in}}{\pgfqpoint{2.125868in}{4.706115in}}{\pgfqpoint{2.125868in}{4.717165in}}%
\pgfpathcurveto{\pgfqpoint{2.125868in}{4.728216in}}{\pgfqpoint{2.121478in}{4.738815in}}{\pgfqpoint{2.113664in}{4.746628in}}%
\pgfpathcurveto{\pgfqpoint{2.105851in}{4.754442in}}{\pgfqpoint{2.095252in}{4.758832in}}{\pgfqpoint{2.084202in}{4.758832in}}%
\pgfpathcurveto{\pgfqpoint{2.073151in}{4.758832in}}{\pgfqpoint{2.062552in}{4.754442in}}{\pgfqpoint{2.054739in}{4.746628in}}%
\pgfpathcurveto{\pgfqpoint{2.046925in}{4.738815in}}{\pgfqpoint{2.042535in}{4.728216in}}{\pgfqpoint{2.042535in}{4.717165in}}%
\pgfpathcurveto{\pgfqpoint{2.042535in}{4.706115in}}{\pgfqpoint{2.046925in}{4.695516in}}{\pgfqpoint{2.054739in}{4.687703in}}%
\pgfpathcurveto{\pgfqpoint{2.062552in}{4.679889in}}{\pgfqpoint{2.073151in}{4.675499in}}{\pgfqpoint{2.084202in}{4.675499in}}%
\pgfpathlineto{\pgfqpoint{2.084202in}{4.675499in}}%
\pgfpathclose%
\pgfusepath{stroke,fill}%
\end{pgfscope}%
\begin{pgfscope}%
\pgfpathrectangle{\pgfqpoint{0.633874in}{2.920818in}}{\pgfqpoint{2.177280in}{2.201755in}}%
\pgfusepath{clip}%
\pgfsetbuttcap%
\pgfsetroundjoin%
\definecolor{currentfill}{rgb}{0.172549,0.627451,0.172549}%
\pgfsetfillcolor{currentfill}%
\pgfsetlinewidth{0.481800pt}%
\definecolor{currentstroke}{rgb}{1.000000,1.000000,1.000000}%
\pgfsetstrokecolor{currentstroke}%
\pgfsetdash{}{0pt}%
\pgfpathmoveto{\pgfqpoint{1.684815in}{4.268395in}}%
\pgfpathcurveto{\pgfqpoint{1.695865in}{4.268395in}}{\pgfqpoint{1.706464in}{4.272785in}}{\pgfqpoint{1.714277in}{4.280599in}}%
\pgfpathcurveto{\pgfqpoint{1.722091in}{4.288412in}}{\pgfqpoint{1.726481in}{4.299011in}}{\pgfqpoint{1.726481in}{4.310061in}}%
\pgfpathcurveto{\pgfqpoint{1.726481in}{4.321111in}}{\pgfqpoint{1.722091in}{4.331710in}}{\pgfqpoint{1.714277in}{4.339524in}}%
\pgfpathcurveto{\pgfqpoint{1.706464in}{4.347338in}}{\pgfqpoint{1.695865in}{4.351728in}}{\pgfqpoint{1.684815in}{4.351728in}}%
\pgfpathcurveto{\pgfqpoint{1.673764in}{4.351728in}}{\pgfqpoint{1.663165in}{4.347338in}}{\pgfqpoint{1.655352in}{4.339524in}}%
\pgfpathcurveto{\pgfqpoint{1.647538in}{4.331710in}}{\pgfqpoint{1.643148in}{4.321111in}}{\pgfqpoint{1.643148in}{4.310061in}}%
\pgfpathcurveto{\pgfqpoint{1.643148in}{4.299011in}}{\pgfqpoint{1.647538in}{4.288412in}}{\pgfqpoint{1.655352in}{4.280599in}}%
\pgfpathcurveto{\pgfqpoint{1.663165in}{4.272785in}}{\pgfqpoint{1.673764in}{4.268395in}}{\pgfqpoint{1.684815in}{4.268395in}}%
\pgfpathlineto{\pgfqpoint{1.684815in}{4.268395in}}%
\pgfpathclose%
\pgfusepath{stroke,fill}%
\end{pgfscope}%
\begin{pgfscope}%
\pgfpathrectangle{\pgfqpoint{0.633874in}{2.920818in}}{\pgfqpoint{2.177280in}{2.201755in}}%
\pgfusepath{clip}%
\pgfsetbuttcap%
\pgfsetroundjoin%
\definecolor{currentfill}{rgb}{0.172549,0.627451,0.172549}%
\pgfsetfillcolor{currentfill}%
\pgfsetlinewidth{0.481800pt}%
\definecolor{currentstroke}{rgb}{1.000000,1.000000,1.000000}%
\pgfsetstrokecolor{currentstroke}%
\pgfsetdash{}{0pt}%
\pgfpathmoveto{\pgfqpoint{1.644876in}{4.302320in}}%
\pgfpathcurveto{\pgfqpoint{1.655926in}{4.302320in}}{\pgfqpoint{1.666525in}{4.306710in}}{\pgfqpoint{1.674339in}{4.314524in}}%
\pgfpathcurveto{\pgfqpoint{1.682152in}{4.322337in}}{\pgfqpoint{1.686543in}{4.332937in}}{\pgfqpoint{1.686543in}{4.343987in}}%
\pgfpathcurveto{\pgfqpoint{1.686543in}{4.355037in}}{\pgfqpoint{1.682152in}{4.365636in}}{\pgfqpoint{1.674339in}{4.373449in}}%
\pgfpathcurveto{\pgfqpoint{1.666525in}{4.381263in}}{\pgfqpoint{1.655926in}{4.385653in}}{\pgfqpoint{1.644876in}{4.385653in}}%
\pgfpathcurveto{\pgfqpoint{1.633826in}{4.385653in}}{\pgfqpoint{1.623227in}{4.381263in}}{\pgfqpoint{1.615413in}{4.373449in}}%
\pgfpathcurveto{\pgfqpoint{1.607599in}{4.365636in}}{\pgfqpoint{1.603209in}{4.355037in}}{\pgfqpoint{1.603209in}{4.343987in}}%
\pgfpathcurveto{\pgfqpoint{1.603209in}{4.332937in}}{\pgfqpoint{1.607599in}{4.322337in}}{\pgfqpoint{1.615413in}{4.314524in}}%
\pgfpathcurveto{\pgfqpoint{1.623227in}{4.306710in}}{\pgfqpoint{1.633826in}{4.302320in}}{\pgfqpoint{1.644876in}{4.302320in}}%
\pgfpathlineto{\pgfqpoint{1.644876in}{4.302320in}}%
\pgfpathclose%
\pgfusepath{stroke,fill}%
\end{pgfscope}%
\begin{pgfscope}%
\pgfpathrectangle{\pgfqpoint{0.633874in}{2.920818in}}{\pgfqpoint{2.177280in}{2.201755in}}%
\pgfusepath{clip}%
\pgfsetbuttcap%
\pgfsetroundjoin%
\definecolor{currentfill}{rgb}{0.172549,0.627451,0.172549}%
\pgfsetfillcolor{currentfill}%
\pgfsetlinewidth{0.481800pt}%
\definecolor{currentstroke}{rgb}{1.000000,1.000000,1.000000}%
\pgfsetstrokecolor{currentstroke}%
\pgfsetdash{}{0pt}%
\pgfpathmoveto{\pgfqpoint{1.764692in}{4.539797in}}%
\pgfpathcurveto{\pgfqpoint{1.775742in}{4.539797in}}{\pgfqpoint{1.786341in}{4.544188in}}{\pgfqpoint{1.794155in}{4.552001in}}%
\pgfpathcurveto{\pgfqpoint{1.801968in}{4.559815in}}{\pgfqpoint{1.806359in}{4.570414in}}{\pgfqpoint{1.806359in}{4.581464in}}%
\pgfpathcurveto{\pgfqpoint{1.806359in}{4.592514in}}{\pgfqpoint{1.801968in}{4.603113in}}{\pgfqpoint{1.794155in}{4.610927in}}%
\pgfpathcurveto{\pgfqpoint{1.786341in}{4.618740in}}{\pgfqpoint{1.775742in}{4.623131in}}{\pgfqpoint{1.764692in}{4.623131in}}%
\pgfpathcurveto{\pgfqpoint{1.753642in}{4.623131in}}{\pgfqpoint{1.743043in}{4.618740in}}{\pgfqpoint{1.735229in}{4.610927in}}%
\pgfpathcurveto{\pgfqpoint{1.727416in}{4.603113in}}{\pgfqpoint{1.723025in}{4.592514in}}{\pgfqpoint{1.723025in}{4.581464in}}%
\pgfpathcurveto{\pgfqpoint{1.723025in}{4.570414in}}{\pgfqpoint{1.727416in}{4.559815in}}{\pgfqpoint{1.735229in}{4.552001in}}%
\pgfpathcurveto{\pgfqpoint{1.743043in}{4.544188in}}{\pgfqpoint{1.753642in}{4.539797in}}{\pgfqpoint{1.764692in}{4.539797in}}%
\pgfpathlineto{\pgfqpoint{1.764692in}{4.539797in}}%
\pgfpathclose%
\pgfusepath{stroke,fill}%
\end{pgfscope}%
\begin{pgfscope}%
\pgfpathrectangle{\pgfqpoint{0.633874in}{2.920818in}}{\pgfqpoint{2.177280in}{2.201755in}}%
\pgfusepath{clip}%
\pgfsetbuttcap%
\pgfsetroundjoin%
\definecolor{currentfill}{rgb}{0.172549,0.627451,0.172549}%
\pgfsetfillcolor{currentfill}%
\pgfsetlinewidth{0.481800pt}%
\definecolor{currentstroke}{rgb}{1.000000,1.000000,1.000000}%
\pgfsetstrokecolor{currentstroke}%
\pgfsetdash{}{0pt}%
\pgfpathmoveto{\pgfqpoint{2.084202in}{4.607648in}}%
\pgfpathcurveto{\pgfqpoint{2.095252in}{4.607648in}}{\pgfqpoint{2.105851in}{4.612038in}}{\pgfqpoint{2.113664in}{4.619852in}}%
\pgfpathcurveto{\pgfqpoint{2.121478in}{4.627666in}}{\pgfqpoint{2.125868in}{4.638265in}}{\pgfqpoint{2.125868in}{4.649315in}}%
\pgfpathcurveto{\pgfqpoint{2.125868in}{4.660365in}}{\pgfqpoint{2.121478in}{4.670964in}}{\pgfqpoint{2.113664in}{4.678778in}}%
\pgfpathcurveto{\pgfqpoint{2.105851in}{4.686591in}}{\pgfqpoint{2.095252in}{4.690981in}}{\pgfqpoint{2.084202in}{4.690981in}}%
\pgfpathcurveto{\pgfqpoint{2.073151in}{4.690981in}}{\pgfqpoint{2.062552in}{4.686591in}}{\pgfqpoint{2.054739in}{4.678778in}}%
\pgfpathcurveto{\pgfqpoint{2.046925in}{4.670964in}}{\pgfqpoint{2.042535in}{4.660365in}}{\pgfqpoint{2.042535in}{4.649315in}}%
\pgfpathcurveto{\pgfqpoint{2.042535in}{4.638265in}}{\pgfqpoint{2.046925in}{4.627666in}}{\pgfqpoint{2.054739in}{4.619852in}}%
\pgfpathcurveto{\pgfqpoint{2.062552in}{4.612038in}}{\pgfqpoint{2.073151in}{4.607648in}}{\pgfqpoint{2.084202in}{4.607648in}}%
\pgfpathlineto{\pgfqpoint{2.084202in}{4.607648in}}%
\pgfpathclose%
\pgfusepath{stroke,fill}%
\end{pgfscope}%
\begin{pgfscope}%
\pgfpathrectangle{\pgfqpoint{0.633874in}{2.920818in}}{\pgfqpoint{2.177280in}{2.201755in}}%
\pgfusepath{clip}%
\pgfsetbuttcap%
\pgfsetroundjoin%
\definecolor{currentfill}{rgb}{0.172549,0.627451,0.172549}%
\pgfsetfillcolor{currentfill}%
\pgfsetlinewidth{0.481800pt}%
\definecolor{currentstroke}{rgb}{1.000000,1.000000,1.000000}%
\pgfsetstrokecolor{currentstroke}%
\pgfsetdash{}{0pt}%
\pgfpathmoveto{\pgfqpoint{2.164079in}{4.709424in}}%
\pgfpathcurveto{\pgfqpoint{2.175129in}{4.709424in}}{\pgfqpoint{2.185728in}{4.713814in}}{\pgfqpoint{2.193542in}{4.721628in}}%
\pgfpathcurveto{\pgfqpoint{2.201355in}{4.729442in}}{\pgfqpoint{2.205746in}{4.740041in}}{\pgfqpoint{2.205746in}{4.751091in}}%
\pgfpathcurveto{\pgfqpoint{2.205746in}{4.762141in}}{\pgfqpoint{2.201355in}{4.772740in}}{\pgfqpoint{2.193542in}{4.780554in}}%
\pgfpathcurveto{\pgfqpoint{2.185728in}{4.788367in}}{\pgfqpoint{2.175129in}{4.792757in}}{\pgfqpoint{2.164079in}{4.792757in}}%
\pgfpathcurveto{\pgfqpoint{2.153029in}{4.792757in}}{\pgfqpoint{2.142430in}{4.788367in}}{\pgfqpoint{2.134616in}{4.780554in}}%
\pgfpathcurveto{\pgfqpoint{2.126803in}{4.772740in}}{\pgfqpoint{2.122412in}{4.762141in}}{\pgfqpoint{2.122412in}{4.751091in}}%
\pgfpathcurveto{\pgfqpoint{2.122412in}{4.740041in}}{\pgfqpoint{2.126803in}{4.729442in}}{\pgfqpoint{2.134616in}{4.721628in}}%
\pgfpathcurveto{\pgfqpoint{2.142430in}{4.713814in}}{\pgfqpoint{2.153029in}{4.709424in}}{\pgfqpoint{2.164079in}{4.709424in}}%
\pgfpathlineto{\pgfqpoint{2.164079in}{4.709424in}}%
\pgfpathclose%
\pgfusepath{stroke,fill}%
\end{pgfscope}%
\begin{pgfscope}%
\pgfpathrectangle{\pgfqpoint{0.633874in}{2.920818in}}{\pgfqpoint{2.177280in}{2.201755in}}%
\pgfusepath{clip}%
\pgfsetbuttcap%
\pgfsetroundjoin%
\definecolor{currentfill}{rgb}{0.172549,0.627451,0.172549}%
\pgfsetfillcolor{currentfill}%
\pgfsetlinewidth{0.481800pt}%
\definecolor{currentstroke}{rgb}{1.000000,1.000000,1.000000}%
\pgfsetstrokecolor{currentstroke}%
\pgfsetdash{}{0pt}%
\pgfpathmoveto{\pgfqpoint{2.363772in}{4.811200in}}%
\pgfpathcurveto{\pgfqpoint{2.374823in}{4.811200in}}{\pgfqpoint{2.385422in}{4.815590in}}{\pgfqpoint{2.393235in}{4.823404in}}%
\pgfpathcurveto{\pgfqpoint{2.401049in}{4.831218in}}{\pgfqpoint{2.405439in}{4.841817in}}{\pgfqpoint{2.405439in}{4.852867in}}%
\pgfpathcurveto{\pgfqpoint{2.405439in}{4.863917in}}{\pgfqpoint{2.401049in}{4.874516in}}{\pgfqpoint{2.393235in}{4.882330in}}%
\pgfpathcurveto{\pgfqpoint{2.385422in}{4.890143in}}{\pgfqpoint{2.374823in}{4.894534in}}{\pgfqpoint{2.363772in}{4.894534in}}%
\pgfpathcurveto{\pgfqpoint{2.352722in}{4.894534in}}{\pgfqpoint{2.342123in}{4.890143in}}{\pgfqpoint{2.334310in}{4.882330in}}%
\pgfpathcurveto{\pgfqpoint{2.326496in}{4.874516in}}{\pgfqpoint{2.322106in}{4.863917in}}{\pgfqpoint{2.322106in}{4.852867in}}%
\pgfpathcurveto{\pgfqpoint{2.322106in}{4.841817in}}{\pgfqpoint{2.326496in}{4.831218in}}{\pgfqpoint{2.334310in}{4.823404in}}%
\pgfpathcurveto{\pgfqpoint{2.342123in}{4.815590in}}{\pgfqpoint{2.352722in}{4.811200in}}{\pgfqpoint{2.363772in}{4.811200in}}%
\pgfpathlineto{\pgfqpoint{2.363772in}{4.811200in}}%
\pgfpathclose%
\pgfusepath{stroke,fill}%
\end{pgfscope}%
\begin{pgfscope}%
\pgfpathrectangle{\pgfqpoint{0.633874in}{2.920818in}}{\pgfqpoint{2.177280in}{2.201755in}}%
\pgfusepath{clip}%
\pgfsetbuttcap%
\pgfsetroundjoin%
\definecolor{currentfill}{rgb}{0.172549,0.627451,0.172549}%
\pgfsetfillcolor{currentfill}%
\pgfsetlinewidth{0.481800pt}%
\definecolor{currentstroke}{rgb}{1.000000,1.000000,1.000000}%
\pgfsetstrokecolor{currentstroke}%
\pgfsetdash{}{0pt}%
\pgfpathmoveto{\pgfqpoint{1.764692in}{4.539797in}}%
\pgfpathcurveto{\pgfqpoint{1.775742in}{4.539797in}}{\pgfqpoint{1.786341in}{4.544188in}}{\pgfqpoint{1.794155in}{4.552001in}}%
\pgfpathcurveto{\pgfqpoint{1.801968in}{4.559815in}}{\pgfqpoint{1.806359in}{4.570414in}}{\pgfqpoint{1.806359in}{4.581464in}}%
\pgfpathcurveto{\pgfqpoint{1.806359in}{4.592514in}}{\pgfqpoint{1.801968in}{4.603113in}}{\pgfqpoint{1.794155in}{4.610927in}}%
\pgfpathcurveto{\pgfqpoint{1.786341in}{4.618740in}}{\pgfqpoint{1.775742in}{4.623131in}}{\pgfqpoint{1.764692in}{4.623131in}}%
\pgfpathcurveto{\pgfqpoint{1.753642in}{4.623131in}}{\pgfqpoint{1.743043in}{4.618740in}}{\pgfqpoint{1.735229in}{4.610927in}}%
\pgfpathcurveto{\pgfqpoint{1.727416in}{4.603113in}}{\pgfqpoint{1.723025in}{4.592514in}}{\pgfqpoint{1.723025in}{4.581464in}}%
\pgfpathcurveto{\pgfqpoint{1.723025in}{4.570414in}}{\pgfqpoint{1.727416in}{4.559815in}}{\pgfqpoint{1.735229in}{4.552001in}}%
\pgfpathcurveto{\pgfqpoint{1.743043in}{4.544188in}}{\pgfqpoint{1.753642in}{4.539797in}}{\pgfqpoint{1.764692in}{4.539797in}}%
\pgfpathlineto{\pgfqpoint{1.764692in}{4.539797in}}%
\pgfpathclose%
\pgfusepath{stroke,fill}%
\end{pgfscope}%
\begin{pgfscope}%
\pgfpathrectangle{\pgfqpoint{0.633874in}{2.920818in}}{\pgfqpoint{2.177280in}{2.201755in}}%
\pgfusepath{clip}%
\pgfsetbuttcap%
\pgfsetroundjoin%
\definecolor{currentfill}{rgb}{0.172549,0.627451,0.172549}%
\pgfsetfillcolor{currentfill}%
\pgfsetlinewidth{0.481800pt}%
\definecolor{currentstroke}{rgb}{1.000000,1.000000,1.000000}%
\pgfsetstrokecolor{currentstroke}%
\pgfsetdash{}{0pt}%
\pgfpathmoveto{\pgfqpoint{1.724753in}{4.370171in}}%
\pgfpathcurveto{\pgfqpoint{1.735803in}{4.370171in}}{\pgfqpoint{1.746402in}{4.374561in}}{\pgfqpoint{1.754216in}{4.382375in}}%
\pgfpathcurveto{\pgfqpoint{1.762030in}{4.390188in}}{\pgfqpoint{1.766420in}{4.400787in}}{\pgfqpoint{1.766420in}{4.411837in}}%
\pgfpathcurveto{\pgfqpoint{1.766420in}{4.422887in}}{\pgfqpoint{1.762030in}{4.433486in}}{\pgfqpoint{1.754216in}{4.441300in}}%
\pgfpathcurveto{\pgfqpoint{1.746402in}{4.449114in}}{\pgfqpoint{1.735803in}{4.453504in}}{\pgfqpoint{1.724753in}{4.453504in}}%
\pgfpathcurveto{\pgfqpoint{1.713703in}{4.453504in}}{\pgfqpoint{1.703104in}{4.449114in}}{\pgfqpoint{1.695290in}{4.441300in}}%
\pgfpathcurveto{\pgfqpoint{1.687477in}{4.433486in}}{\pgfqpoint{1.683087in}{4.422887in}}{\pgfqpoint{1.683087in}{4.411837in}}%
\pgfpathcurveto{\pgfqpoint{1.683087in}{4.400787in}}{\pgfqpoint{1.687477in}{4.390188in}}{\pgfqpoint{1.695290in}{4.382375in}}%
\pgfpathcurveto{\pgfqpoint{1.703104in}{4.374561in}}{\pgfqpoint{1.713703in}{4.370171in}}{\pgfqpoint{1.724753in}{4.370171in}}%
\pgfpathlineto{\pgfqpoint{1.724753in}{4.370171in}}%
\pgfpathclose%
\pgfusepath{stroke,fill}%
\end{pgfscope}%
\begin{pgfscope}%
\pgfpathrectangle{\pgfqpoint{0.633874in}{2.920818in}}{\pgfqpoint{2.177280in}{2.201755in}}%
\pgfusepath{clip}%
\pgfsetbuttcap%
\pgfsetroundjoin%
\definecolor{currentfill}{rgb}{0.172549,0.627451,0.172549}%
\pgfsetfillcolor{currentfill}%
\pgfsetlinewidth{0.481800pt}%
\definecolor{currentstroke}{rgb}{1.000000,1.000000,1.000000}%
\pgfsetstrokecolor{currentstroke}%
\pgfsetdash{}{0pt}%
\pgfpathmoveto{\pgfqpoint{1.644876in}{4.539797in}}%
\pgfpathcurveto{\pgfqpoint{1.655926in}{4.539797in}}{\pgfqpoint{1.666525in}{4.544188in}}{\pgfqpoint{1.674339in}{4.552001in}}%
\pgfpathcurveto{\pgfqpoint{1.682152in}{4.559815in}}{\pgfqpoint{1.686543in}{4.570414in}}{\pgfqpoint{1.686543in}{4.581464in}}%
\pgfpathcurveto{\pgfqpoint{1.686543in}{4.592514in}}{\pgfqpoint{1.682152in}{4.603113in}}{\pgfqpoint{1.674339in}{4.610927in}}%
\pgfpathcurveto{\pgfqpoint{1.666525in}{4.618740in}}{\pgfqpoint{1.655926in}{4.623131in}}{\pgfqpoint{1.644876in}{4.623131in}}%
\pgfpathcurveto{\pgfqpoint{1.633826in}{4.623131in}}{\pgfqpoint{1.623227in}{4.618740in}}{\pgfqpoint{1.615413in}{4.610927in}}%
\pgfpathcurveto{\pgfqpoint{1.607599in}{4.603113in}}{\pgfqpoint{1.603209in}{4.592514in}}{\pgfqpoint{1.603209in}{4.581464in}}%
\pgfpathcurveto{\pgfqpoint{1.603209in}{4.570414in}}{\pgfqpoint{1.607599in}{4.559815in}}{\pgfqpoint{1.615413in}{4.552001in}}%
\pgfpathcurveto{\pgfqpoint{1.623227in}{4.544188in}}{\pgfqpoint{1.633826in}{4.539797in}}{\pgfqpoint{1.644876in}{4.539797in}}%
\pgfpathlineto{\pgfqpoint{1.644876in}{4.539797in}}%
\pgfpathclose%
\pgfusepath{stroke,fill}%
\end{pgfscope}%
\begin{pgfscope}%
\pgfpathrectangle{\pgfqpoint{0.633874in}{2.920818in}}{\pgfqpoint{2.177280in}{2.201755in}}%
\pgfusepath{clip}%
\pgfsetbuttcap%
\pgfsetroundjoin%
\definecolor{currentfill}{rgb}{0.172549,0.627451,0.172549}%
\pgfsetfillcolor{currentfill}%
\pgfsetlinewidth{0.481800pt}%
\definecolor{currentstroke}{rgb}{1.000000,1.000000,1.000000}%
\pgfsetstrokecolor{currentstroke}%
\pgfsetdash{}{0pt}%
\pgfpathmoveto{\pgfqpoint{2.283895in}{4.709424in}}%
\pgfpathcurveto{\pgfqpoint{2.294945in}{4.709424in}}{\pgfqpoint{2.305544in}{4.713814in}}{\pgfqpoint{2.313358in}{4.721628in}}%
\pgfpathcurveto{\pgfqpoint{2.321171in}{4.729442in}}{\pgfqpoint{2.325562in}{4.740041in}}{\pgfqpoint{2.325562in}{4.751091in}}%
\pgfpathcurveto{\pgfqpoint{2.325562in}{4.762141in}}{\pgfqpoint{2.321171in}{4.772740in}}{\pgfqpoint{2.313358in}{4.780554in}}%
\pgfpathcurveto{\pgfqpoint{2.305544in}{4.788367in}}{\pgfqpoint{2.294945in}{4.792757in}}{\pgfqpoint{2.283895in}{4.792757in}}%
\pgfpathcurveto{\pgfqpoint{2.272845in}{4.792757in}}{\pgfqpoint{2.262246in}{4.788367in}}{\pgfqpoint{2.254432in}{4.780554in}}%
\pgfpathcurveto{\pgfqpoint{2.246619in}{4.772740in}}{\pgfqpoint{2.242228in}{4.762141in}}{\pgfqpoint{2.242228in}{4.751091in}}%
\pgfpathcurveto{\pgfqpoint{2.242228in}{4.740041in}}{\pgfqpoint{2.246619in}{4.729442in}}{\pgfqpoint{2.254432in}{4.721628in}}%
\pgfpathcurveto{\pgfqpoint{2.262246in}{4.713814in}}{\pgfqpoint{2.272845in}{4.709424in}}{\pgfqpoint{2.283895in}{4.709424in}}%
\pgfpathlineto{\pgfqpoint{2.283895in}{4.709424in}}%
\pgfpathclose%
\pgfusepath{stroke,fill}%
\end{pgfscope}%
\begin{pgfscope}%
\pgfpathrectangle{\pgfqpoint{0.633874in}{2.920818in}}{\pgfqpoint{2.177280in}{2.201755in}}%
\pgfusepath{clip}%
\pgfsetbuttcap%
\pgfsetroundjoin%
\definecolor{currentfill}{rgb}{0.172549,0.627451,0.172549}%
\pgfsetfillcolor{currentfill}%
\pgfsetlinewidth{0.481800pt}%
\definecolor{currentstroke}{rgb}{1.000000,1.000000,1.000000}%
\pgfsetstrokecolor{currentstroke}%
\pgfsetdash{}{0pt}%
\pgfpathmoveto{\pgfqpoint{1.724753in}{4.539797in}}%
\pgfpathcurveto{\pgfqpoint{1.735803in}{4.539797in}}{\pgfqpoint{1.746402in}{4.544188in}}{\pgfqpoint{1.754216in}{4.552001in}}%
\pgfpathcurveto{\pgfqpoint{1.762030in}{4.559815in}}{\pgfqpoint{1.766420in}{4.570414in}}{\pgfqpoint{1.766420in}{4.581464in}}%
\pgfpathcurveto{\pgfqpoint{1.766420in}{4.592514in}}{\pgfqpoint{1.762030in}{4.603113in}}{\pgfqpoint{1.754216in}{4.610927in}}%
\pgfpathcurveto{\pgfqpoint{1.746402in}{4.618740in}}{\pgfqpoint{1.735803in}{4.623131in}}{\pgfqpoint{1.724753in}{4.623131in}}%
\pgfpathcurveto{\pgfqpoint{1.713703in}{4.623131in}}{\pgfqpoint{1.703104in}{4.618740in}}{\pgfqpoint{1.695290in}{4.610927in}}%
\pgfpathcurveto{\pgfqpoint{1.687477in}{4.603113in}}{\pgfqpoint{1.683087in}{4.592514in}}{\pgfqpoint{1.683087in}{4.581464in}}%
\pgfpathcurveto{\pgfqpoint{1.683087in}{4.570414in}}{\pgfqpoint{1.687477in}{4.559815in}}{\pgfqpoint{1.695290in}{4.552001in}}%
\pgfpathcurveto{\pgfqpoint{1.703104in}{4.544188in}}{\pgfqpoint{1.713703in}{4.539797in}}{\pgfqpoint{1.724753in}{4.539797in}}%
\pgfpathlineto{\pgfqpoint{1.724753in}{4.539797in}}%
\pgfpathclose%
\pgfusepath{stroke,fill}%
\end{pgfscope}%
\begin{pgfscope}%
\pgfpathrectangle{\pgfqpoint{0.633874in}{2.920818in}}{\pgfqpoint{2.177280in}{2.201755in}}%
\pgfusepath{clip}%
\pgfsetbuttcap%
\pgfsetroundjoin%
\definecolor{currentfill}{rgb}{0.172549,0.627451,0.172549}%
\pgfsetfillcolor{currentfill}%
\pgfsetlinewidth{0.481800pt}%
\definecolor{currentstroke}{rgb}{1.000000,1.000000,1.000000}%
\pgfsetstrokecolor{currentstroke}%
\pgfsetdash{}{0pt}%
\pgfpathmoveto{\pgfqpoint{1.764692in}{4.505872in}}%
\pgfpathcurveto{\pgfqpoint{1.775742in}{4.505872in}}{\pgfqpoint{1.786341in}{4.510262in}}{\pgfqpoint{1.794155in}{4.518076in}}%
\pgfpathcurveto{\pgfqpoint{1.801968in}{4.525890in}}{\pgfqpoint{1.806359in}{4.536489in}}{\pgfqpoint{1.806359in}{4.547539in}}%
\pgfpathcurveto{\pgfqpoint{1.806359in}{4.558589in}}{\pgfqpoint{1.801968in}{4.569188in}}{\pgfqpoint{1.794155in}{4.577002in}}%
\pgfpathcurveto{\pgfqpoint{1.786341in}{4.584815in}}{\pgfqpoint{1.775742in}{4.589205in}}{\pgfqpoint{1.764692in}{4.589205in}}%
\pgfpathcurveto{\pgfqpoint{1.753642in}{4.589205in}}{\pgfqpoint{1.743043in}{4.584815in}}{\pgfqpoint{1.735229in}{4.577002in}}%
\pgfpathcurveto{\pgfqpoint{1.727416in}{4.569188in}}{\pgfqpoint{1.723025in}{4.558589in}}{\pgfqpoint{1.723025in}{4.547539in}}%
\pgfpathcurveto{\pgfqpoint{1.723025in}{4.536489in}}{\pgfqpoint{1.727416in}{4.525890in}}{\pgfqpoint{1.735229in}{4.518076in}}%
\pgfpathcurveto{\pgfqpoint{1.743043in}{4.510262in}}{\pgfqpoint{1.753642in}{4.505872in}}{\pgfqpoint{1.764692in}{4.505872in}}%
\pgfpathlineto{\pgfqpoint{1.764692in}{4.505872in}}%
\pgfpathclose%
\pgfusepath{stroke,fill}%
\end{pgfscope}%
\begin{pgfscope}%
\pgfpathrectangle{\pgfqpoint{0.633874in}{2.920818in}}{\pgfqpoint{2.177280in}{2.201755in}}%
\pgfusepath{clip}%
\pgfsetbuttcap%
\pgfsetroundjoin%
\definecolor{currentfill}{rgb}{0.172549,0.627451,0.172549}%
\pgfsetfillcolor{currentfill}%
\pgfsetlinewidth{0.481800pt}%
\definecolor{currentstroke}{rgb}{1.000000,1.000000,1.000000}%
\pgfsetstrokecolor{currentstroke}%
\pgfsetdash{}{0pt}%
\pgfpathmoveto{\pgfqpoint{1.604937in}{4.268395in}}%
\pgfpathcurveto{\pgfqpoint{1.615987in}{4.268395in}}{\pgfqpoint{1.626586in}{4.272785in}}{\pgfqpoint{1.634400in}{4.280599in}}%
\pgfpathcurveto{\pgfqpoint{1.642214in}{4.288412in}}{\pgfqpoint{1.646604in}{4.299011in}}{\pgfqpoint{1.646604in}{4.310061in}}%
\pgfpathcurveto{\pgfqpoint{1.646604in}{4.321111in}}{\pgfqpoint{1.642214in}{4.331710in}}{\pgfqpoint{1.634400in}{4.339524in}}%
\pgfpathcurveto{\pgfqpoint{1.626586in}{4.347338in}}{\pgfqpoint{1.615987in}{4.351728in}}{\pgfqpoint{1.604937in}{4.351728in}}%
\pgfpathcurveto{\pgfqpoint{1.593887in}{4.351728in}}{\pgfqpoint{1.583288in}{4.347338in}}{\pgfqpoint{1.575474in}{4.339524in}}%
\pgfpathcurveto{\pgfqpoint{1.567661in}{4.331710in}}{\pgfqpoint{1.563270in}{4.321111in}}{\pgfqpoint{1.563270in}{4.310061in}}%
\pgfpathcurveto{\pgfqpoint{1.563270in}{4.299011in}}{\pgfqpoint{1.567661in}{4.288412in}}{\pgfqpoint{1.575474in}{4.280599in}}%
\pgfpathcurveto{\pgfqpoint{1.583288in}{4.272785in}}{\pgfqpoint{1.593887in}{4.268395in}}{\pgfqpoint{1.604937in}{4.268395in}}%
\pgfpathlineto{\pgfqpoint{1.604937in}{4.268395in}}%
\pgfpathclose%
\pgfusepath{stroke,fill}%
\end{pgfscope}%
\begin{pgfscope}%
\pgfpathrectangle{\pgfqpoint{0.633874in}{2.920818in}}{\pgfqpoint{2.177280in}{2.201755in}}%
\pgfusepath{clip}%
\pgfsetbuttcap%
\pgfsetroundjoin%
\definecolor{currentfill}{rgb}{0.172549,0.627451,0.172549}%
\pgfsetfillcolor{currentfill}%
\pgfsetlinewidth{0.481800pt}%
\definecolor{currentstroke}{rgb}{1.000000,1.000000,1.000000}%
\pgfsetstrokecolor{currentstroke}%
\pgfsetdash{}{0pt}%
\pgfpathmoveto{\pgfqpoint{1.964385in}{4.471947in}}%
\pgfpathcurveto{\pgfqpoint{1.975436in}{4.471947in}}{\pgfqpoint{1.986035in}{4.476337in}}{\pgfqpoint{1.993848in}{4.484151in}}%
\pgfpathcurveto{\pgfqpoint{2.001662in}{4.491964in}}{\pgfqpoint{2.006052in}{4.502563in}}{\pgfqpoint{2.006052in}{4.513613in}}%
\pgfpathcurveto{\pgfqpoint{2.006052in}{4.524664in}}{\pgfqpoint{2.001662in}{4.535263in}}{\pgfqpoint{1.993848in}{4.543076in}}%
\pgfpathcurveto{\pgfqpoint{1.986035in}{4.550890in}}{\pgfqpoint{1.975436in}{4.555280in}}{\pgfqpoint{1.964385in}{4.555280in}}%
\pgfpathcurveto{\pgfqpoint{1.953335in}{4.555280in}}{\pgfqpoint{1.942736in}{4.550890in}}{\pgfqpoint{1.934923in}{4.543076in}}%
\pgfpathcurveto{\pgfqpoint{1.927109in}{4.535263in}}{\pgfqpoint{1.922719in}{4.524664in}}{\pgfqpoint{1.922719in}{4.513613in}}%
\pgfpathcurveto{\pgfqpoint{1.922719in}{4.502563in}}{\pgfqpoint{1.927109in}{4.491964in}}{\pgfqpoint{1.934923in}{4.484151in}}%
\pgfpathcurveto{\pgfqpoint{1.942736in}{4.476337in}}{\pgfqpoint{1.953335in}{4.471947in}}{\pgfqpoint{1.964385in}{4.471947in}}%
\pgfpathlineto{\pgfqpoint{1.964385in}{4.471947in}}%
\pgfpathclose%
\pgfusepath{stroke,fill}%
\end{pgfscope}%
\begin{pgfscope}%
\pgfpathrectangle{\pgfqpoint{0.633874in}{2.920818in}}{\pgfqpoint{2.177280in}{2.201755in}}%
\pgfusepath{clip}%
\pgfsetbuttcap%
\pgfsetroundjoin%
\definecolor{currentfill}{rgb}{0.172549,0.627451,0.172549}%
\pgfsetfillcolor{currentfill}%
\pgfsetlinewidth{0.481800pt}%
\definecolor{currentstroke}{rgb}{1.000000,1.000000,1.000000}%
\pgfsetstrokecolor{currentstroke}%
\pgfsetdash{}{0pt}%
\pgfpathmoveto{\pgfqpoint{1.884508in}{4.539797in}}%
\pgfpathcurveto{\pgfqpoint{1.895558in}{4.539797in}}{\pgfqpoint{1.906157in}{4.544188in}}{\pgfqpoint{1.913971in}{4.552001in}}%
\pgfpathcurveto{\pgfqpoint{1.921784in}{4.559815in}}{\pgfqpoint{1.926175in}{4.570414in}}{\pgfqpoint{1.926175in}{4.581464in}}%
\pgfpathcurveto{\pgfqpoint{1.926175in}{4.592514in}}{\pgfqpoint{1.921784in}{4.603113in}}{\pgfqpoint{1.913971in}{4.610927in}}%
\pgfpathcurveto{\pgfqpoint{1.906157in}{4.618740in}}{\pgfqpoint{1.895558in}{4.623131in}}{\pgfqpoint{1.884508in}{4.623131in}}%
\pgfpathcurveto{\pgfqpoint{1.873458in}{4.623131in}}{\pgfqpoint{1.862859in}{4.618740in}}{\pgfqpoint{1.855045in}{4.610927in}}%
\pgfpathcurveto{\pgfqpoint{1.847232in}{4.603113in}}{\pgfqpoint{1.842841in}{4.592514in}}{\pgfqpoint{1.842841in}{4.581464in}}%
\pgfpathcurveto{\pgfqpoint{1.842841in}{4.570414in}}{\pgfqpoint{1.847232in}{4.559815in}}{\pgfqpoint{1.855045in}{4.552001in}}%
\pgfpathcurveto{\pgfqpoint{1.862859in}{4.544188in}}{\pgfqpoint{1.873458in}{4.539797in}}{\pgfqpoint{1.884508in}{4.539797in}}%
\pgfpathlineto{\pgfqpoint{1.884508in}{4.539797in}}%
\pgfpathclose%
\pgfusepath{stroke,fill}%
\end{pgfscope}%
\begin{pgfscope}%
\pgfpathrectangle{\pgfqpoint{0.633874in}{2.920818in}}{\pgfqpoint{2.177280in}{2.201755in}}%
\pgfusepath{clip}%
\pgfsetbuttcap%
\pgfsetroundjoin%
\definecolor{currentfill}{rgb}{0.172549,0.627451,0.172549}%
\pgfsetfillcolor{currentfill}%
\pgfsetlinewidth{0.481800pt}%
\definecolor{currentstroke}{rgb}{1.000000,1.000000,1.000000}%
\pgfsetstrokecolor{currentstroke}%
\pgfsetdash{}{0pt}%
\pgfpathmoveto{\pgfqpoint{1.964385in}{4.370171in}}%
\pgfpathcurveto{\pgfqpoint{1.975436in}{4.370171in}}{\pgfqpoint{1.986035in}{4.374561in}}{\pgfqpoint{1.993848in}{4.382375in}}%
\pgfpathcurveto{\pgfqpoint{2.001662in}{4.390188in}}{\pgfqpoint{2.006052in}{4.400787in}}{\pgfqpoint{2.006052in}{4.411837in}}%
\pgfpathcurveto{\pgfqpoint{2.006052in}{4.422887in}}{\pgfqpoint{2.001662in}{4.433486in}}{\pgfqpoint{1.993848in}{4.441300in}}%
\pgfpathcurveto{\pgfqpoint{1.986035in}{4.449114in}}{\pgfqpoint{1.975436in}{4.453504in}}{\pgfqpoint{1.964385in}{4.453504in}}%
\pgfpathcurveto{\pgfqpoint{1.953335in}{4.453504in}}{\pgfqpoint{1.942736in}{4.449114in}}{\pgfqpoint{1.934923in}{4.441300in}}%
\pgfpathcurveto{\pgfqpoint{1.927109in}{4.433486in}}{\pgfqpoint{1.922719in}{4.422887in}}{\pgfqpoint{1.922719in}{4.411837in}}%
\pgfpathcurveto{\pgfqpoint{1.922719in}{4.400787in}}{\pgfqpoint{1.927109in}{4.390188in}}{\pgfqpoint{1.934923in}{4.382375in}}%
\pgfpathcurveto{\pgfqpoint{1.942736in}{4.374561in}}{\pgfqpoint{1.953335in}{4.370171in}}{\pgfqpoint{1.964385in}{4.370171in}}%
\pgfpathlineto{\pgfqpoint{1.964385in}{4.370171in}}%
\pgfpathclose%
\pgfusepath{stroke,fill}%
\end{pgfscope}%
\begin{pgfscope}%
\pgfpathrectangle{\pgfqpoint{0.633874in}{2.920818in}}{\pgfqpoint{2.177280in}{2.201755in}}%
\pgfusepath{clip}%
\pgfsetbuttcap%
\pgfsetroundjoin%
\definecolor{currentfill}{rgb}{0.172549,0.627451,0.172549}%
\pgfsetfillcolor{currentfill}%
\pgfsetlinewidth{0.481800pt}%
\definecolor{currentstroke}{rgb}{1.000000,1.000000,1.000000}%
\pgfsetstrokecolor{currentstroke}%
\pgfsetdash{}{0pt}%
\pgfpathmoveto{\pgfqpoint{1.525060in}{4.370171in}}%
\pgfpathcurveto{\pgfqpoint{1.536110in}{4.370171in}}{\pgfqpoint{1.546709in}{4.374561in}}{\pgfqpoint{1.554523in}{4.382375in}}%
\pgfpathcurveto{\pgfqpoint{1.562336in}{4.390188in}}{\pgfqpoint{1.566726in}{4.400787in}}{\pgfqpoint{1.566726in}{4.411837in}}%
\pgfpathcurveto{\pgfqpoint{1.566726in}{4.422887in}}{\pgfqpoint{1.562336in}{4.433486in}}{\pgfqpoint{1.554523in}{4.441300in}}%
\pgfpathcurveto{\pgfqpoint{1.546709in}{4.449114in}}{\pgfqpoint{1.536110in}{4.453504in}}{\pgfqpoint{1.525060in}{4.453504in}}%
\pgfpathcurveto{\pgfqpoint{1.514010in}{4.453504in}}{\pgfqpoint{1.503411in}{4.449114in}}{\pgfqpoint{1.495597in}{4.441300in}}%
\pgfpathcurveto{\pgfqpoint{1.487783in}{4.433486in}}{\pgfqpoint{1.483393in}{4.422887in}}{\pgfqpoint{1.483393in}{4.411837in}}%
\pgfpathcurveto{\pgfqpoint{1.483393in}{4.400787in}}{\pgfqpoint{1.487783in}{4.390188in}}{\pgfqpoint{1.495597in}{4.382375in}}%
\pgfpathcurveto{\pgfqpoint{1.503411in}{4.374561in}}{\pgfqpoint{1.514010in}{4.370171in}}{\pgfqpoint{1.525060in}{4.370171in}}%
\pgfpathlineto{\pgfqpoint{1.525060in}{4.370171in}}%
\pgfpathclose%
\pgfusepath{stroke,fill}%
\end{pgfscope}%
\begin{pgfscope}%
\pgfpathrectangle{\pgfqpoint{0.633874in}{2.920818in}}{\pgfqpoint{2.177280in}{2.201755in}}%
\pgfusepath{clip}%
\pgfsetbuttcap%
\pgfsetroundjoin%
\definecolor{currentfill}{rgb}{0.172549,0.627451,0.172549}%
\pgfsetfillcolor{currentfill}%
\pgfsetlinewidth{0.481800pt}%
\definecolor{currentstroke}{rgb}{1.000000,1.000000,1.000000}%
\pgfsetstrokecolor{currentstroke}%
\pgfsetdash{}{0pt}%
\pgfpathmoveto{\pgfqpoint{1.924447in}{4.641573in}}%
\pgfpathcurveto{\pgfqpoint{1.935497in}{4.641573in}}{\pgfqpoint{1.946096in}{4.645964in}}{\pgfqpoint{1.953910in}{4.653777in}}%
\pgfpathcurveto{\pgfqpoint{1.961723in}{4.661591in}}{\pgfqpoint{1.966113in}{4.672190in}}{\pgfqpoint{1.966113in}{4.683240in}}%
\pgfpathcurveto{\pgfqpoint{1.966113in}{4.694290in}}{\pgfqpoint{1.961723in}{4.704889in}}{\pgfqpoint{1.953910in}{4.712703in}}%
\pgfpathcurveto{\pgfqpoint{1.946096in}{4.720517in}}{\pgfqpoint{1.935497in}{4.724907in}}{\pgfqpoint{1.924447in}{4.724907in}}%
\pgfpathcurveto{\pgfqpoint{1.913397in}{4.724907in}}{\pgfqpoint{1.902798in}{4.720517in}}{\pgfqpoint{1.894984in}{4.712703in}}%
\pgfpathcurveto{\pgfqpoint{1.887170in}{4.704889in}}{\pgfqpoint{1.882780in}{4.694290in}}{\pgfqpoint{1.882780in}{4.683240in}}%
\pgfpathcurveto{\pgfqpoint{1.882780in}{4.672190in}}{\pgfqpoint{1.887170in}{4.661591in}}{\pgfqpoint{1.894984in}{4.653777in}}%
\pgfpathcurveto{\pgfqpoint{1.902798in}{4.645964in}}{\pgfqpoint{1.913397in}{4.641573in}}{\pgfqpoint{1.924447in}{4.641573in}}%
\pgfpathlineto{\pgfqpoint{1.924447in}{4.641573in}}%
\pgfpathclose%
\pgfusepath{stroke,fill}%
\end{pgfscope}%
\begin{pgfscope}%
\pgfpathrectangle{\pgfqpoint{0.633874in}{2.920818in}}{\pgfqpoint{2.177280in}{2.201755in}}%
\pgfusepath{clip}%
\pgfsetbuttcap%
\pgfsetroundjoin%
\definecolor{currentfill}{rgb}{0.172549,0.627451,0.172549}%
\pgfsetfillcolor{currentfill}%
\pgfsetlinewidth{0.481800pt}%
\definecolor{currentstroke}{rgb}{1.000000,1.000000,1.000000}%
\pgfsetstrokecolor{currentstroke}%
\pgfsetdash{}{0pt}%
\pgfpathmoveto{\pgfqpoint{1.884508in}{4.573723in}}%
\pgfpathcurveto{\pgfqpoint{1.895558in}{4.573723in}}{\pgfqpoint{1.906157in}{4.578113in}}{\pgfqpoint{1.913971in}{4.585927in}}%
\pgfpathcurveto{\pgfqpoint{1.921784in}{4.593740in}}{\pgfqpoint{1.926175in}{4.604339in}}{\pgfqpoint{1.926175in}{4.615389in}}%
\pgfpathcurveto{\pgfqpoint{1.926175in}{4.626440in}}{\pgfqpoint{1.921784in}{4.637039in}}{\pgfqpoint{1.913971in}{4.644852in}}%
\pgfpathcurveto{\pgfqpoint{1.906157in}{4.652666in}}{\pgfqpoint{1.895558in}{4.657056in}}{\pgfqpoint{1.884508in}{4.657056in}}%
\pgfpathcurveto{\pgfqpoint{1.873458in}{4.657056in}}{\pgfqpoint{1.862859in}{4.652666in}}{\pgfqpoint{1.855045in}{4.644852in}}%
\pgfpathcurveto{\pgfqpoint{1.847232in}{4.637039in}}{\pgfqpoint{1.842841in}{4.626440in}}{\pgfqpoint{1.842841in}{4.615389in}}%
\pgfpathcurveto{\pgfqpoint{1.842841in}{4.604339in}}{\pgfqpoint{1.847232in}{4.593740in}}{\pgfqpoint{1.855045in}{4.585927in}}%
\pgfpathcurveto{\pgfqpoint{1.862859in}{4.578113in}}{\pgfqpoint{1.873458in}{4.573723in}}{\pgfqpoint{1.884508in}{4.573723in}}%
\pgfpathlineto{\pgfqpoint{1.884508in}{4.573723in}}%
\pgfpathclose%
\pgfusepath{stroke,fill}%
\end{pgfscope}%
\begin{pgfscope}%
\pgfpathrectangle{\pgfqpoint{0.633874in}{2.920818in}}{\pgfqpoint{2.177280in}{2.201755in}}%
\pgfusepath{clip}%
\pgfsetbuttcap%
\pgfsetroundjoin%
\definecolor{currentfill}{rgb}{0.172549,0.627451,0.172549}%
\pgfsetfillcolor{currentfill}%
\pgfsetlinewidth{0.481800pt}%
\definecolor{currentstroke}{rgb}{1.000000,1.000000,1.000000}%
\pgfsetstrokecolor{currentstroke}%
\pgfsetdash{}{0pt}%
\pgfpathmoveto{\pgfqpoint{1.884508in}{4.404096in}}%
\pgfpathcurveto{\pgfqpoint{1.895558in}{4.404096in}}{\pgfqpoint{1.906157in}{4.408486in}}{\pgfqpoint{1.913971in}{4.416300in}}%
\pgfpathcurveto{\pgfqpoint{1.921784in}{4.424114in}}{\pgfqpoint{1.926175in}{4.434713in}}{\pgfqpoint{1.926175in}{4.445763in}}%
\pgfpathcurveto{\pgfqpoint{1.926175in}{4.456813in}}{\pgfqpoint{1.921784in}{4.467412in}}{\pgfqpoint{1.913971in}{4.475225in}}%
\pgfpathcurveto{\pgfqpoint{1.906157in}{4.483039in}}{\pgfqpoint{1.895558in}{4.487429in}}{\pgfqpoint{1.884508in}{4.487429in}}%
\pgfpathcurveto{\pgfqpoint{1.873458in}{4.487429in}}{\pgfqpoint{1.862859in}{4.483039in}}{\pgfqpoint{1.855045in}{4.475225in}}%
\pgfpathcurveto{\pgfqpoint{1.847232in}{4.467412in}}{\pgfqpoint{1.842841in}{4.456813in}}{\pgfqpoint{1.842841in}{4.445763in}}%
\pgfpathcurveto{\pgfqpoint{1.842841in}{4.434713in}}{\pgfqpoint{1.847232in}{4.424114in}}{\pgfqpoint{1.855045in}{4.416300in}}%
\pgfpathcurveto{\pgfqpoint{1.862859in}{4.408486in}}{\pgfqpoint{1.873458in}{4.404096in}}{\pgfqpoint{1.884508in}{4.404096in}}%
\pgfpathlineto{\pgfqpoint{1.884508in}{4.404096in}}%
\pgfpathclose%
\pgfusepath{stroke,fill}%
\end{pgfscope}%
\begin{pgfscope}%
\pgfpathrectangle{\pgfqpoint{0.633874in}{2.920818in}}{\pgfqpoint{2.177280in}{2.201755in}}%
\pgfusepath{clip}%
\pgfsetbuttcap%
\pgfsetroundjoin%
\definecolor{currentfill}{rgb}{0.172549,0.627451,0.172549}%
\pgfsetfillcolor{currentfill}%
\pgfsetlinewidth{0.481800pt}%
\definecolor{currentstroke}{rgb}{1.000000,1.000000,1.000000}%
\pgfsetstrokecolor{currentstroke}%
\pgfsetdash{}{0pt}%
\pgfpathmoveto{\pgfqpoint{1.724753in}{4.336245in}}%
\pgfpathcurveto{\pgfqpoint{1.735803in}{4.336245in}}{\pgfqpoint{1.746402in}{4.340636in}}{\pgfqpoint{1.754216in}{4.348449in}}%
\pgfpathcurveto{\pgfqpoint{1.762030in}{4.356263in}}{\pgfqpoint{1.766420in}{4.366862in}}{\pgfqpoint{1.766420in}{4.377912in}}%
\pgfpathcurveto{\pgfqpoint{1.766420in}{4.388962in}}{\pgfqpoint{1.762030in}{4.399561in}}{\pgfqpoint{1.754216in}{4.407375in}}%
\pgfpathcurveto{\pgfqpoint{1.746402in}{4.415188in}}{\pgfqpoint{1.735803in}{4.419579in}}{\pgfqpoint{1.724753in}{4.419579in}}%
\pgfpathcurveto{\pgfqpoint{1.713703in}{4.419579in}}{\pgfqpoint{1.703104in}{4.415188in}}{\pgfqpoint{1.695290in}{4.407375in}}%
\pgfpathcurveto{\pgfqpoint{1.687477in}{4.399561in}}{\pgfqpoint{1.683087in}{4.388962in}}{\pgfqpoint{1.683087in}{4.377912in}}%
\pgfpathcurveto{\pgfqpoint{1.683087in}{4.366862in}}{\pgfqpoint{1.687477in}{4.356263in}}{\pgfqpoint{1.695290in}{4.348449in}}%
\pgfpathcurveto{\pgfqpoint{1.703104in}{4.340636in}}{\pgfqpoint{1.713703in}{4.336245in}}{\pgfqpoint{1.724753in}{4.336245in}}%
\pgfpathlineto{\pgfqpoint{1.724753in}{4.336245in}}%
\pgfpathclose%
\pgfusepath{stroke,fill}%
\end{pgfscope}%
\begin{pgfscope}%
\pgfpathrectangle{\pgfqpoint{0.633874in}{2.920818in}}{\pgfqpoint{2.177280in}{2.201755in}}%
\pgfusepath{clip}%
\pgfsetbuttcap%
\pgfsetroundjoin%
\definecolor{currentfill}{rgb}{0.172549,0.627451,0.172549}%
\pgfsetfillcolor{currentfill}%
\pgfsetlinewidth{0.481800pt}%
\definecolor{currentstroke}{rgb}{1.000000,1.000000,1.000000}%
\pgfsetstrokecolor{currentstroke}%
\pgfsetdash{}{0pt}%
\pgfpathmoveto{\pgfqpoint{1.804631in}{4.404096in}}%
\pgfpathcurveto{\pgfqpoint{1.815681in}{4.404096in}}{\pgfqpoint{1.826280in}{4.408486in}}{\pgfqpoint{1.834093in}{4.416300in}}%
\pgfpathcurveto{\pgfqpoint{1.841907in}{4.424114in}}{\pgfqpoint{1.846297in}{4.434713in}}{\pgfqpoint{1.846297in}{4.445763in}}%
\pgfpathcurveto{\pgfqpoint{1.846297in}{4.456813in}}{\pgfqpoint{1.841907in}{4.467412in}}{\pgfqpoint{1.834093in}{4.475225in}}%
\pgfpathcurveto{\pgfqpoint{1.826280in}{4.483039in}}{\pgfqpoint{1.815681in}{4.487429in}}{\pgfqpoint{1.804631in}{4.487429in}}%
\pgfpathcurveto{\pgfqpoint{1.793581in}{4.487429in}}{\pgfqpoint{1.782981in}{4.483039in}}{\pgfqpoint{1.775168in}{4.475225in}}%
\pgfpathcurveto{\pgfqpoint{1.767354in}{4.467412in}}{\pgfqpoint{1.762964in}{4.456813in}}{\pgfqpoint{1.762964in}{4.445763in}}%
\pgfpathcurveto{\pgfqpoint{1.762964in}{4.434713in}}{\pgfqpoint{1.767354in}{4.424114in}}{\pgfqpoint{1.775168in}{4.416300in}}%
\pgfpathcurveto{\pgfqpoint{1.782981in}{4.408486in}}{\pgfqpoint{1.793581in}{4.404096in}}{\pgfqpoint{1.804631in}{4.404096in}}%
\pgfpathlineto{\pgfqpoint{1.804631in}{4.404096in}}%
\pgfpathclose%
\pgfusepath{stroke,fill}%
\end{pgfscope}%
\begin{pgfscope}%
\pgfpathrectangle{\pgfqpoint{0.633874in}{2.920818in}}{\pgfqpoint{2.177280in}{2.201755in}}%
\pgfusepath{clip}%
\pgfsetbuttcap%
\pgfsetroundjoin%
\definecolor{currentfill}{rgb}{0.172549,0.627451,0.172549}%
\pgfsetfillcolor{currentfill}%
\pgfsetlinewidth{0.481800pt}%
\definecolor{currentstroke}{rgb}{1.000000,1.000000,1.000000}%
\pgfsetstrokecolor{currentstroke}%
\pgfsetdash{}{0pt}%
\pgfpathmoveto{\pgfqpoint{1.684815in}{4.471947in}}%
\pgfpathcurveto{\pgfqpoint{1.695865in}{4.471947in}}{\pgfqpoint{1.706464in}{4.476337in}}{\pgfqpoint{1.714277in}{4.484151in}}%
\pgfpathcurveto{\pgfqpoint{1.722091in}{4.491964in}}{\pgfqpoint{1.726481in}{4.502563in}}{\pgfqpoint{1.726481in}{4.513613in}}%
\pgfpathcurveto{\pgfqpoint{1.726481in}{4.524664in}}{\pgfqpoint{1.722091in}{4.535263in}}{\pgfqpoint{1.714277in}{4.543076in}}%
\pgfpathcurveto{\pgfqpoint{1.706464in}{4.550890in}}{\pgfqpoint{1.695865in}{4.555280in}}{\pgfqpoint{1.684815in}{4.555280in}}%
\pgfpathcurveto{\pgfqpoint{1.673764in}{4.555280in}}{\pgfqpoint{1.663165in}{4.550890in}}{\pgfqpoint{1.655352in}{4.543076in}}%
\pgfpathcurveto{\pgfqpoint{1.647538in}{4.535263in}}{\pgfqpoint{1.643148in}{4.524664in}}{\pgfqpoint{1.643148in}{4.513613in}}%
\pgfpathcurveto{\pgfqpoint{1.643148in}{4.502563in}}{\pgfqpoint{1.647538in}{4.491964in}}{\pgfqpoint{1.655352in}{4.484151in}}%
\pgfpathcurveto{\pgfqpoint{1.663165in}{4.476337in}}{\pgfqpoint{1.673764in}{4.471947in}}{\pgfqpoint{1.684815in}{4.471947in}}%
\pgfpathlineto{\pgfqpoint{1.684815in}{4.471947in}}%
\pgfpathclose%
\pgfusepath{stroke,fill}%
\end{pgfscope}%
\begin{pgfscope}%
\pgfpathrectangle{\pgfqpoint{0.633874in}{2.920818in}}{\pgfqpoint{2.177280in}{2.201755in}}%
\pgfusepath{clip}%
\pgfsetbuttcap%
\pgfsetroundjoin%
\definecolor{currentfill}{rgb}{0.172549,0.627451,0.172549}%
\pgfsetfillcolor{currentfill}%
\pgfsetlinewidth{0.481800pt}%
\definecolor{currentstroke}{rgb}{1.000000,1.000000,1.000000}%
\pgfsetstrokecolor{currentstroke}%
\pgfsetdash{}{0pt}%
\pgfpathmoveto{\pgfqpoint{1.564998in}{4.370171in}}%
\pgfpathcurveto{\pgfqpoint{1.576049in}{4.370171in}}{\pgfqpoint{1.586648in}{4.374561in}}{\pgfqpoint{1.594461in}{4.382375in}}%
\pgfpathcurveto{\pgfqpoint{1.602275in}{4.390188in}}{\pgfqpoint{1.606665in}{4.400787in}}{\pgfqpoint{1.606665in}{4.411837in}}%
\pgfpathcurveto{\pgfqpoint{1.606665in}{4.422887in}}{\pgfqpoint{1.602275in}{4.433486in}}{\pgfqpoint{1.594461in}{4.441300in}}%
\pgfpathcurveto{\pgfqpoint{1.586648in}{4.449114in}}{\pgfqpoint{1.576049in}{4.453504in}}{\pgfqpoint{1.564998in}{4.453504in}}%
\pgfpathcurveto{\pgfqpoint{1.553948in}{4.453504in}}{\pgfqpoint{1.543349in}{4.449114in}}{\pgfqpoint{1.535536in}{4.441300in}}%
\pgfpathcurveto{\pgfqpoint{1.527722in}{4.433486in}}{\pgfqpoint{1.523332in}{4.422887in}}{\pgfqpoint{1.523332in}{4.411837in}}%
\pgfpathcurveto{\pgfqpoint{1.523332in}{4.400787in}}{\pgfqpoint{1.527722in}{4.390188in}}{\pgfqpoint{1.535536in}{4.382375in}}%
\pgfpathcurveto{\pgfqpoint{1.543349in}{4.374561in}}{\pgfqpoint{1.553948in}{4.370171in}}{\pgfqpoint{1.564998in}{4.370171in}}%
\pgfpathlineto{\pgfqpoint{1.564998in}{4.370171in}}%
\pgfpathclose%
\pgfusepath{stroke,fill}%
\end{pgfscope}%
\begin{pgfscope}%
\pgfpathrectangle{\pgfqpoint{0.633874in}{2.920818in}}{\pgfqpoint{2.177280in}{2.201755in}}%
\pgfusepath{clip}%
\pgfsetbuttcap%
\pgfsetroundjoin%
\definecolor{currentfill}{rgb}{0.121569,0.466667,0.705882}%
\pgfsetfillcolor{currentfill}%
\pgfsetlinewidth{1.003750pt}%
\definecolor{currentstroke}{rgb}{0.121569,0.466667,0.705882}%
\pgfsetstrokecolor{currentstroke}%
\pgfsetdash{}{0pt}%
\pgfsys@defobject{currentmarker}{\pgfqpoint{-0.041667in}{-0.041667in}}{\pgfqpoint{0.041667in}{0.041667in}}{%
\pgfpathmoveto{\pgfqpoint{0.000000in}{-0.041667in}}%
\pgfpathcurveto{\pgfqpoint{0.011050in}{-0.041667in}}{\pgfqpoint{0.021649in}{-0.037276in}}{\pgfqpoint{0.029463in}{-0.029463in}}%
\pgfpathcurveto{\pgfqpoint{0.037276in}{-0.021649in}}{\pgfqpoint{0.041667in}{-0.011050in}}{\pgfqpoint{0.041667in}{0.000000in}}%
\pgfpathcurveto{\pgfqpoint{0.041667in}{0.011050in}}{\pgfqpoint{0.037276in}{0.021649in}}{\pgfqpoint{0.029463in}{0.029463in}}%
\pgfpathcurveto{\pgfqpoint{0.021649in}{0.037276in}}{\pgfqpoint{0.011050in}{0.041667in}}{\pgfqpoint{0.000000in}{0.041667in}}%
\pgfpathcurveto{\pgfqpoint{-0.011050in}{0.041667in}}{\pgfqpoint{-0.021649in}{0.037276in}}{\pgfqpoint{-0.029463in}{0.029463in}}%
\pgfpathcurveto{\pgfqpoint{-0.037276in}{0.021649in}}{\pgfqpoint{-0.041667in}{0.011050in}}{\pgfqpoint{-0.041667in}{0.000000in}}%
\pgfpathcurveto{\pgfqpoint{-0.041667in}{-0.011050in}}{\pgfqpoint{-0.037276in}{-0.021649in}}{\pgfqpoint{-0.029463in}{-0.029463in}}%
\pgfpathcurveto{\pgfqpoint{-0.021649in}{-0.037276in}}{\pgfqpoint{-0.011050in}{-0.041667in}}{\pgfqpoint{0.000000in}{-0.041667in}}%
\pgfpathlineto{\pgfqpoint{0.000000in}{-0.041667in}}%
\pgfpathclose%
\pgfusepath{stroke,fill}%
}%
\end{pgfscope}%
\begin{pgfscope}%
\pgfpathrectangle{\pgfqpoint{0.633874in}{2.920818in}}{\pgfqpoint{2.177280in}{2.201755in}}%
\pgfusepath{clip}%
\pgfsetbuttcap%
\pgfsetroundjoin%
\definecolor{currentfill}{rgb}{1.000000,0.498039,0.054902}%
\pgfsetfillcolor{currentfill}%
\pgfsetlinewidth{1.003750pt}%
\definecolor{currentstroke}{rgb}{1.000000,0.498039,0.054902}%
\pgfsetstrokecolor{currentstroke}%
\pgfsetdash{}{0pt}%
\pgfsys@defobject{currentmarker}{\pgfqpoint{-0.041667in}{-0.041667in}}{\pgfqpoint{0.041667in}{0.041667in}}{%
\pgfpathmoveto{\pgfqpoint{0.000000in}{-0.041667in}}%
\pgfpathcurveto{\pgfqpoint{0.011050in}{-0.041667in}}{\pgfqpoint{0.021649in}{-0.037276in}}{\pgfqpoint{0.029463in}{-0.029463in}}%
\pgfpathcurveto{\pgfqpoint{0.037276in}{-0.021649in}}{\pgfqpoint{0.041667in}{-0.011050in}}{\pgfqpoint{0.041667in}{0.000000in}}%
\pgfpathcurveto{\pgfqpoint{0.041667in}{0.011050in}}{\pgfqpoint{0.037276in}{0.021649in}}{\pgfqpoint{0.029463in}{0.029463in}}%
\pgfpathcurveto{\pgfqpoint{0.021649in}{0.037276in}}{\pgfqpoint{0.011050in}{0.041667in}}{\pgfqpoint{0.000000in}{0.041667in}}%
\pgfpathcurveto{\pgfqpoint{-0.011050in}{0.041667in}}{\pgfqpoint{-0.021649in}{0.037276in}}{\pgfqpoint{-0.029463in}{0.029463in}}%
\pgfpathcurveto{\pgfqpoint{-0.037276in}{0.021649in}}{\pgfqpoint{-0.041667in}{0.011050in}}{\pgfqpoint{-0.041667in}{0.000000in}}%
\pgfpathcurveto{\pgfqpoint{-0.041667in}{-0.011050in}}{\pgfqpoint{-0.037276in}{-0.021649in}}{\pgfqpoint{-0.029463in}{-0.029463in}}%
\pgfpathcurveto{\pgfqpoint{-0.021649in}{-0.037276in}}{\pgfqpoint{-0.011050in}{-0.041667in}}{\pgfqpoint{0.000000in}{-0.041667in}}%
\pgfpathlineto{\pgfqpoint{0.000000in}{-0.041667in}}%
\pgfpathclose%
\pgfusepath{stroke,fill}%
}%
\end{pgfscope}%
\begin{pgfscope}%
\pgfpathrectangle{\pgfqpoint{0.633874in}{2.920818in}}{\pgfqpoint{2.177280in}{2.201755in}}%
\pgfusepath{clip}%
\pgfsetbuttcap%
\pgfsetroundjoin%
\definecolor{currentfill}{rgb}{0.172549,0.627451,0.172549}%
\pgfsetfillcolor{currentfill}%
\pgfsetlinewidth{1.003750pt}%
\definecolor{currentstroke}{rgb}{0.172549,0.627451,0.172549}%
\pgfsetstrokecolor{currentstroke}%
\pgfsetdash{}{0pt}%
\pgfsys@defobject{currentmarker}{\pgfqpoint{-0.041667in}{-0.041667in}}{\pgfqpoint{0.041667in}{0.041667in}}{%
\pgfpathmoveto{\pgfqpoint{0.000000in}{-0.041667in}}%
\pgfpathcurveto{\pgfqpoint{0.011050in}{-0.041667in}}{\pgfqpoint{0.021649in}{-0.037276in}}{\pgfqpoint{0.029463in}{-0.029463in}}%
\pgfpathcurveto{\pgfqpoint{0.037276in}{-0.021649in}}{\pgfqpoint{0.041667in}{-0.011050in}}{\pgfqpoint{0.041667in}{0.000000in}}%
\pgfpathcurveto{\pgfqpoint{0.041667in}{0.011050in}}{\pgfqpoint{0.037276in}{0.021649in}}{\pgfqpoint{0.029463in}{0.029463in}}%
\pgfpathcurveto{\pgfqpoint{0.021649in}{0.037276in}}{\pgfqpoint{0.011050in}{0.041667in}}{\pgfqpoint{0.000000in}{0.041667in}}%
\pgfpathcurveto{\pgfqpoint{-0.011050in}{0.041667in}}{\pgfqpoint{-0.021649in}{0.037276in}}{\pgfqpoint{-0.029463in}{0.029463in}}%
\pgfpathcurveto{\pgfqpoint{-0.037276in}{0.021649in}}{\pgfqpoint{-0.041667in}{0.011050in}}{\pgfqpoint{-0.041667in}{0.000000in}}%
\pgfpathcurveto{\pgfqpoint{-0.041667in}{-0.011050in}}{\pgfqpoint{-0.037276in}{-0.021649in}}{\pgfqpoint{-0.029463in}{-0.029463in}}%
\pgfpathcurveto{\pgfqpoint{-0.021649in}{-0.037276in}}{\pgfqpoint{-0.011050in}{-0.041667in}}{\pgfqpoint{0.000000in}{-0.041667in}}%
\pgfpathlineto{\pgfqpoint{0.000000in}{-0.041667in}}%
\pgfpathclose%
\pgfusepath{stroke,fill}%
}%
\end{pgfscope}%
\begin{pgfscope}%
\pgfsetbuttcap%
\pgfsetroundjoin%
\definecolor{currentfill}{rgb}{0.000000,0.000000,0.000000}%
\pgfsetfillcolor{currentfill}%
\pgfsetlinewidth{0.803000pt}%
\definecolor{currentstroke}{rgb}{0.000000,0.000000,0.000000}%
\pgfsetstrokecolor{currentstroke}%
\pgfsetdash{}{0pt}%
\pgfsys@defobject{currentmarker}{\pgfqpoint{0.000000in}{-0.048611in}}{\pgfqpoint{0.000000in}{0.000000in}}{%
\pgfpathmoveto{\pgfqpoint{0.000000in}{0.000000in}}%
\pgfpathlineto{\pgfqpoint{0.000000in}{-0.048611in}}%
\pgfusepath{stroke,fill}%
}%
\begin{pgfscope}%
\pgfsys@transformshift{0.806163in}{2.920818in}%
\pgfsys@useobject{currentmarker}{}%
\end{pgfscope}%
\end{pgfscope}%
\begin{pgfscope}%
\pgfsetbuttcap%
\pgfsetroundjoin%
\definecolor{currentfill}{rgb}{0.000000,0.000000,0.000000}%
\pgfsetfillcolor{currentfill}%
\pgfsetlinewidth{0.803000pt}%
\definecolor{currentstroke}{rgb}{0.000000,0.000000,0.000000}%
\pgfsetstrokecolor{currentstroke}%
\pgfsetdash{}{0pt}%
\pgfsys@defobject{currentmarker}{\pgfqpoint{0.000000in}{-0.048611in}}{\pgfqpoint{0.000000in}{0.000000in}}{%
\pgfpathmoveto{\pgfqpoint{0.000000in}{0.000000in}}%
\pgfpathlineto{\pgfqpoint{0.000000in}{-0.048611in}}%
\pgfusepath{stroke,fill}%
}%
\begin{pgfscope}%
\pgfsys@transformshift{1.604937in}{2.920818in}%
\pgfsys@useobject{currentmarker}{}%
\end{pgfscope}%
\end{pgfscope}%
\begin{pgfscope}%
\pgfsetbuttcap%
\pgfsetroundjoin%
\definecolor{currentfill}{rgb}{0.000000,0.000000,0.000000}%
\pgfsetfillcolor{currentfill}%
\pgfsetlinewidth{0.803000pt}%
\definecolor{currentstroke}{rgb}{0.000000,0.000000,0.000000}%
\pgfsetstrokecolor{currentstroke}%
\pgfsetdash{}{0pt}%
\pgfsys@defobject{currentmarker}{\pgfqpoint{0.000000in}{-0.048611in}}{\pgfqpoint{0.000000in}{0.000000in}}{%
\pgfpathmoveto{\pgfqpoint{0.000000in}{0.000000in}}%
\pgfpathlineto{\pgfqpoint{0.000000in}{-0.048611in}}%
\pgfusepath{stroke,fill}%
}%
\begin{pgfscope}%
\pgfsys@transformshift{2.403711in}{2.920818in}%
\pgfsys@useobject{currentmarker}{}%
\end{pgfscope}%
\end{pgfscope}%
\begin{pgfscope}%
\pgfsetbuttcap%
\pgfsetroundjoin%
\definecolor{currentfill}{rgb}{0.000000,0.000000,0.000000}%
\pgfsetfillcolor{currentfill}%
\pgfsetlinewidth{0.803000pt}%
\definecolor{currentstroke}{rgb}{0.000000,0.000000,0.000000}%
\pgfsetstrokecolor{currentstroke}%
\pgfsetdash{}{0pt}%
\pgfsys@defobject{currentmarker}{\pgfqpoint{-0.048611in}{0.000000in}}{\pgfqpoint{-0.000000in}{0.000000in}}{%
\pgfpathmoveto{\pgfqpoint{-0.000000in}{0.000000in}}%
\pgfpathlineto{\pgfqpoint{-0.048611in}{0.000000in}}%
\pgfusepath{stroke,fill}%
}%
\begin{pgfscope}%
\pgfsys@transformshift{0.633874in}{3.020898in}%
\pgfsys@useobject{currentmarker}{}%
\end{pgfscope}%
\end{pgfscope}%
\begin{pgfscope}%
\definecolor{textcolor}{rgb}{0.000000,0.000000,0.000000}%
\pgfsetstrokecolor{textcolor}%
\pgfsetfillcolor{textcolor}%
\pgftext[x=0.467207in, y=2.972673in, left, base]{\color{textcolor}\rmfamily\fontsize{10.000000}{12.000000}\selectfont \(\displaystyle {1}\)}%
\end{pgfscope}%
\begin{pgfscope}%
\pgfsetbuttcap%
\pgfsetroundjoin%
\definecolor{currentfill}{rgb}{0.000000,0.000000,0.000000}%
\pgfsetfillcolor{currentfill}%
\pgfsetlinewidth{0.803000pt}%
\definecolor{currentstroke}{rgb}{0.000000,0.000000,0.000000}%
\pgfsetstrokecolor{currentstroke}%
\pgfsetdash{}{0pt}%
\pgfsys@defobject{currentmarker}{\pgfqpoint{-0.048611in}{0.000000in}}{\pgfqpoint{-0.000000in}{0.000000in}}{%
\pgfpathmoveto{\pgfqpoint{-0.000000in}{0.000000in}}%
\pgfpathlineto{\pgfqpoint{-0.048611in}{0.000000in}}%
\pgfusepath{stroke,fill}%
}%
\begin{pgfscope}%
\pgfsys@transformshift{0.633874in}{3.360152in}%
\pgfsys@useobject{currentmarker}{}%
\end{pgfscope}%
\end{pgfscope}%
\begin{pgfscope}%
\definecolor{textcolor}{rgb}{0.000000,0.000000,0.000000}%
\pgfsetstrokecolor{textcolor}%
\pgfsetfillcolor{textcolor}%
\pgftext[x=0.467207in, y=3.311926in, left, base]{\color{textcolor}\rmfamily\fontsize{10.000000}{12.000000}\selectfont \(\displaystyle {2}\)}%
\end{pgfscope}%
\begin{pgfscope}%
\pgfsetbuttcap%
\pgfsetroundjoin%
\definecolor{currentfill}{rgb}{0.000000,0.000000,0.000000}%
\pgfsetfillcolor{currentfill}%
\pgfsetlinewidth{0.803000pt}%
\definecolor{currentstroke}{rgb}{0.000000,0.000000,0.000000}%
\pgfsetstrokecolor{currentstroke}%
\pgfsetdash{}{0pt}%
\pgfsys@defobject{currentmarker}{\pgfqpoint{-0.048611in}{0.000000in}}{\pgfqpoint{-0.000000in}{0.000000in}}{%
\pgfpathmoveto{\pgfqpoint{-0.000000in}{0.000000in}}%
\pgfpathlineto{\pgfqpoint{-0.048611in}{0.000000in}}%
\pgfusepath{stroke,fill}%
}%
\begin{pgfscope}%
\pgfsys@transformshift{0.633874in}{3.699405in}%
\pgfsys@useobject{currentmarker}{}%
\end{pgfscope}%
\end{pgfscope}%
\begin{pgfscope}%
\definecolor{textcolor}{rgb}{0.000000,0.000000,0.000000}%
\pgfsetstrokecolor{textcolor}%
\pgfsetfillcolor{textcolor}%
\pgftext[x=0.467207in, y=3.651180in, left, base]{\color{textcolor}\rmfamily\fontsize{10.000000}{12.000000}\selectfont \(\displaystyle {3}\)}%
\end{pgfscope}%
\begin{pgfscope}%
\pgfsetbuttcap%
\pgfsetroundjoin%
\definecolor{currentfill}{rgb}{0.000000,0.000000,0.000000}%
\pgfsetfillcolor{currentfill}%
\pgfsetlinewidth{0.803000pt}%
\definecolor{currentstroke}{rgb}{0.000000,0.000000,0.000000}%
\pgfsetstrokecolor{currentstroke}%
\pgfsetdash{}{0pt}%
\pgfsys@defobject{currentmarker}{\pgfqpoint{-0.048611in}{0.000000in}}{\pgfqpoint{-0.000000in}{0.000000in}}{%
\pgfpathmoveto{\pgfqpoint{-0.000000in}{0.000000in}}%
\pgfpathlineto{\pgfqpoint{-0.048611in}{0.000000in}}%
\pgfusepath{stroke,fill}%
}%
\begin{pgfscope}%
\pgfsys@transformshift{0.633874in}{4.038659in}%
\pgfsys@useobject{currentmarker}{}%
\end{pgfscope}%
\end{pgfscope}%
\begin{pgfscope}%
\definecolor{textcolor}{rgb}{0.000000,0.000000,0.000000}%
\pgfsetstrokecolor{textcolor}%
\pgfsetfillcolor{textcolor}%
\pgftext[x=0.467207in, y=3.990433in, left, base]{\color{textcolor}\rmfamily\fontsize{10.000000}{12.000000}\selectfont \(\displaystyle {4}\)}%
\end{pgfscope}%
\begin{pgfscope}%
\pgfsetbuttcap%
\pgfsetroundjoin%
\definecolor{currentfill}{rgb}{0.000000,0.000000,0.000000}%
\pgfsetfillcolor{currentfill}%
\pgfsetlinewidth{0.803000pt}%
\definecolor{currentstroke}{rgb}{0.000000,0.000000,0.000000}%
\pgfsetstrokecolor{currentstroke}%
\pgfsetdash{}{0pt}%
\pgfsys@defobject{currentmarker}{\pgfqpoint{-0.048611in}{0.000000in}}{\pgfqpoint{-0.000000in}{0.000000in}}{%
\pgfpathmoveto{\pgfqpoint{-0.000000in}{0.000000in}}%
\pgfpathlineto{\pgfqpoint{-0.048611in}{0.000000in}}%
\pgfusepath{stroke,fill}%
}%
\begin{pgfscope}%
\pgfsys@transformshift{0.633874in}{4.377912in}%
\pgfsys@useobject{currentmarker}{}%
\end{pgfscope}%
\end{pgfscope}%
\begin{pgfscope}%
\definecolor{textcolor}{rgb}{0.000000,0.000000,0.000000}%
\pgfsetstrokecolor{textcolor}%
\pgfsetfillcolor{textcolor}%
\pgftext[x=0.467207in, y=4.329687in, left, base]{\color{textcolor}\rmfamily\fontsize{10.000000}{12.000000}\selectfont \(\displaystyle {5}\)}%
\end{pgfscope}%
\begin{pgfscope}%
\pgfsetbuttcap%
\pgfsetroundjoin%
\definecolor{currentfill}{rgb}{0.000000,0.000000,0.000000}%
\pgfsetfillcolor{currentfill}%
\pgfsetlinewidth{0.803000pt}%
\definecolor{currentstroke}{rgb}{0.000000,0.000000,0.000000}%
\pgfsetstrokecolor{currentstroke}%
\pgfsetdash{}{0pt}%
\pgfsys@defobject{currentmarker}{\pgfqpoint{-0.048611in}{0.000000in}}{\pgfqpoint{-0.000000in}{0.000000in}}{%
\pgfpathmoveto{\pgfqpoint{-0.000000in}{0.000000in}}%
\pgfpathlineto{\pgfqpoint{-0.048611in}{0.000000in}}%
\pgfusepath{stroke,fill}%
}%
\begin{pgfscope}%
\pgfsys@transformshift{0.633874in}{4.717165in}%
\pgfsys@useobject{currentmarker}{}%
\end{pgfscope}%
\end{pgfscope}%
\begin{pgfscope}%
\definecolor{textcolor}{rgb}{0.000000,0.000000,0.000000}%
\pgfsetstrokecolor{textcolor}%
\pgfsetfillcolor{textcolor}%
\pgftext[x=0.467207in, y=4.668940in, left, base]{\color{textcolor}\rmfamily\fontsize{10.000000}{12.000000}\selectfont \(\displaystyle {6}\)}%
\end{pgfscope}%
\begin{pgfscope}%
\pgfsetbuttcap%
\pgfsetroundjoin%
\definecolor{currentfill}{rgb}{0.000000,0.000000,0.000000}%
\pgfsetfillcolor{currentfill}%
\pgfsetlinewidth{0.803000pt}%
\definecolor{currentstroke}{rgb}{0.000000,0.000000,0.000000}%
\pgfsetstrokecolor{currentstroke}%
\pgfsetdash{}{0pt}%
\pgfsys@defobject{currentmarker}{\pgfqpoint{-0.048611in}{0.000000in}}{\pgfqpoint{-0.000000in}{0.000000in}}{%
\pgfpathmoveto{\pgfqpoint{-0.000000in}{0.000000in}}%
\pgfpathlineto{\pgfqpoint{-0.048611in}{0.000000in}}%
\pgfusepath{stroke,fill}%
}%
\begin{pgfscope}%
\pgfsys@transformshift{0.633874in}{5.056419in}%
\pgfsys@useobject{currentmarker}{}%
\end{pgfscope}%
\end{pgfscope}%
\begin{pgfscope}%
\definecolor{textcolor}{rgb}{0.000000,0.000000,0.000000}%
\pgfsetstrokecolor{textcolor}%
\pgfsetfillcolor{textcolor}%
\pgftext[x=0.467207in, y=5.008194in, left, base]{\color{textcolor}\rmfamily\fontsize{10.000000}{12.000000}\selectfont \(\displaystyle {7}\)}%
\end{pgfscope}%
\begin{pgfscope}%
\definecolor{textcolor}{rgb}{0.000000,0.000000,0.000000}%
\pgfsetstrokecolor{textcolor}%
\pgfsetfillcolor{textcolor}%
\pgftext[x=0.411651in,y=4.021696in,,bottom,rotate=90.000000]{\color{textcolor}\rmfamily\fontsize{10.000000}{12.000000}\selectfont petal\_length}%
\end{pgfscope}%
\begin{pgfscope}%
\pgfsetrectcap%
\pgfsetmiterjoin%
\pgfsetlinewidth{0.803000pt}%
\definecolor{currentstroke}{rgb}{0.000000,0.000000,0.000000}%
\pgfsetstrokecolor{currentstroke}%
\pgfsetdash{}{0pt}%
\pgfpathmoveto{\pgfqpoint{0.633874in}{2.920818in}}%
\pgfpathlineto{\pgfqpoint{0.633874in}{5.122573in}}%
\pgfusepath{stroke}%
\end{pgfscope}%
\begin{pgfscope}%
\pgfsetrectcap%
\pgfsetmiterjoin%
\pgfsetlinewidth{0.803000pt}%
\definecolor{currentstroke}{rgb}{0.000000,0.000000,0.000000}%
\pgfsetstrokecolor{currentstroke}%
\pgfsetdash{}{0pt}%
\pgfpathmoveto{\pgfqpoint{0.633874in}{2.920818in}}%
\pgfpathlineto{\pgfqpoint{2.811154in}{2.920818in}}%
\pgfusepath{stroke}%
\end{pgfscope}%
\begin{pgfscope}%
\pgfsetbuttcap%
\pgfsetmiterjoin%
\definecolor{currentfill}{rgb}{1.000000,1.000000,1.000000}%
\pgfsetfillcolor{currentfill}%
\pgfsetlinewidth{0.000000pt}%
\definecolor{currentstroke}{rgb}{0.000000,0.000000,0.000000}%
\pgfsetstrokecolor{currentstroke}%
\pgfsetstrokeopacity{0.000000}%
\pgfsetdash{}{0pt}%
\pgfpathmoveto{\pgfqpoint{2.963410in}{2.920818in}}%
\pgfpathlineto{\pgfqpoint{5.140690in}{2.920818in}}%
\pgfpathlineto{\pgfqpoint{5.140690in}{5.122573in}}%
\pgfpathlineto{\pgfqpoint{2.963410in}{5.122573in}}%
\pgfpathlineto{\pgfqpoint{2.963410in}{2.920818in}}%
\pgfpathclose%
\pgfusepath{fill}%
\end{pgfscope}%
\begin{pgfscope}%
\pgfpathrectangle{\pgfqpoint{2.963410in}{2.920818in}}{\pgfqpoint{2.177280in}{2.201755in}}%
\pgfusepath{clip}%
\pgfsetbuttcap%
\pgfsetroundjoin%
\definecolor{currentfill}{rgb}{0.121569,0.466667,0.705882}%
\pgfsetfillcolor{currentfill}%
\pgfsetlinewidth{0.481800pt}%
\definecolor{currentstroke}{rgb}{1.000000,1.000000,1.000000}%
\pgfsetstrokecolor{currentstroke}%
\pgfsetdash{}{0pt}%
\pgfpathmoveto{\pgfqpoint{4.202828in}{3.114933in}}%
\pgfpathcurveto{\pgfqpoint{4.213878in}{3.114933in}}{\pgfqpoint{4.224477in}{3.119323in}}{\pgfqpoint{4.232291in}{3.127137in}}%
\pgfpathcurveto{\pgfqpoint{4.240104in}{3.134950in}}{\pgfqpoint{4.244495in}{3.145549in}}{\pgfqpoint{4.244495in}{3.156599in}}%
\pgfpathcurveto{\pgfqpoint{4.244495in}{3.167650in}}{\pgfqpoint{4.240104in}{3.178249in}}{\pgfqpoint{4.232291in}{3.186062in}}%
\pgfpathcurveto{\pgfqpoint{4.224477in}{3.193876in}}{\pgfqpoint{4.213878in}{3.198266in}}{\pgfqpoint{4.202828in}{3.198266in}}%
\pgfpathcurveto{\pgfqpoint{4.191778in}{3.198266in}}{\pgfqpoint{4.181179in}{3.193876in}}{\pgfqpoint{4.173365in}{3.186062in}}%
\pgfpathcurveto{\pgfqpoint{4.165552in}{3.178249in}}{\pgfqpoint{4.161161in}{3.167650in}}{\pgfqpoint{4.161161in}{3.156599in}}%
\pgfpathcurveto{\pgfqpoint{4.161161in}{3.145549in}}{\pgfqpoint{4.165552in}{3.134950in}}{\pgfqpoint{4.173365in}{3.127137in}}%
\pgfpathcurveto{\pgfqpoint{4.181179in}{3.119323in}}{\pgfqpoint{4.191778in}{3.114933in}}{\pgfqpoint{4.202828in}{3.114933in}}%
\pgfpathlineto{\pgfqpoint{4.202828in}{3.114933in}}%
\pgfpathclose%
\pgfusepath{stroke,fill}%
\end{pgfscope}%
\begin{pgfscope}%
\pgfpathrectangle{\pgfqpoint{2.963410in}{2.920818in}}{\pgfqpoint{2.177280in}{2.201755in}}%
\pgfusepath{clip}%
\pgfsetbuttcap%
\pgfsetroundjoin%
\definecolor{currentfill}{rgb}{0.121569,0.466667,0.705882}%
\pgfsetfillcolor{currentfill}%
\pgfsetlinewidth{0.481800pt}%
\definecolor{currentstroke}{rgb}{1.000000,1.000000,1.000000}%
\pgfsetstrokecolor{currentstroke}%
\pgfsetdash{}{0pt}%
\pgfpathmoveto{\pgfqpoint{3.907452in}{3.114933in}}%
\pgfpathcurveto{\pgfqpoint{3.918502in}{3.114933in}}{\pgfqpoint{3.929101in}{3.119323in}}{\pgfqpoint{3.936915in}{3.127137in}}%
\pgfpathcurveto{\pgfqpoint{3.944728in}{3.134950in}}{\pgfqpoint{3.949118in}{3.145549in}}{\pgfqpoint{3.949118in}{3.156599in}}%
\pgfpathcurveto{\pgfqpoint{3.949118in}{3.167650in}}{\pgfqpoint{3.944728in}{3.178249in}}{\pgfqpoint{3.936915in}{3.186062in}}%
\pgfpathcurveto{\pgfqpoint{3.929101in}{3.193876in}}{\pgfqpoint{3.918502in}{3.198266in}}{\pgfqpoint{3.907452in}{3.198266in}}%
\pgfpathcurveto{\pgfqpoint{3.896402in}{3.198266in}}{\pgfqpoint{3.885803in}{3.193876in}}{\pgfqpoint{3.877989in}{3.186062in}}%
\pgfpathcurveto{\pgfqpoint{3.870175in}{3.178249in}}{\pgfqpoint{3.865785in}{3.167650in}}{\pgfqpoint{3.865785in}{3.156599in}}%
\pgfpathcurveto{\pgfqpoint{3.865785in}{3.145549in}}{\pgfqpoint{3.870175in}{3.134950in}}{\pgfqpoint{3.877989in}{3.127137in}}%
\pgfpathcurveto{\pgfqpoint{3.885803in}{3.119323in}}{\pgfqpoint{3.896402in}{3.114933in}}{\pgfqpoint{3.907452in}{3.114933in}}%
\pgfpathlineto{\pgfqpoint{3.907452in}{3.114933in}}%
\pgfpathclose%
\pgfusepath{stroke,fill}%
\end{pgfscope}%
\begin{pgfscope}%
\pgfpathrectangle{\pgfqpoint{2.963410in}{2.920818in}}{\pgfqpoint{2.177280in}{2.201755in}}%
\pgfusepath{clip}%
\pgfsetbuttcap%
\pgfsetroundjoin%
\definecolor{currentfill}{rgb}{0.121569,0.466667,0.705882}%
\pgfsetfillcolor{currentfill}%
\pgfsetlinewidth{0.481800pt}%
\definecolor{currentstroke}{rgb}{1.000000,1.000000,1.000000}%
\pgfsetstrokecolor{currentstroke}%
\pgfsetdash{}{0pt}%
\pgfpathmoveto{\pgfqpoint{4.025602in}{3.081007in}}%
\pgfpathcurveto{\pgfqpoint{4.036652in}{3.081007in}}{\pgfqpoint{4.047251in}{3.085398in}}{\pgfqpoint{4.055065in}{3.093211in}}%
\pgfpathcurveto{\pgfqpoint{4.062879in}{3.101025in}}{\pgfqpoint{4.067269in}{3.111624in}}{\pgfqpoint{4.067269in}{3.122674in}}%
\pgfpathcurveto{\pgfqpoint{4.067269in}{3.133724in}}{\pgfqpoint{4.062879in}{3.144323in}}{\pgfqpoint{4.055065in}{3.152137in}}%
\pgfpathcurveto{\pgfqpoint{4.047251in}{3.159951in}}{\pgfqpoint{4.036652in}{3.164341in}}{\pgfqpoint{4.025602in}{3.164341in}}%
\pgfpathcurveto{\pgfqpoint{4.014552in}{3.164341in}}{\pgfqpoint{4.003953in}{3.159951in}}{\pgfqpoint{3.996139in}{3.152137in}}%
\pgfpathcurveto{\pgfqpoint{3.988326in}{3.144323in}}{\pgfqpoint{3.983936in}{3.133724in}}{\pgfqpoint{3.983936in}{3.122674in}}%
\pgfpathcurveto{\pgfqpoint{3.983936in}{3.111624in}}{\pgfqpoint{3.988326in}{3.101025in}}{\pgfqpoint{3.996139in}{3.093211in}}%
\pgfpathcurveto{\pgfqpoint{4.003953in}{3.085398in}}{\pgfqpoint{4.014552in}{3.081007in}}{\pgfqpoint{4.025602in}{3.081007in}}%
\pgfpathlineto{\pgfqpoint{4.025602in}{3.081007in}}%
\pgfpathclose%
\pgfusepath{stroke,fill}%
\end{pgfscope}%
\begin{pgfscope}%
\pgfpathrectangle{\pgfqpoint{2.963410in}{2.920818in}}{\pgfqpoint{2.177280in}{2.201755in}}%
\pgfusepath{clip}%
\pgfsetbuttcap%
\pgfsetroundjoin%
\definecolor{currentfill}{rgb}{0.121569,0.466667,0.705882}%
\pgfsetfillcolor{currentfill}%
\pgfsetlinewidth{0.481800pt}%
\definecolor{currentstroke}{rgb}{1.000000,1.000000,1.000000}%
\pgfsetstrokecolor{currentstroke}%
\pgfsetdash{}{0pt}%
\pgfpathmoveto{\pgfqpoint{3.966527in}{3.148858in}}%
\pgfpathcurveto{\pgfqpoint{3.977577in}{3.148858in}}{\pgfqpoint{3.988176in}{3.153248in}}{\pgfqpoint{3.995990in}{3.161062in}}%
\pgfpathcurveto{\pgfqpoint{4.003803in}{3.168876in}}{\pgfqpoint{4.008194in}{3.179475in}}{\pgfqpoint{4.008194in}{3.190525in}}%
\pgfpathcurveto{\pgfqpoint{4.008194in}{3.201575in}}{\pgfqpoint{4.003803in}{3.212174in}}{\pgfqpoint{3.995990in}{3.219988in}}%
\pgfpathcurveto{\pgfqpoint{3.988176in}{3.227801in}}{\pgfqpoint{3.977577in}{3.232191in}}{\pgfqpoint{3.966527in}{3.232191in}}%
\pgfpathcurveto{\pgfqpoint{3.955477in}{3.232191in}}{\pgfqpoint{3.944878in}{3.227801in}}{\pgfqpoint{3.937064in}{3.219988in}}%
\pgfpathcurveto{\pgfqpoint{3.929251in}{3.212174in}}{\pgfqpoint{3.924860in}{3.201575in}}{\pgfqpoint{3.924860in}{3.190525in}}%
\pgfpathcurveto{\pgfqpoint{3.924860in}{3.179475in}}{\pgfqpoint{3.929251in}{3.168876in}}{\pgfqpoint{3.937064in}{3.161062in}}%
\pgfpathcurveto{\pgfqpoint{3.944878in}{3.153248in}}{\pgfqpoint{3.955477in}{3.148858in}}{\pgfqpoint{3.966527in}{3.148858in}}%
\pgfpathlineto{\pgfqpoint{3.966527in}{3.148858in}}%
\pgfpathclose%
\pgfusepath{stroke,fill}%
\end{pgfscope}%
\begin{pgfscope}%
\pgfpathrectangle{\pgfqpoint{2.963410in}{2.920818in}}{\pgfqpoint{2.177280in}{2.201755in}}%
\pgfusepath{clip}%
\pgfsetbuttcap%
\pgfsetroundjoin%
\definecolor{currentfill}{rgb}{0.121569,0.466667,0.705882}%
\pgfsetfillcolor{currentfill}%
\pgfsetlinewidth{0.481800pt}%
\definecolor{currentstroke}{rgb}{1.000000,1.000000,1.000000}%
\pgfsetstrokecolor{currentstroke}%
\pgfsetdash{}{0pt}%
\pgfpathmoveto{\pgfqpoint{4.261903in}{3.114933in}}%
\pgfpathcurveto{\pgfqpoint{4.272953in}{3.114933in}}{\pgfqpoint{4.283552in}{3.119323in}}{\pgfqpoint{4.291366in}{3.127137in}}%
\pgfpathcurveto{\pgfqpoint{4.299180in}{3.134950in}}{\pgfqpoint{4.303570in}{3.145549in}}{\pgfqpoint{4.303570in}{3.156599in}}%
\pgfpathcurveto{\pgfqpoint{4.303570in}{3.167650in}}{\pgfqpoint{4.299180in}{3.178249in}}{\pgfqpoint{4.291366in}{3.186062in}}%
\pgfpathcurveto{\pgfqpoint{4.283552in}{3.193876in}}{\pgfqpoint{4.272953in}{3.198266in}}{\pgfqpoint{4.261903in}{3.198266in}}%
\pgfpathcurveto{\pgfqpoint{4.250853in}{3.198266in}}{\pgfqpoint{4.240254in}{3.193876in}}{\pgfqpoint{4.232441in}{3.186062in}}%
\pgfpathcurveto{\pgfqpoint{4.224627in}{3.178249in}}{\pgfqpoint{4.220237in}{3.167650in}}{\pgfqpoint{4.220237in}{3.156599in}}%
\pgfpathcurveto{\pgfqpoint{4.220237in}{3.145549in}}{\pgfqpoint{4.224627in}{3.134950in}}{\pgfqpoint{4.232441in}{3.127137in}}%
\pgfpathcurveto{\pgfqpoint{4.240254in}{3.119323in}}{\pgfqpoint{4.250853in}{3.114933in}}{\pgfqpoint{4.261903in}{3.114933in}}%
\pgfpathlineto{\pgfqpoint{4.261903in}{3.114933in}}%
\pgfpathclose%
\pgfusepath{stroke,fill}%
\end{pgfscope}%
\begin{pgfscope}%
\pgfpathrectangle{\pgfqpoint{2.963410in}{2.920818in}}{\pgfqpoint{2.177280in}{2.201755in}}%
\pgfusepath{clip}%
\pgfsetbuttcap%
\pgfsetroundjoin%
\definecolor{currentfill}{rgb}{0.121569,0.466667,0.705882}%
\pgfsetfillcolor{currentfill}%
\pgfsetlinewidth{0.481800pt}%
\definecolor{currentstroke}{rgb}{1.000000,1.000000,1.000000}%
\pgfsetstrokecolor{currentstroke}%
\pgfsetdash{}{0pt}%
\pgfpathmoveto{\pgfqpoint{4.439129in}{3.216709in}}%
\pgfpathcurveto{\pgfqpoint{4.450179in}{3.216709in}}{\pgfqpoint{4.460778in}{3.221099in}}{\pgfqpoint{4.468592in}{3.228913in}}%
\pgfpathcurveto{\pgfqpoint{4.476406in}{3.236726in}}{\pgfqpoint{4.480796in}{3.247325in}}{\pgfqpoint{4.480796in}{3.258375in}}%
\pgfpathcurveto{\pgfqpoint{4.480796in}{3.269426in}}{\pgfqpoint{4.476406in}{3.280025in}}{\pgfqpoint{4.468592in}{3.287838in}}%
\pgfpathcurveto{\pgfqpoint{4.460778in}{3.295652in}}{\pgfqpoint{4.450179in}{3.300042in}}{\pgfqpoint{4.439129in}{3.300042in}}%
\pgfpathcurveto{\pgfqpoint{4.428079in}{3.300042in}}{\pgfqpoint{4.417480in}{3.295652in}}{\pgfqpoint{4.409666in}{3.287838in}}%
\pgfpathcurveto{\pgfqpoint{4.401853in}{3.280025in}}{\pgfqpoint{4.397462in}{3.269426in}}{\pgfqpoint{4.397462in}{3.258375in}}%
\pgfpathcurveto{\pgfqpoint{4.397462in}{3.247325in}}{\pgfqpoint{4.401853in}{3.236726in}}{\pgfqpoint{4.409666in}{3.228913in}}%
\pgfpathcurveto{\pgfqpoint{4.417480in}{3.221099in}}{\pgfqpoint{4.428079in}{3.216709in}}{\pgfqpoint{4.439129in}{3.216709in}}%
\pgfpathlineto{\pgfqpoint{4.439129in}{3.216709in}}%
\pgfpathclose%
\pgfusepath{stroke,fill}%
\end{pgfscope}%
\begin{pgfscope}%
\pgfpathrectangle{\pgfqpoint{2.963410in}{2.920818in}}{\pgfqpoint{2.177280in}{2.201755in}}%
\pgfusepath{clip}%
\pgfsetbuttcap%
\pgfsetroundjoin%
\definecolor{currentfill}{rgb}{0.121569,0.466667,0.705882}%
\pgfsetfillcolor{currentfill}%
\pgfsetlinewidth{0.481800pt}%
\definecolor{currentstroke}{rgb}{1.000000,1.000000,1.000000}%
\pgfsetstrokecolor{currentstroke}%
\pgfsetdash{}{0pt}%
\pgfpathmoveto{\pgfqpoint{4.143753in}{3.114933in}}%
\pgfpathcurveto{\pgfqpoint{4.154803in}{3.114933in}}{\pgfqpoint{4.165402in}{3.119323in}}{\pgfqpoint{4.173216in}{3.127137in}}%
\pgfpathcurveto{\pgfqpoint{4.181029in}{3.134950in}}{\pgfqpoint{4.185419in}{3.145549in}}{\pgfqpoint{4.185419in}{3.156599in}}%
\pgfpathcurveto{\pgfqpoint{4.185419in}{3.167650in}}{\pgfqpoint{4.181029in}{3.178249in}}{\pgfqpoint{4.173216in}{3.186062in}}%
\pgfpathcurveto{\pgfqpoint{4.165402in}{3.193876in}}{\pgfqpoint{4.154803in}{3.198266in}}{\pgfqpoint{4.143753in}{3.198266in}}%
\pgfpathcurveto{\pgfqpoint{4.132703in}{3.198266in}}{\pgfqpoint{4.122104in}{3.193876in}}{\pgfqpoint{4.114290in}{3.186062in}}%
\pgfpathcurveto{\pgfqpoint{4.106476in}{3.178249in}}{\pgfqpoint{4.102086in}{3.167650in}}{\pgfqpoint{4.102086in}{3.156599in}}%
\pgfpathcurveto{\pgfqpoint{4.102086in}{3.145549in}}{\pgfqpoint{4.106476in}{3.134950in}}{\pgfqpoint{4.114290in}{3.127137in}}%
\pgfpathcurveto{\pgfqpoint{4.122104in}{3.119323in}}{\pgfqpoint{4.132703in}{3.114933in}}{\pgfqpoint{4.143753in}{3.114933in}}%
\pgfpathlineto{\pgfqpoint{4.143753in}{3.114933in}}%
\pgfpathclose%
\pgfusepath{stroke,fill}%
\end{pgfscope}%
\begin{pgfscope}%
\pgfpathrectangle{\pgfqpoint{2.963410in}{2.920818in}}{\pgfqpoint{2.177280in}{2.201755in}}%
\pgfusepath{clip}%
\pgfsetbuttcap%
\pgfsetroundjoin%
\definecolor{currentfill}{rgb}{0.121569,0.466667,0.705882}%
\pgfsetfillcolor{currentfill}%
\pgfsetlinewidth{0.481800pt}%
\definecolor{currentstroke}{rgb}{1.000000,1.000000,1.000000}%
\pgfsetstrokecolor{currentstroke}%
\pgfsetdash{}{0pt}%
\pgfpathmoveto{\pgfqpoint{4.143753in}{3.148858in}}%
\pgfpathcurveto{\pgfqpoint{4.154803in}{3.148858in}}{\pgfqpoint{4.165402in}{3.153248in}}{\pgfqpoint{4.173216in}{3.161062in}}%
\pgfpathcurveto{\pgfqpoint{4.181029in}{3.168876in}}{\pgfqpoint{4.185419in}{3.179475in}}{\pgfqpoint{4.185419in}{3.190525in}}%
\pgfpathcurveto{\pgfqpoint{4.185419in}{3.201575in}}{\pgfqpoint{4.181029in}{3.212174in}}{\pgfqpoint{4.173216in}{3.219988in}}%
\pgfpathcurveto{\pgfqpoint{4.165402in}{3.227801in}}{\pgfqpoint{4.154803in}{3.232191in}}{\pgfqpoint{4.143753in}{3.232191in}}%
\pgfpathcurveto{\pgfqpoint{4.132703in}{3.232191in}}{\pgfqpoint{4.122104in}{3.227801in}}{\pgfqpoint{4.114290in}{3.219988in}}%
\pgfpathcurveto{\pgfqpoint{4.106476in}{3.212174in}}{\pgfqpoint{4.102086in}{3.201575in}}{\pgfqpoint{4.102086in}{3.190525in}}%
\pgfpathcurveto{\pgfqpoint{4.102086in}{3.179475in}}{\pgfqpoint{4.106476in}{3.168876in}}{\pgfqpoint{4.114290in}{3.161062in}}%
\pgfpathcurveto{\pgfqpoint{4.122104in}{3.153248in}}{\pgfqpoint{4.132703in}{3.148858in}}{\pgfqpoint{4.143753in}{3.148858in}}%
\pgfpathlineto{\pgfqpoint{4.143753in}{3.148858in}}%
\pgfpathclose%
\pgfusepath{stroke,fill}%
\end{pgfscope}%
\begin{pgfscope}%
\pgfpathrectangle{\pgfqpoint{2.963410in}{2.920818in}}{\pgfqpoint{2.177280in}{2.201755in}}%
\pgfusepath{clip}%
\pgfsetbuttcap%
\pgfsetroundjoin%
\definecolor{currentfill}{rgb}{0.121569,0.466667,0.705882}%
\pgfsetfillcolor{currentfill}%
\pgfsetlinewidth{0.481800pt}%
\definecolor{currentstroke}{rgb}{1.000000,1.000000,1.000000}%
\pgfsetstrokecolor{currentstroke}%
\pgfsetdash{}{0pt}%
\pgfpathmoveto{\pgfqpoint{3.848376in}{3.114933in}}%
\pgfpathcurveto{\pgfqpoint{3.859427in}{3.114933in}}{\pgfqpoint{3.870026in}{3.119323in}}{\pgfqpoint{3.877839in}{3.127137in}}%
\pgfpathcurveto{\pgfqpoint{3.885653in}{3.134950in}}{\pgfqpoint{3.890043in}{3.145549in}}{\pgfqpoint{3.890043in}{3.156599in}}%
\pgfpathcurveto{\pgfqpoint{3.890043in}{3.167650in}}{\pgfqpoint{3.885653in}{3.178249in}}{\pgfqpoint{3.877839in}{3.186062in}}%
\pgfpathcurveto{\pgfqpoint{3.870026in}{3.193876in}}{\pgfqpoint{3.859427in}{3.198266in}}{\pgfqpoint{3.848376in}{3.198266in}}%
\pgfpathcurveto{\pgfqpoint{3.837326in}{3.198266in}}{\pgfqpoint{3.826727in}{3.193876in}}{\pgfqpoint{3.818914in}{3.186062in}}%
\pgfpathcurveto{\pgfqpoint{3.811100in}{3.178249in}}{\pgfqpoint{3.806710in}{3.167650in}}{\pgfqpoint{3.806710in}{3.156599in}}%
\pgfpathcurveto{\pgfqpoint{3.806710in}{3.145549in}}{\pgfqpoint{3.811100in}{3.134950in}}{\pgfqpoint{3.818914in}{3.127137in}}%
\pgfpathcurveto{\pgfqpoint{3.826727in}{3.119323in}}{\pgfqpoint{3.837326in}{3.114933in}}{\pgfqpoint{3.848376in}{3.114933in}}%
\pgfpathlineto{\pgfqpoint{3.848376in}{3.114933in}}%
\pgfpathclose%
\pgfusepath{stroke,fill}%
\end{pgfscope}%
\begin{pgfscope}%
\pgfpathrectangle{\pgfqpoint{2.963410in}{2.920818in}}{\pgfqpoint{2.177280in}{2.201755in}}%
\pgfusepath{clip}%
\pgfsetbuttcap%
\pgfsetroundjoin%
\definecolor{currentfill}{rgb}{0.121569,0.466667,0.705882}%
\pgfsetfillcolor{currentfill}%
\pgfsetlinewidth{0.481800pt}%
\definecolor{currentstroke}{rgb}{1.000000,1.000000,1.000000}%
\pgfsetstrokecolor{currentstroke}%
\pgfsetdash{}{0pt}%
\pgfpathmoveto{\pgfqpoint{3.966527in}{3.148858in}}%
\pgfpathcurveto{\pgfqpoint{3.977577in}{3.148858in}}{\pgfqpoint{3.988176in}{3.153248in}}{\pgfqpoint{3.995990in}{3.161062in}}%
\pgfpathcurveto{\pgfqpoint{4.003803in}{3.168876in}}{\pgfqpoint{4.008194in}{3.179475in}}{\pgfqpoint{4.008194in}{3.190525in}}%
\pgfpathcurveto{\pgfqpoint{4.008194in}{3.201575in}}{\pgfqpoint{4.003803in}{3.212174in}}{\pgfqpoint{3.995990in}{3.219988in}}%
\pgfpathcurveto{\pgfqpoint{3.988176in}{3.227801in}}{\pgfqpoint{3.977577in}{3.232191in}}{\pgfqpoint{3.966527in}{3.232191in}}%
\pgfpathcurveto{\pgfqpoint{3.955477in}{3.232191in}}{\pgfqpoint{3.944878in}{3.227801in}}{\pgfqpoint{3.937064in}{3.219988in}}%
\pgfpathcurveto{\pgfqpoint{3.929251in}{3.212174in}}{\pgfqpoint{3.924860in}{3.201575in}}{\pgfqpoint{3.924860in}{3.190525in}}%
\pgfpathcurveto{\pgfqpoint{3.924860in}{3.179475in}}{\pgfqpoint{3.929251in}{3.168876in}}{\pgfqpoint{3.937064in}{3.161062in}}%
\pgfpathcurveto{\pgfqpoint{3.944878in}{3.153248in}}{\pgfqpoint{3.955477in}{3.148858in}}{\pgfqpoint{3.966527in}{3.148858in}}%
\pgfpathlineto{\pgfqpoint{3.966527in}{3.148858in}}%
\pgfpathclose%
\pgfusepath{stroke,fill}%
\end{pgfscope}%
\begin{pgfscope}%
\pgfpathrectangle{\pgfqpoint{2.963410in}{2.920818in}}{\pgfqpoint{2.177280in}{2.201755in}}%
\pgfusepath{clip}%
\pgfsetbuttcap%
\pgfsetroundjoin%
\definecolor{currentfill}{rgb}{0.121569,0.466667,0.705882}%
\pgfsetfillcolor{currentfill}%
\pgfsetlinewidth{0.481800pt}%
\definecolor{currentstroke}{rgb}{1.000000,1.000000,1.000000}%
\pgfsetstrokecolor{currentstroke}%
\pgfsetdash{}{0pt}%
\pgfpathmoveto{\pgfqpoint{4.320979in}{3.148858in}}%
\pgfpathcurveto{\pgfqpoint{4.332029in}{3.148858in}}{\pgfqpoint{4.342628in}{3.153248in}}{\pgfqpoint{4.350441in}{3.161062in}}%
\pgfpathcurveto{\pgfqpoint{4.358255in}{3.168876in}}{\pgfqpoint{4.362645in}{3.179475in}}{\pgfqpoint{4.362645in}{3.190525in}}%
\pgfpathcurveto{\pgfqpoint{4.362645in}{3.201575in}}{\pgfqpoint{4.358255in}{3.212174in}}{\pgfqpoint{4.350441in}{3.219988in}}%
\pgfpathcurveto{\pgfqpoint{4.342628in}{3.227801in}}{\pgfqpoint{4.332029in}{3.232191in}}{\pgfqpoint{4.320979in}{3.232191in}}%
\pgfpathcurveto{\pgfqpoint{4.309928in}{3.232191in}}{\pgfqpoint{4.299329in}{3.227801in}}{\pgfqpoint{4.291516in}{3.219988in}}%
\pgfpathcurveto{\pgfqpoint{4.283702in}{3.212174in}}{\pgfqpoint{4.279312in}{3.201575in}}{\pgfqpoint{4.279312in}{3.190525in}}%
\pgfpathcurveto{\pgfqpoint{4.279312in}{3.179475in}}{\pgfqpoint{4.283702in}{3.168876in}}{\pgfqpoint{4.291516in}{3.161062in}}%
\pgfpathcurveto{\pgfqpoint{4.299329in}{3.153248in}}{\pgfqpoint{4.309928in}{3.148858in}}{\pgfqpoint{4.320979in}{3.148858in}}%
\pgfpathlineto{\pgfqpoint{4.320979in}{3.148858in}}%
\pgfpathclose%
\pgfusepath{stroke,fill}%
\end{pgfscope}%
\begin{pgfscope}%
\pgfpathrectangle{\pgfqpoint{2.963410in}{2.920818in}}{\pgfqpoint{2.177280in}{2.201755in}}%
\pgfusepath{clip}%
\pgfsetbuttcap%
\pgfsetroundjoin%
\definecolor{currentfill}{rgb}{0.121569,0.466667,0.705882}%
\pgfsetfillcolor{currentfill}%
\pgfsetlinewidth{0.481800pt}%
\definecolor{currentstroke}{rgb}{1.000000,1.000000,1.000000}%
\pgfsetstrokecolor{currentstroke}%
\pgfsetdash{}{0pt}%
\pgfpathmoveto{\pgfqpoint{4.143753in}{3.182783in}}%
\pgfpathcurveto{\pgfqpoint{4.154803in}{3.182783in}}{\pgfqpoint{4.165402in}{3.187174in}}{\pgfqpoint{4.173216in}{3.194987in}}%
\pgfpathcurveto{\pgfqpoint{4.181029in}{3.202801in}}{\pgfqpoint{4.185419in}{3.213400in}}{\pgfqpoint{4.185419in}{3.224450in}}%
\pgfpathcurveto{\pgfqpoint{4.185419in}{3.235500in}}{\pgfqpoint{4.181029in}{3.246099in}}{\pgfqpoint{4.173216in}{3.253913in}}%
\pgfpathcurveto{\pgfqpoint{4.165402in}{3.261727in}}{\pgfqpoint{4.154803in}{3.266117in}}{\pgfqpoint{4.143753in}{3.266117in}}%
\pgfpathcurveto{\pgfqpoint{4.132703in}{3.266117in}}{\pgfqpoint{4.122104in}{3.261727in}}{\pgfqpoint{4.114290in}{3.253913in}}%
\pgfpathcurveto{\pgfqpoint{4.106476in}{3.246099in}}{\pgfqpoint{4.102086in}{3.235500in}}{\pgfqpoint{4.102086in}{3.224450in}}%
\pgfpathcurveto{\pgfqpoint{4.102086in}{3.213400in}}{\pgfqpoint{4.106476in}{3.202801in}}{\pgfqpoint{4.114290in}{3.194987in}}%
\pgfpathcurveto{\pgfqpoint{4.122104in}{3.187174in}}{\pgfqpoint{4.132703in}{3.182783in}}{\pgfqpoint{4.143753in}{3.182783in}}%
\pgfpathlineto{\pgfqpoint{4.143753in}{3.182783in}}%
\pgfpathclose%
\pgfusepath{stroke,fill}%
\end{pgfscope}%
\begin{pgfscope}%
\pgfpathrectangle{\pgfqpoint{2.963410in}{2.920818in}}{\pgfqpoint{2.177280in}{2.201755in}}%
\pgfusepath{clip}%
\pgfsetbuttcap%
\pgfsetroundjoin%
\definecolor{currentfill}{rgb}{0.121569,0.466667,0.705882}%
\pgfsetfillcolor{currentfill}%
\pgfsetlinewidth{0.481800pt}%
\definecolor{currentstroke}{rgb}{1.000000,1.000000,1.000000}%
\pgfsetstrokecolor{currentstroke}%
\pgfsetdash{}{0pt}%
\pgfpathmoveto{\pgfqpoint{3.907452in}{3.114933in}}%
\pgfpathcurveto{\pgfqpoint{3.918502in}{3.114933in}}{\pgfqpoint{3.929101in}{3.119323in}}{\pgfqpoint{3.936915in}{3.127137in}}%
\pgfpathcurveto{\pgfqpoint{3.944728in}{3.134950in}}{\pgfqpoint{3.949118in}{3.145549in}}{\pgfqpoint{3.949118in}{3.156599in}}%
\pgfpathcurveto{\pgfqpoint{3.949118in}{3.167650in}}{\pgfqpoint{3.944728in}{3.178249in}}{\pgfqpoint{3.936915in}{3.186062in}}%
\pgfpathcurveto{\pgfqpoint{3.929101in}{3.193876in}}{\pgfqpoint{3.918502in}{3.198266in}}{\pgfqpoint{3.907452in}{3.198266in}}%
\pgfpathcurveto{\pgfqpoint{3.896402in}{3.198266in}}{\pgfqpoint{3.885803in}{3.193876in}}{\pgfqpoint{3.877989in}{3.186062in}}%
\pgfpathcurveto{\pgfqpoint{3.870175in}{3.178249in}}{\pgfqpoint{3.865785in}{3.167650in}}{\pgfqpoint{3.865785in}{3.156599in}}%
\pgfpathcurveto{\pgfqpoint{3.865785in}{3.145549in}}{\pgfqpoint{3.870175in}{3.134950in}}{\pgfqpoint{3.877989in}{3.127137in}}%
\pgfpathcurveto{\pgfqpoint{3.885803in}{3.119323in}}{\pgfqpoint{3.896402in}{3.114933in}}{\pgfqpoint{3.907452in}{3.114933in}}%
\pgfpathlineto{\pgfqpoint{3.907452in}{3.114933in}}%
\pgfpathclose%
\pgfusepath{stroke,fill}%
\end{pgfscope}%
\begin{pgfscope}%
\pgfpathrectangle{\pgfqpoint{2.963410in}{2.920818in}}{\pgfqpoint{2.177280in}{2.201755in}}%
\pgfusepath{clip}%
\pgfsetbuttcap%
\pgfsetroundjoin%
\definecolor{currentfill}{rgb}{0.121569,0.466667,0.705882}%
\pgfsetfillcolor{currentfill}%
\pgfsetlinewidth{0.481800pt}%
\definecolor{currentstroke}{rgb}{1.000000,1.000000,1.000000}%
\pgfsetstrokecolor{currentstroke}%
\pgfsetdash{}{0pt}%
\pgfpathmoveto{\pgfqpoint{3.907452in}{3.013157in}}%
\pgfpathcurveto{\pgfqpoint{3.918502in}{3.013157in}}{\pgfqpoint{3.929101in}{3.017547in}}{\pgfqpoint{3.936915in}{3.025361in}}%
\pgfpathcurveto{\pgfqpoint{3.944728in}{3.033174in}}{\pgfqpoint{3.949118in}{3.043773in}}{\pgfqpoint{3.949118in}{3.054823in}}%
\pgfpathcurveto{\pgfqpoint{3.949118in}{3.065874in}}{\pgfqpoint{3.944728in}{3.076473in}}{\pgfqpoint{3.936915in}{3.084286in}}%
\pgfpathcurveto{\pgfqpoint{3.929101in}{3.092100in}}{\pgfqpoint{3.918502in}{3.096490in}}{\pgfqpoint{3.907452in}{3.096490in}}%
\pgfpathcurveto{\pgfqpoint{3.896402in}{3.096490in}}{\pgfqpoint{3.885803in}{3.092100in}}{\pgfqpoint{3.877989in}{3.084286in}}%
\pgfpathcurveto{\pgfqpoint{3.870175in}{3.076473in}}{\pgfqpoint{3.865785in}{3.065874in}}{\pgfqpoint{3.865785in}{3.054823in}}%
\pgfpathcurveto{\pgfqpoint{3.865785in}{3.043773in}}{\pgfqpoint{3.870175in}{3.033174in}}{\pgfqpoint{3.877989in}{3.025361in}}%
\pgfpathcurveto{\pgfqpoint{3.885803in}{3.017547in}}{\pgfqpoint{3.896402in}{3.013157in}}{\pgfqpoint{3.907452in}{3.013157in}}%
\pgfpathlineto{\pgfqpoint{3.907452in}{3.013157in}}%
\pgfpathclose%
\pgfusepath{stroke,fill}%
\end{pgfscope}%
\begin{pgfscope}%
\pgfpathrectangle{\pgfqpoint{2.963410in}{2.920818in}}{\pgfqpoint{2.177280in}{2.201755in}}%
\pgfusepath{clip}%
\pgfsetbuttcap%
\pgfsetroundjoin%
\definecolor{currentfill}{rgb}{0.121569,0.466667,0.705882}%
\pgfsetfillcolor{currentfill}%
\pgfsetlinewidth{0.481800pt}%
\definecolor{currentstroke}{rgb}{1.000000,1.000000,1.000000}%
\pgfsetstrokecolor{currentstroke}%
\pgfsetdash{}{0pt}%
\pgfpathmoveto{\pgfqpoint{4.498204in}{3.047082in}}%
\pgfpathcurveto{\pgfqpoint{4.509255in}{3.047082in}}{\pgfqpoint{4.519854in}{3.051472in}}{\pgfqpoint{4.527667in}{3.059286in}}%
\pgfpathcurveto{\pgfqpoint{4.535481in}{3.067100in}}{\pgfqpoint{4.539871in}{3.077699in}}{\pgfqpoint{4.539871in}{3.088749in}}%
\pgfpathcurveto{\pgfqpoint{4.539871in}{3.099799in}}{\pgfqpoint{4.535481in}{3.110398in}}{\pgfqpoint{4.527667in}{3.118212in}}%
\pgfpathcurveto{\pgfqpoint{4.519854in}{3.126025in}}{\pgfqpoint{4.509255in}{3.130415in}}{\pgfqpoint{4.498204in}{3.130415in}}%
\pgfpathcurveto{\pgfqpoint{4.487154in}{3.130415in}}{\pgfqpoint{4.476555in}{3.126025in}}{\pgfqpoint{4.468742in}{3.118212in}}%
\pgfpathcurveto{\pgfqpoint{4.460928in}{3.110398in}}{\pgfqpoint{4.456538in}{3.099799in}}{\pgfqpoint{4.456538in}{3.088749in}}%
\pgfpathcurveto{\pgfqpoint{4.456538in}{3.077699in}}{\pgfqpoint{4.460928in}{3.067100in}}{\pgfqpoint{4.468742in}{3.059286in}}%
\pgfpathcurveto{\pgfqpoint{4.476555in}{3.051472in}}{\pgfqpoint{4.487154in}{3.047082in}}{\pgfqpoint{4.498204in}{3.047082in}}%
\pgfpathlineto{\pgfqpoint{4.498204in}{3.047082in}}%
\pgfpathclose%
\pgfusepath{stroke,fill}%
\end{pgfscope}%
\begin{pgfscope}%
\pgfpathrectangle{\pgfqpoint{2.963410in}{2.920818in}}{\pgfqpoint{2.177280in}{2.201755in}}%
\pgfusepath{clip}%
\pgfsetbuttcap%
\pgfsetroundjoin%
\definecolor{currentfill}{rgb}{0.121569,0.466667,0.705882}%
\pgfsetfillcolor{currentfill}%
\pgfsetlinewidth{0.481800pt}%
\definecolor{currentstroke}{rgb}{1.000000,1.000000,1.000000}%
\pgfsetstrokecolor{currentstroke}%
\pgfsetdash{}{0pt}%
\pgfpathmoveto{\pgfqpoint{4.734505in}{3.148858in}}%
\pgfpathcurveto{\pgfqpoint{4.745556in}{3.148858in}}{\pgfqpoint{4.756155in}{3.153248in}}{\pgfqpoint{4.763968in}{3.161062in}}%
\pgfpathcurveto{\pgfqpoint{4.771782in}{3.168876in}}{\pgfqpoint{4.776172in}{3.179475in}}{\pgfqpoint{4.776172in}{3.190525in}}%
\pgfpathcurveto{\pgfqpoint{4.776172in}{3.201575in}}{\pgfqpoint{4.771782in}{3.212174in}}{\pgfqpoint{4.763968in}{3.219988in}}%
\pgfpathcurveto{\pgfqpoint{4.756155in}{3.227801in}}{\pgfqpoint{4.745556in}{3.232191in}}{\pgfqpoint{4.734505in}{3.232191in}}%
\pgfpathcurveto{\pgfqpoint{4.723455in}{3.232191in}}{\pgfqpoint{4.712856in}{3.227801in}}{\pgfqpoint{4.705043in}{3.219988in}}%
\pgfpathcurveto{\pgfqpoint{4.697229in}{3.212174in}}{\pgfqpoint{4.692839in}{3.201575in}}{\pgfqpoint{4.692839in}{3.190525in}}%
\pgfpathcurveto{\pgfqpoint{4.692839in}{3.179475in}}{\pgfqpoint{4.697229in}{3.168876in}}{\pgfqpoint{4.705043in}{3.161062in}}%
\pgfpathcurveto{\pgfqpoint{4.712856in}{3.153248in}}{\pgfqpoint{4.723455in}{3.148858in}}{\pgfqpoint{4.734505in}{3.148858in}}%
\pgfpathlineto{\pgfqpoint{4.734505in}{3.148858in}}%
\pgfpathclose%
\pgfusepath{stroke,fill}%
\end{pgfscope}%
\begin{pgfscope}%
\pgfpathrectangle{\pgfqpoint{2.963410in}{2.920818in}}{\pgfqpoint{2.177280in}{2.201755in}}%
\pgfusepath{clip}%
\pgfsetbuttcap%
\pgfsetroundjoin%
\definecolor{currentfill}{rgb}{0.121569,0.466667,0.705882}%
\pgfsetfillcolor{currentfill}%
\pgfsetlinewidth{0.481800pt}%
\definecolor{currentstroke}{rgb}{1.000000,1.000000,1.000000}%
\pgfsetstrokecolor{currentstroke}%
\pgfsetdash{}{0pt}%
\pgfpathmoveto{\pgfqpoint{4.439129in}{3.081007in}}%
\pgfpathcurveto{\pgfqpoint{4.450179in}{3.081007in}}{\pgfqpoint{4.460778in}{3.085398in}}{\pgfqpoint{4.468592in}{3.093211in}}%
\pgfpathcurveto{\pgfqpoint{4.476406in}{3.101025in}}{\pgfqpoint{4.480796in}{3.111624in}}{\pgfqpoint{4.480796in}{3.122674in}}%
\pgfpathcurveto{\pgfqpoint{4.480796in}{3.133724in}}{\pgfqpoint{4.476406in}{3.144323in}}{\pgfqpoint{4.468592in}{3.152137in}}%
\pgfpathcurveto{\pgfqpoint{4.460778in}{3.159951in}}{\pgfqpoint{4.450179in}{3.164341in}}{\pgfqpoint{4.439129in}{3.164341in}}%
\pgfpathcurveto{\pgfqpoint{4.428079in}{3.164341in}}{\pgfqpoint{4.417480in}{3.159951in}}{\pgfqpoint{4.409666in}{3.152137in}}%
\pgfpathcurveto{\pgfqpoint{4.401853in}{3.144323in}}{\pgfqpoint{4.397462in}{3.133724in}}{\pgfqpoint{4.397462in}{3.122674in}}%
\pgfpathcurveto{\pgfqpoint{4.397462in}{3.111624in}}{\pgfqpoint{4.401853in}{3.101025in}}{\pgfqpoint{4.409666in}{3.093211in}}%
\pgfpathcurveto{\pgfqpoint{4.417480in}{3.085398in}}{\pgfqpoint{4.428079in}{3.081007in}}{\pgfqpoint{4.439129in}{3.081007in}}%
\pgfpathlineto{\pgfqpoint{4.439129in}{3.081007in}}%
\pgfpathclose%
\pgfusepath{stroke,fill}%
\end{pgfscope}%
\begin{pgfscope}%
\pgfpathrectangle{\pgfqpoint{2.963410in}{2.920818in}}{\pgfqpoint{2.177280in}{2.201755in}}%
\pgfusepath{clip}%
\pgfsetbuttcap%
\pgfsetroundjoin%
\definecolor{currentfill}{rgb}{0.121569,0.466667,0.705882}%
\pgfsetfillcolor{currentfill}%
\pgfsetlinewidth{0.481800pt}%
\definecolor{currentstroke}{rgb}{1.000000,1.000000,1.000000}%
\pgfsetstrokecolor{currentstroke}%
\pgfsetdash{}{0pt}%
\pgfpathmoveto{\pgfqpoint{4.202828in}{3.114933in}}%
\pgfpathcurveto{\pgfqpoint{4.213878in}{3.114933in}}{\pgfqpoint{4.224477in}{3.119323in}}{\pgfqpoint{4.232291in}{3.127137in}}%
\pgfpathcurveto{\pgfqpoint{4.240104in}{3.134950in}}{\pgfqpoint{4.244495in}{3.145549in}}{\pgfqpoint{4.244495in}{3.156599in}}%
\pgfpathcurveto{\pgfqpoint{4.244495in}{3.167650in}}{\pgfqpoint{4.240104in}{3.178249in}}{\pgfqpoint{4.232291in}{3.186062in}}%
\pgfpathcurveto{\pgfqpoint{4.224477in}{3.193876in}}{\pgfqpoint{4.213878in}{3.198266in}}{\pgfqpoint{4.202828in}{3.198266in}}%
\pgfpathcurveto{\pgfqpoint{4.191778in}{3.198266in}}{\pgfqpoint{4.181179in}{3.193876in}}{\pgfqpoint{4.173365in}{3.186062in}}%
\pgfpathcurveto{\pgfqpoint{4.165552in}{3.178249in}}{\pgfqpoint{4.161161in}{3.167650in}}{\pgfqpoint{4.161161in}{3.156599in}}%
\pgfpathcurveto{\pgfqpoint{4.161161in}{3.145549in}}{\pgfqpoint{4.165552in}{3.134950in}}{\pgfqpoint{4.173365in}{3.127137in}}%
\pgfpathcurveto{\pgfqpoint{4.181179in}{3.119323in}}{\pgfqpoint{4.191778in}{3.114933in}}{\pgfqpoint{4.202828in}{3.114933in}}%
\pgfpathlineto{\pgfqpoint{4.202828in}{3.114933in}}%
\pgfpathclose%
\pgfusepath{stroke,fill}%
\end{pgfscope}%
\begin{pgfscope}%
\pgfpathrectangle{\pgfqpoint{2.963410in}{2.920818in}}{\pgfqpoint{2.177280in}{2.201755in}}%
\pgfusepath{clip}%
\pgfsetbuttcap%
\pgfsetroundjoin%
\definecolor{currentfill}{rgb}{0.121569,0.466667,0.705882}%
\pgfsetfillcolor{currentfill}%
\pgfsetlinewidth{0.481800pt}%
\definecolor{currentstroke}{rgb}{1.000000,1.000000,1.000000}%
\pgfsetstrokecolor{currentstroke}%
\pgfsetdash{}{0pt}%
\pgfpathmoveto{\pgfqpoint{4.380054in}{3.216709in}}%
\pgfpathcurveto{\pgfqpoint{4.391104in}{3.216709in}}{\pgfqpoint{4.401703in}{3.221099in}}{\pgfqpoint{4.409517in}{3.228913in}}%
\pgfpathcurveto{\pgfqpoint{4.417330in}{3.236726in}}{\pgfqpoint{4.421721in}{3.247325in}}{\pgfqpoint{4.421721in}{3.258375in}}%
\pgfpathcurveto{\pgfqpoint{4.421721in}{3.269426in}}{\pgfqpoint{4.417330in}{3.280025in}}{\pgfqpoint{4.409517in}{3.287838in}}%
\pgfpathcurveto{\pgfqpoint{4.401703in}{3.295652in}}{\pgfqpoint{4.391104in}{3.300042in}}{\pgfqpoint{4.380054in}{3.300042in}}%
\pgfpathcurveto{\pgfqpoint{4.369004in}{3.300042in}}{\pgfqpoint{4.358405in}{3.295652in}}{\pgfqpoint{4.350591in}{3.287838in}}%
\pgfpathcurveto{\pgfqpoint{4.342777in}{3.280025in}}{\pgfqpoint{4.338387in}{3.269426in}}{\pgfqpoint{4.338387in}{3.258375in}}%
\pgfpathcurveto{\pgfqpoint{4.338387in}{3.247325in}}{\pgfqpoint{4.342777in}{3.236726in}}{\pgfqpoint{4.350591in}{3.228913in}}%
\pgfpathcurveto{\pgfqpoint{4.358405in}{3.221099in}}{\pgfqpoint{4.369004in}{3.216709in}}{\pgfqpoint{4.380054in}{3.216709in}}%
\pgfpathlineto{\pgfqpoint{4.380054in}{3.216709in}}%
\pgfpathclose%
\pgfusepath{stroke,fill}%
\end{pgfscope}%
\begin{pgfscope}%
\pgfpathrectangle{\pgfqpoint{2.963410in}{2.920818in}}{\pgfqpoint{2.177280in}{2.201755in}}%
\pgfusepath{clip}%
\pgfsetbuttcap%
\pgfsetroundjoin%
\definecolor{currentfill}{rgb}{0.121569,0.466667,0.705882}%
\pgfsetfillcolor{currentfill}%
\pgfsetlinewidth{0.481800pt}%
\definecolor{currentstroke}{rgb}{1.000000,1.000000,1.000000}%
\pgfsetstrokecolor{currentstroke}%
\pgfsetdash{}{0pt}%
\pgfpathmoveto{\pgfqpoint{4.380054in}{3.148858in}}%
\pgfpathcurveto{\pgfqpoint{4.391104in}{3.148858in}}{\pgfqpoint{4.401703in}{3.153248in}}{\pgfqpoint{4.409517in}{3.161062in}}%
\pgfpathcurveto{\pgfqpoint{4.417330in}{3.168876in}}{\pgfqpoint{4.421721in}{3.179475in}}{\pgfqpoint{4.421721in}{3.190525in}}%
\pgfpathcurveto{\pgfqpoint{4.421721in}{3.201575in}}{\pgfqpoint{4.417330in}{3.212174in}}{\pgfqpoint{4.409517in}{3.219988in}}%
\pgfpathcurveto{\pgfqpoint{4.401703in}{3.227801in}}{\pgfqpoint{4.391104in}{3.232191in}}{\pgfqpoint{4.380054in}{3.232191in}}%
\pgfpathcurveto{\pgfqpoint{4.369004in}{3.232191in}}{\pgfqpoint{4.358405in}{3.227801in}}{\pgfqpoint{4.350591in}{3.219988in}}%
\pgfpathcurveto{\pgfqpoint{4.342777in}{3.212174in}}{\pgfqpoint{4.338387in}{3.201575in}}{\pgfqpoint{4.338387in}{3.190525in}}%
\pgfpathcurveto{\pgfqpoint{4.338387in}{3.179475in}}{\pgfqpoint{4.342777in}{3.168876in}}{\pgfqpoint{4.350591in}{3.161062in}}%
\pgfpathcurveto{\pgfqpoint{4.358405in}{3.153248in}}{\pgfqpoint{4.369004in}{3.148858in}}{\pgfqpoint{4.380054in}{3.148858in}}%
\pgfpathlineto{\pgfqpoint{4.380054in}{3.148858in}}%
\pgfpathclose%
\pgfusepath{stroke,fill}%
\end{pgfscope}%
\begin{pgfscope}%
\pgfpathrectangle{\pgfqpoint{2.963410in}{2.920818in}}{\pgfqpoint{2.177280in}{2.201755in}}%
\pgfusepath{clip}%
\pgfsetbuttcap%
\pgfsetroundjoin%
\definecolor{currentfill}{rgb}{0.121569,0.466667,0.705882}%
\pgfsetfillcolor{currentfill}%
\pgfsetlinewidth{0.481800pt}%
\definecolor{currentstroke}{rgb}{1.000000,1.000000,1.000000}%
\pgfsetstrokecolor{currentstroke}%
\pgfsetdash{}{0pt}%
\pgfpathmoveto{\pgfqpoint{4.143753in}{3.216709in}}%
\pgfpathcurveto{\pgfqpoint{4.154803in}{3.216709in}}{\pgfqpoint{4.165402in}{3.221099in}}{\pgfqpoint{4.173216in}{3.228913in}}%
\pgfpathcurveto{\pgfqpoint{4.181029in}{3.236726in}}{\pgfqpoint{4.185419in}{3.247325in}}{\pgfqpoint{4.185419in}{3.258375in}}%
\pgfpathcurveto{\pgfqpoint{4.185419in}{3.269426in}}{\pgfqpoint{4.181029in}{3.280025in}}{\pgfqpoint{4.173216in}{3.287838in}}%
\pgfpathcurveto{\pgfqpoint{4.165402in}{3.295652in}}{\pgfqpoint{4.154803in}{3.300042in}}{\pgfqpoint{4.143753in}{3.300042in}}%
\pgfpathcurveto{\pgfqpoint{4.132703in}{3.300042in}}{\pgfqpoint{4.122104in}{3.295652in}}{\pgfqpoint{4.114290in}{3.287838in}}%
\pgfpathcurveto{\pgfqpoint{4.106476in}{3.280025in}}{\pgfqpoint{4.102086in}{3.269426in}}{\pgfqpoint{4.102086in}{3.258375in}}%
\pgfpathcurveto{\pgfqpoint{4.102086in}{3.247325in}}{\pgfqpoint{4.106476in}{3.236726in}}{\pgfqpoint{4.114290in}{3.228913in}}%
\pgfpathcurveto{\pgfqpoint{4.122104in}{3.221099in}}{\pgfqpoint{4.132703in}{3.216709in}}{\pgfqpoint{4.143753in}{3.216709in}}%
\pgfpathlineto{\pgfqpoint{4.143753in}{3.216709in}}%
\pgfpathclose%
\pgfusepath{stroke,fill}%
\end{pgfscope}%
\begin{pgfscope}%
\pgfpathrectangle{\pgfqpoint{2.963410in}{2.920818in}}{\pgfqpoint{2.177280in}{2.201755in}}%
\pgfusepath{clip}%
\pgfsetbuttcap%
\pgfsetroundjoin%
\definecolor{currentfill}{rgb}{0.121569,0.466667,0.705882}%
\pgfsetfillcolor{currentfill}%
\pgfsetlinewidth{0.481800pt}%
\definecolor{currentstroke}{rgb}{1.000000,1.000000,1.000000}%
\pgfsetstrokecolor{currentstroke}%
\pgfsetdash{}{0pt}%
\pgfpathmoveto{\pgfqpoint{4.320979in}{3.148858in}}%
\pgfpathcurveto{\pgfqpoint{4.332029in}{3.148858in}}{\pgfqpoint{4.342628in}{3.153248in}}{\pgfqpoint{4.350441in}{3.161062in}}%
\pgfpathcurveto{\pgfqpoint{4.358255in}{3.168876in}}{\pgfqpoint{4.362645in}{3.179475in}}{\pgfqpoint{4.362645in}{3.190525in}}%
\pgfpathcurveto{\pgfqpoint{4.362645in}{3.201575in}}{\pgfqpoint{4.358255in}{3.212174in}}{\pgfqpoint{4.350441in}{3.219988in}}%
\pgfpathcurveto{\pgfqpoint{4.342628in}{3.227801in}}{\pgfqpoint{4.332029in}{3.232191in}}{\pgfqpoint{4.320979in}{3.232191in}}%
\pgfpathcurveto{\pgfqpoint{4.309928in}{3.232191in}}{\pgfqpoint{4.299329in}{3.227801in}}{\pgfqpoint{4.291516in}{3.219988in}}%
\pgfpathcurveto{\pgfqpoint{4.283702in}{3.212174in}}{\pgfqpoint{4.279312in}{3.201575in}}{\pgfqpoint{4.279312in}{3.190525in}}%
\pgfpathcurveto{\pgfqpoint{4.279312in}{3.179475in}}{\pgfqpoint{4.283702in}{3.168876in}}{\pgfqpoint{4.291516in}{3.161062in}}%
\pgfpathcurveto{\pgfqpoint{4.299329in}{3.153248in}}{\pgfqpoint{4.309928in}{3.148858in}}{\pgfqpoint{4.320979in}{3.148858in}}%
\pgfpathlineto{\pgfqpoint{4.320979in}{3.148858in}}%
\pgfpathclose%
\pgfusepath{stroke,fill}%
\end{pgfscope}%
\begin{pgfscope}%
\pgfpathrectangle{\pgfqpoint{2.963410in}{2.920818in}}{\pgfqpoint{2.177280in}{2.201755in}}%
\pgfusepath{clip}%
\pgfsetbuttcap%
\pgfsetroundjoin%
\definecolor{currentfill}{rgb}{0.121569,0.466667,0.705882}%
\pgfsetfillcolor{currentfill}%
\pgfsetlinewidth{0.481800pt}%
\definecolor{currentstroke}{rgb}{1.000000,1.000000,1.000000}%
\pgfsetstrokecolor{currentstroke}%
\pgfsetdash{}{0pt}%
\pgfpathmoveto{\pgfqpoint{4.261903in}{2.979231in}}%
\pgfpathcurveto{\pgfqpoint{4.272953in}{2.979231in}}{\pgfqpoint{4.283552in}{2.983622in}}{\pgfqpoint{4.291366in}{2.991435in}}%
\pgfpathcurveto{\pgfqpoint{4.299180in}{2.999249in}}{\pgfqpoint{4.303570in}{3.009848in}}{\pgfqpoint{4.303570in}{3.020898in}}%
\pgfpathcurveto{\pgfqpoint{4.303570in}{3.031948in}}{\pgfqpoint{4.299180in}{3.042547in}}{\pgfqpoint{4.291366in}{3.050361in}}%
\pgfpathcurveto{\pgfqpoint{4.283552in}{3.058174in}}{\pgfqpoint{4.272953in}{3.062565in}}{\pgfqpoint{4.261903in}{3.062565in}}%
\pgfpathcurveto{\pgfqpoint{4.250853in}{3.062565in}}{\pgfqpoint{4.240254in}{3.058174in}}{\pgfqpoint{4.232441in}{3.050361in}}%
\pgfpathcurveto{\pgfqpoint{4.224627in}{3.042547in}}{\pgfqpoint{4.220237in}{3.031948in}}{\pgfqpoint{4.220237in}{3.020898in}}%
\pgfpathcurveto{\pgfqpoint{4.220237in}{3.009848in}}{\pgfqpoint{4.224627in}{2.999249in}}{\pgfqpoint{4.232441in}{2.991435in}}%
\pgfpathcurveto{\pgfqpoint{4.240254in}{2.983622in}}{\pgfqpoint{4.250853in}{2.979231in}}{\pgfqpoint{4.261903in}{2.979231in}}%
\pgfpathlineto{\pgfqpoint{4.261903in}{2.979231in}}%
\pgfpathclose%
\pgfusepath{stroke,fill}%
\end{pgfscope}%
\begin{pgfscope}%
\pgfpathrectangle{\pgfqpoint{2.963410in}{2.920818in}}{\pgfqpoint{2.177280in}{2.201755in}}%
\pgfusepath{clip}%
\pgfsetbuttcap%
\pgfsetroundjoin%
\definecolor{currentfill}{rgb}{0.121569,0.466667,0.705882}%
\pgfsetfillcolor{currentfill}%
\pgfsetlinewidth{0.481800pt}%
\definecolor{currentstroke}{rgb}{1.000000,1.000000,1.000000}%
\pgfsetstrokecolor{currentstroke}%
\pgfsetdash{}{0pt}%
\pgfpathmoveto{\pgfqpoint{4.084678in}{3.216709in}}%
\pgfpathcurveto{\pgfqpoint{4.095728in}{3.216709in}}{\pgfqpoint{4.106327in}{3.221099in}}{\pgfqpoint{4.114140in}{3.228913in}}%
\pgfpathcurveto{\pgfqpoint{4.121954in}{3.236726in}}{\pgfqpoint{4.126344in}{3.247325in}}{\pgfqpoint{4.126344in}{3.258375in}}%
\pgfpathcurveto{\pgfqpoint{4.126344in}{3.269426in}}{\pgfqpoint{4.121954in}{3.280025in}}{\pgfqpoint{4.114140in}{3.287838in}}%
\pgfpathcurveto{\pgfqpoint{4.106327in}{3.295652in}}{\pgfqpoint{4.095728in}{3.300042in}}{\pgfqpoint{4.084678in}{3.300042in}}%
\pgfpathcurveto{\pgfqpoint{4.073627in}{3.300042in}}{\pgfqpoint{4.063028in}{3.295652in}}{\pgfqpoint{4.055215in}{3.287838in}}%
\pgfpathcurveto{\pgfqpoint{4.047401in}{3.280025in}}{\pgfqpoint{4.043011in}{3.269426in}}{\pgfqpoint{4.043011in}{3.258375in}}%
\pgfpathcurveto{\pgfqpoint{4.043011in}{3.247325in}}{\pgfqpoint{4.047401in}{3.236726in}}{\pgfqpoint{4.055215in}{3.228913in}}%
\pgfpathcurveto{\pgfqpoint{4.063028in}{3.221099in}}{\pgfqpoint{4.073627in}{3.216709in}}{\pgfqpoint{4.084678in}{3.216709in}}%
\pgfpathlineto{\pgfqpoint{4.084678in}{3.216709in}}%
\pgfpathclose%
\pgfusepath{stroke,fill}%
\end{pgfscope}%
\begin{pgfscope}%
\pgfpathrectangle{\pgfqpoint{2.963410in}{2.920818in}}{\pgfqpoint{2.177280in}{2.201755in}}%
\pgfusepath{clip}%
\pgfsetbuttcap%
\pgfsetroundjoin%
\definecolor{currentfill}{rgb}{0.121569,0.466667,0.705882}%
\pgfsetfillcolor{currentfill}%
\pgfsetlinewidth{0.481800pt}%
\definecolor{currentstroke}{rgb}{1.000000,1.000000,1.000000}%
\pgfsetstrokecolor{currentstroke}%
\pgfsetdash{}{0pt}%
\pgfpathmoveto{\pgfqpoint{4.143753in}{3.284560in}}%
\pgfpathcurveto{\pgfqpoint{4.154803in}{3.284560in}}{\pgfqpoint{4.165402in}{3.288950in}}{\pgfqpoint{4.173216in}{3.296763in}}%
\pgfpathcurveto{\pgfqpoint{4.181029in}{3.304577in}}{\pgfqpoint{4.185419in}{3.315176in}}{\pgfqpoint{4.185419in}{3.326226in}}%
\pgfpathcurveto{\pgfqpoint{4.185419in}{3.337276in}}{\pgfqpoint{4.181029in}{3.347875in}}{\pgfqpoint{4.173216in}{3.355689in}}%
\pgfpathcurveto{\pgfqpoint{4.165402in}{3.363503in}}{\pgfqpoint{4.154803in}{3.367893in}}{\pgfqpoint{4.143753in}{3.367893in}}%
\pgfpathcurveto{\pgfqpoint{4.132703in}{3.367893in}}{\pgfqpoint{4.122104in}{3.363503in}}{\pgfqpoint{4.114290in}{3.355689in}}%
\pgfpathcurveto{\pgfqpoint{4.106476in}{3.347875in}}{\pgfqpoint{4.102086in}{3.337276in}}{\pgfqpoint{4.102086in}{3.326226in}}%
\pgfpathcurveto{\pgfqpoint{4.102086in}{3.315176in}}{\pgfqpoint{4.106476in}{3.304577in}}{\pgfqpoint{4.114290in}{3.296763in}}%
\pgfpathcurveto{\pgfqpoint{4.122104in}{3.288950in}}{\pgfqpoint{4.132703in}{3.284560in}}{\pgfqpoint{4.143753in}{3.284560in}}%
\pgfpathlineto{\pgfqpoint{4.143753in}{3.284560in}}%
\pgfpathclose%
\pgfusepath{stroke,fill}%
\end{pgfscope}%
\begin{pgfscope}%
\pgfpathrectangle{\pgfqpoint{2.963410in}{2.920818in}}{\pgfqpoint{2.177280in}{2.201755in}}%
\pgfusepath{clip}%
\pgfsetbuttcap%
\pgfsetroundjoin%
\definecolor{currentfill}{rgb}{0.121569,0.466667,0.705882}%
\pgfsetfillcolor{currentfill}%
\pgfsetlinewidth{0.481800pt}%
\definecolor{currentstroke}{rgb}{1.000000,1.000000,1.000000}%
\pgfsetstrokecolor{currentstroke}%
\pgfsetdash{}{0pt}%
\pgfpathmoveto{\pgfqpoint{3.907452in}{3.182783in}}%
\pgfpathcurveto{\pgfqpoint{3.918502in}{3.182783in}}{\pgfqpoint{3.929101in}{3.187174in}}{\pgfqpoint{3.936915in}{3.194987in}}%
\pgfpathcurveto{\pgfqpoint{3.944728in}{3.202801in}}{\pgfqpoint{3.949118in}{3.213400in}}{\pgfqpoint{3.949118in}{3.224450in}}%
\pgfpathcurveto{\pgfqpoint{3.949118in}{3.235500in}}{\pgfqpoint{3.944728in}{3.246099in}}{\pgfqpoint{3.936915in}{3.253913in}}%
\pgfpathcurveto{\pgfqpoint{3.929101in}{3.261727in}}{\pgfqpoint{3.918502in}{3.266117in}}{\pgfqpoint{3.907452in}{3.266117in}}%
\pgfpathcurveto{\pgfqpoint{3.896402in}{3.266117in}}{\pgfqpoint{3.885803in}{3.261727in}}{\pgfqpoint{3.877989in}{3.253913in}}%
\pgfpathcurveto{\pgfqpoint{3.870175in}{3.246099in}}{\pgfqpoint{3.865785in}{3.235500in}}{\pgfqpoint{3.865785in}{3.224450in}}%
\pgfpathcurveto{\pgfqpoint{3.865785in}{3.213400in}}{\pgfqpoint{3.870175in}{3.202801in}}{\pgfqpoint{3.877989in}{3.194987in}}%
\pgfpathcurveto{\pgfqpoint{3.885803in}{3.187174in}}{\pgfqpoint{3.896402in}{3.182783in}}{\pgfqpoint{3.907452in}{3.182783in}}%
\pgfpathlineto{\pgfqpoint{3.907452in}{3.182783in}}%
\pgfpathclose%
\pgfusepath{stroke,fill}%
\end{pgfscope}%
\begin{pgfscope}%
\pgfpathrectangle{\pgfqpoint{2.963410in}{2.920818in}}{\pgfqpoint{2.177280in}{2.201755in}}%
\pgfusepath{clip}%
\pgfsetbuttcap%
\pgfsetroundjoin%
\definecolor{currentfill}{rgb}{0.121569,0.466667,0.705882}%
\pgfsetfillcolor{currentfill}%
\pgfsetlinewidth{0.481800pt}%
\definecolor{currentstroke}{rgb}{1.000000,1.000000,1.000000}%
\pgfsetstrokecolor{currentstroke}%
\pgfsetdash{}{0pt}%
\pgfpathmoveto{\pgfqpoint{4.143753in}{3.182783in}}%
\pgfpathcurveto{\pgfqpoint{4.154803in}{3.182783in}}{\pgfqpoint{4.165402in}{3.187174in}}{\pgfqpoint{4.173216in}{3.194987in}}%
\pgfpathcurveto{\pgfqpoint{4.181029in}{3.202801in}}{\pgfqpoint{4.185419in}{3.213400in}}{\pgfqpoint{4.185419in}{3.224450in}}%
\pgfpathcurveto{\pgfqpoint{4.185419in}{3.235500in}}{\pgfqpoint{4.181029in}{3.246099in}}{\pgfqpoint{4.173216in}{3.253913in}}%
\pgfpathcurveto{\pgfqpoint{4.165402in}{3.261727in}}{\pgfqpoint{4.154803in}{3.266117in}}{\pgfqpoint{4.143753in}{3.266117in}}%
\pgfpathcurveto{\pgfqpoint{4.132703in}{3.266117in}}{\pgfqpoint{4.122104in}{3.261727in}}{\pgfqpoint{4.114290in}{3.253913in}}%
\pgfpathcurveto{\pgfqpoint{4.106476in}{3.246099in}}{\pgfqpoint{4.102086in}{3.235500in}}{\pgfqpoint{4.102086in}{3.224450in}}%
\pgfpathcurveto{\pgfqpoint{4.102086in}{3.213400in}}{\pgfqpoint{4.106476in}{3.202801in}}{\pgfqpoint{4.114290in}{3.194987in}}%
\pgfpathcurveto{\pgfqpoint{4.122104in}{3.187174in}}{\pgfqpoint{4.132703in}{3.182783in}}{\pgfqpoint{4.143753in}{3.182783in}}%
\pgfpathlineto{\pgfqpoint{4.143753in}{3.182783in}}%
\pgfpathclose%
\pgfusepath{stroke,fill}%
\end{pgfscope}%
\begin{pgfscope}%
\pgfpathrectangle{\pgfqpoint{2.963410in}{2.920818in}}{\pgfqpoint{2.177280in}{2.201755in}}%
\pgfusepath{clip}%
\pgfsetbuttcap%
\pgfsetroundjoin%
\definecolor{currentfill}{rgb}{0.121569,0.466667,0.705882}%
\pgfsetfillcolor{currentfill}%
\pgfsetlinewidth{0.481800pt}%
\definecolor{currentstroke}{rgb}{1.000000,1.000000,1.000000}%
\pgfsetstrokecolor{currentstroke}%
\pgfsetdash{}{0pt}%
\pgfpathmoveto{\pgfqpoint{4.202828in}{3.148858in}}%
\pgfpathcurveto{\pgfqpoint{4.213878in}{3.148858in}}{\pgfqpoint{4.224477in}{3.153248in}}{\pgfqpoint{4.232291in}{3.161062in}}%
\pgfpathcurveto{\pgfqpoint{4.240104in}{3.168876in}}{\pgfqpoint{4.244495in}{3.179475in}}{\pgfqpoint{4.244495in}{3.190525in}}%
\pgfpathcurveto{\pgfqpoint{4.244495in}{3.201575in}}{\pgfqpoint{4.240104in}{3.212174in}}{\pgfqpoint{4.232291in}{3.219988in}}%
\pgfpathcurveto{\pgfqpoint{4.224477in}{3.227801in}}{\pgfqpoint{4.213878in}{3.232191in}}{\pgfqpoint{4.202828in}{3.232191in}}%
\pgfpathcurveto{\pgfqpoint{4.191778in}{3.232191in}}{\pgfqpoint{4.181179in}{3.227801in}}{\pgfqpoint{4.173365in}{3.219988in}}%
\pgfpathcurveto{\pgfqpoint{4.165552in}{3.212174in}}{\pgfqpoint{4.161161in}{3.201575in}}{\pgfqpoint{4.161161in}{3.190525in}}%
\pgfpathcurveto{\pgfqpoint{4.161161in}{3.179475in}}{\pgfqpoint{4.165552in}{3.168876in}}{\pgfqpoint{4.173365in}{3.161062in}}%
\pgfpathcurveto{\pgfqpoint{4.181179in}{3.153248in}}{\pgfqpoint{4.191778in}{3.148858in}}{\pgfqpoint{4.202828in}{3.148858in}}%
\pgfpathlineto{\pgfqpoint{4.202828in}{3.148858in}}%
\pgfpathclose%
\pgfusepath{stroke,fill}%
\end{pgfscope}%
\begin{pgfscope}%
\pgfpathrectangle{\pgfqpoint{2.963410in}{2.920818in}}{\pgfqpoint{2.177280in}{2.201755in}}%
\pgfusepath{clip}%
\pgfsetbuttcap%
\pgfsetroundjoin%
\definecolor{currentfill}{rgb}{0.121569,0.466667,0.705882}%
\pgfsetfillcolor{currentfill}%
\pgfsetlinewidth{0.481800pt}%
\definecolor{currentstroke}{rgb}{1.000000,1.000000,1.000000}%
\pgfsetstrokecolor{currentstroke}%
\pgfsetdash{}{0pt}%
\pgfpathmoveto{\pgfqpoint{4.143753in}{3.114933in}}%
\pgfpathcurveto{\pgfqpoint{4.154803in}{3.114933in}}{\pgfqpoint{4.165402in}{3.119323in}}{\pgfqpoint{4.173216in}{3.127137in}}%
\pgfpathcurveto{\pgfqpoint{4.181029in}{3.134950in}}{\pgfqpoint{4.185419in}{3.145549in}}{\pgfqpoint{4.185419in}{3.156599in}}%
\pgfpathcurveto{\pgfqpoint{4.185419in}{3.167650in}}{\pgfqpoint{4.181029in}{3.178249in}}{\pgfqpoint{4.173216in}{3.186062in}}%
\pgfpathcurveto{\pgfqpoint{4.165402in}{3.193876in}}{\pgfqpoint{4.154803in}{3.198266in}}{\pgfqpoint{4.143753in}{3.198266in}}%
\pgfpathcurveto{\pgfqpoint{4.132703in}{3.198266in}}{\pgfqpoint{4.122104in}{3.193876in}}{\pgfqpoint{4.114290in}{3.186062in}}%
\pgfpathcurveto{\pgfqpoint{4.106476in}{3.178249in}}{\pgfqpoint{4.102086in}{3.167650in}}{\pgfqpoint{4.102086in}{3.156599in}}%
\pgfpathcurveto{\pgfqpoint{4.102086in}{3.145549in}}{\pgfqpoint{4.106476in}{3.134950in}}{\pgfqpoint{4.114290in}{3.127137in}}%
\pgfpathcurveto{\pgfqpoint{4.122104in}{3.119323in}}{\pgfqpoint{4.132703in}{3.114933in}}{\pgfqpoint{4.143753in}{3.114933in}}%
\pgfpathlineto{\pgfqpoint{4.143753in}{3.114933in}}%
\pgfpathclose%
\pgfusepath{stroke,fill}%
\end{pgfscope}%
\begin{pgfscope}%
\pgfpathrectangle{\pgfqpoint{2.963410in}{2.920818in}}{\pgfqpoint{2.177280in}{2.201755in}}%
\pgfusepath{clip}%
\pgfsetbuttcap%
\pgfsetroundjoin%
\definecolor{currentfill}{rgb}{0.121569,0.466667,0.705882}%
\pgfsetfillcolor{currentfill}%
\pgfsetlinewidth{0.481800pt}%
\definecolor{currentstroke}{rgb}{1.000000,1.000000,1.000000}%
\pgfsetstrokecolor{currentstroke}%
\pgfsetdash{}{0pt}%
\pgfpathmoveto{\pgfqpoint{4.025602in}{3.182783in}}%
\pgfpathcurveto{\pgfqpoint{4.036652in}{3.182783in}}{\pgfqpoint{4.047251in}{3.187174in}}{\pgfqpoint{4.055065in}{3.194987in}}%
\pgfpathcurveto{\pgfqpoint{4.062879in}{3.202801in}}{\pgfqpoint{4.067269in}{3.213400in}}{\pgfqpoint{4.067269in}{3.224450in}}%
\pgfpathcurveto{\pgfqpoint{4.067269in}{3.235500in}}{\pgfqpoint{4.062879in}{3.246099in}}{\pgfqpoint{4.055065in}{3.253913in}}%
\pgfpathcurveto{\pgfqpoint{4.047251in}{3.261727in}}{\pgfqpoint{4.036652in}{3.266117in}}{\pgfqpoint{4.025602in}{3.266117in}}%
\pgfpathcurveto{\pgfqpoint{4.014552in}{3.266117in}}{\pgfqpoint{4.003953in}{3.261727in}}{\pgfqpoint{3.996139in}{3.253913in}}%
\pgfpathcurveto{\pgfqpoint{3.988326in}{3.246099in}}{\pgfqpoint{3.983936in}{3.235500in}}{\pgfqpoint{3.983936in}{3.224450in}}%
\pgfpathcurveto{\pgfqpoint{3.983936in}{3.213400in}}{\pgfqpoint{3.988326in}{3.202801in}}{\pgfqpoint{3.996139in}{3.194987in}}%
\pgfpathcurveto{\pgfqpoint{4.003953in}{3.187174in}}{\pgfqpoint{4.014552in}{3.182783in}}{\pgfqpoint{4.025602in}{3.182783in}}%
\pgfpathlineto{\pgfqpoint{4.025602in}{3.182783in}}%
\pgfpathclose%
\pgfusepath{stroke,fill}%
\end{pgfscope}%
\begin{pgfscope}%
\pgfpathrectangle{\pgfqpoint{2.963410in}{2.920818in}}{\pgfqpoint{2.177280in}{2.201755in}}%
\pgfusepath{clip}%
\pgfsetbuttcap%
\pgfsetroundjoin%
\definecolor{currentfill}{rgb}{0.121569,0.466667,0.705882}%
\pgfsetfillcolor{currentfill}%
\pgfsetlinewidth{0.481800pt}%
\definecolor{currentstroke}{rgb}{1.000000,1.000000,1.000000}%
\pgfsetstrokecolor{currentstroke}%
\pgfsetdash{}{0pt}%
\pgfpathmoveto{\pgfqpoint{3.966527in}{3.182783in}}%
\pgfpathcurveto{\pgfqpoint{3.977577in}{3.182783in}}{\pgfqpoint{3.988176in}{3.187174in}}{\pgfqpoint{3.995990in}{3.194987in}}%
\pgfpathcurveto{\pgfqpoint{4.003803in}{3.202801in}}{\pgfqpoint{4.008194in}{3.213400in}}{\pgfqpoint{4.008194in}{3.224450in}}%
\pgfpathcurveto{\pgfqpoint{4.008194in}{3.235500in}}{\pgfqpoint{4.003803in}{3.246099in}}{\pgfqpoint{3.995990in}{3.253913in}}%
\pgfpathcurveto{\pgfqpoint{3.988176in}{3.261727in}}{\pgfqpoint{3.977577in}{3.266117in}}{\pgfqpoint{3.966527in}{3.266117in}}%
\pgfpathcurveto{\pgfqpoint{3.955477in}{3.266117in}}{\pgfqpoint{3.944878in}{3.261727in}}{\pgfqpoint{3.937064in}{3.253913in}}%
\pgfpathcurveto{\pgfqpoint{3.929251in}{3.246099in}}{\pgfqpoint{3.924860in}{3.235500in}}{\pgfqpoint{3.924860in}{3.224450in}}%
\pgfpathcurveto{\pgfqpoint{3.924860in}{3.213400in}}{\pgfqpoint{3.929251in}{3.202801in}}{\pgfqpoint{3.937064in}{3.194987in}}%
\pgfpathcurveto{\pgfqpoint{3.944878in}{3.187174in}}{\pgfqpoint{3.955477in}{3.182783in}}{\pgfqpoint{3.966527in}{3.182783in}}%
\pgfpathlineto{\pgfqpoint{3.966527in}{3.182783in}}%
\pgfpathclose%
\pgfusepath{stroke,fill}%
\end{pgfscope}%
\begin{pgfscope}%
\pgfpathrectangle{\pgfqpoint{2.963410in}{2.920818in}}{\pgfqpoint{2.177280in}{2.201755in}}%
\pgfusepath{clip}%
\pgfsetbuttcap%
\pgfsetroundjoin%
\definecolor{currentfill}{rgb}{0.121569,0.466667,0.705882}%
\pgfsetfillcolor{currentfill}%
\pgfsetlinewidth{0.481800pt}%
\definecolor{currentstroke}{rgb}{1.000000,1.000000,1.000000}%
\pgfsetstrokecolor{currentstroke}%
\pgfsetdash{}{0pt}%
\pgfpathmoveto{\pgfqpoint{4.143753in}{3.148858in}}%
\pgfpathcurveto{\pgfqpoint{4.154803in}{3.148858in}}{\pgfqpoint{4.165402in}{3.153248in}}{\pgfqpoint{4.173216in}{3.161062in}}%
\pgfpathcurveto{\pgfqpoint{4.181029in}{3.168876in}}{\pgfqpoint{4.185419in}{3.179475in}}{\pgfqpoint{4.185419in}{3.190525in}}%
\pgfpathcurveto{\pgfqpoint{4.185419in}{3.201575in}}{\pgfqpoint{4.181029in}{3.212174in}}{\pgfqpoint{4.173216in}{3.219988in}}%
\pgfpathcurveto{\pgfqpoint{4.165402in}{3.227801in}}{\pgfqpoint{4.154803in}{3.232191in}}{\pgfqpoint{4.143753in}{3.232191in}}%
\pgfpathcurveto{\pgfqpoint{4.132703in}{3.232191in}}{\pgfqpoint{4.122104in}{3.227801in}}{\pgfqpoint{4.114290in}{3.219988in}}%
\pgfpathcurveto{\pgfqpoint{4.106476in}{3.212174in}}{\pgfqpoint{4.102086in}{3.201575in}}{\pgfqpoint{4.102086in}{3.190525in}}%
\pgfpathcurveto{\pgfqpoint{4.102086in}{3.179475in}}{\pgfqpoint{4.106476in}{3.168876in}}{\pgfqpoint{4.114290in}{3.161062in}}%
\pgfpathcurveto{\pgfqpoint{4.122104in}{3.153248in}}{\pgfqpoint{4.132703in}{3.148858in}}{\pgfqpoint{4.143753in}{3.148858in}}%
\pgfpathlineto{\pgfqpoint{4.143753in}{3.148858in}}%
\pgfpathclose%
\pgfusepath{stroke,fill}%
\end{pgfscope}%
\begin{pgfscope}%
\pgfpathrectangle{\pgfqpoint{2.963410in}{2.920818in}}{\pgfqpoint{2.177280in}{2.201755in}}%
\pgfusepath{clip}%
\pgfsetbuttcap%
\pgfsetroundjoin%
\definecolor{currentfill}{rgb}{0.121569,0.466667,0.705882}%
\pgfsetfillcolor{currentfill}%
\pgfsetlinewidth{0.481800pt}%
\definecolor{currentstroke}{rgb}{1.000000,1.000000,1.000000}%
\pgfsetstrokecolor{currentstroke}%
\pgfsetdash{}{0pt}%
\pgfpathmoveto{\pgfqpoint{4.557280in}{3.148858in}}%
\pgfpathcurveto{\pgfqpoint{4.568330in}{3.148858in}}{\pgfqpoint{4.578929in}{3.153248in}}{\pgfqpoint{4.586742in}{3.161062in}}%
\pgfpathcurveto{\pgfqpoint{4.594556in}{3.168876in}}{\pgfqpoint{4.598946in}{3.179475in}}{\pgfqpoint{4.598946in}{3.190525in}}%
\pgfpathcurveto{\pgfqpoint{4.598946in}{3.201575in}}{\pgfqpoint{4.594556in}{3.212174in}}{\pgfqpoint{4.586742in}{3.219988in}}%
\pgfpathcurveto{\pgfqpoint{4.578929in}{3.227801in}}{\pgfqpoint{4.568330in}{3.232191in}}{\pgfqpoint{4.557280in}{3.232191in}}%
\pgfpathcurveto{\pgfqpoint{4.546230in}{3.232191in}}{\pgfqpoint{4.535631in}{3.227801in}}{\pgfqpoint{4.527817in}{3.219988in}}%
\pgfpathcurveto{\pgfqpoint{4.520003in}{3.212174in}}{\pgfqpoint{4.515613in}{3.201575in}}{\pgfqpoint{4.515613in}{3.190525in}}%
\pgfpathcurveto{\pgfqpoint{4.515613in}{3.179475in}}{\pgfqpoint{4.520003in}{3.168876in}}{\pgfqpoint{4.527817in}{3.161062in}}%
\pgfpathcurveto{\pgfqpoint{4.535631in}{3.153248in}}{\pgfqpoint{4.546230in}{3.148858in}}{\pgfqpoint{4.557280in}{3.148858in}}%
\pgfpathlineto{\pgfqpoint{4.557280in}{3.148858in}}%
\pgfpathclose%
\pgfusepath{stroke,fill}%
\end{pgfscope}%
\begin{pgfscope}%
\pgfpathrectangle{\pgfqpoint{2.963410in}{2.920818in}}{\pgfqpoint{2.177280in}{2.201755in}}%
\pgfusepath{clip}%
\pgfsetbuttcap%
\pgfsetroundjoin%
\definecolor{currentfill}{rgb}{0.121569,0.466667,0.705882}%
\pgfsetfillcolor{currentfill}%
\pgfsetlinewidth{0.481800pt}%
\definecolor{currentstroke}{rgb}{1.000000,1.000000,1.000000}%
\pgfsetstrokecolor{currentstroke}%
\pgfsetdash{}{0pt}%
\pgfpathmoveto{\pgfqpoint{4.616355in}{3.114933in}}%
\pgfpathcurveto{\pgfqpoint{4.627405in}{3.114933in}}{\pgfqpoint{4.638004in}{3.119323in}}{\pgfqpoint{4.645818in}{3.127137in}}%
\pgfpathcurveto{\pgfqpoint{4.653631in}{3.134950in}}{\pgfqpoint{4.658022in}{3.145549in}}{\pgfqpoint{4.658022in}{3.156599in}}%
\pgfpathcurveto{\pgfqpoint{4.658022in}{3.167650in}}{\pgfqpoint{4.653631in}{3.178249in}}{\pgfqpoint{4.645818in}{3.186062in}}%
\pgfpathcurveto{\pgfqpoint{4.638004in}{3.193876in}}{\pgfqpoint{4.627405in}{3.198266in}}{\pgfqpoint{4.616355in}{3.198266in}}%
\pgfpathcurveto{\pgfqpoint{4.605305in}{3.198266in}}{\pgfqpoint{4.594706in}{3.193876in}}{\pgfqpoint{4.586892in}{3.186062in}}%
\pgfpathcurveto{\pgfqpoint{4.579079in}{3.178249in}}{\pgfqpoint{4.574688in}{3.167650in}}{\pgfqpoint{4.574688in}{3.156599in}}%
\pgfpathcurveto{\pgfqpoint{4.574688in}{3.145549in}}{\pgfqpoint{4.579079in}{3.134950in}}{\pgfqpoint{4.586892in}{3.127137in}}%
\pgfpathcurveto{\pgfqpoint{4.594706in}{3.119323in}}{\pgfqpoint{4.605305in}{3.114933in}}{\pgfqpoint{4.616355in}{3.114933in}}%
\pgfpathlineto{\pgfqpoint{4.616355in}{3.114933in}}%
\pgfpathclose%
\pgfusepath{stroke,fill}%
\end{pgfscope}%
\begin{pgfscope}%
\pgfpathrectangle{\pgfqpoint{2.963410in}{2.920818in}}{\pgfqpoint{2.177280in}{2.201755in}}%
\pgfusepath{clip}%
\pgfsetbuttcap%
\pgfsetroundjoin%
\definecolor{currentfill}{rgb}{0.121569,0.466667,0.705882}%
\pgfsetfillcolor{currentfill}%
\pgfsetlinewidth{0.481800pt}%
\definecolor{currentstroke}{rgb}{1.000000,1.000000,1.000000}%
\pgfsetstrokecolor{currentstroke}%
\pgfsetdash{}{0pt}%
\pgfpathmoveto{\pgfqpoint{3.966527in}{3.148858in}}%
\pgfpathcurveto{\pgfqpoint{3.977577in}{3.148858in}}{\pgfqpoint{3.988176in}{3.153248in}}{\pgfqpoint{3.995990in}{3.161062in}}%
\pgfpathcurveto{\pgfqpoint{4.003803in}{3.168876in}}{\pgfqpoint{4.008194in}{3.179475in}}{\pgfqpoint{4.008194in}{3.190525in}}%
\pgfpathcurveto{\pgfqpoint{4.008194in}{3.201575in}}{\pgfqpoint{4.003803in}{3.212174in}}{\pgfqpoint{3.995990in}{3.219988in}}%
\pgfpathcurveto{\pgfqpoint{3.988176in}{3.227801in}}{\pgfqpoint{3.977577in}{3.232191in}}{\pgfqpoint{3.966527in}{3.232191in}}%
\pgfpathcurveto{\pgfqpoint{3.955477in}{3.232191in}}{\pgfqpoint{3.944878in}{3.227801in}}{\pgfqpoint{3.937064in}{3.219988in}}%
\pgfpathcurveto{\pgfqpoint{3.929251in}{3.212174in}}{\pgfqpoint{3.924860in}{3.201575in}}{\pgfqpoint{3.924860in}{3.190525in}}%
\pgfpathcurveto{\pgfqpoint{3.924860in}{3.179475in}}{\pgfqpoint{3.929251in}{3.168876in}}{\pgfqpoint{3.937064in}{3.161062in}}%
\pgfpathcurveto{\pgfqpoint{3.944878in}{3.153248in}}{\pgfqpoint{3.955477in}{3.148858in}}{\pgfqpoint{3.966527in}{3.148858in}}%
\pgfpathlineto{\pgfqpoint{3.966527in}{3.148858in}}%
\pgfpathclose%
\pgfusepath{stroke,fill}%
\end{pgfscope}%
\begin{pgfscope}%
\pgfpathrectangle{\pgfqpoint{2.963410in}{2.920818in}}{\pgfqpoint{2.177280in}{2.201755in}}%
\pgfusepath{clip}%
\pgfsetbuttcap%
\pgfsetroundjoin%
\definecolor{currentfill}{rgb}{0.121569,0.466667,0.705882}%
\pgfsetfillcolor{currentfill}%
\pgfsetlinewidth{0.481800pt}%
\definecolor{currentstroke}{rgb}{1.000000,1.000000,1.000000}%
\pgfsetstrokecolor{currentstroke}%
\pgfsetdash{}{0pt}%
\pgfpathmoveto{\pgfqpoint{4.025602in}{3.047082in}}%
\pgfpathcurveto{\pgfqpoint{4.036652in}{3.047082in}}{\pgfqpoint{4.047251in}{3.051472in}}{\pgfqpoint{4.055065in}{3.059286in}}%
\pgfpathcurveto{\pgfqpoint{4.062879in}{3.067100in}}{\pgfqpoint{4.067269in}{3.077699in}}{\pgfqpoint{4.067269in}{3.088749in}}%
\pgfpathcurveto{\pgfqpoint{4.067269in}{3.099799in}}{\pgfqpoint{4.062879in}{3.110398in}}{\pgfqpoint{4.055065in}{3.118212in}}%
\pgfpathcurveto{\pgfqpoint{4.047251in}{3.126025in}}{\pgfqpoint{4.036652in}{3.130415in}}{\pgfqpoint{4.025602in}{3.130415in}}%
\pgfpathcurveto{\pgfqpoint{4.014552in}{3.130415in}}{\pgfqpoint{4.003953in}{3.126025in}}{\pgfqpoint{3.996139in}{3.118212in}}%
\pgfpathcurveto{\pgfqpoint{3.988326in}{3.110398in}}{\pgfqpoint{3.983936in}{3.099799in}}{\pgfqpoint{3.983936in}{3.088749in}}%
\pgfpathcurveto{\pgfqpoint{3.983936in}{3.077699in}}{\pgfqpoint{3.988326in}{3.067100in}}{\pgfqpoint{3.996139in}{3.059286in}}%
\pgfpathcurveto{\pgfqpoint{4.003953in}{3.051472in}}{\pgfqpoint{4.014552in}{3.047082in}}{\pgfqpoint{4.025602in}{3.047082in}}%
\pgfpathlineto{\pgfqpoint{4.025602in}{3.047082in}}%
\pgfpathclose%
\pgfusepath{stroke,fill}%
\end{pgfscope}%
\begin{pgfscope}%
\pgfpathrectangle{\pgfqpoint{2.963410in}{2.920818in}}{\pgfqpoint{2.177280in}{2.201755in}}%
\pgfusepath{clip}%
\pgfsetbuttcap%
\pgfsetroundjoin%
\definecolor{currentfill}{rgb}{0.121569,0.466667,0.705882}%
\pgfsetfillcolor{currentfill}%
\pgfsetlinewidth{0.481800pt}%
\definecolor{currentstroke}{rgb}{1.000000,1.000000,1.000000}%
\pgfsetstrokecolor{currentstroke}%
\pgfsetdash{}{0pt}%
\pgfpathmoveto{\pgfqpoint{4.202828in}{3.081007in}}%
\pgfpathcurveto{\pgfqpoint{4.213878in}{3.081007in}}{\pgfqpoint{4.224477in}{3.085398in}}{\pgfqpoint{4.232291in}{3.093211in}}%
\pgfpathcurveto{\pgfqpoint{4.240104in}{3.101025in}}{\pgfqpoint{4.244495in}{3.111624in}}{\pgfqpoint{4.244495in}{3.122674in}}%
\pgfpathcurveto{\pgfqpoint{4.244495in}{3.133724in}}{\pgfqpoint{4.240104in}{3.144323in}}{\pgfqpoint{4.232291in}{3.152137in}}%
\pgfpathcurveto{\pgfqpoint{4.224477in}{3.159951in}}{\pgfqpoint{4.213878in}{3.164341in}}{\pgfqpoint{4.202828in}{3.164341in}}%
\pgfpathcurveto{\pgfqpoint{4.191778in}{3.164341in}}{\pgfqpoint{4.181179in}{3.159951in}}{\pgfqpoint{4.173365in}{3.152137in}}%
\pgfpathcurveto{\pgfqpoint{4.165552in}{3.144323in}}{\pgfqpoint{4.161161in}{3.133724in}}{\pgfqpoint{4.161161in}{3.122674in}}%
\pgfpathcurveto{\pgfqpoint{4.161161in}{3.111624in}}{\pgfqpoint{4.165552in}{3.101025in}}{\pgfqpoint{4.173365in}{3.093211in}}%
\pgfpathcurveto{\pgfqpoint{4.181179in}{3.085398in}}{\pgfqpoint{4.191778in}{3.081007in}}{\pgfqpoint{4.202828in}{3.081007in}}%
\pgfpathlineto{\pgfqpoint{4.202828in}{3.081007in}}%
\pgfpathclose%
\pgfusepath{stroke,fill}%
\end{pgfscope}%
\begin{pgfscope}%
\pgfpathrectangle{\pgfqpoint{2.963410in}{2.920818in}}{\pgfqpoint{2.177280in}{2.201755in}}%
\pgfusepath{clip}%
\pgfsetbuttcap%
\pgfsetroundjoin%
\definecolor{currentfill}{rgb}{0.121569,0.466667,0.705882}%
\pgfsetfillcolor{currentfill}%
\pgfsetlinewidth{0.481800pt}%
\definecolor{currentstroke}{rgb}{1.000000,1.000000,1.000000}%
\pgfsetstrokecolor{currentstroke}%
\pgfsetdash{}{0pt}%
\pgfpathmoveto{\pgfqpoint{4.261903in}{3.114933in}}%
\pgfpathcurveto{\pgfqpoint{4.272953in}{3.114933in}}{\pgfqpoint{4.283552in}{3.119323in}}{\pgfqpoint{4.291366in}{3.127137in}}%
\pgfpathcurveto{\pgfqpoint{4.299180in}{3.134950in}}{\pgfqpoint{4.303570in}{3.145549in}}{\pgfqpoint{4.303570in}{3.156599in}}%
\pgfpathcurveto{\pgfqpoint{4.303570in}{3.167650in}}{\pgfqpoint{4.299180in}{3.178249in}}{\pgfqpoint{4.291366in}{3.186062in}}%
\pgfpathcurveto{\pgfqpoint{4.283552in}{3.193876in}}{\pgfqpoint{4.272953in}{3.198266in}}{\pgfqpoint{4.261903in}{3.198266in}}%
\pgfpathcurveto{\pgfqpoint{4.250853in}{3.198266in}}{\pgfqpoint{4.240254in}{3.193876in}}{\pgfqpoint{4.232441in}{3.186062in}}%
\pgfpathcurveto{\pgfqpoint{4.224627in}{3.178249in}}{\pgfqpoint{4.220237in}{3.167650in}}{\pgfqpoint{4.220237in}{3.156599in}}%
\pgfpathcurveto{\pgfqpoint{4.220237in}{3.145549in}}{\pgfqpoint{4.224627in}{3.134950in}}{\pgfqpoint{4.232441in}{3.127137in}}%
\pgfpathcurveto{\pgfqpoint{4.240254in}{3.119323in}}{\pgfqpoint{4.250853in}{3.114933in}}{\pgfqpoint{4.261903in}{3.114933in}}%
\pgfpathlineto{\pgfqpoint{4.261903in}{3.114933in}}%
\pgfpathclose%
\pgfusepath{stroke,fill}%
\end{pgfscope}%
\begin{pgfscope}%
\pgfpathrectangle{\pgfqpoint{2.963410in}{2.920818in}}{\pgfqpoint{2.177280in}{2.201755in}}%
\pgfusepath{clip}%
\pgfsetbuttcap%
\pgfsetroundjoin%
\definecolor{currentfill}{rgb}{0.121569,0.466667,0.705882}%
\pgfsetfillcolor{currentfill}%
\pgfsetlinewidth{0.481800pt}%
\definecolor{currentstroke}{rgb}{1.000000,1.000000,1.000000}%
\pgfsetstrokecolor{currentstroke}%
\pgfsetdash{}{0pt}%
\pgfpathmoveto{\pgfqpoint{3.907452in}{3.081007in}}%
\pgfpathcurveto{\pgfqpoint{3.918502in}{3.081007in}}{\pgfqpoint{3.929101in}{3.085398in}}{\pgfqpoint{3.936915in}{3.093211in}}%
\pgfpathcurveto{\pgfqpoint{3.944728in}{3.101025in}}{\pgfqpoint{3.949118in}{3.111624in}}{\pgfqpoint{3.949118in}{3.122674in}}%
\pgfpathcurveto{\pgfqpoint{3.949118in}{3.133724in}}{\pgfqpoint{3.944728in}{3.144323in}}{\pgfqpoint{3.936915in}{3.152137in}}%
\pgfpathcurveto{\pgfqpoint{3.929101in}{3.159951in}}{\pgfqpoint{3.918502in}{3.164341in}}{\pgfqpoint{3.907452in}{3.164341in}}%
\pgfpathcurveto{\pgfqpoint{3.896402in}{3.164341in}}{\pgfqpoint{3.885803in}{3.159951in}}{\pgfqpoint{3.877989in}{3.152137in}}%
\pgfpathcurveto{\pgfqpoint{3.870175in}{3.144323in}}{\pgfqpoint{3.865785in}{3.133724in}}{\pgfqpoint{3.865785in}{3.122674in}}%
\pgfpathcurveto{\pgfqpoint{3.865785in}{3.111624in}}{\pgfqpoint{3.870175in}{3.101025in}}{\pgfqpoint{3.877989in}{3.093211in}}%
\pgfpathcurveto{\pgfqpoint{3.885803in}{3.085398in}}{\pgfqpoint{3.896402in}{3.081007in}}{\pgfqpoint{3.907452in}{3.081007in}}%
\pgfpathlineto{\pgfqpoint{3.907452in}{3.081007in}}%
\pgfpathclose%
\pgfusepath{stroke,fill}%
\end{pgfscope}%
\begin{pgfscope}%
\pgfpathrectangle{\pgfqpoint{2.963410in}{2.920818in}}{\pgfqpoint{2.177280in}{2.201755in}}%
\pgfusepath{clip}%
\pgfsetbuttcap%
\pgfsetroundjoin%
\definecolor{currentfill}{rgb}{0.121569,0.466667,0.705882}%
\pgfsetfillcolor{currentfill}%
\pgfsetlinewidth{0.481800pt}%
\definecolor{currentstroke}{rgb}{1.000000,1.000000,1.000000}%
\pgfsetstrokecolor{currentstroke}%
\pgfsetdash{}{0pt}%
\pgfpathmoveto{\pgfqpoint{4.143753in}{3.148858in}}%
\pgfpathcurveto{\pgfqpoint{4.154803in}{3.148858in}}{\pgfqpoint{4.165402in}{3.153248in}}{\pgfqpoint{4.173216in}{3.161062in}}%
\pgfpathcurveto{\pgfqpoint{4.181029in}{3.168876in}}{\pgfqpoint{4.185419in}{3.179475in}}{\pgfqpoint{4.185419in}{3.190525in}}%
\pgfpathcurveto{\pgfqpoint{4.185419in}{3.201575in}}{\pgfqpoint{4.181029in}{3.212174in}}{\pgfqpoint{4.173216in}{3.219988in}}%
\pgfpathcurveto{\pgfqpoint{4.165402in}{3.227801in}}{\pgfqpoint{4.154803in}{3.232191in}}{\pgfqpoint{4.143753in}{3.232191in}}%
\pgfpathcurveto{\pgfqpoint{4.132703in}{3.232191in}}{\pgfqpoint{4.122104in}{3.227801in}}{\pgfqpoint{4.114290in}{3.219988in}}%
\pgfpathcurveto{\pgfqpoint{4.106476in}{3.212174in}}{\pgfqpoint{4.102086in}{3.201575in}}{\pgfqpoint{4.102086in}{3.190525in}}%
\pgfpathcurveto{\pgfqpoint{4.102086in}{3.179475in}}{\pgfqpoint{4.106476in}{3.168876in}}{\pgfqpoint{4.114290in}{3.161062in}}%
\pgfpathcurveto{\pgfqpoint{4.122104in}{3.153248in}}{\pgfqpoint{4.132703in}{3.148858in}}{\pgfqpoint{4.143753in}{3.148858in}}%
\pgfpathlineto{\pgfqpoint{4.143753in}{3.148858in}}%
\pgfpathclose%
\pgfusepath{stroke,fill}%
\end{pgfscope}%
\begin{pgfscope}%
\pgfpathrectangle{\pgfqpoint{2.963410in}{2.920818in}}{\pgfqpoint{2.177280in}{2.201755in}}%
\pgfusepath{clip}%
\pgfsetbuttcap%
\pgfsetroundjoin%
\definecolor{currentfill}{rgb}{0.121569,0.466667,0.705882}%
\pgfsetfillcolor{currentfill}%
\pgfsetlinewidth{0.481800pt}%
\definecolor{currentstroke}{rgb}{1.000000,1.000000,1.000000}%
\pgfsetstrokecolor{currentstroke}%
\pgfsetdash{}{0pt}%
\pgfpathmoveto{\pgfqpoint{4.202828in}{3.081007in}}%
\pgfpathcurveto{\pgfqpoint{4.213878in}{3.081007in}}{\pgfqpoint{4.224477in}{3.085398in}}{\pgfqpoint{4.232291in}{3.093211in}}%
\pgfpathcurveto{\pgfqpoint{4.240104in}{3.101025in}}{\pgfqpoint{4.244495in}{3.111624in}}{\pgfqpoint{4.244495in}{3.122674in}}%
\pgfpathcurveto{\pgfqpoint{4.244495in}{3.133724in}}{\pgfqpoint{4.240104in}{3.144323in}}{\pgfqpoint{4.232291in}{3.152137in}}%
\pgfpathcurveto{\pgfqpoint{4.224477in}{3.159951in}}{\pgfqpoint{4.213878in}{3.164341in}}{\pgfqpoint{4.202828in}{3.164341in}}%
\pgfpathcurveto{\pgfqpoint{4.191778in}{3.164341in}}{\pgfqpoint{4.181179in}{3.159951in}}{\pgfqpoint{4.173365in}{3.152137in}}%
\pgfpathcurveto{\pgfqpoint{4.165552in}{3.144323in}}{\pgfqpoint{4.161161in}{3.133724in}}{\pgfqpoint{4.161161in}{3.122674in}}%
\pgfpathcurveto{\pgfqpoint{4.161161in}{3.111624in}}{\pgfqpoint{4.165552in}{3.101025in}}{\pgfqpoint{4.173365in}{3.093211in}}%
\pgfpathcurveto{\pgfqpoint{4.181179in}{3.085398in}}{\pgfqpoint{4.191778in}{3.081007in}}{\pgfqpoint{4.202828in}{3.081007in}}%
\pgfpathlineto{\pgfqpoint{4.202828in}{3.081007in}}%
\pgfpathclose%
\pgfusepath{stroke,fill}%
\end{pgfscope}%
\begin{pgfscope}%
\pgfpathrectangle{\pgfqpoint{2.963410in}{2.920818in}}{\pgfqpoint{2.177280in}{2.201755in}}%
\pgfusepath{clip}%
\pgfsetbuttcap%
\pgfsetroundjoin%
\definecolor{currentfill}{rgb}{0.121569,0.466667,0.705882}%
\pgfsetfillcolor{currentfill}%
\pgfsetlinewidth{0.481800pt}%
\definecolor{currentstroke}{rgb}{1.000000,1.000000,1.000000}%
\pgfsetstrokecolor{currentstroke}%
\pgfsetdash{}{0pt}%
\pgfpathmoveto{\pgfqpoint{3.493925in}{3.081007in}}%
\pgfpathcurveto{\pgfqpoint{3.504975in}{3.081007in}}{\pgfqpoint{3.515574in}{3.085398in}}{\pgfqpoint{3.523388in}{3.093211in}}%
\pgfpathcurveto{\pgfqpoint{3.531201in}{3.101025in}}{\pgfqpoint{3.535592in}{3.111624in}}{\pgfqpoint{3.535592in}{3.122674in}}%
\pgfpathcurveto{\pgfqpoint{3.535592in}{3.133724in}}{\pgfqpoint{3.531201in}{3.144323in}}{\pgfqpoint{3.523388in}{3.152137in}}%
\pgfpathcurveto{\pgfqpoint{3.515574in}{3.159951in}}{\pgfqpoint{3.504975in}{3.164341in}}{\pgfqpoint{3.493925in}{3.164341in}}%
\pgfpathcurveto{\pgfqpoint{3.482875in}{3.164341in}}{\pgfqpoint{3.472276in}{3.159951in}}{\pgfqpoint{3.464462in}{3.152137in}}%
\pgfpathcurveto{\pgfqpoint{3.456648in}{3.144323in}}{\pgfqpoint{3.452258in}{3.133724in}}{\pgfqpoint{3.452258in}{3.122674in}}%
\pgfpathcurveto{\pgfqpoint{3.452258in}{3.111624in}}{\pgfqpoint{3.456648in}{3.101025in}}{\pgfqpoint{3.464462in}{3.093211in}}%
\pgfpathcurveto{\pgfqpoint{3.472276in}{3.085398in}}{\pgfqpoint{3.482875in}{3.081007in}}{\pgfqpoint{3.493925in}{3.081007in}}%
\pgfpathlineto{\pgfqpoint{3.493925in}{3.081007in}}%
\pgfpathclose%
\pgfusepath{stroke,fill}%
\end{pgfscope}%
\begin{pgfscope}%
\pgfpathrectangle{\pgfqpoint{2.963410in}{2.920818in}}{\pgfqpoint{2.177280in}{2.201755in}}%
\pgfusepath{clip}%
\pgfsetbuttcap%
\pgfsetroundjoin%
\definecolor{currentfill}{rgb}{0.121569,0.466667,0.705882}%
\pgfsetfillcolor{currentfill}%
\pgfsetlinewidth{0.481800pt}%
\definecolor{currentstroke}{rgb}{1.000000,1.000000,1.000000}%
\pgfsetstrokecolor{currentstroke}%
\pgfsetdash{}{0pt}%
\pgfpathmoveto{\pgfqpoint{4.025602in}{3.081007in}}%
\pgfpathcurveto{\pgfqpoint{4.036652in}{3.081007in}}{\pgfqpoint{4.047251in}{3.085398in}}{\pgfqpoint{4.055065in}{3.093211in}}%
\pgfpathcurveto{\pgfqpoint{4.062879in}{3.101025in}}{\pgfqpoint{4.067269in}{3.111624in}}{\pgfqpoint{4.067269in}{3.122674in}}%
\pgfpathcurveto{\pgfqpoint{4.067269in}{3.133724in}}{\pgfqpoint{4.062879in}{3.144323in}}{\pgfqpoint{4.055065in}{3.152137in}}%
\pgfpathcurveto{\pgfqpoint{4.047251in}{3.159951in}}{\pgfqpoint{4.036652in}{3.164341in}}{\pgfqpoint{4.025602in}{3.164341in}}%
\pgfpathcurveto{\pgfqpoint{4.014552in}{3.164341in}}{\pgfqpoint{4.003953in}{3.159951in}}{\pgfqpoint{3.996139in}{3.152137in}}%
\pgfpathcurveto{\pgfqpoint{3.988326in}{3.144323in}}{\pgfqpoint{3.983936in}{3.133724in}}{\pgfqpoint{3.983936in}{3.122674in}}%
\pgfpathcurveto{\pgfqpoint{3.983936in}{3.111624in}}{\pgfqpoint{3.988326in}{3.101025in}}{\pgfqpoint{3.996139in}{3.093211in}}%
\pgfpathcurveto{\pgfqpoint{4.003953in}{3.085398in}}{\pgfqpoint{4.014552in}{3.081007in}}{\pgfqpoint{4.025602in}{3.081007in}}%
\pgfpathlineto{\pgfqpoint{4.025602in}{3.081007in}}%
\pgfpathclose%
\pgfusepath{stroke,fill}%
\end{pgfscope}%
\begin{pgfscope}%
\pgfpathrectangle{\pgfqpoint{2.963410in}{2.920818in}}{\pgfqpoint{2.177280in}{2.201755in}}%
\pgfusepath{clip}%
\pgfsetbuttcap%
\pgfsetroundjoin%
\definecolor{currentfill}{rgb}{0.121569,0.466667,0.705882}%
\pgfsetfillcolor{currentfill}%
\pgfsetlinewidth{0.481800pt}%
\definecolor{currentstroke}{rgb}{1.000000,1.000000,1.000000}%
\pgfsetstrokecolor{currentstroke}%
\pgfsetdash{}{0pt}%
\pgfpathmoveto{\pgfqpoint{4.202828in}{3.182783in}}%
\pgfpathcurveto{\pgfqpoint{4.213878in}{3.182783in}}{\pgfqpoint{4.224477in}{3.187174in}}{\pgfqpoint{4.232291in}{3.194987in}}%
\pgfpathcurveto{\pgfqpoint{4.240104in}{3.202801in}}{\pgfqpoint{4.244495in}{3.213400in}}{\pgfqpoint{4.244495in}{3.224450in}}%
\pgfpathcurveto{\pgfqpoint{4.244495in}{3.235500in}}{\pgfqpoint{4.240104in}{3.246099in}}{\pgfqpoint{4.232291in}{3.253913in}}%
\pgfpathcurveto{\pgfqpoint{4.224477in}{3.261727in}}{\pgfqpoint{4.213878in}{3.266117in}}{\pgfqpoint{4.202828in}{3.266117in}}%
\pgfpathcurveto{\pgfqpoint{4.191778in}{3.266117in}}{\pgfqpoint{4.181179in}{3.261727in}}{\pgfqpoint{4.173365in}{3.253913in}}%
\pgfpathcurveto{\pgfqpoint{4.165552in}{3.246099in}}{\pgfqpoint{4.161161in}{3.235500in}}{\pgfqpoint{4.161161in}{3.224450in}}%
\pgfpathcurveto{\pgfqpoint{4.161161in}{3.213400in}}{\pgfqpoint{4.165552in}{3.202801in}}{\pgfqpoint{4.173365in}{3.194987in}}%
\pgfpathcurveto{\pgfqpoint{4.181179in}{3.187174in}}{\pgfqpoint{4.191778in}{3.182783in}}{\pgfqpoint{4.202828in}{3.182783in}}%
\pgfpathlineto{\pgfqpoint{4.202828in}{3.182783in}}%
\pgfpathclose%
\pgfusepath{stroke,fill}%
\end{pgfscope}%
\begin{pgfscope}%
\pgfpathrectangle{\pgfqpoint{2.963410in}{2.920818in}}{\pgfqpoint{2.177280in}{2.201755in}}%
\pgfusepath{clip}%
\pgfsetbuttcap%
\pgfsetroundjoin%
\definecolor{currentfill}{rgb}{0.121569,0.466667,0.705882}%
\pgfsetfillcolor{currentfill}%
\pgfsetlinewidth{0.481800pt}%
\definecolor{currentstroke}{rgb}{1.000000,1.000000,1.000000}%
\pgfsetstrokecolor{currentstroke}%
\pgfsetdash{}{0pt}%
\pgfpathmoveto{\pgfqpoint{4.380054in}{3.284560in}}%
\pgfpathcurveto{\pgfqpoint{4.391104in}{3.284560in}}{\pgfqpoint{4.401703in}{3.288950in}}{\pgfqpoint{4.409517in}{3.296763in}}%
\pgfpathcurveto{\pgfqpoint{4.417330in}{3.304577in}}{\pgfqpoint{4.421721in}{3.315176in}}{\pgfqpoint{4.421721in}{3.326226in}}%
\pgfpathcurveto{\pgfqpoint{4.421721in}{3.337276in}}{\pgfqpoint{4.417330in}{3.347875in}}{\pgfqpoint{4.409517in}{3.355689in}}%
\pgfpathcurveto{\pgfqpoint{4.401703in}{3.363503in}}{\pgfqpoint{4.391104in}{3.367893in}}{\pgfqpoint{4.380054in}{3.367893in}}%
\pgfpathcurveto{\pgfqpoint{4.369004in}{3.367893in}}{\pgfqpoint{4.358405in}{3.363503in}}{\pgfqpoint{4.350591in}{3.355689in}}%
\pgfpathcurveto{\pgfqpoint{4.342777in}{3.347875in}}{\pgfqpoint{4.338387in}{3.337276in}}{\pgfqpoint{4.338387in}{3.326226in}}%
\pgfpathcurveto{\pgfqpoint{4.338387in}{3.315176in}}{\pgfqpoint{4.342777in}{3.304577in}}{\pgfqpoint{4.350591in}{3.296763in}}%
\pgfpathcurveto{\pgfqpoint{4.358405in}{3.288950in}}{\pgfqpoint{4.369004in}{3.284560in}}{\pgfqpoint{4.380054in}{3.284560in}}%
\pgfpathlineto{\pgfqpoint{4.380054in}{3.284560in}}%
\pgfpathclose%
\pgfusepath{stroke,fill}%
\end{pgfscope}%
\begin{pgfscope}%
\pgfpathrectangle{\pgfqpoint{2.963410in}{2.920818in}}{\pgfqpoint{2.177280in}{2.201755in}}%
\pgfusepath{clip}%
\pgfsetbuttcap%
\pgfsetroundjoin%
\definecolor{currentfill}{rgb}{0.121569,0.466667,0.705882}%
\pgfsetfillcolor{currentfill}%
\pgfsetlinewidth{0.481800pt}%
\definecolor{currentstroke}{rgb}{1.000000,1.000000,1.000000}%
\pgfsetstrokecolor{currentstroke}%
\pgfsetdash{}{0pt}%
\pgfpathmoveto{\pgfqpoint{3.907452in}{3.114933in}}%
\pgfpathcurveto{\pgfqpoint{3.918502in}{3.114933in}}{\pgfqpoint{3.929101in}{3.119323in}}{\pgfqpoint{3.936915in}{3.127137in}}%
\pgfpathcurveto{\pgfqpoint{3.944728in}{3.134950in}}{\pgfqpoint{3.949118in}{3.145549in}}{\pgfqpoint{3.949118in}{3.156599in}}%
\pgfpathcurveto{\pgfqpoint{3.949118in}{3.167650in}}{\pgfqpoint{3.944728in}{3.178249in}}{\pgfqpoint{3.936915in}{3.186062in}}%
\pgfpathcurveto{\pgfqpoint{3.929101in}{3.193876in}}{\pgfqpoint{3.918502in}{3.198266in}}{\pgfqpoint{3.907452in}{3.198266in}}%
\pgfpathcurveto{\pgfqpoint{3.896402in}{3.198266in}}{\pgfqpoint{3.885803in}{3.193876in}}{\pgfqpoint{3.877989in}{3.186062in}}%
\pgfpathcurveto{\pgfqpoint{3.870175in}{3.178249in}}{\pgfqpoint{3.865785in}{3.167650in}}{\pgfqpoint{3.865785in}{3.156599in}}%
\pgfpathcurveto{\pgfqpoint{3.865785in}{3.145549in}}{\pgfqpoint{3.870175in}{3.134950in}}{\pgfqpoint{3.877989in}{3.127137in}}%
\pgfpathcurveto{\pgfqpoint{3.885803in}{3.119323in}}{\pgfqpoint{3.896402in}{3.114933in}}{\pgfqpoint{3.907452in}{3.114933in}}%
\pgfpathlineto{\pgfqpoint{3.907452in}{3.114933in}}%
\pgfpathclose%
\pgfusepath{stroke,fill}%
\end{pgfscope}%
\begin{pgfscope}%
\pgfpathrectangle{\pgfqpoint{2.963410in}{2.920818in}}{\pgfqpoint{2.177280in}{2.201755in}}%
\pgfusepath{clip}%
\pgfsetbuttcap%
\pgfsetroundjoin%
\definecolor{currentfill}{rgb}{0.121569,0.466667,0.705882}%
\pgfsetfillcolor{currentfill}%
\pgfsetlinewidth{0.481800pt}%
\definecolor{currentstroke}{rgb}{1.000000,1.000000,1.000000}%
\pgfsetstrokecolor{currentstroke}%
\pgfsetdash{}{0pt}%
\pgfpathmoveto{\pgfqpoint{4.380054in}{3.182783in}}%
\pgfpathcurveto{\pgfqpoint{4.391104in}{3.182783in}}{\pgfqpoint{4.401703in}{3.187174in}}{\pgfqpoint{4.409517in}{3.194987in}}%
\pgfpathcurveto{\pgfqpoint{4.417330in}{3.202801in}}{\pgfqpoint{4.421721in}{3.213400in}}{\pgfqpoint{4.421721in}{3.224450in}}%
\pgfpathcurveto{\pgfqpoint{4.421721in}{3.235500in}}{\pgfqpoint{4.417330in}{3.246099in}}{\pgfqpoint{4.409517in}{3.253913in}}%
\pgfpathcurveto{\pgfqpoint{4.401703in}{3.261727in}}{\pgfqpoint{4.391104in}{3.266117in}}{\pgfqpoint{4.380054in}{3.266117in}}%
\pgfpathcurveto{\pgfqpoint{4.369004in}{3.266117in}}{\pgfqpoint{4.358405in}{3.261727in}}{\pgfqpoint{4.350591in}{3.253913in}}%
\pgfpathcurveto{\pgfqpoint{4.342777in}{3.246099in}}{\pgfqpoint{4.338387in}{3.235500in}}{\pgfqpoint{4.338387in}{3.224450in}}%
\pgfpathcurveto{\pgfqpoint{4.338387in}{3.213400in}}{\pgfqpoint{4.342777in}{3.202801in}}{\pgfqpoint{4.350591in}{3.194987in}}%
\pgfpathcurveto{\pgfqpoint{4.358405in}{3.187174in}}{\pgfqpoint{4.369004in}{3.182783in}}{\pgfqpoint{4.380054in}{3.182783in}}%
\pgfpathlineto{\pgfqpoint{4.380054in}{3.182783in}}%
\pgfpathclose%
\pgfusepath{stroke,fill}%
\end{pgfscope}%
\begin{pgfscope}%
\pgfpathrectangle{\pgfqpoint{2.963410in}{2.920818in}}{\pgfqpoint{2.177280in}{2.201755in}}%
\pgfusepath{clip}%
\pgfsetbuttcap%
\pgfsetroundjoin%
\definecolor{currentfill}{rgb}{0.121569,0.466667,0.705882}%
\pgfsetfillcolor{currentfill}%
\pgfsetlinewidth{0.481800pt}%
\definecolor{currentstroke}{rgb}{1.000000,1.000000,1.000000}%
\pgfsetstrokecolor{currentstroke}%
\pgfsetdash{}{0pt}%
\pgfpathmoveto{\pgfqpoint{4.025602in}{3.114933in}}%
\pgfpathcurveto{\pgfqpoint{4.036652in}{3.114933in}}{\pgfqpoint{4.047251in}{3.119323in}}{\pgfqpoint{4.055065in}{3.127137in}}%
\pgfpathcurveto{\pgfqpoint{4.062879in}{3.134950in}}{\pgfqpoint{4.067269in}{3.145549in}}{\pgfqpoint{4.067269in}{3.156599in}}%
\pgfpathcurveto{\pgfqpoint{4.067269in}{3.167650in}}{\pgfqpoint{4.062879in}{3.178249in}}{\pgfqpoint{4.055065in}{3.186062in}}%
\pgfpathcurveto{\pgfqpoint{4.047251in}{3.193876in}}{\pgfqpoint{4.036652in}{3.198266in}}{\pgfqpoint{4.025602in}{3.198266in}}%
\pgfpathcurveto{\pgfqpoint{4.014552in}{3.198266in}}{\pgfqpoint{4.003953in}{3.193876in}}{\pgfqpoint{3.996139in}{3.186062in}}%
\pgfpathcurveto{\pgfqpoint{3.988326in}{3.178249in}}{\pgfqpoint{3.983936in}{3.167650in}}{\pgfqpoint{3.983936in}{3.156599in}}%
\pgfpathcurveto{\pgfqpoint{3.983936in}{3.145549in}}{\pgfqpoint{3.988326in}{3.134950in}}{\pgfqpoint{3.996139in}{3.127137in}}%
\pgfpathcurveto{\pgfqpoint{4.003953in}{3.119323in}}{\pgfqpoint{4.014552in}{3.114933in}}{\pgfqpoint{4.025602in}{3.114933in}}%
\pgfpathlineto{\pgfqpoint{4.025602in}{3.114933in}}%
\pgfpathclose%
\pgfusepath{stroke,fill}%
\end{pgfscope}%
\begin{pgfscope}%
\pgfpathrectangle{\pgfqpoint{2.963410in}{2.920818in}}{\pgfqpoint{2.177280in}{2.201755in}}%
\pgfusepath{clip}%
\pgfsetbuttcap%
\pgfsetroundjoin%
\definecolor{currentfill}{rgb}{0.121569,0.466667,0.705882}%
\pgfsetfillcolor{currentfill}%
\pgfsetlinewidth{0.481800pt}%
\definecolor{currentstroke}{rgb}{1.000000,1.000000,1.000000}%
\pgfsetstrokecolor{currentstroke}%
\pgfsetdash{}{0pt}%
\pgfpathmoveto{\pgfqpoint{4.320979in}{3.148858in}}%
\pgfpathcurveto{\pgfqpoint{4.332029in}{3.148858in}}{\pgfqpoint{4.342628in}{3.153248in}}{\pgfqpoint{4.350441in}{3.161062in}}%
\pgfpathcurveto{\pgfqpoint{4.358255in}{3.168876in}}{\pgfqpoint{4.362645in}{3.179475in}}{\pgfqpoint{4.362645in}{3.190525in}}%
\pgfpathcurveto{\pgfqpoint{4.362645in}{3.201575in}}{\pgfqpoint{4.358255in}{3.212174in}}{\pgfqpoint{4.350441in}{3.219988in}}%
\pgfpathcurveto{\pgfqpoint{4.342628in}{3.227801in}}{\pgfqpoint{4.332029in}{3.232191in}}{\pgfqpoint{4.320979in}{3.232191in}}%
\pgfpathcurveto{\pgfqpoint{4.309928in}{3.232191in}}{\pgfqpoint{4.299329in}{3.227801in}}{\pgfqpoint{4.291516in}{3.219988in}}%
\pgfpathcurveto{\pgfqpoint{4.283702in}{3.212174in}}{\pgfqpoint{4.279312in}{3.201575in}}{\pgfqpoint{4.279312in}{3.190525in}}%
\pgfpathcurveto{\pgfqpoint{4.279312in}{3.179475in}}{\pgfqpoint{4.283702in}{3.168876in}}{\pgfqpoint{4.291516in}{3.161062in}}%
\pgfpathcurveto{\pgfqpoint{4.299329in}{3.153248in}}{\pgfqpoint{4.309928in}{3.148858in}}{\pgfqpoint{4.320979in}{3.148858in}}%
\pgfpathlineto{\pgfqpoint{4.320979in}{3.148858in}}%
\pgfpathclose%
\pgfusepath{stroke,fill}%
\end{pgfscope}%
\begin{pgfscope}%
\pgfpathrectangle{\pgfqpoint{2.963410in}{2.920818in}}{\pgfqpoint{2.177280in}{2.201755in}}%
\pgfusepath{clip}%
\pgfsetbuttcap%
\pgfsetroundjoin%
\definecolor{currentfill}{rgb}{0.121569,0.466667,0.705882}%
\pgfsetfillcolor{currentfill}%
\pgfsetlinewidth{0.481800pt}%
\definecolor{currentstroke}{rgb}{1.000000,1.000000,1.000000}%
\pgfsetstrokecolor{currentstroke}%
\pgfsetdash{}{0pt}%
\pgfpathmoveto{\pgfqpoint{4.084678in}{3.114933in}}%
\pgfpathcurveto{\pgfqpoint{4.095728in}{3.114933in}}{\pgfqpoint{4.106327in}{3.119323in}}{\pgfqpoint{4.114140in}{3.127137in}}%
\pgfpathcurveto{\pgfqpoint{4.121954in}{3.134950in}}{\pgfqpoint{4.126344in}{3.145549in}}{\pgfqpoint{4.126344in}{3.156599in}}%
\pgfpathcurveto{\pgfqpoint{4.126344in}{3.167650in}}{\pgfqpoint{4.121954in}{3.178249in}}{\pgfqpoint{4.114140in}{3.186062in}}%
\pgfpathcurveto{\pgfqpoint{4.106327in}{3.193876in}}{\pgfqpoint{4.095728in}{3.198266in}}{\pgfqpoint{4.084678in}{3.198266in}}%
\pgfpathcurveto{\pgfqpoint{4.073627in}{3.198266in}}{\pgfqpoint{4.063028in}{3.193876in}}{\pgfqpoint{4.055215in}{3.186062in}}%
\pgfpathcurveto{\pgfqpoint{4.047401in}{3.178249in}}{\pgfqpoint{4.043011in}{3.167650in}}{\pgfqpoint{4.043011in}{3.156599in}}%
\pgfpathcurveto{\pgfqpoint{4.043011in}{3.145549in}}{\pgfqpoint{4.047401in}{3.134950in}}{\pgfqpoint{4.055215in}{3.127137in}}%
\pgfpathcurveto{\pgfqpoint{4.063028in}{3.119323in}}{\pgfqpoint{4.073627in}{3.114933in}}{\pgfqpoint{4.084678in}{3.114933in}}%
\pgfpathlineto{\pgfqpoint{4.084678in}{3.114933in}}%
\pgfpathclose%
\pgfusepath{stroke,fill}%
\end{pgfscope}%
\begin{pgfscope}%
\pgfpathrectangle{\pgfqpoint{2.963410in}{2.920818in}}{\pgfqpoint{2.177280in}{2.201755in}}%
\pgfusepath{clip}%
\pgfsetbuttcap%
\pgfsetroundjoin%
\definecolor{currentfill}{rgb}{1.000000,0.498039,0.054902}%
\pgfsetfillcolor{currentfill}%
\pgfsetlinewidth{0.481800pt}%
\definecolor{currentstroke}{rgb}{1.000000,1.000000,1.000000}%
\pgfsetstrokecolor{currentstroke}%
\pgfsetdash{}{0pt}%
\pgfpathmoveto{\pgfqpoint{4.025602in}{4.234469in}}%
\pgfpathcurveto{\pgfqpoint{4.036652in}{4.234469in}}{\pgfqpoint{4.047251in}{4.238860in}}{\pgfqpoint{4.055065in}{4.246673in}}%
\pgfpathcurveto{\pgfqpoint{4.062879in}{4.254487in}}{\pgfqpoint{4.067269in}{4.265086in}}{\pgfqpoint{4.067269in}{4.276136in}}%
\pgfpathcurveto{\pgfqpoint{4.067269in}{4.287186in}}{\pgfqpoint{4.062879in}{4.297785in}}{\pgfqpoint{4.055065in}{4.305599in}}%
\pgfpathcurveto{\pgfqpoint{4.047251in}{4.313412in}}{\pgfqpoint{4.036652in}{4.317803in}}{\pgfqpoint{4.025602in}{4.317803in}}%
\pgfpathcurveto{\pgfqpoint{4.014552in}{4.317803in}}{\pgfqpoint{4.003953in}{4.313412in}}{\pgfqpoint{3.996139in}{4.305599in}}%
\pgfpathcurveto{\pgfqpoint{3.988326in}{4.297785in}}{\pgfqpoint{3.983936in}{4.287186in}}{\pgfqpoint{3.983936in}{4.276136in}}%
\pgfpathcurveto{\pgfqpoint{3.983936in}{4.265086in}}{\pgfqpoint{3.988326in}{4.254487in}}{\pgfqpoint{3.996139in}{4.246673in}}%
\pgfpathcurveto{\pgfqpoint{4.003953in}{4.238860in}}{\pgfqpoint{4.014552in}{4.234469in}}{\pgfqpoint{4.025602in}{4.234469in}}%
\pgfpathlineto{\pgfqpoint{4.025602in}{4.234469in}}%
\pgfpathclose%
\pgfusepath{stroke,fill}%
\end{pgfscope}%
\begin{pgfscope}%
\pgfpathrectangle{\pgfqpoint{2.963410in}{2.920818in}}{\pgfqpoint{2.177280in}{2.201755in}}%
\pgfusepath{clip}%
\pgfsetbuttcap%
\pgfsetroundjoin%
\definecolor{currentfill}{rgb}{1.000000,0.498039,0.054902}%
\pgfsetfillcolor{currentfill}%
\pgfsetlinewidth{0.481800pt}%
\definecolor{currentstroke}{rgb}{1.000000,1.000000,1.000000}%
\pgfsetstrokecolor{currentstroke}%
\pgfsetdash{}{0pt}%
\pgfpathmoveto{\pgfqpoint{4.025602in}{4.166619in}}%
\pgfpathcurveto{\pgfqpoint{4.036652in}{4.166619in}}{\pgfqpoint{4.047251in}{4.171009in}}{\pgfqpoint{4.055065in}{4.178822in}}%
\pgfpathcurveto{\pgfqpoint{4.062879in}{4.186636in}}{\pgfqpoint{4.067269in}{4.197235in}}{\pgfqpoint{4.067269in}{4.208285in}}%
\pgfpathcurveto{\pgfqpoint{4.067269in}{4.219335in}}{\pgfqpoint{4.062879in}{4.229934in}}{\pgfqpoint{4.055065in}{4.237748in}}%
\pgfpathcurveto{\pgfqpoint{4.047251in}{4.245562in}}{\pgfqpoint{4.036652in}{4.249952in}}{\pgfqpoint{4.025602in}{4.249952in}}%
\pgfpathcurveto{\pgfqpoint{4.014552in}{4.249952in}}{\pgfqpoint{4.003953in}{4.245562in}}{\pgfqpoint{3.996139in}{4.237748in}}%
\pgfpathcurveto{\pgfqpoint{3.988326in}{4.229934in}}{\pgfqpoint{3.983936in}{4.219335in}}{\pgfqpoint{3.983936in}{4.208285in}}%
\pgfpathcurveto{\pgfqpoint{3.983936in}{4.197235in}}{\pgfqpoint{3.988326in}{4.186636in}}{\pgfqpoint{3.996139in}{4.178822in}}%
\pgfpathcurveto{\pgfqpoint{4.003953in}{4.171009in}}{\pgfqpoint{4.014552in}{4.166619in}}{\pgfqpoint{4.025602in}{4.166619in}}%
\pgfpathlineto{\pgfqpoint{4.025602in}{4.166619in}}%
\pgfpathclose%
\pgfusepath{stroke,fill}%
\end{pgfscope}%
\begin{pgfscope}%
\pgfpathrectangle{\pgfqpoint{2.963410in}{2.920818in}}{\pgfqpoint{2.177280in}{2.201755in}}%
\pgfusepath{clip}%
\pgfsetbuttcap%
\pgfsetroundjoin%
\definecolor{currentfill}{rgb}{1.000000,0.498039,0.054902}%
\pgfsetfillcolor{currentfill}%
\pgfsetlinewidth{0.481800pt}%
\definecolor{currentstroke}{rgb}{1.000000,1.000000,1.000000}%
\pgfsetstrokecolor{currentstroke}%
\pgfsetdash{}{0pt}%
\pgfpathmoveto{\pgfqpoint{3.966527in}{4.302320in}}%
\pgfpathcurveto{\pgfqpoint{3.977577in}{4.302320in}}{\pgfqpoint{3.988176in}{4.306710in}}{\pgfqpoint{3.995990in}{4.314524in}}%
\pgfpathcurveto{\pgfqpoint{4.003803in}{4.322337in}}{\pgfqpoint{4.008194in}{4.332937in}}{\pgfqpoint{4.008194in}{4.343987in}}%
\pgfpathcurveto{\pgfqpoint{4.008194in}{4.355037in}}{\pgfqpoint{4.003803in}{4.365636in}}{\pgfqpoint{3.995990in}{4.373449in}}%
\pgfpathcurveto{\pgfqpoint{3.988176in}{4.381263in}}{\pgfqpoint{3.977577in}{4.385653in}}{\pgfqpoint{3.966527in}{4.385653in}}%
\pgfpathcurveto{\pgfqpoint{3.955477in}{4.385653in}}{\pgfqpoint{3.944878in}{4.381263in}}{\pgfqpoint{3.937064in}{4.373449in}}%
\pgfpathcurveto{\pgfqpoint{3.929251in}{4.365636in}}{\pgfqpoint{3.924860in}{4.355037in}}{\pgfqpoint{3.924860in}{4.343987in}}%
\pgfpathcurveto{\pgfqpoint{3.924860in}{4.332937in}}{\pgfqpoint{3.929251in}{4.322337in}}{\pgfqpoint{3.937064in}{4.314524in}}%
\pgfpathcurveto{\pgfqpoint{3.944878in}{4.306710in}}{\pgfqpoint{3.955477in}{4.302320in}}{\pgfqpoint{3.966527in}{4.302320in}}%
\pgfpathlineto{\pgfqpoint{3.966527in}{4.302320in}}%
\pgfpathclose%
\pgfusepath{stroke,fill}%
\end{pgfscope}%
\begin{pgfscope}%
\pgfpathrectangle{\pgfqpoint{2.963410in}{2.920818in}}{\pgfqpoint{2.177280in}{2.201755in}}%
\pgfusepath{clip}%
\pgfsetbuttcap%
\pgfsetroundjoin%
\definecolor{currentfill}{rgb}{1.000000,0.498039,0.054902}%
\pgfsetfillcolor{currentfill}%
\pgfsetlinewidth{0.481800pt}%
\definecolor{currentstroke}{rgb}{1.000000,1.000000,1.000000}%
\pgfsetstrokecolor{currentstroke}%
\pgfsetdash{}{0pt}%
\pgfpathmoveto{\pgfqpoint{3.493925in}{3.996992in}}%
\pgfpathcurveto{\pgfqpoint{3.504975in}{3.996992in}}{\pgfqpoint{3.515574in}{4.001382in}}{\pgfqpoint{3.523388in}{4.009196in}}%
\pgfpathcurveto{\pgfqpoint{3.531201in}{4.017009in}}{\pgfqpoint{3.535592in}{4.027608in}}{\pgfqpoint{3.535592in}{4.038659in}}%
\pgfpathcurveto{\pgfqpoint{3.535592in}{4.049709in}}{\pgfqpoint{3.531201in}{4.060308in}}{\pgfqpoint{3.523388in}{4.068121in}}%
\pgfpathcurveto{\pgfqpoint{3.515574in}{4.075935in}}{\pgfqpoint{3.504975in}{4.080325in}}{\pgfqpoint{3.493925in}{4.080325in}}%
\pgfpathcurveto{\pgfqpoint{3.482875in}{4.080325in}}{\pgfqpoint{3.472276in}{4.075935in}}{\pgfqpoint{3.464462in}{4.068121in}}%
\pgfpathcurveto{\pgfqpoint{3.456648in}{4.060308in}}{\pgfqpoint{3.452258in}{4.049709in}}{\pgfqpoint{3.452258in}{4.038659in}}%
\pgfpathcurveto{\pgfqpoint{3.452258in}{4.027608in}}{\pgfqpoint{3.456648in}{4.017009in}}{\pgfqpoint{3.464462in}{4.009196in}}%
\pgfpathcurveto{\pgfqpoint{3.472276in}{4.001382in}}{\pgfqpoint{3.482875in}{3.996992in}}{\pgfqpoint{3.493925in}{3.996992in}}%
\pgfpathlineto{\pgfqpoint{3.493925in}{3.996992in}}%
\pgfpathclose%
\pgfusepath{stroke,fill}%
\end{pgfscope}%
\begin{pgfscope}%
\pgfpathrectangle{\pgfqpoint{2.963410in}{2.920818in}}{\pgfqpoint{2.177280in}{2.201755in}}%
\pgfusepath{clip}%
\pgfsetbuttcap%
\pgfsetroundjoin%
\definecolor{currentfill}{rgb}{1.000000,0.498039,0.054902}%
\pgfsetfillcolor{currentfill}%
\pgfsetlinewidth{0.481800pt}%
\definecolor{currentstroke}{rgb}{1.000000,1.000000,1.000000}%
\pgfsetstrokecolor{currentstroke}%
\pgfsetdash{}{0pt}%
\pgfpathmoveto{\pgfqpoint{3.789301in}{4.200544in}}%
\pgfpathcurveto{\pgfqpoint{3.800351in}{4.200544in}}{\pgfqpoint{3.810950in}{4.204934in}}{\pgfqpoint{3.818764in}{4.212748in}}%
\pgfpathcurveto{\pgfqpoint{3.826578in}{4.220561in}}{\pgfqpoint{3.830968in}{4.231160in}}{\pgfqpoint{3.830968in}{4.242211in}}%
\pgfpathcurveto{\pgfqpoint{3.830968in}{4.253261in}}{\pgfqpoint{3.826578in}{4.263860in}}{\pgfqpoint{3.818764in}{4.271673in}}%
\pgfpathcurveto{\pgfqpoint{3.810950in}{4.279487in}}{\pgfqpoint{3.800351in}{4.283877in}}{\pgfqpoint{3.789301in}{4.283877in}}%
\pgfpathcurveto{\pgfqpoint{3.778251in}{4.283877in}}{\pgfqpoint{3.767652in}{4.279487in}}{\pgfqpoint{3.759838in}{4.271673in}}%
\pgfpathcurveto{\pgfqpoint{3.752025in}{4.263860in}}{\pgfqpoint{3.747635in}{4.253261in}}{\pgfqpoint{3.747635in}{4.242211in}}%
\pgfpathcurveto{\pgfqpoint{3.747635in}{4.231160in}}{\pgfqpoint{3.752025in}{4.220561in}}{\pgfqpoint{3.759838in}{4.212748in}}%
\pgfpathcurveto{\pgfqpoint{3.767652in}{4.204934in}}{\pgfqpoint{3.778251in}{4.200544in}}{\pgfqpoint{3.789301in}{4.200544in}}%
\pgfpathlineto{\pgfqpoint{3.789301in}{4.200544in}}%
\pgfpathclose%
\pgfusepath{stroke,fill}%
\end{pgfscope}%
\begin{pgfscope}%
\pgfpathrectangle{\pgfqpoint{2.963410in}{2.920818in}}{\pgfqpoint{2.177280in}{2.201755in}}%
\pgfusepath{clip}%
\pgfsetbuttcap%
\pgfsetroundjoin%
\definecolor{currentfill}{rgb}{1.000000,0.498039,0.054902}%
\pgfsetfillcolor{currentfill}%
\pgfsetlinewidth{0.481800pt}%
\definecolor{currentstroke}{rgb}{1.000000,1.000000,1.000000}%
\pgfsetstrokecolor{currentstroke}%
\pgfsetdash{}{0pt}%
\pgfpathmoveto{\pgfqpoint{3.789301in}{4.166619in}}%
\pgfpathcurveto{\pgfqpoint{3.800351in}{4.166619in}}{\pgfqpoint{3.810950in}{4.171009in}}{\pgfqpoint{3.818764in}{4.178822in}}%
\pgfpathcurveto{\pgfqpoint{3.826578in}{4.186636in}}{\pgfqpoint{3.830968in}{4.197235in}}{\pgfqpoint{3.830968in}{4.208285in}}%
\pgfpathcurveto{\pgfqpoint{3.830968in}{4.219335in}}{\pgfqpoint{3.826578in}{4.229934in}}{\pgfqpoint{3.818764in}{4.237748in}}%
\pgfpathcurveto{\pgfqpoint{3.810950in}{4.245562in}}{\pgfqpoint{3.800351in}{4.249952in}}{\pgfqpoint{3.789301in}{4.249952in}}%
\pgfpathcurveto{\pgfqpoint{3.778251in}{4.249952in}}{\pgfqpoint{3.767652in}{4.245562in}}{\pgfqpoint{3.759838in}{4.237748in}}%
\pgfpathcurveto{\pgfqpoint{3.752025in}{4.229934in}}{\pgfqpoint{3.747635in}{4.219335in}}{\pgfqpoint{3.747635in}{4.208285in}}%
\pgfpathcurveto{\pgfqpoint{3.747635in}{4.197235in}}{\pgfqpoint{3.752025in}{4.186636in}}{\pgfqpoint{3.759838in}{4.178822in}}%
\pgfpathcurveto{\pgfqpoint{3.767652in}{4.171009in}}{\pgfqpoint{3.778251in}{4.166619in}}{\pgfqpoint{3.789301in}{4.166619in}}%
\pgfpathlineto{\pgfqpoint{3.789301in}{4.166619in}}%
\pgfpathclose%
\pgfusepath{stroke,fill}%
\end{pgfscope}%
\begin{pgfscope}%
\pgfpathrectangle{\pgfqpoint{2.963410in}{2.920818in}}{\pgfqpoint{2.177280in}{2.201755in}}%
\pgfusepath{clip}%
\pgfsetbuttcap%
\pgfsetroundjoin%
\definecolor{currentfill}{rgb}{1.000000,0.498039,0.054902}%
\pgfsetfillcolor{currentfill}%
\pgfsetlinewidth{0.481800pt}%
\definecolor{currentstroke}{rgb}{1.000000,1.000000,1.000000}%
\pgfsetstrokecolor{currentstroke}%
\pgfsetdash{}{0pt}%
\pgfpathmoveto{\pgfqpoint{4.084678in}{4.234469in}}%
\pgfpathcurveto{\pgfqpoint{4.095728in}{4.234469in}}{\pgfqpoint{4.106327in}{4.238860in}}{\pgfqpoint{4.114140in}{4.246673in}}%
\pgfpathcurveto{\pgfqpoint{4.121954in}{4.254487in}}{\pgfqpoint{4.126344in}{4.265086in}}{\pgfqpoint{4.126344in}{4.276136in}}%
\pgfpathcurveto{\pgfqpoint{4.126344in}{4.287186in}}{\pgfqpoint{4.121954in}{4.297785in}}{\pgfqpoint{4.114140in}{4.305599in}}%
\pgfpathcurveto{\pgfqpoint{4.106327in}{4.313412in}}{\pgfqpoint{4.095728in}{4.317803in}}{\pgfqpoint{4.084678in}{4.317803in}}%
\pgfpathcurveto{\pgfqpoint{4.073627in}{4.317803in}}{\pgfqpoint{4.063028in}{4.313412in}}{\pgfqpoint{4.055215in}{4.305599in}}%
\pgfpathcurveto{\pgfqpoint{4.047401in}{4.297785in}}{\pgfqpoint{4.043011in}{4.287186in}}{\pgfqpoint{4.043011in}{4.276136in}}%
\pgfpathcurveto{\pgfqpoint{4.043011in}{4.265086in}}{\pgfqpoint{4.047401in}{4.254487in}}{\pgfqpoint{4.055215in}{4.246673in}}%
\pgfpathcurveto{\pgfqpoint{4.063028in}{4.238860in}}{\pgfqpoint{4.073627in}{4.234469in}}{\pgfqpoint{4.084678in}{4.234469in}}%
\pgfpathlineto{\pgfqpoint{4.084678in}{4.234469in}}%
\pgfpathclose%
\pgfusepath{stroke,fill}%
\end{pgfscope}%
\begin{pgfscope}%
\pgfpathrectangle{\pgfqpoint{2.963410in}{2.920818in}}{\pgfqpoint{2.177280in}{2.201755in}}%
\pgfusepath{clip}%
\pgfsetbuttcap%
\pgfsetroundjoin%
\definecolor{currentfill}{rgb}{1.000000,0.498039,0.054902}%
\pgfsetfillcolor{currentfill}%
\pgfsetlinewidth{0.481800pt}%
\definecolor{currentstroke}{rgb}{1.000000,1.000000,1.000000}%
\pgfsetstrokecolor{currentstroke}%
\pgfsetdash{}{0pt}%
\pgfpathmoveto{\pgfqpoint{3.553000in}{3.759514in}}%
\pgfpathcurveto{\pgfqpoint{3.564050in}{3.759514in}}{\pgfqpoint{3.574649in}{3.763905in}}{\pgfqpoint{3.582463in}{3.771718in}}%
\pgfpathcurveto{\pgfqpoint{3.590277in}{3.779532in}}{\pgfqpoint{3.594667in}{3.790131in}}{\pgfqpoint{3.594667in}{3.801181in}}%
\pgfpathcurveto{\pgfqpoint{3.594667in}{3.812231in}}{\pgfqpoint{3.590277in}{3.822830in}}{\pgfqpoint{3.582463in}{3.830644in}}%
\pgfpathcurveto{\pgfqpoint{3.574649in}{3.838457in}}{\pgfqpoint{3.564050in}{3.842848in}}{\pgfqpoint{3.553000in}{3.842848in}}%
\pgfpathcurveto{\pgfqpoint{3.541950in}{3.842848in}}{\pgfqpoint{3.531351in}{3.838457in}}{\pgfqpoint{3.523537in}{3.830644in}}%
\pgfpathcurveto{\pgfqpoint{3.515724in}{3.822830in}}{\pgfqpoint{3.511333in}{3.812231in}}{\pgfqpoint{3.511333in}{3.801181in}}%
\pgfpathcurveto{\pgfqpoint{3.511333in}{3.790131in}}{\pgfqpoint{3.515724in}{3.779532in}}{\pgfqpoint{3.523537in}{3.771718in}}%
\pgfpathcurveto{\pgfqpoint{3.531351in}{3.763905in}}{\pgfqpoint{3.541950in}{3.759514in}}{\pgfqpoint{3.553000in}{3.759514in}}%
\pgfpathlineto{\pgfqpoint{3.553000in}{3.759514in}}%
\pgfpathclose%
\pgfusepath{stroke,fill}%
\end{pgfscope}%
\begin{pgfscope}%
\pgfpathrectangle{\pgfqpoint{2.963410in}{2.920818in}}{\pgfqpoint{2.177280in}{2.201755in}}%
\pgfusepath{clip}%
\pgfsetbuttcap%
\pgfsetroundjoin%
\definecolor{currentfill}{rgb}{1.000000,0.498039,0.054902}%
\pgfsetfillcolor{currentfill}%
\pgfsetlinewidth{0.481800pt}%
\definecolor{currentstroke}{rgb}{1.000000,1.000000,1.000000}%
\pgfsetstrokecolor{currentstroke}%
\pgfsetdash{}{0pt}%
\pgfpathmoveto{\pgfqpoint{3.848376in}{4.200544in}}%
\pgfpathcurveto{\pgfqpoint{3.859427in}{4.200544in}}{\pgfqpoint{3.870026in}{4.204934in}}{\pgfqpoint{3.877839in}{4.212748in}}%
\pgfpathcurveto{\pgfqpoint{3.885653in}{4.220561in}}{\pgfqpoint{3.890043in}{4.231160in}}{\pgfqpoint{3.890043in}{4.242211in}}%
\pgfpathcurveto{\pgfqpoint{3.890043in}{4.253261in}}{\pgfqpoint{3.885653in}{4.263860in}}{\pgfqpoint{3.877839in}{4.271673in}}%
\pgfpathcurveto{\pgfqpoint{3.870026in}{4.279487in}}{\pgfqpoint{3.859427in}{4.283877in}}{\pgfqpoint{3.848376in}{4.283877in}}%
\pgfpathcurveto{\pgfqpoint{3.837326in}{4.283877in}}{\pgfqpoint{3.826727in}{4.279487in}}{\pgfqpoint{3.818914in}{4.271673in}}%
\pgfpathcurveto{\pgfqpoint{3.811100in}{4.263860in}}{\pgfqpoint{3.806710in}{4.253261in}}{\pgfqpoint{3.806710in}{4.242211in}}%
\pgfpathcurveto{\pgfqpoint{3.806710in}{4.231160in}}{\pgfqpoint{3.811100in}{4.220561in}}{\pgfqpoint{3.818914in}{4.212748in}}%
\pgfpathcurveto{\pgfqpoint{3.826727in}{4.204934in}}{\pgfqpoint{3.837326in}{4.200544in}}{\pgfqpoint{3.848376in}{4.200544in}}%
\pgfpathlineto{\pgfqpoint{3.848376in}{4.200544in}}%
\pgfpathclose%
\pgfusepath{stroke,fill}%
\end{pgfscope}%
\begin{pgfscope}%
\pgfpathrectangle{\pgfqpoint{2.963410in}{2.920818in}}{\pgfqpoint{2.177280in}{2.201755in}}%
\pgfusepath{clip}%
\pgfsetbuttcap%
\pgfsetroundjoin%
\definecolor{currentfill}{rgb}{1.000000,0.498039,0.054902}%
\pgfsetfillcolor{currentfill}%
\pgfsetlinewidth{0.481800pt}%
\definecolor{currentstroke}{rgb}{1.000000,1.000000,1.000000}%
\pgfsetstrokecolor{currentstroke}%
\pgfsetdash{}{0pt}%
\pgfpathmoveto{\pgfqpoint{3.730226in}{3.963066in}}%
\pgfpathcurveto{\pgfqpoint{3.741276in}{3.963066in}}{\pgfqpoint{3.751875in}{3.967457in}}{\pgfqpoint{3.759689in}{3.975270in}}%
\pgfpathcurveto{\pgfqpoint{3.767502in}{3.983084in}}{\pgfqpoint{3.771893in}{3.993683in}}{\pgfqpoint{3.771893in}{4.004733in}}%
\pgfpathcurveto{\pgfqpoint{3.771893in}{4.015783in}}{\pgfqpoint{3.767502in}{4.026382in}}{\pgfqpoint{3.759689in}{4.034196in}}%
\pgfpathcurveto{\pgfqpoint{3.751875in}{4.042010in}}{\pgfqpoint{3.741276in}{4.046400in}}{\pgfqpoint{3.730226in}{4.046400in}}%
\pgfpathcurveto{\pgfqpoint{3.719176in}{4.046400in}}{\pgfqpoint{3.708577in}{4.042010in}}{\pgfqpoint{3.700763in}{4.034196in}}%
\pgfpathcurveto{\pgfqpoint{3.692950in}{4.026382in}}{\pgfqpoint{3.688559in}{4.015783in}}{\pgfqpoint{3.688559in}{4.004733in}}%
\pgfpathcurveto{\pgfqpoint{3.688559in}{3.993683in}}{\pgfqpoint{3.692950in}{3.983084in}}{\pgfqpoint{3.700763in}{3.975270in}}%
\pgfpathcurveto{\pgfqpoint{3.708577in}{3.967457in}}{\pgfqpoint{3.719176in}{3.963066in}}{\pgfqpoint{3.730226in}{3.963066in}}%
\pgfpathlineto{\pgfqpoint{3.730226in}{3.963066in}}%
\pgfpathclose%
\pgfusepath{stroke,fill}%
\end{pgfscope}%
\begin{pgfscope}%
\pgfpathrectangle{\pgfqpoint{2.963410in}{2.920818in}}{\pgfqpoint{2.177280in}{2.201755in}}%
\pgfusepath{clip}%
\pgfsetbuttcap%
\pgfsetroundjoin%
\definecolor{currentfill}{rgb}{1.000000,0.498039,0.054902}%
\pgfsetfillcolor{currentfill}%
\pgfsetlinewidth{0.481800pt}%
\definecolor{currentstroke}{rgb}{1.000000,1.000000,1.000000}%
\pgfsetstrokecolor{currentstroke}%
\pgfsetdash{}{0pt}%
\pgfpathmoveto{\pgfqpoint{3.316699in}{3.827365in}}%
\pgfpathcurveto{\pgfqpoint{3.327749in}{3.827365in}}{\pgfqpoint{3.338348in}{3.831755in}}{\pgfqpoint{3.346162in}{3.839569in}}%
\pgfpathcurveto{\pgfqpoint{3.353975in}{3.847383in}}{\pgfqpoint{3.358366in}{3.857982in}}{\pgfqpoint{3.358366in}{3.869032in}}%
\pgfpathcurveto{\pgfqpoint{3.358366in}{3.880082in}}{\pgfqpoint{3.353975in}{3.890681in}}{\pgfqpoint{3.346162in}{3.898495in}}%
\pgfpathcurveto{\pgfqpoint{3.338348in}{3.906308in}}{\pgfqpoint{3.327749in}{3.910698in}}{\pgfqpoint{3.316699in}{3.910698in}}%
\pgfpathcurveto{\pgfqpoint{3.305649in}{3.910698in}}{\pgfqpoint{3.295050in}{3.906308in}}{\pgfqpoint{3.287236in}{3.898495in}}%
\pgfpathcurveto{\pgfqpoint{3.279423in}{3.890681in}}{\pgfqpoint{3.275032in}{3.880082in}}{\pgfqpoint{3.275032in}{3.869032in}}%
\pgfpathcurveto{\pgfqpoint{3.275032in}{3.857982in}}{\pgfqpoint{3.279423in}{3.847383in}}{\pgfqpoint{3.287236in}{3.839569in}}%
\pgfpathcurveto{\pgfqpoint{3.295050in}{3.831755in}}{\pgfqpoint{3.305649in}{3.827365in}}{\pgfqpoint{3.316699in}{3.827365in}}%
\pgfpathlineto{\pgfqpoint{3.316699in}{3.827365in}}%
\pgfpathclose%
\pgfusepath{stroke,fill}%
\end{pgfscope}%
\begin{pgfscope}%
\pgfpathrectangle{\pgfqpoint{2.963410in}{2.920818in}}{\pgfqpoint{2.177280in}{2.201755in}}%
\pgfusepath{clip}%
\pgfsetbuttcap%
\pgfsetroundjoin%
\definecolor{currentfill}{rgb}{1.000000,0.498039,0.054902}%
\pgfsetfillcolor{currentfill}%
\pgfsetlinewidth{0.481800pt}%
\definecolor{currentstroke}{rgb}{1.000000,1.000000,1.000000}%
\pgfsetstrokecolor{currentstroke}%
\pgfsetdash{}{0pt}%
\pgfpathmoveto{\pgfqpoint{3.907452in}{4.064843in}}%
\pgfpathcurveto{\pgfqpoint{3.918502in}{4.064843in}}{\pgfqpoint{3.929101in}{4.069233in}}{\pgfqpoint{3.936915in}{4.077046in}}%
\pgfpathcurveto{\pgfqpoint{3.944728in}{4.084860in}}{\pgfqpoint{3.949118in}{4.095459in}}{\pgfqpoint{3.949118in}{4.106509in}}%
\pgfpathcurveto{\pgfqpoint{3.949118in}{4.117559in}}{\pgfqpoint{3.944728in}{4.128158in}}{\pgfqpoint{3.936915in}{4.135972in}}%
\pgfpathcurveto{\pgfqpoint{3.929101in}{4.143786in}}{\pgfqpoint{3.918502in}{4.148176in}}{\pgfqpoint{3.907452in}{4.148176in}}%
\pgfpathcurveto{\pgfqpoint{3.896402in}{4.148176in}}{\pgfqpoint{3.885803in}{4.143786in}}{\pgfqpoint{3.877989in}{4.135972in}}%
\pgfpathcurveto{\pgfqpoint{3.870175in}{4.128158in}}{\pgfqpoint{3.865785in}{4.117559in}}{\pgfqpoint{3.865785in}{4.106509in}}%
\pgfpathcurveto{\pgfqpoint{3.865785in}{4.095459in}}{\pgfqpoint{3.870175in}{4.084860in}}{\pgfqpoint{3.877989in}{4.077046in}}%
\pgfpathcurveto{\pgfqpoint{3.885803in}{4.069233in}}{\pgfqpoint{3.896402in}{4.064843in}}{\pgfqpoint{3.907452in}{4.064843in}}%
\pgfpathlineto{\pgfqpoint{3.907452in}{4.064843in}}%
\pgfpathclose%
\pgfusepath{stroke,fill}%
\end{pgfscope}%
\begin{pgfscope}%
\pgfpathrectangle{\pgfqpoint{2.963410in}{2.920818in}}{\pgfqpoint{2.177280in}{2.201755in}}%
\pgfusepath{clip}%
\pgfsetbuttcap%
\pgfsetroundjoin%
\definecolor{currentfill}{rgb}{1.000000,0.498039,0.054902}%
\pgfsetfillcolor{currentfill}%
\pgfsetlinewidth{0.481800pt}%
\definecolor{currentstroke}{rgb}{1.000000,1.000000,1.000000}%
\pgfsetstrokecolor{currentstroke}%
\pgfsetdash{}{0pt}%
\pgfpathmoveto{\pgfqpoint{3.434850in}{3.996992in}}%
\pgfpathcurveto{\pgfqpoint{3.445900in}{3.996992in}}{\pgfqpoint{3.456499in}{4.001382in}}{\pgfqpoint{3.464312in}{4.009196in}}%
\pgfpathcurveto{\pgfqpoint{3.472126in}{4.017009in}}{\pgfqpoint{3.476516in}{4.027608in}}{\pgfqpoint{3.476516in}{4.038659in}}%
\pgfpathcurveto{\pgfqpoint{3.476516in}{4.049709in}}{\pgfqpoint{3.472126in}{4.060308in}}{\pgfqpoint{3.464312in}{4.068121in}}%
\pgfpathcurveto{\pgfqpoint{3.456499in}{4.075935in}}{\pgfqpoint{3.445900in}{4.080325in}}{\pgfqpoint{3.434850in}{4.080325in}}%
\pgfpathcurveto{\pgfqpoint{3.423799in}{4.080325in}}{\pgfqpoint{3.413200in}{4.075935in}}{\pgfqpoint{3.405387in}{4.068121in}}%
\pgfpathcurveto{\pgfqpoint{3.397573in}{4.060308in}}{\pgfqpoint{3.393183in}{4.049709in}}{\pgfqpoint{3.393183in}{4.038659in}}%
\pgfpathcurveto{\pgfqpoint{3.393183in}{4.027608in}}{\pgfqpoint{3.397573in}{4.017009in}}{\pgfqpoint{3.405387in}{4.009196in}}%
\pgfpathcurveto{\pgfqpoint{3.413200in}{4.001382in}}{\pgfqpoint{3.423799in}{3.996992in}}{\pgfqpoint{3.434850in}{3.996992in}}%
\pgfpathlineto{\pgfqpoint{3.434850in}{3.996992in}}%
\pgfpathclose%
\pgfusepath{stroke,fill}%
\end{pgfscope}%
\begin{pgfscope}%
\pgfpathrectangle{\pgfqpoint{2.963410in}{2.920818in}}{\pgfqpoint{2.177280in}{2.201755in}}%
\pgfusepath{clip}%
\pgfsetbuttcap%
\pgfsetroundjoin%
\definecolor{currentfill}{rgb}{1.000000,0.498039,0.054902}%
\pgfsetfillcolor{currentfill}%
\pgfsetlinewidth{0.481800pt}%
\definecolor{currentstroke}{rgb}{1.000000,1.000000,1.000000}%
\pgfsetstrokecolor{currentstroke}%
\pgfsetdash{}{0pt}%
\pgfpathmoveto{\pgfqpoint{3.848376in}{4.234469in}}%
\pgfpathcurveto{\pgfqpoint{3.859427in}{4.234469in}}{\pgfqpoint{3.870026in}{4.238860in}}{\pgfqpoint{3.877839in}{4.246673in}}%
\pgfpathcurveto{\pgfqpoint{3.885653in}{4.254487in}}{\pgfqpoint{3.890043in}{4.265086in}}{\pgfqpoint{3.890043in}{4.276136in}}%
\pgfpathcurveto{\pgfqpoint{3.890043in}{4.287186in}}{\pgfqpoint{3.885653in}{4.297785in}}{\pgfqpoint{3.877839in}{4.305599in}}%
\pgfpathcurveto{\pgfqpoint{3.870026in}{4.313412in}}{\pgfqpoint{3.859427in}{4.317803in}}{\pgfqpoint{3.848376in}{4.317803in}}%
\pgfpathcurveto{\pgfqpoint{3.837326in}{4.317803in}}{\pgfqpoint{3.826727in}{4.313412in}}{\pgfqpoint{3.818914in}{4.305599in}}%
\pgfpathcurveto{\pgfqpoint{3.811100in}{4.297785in}}{\pgfqpoint{3.806710in}{4.287186in}}{\pgfqpoint{3.806710in}{4.276136in}}%
\pgfpathcurveto{\pgfqpoint{3.806710in}{4.265086in}}{\pgfqpoint{3.811100in}{4.254487in}}{\pgfqpoint{3.818914in}{4.246673in}}%
\pgfpathcurveto{\pgfqpoint{3.826727in}{4.238860in}}{\pgfqpoint{3.837326in}{4.234469in}}{\pgfqpoint{3.848376in}{4.234469in}}%
\pgfpathlineto{\pgfqpoint{3.848376in}{4.234469in}}%
\pgfpathclose%
\pgfusepath{stroke,fill}%
\end{pgfscope}%
\begin{pgfscope}%
\pgfpathrectangle{\pgfqpoint{2.963410in}{2.920818in}}{\pgfqpoint{2.177280in}{2.201755in}}%
\pgfusepath{clip}%
\pgfsetbuttcap%
\pgfsetroundjoin%
\definecolor{currentfill}{rgb}{1.000000,0.498039,0.054902}%
\pgfsetfillcolor{currentfill}%
\pgfsetlinewidth{0.481800pt}%
\definecolor{currentstroke}{rgb}{1.000000,1.000000,1.000000}%
\pgfsetstrokecolor{currentstroke}%
\pgfsetdash{}{0pt}%
\pgfpathmoveto{\pgfqpoint{3.848376in}{3.861290in}}%
\pgfpathcurveto{\pgfqpoint{3.859427in}{3.861290in}}{\pgfqpoint{3.870026in}{3.865681in}}{\pgfqpoint{3.877839in}{3.873494in}}%
\pgfpathcurveto{\pgfqpoint{3.885653in}{3.881308in}}{\pgfqpoint{3.890043in}{3.891907in}}{\pgfqpoint{3.890043in}{3.902957in}}%
\pgfpathcurveto{\pgfqpoint{3.890043in}{3.914007in}}{\pgfqpoint{3.885653in}{3.924606in}}{\pgfqpoint{3.877839in}{3.932420in}}%
\pgfpathcurveto{\pgfqpoint{3.870026in}{3.940234in}}{\pgfqpoint{3.859427in}{3.944624in}}{\pgfqpoint{3.848376in}{3.944624in}}%
\pgfpathcurveto{\pgfqpoint{3.837326in}{3.944624in}}{\pgfqpoint{3.826727in}{3.940234in}}{\pgfqpoint{3.818914in}{3.932420in}}%
\pgfpathcurveto{\pgfqpoint{3.811100in}{3.924606in}}{\pgfqpoint{3.806710in}{3.914007in}}{\pgfqpoint{3.806710in}{3.902957in}}%
\pgfpathcurveto{\pgfqpoint{3.806710in}{3.891907in}}{\pgfqpoint{3.811100in}{3.881308in}}{\pgfqpoint{3.818914in}{3.873494in}}%
\pgfpathcurveto{\pgfqpoint{3.826727in}{3.865681in}}{\pgfqpoint{3.837326in}{3.861290in}}{\pgfqpoint{3.848376in}{3.861290in}}%
\pgfpathlineto{\pgfqpoint{3.848376in}{3.861290in}}%
\pgfpathclose%
\pgfusepath{stroke,fill}%
\end{pgfscope}%
\begin{pgfscope}%
\pgfpathrectangle{\pgfqpoint{2.963410in}{2.920818in}}{\pgfqpoint{2.177280in}{2.201755in}}%
\pgfusepath{clip}%
\pgfsetbuttcap%
\pgfsetroundjoin%
\definecolor{currentfill}{rgb}{1.000000,0.498039,0.054902}%
\pgfsetfillcolor{currentfill}%
\pgfsetlinewidth{0.481800pt}%
\definecolor{currentstroke}{rgb}{1.000000,1.000000,1.000000}%
\pgfsetstrokecolor{currentstroke}%
\pgfsetdash{}{0pt}%
\pgfpathmoveto{\pgfqpoint{3.966527in}{4.132693in}}%
\pgfpathcurveto{\pgfqpoint{3.977577in}{4.132693in}}{\pgfqpoint{3.988176in}{4.137083in}}{\pgfqpoint{3.995990in}{4.144897in}}%
\pgfpathcurveto{\pgfqpoint{4.003803in}{4.152711in}}{\pgfqpoint{4.008194in}{4.163310in}}{\pgfqpoint{4.008194in}{4.174360in}}%
\pgfpathcurveto{\pgfqpoint{4.008194in}{4.185410in}}{\pgfqpoint{4.003803in}{4.196009in}}{\pgfqpoint{3.995990in}{4.203823in}}%
\pgfpathcurveto{\pgfqpoint{3.988176in}{4.211636in}}{\pgfqpoint{3.977577in}{4.216027in}}{\pgfqpoint{3.966527in}{4.216027in}}%
\pgfpathcurveto{\pgfqpoint{3.955477in}{4.216027in}}{\pgfqpoint{3.944878in}{4.211636in}}{\pgfqpoint{3.937064in}{4.203823in}}%
\pgfpathcurveto{\pgfqpoint{3.929251in}{4.196009in}}{\pgfqpoint{3.924860in}{4.185410in}}{\pgfqpoint{3.924860in}{4.174360in}}%
\pgfpathcurveto{\pgfqpoint{3.924860in}{4.163310in}}{\pgfqpoint{3.929251in}{4.152711in}}{\pgfqpoint{3.937064in}{4.144897in}}%
\pgfpathcurveto{\pgfqpoint{3.944878in}{4.137083in}}{\pgfqpoint{3.955477in}{4.132693in}}{\pgfqpoint{3.966527in}{4.132693in}}%
\pgfpathlineto{\pgfqpoint{3.966527in}{4.132693in}}%
\pgfpathclose%
\pgfusepath{stroke,fill}%
\end{pgfscope}%
\begin{pgfscope}%
\pgfpathrectangle{\pgfqpoint{2.963410in}{2.920818in}}{\pgfqpoint{2.177280in}{2.201755in}}%
\pgfusepath{clip}%
\pgfsetbuttcap%
\pgfsetroundjoin%
\definecolor{currentfill}{rgb}{1.000000,0.498039,0.054902}%
\pgfsetfillcolor{currentfill}%
\pgfsetlinewidth{0.481800pt}%
\definecolor{currentstroke}{rgb}{1.000000,1.000000,1.000000}%
\pgfsetstrokecolor{currentstroke}%
\pgfsetdash{}{0pt}%
\pgfpathmoveto{\pgfqpoint{3.907452in}{4.166619in}}%
\pgfpathcurveto{\pgfqpoint{3.918502in}{4.166619in}}{\pgfqpoint{3.929101in}{4.171009in}}{\pgfqpoint{3.936915in}{4.178822in}}%
\pgfpathcurveto{\pgfqpoint{3.944728in}{4.186636in}}{\pgfqpoint{3.949118in}{4.197235in}}{\pgfqpoint{3.949118in}{4.208285in}}%
\pgfpathcurveto{\pgfqpoint{3.949118in}{4.219335in}}{\pgfqpoint{3.944728in}{4.229934in}}{\pgfqpoint{3.936915in}{4.237748in}}%
\pgfpathcurveto{\pgfqpoint{3.929101in}{4.245562in}}{\pgfqpoint{3.918502in}{4.249952in}}{\pgfqpoint{3.907452in}{4.249952in}}%
\pgfpathcurveto{\pgfqpoint{3.896402in}{4.249952in}}{\pgfqpoint{3.885803in}{4.245562in}}{\pgfqpoint{3.877989in}{4.237748in}}%
\pgfpathcurveto{\pgfqpoint{3.870175in}{4.229934in}}{\pgfqpoint{3.865785in}{4.219335in}}{\pgfqpoint{3.865785in}{4.208285in}}%
\pgfpathcurveto{\pgfqpoint{3.865785in}{4.197235in}}{\pgfqpoint{3.870175in}{4.186636in}}{\pgfqpoint{3.877989in}{4.178822in}}%
\pgfpathcurveto{\pgfqpoint{3.885803in}{4.171009in}}{\pgfqpoint{3.896402in}{4.166619in}}{\pgfqpoint{3.907452in}{4.166619in}}%
\pgfpathlineto{\pgfqpoint{3.907452in}{4.166619in}}%
\pgfpathclose%
\pgfusepath{stroke,fill}%
\end{pgfscope}%
\begin{pgfscope}%
\pgfpathrectangle{\pgfqpoint{2.963410in}{2.920818in}}{\pgfqpoint{2.177280in}{2.201755in}}%
\pgfusepath{clip}%
\pgfsetbuttcap%
\pgfsetroundjoin%
\definecolor{currentfill}{rgb}{1.000000,0.498039,0.054902}%
\pgfsetfillcolor{currentfill}%
\pgfsetlinewidth{0.481800pt}%
\definecolor{currentstroke}{rgb}{1.000000,1.000000,1.000000}%
\pgfsetstrokecolor{currentstroke}%
\pgfsetdash{}{0pt}%
\pgfpathmoveto{\pgfqpoint{3.730226in}{4.030917in}}%
\pgfpathcurveto{\pgfqpoint{3.741276in}{4.030917in}}{\pgfqpoint{3.751875in}{4.035307in}}{\pgfqpoint{3.759689in}{4.043121in}}%
\pgfpathcurveto{\pgfqpoint{3.767502in}{4.050935in}}{\pgfqpoint{3.771893in}{4.061534in}}{\pgfqpoint{3.771893in}{4.072584in}}%
\pgfpathcurveto{\pgfqpoint{3.771893in}{4.083634in}}{\pgfqpoint{3.767502in}{4.094233in}}{\pgfqpoint{3.759689in}{4.102047in}}%
\pgfpathcurveto{\pgfqpoint{3.751875in}{4.109860in}}{\pgfqpoint{3.741276in}{4.114251in}}{\pgfqpoint{3.730226in}{4.114251in}}%
\pgfpathcurveto{\pgfqpoint{3.719176in}{4.114251in}}{\pgfqpoint{3.708577in}{4.109860in}}{\pgfqpoint{3.700763in}{4.102047in}}%
\pgfpathcurveto{\pgfqpoint{3.692950in}{4.094233in}}{\pgfqpoint{3.688559in}{4.083634in}}{\pgfqpoint{3.688559in}{4.072584in}}%
\pgfpathcurveto{\pgfqpoint{3.688559in}{4.061534in}}{\pgfqpoint{3.692950in}{4.050935in}}{\pgfqpoint{3.700763in}{4.043121in}}%
\pgfpathcurveto{\pgfqpoint{3.708577in}{4.035307in}}{\pgfqpoint{3.719176in}{4.030917in}}{\pgfqpoint{3.730226in}{4.030917in}}%
\pgfpathlineto{\pgfqpoint{3.730226in}{4.030917in}}%
\pgfpathclose%
\pgfusepath{stroke,fill}%
\end{pgfscope}%
\begin{pgfscope}%
\pgfpathrectangle{\pgfqpoint{2.963410in}{2.920818in}}{\pgfqpoint{2.177280in}{2.201755in}}%
\pgfusepath{clip}%
\pgfsetbuttcap%
\pgfsetroundjoin%
\definecolor{currentfill}{rgb}{1.000000,0.498039,0.054902}%
\pgfsetfillcolor{currentfill}%
\pgfsetlinewidth{0.481800pt}%
\definecolor{currentstroke}{rgb}{1.000000,1.000000,1.000000}%
\pgfsetstrokecolor{currentstroke}%
\pgfsetdash{}{0pt}%
\pgfpathmoveto{\pgfqpoint{3.434850in}{4.166619in}}%
\pgfpathcurveto{\pgfqpoint{3.445900in}{4.166619in}}{\pgfqpoint{3.456499in}{4.171009in}}{\pgfqpoint{3.464312in}{4.178822in}}%
\pgfpathcurveto{\pgfqpoint{3.472126in}{4.186636in}}{\pgfqpoint{3.476516in}{4.197235in}}{\pgfqpoint{3.476516in}{4.208285in}}%
\pgfpathcurveto{\pgfqpoint{3.476516in}{4.219335in}}{\pgfqpoint{3.472126in}{4.229934in}}{\pgfqpoint{3.464312in}{4.237748in}}%
\pgfpathcurveto{\pgfqpoint{3.456499in}{4.245562in}}{\pgfqpoint{3.445900in}{4.249952in}}{\pgfqpoint{3.434850in}{4.249952in}}%
\pgfpathcurveto{\pgfqpoint{3.423799in}{4.249952in}}{\pgfqpoint{3.413200in}{4.245562in}}{\pgfqpoint{3.405387in}{4.237748in}}%
\pgfpathcurveto{\pgfqpoint{3.397573in}{4.229934in}}{\pgfqpoint{3.393183in}{4.219335in}}{\pgfqpoint{3.393183in}{4.208285in}}%
\pgfpathcurveto{\pgfqpoint{3.393183in}{4.197235in}}{\pgfqpoint{3.397573in}{4.186636in}}{\pgfqpoint{3.405387in}{4.178822in}}%
\pgfpathcurveto{\pgfqpoint{3.413200in}{4.171009in}}{\pgfqpoint{3.423799in}{4.166619in}}{\pgfqpoint{3.434850in}{4.166619in}}%
\pgfpathlineto{\pgfqpoint{3.434850in}{4.166619in}}%
\pgfpathclose%
\pgfusepath{stroke,fill}%
\end{pgfscope}%
\begin{pgfscope}%
\pgfpathrectangle{\pgfqpoint{2.963410in}{2.920818in}}{\pgfqpoint{2.177280in}{2.201755in}}%
\pgfusepath{clip}%
\pgfsetbuttcap%
\pgfsetroundjoin%
\definecolor{currentfill}{rgb}{1.000000,0.498039,0.054902}%
\pgfsetfillcolor{currentfill}%
\pgfsetlinewidth{0.481800pt}%
\definecolor{currentstroke}{rgb}{1.000000,1.000000,1.000000}%
\pgfsetstrokecolor{currentstroke}%
\pgfsetdash{}{0pt}%
\pgfpathmoveto{\pgfqpoint{3.612075in}{3.963066in}}%
\pgfpathcurveto{\pgfqpoint{3.623126in}{3.963066in}}{\pgfqpoint{3.633725in}{3.967457in}}{\pgfqpoint{3.641538in}{3.975270in}}%
\pgfpathcurveto{\pgfqpoint{3.649352in}{3.983084in}}{\pgfqpoint{3.653742in}{3.993683in}}{\pgfqpoint{3.653742in}{4.004733in}}%
\pgfpathcurveto{\pgfqpoint{3.653742in}{4.015783in}}{\pgfqpoint{3.649352in}{4.026382in}}{\pgfqpoint{3.641538in}{4.034196in}}%
\pgfpathcurveto{\pgfqpoint{3.633725in}{4.042010in}}{\pgfqpoint{3.623126in}{4.046400in}}{\pgfqpoint{3.612075in}{4.046400in}}%
\pgfpathcurveto{\pgfqpoint{3.601025in}{4.046400in}}{\pgfqpoint{3.590426in}{4.042010in}}{\pgfqpoint{3.582613in}{4.034196in}}%
\pgfpathcurveto{\pgfqpoint{3.574799in}{4.026382in}}{\pgfqpoint{3.570409in}{4.015783in}}{\pgfqpoint{3.570409in}{4.004733in}}%
\pgfpathcurveto{\pgfqpoint{3.570409in}{3.993683in}}{\pgfqpoint{3.574799in}{3.983084in}}{\pgfqpoint{3.582613in}{3.975270in}}%
\pgfpathcurveto{\pgfqpoint{3.590426in}{3.967457in}}{\pgfqpoint{3.601025in}{3.963066in}}{\pgfqpoint{3.612075in}{3.963066in}}%
\pgfpathlineto{\pgfqpoint{3.612075in}{3.963066in}}%
\pgfpathclose%
\pgfusepath{stroke,fill}%
\end{pgfscope}%
\begin{pgfscope}%
\pgfpathrectangle{\pgfqpoint{2.963410in}{2.920818in}}{\pgfqpoint{2.177280in}{2.201755in}}%
\pgfusepath{clip}%
\pgfsetbuttcap%
\pgfsetroundjoin%
\definecolor{currentfill}{rgb}{1.000000,0.498039,0.054902}%
\pgfsetfillcolor{currentfill}%
\pgfsetlinewidth{0.481800pt}%
\definecolor{currentstroke}{rgb}{1.000000,1.000000,1.000000}%
\pgfsetstrokecolor{currentstroke}%
\pgfsetdash{}{0pt}%
\pgfpathmoveto{\pgfqpoint{4.025602in}{4.268395in}}%
\pgfpathcurveto{\pgfqpoint{4.036652in}{4.268395in}}{\pgfqpoint{4.047251in}{4.272785in}}{\pgfqpoint{4.055065in}{4.280599in}}%
\pgfpathcurveto{\pgfqpoint{4.062879in}{4.288412in}}{\pgfqpoint{4.067269in}{4.299011in}}{\pgfqpoint{4.067269in}{4.310061in}}%
\pgfpathcurveto{\pgfqpoint{4.067269in}{4.321111in}}{\pgfqpoint{4.062879in}{4.331710in}}{\pgfqpoint{4.055065in}{4.339524in}}%
\pgfpathcurveto{\pgfqpoint{4.047251in}{4.347338in}}{\pgfqpoint{4.036652in}{4.351728in}}{\pgfqpoint{4.025602in}{4.351728in}}%
\pgfpathcurveto{\pgfqpoint{4.014552in}{4.351728in}}{\pgfqpoint{4.003953in}{4.347338in}}{\pgfqpoint{3.996139in}{4.339524in}}%
\pgfpathcurveto{\pgfqpoint{3.988326in}{4.331710in}}{\pgfqpoint{3.983936in}{4.321111in}}{\pgfqpoint{3.983936in}{4.310061in}}%
\pgfpathcurveto{\pgfqpoint{3.983936in}{4.299011in}}{\pgfqpoint{3.988326in}{4.288412in}}{\pgfqpoint{3.996139in}{4.280599in}}%
\pgfpathcurveto{\pgfqpoint{4.003953in}{4.272785in}}{\pgfqpoint{4.014552in}{4.268395in}}{\pgfqpoint{4.025602in}{4.268395in}}%
\pgfpathlineto{\pgfqpoint{4.025602in}{4.268395in}}%
\pgfpathclose%
\pgfusepath{stroke,fill}%
\end{pgfscope}%
\begin{pgfscope}%
\pgfpathrectangle{\pgfqpoint{2.963410in}{2.920818in}}{\pgfqpoint{2.177280in}{2.201755in}}%
\pgfusepath{clip}%
\pgfsetbuttcap%
\pgfsetroundjoin%
\definecolor{currentfill}{rgb}{1.000000,0.498039,0.054902}%
\pgfsetfillcolor{currentfill}%
\pgfsetlinewidth{0.481800pt}%
\definecolor{currentstroke}{rgb}{1.000000,1.000000,1.000000}%
\pgfsetstrokecolor{currentstroke}%
\pgfsetdash{}{0pt}%
\pgfpathmoveto{\pgfqpoint{3.789301in}{3.996992in}}%
\pgfpathcurveto{\pgfqpoint{3.800351in}{3.996992in}}{\pgfqpoint{3.810950in}{4.001382in}}{\pgfqpoint{3.818764in}{4.009196in}}%
\pgfpathcurveto{\pgfqpoint{3.826578in}{4.017009in}}{\pgfqpoint{3.830968in}{4.027608in}}{\pgfqpoint{3.830968in}{4.038659in}}%
\pgfpathcurveto{\pgfqpoint{3.830968in}{4.049709in}}{\pgfqpoint{3.826578in}{4.060308in}}{\pgfqpoint{3.818764in}{4.068121in}}%
\pgfpathcurveto{\pgfqpoint{3.810950in}{4.075935in}}{\pgfqpoint{3.800351in}{4.080325in}}{\pgfqpoint{3.789301in}{4.080325in}}%
\pgfpathcurveto{\pgfqpoint{3.778251in}{4.080325in}}{\pgfqpoint{3.767652in}{4.075935in}}{\pgfqpoint{3.759838in}{4.068121in}}%
\pgfpathcurveto{\pgfqpoint{3.752025in}{4.060308in}}{\pgfqpoint{3.747635in}{4.049709in}}{\pgfqpoint{3.747635in}{4.038659in}}%
\pgfpathcurveto{\pgfqpoint{3.747635in}{4.027608in}}{\pgfqpoint{3.752025in}{4.017009in}}{\pgfqpoint{3.759838in}{4.009196in}}%
\pgfpathcurveto{\pgfqpoint{3.767652in}{4.001382in}}{\pgfqpoint{3.778251in}{3.996992in}}{\pgfqpoint{3.789301in}{3.996992in}}%
\pgfpathlineto{\pgfqpoint{3.789301in}{3.996992in}}%
\pgfpathclose%
\pgfusepath{stroke,fill}%
\end{pgfscope}%
\begin{pgfscope}%
\pgfpathrectangle{\pgfqpoint{2.963410in}{2.920818in}}{\pgfqpoint{2.177280in}{2.201755in}}%
\pgfusepath{clip}%
\pgfsetbuttcap%
\pgfsetroundjoin%
\definecolor{currentfill}{rgb}{1.000000,0.498039,0.054902}%
\pgfsetfillcolor{currentfill}%
\pgfsetlinewidth{0.481800pt}%
\definecolor{currentstroke}{rgb}{1.000000,1.000000,1.000000}%
\pgfsetstrokecolor{currentstroke}%
\pgfsetdash{}{0pt}%
\pgfpathmoveto{\pgfqpoint{3.612075in}{4.302320in}}%
\pgfpathcurveto{\pgfqpoint{3.623126in}{4.302320in}}{\pgfqpoint{3.633725in}{4.306710in}}{\pgfqpoint{3.641538in}{4.314524in}}%
\pgfpathcurveto{\pgfqpoint{3.649352in}{4.322337in}}{\pgfqpoint{3.653742in}{4.332937in}}{\pgfqpoint{3.653742in}{4.343987in}}%
\pgfpathcurveto{\pgfqpoint{3.653742in}{4.355037in}}{\pgfqpoint{3.649352in}{4.365636in}}{\pgfqpoint{3.641538in}{4.373449in}}%
\pgfpathcurveto{\pgfqpoint{3.633725in}{4.381263in}}{\pgfqpoint{3.623126in}{4.385653in}}{\pgfqpoint{3.612075in}{4.385653in}}%
\pgfpathcurveto{\pgfqpoint{3.601025in}{4.385653in}}{\pgfqpoint{3.590426in}{4.381263in}}{\pgfqpoint{3.582613in}{4.373449in}}%
\pgfpathcurveto{\pgfqpoint{3.574799in}{4.365636in}}{\pgfqpoint{3.570409in}{4.355037in}}{\pgfqpoint{3.570409in}{4.343987in}}%
\pgfpathcurveto{\pgfqpoint{3.570409in}{4.332937in}}{\pgfqpoint{3.574799in}{4.322337in}}{\pgfqpoint{3.582613in}{4.314524in}}%
\pgfpathcurveto{\pgfqpoint{3.590426in}{4.306710in}}{\pgfqpoint{3.601025in}{4.302320in}}{\pgfqpoint{3.612075in}{4.302320in}}%
\pgfpathlineto{\pgfqpoint{3.612075in}{4.302320in}}%
\pgfpathclose%
\pgfusepath{stroke,fill}%
\end{pgfscope}%
\begin{pgfscope}%
\pgfpathrectangle{\pgfqpoint{2.963410in}{2.920818in}}{\pgfqpoint{2.177280in}{2.201755in}}%
\pgfusepath{clip}%
\pgfsetbuttcap%
\pgfsetroundjoin%
\definecolor{currentfill}{rgb}{1.000000,0.498039,0.054902}%
\pgfsetfillcolor{currentfill}%
\pgfsetlinewidth{0.481800pt}%
\definecolor{currentstroke}{rgb}{1.000000,1.000000,1.000000}%
\pgfsetstrokecolor{currentstroke}%
\pgfsetdash{}{0pt}%
\pgfpathmoveto{\pgfqpoint{3.789301in}{4.234469in}}%
\pgfpathcurveto{\pgfqpoint{3.800351in}{4.234469in}}{\pgfqpoint{3.810950in}{4.238860in}}{\pgfqpoint{3.818764in}{4.246673in}}%
\pgfpathcurveto{\pgfqpoint{3.826578in}{4.254487in}}{\pgfqpoint{3.830968in}{4.265086in}}{\pgfqpoint{3.830968in}{4.276136in}}%
\pgfpathcurveto{\pgfqpoint{3.830968in}{4.287186in}}{\pgfqpoint{3.826578in}{4.297785in}}{\pgfqpoint{3.818764in}{4.305599in}}%
\pgfpathcurveto{\pgfqpoint{3.810950in}{4.313412in}}{\pgfqpoint{3.800351in}{4.317803in}}{\pgfqpoint{3.789301in}{4.317803in}}%
\pgfpathcurveto{\pgfqpoint{3.778251in}{4.317803in}}{\pgfqpoint{3.767652in}{4.313412in}}{\pgfqpoint{3.759838in}{4.305599in}}%
\pgfpathcurveto{\pgfqpoint{3.752025in}{4.297785in}}{\pgfqpoint{3.747635in}{4.287186in}}{\pgfqpoint{3.747635in}{4.276136in}}%
\pgfpathcurveto{\pgfqpoint{3.747635in}{4.265086in}}{\pgfqpoint{3.752025in}{4.254487in}}{\pgfqpoint{3.759838in}{4.246673in}}%
\pgfpathcurveto{\pgfqpoint{3.767652in}{4.238860in}}{\pgfqpoint{3.778251in}{4.234469in}}{\pgfqpoint{3.789301in}{4.234469in}}%
\pgfpathlineto{\pgfqpoint{3.789301in}{4.234469in}}%
\pgfpathclose%
\pgfusepath{stroke,fill}%
\end{pgfscope}%
\begin{pgfscope}%
\pgfpathrectangle{\pgfqpoint{2.963410in}{2.920818in}}{\pgfqpoint{2.177280in}{2.201755in}}%
\pgfusepath{clip}%
\pgfsetbuttcap%
\pgfsetroundjoin%
\definecolor{currentfill}{rgb}{1.000000,0.498039,0.054902}%
\pgfsetfillcolor{currentfill}%
\pgfsetlinewidth{0.481800pt}%
\definecolor{currentstroke}{rgb}{1.000000,1.000000,1.000000}%
\pgfsetstrokecolor{currentstroke}%
\pgfsetdash{}{0pt}%
\pgfpathmoveto{\pgfqpoint{3.848376in}{4.098768in}}%
\pgfpathcurveto{\pgfqpoint{3.859427in}{4.098768in}}{\pgfqpoint{3.870026in}{4.103158in}}{\pgfqpoint{3.877839in}{4.110972in}}%
\pgfpathcurveto{\pgfqpoint{3.885653in}{4.118785in}}{\pgfqpoint{3.890043in}{4.129384in}}{\pgfqpoint{3.890043in}{4.140435in}}%
\pgfpathcurveto{\pgfqpoint{3.890043in}{4.151485in}}{\pgfqpoint{3.885653in}{4.162084in}}{\pgfqpoint{3.877839in}{4.169897in}}%
\pgfpathcurveto{\pgfqpoint{3.870026in}{4.177711in}}{\pgfqpoint{3.859427in}{4.182101in}}{\pgfqpoint{3.848376in}{4.182101in}}%
\pgfpathcurveto{\pgfqpoint{3.837326in}{4.182101in}}{\pgfqpoint{3.826727in}{4.177711in}}{\pgfqpoint{3.818914in}{4.169897in}}%
\pgfpathcurveto{\pgfqpoint{3.811100in}{4.162084in}}{\pgfqpoint{3.806710in}{4.151485in}}{\pgfqpoint{3.806710in}{4.140435in}}%
\pgfpathcurveto{\pgfqpoint{3.806710in}{4.129384in}}{\pgfqpoint{3.811100in}{4.118785in}}{\pgfqpoint{3.818914in}{4.110972in}}%
\pgfpathcurveto{\pgfqpoint{3.826727in}{4.103158in}}{\pgfqpoint{3.837326in}{4.098768in}}{\pgfqpoint{3.848376in}{4.098768in}}%
\pgfpathlineto{\pgfqpoint{3.848376in}{4.098768in}}%
\pgfpathclose%
\pgfusepath{stroke,fill}%
\end{pgfscope}%
\begin{pgfscope}%
\pgfpathrectangle{\pgfqpoint{2.963410in}{2.920818in}}{\pgfqpoint{2.177280in}{2.201755in}}%
\pgfusepath{clip}%
\pgfsetbuttcap%
\pgfsetroundjoin%
\definecolor{currentfill}{rgb}{1.000000,0.498039,0.054902}%
\pgfsetfillcolor{currentfill}%
\pgfsetlinewidth{0.481800pt}%
\definecolor{currentstroke}{rgb}{1.000000,1.000000,1.000000}%
\pgfsetstrokecolor{currentstroke}%
\pgfsetdash{}{0pt}%
\pgfpathmoveto{\pgfqpoint{3.907452in}{4.132693in}}%
\pgfpathcurveto{\pgfqpoint{3.918502in}{4.132693in}}{\pgfqpoint{3.929101in}{4.137083in}}{\pgfqpoint{3.936915in}{4.144897in}}%
\pgfpathcurveto{\pgfqpoint{3.944728in}{4.152711in}}{\pgfqpoint{3.949118in}{4.163310in}}{\pgfqpoint{3.949118in}{4.174360in}}%
\pgfpathcurveto{\pgfqpoint{3.949118in}{4.185410in}}{\pgfqpoint{3.944728in}{4.196009in}}{\pgfqpoint{3.936915in}{4.203823in}}%
\pgfpathcurveto{\pgfqpoint{3.929101in}{4.211636in}}{\pgfqpoint{3.918502in}{4.216027in}}{\pgfqpoint{3.907452in}{4.216027in}}%
\pgfpathcurveto{\pgfqpoint{3.896402in}{4.216027in}}{\pgfqpoint{3.885803in}{4.211636in}}{\pgfqpoint{3.877989in}{4.203823in}}%
\pgfpathcurveto{\pgfqpoint{3.870175in}{4.196009in}}{\pgfqpoint{3.865785in}{4.185410in}}{\pgfqpoint{3.865785in}{4.174360in}}%
\pgfpathcurveto{\pgfqpoint{3.865785in}{4.163310in}}{\pgfqpoint{3.870175in}{4.152711in}}{\pgfqpoint{3.877989in}{4.144897in}}%
\pgfpathcurveto{\pgfqpoint{3.885803in}{4.137083in}}{\pgfqpoint{3.896402in}{4.132693in}}{\pgfqpoint{3.907452in}{4.132693in}}%
\pgfpathlineto{\pgfqpoint{3.907452in}{4.132693in}}%
\pgfpathclose%
\pgfusepath{stroke,fill}%
\end{pgfscope}%
\begin{pgfscope}%
\pgfpathrectangle{\pgfqpoint{2.963410in}{2.920818in}}{\pgfqpoint{2.177280in}{2.201755in}}%
\pgfusepath{clip}%
\pgfsetbuttcap%
\pgfsetroundjoin%
\definecolor{currentfill}{rgb}{1.000000,0.498039,0.054902}%
\pgfsetfillcolor{currentfill}%
\pgfsetlinewidth{0.481800pt}%
\definecolor{currentstroke}{rgb}{1.000000,1.000000,1.000000}%
\pgfsetstrokecolor{currentstroke}%
\pgfsetdash{}{0pt}%
\pgfpathmoveto{\pgfqpoint{3.789301in}{4.268395in}}%
\pgfpathcurveto{\pgfqpoint{3.800351in}{4.268395in}}{\pgfqpoint{3.810950in}{4.272785in}}{\pgfqpoint{3.818764in}{4.280599in}}%
\pgfpathcurveto{\pgfqpoint{3.826578in}{4.288412in}}{\pgfqpoint{3.830968in}{4.299011in}}{\pgfqpoint{3.830968in}{4.310061in}}%
\pgfpathcurveto{\pgfqpoint{3.830968in}{4.321111in}}{\pgfqpoint{3.826578in}{4.331710in}}{\pgfqpoint{3.818764in}{4.339524in}}%
\pgfpathcurveto{\pgfqpoint{3.810950in}{4.347338in}}{\pgfqpoint{3.800351in}{4.351728in}}{\pgfqpoint{3.789301in}{4.351728in}}%
\pgfpathcurveto{\pgfqpoint{3.778251in}{4.351728in}}{\pgfqpoint{3.767652in}{4.347338in}}{\pgfqpoint{3.759838in}{4.339524in}}%
\pgfpathcurveto{\pgfqpoint{3.752025in}{4.331710in}}{\pgfqpoint{3.747635in}{4.321111in}}{\pgfqpoint{3.747635in}{4.310061in}}%
\pgfpathcurveto{\pgfqpoint{3.747635in}{4.299011in}}{\pgfqpoint{3.752025in}{4.288412in}}{\pgfqpoint{3.759838in}{4.280599in}}%
\pgfpathcurveto{\pgfqpoint{3.767652in}{4.272785in}}{\pgfqpoint{3.778251in}{4.268395in}}{\pgfqpoint{3.789301in}{4.268395in}}%
\pgfpathlineto{\pgfqpoint{3.789301in}{4.268395in}}%
\pgfpathclose%
\pgfusepath{stroke,fill}%
\end{pgfscope}%
\begin{pgfscope}%
\pgfpathrectangle{\pgfqpoint{2.963410in}{2.920818in}}{\pgfqpoint{2.177280in}{2.201755in}}%
\pgfusepath{clip}%
\pgfsetbuttcap%
\pgfsetroundjoin%
\definecolor{currentfill}{rgb}{1.000000,0.498039,0.054902}%
\pgfsetfillcolor{currentfill}%
\pgfsetlinewidth{0.481800pt}%
\definecolor{currentstroke}{rgb}{1.000000,1.000000,1.000000}%
\pgfsetstrokecolor{currentstroke}%
\pgfsetdash{}{0pt}%
\pgfpathmoveto{\pgfqpoint{3.907452in}{4.336245in}}%
\pgfpathcurveto{\pgfqpoint{3.918502in}{4.336245in}}{\pgfqpoint{3.929101in}{4.340636in}}{\pgfqpoint{3.936915in}{4.348449in}}%
\pgfpathcurveto{\pgfqpoint{3.944728in}{4.356263in}}{\pgfqpoint{3.949118in}{4.366862in}}{\pgfqpoint{3.949118in}{4.377912in}}%
\pgfpathcurveto{\pgfqpoint{3.949118in}{4.388962in}}{\pgfqpoint{3.944728in}{4.399561in}}{\pgfqpoint{3.936915in}{4.407375in}}%
\pgfpathcurveto{\pgfqpoint{3.929101in}{4.415188in}}{\pgfqpoint{3.918502in}{4.419579in}}{\pgfqpoint{3.907452in}{4.419579in}}%
\pgfpathcurveto{\pgfqpoint{3.896402in}{4.419579in}}{\pgfqpoint{3.885803in}{4.415188in}}{\pgfqpoint{3.877989in}{4.407375in}}%
\pgfpathcurveto{\pgfqpoint{3.870175in}{4.399561in}}{\pgfqpoint{3.865785in}{4.388962in}}{\pgfqpoint{3.865785in}{4.377912in}}%
\pgfpathcurveto{\pgfqpoint{3.865785in}{4.366862in}}{\pgfqpoint{3.870175in}{4.356263in}}{\pgfqpoint{3.877989in}{4.348449in}}%
\pgfpathcurveto{\pgfqpoint{3.885803in}{4.340636in}}{\pgfqpoint{3.896402in}{4.336245in}}{\pgfqpoint{3.907452in}{4.336245in}}%
\pgfpathlineto{\pgfqpoint{3.907452in}{4.336245in}}%
\pgfpathclose%
\pgfusepath{stroke,fill}%
\end{pgfscope}%
\begin{pgfscope}%
\pgfpathrectangle{\pgfqpoint{2.963410in}{2.920818in}}{\pgfqpoint{2.177280in}{2.201755in}}%
\pgfusepath{clip}%
\pgfsetbuttcap%
\pgfsetroundjoin%
\definecolor{currentfill}{rgb}{1.000000,0.498039,0.054902}%
\pgfsetfillcolor{currentfill}%
\pgfsetlinewidth{0.481800pt}%
\definecolor{currentstroke}{rgb}{1.000000,1.000000,1.000000}%
\pgfsetstrokecolor{currentstroke}%
\pgfsetdash{}{0pt}%
\pgfpathmoveto{\pgfqpoint{3.848376in}{4.166619in}}%
\pgfpathcurveto{\pgfqpoint{3.859427in}{4.166619in}}{\pgfqpoint{3.870026in}{4.171009in}}{\pgfqpoint{3.877839in}{4.178822in}}%
\pgfpathcurveto{\pgfqpoint{3.885653in}{4.186636in}}{\pgfqpoint{3.890043in}{4.197235in}}{\pgfqpoint{3.890043in}{4.208285in}}%
\pgfpathcurveto{\pgfqpoint{3.890043in}{4.219335in}}{\pgfqpoint{3.885653in}{4.229934in}}{\pgfqpoint{3.877839in}{4.237748in}}%
\pgfpathcurveto{\pgfqpoint{3.870026in}{4.245562in}}{\pgfqpoint{3.859427in}{4.249952in}}{\pgfqpoint{3.848376in}{4.249952in}}%
\pgfpathcurveto{\pgfqpoint{3.837326in}{4.249952in}}{\pgfqpoint{3.826727in}{4.245562in}}{\pgfqpoint{3.818914in}{4.237748in}}%
\pgfpathcurveto{\pgfqpoint{3.811100in}{4.229934in}}{\pgfqpoint{3.806710in}{4.219335in}}{\pgfqpoint{3.806710in}{4.208285in}}%
\pgfpathcurveto{\pgfqpoint{3.806710in}{4.197235in}}{\pgfqpoint{3.811100in}{4.186636in}}{\pgfqpoint{3.818914in}{4.178822in}}%
\pgfpathcurveto{\pgfqpoint{3.826727in}{4.171009in}}{\pgfqpoint{3.837326in}{4.166619in}}{\pgfqpoint{3.848376in}{4.166619in}}%
\pgfpathlineto{\pgfqpoint{3.848376in}{4.166619in}}%
\pgfpathclose%
\pgfusepath{stroke,fill}%
\end{pgfscope}%
\begin{pgfscope}%
\pgfpathrectangle{\pgfqpoint{2.963410in}{2.920818in}}{\pgfqpoint{2.177280in}{2.201755in}}%
\pgfusepath{clip}%
\pgfsetbuttcap%
\pgfsetroundjoin%
\definecolor{currentfill}{rgb}{1.000000,0.498039,0.054902}%
\pgfsetfillcolor{currentfill}%
\pgfsetlinewidth{0.481800pt}%
\definecolor{currentstroke}{rgb}{1.000000,1.000000,1.000000}%
\pgfsetstrokecolor{currentstroke}%
\pgfsetdash{}{0pt}%
\pgfpathmoveto{\pgfqpoint{3.671151in}{3.827365in}}%
\pgfpathcurveto{\pgfqpoint{3.682201in}{3.827365in}}{\pgfqpoint{3.692800in}{3.831755in}}{\pgfqpoint{3.700613in}{3.839569in}}%
\pgfpathcurveto{\pgfqpoint{3.708427in}{3.847383in}}{\pgfqpoint{3.712817in}{3.857982in}}{\pgfqpoint{3.712817in}{3.869032in}}%
\pgfpathcurveto{\pgfqpoint{3.712817in}{3.880082in}}{\pgfqpoint{3.708427in}{3.890681in}}{\pgfqpoint{3.700613in}{3.898495in}}%
\pgfpathcurveto{\pgfqpoint{3.692800in}{3.906308in}}{\pgfqpoint{3.682201in}{3.910698in}}{\pgfqpoint{3.671151in}{3.910698in}}%
\pgfpathcurveto{\pgfqpoint{3.660101in}{3.910698in}}{\pgfqpoint{3.649501in}{3.906308in}}{\pgfqpoint{3.641688in}{3.898495in}}%
\pgfpathcurveto{\pgfqpoint{3.633874in}{3.890681in}}{\pgfqpoint{3.629484in}{3.880082in}}{\pgfqpoint{3.629484in}{3.869032in}}%
\pgfpathcurveto{\pgfqpoint{3.629484in}{3.857982in}}{\pgfqpoint{3.633874in}{3.847383in}}{\pgfqpoint{3.641688in}{3.839569in}}%
\pgfpathcurveto{\pgfqpoint{3.649501in}{3.831755in}}{\pgfqpoint{3.660101in}{3.827365in}}{\pgfqpoint{3.671151in}{3.827365in}}%
\pgfpathlineto{\pgfqpoint{3.671151in}{3.827365in}}%
\pgfpathclose%
\pgfusepath{stroke,fill}%
\end{pgfscope}%
\begin{pgfscope}%
\pgfpathrectangle{\pgfqpoint{2.963410in}{2.920818in}}{\pgfqpoint{2.177280in}{2.201755in}}%
\pgfusepath{clip}%
\pgfsetbuttcap%
\pgfsetroundjoin%
\definecolor{currentfill}{rgb}{1.000000,0.498039,0.054902}%
\pgfsetfillcolor{currentfill}%
\pgfsetlinewidth{0.481800pt}%
\definecolor{currentstroke}{rgb}{1.000000,1.000000,1.000000}%
\pgfsetstrokecolor{currentstroke}%
\pgfsetdash{}{0pt}%
\pgfpathmoveto{\pgfqpoint{3.553000in}{3.929141in}}%
\pgfpathcurveto{\pgfqpoint{3.564050in}{3.929141in}}{\pgfqpoint{3.574649in}{3.933531in}}{\pgfqpoint{3.582463in}{3.941345in}}%
\pgfpathcurveto{\pgfqpoint{3.590277in}{3.949159in}}{\pgfqpoint{3.594667in}{3.959758in}}{\pgfqpoint{3.594667in}{3.970808in}}%
\pgfpathcurveto{\pgfqpoint{3.594667in}{3.981858in}}{\pgfqpoint{3.590277in}{3.992457in}}{\pgfqpoint{3.582463in}{4.000271in}}%
\pgfpathcurveto{\pgfqpoint{3.574649in}{4.008084in}}{\pgfqpoint{3.564050in}{4.012474in}}{\pgfqpoint{3.553000in}{4.012474in}}%
\pgfpathcurveto{\pgfqpoint{3.541950in}{4.012474in}}{\pgfqpoint{3.531351in}{4.008084in}}{\pgfqpoint{3.523537in}{4.000271in}}%
\pgfpathcurveto{\pgfqpoint{3.515724in}{3.992457in}}{\pgfqpoint{3.511333in}{3.981858in}}{\pgfqpoint{3.511333in}{3.970808in}}%
\pgfpathcurveto{\pgfqpoint{3.511333in}{3.959758in}}{\pgfqpoint{3.515724in}{3.949159in}}{\pgfqpoint{3.523537in}{3.941345in}}%
\pgfpathcurveto{\pgfqpoint{3.531351in}{3.933531in}}{\pgfqpoint{3.541950in}{3.929141in}}{\pgfqpoint{3.553000in}{3.929141in}}%
\pgfpathlineto{\pgfqpoint{3.553000in}{3.929141in}}%
\pgfpathclose%
\pgfusepath{stroke,fill}%
\end{pgfscope}%
\begin{pgfscope}%
\pgfpathrectangle{\pgfqpoint{2.963410in}{2.920818in}}{\pgfqpoint{2.177280in}{2.201755in}}%
\pgfusepath{clip}%
\pgfsetbuttcap%
\pgfsetroundjoin%
\definecolor{currentfill}{rgb}{1.000000,0.498039,0.054902}%
\pgfsetfillcolor{currentfill}%
\pgfsetlinewidth{0.481800pt}%
\definecolor{currentstroke}{rgb}{1.000000,1.000000,1.000000}%
\pgfsetstrokecolor{currentstroke}%
\pgfsetdash{}{0pt}%
\pgfpathmoveto{\pgfqpoint{3.553000in}{3.895216in}}%
\pgfpathcurveto{\pgfqpoint{3.564050in}{3.895216in}}{\pgfqpoint{3.574649in}{3.899606in}}{\pgfqpoint{3.582463in}{3.907420in}}%
\pgfpathcurveto{\pgfqpoint{3.590277in}{3.915233in}}{\pgfqpoint{3.594667in}{3.925832in}}{\pgfqpoint{3.594667in}{3.936882in}}%
\pgfpathcurveto{\pgfqpoint{3.594667in}{3.947933in}}{\pgfqpoint{3.590277in}{3.958532in}}{\pgfqpoint{3.582463in}{3.966345in}}%
\pgfpathcurveto{\pgfqpoint{3.574649in}{3.974159in}}{\pgfqpoint{3.564050in}{3.978549in}}{\pgfqpoint{3.553000in}{3.978549in}}%
\pgfpathcurveto{\pgfqpoint{3.541950in}{3.978549in}}{\pgfqpoint{3.531351in}{3.974159in}}{\pgfqpoint{3.523537in}{3.966345in}}%
\pgfpathcurveto{\pgfqpoint{3.515724in}{3.958532in}}{\pgfqpoint{3.511333in}{3.947933in}}{\pgfqpoint{3.511333in}{3.936882in}}%
\pgfpathcurveto{\pgfqpoint{3.511333in}{3.925832in}}{\pgfqpoint{3.515724in}{3.915233in}}{\pgfqpoint{3.523537in}{3.907420in}}%
\pgfpathcurveto{\pgfqpoint{3.531351in}{3.899606in}}{\pgfqpoint{3.541950in}{3.895216in}}{\pgfqpoint{3.553000in}{3.895216in}}%
\pgfpathlineto{\pgfqpoint{3.553000in}{3.895216in}}%
\pgfpathclose%
\pgfusepath{stroke,fill}%
\end{pgfscope}%
\begin{pgfscope}%
\pgfpathrectangle{\pgfqpoint{2.963410in}{2.920818in}}{\pgfqpoint{2.177280in}{2.201755in}}%
\pgfusepath{clip}%
\pgfsetbuttcap%
\pgfsetroundjoin%
\definecolor{currentfill}{rgb}{1.000000,0.498039,0.054902}%
\pgfsetfillcolor{currentfill}%
\pgfsetlinewidth{0.481800pt}%
\definecolor{currentstroke}{rgb}{1.000000,1.000000,1.000000}%
\pgfsetstrokecolor{currentstroke}%
\pgfsetdash{}{0pt}%
\pgfpathmoveto{\pgfqpoint{3.730226in}{3.963066in}}%
\pgfpathcurveto{\pgfqpoint{3.741276in}{3.963066in}}{\pgfqpoint{3.751875in}{3.967457in}}{\pgfqpoint{3.759689in}{3.975270in}}%
\pgfpathcurveto{\pgfqpoint{3.767502in}{3.983084in}}{\pgfqpoint{3.771893in}{3.993683in}}{\pgfqpoint{3.771893in}{4.004733in}}%
\pgfpathcurveto{\pgfqpoint{3.771893in}{4.015783in}}{\pgfqpoint{3.767502in}{4.026382in}}{\pgfqpoint{3.759689in}{4.034196in}}%
\pgfpathcurveto{\pgfqpoint{3.751875in}{4.042010in}}{\pgfqpoint{3.741276in}{4.046400in}}{\pgfqpoint{3.730226in}{4.046400in}}%
\pgfpathcurveto{\pgfqpoint{3.719176in}{4.046400in}}{\pgfqpoint{3.708577in}{4.042010in}}{\pgfqpoint{3.700763in}{4.034196in}}%
\pgfpathcurveto{\pgfqpoint{3.692950in}{4.026382in}}{\pgfqpoint{3.688559in}{4.015783in}}{\pgfqpoint{3.688559in}{4.004733in}}%
\pgfpathcurveto{\pgfqpoint{3.688559in}{3.993683in}}{\pgfqpoint{3.692950in}{3.983084in}}{\pgfqpoint{3.700763in}{3.975270in}}%
\pgfpathcurveto{\pgfqpoint{3.708577in}{3.967457in}}{\pgfqpoint{3.719176in}{3.963066in}}{\pgfqpoint{3.730226in}{3.963066in}}%
\pgfpathlineto{\pgfqpoint{3.730226in}{3.963066in}}%
\pgfpathclose%
\pgfusepath{stroke,fill}%
\end{pgfscope}%
\begin{pgfscope}%
\pgfpathrectangle{\pgfqpoint{2.963410in}{2.920818in}}{\pgfqpoint{2.177280in}{2.201755in}}%
\pgfusepath{clip}%
\pgfsetbuttcap%
\pgfsetroundjoin%
\definecolor{currentfill}{rgb}{1.000000,0.498039,0.054902}%
\pgfsetfillcolor{currentfill}%
\pgfsetlinewidth{0.481800pt}%
\definecolor{currentstroke}{rgb}{1.000000,1.000000,1.000000}%
\pgfsetstrokecolor{currentstroke}%
\pgfsetdash{}{0pt}%
\pgfpathmoveto{\pgfqpoint{3.730226in}{4.370171in}}%
\pgfpathcurveto{\pgfqpoint{3.741276in}{4.370171in}}{\pgfqpoint{3.751875in}{4.374561in}}{\pgfqpoint{3.759689in}{4.382375in}}%
\pgfpathcurveto{\pgfqpoint{3.767502in}{4.390188in}}{\pgfqpoint{3.771893in}{4.400787in}}{\pgfqpoint{3.771893in}{4.411837in}}%
\pgfpathcurveto{\pgfqpoint{3.771893in}{4.422887in}}{\pgfqpoint{3.767502in}{4.433486in}}{\pgfqpoint{3.759689in}{4.441300in}}%
\pgfpathcurveto{\pgfqpoint{3.751875in}{4.449114in}}{\pgfqpoint{3.741276in}{4.453504in}}{\pgfqpoint{3.730226in}{4.453504in}}%
\pgfpathcurveto{\pgfqpoint{3.719176in}{4.453504in}}{\pgfqpoint{3.708577in}{4.449114in}}{\pgfqpoint{3.700763in}{4.441300in}}%
\pgfpathcurveto{\pgfqpoint{3.692950in}{4.433486in}}{\pgfqpoint{3.688559in}{4.422887in}}{\pgfqpoint{3.688559in}{4.411837in}}%
\pgfpathcurveto{\pgfqpoint{3.688559in}{4.400787in}}{\pgfqpoint{3.692950in}{4.390188in}}{\pgfqpoint{3.700763in}{4.382375in}}%
\pgfpathcurveto{\pgfqpoint{3.708577in}{4.374561in}}{\pgfqpoint{3.719176in}{4.370171in}}{\pgfqpoint{3.730226in}{4.370171in}}%
\pgfpathlineto{\pgfqpoint{3.730226in}{4.370171in}}%
\pgfpathclose%
\pgfusepath{stroke,fill}%
\end{pgfscope}%
\begin{pgfscope}%
\pgfpathrectangle{\pgfqpoint{2.963410in}{2.920818in}}{\pgfqpoint{2.177280in}{2.201755in}}%
\pgfusepath{clip}%
\pgfsetbuttcap%
\pgfsetroundjoin%
\definecolor{currentfill}{rgb}{1.000000,0.498039,0.054902}%
\pgfsetfillcolor{currentfill}%
\pgfsetlinewidth{0.481800pt}%
\definecolor{currentstroke}{rgb}{1.000000,1.000000,1.000000}%
\pgfsetstrokecolor{currentstroke}%
\pgfsetdash{}{0pt}%
\pgfpathmoveto{\pgfqpoint{3.907452in}{4.166619in}}%
\pgfpathcurveto{\pgfqpoint{3.918502in}{4.166619in}}{\pgfqpoint{3.929101in}{4.171009in}}{\pgfqpoint{3.936915in}{4.178822in}}%
\pgfpathcurveto{\pgfqpoint{3.944728in}{4.186636in}}{\pgfqpoint{3.949118in}{4.197235in}}{\pgfqpoint{3.949118in}{4.208285in}}%
\pgfpathcurveto{\pgfqpoint{3.949118in}{4.219335in}}{\pgfqpoint{3.944728in}{4.229934in}}{\pgfqpoint{3.936915in}{4.237748in}}%
\pgfpathcurveto{\pgfqpoint{3.929101in}{4.245562in}}{\pgfqpoint{3.918502in}{4.249952in}}{\pgfqpoint{3.907452in}{4.249952in}}%
\pgfpathcurveto{\pgfqpoint{3.896402in}{4.249952in}}{\pgfqpoint{3.885803in}{4.245562in}}{\pgfqpoint{3.877989in}{4.237748in}}%
\pgfpathcurveto{\pgfqpoint{3.870175in}{4.229934in}}{\pgfqpoint{3.865785in}{4.219335in}}{\pgfqpoint{3.865785in}{4.208285in}}%
\pgfpathcurveto{\pgfqpoint{3.865785in}{4.197235in}}{\pgfqpoint{3.870175in}{4.186636in}}{\pgfqpoint{3.877989in}{4.178822in}}%
\pgfpathcurveto{\pgfqpoint{3.885803in}{4.171009in}}{\pgfqpoint{3.896402in}{4.166619in}}{\pgfqpoint{3.907452in}{4.166619in}}%
\pgfpathlineto{\pgfqpoint{3.907452in}{4.166619in}}%
\pgfpathclose%
\pgfusepath{stroke,fill}%
\end{pgfscope}%
\begin{pgfscope}%
\pgfpathrectangle{\pgfqpoint{2.963410in}{2.920818in}}{\pgfqpoint{2.177280in}{2.201755in}}%
\pgfusepath{clip}%
\pgfsetbuttcap%
\pgfsetroundjoin%
\definecolor{currentfill}{rgb}{1.000000,0.498039,0.054902}%
\pgfsetfillcolor{currentfill}%
\pgfsetlinewidth{0.481800pt}%
\definecolor{currentstroke}{rgb}{1.000000,1.000000,1.000000}%
\pgfsetstrokecolor{currentstroke}%
\pgfsetdash{}{0pt}%
\pgfpathmoveto{\pgfqpoint{4.143753in}{4.166619in}}%
\pgfpathcurveto{\pgfqpoint{4.154803in}{4.166619in}}{\pgfqpoint{4.165402in}{4.171009in}}{\pgfqpoint{4.173216in}{4.178822in}}%
\pgfpathcurveto{\pgfqpoint{4.181029in}{4.186636in}}{\pgfqpoint{4.185419in}{4.197235in}}{\pgfqpoint{4.185419in}{4.208285in}}%
\pgfpathcurveto{\pgfqpoint{4.185419in}{4.219335in}}{\pgfqpoint{4.181029in}{4.229934in}}{\pgfqpoint{4.173216in}{4.237748in}}%
\pgfpathcurveto{\pgfqpoint{4.165402in}{4.245562in}}{\pgfqpoint{4.154803in}{4.249952in}}{\pgfqpoint{4.143753in}{4.249952in}}%
\pgfpathcurveto{\pgfqpoint{4.132703in}{4.249952in}}{\pgfqpoint{4.122104in}{4.245562in}}{\pgfqpoint{4.114290in}{4.237748in}}%
\pgfpathcurveto{\pgfqpoint{4.106476in}{4.229934in}}{\pgfqpoint{4.102086in}{4.219335in}}{\pgfqpoint{4.102086in}{4.208285in}}%
\pgfpathcurveto{\pgfqpoint{4.102086in}{4.197235in}}{\pgfqpoint{4.106476in}{4.186636in}}{\pgfqpoint{4.114290in}{4.178822in}}%
\pgfpathcurveto{\pgfqpoint{4.122104in}{4.171009in}}{\pgfqpoint{4.132703in}{4.166619in}}{\pgfqpoint{4.143753in}{4.166619in}}%
\pgfpathlineto{\pgfqpoint{4.143753in}{4.166619in}}%
\pgfpathclose%
\pgfusepath{stroke,fill}%
\end{pgfscope}%
\begin{pgfscope}%
\pgfpathrectangle{\pgfqpoint{2.963410in}{2.920818in}}{\pgfqpoint{2.177280in}{2.201755in}}%
\pgfusepath{clip}%
\pgfsetbuttcap%
\pgfsetroundjoin%
\definecolor{currentfill}{rgb}{1.000000,0.498039,0.054902}%
\pgfsetfillcolor{currentfill}%
\pgfsetlinewidth{0.481800pt}%
\definecolor{currentstroke}{rgb}{1.000000,1.000000,1.000000}%
\pgfsetstrokecolor{currentstroke}%
\pgfsetdash{}{0pt}%
\pgfpathmoveto{\pgfqpoint{3.966527in}{4.234469in}}%
\pgfpathcurveto{\pgfqpoint{3.977577in}{4.234469in}}{\pgfqpoint{3.988176in}{4.238860in}}{\pgfqpoint{3.995990in}{4.246673in}}%
\pgfpathcurveto{\pgfqpoint{4.003803in}{4.254487in}}{\pgfqpoint{4.008194in}{4.265086in}}{\pgfqpoint{4.008194in}{4.276136in}}%
\pgfpathcurveto{\pgfqpoint{4.008194in}{4.287186in}}{\pgfqpoint{4.003803in}{4.297785in}}{\pgfqpoint{3.995990in}{4.305599in}}%
\pgfpathcurveto{\pgfqpoint{3.988176in}{4.313412in}}{\pgfqpoint{3.977577in}{4.317803in}}{\pgfqpoint{3.966527in}{4.317803in}}%
\pgfpathcurveto{\pgfqpoint{3.955477in}{4.317803in}}{\pgfqpoint{3.944878in}{4.313412in}}{\pgfqpoint{3.937064in}{4.305599in}}%
\pgfpathcurveto{\pgfqpoint{3.929251in}{4.297785in}}{\pgfqpoint{3.924860in}{4.287186in}}{\pgfqpoint{3.924860in}{4.276136in}}%
\pgfpathcurveto{\pgfqpoint{3.924860in}{4.265086in}}{\pgfqpoint{3.929251in}{4.254487in}}{\pgfqpoint{3.937064in}{4.246673in}}%
\pgfpathcurveto{\pgfqpoint{3.944878in}{4.238860in}}{\pgfqpoint{3.955477in}{4.234469in}}{\pgfqpoint{3.966527in}{4.234469in}}%
\pgfpathlineto{\pgfqpoint{3.966527in}{4.234469in}}%
\pgfpathclose%
\pgfusepath{stroke,fill}%
\end{pgfscope}%
\begin{pgfscope}%
\pgfpathrectangle{\pgfqpoint{2.963410in}{2.920818in}}{\pgfqpoint{2.177280in}{2.201755in}}%
\pgfusepath{clip}%
\pgfsetbuttcap%
\pgfsetroundjoin%
\definecolor{currentfill}{rgb}{1.000000,0.498039,0.054902}%
\pgfsetfillcolor{currentfill}%
\pgfsetlinewidth{0.481800pt}%
\definecolor{currentstroke}{rgb}{1.000000,1.000000,1.000000}%
\pgfsetstrokecolor{currentstroke}%
\pgfsetdash{}{0pt}%
\pgfpathmoveto{\pgfqpoint{3.493925in}{4.132693in}}%
\pgfpathcurveto{\pgfqpoint{3.504975in}{4.132693in}}{\pgfqpoint{3.515574in}{4.137083in}}{\pgfqpoint{3.523388in}{4.144897in}}%
\pgfpathcurveto{\pgfqpoint{3.531201in}{4.152711in}}{\pgfqpoint{3.535592in}{4.163310in}}{\pgfqpoint{3.535592in}{4.174360in}}%
\pgfpathcurveto{\pgfqpoint{3.535592in}{4.185410in}}{\pgfqpoint{3.531201in}{4.196009in}}{\pgfqpoint{3.523388in}{4.203823in}}%
\pgfpathcurveto{\pgfqpoint{3.515574in}{4.211636in}}{\pgfqpoint{3.504975in}{4.216027in}}{\pgfqpoint{3.493925in}{4.216027in}}%
\pgfpathcurveto{\pgfqpoint{3.482875in}{4.216027in}}{\pgfqpoint{3.472276in}{4.211636in}}{\pgfqpoint{3.464462in}{4.203823in}}%
\pgfpathcurveto{\pgfqpoint{3.456648in}{4.196009in}}{\pgfqpoint{3.452258in}{4.185410in}}{\pgfqpoint{3.452258in}{4.174360in}}%
\pgfpathcurveto{\pgfqpoint{3.452258in}{4.163310in}}{\pgfqpoint{3.456648in}{4.152711in}}{\pgfqpoint{3.464462in}{4.144897in}}%
\pgfpathcurveto{\pgfqpoint{3.472276in}{4.137083in}}{\pgfqpoint{3.482875in}{4.132693in}}{\pgfqpoint{3.493925in}{4.132693in}}%
\pgfpathlineto{\pgfqpoint{3.493925in}{4.132693in}}%
\pgfpathclose%
\pgfusepath{stroke,fill}%
\end{pgfscope}%
\begin{pgfscope}%
\pgfpathrectangle{\pgfqpoint{2.963410in}{2.920818in}}{\pgfqpoint{2.177280in}{2.201755in}}%
\pgfusepath{clip}%
\pgfsetbuttcap%
\pgfsetroundjoin%
\definecolor{currentfill}{rgb}{1.000000,0.498039,0.054902}%
\pgfsetfillcolor{currentfill}%
\pgfsetlinewidth{0.481800pt}%
\definecolor{currentstroke}{rgb}{1.000000,1.000000,1.000000}%
\pgfsetstrokecolor{currentstroke}%
\pgfsetdash{}{0pt}%
\pgfpathmoveto{\pgfqpoint{3.907452in}{4.030917in}}%
\pgfpathcurveto{\pgfqpoint{3.918502in}{4.030917in}}{\pgfqpoint{3.929101in}{4.035307in}}{\pgfqpoint{3.936915in}{4.043121in}}%
\pgfpathcurveto{\pgfqpoint{3.944728in}{4.050935in}}{\pgfqpoint{3.949118in}{4.061534in}}{\pgfqpoint{3.949118in}{4.072584in}}%
\pgfpathcurveto{\pgfqpoint{3.949118in}{4.083634in}}{\pgfqpoint{3.944728in}{4.094233in}}{\pgfqpoint{3.936915in}{4.102047in}}%
\pgfpathcurveto{\pgfqpoint{3.929101in}{4.109860in}}{\pgfqpoint{3.918502in}{4.114251in}}{\pgfqpoint{3.907452in}{4.114251in}}%
\pgfpathcurveto{\pgfqpoint{3.896402in}{4.114251in}}{\pgfqpoint{3.885803in}{4.109860in}}{\pgfqpoint{3.877989in}{4.102047in}}%
\pgfpathcurveto{\pgfqpoint{3.870175in}{4.094233in}}{\pgfqpoint{3.865785in}{4.083634in}}{\pgfqpoint{3.865785in}{4.072584in}}%
\pgfpathcurveto{\pgfqpoint{3.865785in}{4.061534in}}{\pgfqpoint{3.870175in}{4.050935in}}{\pgfqpoint{3.877989in}{4.043121in}}%
\pgfpathcurveto{\pgfqpoint{3.885803in}{4.035307in}}{\pgfqpoint{3.896402in}{4.030917in}}{\pgfqpoint{3.907452in}{4.030917in}}%
\pgfpathlineto{\pgfqpoint{3.907452in}{4.030917in}}%
\pgfpathclose%
\pgfusepath{stroke,fill}%
\end{pgfscope}%
\begin{pgfscope}%
\pgfpathrectangle{\pgfqpoint{2.963410in}{2.920818in}}{\pgfqpoint{2.177280in}{2.201755in}}%
\pgfusepath{clip}%
\pgfsetbuttcap%
\pgfsetroundjoin%
\definecolor{currentfill}{rgb}{1.000000,0.498039,0.054902}%
\pgfsetfillcolor{currentfill}%
\pgfsetlinewidth{0.481800pt}%
\definecolor{currentstroke}{rgb}{1.000000,1.000000,1.000000}%
\pgfsetstrokecolor{currentstroke}%
\pgfsetdash{}{0pt}%
\pgfpathmoveto{\pgfqpoint{3.612075in}{3.996992in}}%
\pgfpathcurveto{\pgfqpoint{3.623126in}{3.996992in}}{\pgfqpoint{3.633725in}{4.001382in}}{\pgfqpoint{3.641538in}{4.009196in}}%
\pgfpathcurveto{\pgfqpoint{3.649352in}{4.017009in}}{\pgfqpoint{3.653742in}{4.027608in}}{\pgfqpoint{3.653742in}{4.038659in}}%
\pgfpathcurveto{\pgfqpoint{3.653742in}{4.049709in}}{\pgfqpoint{3.649352in}{4.060308in}}{\pgfqpoint{3.641538in}{4.068121in}}%
\pgfpathcurveto{\pgfqpoint{3.633725in}{4.075935in}}{\pgfqpoint{3.623126in}{4.080325in}}{\pgfqpoint{3.612075in}{4.080325in}}%
\pgfpathcurveto{\pgfqpoint{3.601025in}{4.080325in}}{\pgfqpoint{3.590426in}{4.075935in}}{\pgfqpoint{3.582613in}{4.068121in}}%
\pgfpathcurveto{\pgfqpoint{3.574799in}{4.060308in}}{\pgfqpoint{3.570409in}{4.049709in}}{\pgfqpoint{3.570409in}{4.038659in}}%
\pgfpathcurveto{\pgfqpoint{3.570409in}{4.027608in}}{\pgfqpoint{3.574799in}{4.017009in}}{\pgfqpoint{3.582613in}{4.009196in}}%
\pgfpathcurveto{\pgfqpoint{3.590426in}{4.001382in}}{\pgfqpoint{3.601025in}{3.996992in}}{\pgfqpoint{3.612075in}{3.996992in}}%
\pgfpathlineto{\pgfqpoint{3.612075in}{3.996992in}}%
\pgfpathclose%
\pgfusepath{stroke,fill}%
\end{pgfscope}%
\begin{pgfscope}%
\pgfpathrectangle{\pgfqpoint{2.963410in}{2.920818in}}{\pgfqpoint{2.177280in}{2.201755in}}%
\pgfusepath{clip}%
\pgfsetbuttcap%
\pgfsetroundjoin%
\definecolor{currentfill}{rgb}{1.000000,0.498039,0.054902}%
\pgfsetfillcolor{currentfill}%
\pgfsetlinewidth{0.481800pt}%
\definecolor{currentstroke}{rgb}{1.000000,1.000000,1.000000}%
\pgfsetstrokecolor{currentstroke}%
\pgfsetdash{}{0pt}%
\pgfpathmoveto{\pgfqpoint{3.671151in}{4.132693in}}%
\pgfpathcurveto{\pgfqpoint{3.682201in}{4.132693in}}{\pgfqpoint{3.692800in}{4.137083in}}{\pgfqpoint{3.700613in}{4.144897in}}%
\pgfpathcurveto{\pgfqpoint{3.708427in}{4.152711in}}{\pgfqpoint{3.712817in}{4.163310in}}{\pgfqpoint{3.712817in}{4.174360in}}%
\pgfpathcurveto{\pgfqpoint{3.712817in}{4.185410in}}{\pgfqpoint{3.708427in}{4.196009in}}{\pgfqpoint{3.700613in}{4.203823in}}%
\pgfpathcurveto{\pgfqpoint{3.692800in}{4.211636in}}{\pgfqpoint{3.682201in}{4.216027in}}{\pgfqpoint{3.671151in}{4.216027in}}%
\pgfpathcurveto{\pgfqpoint{3.660101in}{4.216027in}}{\pgfqpoint{3.649501in}{4.211636in}}{\pgfqpoint{3.641688in}{4.203823in}}%
\pgfpathcurveto{\pgfqpoint{3.633874in}{4.196009in}}{\pgfqpoint{3.629484in}{4.185410in}}{\pgfqpoint{3.629484in}{4.174360in}}%
\pgfpathcurveto{\pgfqpoint{3.629484in}{4.163310in}}{\pgfqpoint{3.633874in}{4.152711in}}{\pgfqpoint{3.641688in}{4.144897in}}%
\pgfpathcurveto{\pgfqpoint{3.649501in}{4.137083in}}{\pgfqpoint{3.660101in}{4.132693in}}{\pgfqpoint{3.671151in}{4.132693in}}%
\pgfpathlineto{\pgfqpoint{3.671151in}{4.132693in}}%
\pgfpathclose%
\pgfusepath{stroke,fill}%
\end{pgfscope}%
\begin{pgfscope}%
\pgfpathrectangle{\pgfqpoint{2.963410in}{2.920818in}}{\pgfqpoint{2.177280in}{2.201755in}}%
\pgfusepath{clip}%
\pgfsetbuttcap%
\pgfsetroundjoin%
\definecolor{currentfill}{rgb}{1.000000,0.498039,0.054902}%
\pgfsetfillcolor{currentfill}%
\pgfsetlinewidth{0.481800pt}%
\definecolor{currentstroke}{rgb}{1.000000,1.000000,1.000000}%
\pgfsetstrokecolor{currentstroke}%
\pgfsetdash{}{0pt}%
\pgfpathmoveto{\pgfqpoint{3.907452in}{4.200544in}}%
\pgfpathcurveto{\pgfqpoint{3.918502in}{4.200544in}}{\pgfqpoint{3.929101in}{4.204934in}}{\pgfqpoint{3.936915in}{4.212748in}}%
\pgfpathcurveto{\pgfqpoint{3.944728in}{4.220561in}}{\pgfqpoint{3.949118in}{4.231160in}}{\pgfqpoint{3.949118in}{4.242211in}}%
\pgfpathcurveto{\pgfqpoint{3.949118in}{4.253261in}}{\pgfqpoint{3.944728in}{4.263860in}}{\pgfqpoint{3.936915in}{4.271673in}}%
\pgfpathcurveto{\pgfqpoint{3.929101in}{4.279487in}}{\pgfqpoint{3.918502in}{4.283877in}}{\pgfqpoint{3.907452in}{4.283877in}}%
\pgfpathcurveto{\pgfqpoint{3.896402in}{4.283877in}}{\pgfqpoint{3.885803in}{4.279487in}}{\pgfqpoint{3.877989in}{4.271673in}}%
\pgfpathcurveto{\pgfqpoint{3.870175in}{4.263860in}}{\pgfqpoint{3.865785in}{4.253261in}}{\pgfqpoint{3.865785in}{4.242211in}}%
\pgfpathcurveto{\pgfqpoint{3.865785in}{4.231160in}}{\pgfqpoint{3.870175in}{4.220561in}}{\pgfqpoint{3.877989in}{4.212748in}}%
\pgfpathcurveto{\pgfqpoint{3.885803in}{4.204934in}}{\pgfqpoint{3.896402in}{4.200544in}}{\pgfqpoint{3.907452in}{4.200544in}}%
\pgfpathlineto{\pgfqpoint{3.907452in}{4.200544in}}%
\pgfpathclose%
\pgfusepath{stroke,fill}%
\end{pgfscope}%
\begin{pgfscope}%
\pgfpathrectangle{\pgfqpoint{2.963410in}{2.920818in}}{\pgfqpoint{2.177280in}{2.201755in}}%
\pgfusepath{clip}%
\pgfsetbuttcap%
\pgfsetroundjoin%
\definecolor{currentfill}{rgb}{1.000000,0.498039,0.054902}%
\pgfsetfillcolor{currentfill}%
\pgfsetlinewidth{0.481800pt}%
\definecolor{currentstroke}{rgb}{1.000000,1.000000,1.000000}%
\pgfsetstrokecolor{currentstroke}%
\pgfsetdash{}{0pt}%
\pgfpathmoveto{\pgfqpoint{3.671151in}{3.996992in}}%
\pgfpathcurveto{\pgfqpoint{3.682201in}{3.996992in}}{\pgfqpoint{3.692800in}{4.001382in}}{\pgfqpoint{3.700613in}{4.009196in}}%
\pgfpathcurveto{\pgfqpoint{3.708427in}{4.017009in}}{\pgfqpoint{3.712817in}{4.027608in}}{\pgfqpoint{3.712817in}{4.038659in}}%
\pgfpathcurveto{\pgfqpoint{3.712817in}{4.049709in}}{\pgfqpoint{3.708427in}{4.060308in}}{\pgfqpoint{3.700613in}{4.068121in}}%
\pgfpathcurveto{\pgfqpoint{3.692800in}{4.075935in}}{\pgfqpoint{3.682201in}{4.080325in}}{\pgfqpoint{3.671151in}{4.080325in}}%
\pgfpathcurveto{\pgfqpoint{3.660101in}{4.080325in}}{\pgfqpoint{3.649501in}{4.075935in}}{\pgfqpoint{3.641688in}{4.068121in}}%
\pgfpathcurveto{\pgfqpoint{3.633874in}{4.060308in}}{\pgfqpoint{3.629484in}{4.049709in}}{\pgfqpoint{3.629484in}{4.038659in}}%
\pgfpathcurveto{\pgfqpoint{3.629484in}{4.027608in}}{\pgfqpoint{3.633874in}{4.017009in}}{\pgfqpoint{3.641688in}{4.009196in}}%
\pgfpathcurveto{\pgfqpoint{3.649501in}{4.001382in}}{\pgfqpoint{3.660101in}{3.996992in}}{\pgfqpoint{3.671151in}{3.996992in}}%
\pgfpathlineto{\pgfqpoint{3.671151in}{3.996992in}}%
\pgfpathclose%
\pgfusepath{stroke,fill}%
\end{pgfscope}%
\begin{pgfscope}%
\pgfpathrectangle{\pgfqpoint{2.963410in}{2.920818in}}{\pgfqpoint{2.177280in}{2.201755in}}%
\pgfusepath{clip}%
\pgfsetbuttcap%
\pgfsetroundjoin%
\definecolor{currentfill}{rgb}{1.000000,0.498039,0.054902}%
\pgfsetfillcolor{currentfill}%
\pgfsetlinewidth{0.481800pt}%
\definecolor{currentstroke}{rgb}{1.000000,1.000000,1.000000}%
\pgfsetstrokecolor{currentstroke}%
\pgfsetdash{}{0pt}%
\pgfpathmoveto{\pgfqpoint{3.493925in}{3.759514in}}%
\pgfpathcurveto{\pgfqpoint{3.504975in}{3.759514in}}{\pgfqpoint{3.515574in}{3.763905in}}{\pgfqpoint{3.523388in}{3.771718in}}%
\pgfpathcurveto{\pgfqpoint{3.531201in}{3.779532in}}{\pgfqpoint{3.535592in}{3.790131in}}{\pgfqpoint{3.535592in}{3.801181in}}%
\pgfpathcurveto{\pgfqpoint{3.535592in}{3.812231in}}{\pgfqpoint{3.531201in}{3.822830in}}{\pgfqpoint{3.523388in}{3.830644in}}%
\pgfpathcurveto{\pgfqpoint{3.515574in}{3.838457in}}{\pgfqpoint{3.504975in}{3.842848in}}{\pgfqpoint{3.493925in}{3.842848in}}%
\pgfpathcurveto{\pgfqpoint{3.482875in}{3.842848in}}{\pgfqpoint{3.472276in}{3.838457in}}{\pgfqpoint{3.464462in}{3.830644in}}%
\pgfpathcurveto{\pgfqpoint{3.456648in}{3.822830in}}{\pgfqpoint{3.452258in}{3.812231in}}{\pgfqpoint{3.452258in}{3.801181in}}%
\pgfpathcurveto{\pgfqpoint{3.452258in}{3.790131in}}{\pgfqpoint{3.456648in}{3.779532in}}{\pgfqpoint{3.464462in}{3.771718in}}%
\pgfpathcurveto{\pgfqpoint{3.472276in}{3.763905in}}{\pgfqpoint{3.482875in}{3.759514in}}{\pgfqpoint{3.493925in}{3.759514in}}%
\pgfpathlineto{\pgfqpoint{3.493925in}{3.759514in}}%
\pgfpathclose%
\pgfusepath{stroke,fill}%
\end{pgfscope}%
\begin{pgfscope}%
\pgfpathrectangle{\pgfqpoint{2.963410in}{2.920818in}}{\pgfqpoint{2.177280in}{2.201755in}}%
\pgfusepath{clip}%
\pgfsetbuttcap%
\pgfsetroundjoin%
\definecolor{currentfill}{rgb}{1.000000,0.498039,0.054902}%
\pgfsetfillcolor{currentfill}%
\pgfsetlinewidth{0.481800pt}%
\definecolor{currentstroke}{rgb}{1.000000,1.000000,1.000000}%
\pgfsetstrokecolor{currentstroke}%
\pgfsetdash{}{0pt}%
\pgfpathmoveto{\pgfqpoint{3.730226in}{4.064843in}}%
\pgfpathcurveto{\pgfqpoint{3.741276in}{4.064843in}}{\pgfqpoint{3.751875in}{4.069233in}}{\pgfqpoint{3.759689in}{4.077046in}}%
\pgfpathcurveto{\pgfqpoint{3.767502in}{4.084860in}}{\pgfqpoint{3.771893in}{4.095459in}}{\pgfqpoint{3.771893in}{4.106509in}}%
\pgfpathcurveto{\pgfqpoint{3.771893in}{4.117559in}}{\pgfqpoint{3.767502in}{4.128158in}}{\pgfqpoint{3.759689in}{4.135972in}}%
\pgfpathcurveto{\pgfqpoint{3.751875in}{4.143786in}}{\pgfqpoint{3.741276in}{4.148176in}}{\pgfqpoint{3.730226in}{4.148176in}}%
\pgfpathcurveto{\pgfqpoint{3.719176in}{4.148176in}}{\pgfqpoint{3.708577in}{4.143786in}}{\pgfqpoint{3.700763in}{4.135972in}}%
\pgfpathcurveto{\pgfqpoint{3.692950in}{4.128158in}}{\pgfqpoint{3.688559in}{4.117559in}}{\pgfqpoint{3.688559in}{4.106509in}}%
\pgfpathcurveto{\pgfqpoint{3.688559in}{4.095459in}}{\pgfqpoint{3.692950in}{4.084860in}}{\pgfqpoint{3.700763in}{4.077046in}}%
\pgfpathcurveto{\pgfqpoint{3.708577in}{4.069233in}}{\pgfqpoint{3.719176in}{4.064843in}}{\pgfqpoint{3.730226in}{4.064843in}}%
\pgfpathlineto{\pgfqpoint{3.730226in}{4.064843in}}%
\pgfpathclose%
\pgfusepath{stroke,fill}%
\end{pgfscope}%
\begin{pgfscope}%
\pgfpathrectangle{\pgfqpoint{2.963410in}{2.920818in}}{\pgfqpoint{2.177280in}{2.201755in}}%
\pgfusepath{clip}%
\pgfsetbuttcap%
\pgfsetroundjoin%
\definecolor{currentfill}{rgb}{1.000000,0.498039,0.054902}%
\pgfsetfillcolor{currentfill}%
\pgfsetlinewidth{0.481800pt}%
\definecolor{currentstroke}{rgb}{1.000000,1.000000,1.000000}%
\pgfsetstrokecolor{currentstroke}%
\pgfsetdash{}{0pt}%
\pgfpathmoveto{\pgfqpoint{3.907452in}{4.064843in}}%
\pgfpathcurveto{\pgfqpoint{3.918502in}{4.064843in}}{\pgfqpoint{3.929101in}{4.069233in}}{\pgfqpoint{3.936915in}{4.077046in}}%
\pgfpathcurveto{\pgfqpoint{3.944728in}{4.084860in}}{\pgfqpoint{3.949118in}{4.095459in}}{\pgfqpoint{3.949118in}{4.106509in}}%
\pgfpathcurveto{\pgfqpoint{3.949118in}{4.117559in}}{\pgfqpoint{3.944728in}{4.128158in}}{\pgfqpoint{3.936915in}{4.135972in}}%
\pgfpathcurveto{\pgfqpoint{3.929101in}{4.143786in}}{\pgfqpoint{3.918502in}{4.148176in}}{\pgfqpoint{3.907452in}{4.148176in}}%
\pgfpathcurveto{\pgfqpoint{3.896402in}{4.148176in}}{\pgfqpoint{3.885803in}{4.143786in}}{\pgfqpoint{3.877989in}{4.135972in}}%
\pgfpathcurveto{\pgfqpoint{3.870175in}{4.128158in}}{\pgfqpoint{3.865785in}{4.117559in}}{\pgfqpoint{3.865785in}{4.106509in}}%
\pgfpathcurveto{\pgfqpoint{3.865785in}{4.095459in}}{\pgfqpoint{3.870175in}{4.084860in}}{\pgfqpoint{3.877989in}{4.077046in}}%
\pgfpathcurveto{\pgfqpoint{3.885803in}{4.069233in}}{\pgfqpoint{3.896402in}{4.064843in}}{\pgfqpoint{3.907452in}{4.064843in}}%
\pgfpathlineto{\pgfqpoint{3.907452in}{4.064843in}}%
\pgfpathclose%
\pgfusepath{stroke,fill}%
\end{pgfscope}%
\begin{pgfscope}%
\pgfpathrectangle{\pgfqpoint{2.963410in}{2.920818in}}{\pgfqpoint{2.177280in}{2.201755in}}%
\pgfusepath{clip}%
\pgfsetbuttcap%
\pgfsetroundjoin%
\definecolor{currentfill}{rgb}{1.000000,0.498039,0.054902}%
\pgfsetfillcolor{currentfill}%
\pgfsetlinewidth{0.481800pt}%
\definecolor{currentstroke}{rgb}{1.000000,1.000000,1.000000}%
\pgfsetstrokecolor{currentstroke}%
\pgfsetdash{}{0pt}%
\pgfpathmoveto{\pgfqpoint{3.848376in}{4.064843in}}%
\pgfpathcurveto{\pgfqpoint{3.859427in}{4.064843in}}{\pgfqpoint{3.870026in}{4.069233in}}{\pgfqpoint{3.877839in}{4.077046in}}%
\pgfpathcurveto{\pgfqpoint{3.885653in}{4.084860in}}{\pgfqpoint{3.890043in}{4.095459in}}{\pgfqpoint{3.890043in}{4.106509in}}%
\pgfpathcurveto{\pgfqpoint{3.890043in}{4.117559in}}{\pgfqpoint{3.885653in}{4.128158in}}{\pgfqpoint{3.877839in}{4.135972in}}%
\pgfpathcurveto{\pgfqpoint{3.870026in}{4.143786in}}{\pgfqpoint{3.859427in}{4.148176in}}{\pgfqpoint{3.848376in}{4.148176in}}%
\pgfpathcurveto{\pgfqpoint{3.837326in}{4.148176in}}{\pgfqpoint{3.826727in}{4.143786in}}{\pgfqpoint{3.818914in}{4.135972in}}%
\pgfpathcurveto{\pgfqpoint{3.811100in}{4.128158in}}{\pgfqpoint{3.806710in}{4.117559in}}{\pgfqpoint{3.806710in}{4.106509in}}%
\pgfpathcurveto{\pgfqpoint{3.806710in}{4.095459in}}{\pgfqpoint{3.811100in}{4.084860in}}{\pgfqpoint{3.818914in}{4.077046in}}%
\pgfpathcurveto{\pgfqpoint{3.826727in}{4.069233in}}{\pgfqpoint{3.837326in}{4.064843in}}{\pgfqpoint{3.848376in}{4.064843in}}%
\pgfpathlineto{\pgfqpoint{3.848376in}{4.064843in}}%
\pgfpathclose%
\pgfusepath{stroke,fill}%
\end{pgfscope}%
\begin{pgfscope}%
\pgfpathrectangle{\pgfqpoint{2.963410in}{2.920818in}}{\pgfqpoint{2.177280in}{2.201755in}}%
\pgfusepath{clip}%
\pgfsetbuttcap%
\pgfsetroundjoin%
\definecolor{currentfill}{rgb}{1.000000,0.498039,0.054902}%
\pgfsetfillcolor{currentfill}%
\pgfsetlinewidth{0.481800pt}%
\definecolor{currentstroke}{rgb}{1.000000,1.000000,1.000000}%
\pgfsetstrokecolor{currentstroke}%
\pgfsetdash{}{0pt}%
\pgfpathmoveto{\pgfqpoint{3.848376in}{4.098768in}}%
\pgfpathcurveto{\pgfqpoint{3.859427in}{4.098768in}}{\pgfqpoint{3.870026in}{4.103158in}}{\pgfqpoint{3.877839in}{4.110972in}}%
\pgfpathcurveto{\pgfqpoint{3.885653in}{4.118785in}}{\pgfqpoint{3.890043in}{4.129384in}}{\pgfqpoint{3.890043in}{4.140435in}}%
\pgfpathcurveto{\pgfqpoint{3.890043in}{4.151485in}}{\pgfqpoint{3.885653in}{4.162084in}}{\pgfqpoint{3.877839in}{4.169897in}}%
\pgfpathcurveto{\pgfqpoint{3.870026in}{4.177711in}}{\pgfqpoint{3.859427in}{4.182101in}}{\pgfqpoint{3.848376in}{4.182101in}}%
\pgfpathcurveto{\pgfqpoint{3.837326in}{4.182101in}}{\pgfqpoint{3.826727in}{4.177711in}}{\pgfqpoint{3.818914in}{4.169897in}}%
\pgfpathcurveto{\pgfqpoint{3.811100in}{4.162084in}}{\pgfqpoint{3.806710in}{4.151485in}}{\pgfqpoint{3.806710in}{4.140435in}}%
\pgfpathcurveto{\pgfqpoint{3.806710in}{4.129384in}}{\pgfqpoint{3.811100in}{4.118785in}}{\pgfqpoint{3.818914in}{4.110972in}}%
\pgfpathcurveto{\pgfqpoint{3.826727in}{4.103158in}}{\pgfqpoint{3.837326in}{4.098768in}}{\pgfqpoint{3.848376in}{4.098768in}}%
\pgfpathlineto{\pgfqpoint{3.848376in}{4.098768in}}%
\pgfpathclose%
\pgfusepath{stroke,fill}%
\end{pgfscope}%
\begin{pgfscope}%
\pgfpathrectangle{\pgfqpoint{2.963410in}{2.920818in}}{\pgfqpoint{2.177280in}{2.201755in}}%
\pgfusepath{clip}%
\pgfsetbuttcap%
\pgfsetroundjoin%
\definecolor{currentfill}{rgb}{1.000000,0.498039,0.054902}%
\pgfsetfillcolor{currentfill}%
\pgfsetlinewidth{0.481800pt}%
\definecolor{currentstroke}{rgb}{1.000000,1.000000,1.000000}%
\pgfsetstrokecolor{currentstroke}%
\pgfsetdash{}{0pt}%
\pgfpathmoveto{\pgfqpoint{3.612075in}{3.657738in}}%
\pgfpathcurveto{\pgfqpoint{3.623126in}{3.657738in}}{\pgfqpoint{3.633725in}{3.662129in}}{\pgfqpoint{3.641538in}{3.669942in}}%
\pgfpathcurveto{\pgfqpoint{3.649352in}{3.677756in}}{\pgfqpoint{3.653742in}{3.688355in}}{\pgfqpoint{3.653742in}{3.699405in}}%
\pgfpathcurveto{\pgfqpoint{3.653742in}{3.710455in}}{\pgfqpoint{3.649352in}{3.721054in}}{\pgfqpoint{3.641538in}{3.728868in}}%
\pgfpathcurveto{\pgfqpoint{3.633725in}{3.736681in}}{\pgfqpoint{3.623126in}{3.741072in}}{\pgfqpoint{3.612075in}{3.741072in}}%
\pgfpathcurveto{\pgfqpoint{3.601025in}{3.741072in}}{\pgfqpoint{3.590426in}{3.736681in}}{\pgfqpoint{3.582613in}{3.728868in}}%
\pgfpathcurveto{\pgfqpoint{3.574799in}{3.721054in}}{\pgfqpoint{3.570409in}{3.710455in}}{\pgfqpoint{3.570409in}{3.699405in}}%
\pgfpathcurveto{\pgfqpoint{3.570409in}{3.688355in}}{\pgfqpoint{3.574799in}{3.677756in}}{\pgfqpoint{3.582613in}{3.669942in}}%
\pgfpathcurveto{\pgfqpoint{3.590426in}{3.662129in}}{\pgfqpoint{3.601025in}{3.657738in}}{\pgfqpoint{3.612075in}{3.657738in}}%
\pgfpathlineto{\pgfqpoint{3.612075in}{3.657738in}}%
\pgfpathclose%
\pgfusepath{stroke,fill}%
\end{pgfscope}%
\begin{pgfscope}%
\pgfpathrectangle{\pgfqpoint{2.963410in}{2.920818in}}{\pgfqpoint{2.177280in}{2.201755in}}%
\pgfusepath{clip}%
\pgfsetbuttcap%
\pgfsetroundjoin%
\definecolor{currentfill}{rgb}{1.000000,0.498039,0.054902}%
\pgfsetfillcolor{currentfill}%
\pgfsetlinewidth{0.481800pt}%
\definecolor{currentstroke}{rgb}{1.000000,1.000000,1.000000}%
\pgfsetstrokecolor{currentstroke}%
\pgfsetdash{}{0pt}%
\pgfpathmoveto{\pgfqpoint{3.789301in}{4.030917in}}%
\pgfpathcurveto{\pgfqpoint{3.800351in}{4.030917in}}{\pgfqpoint{3.810950in}{4.035307in}}{\pgfqpoint{3.818764in}{4.043121in}}%
\pgfpathcurveto{\pgfqpoint{3.826578in}{4.050935in}}{\pgfqpoint{3.830968in}{4.061534in}}{\pgfqpoint{3.830968in}{4.072584in}}%
\pgfpathcurveto{\pgfqpoint{3.830968in}{4.083634in}}{\pgfqpoint{3.826578in}{4.094233in}}{\pgfqpoint{3.818764in}{4.102047in}}%
\pgfpathcurveto{\pgfqpoint{3.810950in}{4.109860in}}{\pgfqpoint{3.800351in}{4.114251in}}{\pgfqpoint{3.789301in}{4.114251in}}%
\pgfpathcurveto{\pgfqpoint{3.778251in}{4.114251in}}{\pgfqpoint{3.767652in}{4.109860in}}{\pgfqpoint{3.759838in}{4.102047in}}%
\pgfpathcurveto{\pgfqpoint{3.752025in}{4.094233in}}{\pgfqpoint{3.747635in}{4.083634in}}{\pgfqpoint{3.747635in}{4.072584in}}%
\pgfpathcurveto{\pgfqpoint{3.747635in}{4.061534in}}{\pgfqpoint{3.752025in}{4.050935in}}{\pgfqpoint{3.759838in}{4.043121in}}%
\pgfpathcurveto{\pgfqpoint{3.767652in}{4.035307in}}{\pgfqpoint{3.778251in}{4.030917in}}{\pgfqpoint{3.789301in}{4.030917in}}%
\pgfpathlineto{\pgfqpoint{3.789301in}{4.030917in}}%
\pgfpathclose%
\pgfusepath{stroke,fill}%
\end{pgfscope}%
\begin{pgfscope}%
\pgfpathrectangle{\pgfqpoint{2.963410in}{2.920818in}}{\pgfqpoint{2.177280in}{2.201755in}}%
\pgfusepath{clip}%
\pgfsetbuttcap%
\pgfsetroundjoin%
\definecolor{currentfill}{rgb}{0.172549,0.627451,0.172549}%
\pgfsetfillcolor{currentfill}%
\pgfsetlinewidth{0.481800pt}%
\definecolor{currentstroke}{rgb}{1.000000,1.000000,1.000000}%
\pgfsetstrokecolor{currentstroke}%
\pgfsetdash{}{0pt}%
\pgfpathmoveto{\pgfqpoint{4.084678in}{4.675499in}}%
\pgfpathcurveto{\pgfqpoint{4.095728in}{4.675499in}}{\pgfqpoint{4.106327in}{4.679889in}}{\pgfqpoint{4.114140in}{4.687703in}}%
\pgfpathcurveto{\pgfqpoint{4.121954in}{4.695516in}}{\pgfqpoint{4.126344in}{4.706115in}}{\pgfqpoint{4.126344in}{4.717165in}}%
\pgfpathcurveto{\pgfqpoint{4.126344in}{4.728216in}}{\pgfqpoint{4.121954in}{4.738815in}}{\pgfqpoint{4.114140in}{4.746628in}}%
\pgfpathcurveto{\pgfqpoint{4.106327in}{4.754442in}}{\pgfqpoint{4.095728in}{4.758832in}}{\pgfqpoint{4.084678in}{4.758832in}}%
\pgfpathcurveto{\pgfqpoint{4.073627in}{4.758832in}}{\pgfqpoint{4.063028in}{4.754442in}}{\pgfqpoint{4.055215in}{4.746628in}}%
\pgfpathcurveto{\pgfqpoint{4.047401in}{4.738815in}}{\pgfqpoint{4.043011in}{4.728216in}}{\pgfqpoint{4.043011in}{4.717165in}}%
\pgfpathcurveto{\pgfqpoint{4.043011in}{4.706115in}}{\pgfqpoint{4.047401in}{4.695516in}}{\pgfqpoint{4.055215in}{4.687703in}}%
\pgfpathcurveto{\pgfqpoint{4.063028in}{4.679889in}}{\pgfqpoint{4.073627in}{4.675499in}}{\pgfqpoint{4.084678in}{4.675499in}}%
\pgfpathlineto{\pgfqpoint{4.084678in}{4.675499in}}%
\pgfpathclose%
\pgfusepath{stroke,fill}%
\end{pgfscope}%
\begin{pgfscope}%
\pgfpathrectangle{\pgfqpoint{2.963410in}{2.920818in}}{\pgfqpoint{2.177280in}{2.201755in}}%
\pgfusepath{clip}%
\pgfsetbuttcap%
\pgfsetroundjoin%
\definecolor{currentfill}{rgb}{0.172549,0.627451,0.172549}%
\pgfsetfillcolor{currentfill}%
\pgfsetlinewidth{0.481800pt}%
\definecolor{currentstroke}{rgb}{1.000000,1.000000,1.000000}%
\pgfsetstrokecolor{currentstroke}%
\pgfsetdash{}{0pt}%
\pgfpathmoveto{\pgfqpoint{3.730226in}{4.370171in}}%
\pgfpathcurveto{\pgfqpoint{3.741276in}{4.370171in}}{\pgfqpoint{3.751875in}{4.374561in}}{\pgfqpoint{3.759689in}{4.382375in}}%
\pgfpathcurveto{\pgfqpoint{3.767502in}{4.390188in}}{\pgfqpoint{3.771893in}{4.400787in}}{\pgfqpoint{3.771893in}{4.411837in}}%
\pgfpathcurveto{\pgfqpoint{3.771893in}{4.422887in}}{\pgfqpoint{3.767502in}{4.433486in}}{\pgfqpoint{3.759689in}{4.441300in}}%
\pgfpathcurveto{\pgfqpoint{3.751875in}{4.449114in}}{\pgfqpoint{3.741276in}{4.453504in}}{\pgfqpoint{3.730226in}{4.453504in}}%
\pgfpathcurveto{\pgfqpoint{3.719176in}{4.453504in}}{\pgfqpoint{3.708577in}{4.449114in}}{\pgfqpoint{3.700763in}{4.441300in}}%
\pgfpathcurveto{\pgfqpoint{3.692950in}{4.433486in}}{\pgfqpoint{3.688559in}{4.422887in}}{\pgfqpoint{3.688559in}{4.411837in}}%
\pgfpathcurveto{\pgfqpoint{3.688559in}{4.400787in}}{\pgfqpoint{3.692950in}{4.390188in}}{\pgfqpoint{3.700763in}{4.382375in}}%
\pgfpathcurveto{\pgfqpoint{3.708577in}{4.374561in}}{\pgfqpoint{3.719176in}{4.370171in}}{\pgfqpoint{3.730226in}{4.370171in}}%
\pgfpathlineto{\pgfqpoint{3.730226in}{4.370171in}}%
\pgfpathclose%
\pgfusepath{stroke,fill}%
\end{pgfscope}%
\begin{pgfscope}%
\pgfpathrectangle{\pgfqpoint{2.963410in}{2.920818in}}{\pgfqpoint{2.177280in}{2.201755in}}%
\pgfusepath{clip}%
\pgfsetbuttcap%
\pgfsetroundjoin%
\definecolor{currentfill}{rgb}{0.172549,0.627451,0.172549}%
\pgfsetfillcolor{currentfill}%
\pgfsetlinewidth{0.481800pt}%
\definecolor{currentstroke}{rgb}{1.000000,1.000000,1.000000}%
\pgfsetstrokecolor{currentstroke}%
\pgfsetdash{}{0pt}%
\pgfpathmoveto{\pgfqpoint{3.907452in}{4.641573in}}%
\pgfpathcurveto{\pgfqpoint{3.918502in}{4.641573in}}{\pgfqpoint{3.929101in}{4.645964in}}{\pgfqpoint{3.936915in}{4.653777in}}%
\pgfpathcurveto{\pgfqpoint{3.944728in}{4.661591in}}{\pgfqpoint{3.949118in}{4.672190in}}{\pgfqpoint{3.949118in}{4.683240in}}%
\pgfpathcurveto{\pgfqpoint{3.949118in}{4.694290in}}{\pgfqpoint{3.944728in}{4.704889in}}{\pgfqpoint{3.936915in}{4.712703in}}%
\pgfpathcurveto{\pgfqpoint{3.929101in}{4.720517in}}{\pgfqpoint{3.918502in}{4.724907in}}{\pgfqpoint{3.907452in}{4.724907in}}%
\pgfpathcurveto{\pgfqpoint{3.896402in}{4.724907in}}{\pgfqpoint{3.885803in}{4.720517in}}{\pgfqpoint{3.877989in}{4.712703in}}%
\pgfpathcurveto{\pgfqpoint{3.870175in}{4.704889in}}{\pgfqpoint{3.865785in}{4.694290in}}{\pgfqpoint{3.865785in}{4.683240in}}%
\pgfpathcurveto{\pgfqpoint{3.865785in}{4.672190in}}{\pgfqpoint{3.870175in}{4.661591in}}{\pgfqpoint{3.877989in}{4.653777in}}%
\pgfpathcurveto{\pgfqpoint{3.885803in}{4.645964in}}{\pgfqpoint{3.896402in}{4.641573in}}{\pgfqpoint{3.907452in}{4.641573in}}%
\pgfpathlineto{\pgfqpoint{3.907452in}{4.641573in}}%
\pgfpathclose%
\pgfusepath{stroke,fill}%
\end{pgfscope}%
\begin{pgfscope}%
\pgfpathrectangle{\pgfqpoint{2.963410in}{2.920818in}}{\pgfqpoint{2.177280in}{2.201755in}}%
\pgfusepath{clip}%
\pgfsetbuttcap%
\pgfsetroundjoin%
\definecolor{currentfill}{rgb}{0.172549,0.627451,0.172549}%
\pgfsetfillcolor{currentfill}%
\pgfsetlinewidth{0.481800pt}%
\definecolor{currentstroke}{rgb}{1.000000,1.000000,1.000000}%
\pgfsetstrokecolor{currentstroke}%
\pgfsetdash{}{0pt}%
\pgfpathmoveto{\pgfqpoint{3.848376in}{4.539797in}}%
\pgfpathcurveto{\pgfqpoint{3.859427in}{4.539797in}}{\pgfqpoint{3.870026in}{4.544188in}}{\pgfqpoint{3.877839in}{4.552001in}}%
\pgfpathcurveto{\pgfqpoint{3.885653in}{4.559815in}}{\pgfqpoint{3.890043in}{4.570414in}}{\pgfqpoint{3.890043in}{4.581464in}}%
\pgfpathcurveto{\pgfqpoint{3.890043in}{4.592514in}}{\pgfqpoint{3.885653in}{4.603113in}}{\pgfqpoint{3.877839in}{4.610927in}}%
\pgfpathcurveto{\pgfqpoint{3.870026in}{4.618740in}}{\pgfqpoint{3.859427in}{4.623131in}}{\pgfqpoint{3.848376in}{4.623131in}}%
\pgfpathcurveto{\pgfqpoint{3.837326in}{4.623131in}}{\pgfqpoint{3.826727in}{4.618740in}}{\pgfqpoint{3.818914in}{4.610927in}}%
\pgfpathcurveto{\pgfqpoint{3.811100in}{4.603113in}}{\pgfqpoint{3.806710in}{4.592514in}}{\pgfqpoint{3.806710in}{4.581464in}}%
\pgfpathcurveto{\pgfqpoint{3.806710in}{4.570414in}}{\pgfqpoint{3.811100in}{4.559815in}}{\pgfqpoint{3.818914in}{4.552001in}}%
\pgfpathcurveto{\pgfqpoint{3.826727in}{4.544188in}}{\pgfqpoint{3.837326in}{4.539797in}}{\pgfqpoint{3.848376in}{4.539797in}}%
\pgfpathlineto{\pgfqpoint{3.848376in}{4.539797in}}%
\pgfpathclose%
\pgfusepath{stroke,fill}%
\end{pgfscope}%
\begin{pgfscope}%
\pgfpathrectangle{\pgfqpoint{2.963410in}{2.920818in}}{\pgfqpoint{2.177280in}{2.201755in}}%
\pgfusepath{clip}%
\pgfsetbuttcap%
\pgfsetroundjoin%
\definecolor{currentfill}{rgb}{0.172549,0.627451,0.172549}%
\pgfsetfillcolor{currentfill}%
\pgfsetlinewidth{0.481800pt}%
\definecolor{currentstroke}{rgb}{1.000000,1.000000,1.000000}%
\pgfsetstrokecolor{currentstroke}%
\pgfsetdash{}{0pt}%
\pgfpathmoveto{\pgfqpoint{3.907452in}{4.607648in}}%
\pgfpathcurveto{\pgfqpoint{3.918502in}{4.607648in}}{\pgfqpoint{3.929101in}{4.612038in}}{\pgfqpoint{3.936915in}{4.619852in}}%
\pgfpathcurveto{\pgfqpoint{3.944728in}{4.627666in}}{\pgfqpoint{3.949118in}{4.638265in}}{\pgfqpoint{3.949118in}{4.649315in}}%
\pgfpathcurveto{\pgfqpoint{3.949118in}{4.660365in}}{\pgfqpoint{3.944728in}{4.670964in}}{\pgfqpoint{3.936915in}{4.678778in}}%
\pgfpathcurveto{\pgfqpoint{3.929101in}{4.686591in}}{\pgfqpoint{3.918502in}{4.690981in}}{\pgfqpoint{3.907452in}{4.690981in}}%
\pgfpathcurveto{\pgfqpoint{3.896402in}{4.690981in}}{\pgfqpoint{3.885803in}{4.686591in}}{\pgfqpoint{3.877989in}{4.678778in}}%
\pgfpathcurveto{\pgfqpoint{3.870175in}{4.670964in}}{\pgfqpoint{3.865785in}{4.660365in}}{\pgfqpoint{3.865785in}{4.649315in}}%
\pgfpathcurveto{\pgfqpoint{3.865785in}{4.638265in}}{\pgfqpoint{3.870175in}{4.627666in}}{\pgfqpoint{3.877989in}{4.619852in}}%
\pgfpathcurveto{\pgfqpoint{3.885803in}{4.612038in}}{\pgfqpoint{3.896402in}{4.607648in}}{\pgfqpoint{3.907452in}{4.607648in}}%
\pgfpathlineto{\pgfqpoint{3.907452in}{4.607648in}}%
\pgfpathclose%
\pgfusepath{stroke,fill}%
\end{pgfscope}%
\begin{pgfscope}%
\pgfpathrectangle{\pgfqpoint{2.963410in}{2.920818in}}{\pgfqpoint{2.177280in}{2.201755in}}%
\pgfusepath{clip}%
\pgfsetbuttcap%
\pgfsetroundjoin%
\definecolor{currentfill}{rgb}{0.172549,0.627451,0.172549}%
\pgfsetfillcolor{currentfill}%
\pgfsetlinewidth{0.481800pt}%
\definecolor{currentstroke}{rgb}{1.000000,1.000000,1.000000}%
\pgfsetstrokecolor{currentstroke}%
\pgfsetdash{}{0pt}%
\pgfpathmoveto{\pgfqpoint{3.907452in}{4.879051in}}%
\pgfpathcurveto{\pgfqpoint{3.918502in}{4.879051in}}{\pgfqpoint{3.929101in}{4.883441in}}{\pgfqpoint{3.936915in}{4.891255in}}%
\pgfpathcurveto{\pgfqpoint{3.944728in}{4.899068in}}{\pgfqpoint{3.949118in}{4.909667in}}{\pgfqpoint{3.949118in}{4.920718in}}%
\pgfpathcurveto{\pgfqpoint{3.949118in}{4.931768in}}{\pgfqpoint{3.944728in}{4.942367in}}{\pgfqpoint{3.936915in}{4.950180in}}%
\pgfpathcurveto{\pgfqpoint{3.929101in}{4.957994in}}{\pgfqpoint{3.918502in}{4.962384in}}{\pgfqpoint{3.907452in}{4.962384in}}%
\pgfpathcurveto{\pgfqpoint{3.896402in}{4.962384in}}{\pgfqpoint{3.885803in}{4.957994in}}{\pgfqpoint{3.877989in}{4.950180in}}%
\pgfpathcurveto{\pgfqpoint{3.870175in}{4.942367in}}{\pgfqpoint{3.865785in}{4.931768in}}{\pgfqpoint{3.865785in}{4.920718in}}%
\pgfpathcurveto{\pgfqpoint{3.865785in}{4.909667in}}{\pgfqpoint{3.870175in}{4.899068in}}{\pgfqpoint{3.877989in}{4.891255in}}%
\pgfpathcurveto{\pgfqpoint{3.885803in}{4.883441in}}{\pgfqpoint{3.896402in}{4.879051in}}{\pgfqpoint{3.907452in}{4.879051in}}%
\pgfpathlineto{\pgfqpoint{3.907452in}{4.879051in}}%
\pgfpathclose%
\pgfusepath{stroke,fill}%
\end{pgfscope}%
\begin{pgfscope}%
\pgfpathrectangle{\pgfqpoint{2.963410in}{2.920818in}}{\pgfqpoint{2.177280in}{2.201755in}}%
\pgfusepath{clip}%
\pgfsetbuttcap%
\pgfsetroundjoin%
\definecolor{currentfill}{rgb}{0.172549,0.627451,0.172549}%
\pgfsetfillcolor{currentfill}%
\pgfsetlinewidth{0.481800pt}%
\definecolor{currentstroke}{rgb}{1.000000,1.000000,1.000000}%
\pgfsetstrokecolor{currentstroke}%
\pgfsetdash{}{0pt}%
\pgfpathmoveto{\pgfqpoint{3.612075in}{4.166619in}}%
\pgfpathcurveto{\pgfqpoint{3.623126in}{4.166619in}}{\pgfqpoint{3.633725in}{4.171009in}}{\pgfqpoint{3.641538in}{4.178822in}}%
\pgfpathcurveto{\pgfqpoint{3.649352in}{4.186636in}}{\pgfqpoint{3.653742in}{4.197235in}}{\pgfqpoint{3.653742in}{4.208285in}}%
\pgfpathcurveto{\pgfqpoint{3.653742in}{4.219335in}}{\pgfqpoint{3.649352in}{4.229934in}}{\pgfqpoint{3.641538in}{4.237748in}}%
\pgfpathcurveto{\pgfqpoint{3.633725in}{4.245562in}}{\pgfqpoint{3.623126in}{4.249952in}}{\pgfqpoint{3.612075in}{4.249952in}}%
\pgfpathcurveto{\pgfqpoint{3.601025in}{4.249952in}}{\pgfqpoint{3.590426in}{4.245562in}}{\pgfqpoint{3.582613in}{4.237748in}}%
\pgfpathcurveto{\pgfqpoint{3.574799in}{4.229934in}}{\pgfqpoint{3.570409in}{4.219335in}}{\pgfqpoint{3.570409in}{4.208285in}}%
\pgfpathcurveto{\pgfqpoint{3.570409in}{4.197235in}}{\pgfqpoint{3.574799in}{4.186636in}}{\pgfqpoint{3.582613in}{4.178822in}}%
\pgfpathcurveto{\pgfqpoint{3.590426in}{4.171009in}}{\pgfqpoint{3.601025in}{4.166619in}}{\pgfqpoint{3.612075in}{4.166619in}}%
\pgfpathlineto{\pgfqpoint{3.612075in}{4.166619in}}%
\pgfpathclose%
\pgfusepath{stroke,fill}%
\end{pgfscope}%
\begin{pgfscope}%
\pgfpathrectangle{\pgfqpoint{2.963410in}{2.920818in}}{\pgfqpoint{2.177280in}{2.201755in}}%
\pgfusepath{clip}%
\pgfsetbuttcap%
\pgfsetroundjoin%
\definecolor{currentfill}{rgb}{0.172549,0.627451,0.172549}%
\pgfsetfillcolor{currentfill}%
\pgfsetlinewidth{0.481800pt}%
\definecolor{currentstroke}{rgb}{1.000000,1.000000,1.000000}%
\pgfsetstrokecolor{currentstroke}%
\pgfsetdash{}{0pt}%
\pgfpathmoveto{\pgfqpoint{3.848376in}{4.777275in}}%
\pgfpathcurveto{\pgfqpoint{3.859427in}{4.777275in}}{\pgfqpoint{3.870026in}{4.781665in}}{\pgfqpoint{3.877839in}{4.789479in}}%
\pgfpathcurveto{\pgfqpoint{3.885653in}{4.797292in}}{\pgfqpoint{3.890043in}{4.807891in}}{\pgfqpoint{3.890043in}{4.818942in}}%
\pgfpathcurveto{\pgfqpoint{3.890043in}{4.829992in}}{\pgfqpoint{3.885653in}{4.840591in}}{\pgfqpoint{3.877839in}{4.848404in}}%
\pgfpathcurveto{\pgfqpoint{3.870026in}{4.856218in}}{\pgfqpoint{3.859427in}{4.860608in}}{\pgfqpoint{3.848376in}{4.860608in}}%
\pgfpathcurveto{\pgfqpoint{3.837326in}{4.860608in}}{\pgfqpoint{3.826727in}{4.856218in}}{\pgfqpoint{3.818914in}{4.848404in}}%
\pgfpathcurveto{\pgfqpoint{3.811100in}{4.840591in}}{\pgfqpoint{3.806710in}{4.829992in}}{\pgfqpoint{3.806710in}{4.818942in}}%
\pgfpathcurveto{\pgfqpoint{3.806710in}{4.807891in}}{\pgfqpoint{3.811100in}{4.797292in}}{\pgfqpoint{3.818914in}{4.789479in}}%
\pgfpathcurveto{\pgfqpoint{3.826727in}{4.781665in}}{\pgfqpoint{3.837326in}{4.777275in}}{\pgfqpoint{3.848376in}{4.777275in}}%
\pgfpathlineto{\pgfqpoint{3.848376in}{4.777275in}}%
\pgfpathclose%
\pgfusepath{stroke,fill}%
\end{pgfscope}%
\begin{pgfscope}%
\pgfpathrectangle{\pgfqpoint{2.963410in}{2.920818in}}{\pgfqpoint{2.177280in}{2.201755in}}%
\pgfusepath{clip}%
\pgfsetbuttcap%
\pgfsetroundjoin%
\definecolor{currentfill}{rgb}{0.172549,0.627451,0.172549}%
\pgfsetfillcolor{currentfill}%
\pgfsetlinewidth{0.481800pt}%
\definecolor{currentstroke}{rgb}{1.000000,1.000000,1.000000}%
\pgfsetstrokecolor{currentstroke}%
\pgfsetdash{}{0pt}%
\pgfpathmoveto{\pgfqpoint{3.612075in}{4.607648in}}%
\pgfpathcurveto{\pgfqpoint{3.623126in}{4.607648in}}{\pgfqpoint{3.633725in}{4.612038in}}{\pgfqpoint{3.641538in}{4.619852in}}%
\pgfpathcurveto{\pgfqpoint{3.649352in}{4.627666in}}{\pgfqpoint{3.653742in}{4.638265in}}{\pgfqpoint{3.653742in}{4.649315in}}%
\pgfpathcurveto{\pgfqpoint{3.653742in}{4.660365in}}{\pgfqpoint{3.649352in}{4.670964in}}{\pgfqpoint{3.641538in}{4.678778in}}%
\pgfpathcurveto{\pgfqpoint{3.633725in}{4.686591in}}{\pgfqpoint{3.623126in}{4.690981in}}{\pgfqpoint{3.612075in}{4.690981in}}%
\pgfpathcurveto{\pgfqpoint{3.601025in}{4.690981in}}{\pgfqpoint{3.590426in}{4.686591in}}{\pgfqpoint{3.582613in}{4.678778in}}%
\pgfpathcurveto{\pgfqpoint{3.574799in}{4.670964in}}{\pgfqpoint{3.570409in}{4.660365in}}{\pgfqpoint{3.570409in}{4.649315in}}%
\pgfpathcurveto{\pgfqpoint{3.570409in}{4.638265in}}{\pgfqpoint{3.574799in}{4.627666in}}{\pgfqpoint{3.582613in}{4.619852in}}%
\pgfpathcurveto{\pgfqpoint{3.590426in}{4.612038in}}{\pgfqpoint{3.601025in}{4.607648in}}{\pgfqpoint{3.612075in}{4.607648in}}%
\pgfpathlineto{\pgfqpoint{3.612075in}{4.607648in}}%
\pgfpathclose%
\pgfusepath{stroke,fill}%
\end{pgfscope}%
\begin{pgfscope}%
\pgfpathrectangle{\pgfqpoint{2.963410in}{2.920818in}}{\pgfqpoint{2.177280in}{2.201755in}}%
\pgfusepath{clip}%
\pgfsetbuttcap%
\pgfsetroundjoin%
\definecolor{currentfill}{rgb}{0.172549,0.627451,0.172549}%
\pgfsetfillcolor{currentfill}%
\pgfsetlinewidth{0.481800pt}%
\definecolor{currentstroke}{rgb}{1.000000,1.000000,1.000000}%
\pgfsetstrokecolor{currentstroke}%
\pgfsetdash{}{0pt}%
\pgfpathmoveto{\pgfqpoint{4.261903in}{4.709424in}}%
\pgfpathcurveto{\pgfqpoint{4.272953in}{4.709424in}}{\pgfqpoint{4.283552in}{4.713814in}}{\pgfqpoint{4.291366in}{4.721628in}}%
\pgfpathcurveto{\pgfqpoint{4.299180in}{4.729442in}}{\pgfqpoint{4.303570in}{4.740041in}}{\pgfqpoint{4.303570in}{4.751091in}}%
\pgfpathcurveto{\pgfqpoint{4.303570in}{4.762141in}}{\pgfqpoint{4.299180in}{4.772740in}}{\pgfqpoint{4.291366in}{4.780554in}}%
\pgfpathcurveto{\pgfqpoint{4.283552in}{4.788367in}}{\pgfqpoint{4.272953in}{4.792757in}}{\pgfqpoint{4.261903in}{4.792757in}}%
\pgfpathcurveto{\pgfqpoint{4.250853in}{4.792757in}}{\pgfqpoint{4.240254in}{4.788367in}}{\pgfqpoint{4.232441in}{4.780554in}}%
\pgfpathcurveto{\pgfqpoint{4.224627in}{4.772740in}}{\pgfqpoint{4.220237in}{4.762141in}}{\pgfqpoint{4.220237in}{4.751091in}}%
\pgfpathcurveto{\pgfqpoint{4.220237in}{4.740041in}}{\pgfqpoint{4.224627in}{4.729442in}}{\pgfqpoint{4.232441in}{4.721628in}}%
\pgfpathcurveto{\pgfqpoint{4.240254in}{4.713814in}}{\pgfqpoint{4.250853in}{4.709424in}}{\pgfqpoint{4.261903in}{4.709424in}}%
\pgfpathlineto{\pgfqpoint{4.261903in}{4.709424in}}%
\pgfpathclose%
\pgfusepath{stroke,fill}%
\end{pgfscope}%
\begin{pgfscope}%
\pgfpathrectangle{\pgfqpoint{2.963410in}{2.920818in}}{\pgfqpoint{2.177280in}{2.201755in}}%
\pgfusepath{clip}%
\pgfsetbuttcap%
\pgfsetroundjoin%
\definecolor{currentfill}{rgb}{0.172549,0.627451,0.172549}%
\pgfsetfillcolor{currentfill}%
\pgfsetlinewidth{0.481800pt}%
\definecolor{currentstroke}{rgb}{1.000000,1.000000,1.000000}%
\pgfsetstrokecolor{currentstroke}%
\pgfsetdash{}{0pt}%
\pgfpathmoveto{\pgfqpoint{4.025602in}{4.370171in}}%
\pgfpathcurveto{\pgfqpoint{4.036652in}{4.370171in}}{\pgfqpoint{4.047251in}{4.374561in}}{\pgfqpoint{4.055065in}{4.382375in}}%
\pgfpathcurveto{\pgfqpoint{4.062879in}{4.390188in}}{\pgfqpoint{4.067269in}{4.400787in}}{\pgfqpoint{4.067269in}{4.411837in}}%
\pgfpathcurveto{\pgfqpoint{4.067269in}{4.422887in}}{\pgfqpoint{4.062879in}{4.433486in}}{\pgfqpoint{4.055065in}{4.441300in}}%
\pgfpathcurveto{\pgfqpoint{4.047251in}{4.449114in}}{\pgfqpoint{4.036652in}{4.453504in}}{\pgfqpoint{4.025602in}{4.453504in}}%
\pgfpathcurveto{\pgfqpoint{4.014552in}{4.453504in}}{\pgfqpoint{4.003953in}{4.449114in}}{\pgfqpoint{3.996139in}{4.441300in}}%
\pgfpathcurveto{\pgfqpoint{3.988326in}{4.433486in}}{\pgfqpoint{3.983936in}{4.422887in}}{\pgfqpoint{3.983936in}{4.411837in}}%
\pgfpathcurveto{\pgfqpoint{3.983936in}{4.400787in}}{\pgfqpoint{3.988326in}{4.390188in}}{\pgfqpoint{3.996139in}{4.382375in}}%
\pgfpathcurveto{\pgfqpoint{4.003953in}{4.374561in}}{\pgfqpoint{4.014552in}{4.370171in}}{\pgfqpoint{4.025602in}{4.370171in}}%
\pgfpathlineto{\pgfqpoint{4.025602in}{4.370171in}}%
\pgfpathclose%
\pgfusepath{stroke,fill}%
\end{pgfscope}%
\begin{pgfscope}%
\pgfpathrectangle{\pgfqpoint{2.963410in}{2.920818in}}{\pgfqpoint{2.177280in}{2.201755in}}%
\pgfusepath{clip}%
\pgfsetbuttcap%
\pgfsetroundjoin%
\definecolor{currentfill}{rgb}{0.172549,0.627451,0.172549}%
\pgfsetfillcolor{currentfill}%
\pgfsetlinewidth{0.481800pt}%
\definecolor{currentstroke}{rgb}{1.000000,1.000000,1.000000}%
\pgfsetstrokecolor{currentstroke}%
\pgfsetdash{}{0pt}%
\pgfpathmoveto{\pgfqpoint{3.730226in}{4.438021in}}%
\pgfpathcurveto{\pgfqpoint{3.741276in}{4.438021in}}{\pgfqpoint{3.751875in}{4.442412in}}{\pgfqpoint{3.759689in}{4.450225in}}%
\pgfpathcurveto{\pgfqpoint{3.767502in}{4.458039in}}{\pgfqpoint{3.771893in}{4.468638in}}{\pgfqpoint{3.771893in}{4.479688in}}%
\pgfpathcurveto{\pgfqpoint{3.771893in}{4.490738in}}{\pgfqpoint{3.767502in}{4.501337in}}{\pgfqpoint{3.759689in}{4.509151in}}%
\pgfpathcurveto{\pgfqpoint{3.751875in}{4.516964in}}{\pgfqpoint{3.741276in}{4.521355in}}{\pgfqpoint{3.730226in}{4.521355in}}%
\pgfpathcurveto{\pgfqpoint{3.719176in}{4.521355in}}{\pgfqpoint{3.708577in}{4.516964in}}{\pgfqpoint{3.700763in}{4.509151in}}%
\pgfpathcurveto{\pgfqpoint{3.692950in}{4.501337in}}{\pgfqpoint{3.688559in}{4.490738in}}{\pgfqpoint{3.688559in}{4.479688in}}%
\pgfpathcurveto{\pgfqpoint{3.688559in}{4.468638in}}{\pgfqpoint{3.692950in}{4.458039in}}{\pgfqpoint{3.700763in}{4.450225in}}%
\pgfpathcurveto{\pgfqpoint{3.708577in}{4.442412in}}{\pgfqpoint{3.719176in}{4.438021in}}{\pgfqpoint{3.730226in}{4.438021in}}%
\pgfpathlineto{\pgfqpoint{3.730226in}{4.438021in}}%
\pgfpathclose%
\pgfusepath{stroke,fill}%
\end{pgfscope}%
\begin{pgfscope}%
\pgfpathrectangle{\pgfqpoint{2.963410in}{2.920818in}}{\pgfqpoint{2.177280in}{2.201755in}}%
\pgfusepath{clip}%
\pgfsetbuttcap%
\pgfsetroundjoin%
\definecolor{currentfill}{rgb}{0.172549,0.627451,0.172549}%
\pgfsetfillcolor{currentfill}%
\pgfsetlinewidth{0.481800pt}%
\definecolor{currentstroke}{rgb}{1.000000,1.000000,1.000000}%
\pgfsetstrokecolor{currentstroke}%
\pgfsetdash{}{0pt}%
\pgfpathmoveto{\pgfqpoint{3.907452in}{4.505872in}}%
\pgfpathcurveto{\pgfqpoint{3.918502in}{4.505872in}}{\pgfqpoint{3.929101in}{4.510262in}}{\pgfqpoint{3.936915in}{4.518076in}}%
\pgfpathcurveto{\pgfqpoint{3.944728in}{4.525890in}}{\pgfqpoint{3.949118in}{4.536489in}}{\pgfqpoint{3.949118in}{4.547539in}}%
\pgfpathcurveto{\pgfqpoint{3.949118in}{4.558589in}}{\pgfqpoint{3.944728in}{4.569188in}}{\pgfqpoint{3.936915in}{4.577002in}}%
\pgfpathcurveto{\pgfqpoint{3.929101in}{4.584815in}}{\pgfqpoint{3.918502in}{4.589205in}}{\pgfqpoint{3.907452in}{4.589205in}}%
\pgfpathcurveto{\pgfqpoint{3.896402in}{4.589205in}}{\pgfqpoint{3.885803in}{4.584815in}}{\pgfqpoint{3.877989in}{4.577002in}}%
\pgfpathcurveto{\pgfqpoint{3.870175in}{4.569188in}}{\pgfqpoint{3.865785in}{4.558589in}}{\pgfqpoint{3.865785in}{4.547539in}}%
\pgfpathcurveto{\pgfqpoint{3.865785in}{4.536489in}}{\pgfqpoint{3.870175in}{4.525890in}}{\pgfqpoint{3.877989in}{4.518076in}}%
\pgfpathcurveto{\pgfqpoint{3.885803in}{4.510262in}}{\pgfqpoint{3.896402in}{4.505872in}}{\pgfqpoint{3.907452in}{4.505872in}}%
\pgfpathlineto{\pgfqpoint{3.907452in}{4.505872in}}%
\pgfpathclose%
\pgfusepath{stroke,fill}%
\end{pgfscope}%
\begin{pgfscope}%
\pgfpathrectangle{\pgfqpoint{2.963410in}{2.920818in}}{\pgfqpoint{2.177280in}{2.201755in}}%
\pgfusepath{clip}%
\pgfsetbuttcap%
\pgfsetroundjoin%
\definecolor{currentfill}{rgb}{0.172549,0.627451,0.172549}%
\pgfsetfillcolor{currentfill}%
\pgfsetlinewidth{0.481800pt}%
\definecolor{currentstroke}{rgb}{1.000000,1.000000,1.000000}%
\pgfsetstrokecolor{currentstroke}%
\pgfsetdash{}{0pt}%
\pgfpathmoveto{\pgfqpoint{3.612075in}{4.336245in}}%
\pgfpathcurveto{\pgfqpoint{3.623126in}{4.336245in}}{\pgfqpoint{3.633725in}{4.340636in}}{\pgfqpoint{3.641538in}{4.348449in}}%
\pgfpathcurveto{\pgfqpoint{3.649352in}{4.356263in}}{\pgfqpoint{3.653742in}{4.366862in}}{\pgfqpoint{3.653742in}{4.377912in}}%
\pgfpathcurveto{\pgfqpoint{3.653742in}{4.388962in}}{\pgfqpoint{3.649352in}{4.399561in}}{\pgfqpoint{3.641538in}{4.407375in}}%
\pgfpathcurveto{\pgfqpoint{3.633725in}{4.415188in}}{\pgfqpoint{3.623126in}{4.419579in}}{\pgfqpoint{3.612075in}{4.419579in}}%
\pgfpathcurveto{\pgfqpoint{3.601025in}{4.419579in}}{\pgfqpoint{3.590426in}{4.415188in}}{\pgfqpoint{3.582613in}{4.407375in}}%
\pgfpathcurveto{\pgfqpoint{3.574799in}{4.399561in}}{\pgfqpoint{3.570409in}{4.388962in}}{\pgfqpoint{3.570409in}{4.377912in}}%
\pgfpathcurveto{\pgfqpoint{3.570409in}{4.366862in}}{\pgfqpoint{3.574799in}{4.356263in}}{\pgfqpoint{3.582613in}{4.348449in}}%
\pgfpathcurveto{\pgfqpoint{3.590426in}{4.340636in}}{\pgfqpoint{3.601025in}{4.336245in}}{\pgfqpoint{3.612075in}{4.336245in}}%
\pgfpathlineto{\pgfqpoint{3.612075in}{4.336245in}}%
\pgfpathclose%
\pgfusepath{stroke,fill}%
\end{pgfscope}%
\begin{pgfscope}%
\pgfpathrectangle{\pgfqpoint{2.963410in}{2.920818in}}{\pgfqpoint{2.177280in}{2.201755in}}%
\pgfusepath{clip}%
\pgfsetbuttcap%
\pgfsetroundjoin%
\definecolor{currentfill}{rgb}{0.172549,0.627451,0.172549}%
\pgfsetfillcolor{currentfill}%
\pgfsetlinewidth{0.481800pt}%
\definecolor{currentstroke}{rgb}{1.000000,1.000000,1.000000}%
\pgfsetstrokecolor{currentstroke}%
\pgfsetdash{}{0pt}%
\pgfpathmoveto{\pgfqpoint{3.789301in}{4.370171in}}%
\pgfpathcurveto{\pgfqpoint{3.800351in}{4.370171in}}{\pgfqpoint{3.810950in}{4.374561in}}{\pgfqpoint{3.818764in}{4.382375in}}%
\pgfpathcurveto{\pgfqpoint{3.826578in}{4.390188in}}{\pgfqpoint{3.830968in}{4.400787in}}{\pgfqpoint{3.830968in}{4.411837in}}%
\pgfpathcurveto{\pgfqpoint{3.830968in}{4.422887in}}{\pgfqpoint{3.826578in}{4.433486in}}{\pgfqpoint{3.818764in}{4.441300in}}%
\pgfpathcurveto{\pgfqpoint{3.810950in}{4.449114in}}{\pgfqpoint{3.800351in}{4.453504in}}{\pgfqpoint{3.789301in}{4.453504in}}%
\pgfpathcurveto{\pgfqpoint{3.778251in}{4.453504in}}{\pgfqpoint{3.767652in}{4.449114in}}{\pgfqpoint{3.759838in}{4.441300in}}%
\pgfpathcurveto{\pgfqpoint{3.752025in}{4.433486in}}{\pgfqpoint{3.747635in}{4.422887in}}{\pgfqpoint{3.747635in}{4.411837in}}%
\pgfpathcurveto{\pgfqpoint{3.747635in}{4.400787in}}{\pgfqpoint{3.752025in}{4.390188in}}{\pgfqpoint{3.759838in}{4.382375in}}%
\pgfpathcurveto{\pgfqpoint{3.767652in}{4.374561in}}{\pgfqpoint{3.778251in}{4.370171in}}{\pgfqpoint{3.789301in}{4.370171in}}%
\pgfpathlineto{\pgfqpoint{3.789301in}{4.370171in}}%
\pgfpathclose%
\pgfusepath{stroke,fill}%
\end{pgfscope}%
\begin{pgfscope}%
\pgfpathrectangle{\pgfqpoint{2.963410in}{2.920818in}}{\pgfqpoint{2.177280in}{2.201755in}}%
\pgfusepath{clip}%
\pgfsetbuttcap%
\pgfsetroundjoin%
\definecolor{currentfill}{rgb}{0.172549,0.627451,0.172549}%
\pgfsetfillcolor{currentfill}%
\pgfsetlinewidth{0.481800pt}%
\definecolor{currentstroke}{rgb}{1.000000,1.000000,1.000000}%
\pgfsetstrokecolor{currentstroke}%
\pgfsetdash{}{0pt}%
\pgfpathmoveto{\pgfqpoint{4.025602in}{4.438021in}}%
\pgfpathcurveto{\pgfqpoint{4.036652in}{4.438021in}}{\pgfqpoint{4.047251in}{4.442412in}}{\pgfqpoint{4.055065in}{4.450225in}}%
\pgfpathcurveto{\pgfqpoint{4.062879in}{4.458039in}}{\pgfqpoint{4.067269in}{4.468638in}}{\pgfqpoint{4.067269in}{4.479688in}}%
\pgfpathcurveto{\pgfqpoint{4.067269in}{4.490738in}}{\pgfqpoint{4.062879in}{4.501337in}}{\pgfqpoint{4.055065in}{4.509151in}}%
\pgfpathcurveto{\pgfqpoint{4.047251in}{4.516964in}}{\pgfqpoint{4.036652in}{4.521355in}}{\pgfqpoint{4.025602in}{4.521355in}}%
\pgfpathcurveto{\pgfqpoint{4.014552in}{4.521355in}}{\pgfqpoint{4.003953in}{4.516964in}}{\pgfqpoint{3.996139in}{4.509151in}}%
\pgfpathcurveto{\pgfqpoint{3.988326in}{4.501337in}}{\pgfqpoint{3.983936in}{4.490738in}}{\pgfqpoint{3.983936in}{4.479688in}}%
\pgfpathcurveto{\pgfqpoint{3.983936in}{4.468638in}}{\pgfqpoint{3.988326in}{4.458039in}}{\pgfqpoint{3.996139in}{4.450225in}}%
\pgfpathcurveto{\pgfqpoint{4.003953in}{4.442412in}}{\pgfqpoint{4.014552in}{4.438021in}}{\pgfqpoint{4.025602in}{4.438021in}}%
\pgfpathlineto{\pgfqpoint{4.025602in}{4.438021in}}%
\pgfpathclose%
\pgfusepath{stroke,fill}%
\end{pgfscope}%
\begin{pgfscope}%
\pgfpathrectangle{\pgfqpoint{2.963410in}{2.920818in}}{\pgfqpoint{2.177280in}{2.201755in}}%
\pgfusepath{clip}%
\pgfsetbuttcap%
\pgfsetroundjoin%
\definecolor{currentfill}{rgb}{0.172549,0.627451,0.172549}%
\pgfsetfillcolor{currentfill}%
\pgfsetlinewidth{0.481800pt}%
\definecolor{currentstroke}{rgb}{1.000000,1.000000,1.000000}%
\pgfsetstrokecolor{currentstroke}%
\pgfsetdash{}{0pt}%
\pgfpathmoveto{\pgfqpoint{3.907452in}{4.505872in}}%
\pgfpathcurveto{\pgfqpoint{3.918502in}{4.505872in}}{\pgfqpoint{3.929101in}{4.510262in}}{\pgfqpoint{3.936915in}{4.518076in}}%
\pgfpathcurveto{\pgfqpoint{3.944728in}{4.525890in}}{\pgfqpoint{3.949118in}{4.536489in}}{\pgfqpoint{3.949118in}{4.547539in}}%
\pgfpathcurveto{\pgfqpoint{3.949118in}{4.558589in}}{\pgfqpoint{3.944728in}{4.569188in}}{\pgfqpoint{3.936915in}{4.577002in}}%
\pgfpathcurveto{\pgfqpoint{3.929101in}{4.584815in}}{\pgfqpoint{3.918502in}{4.589205in}}{\pgfqpoint{3.907452in}{4.589205in}}%
\pgfpathcurveto{\pgfqpoint{3.896402in}{4.589205in}}{\pgfqpoint{3.885803in}{4.584815in}}{\pgfqpoint{3.877989in}{4.577002in}}%
\pgfpathcurveto{\pgfqpoint{3.870175in}{4.569188in}}{\pgfqpoint{3.865785in}{4.558589in}}{\pgfqpoint{3.865785in}{4.547539in}}%
\pgfpathcurveto{\pgfqpoint{3.865785in}{4.536489in}}{\pgfqpoint{3.870175in}{4.525890in}}{\pgfqpoint{3.877989in}{4.518076in}}%
\pgfpathcurveto{\pgfqpoint{3.885803in}{4.510262in}}{\pgfqpoint{3.896402in}{4.505872in}}{\pgfqpoint{3.907452in}{4.505872in}}%
\pgfpathlineto{\pgfqpoint{3.907452in}{4.505872in}}%
\pgfpathclose%
\pgfusepath{stroke,fill}%
\end{pgfscope}%
\begin{pgfscope}%
\pgfpathrectangle{\pgfqpoint{2.963410in}{2.920818in}}{\pgfqpoint{2.177280in}{2.201755in}}%
\pgfusepath{clip}%
\pgfsetbuttcap%
\pgfsetroundjoin%
\definecolor{currentfill}{rgb}{0.172549,0.627451,0.172549}%
\pgfsetfillcolor{currentfill}%
\pgfsetlinewidth{0.481800pt}%
\definecolor{currentstroke}{rgb}{1.000000,1.000000,1.000000}%
\pgfsetstrokecolor{currentstroke}%
\pgfsetdash{}{0pt}%
\pgfpathmoveto{\pgfqpoint{4.380054in}{4.912976in}}%
\pgfpathcurveto{\pgfqpoint{4.391104in}{4.912976in}}{\pgfqpoint{4.401703in}{4.917366in}}{\pgfqpoint{4.409517in}{4.925180in}}%
\pgfpathcurveto{\pgfqpoint{4.417330in}{4.932994in}}{\pgfqpoint{4.421721in}{4.943593in}}{\pgfqpoint{4.421721in}{4.954643in}}%
\pgfpathcurveto{\pgfqpoint{4.421721in}{4.965693in}}{\pgfqpoint{4.417330in}{4.976292in}}{\pgfqpoint{4.409517in}{4.984106in}}%
\pgfpathcurveto{\pgfqpoint{4.401703in}{4.991919in}}{\pgfqpoint{4.391104in}{4.996310in}}{\pgfqpoint{4.380054in}{4.996310in}}%
\pgfpathcurveto{\pgfqpoint{4.369004in}{4.996310in}}{\pgfqpoint{4.358405in}{4.991919in}}{\pgfqpoint{4.350591in}{4.984106in}}%
\pgfpathcurveto{\pgfqpoint{4.342777in}{4.976292in}}{\pgfqpoint{4.338387in}{4.965693in}}{\pgfqpoint{4.338387in}{4.954643in}}%
\pgfpathcurveto{\pgfqpoint{4.338387in}{4.943593in}}{\pgfqpoint{4.342777in}{4.932994in}}{\pgfqpoint{4.350591in}{4.925180in}}%
\pgfpathcurveto{\pgfqpoint{4.358405in}{4.917366in}}{\pgfqpoint{4.369004in}{4.912976in}}{\pgfqpoint{4.380054in}{4.912976in}}%
\pgfpathlineto{\pgfqpoint{4.380054in}{4.912976in}}%
\pgfpathclose%
\pgfusepath{stroke,fill}%
\end{pgfscope}%
\begin{pgfscope}%
\pgfpathrectangle{\pgfqpoint{2.963410in}{2.920818in}}{\pgfqpoint{2.177280in}{2.201755in}}%
\pgfusepath{clip}%
\pgfsetbuttcap%
\pgfsetroundjoin%
\definecolor{currentfill}{rgb}{0.172549,0.627451,0.172549}%
\pgfsetfillcolor{currentfill}%
\pgfsetlinewidth{0.481800pt}%
\definecolor{currentstroke}{rgb}{1.000000,1.000000,1.000000}%
\pgfsetstrokecolor{currentstroke}%
\pgfsetdash{}{0pt}%
\pgfpathmoveto{\pgfqpoint{3.671151in}{4.980827in}}%
\pgfpathcurveto{\pgfqpoint{3.682201in}{4.980827in}}{\pgfqpoint{3.692800in}{4.985217in}}{\pgfqpoint{3.700613in}{4.993031in}}%
\pgfpathcurveto{\pgfqpoint{3.708427in}{5.000844in}}{\pgfqpoint{3.712817in}{5.011443in}}{\pgfqpoint{3.712817in}{5.022494in}}%
\pgfpathcurveto{\pgfqpoint{3.712817in}{5.033544in}}{\pgfqpoint{3.708427in}{5.044143in}}{\pgfqpoint{3.700613in}{5.051956in}}%
\pgfpathcurveto{\pgfqpoint{3.692800in}{5.059770in}}{\pgfqpoint{3.682201in}{5.064160in}}{\pgfqpoint{3.671151in}{5.064160in}}%
\pgfpathcurveto{\pgfqpoint{3.660101in}{5.064160in}}{\pgfqpoint{3.649501in}{5.059770in}}{\pgfqpoint{3.641688in}{5.051956in}}%
\pgfpathcurveto{\pgfqpoint{3.633874in}{5.044143in}}{\pgfqpoint{3.629484in}{5.033544in}}{\pgfqpoint{3.629484in}{5.022494in}}%
\pgfpathcurveto{\pgfqpoint{3.629484in}{5.011443in}}{\pgfqpoint{3.633874in}{5.000844in}}{\pgfqpoint{3.641688in}{4.993031in}}%
\pgfpathcurveto{\pgfqpoint{3.649501in}{4.985217in}}{\pgfqpoint{3.660101in}{4.980827in}}{\pgfqpoint{3.671151in}{4.980827in}}%
\pgfpathlineto{\pgfqpoint{3.671151in}{4.980827in}}%
\pgfpathclose%
\pgfusepath{stroke,fill}%
\end{pgfscope}%
\begin{pgfscope}%
\pgfpathrectangle{\pgfqpoint{2.963410in}{2.920818in}}{\pgfqpoint{2.177280in}{2.201755in}}%
\pgfusepath{clip}%
\pgfsetbuttcap%
\pgfsetroundjoin%
\definecolor{currentfill}{rgb}{0.172549,0.627451,0.172549}%
\pgfsetfillcolor{currentfill}%
\pgfsetlinewidth{0.481800pt}%
\definecolor{currentstroke}{rgb}{1.000000,1.000000,1.000000}%
\pgfsetstrokecolor{currentstroke}%
\pgfsetdash{}{0pt}%
\pgfpathmoveto{\pgfqpoint{3.434850in}{4.336245in}}%
\pgfpathcurveto{\pgfqpoint{3.445900in}{4.336245in}}{\pgfqpoint{3.456499in}{4.340636in}}{\pgfqpoint{3.464312in}{4.348449in}}%
\pgfpathcurveto{\pgfqpoint{3.472126in}{4.356263in}}{\pgfqpoint{3.476516in}{4.366862in}}{\pgfqpoint{3.476516in}{4.377912in}}%
\pgfpathcurveto{\pgfqpoint{3.476516in}{4.388962in}}{\pgfqpoint{3.472126in}{4.399561in}}{\pgfqpoint{3.464312in}{4.407375in}}%
\pgfpathcurveto{\pgfqpoint{3.456499in}{4.415188in}}{\pgfqpoint{3.445900in}{4.419579in}}{\pgfqpoint{3.434850in}{4.419579in}}%
\pgfpathcurveto{\pgfqpoint{3.423799in}{4.419579in}}{\pgfqpoint{3.413200in}{4.415188in}}{\pgfqpoint{3.405387in}{4.407375in}}%
\pgfpathcurveto{\pgfqpoint{3.397573in}{4.399561in}}{\pgfqpoint{3.393183in}{4.388962in}}{\pgfqpoint{3.393183in}{4.377912in}}%
\pgfpathcurveto{\pgfqpoint{3.393183in}{4.366862in}}{\pgfqpoint{3.397573in}{4.356263in}}{\pgfqpoint{3.405387in}{4.348449in}}%
\pgfpathcurveto{\pgfqpoint{3.413200in}{4.340636in}}{\pgfqpoint{3.423799in}{4.336245in}}{\pgfqpoint{3.434850in}{4.336245in}}%
\pgfpathlineto{\pgfqpoint{3.434850in}{4.336245in}}%
\pgfpathclose%
\pgfusepath{stroke,fill}%
\end{pgfscope}%
\begin{pgfscope}%
\pgfpathrectangle{\pgfqpoint{2.963410in}{2.920818in}}{\pgfqpoint{2.177280in}{2.201755in}}%
\pgfusepath{clip}%
\pgfsetbuttcap%
\pgfsetroundjoin%
\definecolor{currentfill}{rgb}{0.172549,0.627451,0.172549}%
\pgfsetfillcolor{currentfill}%
\pgfsetlinewidth{0.481800pt}%
\definecolor{currentstroke}{rgb}{1.000000,1.000000,1.000000}%
\pgfsetstrokecolor{currentstroke}%
\pgfsetdash{}{0pt}%
\pgfpathmoveto{\pgfqpoint{4.025602in}{4.573723in}}%
\pgfpathcurveto{\pgfqpoint{4.036652in}{4.573723in}}{\pgfqpoint{4.047251in}{4.578113in}}{\pgfqpoint{4.055065in}{4.585927in}}%
\pgfpathcurveto{\pgfqpoint{4.062879in}{4.593740in}}{\pgfqpoint{4.067269in}{4.604339in}}{\pgfqpoint{4.067269in}{4.615389in}}%
\pgfpathcurveto{\pgfqpoint{4.067269in}{4.626440in}}{\pgfqpoint{4.062879in}{4.637039in}}{\pgfqpoint{4.055065in}{4.644852in}}%
\pgfpathcurveto{\pgfqpoint{4.047251in}{4.652666in}}{\pgfqpoint{4.036652in}{4.657056in}}{\pgfqpoint{4.025602in}{4.657056in}}%
\pgfpathcurveto{\pgfqpoint{4.014552in}{4.657056in}}{\pgfqpoint{4.003953in}{4.652666in}}{\pgfqpoint{3.996139in}{4.644852in}}%
\pgfpathcurveto{\pgfqpoint{3.988326in}{4.637039in}}{\pgfqpoint{3.983936in}{4.626440in}}{\pgfqpoint{3.983936in}{4.615389in}}%
\pgfpathcurveto{\pgfqpoint{3.983936in}{4.604339in}}{\pgfqpoint{3.988326in}{4.593740in}}{\pgfqpoint{3.996139in}{4.585927in}}%
\pgfpathcurveto{\pgfqpoint{4.003953in}{4.578113in}}{\pgfqpoint{4.014552in}{4.573723in}}{\pgfqpoint{4.025602in}{4.573723in}}%
\pgfpathlineto{\pgfqpoint{4.025602in}{4.573723in}}%
\pgfpathclose%
\pgfusepath{stroke,fill}%
\end{pgfscope}%
\begin{pgfscope}%
\pgfpathrectangle{\pgfqpoint{2.963410in}{2.920818in}}{\pgfqpoint{2.177280in}{2.201755in}}%
\pgfusepath{clip}%
\pgfsetbuttcap%
\pgfsetroundjoin%
\definecolor{currentfill}{rgb}{0.172549,0.627451,0.172549}%
\pgfsetfillcolor{currentfill}%
\pgfsetlinewidth{0.481800pt}%
\definecolor{currentstroke}{rgb}{1.000000,1.000000,1.000000}%
\pgfsetstrokecolor{currentstroke}%
\pgfsetdash{}{0pt}%
\pgfpathmoveto{\pgfqpoint{3.789301in}{4.302320in}}%
\pgfpathcurveto{\pgfqpoint{3.800351in}{4.302320in}}{\pgfqpoint{3.810950in}{4.306710in}}{\pgfqpoint{3.818764in}{4.314524in}}%
\pgfpathcurveto{\pgfqpoint{3.826578in}{4.322337in}}{\pgfqpoint{3.830968in}{4.332937in}}{\pgfqpoint{3.830968in}{4.343987in}}%
\pgfpathcurveto{\pgfqpoint{3.830968in}{4.355037in}}{\pgfqpoint{3.826578in}{4.365636in}}{\pgfqpoint{3.818764in}{4.373449in}}%
\pgfpathcurveto{\pgfqpoint{3.810950in}{4.381263in}}{\pgfqpoint{3.800351in}{4.385653in}}{\pgfqpoint{3.789301in}{4.385653in}}%
\pgfpathcurveto{\pgfqpoint{3.778251in}{4.385653in}}{\pgfqpoint{3.767652in}{4.381263in}}{\pgfqpoint{3.759838in}{4.373449in}}%
\pgfpathcurveto{\pgfqpoint{3.752025in}{4.365636in}}{\pgfqpoint{3.747635in}{4.355037in}}{\pgfqpoint{3.747635in}{4.343987in}}%
\pgfpathcurveto{\pgfqpoint{3.747635in}{4.332937in}}{\pgfqpoint{3.752025in}{4.322337in}}{\pgfqpoint{3.759838in}{4.314524in}}%
\pgfpathcurveto{\pgfqpoint{3.767652in}{4.306710in}}{\pgfqpoint{3.778251in}{4.302320in}}{\pgfqpoint{3.789301in}{4.302320in}}%
\pgfpathlineto{\pgfqpoint{3.789301in}{4.302320in}}%
\pgfpathclose%
\pgfusepath{stroke,fill}%
\end{pgfscope}%
\begin{pgfscope}%
\pgfpathrectangle{\pgfqpoint{2.963410in}{2.920818in}}{\pgfqpoint{2.177280in}{2.201755in}}%
\pgfusepath{clip}%
\pgfsetbuttcap%
\pgfsetroundjoin%
\definecolor{currentfill}{rgb}{0.172549,0.627451,0.172549}%
\pgfsetfillcolor{currentfill}%
\pgfsetlinewidth{0.481800pt}%
\definecolor{currentstroke}{rgb}{1.000000,1.000000,1.000000}%
\pgfsetstrokecolor{currentstroke}%
\pgfsetdash{}{0pt}%
\pgfpathmoveto{\pgfqpoint{3.789301in}{4.912976in}}%
\pgfpathcurveto{\pgfqpoint{3.800351in}{4.912976in}}{\pgfqpoint{3.810950in}{4.917366in}}{\pgfqpoint{3.818764in}{4.925180in}}%
\pgfpathcurveto{\pgfqpoint{3.826578in}{4.932994in}}{\pgfqpoint{3.830968in}{4.943593in}}{\pgfqpoint{3.830968in}{4.954643in}}%
\pgfpathcurveto{\pgfqpoint{3.830968in}{4.965693in}}{\pgfqpoint{3.826578in}{4.976292in}}{\pgfqpoint{3.818764in}{4.984106in}}%
\pgfpathcurveto{\pgfqpoint{3.810950in}{4.991919in}}{\pgfqpoint{3.800351in}{4.996310in}}{\pgfqpoint{3.789301in}{4.996310in}}%
\pgfpathcurveto{\pgfqpoint{3.778251in}{4.996310in}}{\pgfqpoint{3.767652in}{4.991919in}}{\pgfqpoint{3.759838in}{4.984106in}}%
\pgfpathcurveto{\pgfqpoint{3.752025in}{4.976292in}}{\pgfqpoint{3.747635in}{4.965693in}}{\pgfqpoint{3.747635in}{4.954643in}}%
\pgfpathcurveto{\pgfqpoint{3.747635in}{4.943593in}}{\pgfqpoint{3.752025in}{4.932994in}}{\pgfqpoint{3.759838in}{4.925180in}}%
\pgfpathcurveto{\pgfqpoint{3.767652in}{4.917366in}}{\pgfqpoint{3.778251in}{4.912976in}}{\pgfqpoint{3.789301in}{4.912976in}}%
\pgfpathlineto{\pgfqpoint{3.789301in}{4.912976in}}%
\pgfpathclose%
\pgfusepath{stroke,fill}%
\end{pgfscope}%
\begin{pgfscope}%
\pgfpathrectangle{\pgfqpoint{2.963410in}{2.920818in}}{\pgfqpoint{2.177280in}{2.201755in}}%
\pgfusepath{clip}%
\pgfsetbuttcap%
\pgfsetroundjoin%
\definecolor{currentfill}{rgb}{0.172549,0.627451,0.172549}%
\pgfsetfillcolor{currentfill}%
\pgfsetlinewidth{0.481800pt}%
\definecolor{currentstroke}{rgb}{1.000000,1.000000,1.000000}%
\pgfsetstrokecolor{currentstroke}%
\pgfsetdash{}{0pt}%
\pgfpathmoveto{\pgfqpoint{3.730226in}{4.302320in}}%
\pgfpathcurveto{\pgfqpoint{3.741276in}{4.302320in}}{\pgfqpoint{3.751875in}{4.306710in}}{\pgfqpoint{3.759689in}{4.314524in}}%
\pgfpathcurveto{\pgfqpoint{3.767502in}{4.322337in}}{\pgfqpoint{3.771893in}{4.332937in}}{\pgfqpoint{3.771893in}{4.343987in}}%
\pgfpathcurveto{\pgfqpoint{3.771893in}{4.355037in}}{\pgfqpoint{3.767502in}{4.365636in}}{\pgfqpoint{3.759689in}{4.373449in}}%
\pgfpathcurveto{\pgfqpoint{3.751875in}{4.381263in}}{\pgfqpoint{3.741276in}{4.385653in}}{\pgfqpoint{3.730226in}{4.385653in}}%
\pgfpathcurveto{\pgfqpoint{3.719176in}{4.385653in}}{\pgfqpoint{3.708577in}{4.381263in}}{\pgfqpoint{3.700763in}{4.373449in}}%
\pgfpathcurveto{\pgfqpoint{3.692950in}{4.365636in}}{\pgfqpoint{3.688559in}{4.355037in}}{\pgfqpoint{3.688559in}{4.343987in}}%
\pgfpathcurveto{\pgfqpoint{3.688559in}{4.332937in}}{\pgfqpoint{3.692950in}{4.322337in}}{\pgfqpoint{3.700763in}{4.314524in}}%
\pgfpathcurveto{\pgfqpoint{3.708577in}{4.306710in}}{\pgfqpoint{3.719176in}{4.302320in}}{\pgfqpoint{3.730226in}{4.302320in}}%
\pgfpathlineto{\pgfqpoint{3.730226in}{4.302320in}}%
\pgfpathclose%
\pgfusepath{stroke,fill}%
\end{pgfscope}%
\begin{pgfscope}%
\pgfpathrectangle{\pgfqpoint{2.963410in}{2.920818in}}{\pgfqpoint{2.177280in}{2.201755in}}%
\pgfusepath{clip}%
\pgfsetbuttcap%
\pgfsetroundjoin%
\definecolor{currentfill}{rgb}{0.172549,0.627451,0.172549}%
\pgfsetfillcolor{currentfill}%
\pgfsetlinewidth{0.481800pt}%
\definecolor{currentstroke}{rgb}{1.000000,1.000000,1.000000}%
\pgfsetstrokecolor{currentstroke}%
\pgfsetdash{}{0pt}%
\pgfpathmoveto{\pgfqpoint{4.084678in}{4.573723in}}%
\pgfpathcurveto{\pgfqpoint{4.095728in}{4.573723in}}{\pgfqpoint{4.106327in}{4.578113in}}{\pgfqpoint{4.114140in}{4.585927in}}%
\pgfpathcurveto{\pgfqpoint{4.121954in}{4.593740in}}{\pgfqpoint{4.126344in}{4.604339in}}{\pgfqpoint{4.126344in}{4.615389in}}%
\pgfpathcurveto{\pgfqpoint{4.126344in}{4.626440in}}{\pgfqpoint{4.121954in}{4.637039in}}{\pgfqpoint{4.114140in}{4.644852in}}%
\pgfpathcurveto{\pgfqpoint{4.106327in}{4.652666in}}{\pgfqpoint{4.095728in}{4.657056in}}{\pgfqpoint{4.084678in}{4.657056in}}%
\pgfpathcurveto{\pgfqpoint{4.073627in}{4.657056in}}{\pgfqpoint{4.063028in}{4.652666in}}{\pgfqpoint{4.055215in}{4.644852in}}%
\pgfpathcurveto{\pgfqpoint{4.047401in}{4.637039in}}{\pgfqpoint{4.043011in}{4.626440in}}{\pgfqpoint{4.043011in}{4.615389in}}%
\pgfpathcurveto{\pgfqpoint{4.043011in}{4.604339in}}{\pgfqpoint{4.047401in}{4.593740in}}{\pgfqpoint{4.055215in}{4.585927in}}%
\pgfpathcurveto{\pgfqpoint{4.063028in}{4.578113in}}{\pgfqpoint{4.073627in}{4.573723in}}{\pgfqpoint{4.084678in}{4.573723in}}%
\pgfpathlineto{\pgfqpoint{4.084678in}{4.573723in}}%
\pgfpathclose%
\pgfusepath{stroke,fill}%
\end{pgfscope}%
\begin{pgfscope}%
\pgfpathrectangle{\pgfqpoint{2.963410in}{2.920818in}}{\pgfqpoint{2.177280in}{2.201755in}}%
\pgfusepath{clip}%
\pgfsetbuttcap%
\pgfsetroundjoin%
\definecolor{currentfill}{rgb}{0.172549,0.627451,0.172549}%
\pgfsetfillcolor{currentfill}%
\pgfsetlinewidth{0.481800pt}%
\definecolor{currentstroke}{rgb}{1.000000,1.000000,1.000000}%
\pgfsetstrokecolor{currentstroke}%
\pgfsetdash{}{0pt}%
\pgfpathmoveto{\pgfqpoint{4.025602in}{4.675499in}}%
\pgfpathcurveto{\pgfqpoint{4.036652in}{4.675499in}}{\pgfqpoint{4.047251in}{4.679889in}}{\pgfqpoint{4.055065in}{4.687703in}}%
\pgfpathcurveto{\pgfqpoint{4.062879in}{4.695516in}}{\pgfqpoint{4.067269in}{4.706115in}}{\pgfqpoint{4.067269in}{4.717165in}}%
\pgfpathcurveto{\pgfqpoint{4.067269in}{4.728216in}}{\pgfqpoint{4.062879in}{4.738815in}}{\pgfqpoint{4.055065in}{4.746628in}}%
\pgfpathcurveto{\pgfqpoint{4.047251in}{4.754442in}}{\pgfqpoint{4.036652in}{4.758832in}}{\pgfqpoint{4.025602in}{4.758832in}}%
\pgfpathcurveto{\pgfqpoint{4.014552in}{4.758832in}}{\pgfqpoint{4.003953in}{4.754442in}}{\pgfqpoint{3.996139in}{4.746628in}}%
\pgfpathcurveto{\pgfqpoint{3.988326in}{4.738815in}}{\pgfqpoint{3.983936in}{4.728216in}}{\pgfqpoint{3.983936in}{4.717165in}}%
\pgfpathcurveto{\pgfqpoint{3.983936in}{4.706115in}}{\pgfqpoint{3.988326in}{4.695516in}}{\pgfqpoint{3.996139in}{4.687703in}}%
\pgfpathcurveto{\pgfqpoint{4.003953in}{4.679889in}}{\pgfqpoint{4.014552in}{4.675499in}}{\pgfqpoint{4.025602in}{4.675499in}}%
\pgfpathlineto{\pgfqpoint{4.025602in}{4.675499in}}%
\pgfpathclose%
\pgfusepath{stroke,fill}%
\end{pgfscope}%
\begin{pgfscope}%
\pgfpathrectangle{\pgfqpoint{2.963410in}{2.920818in}}{\pgfqpoint{2.177280in}{2.201755in}}%
\pgfusepath{clip}%
\pgfsetbuttcap%
\pgfsetroundjoin%
\definecolor{currentfill}{rgb}{0.172549,0.627451,0.172549}%
\pgfsetfillcolor{currentfill}%
\pgfsetlinewidth{0.481800pt}%
\definecolor{currentstroke}{rgb}{1.000000,1.000000,1.000000}%
\pgfsetstrokecolor{currentstroke}%
\pgfsetdash{}{0pt}%
\pgfpathmoveto{\pgfqpoint{3.789301in}{4.268395in}}%
\pgfpathcurveto{\pgfqpoint{3.800351in}{4.268395in}}{\pgfqpoint{3.810950in}{4.272785in}}{\pgfqpoint{3.818764in}{4.280599in}}%
\pgfpathcurveto{\pgfqpoint{3.826578in}{4.288412in}}{\pgfqpoint{3.830968in}{4.299011in}}{\pgfqpoint{3.830968in}{4.310061in}}%
\pgfpathcurveto{\pgfqpoint{3.830968in}{4.321111in}}{\pgfqpoint{3.826578in}{4.331710in}}{\pgfqpoint{3.818764in}{4.339524in}}%
\pgfpathcurveto{\pgfqpoint{3.810950in}{4.347338in}}{\pgfqpoint{3.800351in}{4.351728in}}{\pgfqpoint{3.789301in}{4.351728in}}%
\pgfpathcurveto{\pgfqpoint{3.778251in}{4.351728in}}{\pgfqpoint{3.767652in}{4.347338in}}{\pgfqpoint{3.759838in}{4.339524in}}%
\pgfpathcurveto{\pgfqpoint{3.752025in}{4.331710in}}{\pgfqpoint{3.747635in}{4.321111in}}{\pgfqpoint{3.747635in}{4.310061in}}%
\pgfpathcurveto{\pgfqpoint{3.747635in}{4.299011in}}{\pgfqpoint{3.752025in}{4.288412in}}{\pgfqpoint{3.759838in}{4.280599in}}%
\pgfpathcurveto{\pgfqpoint{3.767652in}{4.272785in}}{\pgfqpoint{3.778251in}{4.268395in}}{\pgfqpoint{3.789301in}{4.268395in}}%
\pgfpathlineto{\pgfqpoint{3.789301in}{4.268395in}}%
\pgfpathclose%
\pgfusepath{stroke,fill}%
\end{pgfscope}%
\begin{pgfscope}%
\pgfpathrectangle{\pgfqpoint{2.963410in}{2.920818in}}{\pgfqpoint{2.177280in}{2.201755in}}%
\pgfusepath{clip}%
\pgfsetbuttcap%
\pgfsetroundjoin%
\definecolor{currentfill}{rgb}{0.172549,0.627451,0.172549}%
\pgfsetfillcolor{currentfill}%
\pgfsetlinewidth{0.481800pt}%
\definecolor{currentstroke}{rgb}{1.000000,1.000000,1.000000}%
\pgfsetstrokecolor{currentstroke}%
\pgfsetdash{}{0pt}%
\pgfpathmoveto{\pgfqpoint{3.907452in}{4.302320in}}%
\pgfpathcurveto{\pgfqpoint{3.918502in}{4.302320in}}{\pgfqpoint{3.929101in}{4.306710in}}{\pgfqpoint{3.936915in}{4.314524in}}%
\pgfpathcurveto{\pgfqpoint{3.944728in}{4.322337in}}{\pgfqpoint{3.949118in}{4.332937in}}{\pgfqpoint{3.949118in}{4.343987in}}%
\pgfpathcurveto{\pgfqpoint{3.949118in}{4.355037in}}{\pgfqpoint{3.944728in}{4.365636in}}{\pgfqpoint{3.936915in}{4.373449in}}%
\pgfpathcurveto{\pgfqpoint{3.929101in}{4.381263in}}{\pgfqpoint{3.918502in}{4.385653in}}{\pgfqpoint{3.907452in}{4.385653in}}%
\pgfpathcurveto{\pgfqpoint{3.896402in}{4.385653in}}{\pgfqpoint{3.885803in}{4.381263in}}{\pgfqpoint{3.877989in}{4.373449in}}%
\pgfpathcurveto{\pgfqpoint{3.870175in}{4.365636in}}{\pgfqpoint{3.865785in}{4.355037in}}{\pgfqpoint{3.865785in}{4.343987in}}%
\pgfpathcurveto{\pgfqpoint{3.865785in}{4.332937in}}{\pgfqpoint{3.870175in}{4.322337in}}{\pgfqpoint{3.877989in}{4.314524in}}%
\pgfpathcurveto{\pgfqpoint{3.885803in}{4.306710in}}{\pgfqpoint{3.896402in}{4.302320in}}{\pgfqpoint{3.907452in}{4.302320in}}%
\pgfpathlineto{\pgfqpoint{3.907452in}{4.302320in}}%
\pgfpathclose%
\pgfusepath{stroke,fill}%
\end{pgfscope}%
\begin{pgfscope}%
\pgfpathrectangle{\pgfqpoint{2.963410in}{2.920818in}}{\pgfqpoint{2.177280in}{2.201755in}}%
\pgfusepath{clip}%
\pgfsetbuttcap%
\pgfsetroundjoin%
\definecolor{currentfill}{rgb}{0.172549,0.627451,0.172549}%
\pgfsetfillcolor{currentfill}%
\pgfsetlinewidth{0.481800pt}%
\definecolor{currentstroke}{rgb}{1.000000,1.000000,1.000000}%
\pgfsetstrokecolor{currentstroke}%
\pgfsetdash{}{0pt}%
\pgfpathmoveto{\pgfqpoint{3.789301in}{4.539797in}}%
\pgfpathcurveto{\pgfqpoint{3.800351in}{4.539797in}}{\pgfqpoint{3.810950in}{4.544188in}}{\pgfqpoint{3.818764in}{4.552001in}}%
\pgfpathcurveto{\pgfqpoint{3.826578in}{4.559815in}}{\pgfqpoint{3.830968in}{4.570414in}}{\pgfqpoint{3.830968in}{4.581464in}}%
\pgfpathcurveto{\pgfqpoint{3.830968in}{4.592514in}}{\pgfqpoint{3.826578in}{4.603113in}}{\pgfqpoint{3.818764in}{4.610927in}}%
\pgfpathcurveto{\pgfqpoint{3.810950in}{4.618740in}}{\pgfqpoint{3.800351in}{4.623131in}}{\pgfqpoint{3.789301in}{4.623131in}}%
\pgfpathcurveto{\pgfqpoint{3.778251in}{4.623131in}}{\pgfqpoint{3.767652in}{4.618740in}}{\pgfqpoint{3.759838in}{4.610927in}}%
\pgfpathcurveto{\pgfqpoint{3.752025in}{4.603113in}}{\pgfqpoint{3.747635in}{4.592514in}}{\pgfqpoint{3.747635in}{4.581464in}}%
\pgfpathcurveto{\pgfqpoint{3.747635in}{4.570414in}}{\pgfqpoint{3.752025in}{4.559815in}}{\pgfqpoint{3.759838in}{4.552001in}}%
\pgfpathcurveto{\pgfqpoint{3.767652in}{4.544188in}}{\pgfqpoint{3.778251in}{4.539797in}}{\pgfqpoint{3.789301in}{4.539797in}}%
\pgfpathlineto{\pgfqpoint{3.789301in}{4.539797in}}%
\pgfpathclose%
\pgfusepath{stroke,fill}%
\end{pgfscope}%
\begin{pgfscope}%
\pgfpathrectangle{\pgfqpoint{2.963410in}{2.920818in}}{\pgfqpoint{2.177280in}{2.201755in}}%
\pgfusepath{clip}%
\pgfsetbuttcap%
\pgfsetroundjoin%
\definecolor{currentfill}{rgb}{0.172549,0.627451,0.172549}%
\pgfsetfillcolor{currentfill}%
\pgfsetlinewidth{0.481800pt}%
\definecolor{currentstroke}{rgb}{1.000000,1.000000,1.000000}%
\pgfsetstrokecolor{currentstroke}%
\pgfsetdash{}{0pt}%
\pgfpathmoveto{\pgfqpoint{3.907452in}{4.607648in}}%
\pgfpathcurveto{\pgfqpoint{3.918502in}{4.607648in}}{\pgfqpoint{3.929101in}{4.612038in}}{\pgfqpoint{3.936915in}{4.619852in}}%
\pgfpathcurveto{\pgfqpoint{3.944728in}{4.627666in}}{\pgfqpoint{3.949118in}{4.638265in}}{\pgfqpoint{3.949118in}{4.649315in}}%
\pgfpathcurveto{\pgfqpoint{3.949118in}{4.660365in}}{\pgfqpoint{3.944728in}{4.670964in}}{\pgfqpoint{3.936915in}{4.678778in}}%
\pgfpathcurveto{\pgfqpoint{3.929101in}{4.686591in}}{\pgfqpoint{3.918502in}{4.690981in}}{\pgfqpoint{3.907452in}{4.690981in}}%
\pgfpathcurveto{\pgfqpoint{3.896402in}{4.690981in}}{\pgfqpoint{3.885803in}{4.686591in}}{\pgfqpoint{3.877989in}{4.678778in}}%
\pgfpathcurveto{\pgfqpoint{3.870175in}{4.670964in}}{\pgfqpoint{3.865785in}{4.660365in}}{\pgfqpoint{3.865785in}{4.649315in}}%
\pgfpathcurveto{\pgfqpoint{3.865785in}{4.638265in}}{\pgfqpoint{3.870175in}{4.627666in}}{\pgfqpoint{3.877989in}{4.619852in}}%
\pgfpathcurveto{\pgfqpoint{3.885803in}{4.612038in}}{\pgfqpoint{3.896402in}{4.607648in}}{\pgfqpoint{3.907452in}{4.607648in}}%
\pgfpathlineto{\pgfqpoint{3.907452in}{4.607648in}}%
\pgfpathclose%
\pgfusepath{stroke,fill}%
\end{pgfscope}%
\begin{pgfscope}%
\pgfpathrectangle{\pgfqpoint{2.963410in}{2.920818in}}{\pgfqpoint{2.177280in}{2.201755in}}%
\pgfusepath{clip}%
\pgfsetbuttcap%
\pgfsetroundjoin%
\definecolor{currentfill}{rgb}{0.172549,0.627451,0.172549}%
\pgfsetfillcolor{currentfill}%
\pgfsetlinewidth{0.481800pt}%
\definecolor{currentstroke}{rgb}{1.000000,1.000000,1.000000}%
\pgfsetstrokecolor{currentstroke}%
\pgfsetdash{}{0pt}%
\pgfpathmoveto{\pgfqpoint{3.789301in}{4.709424in}}%
\pgfpathcurveto{\pgfqpoint{3.800351in}{4.709424in}}{\pgfqpoint{3.810950in}{4.713814in}}{\pgfqpoint{3.818764in}{4.721628in}}%
\pgfpathcurveto{\pgfqpoint{3.826578in}{4.729442in}}{\pgfqpoint{3.830968in}{4.740041in}}{\pgfqpoint{3.830968in}{4.751091in}}%
\pgfpathcurveto{\pgfqpoint{3.830968in}{4.762141in}}{\pgfqpoint{3.826578in}{4.772740in}}{\pgfqpoint{3.818764in}{4.780554in}}%
\pgfpathcurveto{\pgfqpoint{3.810950in}{4.788367in}}{\pgfqpoint{3.800351in}{4.792757in}}{\pgfqpoint{3.789301in}{4.792757in}}%
\pgfpathcurveto{\pgfqpoint{3.778251in}{4.792757in}}{\pgfqpoint{3.767652in}{4.788367in}}{\pgfqpoint{3.759838in}{4.780554in}}%
\pgfpathcurveto{\pgfqpoint{3.752025in}{4.772740in}}{\pgfqpoint{3.747635in}{4.762141in}}{\pgfqpoint{3.747635in}{4.751091in}}%
\pgfpathcurveto{\pgfqpoint{3.747635in}{4.740041in}}{\pgfqpoint{3.752025in}{4.729442in}}{\pgfqpoint{3.759838in}{4.721628in}}%
\pgfpathcurveto{\pgfqpoint{3.767652in}{4.713814in}}{\pgfqpoint{3.778251in}{4.709424in}}{\pgfqpoint{3.789301in}{4.709424in}}%
\pgfpathlineto{\pgfqpoint{3.789301in}{4.709424in}}%
\pgfpathclose%
\pgfusepath{stroke,fill}%
\end{pgfscope}%
\begin{pgfscope}%
\pgfpathrectangle{\pgfqpoint{2.963410in}{2.920818in}}{\pgfqpoint{2.177280in}{2.201755in}}%
\pgfusepath{clip}%
\pgfsetbuttcap%
\pgfsetroundjoin%
\definecolor{currentfill}{rgb}{0.172549,0.627451,0.172549}%
\pgfsetfillcolor{currentfill}%
\pgfsetlinewidth{0.481800pt}%
\definecolor{currentstroke}{rgb}{1.000000,1.000000,1.000000}%
\pgfsetstrokecolor{currentstroke}%
\pgfsetdash{}{0pt}%
\pgfpathmoveto{\pgfqpoint{4.380054in}{4.811200in}}%
\pgfpathcurveto{\pgfqpoint{4.391104in}{4.811200in}}{\pgfqpoint{4.401703in}{4.815590in}}{\pgfqpoint{4.409517in}{4.823404in}}%
\pgfpathcurveto{\pgfqpoint{4.417330in}{4.831218in}}{\pgfqpoint{4.421721in}{4.841817in}}{\pgfqpoint{4.421721in}{4.852867in}}%
\pgfpathcurveto{\pgfqpoint{4.421721in}{4.863917in}}{\pgfqpoint{4.417330in}{4.874516in}}{\pgfqpoint{4.409517in}{4.882330in}}%
\pgfpathcurveto{\pgfqpoint{4.401703in}{4.890143in}}{\pgfqpoint{4.391104in}{4.894534in}}{\pgfqpoint{4.380054in}{4.894534in}}%
\pgfpathcurveto{\pgfqpoint{4.369004in}{4.894534in}}{\pgfqpoint{4.358405in}{4.890143in}}{\pgfqpoint{4.350591in}{4.882330in}}%
\pgfpathcurveto{\pgfqpoint{4.342777in}{4.874516in}}{\pgfqpoint{4.338387in}{4.863917in}}{\pgfqpoint{4.338387in}{4.852867in}}%
\pgfpathcurveto{\pgfqpoint{4.338387in}{4.841817in}}{\pgfqpoint{4.342777in}{4.831218in}}{\pgfqpoint{4.350591in}{4.823404in}}%
\pgfpathcurveto{\pgfqpoint{4.358405in}{4.815590in}}{\pgfqpoint{4.369004in}{4.811200in}}{\pgfqpoint{4.380054in}{4.811200in}}%
\pgfpathlineto{\pgfqpoint{4.380054in}{4.811200in}}%
\pgfpathclose%
\pgfusepath{stroke,fill}%
\end{pgfscope}%
\begin{pgfscope}%
\pgfpathrectangle{\pgfqpoint{2.963410in}{2.920818in}}{\pgfqpoint{2.177280in}{2.201755in}}%
\pgfusepath{clip}%
\pgfsetbuttcap%
\pgfsetroundjoin%
\definecolor{currentfill}{rgb}{0.172549,0.627451,0.172549}%
\pgfsetfillcolor{currentfill}%
\pgfsetlinewidth{0.481800pt}%
\definecolor{currentstroke}{rgb}{1.000000,1.000000,1.000000}%
\pgfsetstrokecolor{currentstroke}%
\pgfsetdash{}{0pt}%
\pgfpathmoveto{\pgfqpoint{3.789301in}{4.539797in}}%
\pgfpathcurveto{\pgfqpoint{3.800351in}{4.539797in}}{\pgfqpoint{3.810950in}{4.544188in}}{\pgfqpoint{3.818764in}{4.552001in}}%
\pgfpathcurveto{\pgfqpoint{3.826578in}{4.559815in}}{\pgfqpoint{3.830968in}{4.570414in}}{\pgfqpoint{3.830968in}{4.581464in}}%
\pgfpathcurveto{\pgfqpoint{3.830968in}{4.592514in}}{\pgfqpoint{3.826578in}{4.603113in}}{\pgfqpoint{3.818764in}{4.610927in}}%
\pgfpathcurveto{\pgfqpoint{3.810950in}{4.618740in}}{\pgfqpoint{3.800351in}{4.623131in}}{\pgfqpoint{3.789301in}{4.623131in}}%
\pgfpathcurveto{\pgfqpoint{3.778251in}{4.623131in}}{\pgfqpoint{3.767652in}{4.618740in}}{\pgfqpoint{3.759838in}{4.610927in}}%
\pgfpathcurveto{\pgfqpoint{3.752025in}{4.603113in}}{\pgfqpoint{3.747635in}{4.592514in}}{\pgfqpoint{3.747635in}{4.581464in}}%
\pgfpathcurveto{\pgfqpoint{3.747635in}{4.570414in}}{\pgfqpoint{3.752025in}{4.559815in}}{\pgfqpoint{3.759838in}{4.552001in}}%
\pgfpathcurveto{\pgfqpoint{3.767652in}{4.544188in}}{\pgfqpoint{3.778251in}{4.539797in}}{\pgfqpoint{3.789301in}{4.539797in}}%
\pgfpathlineto{\pgfqpoint{3.789301in}{4.539797in}}%
\pgfpathclose%
\pgfusepath{stroke,fill}%
\end{pgfscope}%
\begin{pgfscope}%
\pgfpathrectangle{\pgfqpoint{2.963410in}{2.920818in}}{\pgfqpoint{2.177280in}{2.201755in}}%
\pgfusepath{clip}%
\pgfsetbuttcap%
\pgfsetroundjoin%
\definecolor{currentfill}{rgb}{0.172549,0.627451,0.172549}%
\pgfsetfillcolor{currentfill}%
\pgfsetlinewidth{0.481800pt}%
\definecolor{currentstroke}{rgb}{1.000000,1.000000,1.000000}%
\pgfsetstrokecolor{currentstroke}%
\pgfsetdash{}{0pt}%
\pgfpathmoveto{\pgfqpoint{3.789301in}{4.370171in}}%
\pgfpathcurveto{\pgfqpoint{3.800351in}{4.370171in}}{\pgfqpoint{3.810950in}{4.374561in}}{\pgfqpoint{3.818764in}{4.382375in}}%
\pgfpathcurveto{\pgfqpoint{3.826578in}{4.390188in}}{\pgfqpoint{3.830968in}{4.400787in}}{\pgfqpoint{3.830968in}{4.411837in}}%
\pgfpathcurveto{\pgfqpoint{3.830968in}{4.422887in}}{\pgfqpoint{3.826578in}{4.433486in}}{\pgfqpoint{3.818764in}{4.441300in}}%
\pgfpathcurveto{\pgfqpoint{3.810950in}{4.449114in}}{\pgfqpoint{3.800351in}{4.453504in}}{\pgfqpoint{3.789301in}{4.453504in}}%
\pgfpathcurveto{\pgfqpoint{3.778251in}{4.453504in}}{\pgfqpoint{3.767652in}{4.449114in}}{\pgfqpoint{3.759838in}{4.441300in}}%
\pgfpathcurveto{\pgfqpoint{3.752025in}{4.433486in}}{\pgfqpoint{3.747635in}{4.422887in}}{\pgfqpoint{3.747635in}{4.411837in}}%
\pgfpathcurveto{\pgfqpoint{3.747635in}{4.400787in}}{\pgfqpoint{3.752025in}{4.390188in}}{\pgfqpoint{3.759838in}{4.382375in}}%
\pgfpathcurveto{\pgfqpoint{3.767652in}{4.374561in}}{\pgfqpoint{3.778251in}{4.370171in}}{\pgfqpoint{3.789301in}{4.370171in}}%
\pgfpathlineto{\pgfqpoint{3.789301in}{4.370171in}}%
\pgfpathclose%
\pgfusepath{stroke,fill}%
\end{pgfscope}%
\begin{pgfscope}%
\pgfpathrectangle{\pgfqpoint{2.963410in}{2.920818in}}{\pgfqpoint{2.177280in}{2.201755in}}%
\pgfusepath{clip}%
\pgfsetbuttcap%
\pgfsetroundjoin%
\definecolor{currentfill}{rgb}{0.172549,0.627451,0.172549}%
\pgfsetfillcolor{currentfill}%
\pgfsetlinewidth{0.481800pt}%
\definecolor{currentstroke}{rgb}{1.000000,1.000000,1.000000}%
\pgfsetstrokecolor{currentstroke}%
\pgfsetdash{}{0pt}%
\pgfpathmoveto{\pgfqpoint{3.671151in}{4.539797in}}%
\pgfpathcurveto{\pgfqpoint{3.682201in}{4.539797in}}{\pgfqpoint{3.692800in}{4.544188in}}{\pgfqpoint{3.700613in}{4.552001in}}%
\pgfpathcurveto{\pgfqpoint{3.708427in}{4.559815in}}{\pgfqpoint{3.712817in}{4.570414in}}{\pgfqpoint{3.712817in}{4.581464in}}%
\pgfpathcurveto{\pgfqpoint{3.712817in}{4.592514in}}{\pgfqpoint{3.708427in}{4.603113in}}{\pgfqpoint{3.700613in}{4.610927in}}%
\pgfpathcurveto{\pgfqpoint{3.692800in}{4.618740in}}{\pgfqpoint{3.682201in}{4.623131in}}{\pgfqpoint{3.671151in}{4.623131in}}%
\pgfpathcurveto{\pgfqpoint{3.660101in}{4.623131in}}{\pgfqpoint{3.649501in}{4.618740in}}{\pgfqpoint{3.641688in}{4.610927in}}%
\pgfpathcurveto{\pgfqpoint{3.633874in}{4.603113in}}{\pgfqpoint{3.629484in}{4.592514in}}{\pgfqpoint{3.629484in}{4.581464in}}%
\pgfpathcurveto{\pgfqpoint{3.629484in}{4.570414in}}{\pgfqpoint{3.633874in}{4.559815in}}{\pgfqpoint{3.641688in}{4.552001in}}%
\pgfpathcurveto{\pgfqpoint{3.649501in}{4.544188in}}{\pgfqpoint{3.660101in}{4.539797in}}{\pgfqpoint{3.671151in}{4.539797in}}%
\pgfpathlineto{\pgfqpoint{3.671151in}{4.539797in}}%
\pgfpathclose%
\pgfusepath{stroke,fill}%
\end{pgfscope}%
\begin{pgfscope}%
\pgfpathrectangle{\pgfqpoint{2.963410in}{2.920818in}}{\pgfqpoint{2.177280in}{2.201755in}}%
\pgfusepath{clip}%
\pgfsetbuttcap%
\pgfsetroundjoin%
\definecolor{currentfill}{rgb}{0.172549,0.627451,0.172549}%
\pgfsetfillcolor{currentfill}%
\pgfsetlinewidth{0.481800pt}%
\definecolor{currentstroke}{rgb}{1.000000,1.000000,1.000000}%
\pgfsetstrokecolor{currentstroke}%
\pgfsetdash{}{0pt}%
\pgfpathmoveto{\pgfqpoint{3.907452in}{4.709424in}}%
\pgfpathcurveto{\pgfqpoint{3.918502in}{4.709424in}}{\pgfqpoint{3.929101in}{4.713814in}}{\pgfqpoint{3.936915in}{4.721628in}}%
\pgfpathcurveto{\pgfqpoint{3.944728in}{4.729442in}}{\pgfqpoint{3.949118in}{4.740041in}}{\pgfqpoint{3.949118in}{4.751091in}}%
\pgfpathcurveto{\pgfqpoint{3.949118in}{4.762141in}}{\pgfqpoint{3.944728in}{4.772740in}}{\pgfqpoint{3.936915in}{4.780554in}}%
\pgfpathcurveto{\pgfqpoint{3.929101in}{4.788367in}}{\pgfqpoint{3.918502in}{4.792757in}}{\pgfqpoint{3.907452in}{4.792757in}}%
\pgfpathcurveto{\pgfqpoint{3.896402in}{4.792757in}}{\pgfqpoint{3.885803in}{4.788367in}}{\pgfqpoint{3.877989in}{4.780554in}}%
\pgfpathcurveto{\pgfqpoint{3.870175in}{4.772740in}}{\pgfqpoint{3.865785in}{4.762141in}}{\pgfqpoint{3.865785in}{4.751091in}}%
\pgfpathcurveto{\pgfqpoint{3.865785in}{4.740041in}}{\pgfqpoint{3.870175in}{4.729442in}}{\pgfqpoint{3.877989in}{4.721628in}}%
\pgfpathcurveto{\pgfqpoint{3.885803in}{4.713814in}}{\pgfqpoint{3.896402in}{4.709424in}}{\pgfqpoint{3.907452in}{4.709424in}}%
\pgfpathlineto{\pgfqpoint{3.907452in}{4.709424in}}%
\pgfpathclose%
\pgfusepath{stroke,fill}%
\end{pgfscope}%
\begin{pgfscope}%
\pgfpathrectangle{\pgfqpoint{2.963410in}{2.920818in}}{\pgfqpoint{2.177280in}{2.201755in}}%
\pgfusepath{clip}%
\pgfsetbuttcap%
\pgfsetroundjoin%
\definecolor{currentfill}{rgb}{0.172549,0.627451,0.172549}%
\pgfsetfillcolor{currentfill}%
\pgfsetlinewidth{0.481800pt}%
\definecolor{currentstroke}{rgb}{1.000000,1.000000,1.000000}%
\pgfsetstrokecolor{currentstroke}%
\pgfsetdash{}{0pt}%
\pgfpathmoveto{\pgfqpoint{4.143753in}{4.539797in}}%
\pgfpathcurveto{\pgfqpoint{4.154803in}{4.539797in}}{\pgfqpoint{4.165402in}{4.544188in}}{\pgfqpoint{4.173216in}{4.552001in}}%
\pgfpathcurveto{\pgfqpoint{4.181029in}{4.559815in}}{\pgfqpoint{4.185419in}{4.570414in}}{\pgfqpoint{4.185419in}{4.581464in}}%
\pgfpathcurveto{\pgfqpoint{4.185419in}{4.592514in}}{\pgfqpoint{4.181029in}{4.603113in}}{\pgfqpoint{4.173216in}{4.610927in}}%
\pgfpathcurveto{\pgfqpoint{4.165402in}{4.618740in}}{\pgfqpoint{4.154803in}{4.623131in}}{\pgfqpoint{4.143753in}{4.623131in}}%
\pgfpathcurveto{\pgfqpoint{4.132703in}{4.623131in}}{\pgfqpoint{4.122104in}{4.618740in}}{\pgfqpoint{4.114290in}{4.610927in}}%
\pgfpathcurveto{\pgfqpoint{4.106476in}{4.603113in}}{\pgfqpoint{4.102086in}{4.592514in}}{\pgfqpoint{4.102086in}{4.581464in}}%
\pgfpathcurveto{\pgfqpoint{4.102086in}{4.570414in}}{\pgfqpoint{4.106476in}{4.559815in}}{\pgfqpoint{4.114290in}{4.552001in}}%
\pgfpathcurveto{\pgfqpoint{4.122104in}{4.544188in}}{\pgfqpoint{4.132703in}{4.539797in}}{\pgfqpoint{4.143753in}{4.539797in}}%
\pgfpathlineto{\pgfqpoint{4.143753in}{4.539797in}}%
\pgfpathclose%
\pgfusepath{stroke,fill}%
\end{pgfscope}%
\begin{pgfscope}%
\pgfpathrectangle{\pgfqpoint{2.963410in}{2.920818in}}{\pgfqpoint{2.177280in}{2.201755in}}%
\pgfusepath{clip}%
\pgfsetbuttcap%
\pgfsetroundjoin%
\definecolor{currentfill}{rgb}{0.172549,0.627451,0.172549}%
\pgfsetfillcolor{currentfill}%
\pgfsetlinewidth{0.481800pt}%
\definecolor{currentstroke}{rgb}{1.000000,1.000000,1.000000}%
\pgfsetstrokecolor{currentstroke}%
\pgfsetdash{}{0pt}%
\pgfpathmoveto{\pgfqpoint{3.966527in}{4.505872in}}%
\pgfpathcurveto{\pgfqpoint{3.977577in}{4.505872in}}{\pgfqpoint{3.988176in}{4.510262in}}{\pgfqpoint{3.995990in}{4.518076in}}%
\pgfpathcurveto{\pgfqpoint{4.003803in}{4.525890in}}{\pgfqpoint{4.008194in}{4.536489in}}{\pgfqpoint{4.008194in}{4.547539in}}%
\pgfpathcurveto{\pgfqpoint{4.008194in}{4.558589in}}{\pgfqpoint{4.003803in}{4.569188in}}{\pgfqpoint{3.995990in}{4.577002in}}%
\pgfpathcurveto{\pgfqpoint{3.988176in}{4.584815in}}{\pgfqpoint{3.977577in}{4.589205in}}{\pgfqpoint{3.966527in}{4.589205in}}%
\pgfpathcurveto{\pgfqpoint{3.955477in}{4.589205in}}{\pgfqpoint{3.944878in}{4.584815in}}{\pgfqpoint{3.937064in}{4.577002in}}%
\pgfpathcurveto{\pgfqpoint{3.929251in}{4.569188in}}{\pgfqpoint{3.924860in}{4.558589in}}{\pgfqpoint{3.924860in}{4.547539in}}%
\pgfpathcurveto{\pgfqpoint{3.924860in}{4.536489in}}{\pgfqpoint{3.929251in}{4.525890in}}{\pgfqpoint{3.937064in}{4.518076in}}%
\pgfpathcurveto{\pgfqpoint{3.944878in}{4.510262in}}{\pgfqpoint{3.955477in}{4.505872in}}{\pgfqpoint{3.966527in}{4.505872in}}%
\pgfpathlineto{\pgfqpoint{3.966527in}{4.505872in}}%
\pgfpathclose%
\pgfusepath{stroke,fill}%
\end{pgfscope}%
\begin{pgfscope}%
\pgfpathrectangle{\pgfqpoint{2.963410in}{2.920818in}}{\pgfqpoint{2.177280in}{2.201755in}}%
\pgfusepath{clip}%
\pgfsetbuttcap%
\pgfsetroundjoin%
\definecolor{currentfill}{rgb}{0.172549,0.627451,0.172549}%
\pgfsetfillcolor{currentfill}%
\pgfsetlinewidth{0.481800pt}%
\definecolor{currentstroke}{rgb}{1.000000,1.000000,1.000000}%
\pgfsetstrokecolor{currentstroke}%
\pgfsetdash{}{0pt}%
\pgfpathmoveto{\pgfqpoint{3.907452in}{4.268395in}}%
\pgfpathcurveto{\pgfqpoint{3.918502in}{4.268395in}}{\pgfqpoint{3.929101in}{4.272785in}}{\pgfqpoint{3.936915in}{4.280599in}}%
\pgfpathcurveto{\pgfqpoint{3.944728in}{4.288412in}}{\pgfqpoint{3.949118in}{4.299011in}}{\pgfqpoint{3.949118in}{4.310061in}}%
\pgfpathcurveto{\pgfqpoint{3.949118in}{4.321111in}}{\pgfqpoint{3.944728in}{4.331710in}}{\pgfqpoint{3.936915in}{4.339524in}}%
\pgfpathcurveto{\pgfqpoint{3.929101in}{4.347338in}}{\pgfqpoint{3.918502in}{4.351728in}}{\pgfqpoint{3.907452in}{4.351728in}}%
\pgfpathcurveto{\pgfqpoint{3.896402in}{4.351728in}}{\pgfqpoint{3.885803in}{4.347338in}}{\pgfqpoint{3.877989in}{4.339524in}}%
\pgfpathcurveto{\pgfqpoint{3.870175in}{4.331710in}}{\pgfqpoint{3.865785in}{4.321111in}}{\pgfqpoint{3.865785in}{4.310061in}}%
\pgfpathcurveto{\pgfqpoint{3.865785in}{4.299011in}}{\pgfqpoint{3.870175in}{4.288412in}}{\pgfqpoint{3.877989in}{4.280599in}}%
\pgfpathcurveto{\pgfqpoint{3.885803in}{4.272785in}}{\pgfqpoint{3.896402in}{4.268395in}}{\pgfqpoint{3.907452in}{4.268395in}}%
\pgfpathlineto{\pgfqpoint{3.907452in}{4.268395in}}%
\pgfpathclose%
\pgfusepath{stroke,fill}%
\end{pgfscope}%
\begin{pgfscope}%
\pgfpathrectangle{\pgfqpoint{2.963410in}{2.920818in}}{\pgfqpoint{2.177280in}{2.201755in}}%
\pgfusepath{clip}%
\pgfsetbuttcap%
\pgfsetroundjoin%
\definecolor{currentfill}{rgb}{0.172549,0.627451,0.172549}%
\pgfsetfillcolor{currentfill}%
\pgfsetlinewidth{0.481800pt}%
\definecolor{currentstroke}{rgb}{1.000000,1.000000,1.000000}%
\pgfsetstrokecolor{currentstroke}%
\pgfsetdash{}{0pt}%
\pgfpathmoveto{\pgfqpoint{3.966527in}{4.471947in}}%
\pgfpathcurveto{\pgfqpoint{3.977577in}{4.471947in}}{\pgfqpoint{3.988176in}{4.476337in}}{\pgfqpoint{3.995990in}{4.484151in}}%
\pgfpathcurveto{\pgfqpoint{4.003803in}{4.491964in}}{\pgfqpoint{4.008194in}{4.502563in}}{\pgfqpoint{4.008194in}{4.513613in}}%
\pgfpathcurveto{\pgfqpoint{4.008194in}{4.524664in}}{\pgfqpoint{4.003803in}{4.535263in}}{\pgfqpoint{3.995990in}{4.543076in}}%
\pgfpathcurveto{\pgfqpoint{3.988176in}{4.550890in}}{\pgfqpoint{3.977577in}{4.555280in}}{\pgfqpoint{3.966527in}{4.555280in}}%
\pgfpathcurveto{\pgfqpoint{3.955477in}{4.555280in}}{\pgfqpoint{3.944878in}{4.550890in}}{\pgfqpoint{3.937064in}{4.543076in}}%
\pgfpathcurveto{\pgfqpoint{3.929251in}{4.535263in}}{\pgfqpoint{3.924860in}{4.524664in}}{\pgfqpoint{3.924860in}{4.513613in}}%
\pgfpathcurveto{\pgfqpoint{3.924860in}{4.502563in}}{\pgfqpoint{3.929251in}{4.491964in}}{\pgfqpoint{3.937064in}{4.484151in}}%
\pgfpathcurveto{\pgfqpoint{3.944878in}{4.476337in}}{\pgfqpoint{3.955477in}{4.471947in}}{\pgfqpoint{3.966527in}{4.471947in}}%
\pgfpathlineto{\pgfqpoint{3.966527in}{4.471947in}}%
\pgfpathclose%
\pgfusepath{stroke,fill}%
\end{pgfscope}%
\begin{pgfscope}%
\pgfpathrectangle{\pgfqpoint{2.963410in}{2.920818in}}{\pgfqpoint{2.177280in}{2.201755in}}%
\pgfusepath{clip}%
\pgfsetbuttcap%
\pgfsetroundjoin%
\definecolor{currentfill}{rgb}{0.172549,0.627451,0.172549}%
\pgfsetfillcolor{currentfill}%
\pgfsetlinewidth{0.481800pt}%
\definecolor{currentstroke}{rgb}{1.000000,1.000000,1.000000}%
\pgfsetstrokecolor{currentstroke}%
\pgfsetdash{}{0pt}%
\pgfpathmoveto{\pgfqpoint{3.966527in}{4.539797in}}%
\pgfpathcurveto{\pgfqpoint{3.977577in}{4.539797in}}{\pgfqpoint{3.988176in}{4.544188in}}{\pgfqpoint{3.995990in}{4.552001in}}%
\pgfpathcurveto{\pgfqpoint{4.003803in}{4.559815in}}{\pgfqpoint{4.008194in}{4.570414in}}{\pgfqpoint{4.008194in}{4.581464in}}%
\pgfpathcurveto{\pgfqpoint{4.008194in}{4.592514in}}{\pgfqpoint{4.003803in}{4.603113in}}{\pgfqpoint{3.995990in}{4.610927in}}%
\pgfpathcurveto{\pgfqpoint{3.988176in}{4.618740in}}{\pgfqpoint{3.977577in}{4.623131in}}{\pgfqpoint{3.966527in}{4.623131in}}%
\pgfpathcurveto{\pgfqpoint{3.955477in}{4.623131in}}{\pgfqpoint{3.944878in}{4.618740in}}{\pgfqpoint{3.937064in}{4.610927in}}%
\pgfpathcurveto{\pgfqpoint{3.929251in}{4.603113in}}{\pgfqpoint{3.924860in}{4.592514in}}{\pgfqpoint{3.924860in}{4.581464in}}%
\pgfpathcurveto{\pgfqpoint{3.924860in}{4.570414in}}{\pgfqpoint{3.929251in}{4.559815in}}{\pgfqpoint{3.937064in}{4.552001in}}%
\pgfpathcurveto{\pgfqpoint{3.944878in}{4.544188in}}{\pgfqpoint{3.955477in}{4.539797in}}{\pgfqpoint{3.966527in}{4.539797in}}%
\pgfpathlineto{\pgfqpoint{3.966527in}{4.539797in}}%
\pgfpathclose%
\pgfusepath{stroke,fill}%
\end{pgfscope}%
\begin{pgfscope}%
\pgfpathrectangle{\pgfqpoint{2.963410in}{2.920818in}}{\pgfqpoint{2.177280in}{2.201755in}}%
\pgfusepath{clip}%
\pgfsetbuttcap%
\pgfsetroundjoin%
\definecolor{currentfill}{rgb}{0.172549,0.627451,0.172549}%
\pgfsetfillcolor{currentfill}%
\pgfsetlinewidth{0.481800pt}%
\definecolor{currentstroke}{rgb}{1.000000,1.000000,1.000000}%
\pgfsetstrokecolor{currentstroke}%
\pgfsetdash{}{0pt}%
\pgfpathmoveto{\pgfqpoint{3.966527in}{4.370171in}}%
\pgfpathcurveto{\pgfqpoint{3.977577in}{4.370171in}}{\pgfqpoint{3.988176in}{4.374561in}}{\pgfqpoint{3.995990in}{4.382375in}}%
\pgfpathcurveto{\pgfqpoint{4.003803in}{4.390188in}}{\pgfqpoint{4.008194in}{4.400787in}}{\pgfqpoint{4.008194in}{4.411837in}}%
\pgfpathcurveto{\pgfqpoint{4.008194in}{4.422887in}}{\pgfqpoint{4.003803in}{4.433486in}}{\pgfqpoint{3.995990in}{4.441300in}}%
\pgfpathcurveto{\pgfqpoint{3.988176in}{4.449114in}}{\pgfqpoint{3.977577in}{4.453504in}}{\pgfqpoint{3.966527in}{4.453504in}}%
\pgfpathcurveto{\pgfqpoint{3.955477in}{4.453504in}}{\pgfqpoint{3.944878in}{4.449114in}}{\pgfqpoint{3.937064in}{4.441300in}}%
\pgfpathcurveto{\pgfqpoint{3.929251in}{4.433486in}}{\pgfqpoint{3.924860in}{4.422887in}}{\pgfqpoint{3.924860in}{4.411837in}}%
\pgfpathcurveto{\pgfqpoint{3.924860in}{4.400787in}}{\pgfqpoint{3.929251in}{4.390188in}}{\pgfqpoint{3.937064in}{4.382375in}}%
\pgfpathcurveto{\pgfqpoint{3.944878in}{4.374561in}}{\pgfqpoint{3.955477in}{4.370171in}}{\pgfqpoint{3.966527in}{4.370171in}}%
\pgfpathlineto{\pgfqpoint{3.966527in}{4.370171in}}%
\pgfpathclose%
\pgfusepath{stroke,fill}%
\end{pgfscope}%
\begin{pgfscope}%
\pgfpathrectangle{\pgfqpoint{2.963410in}{2.920818in}}{\pgfqpoint{2.177280in}{2.201755in}}%
\pgfusepath{clip}%
\pgfsetbuttcap%
\pgfsetroundjoin%
\definecolor{currentfill}{rgb}{0.172549,0.627451,0.172549}%
\pgfsetfillcolor{currentfill}%
\pgfsetlinewidth{0.481800pt}%
\definecolor{currentstroke}{rgb}{1.000000,1.000000,1.000000}%
\pgfsetstrokecolor{currentstroke}%
\pgfsetdash{}{0pt}%
\pgfpathmoveto{\pgfqpoint{3.730226in}{4.370171in}}%
\pgfpathcurveto{\pgfqpoint{3.741276in}{4.370171in}}{\pgfqpoint{3.751875in}{4.374561in}}{\pgfqpoint{3.759689in}{4.382375in}}%
\pgfpathcurveto{\pgfqpoint{3.767502in}{4.390188in}}{\pgfqpoint{3.771893in}{4.400787in}}{\pgfqpoint{3.771893in}{4.411837in}}%
\pgfpathcurveto{\pgfqpoint{3.771893in}{4.422887in}}{\pgfqpoint{3.767502in}{4.433486in}}{\pgfqpoint{3.759689in}{4.441300in}}%
\pgfpathcurveto{\pgfqpoint{3.751875in}{4.449114in}}{\pgfqpoint{3.741276in}{4.453504in}}{\pgfqpoint{3.730226in}{4.453504in}}%
\pgfpathcurveto{\pgfqpoint{3.719176in}{4.453504in}}{\pgfqpoint{3.708577in}{4.449114in}}{\pgfqpoint{3.700763in}{4.441300in}}%
\pgfpathcurveto{\pgfqpoint{3.692950in}{4.433486in}}{\pgfqpoint{3.688559in}{4.422887in}}{\pgfqpoint{3.688559in}{4.411837in}}%
\pgfpathcurveto{\pgfqpoint{3.688559in}{4.400787in}}{\pgfqpoint{3.692950in}{4.390188in}}{\pgfqpoint{3.700763in}{4.382375in}}%
\pgfpathcurveto{\pgfqpoint{3.708577in}{4.374561in}}{\pgfqpoint{3.719176in}{4.370171in}}{\pgfqpoint{3.730226in}{4.370171in}}%
\pgfpathlineto{\pgfqpoint{3.730226in}{4.370171in}}%
\pgfpathclose%
\pgfusepath{stroke,fill}%
\end{pgfscope}%
\begin{pgfscope}%
\pgfpathrectangle{\pgfqpoint{2.963410in}{2.920818in}}{\pgfqpoint{2.177280in}{2.201755in}}%
\pgfusepath{clip}%
\pgfsetbuttcap%
\pgfsetroundjoin%
\definecolor{currentfill}{rgb}{0.172549,0.627451,0.172549}%
\pgfsetfillcolor{currentfill}%
\pgfsetlinewidth{0.481800pt}%
\definecolor{currentstroke}{rgb}{1.000000,1.000000,1.000000}%
\pgfsetstrokecolor{currentstroke}%
\pgfsetdash{}{0pt}%
\pgfpathmoveto{\pgfqpoint{4.025602in}{4.641573in}}%
\pgfpathcurveto{\pgfqpoint{4.036652in}{4.641573in}}{\pgfqpoint{4.047251in}{4.645964in}}{\pgfqpoint{4.055065in}{4.653777in}}%
\pgfpathcurveto{\pgfqpoint{4.062879in}{4.661591in}}{\pgfqpoint{4.067269in}{4.672190in}}{\pgfqpoint{4.067269in}{4.683240in}}%
\pgfpathcurveto{\pgfqpoint{4.067269in}{4.694290in}}{\pgfqpoint{4.062879in}{4.704889in}}{\pgfqpoint{4.055065in}{4.712703in}}%
\pgfpathcurveto{\pgfqpoint{4.047251in}{4.720517in}}{\pgfqpoint{4.036652in}{4.724907in}}{\pgfqpoint{4.025602in}{4.724907in}}%
\pgfpathcurveto{\pgfqpoint{4.014552in}{4.724907in}}{\pgfqpoint{4.003953in}{4.720517in}}{\pgfqpoint{3.996139in}{4.712703in}}%
\pgfpathcurveto{\pgfqpoint{3.988326in}{4.704889in}}{\pgfqpoint{3.983936in}{4.694290in}}{\pgfqpoint{3.983936in}{4.683240in}}%
\pgfpathcurveto{\pgfqpoint{3.983936in}{4.672190in}}{\pgfqpoint{3.988326in}{4.661591in}}{\pgfqpoint{3.996139in}{4.653777in}}%
\pgfpathcurveto{\pgfqpoint{4.003953in}{4.645964in}}{\pgfqpoint{4.014552in}{4.641573in}}{\pgfqpoint{4.025602in}{4.641573in}}%
\pgfpathlineto{\pgfqpoint{4.025602in}{4.641573in}}%
\pgfpathclose%
\pgfusepath{stroke,fill}%
\end{pgfscope}%
\begin{pgfscope}%
\pgfpathrectangle{\pgfqpoint{2.963410in}{2.920818in}}{\pgfqpoint{2.177280in}{2.201755in}}%
\pgfusepath{clip}%
\pgfsetbuttcap%
\pgfsetroundjoin%
\definecolor{currentfill}{rgb}{0.172549,0.627451,0.172549}%
\pgfsetfillcolor{currentfill}%
\pgfsetlinewidth{0.481800pt}%
\definecolor{currentstroke}{rgb}{1.000000,1.000000,1.000000}%
\pgfsetstrokecolor{currentstroke}%
\pgfsetdash{}{0pt}%
\pgfpathmoveto{\pgfqpoint{4.084678in}{4.573723in}}%
\pgfpathcurveto{\pgfqpoint{4.095728in}{4.573723in}}{\pgfqpoint{4.106327in}{4.578113in}}{\pgfqpoint{4.114140in}{4.585927in}}%
\pgfpathcurveto{\pgfqpoint{4.121954in}{4.593740in}}{\pgfqpoint{4.126344in}{4.604339in}}{\pgfqpoint{4.126344in}{4.615389in}}%
\pgfpathcurveto{\pgfqpoint{4.126344in}{4.626440in}}{\pgfqpoint{4.121954in}{4.637039in}}{\pgfqpoint{4.114140in}{4.644852in}}%
\pgfpathcurveto{\pgfqpoint{4.106327in}{4.652666in}}{\pgfqpoint{4.095728in}{4.657056in}}{\pgfqpoint{4.084678in}{4.657056in}}%
\pgfpathcurveto{\pgfqpoint{4.073627in}{4.657056in}}{\pgfqpoint{4.063028in}{4.652666in}}{\pgfqpoint{4.055215in}{4.644852in}}%
\pgfpathcurveto{\pgfqpoint{4.047401in}{4.637039in}}{\pgfqpoint{4.043011in}{4.626440in}}{\pgfqpoint{4.043011in}{4.615389in}}%
\pgfpathcurveto{\pgfqpoint{4.043011in}{4.604339in}}{\pgfqpoint{4.047401in}{4.593740in}}{\pgfqpoint{4.055215in}{4.585927in}}%
\pgfpathcurveto{\pgfqpoint{4.063028in}{4.578113in}}{\pgfqpoint{4.073627in}{4.573723in}}{\pgfqpoint{4.084678in}{4.573723in}}%
\pgfpathlineto{\pgfqpoint{4.084678in}{4.573723in}}%
\pgfpathclose%
\pgfusepath{stroke,fill}%
\end{pgfscope}%
\begin{pgfscope}%
\pgfpathrectangle{\pgfqpoint{2.963410in}{2.920818in}}{\pgfqpoint{2.177280in}{2.201755in}}%
\pgfusepath{clip}%
\pgfsetbuttcap%
\pgfsetroundjoin%
\definecolor{currentfill}{rgb}{0.172549,0.627451,0.172549}%
\pgfsetfillcolor{currentfill}%
\pgfsetlinewidth{0.481800pt}%
\definecolor{currentstroke}{rgb}{1.000000,1.000000,1.000000}%
\pgfsetstrokecolor{currentstroke}%
\pgfsetdash{}{0pt}%
\pgfpathmoveto{\pgfqpoint{3.907452in}{4.404096in}}%
\pgfpathcurveto{\pgfqpoint{3.918502in}{4.404096in}}{\pgfqpoint{3.929101in}{4.408486in}}{\pgfqpoint{3.936915in}{4.416300in}}%
\pgfpathcurveto{\pgfqpoint{3.944728in}{4.424114in}}{\pgfqpoint{3.949118in}{4.434713in}}{\pgfqpoint{3.949118in}{4.445763in}}%
\pgfpathcurveto{\pgfqpoint{3.949118in}{4.456813in}}{\pgfqpoint{3.944728in}{4.467412in}}{\pgfqpoint{3.936915in}{4.475225in}}%
\pgfpathcurveto{\pgfqpoint{3.929101in}{4.483039in}}{\pgfqpoint{3.918502in}{4.487429in}}{\pgfqpoint{3.907452in}{4.487429in}}%
\pgfpathcurveto{\pgfqpoint{3.896402in}{4.487429in}}{\pgfqpoint{3.885803in}{4.483039in}}{\pgfqpoint{3.877989in}{4.475225in}}%
\pgfpathcurveto{\pgfqpoint{3.870175in}{4.467412in}}{\pgfqpoint{3.865785in}{4.456813in}}{\pgfqpoint{3.865785in}{4.445763in}}%
\pgfpathcurveto{\pgfqpoint{3.865785in}{4.434713in}}{\pgfqpoint{3.870175in}{4.424114in}}{\pgfqpoint{3.877989in}{4.416300in}}%
\pgfpathcurveto{\pgfqpoint{3.885803in}{4.408486in}}{\pgfqpoint{3.896402in}{4.404096in}}{\pgfqpoint{3.907452in}{4.404096in}}%
\pgfpathlineto{\pgfqpoint{3.907452in}{4.404096in}}%
\pgfpathclose%
\pgfusepath{stroke,fill}%
\end{pgfscope}%
\begin{pgfscope}%
\pgfpathrectangle{\pgfqpoint{2.963410in}{2.920818in}}{\pgfqpoint{2.177280in}{2.201755in}}%
\pgfusepath{clip}%
\pgfsetbuttcap%
\pgfsetroundjoin%
\definecolor{currentfill}{rgb}{0.172549,0.627451,0.172549}%
\pgfsetfillcolor{currentfill}%
\pgfsetlinewidth{0.481800pt}%
\definecolor{currentstroke}{rgb}{1.000000,1.000000,1.000000}%
\pgfsetstrokecolor{currentstroke}%
\pgfsetdash{}{0pt}%
\pgfpathmoveto{\pgfqpoint{3.612075in}{4.336245in}}%
\pgfpathcurveto{\pgfqpoint{3.623126in}{4.336245in}}{\pgfqpoint{3.633725in}{4.340636in}}{\pgfqpoint{3.641538in}{4.348449in}}%
\pgfpathcurveto{\pgfqpoint{3.649352in}{4.356263in}}{\pgfqpoint{3.653742in}{4.366862in}}{\pgfqpoint{3.653742in}{4.377912in}}%
\pgfpathcurveto{\pgfqpoint{3.653742in}{4.388962in}}{\pgfqpoint{3.649352in}{4.399561in}}{\pgfqpoint{3.641538in}{4.407375in}}%
\pgfpathcurveto{\pgfqpoint{3.633725in}{4.415188in}}{\pgfqpoint{3.623126in}{4.419579in}}{\pgfqpoint{3.612075in}{4.419579in}}%
\pgfpathcurveto{\pgfqpoint{3.601025in}{4.419579in}}{\pgfqpoint{3.590426in}{4.415188in}}{\pgfqpoint{3.582613in}{4.407375in}}%
\pgfpathcurveto{\pgfqpoint{3.574799in}{4.399561in}}{\pgfqpoint{3.570409in}{4.388962in}}{\pgfqpoint{3.570409in}{4.377912in}}%
\pgfpathcurveto{\pgfqpoint{3.570409in}{4.366862in}}{\pgfqpoint{3.574799in}{4.356263in}}{\pgfqpoint{3.582613in}{4.348449in}}%
\pgfpathcurveto{\pgfqpoint{3.590426in}{4.340636in}}{\pgfqpoint{3.601025in}{4.336245in}}{\pgfqpoint{3.612075in}{4.336245in}}%
\pgfpathlineto{\pgfqpoint{3.612075in}{4.336245in}}%
\pgfpathclose%
\pgfusepath{stroke,fill}%
\end{pgfscope}%
\begin{pgfscope}%
\pgfpathrectangle{\pgfqpoint{2.963410in}{2.920818in}}{\pgfqpoint{2.177280in}{2.201755in}}%
\pgfusepath{clip}%
\pgfsetbuttcap%
\pgfsetroundjoin%
\definecolor{currentfill}{rgb}{0.172549,0.627451,0.172549}%
\pgfsetfillcolor{currentfill}%
\pgfsetlinewidth{0.481800pt}%
\definecolor{currentstroke}{rgb}{1.000000,1.000000,1.000000}%
\pgfsetstrokecolor{currentstroke}%
\pgfsetdash{}{0pt}%
\pgfpathmoveto{\pgfqpoint{3.907452in}{4.404096in}}%
\pgfpathcurveto{\pgfqpoint{3.918502in}{4.404096in}}{\pgfqpoint{3.929101in}{4.408486in}}{\pgfqpoint{3.936915in}{4.416300in}}%
\pgfpathcurveto{\pgfqpoint{3.944728in}{4.424114in}}{\pgfqpoint{3.949118in}{4.434713in}}{\pgfqpoint{3.949118in}{4.445763in}}%
\pgfpathcurveto{\pgfqpoint{3.949118in}{4.456813in}}{\pgfqpoint{3.944728in}{4.467412in}}{\pgfqpoint{3.936915in}{4.475225in}}%
\pgfpathcurveto{\pgfqpoint{3.929101in}{4.483039in}}{\pgfqpoint{3.918502in}{4.487429in}}{\pgfqpoint{3.907452in}{4.487429in}}%
\pgfpathcurveto{\pgfqpoint{3.896402in}{4.487429in}}{\pgfqpoint{3.885803in}{4.483039in}}{\pgfqpoint{3.877989in}{4.475225in}}%
\pgfpathcurveto{\pgfqpoint{3.870175in}{4.467412in}}{\pgfqpoint{3.865785in}{4.456813in}}{\pgfqpoint{3.865785in}{4.445763in}}%
\pgfpathcurveto{\pgfqpoint{3.865785in}{4.434713in}}{\pgfqpoint{3.870175in}{4.424114in}}{\pgfqpoint{3.877989in}{4.416300in}}%
\pgfpathcurveto{\pgfqpoint{3.885803in}{4.408486in}}{\pgfqpoint{3.896402in}{4.404096in}}{\pgfqpoint{3.907452in}{4.404096in}}%
\pgfpathlineto{\pgfqpoint{3.907452in}{4.404096in}}%
\pgfpathclose%
\pgfusepath{stroke,fill}%
\end{pgfscope}%
\begin{pgfscope}%
\pgfpathrectangle{\pgfqpoint{2.963410in}{2.920818in}}{\pgfqpoint{2.177280in}{2.201755in}}%
\pgfusepath{clip}%
\pgfsetbuttcap%
\pgfsetroundjoin%
\definecolor{currentfill}{rgb}{0.172549,0.627451,0.172549}%
\pgfsetfillcolor{currentfill}%
\pgfsetlinewidth{0.481800pt}%
\definecolor{currentstroke}{rgb}{1.000000,1.000000,1.000000}%
\pgfsetstrokecolor{currentstroke}%
\pgfsetdash{}{0pt}%
\pgfpathmoveto{\pgfqpoint{4.143753in}{4.471947in}}%
\pgfpathcurveto{\pgfqpoint{4.154803in}{4.471947in}}{\pgfqpoint{4.165402in}{4.476337in}}{\pgfqpoint{4.173216in}{4.484151in}}%
\pgfpathcurveto{\pgfqpoint{4.181029in}{4.491964in}}{\pgfqpoint{4.185419in}{4.502563in}}{\pgfqpoint{4.185419in}{4.513613in}}%
\pgfpathcurveto{\pgfqpoint{4.185419in}{4.524664in}}{\pgfqpoint{4.181029in}{4.535263in}}{\pgfqpoint{4.173216in}{4.543076in}}%
\pgfpathcurveto{\pgfqpoint{4.165402in}{4.550890in}}{\pgfqpoint{4.154803in}{4.555280in}}{\pgfqpoint{4.143753in}{4.555280in}}%
\pgfpathcurveto{\pgfqpoint{4.132703in}{4.555280in}}{\pgfqpoint{4.122104in}{4.550890in}}{\pgfqpoint{4.114290in}{4.543076in}}%
\pgfpathcurveto{\pgfqpoint{4.106476in}{4.535263in}}{\pgfqpoint{4.102086in}{4.524664in}}{\pgfqpoint{4.102086in}{4.513613in}}%
\pgfpathcurveto{\pgfqpoint{4.102086in}{4.502563in}}{\pgfqpoint{4.106476in}{4.491964in}}{\pgfqpoint{4.114290in}{4.484151in}}%
\pgfpathcurveto{\pgfqpoint{4.122104in}{4.476337in}}{\pgfqpoint{4.132703in}{4.471947in}}{\pgfqpoint{4.143753in}{4.471947in}}%
\pgfpathlineto{\pgfqpoint{4.143753in}{4.471947in}}%
\pgfpathclose%
\pgfusepath{stroke,fill}%
\end{pgfscope}%
\begin{pgfscope}%
\pgfpathrectangle{\pgfqpoint{2.963410in}{2.920818in}}{\pgfqpoint{2.177280in}{2.201755in}}%
\pgfusepath{clip}%
\pgfsetbuttcap%
\pgfsetroundjoin%
\definecolor{currentfill}{rgb}{0.172549,0.627451,0.172549}%
\pgfsetfillcolor{currentfill}%
\pgfsetlinewidth{0.481800pt}%
\definecolor{currentstroke}{rgb}{1.000000,1.000000,1.000000}%
\pgfsetstrokecolor{currentstroke}%
\pgfsetdash{}{0pt}%
\pgfpathmoveto{\pgfqpoint{3.907452in}{4.370171in}}%
\pgfpathcurveto{\pgfqpoint{3.918502in}{4.370171in}}{\pgfqpoint{3.929101in}{4.374561in}}{\pgfqpoint{3.936915in}{4.382375in}}%
\pgfpathcurveto{\pgfqpoint{3.944728in}{4.390188in}}{\pgfqpoint{3.949118in}{4.400787in}}{\pgfqpoint{3.949118in}{4.411837in}}%
\pgfpathcurveto{\pgfqpoint{3.949118in}{4.422887in}}{\pgfqpoint{3.944728in}{4.433486in}}{\pgfqpoint{3.936915in}{4.441300in}}%
\pgfpathcurveto{\pgfqpoint{3.929101in}{4.449114in}}{\pgfqpoint{3.918502in}{4.453504in}}{\pgfqpoint{3.907452in}{4.453504in}}%
\pgfpathcurveto{\pgfqpoint{3.896402in}{4.453504in}}{\pgfqpoint{3.885803in}{4.449114in}}{\pgfqpoint{3.877989in}{4.441300in}}%
\pgfpathcurveto{\pgfqpoint{3.870175in}{4.433486in}}{\pgfqpoint{3.865785in}{4.422887in}}{\pgfqpoint{3.865785in}{4.411837in}}%
\pgfpathcurveto{\pgfqpoint{3.865785in}{4.400787in}}{\pgfqpoint{3.870175in}{4.390188in}}{\pgfqpoint{3.877989in}{4.382375in}}%
\pgfpathcurveto{\pgfqpoint{3.885803in}{4.374561in}}{\pgfqpoint{3.896402in}{4.370171in}}{\pgfqpoint{3.907452in}{4.370171in}}%
\pgfpathlineto{\pgfqpoint{3.907452in}{4.370171in}}%
\pgfpathclose%
\pgfusepath{stroke,fill}%
\end{pgfscope}%
\begin{pgfscope}%
\pgfpathrectangle{\pgfqpoint{2.963410in}{2.920818in}}{\pgfqpoint{2.177280in}{2.201755in}}%
\pgfusepath{clip}%
\pgfsetbuttcap%
\pgfsetroundjoin%
\definecolor{currentfill}{rgb}{0.121569,0.466667,0.705882}%
\pgfsetfillcolor{currentfill}%
\pgfsetlinewidth{1.003750pt}%
\definecolor{currentstroke}{rgb}{0.121569,0.466667,0.705882}%
\pgfsetstrokecolor{currentstroke}%
\pgfsetdash{}{0pt}%
\pgfsys@defobject{currentmarker}{\pgfqpoint{-0.041667in}{-0.041667in}}{\pgfqpoint{0.041667in}{0.041667in}}{%
\pgfpathmoveto{\pgfqpoint{0.000000in}{-0.041667in}}%
\pgfpathcurveto{\pgfqpoint{0.011050in}{-0.041667in}}{\pgfqpoint{0.021649in}{-0.037276in}}{\pgfqpoint{0.029463in}{-0.029463in}}%
\pgfpathcurveto{\pgfqpoint{0.037276in}{-0.021649in}}{\pgfqpoint{0.041667in}{-0.011050in}}{\pgfqpoint{0.041667in}{0.000000in}}%
\pgfpathcurveto{\pgfqpoint{0.041667in}{0.011050in}}{\pgfqpoint{0.037276in}{0.021649in}}{\pgfqpoint{0.029463in}{0.029463in}}%
\pgfpathcurveto{\pgfqpoint{0.021649in}{0.037276in}}{\pgfqpoint{0.011050in}{0.041667in}}{\pgfqpoint{0.000000in}{0.041667in}}%
\pgfpathcurveto{\pgfqpoint{-0.011050in}{0.041667in}}{\pgfqpoint{-0.021649in}{0.037276in}}{\pgfqpoint{-0.029463in}{0.029463in}}%
\pgfpathcurveto{\pgfqpoint{-0.037276in}{0.021649in}}{\pgfqpoint{-0.041667in}{0.011050in}}{\pgfqpoint{-0.041667in}{0.000000in}}%
\pgfpathcurveto{\pgfqpoint{-0.041667in}{-0.011050in}}{\pgfqpoint{-0.037276in}{-0.021649in}}{\pgfqpoint{-0.029463in}{-0.029463in}}%
\pgfpathcurveto{\pgfqpoint{-0.021649in}{-0.037276in}}{\pgfqpoint{-0.011050in}{-0.041667in}}{\pgfqpoint{0.000000in}{-0.041667in}}%
\pgfpathlineto{\pgfqpoint{0.000000in}{-0.041667in}}%
\pgfpathclose%
\pgfusepath{stroke,fill}%
}%
\end{pgfscope}%
\begin{pgfscope}%
\pgfpathrectangle{\pgfqpoint{2.963410in}{2.920818in}}{\pgfqpoint{2.177280in}{2.201755in}}%
\pgfusepath{clip}%
\pgfsetbuttcap%
\pgfsetroundjoin%
\definecolor{currentfill}{rgb}{1.000000,0.498039,0.054902}%
\pgfsetfillcolor{currentfill}%
\pgfsetlinewidth{1.003750pt}%
\definecolor{currentstroke}{rgb}{1.000000,0.498039,0.054902}%
\pgfsetstrokecolor{currentstroke}%
\pgfsetdash{}{0pt}%
\pgfsys@defobject{currentmarker}{\pgfqpoint{-0.041667in}{-0.041667in}}{\pgfqpoint{0.041667in}{0.041667in}}{%
\pgfpathmoveto{\pgfqpoint{0.000000in}{-0.041667in}}%
\pgfpathcurveto{\pgfqpoint{0.011050in}{-0.041667in}}{\pgfqpoint{0.021649in}{-0.037276in}}{\pgfqpoint{0.029463in}{-0.029463in}}%
\pgfpathcurveto{\pgfqpoint{0.037276in}{-0.021649in}}{\pgfqpoint{0.041667in}{-0.011050in}}{\pgfqpoint{0.041667in}{0.000000in}}%
\pgfpathcurveto{\pgfqpoint{0.041667in}{0.011050in}}{\pgfqpoint{0.037276in}{0.021649in}}{\pgfqpoint{0.029463in}{0.029463in}}%
\pgfpathcurveto{\pgfqpoint{0.021649in}{0.037276in}}{\pgfqpoint{0.011050in}{0.041667in}}{\pgfqpoint{0.000000in}{0.041667in}}%
\pgfpathcurveto{\pgfqpoint{-0.011050in}{0.041667in}}{\pgfqpoint{-0.021649in}{0.037276in}}{\pgfqpoint{-0.029463in}{0.029463in}}%
\pgfpathcurveto{\pgfqpoint{-0.037276in}{0.021649in}}{\pgfqpoint{-0.041667in}{0.011050in}}{\pgfqpoint{-0.041667in}{0.000000in}}%
\pgfpathcurveto{\pgfqpoint{-0.041667in}{-0.011050in}}{\pgfqpoint{-0.037276in}{-0.021649in}}{\pgfqpoint{-0.029463in}{-0.029463in}}%
\pgfpathcurveto{\pgfqpoint{-0.021649in}{-0.037276in}}{\pgfqpoint{-0.011050in}{-0.041667in}}{\pgfqpoint{0.000000in}{-0.041667in}}%
\pgfpathlineto{\pgfqpoint{0.000000in}{-0.041667in}}%
\pgfpathclose%
\pgfusepath{stroke,fill}%
}%
\end{pgfscope}%
\begin{pgfscope}%
\pgfpathrectangle{\pgfqpoint{2.963410in}{2.920818in}}{\pgfqpoint{2.177280in}{2.201755in}}%
\pgfusepath{clip}%
\pgfsetbuttcap%
\pgfsetroundjoin%
\definecolor{currentfill}{rgb}{0.172549,0.627451,0.172549}%
\pgfsetfillcolor{currentfill}%
\pgfsetlinewidth{1.003750pt}%
\definecolor{currentstroke}{rgb}{0.172549,0.627451,0.172549}%
\pgfsetstrokecolor{currentstroke}%
\pgfsetdash{}{0pt}%
\pgfsys@defobject{currentmarker}{\pgfqpoint{-0.041667in}{-0.041667in}}{\pgfqpoint{0.041667in}{0.041667in}}{%
\pgfpathmoveto{\pgfqpoint{0.000000in}{-0.041667in}}%
\pgfpathcurveto{\pgfqpoint{0.011050in}{-0.041667in}}{\pgfqpoint{0.021649in}{-0.037276in}}{\pgfqpoint{0.029463in}{-0.029463in}}%
\pgfpathcurveto{\pgfqpoint{0.037276in}{-0.021649in}}{\pgfqpoint{0.041667in}{-0.011050in}}{\pgfqpoint{0.041667in}{0.000000in}}%
\pgfpathcurveto{\pgfqpoint{0.041667in}{0.011050in}}{\pgfqpoint{0.037276in}{0.021649in}}{\pgfqpoint{0.029463in}{0.029463in}}%
\pgfpathcurveto{\pgfqpoint{0.021649in}{0.037276in}}{\pgfqpoint{0.011050in}{0.041667in}}{\pgfqpoint{0.000000in}{0.041667in}}%
\pgfpathcurveto{\pgfqpoint{-0.011050in}{0.041667in}}{\pgfqpoint{-0.021649in}{0.037276in}}{\pgfqpoint{-0.029463in}{0.029463in}}%
\pgfpathcurveto{\pgfqpoint{-0.037276in}{0.021649in}}{\pgfqpoint{-0.041667in}{0.011050in}}{\pgfqpoint{-0.041667in}{0.000000in}}%
\pgfpathcurveto{\pgfqpoint{-0.041667in}{-0.011050in}}{\pgfqpoint{-0.037276in}{-0.021649in}}{\pgfqpoint{-0.029463in}{-0.029463in}}%
\pgfpathcurveto{\pgfqpoint{-0.021649in}{-0.037276in}}{\pgfqpoint{-0.011050in}{-0.041667in}}{\pgfqpoint{0.000000in}{-0.041667in}}%
\pgfpathlineto{\pgfqpoint{0.000000in}{-0.041667in}}%
\pgfpathclose%
\pgfusepath{stroke,fill}%
}%
\end{pgfscope}%
\begin{pgfscope}%
\pgfsetbuttcap%
\pgfsetroundjoin%
\definecolor{currentfill}{rgb}{0.000000,0.000000,0.000000}%
\pgfsetfillcolor{currentfill}%
\pgfsetlinewidth{0.803000pt}%
\definecolor{currentstroke}{rgb}{0.000000,0.000000,0.000000}%
\pgfsetstrokecolor{currentstroke}%
\pgfsetdash{}{0pt}%
\pgfsys@defobject{currentmarker}{\pgfqpoint{0.000000in}{-0.048611in}}{\pgfqpoint{0.000000in}{0.000000in}}{%
\pgfpathmoveto{\pgfqpoint{0.000000in}{0.000000in}}%
\pgfpathlineto{\pgfqpoint{0.000000in}{-0.048611in}}%
\pgfusepath{stroke,fill}%
}%
\begin{pgfscope}%
\pgfsys@transformshift{3.316699in}{2.920818in}%
\pgfsys@useobject{currentmarker}{}%
\end{pgfscope}%
\end{pgfscope}%
\begin{pgfscope}%
\pgfsetbuttcap%
\pgfsetroundjoin%
\definecolor{currentfill}{rgb}{0.000000,0.000000,0.000000}%
\pgfsetfillcolor{currentfill}%
\pgfsetlinewidth{0.803000pt}%
\definecolor{currentstroke}{rgb}{0.000000,0.000000,0.000000}%
\pgfsetstrokecolor{currentstroke}%
\pgfsetdash{}{0pt}%
\pgfsys@defobject{currentmarker}{\pgfqpoint{0.000000in}{-0.048611in}}{\pgfqpoint{0.000000in}{0.000000in}}{%
\pgfpathmoveto{\pgfqpoint{0.000000in}{0.000000in}}%
\pgfpathlineto{\pgfqpoint{0.000000in}{-0.048611in}}%
\pgfusepath{stroke,fill}%
}%
\begin{pgfscope}%
\pgfsys@transformshift{3.907452in}{2.920818in}%
\pgfsys@useobject{currentmarker}{}%
\end{pgfscope}%
\end{pgfscope}%
\begin{pgfscope}%
\pgfsetbuttcap%
\pgfsetroundjoin%
\definecolor{currentfill}{rgb}{0.000000,0.000000,0.000000}%
\pgfsetfillcolor{currentfill}%
\pgfsetlinewidth{0.803000pt}%
\definecolor{currentstroke}{rgb}{0.000000,0.000000,0.000000}%
\pgfsetstrokecolor{currentstroke}%
\pgfsetdash{}{0pt}%
\pgfsys@defobject{currentmarker}{\pgfqpoint{0.000000in}{-0.048611in}}{\pgfqpoint{0.000000in}{0.000000in}}{%
\pgfpathmoveto{\pgfqpoint{0.000000in}{0.000000in}}%
\pgfpathlineto{\pgfqpoint{0.000000in}{-0.048611in}}%
\pgfusepath{stroke,fill}%
}%
\begin{pgfscope}%
\pgfsys@transformshift{4.498204in}{2.920818in}%
\pgfsys@useobject{currentmarker}{}%
\end{pgfscope}%
\end{pgfscope}%
\begin{pgfscope}%
\pgfsetbuttcap%
\pgfsetroundjoin%
\definecolor{currentfill}{rgb}{0.000000,0.000000,0.000000}%
\pgfsetfillcolor{currentfill}%
\pgfsetlinewidth{0.803000pt}%
\definecolor{currentstroke}{rgb}{0.000000,0.000000,0.000000}%
\pgfsetstrokecolor{currentstroke}%
\pgfsetdash{}{0pt}%
\pgfsys@defobject{currentmarker}{\pgfqpoint{0.000000in}{-0.048611in}}{\pgfqpoint{0.000000in}{0.000000in}}{%
\pgfpathmoveto{\pgfqpoint{0.000000in}{0.000000in}}%
\pgfpathlineto{\pgfqpoint{0.000000in}{-0.048611in}}%
\pgfusepath{stroke,fill}%
}%
\begin{pgfscope}%
\pgfsys@transformshift{5.088957in}{2.920818in}%
\pgfsys@useobject{currentmarker}{}%
\end{pgfscope}%
\end{pgfscope}%
\begin{pgfscope}%
\pgfsetbuttcap%
\pgfsetroundjoin%
\definecolor{currentfill}{rgb}{0.000000,0.000000,0.000000}%
\pgfsetfillcolor{currentfill}%
\pgfsetlinewidth{0.803000pt}%
\definecolor{currentstroke}{rgb}{0.000000,0.000000,0.000000}%
\pgfsetstrokecolor{currentstroke}%
\pgfsetdash{}{0pt}%
\pgfsys@defobject{currentmarker}{\pgfqpoint{-0.048611in}{0.000000in}}{\pgfqpoint{-0.000000in}{0.000000in}}{%
\pgfpathmoveto{\pgfqpoint{-0.000000in}{0.000000in}}%
\pgfpathlineto{\pgfqpoint{-0.048611in}{0.000000in}}%
\pgfusepath{stroke,fill}%
}%
\begin{pgfscope}%
\pgfsys@transformshift{2.963410in}{3.020898in}%
\pgfsys@useobject{currentmarker}{}%
\end{pgfscope}%
\end{pgfscope}%
\begin{pgfscope}%
\pgfsetbuttcap%
\pgfsetroundjoin%
\definecolor{currentfill}{rgb}{0.000000,0.000000,0.000000}%
\pgfsetfillcolor{currentfill}%
\pgfsetlinewidth{0.803000pt}%
\definecolor{currentstroke}{rgb}{0.000000,0.000000,0.000000}%
\pgfsetstrokecolor{currentstroke}%
\pgfsetdash{}{0pt}%
\pgfsys@defobject{currentmarker}{\pgfqpoint{-0.048611in}{0.000000in}}{\pgfqpoint{-0.000000in}{0.000000in}}{%
\pgfpathmoveto{\pgfqpoint{-0.000000in}{0.000000in}}%
\pgfpathlineto{\pgfqpoint{-0.048611in}{0.000000in}}%
\pgfusepath{stroke,fill}%
}%
\begin{pgfscope}%
\pgfsys@transformshift{2.963410in}{3.360152in}%
\pgfsys@useobject{currentmarker}{}%
\end{pgfscope}%
\end{pgfscope}%
\begin{pgfscope}%
\pgfsetbuttcap%
\pgfsetroundjoin%
\definecolor{currentfill}{rgb}{0.000000,0.000000,0.000000}%
\pgfsetfillcolor{currentfill}%
\pgfsetlinewidth{0.803000pt}%
\definecolor{currentstroke}{rgb}{0.000000,0.000000,0.000000}%
\pgfsetstrokecolor{currentstroke}%
\pgfsetdash{}{0pt}%
\pgfsys@defobject{currentmarker}{\pgfqpoint{-0.048611in}{0.000000in}}{\pgfqpoint{-0.000000in}{0.000000in}}{%
\pgfpathmoveto{\pgfqpoint{-0.000000in}{0.000000in}}%
\pgfpathlineto{\pgfqpoint{-0.048611in}{0.000000in}}%
\pgfusepath{stroke,fill}%
}%
\begin{pgfscope}%
\pgfsys@transformshift{2.963410in}{3.699405in}%
\pgfsys@useobject{currentmarker}{}%
\end{pgfscope}%
\end{pgfscope}%
\begin{pgfscope}%
\pgfsetbuttcap%
\pgfsetroundjoin%
\definecolor{currentfill}{rgb}{0.000000,0.000000,0.000000}%
\pgfsetfillcolor{currentfill}%
\pgfsetlinewidth{0.803000pt}%
\definecolor{currentstroke}{rgb}{0.000000,0.000000,0.000000}%
\pgfsetstrokecolor{currentstroke}%
\pgfsetdash{}{0pt}%
\pgfsys@defobject{currentmarker}{\pgfqpoint{-0.048611in}{0.000000in}}{\pgfqpoint{-0.000000in}{0.000000in}}{%
\pgfpathmoveto{\pgfqpoint{-0.000000in}{0.000000in}}%
\pgfpathlineto{\pgfqpoint{-0.048611in}{0.000000in}}%
\pgfusepath{stroke,fill}%
}%
\begin{pgfscope}%
\pgfsys@transformshift{2.963410in}{4.038659in}%
\pgfsys@useobject{currentmarker}{}%
\end{pgfscope}%
\end{pgfscope}%
\begin{pgfscope}%
\pgfsetbuttcap%
\pgfsetroundjoin%
\definecolor{currentfill}{rgb}{0.000000,0.000000,0.000000}%
\pgfsetfillcolor{currentfill}%
\pgfsetlinewidth{0.803000pt}%
\definecolor{currentstroke}{rgb}{0.000000,0.000000,0.000000}%
\pgfsetstrokecolor{currentstroke}%
\pgfsetdash{}{0pt}%
\pgfsys@defobject{currentmarker}{\pgfqpoint{-0.048611in}{0.000000in}}{\pgfqpoint{-0.000000in}{0.000000in}}{%
\pgfpathmoveto{\pgfqpoint{-0.000000in}{0.000000in}}%
\pgfpathlineto{\pgfqpoint{-0.048611in}{0.000000in}}%
\pgfusepath{stroke,fill}%
}%
\begin{pgfscope}%
\pgfsys@transformshift{2.963410in}{4.377912in}%
\pgfsys@useobject{currentmarker}{}%
\end{pgfscope}%
\end{pgfscope}%
\begin{pgfscope}%
\pgfsetbuttcap%
\pgfsetroundjoin%
\definecolor{currentfill}{rgb}{0.000000,0.000000,0.000000}%
\pgfsetfillcolor{currentfill}%
\pgfsetlinewidth{0.803000pt}%
\definecolor{currentstroke}{rgb}{0.000000,0.000000,0.000000}%
\pgfsetstrokecolor{currentstroke}%
\pgfsetdash{}{0pt}%
\pgfsys@defobject{currentmarker}{\pgfqpoint{-0.048611in}{0.000000in}}{\pgfqpoint{-0.000000in}{0.000000in}}{%
\pgfpathmoveto{\pgfqpoint{-0.000000in}{0.000000in}}%
\pgfpathlineto{\pgfqpoint{-0.048611in}{0.000000in}}%
\pgfusepath{stroke,fill}%
}%
\begin{pgfscope}%
\pgfsys@transformshift{2.963410in}{4.717165in}%
\pgfsys@useobject{currentmarker}{}%
\end{pgfscope}%
\end{pgfscope}%
\begin{pgfscope}%
\pgfsetbuttcap%
\pgfsetroundjoin%
\definecolor{currentfill}{rgb}{0.000000,0.000000,0.000000}%
\pgfsetfillcolor{currentfill}%
\pgfsetlinewidth{0.803000pt}%
\definecolor{currentstroke}{rgb}{0.000000,0.000000,0.000000}%
\pgfsetstrokecolor{currentstroke}%
\pgfsetdash{}{0pt}%
\pgfsys@defobject{currentmarker}{\pgfqpoint{-0.048611in}{0.000000in}}{\pgfqpoint{-0.000000in}{0.000000in}}{%
\pgfpathmoveto{\pgfqpoint{-0.000000in}{0.000000in}}%
\pgfpathlineto{\pgfqpoint{-0.048611in}{0.000000in}}%
\pgfusepath{stroke,fill}%
}%
\begin{pgfscope}%
\pgfsys@transformshift{2.963410in}{5.056419in}%
\pgfsys@useobject{currentmarker}{}%
\end{pgfscope}%
\end{pgfscope}%
\begin{pgfscope}%
\pgfsetrectcap%
\pgfsetmiterjoin%
\pgfsetlinewidth{0.803000pt}%
\definecolor{currentstroke}{rgb}{0.000000,0.000000,0.000000}%
\pgfsetstrokecolor{currentstroke}%
\pgfsetdash{}{0pt}%
\pgfpathmoveto{\pgfqpoint{2.963410in}{2.920818in}}%
\pgfpathlineto{\pgfqpoint{2.963410in}{5.122573in}}%
\pgfusepath{stroke}%
\end{pgfscope}%
\begin{pgfscope}%
\pgfsetrectcap%
\pgfsetmiterjoin%
\pgfsetlinewidth{0.803000pt}%
\definecolor{currentstroke}{rgb}{0.000000,0.000000,0.000000}%
\pgfsetstrokecolor{currentstroke}%
\pgfsetdash{}{0pt}%
\pgfpathmoveto{\pgfqpoint{2.963410in}{2.920818in}}%
\pgfpathlineto{\pgfqpoint{5.140690in}{2.920818in}}%
\pgfusepath{stroke}%
\end{pgfscope}%
\begin{pgfscope}%
\pgfsetbuttcap%
\pgfsetmiterjoin%
\definecolor{currentfill}{rgb}{1.000000,1.000000,1.000000}%
\pgfsetfillcolor{currentfill}%
\pgfsetlinewidth{0.000000pt}%
\definecolor{currentstroke}{rgb}{0.000000,0.000000,0.000000}%
\pgfsetstrokecolor{currentstroke}%
\pgfsetstrokeopacity{0.000000}%
\pgfsetdash{}{0pt}%
\pgfpathmoveto{\pgfqpoint{5.292946in}{2.920818in}}%
\pgfpathlineto{\pgfqpoint{7.470226in}{2.920818in}}%
\pgfpathlineto{\pgfqpoint{7.470226in}{5.122573in}}%
\pgfpathlineto{\pgfqpoint{5.292946in}{5.122573in}}%
\pgfpathlineto{\pgfqpoint{5.292946in}{2.920818in}}%
\pgfpathclose%
\pgfusepath{fill}%
\end{pgfscope}%
\begin{pgfscope}%
\pgfsetbuttcap%
\pgfsetroundjoin%
\definecolor{currentfill}{rgb}{0.000000,0.000000,0.000000}%
\pgfsetfillcolor{currentfill}%
\pgfsetlinewidth{0.803000pt}%
\definecolor{currentstroke}{rgb}{0.000000,0.000000,0.000000}%
\pgfsetstrokecolor{currentstroke}%
\pgfsetdash{}{0pt}%
\pgfsys@defobject{currentmarker}{\pgfqpoint{0.000000in}{-0.048611in}}{\pgfqpoint{0.000000in}{0.000000in}}{%
\pgfpathmoveto{\pgfqpoint{0.000000in}{0.000000in}}%
\pgfpathlineto{\pgfqpoint{0.000000in}{-0.048611in}}%
\pgfusepath{stroke,fill}%
}%
\begin{pgfscope}%
\pgfsys@transformshift{5.747357in}{2.920818in}%
\pgfsys@useobject{currentmarker}{}%
\end{pgfscope}%
\end{pgfscope}%
\begin{pgfscope}%
\pgfsetbuttcap%
\pgfsetroundjoin%
\definecolor{currentfill}{rgb}{0.000000,0.000000,0.000000}%
\pgfsetfillcolor{currentfill}%
\pgfsetlinewidth{0.803000pt}%
\definecolor{currentstroke}{rgb}{0.000000,0.000000,0.000000}%
\pgfsetstrokecolor{currentstroke}%
\pgfsetdash{}{0pt}%
\pgfsys@defobject{currentmarker}{\pgfqpoint{0.000000in}{-0.048611in}}{\pgfqpoint{0.000000in}{0.000000in}}{%
\pgfpathmoveto{\pgfqpoint{0.000000in}{0.000000in}}%
\pgfpathlineto{\pgfqpoint{0.000000in}{-0.048611in}}%
\pgfusepath{stroke,fill}%
}%
\begin{pgfscope}%
\pgfsys@transformshift{6.321463in}{2.920818in}%
\pgfsys@useobject{currentmarker}{}%
\end{pgfscope}%
\end{pgfscope}%
\begin{pgfscope}%
\pgfsetbuttcap%
\pgfsetroundjoin%
\definecolor{currentfill}{rgb}{0.000000,0.000000,0.000000}%
\pgfsetfillcolor{currentfill}%
\pgfsetlinewidth{0.803000pt}%
\definecolor{currentstroke}{rgb}{0.000000,0.000000,0.000000}%
\pgfsetstrokecolor{currentstroke}%
\pgfsetdash{}{0pt}%
\pgfsys@defobject{currentmarker}{\pgfqpoint{0.000000in}{-0.048611in}}{\pgfqpoint{0.000000in}{0.000000in}}{%
\pgfpathmoveto{\pgfqpoint{0.000000in}{0.000000in}}%
\pgfpathlineto{\pgfqpoint{0.000000in}{-0.048611in}}%
\pgfusepath{stroke,fill}%
}%
\begin{pgfscope}%
\pgfsys@transformshift{6.895569in}{2.920818in}%
\pgfsys@useobject{currentmarker}{}%
\end{pgfscope}%
\end{pgfscope}%
\begin{pgfscope}%
\pgfsetbuttcap%
\pgfsetroundjoin%
\definecolor{currentfill}{rgb}{0.000000,0.000000,0.000000}%
\pgfsetfillcolor{currentfill}%
\pgfsetlinewidth{0.803000pt}%
\definecolor{currentstroke}{rgb}{0.000000,0.000000,0.000000}%
\pgfsetstrokecolor{currentstroke}%
\pgfsetdash{}{0pt}%
\pgfsys@defobject{currentmarker}{\pgfqpoint{0.000000in}{-0.048611in}}{\pgfqpoint{0.000000in}{0.000000in}}{%
\pgfpathmoveto{\pgfqpoint{0.000000in}{0.000000in}}%
\pgfpathlineto{\pgfqpoint{0.000000in}{-0.048611in}}%
\pgfusepath{stroke,fill}%
}%
\begin{pgfscope}%
\pgfsys@transformshift{7.469674in}{2.920818in}%
\pgfsys@useobject{currentmarker}{}%
\end{pgfscope}%
\end{pgfscope}%
\begin{pgfscope}%
\pgfsetbuttcap%
\pgfsetroundjoin%
\definecolor{currentfill}{rgb}{0.000000,0.000000,0.000000}%
\pgfsetfillcolor{currentfill}%
\pgfsetlinewidth{0.803000pt}%
\definecolor{currentstroke}{rgb}{0.000000,0.000000,0.000000}%
\pgfsetstrokecolor{currentstroke}%
\pgfsetdash{}{0pt}%
\pgfsys@defobject{currentmarker}{\pgfqpoint{-0.048611in}{0.000000in}}{\pgfqpoint{-0.000000in}{0.000000in}}{%
\pgfpathmoveto{\pgfqpoint{-0.000000in}{0.000000in}}%
\pgfpathlineto{\pgfqpoint{-0.048611in}{0.000000in}}%
\pgfusepath{stroke,fill}%
}%
\begin{pgfscope}%
\pgfsys@transformshift{5.292946in}{3.020898in}%
\pgfsys@useobject{currentmarker}{}%
\end{pgfscope}%
\end{pgfscope}%
\begin{pgfscope}%
\pgfsetbuttcap%
\pgfsetroundjoin%
\definecolor{currentfill}{rgb}{0.000000,0.000000,0.000000}%
\pgfsetfillcolor{currentfill}%
\pgfsetlinewidth{0.803000pt}%
\definecolor{currentstroke}{rgb}{0.000000,0.000000,0.000000}%
\pgfsetstrokecolor{currentstroke}%
\pgfsetdash{}{0pt}%
\pgfsys@defobject{currentmarker}{\pgfqpoint{-0.048611in}{0.000000in}}{\pgfqpoint{-0.000000in}{0.000000in}}{%
\pgfpathmoveto{\pgfqpoint{-0.000000in}{0.000000in}}%
\pgfpathlineto{\pgfqpoint{-0.048611in}{0.000000in}}%
\pgfusepath{stroke,fill}%
}%
\begin{pgfscope}%
\pgfsys@transformshift{5.292946in}{3.360152in}%
\pgfsys@useobject{currentmarker}{}%
\end{pgfscope}%
\end{pgfscope}%
\begin{pgfscope}%
\pgfsetbuttcap%
\pgfsetroundjoin%
\definecolor{currentfill}{rgb}{0.000000,0.000000,0.000000}%
\pgfsetfillcolor{currentfill}%
\pgfsetlinewidth{0.803000pt}%
\definecolor{currentstroke}{rgb}{0.000000,0.000000,0.000000}%
\pgfsetstrokecolor{currentstroke}%
\pgfsetdash{}{0pt}%
\pgfsys@defobject{currentmarker}{\pgfqpoint{-0.048611in}{0.000000in}}{\pgfqpoint{-0.000000in}{0.000000in}}{%
\pgfpathmoveto{\pgfqpoint{-0.000000in}{0.000000in}}%
\pgfpathlineto{\pgfqpoint{-0.048611in}{0.000000in}}%
\pgfusepath{stroke,fill}%
}%
\begin{pgfscope}%
\pgfsys@transformshift{5.292946in}{3.699405in}%
\pgfsys@useobject{currentmarker}{}%
\end{pgfscope}%
\end{pgfscope}%
\begin{pgfscope}%
\pgfsetbuttcap%
\pgfsetroundjoin%
\definecolor{currentfill}{rgb}{0.000000,0.000000,0.000000}%
\pgfsetfillcolor{currentfill}%
\pgfsetlinewidth{0.803000pt}%
\definecolor{currentstroke}{rgb}{0.000000,0.000000,0.000000}%
\pgfsetstrokecolor{currentstroke}%
\pgfsetdash{}{0pt}%
\pgfsys@defobject{currentmarker}{\pgfqpoint{-0.048611in}{0.000000in}}{\pgfqpoint{-0.000000in}{0.000000in}}{%
\pgfpathmoveto{\pgfqpoint{-0.000000in}{0.000000in}}%
\pgfpathlineto{\pgfqpoint{-0.048611in}{0.000000in}}%
\pgfusepath{stroke,fill}%
}%
\begin{pgfscope}%
\pgfsys@transformshift{5.292946in}{4.038659in}%
\pgfsys@useobject{currentmarker}{}%
\end{pgfscope}%
\end{pgfscope}%
\begin{pgfscope}%
\pgfsetbuttcap%
\pgfsetroundjoin%
\definecolor{currentfill}{rgb}{0.000000,0.000000,0.000000}%
\pgfsetfillcolor{currentfill}%
\pgfsetlinewidth{0.803000pt}%
\definecolor{currentstroke}{rgb}{0.000000,0.000000,0.000000}%
\pgfsetstrokecolor{currentstroke}%
\pgfsetdash{}{0pt}%
\pgfsys@defobject{currentmarker}{\pgfqpoint{-0.048611in}{0.000000in}}{\pgfqpoint{-0.000000in}{0.000000in}}{%
\pgfpathmoveto{\pgfqpoint{-0.000000in}{0.000000in}}%
\pgfpathlineto{\pgfqpoint{-0.048611in}{0.000000in}}%
\pgfusepath{stroke,fill}%
}%
\begin{pgfscope}%
\pgfsys@transformshift{5.292946in}{4.377912in}%
\pgfsys@useobject{currentmarker}{}%
\end{pgfscope}%
\end{pgfscope}%
\begin{pgfscope}%
\pgfsetbuttcap%
\pgfsetroundjoin%
\definecolor{currentfill}{rgb}{0.000000,0.000000,0.000000}%
\pgfsetfillcolor{currentfill}%
\pgfsetlinewidth{0.803000pt}%
\definecolor{currentstroke}{rgb}{0.000000,0.000000,0.000000}%
\pgfsetstrokecolor{currentstroke}%
\pgfsetdash{}{0pt}%
\pgfsys@defobject{currentmarker}{\pgfqpoint{-0.048611in}{0.000000in}}{\pgfqpoint{-0.000000in}{0.000000in}}{%
\pgfpathmoveto{\pgfqpoint{-0.000000in}{0.000000in}}%
\pgfpathlineto{\pgfqpoint{-0.048611in}{0.000000in}}%
\pgfusepath{stroke,fill}%
}%
\begin{pgfscope}%
\pgfsys@transformshift{5.292946in}{4.717165in}%
\pgfsys@useobject{currentmarker}{}%
\end{pgfscope}%
\end{pgfscope}%
\begin{pgfscope}%
\pgfsetbuttcap%
\pgfsetroundjoin%
\definecolor{currentfill}{rgb}{0.000000,0.000000,0.000000}%
\pgfsetfillcolor{currentfill}%
\pgfsetlinewidth{0.803000pt}%
\definecolor{currentstroke}{rgb}{0.000000,0.000000,0.000000}%
\pgfsetstrokecolor{currentstroke}%
\pgfsetdash{}{0pt}%
\pgfsys@defobject{currentmarker}{\pgfqpoint{-0.048611in}{0.000000in}}{\pgfqpoint{-0.000000in}{0.000000in}}{%
\pgfpathmoveto{\pgfqpoint{-0.000000in}{0.000000in}}%
\pgfpathlineto{\pgfqpoint{-0.048611in}{0.000000in}}%
\pgfusepath{stroke,fill}%
}%
\begin{pgfscope}%
\pgfsys@transformshift{5.292946in}{5.056419in}%
\pgfsys@useobject{currentmarker}{}%
\end{pgfscope}%
\end{pgfscope}%
\begin{pgfscope}%
\pgfsetrectcap%
\pgfsetmiterjoin%
\pgfsetlinewidth{0.803000pt}%
\definecolor{currentstroke}{rgb}{0.000000,0.000000,0.000000}%
\pgfsetstrokecolor{currentstroke}%
\pgfsetdash{}{0pt}%
\pgfpathmoveto{\pgfqpoint{5.292946in}{2.920818in}}%
\pgfpathlineto{\pgfqpoint{5.292946in}{5.122573in}}%
\pgfusepath{stroke}%
\end{pgfscope}%
\begin{pgfscope}%
\pgfsetrectcap%
\pgfsetmiterjoin%
\pgfsetlinewidth{0.803000pt}%
\definecolor{currentstroke}{rgb}{0.000000,0.000000,0.000000}%
\pgfsetstrokecolor{currentstroke}%
\pgfsetdash{}{0pt}%
\pgfpathmoveto{\pgfqpoint{5.292946in}{2.920818in}}%
\pgfpathlineto{\pgfqpoint{7.470226in}{2.920818in}}%
\pgfusepath{stroke}%
\end{pgfscope}%
\begin{pgfscope}%
\pgfsetbuttcap%
\pgfsetmiterjoin%
\definecolor{currentfill}{rgb}{1.000000,1.000000,1.000000}%
\pgfsetfillcolor{currentfill}%
\pgfsetlinewidth{0.000000pt}%
\definecolor{currentstroke}{rgb}{0.000000,0.000000,0.000000}%
\pgfsetstrokecolor{currentstroke}%
\pgfsetstrokeopacity{0.000000}%
\pgfsetdash{}{0pt}%
\pgfpathmoveto{\pgfqpoint{7.622482in}{2.920818in}}%
\pgfpathlineto{\pgfqpoint{9.799762in}{2.920818in}}%
\pgfpathlineto{\pgfqpoint{9.799762in}{5.122573in}}%
\pgfpathlineto{\pgfqpoint{7.622482in}{5.122573in}}%
\pgfpathlineto{\pgfqpoint{7.622482in}{2.920818in}}%
\pgfpathclose%
\pgfusepath{fill}%
\end{pgfscope}%
\begin{pgfscope}%
\pgfpathrectangle{\pgfqpoint{7.622482in}{2.920818in}}{\pgfqpoint{2.177280in}{2.201755in}}%
\pgfusepath{clip}%
\pgfsetbuttcap%
\pgfsetroundjoin%
\definecolor{currentfill}{rgb}{0.121569,0.466667,0.705882}%
\pgfsetfillcolor{currentfill}%
\pgfsetlinewidth{0.481800pt}%
\definecolor{currentstroke}{rgb}{1.000000,1.000000,1.000000}%
\pgfsetstrokecolor{currentstroke}%
\pgfsetdash{}{0pt}%
\pgfpathmoveto{\pgfqpoint{7.887162in}{3.114933in}}%
\pgfpathcurveto{\pgfqpoint{7.898212in}{3.114933in}}{\pgfqpoint{7.908811in}{3.119323in}}{\pgfqpoint{7.916625in}{3.127137in}}%
\pgfpathcurveto{\pgfqpoint{7.924438in}{3.134950in}}{\pgfqpoint{7.928828in}{3.145549in}}{\pgfqpoint{7.928828in}{3.156599in}}%
\pgfpathcurveto{\pgfqpoint{7.928828in}{3.167650in}}{\pgfqpoint{7.924438in}{3.178249in}}{\pgfqpoint{7.916625in}{3.186062in}}%
\pgfpathcurveto{\pgfqpoint{7.908811in}{3.193876in}}{\pgfqpoint{7.898212in}{3.198266in}}{\pgfqpoint{7.887162in}{3.198266in}}%
\pgfpathcurveto{\pgfqpoint{7.876112in}{3.198266in}}{\pgfqpoint{7.865513in}{3.193876in}}{\pgfqpoint{7.857699in}{3.186062in}}%
\pgfpathcurveto{\pgfqpoint{7.849885in}{3.178249in}}{\pgfqpoint{7.845495in}{3.167650in}}{\pgfqpoint{7.845495in}{3.156599in}}%
\pgfpathcurveto{\pgfqpoint{7.845495in}{3.145549in}}{\pgfqpoint{7.849885in}{3.134950in}}{\pgfqpoint{7.857699in}{3.127137in}}%
\pgfpathcurveto{\pgfqpoint{7.865513in}{3.119323in}}{\pgfqpoint{7.876112in}{3.114933in}}{\pgfqpoint{7.887162in}{3.114933in}}%
\pgfpathlineto{\pgfqpoint{7.887162in}{3.114933in}}%
\pgfpathclose%
\pgfusepath{stroke,fill}%
\end{pgfscope}%
\begin{pgfscope}%
\pgfpathrectangle{\pgfqpoint{7.622482in}{2.920818in}}{\pgfqpoint{2.177280in}{2.201755in}}%
\pgfusepath{clip}%
\pgfsetbuttcap%
\pgfsetroundjoin%
\definecolor{currentfill}{rgb}{0.121569,0.466667,0.705882}%
\pgfsetfillcolor{currentfill}%
\pgfsetlinewidth{0.481800pt}%
\definecolor{currentstroke}{rgb}{1.000000,1.000000,1.000000}%
\pgfsetstrokecolor{currentstroke}%
\pgfsetdash{}{0pt}%
\pgfpathmoveto{\pgfqpoint{7.887162in}{3.114933in}}%
\pgfpathcurveto{\pgfqpoint{7.898212in}{3.114933in}}{\pgfqpoint{7.908811in}{3.119323in}}{\pgfqpoint{7.916625in}{3.127137in}}%
\pgfpathcurveto{\pgfqpoint{7.924438in}{3.134950in}}{\pgfqpoint{7.928828in}{3.145549in}}{\pgfqpoint{7.928828in}{3.156599in}}%
\pgfpathcurveto{\pgfqpoint{7.928828in}{3.167650in}}{\pgfqpoint{7.924438in}{3.178249in}}{\pgfqpoint{7.916625in}{3.186062in}}%
\pgfpathcurveto{\pgfqpoint{7.908811in}{3.193876in}}{\pgfqpoint{7.898212in}{3.198266in}}{\pgfqpoint{7.887162in}{3.198266in}}%
\pgfpathcurveto{\pgfqpoint{7.876112in}{3.198266in}}{\pgfqpoint{7.865513in}{3.193876in}}{\pgfqpoint{7.857699in}{3.186062in}}%
\pgfpathcurveto{\pgfqpoint{7.849885in}{3.178249in}}{\pgfqpoint{7.845495in}{3.167650in}}{\pgfqpoint{7.845495in}{3.156599in}}%
\pgfpathcurveto{\pgfqpoint{7.845495in}{3.145549in}}{\pgfqpoint{7.849885in}{3.134950in}}{\pgfqpoint{7.857699in}{3.127137in}}%
\pgfpathcurveto{\pgfqpoint{7.865513in}{3.119323in}}{\pgfqpoint{7.876112in}{3.114933in}}{\pgfqpoint{7.887162in}{3.114933in}}%
\pgfpathlineto{\pgfqpoint{7.887162in}{3.114933in}}%
\pgfpathclose%
\pgfusepath{stroke,fill}%
\end{pgfscope}%
\begin{pgfscope}%
\pgfpathrectangle{\pgfqpoint{7.622482in}{2.920818in}}{\pgfqpoint{2.177280in}{2.201755in}}%
\pgfusepath{clip}%
\pgfsetbuttcap%
\pgfsetroundjoin%
\definecolor{currentfill}{rgb}{0.121569,0.466667,0.705882}%
\pgfsetfillcolor{currentfill}%
\pgfsetlinewidth{0.481800pt}%
\definecolor{currentstroke}{rgb}{1.000000,1.000000,1.000000}%
\pgfsetstrokecolor{currentstroke}%
\pgfsetdash{}{0pt}%
\pgfpathmoveto{\pgfqpoint{7.887162in}{3.081007in}}%
\pgfpathcurveto{\pgfqpoint{7.898212in}{3.081007in}}{\pgfqpoint{7.908811in}{3.085398in}}{\pgfqpoint{7.916625in}{3.093211in}}%
\pgfpathcurveto{\pgfqpoint{7.924438in}{3.101025in}}{\pgfqpoint{7.928828in}{3.111624in}}{\pgfqpoint{7.928828in}{3.122674in}}%
\pgfpathcurveto{\pgfqpoint{7.928828in}{3.133724in}}{\pgfqpoint{7.924438in}{3.144323in}}{\pgfqpoint{7.916625in}{3.152137in}}%
\pgfpathcurveto{\pgfqpoint{7.908811in}{3.159951in}}{\pgfqpoint{7.898212in}{3.164341in}}{\pgfqpoint{7.887162in}{3.164341in}}%
\pgfpathcurveto{\pgfqpoint{7.876112in}{3.164341in}}{\pgfqpoint{7.865513in}{3.159951in}}{\pgfqpoint{7.857699in}{3.152137in}}%
\pgfpathcurveto{\pgfqpoint{7.849885in}{3.144323in}}{\pgfqpoint{7.845495in}{3.133724in}}{\pgfqpoint{7.845495in}{3.122674in}}%
\pgfpathcurveto{\pgfqpoint{7.845495in}{3.111624in}}{\pgfqpoint{7.849885in}{3.101025in}}{\pgfqpoint{7.857699in}{3.093211in}}%
\pgfpathcurveto{\pgfqpoint{7.865513in}{3.085398in}}{\pgfqpoint{7.876112in}{3.081007in}}{\pgfqpoint{7.887162in}{3.081007in}}%
\pgfpathlineto{\pgfqpoint{7.887162in}{3.081007in}}%
\pgfpathclose%
\pgfusepath{stroke,fill}%
\end{pgfscope}%
\begin{pgfscope}%
\pgfpathrectangle{\pgfqpoint{7.622482in}{2.920818in}}{\pgfqpoint{2.177280in}{2.201755in}}%
\pgfusepath{clip}%
\pgfsetbuttcap%
\pgfsetroundjoin%
\definecolor{currentfill}{rgb}{0.121569,0.466667,0.705882}%
\pgfsetfillcolor{currentfill}%
\pgfsetlinewidth{0.481800pt}%
\definecolor{currentstroke}{rgb}{1.000000,1.000000,1.000000}%
\pgfsetstrokecolor{currentstroke}%
\pgfsetdash{}{0pt}%
\pgfpathmoveto{\pgfqpoint{7.887162in}{3.148858in}}%
\pgfpathcurveto{\pgfqpoint{7.898212in}{3.148858in}}{\pgfqpoint{7.908811in}{3.153248in}}{\pgfqpoint{7.916625in}{3.161062in}}%
\pgfpathcurveto{\pgfqpoint{7.924438in}{3.168876in}}{\pgfqpoint{7.928828in}{3.179475in}}{\pgfqpoint{7.928828in}{3.190525in}}%
\pgfpathcurveto{\pgfqpoint{7.928828in}{3.201575in}}{\pgfqpoint{7.924438in}{3.212174in}}{\pgfqpoint{7.916625in}{3.219988in}}%
\pgfpathcurveto{\pgfqpoint{7.908811in}{3.227801in}}{\pgfqpoint{7.898212in}{3.232191in}}{\pgfqpoint{7.887162in}{3.232191in}}%
\pgfpathcurveto{\pgfqpoint{7.876112in}{3.232191in}}{\pgfqpoint{7.865513in}{3.227801in}}{\pgfqpoint{7.857699in}{3.219988in}}%
\pgfpathcurveto{\pgfqpoint{7.849885in}{3.212174in}}{\pgfqpoint{7.845495in}{3.201575in}}{\pgfqpoint{7.845495in}{3.190525in}}%
\pgfpathcurveto{\pgfqpoint{7.845495in}{3.179475in}}{\pgfqpoint{7.849885in}{3.168876in}}{\pgfqpoint{7.857699in}{3.161062in}}%
\pgfpathcurveto{\pgfqpoint{7.865513in}{3.153248in}}{\pgfqpoint{7.876112in}{3.148858in}}{\pgfqpoint{7.887162in}{3.148858in}}%
\pgfpathlineto{\pgfqpoint{7.887162in}{3.148858in}}%
\pgfpathclose%
\pgfusepath{stroke,fill}%
\end{pgfscope}%
\begin{pgfscope}%
\pgfpathrectangle{\pgfqpoint{7.622482in}{2.920818in}}{\pgfqpoint{2.177280in}{2.201755in}}%
\pgfusepath{clip}%
\pgfsetbuttcap%
\pgfsetroundjoin%
\definecolor{currentfill}{rgb}{0.121569,0.466667,0.705882}%
\pgfsetfillcolor{currentfill}%
\pgfsetlinewidth{0.481800pt}%
\definecolor{currentstroke}{rgb}{1.000000,1.000000,1.000000}%
\pgfsetstrokecolor{currentstroke}%
\pgfsetdash{}{0pt}%
\pgfpathmoveto{\pgfqpoint{7.887162in}{3.114933in}}%
\pgfpathcurveto{\pgfqpoint{7.898212in}{3.114933in}}{\pgfqpoint{7.908811in}{3.119323in}}{\pgfqpoint{7.916625in}{3.127137in}}%
\pgfpathcurveto{\pgfqpoint{7.924438in}{3.134950in}}{\pgfqpoint{7.928828in}{3.145549in}}{\pgfqpoint{7.928828in}{3.156599in}}%
\pgfpathcurveto{\pgfqpoint{7.928828in}{3.167650in}}{\pgfqpoint{7.924438in}{3.178249in}}{\pgfqpoint{7.916625in}{3.186062in}}%
\pgfpathcurveto{\pgfqpoint{7.908811in}{3.193876in}}{\pgfqpoint{7.898212in}{3.198266in}}{\pgfqpoint{7.887162in}{3.198266in}}%
\pgfpathcurveto{\pgfqpoint{7.876112in}{3.198266in}}{\pgfqpoint{7.865513in}{3.193876in}}{\pgfqpoint{7.857699in}{3.186062in}}%
\pgfpathcurveto{\pgfqpoint{7.849885in}{3.178249in}}{\pgfqpoint{7.845495in}{3.167650in}}{\pgfqpoint{7.845495in}{3.156599in}}%
\pgfpathcurveto{\pgfqpoint{7.845495in}{3.145549in}}{\pgfqpoint{7.849885in}{3.134950in}}{\pgfqpoint{7.857699in}{3.127137in}}%
\pgfpathcurveto{\pgfqpoint{7.865513in}{3.119323in}}{\pgfqpoint{7.876112in}{3.114933in}}{\pgfqpoint{7.887162in}{3.114933in}}%
\pgfpathlineto{\pgfqpoint{7.887162in}{3.114933in}}%
\pgfpathclose%
\pgfusepath{stroke,fill}%
\end{pgfscope}%
\begin{pgfscope}%
\pgfpathrectangle{\pgfqpoint{7.622482in}{2.920818in}}{\pgfqpoint{2.177280in}{2.201755in}}%
\pgfusepath{clip}%
\pgfsetbuttcap%
\pgfsetroundjoin%
\definecolor{currentfill}{rgb}{0.121569,0.466667,0.705882}%
\pgfsetfillcolor{currentfill}%
\pgfsetlinewidth{0.481800pt}%
\definecolor{currentstroke}{rgb}{1.000000,1.000000,1.000000}%
\pgfsetstrokecolor{currentstroke}%
\pgfsetdash{}{0pt}%
\pgfpathmoveto{\pgfqpoint{8.022670in}{3.216709in}}%
\pgfpathcurveto{\pgfqpoint{8.033720in}{3.216709in}}{\pgfqpoint{8.044319in}{3.221099in}}{\pgfqpoint{8.052132in}{3.228913in}}%
\pgfpathcurveto{\pgfqpoint{8.059946in}{3.236726in}}{\pgfqpoint{8.064336in}{3.247325in}}{\pgfqpoint{8.064336in}{3.258375in}}%
\pgfpathcurveto{\pgfqpoint{8.064336in}{3.269426in}}{\pgfqpoint{8.059946in}{3.280025in}}{\pgfqpoint{8.052132in}{3.287838in}}%
\pgfpathcurveto{\pgfqpoint{8.044319in}{3.295652in}}{\pgfqpoint{8.033720in}{3.300042in}}{\pgfqpoint{8.022670in}{3.300042in}}%
\pgfpathcurveto{\pgfqpoint{8.011619in}{3.300042in}}{\pgfqpoint{8.001020in}{3.295652in}}{\pgfqpoint{7.993207in}{3.287838in}}%
\pgfpathcurveto{\pgfqpoint{7.985393in}{3.280025in}}{\pgfqpoint{7.981003in}{3.269426in}}{\pgfqpoint{7.981003in}{3.258375in}}%
\pgfpathcurveto{\pgfqpoint{7.981003in}{3.247325in}}{\pgfqpoint{7.985393in}{3.236726in}}{\pgfqpoint{7.993207in}{3.228913in}}%
\pgfpathcurveto{\pgfqpoint{8.001020in}{3.221099in}}{\pgfqpoint{8.011619in}{3.216709in}}{\pgfqpoint{8.022670in}{3.216709in}}%
\pgfpathlineto{\pgfqpoint{8.022670in}{3.216709in}}%
\pgfpathclose%
\pgfusepath{stroke,fill}%
\end{pgfscope}%
\begin{pgfscope}%
\pgfpathrectangle{\pgfqpoint{7.622482in}{2.920818in}}{\pgfqpoint{2.177280in}{2.201755in}}%
\pgfusepath{clip}%
\pgfsetbuttcap%
\pgfsetroundjoin%
\definecolor{currentfill}{rgb}{0.121569,0.466667,0.705882}%
\pgfsetfillcolor{currentfill}%
\pgfsetlinewidth{0.481800pt}%
\definecolor{currentstroke}{rgb}{1.000000,1.000000,1.000000}%
\pgfsetstrokecolor{currentstroke}%
\pgfsetdash{}{0pt}%
\pgfpathmoveto{\pgfqpoint{7.954916in}{3.114933in}}%
\pgfpathcurveto{\pgfqpoint{7.965966in}{3.114933in}}{\pgfqpoint{7.976565in}{3.119323in}}{\pgfqpoint{7.984378in}{3.127137in}}%
\pgfpathcurveto{\pgfqpoint{7.992192in}{3.134950in}}{\pgfqpoint{7.996582in}{3.145549in}}{\pgfqpoint{7.996582in}{3.156599in}}%
\pgfpathcurveto{\pgfqpoint{7.996582in}{3.167650in}}{\pgfqpoint{7.992192in}{3.178249in}}{\pgfqpoint{7.984378in}{3.186062in}}%
\pgfpathcurveto{\pgfqpoint{7.976565in}{3.193876in}}{\pgfqpoint{7.965966in}{3.198266in}}{\pgfqpoint{7.954916in}{3.198266in}}%
\pgfpathcurveto{\pgfqpoint{7.943866in}{3.198266in}}{\pgfqpoint{7.933267in}{3.193876in}}{\pgfqpoint{7.925453in}{3.186062in}}%
\pgfpathcurveto{\pgfqpoint{7.917639in}{3.178249in}}{\pgfqpoint{7.913249in}{3.167650in}}{\pgfqpoint{7.913249in}{3.156599in}}%
\pgfpathcurveto{\pgfqpoint{7.913249in}{3.145549in}}{\pgfqpoint{7.917639in}{3.134950in}}{\pgfqpoint{7.925453in}{3.127137in}}%
\pgfpathcurveto{\pgfqpoint{7.933267in}{3.119323in}}{\pgfqpoint{7.943866in}{3.114933in}}{\pgfqpoint{7.954916in}{3.114933in}}%
\pgfpathlineto{\pgfqpoint{7.954916in}{3.114933in}}%
\pgfpathclose%
\pgfusepath{stroke,fill}%
\end{pgfscope}%
\begin{pgfscope}%
\pgfpathrectangle{\pgfqpoint{7.622482in}{2.920818in}}{\pgfqpoint{2.177280in}{2.201755in}}%
\pgfusepath{clip}%
\pgfsetbuttcap%
\pgfsetroundjoin%
\definecolor{currentfill}{rgb}{0.121569,0.466667,0.705882}%
\pgfsetfillcolor{currentfill}%
\pgfsetlinewidth{0.481800pt}%
\definecolor{currentstroke}{rgb}{1.000000,1.000000,1.000000}%
\pgfsetstrokecolor{currentstroke}%
\pgfsetdash{}{0pt}%
\pgfpathmoveto{\pgfqpoint{7.887162in}{3.148858in}}%
\pgfpathcurveto{\pgfqpoint{7.898212in}{3.148858in}}{\pgfqpoint{7.908811in}{3.153248in}}{\pgfqpoint{7.916625in}{3.161062in}}%
\pgfpathcurveto{\pgfqpoint{7.924438in}{3.168876in}}{\pgfqpoint{7.928828in}{3.179475in}}{\pgfqpoint{7.928828in}{3.190525in}}%
\pgfpathcurveto{\pgfqpoint{7.928828in}{3.201575in}}{\pgfqpoint{7.924438in}{3.212174in}}{\pgfqpoint{7.916625in}{3.219988in}}%
\pgfpathcurveto{\pgfqpoint{7.908811in}{3.227801in}}{\pgfqpoint{7.898212in}{3.232191in}}{\pgfqpoint{7.887162in}{3.232191in}}%
\pgfpathcurveto{\pgfqpoint{7.876112in}{3.232191in}}{\pgfqpoint{7.865513in}{3.227801in}}{\pgfqpoint{7.857699in}{3.219988in}}%
\pgfpathcurveto{\pgfqpoint{7.849885in}{3.212174in}}{\pgfqpoint{7.845495in}{3.201575in}}{\pgfqpoint{7.845495in}{3.190525in}}%
\pgfpathcurveto{\pgfqpoint{7.845495in}{3.179475in}}{\pgfqpoint{7.849885in}{3.168876in}}{\pgfqpoint{7.857699in}{3.161062in}}%
\pgfpathcurveto{\pgfqpoint{7.865513in}{3.153248in}}{\pgfqpoint{7.876112in}{3.148858in}}{\pgfqpoint{7.887162in}{3.148858in}}%
\pgfpathlineto{\pgfqpoint{7.887162in}{3.148858in}}%
\pgfpathclose%
\pgfusepath{stroke,fill}%
\end{pgfscope}%
\begin{pgfscope}%
\pgfpathrectangle{\pgfqpoint{7.622482in}{2.920818in}}{\pgfqpoint{2.177280in}{2.201755in}}%
\pgfusepath{clip}%
\pgfsetbuttcap%
\pgfsetroundjoin%
\definecolor{currentfill}{rgb}{0.121569,0.466667,0.705882}%
\pgfsetfillcolor{currentfill}%
\pgfsetlinewidth{0.481800pt}%
\definecolor{currentstroke}{rgb}{1.000000,1.000000,1.000000}%
\pgfsetstrokecolor{currentstroke}%
\pgfsetdash{}{0pt}%
\pgfpathmoveto{\pgfqpoint{7.887162in}{3.114933in}}%
\pgfpathcurveto{\pgfqpoint{7.898212in}{3.114933in}}{\pgfqpoint{7.908811in}{3.119323in}}{\pgfqpoint{7.916625in}{3.127137in}}%
\pgfpathcurveto{\pgfqpoint{7.924438in}{3.134950in}}{\pgfqpoint{7.928828in}{3.145549in}}{\pgfqpoint{7.928828in}{3.156599in}}%
\pgfpathcurveto{\pgfqpoint{7.928828in}{3.167650in}}{\pgfqpoint{7.924438in}{3.178249in}}{\pgfqpoint{7.916625in}{3.186062in}}%
\pgfpathcurveto{\pgfqpoint{7.908811in}{3.193876in}}{\pgfqpoint{7.898212in}{3.198266in}}{\pgfqpoint{7.887162in}{3.198266in}}%
\pgfpathcurveto{\pgfqpoint{7.876112in}{3.198266in}}{\pgfqpoint{7.865513in}{3.193876in}}{\pgfqpoint{7.857699in}{3.186062in}}%
\pgfpathcurveto{\pgfqpoint{7.849885in}{3.178249in}}{\pgfqpoint{7.845495in}{3.167650in}}{\pgfqpoint{7.845495in}{3.156599in}}%
\pgfpathcurveto{\pgfqpoint{7.845495in}{3.145549in}}{\pgfqpoint{7.849885in}{3.134950in}}{\pgfqpoint{7.857699in}{3.127137in}}%
\pgfpathcurveto{\pgfqpoint{7.865513in}{3.119323in}}{\pgfqpoint{7.876112in}{3.114933in}}{\pgfqpoint{7.887162in}{3.114933in}}%
\pgfpathlineto{\pgfqpoint{7.887162in}{3.114933in}}%
\pgfpathclose%
\pgfusepath{stroke,fill}%
\end{pgfscope}%
\begin{pgfscope}%
\pgfpathrectangle{\pgfqpoint{7.622482in}{2.920818in}}{\pgfqpoint{2.177280in}{2.201755in}}%
\pgfusepath{clip}%
\pgfsetbuttcap%
\pgfsetroundjoin%
\definecolor{currentfill}{rgb}{0.121569,0.466667,0.705882}%
\pgfsetfillcolor{currentfill}%
\pgfsetlinewidth{0.481800pt}%
\definecolor{currentstroke}{rgb}{1.000000,1.000000,1.000000}%
\pgfsetstrokecolor{currentstroke}%
\pgfsetdash{}{0pt}%
\pgfpathmoveto{\pgfqpoint{7.819408in}{3.148858in}}%
\pgfpathcurveto{\pgfqpoint{7.830458in}{3.148858in}}{\pgfqpoint{7.841057in}{3.153248in}}{\pgfqpoint{7.848871in}{3.161062in}}%
\pgfpathcurveto{\pgfqpoint{7.856684in}{3.168876in}}{\pgfqpoint{7.861075in}{3.179475in}}{\pgfqpoint{7.861075in}{3.190525in}}%
\pgfpathcurveto{\pgfqpoint{7.861075in}{3.201575in}}{\pgfqpoint{7.856684in}{3.212174in}}{\pgfqpoint{7.848871in}{3.219988in}}%
\pgfpathcurveto{\pgfqpoint{7.841057in}{3.227801in}}{\pgfqpoint{7.830458in}{3.232191in}}{\pgfqpoint{7.819408in}{3.232191in}}%
\pgfpathcurveto{\pgfqpoint{7.808358in}{3.232191in}}{\pgfqpoint{7.797759in}{3.227801in}}{\pgfqpoint{7.789945in}{3.219988in}}%
\pgfpathcurveto{\pgfqpoint{7.782132in}{3.212174in}}{\pgfqpoint{7.777741in}{3.201575in}}{\pgfqpoint{7.777741in}{3.190525in}}%
\pgfpathcurveto{\pgfqpoint{7.777741in}{3.179475in}}{\pgfqpoint{7.782132in}{3.168876in}}{\pgfqpoint{7.789945in}{3.161062in}}%
\pgfpathcurveto{\pgfqpoint{7.797759in}{3.153248in}}{\pgfqpoint{7.808358in}{3.148858in}}{\pgfqpoint{7.819408in}{3.148858in}}%
\pgfpathlineto{\pgfqpoint{7.819408in}{3.148858in}}%
\pgfpathclose%
\pgfusepath{stroke,fill}%
\end{pgfscope}%
\begin{pgfscope}%
\pgfpathrectangle{\pgfqpoint{7.622482in}{2.920818in}}{\pgfqpoint{2.177280in}{2.201755in}}%
\pgfusepath{clip}%
\pgfsetbuttcap%
\pgfsetroundjoin%
\definecolor{currentfill}{rgb}{0.121569,0.466667,0.705882}%
\pgfsetfillcolor{currentfill}%
\pgfsetlinewidth{0.481800pt}%
\definecolor{currentstroke}{rgb}{1.000000,1.000000,1.000000}%
\pgfsetstrokecolor{currentstroke}%
\pgfsetdash{}{0pt}%
\pgfpathmoveto{\pgfqpoint{7.887162in}{3.148858in}}%
\pgfpathcurveto{\pgfqpoint{7.898212in}{3.148858in}}{\pgfqpoint{7.908811in}{3.153248in}}{\pgfqpoint{7.916625in}{3.161062in}}%
\pgfpathcurveto{\pgfqpoint{7.924438in}{3.168876in}}{\pgfqpoint{7.928828in}{3.179475in}}{\pgfqpoint{7.928828in}{3.190525in}}%
\pgfpathcurveto{\pgfqpoint{7.928828in}{3.201575in}}{\pgfqpoint{7.924438in}{3.212174in}}{\pgfqpoint{7.916625in}{3.219988in}}%
\pgfpathcurveto{\pgfqpoint{7.908811in}{3.227801in}}{\pgfqpoint{7.898212in}{3.232191in}}{\pgfqpoint{7.887162in}{3.232191in}}%
\pgfpathcurveto{\pgfqpoint{7.876112in}{3.232191in}}{\pgfqpoint{7.865513in}{3.227801in}}{\pgfqpoint{7.857699in}{3.219988in}}%
\pgfpathcurveto{\pgfqpoint{7.849885in}{3.212174in}}{\pgfqpoint{7.845495in}{3.201575in}}{\pgfqpoint{7.845495in}{3.190525in}}%
\pgfpathcurveto{\pgfqpoint{7.845495in}{3.179475in}}{\pgfqpoint{7.849885in}{3.168876in}}{\pgfqpoint{7.857699in}{3.161062in}}%
\pgfpathcurveto{\pgfqpoint{7.865513in}{3.153248in}}{\pgfqpoint{7.876112in}{3.148858in}}{\pgfqpoint{7.887162in}{3.148858in}}%
\pgfpathlineto{\pgfqpoint{7.887162in}{3.148858in}}%
\pgfpathclose%
\pgfusepath{stroke,fill}%
\end{pgfscope}%
\begin{pgfscope}%
\pgfpathrectangle{\pgfqpoint{7.622482in}{2.920818in}}{\pgfqpoint{2.177280in}{2.201755in}}%
\pgfusepath{clip}%
\pgfsetbuttcap%
\pgfsetroundjoin%
\definecolor{currentfill}{rgb}{0.121569,0.466667,0.705882}%
\pgfsetfillcolor{currentfill}%
\pgfsetlinewidth{0.481800pt}%
\definecolor{currentstroke}{rgb}{1.000000,1.000000,1.000000}%
\pgfsetstrokecolor{currentstroke}%
\pgfsetdash{}{0pt}%
\pgfpathmoveto{\pgfqpoint{7.887162in}{3.182783in}}%
\pgfpathcurveto{\pgfqpoint{7.898212in}{3.182783in}}{\pgfqpoint{7.908811in}{3.187174in}}{\pgfqpoint{7.916625in}{3.194987in}}%
\pgfpathcurveto{\pgfqpoint{7.924438in}{3.202801in}}{\pgfqpoint{7.928828in}{3.213400in}}{\pgfqpoint{7.928828in}{3.224450in}}%
\pgfpathcurveto{\pgfqpoint{7.928828in}{3.235500in}}{\pgfqpoint{7.924438in}{3.246099in}}{\pgfqpoint{7.916625in}{3.253913in}}%
\pgfpathcurveto{\pgfqpoint{7.908811in}{3.261727in}}{\pgfqpoint{7.898212in}{3.266117in}}{\pgfqpoint{7.887162in}{3.266117in}}%
\pgfpathcurveto{\pgfqpoint{7.876112in}{3.266117in}}{\pgfqpoint{7.865513in}{3.261727in}}{\pgfqpoint{7.857699in}{3.253913in}}%
\pgfpathcurveto{\pgfqpoint{7.849885in}{3.246099in}}{\pgfqpoint{7.845495in}{3.235500in}}{\pgfqpoint{7.845495in}{3.224450in}}%
\pgfpathcurveto{\pgfqpoint{7.845495in}{3.213400in}}{\pgfqpoint{7.849885in}{3.202801in}}{\pgfqpoint{7.857699in}{3.194987in}}%
\pgfpathcurveto{\pgfqpoint{7.865513in}{3.187174in}}{\pgfqpoint{7.876112in}{3.182783in}}{\pgfqpoint{7.887162in}{3.182783in}}%
\pgfpathlineto{\pgfqpoint{7.887162in}{3.182783in}}%
\pgfpathclose%
\pgfusepath{stroke,fill}%
\end{pgfscope}%
\begin{pgfscope}%
\pgfpathrectangle{\pgfqpoint{7.622482in}{2.920818in}}{\pgfqpoint{2.177280in}{2.201755in}}%
\pgfusepath{clip}%
\pgfsetbuttcap%
\pgfsetroundjoin%
\definecolor{currentfill}{rgb}{0.121569,0.466667,0.705882}%
\pgfsetfillcolor{currentfill}%
\pgfsetlinewidth{0.481800pt}%
\definecolor{currentstroke}{rgb}{1.000000,1.000000,1.000000}%
\pgfsetstrokecolor{currentstroke}%
\pgfsetdash{}{0pt}%
\pgfpathmoveto{\pgfqpoint{7.819408in}{3.114933in}}%
\pgfpathcurveto{\pgfqpoint{7.830458in}{3.114933in}}{\pgfqpoint{7.841057in}{3.119323in}}{\pgfqpoint{7.848871in}{3.127137in}}%
\pgfpathcurveto{\pgfqpoint{7.856684in}{3.134950in}}{\pgfqpoint{7.861075in}{3.145549in}}{\pgfqpoint{7.861075in}{3.156599in}}%
\pgfpathcurveto{\pgfqpoint{7.861075in}{3.167650in}}{\pgfqpoint{7.856684in}{3.178249in}}{\pgfqpoint{7.848871in}{3.186062in}}%
\pgfpathcurveto{\pgfqpoint{7.841057in}{3.193876in}}{\pgfqpoint{7.830458in}{3.198266in}}{\pgfqpoint{7.819408in}{3.198266in}}%
\pgfpathcurveto{\pgfqpoint{7.808358in}{3.198266in}}{\pgfqpoint{7.797759in}{3.193876in}}{\pgfqpoint{7.789945in}{3.186062in}}%
\pgfpathcurveto{\pgfqpoint{7.782132in}{3.178249in}}{\pgfqpoint{7.777741in}{3.167650in}}{\pgfqpoint{7.777741in}{3.156599in}}%
\pgfpathcurveto{\pgfqpoint{7.777741in}{3.145549in}}{\pgfqpoint{7.782132in}{3.134950in}}{\pgfqpoint{7.789945in}{3.127137in}}%
\pgfpathcurveto{\pgfqpoint{7.797759in}{3.119323in}}{\pgfqpoint{7.808358in}{3.114933in}}{\pgfqpoint{7.819408in}{3.114933in}}%
\pgfpathlineto{\pgfqpoint{7.819408in}{3.114933in}}%
\pgfpathclose%
\pgfusepath{stroke,fill}%
\end{pgfscope}%
\begin{pgfscope}%
\pgfpathrectangle{\pgfqpoint{7.622482in}{2.920818in}}{\pgfqpoint{2.177280in}{2.201755in}}%
\pgfusepath{clip}%
\pgfsetbuttcap%
\pgfsetroundjoin%
\definecolor{currentfill}{rgb}{0.121569,0.466667,0.705882}%
\pgfsetfillcolor{currentfill}%
\pgfsetlinewidth{0.481800pt}%
\definecolor{currentstroke}{rgb}{1.000000,1.000000,1.000000}%
\pgfsetstrokecolor{currentstroke}%
\pgfsetdash{}{0pt}%
\pgfpathmoveto{\pgfqpoint{7.819408in}{3.013157in}}%
\pgfpathcurveto{\pgfqpoint{7.830458in}{3.013157in}}{\pgfqpoint{7.841057in}{3.017547in}}{\pgfqpoint{7.848871in}{3.025361in}}%
\pgfpathcurveto{\pgfqpoint{7.856684in}{3.033174in}}{\pgfqpoint{7.861075in}{3.043773in}}{\pgfqpoint{7.861075in}{3.054823in}}%
\pgfpathcurveto{\pgfqpoint{7.861075in}{3.065874in}}{\pgfqpoint{7.856684in}{3.076473in}}{\pgfqpoint{7.848871in}{3.084286in}}%
\pgfpathcurveto{\pgfqpoint{7.841057in}{3.092100in}}{\pgfqpoint{7.830458in}{3.096490in}}{\pgfqpoint{7.819408in}{3.096490in}}%
\pgfpathcurveto{\pgfqpoint{7.808358in}{3.096490in}}{\pgfqpoint{7.797759in}{3.092100in}}{\pgfqpoint{7.789945in}{3.084286in}}%
\pgfpathcurveto{\pgfqpoint{7.782132in}{3.076473in}}{\pgfqpoint{7.777741in}{3.065874in}}{\pgfqpoint{7.777741in}{3.054823in}}%
\pgfpathcurveto{\pgfqpoint{7.777741in}{3.043773in}}{\pgfqpoint{7.782132in}{3.033174in}}{\pgfqpoint{7.789945in}{3.025361in}}%
\pgfpathcurveto{\pgfqpoint{7.797759in}{3.017547in}}{\pgfqpoint{7.808358in}{3.013157in}}{\pgfqpoint{7.819408in}{3.013157in}}%
\pgfpathlineto{\pgfqpoint{7.819408in}{3.013157in}}%
\pgfpathclose%
\pgfusepath{stroke,fill}%
\end{pgfscope}%
\begin{pgfscope}%
\pgfpathrectangle{\pgfqpoint{7.622482in}{2.920818in}}{\pgfqpoint{2.177280in}{2.201755in}}%
\pgfusepath{clip}%
\pgfsetbuttcap%
\pgfsetroundjoin%
\definecolor{currentfill}{rgb}{0.121569,0.466667,0.705882}%
\pgfsetfillcolor{currentfill}%
\pgfsetlinewidth{0.481800pt}%
\definecolor{currentstroke}{rgb}{1.000000,1.000000,1.000000}%
\pgfsetstrokecolor{currentstroke}%
\pgfsetdash{}{0pt}%
\pgfpathmoveto{\pgfqpoint{7.887162in}{3.047082in}}%
\pgfpathcurveto{\pgfqpoint{7.898212in}{3.047082in}}{\pgfqpoint{7.908811in}{3.051472in}}{\pgfqpoint{7.916625in}{3.059286in}}%
\pgfpathcurveto{\pgfqpoint{7.924438in}{3.067100in}}{\pgfqpoint{7.928828in}{3.077699in}}{\pgfqpoint{7.928828in}{3.088749in}}%
\pgfpathcurveto{\pgfqpoint{7.928828in}{3.099799in}}{\pgfqpoint{7.924438in}{3.110398in}}{\pgfqpoint{7.916625in}{3.118212in}}%
\pgfpathcurveto{\pgfqpoint{7.908811in}{3.126025in}}{\pgfqpoint{7.898212in}{3.130415in}}{\pgfqpoint{7.887162in}{3.130415in}}%
\pgfpathcurveto{\pgfqpoint{7.876112in}{3.130415in}}{\pgfqpoint{7.865513in}{3.126025in}}{\pgfqpoint{7.857699in}{3.118212in}}%
\pgfpathcurveto{\pgfqpoint{7.849885in}{3.110398in}}{\pgfqpoint{7.845495in}{3.099799in}}{\pgfqpoint{7.845495in}{3.088749in}}%
\pgfpathcurveto{\pgfqpoint{7.845495in}{3.077699in}}{\pgfqpoint{7.849885in}{3.067100in}}{\pgfqpoint{7.857699in}{3.059286in}}%
\pgfpathcurveto{\pgfqpoint{7.865513in}{3.051472in}}{\pgfqpoint{7.876112in}{3.047082in}}{\pgfqpoint{7.887162in}{3.047082in}}%
\pgfpathlineto{\pgfqpoint{7.887162in}{3.047082in}}%
\pgfpathclose%
\pgfusepath{stroke,fill}%
\end{pgfscope}%
\begin{pgfscope}%
\pgfpathrectangle{\pgfqpoint{7.622482in}{2.920818in}}{\pgfqpoint{2.177280in}{2.201755in}}%
\pgfusepath{clip}%
\pgfsetbuttcap%
\pgfsetroundjoin%
\definecolor{currentfill}{rgb}{0.121569,0.466667,0.705882}%
\pgfsetfillcolor{currentfill}%
\pgfsetlinewidth{0.481800pt}%
\definecolor{currentstroke}{rgb}{1.000000,1.000000,1.000000}%
\pgfsetstrokecolor{currentstroke}%
\pgfsetdash{}{0pt}%
\pgfpathmoveto{\pgfqpoint{8.022670in}{3.148858in}}%
\pgfpathcurveto{\pgfqpoint{8.033720in}{3.148858in}}{\pgfqpoint{8.044319in}{3.153248in}}{\pgfqpoint{8.052132in}{3.161062in}}%
\pgfpathcurveto{\pgfqpoint{8.059946in}{3.168876in}}{\pgfqpoint{8.064336in}{3.179475in}}{\pgfqpoint{8.064336in}{3.190525in}}%
\pgfpathcurveto{\pgfqpoint{8.064336in}{3.201575in}}{\pgfqpoint{8.059946in}{3.212174in}}{\pgfqpoint{8.052132in}{3.219988in}}%
\pgfpathcurveto{\pgfqpoint{8.044319in}{3.227801in}}{\pgfqpoint{8.033720in}{3.232191in}}{\pgfqpoint{8.022670in}{3.232191in}}%
\pgfpathcurveto{\pgfqpoint{8.011619in}{3.232191in}}{\pgfqpoint{8.001020in}{3.227801in}}{\pgfqpoint{7.993207in}{3.219988in}}%
\pgfpathcurveto{\pgfqpoint{7.985393in}{3.212174in}}{\pgfqpoint{7.981003in}{3.201575in}}{\pgfqpoint{7.981003in}{3.190525in}}%
\pgfpathcurveto{\pgfqpoint{7.981003in}{3.179475in}}{\pgfqpoint{7.985393in}{3.168876in}}{\pgfqpoint{7.993207in}{3.161062in}}%
\pgfpathcurveto{\pgfqpoint{8.001020in}{3.153248in}}{\pgfqpoint{8.011619in}{3.148858in}}{\pgfqpoint{8.022670in}{3.148858in}}%
\pgfpathlineto{\pgfqpoint{8.022670in}{3.148858in}}%
\pgfpathclose%
\pgfusepath{stroke,fill}%
\end{pgfscope}%
\begin{pgfscope}%
\pgfpathrectangle{\pgfqpoint{7.622482in}{2.920818in}}{\pgfqpoint{2.177280in}{2.201755in}}%
\pgfusepath{clip}%
\pgfsetbuttcap%
\pgfsetroundjoin%
\definecolor{currentfill}{rgb}{0.121569,0.466667,0.705882}%
\pgfsetfillcolor{currentfill}%
\pgfsetlinewidth{0.481800pt}%
\definecolor{currentstroke}{rgb}{1.000000,1.000000,1.000000}%
\pgfsetstrokecolor{currentstroke}%
\pgfsetdash{}{0pt}%
\pgfpathmoveto{\pgfqpoint{8.022670in}{3.081007in}}%
\pgfpathcurveto{\pgfqpoint{8.033720in}{3.081007in}}{\pgfqpoint{8.044319in}{3.085398in}}{\pgfqpoint{8.052132in}{3.093211in}}%
\pgfpathcurveto{\pgfqpoint{8.059946in}{3.101025in}}{\pgfqpoint{8.064336in}{3.111624in}}{\pgfqpoint{8.064336in}{3.122674in}}%
\pgfpathcurveto{\pgfqpoint{8.064336in}{3.133724in}}{\pgfqpoint{8.059946in}{3.144323in}}{\pgfqpoint{8.052132in}{3.152137in}}%
\pgfpathcurveto{\pgfqpoint{8.044319in}{3.159951in}}{\pgfqpoint{8.033720in}{3.164341in}}{\pgfqpoint{8.022670in}{3.164341in}}%
\pgfpathcurveto{\pgfqpoint{8.011619in}{3.164341in}}{\pgfqpoint{8.001020in}{3.159951in}}{\pgfqpoint{7.993207in}{3.152137in}}%
\pgfpathcurveto{\pgfqpoint{7.985393in}{3.144323in}}{\pgfqpoint{7.981003in}{3.133724in}}{\pgfqpoint{7.981003in}{3.122674in}}%
\pgfpathcurveto{\pgfqpoint{7.981003in}{3.111624in}}{\pgfqpoint{7.985393in}{3.101025in}}{\pgfqpoint{7.993207in}{3.093211in}}%
\pgfpathcurveto{\pgfqpoint{8.001020in}{3.085398in}}{\pgfqpoint{8.011619in}{3.081007in}}{\pgfqpoint{8.022670in}{3.081007in}}%
\pgfpathlineto{\pgfqpoint{8.022670in}{3.081007in}}%
\pgfpathclose%
\pgfusepath{stroke,fill}%
\end{pgfscope}%
\begin{pgfscope}%
\pgfpathrectangle{\pgfqpoint{7.622482in}{2.920818in}}{\pgfqpoint{2.177280in}{2.201755in}}%
\pgfusepath{clip}%
\pgfsetbuttcap%
\pgfsetroundjoin%
\definecolor{currentfill}{rgb}{0.121569,0.466667,0.705882}%
\pgfsetfillcolor{currentfill}%
\pgfsetlinewidth{0.481800pt}%
\definecolor{currentstroke}{rgb}{1.000000,1.000000,1.000000}%
\pgfsetstrokecolor{currentstroke}%
\pgfsetdash{}{0pt}%
\pgfpathmoveto{\pgfqpoint{7.954916in}{3.114933in}}%
\pgfpathcurveto{\pgfqpoint{7.965966in}{3.114933in}}{\pgfqpoint{7.976565in}{3.119323in}}{\pgfqpoint{7.984378in}{3.127137in}}%
\pgfpathcurveto{\pgfqpoint{7.992192in}{3.134950in}}{\pgfqpoint{7.996582in}{3.145549in}}{\pgfqpoint{7.996582in}{3.156599in}}%
\pgfpathcurveto{\pgfqpoint{7.996582in}{3.167650in}}{\pgfqpoint{7.992192in}{3.178249in}}{\pgfqpoint{7.984378in}{3.186062in}}%
\pgfpathcurveto{\pgfqpoint{7.976565in}{3.193876in}}{\pgfqpoint{7.965966in}{3.198266in}}{\pgfqpoint{7.954916in}{3.198266in}}%
\pgfpathcurveto{\pgfqpoint{7.943866in}{3.198266in}}{\pgfqpoint{7.933267in}{3.193876in}}{\pgfqpoint{7.925453in}{3.186062in}}%
\pgfpathcurveto{\pgfqpoint{7.917639in}{3.178249in}}{\pgfqpoint{7.913249in}{3.167650in}}{\pgfqpoint{7.913249in}{3.156599in}}%
\pgfpathcurveto{\pgfqpoint{7.913249in}{3.145549in}}{\pgfqpoint{7.917639in}{3.134950in}}{\pgfqpoint{7.925453in}{3.127137in}}%
\pgfpathcurveto{\pgfqpoint{7.933267in}{3.119323in}}{\pgfqpoint{7.943866in}{3.114933in}}{\pgfqpoint{7.954916in}{3.114933in}}%
\pgfpathlineto{\pgfqpoint{7.954916in}{3.114933in}}%
\pgfpathclose%
\pgfusepath{stroke,fill}%
\end{pgfscope}%
\begin{pgfscope}%
\pgfpathrectangle{\pgfqpoint{7.622482in}{2.920818in}}{\pgfqpoint{2.177280in}{2.201755in}}%
\pgfusepath{clip}%
\pgfsetbuttcap%
\pgfsetroundjoin%
\definecolor{currentfill}{rgb}{0.121569,0.466667,0.705882}%
\pgfsetfillcolor{currentfill}%
\pgfsetlinewidth{0.481800pt}%
\definecolor{currentstroke}{rgb}{1.000000,1.000000,1.000000}%
\pgfsetstrokecolor{currentstroke}%
\pgfsetdash{}{0pt}%
\pgfpathmoveto{\pgfqpoint{7.954916in}{3.216709in}}%
\pgfpathcurveto{\pgfqpoint{7.965966in}{3.216709in}}{\pgfqpoint{7.976565in}{3.221099in}}{\pgfqpoint{7.984378in}{3.228913in}}%
\pgfpathcurveto{\pgfqpoint{7.992192in}{3.236726in}}{\pgfqpoint{7.996582in}{3.247325in}}{\pgfqpoint{7.996582in}{3.258375in}}%
\pgfpathcurveto{\pgfqpoint{7.996582in}{3.269426in}}{\pgfqpoint{7.992192in}{3.280025in}}{\pgfqpoint{7.984378in}{3.287838in}}%
\pgfpathcurveto{\pgfqpoint{7.976565in}{3.295652in}}{\pgfqpoint{7.965966in}{3.300042in}}{\pgfqpoint{7.954916in}{3.300042in}}%
\pgfpathcurveto{\pgfqpoint{7.943866in}{3.300042in}}{\pgfqpoint{7.933267in}{3.295652in}}{\pgfqpoint{7.925453in}{3.287838in}}%
\pgfpathcurveto{\pgfqpoint{7.917639in}{3.280025in}}{\pgfqpoint{7.913249in}{3.269426in}}{\pgfqpoint{7.913249in}{3.258375in}}%
\pgfpathcurveto{\pgfqpoint{7.913249in}{3.247325in}}{\pgfqpoint{7.917639in}{3.236726in}}{\pgfqpoint{7.925453in}{3.228913in}}%
\pgfpathcurveto{\pgfqpoint{7.933267in}{3.221099in}}{\pgfqpoint{7.943866in}{3.216709in}}{\pgfqpoint{7.954916in}{3.216709in}}%
\pgfpathlineto{\pgfqpoint{7.954916in}{3.216709in}}%
\pgfpathclose%
\pgfusepath{stroke,fill}%
\end{pgfscope}%
\begin{pgfscope}%
\pgfpathrectangle{\pgfqpoint{7.622482in}{2.920818in}}{\pgfqpoint{2.177280in}{2.201755in}}%
\pgfusepath{clip}%
\pgfsetbuttcap%
\pgfsetroundjoin%
\definecolor{currentfill}{rgb}{0.121569,0.466667,0.705882}%
\pgfsetfillcolor{currentfill}%
\pgfsetlinewidth{0.481800pt}%
\definecolor{currentstroke}{rgb}{1.000000,1.000000,1.000000}%
\pgfsetstrokecolor{currentstroke}%
\pgfsetdash{}{0pt}%
\pgfpathmoveto{\pgfqpoint{7.954916in}{3.148858in}}%
\pgfpathcurveto{\pgfqpoint{7.965966in}{3.148858in}}{\pgfqpoint{7.976565in}{3.153248in}}{\pgfqpoint{7.984378in}{3.161062in}}%
\pgfpathcurveto{\pgfqpoint{7.992192in}{3.168876in}}{\pgfqpoint{7.996582in}{3.179475in}}{\pgfqpoint{7.996582in}{3.190525in}}%
\pgfpathcurveto{\pgfqpoint{7.996582in}{3.201575in}}{\pgfqpoint{7.992192in}{3.212174in}}{\pgfqpoint{7.984378in}{3.219988in}}%
\pgfpathcurveto{\pgfqpoint{7.976565in}{3.227801in}}{\pgfqpoint{7.965966in}{3.232191in}}{\pgfqpoint{7.954916in}{3.232191in}}%
\pgfpathcurveto{\pgfqpoint{7.943866in}{3.232191in}}{\pgfqpoint{7.933267in}{3.227801in}}{\pgfqpoint{7.925453in}{3.219988in}}%
\pgfpathcurveto{\pgfqpoint{7.917639in}{3.212174in}}{\pgfqpoint{7.913249in}{3.201575in}}{\pgfqpoint{7.913249in}{3.190525in}}%
\pgfpathcurveto{\pgfqpoint{7.913249in}{3.179475in}}{\pgfqpoint{7.917639in}{3.168876in}}{\pgfqpoint{7.925453in}{3.161062in}}%
\pgfpathcurveto{\pgfqpoint{7.933267in}{3.153248in}}{\pgfqpoint{7.943866in}{3.148858in}}{\pgfqpoint{7.954916in}{3.148858in}}%
\pgfpathlineto{\pgfqpoint{7.954916in}{3.148858in}}%
\pgfpathclose%
\pgfusepath{stroke,fill}%
\end{pgfscope}%
\begin{pgfscope}%
\pgfpathrectangle{\pgfqpoint{7.622482in}{2.920818in}}{\pgfqpoint{2.177280in}{2.201755in}}%
\pgfusepath{clip}%
\pgfsetbuttcap%
\pgfsetroundjoin%
\definecolor{currentfill}{rgb}{0.121569,0.466667,0.705882}%
\pgfsetfillcolor{currentfill}%
\pgfsetlinewidth{0.481800pt}%
\definecolor{currentstroke}{rgb}{1.000000,1.000000,1.000000}%
\pgfsetstrokecolor{currentstroke}%
\pgfsetdash{}{0pt}%
\pgfpathmoveto{\pgfqpoint{7.887162in}{3.216709in}}%
\pgfpathcurveto{\pgfqpoint{7.898212in}{3.216709in}}{\pgfqpoint{7.908811in}{3.221099in}}{\pgfqpoint{7.916625in}{3.228913in}}%
\pgfpathcurveto{\pgfqpoint{7.924438in}{3.236726in}}{\pgfqpoint{7.928828in}{3.247325in}}{\pgfqpoint{7.928828in}{3.258375in}}%
\pgfpathcurveto{\pgfqpoint{7.928828in}{3.269426in}}{\pgfqpoint{7.924438in}{3.280025in}}{\pgfqpoint{7.916625in}{3.287838in}}%
\pgfpathcurveto{\pgfqpoint{7.908811in}{3.295652in}}{\pgfqpoint{7.898212in}{3.300042in}}{\pgfqpoint{7.887162in}{3.300042in}}%
\pgfpathcurveto{\pgfqpoint{7.876112in}{3.300042in}}{\pgfqpoint{7.865513in}{3.295652in}}{\pgfqpoint{7.857699in}{3.287838in}}%
\pgfpathcurveto{\pgfqpoint{7.849885in}{3.280025in}}{\pgfqpoint{7.845495in}{3.269426in}}{\pgfqpoint{7.845495in}{3.258375in}}%
\pgfpathcurveto{\pgfqpoint{7.845495in}{3.247325in}}{\pgfqpoint{7.849885in}{3.236726in}}{\pgfqpoint{7.857699in}{3.228913in}}%
\pgfpathcurveto{\pgfqpoint{7.865513in}{3.221099in}}{\pgfqpoint{7.876112in}{3.216709in}}{\pgfqpoint{7.887162in}{3.216709in}}%
\pgfpathlineto{\pgfqpoint{7.887162in}{3.216709in}}%
\pgfpathclose%
\pgfusepath{stroke,fill}%
\end{pgfscope}%
\begin{pgfscope}%
\pgfpathrectangle{\pgfqpoint{7.622482in}{2.920818in}}{\pgfqpoint{2.177280in}{2.201755in}}%
\pgfusepath{clip}%
\pgfsetbuttcap%
\pgfsetroundjoin%
\definecolor{currentfill}{rgb}{0.121569,0.466667,0.705882}%
\pgfsetfillcolor{currentfill}%
\pgfsetlinewidth{0.481800pt}%
\definecolor{currentstroke}{rgb}{1.000000,1.000000,1.000000}%
\pgfsetstrokecolor{currentstroke}%
\pgfsetdash{}{0pt}%
\pgfpathmoveto{\pgfqpoint{8.022670in}{3.148858in}}%
\pgfpathcurveto{\pgfqpoint{8.033720in}{3.148858in}}{\pgfqpoint{8.044319in}{3.153248in}}{\pgfqpoint{8.052132in}{3.161062in}}%
\pgfpathcurveto{\pgfqpoint{8.059946in}{3.168876in}}{\pgfqpoint{8.064336in}{3.179475in}}{\pgfqpoint{8.064336in}{3.190525in}}%
\pgfpathcurveto{\pgfqpoint{8.064336in}{3.201575in}}{\pgfqpoint{8.059946in}{3.212174in}}{\pgfqpoint{8.052132in}{3.219988in}}%
\pgfpathcurveto{\pgfqpoint{8.044319in}{3.227801in}}{\pgfqpoint{8.033720in}{3.232191in}}{\pgfqpoint{8.022670in}{3.232191in}}%
\pgfpathcurveto{\pgfqpoint{8.011619in}{3.232191in}}{\pgfqpoint{8.001020in}{3.227801in}}{\pgfqpoint{7.993207in}{3.219988in}}%
\pgfpathcurveto{\pgfqpoint{7.985393in}{3.212174in}}{\pgfqpoint{7.981003in}{3.201575in}}{\pgfqpoint{7.981003in}{3.190525in}}%
\pgfpathcurveto{\pgfqpoint{7.981003in}{3.179475in}}{\pgfqpoint{7.985393in}{3.168876in}}{\pgfqpoint{7.993207in}{3.161062in}}%
\pgfpathcurveto{\pgfqpoint{8.001020in}{3.153248in}}{\pgfqpoint{8.011619in}{3.148858in}}{\pgfqpoint{8.022670in}{3.148858in}}%
\pgfpathlineto{\pgfqpoint{8.022670in}{3.148858in}}%
\pgfpathclose%
\pgfusepath{stroke,fill}%
\end{pgfscope}%
\begin{pgfscope}%
\pgfpathrectangle{\pgfqpoint{7.622482in}{2.920818in}}{\pgfqpoint{2.177280in}{2.201755in}}%
\pgfusepath{clip}%
\pgfsetbuttcap%
\pgfsetroundjoin%
\definecolor{currentfill}{rgb}{0.121569,0.466667,0.705882}%
\pgfsetfillcolor{currentfill}%
\pgfsetlinewidth{0.481800pt}%
\definecolor{currentstroke}{rgb}{1.000000,1.000000,1.000000}%
\pgfsetstrokecolor{currentstroke}%
\pgfsetdash{}{0pt}%
\pgfpathmoveto{\pgfqpoint{7.887162in}{2.979231in}}%
\pgfpathcurveto{\pgfqpoint{7.898212in}{2.979231in}}{\pgfqpoint{7.908811in}{2.983622in}}{\pgfqpoint{7.916625in}{2.991435in}}%
\pgfpathcurveto{\pgfqpoint{7.924438in}{2.999249in}}{\pgfqpoint{7.928828in}{3.009848in}}{\pgfqpoint{7.928828in}{3.020898in}}%
\pgfpathcurveto{\pgfqpoint{7.928828in}{3.031948in}}{\pgfqpoint{7.924438in}{3.042547in}}{\pgfqpoint{7.916625in}{3.050361in}}%
\pgfpathcurveto{\pgfqpoint{7.908811in}{3.058174in}}{\pgfqpoint{7.898212in}{3.062565in}}{\pgfqpoint{7.887162in}{3.062565in}}%
\pgfpathcurveto{\pgfqpoint{7.876112in}{3.062565in}}{\pgfqpoint{7.865513in}{3.058174in}}{\pgfqpoint{7.857699in}{3.050361in}}%
\pgfpathcurveto{\pgfqpoint{7.849885in}{3.042547in}}{\pgfqpoint{7.845495in}{3.031948in}}{\pgfqpoint{7.845495in}{3.020898in}}%
\pgfpathcurveto{\pgfqpoint{7.845495in}{3.009848in}}{\pgfqpoint{7.849885in}{2.999249in}}{\pgfqpoint{7.857699in}{2.991435in}}%
\pgfpathcurveto{\pgfqpoint{7.865513in}{2.983622in}}{\pgfqpoint{7.876112in}{2.979231in}}{\pgfqpoint{7.887162in}{2.979231in}}%
\pgfpathlineto{\pgfqpoint{7.887162in}{2.979231in}}%
\pgfpathclose%
\pgfusepath{stroke,fill}%
\end{pgfscope}%
\begin{pgfscope}%
\pgfpathrectangle{\pgfqpoint{7.622482in}{2.920818in}}{\pgfqpoint{2.177280in}{2.201755in}}%
\pgfusepath{clip}%
\pgfsetbuttcap%
\pgfsetroundjoin%
\definecolor{currentfill}{rgb}{0.121569,0.466667,0.705882}%
\pgfsetfillcolor{currentfill}%
\pgfsetlinewidth{0.481800pt}%
\definecolor{currentstroke}{rgb}{1.000000,1.000000,1.000000}%
\pgfsetstrokecolor{currentstroke}%
\pgfsetdash{}{0pt}%
\pgfpathmoveto{\pgfqpoint{8.090423in}{3.216709in}}%
\pgfpathcurveto{\pgfqpoint{8.101474in}{3.216709in}}{\pgfqpoint{8.112073in}{3.221099in}}{\pgfqpoint{8.119886in}{3.228913in}}%
\pgfpathcurveto{\pgfqpoint{8.127700in}{3.236726in}}{\pgfqpoint{8.132090in}{3.247325in}}{\pgfqpoint{8.132090in}{3.258375in}}%
\pgfpathcurveto{\pgfqpoint{8.132090in}{3.269426in}}{\pgfqpoint{8.127700in}{3.280025in}}{\pgfqpoint{8.119886in}{3.287838in}}%
\pgfpathcurveto{\pgfqpoint{8.112073in}{3.295652in}}{\pgfqpoint{8.101474in}{3.300042in}}{\pgfqpoint{8.090423in}{3.300042in}}%
\pgfpathcurveto{\pgfqpoint{8.079373in}{3.300042in}}{\pgfqpoint{8.068774in}{3.295652in}}{\pgfqpoint{8.060961in}{3.287838in}}%
\pgfpathcurveto{\pgfqpoint{8.053147in}{3.280025in}}{\pgfqpoint{8.048757in}{3.269426in}}{\pgfqpoint{8.048757in}{3.258375in}}%
\pgfpathcurveto{\pgfqpoint{8.048757in}{3.247325in}}{\pgfqpoint{8.053147in}{3.236726in}}{\pgfqpoint{8.060961in}{3.228913in}}%
\pgfpathcurveto{\pgfqpoint{8.068774in}{3.221099in}}{\pgfqpoint{8.079373in}{3.216709in}}{\pgfqpoint{8.090423in}{3.216709in}}%
\pgfpathlineto{\pgfqpoint{8.090423in}{3.216709in}}%
\pgfpathclose%
\pgfusepath{stroke,fill}%
\end{pgfscope}%
\begin{pgfscope}%
\pgfpathrectangle{\pgfqpoint{7.622482in}{2.920818in}}{\pgfqpoint{2.177280in}{2.201755in}}%
\pgfusepath{clip}%
\pgfsetbuttcap%
\pgfsetroundjoin%
\definecolor{currentfill}{rgb}{0.121569,0.466667,0.705882}%
\pgfsetfillcolor{currentfill}%
\pgfsetlinewidth{0.481800pt}%
\definecolor{currentstroke}{rgb}{1.000000,1.000000,1.000000}%
\pgfsetstrokecolor{currentstroke}%
\pgfsetdash{}{0pt}%
\pgfpathmoveto{\pgfqpoint{7.887162in}{3.284560in}}%
\pgfpathcurveto{\pgfqpoint{7.898212in}{3.284560in}}{\pgfqpoint{7.908811in}{3.288950in}}{\pgfqpoint{7.916625in}{3.296763in}}%
\pgfpathcurveto{\pgfqpoint{7.924438in}{3.304577in}}{\pgfqpoint{7.928828in}{3.315176in}}{\pgfqpoint{7.928828in}{3.326226in}}%
\pgfpathcurveto{\pgfqpoint{7.928828in}{3.337276in}}{\pgfqpoint{7.924438in}{3.347875in}}{\pgfqpoint{7.916625in}{3.355689in}}%
\pgfpathcurveto{\pgfqpoint{7.908811in}{3.363503in}}{\pgfqpoint{7.898212in}{3.367893in}}{\pgfqpoint{7.887162in}{3.367893in}}%
\pgfpathcurveto{\pgfqpoint{7.876112in}{3.367893in}}{\pgfqpoint{7.865513in}{3.363503in}}{\pgfqpoint{7.857699in}{3.355689in}}%
\pgfpathcurveto{\pgfqpoint{7.849885in}{3.347875in}}{\pgfqpoint{7.845495in}{3.337276in}}{\pgfqpoint{7.845495in}{3.326226in}}%
\pgfpathcurveto{\pgfqpoint{7.845495in}{3.315176in}}{\pgfqpoint{7.849885in}{3.304577in}}{\pgfqpoint{7.857699in}{3.296763in}}%
\pgfpathcurveto{\pgfqpoint{7.865513in}{3.288950in}}{\pgfqpoint{7.876112in}{3.284560in}}{\pgfqpoint{7.887162in}{3.284560in}}%
\pgfpathlineto{\pgfqpoint{7.887162in}{3.284560in}}%
\pgfpathclose%
\pgfusepath{stroke,fill}%
\end{pgfscope}%
\begin{pgfscope}%
\pgfpathrectangle{\pgfqpoint{7.622482in}{2.920818in}}{\pgfqpoint{2.177280in}{2.201755in}}%
\pgfusepath{clip}%
\pgfsetbuttcap%
\pgfsetroundjoin%
\definecolor{currentfill}{rgb}{0.121569,0.466667,0.705882}%
\pgfsetfillcolor{currentfill}%
\pgfsetlinewidth{0.481800pt}%
\definecolor{currentstroke}{rgb}{1.000000,1.000000,1.000000}%
\pgfsetstrokecolor{currentstroke}%
\pgfsetdash{}{0pt}%
\pgfpathmoveto{\pgfqpoint{7.887162in}{3.182783in}}%
\pgfpathcurveto{\pgfqpoint{7.898212in}{3.182783in}}{\pgfqpoint{7.908811in}{3.187174in}}{\pgfqpoint{7.916625in}{3.194987in}}%
\pgfpathcurveto{\pgfqpoint{7.924438in}{3.202801in}}{\pgfqpoint{7.928828in}{3.213400in}}{\pgfqpoint{7.928828in}{3.224450in}}%
\pgfpathcurveto{\pgfqpoint{7.928828in}{3.235500in}}{\pgfqpoint{7.924438in}{3.246099in}}{\pgfqpoint{7.916625in}{3.253913in}}%
\pgfpathcurveto{\pgfqpoint{7.908811in}{3.261727in}}{\pgfqpoint{7.898212in}{3.266117in}}{\pgfqpoint{7.887162in}{3.266117in}}%
\pgfpathcurveto{\pgfqpoint{7.876112in}{3.266117in}}{\pgfqpoint{7.865513in}{3.261727in}}{\pgfqpoint{7.857699in}{3.253913in}}%
\pgfpathcurveto{\pgfqpoint{7.849885in}{3.246099in}}{\pgfqpoint{7.845495in}{3.235500in}}{\pgfqpoint{7.845495in}{3.224450in}}%
\pgfpathcurveto{\pgfqpoint{7.845495in}{3.213400in}}{\pgfqpoint{7.849885in}{3.202801in}}{\pgfqpoint{7.857699in}{3.194987in}}%
\pgfpathcurveto{\pgfqpoint{7.865513in}{3.187174in}}{\pgfqpoint{7.876112in}{3.182783in}}{\pgfqpoint{7.887162in}{3.182783in}}%
\pgfpathlineto{\pgfqpoint{7.887162in}{3.182783in}}%
\pgfpathclose%
\pgfusepath{stroke,fill}%
\end{pgfscope}%
\begin{pgfscope}%
\pgfpathrectangle{\pgfqpoint{7.622482in}{2.920818in}}{\pgfqpoint{2.177280in}{2.201755in}}%
\pgfusepath{clip}%
\pgfsetbuttcap%
\pgfsetroundjoin%
\definecolor{currentfill}{rgb}{0.121569,0.466667,0.705882}%
\pgfsetfillcolor{currentfill}%
\pgfsetlinewidth{0.481800pt}%
\definecolor{currentstroke}{rgb}{1.000000,1.000000,1.000000}%
\pgfsetstrokecolor{currentstroke}%
\pgfsetdash{}{0pt}%
\pgfpathmoveto{\pgfqpoint{8.022670in}{3.182783in}}%
\pgfpathcurveto{\pgfqpoint{8.033720in}{3.182783in}}{\pgfqpoint{8.044319in}{3.187174in}}{\pgfqpoint{8.052132in}{3.194987in}}%
\pgfpathcurveto{\pgfqpoint{8.059946in}{3.202801in}}{\pgfqpoint{8.064336in}{3.213400in}}{\pgfqpoint{8.064336in}{3.224450in}}%
\pgfpathcurveto{\pgfqpoint{8.064336in}{3.235500in}}{\pgfqpoint{8.059946in}{3.246099in}}{\pgfqpoint{8.052132in}{3.253913in}}%
\pgfpathcurveto{\pgfqpoint{8.044319in}{3.261727in}}{\pgfqpoint{8.033720in}{3.266117in}}{\pgfqpoint{8.022670in}{3.266117in}}%
\pgfpathcurveto{\pgfqpoint{8.011619in}{3.266117in}}{\pgfqpoint{8.001020in}{3.261727in}}{\pgfqpoint{7.993207in}{3.253913in}}%
\pgfpathcurveto{\pgfqpoint{7.985393in}{3.246099in}}{\pgfqpoint{7.981003in}{3.235500in}}{\pgfqpoint{7.981003in}{3.224450in}}%
\pgfpathcurveto{\pgfqpoint{7.981003in}{3.213400in}}{\pgfqpoint{7.985393in}{3.202801in}}{\pgfqpoint{7.993207in}{3.194987in}}%
\pgfpathcurveto{\pgfqpoint{8.001020in}{3.187174in}}{\pgfqpoint{8.011619in}{3.182783in}}{\pgfqpoint{8.022670in}{3.182783in}}%
\pgfpathlineto{\pgfqpoint{8.022670in}{3.182783in}}%
\pgfpathclose%
\pgfusepath{stroke,fill}%
\end{pgfscope}%
\begin{pgfscope}%
\pgfpathrectangle{\pgfqpoint{7.622482in}{2.920818in}}{\pgfqpoint{2.177280in}{2.201755in}}%
\pgfusepath{clip}%
\pgfsetbuttcap%
\pgfsetroundjoin%
\definecolor{currentfill}{rgb}{0.121569,0.466667,0.705882}%
\pgfsetfillcolor{currentfill}%
\pgfsetlinewidth{0.481800pt}%
\definecolor{currentstroke}{rgb}{1.000000,1.000000,1.000000}%
\pgfsetstrokecolor{currentstroke}%
\pgfsetdash{}{0pt}%
\pgfpathmoveto{\pgfqpoint{7.887162in}{3.148858in}}%
\pgfpathcurveto{\pgfqpoint{7.898212in}{3.148858in}}{\pgfqpoint{7.908811in}{3.153248in}}{\pgfqpoint{7.916625in}{3.161062in}}%
\pgfpathcurveto{\pgfqpoint{7.924438in}{3.168876in}}{\pgfqpoint{7.928828in}{3.179475in}}{\pgfqpoint{7.928828in}{3.190525in}}%
\pgfpathcurveto{\pgfqpoint{7.928828in}{3.201575in}}{\pgfqpoint{7.924438in}{3.212174in}}{\pgfqpoint{7.916625in}{3.219988in}}%
\pgfpathcurveto{\pgfqpoint{7.908811in}{3.227801in}}{\pgfqpoint{7.898212in}{3.232191in}}{\pgfqpoint{7.887162in}{3.232191in}}%
\pgfpathcurveto{\pgfqpoint{7.876112in}{3.232191in}}{\pgfqpoint{7.865513in}{3.227801in}}{\pgfqpoint{7.857699in}{3.219988in}}%
\pgfpathcurveto{\pgfqpoint{7.849885in}{3.212174in}}{\pgfqpoint{7.845495in}{3.201575in}}{\pgfqpoint{7.845495in}{3.190525in}}%
\pgfpathcurveto{\pgfqpoint{7.845495in}{3.179475in}}{\pgfqpoint{7.849885in}{3.168876in}}{\pgfqpoint{7.857699in}{3.161062in}}%
\pgfpathcurveto{\pgfqpoint{7.865513in}{3.153248in}}{\pgfqpoint{7.876112in}{3.148858in}}{\pgfqpoint{7.887162in}{3.148858in}}%
\pgfpathlineto{\pgfqpoint{7.887162in}{3.148858in}}%
\pgfpathclose%
\pgfusepath{stroke,fill}%
\end{pgfscope}%
\begin{pgfscope}%
\pgfpathrectangle{\pgfqpoint{7.622482in}{2.920818in}}{\pgfqpoint{2.177280in}{2.201755in}}%
\pgfusepath{clip}%
\pgfsetbuttcap%
\pgfsetroundjoin%
\definecolor{currentfill}{rgb}{0.121569,0.466667,0.705882}%
\pgfsetfillcolor{currentfill}%
\pgfsetlinewidth{0.481800pt}%
\definecolor{currentstroke}{rgb}{1.000000,1.000000,1.000000}%
\pgfsetstrokecolor{currentstroke}%
\pgfsetdash{}{0pt}%
\pgfpathmoveto{\pgfqpoint{7.887162in}{3.114933in}}%
\pgfpathcurveto{\pgfqpoint{7.898212in}{3.114933in}}{\pgfqpoint{7.908811in}{3.119323in}}{\pgfqpoint{7.916625in}{3.127137in}}%
\pgfpathcurveto{\pgfqpoint{7.924438in}{3.134950in}}{\pgfqpoint{7.928828in}{3.145549in}}{\pgfqpoint{7.928828in}{3.156599in}}%
\pgfpathcurveto{\pgfqpoint{7.928828in}{3.167650in}}{\pgfqpoint{7.924438in}{3.178249in}}{\pgfqpoint{7.916625in}{3.186062in}}%
\pgfpathcurveto{\pgfqpoint{7.908811in}{3.193876in}}{\pgfqpoint{7.898212in}{3.198266in}}{\pgfqpoint{7.887162in}{3.198266in}}%
\pgfpathcurveto{\pgfqpoint{7.876112in}{3.198266in}}{\pgfqpoint{7.865513in}{3.193876in}}{\pgfqpoint{7.857699in}{3.186062in}}%
\pgfpathcurveto{\pgfqpoint{7.849885in}{3.178249in}}{\pgfqpoint{7.845495in}{3.167650in}}{\pgfqpoint{7.845495in}{3.156599in}}%
\pgfpathcurveto{\pgfqpoint{7.845495in}{3.145549in}}{\pgfqpoint{7.849885in}{3.134950in}}{\pgfqpoint{7.857699in}{3.127137in}}%
\pgfpathcurveto{\pgfqpoint{7.865513in}{3.119323in}}{\pgfqpoint{7.876112in}{3.114933in}}{\pgfqpoint{7.887162in}{3.114933in}}%
\pgfpathlineto{\pgfqpoint{7.887162in}{3.114933in}}%
\pgfpathclose%
\pgfusepath{stroke,fill}%
\end{pgfscope}%
\begin{pgfscope}%
\pgfpathrectangle{\pgfqpoint{7.622482in}{2.920818in}}{\pgfqpoint{2.177280in}{2.201755in}}%
\pgfusepath{clip}%
\pgfsetbuttcap%
\pgfsetroundjoin%
\definecolor{currentfill}{rgb}{0.121569,0.466667,0.705882}%
\pgfsetfillcolor{currentfill}%
\pgfsetlinewidth{0.481800pt}%
\definecolor{currentstroke}{rgb}{1.000000,1.000000,1.000000}%
\pgfsetstrokecolor{currentstroke}%
\pgfsetdash{}{0pt}%
\pgfpathmoveto{\pgfqpoint{7.887162in}{3.182783in}}%
\pgfpathcurveto{\pgfqpoint{7.898212in}{3.182783in}}{\pgfqpoint{7.908811in}{3.187174in}}{\pgfqpoint{7.916625in}{3.194987in}}%
\pgfpathcurveto{\pgfqpoint{7.924438in}{3.202801in}}{\pgfqpoint{7.928828in}{3.213400in}}{\pgfqpoint{7.928828in}{3.224450in}}%
\pgfpathcurveto{\pgfqpoint{7.928828in}{3.235500in}}{\pgfqpoint{7.924438in}{3.246099in}}{\pgfqpoint{7.916625in}{3.253913in}}%
\pgfpathcurveto{\pgfqpoint{7.908811in}{3.261727in}}{\pgfqpoint{7.898212in}{3.266117in}}{\pgfqpoint{7.887162in}{3.266117in}}%
\pgfpathcurveto{\pgfqpoint{7.876112in}{3.266117in}}{\pgfqpoint{7.865513in}{3.261727in}}{\pgfqpoint{7.857699in}{3.253913in}}%
\pgfpathcurveto{\pgfqpoint{7.849885in}{3.246099in}}{\pgfqpoint{7.845495in}{3.235500in}}{\pgfqpoint{7.845495in}{3.224450in}}%
\pgfpathcurveto{\pgfqpoint{7.845495in}{3.213400in}}{\pgfqpoint{7.849885in}{3.202801in}}{\pgfqpoint{7.857699in}{3.194987in}}%
\pgfpathcurveto{\pgfqpoint{7.865513in}{3.187174in}}{\pgfqpoint{7.876112in}{3.182783in}}{\pgfqpoint{7.887162in}{3.182783in}}%
\pgfpathlineto{\pgfqpoint{7.887162in}{3.182783in}}%
\pgfpathclose%
\pgfusepath{stroke,fill}%
\end{pgfscope}%
\begin{pgfscope}%
\pgfpathrectangle{\pgfqpoint{7.622482in}{2.920818in}}{\pgfqpoint{2.177280in}{2.201755in}}%
\pgfusepath{clip}%
\pgfsetbuttcap%
\pgfsetroundjoin%
\definecolor{currentfill}{rgb}{0.121569,0.466667,0.705882}%
\pgfsetfillcolor{currentfill}%
\pgfsetlinewidth{0.481800pt}%
\definecolor{currentstroke}{rgb}{1.000000,1.000000,1.000000}%
\pgfsetstrokecolor{currentstroke}%
\pgfsetdash{}{0pt}%
\pgfpathmoveto{\pgfqpoint{7.887162in}{3.182783in}}%
\pgfpathcurveto{\pgfqpoint{7.898212in}{3.182783in}}{\pgfqpoint{7.908811in}{3.187174in}}{\pgfqpoint{7.916625in}{3.194987in}}%
\pgfpathcurveto{\pgfqpoint{7.924438in}{3.202801in}}{\pgfqpoint{7.928828in}{3.213400in}}{\pgfqpoint{7.928828in}{3.224450in}}%
\pgfpathcurveto{\pgfqpoint{7.928828in}{3.235500in}}{\pgfqpoint{7.924438in}{3.246099in}}{\pgfqpoint{7.916625in}{3.253913in}}%
\pgfpathcurveto{\pgfqpoint{7.908811in}{3.261727in}}{\pgfqpoint{7.898212in}{3.266117in}}{\pgfqpoint{7.887162in}{3.266117in}}%
\pgfpathcurveto{\pgfqpoint{7.876112in}{3.266117in}}{\pgfqpoint{7.865513in}{3.261727in}}{\pgfqpoint{7.857699in}{3.253913in}}%
\pgfpathcurveto{\pgfqpoint{7.849885in}{3.246099in}}{\pgfqpoint{7.845495in}{3.235500in}}{\pgfqpoint{7.845495in}{3.224450in}}%
\pgfpathcurveto{\pgfqpoint{7.845495in}{3.213400in}}{\pgfqpoint{7.849885in}{3.202801in}}{\pgfqpoint{7.857699in}{3.194987in}}%
\pgfpathcurveto{\pgfqpoint{7.865513in}{3.187174in}}{\pgfqpoint{7.876112in}{3.182783in}}{\pgfqpoint{7.887162in}{3.182783in}}%
\pgfpathlineto{\pgfqpoint{7.887162in}{3.182783in}}%
\pgfpathclose%
\pgfusepath{stroke,fill}%
\end{pgfscope}%
\begin{pgfscope}%
\pgfpathrectangle{\pgfqpoint{7.622482in}{2.920818in}}{\pgfqpoint{2.177280in}{2.201755in}}%
\pgfusepath{clip}%
\pgfsetbuttcap%
\pgfsetroundjoin%
\definecolor{currentfill}{rgb}{0.121569,0.466667,0.705882}%
\pgfsetfillcolor{currentfill}%
\pgfsetlinewidth{0.481800pt}%
\definecolor{currentstroke}{rgb}{1.000000,1.000000,1.000000}%
\pgfsetstrokecolor{currentstroke}%
\pgfsetdash{}{0pt}%
\pgfpathmoveto{\pgfqpoint{8.022670in}{3.148858in}}%
\pgfpathcurveto{\pgfqpoint{8.033720in}{3.148858in}}{\pgfqpoint{8.044319in}{3.153248in}}{\pgfqpoint{8.052132in}{3.161062in}}%
\pgfpathcurveto{\pgfqpoint{8.059946in}{3.168876in}}{\pgfqpoint{8.064336in}{3.179475in}}{\pgfqpoint{8.064336in}{3.190525in}}%
\pgfpathcurveto{\pgfqpoint{8.064336in}{3.201575in}}{\pgfqpoint{8.059946in}{3.212174in}}{\pgfqpoint{8.052132in}{3.219988in}}%
\pgfpathcurveto{\pgfqpoint{8.044319in}{3.227801in}}{\pgfqpoint{8.033720in}{3.232191in}}{\pgfqpoint{8.022670in}{3.232191in}}%
\pgfpathcurveto{\pgfqpoint{8.011619in}{3.232191in}}{\pgfqpoint{8.001020in}{3.227801in}}{\pgfqpoint{7.993207in}{3.219988in}}%
\pgfpathcurveto{\pgfqpoint{7.985393in}{3.212174in}}{\pgfqpoint{7.981003in}{3.201575in}}{\pgfqpoint{7.981003in}{3.190525in}}%
\pgfpathcurveto{\pgfqpoint{7.981003in}{3.179475in}}{\pgfqpoint{7.985393in}{3.168876in}}{\pgfqpoint{7.993207in}{3.161062in}}%
\pgfpathcurveto{\pgfqpoint{8.001020in}{3.153248in}}{\pgfqpoint{8.011619in}{3.148858in}}{\pgfqpoint{8.022670in}{3.148858in}}%
\pgfpathlineto{\pgfqpoint{8.022670in}{3.148858in}}%
\pgfpathclose%
\pgfusepath{stroke,fill}%
\end{pgfscope}%
\begin{pgfscope}%
\pgfpathrectangle{\pgfqpoint{7.622482in}{2.920818in}}{\pgfqpoint{2.177280in}{2.201755in}}%
\pgfusepath{clip}%
\pgfsetbuttcap%
\pgfsetroundjoin%
\definecolor{currentfill}{rgb}{0.121569,0.466667,0.705882}%
\pgfsetfillcolor{currentfill}%
\pgfsetlinewidth{0.481800pt}%
\definecolor{currentstroke}{rgb}{1.000000,1.000000,1.000000}%
\pgfsetstrokecolor{currentstroke}%
\pgfsetdash{}{0pt}%
\pgfpathmoveto{\pgfqpoint{7.819408in}{3.148858in}}%
\pgfpathcurveto{\pgfqpoint{7.830458in}{3.148858in}}{\pgfqpoint{7.841057in}{3.153248in}}{\pgfqpoint{7.848871in}{3.161062in}}%
\pgfpathcurveto{\pgfqpoint{7.856684in}{3.168876in}}{\pgfqpoint{7.861075in}{3.179475in}}{\pgfqpoint{7.861075in}{3.190525in}}%
\pgfpathcurveto{\pgfqpoint{7.861075in}{3.201575in}}{\pgfqpoint{7.856684in}{3.212174in}}{\pgfqpoint{7.848871in}{3.219988in}}%
\pgfpathcurveto{\pgfqpoint{7.841057in}{3.227801in}}{\pgfqpoint{7.830458in}{3.232191in}}{\pgfqpoint{7.819408in}{3.232191in}}%
\pgfpathcurveto{\pgfqpoint{7.808358in}{3.232191in}}{\pgfqpoint{7.797759in}{3.227801in}}{\pgfqpoint{7.789945in}{3.219988in}}%
\pgfpathcurveto{\pgfqpoint{7.782132in}{3.212174in}}{\pgfqpoint{7.777741in}{3.201575in}}{\pgfqpoint{7.777741in}{3.190525in}}%
\pgfpathcurveto{\pgfqpoint{7.777741in}{3.179475in}}{\pgfqpoint{7.782132in}{3.168876in}}{\pgfqpoint{7.789945in}{3.161062in}}%
\pgfpathcurveto{\pgfqpoint{7.797759in}{3.153248in}}{\pgfqpoint{7.808358in}{3.148858in}}{\pgfqpoint{7.819408in}{3.148858in}}%
\pgfpathlineto{\pgfqpoint{7.819408in}{3.148858in}}%
\pgfpathclose%
\pgfusepath{stroke,fill}%
\end{pgfscope}%
\begin{pgfscope}%
\pgfpathrectangle{\pgfqpoint{7.622482in}{2.920818in}}{\pgfqpoint{2.177280in}{2.201755in}}%
\pgfusepath{clip}%
\pgfsetbuttcap%
\pgfsetroundjoin%
\definecolor{currentfill}{rgb}{0.121569,0.466667,0.705882}%
\pgfsetfillcolor{currentfill}%
\pgfsetlinewidth{0.481800pt}%
\definecolor{currentstroke}{rgb}{1.000000,1.000000,1.000000}%
\pgfsetstrokecolor{currentstroke}%
\pgfsetdash{}{0pt}%
\pgfpathmoveto{\pgfqpoint{7.887162in}{3.114933in}}%
\pgfpathcurveto{\pgfqpoint{7.898212in}{3.114933in}}{\pgfqpoint{7.908811in}{3.119323in}}{\pgfqpoint{7.916625in}{3.127137in}}%
\pgfpathcurveto{\pgfqpoint{7.924438in}{3.134950in}}{\pgfqpoint{7.928828in}{3.145549in}}{\pgfqpoint{7.928828in}{3.156599in}}%
\pgfpathcurveto{\pgfqpoint{7.928828in}{3.167650in}}{\pgfqpoint{7.924438in}{3.178249in}}{\pgfqpoint{7.916625in}{3.186062in}}%
\pgfpathcurveto{\pgfqpoint{7.908811in}{3.193876in}}{\pgfqpoint{7.898212in}{3.198266in}}{\pgfqpoint{7.887162in}{3.198266in}}%
\pgfpathcurveto{\pgfqpoint{7.876112in}{3.198266in}}{\pgfqpoint{7.865513in}{3.193876in}}{\pgfqpoint{7.857699in}{3.186062in}}%
\pgfpathcurveto{\pgfqpoint{7.849885in}{3.178249in}}{\pgfqpoint{7.845495in}{3.167650in}}{\pgfqpoint{7.845495in}{3.156599in}}%
\pgfpathcurveto{\pgfqpoint{7.845495in}{3.145549in}}{\pgfqpoint{7.849885in}{3.134950in}}{\pgfqpoint{7.857699in}{3.127137in}}%
\pgfpathcurveto{\pgfqpoint{7.865513in}{3.119323in}}{\pgfqpoint{7.876112in}{3.114933in}}{\pgfqpoint{7.887162in}{3.114933in}}%
\pgfpathlineto{\pgfqpoint{7.887162in}{3.114933in}}%
\pgfpathclose%
\pgfusepath{stroke,fill}%
\end{pgfscope}%
\begin{pgfscope}%
\pgfpathrectangle{\pgfqpoint{7.622482in}{2.920818in}}{\pgfqpoint{2.177280in}{2.201755in}}%
\pgfusepath{clip}%
\pgfsetbuttcap%
\pgfsetroundjoin%
\definecolor{currentfill}{rgb}{0.121569,0.466667,0.705882}%
\pgfsetfillcolor{currentfill}%
\pgfsetlinewidth{0.481800pt}%
\definecolor{currentstroke}{rgb}{1.000000,1.000000,1.000000}%
\pgfsetstrokecolor{currentstroke}%
\pgfsetdash{}{0pt}%
\pgfpathmoveto{\pgfqpoint{7.887162in}{3.148858in}}%
\pgfpathcurveto{\pgfqpoint{7.898212in}{3.148858in}}{\pgfqpoint{7.908811in}{3.153248in}}{\pgfqpoint{7.916625in}{3.161062in}}%
\pgfpathcurveto{\pgfqpoint{7.924438in}{3.168876in}}{\pgfqpoint{7.928828in}{3.179475in}}{\pgfqpoint{7.928828in}{3.190525in}}%
\pgfpathcurveto{\pgfqpoint{7.928828in}{3.201575in}}{\pgfqpoint{7.924438in}{3.212174in}}{\pgfqpoint{7.916625in}{3.219988in}}%
\pgfpathcurveto{\pgfqpoint{7.908811in}{3.227801in}}{\pgfqpoint{7.898212in}{3.232191in}}{\pgfqpoint{7.887162in}{3.232191in}}%
\pgfpathcurveto{\pgfqpoint{7.876112in}{3.232191in}}{\pgfqpoint{7.865513in}{3.227801in}}{\pgfqpoint{7.857699in}{3.219988in}}%
\pgfpathcurveto{\pgfqpoint{7.849885in}{3.212174in}}{\pgfqpoint{7.845495in}{3.201575in}}{\pgfqpoint{7.845495in}{3.190525in}}%
\pgfpathcurveto{\pgfqpoint{7.845495in}{3.179475in}}{\pgfqpoint{7.849885in}{3.168876in}}{\pgfqpoint{7.857699in}{3.161062in}}%
\pgfpathcurveto{\pgfqpoint{7.865513in}{3.153248in}}{\pgfqpoint{7.876112in}{3.148858in}}{\pgfqpoint{7.887162in}{3.148858in}}%
\pgfpathlineto{\pgfqpoint{7.887162in}{3.148858in}}%
\pgfpathclose%
\pgfusepath{stroke,fill}%
\end{pgfscope}%
\begin{pgfscope}%
\pgfpathrectangle{\pgfqpoint{7.622482in}{2.920818in}}{\pgfqpoint{2.177280in}{2.201755in}}%
\pgfusepath{clip}%
\pgfsetbuttcap%
\pgfsetroundjoin%
\definecolor{currentfill}{rgb}{0.121569,0.466667,0.705882}%
\pgfsetfillcolor{currentfill}%
\pgfsetlinewidth{0.481800pt}%
\definecolor{currentstroke}{rgb}{1.000000,1.000000,1.000000}%
\pgfsetstrokecolor{currentstroke}%
\pgfsetdash{}{0pt}%
\pgfpathmoveto{\pgfqpoint{7.887162in}{3.047082in}}%
\pgfpathcurveto{\pgfqpoint{7.898212in}{3.047082in}}{\pgfqpoint{7.908811in}{3.051472in}}{\pgfqpoint{7.916625in}{3.059286in}}%
\pgfpathcurveto{\pgfqpoint{7.924438in}{3.067100in}}{\pgfqpoint{7.928828in}{3.077699in}}{\pgfqpoint{7.928828in}{3.088749in}}%
\pgfpathcurveto{\pgfqpoint{7.928828in}{3.099799in}}{\pgfqpoint{7.924438in}{3.110398in}}{\pgfqpoint{7.916625in}{3.118212in}}%
\pgfpathcurveto{\pgfqpoint{7.908811in}{3.126025in}}{\pgfqpoint{7.898212in}{3.130415in}}{\pgfqpoint{7.887162in}{3.130415in}}%
\pgfpathcurveto{\pgfqpoint{7.876112in}{3.130415in}}{\pgfqpoint{7.865513in}{3.126025in}}{\pgfqpoint{7.857699in}{3.118212in}}%
\pgfpathcurveto{\pgfqpoint{7.849885in}{3.110398in}}{\pgfqpoint{7.845495in}{3.099799in}}{\pgfqpoint{7.845495in}{3.088749in}}%
\pgfpathcurveto{\pgfqpoint{7.845495in}{3.077699in}}{\pgfqpoint{7.849885in}{3.067100in}}{\pgfqpoint{7.857699in}{3.059286in}}%
\pgfpathcurveto{\pgfqpoint{7.865513in}{3.051472in}}{\pgfqpoint{7.876112in}{3.047082in}}{\pgfqpoint{7.887162in}{3.047082in}}%
\pgfpathlineto{\pgfqpoint{7.887162in}{3.047082in}}%
\pgfpathclose%
\pgfusepath{stroke,fill}%
\end{pgfscope}%
\begin{pgfscope}%
\pgfpathrectangle{\pgfqpoint{7.622482in}{2.920818in}}{\pgfqpoint{2.177280in}{2.201755in}}%
\pgfusepath{clip}%
\pgfsetbuttcap%
\pgfsetroundjoin%
\definecolor{currentfill}{rgb}{0.121569,0.466667,0.705882}%
\pgfsetfillcolor{currentfill}%
\pgfsetlinewidth{0.481800pt}%
\definecolor{currentstroke}{rgb}{1.000000,1.000000,1.000000}%
\pgfsetstrokecolor{currentstroke}%
\pgfsetdash{}{0pt}%
\pgfpathmoveto{\pgfqpoint{7.887162in}{3.081007in}}%
\pgfpathcurveto{\pgfqpoint{7.898212in}{3.081007in}}{\pgfqpoint{7.908811in}{3.085398in}}{\pgfqpoint{7.916625in}{3.093211in}}%
\pgfpathcurveto{\pgfqpoint{7.924438in}{3.101025in}}{\pgfqpoint{7.928828in}{3.111624in}}{\pgfqpoint{7.928828in}{3.122674in}}%
\pgfpathcurveto{\pgfqpoint{7.928828in}{3.133724in}}{\pgfqpoint{7.924438in}{3.144323in}}{\pgfqpoint{7.916625in}{3.152137in}}%
\pgfpathcurveto{\pgfqpoint{7.908811in}{3.159951in}}{\pgfqpoint{7.898212in}{3.164341in}}{\pgfqpoint{7.887162in}{3.164341in}}%
\pgfpathcurveto{\pgfqpoint{7.876112in}{3.164341in}}{\pgfqpoint{7.865513in}{3.159951in}}{\pgfqpoint{7.857699in}{3.152137in}}%
\pgfpathcurveto{\pgfqpoint{7.849885in}{3.144323in}}{\pgfqpoint{7.845495in}{3.133724in}}{\pgfqpoint{7.845495in}{3.122674in}}%
\pgfpathcurveto{\pgfqpoint{7.845495in}{3.111624in}}{\pgfqpoint{7.849885in}{3.101025in}}{\pgfqpoint{7.857699in}{3.093211in}}%
\pgfpathcurveto{\pgfqpoint{7.865513in}{3.085398in}}{\pgfqpoint{7.876112in}{3.081007in}}{\pgfqpoint{7.887162in}{3.081007in}}%
\pgfpathlineto{\pgfqpoint{7.887162in}{3.081007in}}%
\pgfpathclose%
\pgfusepath{stroke,fill}%
\end{pgfscope}%
\begin{pgfscope}%
\pgfpathrectangle{\pgfqpoint{7.622482in}{2.920818in}}{\pgfqpoint{2.177280in}{2.201755in}}%
\pgfusepath{clip}%
\pgfsetbuttcap%
\pgfsetroundjoin%
\definecolor{currentfill}{rgb}{0.121569,0.466667,0.705882}%
\pgfsetfillcolor{currentfill}%
\pgfsetlinewidth{0.481800pt}%
\definecolor{currentstroke}{rgb}{1.000000,1.000000,1.000000}%
\pgfsetstrokecolor{currentstroke}%
\pgfsetdash{}{0pt}%
\pgfpathmoveto{\pgfqpoint{7.819408in}{3.114933in}}%
\pgfpathcurveto{\pgfqpoint{7.830458in}{3.114933in}}{\pgfqpoint{7.841057in}{3.119323in}}{\pgfqpoint{7.848871in}{3.127137in}}%
\pgfpathcurveto{\pgfqpoint{7.856684in}{3.134950in}}{\pgfqpoint{7.861075in}{3.145549in}}{\pgfqpoint{7.861075in}{3.156599in}}%
\pgfpathcurveto{\pgfqpoint{7.861075in}{3.167650in}}{\pgfqpoint{7.856684in}{3.178249in}}{\pgfqpoint{7.848871in}{3.186062in}}%
\pgfpathcurveto{\pgfqpoint{7.841057in}{3.193876in}}{\pgfqpoint{7.830458in}{3.198266in}}{\pgfqpoint{7.819408in}{3.198266in}}%
\pgfpathcurveto{\pgfqpoint{7.808358in}{3.198266in}}{\pgfqpoint{7.797759in}{3.193876in}}{\pgfqpoint{7.789945in}{3.186062in}}%
\pgfpathcurveto{\pgfqpoint{7.782132in}{3.178249in}}{\pgfqpoint{7.777741in}{3.167650in}}{\pgfqpoint{7.777741in}{3.156599in}}%
\pgfpathcurveto{\pgfqpoint{7.777741in}{3.145549in}}{\pgfqpoint{7.782132in}{3.134950in}}{\pgfqpoint{7.789945in}{3.127137in}}%
\pgfpathcurveto{\pgfqpoint{7.797759in}{3.119323in}}{\pgfqpoint{7.808358in}{3.114933in}}{\pgfqpoint{7.819408in}{3.114933in}}%
\pgfpathlineto{\pgfqpoint{7.819408in}{3.114933in}}%
\pgfpathclose%
\pgfusepath{stroke,fill}%
\end{pgfscope}%
\begin{pgfscope}%
\pgfpathrectangle{\pgfqpoint{7.622482in}{2.920818in}}{\pgfqpoint{2.177280in}{2.201755in}}%
\pgfusepath{clip}%
\pgfsetbuttcap%
\pgfsetroundjoin%
\definecolor{currentfill}{rgb}{0.121569,0.466667,0.705882}%
\pgfsetfillcolor{currentfill}%
\pgfsetlinewidth{0.481800pt}%
\definecolor{currentstroke}{rgb}{1.000000,1.000000,1.000000}%
\pgfsetstrokecolor{currentstroke}%
\pgfsetdash{}{0pt}%
\pgfpathmoveto{\pgfqpoint{7.887162in}{3.081007in}}%
\pgfpathcurveto{\pgfqpoint{7.898212in}{3.081007in}}{\pgfqpoint{7.908811in}{3.085398in}}{\pgfqpoint{7.916625in}{3.093211in}}%
\pgfpathcurveto{\pgfqpoint{7.924438in}{3.101025in}}{\pgfqpoint{7.928828in}{3.111624in}}{\pgfqpoint{7.928828in}{3.122674in}}%
\pgfpathcurveto{\pgfqpoint{7.928828in}{3.133724in}}{\pgfqpoint{7.924438in}{3.144323in}}{\pgfqpoint{7.916625in}{3.152137in}}%
\pgfpathcurveto{\pgfqpoint{7.908811in}{3.159951in}}{\pgfqpoint{7.898212in}{3.164341in}}{\pgfqpoint{7.887162in}{3.164341in}}%
\pgfpathcurveto{\pgfqpoint{7.876112in}{3.164341in}}{\pgfqpoint{7.865513in}{3.159951in}}{\pgfqpoint{7.857699in}{3.152137in}}%
\pgfpathcurveto{\pgfqpoint{7.849885in}{3.144323in}}{\pgfqpoint{7.845495in}{3.133724in}}{\pgfqpoint{7.845495in}{3.122674in}}%
\pgfpathcurveto{\pgfqpoint{7.845495in}{3.111624in}}{\pgfqpoint{7.849885in}{3.101025in}}{\pgfqpoint{7.857699in}{3.093211in}}%
\pgfpathcurveto{\pgfqpoint{7.865513in}{3.085398in}}{\pgfqpoint{7.876112in}{3.081007in}}{\pgfqpoint{7.887162in}{3.081007in}}%
\pgfpathlineto{\pgfqpoint{7.887162in}{3.081007in}}%
\pgfpathclose%
\pgfusepath{stroke,fill}%
\end{pgfscope}%
\begin{pgfscope}%
\pgfpathrectangle{\pgfqpoint{7.622482in}{2.920818in}}{\pgfqpoint{2.177280in}{2.201755in}}%
\pgfusepath{clip}%
\pgfsetbuttcap%
\pgfsetroundjoin%
\definecolor{currentfill}{rgb}{0.121569,0.466667,0.705882}%
\pgfsetfillcolor{currentfill}%
\pgfsetlinewidth{0.481800pt}%
\definecolor{currentstroke}{rgb}{1.000000,1.000000,1.000000}%
\pgfsetstrokecolor{currentstroke}%
\pgfsetdash{}{0pt}%
\pgfpathmoveto{\pgfqpoint{7.887162in}{3.148858in}}%
\pgfpathcurveto{\pgfqpoint{7.898212in}{3.148858in}}{\pgfqpoint{7.908811in}{3.153248in}}{\pgfqpoint{7.916625in}{3.161062in}}%
\pgfpathcurveto{\pgfqpoint{7.924438in}{3.168876in}}{\pgfqpoint{7.928828in}{3.179475in}}{\pgfqpoint{7.928828in}{3.190525in}}%
\pgfpathcurveto{\pgfqpoint{7.928828in}{3.201575in}}{\pgfqpoint{7.924438in}{3.212174in}}{\pgfqpoint{7.916625in}{3.219988in}}%
\pgfpathcurveto{\pgfqpoint{7.908811in}{3.227801in}}{\pgfqpoint{7.898212in}{3.232191in}}{\pgfqpoint{7.887162in}{3.232191in}}%
\pgfpathcurveto{\pgfqpoint{7.876112in}{3.232191in}}{\pgfqpoint{7.865513in}{3.227801in}}{\pgfqpoint{7.857699in}{3.219988in}}%
\pgfpathcurveto{\pgfqpoint{7.849885in}{3.212174in}}{\pgfqpoint{7.845495in}{3.201575in}}{\pgfqpoint{7.845495in}{3.190525in}}%
\pgfpathcurveto{\pgfqpoint{7.845495in}{3.179475in}}{\pgfqpoint{7.849885in}{3.168876in}}{\pgfqpoint{7.857699in}{3.161062in}}%
\pgfpathcurveto{\pgfqpoint{7.865513in}{3.153248in}}{\pgfqpoint{7.876112in}{3.148858in}}{\pgfqpoint{7.887162in}{3.148858in}}%
\pgfpathlineto{\pgfqpoint{7.887162in}{3.148858in}}%
\pgfpathclose%
\pgfusepath{stroke,fill}%
\end{pgfscope}%
\begin{pgfscope}%
\pgfpathrectangle{\pgfqpoint{7.622482in}{2.920818in}}{\pgfqpoint{2.177280in}{2.201755in}}%
\pgfusepath{clip}%
\pgfsetbuttcap%
\pgfsetroundjoin%
\definecolor{currentfill}{rgb}{0.121569,0.466667,0.705882}%
\pgfsetfillcolor{currentfill}%
\pgfsetlinewidth{0.481800pt}%
\definecolor{currentstroke}{rgb}{1.000000,1.000000,1.000000}%
\pgfsetstrokecolor{currentstroke}%
\pgfsetdash{}{0pt}%
\pgfpathmoveto{\pgfqpoint{7.954916in}{3.081007in}}%
\pgfpathcurveto{\pgfqpoint{7.965966in}{3.081007in}}{\pgfqpoint{7.976565in}{3.085398in}}{\pgfqpoint{7.984378in}{3.093211in}}%
\pgfpathcurveto{\pgfqpoint{7.992192in}{3.101025in}}{\pgfqpoint{7.996582in}{3.111624in}}{\pgfqpoint{7.996582in}{3.122674in}}%
\pgfpathcurveto{\pgfqpoint{7.996582in}{3.133724in}}{\pgfqpoint{7.992192in}{3.144323in}}{\pgfqpoint{7.984378in}{3.152137in}}%
\pgfpathcurveto{\pgfqpoint{7.976565in}{3.159951in}}{\pgfqpoint{7.965966in}{3.164341in}}{\pgfqpoint{7.954916in}{3.164341in}}%
\pgfpathcurveto{\pgfqpoint{7.943866in}{3.164341in}}{\pgfqpoint{7.933267in}{3.159951in}}{\pgfqpoint{7.925453in}{3.152137in}}%
\pgfpathcurveto{\pgfqpoint{7.917639in}{3.144323in}}{\pgfqpoint{7.913249in}{3.133724in}}{\pgfqpoint{7.913249in}{3.122674in}}%
\pgfpathcurveto{\pgfqpoint{7.913249in}{3.111624in}}{\pgfqpoint{7.917639in}{3.101025in}}{\pgfqpoint{7.925453in}{3.093211in}}%
\pgfpathcurveto{\pgfqpoint{7.933267in}{3.085398in}}{\pgfqpoint{7.943866in}{3.081007in}}{\pgfqpoint{7.954916in}{3.081007in}}%
\pgfpathlineto{\pgfqpoint{7.954916in}{3.081007in}}%
\pgfpathclose%
\pgfusepath{stroke,fill}%
\end{pgfscope}%
\begin{pgfscope}%
\pgfpathrectangle{\pgfqpoint{7.622482in}{2.920818in}}{\pgfqpoint{2.177280in}{2.201755in}}%
\pgfusepath{clip}%
\pgfsetbuttcap%
\pgfsetroundjoin%
\definecolor{currentfill}{rgb}{0.121569,0.466667,0.705882}%
\pgfsetfillcolor{currentfill}%
\pgfsetlinewidth{0.481800pt}%
\definecolor{currentstroke}{rgb}{1.000000,1.000000,1.000000}%
\pgfsetstrokecolor{currentstroke}%
\pgfsetdash{}{0pt}%
\pgfpathmoveto{\pgfqpoint{7.954916in}{3.081007in}}%
\pgfpathcurveto{\pgfqpoint{7.965966in}{3.081007in}}{\pgfqpoint{7.976565in}{3.085398in}}{\pgfqpoint{7.984378in}{3.093211in}}%
\pgfpathcurveto{\pgfqpoint{7.992192in}{3.101025in}}{\pgfqpoint{7.996582in}{3.111624in}}{\pgfqpoint{7.996582in}{3.122674in}}%
\pgfpathcurveto{\pgfqpoint{7.996582in}{3.133724in}}{\pgfqpoint{7.992192in}{3.144323in}}{\pgfqpoint{7.984378in}{3.152137in}}%
\pgfpathcurveto{\pgfqpoint{7.976565in}{3.159951in}}{\pgfqpoint{7.965966in}{3.164341in}}{\pgfqpoint{7.954916in}{3.164341in}}%
\pgfpathcurveto{\pgfqpoint{7.943866in}{3.164341in}}{\pgfqpoint{7.933267in}{3.159951in}}{\pgfqpoint{7.925453in}{3.152137in}}%
\pgfpathcurveto{\pgfqpoint{7.917639in}{3.144323in}}{\pgfqpoint{7.913249in}{3.133724in}}{\pgfqpoint{7.913249in}{3.122674in}}%
\pgfpathcurveto{\pgfqpoint{7.913249in}{3.111624in}}{\pgfqpoint{7.917639in}{3.101025in}}{\pgfqpoint{7.925453in}{3.093211in}}%
\pgfpathcurveto{\pgfqpoint{7.933267in}{3.085398in}}{\pgfqpoint{7.943866in}{3.081007in}}{\pgfqpoint{7.954916in}{3.081007in}}%
\pgfpathlineto{\pgfqpoint{7.954916in}{3.081007in}}%
\pgfpathclose%
\pgfusepath{stroke,fill}%
\end{pgfscope}%
\begin{pgfscope}%
\pgfpathrectangle{\pgfqpoint{7.622482in}{2.920818in}}{\pgfqpoint{2.177280in}{2.201755in}}%
\pgfusepath{clip}%
\pgfsetbuttcap%
\pgfsetroundjoin%
\definecolor{currentfill}{rgb}{0.121569,0.466667,0.705882}%
\pgfsetfillcolor{currentfill}%
\pgfsetlinewidth{0.481800pt}%
\definecolor{currentstroke}{rgb}{1.000000,1.000000,1.000000}%
\pgfsetstrokecolor{currentstroke}%
\pgfsetdash{}{0pt}%
\pgfpathmoveto{\pgfqpoint{7.887162in}{3.081007in}}%
\pgfpathcurveto{\pgfqpoint{7.898212in}{3.081007in}}{\pgfqpoint{7.908811in}{3.085398in}}{\pgfqpoint{7.916625in}{3.093211in}}%
\pgfpathcurveto{\pgfqpoint{7.924438in}{3.101025in}}{\pgfqpoint{7.928828in}{3.111624in}}{\pgfqpoint{7.928828in}{3.122674in}}%
\pgfpathcurveto{\pgfqpoint{7.928828in}{3.133724in}}{\pgfqpoint{7.924438in}{3.144323in}}{\pgfqpoint{7.916625in}{3.152137in}}%
\pgfpathcurveto{\pgfqpoint{7.908811in}{3.159951in}}{\pgfqpoint{7.898212in}{3.164341in}}{\pgfqpoint{7.887162in}{3.164341in}}%
\pgfpathcurveto{\pgfqpoint{7.876112in}{3.164341in}}{\pgfqpoint{7.865513in}{3.159951in}}{\pgfqpoint{7.857699in}{3.152137in}}%
\pgfpathcurveto{\pgfqpoint{7.849885in}{3.144323in}}{\pgfqpoint{7.845495in}{3.133724in}}{\pgfqpoint{7.845495in}{3.122674in}}%
\pgfpathcurveto{\pgfqpoint{7.845495in}{3.111624in}}{\pgfqpoint{7.849885in}{3.101025in}}{\pgfqpoint{7.857699in}{3.093211in}}%
\pgfpathcurveto{\pgfqpoint{7.865513in}{3.085398in}}{\pgfqpoint{7.876112in}{3.081007in}}{\pgfqpoint{7.887162in}{3.081007in}}%
\pgfpathlineto{\pgfqpoint{7.887162in}{3.081007in}}%
\pgfpathclose%
\pgfusepath{stroke,fill}%
\end{pgfscope}%
\begin{pgfscope}%
\pgfpathrectangle{\pgfqpoint{7.622482in}{2.920818in}}{\pgfqpoint{2.177280in}{2.201755in}}%
\pgfusepath{clip}%
\pgfsetbuttcap%
\pgfsetroundjoin%
\definecolor{currentfill}{rgb}{0.121569,0.466667,0.705882}%
\pgfsetfillcolor{currentfill}%
\pgfsetlinewidth{0.481800pt}%
\definecolor{currentstroke}{rgb}{1.000000,1.000000,1.000000}%
\pgfsetstrokecolor{currentstroke}%
\pgfsetdash{}{0pt}%
\pgfpathmoveto{\pgfqpoint{8.158177in}{3.182783in}}%
\pgfpathcurveto{\pgfqpoint{8.169227in}{3.182783in}}{\pgfqpoint{8.179826in}{3.187174in}}{\pgfqpoint{8.187640in}{3.194987in}}%
\pgfpathcurveto{\pgfqpoint{8.195454in}{3.202801in}}{\pgfqpoint{8.199844in}{3.213400in}}{\pgfqpoint{8.199844in}{3.224450in}}%
\pgfpathcurveto{\pgfqpoint{8.199844in}{3.235500in}}{\pgfqpoint{8.195454in}{3.246099in}}{\pgfqpoint{8.187640in}{3.253913in}}%
\pgfpathcurveto{\pgfqpoint{8.179826in}{3.261727in}}{\pgfqpoint{8.169227in}{3.266117in}}{\pgfqpoint{8.158177in}{3.266117in}}%
\pgfpathcurveto{\pgfqpoint{8.147127in}{3.266117in}}{\pgfqpoint{8.136528in}{3.261727in}}{\pgfqpoint{8.128714in}{3.253913in}}%
\pgfpathcurveto{\pgfqpoint{8.120901in}{3.246099in}}{\pgfqpoint{8.116511in}{3.235500in}}{\pgfqpoint{8.116511in}{3.224450in}}%
\pgfpathcurveto{\pgfqpoint{8.116511in}{3.213400in}}{\pgfqpoint{8.120901in}{3.202801in}}{\pgfqpoint{8.128714in}{3.194987in}}%
\pgfpathcurveto{\pgfqpoint{8.136528in}{3.187174in}}{\pgfqpoint{8.147127in}{3.182783in}}{\pgfqpoint{8.158177in}{3.182783in}}%
\pgfpathlineto{\pgfqpoint{8.158177in}{3.182783in}}%
\pgfpathclose%
\pgfusepath{stroke,fill}%
\end{pgfscope}%
\begin{pgfscope}%
\pgfpathrectangle{\pgfqpoint{7.622482in}{2.920818in}}{\pgfqpoint{2.177280in}{2.201755in}}%
\pgfusepath{clip}%
\pgfsetbuttcap%
\pgfsetroundjoin%
\definecolor{currentfill}{rgb}{0.121569,0.466667,0.705882}%
\pgfsetfillcolor{currentfill}%
\pgfsetlinewidth{0.481800pt}%
\definecolor{currentstroke}{rgb}{1.000000,1.000000,1.000000}%
\pgfsetstrokecolor{currentstroke}%
\pgfsetdash{}{0pt}%
\pgfpathmoveto{\pgfqpoint{8.022670in}{3.284560in}}%
\pgfpathcurveto{\pgfqpoint{8.033720in}{3.284560in}}{\pgfqpoint{8.044319in}{3.288950in}}{\pgfqpoint{8.052132in}{3.296763in}}%
\pgfpathcurveto{\pgfqpoint{8.059946in}{3.304577in}}{\pgfqpoint{8.064336in}{3.315176in}}{\pgfqpoint{8.064336in}{3.326226in}}%
\pgfpathcurveto{\pgfqpoint{8.064336in}{3.337276in}}{\pgfqpoint{8.059946in}{3.347875in}}{\pgfqpoint{8.052132in}{3.355689in}}%
\pgfpathcurveto{\pgfqpoint{8.044319in}{3.363503in}}{\pgfqpoint{8.033720in}{3.367893in}}{\pgfqpoint{8.022670in}{3.367893in}}%
\pgfpathcurveto{\pgfqpoint{8.011619in}{3.367893in}}{\pgfqpoint{8.001020in}{3.363503in}}{\pgfqpoint{7.993207in}{3.355689in}}%
\pgfpathcurveto{\pgfqpoint{7.985393in}{3.347875in}}{\pgfqpoint{7.981003in}{3.337276in}}{\pgfqpoint{7.981003in}{3.326226in}}%
\pgfpathcurveto{\pgfqpoint{7.981003in}{3.315176in}}{\pgfqpoint{7.985393in}{3.304577in}}{\pgfqpoint{7.993207in}{3.296763in}}%
\pgfpathcurveto{\pgfqpoint{8.001020in}{3.288950in}}{\pgfqpoint{8.011619in}{3.284560in}}{\pgfqpoint{8.022670in}{3.284560in}}%
\pgfpathlineto{\pgfqpoint{8.022670in}{3.284560in}}%
\pgfpathclose%
\pgfusepath{stroke,fill}%
\end{pgfscope}%
\begin{pgfscope}%
\pgfpathrectangle{\pgfqpoint{7.622482in}{2.920818in}}{\pgfqpoint{2.177280in}{2.201755in}}%
\pgfusepath{clip}%
\pgfsetbuttcap%
\pgfsetroundjoin%
\definecolor{currentfill}{rgb}{0.121569,0.466667,0.705882}%
\pgfsetfillcolor{currentfill}%
\pgfsetlinewidth{0.481800pt}%
\definecolor{currentstroke}{rgb}{1.000000,1.000000,1.000000}%
\pgfsetstrokecolor{currentstroke}%
\pgfsetdash{}{0pt}%
\pgfpathmoveto{\pgfqpoint{7.954916in}{3.114933in}}%
\pgfpathcurveto{\pgfqpoint{7.965966in}{3.114933in}}{\pgfqpoint{7.976565in}{3.119323in}}{\pgfqpoint{7.984378in}{3.127137in}}%
\pgfpathcurveto{\pgfqpoint{7.992192in}{3.134950in}}{\pgfqpoint{7.996582in}{3.145549in}}{\pgfqpoint{7.996582in}{3.156599in}}%
\pgfpathcurveto{\pgfqpoint{7.996582in}{3.167650in}}{\pgfqpoint{7.992192in}{3.178249in}}{\pgfqpoint{7.984378in}{3.186062in}}%
\pgfpathcurveto{\pgfqpoint{7.976565in}{3.193876in}}{\pgfqpoint{7.965966in}{3.198266in}}{\pgfqpoint{7.954916in}{3.198266in}}%
\pgfpathcurveto{\pgfqpoint{7.943866in}{3.198266in}}{\pgfqpoint{7.933267in}{3.193876in}}{\pgfqpoint{7.925453in}{3.186062in}}%
\pgfpathcurveto{\pgfqpoint{7.917639in}{3.178249in}}{\pgfqpoint{7.913249in}{3.167650in}}{\pgfqpoint{7.913249in}{3.156599in}}%
\pgfpathcurveto{\pgfqpoint{7.913249in}{3.145549in}}{\pgfqpoint{7.917639in}{3.134950in}}{\pgfqpoint{7.925453in}{3.127137in}}%
\pgfpathcurveto{\pgfqpoint{7.933267in}{3.119323in}}{\pgfqpoint{7.943866in}{3.114933in}}{\pgfqpoint{7.954916in}{3.114933in}}%
\pgfpathlineto{\pgfqpoint{7.954916in}{3.114933in}}%
\pgfpathclose%
\pgfusepath{stroke,fill}%
\end{pgfscope}%
\begin{pgfscope}%
\pgfpathrectangle{\pgfqpoint{7.622482in}{2.920818in}}{\pgfqpoint{2.177280in}{2.201755in}}%
\pgfusepath{clip}%
\pgfsetbuttcap%
\pgfsetroundjoin%
\definecolor{currentfill}{rgb}{0.121569,0.466667,0.705882}%
\pgfsetfillcolor{currentfill}%
\pgfsetlinewidth{0.481800pt}%
\definecolor{currentstroke}{rgb}{1.000000,1.000000,1.000000}%
\pgfsetstrokecolor{currentstroke}%
\pgfsetdash{}{0pt}%
\pgfpathmoveto{\pgfqpoint{7.887162in}{3.182783in}}%
\pgfpathcurveto{\pgfqpoint{7.898212in}{3.182783in}}{\pgfqpoint{7.908811in}{3.187174in}}{\pgfqpoint{7.916625in}{3.194987in}}%
\pgfpathcurveto{\pgfqpoint{7.924438in}{3.202801in}}{\pgfqpoint{7.928828in}{3.213400in}}{\pgfqpoint{7.928828in}{3.224450in}}%
\pgfpathcurveto{\pgfqpoint{7.928828in}{3.235500in}}{\pgfqpoint{7.924438in}{3.246099in}}{\pgfqpoint{7.916625in}{3.253913in}}%
\pgfpathcurveto{\pgfqpoint{7.908811in}{3.261727in}}{\pgfqpoint{7.898212in}{3.266117in}}{\pgfqpoint{7.887162in}{3.266117in}}%
\pgfpathcurveto{\pgfqpoint{7.876112in}{3.266117in}}{\pgfqpoint{7.865513in}{3.261727in}}{\pgfqpoint{7.857699in}{3.253913in}}%
\pgfpathcurveto{\pgfqpoint{7.849885in}{3.246099in}}{\pgfqpoint{7.845495in}{3.235500in}}{\pgfqpoint{7.845495in}{3.224450in}}%
\pgfpathcurveto{\pgfqpoint{7.845495in}{3.213400in}}{\pgfqpoint{7.849885in}{3.202801in}}{\pgfqpoint{7.857699in}{3.194987in}}%
\pgfpathcurveto{\pgfqpoint{7.865513in}{3.187174in}}{\pgfqpoint{7.876112in}{3.182783in}}{\pgfqpoint{7.887162in}{3.182783in}}%
\pgfpathlineto{\pgfqpoint{7.887162in}{3.182783in}}%
\pgfpathclose%
\pgfusepath{stroke,fill}%
\end{pgfscope}%
\begin{pgfscope}%
\pgfpathrectangle{\pgfqpoint{7.622482in}{2.920818in}}{\pgfqpoint{2.177280in}{2.201755in}}%
\pgfusepath{clip}%
\pgfsetbuttcap%
\pgfsetroundjoin%
\definecolor{currentfill}{rgb}{0.121569,0.466667,0.705882}%
\pgfsetfillcolor{currentfill}%
\pgfsetlinewidth{0.481800pt}%
\definecolor{currentstroke}{rgb}{1.000000,1.000000,1.000000}%
\pgfsetstrokecolor{currentstroke}%
\pgfsetdash{}{0pt}%
\pgfpathmoveto{\pgfqpoint{7.887162in}{3.114933in}}%
\pgfpathcurveto{\pgfqpoint{7.898212in}{3.114933in}}{\pgfqpoint{7.908811in}{3.119323in}}{\pgfqpoint{7.916625in}{3.127137in}}%
\pgfpathcurveto{\pgfqpoint{7.924438in}{3.134950in}}{\pgfqpoint{7.928828in}{3.145549in}}{\pgfqpoint{7.928828in}{3.156599in}}%
\pgfpathcurveto{\pgfqpoint{7.928828in}{3.167650in}}{\pgfqpoint{7.924438in}{3.178249in}}{\pgfqpoint{7.916625in}{3.186062in}}%
\pgfpathcurveto{\pgfqpoint{7.908811in}{3.193876in}}{\pgfqpoint{7.898212in}{3.198266in}}{\pgfqpoint{7.887162in}{3.198266in}}%
\pgfpathcurveto{\pgfqpoint{7.876112in}{3.198266in}}{\pgfqpoint{7.865513in}{3.193876in}}{\pgfqpoint{7.857699in}{3.186062in}}%
\pgfpathcurveto{\pgfqpoint{7.849885in}{3.178249in}}{\pgfqpoint{7.845495in}{3.167650in}}{\pgfqpoint{7.845495in}{3.156599in}}%
\pgfpathcurveto{\pgfqpoint{7.845495in}{3.145549in}}{\pgfqpoint{7.849885in}{3.134950in}}{\pgfqpoint{7.857699in}{3.127137in}}%
\pgfpathcurveto{\pgfqpoint{7.865513in}{3.119323in}}{\pgfqpoint{7.876112in}{3.114933in}}{\pgfqpoint{7.887162in}{3.114933in}}%
\pgfpathlineto{\pgfqpoint{7.887162in}{3.114933in}}%
\pgfpathclose%
\pgfusepath{stroke,fill}%
\end{pgfscope}%
\begin{pgfscope}%
\pgfpathrectangle{\pgfqpoint{7.622482in}{2.920818in}}{\pgfqpoint{2.177280in}{2.201755in}}%
\pgfusepath{clip}%
\pgfsetbuttcap%
\pgfsetroundjoin%
\definecolor{currentfill}{rgb}{0.121569,0.466667,0.705882}%
\pgfsetfillcolor{currentfill}%
\pgfsetlinewidth{0.481800pt}%
\definecolor{currentstroke}{rgb}{1.000000,1.000000,1.000000}%
\pgfsetstrokecolor{currentstroke}%
\pgfsetdash{}{0pt}%
\pgfpathmoveto{\pgfqpoint{7.887162in}{3.148858in}}%
\pgfpathcurveto{\pgfqpoint{7.898212in}{3.148858in}}{\pgfqpoint{7.908811in}{3.153248in}}{\pgfqpoint{7.916625in}{3.161062in}}%
\pgfpathcurveto{\pgfqpoint{7.924438in}{3.168876in}}{\pgfqpoint{7.928828in}{3.179475in}}{\pgfqpoint{7.928828in}{3.190525in}}%
\pgfpathcurveto{\pgfqpoint{7.928828in}{3.201575in}}{\pgfqpoint{7.924438in}{3.212174in}}{\pgfqpoint{7.916625in}{3.219988in}}%
\pgfpathcurveto{\pgfqpoint{7.908811in}{3.227801in}}{\pgfqpoint{7.898212in}{3.232191in}}{\pgfqpoint{7.887162in}{3.232191in}}%
\pgfpathcurveto{\pgfqpoint{7.876112in}{3.232191in}}{\pgfqpoint{7.865513in}{3.227801in}}{\pgfqpoint{7.857699in}{3.219988in}}%
\pgfpathcurveto{\pgfqpoint{7.849885in}{3.212174in}}{\pgfqpoint{7.845495in}{3.201575in}}{\pgfqpoint{7.845495in}{3.190525in}}%
\pgfpathcurveto{\pgfqpoint{7.845495in}{3.179475in}}{\pgfqpoint{7.849885in}{3.168876in}}{\pgfqpoint{7.857699in}{3.161062in}}%
\pgfpathcurveto{\pgfqpoint{7.865513in}{3.153248in}}{\pgfqpoint{7.876112in}{3.148858in}}{\pgfqpoint{7.887162in}{3.148858in}}%
\pgfpathlineto{\pgfqpoint{7.887162in}{3.148858in}}%
\pgfpathclose%
\pgfusepath{stroke,fill}%
\end{pgfscope}%
\begin{pgfscope}%
\pgfpathrectangle{\pgfqpoint{7.622482in}{2.920818in}}{\pgfqpoint{2.177280in}{2.201755in}}%
\pgfusepath{clip}%
\pgfsetbuttcap%
\pgfsetroundjoin%
\definecolor{currentfill}{rgb}{0.121569,0.466667,0.705882}%
\pgfsetfillcolor{currentfill}%
\pgfsetlinewidth{0.481800pt}%
\definecolor{currentstroke}{rgb}{1.000000,1.000000,1.000000}%
\pgfsetstrokecolor{currentstroke}%
\pgfsetdash{}{0pt}%
\pgfpathmoveto{\pgfqpoint{7.887162in}{3.114933in}}%
\pgfpathcurveto{\pgfqpoint{7.898212in}{3.114933in}}{\pgfqpoint{7.908811in}{3.119323in}}{\pgfqpoint{7.916625in}{3.127137in}}%
\pgfpathcurveto{\pgfqpoint{7.924438in}{3.134950in}}{\pgfqpoint{7.928828in}{3.145549in}}{\pgfqpoint{7.928828in}{3.156599in}}%
\pgfpathcurveto{\pgfqpoint{7.928828in}{3.167650in}}{\pgfqpoint{7.924438in}{3.178249in}}{\pgfqpoint{7.916625in}{3.186062in}}%
\pgfpathcurveto{\pgfqpoint{7.908811in}{3.193876in}}{\pgfqpoint{7.898212in}{3.198266in}}{\pgfqpoint{7.887162in}{3.198266in}}%
\pgfpathcurveto{\pgfqpoint{7.876112in}{3.198266in}}{\pgfqpoint{7.865513in}{3.193876in}}{\pgfqpoint{7.857699in}{3.186062in}}%
\pgfpathcurveto{\pgfqpoint{7.849885in}{3.178249in}}{\pgfqpoint{7.845495in}{3.167650in}}{\pgfqpoint{7.845495in}{3.156599in}}%
\pgfpathcurveto{\pgfqpoint{7.845495in}{3.145549in}}{\pgfqpoint{7.849885in}{3.134950in}}{\pgfqpoint{7.857699in}{3.127137in}}%
\pgfpathcurveto{\pgfqpoint{7.865513in}{3.119323in}}{\pgfqpoint{7.876112in}{3.114933in}}{\pgfqpoint{7.887162in}{3.114933in}}%
\pgfpathlineto{\pgfqpoint{7.887162in}{3.114933in}}%
\pgfpathclose%
\pgfusepath{stroke,fill}%
\end{pgfscope}%
\begin{pgfscope}%
\pgfpathrectangle{\pgfqpoint{7.622482in}{2.920818in}}{\pgfqpoint{2.177280in}{2.201755in}}%
\pgfusepath{clip}%
\pgfsetbuttcap%
\pgfsetroundjoin%
\definecolor{currentfill}{rgb}{1.000000,0.498039,0.054902}%
\pgfsetfillcolor{currentfill}%
\pgfsetlinewidth{0.481800pt}%
\definecolor{currentstroke}{rgb}{1.000000,1.000000,1.000000}%
\pgfsetstrokecolor{currentstroke}%
\pgfsetdash{}{0pt}%
\pgfpathmoveto{\pgfqpoint{8.700208in}{4.234469in}}%
\pgfpathcurveto{\pgfqpoint{8.711258in}{4.234469in}}{\pgfqpoint{8.721857in}{4.238860in}}{\pgfqpoint{8.729671in}{4.246673in}}%
\pgfpathcurveto{\pgfqpoint{8.737485in}{4.254487in}}{\pgfqpoint{8.741875in}{4.265086in}}{\pgfqpoint{8.741875in}{4.276136in}}%
\pgfpathcurveto{\pgfqpoint{8.741875in}{4.287186in}}{\pgfqpoint{8.737485in}{4.297785in}}{\pgfqpoint{8.729671in}{4.305599in}}%
\pgfpathcurveto{\pgfqpoint{8.721857in}{4.313412in}}{\pgfqpoint{8.711258in}{4.317803in}}{\pgfqpoint{8.700208in}{4.317803in}}%
\pgfpathcurveto{\pgfqpoint{8.689158in}{4.317803in}}{\pgfqpoint{8.678559in}{4.313412in}}{\pgfqpoint{8.670745in}{4.305599in}}%
\pgfpathcurveto{\pgfqpoint{8.662932in}{4.297785in}}{\pgfqpoint{8.658542in}{4.287186in}}{\pgfqpoint{8.658542in}{4.276136in}}%
\pgfpathcurveto{\pgfqpoint{8.658542in}{4.265086in}}{\pgfqpoint{8.662932in}{4.254487in}}{\pgfqpoint{8.670745in}{4.246673in}}%
\pgfpathcurveto{\pgfqpoint{8.678559in}{4.238860in}}{\pgfqpoint{8.689158in}{4.234469in}}{\pgfqpoint{8.700208in}{4.234469in}}%
\pgfpathlineto{\pgfqpoint{8.700208in}{4.234469in}}%
\pgfpathclose%
\pgfusepath{stroke,fill}%
\end{pgfscope}%
\begin{pgfscope}%
\pgfpathrectangle{\pgfqpoint{7.622482in}{2.920818in}}{\pgfqpoint{2.177280in}{2.201755in}}%
\pgfusepath{clip}%
\pgfsetbuttcap%
\pgfsetroundjoin%
\definecolor{currentfill}{rgb}{1.000000,0.498039,0.054902}%
\pgfsetfillcolor{currentfill}%
\pgfsetlinewidth{0.481800pt}%
\definecolor{currentstroke}{rgb}{1.000000,1.000000,1.000000}%
\pgfsetstrokecolor{currentstroke}%
\pgfsetdash{}{0pt}%
\pgfpathmoveto{\pgfqpoint{8.767962in}{4.166619in}}%
\pgfpathcurveto{\pgfqpoint{8.779012in}{4.166619in}}{\pgfqpoint{8.789611in}{4.171009in}}{\pgfqpoint{8.797425in}{4.178822in}}%
\pgfpathcurveto{\pgfqpoint{8.805238in}{4.186636in}}{\pgfqpoint{8.809629in}{4.197235in}}{\pgfqpoint{8.809629in}{4.208285in}}%
\pgfpathcurveto{\pgfqpoint{8.809629in}{4.219335in}}{\pgfqpoint{8.805238in}{4.229934in}}{\pgfqpoint{8.797425in}{4.237748in}}%
\pgfpathcurveto{\pgfqpoint{8.789611in}{4.245562in}}{\pgfqpoint{8.779012in}{4.249952in}}{\pgfqpoint{8.767962in}{4.249952in}}%
\pgfpathcurveto{\pgfqpoint{8.756912in}{4.249952in}}{\pgfqpoint{8.746313in}{4.245562in}}{\pgfqpoint{8.738499in}{4.237748in}}%
\pgfpathcurveto{\pgfqpoint{8.730686in}{4.229934in}}{\pgfqpoint{8.726295in}{4.219335in}}{\pgfqpoint{8.726295in}{4.208285in}}%
\pgfpathcurveto{\pgfqpoint{8.726295in}{4.197235in}}{\pgfqpoint{8.730686in}{4.186636in}}{\pgfqpoint{8.738499in}{4.178822in}}%
\pgfpathcurveto{\pgfqpoint{8.746313in}{4.171009in}}{\pgfqpoint{8.756912in}{4.166619in}}{\pgfqpoint{8.767962in}{4.166619in}}%
\pgfpathlineto{\pgfqpoint{8.767962in}{4.166619in}}%
\pgfpathclose%
\pgfusepath{stroke,fill}%
\end{pgfscope}%
\begin{pgfscope}%
\pgfpathrectangle{\pgfqpoint{7.622482in}{2.920818in}}{\pgfqpoint{2.177280in}{2.201755in}}%
\pgfusepath{clip}%
\pgfsetbuttcap%
\pgfsetroundjoin%
\definecolor{currentfill}{rgb}{1.000000,0.498039,0.054902}%
\pgfsetfillcolor{currentfill}%
\pgfsetlinewidth{0.481800pt}%
\definecolor{currentstroke}{rgb}{1.000000,1.000000,1.000000}%
\pgfsetstrokecolor{currentstroke}%
\pgfsetdash{}{0pt}%
\pgfpathmoveto{\pgfqpoint{8.767962in}{4.302320in}}%
\pgfpathcurveto{\pgfqpoint{8.779012in}{4.302320in}}{\pgfqpoint{8.789611in}{4.306710in}}{\pgfqpoint{8.797425in}{4.314524in}}%
\pgfpathcurveto{\pgfqpoint{8.805238in}{4.322337in}}{\pgfqpoint{8.809629in}{4.332937in}}{\pgfqpoint{8.809629in}{4.343987in}}%
\pgfpathcurveto{\pgfqpoint{8.809629in}{4.355037in}}{\pgfqpoint{8.805238in}{4.365636in}}{\pgfqpoint{8.797425in}{4.373449in}}%
\pgfpathcurveto{\pgfqpoint{8.789611in}{4.381263in}}{\pgfqpoint{8.779012in}{4.385653in}}{\pgfqpoint{8.767962in}{4.385653in}}%
\pgfpathcurveto{\pgfqpoint{8.756912in}{4.385653in}}{\pgfqpoint{8.746313in}{4.381263in}}{\pgfqpoint{8.738499in}{4.373449in}}%
\pgfpathcurveto{\pgfqpoint{8.730686in}{4.365636in}}{\pgfqpoint{8.726295in}{4.355037in}}{\pgfqpoint{8.726295in}{4.343987in}}%
\pgfpathcurveto{\pgfqpoint{8.726295in}{4.332937in}}{\pgfqpoint{8.730686in}{4.322337in}}{\pgfqpoint{8.738499in}{4.314524in}}%
\pgfpathcurveto{\pgfqpoint{8.746313in}{4.306710in}}{\pgfqpoint{8.756912in}{4.302320in}}{\pgfqpoint{8.767962in}{4.302320in}}%
\pgfpathlineto{\pgfqpoint{8.767962in}{4.302320in}}%
\pgfpathclose%
\pgfusepath{stroke,fill}%
\end{pgfscope}%
\begin{pgfscope}%
\pgfpathrectangle{\pgfqpoint{7.622482in}{2.920818in}}{\pgfqpoint{2.177280in}{2.201755in}}%
\pgfusepath{clip}%
\pgfsetbuttcap%
\pgfsetroundjoin%
\definecolor{currentfill}{rgb}{1.000000,0.498039,0.054902}%
\pgfsetfillcolor{currentfill}%
\pgfsetlinewidth{0.481800pt}%
\definecolor{currentstroke}{rgb}{1.000000,1.000000,1.000000}%
\pgfsetstrokecolor{currentstroke}%
\pgfsetdash{}{0pt}%
\pgfpathmoveto{\pgfqpoint{8.632454in}{3.996992in}}%
\pgfpathcurveto{\pgfqpoint{8.643504in}{3.996992in}}{\pgfqpoint{8.654104in}{4.001382in}}{\pgfqpoint{8.661917in}{4.009196in}}%
\pgfpathcurveto{\pgfqpoint{8.669731in}{4.017009in}}{\pgfqpoint{8.674121in}{4.027608in}}{\pgfqpoint{8.674121in}{4.038659in}}%
\pgfpathcurveto{\pgfqpoint{8.674121in}{4.049709in}}{\pgfqpoint{8.669731in}{4.060308in}}{\pgfqpoint{8.661917in}{4.068121in}}%
\pgfpathcurveto{\pgfqpoint{8.654104in}{4.075935in}}{\pgfqpoint{8.643504in}{4.080325in}}{\pgfqpoint{8.632454in}{4.080325in}}%
\pgfpathcurveto{\pgfqpoint{8.621404in}{4.080325in}}{\pgfqpoint{8.610805in}{4.075935in}}{\pgfqpoint{8.602992in}{4.068121in}}%
\pgfpathcurveto{\pgfqpoint{8.595178in}{4.060308in}}{\pgfqpoint{8.590788in}{4.049709in}}{\pgfqpoint{8.590788in}{4.038659in}}%
\pgfpathcurveto{\pgfqpoint{8.590788in}{4.027608in}}{\pgfqpoint{8.595178in}{4.017009in}}{\pgfqpoint{8.602992in}{4.009196in}}%
\pgfpathcurveto{\pgfqpoint{8.610805in}{4.001382in}}{\pgfqpoint{8.621404in}{3.996992in}}{\pgfqpoint{8.632454in}{3.996992in}}%
\pgfpathlineto{\pgfqpoint{8.632454in}{3.996992in}}%
\pgfpathclose%
\pgfusepath{stroke,fill}%
\end{pgfscope}%
\begin{pgfscope}%
\pgfpathrectangle{\pgfqpoint{7.622482in}{2.920818in}}{\pgfqpoint{2.177280in}{2.201755in}}%
\pgfusepath{clip}%
\pgfsetbuttcap%
\pgfsetroundjoin%
\definecolor{currentfill}{rgb}{1.000000,0.498039,0.054902}%
\pgfsetfillcolor{currentfill}%
\pgfsetlinewidth{0.481800pt}%
\definecolor{currentstroke}{rgb}{1.000000,1.000000,1.000000}%
\pgfsetstrokecolor{currentstroke}%
\pgfsetdash{}{0pt}%
\pgfpathmoveto{\pgfqpoint{8.767962in}{4.200544in}}%
\pgfpathcurveto{\pgfqpoint{8.779012in}{4.200544in}}{\pgfqpoint{8.789611in}{4.204934in}}{\pgfqpoint{8.797425in}{4.212748in}}%
\pgfpathcurveto{\pgfqpoint{8.805238in}{4.220561in}}{\pgfqpoint{8.809629in}{4.231160in}}{\pgfqpoint{8.809629in}{4.242211in}}%
\pgfpathcurveto{\pgfqpoint{8.809629in}{4.253261in}}{\pgfqpoint{8.805238in}{4.263860in}}{\pgfqpoint{8.797425in}{4.271673in}}%
\pgfpathcurveto{\pgfqpoint{8.789611in}{4.279487in}}{\pgfqpoint{8.779012in}{4.283877in}}{\pgfqpoint{8.767962in}{4.283877in}}%
\pgfpathcurveto{\pgfqpoint{8.756912in}{4.283877in}}{\pgfqpoint{8.746313in}{4.279487in}}{\pgfqpoint{8.738499in}{4.271673in}}%
\pgfpathcurveto{\pgfqpoint{8.730686in}{4.263860in}}{\pgfqpoint{8.726295in}{4.253261in}}{\pgfqpoint{8.726295in}{4.242211in}}%
\pgfpathcurveto{\pgfqpoint{8.726295in}{4.231160in}}{\pgfqpoint{8.730686in}{4.220561in}}{\pgfqpoint{8.738499in}{4.212748in}}%
\pgfpathcurveto{\pgfqpoint{8.746313in}{4.204934in}}{\pgfqpoint{8.756912in}{4.200544in}}{\pgfqpoint{8.767962in}{4.200544in}}%
\pgfpathlineto{\pgfqpoint{8.767962in}{4.200544in}}%
\pgfpathclose%
\pgfusepath{stroke,fill}%
\end{pgfscope}%
\begin{pgfscope}%
\pgfpathrectangle{\pgfqpoint{7.622482in}{2.920818in}}{\pgfqpoint{2.177280in}{2.201755in}}%
\pgfusepath{clip}%
\pgfsetbuttcap%
\pgfsetroundjoin%
\definecolor{currentfill}{rgb}{1.000000,0.498039,0.054902}%
\pgfsetfillcolor{currentfill}%
\pgfsetlinewidth{0.481800pt}%
\definecolor{currentstroke}{rgb}{1.000000,1.000000,1.000000}%
\pgfsetstrokecolor{currentstroke}%
\pgfsetdash{}{0pt}%
\pgfpathmoveto{\pgfqpoint{8.632454in}{4.166619in}}%
\pgfpathcurveto{\pgfqpoint{8.643504in}{4.166619in}}{\pgfqpoint{8.654104in}{4.171009in}}{\pgfqpoint{8.661917in}{4.178822in}}%
\pgfpathcurveto{\pgfqpoint{8.669731in}{4.186636in}}{\pgfqpoint{8.674121in}{4.197235in}}{\pgfqpoint{8.674121in}{4.208285in}}%
\pgfpathcurveto{\pgfqpoint{8.674121in}{4.219335in}}{\pgfqpoint{8.669731in}{4.229934in}}{\pgfqpoint{8.661917in}{4.237748in}}%
\pgfpathcurveto{\pgfqpoint{8.654104in}{4.245562in}}{\pgfqpoint{8.643504in}{4.249952in}}{\pgfqpoint{8.632454in}{4.249952in}}%
\pgfpathcurveto{\pgfqpoint{8.621404in}{4.249952in}}{\pgfqpoint{8.610805in}{4.245562in}}{\pgfqpoint{8.602992in}{4.237748in}}%
\pgfpathcurveto{\pgfqpoint{8.595178in}{4.229934in}}{\pgfqpoint{8.590788in}{4.219335in}}{\pgfqpoint{8.590788in}{4.208285in}}%
\pgfpathcurveto{\pgfqpoint{8.590788in}{4.197235in}}{\pgfqpoint{8.595178in}{4.186636in}}{\pgfqpoint{8.602992in}{4.178822in}}%
\pgfpathcurveto{\pgfqpoint{8.610805in}{4.171009in}}{\pgfqpoint{8.621404in}{4.166619in}}{\pgfqpoint{8.632454in}{4.166619in}}%
\pgfpathlineto{\pgfqpoint{8.632454in}{4.166619in}}%
\pgfpathclose%
\pgfusepath{stroke,fill}%
\end{pgfscope}%
\begin{pgfscope}%
\pgfpathrectangle{\pgfqpoint{7.622482in}{2.920818in}}{\pgfqpoint{2.177280in}{2.201755in}}%
\pgfusepath{clip}%
\pgfsetbuttcap%
\pgfsetroundjoin%
\definecolor{currentfill}{rgb}{1.000000,0.498039,0.054902}%
\pgfsetfillcolor{currentfill}%
\pgfsetlinewidth{0.481800pt}%
\definecolor{currentstroke}{rgb}{1.000000,1.000000,1.000000}%
\pgfsetstrokecolor{currentstroke}%
\pgfsetdash{}{0pt}%
\pgfpathmoveto{\pgfqpoint{8.835716in}{4.234469in}}%
\pgfpathcurveto{\pgfqpoint{8.846766in}{4.234469in}}{\pgfqpoint{8.857365in}{4.238860in}}{\pgfqpoint{8.865179in}{4.246673in}}%
\pgfpathcurveto{\pgfqpoint{8.872992in}{4.254487in}}{\pgfqpoint{8.877383in}{4.265086in}}{\pgfqpoint{8.877383in}{4.276136in}}%
\pgfpathcurveto{\pgfqpoint{8.877383in}{4.287186in}}{\pgfqpoint{8.872992in}{4.297785in}}{\pgfqpoint{8.865179in}{4.305599in}}%
\pgfpathcurveto{\pgfqpoint{8.857365in}{4.313412in}}{\pgfqpoint{8.846766in}{4.317803in}}{\pgfqpoint{8.835716in}{4.317803in}}%
\pgfpathcurveto{\pgfqpoint{8.824666in}{4.317803in}}{\pgfqpoint{8.814067in}{4.313412in}}{\pgfqpoint{8.806253in}{4.305599in}}%
\pgfpathcurveto{\pgfqpoint{8.798440in}{4.297785in}}{\pgfqpoint{8.794049in}{4.287186in}}{\pgfqpoint{8.794049in}{4.276136in}}%
\pgfpathcurveto{\pgfqpoint{8.794049in}{4.265086in}}{\pgfqpoint{8.798440in}{4.254487in}}{\pgfqpoint{8.806253in}{4.246673in}}%
\pgfpathcurveto{\pgfqpoint{8.814067in}{4.238860in}}{\pgfqpoint{8.824666in}{4.234469in}}{\pgfqpoint{8.835716in}{4.234469in}}%
\pgfpathlineto{\pgfqpoint{8.835716in}{4.234469in}}%
\pgfpathclose%
\pgfusepath{stroke,fill}%
\end{pgfscope}%
\begin{pgfscope}%
\pgfpathrectangle{\pgfqpoint{7.622482in}{2.920818in}}{\pgfqpoint{2.177280in}{2.201755in}}%
\pgfusepath{clip}%
\pgfsetbuttcap%
\pgfsetroundjoin%
\definecolor{currentfill}{rgb}{1.000000,0.498039,0.054902}%
\pgfsetfillcolor{currentfill}%
\pgfsetlinewidth{0.481800pt}%
\definecolor{currentstroke}{rgb}{1.000000,1.000000,1.000000}%
\pgfsetstrokecolor{currentstroke}%
\pgfsetdash{}{0pt}%
\pgfpathmoveto{\pgfqpoint{8.429193in}{3.759514in}}%
\pgfpathcurveto{\pgfqpoint{8.440243in}{3.759514in}}{\pgfqpoint{8.450842in}{3.763905in}}{\pgfqpoint{8.458656in}{3.771718in}}%
\pgfpathcurveto{\pgfqpoint{8.466469in}{3.779532in}}{\pgfqpoint{8.470859in}{3.790131in}}{\pgfqpoint{8.470859in}{3.801181in}}%
\pgfpathcurveto{\pgfqpoint{8.470859in}{3.812231in}}{\pgfqpoint{8.466469in}{3.822830in}}{\pgfqpoint{8.458656in}{3.830644in}}%
\pgfpathcurveto{\pgfqpoint{8.450842in}{3.838457in}}{\pgfqpoint{8.440243in}{3.842848in}}{\pgfqpoint{8.429193in}{3.842848in}}%
\pgfpathcurveto{\pgfqpoint{8.418143in}{3.842848in}}{\pgfqpoint{8.407544in}{3.838457in}}{\pgfqpoint{8.399730in}{3.830644in}}%
\pgfpathcurveto{\pgfqpoint{8.391916in}{3.822830in}}{\pgfqpoint{8.387526in}{3.812231in}}{\pgfqpoint{8.387526in}{3.801181in}}%
\pgfpathcurveto{\pgfqpoint{8.387526in}{3.790131in}}{\pgfqpoint{8.391916in}{3.779532in}}{\pgfqpoint{8.399730in}{3.771718in}}%
\pgfpathcurveto{\pgfqpoint{8.407544in}{3.763905in}}{\pgfqpoint{8.418143in}{3.759514in}}{\pgfqpoint{8.429193in}{3.759514in}}%
\pgfpathlineto{\pgfqpoint{8.429193in}{3.759514in}}%
\pgfpathclose%
\pgfusepath{stroke,fill}%
\end{pgfscope}%
\begin{pgfscope}%
\pgfpathrectangle{\pgfqpoint{7.622482in}{2.920818in}}{\pgfqpoint{2.177280in}{2.201755in}}%
\pgfusepath{clip}%
\pgfsetbuttcap%
\pgfsetroundjoin%
\definecolor{currentfill}{rgb}{1.000000,0.498039,0.054902}%
\pgfsetfillcolor{currentfill}%
\pgfsetlinewidth{0.481800pt}%
\definecolor{currentstroke}{rgb}{1.000000,1.000000,1.000000}%
\pgfsetstrokecolor{currentstroke}%
\pgfsetdash{}{0pt}%
\pgfpathmoveto{\pgfqpoint{8.632454in}{4.200544in}}%
\pgfpathcurveto{\pgfqpoint{8.643504in}{4.200544in}}{\pgfqpoint{8.654104in}{4.204934in}}{\pgfqpoint{8.661917in}{4.212748in}}%
\pgfpathcurveto{\pgfqpoint{8.669731in}{4.220561in}}{\pgfqpoint{8.674121in}{4.231160in}}{\pgfqpoint{8.674121in}{4.242211in}}%
\pgfpathcurveto{\pgfqpoint{8.674121in}{4.253261in}}{\pgfqpoint{8.669731in}{4.263860in}}{\pgfqpoint{8.661917in}{4.271673in}}%
\pgfpathcurveto{\pgfqpoint{8.654104in}{4.279487in}}{\pgfqpoint{8.643504in}{4.283877in}}{\pgfqpoint{8.632454in}{4.283877in}}%
\pgfpathcurveto{\pgfqpoint{8.621404in}{4.283877in}}{\pgfqpoint{8.610805in}{4.279487in}}{\pgfqpoint{8.602992in}{4.271673in}}%
\pgfpathcurveto{\pgfqpoint{8.595178in}{4.263860in}}{\pgfqpoint{8.590788in}{4.253261in}}{\pgfqpoint{8.590788in}{4.242211in}}%
\pgfpathcurveto{\pgfqpoint{8.590788in}{4.231160in}}{\pgfqpoint{8.595178in}{4.220561in}}{\pgfqpoint{8.602992in}{4.212748in}}%
\pgfpathcurveto{\pgfqpoint{8.610805in}{4.204934in}}{\pgfqpoint{8.621404in}{4.200544in}}{\pgfqpoint{8.632454in}{4.200544in}}%
\pgfpathlineto{\pgfqpoint{8.632454in}{4.200544in}}%
\pgfpathclose%
\pgfusepath{stroke,fill}%
\end{pgfscope}%
\begin{pgfscope}%
\pgfpathrectangle{\pgfqpoint{7.622482in}{2.920818in}}{\pgfqpoint{2.177280in}{2.201755in}}%
\pgfusepath{clip}%
\pgfsetbuttcap%
\pgfsetroundjoin%
\definecolor{currentfill}{rgb}{1.000000,0.498039,0.054902}%
\pgfsetfillcolor{currentfill}%
\pgfsetlinewidth{0.481800pt}%
\definecolor{currentstroke}{rgb}{1.000000,1.000000,1.000000}%
\pgfsetstrokecolor{currentstroke}%
\pgfsetdash{}{0pt}%
\pgfpathmoveto{\pgfqpoint{8.700208in}{3.963066in}}%
\pgfpathcurveto{\pgfqpoint{8.711258in}{3.963066in}}{\pgfqpoint{8.721857in}{3.967457in}}{\pgfqpoint{8.729671in}{3.975270in}}%
\pgfpathcurveto{\pgfqpoint{8.737485in}{3.983084in}}{\pgfqpoint{8.741875in}{3.993683in}}{\pgfqpoint{8.741875in}{4.004733in}}%
\pgfpathcurveto{\pgfqpoint{8.741875in}{4.015783in}}{\pgfqpoint{8.737485in}{4.026382in}}{\pgfqpoint{8.729671in}{4.034196in}}%
\pgfpathcurveto{\pgfqpoint{8.721857in}{4.042010in}}{\pgfqpoint{8.711258in}{4.046400in}}{\pgfqpoint{8.700208in}{4.046400in}}%
\pgfpathcurveto{\pgfqpoint{8.689158in}{4.046400in}}{\pgfqpoint{8.678559in}{4.042010in}}{\pgfqpoint{8.670745in}{4.034196in}}%
\pgfpathcurveto{\pgfqpoint{8.662932in}{4.026382in}}{\pgfqpoint{8.658542in}{4.015783in}}{\pgfqpoint{8.658542in}{4.004733in}}%
\pgfpathcurveto{\pgfqpoint{8.658542in}{3.993683in}}{\pgfqpoint{8.662932in}{3.983084in}}{\pgfqpoint{8.670745in}{3.975270in}}%
\pgfpathcurveto{\pgfqpoint{8.678559in}{3.967457in}}{\pgfqpoint{8.689158in}{3.963066in}}{\pgfqpoint{8.700208in}{3.963066in}}%
\pgfpathlineto{\pgfqpoint{8.700208in}{3.963066in}}%
\pgfpathclose%
\pgfusepath{stroke,fill}%
\end{pgfscope}%
\begin{pgfscope}%
\pgfpathrectangle{\pgfqpoint{7.622482in}{2.920818in}}{\pgfqpoint{2.177280in}{2.201755in}}%
\pgfusepath{clip}%
\pgfsetbuttcap%
\pgfsetroundjoin%
\definecolor{currentfill}{rgb}{1.000000,0.498039,0.054902}%
\pgfsetfillcolor{currentfill}%
\pgfsetlinewidth{0.481800pt}%
\definecolor{currentstroke}{rgb}{1.000000,1.000000,1.000000}%
\pgfsetstrokecolor{currentstroke}%
\pgfsetdash{}{0pt}%
\pgfpathmoveto{\pgfqpoint{8.429193in}{3.827365in}}%
\pgfpathcurveto{\pgfqpoint{8.440243in}{3.827365in}}{\pgfqpoint{8.450842in}{3.831755in}}{\pgfqpoint{8.458656in}{3.839569in}}%
\pgfpathcurveto{\pgfqpoint{8.466469in}{3.847383in}}{\pgfqpoint{8.470859in}{3.857982in}}{\pgfqpoint{8.470859in}{3.869032in}}%
\pgfpathcurveto{\pgfqpoint{8.470859in}{3.880082in}}{\pgfqpoint{8.466469in}{3.890681in}}{\pgfqpoint{8.458656in}{3.898495in}}%
\pgfpathcurveto{\pgfqpoint{8.450842in}{3.906308in}}{\pgfqpoint{8.440243in}{3.910698in}}{\pgfqpoint{8.429193in}{3.910698in}}%
\pgfpathcurveto{\pgfqpoint{8.418143in}{3.910698in}}{\pgfqpoint{8.407544in}{3.906308in}}{\pgfqpoint{8.399730in}{3.898495in}}%
\pgfpathcurveto{\pgfqpoint{8.391916in}{3.890681in}}{\pgfqpoint{8.387526in}{3.880082in}}{\pgfqpoint{8.387526in}{3.869032in}}%
\pgfpathcurveto{\pgfqpoint{8.387526in}{3.857982in}}{\pgfqpoint{8.391916in}{3.847383in}}{\pgfqpoint{8.399730in}{3.839569in}}%
\pgfpathcurveto{\pgfqpoint{8.407544in}{3.831755in}}{\pgfqpoint{8.418143in}{3.827365in}}{\pgfqpoint{8.429193in}{3.827365in}}%
\pgfpathlineto{\pgfqpoint{8.429193in}{3.827365in}}%
\pgfpathclose%
\pgfusepath{stroke,fill}%
\end{pgfscope}%
\begin{pgfscope}%
\pgfpathrectangle{\pgfqpoint{7.622482in}{2.920818in}}{\pgfqpoint{2.177280in}{2.201755in}}%
\pgfusepath{clip}%
\pgfsetbuttcap%
\pgfsetroundjoin%
\definecolor{currentfill}{rgb}{1.000000,0.498039,0.054902}%
\pgfsetfillcolor{currentfill}%
\pgfsetlinewidth{0.481800pt}%
\definecolor{currentstroke}{rgb}{1.000000,1.000000,1.000000}%
\pgfsetstrokecolor{currentstroke}%
\pgfsetdash{}{0pt}%
\pgfpathmoveto{\pgfqpoint{8.767962in}{4.064843in}}%
\pgfpathcurveto{\pgfqpoint{8.779012in}{4.064843in}}{\pgfqpoint{8.789611in}{4.069233in}}{\pgfqpoint{8.797425in}{4.077046in}}%
\pgfpathcurveto{\pgfqpoint{8.805238in}{4.084860in}}{\pgfqpoint{8.809629in}{4.095459in}}{\pgfqpoint{8.809629in}{4.106509in}}%
\pgfpathcurveto{\pgfqpoint{8.809629in}{4.117559in}}{\pgfqpoint{8.805238in}{4.128158in}}{\pgfqpoint{8.797425in}{4.135972in}}%
\pgfpathcurveto{\pgfqpoint{8.789611in}{4.143786in}}{\pgfqpoint{8.779012in}{4.148176in}}{\pgfqpoint{8.767962in}{4.148176in}}%
\pgfpathcurveto{\pgfqpoint{8.756912in}{4.148176in}}{\pgfqpoint{8.746313in}{4.143786in}}{\pgfqpoint{8.738499in}{4.135972in}}%
\pgfpathcurveto{\pgfqpoint{8.730686in}{4.128158in}}{\pgfqpoint{8.726295in}{4.117559in}}{\pgfqpoint{8.726295in}{4.106509in}}%
\pgfpathcurveto{\pgfqpoint{8.726295in}{4.095459in}}{\pgfqpoint{8.730686in}{4.084860in}}{\pgfqpoint{8.738499in}{4.077046in}}%
\pgfpathcurveto{\pgfqpoint{8.746313in}{4.069233in}}{\pgfqpoint{8.756912in}{4.064843in}}{\pgfqpoint{8.767962in}{4.064843in}}%
\pgfpathlineto{\pgfqpoint{8.767962in}{4.064843in}}%
\pgfpathclose%
\pgfusepath{stroke,fill}%
\end{pgfscope}%
\begin{pgfscope}%
\pgfpathrectangle{\pgfqpoint{7.622482in}{2.920818in}}{\pgfqpoint{2.177280in}{2.201755in}}%
\pgfusepath{clip}%
\pgfsetbuttcap%
\pgfsetroundjoin%
\definecolor{currentfill}{rgb}{1.000000,0.498039,0.054902}%
\pgfsetfillcolor{currentfill}%
\pgfsetlinewidth{0.481800pt}%
\definecolor{currentstroke}{rgb}{1.000000,1.000000,1.000000}%
\pgfsetstrokecolor{currentstroke}%
\pgfsetdash{}{0pt}%
\pgfpathmoveto{\pgfqpoint{8.429193in}{3.996992in}}%
\pgfpathcurveto{\pgfqpoint{8.440243in}{3.996992in}}{\pgfqpoint{8.450842in}{4.001382in}}{\pgfqpoint{8.458656in}{4.009196in}}%
\pgfpathcurveto{\pgfqpoint{8.466469in}{4.017009in}}{\pgfqpoint{8.470859in}{4.027608in}}{\pgfqpoint{8.470859in}{4.038659in}}%
\pgfpathcurveto{\pgfqpoint{8.470859in}{4.049709in}}{\pgfqpoint{8.466469in}{4.060308in}}{\pgfqpoint{8.458656in}{4.068121in}}%
\pgfpathcurveto{\pgfqpoint{8.450842in}{4.075935in}}{\pgfqpoint{8.440243in}{4.080325in}}{\pgfqpoint{8.429193in}{4.080325in}}%
\pgfpathcurveto{\pgfqpoint{8.418143in}{4.080325in}}{\pgfqpoint{8.407544in}{4.075935in}}{\pgfqpoint{8.399730in}{4.068121in}}%
\pgfpathcurveto{\pgfqpoint{8.391916in}{4.060308in}}{\pgfqpoint{8.387526in}{4.049709in}}{\pgfqpoint{8.387526in}{4.038659in}}%
\pgfpathcurveto{\pgfqpoint{8.387526in}{4.027608in}}{\pgfqpoint{8.391916in}{4.017009in}}{\pgfqpoint{8.399730in}{4.009196in}}%
\pgfpathcurveto{\pgfqpoint{8.407544in}{4.001382in}}{\pgfqpoint{8.418143in}{3.996992in}}{\pgfqpoint{8.429193in}{3.996992in}}%
\pgfpathlineto{\pgfqpoint{8.429193in}{3.996992in}}%
\pgfpathclose%
\pgfusepath{stroke,fill}%
\end{pgfscope}%
\begin{pgfscope}%
\pgfpathrectangle{\pgfqpoint{7.622482in}{2.920818in}}{\pgfqpoint{2.177280in}{2.201755in}}%
\pgfusepath{clip}%
\pgfsetbuttcap%
\pgfsetroundjoin%
\definecolor{currentfill}{rgb}{1.000000,0.498039,0.054902}%
\pgfsetfillcolor{currentfill}%
\pgfsetlinewidth{0.481800pt}%
\definecolor{currentstroke}{rgb}{1.000000,1.000000,1.000000}%
\pgfsetstrokecolor{currentstroke}%
\pgfsetdash{}{0pt}%
\pgfpathmoveto{\pgfqpoint{8.700208in}{4.234469in}}%
\pgfpathcurveto{\pgfqpoint{8.711258in}{4.234469in}}{\pgfqpoint{8.721857in}{4.238860in}}{\pgfqpoint{8.729671in}{4.246673in}}%
\pgfpathcurveto{\pgfqpoint{8.737485in}{4.254487in}}{\pgfqpoint{8.741875in}{4.265086in}}{\pgfqpoint{8.741875in}{4.276136in}}%
\pgfpathcurveto{\pgfqpoint{8.741875in}{4.287186in}}{\pgfqpoint{8.737485in}{4.297785in}}{\pgfqpoint{8.729671in}{4.305599in}}%
\pgfpathcurveto{\pgfqpoint{8.721857in}{4.313412in}}{\pgfqpoint{8.711258in}{4.317803in}}{\pgfqpoint{8.700208in}{4.317803in}}%
\pgfpathcurveto{\pgfqpoint{8.689158in}{4.317803in}}{\pgfqpoint{8.678559in}{4.313412in}}{\pgfqpoint{8.670745in}{4.305599in}}%
\pgfpathcurveto{\pgfqpoint{8.662932in}{4.297785in}}{\pgfqpoint{8.658542in}{4.287186in}}{\pgfqpoint{8.658542in}{4.276136in}}%
\pgfpathcurveto{\pgfqpoint{8.658542in}{4.265086in}}{\pgfqpoint{8.662932in}{4.254487in}}{\pgfqpoint{8.670745in}{4.246673in}}%
\pgfpathcurveto{\pgfqpoint{8.678559in}{4.238860in}}{\pgfqpoint{8.689158in}{4.234469in}}{\pgfqpoint{8.700208in}{4.234469in}}%
\pgfpathlineto{\pgfqpoint{8.700208in}{4.234469in}}%
\pgfpathclose%
\pgfusepath{stroke,fill}%
\end{pgfscope}%
\begin{pgfscope}%
\pgfpathrectangle{\pgfqpoint{7.622482in}{2.920818in}}{\pgfqpoint{2.177280in}{2.201755in}}%
\pgfusepath{clip}%
\pgfsetbuttcap%
\pgfsetroundjoin%
\definecolor{currentfill}{rgb}{1.000000,0.498039,0.054902}%
\pgfsetfillcolor{currentfill}%
\pgfsetlinewidth{0.481800pt}%
\definecolor{currentstroke}{rgb}{1.000000,1.000000,1.000000}%
\pgfsetstrokecolor{currentstroke}%
\pgfsetdash{}{0pt}%
\pgfpathmoveto{\pgfqpoint{8.632454in}{3.861290in}}%
\pgfpathcurveto{\pgfqpoint{8.643504in}{3.861290in}}{\pgfqpoint{8.654104in}{3.865681in}}{\pgfqpoint{8.661917in}{3.873494in}}%
\pgfpathcurveto{\pgfqpoint{8.669731in}{3.881308in}}{\pgfqpoint{8.674121in}{3.891907in}}{\pgfqpoint{8.674121in}{3.902957in}}%
\pgfpathcurveto{\pgfqpoint{8.674121in}{3.914007in}}{\pgfqpoint{8.669731in}{3.924606in}}{\pgfqpoint{8.661917in}{3.932420in}}%
\pgfpathcurveto{\pgfqpoint{8.654104in}{3.940234in}}{\pgfqpoint{8.643504in}{3.944624in}}{\pgfqpoint{8.632454in}{3.944624in}}%
\pgfpathcurveto{\pgfqpoint{8.621404in}{3.944624in}}{\pgfqpoint{8.610805in}{3.940234in}}{\pgfqpoint{8.602992in}{3.932420in}}%
\pgfpathcurveto{\pgfqpoint{8.595178in}{3.924606in}}{\pgfqpoint{8.590788in}{3.914007in}}{\pgfqpoint{8.590788in}{3.902957in}}%
\pgfpathcurveto{\pgfqpoint{8.590788in}{3.891907in}}{\pgfqpoint{8.595178in}{3.881308in}}{\pgfqpoint{8.602992in}{3.873494in}}%
\pgfpathcurveto{\pgfqpoint{8.610805in}{3.865681in}}{\pgfqpoint{8.621404in}{3.861290in}}{\pgfqpoint{8.632454in}{3.861290in}}%
\pgfpathlineto{\pgfqpoint{8.632454in}{3.861290in}}%
\pgfpathclose%
\pgfusepath{stroke,fill}%
\end{pgfscope}%
\begin{pgfscope}%
\pgfpathrectangle{\pgfqpoint{7.622482in}{2.920818in}}{\pgfqpoint{2.177280in}{2.201755in}}%
\pgfusepath{clip}%
\pgfsetbuttcap%
\pgfsetroundjoin%
\definecolor{currentfill}{rgb}{1.000000,0.498039,0.054902}%
\pgfsetfillcolor{currentfill}%
\pgfsetlinewidth{0.481800pt}%
\definecolor{currentstroke}{rgb}{1.000000,1.000000,1.000000}%
\pgfsetstrokecolor{currentstroke}%
\pgfsetdash{}{0pt}%
\pgfpathmoveto{\pgfqpoint{8.700208in}{4.132693in}}%
\pgfpathcurveto{\pgfqpoint{8.711258in}{4.132693in}}{\pgfqpoint{8.721857in}{4.137083in}}{\pgfqpoint{8.729671in}{4.144897in}}%
\pgfpathcurveto{\pgfqpoint{8.737485in}{4.152711in}}{\pgfqpoint{8.741875in}{4.163310in}}{\pgfqpoint{8.741875in}{4.174360in}}%
\pgfpathcurveto{\pgfqpoint{8.741875in}{4.185410in}}{\pgfqpoint{8.737485in}{4.196009in}}{\pgfqpoint{8.729671in}{4.203823in}}%
\pgfpathcurveto{\pgfqpoint{8.721857in}{4.211636in}}{\pgfqpoint{8.711258in}{4.216027in}}{\pgfqpoint{8.700208in}{4.216027in}}%
\pgfpathcurveto{\pgfqpoint{8.689158in}{4.216027in}}{\pgfqpoint{8.678559in}{4.211636in}}{\pgfqpoint{8.670745in}{4.203823in}}%
\pgfpathcurveto{\pgfqpoint{8.662932in}{4.196009in}}{\pgfqpoint{8.658542in}{4.185410in}}{\pgfqpoint{8.658542in}{4.174360in}}%
\pgfpathcurveto{\pgfqpoint{8.658542in}{4.163310in}}{\pgfqpoint{8.662932in}{4.152711in}}{\pgfqpoint{8.670745in}{4.144897in}}%
\pgfpathcurveto{\pgfqpoint{8.678559in}{4.137083in}}{\pgfqpoint{8.689158in}{4.132693in}}{\pgfqpoint{8.700208in}{4.132693in}}%
\pgfpathlineto{\pgfqpoint{8.700208in}{4.132693in}}%
\pgfpathclose%
\pgfusepath{stroke,fill}%
\end{pgfscope}%
\begin{pgfscope}%
\pgfpathrectangle{\pgfqpoint{7.622482in}{2.920818in}}{\pgfqpoint{2.177280in}{2.201755in}}%
\pgfusepath{clip}%
\pgfsetbuttcap%
\pgfsetroundjoin%
\definecolor{currentfill}{rgb}{1.000000,0.498039,0.054902}%
\pgfsetfillcolor{currentfill}%
\pgfsetlinewidth{0.481800pt}%
\definecolor{currentstroke}{rgb}{1.000000,1.000000,1.000000}%
\pgfsetstrokecolor{currentstroke}%
\pgfsetdash{}{0pt}%
\pgfpathmoveto{\pgfqpoint{8.767962in}{4.166619in}}%
\pgfpathcurveto{\pgfqpoint{8.779012in}{4.166619in}}{\pgfqpoint{8.789611in}{4.171009in}}{\pgfqpoint{8.797425in}{4.178822in}}%
\pgfpathcurveto{\pgfqpoint{8.805238in}{4.186636in}}{\pgfqpoint{8.809629in}{4.197235in}}{\pgfqpoint{8.809629in}{4.208285in}}%
\pgfpathcurveto{\pgfqpoint{8.809629in}{4.219335in}}{\pgfqpoint{8.805238in}{4.229934in}}{\pgfqpoint{8.797425in}{4.237748in}}%
\pgfpathcurveto{\pgfqpoint{8.789611in}{4.245562in}}{\pgfqpoint{8.779012in}{4.249952in}}{\pgfqpoint{8.767962in}{4.249952in}}%
\pgfpathcurveto{\pgfqpoint{8.756912in}{4.249952in}}{\pgfqpoint{8.746313in}{4.245562in}}{\pgfqpoint{8.738499in}{4.237748in}}%
\pgfpathcurveto{\pgfqpoint{8.730686in}{4.229934in}}{\pgfqpoint{8.726295in}{4.219335in}}{\pgfqpoint{8.726295in}{4.208285in}}%
\pgfpathcurveto{\pgfqpoint{8.726295in}{4.197235in}}{\pgfqpoint{8.730686in}{4.186636in}}{\pgfqpoint{8.738499in}{4.178822in}}%
\pgfpathcurveto{\pgfqpoint{8.746313in}{4.171009in}}{\pgfqpoint{8.756912in}{4.166619in}}{\pgfqpoint{8.767962in}{4.166619in}}%
\pgfpathlineto{\pgfqpoint{8.767962in}{4.166619in}}%
\pgfpathclose%
\pgfusepath{stroke,fill}%
\end{pgfscope}%
\begin{pgfscope}%
\pgfpathrectangle{\pgfqpoint{7.622482in}{2.920818in}}{\pgfqpoint{2.177280in}{2.201755in}}%
\pgfusepath{clip}%
\pgfsetbuttcap%
\pgfsetroundjoin%
\definecolor{currentfill}{rgb}{1.000000,0.498039,0.054902}%
\pgfsetfillcolor{currentfill}%
\pgfsetlinewidth{0.481800pt}%
\definecolor{currentstroke}{rgb}{1.000000,1.000000,1.000000}%
\pgfsetstrokecolor{currentstroke}%
\pgfsetdash{}{0pt}%
\pgfpathmoveto{\pgfqpoint{8.429193in}{4.030917in}}%
\pgfpathcurveto{\pgfqpoint{8.440243in}{4.030917in}}{\pgfqpoint{8.450842in}{4.035307in}}{\pgfqpoint{8.458656in}{4.043121in}}%
\pgfpathcurveto{\pgfqpoint{8.466469in}{4.050935in}}{\pgfqpoint{8.470859in}{4.061534in}}{\pgfqpoint{8.470859in}{4.072584in}}%
\pgfpathcurveto{\pgfqpoint{8.470859in}{4.083634in}}{\pgfqpoint{8.466469in}{4.094233in}}{\pgfqpoint{8.458656in}{4.102047in}}%
\pgfpathcurveto{\pgfqpoint{8.450842in}{4.109860in}}{\pgfqpoint{8.440243in}{4.114251in}}{\pgfqpoint{8.429193in}{4.114251in}}%
\pgfpathcurveto{\pgfqpoint{8.418143in}{4.114251in}}{\pgfqpoint{8.407544in}{4.109860in}}{\pgfqpoint{8.399730in}{4.102047in}}%
\pgfpathcurveto{\pgfqpoint{8.391916in}{4.094233in}}{\pgfqpoint{8.387526in}{4.083634in}}{\pgfqpoint{8.387526in}{4.072584in}}%
\pgfpathcurveto{\pgfqpoint{8.387526in}{4.061534in}}{\pgfqpoint{8.391916in}{4.050935in}}{\pgfqpoint{8.399730in}{4.043121in}}%
\pgfpathcurveto{\pgfqpoint{8.407544in}{4.035307in}}{\pgfqpoint{8.418143in}{4.030917in}}{\pgfqpoint{8.429193in}{4.030917in}}%
\pgfpathlineto{\pgfqpoint{8.429193in}{4.030917in}}%
\pgfpathclose%
\pgfusepath{stroke,fill}%
\end{pgfscope}%
\begin{pgfscope}%
\pgfpathrectangle{\pgfqpoint{7.622482in}{2.920818in}}{\pgfqpoint{2.177280in}{2.201755in}}%
\pgfusepath{clip}%
\pgfsetbuttcap%
\pgfsetroundjoin%
\definecolor{currentfill}{rgb}{1.000000,0.498039,0.054902}%
\pgfsetfillcolor{currentfill}%
\pgfsetlinewidth{0.481800pt}%
\definecolor{currentstroke}{rgb}{1.000000,1.000000,1.000000}%
\pgfsetstrokecolor{currentstroke}%
\pgfsetdash{}{0pt}%
\pgfpathmoveto{\pgfqpoint{8.767962in}{4.166619in}}%
\pgfpathcurveto{\pgfqpoint{8.779012in}{4.166619in}}{\pgfqpoint{8.789611in}{4.171009in}}{\pgfqpoint{8.797425in}{4.178822in}}%
\pgfpathcurveto{\pgfqpoint{8.805238in}{4.186636in}}{\pgfqpoint{8.809629in}{4.197235in}}{\pgfqpoint{8.809629in}{4.208285in}}%
\pgfpathcurveto{\pgfqpoint{8.809629in}{4.219335in}}{\pgfqpoint{8.805238in}{4.229934in}}{\pgfqpoint{8.797425in}{4.237748in}}%
\pgfpathcurveto{\pgfqpoint{8.789611in}{4.245562in}}{\pgfqpoint{8.779012in}{4.249952in}}{\pgfqpoint{8.767962in}{4.249952in}}%
\pgfpathcurveto{\pgfqpoint{8.756912in}{4.249952in}}{\pgfqpoint{8.746313in}{4.245562in}}{\pgfqpoint{8.738499in}{4.237748in}}%
\pgfpathcurveto{\pgfqpoint{8.730686in}{4.229934in}}{\pgfqpoint{8.726295in}{4.219335in}}{\pgfqpoint{8.726295in}{4.208285in}}%
\pgfpathcurveto{\pgfqpoint{8.726295in}{4.197235in}}{\pgfqpoint{8.730686in}{4.186636in}}{\pgfqpoint{8.738499in}{4.178822in}}%
\pgfpathcurveto{\pgfqpoint{8.746313in}{4.171009in}}{\pgfqpoint{8.756912in}{4.166619in}}{\pgfqpoint{8.767962in}{4.166619in}}%
\pgfpathlineto{\pgfqpoint{8.767962in}{4.166619in}}%
\pgfpathclose%
\pgfusepath{stroke,fill}%
\end{pgfscope}%
\begin{pgfscope}%
\pgfpathrectangle{\pgfqpoint{7.622482in}{2.920818in}}{\pgfqpoint{2.177280in}{2.201755in}}%
\pgfusepath{clip}%
\pgfsetbuttcap%
\pgfsetroundjoin%
\definecolor{currentfill}{rgb}{1.000000,0.498039,0.054902}%
\pgfsetfillcolor{currentfill}%
\pgfsetlinewidth{0.481800pt}%
\definecolor{currentstroke}{rgb}{1.000000,1.000000,1.000000}%
\pgfsetstrokecolor{currentstroke}%
\pgfsetdash{}{0pt}%
\pgfpathmoveto{\pgfqpoint{8.496947in}{3.963066in}}%
\pgfpathcurveto{\pgfqpoint{8.507997in}{3.963066in}}{\pgfqpoint{8.518596in}{3.967457in}}{\pgfqpoint{8.526409in}{3.975270in}}%
\pgfpathcurveto{\pgfqpoint{8.534223in}{3.983084in}}{\pgfqpoint{8.538613in}{3.993683in}}{\pgfqpoint{8.538613in}{4.004733in}}%
\pgfpathcurveto{\pgfqpoint{8.538613in}{4.015783in}}{\pgfqpoint{8.534223in}{4.026382in}}{\pgfqpoint{8.526409in}{4.034196in}}%
\pgfpathcurveto{\pgfqpoint{8.518596in}{4.042010in}}{\pgfqpoint{8.507997in}{4.046400in}}{\pgfqpoint{8.496947in}{4.046400in}}%
\pgfpathcurveto{\pgfqpoint{8.485896in}{4.046400in}}{\pgfqpoint{8.475297in}{4.042010in}}{\pgfqpoint{8.467484in}{4.034196in}}%
\pgfpathcurveto{\pgfqpoint{8.459670in}{4.026382in}}{\pgfqpoint{8.455280in}{4.015783in}}{\pgfqpoint{8.455280in}{4.004733in}}%
\pgfpathcurveto{\pgfqpoint{8.455280in}{3.993683in}}{\pgfqpoint{8.459670in}{3.983084in}}{\pgfqpoint{8.467484in}{3.975270in}}%
\pgfpathcurveto{\pgfqpoint{8.475297in}{3.967457in}}{\pgfqpoint{8.485896in}{3.963066in}}{\pgfqpoint{8.496947in}{3.963066in}}%
\pgfpathlineto{\pgfqpoint{8.496947in}{3.963066in}}%
\pgfpathclose%
\pgfusepath{stroke,fill}%
\end{pgfscope}%
\begin{pgfscope}%
\pgfpathrectangle{\pgfqpoint{7.622482in}{2.920818in}}{\pgfqpoint{2.177280in}{2.201755in}}%
\pgfusepath{clip}%
\pgfsetbuttcap%
\pgfsetroundjoin%
\definecolor{currentfill}{rgb}{1.000000,0.498039,0.054902}%
\pgfsetfillcolor{currentfill}%
\pgfsetlinewidth{0.481800pt}%
\definecolor{currentstroke}{rgb}{1.000000,1.000000,1.000000}%
\pgfsetstrokecolor{currentstroke}%
\pgfsetdash{}{0pt}%
\pgfpathmoveto{\pgfqpoint{8.971224in}{4.268395in}}%
\pgfpathcurveto{\pgfqpoint{8.982274in}{4.268395in}}{\pgfqpoint{8.992873in}{4.272785in}}{\pgfqpoint{9.000686in}{4.280599in}}%
\pgfpathcurveto{\pgfqpoint{9.008500in}{4.288412in}}{\pgfqpoint{9.012890in}{4.299011in}}{\pgfqpoint{9.012890in}{4.310061in}}%
\pgfpathcurveto{\pgfqpoint{9.012890in}{4.321111in}}{\pgfqpoint{9.008500in}{4.331710in}}{\pgfqpoint{9.000686in}{4.339524in}}%
\pgfpathcurveto{\pgfqpoint{8.992873in}{4.347338in}}{\pgfqpoint{8.982274in}{4.351728in}}{\pgfqpoint{8.971224in}{4.351728in}}%
\pgfpathcurveto{\pgfqpoint{8.960174in}{4.351728in}}{\pgfqpoint{8.949575in}{4.347338in}}{\pgfqpoint{8.941761in}{4.339524in}}%
\pgfpathcurveto{\pgfqpoint{8.933947in}{4.331710in}}{\pgfqpoint{8.929557in}{4.321111in}}{\pgfqpoint{8.929557in}{4.310061in}}%
\pgfpathcurveto{\pgfqpoint{8.929557in}{4.299011in}}{\pgfqpoint{8.933947in}{4.288412in}}{\pgfqpoint{8.941761in}{4.280599in}}%
\pgfpathcurveto{\pgfqpoint{8.949575in}{4.272785in}}{\pgfqpoint{8.960174in}{4.268395in}}{\pgfqpoint{8.971224in}{4.268395in}}%
\pgfpathlineto{\pgfqpoint{8.971224in}{4.268395in}}%
\pgfpathclose%
\pgfusepath{stroke,fill}%
\end{pgfscope}%
\begin{pgfscope}%
\pgfpathrectangle{\pgfqpoint{7.622482in}{2.920818in}}{\pgfqpoint{2.177280in}{2.201755in}}%
\pgfusepath{clip}%
\pgfsetbuttcap%
\pgfsetroundjoin%
\definecolor{currentfill}{rgb}{1.000000,0.498039,0.054902}%
\pgfsetfillcolor{currentfill}%
\pgfsetlinewidth{0.481800pt}%
\definecolor{currentstroke}{rgb}{1.000000,1.000000,1.000000}%
\pgfsetstrokecolor{currentstroke}%
\pgfsetdash{}{0pt}%
\pgfpathmoveto{\pgfqpoint{8.632454in}{3.996992in}}%
\pgfpathcurveto{\pgfqpoint{8.643504in}{3.996992in}}{\pgfqpoint{8.654104in}{4.001382in}}{\pgfqpoint{8.661917in}{4.009196in}}%
\pgfpathcurveto{\pgfqpoint{8.669731in}{4.017009in}}{\pgfqpoint{8.674121in}{4.027608in}}{\pgfqpoint{8.674121in}{4.038659in}}%
\pgfpathcurveto{\pgfqpoint{8.674121in}{4.049709in}}{\pgfqpoint{8.669731in}{4.060308in}}{\pgfqpoint{8.661917in}{4.068121in}}%
\pgfpathcurveto{\pgfqpoint{8.654104in}{4.075935in}}{\pgfqpoint{8.643504in}{4.080325in}}{\pgfqpoint{8.632454in}{4.080325in}}%
\pgfpathcurveto{\pgfqpoint{8.621404in}{4.080325in}}{\pgfqpoint{8.610805in}{4.075935in}}{\pgfqpoint{8.602992in}{4.068121in}}%
\pgfpathcurveto{\pgfqpoint{8.595178in}{4.060308in}}{\pgfqpoint{8.590788in}{4.049709in}}{\pgfqpoint{8.590788in}{4.038659in}}%
\pgfpathcurveto{\pgfqpoint{8.590788in}{4.027608in}}{\pgfqpoint{8.595178in}{4.017009in}}{\pgfqpoint{8.602992in}{4.009196in}}%
\pgfpathcurveto{\pgfqpoint{8.610805in}{4.001382in}}{\pgfqpoint{8.621404in}{3.996992in}}{\pgfqpoint{8.632454in}{3.996992in}}%
\pgfpathlineto{\pgfqpoint{8.632454in}{3.996992in}}%
\pgfpathclose%
\pgfusepath{stroke,fill}%
\end{pgfscope}%
\begin{pgfscope}%
\pgfpathrectangle{\pgfqpoint{7.622482in}{2.920818in}}{\pgfqpoint{2.177280in}{2.201755in}}%
\pgfusepath{clip}%
\pgfsetbuttcap%
\pgfsetroundjoin%
\definecolor{currentfill}{rgb}{1.000000,0.498039,0.054902}%
\pgfsetfillcolor{currentfill}%
\pgfsetlinewidth{0.481800pt}%
\definecolor{currentstroke}{rgb}{1.000000,1.000000,1.000000}%
\pgfsetstrokecolor{currentstroke}%
\pgfsetdash{}{0pt}%
\pgfpathmoveto{\pgfqpoint{8.767962in}{4.302320in}}%
\pgfpathcurveto{\pgfqpoint{8.779012in}{4.302320in}}{\pgfqpoint{8.789611in}{4.306710in}}{\pgfqpoint{8.797425in}{4.314524in}}%
\pgfpathcurveto{\pgfqpoint{8.805238in}{4.322337in}}{\pgfqpoint{8.809629in}{4.332937in}}{\pgfqpoint{8.809629in}{4.343987in}}%
\pgfpathcurveto{\pgfqpoint{8.809629in}{4.355037in}}{\pgfqpoint{8.805238in}{4.365636in}}{\pgfqpoint{8.797425in}{4.373449in}}%
\pgfpathcurveto{\pgfqpoint{8.789611in}{4.381263in}}{\pgfqpoint{8.779012in}{4.385653in}}{\pgfqpoint{8.767962in}{4.385653in}}%
\pgfpathcurveto{\pgfqpoint{8.756912in}{4.385653in}}{\pgfqpoint{8.746313in}{4.381263in}}{\pgfqpoint{8.738499in}{4.373449in}}%
\pgfpathcurveto{\pgfqpoint{8.730686in}{4.365636in}}{\pgfqpoint{8.726295in}{4.355037in}}{\pgfqpoint{8.726295in}{4.343987in}}%
\pgfpathcurveto{\pgfqpoint{8.726295in}{4.332937in}}{\pgfqpoint{8.730686in}{4.322337in}}{\pgfqpoint{8.738499in}{4.314524in}}%
\pgfpathcurveto{\pgfqpoint{8.746313in}{4.306710in}}{\pgfqpoint{8.756912in}{4.302320in}}{\pgfqpoint{8.767962in}{4.302320in}}%
\pgfpathlineto{\pgfqpoint{8.767962in}{4.302320in}}%
\pgfpathclose%
\pgfusepath{stroke,fill}%
\end{pgfscope}%
\begin{pgfscope}%
\pgfpathrectangle{\pgfqpoint{7.622482in}{2.920818in}}{\pgfqpoint{2.177280in}{2.201755in}}%
\pgfusepath{clip}%
\pgfsetbuttcap%
\pgfsetroundjoin%
\definecolor{currentfill}{rgb}{1.000000,0.498039,0.054902}%
\pgfsetfillcolor{currentfill}%
\pgfsetlinewidth{0.481800pt}%
\definecolor{currentstroke}{rgb}{1.000000,1.000000,1.000000}%
\pgfsetstrokecolor{currentstroke}%
\pgfsetdash{}{0pt}%
\pgfpathmoveto{\pgfqpoint{8.564700in}{4.234469in}}%
\pgfpathcurveto{\pgfqpoint{8.575751in}{4.234469in}}{\pgfqpoint{8.586350in}{4.238860in}}{\pgfqpoint{8.594163in}{4.246673in}}%
\pgfpathcurveto{\pgfqpoint{8.601977in}{4.254487in}}{\pgfqpoint{8.606367in}{4.265086in}}{\pgfqpoint{8.606367in}{4.276136in}}%
\pgfpathcurveto{\pgfqpoint{8.606367in}{4.287186in}}{\pgfqpoint{8.601977in}{4.297785in}}{\pgfqpoint{8.594163in}{4.305599in}}%
\pgfpathcurveto{\pgfqpoint{8.586350in}{4.313412in}}{\pgfqpoint{8.575751in}{4.317803in}}{\pgfqpoint{8.564700in}{4.317803in}}%
\pgfpathcurveto{\pgfqpoint{8.553650in}{4.317803in}}{\pgfqpoint{8.543051in}{4.313412in}}{\pgfqpoint{8.535238in}{4.305599in}}%
\pgfpathcurveto{\pgfqpoint{8.527424in}{4.297785in}}{\pgfqpoint{8.523034in}{4.287186in}}{\pgfqpoint{8.523034in}{4.276136in}}%
\pgfpathcurveto{\pgfqpoint{8.523034in}{4.265086in}}{\pgfqpoint{8.527424in}{4.254487in}}{\pgfqpoint{8.535238in}{4.246673in}}%
\pgfpathcurveto{\pgfqpoint{8.543051in}{4.238860in}}{\pgfqpoint{8.553650in}{4.234469in}}{\pgfqpoint{8.564700in}{4.234469in}}%
\pgfpathlineto{\pgfqpoint{8.564700in}{4.234469in}}%
\pgfpathclose%
\pgfusepath{stroke,fill}%
\end{pgfscope}%
\begin{pgfscope}%
\pgfpathrectangle{\pgfqpoint{7.622482in}{2.920818in}}{\pgfqpoint{2.177280in}{2.201755in}}%
\pgfusepath{clip}%
\pgfsetbuttcap%
\pgfsetroundjoin%
\definecolor{currentfill}{rgb}{1.000000,0.498039,0.054902}%
\pgfsetfillcolor{currentfill}%
\pgfsetlinewidth{0.481800pt}%
\definecolor{currentstroke}{rgb}{1.000000,1.000000,1.000000}%
\pgfsetstrokecolor{currentstroke}%
\pgfsetdash{}{0pt}%
\pgfpathmoveto{\pgfqpoint{8.632454in}{4.098768in}}%
\pgfpathcurveto{\pgfqpoint{8.643504in}{4.098768in}}{\pgfqpoint{8.654104in}{4.103158in}}{\pgfqpoint{8.661917in}{4.110972in}}%
\pgfpathcurveto{\pgfqpoint{8.669731in}{4.118785in}}{\pgfqpoint{8.674121in}{4.129384in}}{\pgfqpoint{8.674121in}{4.140435in}}%
\pgfpathcurveto{\pgfqpoint{8.674121in}{4.151485in}}{\pgfqpoint{8.669731in}{4.162084in}}{\pgfqpoint{8.661917in}{4.169897in}}%
\pgfpathcurveto{\pgfqpoint{8.654104in}{4.177711in}}{\pgfqpoint{8.643504in}{4.182101in}}{\pgfqpoint{8.632454in}{4.182101in}}%
\pgfpathcurveto{\pgfqpoint{8.621404in}{4.182101in}}{\pgfqpoint{8.610805in}{4.177711in}}{\pgfqpoint{8.602992in}{4.169897in}}%
\pgfpathcurveto{\pgfqpoint{8.595178in}{4.162084in}}{\pgfqpoint{8.590788in}{4.151485in}}{\pgfqpoint{8.590788in}{4.140435in}}%
\pgfpathcurveto{\pgfqpoint{8.590788in}{4.129384in}}{\pgfqpoint{8.595178in}{4.118785in}}{\pgfqpoint{8.602992in}{4.110972in}}%
\pgfpathcurveto{\pgfqpoint{8.610805in}{4.103158in}}{\pgfqpoint{8.621404in}{4.098768in}}{\pgfqpoint{8.632454in}{4.098768in}}%
\pgfpathlineto{\pgfqpoint{8.632454in}{4.098768in}}%
\pgfpathclose%
\pgfusepath{stroke,fill}%
\end{pgfscope}%
\begin{pgfscope}%
\pgfpathrectangle{\pgfqpoint{7.622482in}{2.920818in}}{\pgfqpoint{2.177280in}{2.201755in}}%
\pgfusepath{clip}%
\pgfsetbuttcap%
\pgfsetroundjoin%
\definecolor{currentfill}{rgb}{1.000000,0.498039,0.054902}%
\pgfsetfillcolor{currentfill}%
\pgfsetlinewidth{0.481800pt}%
\definecolor{currentstroke}{rgb}{1.000000,1.000000,1.000000}%
\pgfsetstrokecolor{currentstroke}%
\pgfsetdash{}{0pt}%
\pgfpathmoveto{\pgfqpoint{8.700208in}{4.132693in}}%
\pgfpathcurveto{\pgfqpoint{8.711258in}{4.132693in}}{\pgfqpoint{8.721857in}{4.137083in}}{\pgfqpoint{8.729671in}{4.144897in}}%
\pgfpathcurveto{\pgfqpoint{8.737485in}{4.152711in}}{\pgfqpoint{8.741875in}{4.163310in}}{\pgfqpoint{8.741875in}{4.174360in}}%
\pgfpathcurveto{\pgfqpoint{8.741875in}{4.185410in}}{\pgfqpoint{8.737485in}{4.196009in}}{\pgfqpoint{8.729671in}{4.203823in}}%
\pgfpathcurveto{\pgfqpoint{8.721857in}{4.211636in}}{\pgfqpoint{8.711258in}{4.216027in}}{\pgfqpoint{8.700208in}{4.216027in}}%
\pgfpathcurveto{\pgfqpoint{8.689158in}{4.216027in}}{\pgfqpoint{8.678559in}{4.211636in}}{\pgfqpoint{8.670745in}{4.203823in}}%
\pgfpathcurveto{\pgfqpoint{8.662932in}{4.196009in}}{\pgfqpoint{8.658542in}{4.185410in}}{\pgfqpoint{8.658542in}{4.174360in}}%
\pgfpathcurveto{\pgfqpoint{8.658542in}{4.163310in}}{\pgfqpoint{8.662932in}{4.152711in}}{\pgfqpoint{8.670745in}{4.144897in}}%
\pgfpathcurveto{\pgfqpoint{8.678559in}{4.137083in}}{\pgfqpoint{8.689158in}{4.132693in}}{\pgfqpoint{8.700208in}{4.132693in}}%
\pgfpathlineto{\pgfqpoint{8.700208in}{4.132693in}}%
\pgfpathclose%
\pgfusepath{stroke,fill}%
\end{pgfscope}%
\begin{pgfscope}%
\pgfpathrectangle{\pgfqpoint{7.622482in}{2.920818in}}{\pgfqpoint{2.177280in}{2.201755in}}%
\pgfusepath{clip}%
\pgfsetbuttcap%
\pgfsetroundjoin%
\definecolor{currentfill}{rgb}{1.000000,0.498039,0.054902}%
\pgfsetfillcolor{currentfill}%
\pgfsetlinewidth{0.481800pt}%
\definecolor{currentstroke}{rgb}{1.000000,1.000000,1.000000}%
\pgfsetstrokecolor{currentstroke}%
\pgfsetdash{}{0pt}%
\pgfpathmoveto{\pgfqpoint{8.700208in}{4.268395in}}%
\pgfpathcurveto{\pgfqpoint{8.711258in}{4.268395in}}{\pgfqpoint{8.721857in}{4.272785in}}{\pgfqpoint{8.729671in}{4.280599in}}%
\pgfpathcurveto{\pgfqpoint{8.737485in}{4.288412in}}{\pgfqpoint{8.741875in}{4.299011in}}{\pgfqpoint{8.741875in}{4.310061in}}%
\pgfpathcurveto{\pgfqpoint{8.741875in}{4.321111in}}{\pgfqpoint{8.737485in}{4.331710in}}{\pgfqpoint{8.729671in}{4.339524in}}%
\pgfpathcurveto{\pgfqpoint{8.721857in}{4.347338in}}{\pgfqpoint{8.711258in}{4.351728in}}{\pgfqpoint{8.700208in}{4.351728in}}%
\pgfpathcurveto{\pgfqpoint{8.689158in}{4.351728in}}{\pgfqpoint{8.678559in}{4.347338in}}{\pgfqpoint{8.670745in}{4.339524in}}%
\pgfpathcurveto{\pgfqpoint{8.662932in}{4.331710in}}{\pgfqpoint{8.658542in}{4.321111in}}{\pgfqpoint{8.658542in}{4.310061in}}%
\pgfpathcurveto{\pgfqpoint{8.658542in}{4.299011in}}{\pgfqpoint{8.662932in}{4.288412in}}{\pgfqpoint{8.670745in}{4.280599in}}%
\pgfpathcurveto{\pgfqpoint{8.678559in}{4.272785in}}{\pgfqpoint{8.689158in}{4.268395in}}{\pgfqpoint{8.700208in}{4.268395in}}%
\pgfpathlineto{\pgfqpoint{8.700208in}{4.268395in}}%
\pgfpathclose%
\pgfusepath{stroke,fill}%
\end{pgfscope}%
\begin{pgfscope}%
\pgfpathrectangle{\pgfqpoint{7.622482in}{2.920818in}}{\pgfqpoint{2.177280in}{2.201755in}}%
\pgfusepath{clip}%
\pgfsetbuttcap%
\pgfsetroundjoin%
\definecolor{currentfill}{rgb}{1.000000,0.498039,0.054902}%
\pgfsetfillcolor{currentfill}%
\pgfsetlinewidth{0.481800pt}%
\definecolor{currentstroke}{rgb}{1.000000,1.000000,1.000000}%
\pgfsetstrokecolor{currentstroke}%
\pgfsetdash{}{0pt}%
\pgfpathmoveto{\pgfqpoint{8.903470in}{4.336245in}}%
\pgfpathcurveto{\pgfqpoint{8.914520in}{4.336245in}}{\pgfqpoint{8.925119in}{4.340636in}}{\pgfqpoint{8.932933in}{4.348449in}}%
\pgfpathcurveto{\pgfqpoint{8.940746in}{4.356263in}}{\pgfqpoint{8.945136in}{4.366862in}}{\pgfqpoint{8.945136in}{4.377912in}}%
\pgfpathcurveto{\pgfqpoint{8.945136in}{4.388962in}}{\pgfqpoint{8.940746in}{4.399561in}}{\pgfqpoint{8.932933in}{4.407375in}}%
\pgfpathcurveto{\pgfqpoint{8.925119in}{4.415188in}}{\pgfqpoint{8.914520in}{4.419579in}}{\pgfqpoint{8.903470in}{4.419579in}}%
\pgfpathcurveto{\pgfqpoint{8.892420in}{4.419579in}}{\pgfqpoint{8.881821in}{4.415188in}}{\pgfqpoint{8.874007in}{4.407375in}}%
\pgfpathcurveto{\pgfqpoint{8.866193in}{4.399561in}}{\pgfqpoint{8.861803in}{4.388962in}}{\pgfqpoint{8.861803in}{4.377912in}}%
\pgfpathcurveto{\pgfqpoint{8.861803in}{4.366862in}}{\pgfqpoint{8.866193in}{4.356263in}}{\pgfqpoint{8.874007in}{4.348449in}}%
\pgfpathcurveto{\pgfqpoint{8.881821in}{4.340636in}}{\pgfqpoint{8.892420in}{4.336245in}}{\pgfqpoint{8.903470in}{4.336245in}}%
\pgfpathlineto{\pgfqpoint{8.903470in}{4.336245in}}%
\pgfpathclose%
\pgfusepath{stroke,fill}%
\end{pgfscope}%
\begin{pgfscope}%
\pgfpathrectangle{\pgfqpoint{7.622482in}{2.920818in}}{\pgfqpoint{2.177280in}{2.201755in}}%
\pgfusepath{clip}%
\pgfsetbuttcap%
\pgfsetroundjoin%
\definecolor{currentfill}{rgb}{1.000000,0.498039,0.054902}%
\pgfsetfillcolor{currentfill}%
\pgfsetlinewidth{0.481800pt}%
\definecolor{currentstroke}{rgb}{1.000000,1.000000,1.000000}%
\pgfsetstrokecolor{currentstroke}%
\pgfsetdash{}{0pt}%
\pgfpathmoveto{\pgfqpoint{8.767962in}{4.166619in}}%
\pgfpathcurveto{\pgfqpoint{8.779012in}{4.166619in}}{\pgfqpoint{8.789611in}{4.171009in}}{\pgfqpoint{8.797425in}{4.178822in}}%
\pgfpathcurveto{\pgfqpoint{8.805238in}{4.186636in}}{\pgfqpoint{8.809629in}{4.197235in}}{\pgfqpoint{8.809629in}{4.208285in}}%
\pgfpathcurveto{\pgfqpoint{8.809629in}{4.219335in}}{\pgfqpoint{8.805238in}{4.229934in}}{\pgfqpoint{8.797425in}{4.237748in}}%
\pgfpathcurveto{\pgfqpoint{8.789611in}{4.245562in}}{\pgfqpoint{8.779012in}{4.249952in}}{\pgfqpoint{8.767962in}{4.249952in}}%
\pgfpathcurveto{\pgfqpoint{8.756912in}{4.249952in}}{\pgfqpoint{8.746313in}{4.245562in}}{\pgfqpoint{8.738499in}{4.237748in}}%
\pgfpathcurveto{\pgfqpoint{8.730686in}{4.229934in}}{\pgfqpoint{8.726295in}{4.219335in}}{\pgfqpoint{8.726295in}{4.208285in}}%
\pgfpathcurveto{\pgfqpoint{8.726295in}{4.197235in}}{\pgfqpoint{8.730686in}{4.186636in}}{\pgfqpoint{8.738499in}{4.178822in}}%
\pgfpathcurveto{\pgfqpoint{8.746313in}{4.171009in}}{\pgfqpoint{8.756912in}{4.166619in}}{\pgfqpoint{8.767962in}{4.166619in}}%
\pgfpathlineto{\pgfqpoint{8.767962in}{4.166619in}}%
\pgfpathclose%
\pgfusepath{stroke,fill}%
\end{pgfscope}%
\begin{pgfscope}%
\pgfpathrectangle{\pgfqpoint{7.622482in}{2.920818in}}{\pgfqpoint{2.177280in}{2.201755in}}%
\pgfusepath{clip}%
\pgfsetbuttcap%
\pgfsetroundjoin%
\definecolor{currentfill}{rgb}{1.000000,0.498039,0.054902}%
\pgfsetfillcolor{currentfill}%
\pgfsetlinewidth{0.481800pt}%
\definecolor{currentstroke}{rgb}{1.000000,1.000000,1.000000}%
\pgfsetstrokecolor{currentstroke}%
\pgfsetdash{}{0pt}%
\pgfpathmoveto{\pgfqpoint{8.429193in}{3.827365in}}%
\pgfpathcurveto{\pgfqpoint{8.440243in}{3.827365in}}{\pgfqpoint{8.450842in}{3.831755in}}{\pgfqpoint{8.458656in}{3.839569in}}%
\pgfpathcurveto{\pgfqpoint{8.466469in}{3.847383in}}{\pgfqpoint{8.470859in}{3.857982in}}{\pgfqpoint{8.470859in}{3.869032in}}%
\pgfpathcurveto{\pgfqpoint{8.470859in}{3.880082in}}{\pgfqpoint{8.466469in}{3.890681in}}{\pgfqpoint{8.458656in}{3.898495in}}%
\pgfpathcurveto{\pgfqpoint{8.450842in}{3.906308in}}{\pgfqpoint{8.440243in}{3.910698in}}{\pgfqpoint{8.429193in}{3.910698in}}%
\pgfpathcurveto{\pgfqpoint{8.418143in}{3.910698in}}{\pgfqpoint{8.407544in}{3.906308in}}{\pgfqpoint{8.399730in}{3.898495in}}%
\pgfpathcurveto{\pgfqpoint{8.391916in}{3.890681in}}{\pgfqpoint{8.387526in}{3.880082in}}{\pgfqpoint{8.387526in}{3.869032in}}%
\pgfpathcurveto{\pgfqpoint{8.387526in}{3.857982in}}{\pgfqpoint{8.391916in}{3.847383in}}{\pgfqpoint{8.399730in}{3.839569in}}%
\pgfpathcurveto{\pgfqpoint{8.407544in}{3.831755in}}{\pgfqpoint{8.418143in}{3.827365in}}{\pgfqpoint{8.429193in}{3.827365in}}%
\pgfpathlineto{\pgfqpoint{8.429193in}{3.827365in}}%
\pgfpathclose%
\pgfusepath{stroke,fill}%
\end{pgfscope}%
\begin{pgfscope}%
\pgfpathrectangle{\pgfqpoint{7.622482in}{2.920818in}}{\pgfqpoint{2.177280in}{2.201755in}}%
\pgfusepath{clip}%
\pgfsetbuttcap%
\pgfsetroundjoin%
\definecolor{currentfill}{rgb}{1.000000,0.498039,0.054902}%
\pgfsetfillcolor{currentfill}%
\pgfsetlinewidth{0.481800pt}%
\definecolor{currentstroke}{rgb}{1.000000,1.000000,1.000000}%
\pgfsetstrokecolor{currentstroke}%
\pgfsetdash{}{0pt}%
\pgfpathmoveto{\pgfqpoint{8.496947in}{3.929141in}}%
\pgfpathcurveto{\pgfqpoint{8.507997in}{3.929141in}}{\pgfqpoint{8.518596in}{3.933531in}}{\pgfqpoint{8.526409in}{3.941345in}}%
\pgfpathcurveto{\pgfqpoint{8.534223in}{3.949159in}}{\pgfqpoint{8.538613in}{3.959758in}}{\pgfqpoint{8.538613in}{3.970808in}}%
\pgfpathcurveto{\pgfqpoint{8.538613in}{3.981858in}}{\pgfqpoint{8.534223in}{3.992457in}}{\pgfqpoint{8.526409in}{4.000271in}}%
\pgfpathcurveto{\pgfqpoint{8.518596in}{4.008084in}}{\pgfqpoint{8.507997in}{4.012474in}}{\pgfqpoint{8.496947in}{4.012474in}}%
\pgfpathcurveto{\pgfqpoint{8.485896in}{4.012474in}}{\pgfqpoint{8.475297in}{4.008084in}}{\pgfqpoint{8.467484in}{4.000271in}}%
\pgfpathcurveto{\pgfqpoint{8.459670in}{3.992457in}}{\pgfqpoint{8.455280in}{3.981858in}}{\pgfqpoint{8.455280in}{3.970808in}}%
\pgfpathcurveto{\pgfqpoint{8.455280in}{3.959758in}}{\pgfqpoint{8.459670in}{3.949159in}}{\pgfqpoint{8.467484in}{3.941345in}}%
\pgfpathcurveto{\pgfqpoint{8.475297in}{3.933531in}}{\pgfqpoint{8.485896in}{3.929141in}}{\pgfqpoint{8.496947in}{3.929141in}}%
\pgfpathlineto{\pgfqpoint{8.496947in}{3.929141in}}%
\pgfpathclose%
\pgfusepath{stroke,fill}%
\end{pgfscope}%
\begin{pgfscope}%
\pgfpathrectangle{\pgfqpoint{7.622482in}{2.920818in}}{\pgfqpoint{2.177280in}{2.201755in}}%
\pgfusepath{clip}%
\pgfsetbuttcap%
\pgfsetroundjoin%
\definecolor{currentfill}{rgb}{1.000000,0.498039,0.054902}%
\pgfsetfillcolor{currentfill}%
\pgfsetlinewidth{0.481800pt}%
\definecolor{currentstroke}{rgb}{1.000000,1.000000,1.000000}%
\pgfsetstrokecolor{currentstroke}%
\pgfsetdash{}{0pt}%
\pgfpathmoveto{\pgfqpoint{8.429193in}{3.895216in}}%
\pgfpathcurveto{\pgfqpoint{8.440243in}{3.895216in}}{\pgfqpoint{8.450842in}{3.899606in}}{\pgfqpoint{8.458656in}{3.907420in}}%
\pgfpathcurveto{\pgfqpoint{8.466469in}{3.915233in}}{\pgfqpoint{8.470859in}{3.925832in}}{\pgfqpoint{8.470859in}{3.936882in}}%
\pgfpathcurveto{\pgfqpoint{8.470859in}{3.947933in}}{\pgfqpoint{8.466469in}{3.958532in}}{\pgfqpoint{8.458656in}{3.966345in}}%
\pgfpathcurveto{\pgfqpoint{8.450842in}{3.974159in}}{\pgfqpoint{8.440243in}{3.978549in}}{\pgfqpoint{8.429193in}{3.978549in}}%
\pgfpathcurveto{\pgfqpoint{8.418143in}{3.978549in}}{\pgfqpoint{8.407544in}{3.974159in}}{\pgfqpoint{8.399730in}{3.966345in}}%
\pgfpathcurveto{\pgfqpoint{8.391916in}{3.958532in}}{\pgfqpoint{8.387526in}{3.947933in}}{\pgfqpoint{8.387526in}{3.936882in}}%
\pgfpathcurveto{\pgfqpoint{8.387526in}{3.925832in}}{\pgfqpoint{8.391916in}{3.915233in}}{\pgfqpoint{8.399730in}{3.907420in}}%
\pgfpathcurveto{\pgfqpoint{8.407544in}{3.899606in}}{\pgfqpoint{8.418143in}{3.895216in}}{\pgfqpoint{8.429193in}{3.895216in}}%
\pgfpathlineto{\pgfqpoint{8.429193in}{3.895216in}}%
\pgfpathclose%
\pgfusepath{stroke,fill}%
\end{pgfscope}%
\begin{pgfscope}%
\pgfpathrectangle{\pgfqpoint{7.622482in}{2.920818in}}{\pgfqpoint{2.177280in}{2.201755in}}%
\pgfusepath{clip}%
\pgfsetbuttcap%
\pgfsetroundjoin%
\definecolor{currentfill}{rgb}{1.000000,0.498039,0.054902}%
\pgfsetfillcolor{currentfill}%
\pgfsetlinewidth{0.481800pt}%
\definecolor{currentstroke}{rgb}{1.000000,1.000000,1.000000}%
\pgfsetstrokecolor{currentstroke}%
\pgfsetdash{}{0pt}%
\pgfpathmoveto{\pgfqpoint{8.564700in}{3.963066in}}%
\pgfpathcurveto{\pgfqpoint{8.575751in}{3.963066in}}{\pgfqpoint{8.586350in}{3.967457in}}{\pgfqpoint{8.594163in}{3.975270in}}%
\pgfpathcurveto{\pgfqpoint{8.601977in}{3.983084in}}{\pgfqpoint{8.606367in}{3.993683in}}{\pgfqpoint{8.606367in}{4.004733in}}%
\pgfpathcurveto{\pgfqpoint{8.606367in}{4.015783in}}{\pgfqpoint{8.601977in}{4.026382in}}{\pgfqpoint{8.594163in}{4.034196in}}%
\pgfpathcurveto{\pgfqpoint{8.586350in}{4.042010in}}{\pgfqpoint{8.575751in}{4.046400in}}{\pgfqpoint{8.564700in}{4.046400in}}%
\pgfpathcurveto{\pgfqpoint{8.553650in}{4.046400in}}{\pgfqpoint{8.543051in}{4.042010in}}{\pgfqpoint{8.535238in}{4.034196in}}%
\pgfpathcurveto{\pgfqpoint{8.527424in}{4.026382in}}{\pgfqpoint{8.523034in}{4.015783in}}{\pgfqpoint{8.523034in}{4.004733in}}%
\pgfpathcurveto{\pgfqpoint{8.523034in}{3.993683in}}{\pgfqpoint{8.527424in}{3.983084in}}{\pgfqpoint{8.535238in}{3.975270in}}%
\pgfpathcurveto{\pgfqpoint{8.543051in}{3.967457in}}{\pgfqpoint{8.553650in}{3.963066in}}{\pgfqpoint{8.564700in}{3.963066in}}%
\pgfpathlineto{\pgfqpoint{8.564700in}{3.963066in}}%
\pgfpathclose%
\pgfusepath{stroke,fill}%
\end{pgfscope}%
\begin{pgfscope}%
\pgfpathrectangle{\pgfqpoint{7.622482in}{2.920818in}}{\pgfqpoint{2.177280in}{2.201755in}}%
\pgfusepath{clip}%
\pgfsetbuttcap%
\pgfsetroundjoin%
\definecolor{currentfill}{rgb}{1.000000,0.498039,0.054902}%
\pgfsetfillcolor{currentfill}%
\pgfsetlinewidth{0.481800pt}%
\definecolor{currentstroke}{rgb}{1.000000,1.000000,1.000000}%
\pgfsetstrokecolor{currentstroke}%
\pgfsetdash{}{0pt}%
\pgfpathmoveto{\pgfqpoint{8.835716in}{4.370171in}}%
\pgfpathcurveto{\pgfqpoint{8.846766in}{4.370171in}}{\pgfqpoint{8.857365in}{4.374561in}}{\pgfqpoint{8.865179in}{4.382375in}}%
\pgfpathcurveto{\pgfqpoint{8.872992in}{4.390188in}}{\pgfqpoint{8.877383in}{4.400787in}}{\pgfqpoint{8.877383in}{4.411837in}}%
\pgfpathcurveto{\pgfqpoint{8.877383in}{4.422887in}}{\pgfqpoint{8.872992in}{4.433486in}}{\pgfqpoint{8.865179in}{4.441300in}}%
\pgfpathcurveto{\pgfqpoint{8.857365in}{4.449114in}}{\pgfqpoint{8.846766in}{4.453504in}}{\pgfqpoint{8.835716in}{4.453504in}}%
\pgfpathcurveto{\pgfqpoint{8.824666in}{4.453504in}}{\pgfqpoint{8.814067in}{4.449114in}}{\pgfqpoint{8.806253in}{4.441300in}}%
\pgfpathcurveto{\pgfqpoint{8.798440in}{4.433486in}}{\pgfqpoint{8.794049in}{4.422887in}}{\pgfqpoint{8.794049in}{4.411837in}}%
\pgfpathcurveto{\pgfqpoint{8.794049in}{4.400787in}}{\pgfqpoint{8.798440in}{4.390188in}}{\pgfqpoint{8.806253in}{4.382375in}}%
\pgfpathcurveto{\pgfqpoint{8.814067in}{4.374561in}}{\pgfqpoint{8.824666in}{4.370171in}}{\pgfqpoint{8.835716in}{4.370171in}}%
\pgfpathlineto{\pgfqpoint{8.835716in}{4.370171in}}%
\pgfpathclose%
\pgfusepath{stroke,fill}%
\end{pgfscope}%
\begin{pgfscope}%
\pgfpathrectangle{\pgfqpoint{7.622482in}{2.920818in}}{\pgfqpoint{2.177280in}{2.201755in}}%
\pgfusepath{clip}%
\pgfsetbuttcap%
\pgfsetroundjoin%
\definecolor{currentfill}{rgb}{1.000000,0.498039,0.054902}%
\pgfsetfillcolor{currentfill}%
\pgfsetlinewidth{0.481800pt}%
\definecolor{currentstroke}{rgb}{1.000000,1.000000,1.000000}%
\pgfsetstrokecolor{currentstroke}%
\pgfsetdash{}{0pt}%
\pgfpathmoveto{\pgfqpoint{8.767962in}{4.166619in}}%
\pgfpathcurveto{\pgfqpoint{8.779012in}{4.166619in}}{\pgfqpoint{8.789611in}{4.171009in}}{\pgfqpoint{8.797425in}{4.178822in}}%
\pgfpathcurveto{\pgfqpoint{8.805238in}{4.186636in}}{\pgfqpoint{8.809629in}{4.197235in}}{\pgfqpoint{8.809629in}{4.208285in}}%
\pgfpathcurveto{\pgfqpoint{8.809629in}{4.219335in}}{\pgfqpoint{8.805238in}{4.229934in}}{\pgfqpoint{8.797425in}{4.237748in}}%
\pgfpathcurveto{\pgfqpoint{8.789611in}{4.245562in}}{\pgfqpoint{8.779012in}{4.249952in}}{\pgfqpoint{8.767962in}{4.249952in}}%
\pgfpathcurveto{\pgfqpoint{8.756912in}{4.249952in}}{\pgfqpoint{8.746313in}{4.245562in}}{\pgfqpoint{8.738499in}{4.237748in}}%
\pgfpathcurveto{\pgfqpoint{8.730686in}{4.229934in}}{\pgfqpoint{8.726295in}{4.219335in}}{\pgfqpoint{8.726295in}{4.208285in}}%
\pgfpathcurveto{\pgfqpoint{8.726295in}{4.197235in}}{\pgfqpoint{8.730686in}{4.186636in}}{\pgfqpoint{8.738499in}{4.178822in}}%
\pgfpathcurveto{\pgfqpoint{8.746313in}{4.171009in}}{\pgfqpoint{8.756912in}{4.166619in}}{\pgfqpoint{8.767962in}{4.166619in}}%
\pgfpathlineto{\pgfqpoint{8.767962in}{4.166619in}}%
\pgfpathclose%
\pgfusepath{stroke,fill}%
\end{pgfscope}%
\begin{pgfscope}%
\pgfpathrectangle{\pgfqpoint{7.622482in}{2.920818in}}{\pgfqpoint{2.177280in}{2.201755in}}%
\pgfusepath{clip}%
\pgfsetbuttcap%
\pgfsetroundjoin%
\definecolor{currentfill}{rgb}{1.000000,0.498039,0.054902}%
\pgfsetfillcolor{currentfill}%
\pgfsetlinewidth{0.481800pt}%
\definecolor{currentstroke}{rgb}{1.000000,1.000000,1.000000}%
\pgfsetstrokecolor{currentstroke}%
\pgfsetdash{}{0pt}%
\pgfpathmoveto{\pgfqpoint{8.835716in}{4.166619in}}%
\pgfpathcurveto{\pgfqpoint{8.846766in}{4.166619in}}{\pgfqpoint{8.857365in}{4.171009in}}{\pgfqpoint{8.865179in}{4.178822in}}%
\pgfpathcurveto{\pgfqpoint{8.872992in}{4.186636in}}{\pgfqpoint{8.877383in}{4.197235in}}{\pgfqpoint{8.877383in}{4.208285in}}%
\pgfpathcurveto{\pgfqpoint{8.877383in}{4.219335in}}{\pgfqpoint{8.872992in}{4.229934in}}{\pgfqpoint{8.865179in}{4.237748in}}%
\pgfpathcurveto{\pgfqpoint{8.857365in}{4.245562in}}{\pgfqpoint{8.846766in}{4.249952in}}{\pgfqpoint{8.835716in}{4.249952in}}%
\pgfpathcurveto{\pgfqpoint{8.824666in}{4.249952in}}{\pgfqpoint{8.814067in}{4.245562in}}{\pgfqpoint{8.806253in}{4.237748in}}%
\pgfpathcurveto{\pgfqpoint{8.798440in}{4.229934in}}{\pgfqpoint{8.794049in}{4.219335in}}{\pgfqpoint{8.794049in}{4.208285in}}%
\pgfpathcurveto{\pgfqpoint{8.794049in}{4.197235in}}{\pgfqpoint{8.798440in}{4.186636in}}{\pgfqpoint{8.806253in}{4.178822in}}%
\pgfpathcurveto{\pgfqpoint{8.814067in}{4.171009in}}{\pgfqpoint{8.824666in}{4.166619in}}{\pgfqpoint{8.835716in}{4.166619in}}%
\pgfpathlineto{\pgfqpoint{8.835716in}{4.166619in}}%
\pgfpathclose%
\pgfusepath{stroke,fill}%
\end{pgfscope}%
\begin{pgfscope}%
\pgfpathrectangle{\pgfqpoint{7.622482in}{2.920818in}}{\pgfqpoint{2.177280in}{2.201755in}}%
\pgfusepath{clip}%
\pgfsetbuttcap%
\pgfsetroundjoin%
\definecolor{currentfill}{rgb}{1.000000,0.498039,0.054902}%
\pgfsetfillcolor{currentfill}%
\pgfsetlinewidth{0.481800pt}%
\definecolor{currentstroke}{rgb}{1.000000,1.000000,1.000000}%
\pgfsetstrokecolor{currentstroke}%
\pgfsetdash{}{0pt}%
\pgfpathmoveto{\pgfqpoint{8.767962in}{4.234469in}}%
\pgfpathcurveto{\pgfqpoint{8.779012in}{4.234469in}}{\pgfqpoint{8.789611in}{4.238860in}}{\pgfqpoint{8.797425in}{4.246673in}}%
\pgfpathcurveto{\pgfqpoint{8.805238in}{4.254487in}}{\pgfqpoint{8.809629in}{4.265086in}}{\pgfqpoint{8.809629in}{4.276136in}}%
\pgfpathcurveto{\pgfqpoint{8.809629in}{4.287186in}}{\pgfqpoint{8.805238in}{4.297785in}}{\pgfqpoint{8.797425in}{4.305599in}}%
\pgfpathcurveto{\pgfqpoint{8.789611in}{4.313412in}}{\pgfqpoint{8.779012in}{4.317803in}}{\pgfqpoint{8.767962in}{4.317803in}}%
\pgfpathcurveto{\pgfqpoint{8.756912in}{4.317803in}}{\pgfqpoint{8.746313in}{4.313412in}}{\pgfqpoint{8.738499in}{4.305599in}}%
\pgfpathcurveto{\pgfqpoint{8.730686in}{4.297785in}}{\pgfqpoint{8.726295in}{4.287186in}}{\pgfqpoint{8.726295in}{4.276136in}}%
\pgfpathcurveto{\pgfqpoint{8.726295in}{4.265086in}}{\pgfqpoint{8.730686in}{4.254487in}}{\pgfqpoint{8.738499in}{4.246673in}}%
\pgfpathcurveto{\pgfqpoint{8.746313in}{4.238860in}}{\pgfqpoint{8.756912in}{4.234469in}}{\pgfqpoint{8.767962in}{4.234469in}}%
\pgfpathlineto{\pgfqpoint{8.767962in}{4.234469in}}%
\pgfpathclose%
\pgfusepath{stroke,fill}%
\end{pgfscope}%
\begin{pgfscope}%
\pgfpathrectangle{\pgfqpoint{7.622482in}{2.920818in}}{\pgfqpoint{2.177280in}{2.201755in}}%
\pgfusepath{clip}%
\pgfsetbuttcap%
\pgfsetroundjoin%
\definecolor{currentfill}{rgb}{1.000000,0.498039,0.054902}%
\pgfsetfillcolor{currentfill}%
\pgfsetlinewidth{0.481800pt}%
\definecolor{currentstroke}{rgb}{1.000000,1.000000,1.000000}%
\pgfsetstrokecolor{currentstroke}%
\pgfsetdash{}{0pt}%
\pgfpathmoveto{\pgfqpoint{8.632454in}{4.132693in}}%
\pgfpathcurveto{\pgfqpoint{8.643504in}{4.132693in}}{\pgfqpoint{8.654104in}{4.137083in}}{\pgfqpoint{8.661917in}{4.144897in}}%
\pgfpathcurveto{\pgfqpoint{8.669731in}{4.152711in}}{\pgfqpoint{8.674121in}{4.163310in}}{\pgfqpoint{8.674121in}{4.174360in}}%
\pgfpathcurveto{\pgfqpoint{8.674121in}{4.185410in}}{\pgfqpoint{8.669731in}{4.196009in}}{\pgfqpoint{8.661917in}{4.203823in}}%
\pgfpathcurveto{\pgfqpoint{8.654104in}{4.211636in}}{\pgfqpoint{8.643504in}{4.216027in}}{\pgfqpoint{8.632454in}{4.216027in}}%
\pgfpathcurveto{\pgfqpoint{8.621404in}{4.216027in}}{\pgfqpoint{8.610805in}{4.211636in}}{\pgfqpoint{8.602992in}{4.203823in}}%
\pgfpathcurveto{\pgfqpoint{8.595178in}{4.196009in}}{\pgfqpoint{8.590788in}{4.185410in}}{\pgfqpoint{8.590788in}{4.174360in}}%
\pgfpathcurveto{\pgfqpoint{8.590788in}{4.163310in}}{\pgfqpoint{8.595178in}{4.152711in}}{\pgfqpoint{8.602992in}{4.144897in}}%
\pgfpathcurveto{\pgfqpoint{8.610805in}{4.137083in}}{\pgfqpoint{8.621404in}{4.132693in}}{\pgfqpoint{8.632454in}{4.132693in}}%
\pgfpathlineto{\pgfqpoint{8.632454in}{4.132693in}}%
\pgfpathclose%
\pgfusepath{stroke,fill}%
\end{pgfscope}%
\begin{pgfscope}%
\pgfpathrectangle{\pgfqpoint{7.622482in}{2.920818in}}{\pgfqpoint{2.177280in}{2.201755in}}%
\pgfusepath{clip}%
\pgfsetbuttcap%
\pgfsetroundjoin%
\definecolor{currentfill}{rgb}{1.000000,0.498039,0.054902}%
\pgfsetfillcolor{currentfill}%
\pgfsetlinewidth{0.481800pt}%
\definecolor{currentstroke}{rgb}{1.000000,1.000000,1.000000}%
\pgfsetstrokecolor{currentstroke}%
\pgfsetdash{}{0pt}%
\pgfpathmoveto{\pgfqpoint{8.632454in}{4.030917in}}%
\pgfpathcurveto{\pgfqpoint{8.643504in}{4.030917in}}{\pgfqpoint{8.654104in}{4.035307in}}{\pgfqpoint{8.661917in}{4.043121in}}%
\pgfpathcurveto{\pgfqpoint{8.669731in}{4.050935in}}{\pgfqpoint{8.674121in}{4.061534in}}{\pgfqpoint{8.674121in}{4.072584in}}%
\pgfpathcurveto{\pgfqpoint{8.674121in}{4.083634in}}{\pgfqpoint{8.669731in}{4.094233in}}{\pgfqpoint{8.661917in}{4.102047in}}%
\pgfpathcurveto{\pgfqpoint{8.654104in}{4.109860in}}{\pgfqpoint{8.643504in}{4.114251in}}{\pgfqpoint{8.632454in}{4.114251in}}%
\pgfpathcurveto{\pgfqpoint{8.621404in}{4.114251in}}{\pgfqpoint{8.610805in}{4.109860in}}{\pgfqpoint{8.602992in}{4.102047in}}%
\pgfpathcurveto{\pgfqpoint{8.595178in}{4.094233in}}{\pgfqpoint{8.590788in}{4.083634in}}{\pgfqpoint{8.590788in}{4.072584in}}%
\pgfpathcurveto{\pgfqpoint{8.590788in}{4.061534in}}{\pgfqpoint{8.595178in}{4.050935in}}{\pgfqpoint{8.602992in}{4.043121in}}%
\pgfpathcurveto{\pgfqpoint{8.610805in}{4.035307in}}{\pgfqpoint{8.621404in}{4.030917in}}{\pgfqpoint{8.632454in}{4.030917in}}%
\pgfpathlineto{\pgfqpoint{8.632454in}{4.030917in}}%
\pgfpathclose%
\pgfusepath{stroke,fill}%
\end{pgfscope}%
\begin{pgfscope}%
\pgfpathrectangle{\pgfqpoint{7.622482in}{2.920818in}}{\pgfqpoint{2.177280in}{2.201755in}}%
\pgfusepath{clip}%
\pgfsetbuttcap%
\pgfsetroundjoin%
\definecolor{currentfill}{rgb}{1.000000,0.498039,0.054902}%
\pgfsetfillcolor{currentfill}%
\pgfsetlinewidth{0.481800pt}%
\definecolor{currentstroke}{rgb}{1.000000,1.000000,1.000000}%
\pgfsetstrokecolor{currentstroke}%
\pgfsetdash{}{0pt}%
\pgfpathmoveto{\pgfqpoint{8.632454in}{3.996992in}}%
\pgfpathcurveto{\pgfqpoint{8.643504in}{3.996992in}}{\pgfqpoint{8.654104in}{4.001382in}}{\pgfqpoint{8.661917in}{4.009196in}}%
\pgfpathcurveto{\pgfqpoint{8.669731in}{4.017009in}}{\pgfqpoint{8.674121in}{4.027608in}}{\pgfqpoint{8.674121in}{4.038659in}}%
\pgfpathcurveto{\pgfqpoint{8.674121in}{4.049709in}}{\pgfqpoint{8.669731in}{4.060308in}}{\pgfqpoint{8.661917in}{4.068121in}}%
\pgfpathcurveto{\pgfqpoint{8.654104in}{4.075935in}}{\pgfqpoint{8.643504in}{4.080325in}}{\pgfqpoint{8.632454in}{4.080325in}}%
\pgfpathcurveto{\pgfqpoint{8.621404in}{4.080325in}}{\pgfqpoint{8.610805in}{4.075935in}}{\pgfqpoint{8.602992in}{4.068121in}}%
\pgfpathcurveto{\pgfqpoint{8.595178in}{4.060308in}}{\pgfqpoint{8.590788in}{4.049709in}}{\pgfqpoint{8.590788in}{4.038659in}}%
\pgfpathcurveto{\pgfqpoint{8.590788in}{4.027608in}}{\pgfqpoint{8.595178in}{4.017009in}}{\pgfqpoint{8.602992in}{4.009196in}}%
\pgfpathcurveto{\pgfqpoint{8.610805in}{4.001382in}}{\pgfqpoint{8.621404in}{3.996992in}}{\pgfqpoint{8.632454in}{3.996992in}}%
\pgfpathlineto{\pgfqpoint{8.632454in}{3.996992in}}%
\pgfpathclose%
\pgfusepath{stroke,fill}%
\end{pgfscope}%
\begin{pgfscope}%
\pgfpathrectangle{\pgfqpoint{7.622482in}{2.920818in}}{\pgfqpoint{2.177280in}{2.201755in}}%
\pgfusepath{clip}%
\pgfsetbuttcap%
\pgfsetroundjoin%
\definecolor{currentfill}{rgb}{1.000000,0.498039,0.054902}%
\pgfsetfillcolor{currentfill}%
\pgfsetlinewidth{0.481800pt}%
\definecolor{currentstroke}{rgb}{1.000000,1.000000,1.000000}%
\pgfsetstrokecolor{currentstroke}%
\pgfsetdash{}{0pt}%
\pgfpathmoveto{\pgfqpoint{8.564700in}{4.132693in}}%
\pgfpathcurveto{\pgfqpoint{8.575751in}{4.132693in}}{\pgfqpoint{8.586350in}{4.137083in}}{\pgfqpoint{8.594163in}{4.144897in}}%
\pgfpathcurveto{\pgfqpoint{8.601977in}{4.152711in}}{\pgfqpoint{8.606367in}{4.163310in}}{\pgfqpoint{8.606367in}{4.174360in}}%
\pgfpathcurveto{\pgfqpoint{8.606367in}{4.185410in}}{\pgfqpoint{8.601977in}{4.196009in}}{\pgfqpoint{8.594163in}{4.203823in}}%
\pgfpathcurveto{\pgfqpoint{8.586350in}{4.211636in}}{\pgfqpoint{8.575751in}{4.216027in}}{\pgfqpoint{8.564700in}{4.216027in}}%
\pgfpathcurveto{\pgfqpoint{8.553650in}{4.216027in}}{\pgfqpoint{8.543051in}{4.211636in}}{\pgfqpoint{8.535238in}{4.203823in}}%
\pgfpathcurveto{\pgfqpoint{8.527424in}{4.196009in}}{\pgfqpoint{8.523034in}{4.185410in}}{\pgfqpoint{8.523034in}{4.174360in}}%
\pgfpathcurveto{\pgfqpoint{8.523034in}{4.163310in}}{\pgfqpoint{8.527424in}{4.152711in}}{\pgfqpoint{8.535238in}{4.144897in}}%
\pgfpathcurveto{\pgfqpoint{8.543051in}{4.137083in}}{\pgfqpoint{8.553650in}{4.132693in}}{\pgfqpoint{8.564700in}{4.132693in}}%
\pgfpathlineto{\pgfqpoint{8.564700in}{4.132693in}}%
\pgfpathclose%
\pgfusepath{stroke,fill}%
\end{pgfscope}%
\begin{pgfscope}%
\pgfpathrectangle{\pgfqpoint{7.622482in}{2.920818in}}{\pgfqpoint{2.177280in}{2.201755in}}%
\pgfusepath{clip}%
\pgfsetbuttcap%
\pgfsetroundjoin%
\definecolor{currentfill}{rgb}{1.000000,0.498039,0.054902}%
\pgfsetfillcolor{currentfill}%
\pgfsetlinewidth{0.481800pt}%
\definecolor{currentstroke}{rgb}{1.000000,1.000000,1.000000}%
\pgfsetstrokecolor{currentstroke}%
\pgfsetdash{}{0pt}%
\pgfpathmoveto{\pgfqpoint{8.700208in}{4.200544in}}%
\pgfpathcurveto{\pgfqpoint{8.711258in}{4.200544in}}{\pgfqpoint{8.721857in}{4.204934in}}{\pgfqpoint{8.729671in}{4.212748in}}%
\pgfpathcurveto{\pgfqpoint{8.737485in}{4.220561in}}{\pgfqpoint{8.741875in}{4.231160in}}{\pgfqpoint{8.741875in}{4.242211in}}%
\pgfpathcurveto{\pgfqpoint{8.741875in}{4.253261in}}{\pgfqpoint{8.737485in}{4.263860in}}{\pgfqpoint{8.729671in}{4.271673in}}%
\pgfpathcurveto{\pgfqpoint{8.721857in}{4.279487in}}{\pgfqpoint{8.711258in}{4.283877in}}{\pgfqpoint{8.700208in}{4.283877in}}%
\pgfpathcurveto{\pgfqpoint{8.689158in}{4.283877in}}{\pgfqpoint{8.678559in}{4.279487in}}{\pgfqpoint{8.670745in}{4.271673in}}%
\pgfpathcurveto{\pgfqpoint{8.662932in}{4.263860in}}{\pgfqpoint{8.658542in}{4.253261in}}{\pgfqpoint{8.658542in}{4.242211in}}%
\pgfpathcurveto{\pgfqpoint{8.658542in}{4.231160in}}{\pgfqpoint{8.662932in}{4.220561in}}{\pgfqpoint{8.670745in}{4.212748in}}%
\pgfpathcurveto{\pgfqpoint{8.678559in}{4.204934in}}{\pgfqpoint{8.689158in}{4.200544in}}{\pgfqpoint{8.700208in}{4.200544in}}%
\pgfpathlineto{\pgfqpoint{8.700208in}{4.200544in}}%
\pgfpathclose%
\pgfusepath{stroke,fill}%
\end{pgfscope}%
\begin{pgfscope}%
\pgfpathrectangle{\pgfqpoint{7.622482in}{2.920818in}}{\pgfqpoint{2.177280in}{2.201755in}}%
\pgfusepath{clip}%
\pgfsetbuttcap%
\pgfsetroundjoin%
\definecolor{currentfill}{rgb}{1.000000,0.498039,0.054902}%
\pgfsetfillcolor{currentfill}%
\pgfsetlinewidth{0.481800pt}%
\definecolor{currentstroke}{rgb}{1.000000,1.000000,1.000000}%
\pgfsetstrokecolor{currentstroke}%
\pgfsetdash{}{0pt}%
\pgfpathmoveto{\pgfqpoint{8.564700in}{3.996992in}}%
\pgfpathcurveto{\pgfqpoint{8.575751in}{3.996992in}}{\pgfqpoint{8.586350in}{4.001382in}}{\pgfqpoint{8.594163in}{4.009196in}}%
\pgfpathcurveto{\pgfqpoint{8.601977in}{4.017009in}}{\pgfqpoint{8.606367in}{4.027608in}}{\pgfqpoint{8.606367in}{4.038659in}}%
\pgfpathcurveto{\pgfqpoint{8.606367in}{4.049709in}}{\pgfqpoint{8.601977in}{4.060308in}}{\pgfqpoint{8.594163in}{4.068121in}}%
\pgfpathcurveto{\pgfqpoint{8.586350in}{4.075935in}}{\pgfqpoint{8.575751in}{4.080325in}}{\pgfqpoint{8.564700in}{4.080325in}}%
\pgfpathcurveto{\pgfqpoint{8.553650in}{4.080325in}}{\pgfqpoint{8.543051in}{4.075935in}}{\pgfqpoint{8.535238in}{4.068121in}}%
\pgfpathcurveto{\pgfqpoint{8.527424in}{4.060308in}}{\pgfqpoint{8.523034in}{4.049709in}}{\pgfqpoint{8.523034in}{4.038659in}}%
\pgfpathcurveto{\pgfqpoint{8.523034in}{4.027608in}}{\pgfqpoint{8.527424in}{4.017009in}}{\pgfqpoint{8.535238in}{4.009196in}}%
\pgfpathcurveto{\pgfqpoint{8.543051in}{4.001382in}}{\pgfqpoint{8.553650in}{3.996992in}}{\pgfqpoint{8.564700in}{3.996992in}}%
\pgfpathlineto{\pgfqpoint{8.564700in}{3.996992in}}%
\pgfpathclose%
\pgfusepath{stroke,fill}%
\end{pgfscope}%
\begin{pgfscope}%
\pgfpathrectangle{\pgfqpoint{7.622482in}{2.920818in}}{\pgfqpoint{2.177280in}{2.201755in}}%
\pgfusepath{clip}%
\pgfsetbuttcap%
\pgfsetroundjoin%
\definecolor{currentfill}{rgb}{1.000000,0.498039,0.054902}%
\pgfsetfillcolor{currentfill}%
\pgfsetlinewidth{0.481800pt}%
\definecolor{currentstroke}{rgb}{1.000000,1.000000,1.000000}%
\pgfsetstrokecolor{currentstroke}%
\pgfsetdash{}{0pt}%
\pgfpathmoveto{\pgfqpoint{8.429193in}{3.759514in}}%
\pgfpathcurveto{\pgfqpoint{8.440243in}{3.759514in}}{\pgfqpoint{8.450842in}{3.763905in}}{\pgfqpoint{8.458656in}{3.771718in}}%
\pgfpathcurveto{\pgfqpoint{8.466469in}{3.779532in}}{\pgfqpoint{8.470859in}{3.790131in}}{\pgfqpoint{8.470859in}{3.801181in}}%
\pgfpathcurveto{\pgfqpoint{8.470859in}{3.812231in}}{\pgfqpoint{8.466469in}{3.822830in}}{\pgfqpoint{8.458656in}{3.830644in}}%
\pgfpathcurveto{\pgfqpoint{8.450842in}{3.838457in}}{\pgfqpoint{8.440243in}{3.842848in}}{\pgfqpoint{8.429193in}{3.842848in}}%
\pgfpathcurveto{\pgfqpoint{8.418143in}{3.842848in}}{\pgfqpoint{8.407544in}{3.838457in}}{\pgfqpoint{8.399730in}{3.830644in}}%
\pgfpathcurveto{\pgfqpoint{8.391916in}{3.822830in}}{\pgfqpoint{8.387526in}{3.812231in}}{\pgfqpoint{8.387526in}{3.801181in}}%
\pgfpathcurveto{\pgfqpoint{8.387526in}{3.790131in}}{\pgfqpoint{8.391916in}{3.779532in}}{\pgfqpoint{8.399730in}{3.771718in}}%
\pgfpathcurveto{\pgfqpoint{8.407544in}{3.763905in}}{\pgfqpoint{8.418143in}{3.759514in}}{\pgfqpoint{8.429193in}{3.759514in}}%
\pgfpathlineto{\pgfqpoint{8.429193in}{3.759514in}}%
\pgfpathclose%
\pgfusepath{stroke,fill}%
\end{pgfscope}%
\begin{pgfscope}%
\pgfpathrectangle{\pgfqpoint{7.622482in}{2.920818in}}{\pgfqpoint{2.177280in}{2.201755in}}%
\pgfusepath{clip}%
\pgfsetbuttcap%
\pgfsetroundjoin%
\definecolor{currentfill}{rgb}{1.000000,0.498039,0.054902}%
\pgfsetfillcolor{currentfill}%
\pgfsetlinewidth{0.481800pt}%
\definecolor{currentstroke}{rgb}{1.000000,1.000000,1.000000}%
\pgfsetstrokecolor{currentstroke}%
\pgfsetdash{}{0pt}%
\pgfpathmoveto{\pgfqpoint{8.632454in}{4.064843in}}%
\pgfpathcurveto{\pgfqpoint{8.643504in}{4.064843in}}{\pgfqpoint{8.654104in}{4.069233in}}{\pgfqpoint{8.661917in}{4.077046in}}%
\pgfpathcurveto{\pgfqpoint{8.669731in}{4.084860in}}{\pgfqpoint{8.674121in}{4.095459in}}{\pgfqpoint{8.674121in}{4.106509in}}%
\pgfpathcurveto{\pgfqpoint{8.674121in}{4.117559in}}{\pgfqpoint{8.669731in}{4.128158in}}{\pgfqpoint{8.661917in}{4.135972in}}%
\pgfpathcurveto{\pgfqpoint{8.654104in}{4.143786in}}{\pgfqpoint{8.643504in}{4.148176in}}{\pgfqpoint{8.632454in}{4.148176in}}%
\pgfpathcurveto{\pgfqpoint{8.621404in}{4.148176in}}{\pgfqpoint{8.610805in}{4.143786in}}{\pgfqpoint{8.602992in}{4.135972in}}%
\pgfpathcurveto{\pgfqpoint{8.595178in}{4.128158in}}{\pgfqpoint{8.590788in}{4.117559in}}{\pgfqpoint{8.590788in}{4.106509in}}%
\pgfpathcurveto{\pgfqpoint{8.590788in}{4.095459in}}{\pgfqpoint{8.595178in}{4.084860in}}{\pgfqpoint{8.602992in}{4.077046in}}%
\pgfpathcurveto{\pgfqpoint{8.610805in}{4.069233in}}{\pgfqpoint{8.621404in}{4.064843in}}{\pgfqpoint{8.632454in}{4.064843in}}%
\pgfpathlineto{\pgfqpoint{8.632454in}{4.064843in}}%
\pgfpathclose%
\pgfusepath{stroke,fill}%
\end{pgfscope}%
\begin{pgfscope}%
\pgfpathrectangle{\pgfqpoint{7.622482in}{2.920818in}}{\pgfqpoint{2.177280in}{2.201755in}}%
\pgfusepath{clip}%
\pgfsetbuttcap%
\pgfsetroundjoin%
\definecolor{currentfill}{rgb}{1.000000,0.498039,0.054902}%
\pgfsetfillcolor{currentfill}%
\pgfsetlinewidth{0.481800pt}%
\definecolor{currentstroke}{rgb}{1.000000,1.000000,1.000000}%
\pgfsetstrokecolor{currentstroke}%
\pgfsetdash{}{0pt}%
\pgfpathmoveto{\pgfqpoint{8.564700in}{4.064843in}}%
\pgfpathcurveto{\pgfqpoint{8.575751in}{4.064843in}}{\pgfqpoint{8.586350in}{4.069233in}}{\pgfqpoint{8.594163in}{4.077046in}}%
\pgfpathcurveto{\pgfqpoint{8.601977in}{4.084860in}}{\pgfqpoint{8.606367in}{4.095459in}}{\pgfqpoint{8.606367in}{4.106509in}}%
\pgfpathcurveto{\pgfqpoint{8.606367in}{4.117559in}}{\pgfqpoint{8.601977in}{4.128158in}}{\pgfqpoint{8.594163in}{4.135972in}}%
\pgfpathcurveto{\pgfqpoint{8.586350in}{4.143786in}}{\pgfqpoint{8.575751in}{4.148176in}}{\pgfqpoint{8.564700in}{4.148176in}}%
\pgfpathcurveto{\pgfqpoint{8.553650in}{4.148176in}}{\pgfqpoint{8.543051in}{4.143786in}}{\pgfqpoint{8.535238in}{4.135972in}}%
\pgfpathcurveto{\pgfqpoint{8.527424in}{4.128158in}}{\pgfqpoint{8.523034in}{4.117559in}}{\pgfqpoint{8.523034in}{4.106509in}}%
\pgfpathcurveto{\pgfqpoint{8.523034in}{4.095459in}}{\pgfqpoint{8.527424in}{4.084860in}}{\pgfqpoint{8.535238in}{4.077046in}}%
\pgfpathcurveto{\pgfqpoint{8.543051in}{4.069233in}}{\pgfqpoint{8.553650in}{4.064843in}}{\pgfqpoint{8.564700in}{4.064843in}}%
\pgfpathlineto{\pgfqpoint{8.564700in}{4.064843in}}%
\pgfpathclose%
\pgfusepath{stroke,fill}%
\end{pgfscope}%
\begin{pgfscope}%
\pgfpathrectangle{\pgfqpoint{7.622482in}{2.920818in}}{\pgfqpoint{2.177280in}{2.201755in}}%
\pgfusepath{clip}%
\pgfsetbuttcap%
\pgfsetroundjoin%
\definecolor{currentfill}{rgb}{1.000000,0.498039,0.054902}%
\pgfsetfillcolor{currentfill}%
\pgfsetlinewidth{0.481800pt}%
\definecolor{currentstroke}{rgb}{1.000000,1.000000,1.000000}%
\pgfsetstrokecolor{currentstroke}%
\pgfsetdash{}{0pt}%
\pgfpathmoveto{\pgfqpoint{8.632454in}{4.064843in}}%
\pgfpathcurveto{\pgfqpoint{8.643504in}{4.064843in}}{\pgfqpoint{8.654104in}{4.069233in}}{\pgfqpoint{8.661917in}{4.077046in}}%
\pgfpathcurveto{\pgfqpoint{8.669731in}{4.084860in}}{\pgfqpoint{8.674121in}{4.095459in}}{\pgfqpoint{8.674121in}{4.106509in}}%
\pgfpathcurveto{\pgfqpoint{8.674121in}{4.117559in}}{\pgfqpoint{8.669731in}{4.128158in}}{\pgfqpoint{8.661917in}{4.135972in}}%
\pgfpathcurveto{\pgfqpoint{8.654104in}{4.143786in}}{\pgfqpoint{8.643504in}{4.148176in}}{\pgfqpoint{8.632454in}{4.148176in}}%
\pgfpathcurveto{\pgfqpoint{8.621404in}{4.148176in}}{\pgfqpoint{8.610805in}{4.143786in}}{\pgfqpoint{8.602992in}{4.135972in}}%
\pgfpathcurveto{\pgfqpoint{8.595178in}{4.128158in}}{\pgfqpoint{8.590788in}{4.117559in}}{\pgfqpoint{8.590788in}{4.106509in}}%
\pgfpathcurveto{\pgfqpoint{8.590788in}{4.095459in}}{\pgfqpoint{8.595178in}{4.084860in}}{\pgfqpoint{8.602992in}{4.077046in}}%
\pgfpathcurveto{\pgfqpoint{8.610805in}{4.069233in}}{\pgfqpoint{8.621404in}{4.064843in}}{\pgfqpoint{8.632454in}{4.064843in}}%
\pgfpathlineto{\pgfqpoint{8.632454in}{4.064843in}}%
\pgfpathclose%
\pgfusepath{stroke,fill}%
\end{pgfscope}%
\begin{pgfscope}%
\pgfpathrectangle{\pgfqpoint{7.622482in}{2.920818in}}{\pgfqpoint{2.177280in}{2.201755in}}%
\pgfusepath{clip}%
\pgfsetbuttcap%
\pgfsetroundjoin%
\definecolor{currentfill}{rgb}{1.000000,0.498039,0.054902}%
\pgfsetfillcolor{currentfill}%
\pgfsetlinewidth{0.481800pt}%
\definecolor{currentstroke}{rgb}{1.000000,1.000000,1.000000}%
\pgfsetstrokecolor{currentstroke}%
\pgfsetdash{}{0pt}%
\pgfpathmoveto{\pgfqpoint{8.632454in}{4.098768in}}%
\pgfpathcurveto{\pgfqpoint{8.643504in}{4.098768in}}{\pgfqpoint{8.654104in}{4.103158in}}{\pgfqpoint{8.661917in}{4.110972in}}%
\pgfpathcurveto{\pgfqpoint{8.669731in}{4.118785in}}{\pgfqpoint{8.674121in}{4.129384in}}{\pgfqpoint{8.674121in}{4.140435in}}%
\pgfpathcurveto{\pgfqpoint{8.674121in}{4.151485in}}{\pgfqpoint{8.669731in}{4.162084in}}{\pgfqpoint{8.661917in}{4.169897in}}%
\pgfpathcurveto{\pgfqpoint{8.654104in}{4.177711in}}{\pgfqpoint{8.643504in}{4.182101in}}{\pgfqpoint{8.632454in}{4.182101in}}%
\pgfpathcurveto{\pgfqpoint{8.621404in}{4.182101in}}{\pgfqpoint{8.610805in}{4.177711in}}{\pgfqpoint{8.602992in}{4.169897in}}%
\pgfpathcurveto{\pgfqpoint{8.595178in}{4.162084in}}{\pgfqpoint{8.590788in}{4.151485in}}{\pgfqpoint{8.590788in}{4.140435in}}%
\pgfpathcurveto{\pgfqpoint{8.590788in}{4.129384in}}{\pgfqpoint{8.595178in}{4.118785in}}{\pgfqpoint{8.602992in}{4.110972in}}%
\pgfpathcurveto{\pgfqpoint{8.610805in}{4.103158in}}{\pgfqpoint{8.621404in}{4.098768in}}{\pgfqpoint{8.632454in}{4.098768in}}%
\pgfpathlineto{\pgfqpoint{8.632454in}{4.098768in}}%
\pgfpathclose%
\pgfusepath{stroke,fill}%
\end{pgfscope}%
\begin{pgfscope}%
\pgfpathrectangle{\pgfqpoint{7.622482in}{2.920818in}}{\pgfqpoint{2.177280in}{2.201755in}}%
\pgfusepath{clip}%
\pgfsetbuttcap%
\pgfsetroundjoin%
\definecolor{currentfill}{rgb}{1.000000,0.498039,0.054902}%
\pgfsetfillcolor{currentfill}%
\pgfsetlinewidth{0.481800pt}%
\definecolor{currentstroke}{rgb}{1.000000,1.000000,1.000000}%
\pgfsetstrokecolor{currentstroke}%
\pgfsetdash{}{0pt}%
\pgfpathmoveto{\pgfqpoint{8.496947in}{3.657738in}}%
\pgfpathcurveto{\pgfqpoint{8.507997in}{3.657738in}}{\pgfqpoint{8.518596in}{3.662129in}}{\pgfqpoint{8.526409in}{3.669942in}}%
\pgfpathcurveto{\pgfqpoint{8.534223in}{3.677756in}}{\pgfqpoint{8.538613in}{3.688355in}}{\pgfqpoint{8.538613in}{3.699405in}}%
\pgfpathcurveto{\pgfqpoint{8.538613in}{3.710455in}}{\pgfqpoint{8.534223in}{3.721054in}}{\pgfqpoint{8.526409in}{3.728868in}}%
\pgfpathcurveto{\pgfqpoint{8.518596in}{3.736681in}}{\pgfqpoint{8.507997in}{3.741072in}}{\pgfqpoint{8.496947in}{3.741072in}}%
\pgfpathcurveto{\pgfqpoint{8.485896in}{3.741072in}}{\pgfqpoint{8.475297in}{3.736681in}}{\pgfqpoint{8.467484in}{3.728868in}}%
\pgfpathcurveto{\pgfqpoint{8.459670in}{3.721054in}}{\pgfqpoint{8.455280in}{3.710455in}}{\pgfqpoint{8.455280in}{3.699405in}}%
\pgfpathcurveto{\pgfqpoint{8.455280in}{3.688355in}}{\pgfqpoint{8.459670in}{3.677756in}}{\pgfqpoint{8.467484in}{3.669942in}}%
\pgfpathcurveto{\pgfqpoint{8.475297in}{3.662129in}}{\pgfqpoint{8.485896in}{3.657738in}}{\pgfqpoint{8.496947in}{3.657738in}}%
\pgfpathlineto{\pgfqpoint{8.496947in}{3.657738in}}%
\pgfpathclose%
\pgfusepath{stroke,fill}%
\end{pgfscope}%
\begin{pgfscope}%
\pgfpathrectangle{\pgfqpoint{7.622482in}{2.920818in}}{\pgfqpoint{2.177280in}{2.201755in}}%
\pgfusepath{clip}%
\pgfsetbuttcap%
\pgfsetroundjoin%
\definecolor{currentfill}{rgb}{1.000000,0.498039,0.054902}%
\pgfsetfillcolor{currentfill}%
\pgfsetlinewidth{0.481800pt}%
\definecolor{currentstroke}{rgb}{1.000000,1.000000,1.000000}%
\pgfsetstrokecolor{currentstroke}%
\pgfsetdash{}{0pt}%
\pgfpathmoveto{\pgfqpoint{8.632454in}{4.030917in}}%
\pgfpathcurveto{\pgfqpoint{8.643504in}{4.030917in}}{\pgfqpoint{8.654104in}{4.035307in}}{\pgfqpoint{8.661917in}{4.043121in}}%
\pgfpathcurveto{\pgfqpoint{8.669731in}{4.050935in}}{\pgfqpoint{8.674121in}{4.061534in}}{\pgfqpoint{8.674121in}{4.072584in}}%
\pgfpathcurveto{\pgfqpoint{8.674121in}{4.083634in}}{\pgfqpoint{8.669731in}{4.094233in}}{\pgfqpoint{8.661917in}{4.102047in}}%
\pgfpathcurveto{\pgfqpoint{8.654104in}{4.109860in}}{\pgfqpoint{8.643504in}{4.114251in}}{\pgfqpoint{8.632454in}{4.114251in}}%
\pgfpathcurveto{\pgfqpoint{8.621404in}{4.114251in}}{\pgfqpoint{8.610805in}{4.109860in}}{\pgfqpoint{8.602992in}{4.102047in}}%
\pgfpathcurveto{\pgfqpoint{8.595178in}{4.094233in}}{\pgfqpoint{8.590788in}{4.083634in}}{\pgfqpoint{8.590788in}{4.072584in}}%
\pgfpathcurveto{\pgfqpoint{8.590788in}{4.061534in}}{\pgfqpoint{8.595178in}{4.050935in}}{\pgfqpoint{8.602992in}{4.043121in}}%
\pgfpathcurveto{\pgfqpoint{8.610805in}{4.035307in}}{\pgfqpoint{8.621404in}{4.030917in}}{\pgfqpoint{8.632454in}{4.030917in}}%
\pgfpathlineto{\pgfqpoint{8.632454in}{4.030917in}}%
\pgfpathclose%
\pgfusepath{stroke,fill}%
\end{pgfscope}%
\begin{pgfscope}%
\pgfpathrectangle{\pgfqpoint{7.622482in}{2.920818in}}{\pgfqpoint{2.177280in}{2.201755in}}%
\pgfusepath{clip}%
\pgfsetbuttcap%
\pgfsetroundjoin%
\definecolor{currentfill}{rgb}{0.172549,0.627451,0.172549}%
\pgfsetfillcolor{currentfill}%
\pgfsetlinewidth{0.481800pt}%
\definecolor{currentstroke}{rgb}{1.000000,1.000000,1.000000}%
\pgfsetstrokecolor{currentstroke}%
\pgfsetdash{}{0pt}%
\pgfpathmoveto{\pgfqpoint{9.445501in}{4.675499in}}%
\pgfpathcurveto{\pgfqpoint{9.456551in}{4.675499in}}{\pgfqpoint{9.467150in}{4.679889in}}{\pgfqpoint{9.474964in}{4.687703in}}%
\pgfpathcurveto{\pgfqpoint{9.482777in}{4.695516in}}{\pgfqpoint{9.487167in}{4.706115in}}{\pgfqpoint{9.487167in}{4.717165in}}%
\pgfpathcurveto{\pgfqpoint{9.487167in}{4.728216in}}{\pgfqpoint{9.482777in}{4.738815in}}{\pgfqpoint{9.474964in}{4.746628in}}%
\pgfpathcurveto{\pgfqpoint{9.467150in}{4.754442in}}{\pgfqpoint{9.456551in}{4.758832in}}{\pgfqpoint{9.445501in}{4.758832in}}%
\pgfpathcurveto{\pgfqpoint{9.434451in}{4.758832in}}{\pgfqpoint{9.423852in}{4.754442in}}{\pgfqpoint{9.416038in}{4.746628in}}%
\pgfpathcurveto{\pgfqpoint{9.408224in}{4.738815in}}{\pgfqpoint{9.403834in}{4.728216in}}{\pgfqpoint{9.403834in}{4.717165in}}%
\pgfpathcurveto{\pgfqpoint{9.403834in}{4.706115in}}{\pgfqpoint{9.408224in}{4.695516in}}{\pgfqpoint{9.416038in}{4.687703in}}%
\pgfpathcurveto{\pgfqpoint{9.423852in}{4.679889in}}{\pgfqpoint{9.434451in}{4.675499in}}{\pgfqpoint{9.445501in}{4.675499in}}%
\pgfpathlineto{\pgfqpoint{9.445501in}{4.675499in}}%
\pgfpathclose%
\pgfusepath{stroke,fill}%
\end{pgfscope}%
\begin{pgfscope}%
\pgfpathrectangle{\pgfqpoint{7.622482in}{2.920818in}}{\pgfqpoint{2.177280in}{2.201755in}}%
\pgfusepath{clip}%
\pgfsetbuttcap%
\pgfsetroundjoin%
\definecolor{currentfill}{rgb}{0.172549,0.627451,0.172549}%
\pgfsetfillcolor{currentfill}%
\pgfsetlinewidth{0.481800pt}%
\definecolor{currentstroke}{rgb}{1.000000,1.000000,1.000000}%
\pgfsetstrokecolor{currentstroke}%
\pgfsetdash{}{0pt}%
\pgfpathmoveto{\pgfqpoint{9.038978in}{4.370171in}}%
\pgfpathcurveto{\pgfqpoint{9.050028in}{4.370171in}}{\pgfqpoint{9.060627in}{4.374561in}}{\pgfqpoint{9.068440in}{4.382375in}}%
\pgfpathcurveto{\pgfqpoint{9.076254in}{4.390188in}}{\pgfqpoint{9.080644in}{4.400787in}}{\pgfqpoint{9.080644in}{4.411837in}}%
\pgfpathcurveto{\pgfqpoint{9.080644in}{4.422887in}}{\pgfqpoint{9.076254in}{4.433486in}}{\pgfqpoint{9.068440in}{4.441300in}}%
\pgfpathcurveto{\pgfqpoint{9.060627in}{4.449114in}}{\pgfqpoint{9.050028in}{4.453504in}}{\pgfqpoint{9.038978in}{4.453504in}}%
\pgfpathcurveto{\pgfqpoint{9.027927in}{4.453504in}}{\pgfqpoint{9.017328in}{4.449114in}}{\pgfqpoint{9.009515in}{4.441300in}}%
\pgfpathcurveto{\pgfqpoint{9.001701in}{4.433486in}}{\pgfqpoint{8.997311in}{4.422887in}}{\pgfqpoint{8.997311in}{4.411837in}}%
\pgfpathcurveto{\pgfqpoint{8.997311in}{4.400787in}}{\pgfqpoint{9.001701in}{4.390188in}}{\pgfqpoint{9.009515in}{4.382375in}}%
\pgfpathcurveto{\pgfqpoint{9.017328in}{4.374561in}}{\pgfqpoint{9.027927in}{4.370171in}}{\pgfqpoint{9.038978in}{4.370171in}}%
\pgfpathlineto{\pgfqpoint{9.038978in}{4.370171in}}%
\pgfpathclose%
\pgfusepath{stroke,fill}%
\end{pgfscope}%
\begin{pgfscope}%
\pgfpathrectangle{\pgfqpoint{7.622482in}{2.920818in}}{\pgfqpoint{2.177280in}{2.201755in}}%
\pgfusepath{clip}%
\pgfsetbuttcap%
\pgfsetroundjoin%
\definecolor{currentfill}{rgb}{0.172549,0.627451,0.172549}%
\pgfsetfillcolor{currentfill}%
\pgfsetlinewidth{0.481800pt}%
\definecolor{currentstroke}{rgb}{1.000000,1.000000,1.000000}%
\pgfsetstrokecolor{currentstroke}%
\pgfsetdash{}{0pt}%
\pgfpathmoveto{\pgfqpoint{9.174485in}{4.641573in}}%
\pgfpathcurveto{\pgfqpoint{9.185535in}{4.641573in}}{\pgfqpoint{9.196134in}{4.645964in}}{\pgfqpoint{9.203948in}{4.653777in}}%
\pgfpathcurveto{\pgfqpoint{9.211762in}{4.661591in}}{\pgfqpoint{9.216152in}{4.672190in}}{\pgfqpoint{9.216152in}{4.683240in}}%
\pgfpathcurveto{\pgfqpoint{9.216152in}{4.694290in}}{\pgfqpoint{9.211762in}{4.704889in}}{\pgfqpoint{9.203948in}{4.712703in}}%
\pgfpathcurveto{\pgfqpoint{9.196134in}{4.720517in}}{\pgfqpoint{9.185535in}{4.724907in}}{\pgfqpoint{9.174485in}{4.724907in}}%
\pgfpathcurveto{\pgfqpoint{9.163435in}{4.724907in}}{\pgfqpoint{9.152836in}{4.720517in}}{\pgfqpoint{9.145023in}{4.712703in}}%
\pgfpathcurveto{\pgfqpoint{9.137209in}{4.704889in}}{\pgfqpoint{9.132819in}{4.694290in}}{\pgfqpoint{9.132819in}{4.683240in}}%
\pgfpathcurveto{\pgfqpoint{9.132819in}{4.672190in}}{\pgfqpoint{9.137209in}{4.661591in}}{\pgfqpoint{9.145023in}{4.653777in}}%
\pgfpathcurveto{\pgfqpoint{9.152836in}{4.645964in}}{\pgfqpoint{9.163435in}{4.641573in}}{\pgfqpoint{9.174485in}{4.641573in}}%
\pgfpathlineto{\pgfqpoint{9.174485in}{4.641573in}}%
\pgfpathclose%
\pgfusepath{stroke,fill}%
\end{pgfscope}%
\begin{pgfscope}%
\pgfpathrectangle{\pgfqpoint{7.622482in}{2.920818in}}{\pgfqpoint{2.177280in}{2.201755in}}%
\pgfusepath{clip}%
\pgfsetbuttcap%
\pgfsetroundjoin%
\definecolor{currentfill}{rgb}{0.172549,0.627451,0.172549}%
\pgfsetfillcolor{currentfill}%
\pgfsetlinewidth{0.481800pt}%
\definecolor{currentstroke}{rgb}{1.000000,1.000000,1.000000}%
\pgfsetstrokecolor{currentstroke}%
\pgfsetdash{}{0pt}%
\pgfpathmoveto{\pgfqpoint{8.971224in}{4.539797in}}%
\pgfpathcurveto{\pgfqpoint{8.982274in}{4.539797in}}{\pgfqpoint{8.992873in}{4.544188in}}{\pgfqpoint{9.000686in}{4.552001in}}%
\pgfpathcurveto{\pgfqpoint{9.008500in}{4.559815in}}{\pgfqpoint{9.012890in}{4.570414in}}{\pgfqpoint{9.012890in}{4.581464in}}%
\pgfpathcurveto{\pgfqpoint{9.012890in}{4.592514in}}{\pgfqpoint{9.008500in}{4.603113in}}{\pgfqpoint{9.000686in}{4.610927in}}%
\pgfpathcurveto{\pgfqpoint{8.992873in}{4.618740in}}{\pgfqpoint{8.982274in}{4.623131in}}{\pgfqpoint{8.971224in}{4.623131in}}%
\pgfpathcurveto{\pgfqpoint{8.960174in}{4.623131in}}{\pgfqpoint{8.949575in}{4.618740in}}{\pgfqpoint{8.941761in}{4.610927in}}%
\pgfpathcurveto{\pgfqpoint{8.933947in}{4.603113in}}{\pgfqpoint{8.929557in}{4.592514in}}{\pgfqpoint{8.929557in}{4.581464in}}%
\pgfpathcurveto{\pgfqpoint{8.929557in}{4.570414in}}{\pgfqpoint{8.933947in}{4.559815in}}{\pgfqpoint{8.941761in}{4.552001in}}%
\pgfpathcurveto{\pgfqpoint{8.949575in}{4.544188in}}{\pgfqpoint{8.960174in}{4.539797in}}{\pgfqpoint{8.971224in}{4.539797in}}%
\pgfpathlineto{\pgfqpoint{8.971224in}{4.539797in}}%
\pgfpathclose%
\pgfusepath{stroke,fill}%
\end{pgfscope}%
\begin{pgfscope}%
\pgfpathrectangle{\pgfqpoint{7.622482in}{2.920818in}}{\pgfqpoint{2.177280in}{2.201755in}}%
\pgfusepath{clip}%
\pgfsetbuttcap%
\pgfsetroundjoin%
\definecolor{currentfill}{rgb}{0.172549,0.627451,0.172549}%
\pgfsetfillcolor{currentfill}%
\pgfsetlinewidth{0.481800pt}%
\definecolor{currentstroke}{rgb}{1.000000,1.000000,1.000000}%
\pgfsetstrokecolor{currentstroke}%
\pgfsetdash{}{0pt}%
\pgfpathmoveto{\pgfqpoint{9.242239in}{4.607648in}}%
\pgfpathcurveto{\pgfqpoint{9.253289in}{4.607648in}}{\pgfqpoint{9.263888in}{4.612038in}}{\pgfqpoint{9.271702in}{4.619852in}}%
\pgfpathcurveto{\pgfqpoint{9.279516in}{4.627666in}}{\pgfqpoint{9.283906in}{4.638265in}}{\pgfqpoint{9.283906in}{4.649315in}}%
\pgfpathcurveto{\pgfqpoint{9.283906in}{4.660365in}}{\pgfqpoint{9.279516in}{4.670964in}}{\pgfqpoint{9.271702in}{4.678778in}}%
\pgfpathcurveto{\pgfqpoint{9.263888in}{4.686591in}}{\pgfqpoint{9.253289in}{4.690981in}}{\pgfqpoint{9.242239in}{4.690981in}}%
\pgfpathcurveto{\pgfqpoint{9.231189in}{4.690981in}}{\pgfqpoint{9.220590in}{4.686591in}}{\pgfqpoint{9.212776in}{4.678778in}}%
\pgfpathcurveto{\pgfqpoint{9.204963in}{4.670964in}}{\pgfqpoint{9.200573in}{4.660365in}}{\pgfqpoint{9.200573in}{4.649315in}}%
\pgfpathcurveto{\pgfqpoint{9.200573in}{4.638265in}}{\pgfqpoint{9.204963in}{4.627666in}}{\pgfqpoint{9.212776in}{4.619852in}}%
\pgfpathcurveto{\pgfqpoint{9.220590in}{4.612038in}}{\pgfqpoint{9.231189in}{4.607648in}}{\pgfqpoint{9.242239in}{4.607648in}}%
\pgfpathlineto{\pgfqpoint{9.242239in}{4.607648in}}%
\pgfpathclose%
\pgfusepath{stroke,fill}%
\end{pgfscope}%
\begin{pgfscope}%
\pgfpathrectangle{\pgfqpoint{7.622482in}{2.920818in}}{\pgfqpoint{2.177280in}{2.201755in}}%
\pgfusepath{clip}%
\pgfsetbuttcap%
\pgfsetroundjoin%
\definecolor{currentfill}{rgb}{0.172549,0.627451,0.172549}%
\pgfsetfillcolor{currentfill}%
\pgfsetlinewidth{0.481800pt}%
\definecolor{currentstroke}{rgb}{1.000000,1.000000,1.000000}%
\pgfsetstrokecolor{currentstroke}%
\pgfsetdash{}{0pt}%
\pgfpathmoveto{\pgfqpoint{9.174485in}{4.879051in}}%
\pgfpathcurveto{\pgfqpoint{9.185535in}{4.879051in}}{\pgfqpoint{9.196134in}{4.883441in}}{\pgfqpoint{9.203948in}{4.891255in}}%
\pgfpathcurveto{\pgfqpoint{9.211762in}{4.899068in}}{\pgfqpoint{9.216152in}{4.909667in}}{\pgfqpoint{9.216152in}{4.920718in}}%
\pgfpathcurveto{\pgfqpoint{9.216152in}{4.931768in}}{\pgfqpoint{9.211762in}{4.942367in}}{\pgfqpoint{9.203948in}{4.950180in}}%
\pgfpathcurveto{\pgfqpoint{9.196134in}{4.957994in}}{\pgfqpoint{9.185535in}{4.962384in}}{\pgfqpoint{9.174485in}{4.962384in}}%
\pgfpathcurveto{\pgfqpoint{9.163435in}{4.962384in}}{\pgfqpoint{9.152836in}{4.957994in}}{\pgfqpoint{9.145023in}{4.950180in}}%
\pgfpathcurveto{\pgfqpoint{9.137209in}{4.942367in}}{\pgfqpoint{9.132819in}{4.931768in}}{\pgfqpoint{9.132819in}{4.920718in}}%
\pgfpathcurveto{\pgfqpoint{9.132819in}{4.909667in}}{\pgfqpoint{9.137209in}{4.899068in}}{\pgfqpoint{9.145023in}{4.891255in}}%
\pgfpathcurveto{\pgfqpoint{9.152836in}{4.883441in}}{\pgfqpoint{9.163435in}{4.879051in}}{\pgfqpoint{9.174485in}{4.879051in}}%
\pgfpathlineto{\pgfqpoint{9.174485in}{4.879051in}}%
\pgfpathclose%
\pgfusepath{stroke,fill}%
\end{pgfscope}%
\begin{pgfscope}%
\pgfpathrectangle{\pgfqpoint{7.622482in}{2.920818in}}{\pgfqpoint{2.177280in}{2.201755in}}%
\pgfusepath{clip}%
\pgfsetbuttcap%
\pgfsetroundjoin%
\definecolor{currentfill}{rgb}{0.172549,0.627451,0.172549}%
\pgfsetfillcolor{currentfill}%
\pgfsetlinewidth{0.481800pt}%
\definecolor{currentstroke}{rgb}{1.000000,1.000000,1.000000}%
\pgfsetstrokecolor{currentstroke}%
\pgfsetdash{}{0pt}%
\pgfpathmoveto{\pgfqpoint{8.903470in}{4.166619in}}%
\pgfpathcurveto{\pgfqpoint{8.914520in}{4.166619in}}{\pgfqpoint{8.925119in}{4.171009in}}{\pgfqpoint{8.932933in}{4.178822in}}%
\pgfpathcurveto{\pgfqpoint{8.940746in}{4.186636in}}{\pgfqpoint{8.945136in}{4.197235in}}{\pgfqpoint{8.945136in}{4.208285in}}%
\pgfpathcurveto{\pgfqpoint{8.945136in}{4.219335in}}{\pgfqpoint{8.940746in}{4.229934in}}{\pgfqpoint{8.932933in}{4.237748in}}%
\pgfpathcurveto{\pgfqpoint{8.925119in}{4.245562in}}{\pgfqpoint{8.914520in}{4.249952in}}{\pgfqpoint{8.903470in}{4.249952in}}%
\pgfpathcurveto{\pgfqpoint{8.892420in}{4.249952in}}{\pgfqpoint{8.881821in}{4.245562in}}{\pgfqpoint{8.874007in}{4.237748in}}%
\pgfpathcurveto{\pgfqpoint{8.866193in}{4.229934in}}{\pgfqpoint{8.861803in}{4.219335in}}{\pgfqpoint{8.861803in}{4.208285in}}%
\pgfpathcurveto{\pgfqpoint{8.861803in}{4.197235in}}{\pgfqpoint{8.866193in}{4.186636in}}{\pgfqpoint{8.874007in}{4.178822in}}%
\pgfpathcurveto{\pgfqpoint{8.881821in}{4.171009in}}{\pgfqpoint{8.892420in}{4.166619in}}{\pgfqpoint{8.903470in}{4.166619in}}%
\pgfpathlineto{\pgfqpoint{8.903470in}{4.166619in}}%
\pgfpathclose%
\pgfusepath{stroke,fill}%
\end{pgfscope}%
\begin{pgfscope}%
\pgfpathrectangle{\pgfqpoint{7.622482in}{2.920818in}}{\pgfqpoint{2.177280in}{2.201755in}}%
\pgfusepath{clip}%
\pgfsetbuttcap%
\pgfsetroundjoin%
\definecolor{currentfill}{rgb}{0.172549,0.627451,0.172549}%
\pgfsetfillcolor{currentfill}%
\pgfsetlinewidth{0.481800pt}%
\definecolor{currentstroke}{rgb}{1.000000,1.000000,1.000000}%
\pgfsetstrokecolor{currentstroke}%
\pgfsetdash{}{0pt}%
\pgfpathmoveto{\pgfqpoint{8.971224in}{4.777275in}}%
\pgfpathcurveto{\pgfqpoint{8.982274in}{4.777275in}}{\pgfqpoint{8.992873in}{4.781665in}}{\pgfqpoint{9.000686in}{4.789479in}}%
\pgfpathcurveto{\pgfqpoint{9.008500in}{4.797292in}}{\pgfqpoint{9.012890in}{4.807891in}}{\pgfqpoint{9.012890in}{4.818942in}}%
\pgfpathcurveto{\pgfqpoint{9.012890in}{4.829992in}}{\pgfqpoint{9.008500in}{4.840591in}}{\pgfqpoint{9.000686in}{4.848404in}}%
\pgfpathcurveto{\pgfqpoint{8.992873in}{4.856218in}}{\pgfqpoint{8.982274in}{4.860608in}}{\pgfqpoint{8.971224in}{4.860608in}}%
\pgfpathcurveto{\pgfqpoint{8.960174in}{4.860608in}}{\pgfqpoint{8.949575in}{4.856218in}}{\pgfqpoint{8.941761in}{4.848404in}}%
\pgfpathcurveto{\pgfqpoint{8.933947in}{4.840591in}}{\pgfqpoint{8.929557in}{4.829992in}}{\pgfqpoint{8.929557in}{4.818942in}}%
\pgfpathcurveto{\pgfqpoint{8.929557in}{4.807891in}}{\pgfqpoint{8.933947in}{4.797292in}}{\pgfqpoint{8.941761in}{4.789479in}}%
\pgfpathcurveto{\pgfqpoint{8.949575in}{4.781665in}}{\pgfqpoint{8.960174in}{4.777275in}}{\pgfqpoint{8.971224in}{4.777275in}}%
\pgfpathlineto{\pgfqpoint{8.971224in}{4.777275in}}%
\pgfpathclose%
\pgfusepath{stroke,fill}%
\end{pgfscope}%
\begin{pgfscope}%
\pgfpathrectangle{\pgfqpoint{7.622482in}{2.920818in}}{\pgfqpoint{2.177280in}{2.201755in}}%
\pgfusepath{clip}%
\pgfsetbuttcap%
\pgfsetroundjoin%
\definecolor{currentfill}{rgb}{0.172549,0.627451,0.172549}%
\pgfsetfillcolor{currentfill}%
\pgfsetlinewidth{0.481800pt}%
\definecolor{currentstroke}{rgb}{1.000000,1.000000,1.000000}%
\pgfsetstrokecolor{currentstroke}%
\pgfsetdash{}{0pt}%
\pgfpathmoveto{\pgfqpoint{8.971224in}{4.607648in}}%
\pgfpathcurveto{\pgfqpoint{8.982274in}{4.607648in}}{\pgfqpoint{8.992873in}{4.612038in}}{\pgfqpoint{9.000686in}{4.619852in}}%
\pgfpathcurveto{\pgfqpoint{9.008500in}{4.627666in}}{\pgfqpoint{9.012890in}{4.638265in}}{\pgfqpoint{9.012890in}{4.649315in}}%
\pgfpathcurveto{\pgfqpoint{9.012890in}{4.660365in}}{\pgfqpoint{9.008500in}{4.670964in}}{\pgfqpoint{9.000686in}{4.678778in}}%
\pgfpathcurveto{\pgfqpoint{8.992873in}{4.686591in}}{\pgfqpoint{8.982274in}{4.690981in}}{\pgfqpoint{8.971224in}{4.690981in}}%
\pgfpathcurveto{\pgfqpoint{8.960174in}{4.690981in}}{\pgfqpoint{8.949575in}{4.686591in}}{\pgfqpoint{8.941761in}{4.678778in}}%
\pgfpathcurveto{\pgfqpoint{8.933947in}{4.670964in}}{\pgfqpoint{8.929557in}{4.660365in}}{\pgfqpoint{8.929557in}{4.649315in}}%
\pgfpathcurveto{\pgfqpoint{8.929557in}{4.638265in}}{\pgfqpoint{8.933947in}{4.627666in}}{\pgfqpoint{8.941761in}{4.619852in}}%
\pgfpathcurveto{\pgfqpoint{8.949575in}{4.612038in}}{\pgfqpoint{8.960174in}{4.607648in}}{\pgfqpoint{8.971224in}{4.607648in}}%
\pgfpathlineto{\pgfqpoint{8.971224in}{4.607648in}}%
\pgfpathclose%
\pgfusepath{stroke,fill}%
\end{pgfscope}%
\begin{pgfscope}%
\pgfpathrectangle{\pgfqpoint{7.622482in}{2.920818in}}{\pgfqpoint{2.177280in}{2.201755in}}%
\pgfusepath{clip}%
\pgfsetbuttcap%
\pgfsetroundjoin%
\definecolor{currentfill}{rgb}{0.172549,0.627451,0.172549}%
\pgfsetfillcolor{currentfill}%
\pgfsetlinewidth{0.481800pt}%
\definecolor{currentstroke}{rgb}{1.000000,1.000000,1.000000}%
\pgfsetstrokecolor{currentstroke}%
\pgfsetdash{}{0pt}%
\pgfpathmoveto{\pgfqpoint{9.445501in}{4.709424in}}%
\pgfpathcurveto{\pgfqpoint{9.456551in}{4.709424in}}{\pgfqpoint{9.467150in}{4.713814in}}{\pgfqpoint{9.474964in}{4.721628in}}%
\pgfpathcurveto{\pgfqpoint{9.482777in}{4.729442in}}{\pgfqpoint{9.487167in}{4.740041in}}{\pgfqpoint{9.487167in}{4.751091in}}%
\pgfpathcurveto{\pgfqpoint{9.487167in}{4.762141in}}{\pgfqpoint{9.482777in}{4.772740in}}{\pgfqpoint{9.474964in}{4.780554in}}%
\pgfpathcurveto{\pgfqpoint{9.467150in}{4.788367in}}{\pgfqpoint{9.456551in}{4.792757in}}{\pgfqpoint{9.445501in}{4.792757in}}%
\pgfpathcurveto{\pgfqpoint{9.434451in}{4.792757in}}{\pgfqpoint{9.423852in}{4.788367in}}{\pgfqpoint{9.416038in}{4.780554in}}%
\pgfpathcurveto{\pgfqpoint{9.408224in}{4.772740in}}{\pgfqpoint{9.403834in}{4.762141in}}{\pgfqpoint{9.403834in}{4.751091in}}%
\pgfpathcurveto{\pgfqpoint{9.403834in}{4.740041in}}{\pgfqpoint{9.408224in}{4.729442in}}{\pgfqpoint{9.416038in}{4.721628in}}%
\pgfpathcurveto{\pgfqpoint{9.423852in}{4.713814in}}{\pgfqpoint{9.434451in}{4.709424in}}{\pgfqpoint{9.445501in}{4.709424in}}%
\pgfpathlineto{\pgfqpoint{9.445501in}{4.709424in}}%
\pgfpathclose%
\pgfusepath{stroke,fill}%
\end{pgfscope}%
\begin{pgfscope}%
\pgfpathrectangle{\pgfqpoint{7.622482in}{2.920818in}}{\pgfqpoint{2.177280in}{2.201755in}}%
\pgfusepath{clip}%
\pgfsetbuttcap%
\pgfsetroundjoin%
\definecolor{currentfill}{rgb}{0.172549,0.627451,0.172549}%
\pgfsetfillcolor{currentfill}%
\pgfsetlinewidth{0.481800pt}%
\definecolor{currentstroke}{rgb}{1.000000,1.000000,1.000000}%
\pgfsetstrokecolor{currentstroke}%
\pgfsetdash{}{0pt}%
\pgfpathmoveto{\pgfqpoint{9.106731in}{4.370171in}}%
\pgfpathcurveto{\pgfqpoint{9.117782in}{4.370171in}}{\pgfqpoint{9.128381in}{4.374561in}}{\pgfqpoint{9.136194in}{4.382375in}}%
\pgfpathcurveto{\pgfqpoint{9.144008in}{4.390188in}}{\pgfqpoint{9.148398in}{4.400787in}}{\pgfqpoint{9.148398in}{4.411837in}}%
\pgfpathcurveto{\pgfqpoint{9.148398in}{4.422887in}}{\pgfqpoint{9.144008in}{4.433486in}}{\pgfqpoint{9.136194in}{4.441300in}}%
\pgfpathcurveto{\pgfqpoint{9.128381in}{4.449114in}}{\pgfqpoint{9.117782in}{4.453504in}}{\pgfqpoint{9.106731in}{4.453504in}}%
\pgfpathcurveto{\pgfqpoint{9.095681in}{4.453504in}}{\pgfqpoint{9.085082in}{4.449114in}}{\pgfqpoint{9.077269in}{4.441300in}}%
\pgfpathcurveto{\pgfqpoint{9.069455in}{4.433486in}}{\pgfqpoint{9.065065in}{4.422887in}}{\pgfqpoint{9.065065in}{4.411837in}}%
\pgfpathcurveto{\pgfqpoint{9.065065in}{4.400787in}}{\pgfqpoint{9.069455in}{4.390188in}}{\pgfqpoint{9.077269in}{4.382375in}}%
\pgfpathcurveto{\pgfqpoint{9.085082in}{4.374561in}}{\pgfqpoint{9.095681in}{4.370171in}}{\pgfqpoint{9.106731in}{4.370171in}}%
\pgfpathlineto{\pgfqpoint{9.106731in}{4.370171in}}%
\pgfpathclose%
\pgfusepath{stroke,fill}%
\end{pgfscope}%
\begin{pgfscope}%
\pgfpathrectangle{\pgfqpoint{7.622482in}{2.920818in}}{\pgfqpoint{2.177280in}{2.201755in}}%
\pgfusepath{clip}%
\pgfsetbuttcap%
\pgfsetroundjoin%
\definecolor{currentfill}{rgb}{0.172549,0.627451,0.172549}%
\pgfsetfillcolor{currentfill}%
\pgfsetlinewidth{0.481800pt}%
\definecolor{currentstroke}{rgb}{1.000000,1.000000,1.000000}%
\pgfsetstrokecolor{currentstroke}%
\pgfsetdash{}{0pt}%
\pgfpathmoveto{\pgfqpoint{9.038978in}{4.438021in}}%
\pgfpathcurveto{\pgfqpoint{9.050028in}{4.438021in}}{\pgfqpoint{9.060627in}{4.442412in}}{\pgfqpoint{9.068440in}{4.450225in}}%
\pgfpathcurveto{\pgfqpoint{9.076254in}{4.458039in}}{\pgfqpoint{9.080644in}{4.468638in}}{\pgfqpoint{9.080644in}{4.479688in}}%
\pgfpathcurveto{\pgfqpoint{9.080644in}{4.490738in}}{\pgfqpoint{9.076254in}{4.501337in}}{\pgfqpoint{9.068440in}{4.509151in}}%
\pgfpathcurveto{\pgfqpoint{9.060627in}{4.516964in}}{\pgfqpoint{9.050028in}{4.521355in}}{\pgfqpoint{9.038978in}{4.521355in}}%
\pgfpathcurveto{\pgfqpoint{9.027927in}{4.521355in}}{\pgfqpoint{9.017328in}{4.516964in}}{\pgfqpoint{9.009515in}{4.509151in}}%
\pgfpathcurveto{\pgfqpoint{9.001701in}{4.501337in}}{\pgfqpoint{8.997311in}{4.490738in}}{\pgfqpoint{8.997311in}{4.479688in}}%
\pgfpathcurveto{\pgfqpoint{8.997311in}{4.468638in}}{\pgfqpoint{9.001701in}{4.458039in}}{\pgfqpoint{9.009515in}{4.450225in}}%
\pgfpathcurveto{\pgfqpoint{9.017328in}{4.442412in}}{\pgfqpoint{9.027927in}{4.438021in}}{\pgfqpoint{9.038978in}{4.438021in}}%
\pgfpathlineto{\pgfqpoint{9.038978in}{4.438021in}}%
\pgfpathclose%
\pgfusepath{stroke,fill}%
\end{pgfscope}%
\begin{pgfscope}%
\pgfpathrectangle{\pgfqpoint{7.622482in}{2.920818in}}{\pgfqpoint{2.177280in}{2.201755in}}%
\pgfusepath{clip}%
\pgfsetbuttcap%
\pgfsetroundjoin%
\definecolor{currentfill}{rgb}{0.172549,0.627451,0.172549}%
\pgfsetfillcolor{currentfill}%
\pgfsetlinewidth{0.481800pt}%
\definecolor{currentstroke}{rgb}{1.000000,1.000000,1.000000}%
\pgfsetstrokecolor{currentstroke}%
\pgfsetdash{}{0pt}%
\pgfpathmoveto{\pgfqpoint{9.174485in}{4.505872in}}%
\pgfpathcurveto{\pgfqpoint{9.185535in}{4.505872in}}{\pgfqpoint{9.196134in}{4.510262in}}{\pgfqpoint{9.203948in}{4.518076in}}%
\pgfpathcurveto{\pgfqpoint{9.211762in}{4.525890in}}{\pgfqpoint{9.216152in}{4.536489in}}{\pgfqpoint{9.216152in}{4.547539in}}%
\pgfpathcurveto{\pgfqpoint{9.216152in}{4.558589in}}{\pgfqpoint{9.211762in}{4.569188in}}{\pgfqpoint{9.203948in}{4.577002in}}%
\pgfpathcurveto{\pgfqpoint{9.196134in}{4.584815in}}{\pgfqpoint{9.185535in}{4.589205in}}{\pgfqpoint{9.174485in}{4.589205in}}%
\pgfpathcurveto{\pgfqpoint{9.163435in}{4.589205in}}{\pgfqpoint{9.152836in}{4.584815in}}{\pgfqpoint{9.145023in}{4.577002in}}%
\pgfpathcurveto{\pgfqpoint{9.137209in}{4.569188in}}{\pgfqpoint{9.132819in}{4.558589in}}{\pgfqpoint{9.132819in}{4.547539in}}%
\pgfpathcurveto{\pgfqpoint{9.132819in}{4.536489in}}{\pgfqpoint{9.137209in}{4.525890in}}{\pgfqpoint{9.145023in}{4.518076in}}%
\pgfpathcurveto{\pgfqpoint{9.152836in}{4.510262in}}{\pgfqpoint{9.163435in}{4.505872in}}{\pgfqpoint{9.174485in}{4.505872in}}%
\pgfpathlineto{\pgfqpoint{9.174485in}{4.505872in}}%
\pgfpathclose%
\pgfusepath{stroke,fill}%
\end{pgfscope}%
\begin{pgfscope}%
\pgfpathrectangle{\pgfqpoint{7.622482in}{2.920818in}}{\pgfqpoint{2.177280in}{2.201755in}}%
\pgfusepath{clip}%
\pgfsetbuttcap%
\pgfsetroundjoin%
\definecolor{currentfill}{rgb}{0.172549,0.627451,0.172549}%
\pgfsetfillcolor{currentfill}%
\pgfsetlinewidth{0.481800pt}%
\definecolor{currentstroke}{rgb}{1.000000,1.000000,1.000000}%
\pgfsetstrokecolor{currentstroke}%
\pgfsetdash{}{0pt}%
\pgfpathmoveto{\pgfqpoint{9.106731in}{4.336245in}}%
\pgfpathcurveto{\pgfqpoint{9.117782in}{4.336245in}}{\pgfqpoint{9.128381in}{4.340636in}}{\pgfqpoint{9.136194in}{4.348449in}}%
\pgfpathcurveto{\pgfqpoint{9.144008in}{4.356263in}}{\pgfqpoint{9.148398in}{4.366862in}}{\pgfqpoint{9.148398in}{4.377912in}}%
\pgfpathcurveto{\pgfqpoint{9.148398in}{4.388962in}}{\pgfqpoint{9.144008in}{4.399561in}}{\pgfqpoint{9.136194in}{4.407375in}}%
\pgfpathcurveto{\pgfqpoint{9.128381in}{4.415188in}}{\pgfqpoint{9.117782in}{4.419579in}}{\pgfqpoint{9.106731in}{4.419579in}}%
\pgfpathcurveto{\pgfqpoint{9.095681in}{4.419579in}}{\pgfqpoint{9.085082in}{4.415188in}}{\pgfqpoint{9.077269in}{4.407375in}}%
\pgfpathcurveto{\pgfqpoint{9.069455in}{4.399561in}}{\pgfqpoint{9.065065in}{4.388962in}}{\pgfqpoint{9.065065in}{4.377912in}}%
\pgfpathcurveto{\pgfqpoint{9.065065in}{4.366862in}}{\pgfqpoint{9.069455in}{4.356263in}}{\pgfqpoint{9.077269in}{4.348449in}}%
\pgfpathcurveto{\pgfqpoint{9.085082in}{4.340636in}}{\pgfqpoint{9.095681in}{4.336245in}}{\pgfqpoint{9.106731in}{4.336245in}}%
\pgfpathlineto{\pgfqpoint{9.106731in}{4.336245in}}%
\pgfpathclose%
\pgfusepath{stroke,fill}%
\end{pgfscope}%
\begin{pgfscope}%
\pgfpathrectangle{\pgfqpoint{7.622482in}{2.920818in}}{\pgfqpoint{2.177280in}{2.201755in}}%
\pgfusepath{clip}%
\pgfsetbuttcap%
\pgfsetroundjoin%
\definecolor{currentfill}{rgb}{0.172549,0.627451,0.172549}%
\pgfsetfillcolor{currentfill}%
\pgfsetlinewidth{0.481800pt}%
\definecolor{currentstroke}{rgb}{1.000000,1.000000,1.000000}%
\pgfsetstrokecolor{currentstroke}%
\pgfsetdash{}{0pt}%
\pgfpathmoveto{\pgfqpoint{9.377747in}{4.370171in}}%
\pgfpathcurveto{\pgfqpoint{9.388797in}{4.370171in}}{\pgfqpoint{9.399396in}{4.374561in}}{\pgfqpoint{9.407210in}{4.382375in}}%
\pgfpathcurveto{\pgfqpoint{9.415023in}{4.390188in}}{\pgfqpoint{9.419414in}{4.400787in}}{\pgfqpoint{9.419414in}{4.411837in}}%
\pgfpathcurveto{\pgfqpoint{9.419414in}{4.422887in}}{\pgfqpoint{9.415023in}{4.433486in}}{\pgfqpoint{9.407210in}{4.441300in}}%
\pgfpathcurveto{\pgfqpoint{9.399396in}{4.449114in}}{\pgfqpoint{9.388797in}{4.453504in}}{\pgfqpoint{9.377747in}{4.453504in}}%
\pgfpathcurveto{\pgfqpoint{9.366697in}{4.453504in}}{\pgfqpoint{9.356098in}{4.449114in}}{\pgfqpoint{9.348284in}{4.441300in}}%
\pgfpathcurveto{\pgfqpoint{9.340471in}{4.433486in}}{\pgfqpoint{9.336080in}{4.422887in}}{\pgfqpoint{9.336080in}{4.411837in}}%
\pgfpathcurveto{\pgfqpoint{9.336080in}{4.400787in}}{\pgfqpoint{9.340471in}{4.390188in}}{\pgfqpoint{9.348284in}{4.382375in}}%
\pgfpathcurveto{\pgfqpoint{9.356098in}{4.374561in}}{\pgfqpoint{9.366697in}{4.370171in}}{\pgfqpoint{9.377747in}{4.370171in}}%
\pgfpathlineto{\pgfqpoint{9.377747in}{4.370171in}}%
\pgfpathclose%
\pgfusepath{stroke,fill}%
\end{pgfscope}%
\begin{pgfscope}%
\pgfpathrectangle{\pgfqpoint{7.622482in}{2.920818in}}{\pgfqpoint{2.177280in}{2.201755in}}%
\pgfusepath{clip}%
\pgfsetbuttcap%
\pgfsetroundjoin%
\definecolor{currentfill}{rgb}{0.172549,0.627451,0.172549}%
\pgfsetfillcolor{currentfill}%
\pgfsetlinewidth{0.481800pt}%
\definecolor{currentstroke}{rgb}{1.000000,1.000000,1.000000}%
\pgfsetstrokecolor{currentstroke}%
\pgfsetdash{}{0pt}%
\pgfpathmoveto{\pgfqpoint{9.309993in}{4.438021in}}%
\pgfpathcurveto{\pgfqpoint{9.321043in}{4.438021in}}{\pgfqpoint{9.331642in}{4.442412in}}{\pgfqpoint{9.339456in}{4.450225in}}%
\pgfpathcurveto{\pgfqpoint{9.347269in}{4.458039in}}{\pgfqpoint{9.351660in}{4.468638in}}{\pgfqpoint{9.351660in}{4.479688in}}%
\pgfpathcurveto{\pgfqpoint{9.351660in}{4.490738in}}{\pgfqpoint{9.347269in}{4.501337in}}{\pgfqpoint{9.339456in}{4.509151in}}%
\pgfpathcurveto{\pgfqpoint{9.331642in}{4.516964in}}{\pgfqpoint{9.321043in}{4.521355in}}{\pgfqpoint{9.309993in}{4.521355in}}%
\pgfpathcurveto{\pgfqpoint{9.298943in}{4.521355in}}{\pgfqpoint{9.288344in}{4.516964in}}{\pgfqpoint{9.280530in}{4.509151in}}%
\pgfpathcurveto{\pgfqpoint{9.272717in}{4.501337in}}{\pgfqpoint{9.268326in}{4.490738in}}{\pgfqpoint{9.268326in}{4.479688in}}%
\pgfpathcurveto{\pgfqpoint{9.268326in}{4.468638in}}{\pgfqpoint{9.272717in}{4.458039in}}{\pgfqpoint{9.280530in}{4.450225in}}%
\pgfpathcurveto{\pgfqpoint{9.288344in}{4.442412in}}{\pgfqpoint{9.298943in}{4.438021in}}{\pgfqpoint{9.309993in}{4.438021in}}%
\pgfpathlineto{\pgfqpoint{9.309993in}{4.438021in}}%
\pgfpathclose%
\pgfusepath{stroke,fill}%
\end{pgfscope}%
\begin{pgfscope}%
\pgfpathrectangle{\pgfqpoint{7.622482in}{2.920818in}}{\pgfqpoint{2.177280in}{2.201755in}}%
\pgfusepath{clip}%
\pgfsetbuttcap%
\pgfsetroundjoin%
\definecolor{currentfill}{rgb}{0.172549,0.627451,0.172549}%
\pgfsetfillcolor{currentfill}%
\pgfsetlinewidth{0.481800pt}%
\definecolor{currentstroke}{rgb}{1.000000,1.000000,1.000000}%
\pgfsetstrokecolor{currentstroke}%
\pgfsetdash{}{0pt}%
\pgfpathmoveto{\pgfqpoint{8.971224in}{4.505872in}}%
\pgfpathcurveto{\pgfqpoint{8.982274in}{4.505872in}}{\pgfqpoint{8.992873in}{4.510262in}}{\pgfqpoint{9.000686in}{4.518076in}}%
\pgfpathcurveto{\pgfqpoint{9.008500in}{4.525890in}}{\pgfqpoint{9.012890in}{4.536489in}}{\pgfqpoint{9.012890in}{4.547539in}}%
\pgfpathcurveto{\pgfqpoint{9.012890in}{4.558589in}}{\pgfqpoint{9.008500in}{4.569188in}}{\pgfqpoint{9.000686in}{4.577002in}}%
\pgfpathcurveto{\pgfqpoint{8.992873in}{4.584815in}}{\pgfqpoint{8.982274in}{4.589205in}}{\pgfqpoint{8.971224in}{4.589205in}}%
\pgfpathcurveto{\pgfqpoint{8.960174in}{4.589205in}}{\pgfqpoint{8.949575in}{4.584815in}}{\pgfqpoint{8.941761in}{4.577002in}}%
\pgfpathcurveto{\pgfqpoint{8.933947in}{4.569188in}}{\pgfqpoint{8.929557in}{4.558589in}}{\pgfqpoint{8.929557in}{4.547539in}}%
\pgfpathcurveto{\pgfqpoint{8.929557in}{4.536489in}}{\pgfqpoint{8.933947in}{4.525890in}}{\pgfqpoint{8.941761in}{4.518076in}}%
\pgfpathcurveto{\pgfqpoint{8.949575in}{4.510262in}}{\pgfqpoint{8.960174in}{4.505872in}}{\pgfqpoint{8.971224in}{4.505872in}}%
\pgfpathlineto{\pgfqpoint{8.971224in}{4.505872in}}%
\pgfpathclose%
\pgfusepath{stroke,fill}%
\end{pgfscope}%
\begin{pgfscope}%
\pgfpathrectangle{\pgfqpoint{7.622482in}{2.920818in}}{\pgfqpoint{2.177280in}{2.201755in}}%
\pgfusepath{clip}%
\pgfsetbuttcap%
\pgfsetroundjoin%
\definecolor{currentfill}{rgb}{0.172549,0.627451,0.172549}%
\pgfsetfillcolor{currentfill}%
\pgfsetlinewidth{0.481800pt}%
\definecolor{currentstroke}{rgb}{1.000000,1.000000,1.000000}%
\pgfsetstrokecolor{currentstroke}%
\pgfsetdash{}{0pt}%
\pgfpathmoveto{\pgfqpoint{9.242239in}{4.912976in}}%
\pgfpathcurveto{\pgfqpoint{9.253289in}{4.912976in}}{\pgfqpoint{9.263888in}{4.917366in}}{\pgfqpoint{9.271702in}{4.925180in}}%
\pgfpathcurveto{\pgfqpoint{9.279516in}{4.932994in}}{\pgfqpoint{9.283906in}{4.943593in}}{\pgfqpoint{9.283906in}{4.954643in}}%
\pgfpathcurveto{\pgfqpoint{9.283906in}{4.965693in}}{\pgfqpoint{9.279516in}{4.976292in}}{\pgfqpoint{9.271702in}{4.984106in}}%
\pgfpathcurveto{\pgfqpoint{9.263888in}{4.991919in}}{\pgfqpoint{9.253289in}{4.996310in}}{\pgfqpoint{9.242239in}{4.996310in}}%
\pgfpathcurveto{\pgfqpoint{9.231189in}{4.996310in}}{\pgfqpoint{9.220590in}{4.991919in}}{\pgfqpoint{9.212776in}{4.984106in}}%
\pgfpathcurveto{\pgfqpoint{9.204963in}{4.976292in}}{\pgfqpoint{9.200573in}{4.965693in}}{\pgfqpoint{9.200573in}{4.954643in}}%
\pgfpathcurveto{\pgfqpoint{9.200573in}{4.943593in}}{\pgfqpoint{9.204963in}{4.932994in}}{\pgfqpoint{9.212776in}{4.925180in}}%
\pgfpathcurveto{\pgfqpoint{9.220590in}{4.917366in}}{\pgfqpoint{9.231189in}{4.912976in}}{\pgfqpoint{9.242239in}{4.912976in}}%
\pgfpathlineto{\pgfqpoint{9.242239in}{4.912976in}}%
\pgfpathclose%
\pgfusepath{stroke,fill}%
\end{pgfscope}%
\begin{pgfscope}%
\pgfpathrectangle{\pgfqpoint{7.622482in}{2.920818in}}{\pgfqpoint{2.177280in}{2.201755in}}%
\pgfusepath{clip}%
\pgfsetbuttcap%
\pgfsetroundjoin%
\definecolor{currentfill}{rgb}{0.172549,0.627451,0.172549}%
\pgfsetfillcolor{currentfill}%
\pgfsetlinewidth{0.481800pt}%
\definecolor{currentstroke}{rgb}{1.000000,1.000000,1.000000}%
\pgfsetstrokecolor{currentstroke}%
\pgfsetdash{}{0pt}%
\pgfpathmoveto{\pgfqpoint{9.309993in}{4.980827in}}%
\pgfpathcurveto{\pgfqpoint{9.321043in}{4.980827in}}{\pgfqpoint{9.331642in}{4.985217in}}{\pgfqpoint{9.339456in}{4.993031in}}%
\pgfpathcurveto{\pgfqpoint{9.347269in}{5.000844in}}{\pgfqpoint{9.351660in}{5.011443in}}{\pgfqpoint{9.351660in}{5.022494in}}%
\pgfpathcurveto{\pgfqpoint{9.351660in}{5.033544in}}{\pgfqpoint{9.347269in}{5.044143in}}{\pgfqpoint{9.339456in}{5.051956in}}%
\pgfpathcurveto{\pgfqpoint{9.331642in}{5.059770in}}{\pgfqpoint{9.321043in}{5.064160in}}{\pgfqpoint{9.309993in}{5.064160in}}%
\pgfpathcurveto{\pgfqpoint{9.298943in}{5.064160in}}{\pgfqpoint{9.288344in}{5.059770in}}{\pgfqpoint{9.280530in}{5.051956in}}%
\pgfpathcurveto{\pgfqpoint{9.272717in}{5.044143in}}{\pgfqpoint{9.268326in}{5.033544in}}{\pgfqpoint{9.268326in}{5.022494in}}%
\pgfpathcurveto{\pgfqpoint{9.268326in}{5.011443in}}{\pgfqpoint{9.272717in}{5.000844in}}{\pgfqpoint{9.280530in}{4.993031in}}%
\pgfpathcurveto{\pgfqpoint{9.288344in}{4.985217in}}{\pgfqpoint{9.298943in}{4.980827in}}{\pgfqpoint{9.309993in}{4.980827in}}%
\pgfpathlineto{\pgfqpoint{9.309993in}{4.980827in}}%
\pgfpathclose%
\pgfusepath{stroke,fill}%
\end{pgfscope}%
\begin{pgfscope}%
\pgfpathrectangle{\pgfqpoint{7.622482in}{2.920818in}}{\pgfqpoint{2.177280in}{2.201755in}}%
\pgfusepath{clip}%
\pgfsetbuttcap%
\pgfsetroundjoin%
\definecolor{currentfill}{rgb}{0.172549,0.627451,0.172549}%
\pgfsetfillcolor{currentfill}%
\pgfsetlinewidth{0.481800pt}%
\definecolor{currentstroke}{rgb}{1.000000,1.000000,1.000000}%
\pgfsetstrokecolor{currentstroke}%
\pgfsetdash{}{0pt}%
\pgfpathmoveto{\pgfqpoint{8.767962in}{4.336245in}}%
\pgfpathcurveto{\pgfqpoint{8.779012in}{4.336245in}}{\pgfqpoint{8.789611in}{4.340636in}}{\pgfqpoint{8.797425in}{4.348449in}}%
\pgfpathcurveto{\pgfqpoint{8.805238in}{4.356263in}}{\pgfqpoint{8.809629in}{4.366862in}}{\pgfqpoint{8.809629in}{4.377912in}}%
\pgfpathcurveto{\pgfqpoint{8.809629in}{4.388962in}}{\pgfqpoint{8.805238in}{4.399561in}}{\pgfqpoint{8.797425in}{4.407375in}}%
\pgfpathcurveto{\pgfqpoint{8.789611in}{4.415188in}}{\pgfqpoint{8.779012in}{4.419579in}}{\pgfqpoint{8.767962in}{4.419579in}}%
\pgfpathcurveto{\pgfqpoint{8.756912in}{4.419579in}}{\pgfqpoint{8.746313in}{4.415188in}}{\pgfqpoint{8.738499in}{4.407375in}}%
\pgfpathcurveto{\pgfqpoint{8.730686in}{4.399561in}}{\pgfqpoint{8.726295in}{4.388962in}}{\pgfqpoint{8.726295in}{4.377912in}}%
\pgfpathcurveto{\pgfqpoint{8.726295in}{4.366862in}}{\pgfqpoint{8.730686in}{4.356263in}}{\pgfqpoint{8.738499in}{4.348449in}}%
\pgfpathcurveto{\pgfqpoint{8.746313in}{4.340636in}}{\pgfqpoint{8.756912in}{4.336245in}}{\pgfqpoint{8.767962in}{4.336245in}}%
\pgfpathlineto{\pgfqpoint{8.767962in}{4.336245in}}%
\pgfpathclose%
\pgfusepath{stroke,fill}%
\end{pgfscope}%
\begin{pgfscope}%
\pgfpathrectangle{\pgfqpoint{7.622482in}{2.920818in}}{\pgfqpoint{2.177280in}{2.201755in}}%
\pgfusepath{clip}%
\pgfsetbuttcap%
\pgfsetroundjoin%
\definecolor{currentfill}{rgb}{0.172549,0.627451,0.172549}%
\pgfsetfillcolor{currentfill}%
\pgfsetlinewidth{0.481800pt}%
\definecolor{currentstroke}{rgb}{1.000000,1.000000,1.000000}%
\pgfsetstrokecolor{currentstroke}%
\pgfsetdash{}{0pt}%
\pgfpathmoveto{\pgfqpoint{9.309993in}{4.573723in}}%
\pgfpathcurveto{\pgfqpoint{9.321043in}{4.573723in}}{\pgfqpoint{9.331642in}{4.578113in}}{\pgfqpoint{9.339456in}{4.585927in}}%
\pgfpathcurveto{\pgfqpoint{9.347269in}{4.593740in}}{\pgfqpoint{9.351660in}{4.604339in}}{\pgfqpoint{9.351660in}{4.615389in}}%
\pgfpathcurveto{\pgfqpoint{9.351660in}{4.626440in}}{\pgfqpoint{9.347269in}{4.637039in}}{\pgfqpoint{9.339456in}{4.644852in}}%
\pgfpathcurveto{\pgfqpoint{9.331642in}{4.652666in}}{\pgfqpoint{9.321043in}{4.657056in}}{\pgfqpoint{9.309993in}{4.657056in}}%
\pgfpathcurveto{\pgfqpoint{9.298943in}{4.657056in}}{\pgfqpoint{9.288344in}{4.652666in}}{\pgfqpoint{9.280530in}{4.644852in}}%
\pgfpathcurveto{\pgfqpoint{9.272717in}{4.637039in}}{\pgfqpoint{9.268326in}{4.626440in}}{\pgfqpoint{9.268326in}{4.615389in}}%
\pgfpathcurveto{\pgfqpoint{9.268326in}{4.604339in}}{\pgfqpoint{9.272717in}{4.593740in}}{\pgfqpoint{9.280530in}{4.585927in}}%
\pgfpathcurveto{\pgfqpoint{9.288344in}{4.578113in}}{\pgfqpoint{9.298943in}{4.573723in}}{\pgfqpoint{9.309993in}{4.573723in}}%
\pgfpathlineto{\pgfqpoint{9.309993in}{4.573723in}}%
\pgfpathclose%
\pgfusepath{stroke,fill}%
\end{pgfscope}%
\begin{pgfscope}%
\pgfpathrectangle{\pgfqpoint{7.622482in}{2.920818in}}{\pgfqpoint{2.177280in}{2.201755in}}%
\pgfusepath{clip}%
\pgfsetbuttcap%
\pgfsetroundjoin%
\definecolor{currentfill}{rgb}{0.172549,0.627451,0.172549}%
\pgfsetfillcolor{currentfill}%
\pgfsetlinewidth{0.481800pt}%
\definecolor{currentstroke}{rgb}{1.000000,1.000000,1.000000}%
\pgfsetstrokecolor{currentstroke}%
\pgfsetdash{}{0pt}%
\pgfpathmoveto{\pgfqpoint{9.106731in}{4.302320in}}%
\pgfpathcurveto{\pgfqpoint{9.117782in}{4.302320in}}{\pgfqpoint{9.128381in}{4.306710in}}{\pgfqpoint{9.136194in}{4.314524in}}%
\pgfpathcurveto{\pgfqpoint{9.144008in}{4.322337in}}{\pgfqpoint{9.148398in}{4.332937in}}{\pgfqpoint{9.148398in}{4.343987in}}%
\pgfpathcurveto{\pgfqpoint{9.148398in}{4.355037in}}{\pgfqpoint{9.144008in}{4.365636in}}{\pgfqpoint{9.136194in}{4.373449in}}%
\pgfpathcurveto{\pgfqpoint{9.128381in}{4.381263in}}{\pgfqpoint{9.117782in}{4.385653in}}{\pgfqpoint{9.106731in}{4.385653in}}%
\pgfpathcurveto{\pgfqpoint{9.095681in}{4.385653in}}{\pgfqpoint{9.085082in}{4.381263in}}{\pgfqpoint{9.077269in}{4.373449in}}%
\pgfpathcurveto{\pgfqpoint{9.069455in}{4.365636in}}{\pgfqpoint{9.065065in}{4.355037in}}{\pgfqpoint{9.065065in}{4.343987in}}%
\pgfpathcurveto{\pgfqpoint{9.065065in}{4.332937in}}{\pgfqpoint{9.069455in}{4.322337in}}{\pgfqpoint{9.077269in}{4.314524in}}%
\pgfpathcurveto{\pgfqpoint{9.085082in}{4.306710in}}{\pgfqpoint{9.095681in}{4.302320in}}{\pgfqpoint{9.106731in}{4.302320in}}%
\pgfpathlineto{\pgfqpoint{9.106731in}{4.302320in}}%
\pgfpathclose%
\pgfusepath{stroke,fill}%
\end{pgfscope}%
\begin{pgfscope}%
\pgfpathrectangle{\pgfqpoint{7.622482in}{2.920818in}}{\pgfqpoint{2.177280in}{2.201755in}}%
\pgfusepath{clip}%
\pgfsetbuttcap%
\pgfsetroundjoin%
\definecolor{currentfill}{rgb}{0.172549,0.627451,0.172549}%
\pgfsetfillcolor{currentfill}%
\pgfsetlinewidth{0.481800pt}%
\definecolor{currentstroke}{rgb}{1.000000,1.000000,1.000000}%
\pgfsetstrokecolor{currentstroke}%
\pgfsetdash{}{0pt}%
\pgfpathmoveto{\pgfqpoint{9.106731in}{4.912976in}}%
\pgfpathcurveto{\pgfqpoint{9.117782in}{4.912976in}}{\pgfqpoint{9.128381in}{4.917366in}}{\pgfqpoint{9.136194in}{4.925180in}}%
\pgfpathcurveto{\pgfqpoint{9.144008in}{4.932994in}}{\pgfqpoint{9.148398in}{4.943593in}}{\pgfqpoint{9.148398in}{4.954643in}}%
\pgfpathcurveto{\pgfqpoint{9.148398in}{4.965693in}}{\pgfqpoint{9.144008in}{4.976292in}}{\pgfqpoint{9.136194in}{4.984106in}}%
\pgfpathcurveto{\pgfqpoint{9.128381in}{4.991919in}}{\pgfqpoint{9.117782in}{4.996310in}}{\pgfqpoint{9.106731in}{4.996310in}}%
\pgfpathcurveto{\pgfqpoint{9.095681in}{4.996310in}}{\pgfqpoint{9.085082in}{4.991919in}}{\pgfqpoint{9.077269in}{4.984106in}}%
\pgfpathcurveto{\pgfqpoint{9.069455in}{4.976292in}}{\pgfqpoint{9.065065in}{4.965693in}}{\pgfqpoint{9.065065in}{4.954643in}}%
\pgfpathcurveto{\pgfqpoint{9.065065in}{4.943593in}}{\pgfqpoint{9.069455in}{4.932994in}}{\pgfqpoint{9.077269in}{4.925180in}}%
\pgfpathcurveto{\pgfqpoint{9.085082in}{4.917366in}}{\pgfqpoint{9.095681in}{4.912976in}}{\pgfqpoint{9.106731in}{4.912976in}}%
\pgfpathlineto{\pgfqpoint{9.106731in}{4.912976in}}%
\pgfpathclose%
\pgfusepath{stroke,fill}%
\end{pgfscope}%
\begin{pgfscope}%
\pgfpathrectangle{\pgfqpoint{7.622482in}{2.920818in}}{\pgfqpoint{2.177280in}{2.201755in}}%
\pgfusepath{clip}%
\pgfsetbuttcap%
\pgfsetroundjoin%
\definecolor{currentfill}{rgb}{0.172549,0.627451,0.172549}%
\pgfsetfillcolor{currentfill}%
\pgfsetlinewidth{0.481800pt}%
\definecolor{currentstroke}{rgb}{1.000000,1.000000,1.000000}%
\pgfsetstrokecolor{currentstroke}%
\pgfsetdash{}{0pt}%
\pgfpathmoveto{\pgfqpoint{8.971224in}{4.302320in}}%
\pgfpathcurveto{\pgfqpoint{8.982274in}{4.302320in}}{\pgfqpoint{8.992873in}{4.306710in}}{\pgfqpoint{9.000686in}{4.314524in}}%
\pgfpathcurveto{\pgfqpoint{9.008500in}{4.322337in}}{\pgfqpoint{9.012890in}{4.332937in}}{\pgfqpoint{9.012890in}{4.343987in}}%
\pgfpathcurveto{\pgfqpoint{9.012890in}{4.355037in}}{\pgfqpoint{9.008500in}{4.365636in}}{\pgfqpoint{9.000686in}{4.373449in}}%
\pgfpathcurveto{\pgfqpoint{8.992873in}{4.381263in}}{\pgfqpoint{8.982274in}{4.385653in}}{\pgfqpoint{8.971224in}{4.385653in}}%
\pgfpathcurveto{\pgfqpoint{8.960174in}{4.385653in}}{\pgfqpoint{8.949575in}{4.381263in}}{\pgfqpoint{8.941761in}{4.373449in}}%
\pgfpathcurveto{\pgfqpoint{8.933947in}{4.365636in}}{\pgfqpoint{8.929557in}{4.355037in}}{\pgfqpoint{8.929557in}{4.343987in}}%
\pgfpathcurveto{\pgfqpoint{8.929557in}{4.332937in}}{\pgfqpoint{8.933947in}{4.322337in}}{\pgfqpoint{8.941761in}{4.314524in}}%
\pgfpathcurveto{\pgfqpoint{8.949575in}{4.306710in}}{\pgfqpoint{8.960174in}{4.302320in}}{\pgfqpoint{8.971224in}{4.302320in}}%
\pgfpathlineto{\pgfqpoint{8.971224in}{4.302320in}}%
\pgfpathclose%
\pgfusepath{stroke,fill}%
\end{pgfscope}%
\begin{pgfscope}%
\pgfpathrectangle{\pgfqpoint{7.622482in}{2.920818in}}{\pgfqpoint{2.177280in}{2.201755in}}%
\pgfusepath{clip}%
\pgfsetbuttcap%
\pgfsetroundjoin%
\definecolor{currentfill}{rgb}{0.172549,0.627451,0.172549}%
\pgfsetfillcolor{currentfill}%
\pgfsetlinewidth{0.481800pt}%
\definecolor{currentstroke}{rgb}{1.000000,1.000000,1.000000}%
\pgfsetstrokecolor{currentstroke}%
\pgfsetdash{}{0pt}%
\pgfpathmoveto{\pgfqpoint{9.174485in}{4.573723in}}%
\pgfpathcurveto{\pgfqpoint{9.185535in}{4.573723in}}{\pgfqpoint{9.196134in}{4.578113in}}{\pgfqpoint{9.203948in}{4.585927in}}%
\pgfpathcurveto{\pgfqpoint{9.211762in}{4.593740in}}{\pgfqpoint{9.216152in}{4.604339in}}{\pgfqpoint{9.216152in}{4.615389in}}%
\pgfpathcurveto{\pgfqpoint{9.216152in}{4.626440in}}{\pgfqpoint{9.211762in}{4.637039in}}{\pgfqpoint{9.203948in}{4.644852in}}%
\pgfpathcurveto{\pgfqpoint{9.196134in}{4.652666in}}{\pgfqpoint{9.185535in}{4.657056in}}{\pgfqpoint{9.174485in}{4.657056in}}%
\pgfpathcurveto{\pgfqpoint{9.163435in}{4.657056in}}{\pgfqpoint{9.152836in}{4.652666in}}{\pgfqpoint{9.145023in}{4.644852in}}%
\pgfpathcurveto{\pgfqpoint{9.137209in}{4.637039in}}{\pgfqpoint{9.132819in}{4.626440in}}{\pgfqpoint{9.132819in}{4.615389in}}%
\pgfpathcurveto{\pgfqpoint{9.132819in}{4.604339in}}{\pgfqpoint{9.137209in}{4.593740in}}{\pgfqpoint{9.145023in}{4.585927in}}%
\pgfpathcurveto{\pgfqpoint{9.152836in}{4.578113in}}{\pgfqpoint{9.163435in}{4.573723in}}{\pgfqpoint{9.174485in}{4.573723in}}%
\pgfpathlineto{\pgfqpoint{9.174485in}{4.573723in}}%
\pgfpathclose%
\pgfusepath{stroke,fill}%
\end{pgfscope}%
\begin{pgfscope}%
\pgfpathrectangle{\pgfqpoint{7.622482in}{2.920818in}}{\pgfqpoint{2.177280in}{2.201755in}}%
\pgfusepath{clip}%
\pgfsetbuttcap%
\pgfsetroundjoin%
\definecolor{currentfill}{rgb}{0.172549,0.627451,0.172549}%
\pgfsetfillcolor{currentfill}%
\pgfsetlinewidth{0.481800pt}%
\definecolor{currentstroke}{rgb}{1.000000,1.000000,1.000000}%
\pgfsetstrokecolor{currentstroke}%
\pgfsetdash{}{0pt}%
\pgfpathmoveto{\pgfqpoint{8.971224in}{4.675499in}}%
\pgfpathcurveto{\pgfqpoint{8.982274in}{4.675499in}}{\pgfqpoint{8.992873in}{4.679889in}}{\pgfqpoint{9.000686in}{4.687703in}}%
\pgfpathcurveto{\pgfqpoint{9.008500in}{4.695516in}}{\pgfqpoint{9.012890in}{4.706115in}}{\pgfqpoint{9.012890in}{4.717165in}}%
\pgfpathcurveto{\pgfqpoint{9.012890in}{4.728216in}}{\pgfqpoint{9.008500in}{4.738815in}}{\pgfqpoint{9.000686in}{4.746628in}}%
\pgfpathcurveto{\pgfqpoint{8.992873in}{4.754442in}}{\pgfqpoint{8.982274in}{4.758832in}}{\pgfqpoint{8.971224in}{4.758832in}}%
\pgfpathcurveto{\pgfqpoint{8.960174in}{4.758832in}}{\pgfqpoint{8.949575in}{4.754442in}}{\pgfqpoint{8.941761in}{4.746628in}}%
\pgfpathcurveto{\pgfqpoint{8.933947in}{4.738815in}}{\pgfqpoint{8.929557in}{4.728216in}}{\pgfqpoint{8.929557in}{4.717165in}}%
\pgfpathcurveto{\pgfqpoint{8.929557in}{4.706115in}}{\pgfqpoint{8.933947in}{4.695516in}}{\pgfqpoint{8.941761in}{4.687703in}}%
\pgfpathcurveto{\pgfqpoint{8.949575in}{4.679889in}}{\pgfqpoint{8.960174in}{4.675499in}}{\pgfqpoint{8.971224in}{4.675499in}}%
\pgfpathlineto{\pgfqpoint{8.971224in}{4.675499in}}%
\pgfpathclose%
\pgfusepath{stroke,fill}%
\end{pgfscope}%
\begin{pgfscope}%
\pgfpathrectangle{\pgfqpoint{7.622482in}{2.920818in}}{\pgfqpoint{2.177280in}{2.201755in}}%
\pgfusepath{clip}%
\pgfsetbuttcap%
\pgfsetroundjoin%
\definecolor{currentfill}{rgb}{0.172549,0.627451,0.172549}%
\pgfsetfillcolor{currentfill}%
\pgfsetlinewidth{0.481800pt}%
\definecolor{currentstroke}{rgb}{1.000000,1.000000,1.000000}%
\pgfsetstrokecolor{currentstroke}%
\pgfsetdash{}{0pt}%
\pgfpathmoveto{\pgfqpoint{8.971224in}{4.268395in}}%
\pgfpathcurveto{\pgfqpoint{8.982274in}{4.268395in}}{\pgfqpoint{8.992873in}{4.272785in}}{\pgfqpoint{9.000686in}{4.280599in}}%
\pgfpathcurveto{\pgfqpoint{9.008500in}{4.288412in}}{\pgfqpoint{9.012890in}{4.299011in}}{\pgfqpoint{9.012890in}{4.310061in}}%
\pgfpathcurveto{\pgfqpoint{9.012890in}{4.321111in}}{\pgfqpoint{9.008500in}{4.331710in}}{\pgfqpoint{9.000686in}{4.339524in}}%
\pgfpathcurveto{\pgfqpoint{8.992873in}{4.347338in}}{\pgfqpoint{8.982274in}{4.351728in}}{\pgfqpoint{8.971224in}{4.351728in}}%
\pgfpathcurveto{\pgfqpoint{8.960174in}{4.351728in}}{\pgfqpoint{8.949575in}{4.347338in}}{\pgfqpoint{8.941761in}{4.339524in}}%
\pgfpathcurveto{\pgfqpoint{8.933947in}{4.331710in}}{\pgfqpoint{8.929557in}{4.321111in}}{\pgfqpoint{8.929557in}{4.310061in}}%
\pgfpathcurveto{\pgfqpoint{8.929557in}{4.299011in}}{\pgfqpoint{8.933947in}{4.288412in}}{\pgfqpoint{8.941761in}{4.280599in}}%
\pgfpathcurveto{\pgfqpoint{8.949575in}{4.272785in}}{\pgfqpoint{8.960174in}{4.268395in}}{\pgfqpoint{8.971224in}{4.268395in}}%
\pgfpathlineto{\pgfqpoint{8.971224in}{4.268395in}}%
\pgfpathclose%
\pgfusepath{stroke,fill}%
\end{pgfscope}%
\begin{pgfscope}%
\pgfpathrectangle{\pgfqpoint{7.622482in}{2.920818in}}{\pgfqpoint{2.177280in}{2.201755in}}%
\pgfusepath{clip}%
\pgfsetbuttcap%
\pgfsetroundjoin%
\definecolor{currentfill}{rgb}{0.172549,0.627451,0.172549}%
\pgfsetfillcolor{currentfill}%
\pgfsetlinewidth{0.481800pt}%
\definecolor{currentstroke}{rgb}{1.000000,1.000000,1.000000}%
\pgfsetstrokecolor{currentstroke}%
\pgfsetdash{}{0pt}%
\pgfpathmoveto{\pgfqpoint{8.971224in}{4.302320in}}%
\pgfpathcurveto{\pgfqpoint{8.982274in}{4.302320in}}{\pgfqpoint{8.992873in}{4.306710in}}{\pgfqpoint{9.000686in}{4.314524in}}%
\pgfpathcurveto{\pgfqpoint{9.008500in}{4.322337in}}{\pgfqpoint{9.012890in}{4.332937in}}{\pgfqpoint{9.012890in}{4.343987in}}%
\pgfpathcurveto{\pgfqpoint{9.012890in}{4.355037in}}{\pgfqpoint{9.008500in}{4.365636in}}{\pgfqpoint{9.000686in}{4.373449in}}%
\pgfpathcurveto{\pgfqpoint{8.992873in}{4.381263in}}{\pgfqpoint{8.982274in}{4.385653in}}{\pgfqpoint{8.971224in}{4.385653in}}%
\pgfpathcurveto{\pgfqpoint{8.960174in}{4.385653in}}{\pgfqpoint{8.949575in}{4.381263in}}{\pgfqpoint{8.941761in}{4.373449in}}%
\pgfpathcurveto{\pgfqpoint{8.933947in}{4.365636in}}{\pgfqpoint{8.929557in}{4.355037in}}{\pgfqpoint{8.929557in}{4.343987in}}%
\pgfpathcurveto{\pgfqpoint{8.929557in}{4.332937in}}{\pgfqpoint{8.933947in}{4.322337in}}{\pgfqpoint{8.941761in}{4.314524in}}%
\pgfpathcurveto{\pgfqpoint{8.949575in}{4.306710in}}{\pgfqpoint{8.960174in}{4.302320in}}{\pgfqpoint{8.971224in}{4.302320in}}%
\pgfpathlineto{\pgfqpoint{8.971224in}{4.302320in}}%
\pgfpathclose%
\pgfusepath{stroke,fill}%
\end{pgfscope}%
\begin{pgfscope}%
\pgfpathrectangle{\pgfqpoint{7.622482in}{2.920818in}}{\pgfqpoint{2.177280in}{2.201755in}}%
\pgfusepath{clip}%
\pgfsetbuttcap%
\pgfsetroundjoin%
\definecolor{currentfill}{rgb}{0.172549,0.627451,0.172549}%
\pgfsetfillcolor{currentfill}%
\pgfsetlinewidth{0.481800pt}%
\definecolor{currentstroke}{rgb}{1.000000,1.000000,1.000000}%
\pgfsetstrokecolor{currentstroke}%
\pgfsetdash{}{0pt}%
\pgfpathmoveto{\pgfqpoint{9.174485in}{4.539797in}}%
\pgfpathcurveto{\pgfqpoint{9.185535in}{4.539797in}}{\pgfqpoint{9.196134in}{4.544188in}}{\pgfqpoint{9.203948in}{4.552001in}}%
\pgfpathcurveto{\pgfqpoint{9.211762in}{4.559815in}}{\pgfqpoint{9.216152in}{4.570414in}}{\pgfqpoint{9.216152in}{4.581464in}}%
\pgfpathcurveto{\pgfqpoint{9.216152in}{4.592514in}}{\pgfqpoint{9.211762in}{4.603113in}}{\pgfqpoint{9.203948in}{4.610927in}}%
\pgfpathcurveto{\pgfqpoint{9.196134in}{4.618740in}}{\pgfqpoint{9.185535in}{4.623131in}}{\pgfqpoint{9.174485in}{4.623131in}}%
\pgfpathcurveto{\pgfqpoint{9.163435in}{4.623131in}}{\pgfqpoint{9.152836in}{4.618740in}}{\pgfqpoint{9.145023in}{4.610927in}}%
\pgfpathcurveto{\pgfqpoint{9.137209in}{4.603113in}}{\pgfqpoint{9.132819in}{4.592514in}}{\pgfqpoint{9.132819in}{4.581464in}}%
\pgfpathcurveto{\pgfqpoint{9.132819in}{4.570414in}}{\pgfqpoint{9.137209in}{4.559815in}}{\pgfqpoint{9.145023in}{4.552001in}}%
\pgfpathcurveto{\pgfqpoint{9.152836in}{4.544188in}}{\pgfqpoint{9.163435in}{4.539797in}}{\pgfqpoint{9.174485in}{4.539797in}}%
\pgfpathlineto{\pgfqpoint{9.174485in}{4.539797in}}%
\pgfpathclose%
\pgfusepath{stroke,fill}%
\end{pgfscope}%
\begin{pgfscope}%
\pgfpathrectangle{\pgfqpoint{7.622482in}{2.920818in}}{\pgfqpoint{2.177280in}{2.201755in}}%
\pgfusepath{clip}%
\pgfsetbuttcap%
\pgfsetroundjoin%
\definecolor{currentfill}{rgb}{0.172549,0.627451,0.172549}%
\pgfsetfillcolor{currentfill}%
\pgfsetlinewidth{0.481800pt}%
\definecolor{currentstroke}{rgb}{1.000000,1.000000,1.000000}%
\pgfsetstrokecolor{currentstroke}%
\pgfsetdash{}{0pt}%
\pgfpathmoveto{\pgfqpoint{8.835716in}{4.607648in}}%
\pgfpathcurveto{\pgfqpoint{8.846766in}{4.607648in}}{\pgfqpoint{8.857365in}{4.612038in}}{\pgfqpoint{8.865179in}{4.619852in}}%
\pgfpathcurveto{\pgfqpoint{8.872992in}{4.627666in}}{\pgfqpoint{8.877383in}{4.638265in}}{\pgfqpoint{8.877383in}{4.649315in}}%
\pgfpathcurveto{\pgfqpoint{8.877383in}{4.660365in}}{\pgfqpoint{8.872992in}{4.670964in}}{\pgfqpoint{8.865179in}{4.678778in}}%
\pgfpathcurveto{\pgfqpoint{8.857365in}{4.686591in}}{\pgfqpoint{8.846766in}{4.690981in}}{\pgfqpoint{8.835716in}{4.690981in}}%
\pgfpathcurveto{\pgfqpoint{8.824666in}{4.690981in}}{\pgfqpoint{8.814067in}{4.686591in}}{\pgfqpoint{8.806253in}{4.678778in}}%
\pgfpathcurveto{\pgfqpoint{8.798440in}{4.670964in}}{\pgfqpoint{8.794049in}{4.660365in}}{\pgfqpoint{8.794049in}{4.649315in}}%
\pgfpathcurveto{\pgfqpoint{8.794049in}{4.638265in}}{\pgfqpoint{8.798440in}{4.627666in}}{\pgfqpoint{8.806253in}{4.619852in}}%
\pgfpathcurveto{\pgfqpoint{8.814067in}{4.612038in}}{\pgfqpoint{8.824666in}{4.607648in}}{\pgfqpoint{8.835716in}{4.607648in}}%
\pgfpathlineto{\pgfqpoint{8.835716in}{4.607648in}}%
\pgfpathclose%
\pgfusepath{stroke,fill}%
\end{pgfscope}%
\begin{pgfscope}%
\pgfpathrectangle{\pgfqpoint{7.622482in}{2.920818in}}{\pgfqpoint{2.177280in}{2.201755in}}%
\pgfusepath{clip}%
\pgfsetbuttcap%
\pgfsetroundjoin%
\definecolor{currentfill}{rgb}{0.172549,0.627451,0.172549}%
\pgfsetfillcolor{currentfill}%
\pgfsetlinewidth{0.481800pt}%
\definecolor{currentstroke}{rgb}{1.000000,1.000000,1.000000}%
\pgfsetstrokecolor{currentstroke}%
\pgfsetdash{}{0pt}%
\pgfpathmoveto{\pgfqpoint{9.038978in}{4.709424in}}%
\pgfpathcurveto{\pgfqpoint{9.050028in}{4.709424in}}{\pgfqpoint{9.060627in}{4.713814in}}{\pgfqpoint{9.068440in}{4.721628in}}%
\pgfpathcurveto{\pgfqpoint{9.076254in}{4.729442in}}{\pgfqpoint{9.080644in}{4.740041in}}{\pgfqpoint{9.080644in}{4.751091in}}%
\pgfpathcurveto{\pgfqpoint{9.080644in}{4.762141in}}{\pgfqpoint{9.076254in}{4.772740in}}{\pgfqpoint{9.068440in}{4.780554in}}%
\pgfpathcurveto{\pgfqpoint{9.060627in}{4.788367in}}{\pgfqpoint{9.050028in}{4.792757in}}{\pgfqpoint{9.038978in}{4.792757in}}%
\pgfpathcurveto{\pgfqpoint{9.027927in}{4.792757in}}{\pgfqpoint{9.017328in}{4.788367in}}{\pgfqpoint{9.009515in}{4.780554in}}%
\pgfpathcurveto{\pgfqpoint{9.001701in}{4.772740in}}{\pgfqpoint{8.997311in}{4.762141in}}{\pgfqpoint{8.997311in}{4.751091in}}%
\pgfpathcurveto{\pgfqpoint{8.997311in}{4.740041in}}{\pgfqpoint{9.001701in}{4.729442in}}{\pgfqpoint{9.009515in}{4.721628in}}%
\pgfpathcurveto{\pgfqpoint{9.017328in}{4.713814in}}{\pgfqpoint{9.027927in}{4.709424in}}{\pgfqpoint{9.038978in}{4.709424in}}%
\pgfpathlineto{\pgfqpoint{9.038978in}{4.709424in}}%
\pgfpathclose%
\pgfusepath{stroke,fill}%
\end{pgfscope}%
\begin{pgfscope}%
\pgfpathrectangle{\pgfqpoint{7.622482in}{2.920818in}}{\pgfqpoint{2.177280in}{2.201755in}}%
\pgfusepath{clip}%
\pgfsetbuttcap%
\pgfsetroundjoin%
\definecolor{currentfill}{rgb}{0.172549,0.627451,0.172549}%
\pgfsetfillcolor{currentfill}%
\pgfsetlinewidth{0.481800pt}%
\definecolor{currentstroke}{rgb}{1.000000,1.000000,1.000000}%
\pgfsetstrokecolor{currentstroke}%
\pgfsetdash{}{0pt}%
\pgfpathmoveto{\pgfqpoint{9.106731in}{4.811200in}}%
\pgfpathcurveto{\pgfqpoint{9.117782in}{4.811200in}}{\pgfqpoint{9.128381in}{4.815590in}}{\pgfqpoint{9.136194in}{4.823404in}}%
\pgfpathcurveto{\pgfqpoint{9.144008in}{4.831218in}}{\pgfqpoint{9.148398in}{4.841817in}}{\pgfqpoint{9.148398in}{4.852867in}}%
\pgfpathcurveto{\pgfqpoint{9.148398in}{4.863917in}}{\pgfqpoint{9.144008in}{4.874516in}}{\pgfqpoint{9.136194in}{4.882330in}}%
\pgfpathcurveto{\pgfqpoint{9.128381in}{4.890143in}}{\pgfqpoint{9.117782in}{4.894534in}}{\pgfqpoint{9.106731in}{4.894534in}}%
\pgfpathcurveto{\pgfqpoint{9.095681in}{4.894534in}}{\pgfqpoint{9.085082in}{4.890143in}}{\pgfqpoint{9.077269in}{4.882330in}}%
\pgfpathcurveto{\pgfqpoint{9.069455in}{4.874516in}}{\pgfqpoint{9.065065in}{4.863917in}}{\pgfqpoint{9.065065in}{4.852867in}}%
\pgfpathcurveto{\pgfqpoint{9.065065in}{4.841817in}}{\pgfqpoint{9.069455in}{4.831218in}}{\pgfqpoint{9.077269in}{4.823404in}}%
\pgfpathcurveto{\pgfqpoint{9.085082in}{4.815590in}}{\pgfqpoint{9.095681in}{4.811200in}}{\pgfqpoint{9.106731in}{4.811200in}}%
\pgfpathlineto{\pgfqpoint{9.106731in}{4.811200in}}%
\pgfpathclose%
\pgfusepath{stroke,fill}%
\end{pgfscope}%
\begin{pgfscope}%
\pgfpathrectangle{\pgfqpoint{7.622482in}{2.920818in}}{\pgfqpoint{2.177280in}{2.201755in}}%
\pgfusepath{clip}%
\pgfsetbuttcap%
\pgfsetroundjoin%
\definecolor{currentfill}{rgb}{0.172549,0.627451,0.172549}%
\pgfsetfillcolor{currentfill}%
\pgfsetlinewidth{0.481800pt}%
\definecolor{currentstroke}{rgb}{1.000000,1.000000,1.000000}%
\pgfsetstrokecolor{currentstroke}%
\pgfsetdash{}{0pt}%
\pgfpathmoveto{\pgfqpoint{9.242239in}{4.539797in}}%
\pgfpathcurveto{\pgfqpoint{9.253289in}{4.539797in}}{\pgfqpoint{9.263888in}{4.544188in}}{\pgfqpoint{9.271702in}{4.552001in}}%
\pgfpathcurveto{\pgfqpoint{9.279516in}{4.559815in}}{\pgfqpoint{9.283906in}{4.570414in}}{\pgfqpoint{9.283906in}{4.581464in}}%
\pgfpathcurveto{\pgfqpoint{9.283906in}{4.592514in}}{\pgfqpoint{9.279516in}{4.603113in}}{\pgfqpoint{9.271702in}{4.610927in}}%
\pgfpathcurveto{\pgfqpoint{9.263888in}{4.618740in}}{\pgfqpoint{9.253289in}{4.623131in}}{\pgfqpoint{9.242239in}{4.623131in}}%
\pgfpathcurveto{\pgfqpoint{9.231189in}{4.623131in}}{\pgfqpoint{9.220590in}{4.618740in}}{\pgfqpoint{9.212776in}{4.610927in}}%
\pgfpathcurveto{\pgfqpoint{9.204963in}{4.603113in}}{\pgfqpoint{9.200573in}{4.592514in}}{\pgfqpoint{9.200573in}{4.581464in}}%
\pgfpathcurveto{\pgfqpoint{9.200573in}{4.570414in}}{\pgfqpoint{9.204963in}{4.559815in}}{\pgfqpoint{9.212776in}{4.552001in}}%
\pgfpathcurveto{\pgfqpoint{9.220590in}{4.544188in}}{\pgfqpoint{9.231189in}{4.539797in}}{\pgfqpoint{9.242239in}{4.539797in}}%
\pgfpathlineto{\pgfqpoint{9.242239in}{4.539797in}}%
\pgfpathclose%
\pgfusepath{stroke,fill}%
\end{pgfscope}%
\begin{pgfscope}%
\pgfpathrectangle{\pgfqpoint{7.622482in}{2.920818in}}{\pgfqpoint{2.177280in}{2.201755in}}%
\pgfusepath{clip}%
\pgfsetbuttcap%
\pgfsetroundjoin%
\definecolor{currentfill}{rgb}{0.172549,0.627451,0.172549}%
\pgfsetfillcolor{currentfill}%
\pgfsetlinewidth{0.481800pt}%
\definecolor{currentstroke}{rgb}{1.000000,1.000000,1.000000}%
\pgfsetstrokecolor{currentstroke}%
\pgfsetdash{}{0pt}%
\pgfpathmoveto{\pgfqpoint{8.767962in}{4.370171in}}%
\pgfpathcurveto{\pgfqpoint{8.779012in}{4.370171in}}{\pgfqpoint{8.789611in}{4.374561in}}{\pgfqpoint{8.797425in}{4.382375in}}%
\pgfpathcurveto{\pgfqpoint{8.805238in}{4.390188in}}{\pgfqpoint{8.809629in}{4.400787in}}{\pgfqpoint{8.809629in}{4.411837in}}%
\pgfpathcurveto{\pgfqpoint{8.809629in}{4.422887in}}{\pgfqpoint{8.805238in}{4.433486in}}{\pgfqpoint{8.797425in}{4.441300in}}%
\pgfpathcurveto{\pgfqpoint{8.789611in}{4.449114in}}{\pgfqpoint{8.779012in}{4.453504in}}{\pgfqpoint{8.767962in}{4.453504in}}%
\pgfpathcurveto{\pgfqpoint{8.756912in}{4.453504in}}{\pgfqpoint{8.746313in}{4.449114in}}{\pgfqpoint{8.738499in}{4.441300in}}%
\pgfpathcurveto{\pgfqpoint{8.730686in}{4.433486in}}{\pgfqpoint{8.726295in}{4.422887in}}{\pgfqpoint{8.726295in}{4.411837in}}%
\pgfpathcurveto{\pgfqpoint{8.726295in}{4.400787in}}{\pgfqpoint{8.730686in}{4.390188in}}{\pgfqpoint{8.738499in}{4.382375in}}%
\pgfpathcurveto{\pgfqpoint{8.746313in}{4.374561in}}{\pgfqpoint{8.756912in}{4.370171in}}{\pgfqpoint{8.767962in}{4.370171in}}%
\pgfpathlineto{\pgfqpoint{8.767962in}{4.370171in}}%
\pgfpathclose%
\pgfusepath{stroke,fill}%
\end{pgfscope}%
\begin{pgfscope}%
\pgfpathrectangle{\pgfqpoint{7.622482in}{2.920818in}}{\pgfqpoint{2.177280in}{2.201755in}}%
\pgfusepath{clip}%
\pgfsetbuttcap%
\pgfsetroundjoin%
\definecolor{currentfill}{rgb}{0.172549,0.627451,0.172549}%
\pgfsetfillcolor{currentfill}%
\pgfsetlinewidth{0.481800pt}%
\definecolor{currentstroke}{rgb}{1.000000,1.000000,1.000000}%
\pgfsetstrokecolor{currentstroke}%
\pgfsetdash{}{0pt}%
\pgfpathmoveto{\pgfqpoint{8.700208in}{4.539797in}}%
\pgfpathcurveto{\pgfqpoint{8.711258in}{4.539797in}}{\pgfqpoint{8.721857in}{4.544188in}}{\pgfqpoint{8.729671in}{4.552001in}}%
\pgfpathcurveto{\pgfqpoint{8.737485in}{4.559815in}}{\pgfqpoint{8.741875in}{4.570414in}}{\pgfqpoint{8.741875in}{4.581464in}}%
\pgfpathcurveto{\pgfqpoint{8.741875in}{4.592514in}}{\pgfqpoint{8.737485in}{4.603113in}}{\pgfqpoint{8.729671in}{4.610927in}}%
\pgfpathcurveto{\pgfqpoint{8.721857in}{4.618740in}}{\pgfqpoint{8.711258in}{4.623131in}}{\pgfqpoint{8.700208in}{4.623131in}}%
\pgfpathcurveto{\pgfqpoint{8.689158in}{4.623131in}}{\pgfqpoint{8.678559in}{4.618740in}}{\pgfqpoint{8.670745in}{4.610927in}}%
\pgfpathcurveto{\pgfqpoint{8.662932in}{4.603113in}}{\pgfqpoint{8.658542in}{4.592514in}}{\pgfqpoint{8.658542in}{4.581464in}}%
\pgfpathcurveto{\pgfqpoint{8.658542in}{4.570414in}}{\pgfqpoint{8.662932in}{4.559815in}}{\pgfqpoint{8.670745in}{4.552001in}}%
\pgfpathcurveto{\pgfqpoint{8.678559in}{4.544188in}}{\pgfqpoint{8.689158in}{4.539797in}}{\pgfqpoint{8.700208in}{4.539797in}}%
\pgfpathlineto{\pgfqpoint{8.700208in}{4.539797in}}%
\pgfpathclose%
\pgfusepath{stroke,fill}%
\end{pgfscope}%
\begin{pgfscope}%
\pgfpathrectangle{\pgfqpoint{7.622482in}{2.920818in}}{\pgfqpoint{2.177280in}{2.201755in}}%
\pgfusepath{clip}%
\pgfsetbuttcap%
\pgfsetroundjoin%
\definecolor{currentfill}{rgb}{0.172549,0.627451,0.172549}%
\pgfsetfillcolor{currentfill}%
\pgfsetlinewidth{0.481800pt}%
\definecolor{currentstroke}{rgb}{1.000000,1.000000,1.000000}%
\pgfsetstrokecolor{currentstroke}%
\pgfsetdash{}{0pt}%
\pgfpathmoveto{\pgfqpoint{9.309993in}{4.709424in}}%
\pgfpathcurveto{\pgfqpoint{9.321043in}{4.709424in}}{\pgfqpoint{9.331642in}{4.713814in}}{\pgfqpoint{9.339456in}{4.721628in}}%
\pgfpathcurveto{\pgfqpoint{9.347269in}{4.729442in}}{\pgfqpoint{9.351660in}{4.740041in}}{\pgfqpoint{9.351660in}{4.751091in}}%
\pgfpathcurveto{\pgfqpoint{9.351660in}{4.762141in}}{\pgfqpoint{9.347269in}{4.772740in}}{\pgfqpoint{9.339456in}{4.780554in}}%
\pgfpathcurveto{\pgfqpoint{9.331642in}{4.788367in}}{\pgfqpoint{9.321043in}{4.792757in}}{\pgfqpoint{9.309993in}{4.792757in}}%
\pgfpathcurveto{\pgfqpoint{9.298943in}{4.792757in}}{\pgfqpoint{9.288344in}{4.788367in}}{\pgfqpoint{9.280530in}{4.780554in}}%
\pgfpathcurveto{\pgfqpoint{9.272717in}{4.772740in}}{\pgfqpoint{9.268326in}{4.762141in}}{\pgfqpoint{9.268326in}{4.751091in}}%
\pgfpathcurveto{\pgfqpoint{9.268326in}{4.740041in}}{\pgfqpoint{9.272717in}{4.729442in}}{\pgfqpoint{9.280530in}{4.721628in}}%
\pgfpathcurveto{\pgfqpoint{9.288344in}{4.713814in}}{\pgfqpoint{9.298943in}{4.709424in}}{\pgfqpoint{9.309993in}{4.709424in}}%
\pgfpathlineto{\pgfqpoint{9.309993in}{4.709424in}}%
\pgfpathclose%
\pgfusepath{stroke,fill}%
\end{pgfscope}%
\begin{pgfscope}%
\pgfpathrectangle{\pgfqpoint{7.622482in}{2.920818in}}{\pgfqpoint{2.177280in}{2.201755in}}%
\pgfusepath{clip}%
\pgfsetbuttcap%
\pgfsetroundjoin%
\definecolor{currentfill}{rgb}{0.172549,0.627451,0.172549}%
\pgfsetfillcolor{currentfill}%
\pgfsetlinewidth{0.481800pt}%
\definecolor{currentstroke}{rgb}{1.000000,1.000000,1.000000}%
\pgfsetstrokecolor{currentstroke}%
\pgfsetdash{}{0pt}%
\pgfpathmoveto{\pgfqpoint{9.377747in}{4.539797in}}%
\pgfpathcurveto{\pgfqpoint{9.388797in}{4.539797in}}{\pgfqpoint{9.399396in}{4.544188in}}{\pgfqpoint{9.407210in}{4.552001in}}%
\pgfpathcurveto{\pgfqpoint{9.415023in}{4.559815in}}{\pgfqpoint{9.419414in}{4.570414in}}{\pgfqpoint{9.419414in}{4.581464in}}%
\pgfpathcurveto{\pgfqpoint{9.419414in}{4.592514in}}{\pgfqpoint{9.415023in}{4.603113in}}{\pgfqpoint{9.407210in}{4.610927in}}%
\pgfpathcurveto{\pgfqpoint{9.399396in}{4.618740in}}{\pgfqpoint{9.388797in}{4.623131in}}{\pgfqpoint{9.377747in}{4.623131in}}%
\pgfpathcurveto{\pgfqpoint{9.366697in}{4.623131in}}{\pgfqpoint{9.356098in}{4.618740in}}{\pgfqpoint{9.348284in}{4.610927in}}%
\pgfpathcurveto{\pgfqpoint{9.340471in}{4.603113in}}{\pgfqpoint{9.336080in}{4.592514in}}{\pgfqpoint{9.336080in}{4.581464in}}%
\pgfpathcurveto{\pgfqpoint{9.336080in}{4.570414in}}{\pgfqpoint{9.340471in}{4.559815in}}{\pgfqpoint{9.348284in}{4.552001in}}%
\pgfpathcurveto{\pgfqpoint{9.356098in}{4.544188in}}{\pgfqpoint{9.366697in}{4.539797in}}{\pgfqpoint{9.377747in}{4.539797in}}%
\pgfpathlineto{\pgfqpoint{9.377747in}{4.539797in}}%
\pgfpathclose%
\pgfusepath{stroke,fill}%
\end{pgfscope}%
\begin{pgfscope}%
\pgfpathrectangle{\pgfqpoint{7.622482in}{2.920818in}}{\pgfqpoint{2.177280in}{2.201755in}}%
\pgfusepath{clip}%
\pgfsetbuttcap%
\pgfsetroundjoin%
\definecolor{currentfill}{rgb}{0.172549,0.627451,0.172549}%
\pgfsetfillcolor{currentfill}%
\pgfsetlinewidth{0.481800pt}%
\definecolor{currentstroke}{rgb}{1.000000,1.000000,1.000000}%
\pgfsetstrokecolor{currentstroke}%
\pgfsetdash{}{0pt}%
\pgfpathmoveto{\pgfqpoint{8.971224in}{4.505872in}}%
\pgfpathcurveto{\pgfqpoint{8.982274in}{4.505872in}}{\pgfqpoint{8.992873in}{4.510262in}}{\pgfqpoint{9.000686in}{4.518076in}}%
\pgfpathcurveto{\pgfqpoint{9.008500in}{4.525890in}}{\pgfqpoint{9.012890in}{4.536489in}}{\pgfqpoint{9.012890in}{4.547539in}}%
\pgfpathcurveto{\pgfqpoint{9.012890in}{4.558589in}}{\pgfqpoint{9.008500in}{4.569188in}}{\pgfqpoint{9.000686in}{4.577002in}}%
\pgfpathcurveto{\pgfqpoint{8.992873in}{4.584815in}}{\pgfqpoint{8.982274in}{4.589205in}}{\pgfqpoint{8.971224in}{4.589205in}}%
\pgfpathcurveto{\pgfqpoint{8.960174in}{4.589205in}}{\pgfqpoint{8.949575in}{4.584815in}}{\pgfqpoint{8.941761in}{4.577002in}}%
\pgfpathcurveto{\pgfqpoint{8.933947in}{4.569188in}}{\pgfqpoint{8.929557in}{4.558589in}}{\pgfqpoint{8.929557in}{4.547539in}}%
\pgfpathcurveto{\pgfqpoint{8.929557in}{4.536489in}}{\pgfqpoint{8.933947in}{4.525890in}}{\pgfqpoint{8.941761in}{4.518076in}}%
\pgfpathcurveto{\pgfqpoint{8.949575in}{4.510262in}}{\pgfqpoint{8.960174in}{4.505872in}}{\pgfqpoint{8.971224in}{4.505872in}}%
\pgfpathlineto{\pgfqpoint{8.971224in}{4.505872in}}%
\pgfpathclose%
\pgfusepath{stroke,fill}%
\end{pgfscope}%
\begin{pgfscope}%
\pgfpathrectangle{\pgfqpoint{7.622482in}{2.920818in}}{\pgfqpoint{2.177280in}{2.201755in}}%
\pgfusepath{clip}%
\pgfsetbuttcap%
\pgfsetroundjoin%
\definecolor{currentfill}{rgb}{0.172549,0.627451,0.172549}%
\pgfsetfillcolor{currentfill}%
\pgfsetlinewidth{0.481800pt}%
\definecolor{currentstroke}{rgb}{1.000000,1.000000,1.000000}%
\pgfsetstrokecolor{currentstroke}%
\pgfsetdash{}{0pt}%
\pgfpathmoveto{\pgfqpoint{8.971224in}{4.268395in}}%
\pgfpathcurveto{\pgfqpoint{8.982274in}{4.268395in}}{\pgfqpoint{8.992873in}{4.272785in}}{\pgfqpoint{9.000686in}{4.280599in}}%
\pgfpathcurveto{\pgfqpoint{9.008500in}{4.288412in}}{\pgfqpoint{9.012890in}{4.299011in}}{\pgfqpoint{9.012890in}{4.310061in}}%
\pgfpathcurveto{\pgfqpoint{9.012890in}{4.321111in}}{\pgfqpoint{9.008500in}{4.331710in}}{\pgfqpoint{9.000686in}{4.339524in}}%
\pgfpathcurveto{\pgfqpoint{8.992873in}{4.347338in}}{\pgfqpoint{8.982274in}{4.351728in}}{\pgfqpoint{8.971224in}{4.351728in}}%
\pgfpathcurveto{\pgfqpoint{8.960174in}{4.351728in}}{\pgfqpoint{8.949575in}{4.347338in}}{\pgfqpoint{8.941761in}{4.339524in}}%
\pgfpathcurveto{\pgfqpoint{8.933947in}{4.331710in}}{\pgfqpoint{8.929557in}{4.321111in}}{\pgfqpoint{8.929557in}{4.310061in}}%
\pgfpathcurveto{\pgfqpoint{8.929557in}{4.299011in}}{\pgfqpoint{8.933947in}{4.288412in}}{\pgfqpoint{8.941761in}{4.280599in}}%
\pgfpathcurveto{\pgfqpoint{8.949575in}{4.272785in}}{\pgfqpoint{8.960174in}{4.268395in}}{\pgfqpoint{8.971224in}{4.268395in}}%
\pgfpathlineto{\pgfqpoint{8.971224in}{4.268395in}}%
\pgfpathclose%
\pgfusepath{stroke,fill}%
\end{pgfscope}%
\begin{pgfscope}%
\pgfpathrectangle{\pgfqpoint{7.622482in}{2.920818in}}{\pgfqpoint{2.177280in}{2.201755in}}%
\pgfusepath{clip}%
\pgfsetbuttcap%
\pgfsetroundjoin%
\definecolor{currentfill}{rgb}{0.172549,0.627451,0.172549}%
\pgfsetfillcolor{currentfill}%
\pgfsetlinewidth{0.481800pt}%
\definecolor{currentstroke}{rgb}{1.000000,1.000000,1.000000}%
\pgfsetstrokecolor{currentstroke}%
\pgfsetdash{}{0pt}%
\pgfpathmoveto{\pgfqpoint{9.174485in}{4.471947in}}%
\pgfpathcurveto{\pgfqpoint{9.185535in}{4.471947in}}{\pgfqpoint{9.196134in}{4.476337in}}{\pgfqpoint{9.203948in}{4.484151in}}%
\pgfpathcurveto{\pgfqpoint{9.211762in}{4.491964in}}{\pgfqpoint{9.216152in}{4.502563in}}{\pgfqpoint{9.216152in}{4.513613in}}%
\pgfpathcurveto{\pgfqpoint{9.216152in}{4.524664in}}{\pgfqpoint{9.211762in}{4.535263in}}{\pgfqpoint{9.203948in}{4.543076in}}%
\pgfpathcurveto{\pgfqpoint{9.196134in}{4.550890in}}{\pgfqpoint{9.185535in}{4.555280in}}{\pgfqpoint{9.174485in}{4.555280in}}%
\pgfpathcurveto{\pgfqpoint{9.163435in}{4.555280in}}{\pgfqpoint{9.152836in}{4.550890in}}{\pgfqpoint{9.145023in}{4.543076in}}%
\pgfpathcurveto{\pgfqpoint{9.137209in}{4.535263in}}{\pgfqpoint{9.132819in}{4.524664in}}{\pgfqpoint{9.132819in}{4.513613in}}%
\pgfpathcurveto{\pgfqpoint{9.132819in}{4.502563in}}{\pgfqpoint{9.137209in}{4.491964in}}{\pgfqpoint{9.145023in}{4.484151in}}%
\pgfpathcurveto{\pgfqpoint{9.152836in}{4.476337in}}{\pgfqpoint{9.163435in}{4.471947in}}{\pgfqpoint{9.174485in}{4.471947in}}%
\pgfpathlineto{\pgfqpoint{9.174485in}{4.471947in}}%
\pgfpathclose%
\pgfusepath{stroke,fill}%
\end{pgfscope}%
\begin{pgfscope}%
\pgfpathrectangle{\pgfqpoint{7.622482in}{2.920818in}}{\pgfqpoint{2.177280in}{2.201755in}}%
\pgfusepath{clip}%
\pgfsetbuttcap%
\pgfsetroundjoin%
\definecolor{currentfill}{rgb}{0.172549,0.627451,0.172549}%
\pgfsetfillcolor{currentfill}%
\pgfsetlinewidth{0.481800pt}%
\definecolor{currentstroke}{rgb}{1.000000,1.000000,1.000000}%
\pgfsetstrokecolor{currentstroke}%
\pgfsetdash{}{0pt}%
\pgfpathmoveto{\pgfqpoint{9.377747in}{4.539797in}}%
\pgfpathcurveto{\pgfqpoint{9.388797in}{4.539797in}}{\pgfqpoint{9.399396in}{4.544188in}}{\pgfqpoint{9.407210in}{4.552001in}}%
\pgfpathcurveto{\pgfqpoint{9.415023in}{4.559815in}}{\pgfqpoint{9.419414in}{4.570414in}}{\pgfqpoint{9.419414in}{4.581464in}}%
\pgfpathcurveto{\pgfqpoint{9.419414in}{4.592514in}}{\pgfqpoint{9.415023in}{4.603113in}}{\pgfqpoint{9.407210in}{4.610927in}}%
\pgfpathcurveto{\pgfqpoint{9.399396in}{4.618740in}}{\pgfqpoint{9.388797in}{4.623131in}}{\pgfqpoint{9.377747in}{4.623131in}}%
\pgfpathcurveto{\pgfqpoint{9.366697in}{4.623131in}}{\pgfqpoint{9.356098in}{4.618740in}}{\pgfqpoint{9.348284in}{4.610927in}}%
\pgfpathcurveto{\pgfqpoint{9.340471in}{4.603113in}}{\pgfqpoint{9.336080in}{4.592514in}}{\pgfqpoint{9.336080in}{4.581464in}}%
\pgfpathcurveto{\pgfqpoint{9.336080in}{4.570414in}}{\pgfqpoint{9.340471in}{4.559815in}}{\pgfqpoint{9.348284in}{4.552001in}}%
\pgfpathcurveto{\pgfqpoint{9.356098in}{4.544188in}}{\pgfqpoint{9.366697in}{4.539797in}}{\pgfqpoint{9.377747in}{4.539797in}}%
\pgfpathlineto{\pgfqpoint{9.377747in}{4.539797in}}%
\pgfpathclose%
\pgfusepath{stroke,fill}%
\end{pgfscope}%
\begin{pgfscope}%
\pgfpathrectangle{\pgfqpoint{7.622482in}{2.920818in}}{\pgfqpoint{2.177280in}{2.201755in}}%
\pgfusepath{clip}%
\pgfsetbuttcap%
\pgfsetroundjoin%
\definecolor{currentfill}{rgb}{0.172549,0.627451,0.172549}%
\pgfsetfillcolor{currentfill}%
\pgfsetlinewidth{0.481800pt}%
\definecolor{currentstroke}{rgb}{1.000000,1.000000,1.000000}%
\pgfsetstrokecolor{currentstroke}%
\pgfsetdash{}{0pt}%
\pgfpathmoveto{\pgfqpoint{9.309993in}{4.370171in}}%
\pgfpathcurveto{\pgfqpoint{9.321043in}{4.370171in}}{\pgfqpoint{9.331642in}{4.374561in}}{\pgfqpoint{9.339456in}{4.382375in}}%
\pgfpathcurveto{\pgfqpoint{9.347269in}{4.390188in}}{\pgfqpoint{9.351660in}{4.400787in}}{\pgfqpoint{9.351660in}{4.411837in}}%
\pgfpathcurveto{\pgfqpoint{9.351660in}{4.422887in}}{\pgfqpoint{9.347269in}{4.433486in}}{\pgfqpoint{9.339456in}{4.441300in}}%
\pgfpathcurveto{\pgfqpoint{9.331642in}{4.449114in}}{\pgfqpoint{9.321043in}{4.453504in}}{\pgfqpoint{9.309993in}{4.453504in}}%
\pgfpathcurveto{\pgfqpoint{9.298943in}{4.453504in}}{\pgfqpoint{9.288344in}{4.449114in}}{\pgfqpoint{9.280530in}{4.441300in}}%
\pgfpathcurveto{\pgfqpoint{9.272717in}{4.433486in}}{\pgfqpoint{9.268326in}{4.422887in}}{\pgfqpoint{9.268326in}{4.411837in}}%
\pgfpathcurveto{\pgfqpoint{9.268326in}{4.400787in}}{\pgfqpoint{9.272717in}{4.390188in}}{\pgfqpoint{9.280530in}{4.382375in}}%
\pgfpathcurveto{\pgfqpoint{9.288344in}{4.374561in}}{\pgfqpoint{9.298943in}{4.370171in}}{\pgfqpoint{9.309993in}{4.370171in}}%
\pgfpathlineto{\pgfqpoint{9.309993in}{4.370171in}}%
\pgfpathclose%
\pgfusepath{stroke,fill}%
\end{pgfscope}%
\begin{pgfscope}%
\pgfpathrectangle{\pgfqpoint{7.622482in}{2.920818in}}{\pgfqpoint{2.177280in}{2.201755in}}%
\pgfusepath{clip}%
\pgfsetbuttcap%
\pgfsetroundjoin%
\definecolor{currentfill}{rgb}{0.172549,0.627451,0.172549}%
\pgfsetfillcolor{currentfill}%
\pgfsetlinewidth{0.481800pt}%
\definecolor{currentstroke}{rgb}{1.000000,1.000000,1.000000}%
\pgfsetstrokecolor{currentstroke}%
\pgfsetdash{}{0pt}%
\pgfpathmoveto{\pgfqpoint{9.038978in}{4.370171in}}%
\pgfpathcurveto{\pgfqpoint{9.050028in}{4.370171in}}{\pgfqpoint{9.060627in}{4.374561in}}{\pgfqpoint{9.068440in}{4.382375in}}%
\pgfpathcurveto{\pgfqpoint{9.076254in}{4.390188in}}{\pgfqpoint{9.080644in}{4.400787in}}{\pgfqpoint{9.080644in}{4.411837in}}%
\pgfpathcurveto{\pgfqpoint{9.080644in}{4.422887in}}{\pgfqpoint{9.076254in}{4.433486in}}{\pgfqpoint{9.068440in}{4.441300in}}%
\pgfpathcurveto{\pgfqpoint{9.060627in}{4.449114in}}{\pgfqpoint{9.050028in}{4.453504in}}{\pgfqpoint{9.038978in}{4.453504in}}%
\pgfpathcurveto{\pgfqpoint{9.027927in}{4.453504in}}{\pgfqpoint{9.017328in}{4.449114in}}{\pgfqpoint{9.009515in}{4.441300in}}%
\pgfpathcurveto{\pgfqpoint{9.001701in}{4.433486in}}{\pgfqpoint{8.997311in}{4.422887in}}{\pgfqpoint{8.997311in}{4.411837in}}%
\pgfpathcurveto{\pgfqpoint{8.997311in}{4.400787in}}{\pgfqpoint{9.001701in}{4.390188in}}{\pgfqpoint{9.009515in}{4.382375in}}%
\pgfpathcurveto{\pgfqpoint{9.017328in}{4.374561in}}{\pgfqpoint{9.027927in}{4.370171in}}{\pgfqpoint{9.038978in}{4.370171in}}%
\pgfpathlineto{\pgfqpoint{9.038978in}{4.370171in}}%
\pgfpathclose%
\pgfusepath{stroke,fill}%
\end{pgfscope}%
\begin{pgfscope}%
\pgfpathrectangle{\pgfqpoint{7.622482in}{2.920818in}}{\pgfqpoint{2.177280in}{2.201755in}}%
\pgfusepath{clip}%
\pgfsetbuttcap%
\pgfsetroundjoin%
\definecolor{currentfill}{rgb}{0.172549,0.627451,0.172549}%
\pgfsetfillcolor{currentfill}%
\pgfsetlinewidth{0.481800pt}%
\definecolor{currentstroke}{rgb}{1.000000,1.000000,1.000000}%
\pgfsetstrokecolor{currentstroke}%
\pgfsetdash{}{0pt}%
\pgfpathmoveto{\pgfqpoint{9.309993in}{4.641573in}}%
\pgfpathcurveto{\pgfqpoint{9.321043in}{4.641573in}}{\pgfqpoint{9.331642in}{4.645964in}}{\pgfqpoint{9.339456in}{4.653777in}}%
\pgfpathcurveto{\pgfqpoint{9.347269in}{4.661591in}}{\pgfqpoint{9.351660in}{4.672190in}}{\pgfqpoint{9.351660in}{4.683240in}}%
\pgfpathcurveto{\pgfqpoint{9.351660in}{4.694290in}}{\pgfqpoint{9.347269in}{4.704889in}}{\pgfqpoint{9.339456in}{4.712703in}}%
\pgfpathcurveto{\pgfqpoint{9.331642in}{4.720517in}}{\pgfqpoint{9.321043in}{4.724907in}}{\pgfqpoint{9.309993in}{4.724907in}}%
\pgfpathcurveto{\pgfqpoint{9.298943in}{4.724907in}}{\pgfqpoint{9.288344in}{4.720517in}}{\pgfqpoint{9.280530in}{4.712703in}}%
\pgfpathcurveto{\pgfqpoint{9.272717in}{4.704889in}}{\pgfqpoint{9.268326in}{4.694290in}}{\pgfqpoint{9.268326in}{4.683240in}}%
\pgfpathcurveto{\pgfqpoint{9.268326in}{4.672190in}}{\pgfqpoint{9.272717in}{4.661591in}}{\pgfqpoint{9.280530in}{4.653777in}}%
\pgfpathcurveto{\pgfqpoint{9.288344in}{4.645964in}}{\pgfqpoint{9.298943in}{4.641573in}}{\pgfqpoint{9.309993in}{4.641573in}}%
\pgfpathlineto{\pgfqpoint{9.309993in}{4.641573in}}%
\pgfpathclose%
\pgfusepath{stroke,fill}%
\end{pgfscope}%
\begin{pgfscope}%
\pgfpathrectangle{\pgfqpoint{7.622482in}{2.920818in}}{\pgfqpoint{2.177280in}{2.201755in}}%
\pgfusepath{clip}%
\pgfsetbuttcap%
\pgfsetroundjoin%
\definecolor{currentfill}{rgb}{0.172549,0.627451,0.172549}%
\pgfsetfillcolor{currentfill}%
\pgfsetlinewidth{0.481800pt}%
\definecolor{currentstroke}{rgb}{1.000000,1.000000,1.000000}%
\pgfsetstrokecolor{currentstroke}%
\pgfsetdash{}{0pt}%
\pgfpathmoveto{\pgfqpoint{9.445501in}{4.573723in}}%
\pgfpathcurveto{\pgfqpoint{9.456551in}{4.573723in}}{\pgfqpoint{9.467150in}{4.578113in}}{\pgfqpoint{9.474964in}{4.585927in}}%
\pgfpathcurveto{\pgfqpoint{9.482777in}{4.593740in}}{\pgfqpoint{9.487167in}{4.604339in}}{\pgfqpoint{9.487167in}{4.615389in}}%
\pgfpathcurveto{\pgfqpoint{9.487167in}{4.626440in}}{\pgfqpoint{9.482777in}{4.637039in}}{\pgfqpoint{9.474964in}{4.644852in}}%
\pgfpathcurveto{\pgfqpoint{9.467150in}{4.652666in}}{\pgfqpoint{9.456551in}{4.657056in}}{\pgfqpoint{9.445501in}{4.657056in}}%
\pgfpathcurveto{\pgfqpoint{9.434451in}{4.657056in}}{\pgfqpoint{9.423852in}{4.652666in}}{\pgfqpoint{9.416038in}{4.644852in}}%
\pgfpathcurveto{\pgfqpoint{9.408224in}{4.637039in}}{\pgfqpoint{9.403834in}{4.626440in}}{\pgfqpoint{9.403834in}{4.615389in}}%
\pgfpathcurveto{\pgfqpoint{9.403834in}{4.604339in}}{\pgfqpoint{9.408224in}{4.593740in}}{\pgfqpoint{9.416038in}{4.585927in}}%
\pgfpathcurveto{\pgfqpoint{9.423852in}{4.578113in}}{\pgfqpoint{9.434451in}{4.573723in}}{\pgfqpoint{9.445501in}{4.573723in}}%
\pgfpathlineto{\pgfqpoint{9.445501in}{4.573723in}}%
\pgfpathclose%
\pgfusepath{stroke,fill}%
\end{pgfscope}%
\begin{pgfscope}%
\pgfpathrectangle{\pgfqpoint{7.622482in}{2.920818in}}{\pgfqpoint{2.177280in}{2.201755in}}%
\pgfusepath{clip}%
\pgfsetbuttcap%
\pgfsetroundjoin%
\definecolor{currentfill}{rgb}{0.172549,0.627451,0.172549}%
\pgfsetfillcolor{currentfill}%
\pgfsetlinewidth{0.481800pt}%
\definecolor{currentstroke}{rgb}{1.000000,1.000000,1.000000}%
\pgfsetstrokecolor{currentstroke}%
\pgfsetdash{}{0pt}%
\pgfpathmoveto{\pgfqpoint{9.309993in}{4.404096in}}%
\pgfpathcurveto{\pgfqpoint{9.321043in}{4.404096in}}{\pgfqpoint{9.331642in}{4.408486in}}{\pgfqpoint{9.339456in}{4.416300in}}%
\pgfpathcurveto{\pgfqpoint{9.347269in}{4.424114in}}{\pgfqpoint{9.351660in}{4.434713in}}{\pgfqpoint{9.351660in}{4.445763in}}%
\pgfpathcurveto{\pgfqpoint{9.351660in}{4.456813in}}{\pgfqpoint{9.347269in}{4.467412in}}{\pgfqpoint{9.339456in}{4.475225in}}%
\pgfpathcurveto{\pgfqpoint{9.331642in}{4.483039in}}{\pgfqpoint{9.321043in}{4.487429in}}{\pgfqpoint{9.309993in}{4.487429in}}%
\pgfpathcurveto{\pgfqpoint{9.298943in}{4.487429in}}{\pgfqpoint{9.288344in}{4.483039in}}{\pgfqpoint{9.280530in}{4.475225in}}%
\pgfpathcurveto{\pgfqpoint{9.272717in}{4.467412in}}{\pgfqpoint{9.268326in}{4.456813in}}{\pgfqpoint{9.268326in}{4.445763in}}%
\pgfpathcurveto{\pgfqpoint{9.268326in}{4.434713in}}{\pgfqpoint{9.272717in}{4.424114in}}{\pgfqpoint{9.280530in}{4.416300in}}%
\pgfpathcurveto{\pgfqpoint{9.288344in}{4.408486in}}{\pgfqpoint{9.298943in}{4.404096in}}{\pgfqpoint{9.309993in}{4.404096in}}%
\pgfpathlineto{\pgfqpoint{9.309993in}{4.404096in}}%
\pgfpathclose%
\pgfusepath{stroke,fill}%
\end{pgfscope}%
\begin{pgfscope}%
\pgfpathrectangle{\pgfqpoint{7.622482in}{2.920818in}}{\pgfqpoint{2.177280in}{2.201755in}}%
\pgfusepath{clip}%
\pgfsetbuttcap%
\pgfsetroundjoin%
\definecolor{currentfill}{rgb}{0.172549,0.627451,0.172549}%
\pgfsetfillcolor{currentfill}%
\pgfsetlinewidth{0.481800pt}%
\definecolor{currentstroke}{rgb}{1.000000,1.000000,1.000000}%
\pgfsetstrokecolor{currentstroke}%
\pgfsetdash{}{0pt}%
\pgfpathmoveto{\pgfqpoint{9.038978in}{4.336245in}}%
\pgfpathcurveto{\pgfqpoint{9.050028in}{4.336245in}}{\pgfqpoint{9.060627in}{4.340636in}}{\pgfqpoint{9.068440in}{4.348449in}}%
\pgfpathcurveto{\pgfqpoint{9.076254in}{4.356263in}}{\pgfqpoint{9.080644in}{4.366862in}}{\pgfqpoint{9.080644in}{4.377912in}}%
\pgfpathcurveto{\pgfqpoint{9.080644in}{4.388962in}}{\pgfqpoint{9.076254in}{4.399561in}}{\pgfqpoint{9.068440in}{4.407375in}}%
\pgfpathcurveto{\pgfqpoint{9.060627in}{4.415188in}}{\pgfqpoint{9.050028in}{4.419579in}}{\pgfqpoint{9.038978in}{4.419579in}}%
\pgfpathcurveto{\pgfqpoint{9.027927in}{4.419579in}}{\pgfqpoint{9.017328in}{4.415188in}}{\pgfqpoint{9.009515in}{4.407375in}}%
\pgfpathcurveto{\pgfqpoint{9.001701in}{4.399561in}}{\pgfqpoint{8.997311in}{4.388962in}}{\pgfqpoint{8.997311in}{4.377912in}}%
\pgfpathcurveto{\pgfqpoint{8.997311in}{4.366862in}}{\pgfqpoint{9.001701in}{4.356263in}}{\pgfqpoint{9.009515in}{4.348449in}}%
\pgfpathcurveto{\pgfqpoint{9.017328in}{4.340636in}}{\pgfqpoint{9.027927in}{4.336245in}}{\pgfqpoint{9.038978in}{4.336245in}}%
\pgfpathlineto{\pgfqpoint{9.038978in}{4.336245in}}%
\pgfpathclose%
\pgfusepath{stroke,fill}%
\end{pgfscope}%
\begin{pgfscope}%
\pgfpathrectangle{\pgfqpoint{7.622482in}{2.920818in}}{\pgfqpoint{2.177280in}{2.201755in}}%
\pgfusepath{clip}%
\pgfsetbuttcap%
\pgfsetroundjoin%
\definecolor{currentfill}{rgb}{0.172549,0.627451,0.172549}%
\pgfsetfillcolor{currentfill}%
\pgfsetlinewidth{0.481800pt}%
\definecolor{currentstroke}{rgb}{1.000000,1.000000,1.000000}%
\pgfsetstrokecolor{currentstroke}%
\pgfsetdash{}{0pt}%
\pgfpathmoveto{\pgfqpoint{9.106731in}{4.404096in}}%
\pgfpathcurveto{\pgfqpoint{9.117782in}{4.404096in}}{\pgfqpoint{9.128381in}{4.408486in}}{\pgfqpoint{9.136194in}{4.416300in}}%
\pgfpathcurveto{\pgfqpoint{9.144008in}{4.424114in}}{\pgfqpoint{9.148398in}{4.434713in}}{\pgfqpoint{9.148398in}{4.445763in}}%
\pgfpathcurveto{\pgfqpoint{9.148398in}{4.456813in}}{\pgfqpoint{9.144008in}{4.467412in}}{\pgfqpoint{9.136194in}{4.475225in}}%
\pgfpathcurveto{\pgfqpoint{9.128381in}{4.483039in}}{\pgfqpoint{9.117782in}{4.487429in}}{\pgfqpoint{9.106731in}{4.487429in}}%
\pgfpathcurveto{\pgfqpoint{9.095681in}{4.487429in}}{\pgfqpoint{9.085082in}{4.483039in}}{\pgfqpoint{9.077269in}{4.475225in}}%
\pgfpathcurveto{\pgfqpoint{9.069455in}{4.467412in}}{\pgfqpoint{9.065065in}{4.456813in}}{\pgfqpoint{9.065065in}{4.445763in}}%
\pgfpathcurveto{\pgfqpoint{9.065065in}{4.434713in}}{\pgfqpoint{9.069455in}{4.424114in}}{\pgfqpoint{9.077269in}{4.416300in}}%
\pgfpathcurveto{\pgfqpoint{9.085082in}{4.408486in}}{\pgfqpoint{9.095681in}{4.404096in}}{\pgfqpoint{9.106731in}{4.404096in}}%
\pgfpathlineto{\pgfqpoint{9.106731in}{4.404096in}}%
\pgfpathclose%
\pgfusepath{stroke,fill}%
\end{pgfscope}%
\begin{pgfscope}%
\pgfpathrectangle{\pgfqpoint{7.622482in}{2.920818in}}{\pgfqpoint{2.177280in}{2.201755in}}%
\pgfusepath{clip}%
\pgfsetbuttcap%
\pgfsetroundjoin%
\definecolor{currentfill}{rgb}{0.172549,0.627451,0.172549}%
\pgfsetfillcolor{currentfill}%
\pgfsetlinewidth{0.481800pt}%
\definecolor{currentstroke}{rgb}{1.000000,1.000000,1.000000}%
\pgfsetstrokecolor{currentstroke}%
\pgfsetdash{}{0pt}%
\pgfpathmoveto{\pgfqpoint{9.309993in}{4.471947in}}%
\pgfpathcurveto{\pgfqpoint{9.321043in}{4.471947in}}{\pgfqpoint{9.331642in}{4.476337in}}{\pgfqpoint{9.339456in}{4.484151in}}%
\pgfpathcurveto{\pgfqpoint{9.347269in}{4.491964in}}{\pgfqpoint{9.351660in}{4.502563in}}{\pgfqpoint{9.351660in}{4.513613in}}%
\pgfpathcurveto{\pgfqpoint{9.351660in}{4.524664in}}{\pgfqpoint{9.347269in}{4.535263in}}{\pgfqpoint{9.339456in}{4.543076in}}%
\pgfpathcurveto{\pgfqpoint{9.331642in}{4.550890in}}{\pgfqpoint{9.321043in}{4.555280in}}{\pgfqpoint{9.309993in}{4.555280in}}%
\pgfpathcurveto{\pgfqpoint{9.298943in}{4.555280in}}{\pgfqpoint{9.288344in}{4.550890in}}{\pgfqpoint{9.280530in}{4.543076in}}%
\pgfpathcurveto{\pgfqpoint{9.272717in}{4.535263in}}{\pgfqpoint{9.268326in}{4.524664in}}{\pgfqpoint{9.268326in}{4.513613in}}%
\pgfpathcurveto{\pgfqpoint{9.268326in}{4.502563in}}{\pgfqpoint{9.272717in}{4.491964in}}{\pgfqpoint{9.280530in}{4.484151in}}%
\pgfpathcurveto{\pgfqpoint{9.288344in}{4.476337in}}{\pgfqpoint{9.298943in}{4.471947in}}{\pgfqpoint{9.309993in}{4.471947in}}%
\pgfpathlineto{\pgfqpoint{9.309993in}{4.471947in}}%
\pgfpathclose%
\pgfusepath{stroke,fill}%
\end{pgfscope}%
\begin{pgfscope}%
\pgfpathrectangle{\pgfqpoint{7.622482in}{2.920818in}}{\pgfqpoint{2.177280in}{2.201755in}}%
\pgfusepath{clip}%
\pgfsetbuttcap%
\pgfsetroundjoin%
\definecolor{currentfill}{rgb}{0.172549,0.627451,0.172549}%
\pgfsetfillcolor{currentfill}%
\pgfsetlinewidth{0.481800pt}%
\definecolor{currentstroke}{rgb}{1.000000,1.000000,1.000000}%
\pgfsetstrokecolor{currentstroke}%
\pgfsetdash{}{0pt}%
\pgfpathmoveto{\pgfqpoint{8.971224in}{4.370171in}}%
\pgfpathcurveto{\pgfqpoint{8.982274in}{4.370171in}}{\pgfqpoint{8.992873in}{4.374561in}}{\pgfqpoint{9.000686in}{4.382375in}}%
\pgfpathcurveto{\pgfqpoint{9.008500in}{4.390188in}}{\pgfqpoint{9.012890in}{4.400787in}}{\pgfqpoint{9.012890in}{4.411837in}}%
\pgfpathcurveto{\pgfqpoint{9.012890in}{4.422887in}}{\pgfqpoint{9.008500in}{4.433486in}}{\pgfqpoint{9.000686in}{4.441300in}}%
\pgfpathcurveto{\pgfqpoint{8.992873in}{4.449114in}}{\pgfqpoint{8.982274in}{4.453504in}}{\pgfqpoint{8.971224in}{4.453504in}}%
\pgfpathcurveto{\pgfqpoint{8.960174in}{4.453504in}}{\pgfqpoint{8.949575in}{4.449114in}}{\pgfqpoint{8.941761in}{4.441300in}}%
\pgfpathcurveto{\pgfqpoint{8.933947in}{4.433486in}}{\pgfqpoint{8.929557in}{4.422887in}}{\pgfqpoint{8.929557in}{4.411837in}}%
\pgfpathcurveto{\pgfqpoint{8.929557in}{4.400787in}}{\pgfqpoint{8.933947in}{4.390188in}}{\pgfqpoint{8.941761in}{4.382375in}}%
\pgfpathcurveto{\pgfqpoint{8.949575in}{4.374561in}}{\pgfqpoint{8.960174in}{4.370171in}}{\pgfqpoint{8.971224in}{4.370171in}}%
\pgfpathlineto{\pgfqpoint{8.971224in}{4.370171in}}%
\pgfpathclose%
\pgfusepath{stroke,fill}%
\end{pgfscope}%
\begin{pgfscope}%
\pgfpathrectangle{\pgfqpoint{7.622482in}{2.920818in}}{\pgfqpoint{2.177280in}{2.201755in}}%
\pgfusepath{clip}%
\pgfsetbuttcap%
\pgfsetroundjoin%
\definecolor{currentfill}{rgb}{0.121569,0.466667,0.705882}%
\pgfsetfillcolor{currentfill}%
\pgfsetlinewidth{1.003750pt}%
\definecolor{currentstroke}{rgb}{0.121569,0.466667,0.705882}%
\pgfsetstrokecolor{currentstroke}%
\pgfsetdash{}{0pt}%
\pgfsys@defobject{currentmarker}{\pgfqpoint{-0.041667in}{-0.041667in}}{\pgfqpoint{0.041667in}{0.041667in}}{%
\pgfpathmoveto{\pgfqpoint{0.000000in}{-0.041667in}}%
\pgfpathcurveto{\pgfqpoint{0.011050in}{-0.041667in}}{\pgfqpoint{0.021649in}{-0.037276in}}{\pgfqpoint{0.029463in}{-0.029463in}}%
\pgfpathcurveto{\pgfqpoint{0.037276in}{-0.021649in}}{\pgfqpoint{0.041667in}{-0.011050in}}{\pgfqpoint{0.041667in}{0.000000in}}%
\pgfpathcurveto{\pgfqpoint{0.041667in}{0.011050in}}{\pgfqpoint{0.037276in}{0.021649in}}{\pgfqpoint{0.029463in}{0.029463in}}%
\pgfpathcurveto{\pgfqpoint{0.021649in}{0.037276in}}{\pgfqpoint{0.011050in}{0.041667in}}{\pgfqpoint{0.000000in}{0.041667in}}%
\pgfpathcurveto{\pgfqpoint{-0.011050in}{0.041667in}}{\pgfqpoint{-0.021649in}{0.037276in}}{\pgfqpoint{-0.029463in}{0.029463in}}%
\pgfpathcurveto{\pgfqpoint{-0.037276in}{0.021649in}}{\pgfqpoint{-0.041667in}{0.011050in}}{\pgfqpoint{-0.041667in}{0.000000in}}%
\pgfpathcurveto{\pgfqpoint{-0.041667in}{-0.011050in}}{\pgfqpoint{-0.037276in}{-0.021649in}}{\pgfqpoint{-0.029463in}{-0.029463in}}%
\pgfpathcurveto{\pgfqpoint{-0.021649in}{-0.037276in}}{\pgfqpoint{-0.011050in}{-0.041667in}}{\pgfqpoint{0.000000in}{-0.041667in}}%
\pgfpathlineto{\pgfqpoint{0.000000in}{-0.041667in}}%
\pgfpathclose%
\pgfusepath{stroke,fill}%
}%
\end{pgfscope}%
\begin{pgfscope}%
\pgfpathrectangle{\pgfqpoint{7.622482in}{2.920818in}}{\pgfqpoint{2.177280in}{2.201755in}}%
\pgfusepath{clip}%
\pgfsetbuttcap%
\pgfsetroundjoin%
\definecolor{currentfill}{rgb}{1.000000,0.498039,0.054902}%
\pgfsetfillcolor{currentfill}%
\pgfsetlinewidth{1.003750pt}%
\definecolor{currentstroke}{rgb}{1.000000,0.498039,0.054902}%
\pgfsetstrokecolor{currentstroke}%
\pgfsetdash{}{0pt}%
\pgfsys@defobject{currentmarker}{\pgfqpoint{-0.041667in}{-0.041667in}}{\pgfqpoint{0.041667in}{0.041667in}}{%
\pgfpathmoveto{\pgfqpoint{0.000000in}{-0.041667in}}%
\pgfpathcurveto{\pgfqpoint{0.011050in}{-0.041667in}}{\pgfqpoint{0.021649in}{-0.037276in}}{\pgfqpoint{0.029463in}{-0.029463in}}%
\pgfpathcurveto{\pgfqpoint{0.037276in}{-0.021649in}}{\pgfqpoint{0.041667in}{-0.011050in}}{\pgfqpoint{0.041667in}{0.000000in}}%
\pgfpathcurveto{\pgfqpoint{0.041667in}{0.011050in}}{\pgfqpoint{0.037276in}{0.021649in}}{\pgfqpoint{0.029463in}{0.029463in}}%
\pgfpathcurveto{\pgfqpoint{0.021649in}{0.037276in}}{\pgfqpoint{0.011050in}{0.041667in}}{\pgfqpoint{0.000000in}{0.041667in}}%
\pgfpathcurveto{\pgfqpoint{-0.011050in}{0.041667in}}{\pgfqpoint{-0.021649in}{0.037276in}}{\pgfqpoint{-0.029463in}{0.029463in}}%
\pgfpathcurveto{\pgfqpoint{-0.037276in}{0.021649in}}{\pgfqpoint{-0.041667in}{0.011050in}}{\pgfqpoint{-0.041667in}{0.000000in}}%
\pgfpathcurveto{\pgfqpoint{-0.041667in}{-0.011050in}}{\pgfqpoint{-0.037276in}{-0.021649in}}{\pgfqpoint{-0.029463in}{-0.029463in}}%
\pgfpathcurveto{\pgfqpoint{-0.021649in}{-0.037276in}}{\pgfqpoint{-0.011050in}{-0.041667in}}{\pgfqpoint{0.000000in}{-0.041667in}}%
\pgfpathlineto{\pgfqpoint{0.000000in}{-0.041667in}}%
\pgfpathclose%
\pgfusepath{stroke,fill}%
}%
\end{pgfscope}%
\begin{pgfscope}%
\pgfpathrectangle{\pgfqpoint{7.622482in}{2.920818in}}{\pgfqpoint{2.177280in}{2.201755in}}%
\pgfusepath{clip}%
\pgfsetbuttcap%
\pgfsetroundjoin%
\definecolor{currentfill}{rgb}{0.172549,0.627451,0.172549}%
\pgfsetfillcolor{currentfill}%
\pgfsetlinewidth{1.003750pt}%
\definecolor{currentstroke}{rgb}{0.172549,0.627451,0.172549}%
\pgfsetstrokecolor{currentstroke}%
\pgfsetdash{}{0pt}%
\pgfsys@defobject{currentmarker}{\pgfqpoint{-0.041667in}{-0.041667in}}{\pgfqpoint{0.041667in}{0.041667in}}{%
\pgfpathmoveto{\pgfqpoint{0.000000in}{-0.041667in}}%
\pgfpathcurveto{\pgfqpoint{0.011050in}{-0.041667in}}{\pgfqpoint{0.021649in}{-0.037276in}}{\pgfqpoint{0.029463in}{-0.029463in}}%
\pgfpathcurveto{\pgfqpoint{0.037276in}{-0.021649in}}{\pgfqpoint{0.041667in}{-0.011050in}}{\pgfqpoint{0.041667in}{0.000000in}}%
\pgfpathcurveto{\pgfqpoint{0.041667in}{0.011050in}}{\pgfqpoint{0.037276in}{0.021649in}}{\pgfqpoint{0.029463in}{0.029463in}}%
\pgfpathcurveto{\pgfqpoint{0.021649in}{0.037276in}}{\pgfqpoint{0.011050in}{0.041667in}}{\pgfqpoint{0.000000in}{0.041667in}}%
\pgfpathcurveto{\pgfqpoint{-0.011050in}{0.041667in}}{\pgfqpoint{-0.021649in}{0.037276in}}{\pgfqpoint{-0.029463in}{0.029463in}}%
\pgfpathcurveto{\pgfqpoint{-0.037276in}{0.021649in}}{\pgfqpoint{-0.041667in}{0.011050in}}{\pgfqpoint{-0.041667in}{0.000000in}}%
\pgfpathcurveto{\pgfqpoint{-0.041667in}{-0.011050in}}{\pgfqpoint{-0.037276in}{-0.021649in}}{\pgfqpoint{-0.029463in}{-0.029463in}}%
\pgfpathcurveto{\pgfqpoint{-0.021649in}{-0.037276in}}{\pgfqpoint{-0.011050in}{-0.041667in}}{\pgfqpoint{0.000000in}{-0.041667in}}%
\pgfpathlineto{\pgfqpoint{0.000000in}{-0.041667in}}%
\pgfpathclose%
\pgfusepath{stroke,fill}%
}%
\end{pgfscope}%
\begin{pgfscope}%
\pgfsetbuttcap%
\pgfsetroundjoin%
\definecolor{currentfill}{rgb}{0.000000,0.000000,0.000000}%
\pgfsetfillcolor{currentfill}%
\pgfsetlinewidth{0.803000pt}%
\definecolor{currentstroke}{rgb}{0.000000,0.000000,0.000000}%
\pgfsetstrokecolor{currentstroke}%
\pgfsetdash{}{0pt}%
\pgfsys@defobject{currentmarker}{\pgfqpoint{0.000000in}{-0.048611in}}{\pgfqpoint{0.000000in}{0.000000in}}{%
\pgfpathmoveto{\pgfqpoint{0.000000in}{0.000000in}}%
\pgfpathlineto{\pgfqpoint{0.000000in}{-0.048611in}}%
\pgfusepath{stroke,fill}%
}%
\begin{pgfscope}%
\pgfsys@transformshift{7.751654in}{2.920818in}%
\pgfsys@useobject{currentmarker}{}%
\end{pgfscope}%
\end{pgfscope}%
\begin{pgfscope}%
\pgfsetbuttcap%
\pgfsetroundjoin%
\definecolor{currentfill}{rgb}{0.000000,0.000000,0.000000}%
\pgfsetfillcolor{currentfill}%
\pgfsetlinewidth{0.803000pt}%
\definecolor{currentstroke}{rgb}{0.000000,0.000000,0.000000}%
\pgfsetstrokecolor{currentstroke}%
\pgfsetdash{}{0pt}%
\pgfsys@defobject{currentmarker}{\pgfqpoint{0.000000in}{-0.048611in}}{\pgfqpoint{0.000000in}{0.000000in}}{%
\pgfpathmoveto{\pgfqpoint{0.000000in}{0.000000in}}%
\pgfpathlineto{\pgfqpoint{0.000000in}{-0.048611in}}%
\pgfusepath{stroke,fill}%
}%
\begin{pgfscope}%
\pgfsys@transformshift{8.429193in}{2.920818in}%
\pgfsys@useobject{currentmarker}{}%
\end{pgfscope}%
\end{pgfscope}%
\begin{pgfscope}%
\pgfsetbuttcap%
\pgfsetroundjoin%
\definecolor{currentfill}{rgb}{0.000000,0.000000,0.000000}%
\pgfsetfillcolor{currentfill}%
\pgfsetlinewidth{0.803000pt}%
\definecolor{currentstroke}{rgb}{0.000000,0.000000,0.000000}%
\pgfsetstrokecolor{currentstroke}%
\pgfsetdash{}{0pt}%
\pgfsys@defobject{currentmarker}{\pgfqpoint{0.000000in}{-0.048611in}}{\pgfqpoint{0.000000in}{0.000000in}}{%
\pgfpathmoveto{\pgfqpoint{0.000000in}{0.000000in}}%
\pgfpathlineto{\pgfqpoint{0.000000in}{-0.048611in}}%
\pgfusepath{stroke,fill}%
}%
\begin{pgfscope}%
\pgfsys@transformshift{9.106731in}{2.920818in}%
\pgfsys@useobject{currentmarker}{}%
\end{pgfscope}%
\end{pgfscope}%
\begin{pgfscope}%
\pgfsetbuttcap%
\pgfsetroundjoin%
\definecolor{currentfill}{rgb}{0.000000,0.000000,0.000000}%
\pgfsetfillcolor{currentfill}%
\pgfsetlinewidth{0.803000pt}%
\definecolor{currentstroke}{rgb}{0.000000,0.000000,0.000000}%
\pgfsetstrokecolor{currentstroke}%
\pgfsetdash{}{0pt}%
\pgfsys@defobject{currentmarker}{\pgfqpoint{0.000000in}{-0.048611in}}{\pgfqpoint{0.000000in}{0.000000in}}{%
\pgfpathmoveto{\pgfqpoint{0.000000in}{0.000000in}}%
\pgfpathlineto{\pgfqpoint{0.000000in}{-0.048611in}}%
\pgfusepath{stroke,fill}%
}%
\begin{pgfscope}%
\pgfsys@transformshift{9.784270in}{2.920818in}%
\pgfsys@useobject{currentmarker}{}%
\end{pgfscope}%
\end{pgfscope}%
\begin{pgfscope}%
\pgfsetbuttcap%
\pgfsetroundjoin%
\definecolor{currentfill}{rgb}{0.000000,0.000000,0.000000}%
\pgfsetfillcolor{currentfill}%
\pgfsetlinewidth{0.803000pt}%
\definecolor{currentstroke}{rgb}{0.000000,0.000000,0.000000}%
\pgfsetstrokecolor{currentstroke}%
\pgfsetdash{}{0pt}%
\pgfsys@defobject{currentmarker}{\pgfqpoint{-0.048611in}{0.000000in}}{\pgfqpoint{-0.000000in}{0.000000in}}{%
\pgfpathmoveto{\pgfqpoint{-0.000000in}{0.000000in}}%
\pgfpathlineto{\pgfqpoint{-0.048611in}{0.000000in}}%
\pgfusepath{stroke,fill}%
}%
\begin{pgfscope}%
\pgfsys@transformshift{7.622482in}{3.020898in}%
\pgfsys@useobject{currentmarker}{}%
\end{pgfscope}%
\end{pgfscope}%
\begin{pgfscope}%
\pgfsetbuttcap%
\pgfsetroundjoin%
\definecolor{currentfill}{rgb}{0.000000,0.000000,0.000000}%
\pgfsetfillcolor{currentfill}%
\pgfsetlinewidth{0.803000pt}%
\definecolor{currentstroke}{rgb}{0.000000,0.000000,0.000000}%
\pgfsetstrokecolor{currentstroke}%
\pgfsetdash{}{0pt}%
\pgfsys@defobject{currentmarker}{\pgfqpoint{-0.048611in}{0.000000in}}{\pgfqpoint{-0.000000in}{0.000000in}}{%
\pgfpathmoveto{\pgfqpoint{-0.000000in}{0.000000in}}%
\pgfpathlineto{\pgfqpoint{-0.048611in}{0.000000in}}%
\pgfusepath{stroke,fill}%
}%
\begin{pgfscope}%
\pgfsys@transformshift{7.622482in}{3.360152in}%
\pgfsys@useobject{currentmarker}{}%
\end{pgfscope}%
\end{pgfscope}%
\begin{pgfscope}%
\pgfsetbuttcap%
\pgfsetroundjoin%
\definecolor{currentfill}{rgb}{0.000000,0.000000,0.000000}%
\pgfsetfillcolor{currentfill}%
\pgfsetlinewidth{0.803000pt}%
\definecolor{currentstroke}{rgb}{0.000000,0.000000,0.000000}%
\pgfsetstrokecolor{currentstroke}%
\pgfsetdash{}{0pt}%
\pgfsys@defobject{currentmarker}{\pgfqpoint{-0.048611in}{0.000000in}}{\pgfqpoint{-0.000000in}{0.000000in}}{%
\pgfpathmoveto{\pgfqpoint{-0.000000in}{0.000000in}}%
\pgfpathlineto{\pgfqpoint{-0.048611in}{0.000000in}}%
\pgfusepath{stroke,fill}%
}%
\begin{pgfscope}%
\pgfsys@transformshift{7.622482in}{3.699405in}%
\pgfsys@useobject{currentmarker}{}%
\end{pgfscope}%
\end{pgfscope}%
\begin{pgfscope}%
\pgfsetbuttcap%
\pgfsetroundjoin%
\definecolor{currentfill}{rgb}{0.000000,0.000000,0.000000}%
\pgfsetfillcolor{currentfill}%
\pgfsetlinewidth{0.803000pt}%
\definecolor{currentstroke}{rgb}{0.000000,0.000000,0.000000}%
\pgfsetstrokecolor{currentstroke}%
\pgfsetdash{}{0pt}%
\pgfsys@defobject{currentmarker}{\pgfqpoint{-0.048611in}{0.000000in}}{\pgfqpoint{-0.000000in}{0.000000in}}{%
\pgfpathmoveto{\pgfqpoint{-0.000000in}{0.000000in}}%
\pgfpathlineto{\pgfqpoint{-0.048611in}{0.000000in}}%
\pgfusepath{stroke,fill}%
}%
\begin{pgfscope}%
\pgfsys@transformshift{7.622482in}{4.038659in}%
\pgfsys@useobject{currentmarker}{}%
\end{pgfscope}%
\end{pgfscope}%
\begin{pgfscope}%
\pgfsetbuttcap%
\pgfsetroundjoin%
\definecolor{currentfill}{rgb}{0.000000,0.000000,0.000000}%
\pgfsetfillcolor{currentfill}%
\pgfsetlinewidth{0.803000pt}%
\definecolor{currentstroke}{rgb}{0.000000,0.000000,0.000000}%
\pgfsetstrokecolor{currentstroke}%
\pgfsetdash{}{0pt}%
\pgfsys@defobject{currentmarker}{\pgfqpoint{-0.048611in}{0.000000in}}{\pgfqpoint{-0.000000in}{0.000000in}}{%
\pgfpathmoveto{\pgfqpoint{-0.000000in}{0.000000in}}%
\pgfpathlineto{\pgfqpoint{-0.048611in}{0.000000in}}%
\pgfusepath{stroke,fill}%
}%
\begin{pgfscope}%
\pgfsys@transformshift{7.622482in}{4.377912in}%
\pgfsys@useobject{currentmarker}{}%
\end{pgfscope}%
\end{pgfscope}%
\begin{pgfscope}%
\pgfsetbuttcap%
\pgfsetroundjoin%
\definecolor{currentfill}{rgb}{0.000000,0.000000,0.000000}%
\pgfsetfillcolor{currentfill}%
\pgfsetlinewidth{0.803000pt}%
\definecolor{currentstroke}{rgb}{0.000000,0.000000,0.000000}%
\pgfsetstrokecolor{currentstroke}%
\pgfsetdash{}{0pt}%
\pgfsys@defobject{currentmarker}{\pgfqpoint{-0.048611in}{0.000000in}}{\pgfqpoint{-0.000000in}{0.000000in}}{%
\pgfpathmoveto{\pgfqpoint{-0.000000in}{0.000000in}}%
\pgfpathlineto{\pgfqpoint{-0.048611in}{0.000000in}}%
\pgfusepath{stroke,fill}%
}%
\begin{pgfscope}%
\pgfsys@transformshift{7.622482in}{4.717165in}%
\pgfsys@useobject{currentmarker}{}%
\end{pgfscope}%
\end{pgfscope}%
\begin{pgfscope}%
\pgfsetbuttcap%
\pgfsetroundjoin%
\definecolor{currentfill}{rgb}{0.000000,0.000000,0.000000}%
\pgfsetfillcolor{currentfill}%
\pgfsetlinewidth{0.803000pt}%
\definecolor{currentstroke}{rgb}{0.000000,0.000000,0.000000}%
\pgfsetstrokecolor{currentstroke}%
\pgfsetdash{}{0pt}%
\pgfsys@defobject{currentmarker}{\pgfqpoint{-0.048611in}{0.000000in}}{\pgfqpoint{-0.000000in}{0.000000in}}{%
\pgfpathmoveto{\pgfqpoint{-0.000000in}{0.000000in}}%
\pgfpathlineto{\pgfqpoint{-0.048611in}{0.000000in}}%
\pgfusepath{stroke,fill}%
}%
\begin{pgfscope}%
\pgfsys@transformshift{7.622482in}{5.056419in}%
\pgfsys@useobject{currentmarker}{}%
\end{pgfscope}%
\end{pgfscope}%
\begin{pgfscope}%
\pgfsetrectcap%
\pgfsetmiterjoin%
\pgfsetlinewidth{0.803000pt}%
\definecolor{currentstroke}{rgb}{0.000000,0.000000,0.000000}%
\pgfsetstrokecolor{currentstroke}%
\pgfsetdash{}{0pt}%
\pgfpathmoveto{\pgfqpoint{7.622482in}{2.920818in}}%
\pgfpathlineto{\pgfqpoint{7.622482in}{5.122573in}}%
\pgfusepath{stroke}%
\end{pgfscope}%
\begin{pgfscope}%
\pgfsetrectcap%
\pgfsetmiterjoin%
\pgfsetlinewidth{0.803000pt}%
\definecolor{currentstroke}{rgb}{0.000000,0.000000,0.000000}%
\pgfsetstrokecolor{currentstroke}%
\pgfsetdash{}{0pt}%
\pgfpathmoveto{\pgfqpoint{7.622482in}{2.920818in}}%
\pgfpathlineto{\pgfqpoint{9.799762in}{2.920818in}}%
\pgfusepath{stroke}%
\end{pgfscope}%
\begin{pgfscope}%
\pgfsetbuttcap%
\pgfsetmiterjoin%
\definecolor{currentfill}{rgb}{1.000000,1.000000,1.000000}%
\pgfsetfillcolor{currentfill}%
\pgfsetlinewidth{0.000000pt}%
\definecolor{currentstroke}{rgb}{0.000000,0.000000,0.000000}%
\pgfsetstrokecolor{currentstroke}%
\pgfsetstrokeopacity{0.000000}%
\pgfsetdash{}{0pt}%
\pgfpathmoveto{\pgfqpoint{0.633874in}{0.569136in}}%
\pgfpathlineto{\pgfqpoint{2.811154in}{0.569136in}}%
\pgfpathlineto{\pgfqpoint{2.811154in}{2.770891in}}%
\pgfpathlineto{\pgfqpoint{0.633874in}{2.770891in}}%
\pgfpathlineto{\pgfqpoint{0.633874in}{0.569136in}}%
\pgfpathclose%
\pgfusepath{fill}%
\end{pgfscope}%
\begin{pgfscope}%
\pgfpathrectangle{\pgfqpoint{0.633874in}{0.569136in}}{\pgfqpoint{2.177280in}{2.201755in}}%
\pgfusepath{clip}%
\pgfsetbuttcap%
\pgfsetroundjoin%
\definecolor{currentfill}{rgb}{0.121569,0.466667,0.705882}%
\pgfsetfillcolor{currentfill}%
\pgfsetlinewidth{0.481800pt}%
\definecolor{currentstroke}{rgb}{1.000000,1.000000,1.000000}%
\pgfsetstrokecolor{currentstroke}%
\pgfsetdash{}{0pt}%
\pgfpathmoveto{\pgfqpoint{1.245489in}{0.710948in}}%
\pgfpathcurveto{\pgfqpoint{1.256539in}{0.710948in}}{\pgfqpoint{1.267138in}{0.715339in}}{\pgfqpoint{1.274952in}{0.723152in}}%
\pgfpathcurveto{\pgfqpoint{1.282765in}{0.730966in}}{\pgfqpoint{1.287155in}{0.741565in}}{\pgfqpoint{1.287155in}{0.752615in}}%
\pgfpathcurveto{\pgfqpoint{1.287155in}{0.763665in}}{\pgfqpoint{1.282765in}{0.774264in}}{\pgfqpoint{1.274952in}{0.782078in}}%
\pgfpathcurveto{\pgfqpoint{1.267138in}{0.789892in}}{\pgfqpoint{1.256539in}{0.794282in}}{\pgfqpoint{1.245489in}{0.794282in}}%
\pgfpathcurveto{\pgfqpoint{1.234439in}{0.794282in}}{\pgfqpoint{1.223840in}{0.789892in}}{\pgfqpoint{1.216026in}{0.782078in}}%
\pgfpathcurveto{\pgfqpoint{1.208212in}{0.774264in}}{\pgfqpoint{1.203822in}{0.763665in}}{\pgfqpoint{1.203822in}{0.752615in}}%
\pgfpathcurveto{\pgfqpoint{1.203822in}{0.741565in}}{\pgfqpoint{1.208212in}{0.730966in}}{\pgfqpoint{1.216026in}{0.723152in}}%
\pgfpathcurveto{\pgfqpoint{1.223840in}{0.715339in}}{\pgfqpoint{1.234439in}{0.710948in}}{\pgfqpoint{1.245489in}{0.710948in}}%
\pgfpathlineto{\pgfqpoint{1.245489in}{0.710948in}}%
\pgfpathclose%
\pgfusepath{stroke,fill}%
\end{pgfscope}%
\begin{pgfscope}%
\pgfpathrectangle{\pgfqpoint{0.633874in}{0.569136in}}{\pgfqpoint{2.177280in}{2.201755in}}%
\pgfusepath{clip}%
\pgfsetbuttcap%
\pgfsetroundjoin%
\definecolor{currentfill}{rgb}{0.121569,0.466667,0.705882}%
\pgfsetfillcolor{currentfill}%
\pgfsetlinewidth{0.481800pt}%
\definecolor{currentstroke}{rgb}{1.000000,1.000000,1.000000}%
\pgfsetstrokecolor{currentstroke}%
\pgfsetdash{}{0pt}%
\pgfpathmoveto{\pgfqpoint{1.165611in}{0.710948in}}%
\pgfpathcurveto{\pgfqpoint{1.176662in}{0.710948in}}{\pgfqpoint{1.187261in}{0.715339in}}{\pgfqpoint{1.195074in}{0.723152in}}%
\pgfpathcurveto{\pgfqpoint{1.202888in}{0.730966in}}{\pgfqpoint{1.207278in}{0.741565in}}{\pgfqpoint{1.207278in}{0.752615in}}%
\pgfpathcurveto{\pgfqpoint{1.207278in}{0.763665in}}{\pgfqpoint{1.202888in}{0.774264in}}{\pgfqpoint{1.195074in}{0.782078in}}%
\pgfpathcurveto{\pgfqpoint{1.187261in}{0.789892in}}{\pgfqpoint{1.176662in}{0.794282in}}{\pgfqpoint{1.165611in}{0.794282in}}%
\pgfpathcurveto{\pgfqpoint{1.154561in}{0.794282in}}{\pgfqpoint{1.143962in}{0.789892in}}{\pgfqpoint{1.136149in}{0.782078in}}%
\pgfpathcurveto{\pgfqpoint{1.128335in}{0.774264in}}{\pgfqpoint{1.123945in}{0.763665in}}{\pgfqpoint{1.123945in}{0.752615in}}%
\pgfpathcurveto{\pgfqpoint{1.123945in}{0.741565in}}{\pgfqpoint{1.128335in}{0.730966in}}{\pgfqpoint{1.136149in}{0.723152in}}%
\pgfpathcurveto{\pgfqpoint{1.143962in}{0.715339in}}{\pgfqpoint{1.154561in}{0.710948in}}{\pgfqpoint{1.165611in}{0.710948in}}%
\pgfpathlineto{\pgfqpoint{1.165611in}{0.710948in}}%
\pgfpathclose%
\pgfusepath{stroke,fill}%
\end{pgfscope}%
\begin{pgfscope}%
\pgfpathrectangle{\pgfqpoint{0.633874in}{0.569136in}}{\pgfqpoint{2.177280in}{2.201755in}}%
\pgfusepath{clip}%
\pgfsetbuttcap%
\pgfsetroundjoin%
\definecolor{currentfill}{rgb}{0.121569,0.466667,0.705882}%
\pgfsetfillcolor{currentfill}%
\pgfsetlinewidth{0.481800pt}%
\definecolor{currentstroke}{rgb}{1.000000,1.000000,1.000000}%
\pgfsetstrokecolor{currentstroke}%
\pgfsetdash{}{0pt}%
\pgfpathmoveto{\pgfqpoint{1.085734in}{0.710948in}}%
\pgfpathcurveto{\pgfqpoint{1.096784in}{0.710948in}}{\pgfqpoint{1.107383in}{0.715339in}}{\pgfqpoint{1.115197in}{0.723152in}}%
\pgfpathcurveto{\pgfqpoint{1.123010in}{0.730966in}}{\pgfqpoint{1.127401in}{0.741565in}}{\pgfqpoint{1.127401in}{0.752615in}}%
\pgfpathcurveto{\pgfqpoint{1.127401in}{0.763665in}}{\pgfqpoint{1.123010in}{0.774264in}}{\pgfqpoint{1.115197in}{0.782078in}}%
\pgfpathcurveto{\pgfqpoint{1.107383in}{0.789892in}}{\pgfqpoint{1.096784in}{0.794282in}}{\pgfqpoint{1.085734in}{0.794282in}}%
\pgfpathcurveto{\pgfqpoint{1.074684in}{0.794282in}}{\pgfqpoint{1.064085in}{0.789892in}}{\pgfqpoint{1.056271in}{0.782078in}}%
\pgfpathcurveto{\pgfqpoint{1.048458in}{0.774264in}}{\pgfqpoint{1.044067in}{0.763665in}}{\pgfqpoint{1.044067in}{0.752615in}}%
\pgfpathcurveto{\pgfqpoint{1.044067in}{0.741565in}}{\pgfqpoint{1.048458in}{0.730966in}}{\pgfqpoint{1.056271in}{0.723152in}}%
\pgfpathcurveto{\pgfqpoint{1.064085in}{0.715339in}}{\pgfqpoint{1.074684in}{0.710948in}}{\pgfqpoint{1.085734in}{0.710948in}}%
\pgfpathlineto{\pgfqpoint{1.085734in}{0.710948in}}%
\pgfpathclose%
\pgfusepath{stroke,fill}%
\end{pgfscope}%
\begin{pgfscope}%
\pgfpathrectangle{\pgfqpoint{0.633874in}{0.569136in}}{\pgfqpoint{2.177280in}{2.201755in}}%
\pgfusepath{clip}%
\pgfsetbuttcap%
\pgfsetroundjoin%
\definecolor{currentfill}{rgb}{0.121569,0.466667,0.705882}%
\pgfsetfillcolor{currentfill}%
\pgfsetlinewidth{0.481800pt}%
\definecolor{currentstroke}{rgb}{1.000000,1.000000,1.000000}%
\pgfsetstrokecolor{currentstroke}%
\pgfsetdash{}{0pt}%
\pgfpathmoveto{\pgfqpoint{1.045795in}{0.710948in}}%
\pgfpathcurveto{\pgfqpoint{1.056845in}{0.710948in}}{\pgfqpoint{1.067444in}{0.715339in}}{\pgfqpoint{1.075258in}{0.723152in}}%
\pgfpathcurveto{\pgfqpoint{1.083072in}{0.730966in}}{\pgfqpoint{1.087462in}{0.741565in}}{\pgfqpoint{1.087462in}{0.752615in}}%
\pgfpathcurveto{\pgfqpoint{1.087462in}{0.763665in}}{\pgfqpoint{1.083072in}{0.774264in}}{\pgfqpoint{1.075258in}{0.782078in}}%
\pgfpathcurveto{\pgfqpoint{1.067444in}{0.789892in}}{\pgfqpoint{1.056845in}{0.794282in}}{\pgfqpoint{1.045795in}{0.794282in}}%
\pgfpathcurveto{\pgfqpoint{1.034745in}{0.794282in}}{\pgfqpoint{1.024146in}{0.789892in}}{\pgfqpoint{1.016332in}{0.782078in}}%
\pgfpathcurveto{\pgfqpoint{1.008519in}{0.774264in}}{\pgfqpoint{1.004129in}{0.763665in}}{\pgfqpoint{1.004129in}{0.752615in}}%
\pgfpathcurveto{\pgfqpoint{1.004129in}{0.741565in}}{\pgfqpoint{1.008519in}{0.730966in}}{\pgfqpoint{1.016332in}{0.723152in}}%
\pgfpathcurveto{\pgfqpoint{1.024146in}{0.715339in}}{\pgfqpoint{1.034745in}{0.710948in}}{\pgfqpoint{1.045795in}{0.710948in}}%
\pgfpathlineto{\pgfqpoint{1.045795in}{0.710948in}}%
\pgfpathclose%
\pgfusepath{stroke,fill}%
\end{pgfscope}%
\begin{pgfscope}%
\pgfpathrectangle{\pgfqpoint{0.633874in}{0.569136in}}{\pgfqpoint{2.177280in}{2.201755in}}%
\pgfusepath{clip}%
\pgfsetbuttcap%
\pgfsetroundjoin%
\definecolor{currentfill}{rgb}{0.121569,0.466667,0.705882}%
\pgfsetfillcolor{currentfill}%
\pgfsetlinewidth{0.481800pt}%
\definecolor{currentstroke}{rgb}{1.000000,1.000000,1.000000}%
\pgfsetstrokecolor{currentstroke}%
\pgfsetdash{}{0pt}%
\pgfpathmoveto{\pgfqpoint{1.205550in}{0.710948in}}%
\pgfpathcurveto{\pgfqpoint{1.216600in}{0.710948in}}{\pgfqpoint{1.227199in}{0.715339in}}{\pgfqpoint{1.235013in}{0.723152in}}%
\pgfpathcurveto{\pgfqpoint{1.242826in}{0.730966in}}{\pgfqpoint{1.247217in}{0.741565in}}{\pgfqpoint{1.247217in}{0.752615in}}%
\pgfpathcurveto{\pgfqpoint{1.247217in}{0.763665in}}{\pgfqpoint{1.242826in}{0.774264in}}{\pgfqpoint{1.235013in}{0.782078in}}%
\pgfpathcurveto{\pgfqpoint{1.227199in}{0.789892in}}{\pgfqpoint{1.216600in}{0.794282in}}{\pgfqpoint{1.205550in}{0.794282in}}%
\pgfpathcurveto{\pgfqpoint{1.194500in}{0.794282in}}{\pgfqpoint{1.183901in}{0.789892in}}{\pgfqpoint{1.176087in}{0.782078in}}%
\pgfpathcurveto{\pgfqpoint{1.168274in}{0.774264in}}{\pgfqpoint{1.163883in}{0.763665in}}{\pgfqpoint{1.163883in}{0.752615in}}%
\pgfpathcurveto{\pgfqpoint{1.163883in}{0.741565in}}{\pgfqpoint{1.168274in}{0.730966in}}{\pgfqpoint{1.176087in}{0.723152in}}%
\pgfpathcurveto{\pgfqpoint{1.183901in}{0.715339in}}{\pgfqpoint{1.194500in}{0.710948in}}{\pgfqpoint{1.205550in}{0.710948in}}%
\pgfpathlineto{\pgfqpoint{1.205550in}{0.710948in}}%
\pgfpathclose%
\pgfusepath{stroke,fill}%
\end{pgfscope}%
\begin{pgfscope}%
\pgfpathrectangle{\pgfqpoint{0.633874in}{0.569136in}}{\pgfqpoint{2.177280in}{2.201755in}}%
\pgfusepath{clip}%
\pgfsetbuttcap%
\pgfsetroundjoin%
\definecolor{currentfill}{rgb}{0.121569,0.466667,0.705882}%
\pgfsetfillcolor{currentfill}%
\pgfsetlinewidth{0.481800pt}%
\definecolor{currentstroke}{rgb}{1.000000,1.000000,1.000000}%
\pgfsetstrokecolor{currentstroke}%
\pgfsetdash{}{0pt}%
\pgfpathmoveto{\pgfqpoint{1.365305in}{0.877748in}}%
\pgfpathcurveto{\pgfqpoint{1.376355in}{0.877748in}}{\pgfqpoint{1.386954in}{0.882138in}}{\pgfqpoint{1.394768in}{0.889952in}}%
\pgfpathcurveto{\pgfqpoint{1.402581in}{0.897766in}}{\pgfqpoint{1.406972in}{0.908365in}}{\pgfqpoint{1.406972in}{0.919415in}}%
\pgfpathcurveto{\pgfqpoint{1.406972in}{0.930465in}}{\pgfqpoint{1.402581in}{0.941064in}}{\pgfqpoint{1.394768in}{0.948878in}}%
\pgfpathcurveto{\pgfqpoint{1.386954in}{0.956691in}}{\pgfqpoint{1.376355in}{0.961081in}}{\pgfqpoint{1.365305in}{0.961081in}}%
\pgfpathcurveto{\pgfqpoint{1.354255in}{0.961081in}}{\pgfqpoint{1.343656in}{0.956691in}}{\pgfqpoint{1.335842in}{0.948878in}}%
\pgfpathcurveto{\pgfqpoint{1.328029in}{0.941064in}}{\pgfqpoint{1.323638in}{0.930465in}}{\pgfqpoint{1.323638in}{0.919415in}}%
\pgfpathcurveto{\pgfqpoint{1.323638in}{0.908365in}}{\pgfqpoint{1.328029in}{0.897766in}}{\pgfqpoint{1.335842in}{0.889952in}}%
\pgfpathcurveto{\pgfqpoint{1.343656in}{0.882138in}}{\pgfqpoint{1.354255in}{0.877748in}}{\pgfqpoint{1.365305in}{0.877748in}}%
\pgfpathlineto{\pgfqpoint{1.365305in}{0.877748in}}%
\pgfpathclose%
\pgfusepath{stroke,fill}%
\end{pgfscope}%
\begin{pgfscope}%
\pgfpathrectangle{\pgfqpoint{0.633874in}{0.569136in}}{\pgfqpoint{2.177280in}{2.201755in}}%
\pgfusepath{clip}%
\pgfsetbuttcap%
\pgfsetroundjoin%
\definecolor{currentfill}{rgb}{0.121569,0.466667,0.705882}%
\pgfsetfillcolor{currentfill}%
\pgfsetlinewidth{0.481800pt}%
\definecolor{currentstroke}{rgb}{1.000000,1.000000,1.000000}%
\pgfsetstrokecolor{currentstroke}%
\pgfsetdash{}{0pt}%
\pgfpathmoveto{\pgfqpoint{1.045795in}{0.794348in}}%
\pgfpathcurveto{\pgfqpoint{1.056845in}{0.794348in}}{\pgfqpoint{1.067444in}{0.798739in}}{\pgfqpoint{1.075258in}{0.806552in}}%
\pgfpathcurveto{\pgfqpoint{1.083072in}{0.814366in}}{\pgfqpoint{1.087462in}{0.824965in}}{\pgfqpoint{1.087462in}{0.836015in}}%
\pgfpathcurveto{\pgfqpoint{1.087462in}{0.847065in}}{\pgfqpoint{1.083072in}{0.857664in}}{\pgfqpoint{1.075258in}{0.865478in}}%
\pgfpathcurveto{\pgfqpoint{1.067444in}{0.873291in}}{\pgfqpoint{1.056845in}{0.877682in}}{\pgfqpoint{1.045795in}{0.877682in}}%
\pgfpathcurveto{\pgfqpoint{1.034745in}{0.877682in}}{\pgfqpoint{1.024146in}{0.873291in}}{\pgfqpoint{1.016332in}{0.865478in}}%
\pgfpathcurveto{\pgfqpoint{1.008519in}{0.857664in}}{\pgfqpoint{1.004129in}{0.847065in}}{\pgfqpoint{1.004129in}{0.836015in}}%
\pgfpathcurveto{\pgfqpoint{1.004129in}{0.824965in}}{\pgfqpoint{1.008519in}{0.814366in}}{\pgfqpoint{1.016332in}{0.806552in}}%
\pgfpathcurveto{\pgfqpoint{1.024146in}{0.798739in}}{\pgfqpoint{1.034745in}{0.794348in}}{\pgfqpoint{1.045795in}{0.794348in}}%
\pgfpathlineto{\pgfqpoint{1.045795in}{0.794348in}}%
\pgfpathclose%
\pgfusepath{stroke,fill}%
\end{pgfscope}%
\begin{pgfscope}%
\pgfpathrectangle{\pgfqpoint{0.633874in}{0.569136in}}{\pgfqpoint{2.177280in}{2.201755in}}%
\pgfusepath{clip}%
\pgfsetbuttcap%
\pgfsetroundjoin%
\definecolor{currentfill}{rgb}{0.121569,0.466667,0.705882}%
\pgfsetfillcolor{currentfill}%
\pgfsetlinewidth{0.481800pt}%
\definecolor{currentstroke}{rgb}{1.000000,1.000000,1.000000}%
\pgfsetstrokecolor{currentstroke}%
\pgfsetdash{}{0pt}%
\pgfpathmoveto{\pgfqpoint{1.205550in}{0.710948in}}%
\pgfpathcurveto{\pgfqpoint{1.216600in}{0.710948in}}{\pgfqpoint{1.227199in}{0.715339in}}{\pgfqpoint{1.235013in}{0.723152in}}%
\pgfpathcurveto{\pgfqpoint{1.242826in}{0.730966in}}{\pgfqpoint{1.247217in}{0.741565in}}{\pgfqpoint{1.247217in}{0.752615in}}%
\pgfpathcurveto{\pgfqpoint{1.247217in}{0.763665in}}{\pgfqpoint{1.242826in}{0.774264in}}{\pgfqpoint{1.235013in}{0.782078in}}%
\pgfpathcurveto{\pgfqpoint{1.227199in}{0.789892in}}{\pgfqpoint{1.216600in}{0.794282in}}{\pgfqpoint{1.205550in}{0.794282in}}%
\pgfpathcurveto{\pgfqpoint{1.194500in}{0.794282in}}{\pgfqpoint{1.183901in}{0.789892in}}{\pgfqpoint{1.176087in}{0.782078in}}%
\pgfpathcurveto{\pgfqpoint{1.168274in}{0.774264in}}{\pgfqpoint{1.163883in}{0.763665in}}{\pgfqpoint{1.163883in}{0.752615in}}%
\pgfpathcurveto{\pgfqpoint{1.163883in}{0.741565in}}{\pgfqpoint{1.168274in}{0.730966in}}{\pgfqpoint{1.176087in}{0.723152in}}%
\pgfpathcurveto{\pgfqpoint{1.183901in}{0.715339in}}{\pgfqpoint{1.194500in}{0.710948in}}{\pgfqpoint{1.205550in}{0.710948in}}%
\pgfpathlineto{\pgfqpoint{1.205550in}{0.710948in}}%
\pgfpathclose%
\pgfusepath{stroke,fill}%
\end{pgfscope}%
\begin{pgfscope}%
\pgfpathrectangle{\pgfqpoint{0.633874in}{0.569136in}}{\pgfqpoint{2.177280in}{2.201755in}}%
\pgfusepath{clip}%
\pgfsetbuttcap%
\pgfsetroundjoin%
\definecolor{currentfill}{rgb}{0.121569,0.466667,0.705882}%
\pgfsetfillcolor{currentfill}%
\pgfsetlinewidth{0.481800pt}%
\definecolor{currentstroke}{rgb}{1.000000,1.000000,1.000000}%
\pgfsetstrokecolor{currentstroke}%
\pgfsetdash{}{0pt}%
\pgfpathmoveto{\pgfqpoint{0.965918in}{0.710948in}}%
\pgfpathcurveto{\pgfqpoint{0.976968in}{0.710948in}}{\pgfqpoint{0.987567in}{0.715339in}}{\pgfqpoint{0.995381in}{0.723152in}}%
\pgfpathcurveto{\pgfqpoint{1.003194in}{0.730966in}}{\pgfqpoint{1.007585in}{0.741565in}}{\pgfqpoint{1.007585in}{0.752615in}}%
\pgfpathcurveto{\pgfqpoint{1.007585in}{0.763665in}}{\pgfqpoint{1.003194in}{0.774264in}}{\pgfqpoint{0.995381in}{0.782078in}}%
\pgfpathcurveto{\pgfqpoint{0.987567in}{0.789892in}}{\pgfqpoint{0.976968in}{0.794282in}}{\pgfqpoint{0.965918in}{0.794282in}}%
\pgfpathcurveto{\pgfqpoint{0.954868in}{0.794282in}}{\pgfqpoint{0.944269in}{0.789892in}}{\pgfqpoint{0.936455in}{0.782078in}}%
\pgfpathcurveto{\pgfqpoint{0.928641in}{0.774264in}}{\pgfqpoint{0.924251in}{0.763665in}}{\pgfqpoint{0.924251in}{0.752615in}}%
\pgfpathcurveto{\pgfqpoint{0.924251in}{0.741565in}}{\pgfqpoint{0.928641in}{0.730966in}}{\pgfqpoint{0.936455in}{0.723152in}}%
\pgfpathcurveto{\pgfqpoint{0.944269in}{0.715339in}}{\pgfqpoint{0.954868in}{0.710948in}}{\pgfqpoint{0.965918in}{0.710948in}}%
\pgfpathlineto{\pgfqpoint{0.965918in}{0.710948in}}%
\pgfpathclose%
\pgfusepath{stroke,fill}%
\end{pgfscope}%
\begin{pgfscope}%
\pgfpathrectangle{\pgfqpoint{0.633874in}{0.569136in}}{\pgfqpoint{2.177280in}{2.201755in}}%
\pgfusepath{clip}%
\pgfsetbuttcap%
\pgfsetroundjoin%
\definecolor{currentfill}{rgb}{0.121569,0.466667,0.705882}%
\pgfsetfillcolor{currentfill}%
\pgfsetlinewidth{0.481800pt}%
\definecolor{currentstroke}{rgb}{1.000000,1.000000,1.000000}%
\pgfsetstrokecolor{currentstroke}%
\pgfsetdash{}{0pt}%
\pgfpathmoveto{\pgfqpoint{1.165611in}{0.627549in}}%
\pgfpathcurveto{\pgfqpoint{1.176662in}{0.627549in}}{\pgfqpoint{1.187261in}{0.631939in}}{\pgfqpoint{1.195074in}{0.639753in}}%
\pgfpathcurveto{\pgfqpoint{1.202888in}{0.647566in}}{\pgfqpoint{1.207278in}{0.658165in}}{\pgfqpoint{1.207278in}{0.669215in}}%
\pgfpathcurveto{\pgfqpoint{1.207278in}{0.680265in}}{\pgfqpoint{1.202888in}{0.690864in}}{\pgfqpoint{1.195074in}{0.698678in}}%
\pgfpathcurveto{\pgfqpoint{1.187261in}{0.706492in}}{\pgfqpoint{1.176662in}{0.710882in}}{\pgfqpoint{1.165611in}{0.710882in}}%
\pgfpathcurveto{\pgfqpoint{1.154561in}{0.710882in}}{\pgfqpoint{1.143962in}{0.706492in}}{\pgfqpoint{1.136149in}{0.698678in}}%
\pgfpathcurveto{\pgfqpoint{1.128335in}{0.690864in}}{\pgfqpoint{1.123945in}{0.680265in}}{\pgfqpoint{1.123945in}{0.669215in}}%
\pgfpathcurveto{\pgfqpoint{1.123945in}{0.658165in}}{\pgfqpoint{1.128335in}{0.647566in}}{\pgfqpoint{1.136149in}{0.639753in}}%
\pgfpathcurveto{\pgfqpoint{1.143962in}{0.631939in}}{\pgfqpoint{1.154561in}{0.627549in}}{\pgfqpoint{1.165611in}{0.627549in}}%
\pgfpathlineto{\pgfqpoint{1.165611in}{0.627549in}}%
\pgfpathclose%
\pgfusepath{stroke,fill}%
\end{pgfscope}%
\begin{pgfscope}%
\pgfpathrectangle{\pgfqpoint{0.633874in}{0.569136in}}{\pgfqpoint{2.177280in}{2.201755in}}%
\pgfusepath{clip}%
\pgfsetbuttcap%
\pgfsetroundjoin%
\definecolor{currentfill}{rgb}{0.121569,0.466667,0.705882}%
\pgfsetfillcolor{currentfill}%
\pgfsetlinewidth{0.481800pt}%
\definecolor{currentstroke}{rgb}{1.000000,1.000000,1.000000}%
\pgfsetstrokecolor{currentstroke}%
\pgfsetdash{}{0pt}%
\pgfpathmoveto{\pgfqpoint{1.365305in}{0.710948in}}%
\pgfpathcurveto{\pgfqpoint{1.376355in}{0.710948in}}{\pgfqpoint{1.386954in}{0.715339in}}{\pgfqpoint{1.394768in}{0.723152in}}%
\pgfpathcurveto{\pgfqpoint{1.402581in}{0.730966in}}{\pgfqpoint{1.406972in}{0.741565in}}{\pgfqpoint{1.406972in}{0.752615in}}%
\pgfpathcurveto{\pgfqpoint{1.406972in}{0.763665in}}{\pgfqpoint{1.402581in}{0.774264in}}{\pgfqpoint{1.394768in}{0.782078in}}%
\pgfpathcurveto{\pgfqpoint{1.386954in}{0.789892in}}{\pgfqpoint{1.376355in}{0.794282in}}{\pgfqpoint{1.365305in}{0.794282in}}%
\pgfpathcurveto{\pgfqpoint{1.354255in}{0.794282in}}{\pgfqpoint{1.343656in}{0.789892in}}{\pgfqpoint{1.335842in}{0.782078in}}%
\pgfpathcurveto{\pgfqpoint{1.328029in}{0.774264in}}{\pgfqpoint{1.323638in}{0.763665in}}{\pgfqpoint{1.323638in}{0.752615in}}%
\pgfpathcurveto{\pgfqpoint{1.323638in}{0.741565in}}{\pgfqpoint{1.328029in}{0.730966in}}{\pgfqpoint{1.335842in}{0.723152in}}%
\pgfpathcurveto{\pgfqpoint{1.343656in}{0.715339in}}{\pgfqpoint{1.354255in}{0.710948in}}{\pgfqpoint{1.365305in}{0.710948in}}%
\pgfpathlineto{\pgfqpoint{1.365305in}{0.710948in}}%
\pgfpathclose%
\pgfusepath{stroke,fill}%
\end{pgfscope}%
\begin{pgfscope}%
\pgfpathrectangle{\pgfqpoint{0.633874in}{0.569136in}}{\pgfqpoint{2.177280in}{2.201755in}}%
\pgfusepath{clip}%
\pgfsetbuttcap%
\pgfsetroundjoin%
\definecolor{currentfill}{rgb}{0.121569,0.466667,0.705882}%
\pgfsetfillcolor{currentfill}%
\pgfsetlinewidth{0.481800pt}%
\definecolor{currentstroke}{rgb}{1.000000,1.000000,1.000000}%
\pgfsetstrokecolor{currentstroke}%
\pgfsetdash{}{0pt}%
\pgfpathmoveto{\pgfqpoint{1.125673in}{0.710948in}}%
\pgfpathcurveto{\pgfqpoint{1.136723in}{0.710948in}}{\pgfqpoint{1.147322in}{0.715339in}}{\pgfqpoint{1.155135in}{0.723152in}}%
\pgfpathcurveto{\pgfqpoint{1.162949in}{0.730966in}}{\pgfqpoint{1.167339in}{0.741565in}}{\pgfqpoint{1.167339in}{0.752615in}}%
\pgfpathcurveto{\pgfqpoint{1.167339in}{0.763665in}}{\pgfqpoint{1.162949in}{0.774264in}}{\pgfqpoint{1.155135in}{0.782078in}}%
\pgfpathcurveto{\pgfqpoint{1.147322in}{0.789892in}}{\pgfqpoint{1.136723in}{0.794282in}}{\pgfqpoint{1.125673in}{0.794282in}}%
\pgfpathcurveto{\pgfqpoint{1.114623in}{0.794282in}}{\pgfqpoint{1.104024in}{0.789892in}}{\pgfqpoint{1.096210in}{0.782078in}}%
\pgfpathcurveto{\pgfqpoint{1.088396in}{0.774264in}}{\pgfqpoint{1.084006in}{0.763665in}}{\pgfqpoint{1.084006in}{0.752615in}}%
\pgfpathcurveto{\pgfqpoint{1.084006in}{0.741565in}}{\pgfqpoint{1.088396in}{0.730966in}}{\pgfqpoint{1.096210in}{0.723152in}}%
\pgfpathcurveto{\pgfqpoint{1.104024in}{0.715339in}}{\pgfqpoint{1.114623in}{0.710948in}}{\pgfqpoint{1.125673in}{0.710948in}}%
\pgfpathlineto{\pgfqpoint{1.125673in}{0.710948in}}%
\pgfpathclose%
\pgfusepath{stroke,fill}%
\end{pgfscope}%
\begin{pgfscope}%
\pgfpathrectangle{\pgfqpoint{0.633874in}{0.569136in}}{\pgfqpoint{2.177280in}{2.201755in}}%
\pgfusepath{clip}%
\pgfsetbuttcap%
\pgfsetroundjoin%
\definecolor{currentfill}{rgb}{0.121569,0.466667,0.705882}%
\pgfsetfillcolor{currentfill}%
\pgfsetlinewidth{0.481800pt}%
\definecolor{currentstroke}{rgb}{1.000000,1.000000,1.000000}%
\pgfsetstrokecolor{currentstroke}%
\pgfsetdash{}{0pt}%
\pgfpathmoveto{\pgfqpoint{1.125673in}{0.627549in}}%
\pgfpathcurveto{\pgfqpoint{1.136723in}{0.627549in}}{\pgfqpoint{1.147322in}{0.631939in}}{\pgfqpoint{1.155135in}{0.639753in}}%
\pgfpathcurveto{\pgfqpoint{1.162949in}{0.647566in}}{\pgfqpoint{1.167339in}{0.658165in}}{\pgfqpoint{1.167339in}{0.669215in}}%
\pgfpathcurveto{\pgfqpoint{1.167339in}{0.680265in}}{\pgfqpoint{1.162949in}{0.690864in}}{\pgfqpoint{1.155135in}{0.698678in}}%
\pgfpathcurveto{\pgfqpoint{1.147322in}{0.706492in}}{\pgfqpoint{1.136723in}{0.710882in}}{\pgfqpoint{1.125673in}{0.710882in}}%
\pgfpathcurveto{\pgfqpoint{1.114623in}{0.710882in}}{\pgfqpoint{1.104024in}{0.706492in}}{\pgfqpoint{1.096210in}{0.698678in}}%
\pgfpathcurveto{\pgfqpoint{1.088396in}{0.690864in}}{\pgfqpoint{1.084006in}{0.680265in}}{\pgfqpoint{1.084006in}{0.669215in}}%
\pgfpathcurveto{\pgfqpoint{1.084006in}{0.658165in}}{\pgfqpoint{1.088396in}{0.647566in}}{\pgfqpoint{1.096210in}{0.639753in}}%
\pgfpathcurveto{\pgfqpoint{1.104024in}{0.631939in}}{\pgfqpoint{1.114623in}{0.627549in}}{\pgfqpoint{1.125673in}{0.627549in}}%
\pgfpathlineto{\pgfqpoint{1.125673in}{0.627549in}}%
\pgfpathclose%
\pgfusepath{stroke,fill}%
\end{pgfscope}%
\begin{pgfscope}%
\pgfpathrectangle{\pgfqpoint{0.633874in}{0.569136in}}{\pgfqpoint{2.177280in}{2.201755in}}%
\pgfusepath{clip}%
\pgfsetbuttcap%
\pgfsetroundjoin%
\definecolor{currentfill}{rgb}{0.121569,0.466667,0.705882}%
\pgfsetfillcolor{currentfill}%
\pgfsetlinewidth{0.481800pt}%
\definecolor{currentstroke}{rgb}{1.000000,1.000000,1.000000}%
\pgfsetstrokecolor{currentstroke}%
\pgfsetdash{}{0pt}%
\pgfpathmoveto{\pgfqpoint{0.925979in}{0.627549in}}%
\pgfpathcurveto{\pgfqpoint{0.937029in}{0.627549in}}{\pgfqpoint{0.947628in}{0.631939in}}{\pgfqpoint{0.955442in}{0.639753in}}%
\pgfpathcurveto{\pgfqpoint{0.963256in}{0.647566in}}{\pgfqpoint{0.967646in}{0.658165in}}{\pgfqpoint{0.967646in}{0.669215in}}%
\pgfpathcurveto{\pgfqpoint{0.967646in}{0.680265in}}{\pgfqpoint{0.963256in}{0.690864in}}{\pgfqpoint{0.955442in}{0.698678in}}%
\pgfpathcurveto{\pgfqpoint{0.947628in}{0.706492in}}{\pgfqpoint{0.937029in}{0.710882in}}{\pgfqpoint{0.925979in}{0.710882in}}%
\pgfpathcurveto{\pgfqpoint{0.914929in}{0.710882in}}{\pgfqpoint{0.904330in}{0.706492in}}{\pgfqpoint{0.896516in}{0.698678in}}%
\pgfpathcurveto{\pgfqpoint{0.888703in}{0.690864in}}{\pgfqpoint{0.884313in}{0.680265in}}{\pgfqpoint{0.884313in}{0.669215in}}%
\pgfpathcurveto{\pgfqpoint{0.884313in}{0.658165in}}{\pgfqpoint{0.888703in}{0.647566in}}{\pgfqpoint{0.896516in}{0.639753in}}%
\pgfpathcurveto{\pgfqpoint{0.904330in}{0.631939in}}{\pgfqpoint{0.914929in}{0.627549in}}{\pgfqpoint{0.925979in}{0.627549in}}%
\pgfpathlineto{\pgfqpoint{0.925979in}{0.627549in}}%
\pgfpathclose%
\pgfusepath{stroke,fill}%
\end{pgfscope}%
\begin{pgfscope}%
\pgfpathrectangle{\pgfqpoint{0.633874in}{0.569136in}}{\pgfqpoint{2.177280in}{2.201755in}}%
\pgfusepath{clip}%
\pgfsetbuttcap%
\pgfsetroundjoin%
\definecolor{currentfill}{rgb}{0.121569,0.466667,0.705882}%
\pgfsetfillcolor{currentfill}%
\pgfsetlinewidth{0.481800pt}%
\definecolor{currentstroke}{rgb}{1.000000,1.000000,1.000000}%
\pgfsetstrokecolor{currentstroke}%
\pgfsetdash{}{0pt}%
\pgfpathmoveto{\pgfqpoint{1.525060in}{0.710948in}}%
\pgfpathcurveto{\pgfqpoint{1.536110in}{0.710948in}}{\pgfqpoint{1.546709in}{0.715339in}}{\pgfqpoint{1.554523in}{0.723152in}}%
\pgfpathcurveto{\pgfqpoint{1.562336in}{0.730966in}}{\pgfqpoint{1.566726in}{0.741565in}}{\pgfqpoint{1.566726in}{0.752615in}}%
\pgfpathcurveto{\pgfqpoint{1.566726in}{0.763665in}}{\pgfqpoint{1.562336in}{0.774264in}}{\pgfqpoint{1.554523in}{0.782078in}}%
\pgfpathcurveto{\pgfqpoint{1.546709in}{0.789892in}}{\pgfqpoint{1.536110in}{0.794282in}}{\pgfqpoint{1.525060in}{0.794282in}}%
\pgfpathcurveto{\pgfqpoint{1.514010in}{0.794282in}}{\pgfqpoint{1.503411in}{0.789892in}}{\pgfqpoint{1.495597in}{0.782078in}}%
\pgfpathcurveto{\pgfqpoint{1.487783in}{0.774264in}}{\pgfqpoint{1.483393in}{0.763665in}}{\pgfqpoint{1.483393in}{0.752615in}}%
\pgfpathcurveto{\pgfqpoint{1.483393in}{0.741565in}}{\pgfqpoint{1.487783in}{0.730966in}}{\pgfqpoint{1.495597in}{0.723152in}}%
\pgfpathcurveto{\pgfqpoint{1.503411in}{0.715339in}}{\pgfqpoint{1.514010in}{0.710948in}}{\pgfqpoint{1.525060in}{0.710948in}}%
\pgfpathlineto{\pgfqpoint{1.525060in}{0.710948in}}%
\pgfpathclose%
\pgfusepath{stroke,fill}%
\end{pgfscope}%
\begin{pgfscope}%
\pgfpathrectangle{\pgfqpoint{0.633874in}{0.569136in}}{\pgfqpoint{2.177280in}{2.201755in}}%
\pgfusepath{clip}%
\pgfsetbuttcap%
\pgfsetroundjoin%
\definecolor{currentfill}{rgb}{0.121569,0.466667,0.705882}%
\pgfsetfillcolor{currentfill}%
\pgfsetlinewidth{0.481800pt}%
\definecolor{currentstroke}{rgb}{1.000000,1.000000,1.000000}%
\pgfsetstrokecolor{currentstroke}%
\pgfsetdash{}{0pt}%
\pgfpathmoveto{\pgfqpoint{1.485121in}{0.877748in}}%
\pgfpathcurveto{\pgfqpoint{1.496171in}{0.877748in}}{\pgfqpoint{1.506770in}{0.882138in}}{\pgfqpoint{1.514584in}{0.889952in}}%
\pgfpathcurveto{\pgfqpoint{1.522397in}{0.897766in}}{\pgfqpoint{1.526788in}{0.908365in}}{\pgfqpoint{1.526788in}{0.919415in}}%
\pgfpathcurveto{\pgfqpoint{1.526788in}{0.930465in}}{\pgfqpoint{1.522397in}{0.941064in}}{\pgfqpoint{1.514584in}{0.948878in}}%
\pgfpathcurveto{\pgfqpoint{1.506770in}{0.956691in}}{\pgfqpoint{1.496171in}{0.961081in}}{\pgfqpoint{1.485121in}{0.961081in}}%
\pgfpathcurveto{\pgfqpoint{1.474071in}{0.961081in}}{\pgfqpoint{1.463472in}{0.956691in}}{\pgfqpoint{1.455658in}{0.948878in}}%
\pgfpathcurveto{\pgfqpoint{1.447845in}{0.941064in}}{\pgfqpoint{1.443454in}{0.930465in}}{\pgfqpoint{1.443454in}{0.919415in}}%
\pgfpathcurveto{\pgfqpoint{1.443454in}{0.908365in}}{\pgfqpoint{1.447845in}{0.897766in}}{\pgfqpoint{1.455658in}{0.889952in}}%
\pgfpathcurveto{\pgfqpoint{1.463472in}{0.882138in}}{\pgfqpoint{1.474071in}{0.877748in}}{\pgfqpoint{1.485121in}{0.877748in}}%
\pgfpathlineto{\pgfqpoint{1.485121in}{0.877748in}}%
\pgfpathclose%
\pgfusepath{stroke,fill}%
\end{pgfscope}%
\begin{pgfscope}%
\pgfpathrectangle{\pgfqpoint{0.633874in}{0.569136in}}{\pgfqpoint{2.177280in}{2.201755in}}%
\pgfusepath{clip}%
\pgfsetbuttcap%
\pgfsetroundjoin%
\definecolor{currentfill}{rgb}{0.121569,0.466667,0.705882}%
\pgfsetfillcolor{currentfill}%
\pgfsetlinewidth{0.481800pt}%
\definecolor{currentstroke}{rgb}{1.000000,1.000000,1.000000}%
\pgfsetstrokecolor{currentstroke}%
\pgfsetdash{}{0pt}%
\pgfpathmoveto{\pgfqpoint{1.365305in}{0.877748in}}%
\pgfpathcurveto{\pgfqpoint{1.376355in}{0.877748in}}{\pgfqpoint{1.386954in}{0.882138in}}{\pgfqpoint{1.394768in}{0.889952in}}%
\pgfpathcurveto{\pgfqpoint{1.402581in}{0.897766in}}{\pgfqpoint{1.406972in}{0.908365in}}{\pgfqpoint{1.406972in}{0.919415in}}%
\pgfpathcurveto{\pgfqpoint{1.406972in}{0.930465in}}{\pgfqpoint{1.402581in}{0.941064in}}{\pgfqpoint{1.394768in}{0.948878in}}%
\pgfpathcurveto{\pgfqpoint{1.386954in}{0.956691in}}{\pgfqpoint{1.376355in}{0.961081in}}{\pgfqpoint{1.365305in}{0.961081in}}%
\pgfpathcurveto{\pgfqpoint{1.354255in}{0.961081in}}{\pgfqpoint{1.343656in}{0.956691in}}{\pgfqpoint{1.335842in}{0.948878in}}%
\pgfpathcurveto{\pgfqpoint{1.328029in}{0.941064in}}{\pgfqpoint{1.323638in}{0.930465in}}{\pgfqpoint{1.323638in}{0.919415in}}%
\pgfpathcurveto{\pgfqpoint{1.323638in}{0.908365in}}{\pgfqpoint{1.328029in}{0.897766in}}{\pgfqpoint{1.335842in}{0.889952in}}%
\pgfpathcurveto{\pgfqpoint{1.343656in}{0.882138in}}{\pgfqpoint{1.354255in}{0.877748in}}{\pgfqpoint{1.365305in}{0.877748in}}%
\pgfpathlineto{\pgfqpoint{1.365305in}{0.877748in}}%
\pgfpathclose%
\pgfusepath{stroke,fill}%
\end{pgfscope}%
\begin{pgfscope}%
\pgfpathrectangle{\pgfqpoint{0.633874in}{0.569136in}}{\pgfqpoint{2.177280in}{2.201755in}}%
\pgfusepath{clip}%
\pgfsetbuttcap%
\pgfsetroundjoin%
\definecolor{currentfill}{rgb}{0.121569,0.466667,0.705882}%
\pgfsetfillcolor{currentfill}%
\pgfsetlinewidth{0.481800pt}%
\definecolor{currentstroke}{rgb}{1.000000,1.000000,1.000000}%
\pgfsetstrokecolor{currentstroke}%
\pgfsetdash{}{0pt}%
\pgfpathmoveto{\pgfqpoint{1.245489in}{0.794348in}}%
\pgfpathcurveto{\pgfqpoint{1.256539in}{0.794348in}}{\pgfqpoint{1.267138in}{0.798739in}}{\pgfqpoint{1.274952in}{0.806552in}}%
\pgfpathcurveto{\pgfqpoint{1.282765in}{0.814366in}}{\pgfqpoint{1.287155in}{0.824965in}}{\pgfqpoint{1.287155in}{0.836015in}}%
\pgfpathcurveto{\pgfqpoint{1.287155in}{0.847065in}}{\pgfqpoint{1.282765in}{0.857664in}}{\pgfqpoint{1.274952in}{0.865478in}}%
\pgfpathcurveto{\pgfqpoint{1.267138in}{0.873291in}}{\pgfqpoint{1.256539in}{0.877682in}}{\pgfqpoint{1.245489in}{0.877682in}}%
\pgfpathcurveto{\pgfqpoint{1.234439in}{0.877682in}}{\pgfqpoint{1.223840in}{0.873291in}}{\pgfqpoint{1.216026in}{0.865478in}}%
\pgfpathcurveto{\pgfqpoint{1.208212in}{0.857664in}}{\pgfqpoint{1.203822in}{0.847065in}}{\pgfqpoint{1.203822in}{0.836015in}}%
\pgfpathcurveto{\pgfqpoint{1.203822in}{0.824965in}}{\pgfqpoint{1.208212in}{0.814366in}}{\pgfqpoint{1.216026in}{0.806552in}}%
\pgfpathcurveto{\pgfqpoint{1.223840in}{0.798739in}}{\pgfqpoint{1.234439in}{0.794348in}}{\pgfqpoint{1.245489in}{0.794348in}}%
\pgfpathlineto{\pgfqpoint{1.245489in}{0.794348in}}%
\pgfpathclose%
\pgfusepath{stroke,fill}%
\end{pgfscope}%
\begin{pgfscope}%
\pgfpathrectangle{\pgfqpoint{0.633874in}{0.569136in}}{\pgfqpoint{2.177280in}{2.201755in}}%
\pgfusepath{clip}%
\pgfsetbuttcap%
\pgfsetroundjoin%
\definecolor{currentfill}{rgb}{0.121569,0.466667,0.705882}%
\pgfsetfillcolor{currentfill}%
\pgfsetlinewidth{0.481800pt}%
\definecolor{currentstroke}{rgb}{1.000000,1.000000,1.000000}%
\pgfsetstrokecolor{currentstroke}%
\pgfsetdash{}{0pt}%
\pgfpathmoveto{\pgfqpoint{1.485121in}{0.794348in}}%
\pgfpathcurveto{\pgfqpoint{1.496171in}{0.794348in}}{\pgfqpoint{1.506770in}{0.798739in}}{\pgfqpoint{1.514584in}{0.806552in}}%
\pgfpathcurveto{\pgfqpoint{1.522397in}{0.814366in}}{\pgfqpoint{1.526788in}{0.824965in}}{\pgfqpoint{1.526788in}{0.836015in}}%
\pgfpathcurveto{\pgfqpoint{1.526788in}{0.847065in}}{\pgfqpoint{1.522397in}{0.857664in}}{\pgfqpoint{1.514584in}{0.865478in}}%
\pgfpathcurveto{\pgfqpoint{1.506770in}{0.873291in}}{\pgfqpoint{1.496171in}{0.877682in}}{\pgfqpoint{1.485121in}{0.877682in}}%
\pgfpathcurveto{\pgfqpoint{1.474071in}{0.877682in}}{\pgfqpoint{1.463472in}{0.873291in}}{\pgfqpoint{1.455658in}{0.865478in}}%
\pgfpathcurveto{\pgfqpoint{1.447845in}{0.857664in}}{\pgfqpoint{1.443454in}{0.847065in}}{\pgfqpoint{1.443454in}{0.836015in}}%
\pgfpathcurveto{\pgfqpoint{1.443454in}{0.824965in}}{\pgfqpoint{1.447845in}{0.814366in}}{\pgfqpoint{1.455658in}{0.806552in}}%
\pgfpathcurveto{\pgfqpoint{1.463472in}{0.798739in}}{\pgfqpoint{1.474071in}{0.794348in}}{\pgfqpoint{1.485121in}{0.794348in}}%
\pgfpathlineto{\pgfqpoint{1.485121in}{0.794348in}}%
\pgfpathclose%
\pgfusepath{stroke,fill}%
\end{pgfscope}%
\begin{pgfscope}%
\pgfpathrectangle{\pgfqpoint{0.633874in}{0.569136in}}{\pgfqpoint{2.177280in}{2.201755in}}%
\pgfusepath{clip}%
\pgfsetbuttcap%
\pgfsetroundjoin%
\definecolor{currentfill}{rgb}{0.121569,0.466667,0.705882}%
\pgfsetfillcolor{currentfill}%
\pgfsetlinewidth{0.481800pt}%
\definecolor{currentstroke}{rgb}{1.000000,1.000000,1.000000}%
\pgfsetstrokecolor{currentstroke}%
\pgfsetdash{}{0pt}%
\pgfpathmoveto{\pgfqpoint{1.245489in}{0.794348in}}%
\pgfpathcurveto{\pgfqpoint{1.256539in}{0.794348in}}{\pgfqpoint{1.267138in}{0.798739in}}{\pgfqpoint{1.274952in}{0.806552in}}%
\pgfpathcurveto{\pgfqpoint{1.282765in}{0.814366in}}{\pgfqpoint{1.287155in}{0.824965in}}{\pgfqpoint{1.287155in}{0.836015in}}%
\pgfpathcurveto{\pgfqpoint{1.287155in}{0.847065in}}{\pgfqpoint{1.282765in}{0.857664in}}{\pgfqpoint{1.274952in}{0.865478in}}%
\pgfpathcurveto{\pgfqpoint{1.267138in}{0.873291in}}{\pgfqpoint{1.256539in}{0.877682in}}{\pgfqpoint{1.245489in}{0.877682in}}%
\pgfpathcurveto{\pgfqpoint{1.234439in}{0.877682in}}{\pgfqpoint{1.223840in}{0.873291in}}{\pgfqpoint{1.216026in}{0.865478in}}%
\pgfpathcurveto{\pgfqpoint{1.208212in}{0.857664in}}{\pgfqpoint{1.203822in}{0.847065in}}{\pgfqpoint{1.203822in}{0.836015in}}%
\pgfpathcurveto{\pgfqpoint{1.203822in}{0.824965in}}{\pgfqpoint{1.208212in}{0.814366in}}{\pgfqpoint{1.216026in}{0.806552in}}%
\pgfpathcurveto{\pgfqpoint{1.223840in}{0.798739in}}{\pgfqpoint{1.234439in}{0.794348in}}{\pgfqpoint{1.245489in}{0.794348in}}%
\pgfpathlineto{\pgfqpoint{1.245489in}{0.794348in}}%
\pgfpathclose%
\pgfusepath{stroke,fill}%
\end{pgfscope}%
\begin{pgfscope}%
\pgfpathrectangle{\pgfqpoint{0.633874in}{0.569136in}}{\pgfqpoint{2.177280in}{2.201755in}}%
\pgfusepath{clip}%
\pgfsetbuttcap%
\pgfsetroundjoin%
\definecolor{currentfill}{rgb}{0.121569,0.466667,0.705882}%
\pgfsetfillcolor{currentfill}%
\pgfsetlinewidth{0.481800pt}%
\definecolor{currentstroke}{rgb}{1.000000,1.000000,1.000000}%
\pgfsetstrokecolor{currentstroke}%
\pgfsetdash{}{0pt}%
\pgfpathmoveto{\pgfqpoint{1.365305in}{0.710948in}}%
\pgfpathcurveto{\pgfqpoint{1.376355in}{0.710948in}}{\pgfqpoint{1.386954in}{0.715339in}}{\pgfqpoint{1.394768in}{0.723152in}}%
\pgfpathcurveto{\pgfqpoint{1.402581in}{0.730966in}}{\pgfqpoint{1.406972in}{0.741565in}}{\pgfqpoint{1.406972in}{0.752615in}}%
\pgfpathcurveto{\pgfqpoint{1.406972in}{0.763665in}}{\pgfqpoint{1.402581in}{0.774264in}}{\pgfqpoint{1.394768in}{0.782078in}}%
\pgfpathcurveto{\pgfqpoint{1.386954in}{0.789892in}}{\pgfqpoint{1.376355in}{0.794282in}}{\pgfqpoint{1.365305in}{0.794282in}}%
\pgfpathcurveto{\pgfqpoint{1.354255in}{0.794282in}}{\pgfqpoint{1.343656in}{0.789892in}}{\pgfqpoint{1.335842in}{0.782078in}}%
\pgfpathcurveto{\pgfqpoint{1.328029in}{0.774264in}}{\pgfqpoint{1.323638in}{0.763665in}}{\pgfqpoint{1.323638in}{0.752615in}}%
\pgfpathcurveto{\pgfqpoint{1.323638in}{0.741565in}}{\pgfqpoint{1.328029in}{0.730966in}}{\pgfqpoint{1.335842in}{0.723152in}}%
\pgfpathcurveto{\pgfqpoint{1.343656in}{0.715339in}}{\pgfqpoint{1.354255in}{0.710948in}}{\pgfqpoint{1.365305in}{0.710948in}}%
\pgfpathlineto{\pgfqpoint{1.365305in}{0.710948in}}%
\pgfpathclose%
\pgfusepath{stroke,fill}%
\end{pgfscope}%
\begin{pgfscope}%
\pgfpathrectangle{\pgfqpoint{0.633874in}{0.569136in}}{\pgfqpoint{2.177280in}{2.201755in}}%
\pgfusepath{clip}%
\pgfsetbuttcap%
\pgfsetroundjoin%
\definecolor{currentfill}{rgb}{0.121569,0.466667,0.705882}%
\pgfsetfillcolor{currentfill}%
\pgfsetlinewidth{0.481800pt}%
\definecolor{currentstroke}{rgb}{1.000000,1.000000,1.000000}%
\pgfsetstrokecolor{currentstroke}%
\pgfsetdash{}{0pt}%
\pgfpathmoveto{\pgfqpoint{1.245489in}{0.877748in}}%
\pgfpathcurveto{\pgfqpoint{1.256539in}{0.877748in}}{\pgfqpoint{1.267138in}{0.882138in}}{\pgfqpoint{1.274952in}{0.889952in}}%
\pgfpathcurveto{\pgfqpoint{1.282765in}{0.897766in}}{\pgfqpoint{1.287155in}{0.908365in}}{\pgfqpoint{1.287155in}{0.919415in}}%
\pgfpathcurveto{\pgfqpoint{1.287155in}{0.930465in}}{\pgfqpoint{1.282765in}{0.941064in}}{\pgfqpoint{1.274952in}{0.948878in}}%
\pgfpathcurveto{\pgfqpoint{1.267138in}{0.956691in}}{\pgfqpoint{1.256539in}{0.961081in}}{\pgfqpoint{1.245489in}{0.961081in}}%
\pgfpathcurveto{\pgfqpoint{1.234439in}{0.961081in}}{\pgfqpoint{1.223840in}{0.956691in}}{\pgfqpoint{1.216026in}{0.948878in}}%
\pgfpathcurveto{\pgfqpoint{1.208212in}{0.941064in}}{\pgfqpoint{1.203822in}{0.930465in}}{\pgfqpoint{1.203822in}{0.919415in}}%
\pgfpathcurveto{\pgfqpoint{1.203822in}{0.908365in}}{\pgfqpoint{1.208212in}{0.897766in}}{\pgfqpoint{1.216026in}{0.889952in}}%
\pgfpathcurveto{\pgfqpoint{1.223840in}{0.882138in}}{\pgfqpoint{1.234439in}{0.877748in}}{\pgfqpoint{1.245489in}{0.877748in}}%
\pgfpathlineto{\pgfqpoint{1.245489in}{0.877748in}}%
\pgfpathclose%
\pgfusepath{stroke,fill}%
\end{pgfscope}%
\begin{pgfscope}%
\pgfpathrectangle{\pgfqpoint{0.633874in}{0.569136in}}{\pgfqpoint{2.177280in}{2.201755in}}%
\pgfusepath{clip}%
\pgfsetbuttcap%
\pgfsetroundjoin%
\definecolor{currentfill}{rgb}{0.121569,0.466667,0.705882}%
\pgfsetfillcolor{currentfill}%
\pgfsetlinewidth{0.481800pt}%
\definecolor{currentstroke}{rgb}{1.000000,1.000000,1.000000}%
\pgfsetstrokecolor{currentstroke}%
\pgfsetdash{}{0pt}%
\pgfpathmoveto{\pgfqpoint{1.045795in}{0.710948in}}%
\pgfpathcurveto{\pgfqpoint{1.056845in}{0.710948in}}{\pgfqpoint{1.067444in}{0.715339in}}{\pgfqpoint{1.075258in}{0.723152in}}%
\pgfpathcurveto{\pgfqpoint{1.083072in}{0.730966in}}{\pgfqpoint{1.087462in}{0.741565in}}{\pgfqpoint{1.087462in}{0.752615in}}%
\pgfpathcurveto{\pgfqpoint{1.087462in}{0.763665in}}{\pgfqpoint{1.083072in}{0.774264in}}{\pgfqpoint{1.075258in}{0.782078in}}%
\pgfpathcurveto{\pgfqpoint{1.067444in}{0.789892in}}{\pgfqpoint{1.056845in}{0.794282in}}{\pgfqpoint{1.045795in}{0.794282in}}%
\pgfpathcurveto{\pgfqpoint{1.034745in}{0.794282in}}{\pgfqpoint{1.024146in}{0.789892in}}{\pgfqpoint{1.016332in}{0.782078in}}%
\pgfpathcurveto{\pgfqpoint{1.008519in}{0.774264in}}{\pgfqpoint{1.004129in}{0.763665in}}{\pgfqpoint{1.004129in}{0.752615in}}%
\pgfpathcurveto{\pgfqpoint{1.004129in}{0.741565in}}{\pgfqpoint{1.008519in}{0.730966in}}{\pgfqpoint{1.016332in}{0.723152in}}%
\pgfpathcurveto{\pgfqpoint{1.024146in}{0.715339in}}{\pgfqpoint{1.034745in}{0.710948in}}{\pgfqpoint{1.045795in}{0.710948in}}%
\pgfpathlineto{\pgfqpoint{1.045795in}{0.710948in}}%
\pgfpathclose%
\pgfusepath{stroke,fill}%
\end{pgfscope}%
\begin{pgfscope}%
\pgfpathrectangle{\pgfqpoint{0.633874in}{0.569136in}}{\pgfqpoint{2.177280in}{2.201755in}}%
\pgfusepath{clip}%
\pgfsetbuttcap%
\pgfsetroundjoin%
\definecolor{currentfill}{rgb}{0.121569,0.466667,0.705882}%
\pgfsetfillcolor{currentfill}%
\pgfsetlinewidth{0.481800pt}%
\definecolor{currentstroke}{rgb}{1.000000,1.000000,1.000000}%
\pgfsetstrokecolor{currentstroke}%
\pgfsetdash{}{0pt}%
\pgfpathmoveto{\pgfqpoint{1.245489in}{0.961148in}}%
\pgfpathcurveto{\pgfqpoint{1.256539in}{0.961148in}}{\pgfqpoint{1.267138in}{0.965538in}}{\pgfqpoint{1.274952in}{0.973352in}}%
\pgfpathcurveto{\pgfqpoint{1.282765in}{0.981165in}}{\pgfqpoint{1.287155in}{0.991764in}}{\pgfqpoint{1.287155in}{1.002815in}}%
\pgfpathcurveto{\pgfqpoint{1.287155in}{1.013865in}}{\pgfqpoint{1.282765in}{1.024464in}}{\pgfqpoint{1.274952in}{1.032277in}}%
\pgfpathcurveto{\pgfqpoint{1.267138in}{1.040091in}}{\pgfqpoint{1.256539in}{1.044481in}}{\pgfqpoint{1.245489in}{1.044481in}}%
\pgfpathcurveto{\pgfqpoint{1.234439in}{1.044481in}}{\pgfqpoint{1.223840in}{1.040091in}}{\pgfqpoint{1.216026in}{1.032277in}}%
\pgfpathcurveto{\pgfqpoint{1.208212in}{1.024464in}}{\pgfqpoint{1.203822in}{1.013865in}}{\pgfqpoint{1.203822in}{1.002815in}}%
\pgfpathcurveto{\pgfqpoint{1.203822in}{0.991764in}}{\pgfqpoint{1.208212in}{0.981165in}}{\pgfqpoint{1.216026in}{0.973352in}}%
\pgfpathcurveto{\pgfqpoint{1.223840in}{0.965538in}}{\pgfqpoint{1.234439in}{0.961148in}}{\pgfqpoint{1.245489in}{0.961148in}}%
\pgfpathlineto{\pgfqpoint{1.245489in}{0.961148in}}%
\pgfpathclose%
\pgfusepath{stroke,fill}%
\end{pgfscope}%
\begin{pgfscope}%
\pgfpathrectangle{\pgfqpoint{0.633874in}{0.569136in}}{\pgfqpoint{2.177280in}{2.201755in}}%
\pgfusepath{clip}%
\pgfsetbuttcap%
\pgfsetroundjoin%
\definecolor{currentfill}{rgb}{0.121569,0.466667,0.705882}%
\pgfsetfillcolor{currentfill}%
\pgfsetlinewidth{0.481800pt}%
\definecolor{currentstroke}{rgb}{1.000000,1.000000,1.000000}%
\pgfsetstrokecolor{currentstroke}%
\pgfsetdash{}{0pt}%
\pgfpathmoveto{\pgfqpoint{1.125673in}{0.710948in}}%
\pgfpathcurveto{\pgfqpoint{1.136723in}{0.710948in}}{\pgfqpoint{1.147322in}{0.715339in}}{\pgfqpoint{1.155135in}{0.723152in}}%
\pgfpathcurveto{\pgfqpoint{1.162949in}{0.730966in}}{\pgfqpoint{1.167339in}{0.741565in}}{\pgfqpoint{1.167339in}{0.752615in}}%
\pgfpathcurveto{\pgfqpoint{1.167339in}{0.763665in}}{\pgfqpoint{1.162949in}{0.774264in}}{\pgfqpoint{1.155135in}{0.782078in}}%
\pgfpathcurveto{\pgfqpoint{1.147322in}{0.789892in}}{\pgfqpoint{1.136723in}{0.794282in}}{\pgfqpoint{1.125673in}{0.794282in}}%
\pgfpathcurveto{\pgfqpoint{1.114623in}{0.794282in}}{\pgfqpoint{1.104024in}{0.789892in}}{\pgfqpoint{1.096210in}{0.782078in}}%
\pgfpathcurveto{\pgfqpoint{1.088396in}{0.774264in}}{\pgfqpoint{1.084006in}{0.763665in}}{\pgfqpoint{1.084006in}{0.752615in}}%
\pgfpathcurveto{\pgfqpoint{1.084006in}{0.741565in}}{\pgfqpoint{1.088396in}{0.730966in}}{\pgfqpoint{1.096210in}{0.723152in}}%
\pgfpathcurveto{\pgfqpoint{1.104024in}{0.715339in}}{\pgfqpoint{1.114623in}{0.710948in}}{\pgfqpoint{1.125673in}{0.710948in}}%
\pgfpathlineto{\pgfqpoint{1.125673in}{0.710948in}}%
\pgfpathclose%
\pgfusepath{stroke,fill}%
\end{pgfscope}%
\begin{pgfscope}%
\pgfpathrectangle{\pgfqpoint{0.633874in}{0.569136in}}{\pgfqpoint{2.177280in}{2.201755in}}%
\pgfusepath{clip}%
\pgfsetbuttcap%
\pgfsetroundjoin%
\definecolor{currentfill}{rgb}{0.121569,0.466667,0.705882}%
\pgfsetfillcolor{currentfill}%
\pgfsetlinewidth{0.481800pt}%
\definecolor{currentstroke}{rgb}{1.000000,1.000000,1.000000}%
\pgfsetstrokecolor{currentstroke}%
\pgfsetdash{}{0pt}%
\pgfpathmoveto{\pgfqpoint{1.205550in}{0.710948in}}%
\pgfpathcurveto{\pgfqpoint{1.216600in}{0.710948in}}{\pgfqpoint{1.227199in}{0.715339in}}{\pgfqpoint{1.235013in}{0.723152in}}%
\pgfpathcurveto{\pgfqpoint{1.242826in}{0.730966in}}{\pgfqpoint{1.247217in}{0.741565in}}{\pgfqpoint{1.247217in}{0.752615in}}%
\pgfpathcurveto{\pgfqpoint{1.247217in}{0.763665in}}{\pgfqpoint{1.242826in}{0.774264in}}{\pgfqpoint{1.235013in}{0.782078in}}%
\pgfpathcurveto{\pgfqpoint{1.227199in}{0.789892in}}{\pgfqpoint{1.216600in}{0.794282in}}{\pgfqpoint{1.205550in}{0.794282in}}%
\pgfpathcurveto{\pgfqpoint{1.194500in}{0.794282in}}{\pgfqpoint{1.183901in}{0.789892in}}{\pgfqpoint{1.176087in}{0.782078in}}%
\pgfpathcurveto{\pgfqpoint{1.168274in}{0.774264in}}{\pgfqpoint{1.163883in}{0.763665in}}{\pgfqpoint{1.163883in}{0.752615in}}%
\pgfpathcurveto{\pgfqpoint{1.163883in}{0.741565in}}{\pgfqpoint{1.168274in}{0.730966in}}{\pgfqpoint{1.176087in}{0.723152in}}%
\pgfpathcurveto{\pgfqpoint{1.183901in}{0.715339in}}{\pgfqpoint{1.194500in}{0.710948in}}{\pgfqpoint{1.205550in}{0.710948in}}%
\pgfpathlineto{\pgfqpoint{1.205550in}{0.710948in}}%
\pgfpathclose%
\pgfusepath{stroke,fill}%
\end{pgfscope}%
\begin{pgfscope}%
\pgfpathrectangle{\pgfqpoint{0.633874in}{0.569136in}}{\pgfqpoint{2.177280in}{2.201755in}}%
\pgfusepath{clip}%
\pgfsetbuttcap%
\pgfsetroundjoin%
\definecolor{currentfill}{rgb}{0.121569,0.466667,0.705882}%
\pgfsetfillcolor{currentfill}%
\pgfsetlinewidth{0.481800pt}%
\definecolor{currentstroke}{rgb}{1.000000,1.000000,1.000000}%
\pgfsetstrokecolor{currentstroke}%
\pgfsetdash{}{0pt}%
\pgfpathmoveto{\pgfqpoint{1.205550in}{0.877748in}}%
\pgfpathcurveto{\pgfqpoint{1.216600in}{0.877748in}}{\pgfqpoint{1.227199in}{0.882138in}}{\pgfqpoint{1.235013in}{0.889952in}}%
\pgfpathcurveto{\pgfqpoint{1.242826in}{0.897766in}}{\pgfqpoint{1.247217in}{0.908365in}}{\pgfqpoint{1.247217in}{0.919415in}}%
\pgfpathcurveto{\pgfqpoint{1.247217in}{0.930465in}}{\pgfqpoint{1.242826in}{0.941064in}}{\pgfqpoint{1.235013in}{0.948878in}}%
\pgfpathcurveto{\pgfqpoint{1.227199in}{0.956691in}}{\pgfqpoint{1.216600in}{0.961081in}}{\pgfqpoint{1.205550in}{0.961081in}}%
\pgfpathcurveto{\pgfqpoint{1.194500in}{0.961081in}}{\pgfqpoint{1.183901in}{0.956691in}}{\pgfqpoint{1.176087in}{0.948878in}}%
\pgfpathcurveto{\pgfqpoint{1.168274in}{0.941064in}}{\pgfqpoint{1.163883in}{0.930465in}}{\pgfqpoint{1.163883in}{0.919415in}}%
\pgfpathcurveto{\pgfqpoint{1.163883in}{0.908365in}}{\pgfqpoint{1.168274in}{0.897766in}}{\pgfqpoint{1.176087in}{0.889952in}}%
\pgfpathcurveto{\pgfqpoint{1.183901in}{0.882138in}}{\pgfqpoint{1.194500in}{0.877748in}}{\pgfqpoint{1.205550in}{0.877748in}}%
\pgfpathlineto{\pgfqpoint{1.205550in}{0.877748in}}%
\pgfpathclose%
\pgfusepath{stroke,fill}%
\end{pgfscope}%
\begin{pgfscope}%
\pgfpathrectangle{\pgfqpoint{0.633874in}{0.569136in}}{\pgfqpoint{2.177280in}{2.201755in}}%
\pgfusepath{clip}%
\pgfsetbuttcap%
\pgfsetroundjoin%
\definecolor{currentfill}{rgb}{0.121569,0.466667,0.705882}%
\pgfsetfillcolor{currentfill}%
\pgfsetlinewidth{0.481800pt}%
\definecolor{currentstroke}{rgb}{1.000000,1.000000,1.000000}%
\pgfsetstrokecolor{currentstroke}%
\pgfsetdash{}{0pt}%
\pgfpathmoveto{\pgfqpoint{1.285428in}{0.710948in}}%
\pgfpathcurveto{\pgfqpoint{1.296478in}{0.710948in}}{\pgfqpoint{1.307077in}{0.715339in}}{\pgfqpoint{1.314890in}{0.723152in}}%
\pgfpathcurveto{\pgfqpoint{1.322704in}{0.730966in}}{\pgfqpoint{1.327094in}{0.741565in}}{\pgfqpoint{1.327094in}{0.752615in}}%
\pgfpathcurveto{\pgfqpoint{1.327094in}{0.763665in}}{\pgfqpoint{1.322704in}{0.774264in}}{\pgfqpoint{1.314890in}{0.782078in}}%
\pgfpathcurveto{\pgfqpoint{1.307077in}{0.789892in}}{\pgfqpoint{1.296478in}{0.794282in}}{\pgfqpoint{1.285428in}{0.794282in}}%
\pgfpathcurveto{\pgfqpoint{1.274377in}{0.794282in}}{\pgfqpoint{1.263778in}{0.789892in}}{\pgfqpoint{1.255965in}{0.782078in}}%
\pgfpathcurveto{\pgfqpoint{1.248151in}{0.774264in}}{\pgfqpoint{1.243761in}{0.763665in}}{\pgfqpoint{1.243761in}{0.752615in}}%
\pgfpathcurveto{\pgfqpoint{1.243761in}{0.741565in}}{\pgfqpoint{1.248151in}{0.730966in}}{\pgfqpoint{1.255965in}{0.723152in}}%
\pgfpathcurveto{\pgfqpoint{1.263778in}{0.715339in}}{\pgfqpoint{1.274377in}{0.710948in}}{\pgfqpoint{1.285428in}{0.710948in}}%
\pgfpathlineto{\pgfqpoint{1.285428in}{0.710948in}}%
\pgfpathclose%
\pgfusepath{stroke,fill}%
\end{pgfscope}%
\begin{pgfscope}%
\pgfpathrectangle{\pgfqpoint{0.633874in}{0.569136in}}{\pgfqpoint{2.177280in}{2.201755in}}%
\pgfusepath{clip}%
\pgfsetbuttcap%
\pgfsetroundjoin%
\definecolor{currentfill}{rgb}{0.121569,0.466667,0.705882}%
\pgfsetfillcolor{currentfill}%
\pgfsetlinewidth{0.481800pt}%
\definecolor{currentstroke}{rgb}{1.000000,1.000000,1.000000}%
\pgfsetstrokecolor{currentstroke}%
\pgfsetdash{}{0pt}%
\pgfpathmoveto{\pgfqpoint{1.285428in}{0.710948in}}%
\pgfpathcurveto{\pgfqpoint{1.296478in}{0.710948in}}{\pgfqpoint{1.307077in}{0.715339in}}{\pgfqpoint{1.314890in}{0.723152in}}%
\pgfpathcurveto{\pgfqpoint{1.322704in}{0.730966in}}{\pgfqpoint{1.327094in}{0.741565in}}{\pgfqpoint{1.327094in}{0.752615in}}%
\pgfpathcurveto{\pgfqpoint{1.327094in}{0.763665in}}{\pgfqpoint{1.322704in}{0.774264in}}{\pgfqpoint{1.314890in}{0.782078in}}%
\pgfpathcurveto{\pgfqpoint{1.307077in}{0.789892in}}{\pgfqpoint{1.296478in}{0.794282in}}{\pgfqpoint{1.285428in}{0.794282in}}%
\pgfpathcurveto{\pgfqpoint{1.274377in}{0.794282in}}{\pgfqpoint{1.263778in}{0.789892in}}{\pgfqpoint{1.255965in}{0.782078in}}%
\pgfpathcurveto{\pgfqpoint{1.248151in}{0.774264in}}{\pgfqpoint{1.243761in}{0.763665in}}{\pgfqpoint{1.243761in}{0.752615in}}%
\pgfpathcurveto{\pgfqpoint{1.243761in}{0.741565in}}{\pgfqpoint{1.248151in}{0.730966in}}{\pgfqpoint{1.255965in}{0.723152in}}%
\pgfpathcurveto{\pgfqpoint{1.263778in}{0.715339in}}{\pgfqpoint{1.274377in}{0.710948in}}{\pgfqpoint{1.285428in}{0.710948in}}%
\pgfpathlineto{\pgfqpoint{1.285428in}{0.710948in}}%
\pgfpathclose%
\pgfusepath{stroke,fill}%
\end{pgfscope}%
\begin{pgfscope}%
\pgfpathrectangle{\pgfqpoint{0.633874in}{0.569136in}}{\pgfqpoint{2.177280in}{2.201755in}}%
\pgfusepath{clip}%
\pgfsetbuttcap%
\pgfsetroundjoin%
\definecolor{currentfill}{rgb}{0.121569,0.466667,0.705882}%
\pgfsetfillcolor{currentfill}%
\pgfsetlinewidth{0.481800pt}%
\definecolor{currentstroke}{rgb}{1.000000,1.000000,1.000000}%
\pgfsetstrokecolor{currentstroke}%
\pgfsetdash{}{0pt}%
\pgfpathmoveto{\pgfqpoint{1.085734in}{0.710948in}}%
\pgfpathcurveto{\pgfqpoint{1.096784in}{0.710948in}}{\pgfqpoint{1.107383in}{0.715339in}}{\pgfqpoint{1.115197in}{0.723152in}}%
\pgfpathcurveto{\pgfqpoint{1.123010in}{0.730966in}}{\pgfqpoint{1.127401in}{0.741565in}}{\pgfqpoint{1.127401in}{0.752615in}}%
\pgfpathcurveto{\pgfqpoint{1.127401in}{0.763665in}}{\pgfqpoint{1.123010in}{0.774264in}}{\pgfqpoint{1.115197in}{0.782078in}}%
\pgfpathcurveto{\pgfqpoint{1.107383in}{0.789892in}}{\pgfqpoint{1.096784in}{0.794282in}}{\pgfqpoint{1.085734in}{0.794282in}}%
\pgfpathcurveto{\pgfqpoint{1.074684in}{0.794282in}}{\pgfqpoint{1.064085in}{0.789892in}}{\pgfqpoint{1.056271in}{0.782078in}}%
\pgfpathcurveto{\pgfqpoint{1.048458in}{0.774264in}}{\pgfqpoint{1.044067in}{0.763665in}}{\pgfqpoint{1.044067in}{0.752615in}}%
\pgfpathcurveto{\pgfqpoint{1.044067in}{0.741565in}}{\pgfqpoint{1.048458in}{0.730966in}}{\pgfqpoint{1.056271in}{0.723152in}}%
\pgfpathcurveto{\pgfqpoint{1.064085in}{0.715339in}}{\pgfqpoint{1.074684in}{0.710948in}}{\pgfqpoint{1.085734in}{0.710948in}}%
\pgfpathlineto{\pgfqpoint{1.085734in}{0.710948in}}%
\pgfpathclose%
\pgfusepath{stroke,fill}%
\end{pgfscope}%
\begin{pgfscope}%
\pgfpathrectangle{\pgfqpoint{0.633874in}{0.569136in}}{\pgfqpoint{2.177280in}{2.201755in}}%
\pgfusepath{clip}%
\pgfsetbuttcap%
\pgfsetroundjoin%
\definecolor{currentfill}{rgb}{0.121569,0.466667,0.705882}%
\pgfsetfillcolor{currentfill}%
\pgfsetlinewidth{0.481800pt}%
\definecolor{currentstroke}{rgb}{1.000000,1.000000,1.000000}%
\pgfsetstrokecolor{currentstroke}%
\pgfsetdash{}{0pt}%
\pgfpathmoveto{\pgfqpoint{1.125673in}{0.710948in}}%
\pgfpathcurveto{\pgfqpoint{1.136723in}{0.710948in}}{\pgfqpoint{1.147322in}{0.715339in}}{\pgfqpoint{1.155135in}{0.723152in}}%
\pgfpathcurveto{\pgfqpoint{1.162949in}{0.730966in}}{\pgfqpoint{1.167339in}{0.741565in}}{\pgfqpoint{1.167339in}{0.752615in}}%
\pgfpathcurveto{\pgfqpoint{1.167339in}{0.763665in}}{\pgfqpoint{1.162949in}{0.774264in}}{\pgfqpoint{1.155135in}{0.782078in}}%
\pgfpathcurveto{\pgfqpoint{1.147322in}{0.789892in}}{\pgfqpoint{1.136723in}{0.794282in}}{\pgfqpoint{1.125673in}{0.794282in}}%
\pgfpathcurveto{\pgfqpoint{1.114623in}{0.794282in}}{\pgfqpoint{1.104024in}{0.789892in}}{\pgfqpoint{1.096210in}{0.782078in}}%
\pgfpathcurveto{\pgfqpoint{1.088396in}{0.774264in}}{\pgfqpoint{1.084006in}{0.763665in}}{\pgfqpoint{1.084006in}{0.752615in}}%
\pgfpathcurveto{\pgfqpoint{1.084006in}{0.741565in}}{\pgfqpoint{1.088396in}{0.730966in}}{\pgfqpoint{1.096210in}{0.723152in}}%
\pgfpathcurveto{\pgfqpoint{1.104024in}{0.715339in}}{\pgfqpoint{1.114623in}{0.710948in}}{\pgfqpoint{1.125673in}{0.710948in}}%
\pgfpathlineto{\pgfqpoint{1.125673in}{0.710948in}}%
\pgfpathclose%
\pgfusepath{stroke,fill}%
\end{pgfscope}%
\begin{pgfscope}%
\pgfpathrectangle{\pgfqpoint{0.633874in}{0.569136in}}{\pgfqpoint{2.177280in}{2.201755in}}%
\pgfusepath{clip}%
\pgfsetbuttcap%
\pgfsetroundjoin%
\definecolor{currentfill}{rgb}{0.121569,0.466667,0.705882}%
\pgfsetfillcolor{currentfill}%
\pgfsetlinewidth{0.481800pt}%
\definecolor{currentstroke}{rgb}{1.000000,1.000000,1.000000}%
\pgfsetstrokecolor{currentstroke}%
\pgfsetdash{}{0pt}%
\pgfpathmoveto{\pgfqpoint{1.365305in}{0.877748in}}%
\pgfpathcurveto{\pgfqpoint{1.376355in}{0.877748in}}{\pgfqpoint{1.386954in}{0.882138in}}{\pgfqpoint{1.394768in}{0.889952in}}%
\pgfpathcurveto{\pgfqpoint{1.402581in}{0.897766in}}{\pgfqpoint{1.406972in}{0.908365in}}{\pgfqpoint{1.406972in}{0.919415in}}%
\pgfpathcurveto{\pgfqpoint{1.406972in}{0.930465in}}{\pgfqpoint{1.402581in}{0.941064in}}{\pgfqpoint{1.394768in}{0.948878in}}%
\pgfpathcurveto{\pgfqpoint{1.386954in}{0.956691in}}{\pgfqpoint{1.376355in}{0.961081in}}{\pgfqpoint{1.365305in}{0.961081in}}%
\pgfpathcurveto{\pgfqpoint{1.354255in}{0.961081in}}{\pgfqpoint{1.343656in}{0.956691in}}{\pgfqpoint{1.335842in}{0.948878in}}%
\pgfpathcurveto{\pgfqpoint{1.328029in}{0.941064in}}{\pgfqpoint{1.323638in}{0.930465in}}{\pgfqpoint{1.323638in}{0.919415in}}%
\pgfpathcurveto{\pgfqpoint{1.323638in}{0.908365in}}{\pgfqpoint{1.328029in}{0.897766in}}{\pgfqpoint{1.335842in}{0.889952in}}%
\pgfpathcurveto{\pgfqpoint{1.343656in}{0.882138in}}{\pgfqpoint{1.354255in}{0.877748in}}{\pgfqpoint{1.365305in}{0.877748in}}%
\pgfpathlineto{\pgfqpoint{1.365305in}{0.877748in}}%
\pgfpathclose%
\pgfusepath{stroke,fill}%
\end{pgfscope}%
\begin{pgfscope}%
\pgfpathrectangle{\pgfqpoint{0.633874in}{0.569136in}}{\pgfqpoint{2.177280in}{2.201755in}}%
\pgfusepath{clip}%
\pgfsetbuttcap%
\pgfsetroundjoin%
\definecolor{currentfill}{rgb}{0.121569,0.466667,0.705882}%
\pgfsetfillcolor{currentfill}%
\pgfsetlinewidth{0.481800pt}%
\definecolor{currentstroke}{rgb}{1.000000,1.000000,1.000000}%
\pgfsetstrokecolor{currentstroke}%
\pgfsetdash{}{0pt}%
\pgfpathmoveto{\pgfqpoint{1.285428in}{0.627549in}}%
\pgfpathcurveto{\pgfqpoint{1.296478in}{0.627549in}}{\pgfqpoint{1.307077in}{0.631939in}}{\pgfqpoint{1.314890in}{0.639753in}}%
\pgfpathcurveto{\pgfqpoint{1.322704in}{0.647566in}}{\pgfqpoint{1.327094in}{0.658165in}}{\pgfqpoint{1.327094in}{0.669215in}}%
\pgfpathcurveto{\pgfqpoint{1.327094in}{0.680265in}}{\pgfqpoint{1.322704in}{0.690864in}}{\pgfqpoint{1.314890in}{0.698678in}}%
\pgfpathcurveto{\pgfqpoint{1.307077in}{0.706492in}}{\pgfqpoint{1.296478in}{0.710882in}}{\pgfqpoint{1.285428in}{0.710882in}}%
\pgfpathcurveto{\pgfqpoint{1.274377in}{0.710882in}}{\pgfqpoint{1.263778in}{0.706492in}}{\pgfqpoint{1.255965in}{0.698678in}}%
\pgfpathcurveto{\pgfqpoint{1.248151in}{0.690864in}}{\pgfqpoint{1.243761in}{0.680265in}}{\pgfqpoint{1.243761in}{0.669215in}}%
\pgfpathcurveto{\pgfqpoint{1.243761in}{0.658165in}}{\pgfqpoint{1.248151in}{0.647566in}}{\pgfqpoint{1.255965in}{0.639753in}}%
\pgfpathcurveto{\pgfqpoint{1.263778in}{0.631939in}}{\pgfqpoint{1.274377in}{0.627549in}}{\pgfqpoint{1.285428in}{0.627549in}}%
\pgfpathlineto{\pgfqpoint{1.285428in}{0.627549in}}%
\pgfpathclose%
\pgfusepath{stroke,fill}%
\end{pgfscope}%
\begin{pgfscope}%
\pgfpathrectangle{\pgfqpoint{0.633874in}{0.569136in}}{\pgfqpoint{2.177280in}{2.201755in}}%
\pgfusepath{clip}%
\pgfsetbuttcap%
\pgfsetroundjoin%
\definecolor{currentfill}{rgb}{0.121569,0.466667,0.705882}%
\pgfsetfillcolor{currentfill}%
\pgfsetlinewidth{0.481800pt}%
\definecolor{currentstroke}{rgb}{1.000000,1.000000,1.000000}%
\pgfsetstrokecolor{currentstroke}%
\pgfsetdash{}{0pt}%
\pgfpathmoveto{\pgfqpoint{1.405244in}{0.710948in}}%
\pgfpathcurveto{\pgfqpoint{1.416294in}{0.710948in}}{\pgfqpoint{1.426893in}{0.715339in}}{\pgfqpoint{1.434706in}{0.723152in}}%
\pgfpathcurveto{\pgfqpoint{1.442520in}{0.730966in}}{\pgfqpoint{1.446910in}{0.741565in}}{\pgfqpoint{1.446910in}{0.752615in}}%
\pgfpathcurveto{\pgfqpoint{1.446910in}{0.763665in}}{\pgfqpoint{1.442520in}{0.774264in}}{\pgfqpoint{1.434706in}{0.782078in}}%
\pgfpathcurveto{\pgfqpoint{1.426893in}{0.789892in}}{\pgfqpoint{1.416294in}{0.794282in}}{\pgfqpoint{1.405244in}{0.794282in}}%
\pgfpathcurveto{\pgfqpoint{1.394193in}{0.794282in}}{\pgfqpoint{1.383594in}{0.789892in}}{\pgfqpoint{1.375781in}{0.782078in}}%
\pgfpathcurveto{\pgfqpoint{1.367967in}{0.774264in}}{\pgfqpoint{1.363577in}{0.763665in}}{\pgfqpoint{1.363577in}{0.752615in}}%
\pgfpathcurveto{\pgfqpoint{1.363577in}{0.741565in}}{\pgfqpoint{1.367967in}{0.730966in}}{\pgfqpoint{1.375781in}{0.723152in}}%
\pgfpathcurveto{\pgfqpoint{1.383594in}{0.715339in}}{\pgfqpoint{1.394193in}{0.710948in}}{\pgfqpoint{1.405244in}{0.710948in}}%
\pgfpathlineto{\pgfqpoint{1.405244in}{0.710948in}}%
\pgfpathclose%
\pgfusepath{stroke,fill}%
\end{pgfscope}%
\begin{pgfscope}%
\pgfpathrectangle{\pgfqpoint{0.633874in}{0.569136in}}{\pgfqpoint{2.177280in}{2.201755in}}%
\pgfusepath{clip}%
\pgfsetbuttcap%
\pgfsetroundjoin%
\definecolor{currentfill}{rgb}{0.121569,0.466667,0.705882}%
\pgfsetfillcolor{currentfill}%
\pgfsetlinewidth{0.481800pt}%
\definecolor{currentstroke}{rgb}{1.000000,1.000000,1.000000}%
\pgfsetstrokecolor{currentstroke}%
\pgfsetdash{}{0pt}%
\pgfpathmoveto{\pgfqpoint{1.165611in}{0.710948in}}%
\pgfpathcurveto{\pgfqpoint{1.176662in}{0.710948in}}{\pgfqpoint{1.187261in}{0.715339in}}{\pgfqpoint{1.195074in}{0.723152in}}%
\pgfpathcurveto{\pgfqpoint{1.202888in}{0.730966in}}{\pgfqpoint{1.207278in}{0.741565in}}{\pgfqpoint{1.207278in}{0.752615in}}%
\pgfpathcurveto{\pgfqpoint{1.207278in}{0.763665in}}{\pgfqpoint{1.202888in}{0.774264in}}{\pgfqpoint{1.195074in}{0.782078in}}%
\pgfpathcurveto{\pgfqpoint{1.187261in}{0.789892in}}{\pgfqpoint{1.176662in}{0.794282in}}{\pgfqpoint{1.165611in}{0.794282in}}%
\pgfpathcurveto{\pgfqpoint{1.154561in}{0.794282in}}{\pgfqpoint{1.143962in}{0.789892in}}{\pgfqpoint{1.136149in}{0.782078in}}%
\pgfpathcurveto{\pgfqpoint{1.128335in}{0.774264in}}{\pgfqpoint{1.123945in}{0.763665in}}{\pgfqpoint{1.123945in}{0.752615in}}%
\pgfpathcurveto{\pgfqpoint{1.123945in}{0.741565in}}{\pgfqpoint{1.128335in}{0.730966in}}{\pgfqpoint{1.136149in}{0.723152in}}%
\pgfpathcurveto{\pgfqpoint{1.143962in}{0.715339in}}{\pgfqpoint{1.154561in}{0.710948in}}{\pgfqpoint{1.165611in}{0.710948in}}%
\pgfpathlineto{\pgfqpoint{1.165611in}{0.710948in}}%
\pgfpathclose%
\pgfusepath{stroke,fill}%
\end{pgfscope}%
\begin{pgfscope}%
\pgfpathrectangle{\pgfqpoint{0.633874in}{0.569136in}}{\pgfqpoint{2.177280in}{2.201755in}}%
\pgfusepath{clip}%
\pgfsetbuttcap%
\pgfsetroundjoin%
\definecolor{currentfill}{rgb}{0.121569,0.466667,0.705882}%
\pgfsetfillcolor{currentfill}%
\pgfsetlinewidth{0.481800pt}%
\definecolor{currentstroke}{rgb}{1.000000,1.000000,1.000000}%
\pgfsetstrokecolor{currentstroke}%
\pgfsetdash{}{0pt}%
\pgfpathmoveto{\pgfqpoint{1.205550in}{0.710948in}}%
\pgfpathcurveto{\pgfqpoint{1.216600in}{0.710948in}}{\pgfqpoint{1.227199in}{0.715339in}}{\pgfqpoint{1.235013in}{0.723152in}}%
\pgfpathcurveto{\pgfqpoint{1.242826in}{0.730966in}}{\pgfqpoint{1.247217in}{0.741565in}}{\pgfqpoint{1.247217in}{0.752615in}}%
\pgfpathcurveto{\pgfqpoint{1.247217in}{0.763665in}}{\pgfqpoint{1.242826in}{0.774264in}}{\pgfqpoint{1.235013in}{0.782078in}}%
\pgfpathcurveto{\pgfqpoint{1.227199in}{0.789892in}}{\pgfqpoint{1.216600in}{0.794282in}}{\pgfqpoint{1.205550in}{0.794282in}}%
\pgfpathcurveto{\pgfqpoint{1.194500in}{0.794282in}}{\pgfqpoint{1.183901in}{0.789892in}}{\pgfqpoint{1.176087in}{0.782078in}}%
\pgfpathcurveto{\pgfqpoint{1.168274in}{0.774264in}}{\pgfqpoint{1.163883in}{0.763665in}}{\pgfqpoint{1.163883in}{0.752615in}}%
\pgfpathcurveto{\pgfqpoint{1.163883in}{0.741565in}}{\pgfqpoint{1.168274in}{0.730966in}}{\pgfqpoint{1.176087in}{0.723152in}}%
\pgfpathcurveto{\pgfqpoint{1.183901in}{0.715339in}}{\pgfqpoint{1.194500in}{0.710948in}}{\pgfqpoint{1.205550in}{0.710948in}}%
\pgfpathlineto{\pgfqpoint{1.205550in}{0.710948in}}%
\pgfpathclose%
\pgfusepath{stroke,fill}%
\end{pgfscope}%
\begin{pgfscope}%
\pgfpathrectangle{\pgfqpoint{0.633874in}{0.569136in}}{\pgfqpoint{2.177280in}{2.201755in}}%
\pgfusepath{clip}%
\pgfsetbuttcap%
\pgfsetroundjoin%
\definecolor{currentfill}{rgb}{0.121569,0.466667,0.705882}%
\pgfsetfillcolor{currentfill}%
\pgfsetlinewidth{0.481800pt}%
\definecolor{currentstroke}{rgb}{1.000000,1.000000,1.000000}%
\pgfsetstrokecolor{currentstroke}%
\pgfsetdash{}{0pt}%
\pgfpathmoveto{\pgfqpoint{1.405244in}{0.710948in}}%
\pgfpathcurveto{\pgfqpoint{1.416294in}{0.710948in}}{\pgfqpoint{1.426893in}{0.715339in}}{\pgfqpoint{1.434706in}{0.723152in}}%
\pgfpathcurveto{\pgfqpoint{1.442520in}{0.730966in}}{\pgfqpoint{1.446910in}{0.741565in}}{\pgfqpoint{1.446910in}{0.752615in}}%
\pgfpathcurveto{\pgfqpoint{1.446910in}{0.763665in}}{\pgfqpoint{1.442520in}{0.774264in}}{\pgfqpoint{1.434706in}{0.782078in}}%
\pgfpathcurveto{\pgfqpoint{1.426893in}{0.789892in}}{\pgfqpoint{1.416294in}{0.794282in}}{\pgfqpoint{1.405244in}{0.794282in}}%
\pgfpathcurveto{\pgfqpoint{1.394193in}{0.794282in}}{\pgfqpoint{1.383594in}{0.789892in}}{\pgfqpoint{1.375781in}{0.782078in}}%
\pgfpathcurveto{\pgfqpoint{1.367967in}{0.774264in}}{\pgfqpoint{1.363577in}{0.763665in}}{\pgfqpoint{1.363577in}{0.752615in}}%
\pgfpathcurveto{\pgfqpoint{1.363577in}{0.741565in}}{\pgfqpoint{1.367967in}{0.730966in}}{\pgfqpoint{1.375781in}{0.723152in}}%
\pgfpathcurveto{\pgfqpoint{1.383594in}{0.715339in}}{\pgfqpoint{1.394193in}{0.710948in}}{\pgfqpoint{1.405244in}{0.710948in}}%
\pgfpathlineto{\pgfqpoint{1.405244in}{0.710948in}}%
\pgfpathclose%
\pgfusepath{stroke,fill}%
\end{pgfscope}%
\begin{pgfscope}%
\pgfpathrectangle{\pgfqpoint{0.633874in}{0.569136in}}{\pgfqpoint{2.177280in}{2.201755in}}%
\pgfusepath{clip}%
\pgfsetbuttcap%
\pgfsetroundjoin%
\definecolor{currentfill}{rgb}{0.121569,0.466667,0.705882}%
\pgfsetfillcolor{currentfill}%
\pgfsetlinewidth{0.481800pt}%
\definecolor{currentstroke}{rgb}{1.000000,1.000000,1.000000}%
\pgfsetstrokecolor{currentstroke}%
\pgfsetdash{}{0pt}%
\pgfpathmoveto{\pgfqpoint{1.165611in}{0.627549in}}%
\pgfpathcurveto{\pgfqpoint{1.176662in}{0.627549in}}{\pgfqpoint{1.187261in}{0.631939in}}{\pgfqpoint{1.195074in}{0.639753in}}%
\pgfpathcurveto{\pgfqpoint{1.202888in}{0.647566in}}{\pgfqpoint{1.207278in}{0.658165in}}{\pgfqpoint{1.207278in}{0.669215in}}%
\pgfpathcurveto{\pgfqpoint{1.207278in}{0.680265in}}{\pgfqpoint{1.202888in}{0.690864in}}{\pgfqpoint{1.195074in}{0.698678in}}%
\pgfpathcurveto{\pgfqpoint{1.187261in}{0.706492in}}{\pgfqpoint{1.176662in}{0.710882in}}{\pgfqpoint{1.165611in}{0.710882in}}%
\pgfpathcurveto{\pgfqpoint{1.154561in}{0.710882in}}{\pgfqpoint{1.143962in}{0.706492in}}{\pgfqpoint{1.136149in}{0.698678in}}%
\pgfpathcurveto{\pgfqpoint{1.128335in}{0.690864in}}{\pgfqpoint{1.123945in}{0.680265in}}{\pgfqpoint{1.123945in}{0.669215in}}%
\pgfpathcurveto{\pgfqpoint{1.123945in}{0.658165in}}{\pgfqpoint{1.128335in}{0.647566in}}{\pgfqpoint{1.136149in}{0.639753in}}%
\pgfpathcurveto{\pgfqpoint{1.143962in}{0.631939in}}{\pgfqpoint{1.154561in}{0.627549in}}{\pgfqpoint{1.165611in}{0.627549in}}%
\pgfpathlineto{\pgfqpoint{1.165611in}{0.627549in}}%
\pgfpathclose%
\pgfusepath{stroke,fill}%
\end{pgfscope}%
\begin{pgfscope}%
\pgfpathrectangle{\pgfqpoint{0.633874in}{0.569136in}}{\pgfqpoint{2.177280in}{2.201755in}}%
\pgfusepath{clip}%
\pgfsetbuttcap%
\pgfsetroundjoin%
\definecolor{currentfill}{rgb}{0.121569,0.466667,0.705882}%
\pgfsetfillcolor{currentfill}%
\pgfsetlinewidth{0.481800pt}%
\definecolor{currentstroke}{rgb}{1.000000,1.000000,1.000000}%
\pgfsetstrokecolor{currentstroke}%
\pgfsetdash{}{0pt}%
\pgfpathmoveto{\pgfqpoint{0.965918in}{0.710948in}}%
\pgfpathcurveto{\pgfqpoint{0.976968in}{0.710948in}}{\pgfqpoint{0.987567in}{0.715339in}}{\pgfqpoint{0.995381in}{0.723152in}}%
\pgfpathcurveto{\pgfqpoint{1.003194in}{0.730966in}}{\pgfqpoint{1.007585in}{0.741565in}}{\pgfqpoint{1.007585in}{0.752615in}}%
\pgfpathcurveto{\pgfqpoint{1.007585in}{0.763665in}}{\pgfqpoint{1.003194in}{0.774264in}}{\pgfqpoint{0.995381in}{0.782078in}}%
\pgfpathcurveto{\pgfqpoint{0.987567in}{0.789892in}}{\pgfqpoint{0.976968in}{0.794282in}}{\pgfqpoint{0.965918in}{0.794282in}}%
\pgfpathcurveto{\pgfqpoint{0.954868in}{0.794282in}}{\pgfqpoint{0.944269in}{0.789892in}}{\pgfqpoint{0.936455in}{0.782078in}}%
\pgfpathcurveto{\pgfqpoint{0.928641in}{0.774264in}}{\pgfqpoint{0.924251in}{0.763665in}}{\pgfqpoint{0.924251in}{0.752615in}}%
\pgfpathcurveto{\pgfqpoint{0.924251in}{0.741565in}}{\pgfqpoint{0.928641in}{0.730966in}}{\pgfqpoint{0.936455in}{0.723152in}}%
\pgfpathcurveto{\pgfqpoint{0.944269in}{0.715339in}}{\pgfqpoint{0.954868in}{0.710948in}}{\pgfqpoint{0.965918in}{0.710948in}}%
\pgfpathlineto{\pgfqpoint{0.965918in}{0.710948in}}%
\pgfpathclose%
\pgfusepath{stroke,fill}%
\end{pgfscope}%
\begin{pgfscope}%
\pgfpathrectangle{\pgfqpoint{0.633874in}{0.569136in}}{\pgfqpoint{2.177280in}{2.201755in}}%
\pgfusepath{clip}%
\pgfsetbuttcap%
\pgfsetroundjoin%
\definecolor{currentfill}{rgb}{0.121569,0.466667,0.705882}%
\pgfsetfillcolor{currentfill}%
\pgfsetlinewidth{0.481800pt}%
\definecolor{currentstroke}{rgb}{1.000000,1.000000,1.000000}%
\pgfsetstrokecolor{currentstroke}%
\pgfsetdash{}{0pt}%
\pgfpathmoveto{\pgfqpoint{1.245489in}{0.710948in}}%
\pgfpathcurveto{\pgfqpoint{1.256539in}{0.710948in}}{\pgfqpoint{1.267138in}{0.715339in}}{\pgfqpoint{1.274952in}{0.723152in}}%
\pgfpathcurveto{\pgfqpoint{1.282765in}{0.730966in}}{\pgfqpoint{1.287155in}{0.741565in}}{\pgfqpoint{1.287155in}{0.752615in}}%
\pgfpathcurveto{\pgfqpoint{1.287155in}{0.763665in}}{\pgfqpoint{1.282765in}{0.774264in}}{\pgfqpoint{1.274952in}{0.782078in}}%
\pgfpathcurveto{\pgfqpoint{1.267138in}{0.789892in}}{\pgfqpoint{1.256539in}{0.794282in}}{\pgfqpoint{1.245489in}{0.794282in}}%
\pgfpathcurveto{\pgfqpoint{1.234439in}{0.794282in}}{\pgfqpoint{1.223840in}{0.789892in}}{\pgfqpoint{1.216026in}{0.782078in}}%
\pgfpathcurveto{\pgfqpoint{1.208212in}{0.774264in}}{\pgfqpoint{1.203822in}{0.763665in}}{\pgfqpoint{1.203822in}{0.752615in}}%
\pgfpathcurveto{\pgfqpoint{1.203822in}{0.741565in}}{\pgfqpoint{1.208212in}{0.730966in}}{\pgfqpoint{1.216026in}{0.723152in}}%
\pgfpathcurveto{\pgfqpoint{1.223840in}{0.715339in}}{\pgfqpoint{1.234439in}{0.710948in}}{\pgfqpoint{1.245489in}{0.710948in}}%
\pgfpathlineto{\pgfqpoint{1.245489in}{0.710948in}}%
\pgfpathclose%
\pgfusepath{stroke,fill}%
\end{pgfscope}%
\begin{pgfscope}%
\pgfpathrectangle{\pgfqpoint{0.633874in}{0.569136in}}{\pgfqpoint{2.177280in}{2.201755in}}%
\pgfusepath{clip}%
\pgfsetbuttcap%
\pgfsetroundjoin%
\definecolor{currentfill}{rgb}{0.121569,0.466667,0.705882}%
\pgfsetfillcolor{currentfill}%
\pgfsetlinewidth{0.481800pt}%
\definecolor{currentstroke}{rgb}{1.000000,1.000000,1.000000}%
\pgfsetstrokecolor{currentstroke}%
\pgfsetdash{}{0pt}%
\pgfpathmoveto{\pgfqpoint{1.205550in}{0.794348in}}%
\pgfpathcurveto{\pgfqpoint{1.216600in}{0.794348in}}{\pgfqpoint{1.227199in}{0.798739in}}{\pgfqpoint{1.235013in}{0.806552in}}%
\pgfpathcurveto{\pgfqpoint{1.242826in}{0.814366in}}{\pgfqpoint{1.247217in}{0.824965in}}{\pgfqpoint{1.247217in}{0.836015in}}%
\pgfpathcurveto{\pgfqpoint{1.247217in}{0.847065in}}{\pgfqpoint{1.242826in}{0.857664in}}{\pgfqpoint{1.235013in}{0.865478in}}%
\pgfpathcurveto{\pgfqpoint{1.227199in}{0.873291in}}{\pgfqpoint{1.216600in}{0.877682in}}{\pgfqpoint{1.205550in}{0.877682in}}%
\pgfpathcurveto{\pgfqpoint{1.194500in}{0.877682in}}{\pgfqpoint{1.183901in}{0.873291in}}{\pgfqpoint{1.176087in}{0.865478in}}%
\pgfpathcurveto{\pgfqpoint{1.168274in}{0.857664in}}{\pgfqpoint{1.163883in}{0.847065in}}{\pgfqpoint{1.163883in}{0.836015in}}%
\pgfpathcurveto{\pgfqpoint{1.163883in}{0.824965in}}{\pgfqpoint{1.168274in}{0.814366in}}{\pgfqpoint{1.176087in}{0.806552in}}%
\pgfpathcurveto{\pgfqpoint{1.183901in}{0.798739in}}{\pgfqpoint{1.194500in}{0.794348in}}{\pgfqpoint{1.205550in}{0.794348in}}%
\pgfpathlineto{\pgfqpoint{1.205550in}{0.794348in}}%
\pgfpathclose%
\pgfusepath{stroke,fill}%
\end{pgfscope}%
\begin{pgfscope}%
\pgfpathrectangle{\pgfqpoint{0.633874in}{0.569136in}}{\pgfqpoint{2.177280in}{2.201755in}}%
\pgfusepath{clip}%
\pgfsetbuttcap%
\pgfsetroundjoin%
\definecolor{currentfill}{rgb}{0.121569,0.466667,0.705882}%
\pgfsetfillcolor{currentfill}%
\pgfsetlinewidth{0.481800pt}%
\definecolor{currentstroke}{rgb}{1.000000,1.000000,1.000000}%
\pgfsetstrokecolor{currentstroke}%
\pgfsetdash{}{0pt}%
\pgfpathmoveto{\pgfqpoint{1.005857in}{0.794348in}}%
\pgfpathcurveto{\pgfqpoint{1.016907in}{0.794348in}}{\pgfqpoint{1.027506in}{0.798739in}}{\pgfqpoint{1.035319in}{0.806552in}}%
\pgfpathcurveto{\pgfqpoint{1.043133in}{0.814366in}}{\pgfqpoint{1.047523in}{0.824965in}}{\pgfqpoint{1.047523in}{0.836015in}}%
\pgfpathcurveto{\pgfqpoint{1.047523in}{0.847065in}}{\pgfqpoint{1.043133in}{0.857664in}}{\pgfqpoint{1.035319in}{0.865478in}}%
\pgfpathcurveto{\pgfqpoint{1.027506in}{0.873291in}}{\pgfqpoint{1.016907in}{0.877682in}}{\pgfqpoint{1.005857in}{0.877682in}}%
\pgfpathcurveto{\pgfqpoint{0.994806in}{0.877682in}}{\pgfqpoint{0.984207in}{0.873291in}}{\pgfqpoint{0.976394in}{0.865478in}}%
\pgfpathcurveto{\pgfqpoint{0.968580in}{0.857664in}}{\pgfqpoint{0.964190in}{0.847065in}}{\pgfqpoint{0.964190in}{0.836015in}}%
\pgfpathcurveto{\pgfqpoint{0.964190in}{0.824965in}}{\pgfqpoint{0.968580in}{0.814366in}}{\pgfqpoint{0.976394in}{0.806552in}}%
\pgfpathcurveto{\pgfqpoint{0.984207in}{0.798739in}}{\pgfqpoint{0.994806in}{0.794348in}}{\pgfqpoint{1.005857in}{0.794348in}}%
\pgfpathlineto{\pgfqpoint{1.005857in}{0.794348in}}%
\pgfpathclose%
\pgfusepath{stroke,fill}%
\end{pgfscope}%
\begin{pgfscope}%
\pgfpathrectangle{\pgfqpoint{0.633874in}{0.569136in}}{\pgfqpoint{2.177280in}{2.201755in}}%
\pgfusepath{clip}%
\pgfsetbuttcap%
\pgfsetroundjoin%
\definecolor{currentfill}{rgb}{0.121569,0.466667,0.705882}%
\pgfsetfillcolor{currentfill}%
\pgfsetlinewidth{0.481800pt}%
\definecolor{currentstroke}{rgb}{1.000000,1.000000,1.000000}%
\pgfsetstrokecolor{currentstroke}%
\pgfsetdash{}{0pt}%
\pgfpathmoveto{\pgfqpoint{0.965918in}{0.710948in}}%
\pgfpathcurveto{\pgfqpoint{0.976968in}{0.710948in}}{\pgfqpoint{0.987567in}{0.715339in}}{\pgfqpoint{0.995381in}{0.723152in}}%
\pgfpathcurveto{\pgfqpoint{1.003194in}{0.730966in}}{\pgfqpoint{1.007585in}{0.741565in}}{\pgfqpoint{1.007585in}{0.752615in}}%
\pgfpathcurveto{\pgfqpoint{1.007585in}{0.763665in}}{\pgfqpoint{1.003194in}{0.774264in}}{\pgfqpoint{0.995381in}{0.782078in}}%
\pgfpathcurveto{\pgfqpoint{0.987567in}{0.789892in}}{\pgfqpoint{0.976968in}{0.794282in}}{\pgfqpoint{0.965918in}{0.794282in}}%
\pgfpathcurveto{\pgfqpoint{0.954868in}{0.794282in}}{\pgfqpoint{0.944269in}{0.789892in}}{\pgfqpoint{0.936455in}{0.782078in}}%
\pgfpathcurveto{\pgfqpoint{0.928641in}{0.774264in}}{\pgfqpoint{0.924251in}{0.763665in}}{\pgfqpoint{0.924251in}{0.752615in}}%
\pgfpathcurveto{\pgfqpoint{0.924251in}{0.741565in}}{\pgfqpoint{0.928641in}{0.730966in}}{\pgfqpoint{0.936455in}{0.723152in}}%
\pgfpathcurveto{\pgfqpoint{0.944269in}{0.715339in}}{\pgfqpoint{0.954868in}{0.710948in}}{\pgfqpoint{0.965918in}{0.710948in}}%
\pgfpathlineto{\pgfqpoint{0.965918in}{0.710948in}}%
\pgfpathclose%
\pgfusepath{stroke,fill}%
\end{pgfscope}%
\begin{pgfscope}%
\pgfpathrectangle{\pgfqpoint{0.633874in}{0.569136in}}{\pgfqpoint{2.177280in}{2.201755in}}%
\pgfusepath{clip}%
\pgfsetbuttcap%
\pgfsetroundjoin%
\definecolor{currentfill}{rgb}{0.121569,0.466667,0.705882}%
\pgfsetfillcolor{currentfill}%
\pgfsetlinewidth{0.481800pt}%
\definecolor{currentstroke}{rgb}{1.000000,1.000000,1.000000}%
\pgfsetstrokecolor{currentstroke}%
\pgfsetdash{}{0pt}%
\pgfpathmoveto{\pgfqpoint{1.205550in}{1.044548in}}%
\pgfpathcurveto{\pgfqpoint{1.216600in}{1.044548in}}{\pgfqpoint{1.227199in}{1.048938in}}{\pgfqpoint{1.235013in}{1.056752in}}%
\pgfpathcurveto{\pgfqpoint{1.242826in}{1.064565in}}{\pgfqpoint{1.247217in}{1.075164in}}{\pgfqpoint{1.247217in}{1.086214in}}%
\pgfpathcurveto{\pgfqpoint{1.247217in}{1.097265in}}{\pgfqpoint{1.242826in}{1.107864in}}{\pgfqpoint{1.235013in}{1.115677in}}%
\pgfpathcurveto{\pgfqpoint{1.227199in}{1.123491in}}{\pgfqpoint{1.216600in}{1.127881in}}{\pgfqpoint{1.205550in}{1.127881in}}%
\pgfpathcurveto{\pgfqpoint{1.194500in}{1.127881in}}{\pgfqpoint{1.183901in}{1.123491in}}{\pgfqpoint{1.176087in}{1.115677in}}%
\pgfpathcurveto{\pgfqpoint{1.168274in}{1.107864in}}{\pgfqpoint{1.163883in}{1.097265in}}{\pgfqpoint{1.163883in}{1.086214in}}%
\pgfpathcurveto{\pgfqpoint{1.163883in}{1.075164in}}{\pgfqpoint{1.168274in}{1.064565in}}{\pgfqpoint{1.176087in}{1.056752in}}%
\pgfpathcurveto{\pgfqpoint{1.183901in}{1.048938in}}{\pgfqpoint{1.194500in}{1.044548in}}{\pgfqpoint{1.205550in}{1.044548in}}%
\pgfpathlineto{\pgfqpoint{1.205550in}{1.044548in}}%
\pgfpathclose%
\pgfusepath{stroke,fill}%
\end{pgfscope}%
\begin{pgfscope}%
\pgfpathrectangle{\pgfqpoint{0.633874in}{0.569136in}}{\pgfqpoint{2.177280in}{2.201755in}}%
\pgfusepath{clip}%
\pgfsetbuttcap%
\pgfsetroundjoin%
\definecolor{currentfill}{rgb}{0.121569,0.466667,0.705882}%
\pgfsetfillcolor{currentfill}%
\pgfsetlinewidth{0.481800pt}%
\definecolor{currentstroke}{rgb}{1.000000,1.000000,1.000000}%
\pgfsetstrokecolor{currentstroke}%
\pgfsetdash{}{0pt}%
\pgfpathmoveto{\pgfqpoint{1.245489in}{0.877748in}}%
\pgfpathcurveto{\pgfqpoint{1.256539in}{0.877748in}}{\pgfqpoint{1.267138in}{0.882138in}}{\pgfqpoint{1.274952in}{0.889952in}}%
\pgfpathcurveto{\pgfqpoint{1.282765in}{0.897766in}}{\pgfqpoint{1.287155in}{0.908365in}}{\pgfqpoint{1.287155in}{0.919415in}}%
\pgfpathcurveto{\pgfqpoint{1.287155in}{0.930465in}}{\pgfqpoint{1.282765in}{0.941064in}}{\pgfqpoint{1.274952in}{0.948878in}}%
\pgfpathcurveto{\pgfqpoint{1.267138in}{0.956691in}}{\pgfqpoint{1.256539in}{0.961081in}}{\pgfqpoint{1.245489in}{0.961081in}}%
\pgfpathcurveto{\pgfqpoint{1.234439in}{0.961081in}}{\pgfqpoint{1.223840in}{0.956691in}}{\pgfqpoint{1.216026in}{0.948878in}}%
\pgfpathcurveto{\pgfqpoint{1.208212in}{0.941064in}}{\pgfqpoint{1.203822in}{0.930465in}}{\pgfqpoint{1.203822in}{0.919415in}}%
\pgfpathcurveto{\pgfqpoint{1.203822in}{0.908365in}}{\pgfqpoint{1.208212in}{0.897766in}}{\pgfqpoint{1.216026in}{0.889952in}}%
\pgfpathcurveto{\pgfqpoint{1.223840in}{0.882138in}}{\pgfqpoint{1.234439in}{0.877748in}}{\pgfqpoint{1.245489in}{0.877748in}}%
\pgfpathlineto{\pgfqpoint{1.245489in}{0.877748in}}%
\pgfpathclose%
\pgfusepath{stroke,fill}%
\end{pgfscope}%
\begin{pgfscope}%
\pgfpathrectangle{\pgfqpoint{0.633874in}{0.569136in}}{\pgfqpoint{2.177280in}{2.201755in}}%
\pgfusepath{clip}%
\pgfsetbuttcap%
\pgfsetroundjoin%
\definecolor{currentfill}{rgb}{0.121569,0.466667,0.705882}%
\pgfsetfillcolor{currentfill}%
\pgfsetlinewidth{0.481800pt}%
\definecolor{currentstroke}{rgb}{1.000000,1.000000,1.000000}%
\pgfsetstrokecolor{currentstroke}%
\pgfsetdash{}{0pt}%
\pgfpathmoveto{\pgfqpoint{1.125673in}{0.794348in}}%
\pgfpathcurveto{\pgfqpoint{1.136723in}{0.794348in}}{\pgfqpoint{1.147322in}{0.798739in}}{\pgfqpoint{1.155135in}{0.806552in}}%
\pgfpathcurveto{\pgfqpoint{1.162949in}{0.814366in}}{\pgfqpoint{1.167339in}{0.824965in}}{\pgfqpoint{1.167339in}{0.836015in}}%
\pgfpathcurveto{\pgfqpoint{1.167339in}{0.847065in}}{\pgfqpoint{1.162949in}{0.857664in}}{\pgfqpoint{1.155135in}{0.865478in}}%
\pgfpathcurveto{\pgfqpoint{1.147322in}{0.873291in}}{\pgfqpoint{1.136723in}{0.877682in}}{\pgfqpoint{1.125673in}{0.877682in}}%
\pgfpathcurveto{\pgfqpoint{1.114623in}{0.877682in}}{\pgfqpoint{1.104024in}{0.873291in}}{\pgfqpoint{1.096210in}{0.865478in}}%
\pgfpathcurveto{\pgfqpoint{1.088396in}{0.857664in}}{\pgfqpoint{1.084006in}{0.847065in}}{\pgfqpoint{1.084006in}{0.836015in}}%
\pgfpathcurveto{\pgfqpoint{1.084006in}{0.824965in}}{\pgfqpoint{1.088396in}{0.814366in}}{\pgfqpoint{1.096210in}{0.806552in}}%
\pgfpathcurveto{\pgfqpoint{1.104024in}{0.798739in}}{\pgfqpoint{1.114623in}{0.794348in}}{\pgfqpoint{1.125673in}{0.794348in}}%
\pgfpathlineto{\pgfqpoint{1.125673in}{0.794348in}}%
\pgfpathclose%
\pgfusepath{stroke,fill}%
\end{pgfscope}%
\begin{pgfscope}%
\pgfpathrectangle{\pgfqpoint{0.633874in}{0.569136in}}{\pgfqpoint{2.177280in}{2.201755in}}%
\pgfusepath{clip}%
\pgfsetbuttcap%
\pgfsetroundjoin%
\definecolor{currentfill}{rgb}{0.121569,0.466667,0.705882}%
\pgfsetfillcolor{currentfill}%
\pgfsetlinewidth{0.481800pt}%
\definecolor{currentstroke}{rgb}{1.000000,1.000000,1.000000}%
\pgfsetstrokecolor{currentstroke}%
\pgfsetdash{}{0pt}%
\pgfpathmoveto{\pgfqpoint{1.245489in}{0.710948in}}%
\pgfpathcurveto{\pgfqpoint{1.256539in}{0.710948in}}{\pgfqpoint{1.267138in}{0.715339in}}{\pgfqpoint{1.274952in}{0.723152in}}%
\pgfpathcurveto{\pgfqpoint{1.282765in}{0.730966in}}{\pgfqpoint{1.287155in}{0.741565in}}{\pgfqpoint{1.287155in}{0.752615in}}%
\pgfpathcurveto{\pgfqpoint{1.287155in}{0.763665in}}{\pgfqpoint{1.282765in}{0.774264in}}{\pgfqpoint{1.274952in}{0.782078in}}%
\pgfpathcurveto{\pgfqpoint{1.267138in}{0.789892in}}{\pgfqpoint{1.256539in}{0.794282in}}{\pgfqpoint{1.245489in}{0.794282in}}%
\pgfpathcurveto{\pgfqpoint{1.234439in}{0.794282in}}{\pgfqpoint{1.223840in}{0.789892in}}{\pgfqpoint{1.216026in}{0.782078in}}%
\pgfpathcurveto{\pgfqpoint{1.208212in}{0.774264in}}{\pgfqpoint{1.203822in}{0.763665in}}{\pgfqpoint{1.203822in}{0.752615in}}%
\pgfpathcurveto{\pgfqpoint{1.203822in}{0.741565in}}{\pgfqpoint{1.208212in}{0.730966in}}{\pgfqpoint{1.216026in}{0.723152in}}%
\pgfpathcurveto{\pgfqpoint{1.223840in}{0.715339in}}{\pgfqpoint{1.234439in}{0.710948in}}{\pgfqpoint{1.245489in}{0.710948in}}%
\pgfpathlineto{\pgfqpoint{1.245489in}{0.710948in}}%
\pgfpathclose%
\pgfusepath{stroke,fill}%
\end{pgfscope}%
\begin{pgfscope}%
\pgfpathrectangle{\pgfqpoint{0.633874in}{0.569136in}}{\pgfqpoint{2.177280in}{2.201755in}}%
\pgfusepath{clip}%
\pgfsetbuttcap%
\pgfsetroundjoin%
\definecolor{currentfill}{rgb}{0.121569,0.466667,0.705882}%
\pgfsetfillcolor{currentfill}%
\pgfsetlinewidth{0.481800pt}%
\definecolor{currentstroke}{rgb}{1.000000,1.000000,1.000000}%
\pgfsetstrokecolor{currentstroke}%
\pgfsetdash{}{0pt}%
\pgfpathmoveto{\pgfqpoint{1.045795in}{0.710948in}}%
\pgfpathcurveto{\pgfqpoint{1.056845in}{0.710948in}}{\pgfqpoint{1.067444in}{0.715339in}}{\pgfqpoint{1.075258in}{0.723152in}}%
\pgfpathcurveto{\pgfqpoint{1.083072in}{0.730966in}}{\pgfqpoint{1.087462in}{0.741565in}}{\pgfqpoint{1.087462in}{0.752615in}}%
\pgfpathcurveto{\pgfqpoint{1.087462in}{0.763665in}}{\pgfqpoint{1.083072in}{0.774264in}}{\pgfqpoint{1.075258in}{0.782078in}}%
\pgfpathcurveto{\pgfqpoint{1.067444in}{0.789892in}}{\pgfqpoint{1.056845in}{0.794282in}}{\pgfqpoint{1.045795in}{0.794282in}}%
\pgfpathcurveto{\pgfqpoint{1.034745in}{0.794282in}}{\pgfqpoint{1.024146in}{0.789892in}}{\pgfqpoint{1.016332in}{0.782078in}}%
\pgfpathcurveto{\pgfqpoint{1.008519in}{0.774264in}}{\pgfqpoint{1.004129in}{0.763665in}}{\pgfqpoint{1.004129in}{0.752615in}}%
\pgfpathcurveto{\pgfqpoint{1.004129in}{0.741565in}}{\pgfqpoint{1.008519in}{0.730966in}}{\pgfqpoint{1.016332in}{0.723152in}}%
\pgfpathcurveto{\pgfqpoint{1.024146in}{0.715339in}}{\pgfqpoint{1.034745in}{0.710948in}}{\pgfqpoint{1.045795in}{0.710948in}}%
\pgfpathlineto{\pgfqpoint{1.045795in}{0.710948in}}%
\pgfpathclose%
\pgfusepath{stroke,fill}%
\end{pgfscope}%
\begin{pgfscope}%
\pgfpathrectangle{\pgfqpoint{0.633874in}{0.569136in}}{\pgfqpoint{2.177280in}{2.201755in}}%
\pgfusepath{clip}%
\pgfsetbuttcap%
\pgfsetroundjoin%
\definecolor{currentfill}{rgb}{0.121569,0.466667,0.705882}%
\pgfsetfillcolor{currentfill}%
\pgfsetlinewidth{0.481800pt}%
\definecolor{currentstroke}{rgb}{1.000000,1.000000,1.000000}%
\pgfsetstrokecolor{currentstroke}%
\pgfsetdash{}{0pt}%
\pgfpathmoveto{\pgfqpoint{1.325366in}{0.710948in}}%
\pgfpathcurveto{\pgfqpoint{1.336416in}{0.710948in}}{\pgfqpoint{1.347015in}{0.715339in}}{\pgfqpoint{1.354829in}{0.723152in}}%
\pgfpathcurveto{\pgfqpoint{1.362643in}{0.730966in}}{\pgfqpoint{1.367033in}{0.741565in}}{\pgfqpoint{1.367033in}{0.752615in}}%
\pgfpathcurveto{\pgfqpoint{1.367033in}{0.763665in}}{\pgfqpoint{1.362643in}{0.774264in}}{\pgfqpoint{1.354829in}{0.782078in}}%
\pgfpathcurveto{\pgfqpoint{1.347015in}{0.789892in}}{\pgfqpoint{1.336416in}{0.794282in}}{\pgfqpoint{1.325366in}{0.794282in}}%
\pgfpathcurveto{\pgfqpoint{1.314316in}{0.794282in}}{\pgfqpoint{1.303717in}{0.789892in}}{\pgfqpoint{1.295903in}{0.782078in}}%
\pgfpathcurveto{\pgfqpoint{1.288090in}{0.774264in}}{\pgfqpoint{1.283700in}{0.763665in}}{\pgfqpoint{1.283700in}{0.752615in}}%
\pgfpathcurveto{\pgfqpoint{1.283700in}{0.741565in}}{\pgfqpoint{1.288090in}{0.730966in}}{\pgfqpoint{1.295903in}{0.723152in}}%
\pgfpathcurveto{\pgfqpoint{1.303717in}{0.715339in}}{\pgfqpoint{1.314316in}{0.710948in}}{\pgfqpoint{1.325366in}{0.710948in}}%
\pgfpathlineto{\pgfqpoint{1.325366in}{0.710948in}}%
\pgfpathclose%
\pgfusepath{stroke,fill}%
\end{pgfscope}%
\begin{pgfscope}%
\pgfpathrectangle{\pgfqpoint{0.633874in}{0.569136in}}{\pgfqpoint{2.177280in}{2.201755in}}%
\pgfusepath{clip}%
\pgfsetbuttcap%
\pgfsetroundjoin%
\definecolor{currentfill}{rgb}{0.121569,0.466667,0.705882}%
\pgfsetfillcolor{currentfill}%
\pgfsetlinewidth{0.481800pt}%
\definecolor{currentstroke}{rgb}{1.000000,1.000000,1.000000}%
\pgfsetstrokecolor{currentstroke}%
\pgfsetdash{}{0pt}%
\pgfpathmoveto{\pgfqpoint{1.205550in}{0.710948in}}%
\pgfpathcurveto{\pgfqpoint{1.216600in}{0.710948in}}{\pgfqpoint{1.227199in}{0.715339in}}{\pgfqpoint{1.235013in}{0.723152in}}%
\pgfpathcurveto{\pgfqpoint{1.242826in}{0.730966in}}{\pgfqpoint{1.247217in}{0.741565in}}{\pgfqpoint{1.247217in}{0.752615in}}%
\pgfpathcurveto{\pgfqpoint{1.247217in}{0.763665in}}{\pgfqpoint{1.242826in}{0.774264in}}{\pgfqpoint{1.235013in}{0.782078in}}%
\pgfpathcurveto{\pgfqpoint{1.227199in}{0.789892in}}{\pgfqpoint{1.216600in}{0.794282in}}{\pgfqpoint{1.205550in}{0.794282in}}%
\pgfpathcurveto{\pgfqpoint{1.194500in}{0.794282in}}{\pgfqpoint{1.183901in}{0.789892in}}{\pgfqpoint{1.176087in}{0.782078in}}%
\pgfpathcurveto{\pgfqpoint{1.168274in}{0.774264in}}{\pgfqpoint{1.163883in}{0.763665in}}{\pgfqpoint{1.163883in}{0.752615in}}%
\pgfpathcurveto{\pgfqpoint{1.163883in}{0.741565in}}{\pgfqpoint{1.168274in}{0.730966in}}{\pgfqpoint{1.176087in}{0.723152in}}%
\pgfpathcurveto{\pgfqpoint{1.183901in}{0.715339in}}{\pgfqpoint{1.194500in}{0.710948in}}{\pgfqpoint{1.205550in}{0.710948in}}%
\pgfpathlineto{\pgfqpoint{1.205550in}{0.710948in}}%
\pgfpathclose%
\pgfusepath{stroke,fill}%
\end{pgfscope}%
\begin{pgfscope}%
\pgfpathrectangle{\pgfqpoint{0.633874in}{0.569136in}}{\pgfqpoint{2.177280in}{2.201755in}}%
\pgfusepath{clip}%
\pgfsetbuttcap%
\pgfsetroundjoin%
\definecolor{currentfill}{rgb}{1.000000,0.498039,0.054902}%
\pgfsetfillcolor{currentfill}%
\pgfsetlinewidth{0.481800pt}%
\definecolor{currentstroke}{rgb}{1.000000,1.000000,1.000000}%
\pgfsetstrokecolor{currentstroke}%
\pgfsetdash{}{0pt}%
\pgfpathmoveto{\pgfqpoint{2.004324in}{1.711746in}}%
\pgfpathcurveto{\pgfqpoint{2.015374in}{1.711746in}}{\pgfqpoint{2.025973in}{1.716137in}}{\pgfqpoint{2.033787in}{1.723950in}}%
\pgfpathcurveto{\pgfqpoint{2.041601in}{1.731764in}}{\pgfqpoint{2.045991in}{1.742363in}}{\pgfqpoint{2.045991in}{1.753413in}}%
\pgfpathcurveto{\pgfqpoint{2.045991in}{1.764463in}}{\pgfqpoint{2.041601in}{1.775062in}}{\pgfqpoint{2.033787in}{1.782876in}}%
\pgfpathcurveto{\pgfqpoint{2.025973in}{1.790689in}}{\pgfqpoint{2.015374in}{1.795080in}}{\pgfqpoint{2.004324in}{1.795080in}}%
\pgfpathcurveto{\pgfqpoint{1.993274in}{1.795080in}}{\pgfqpoint{1.982675in}{1.790689in}}{\pgfqpoint{1.974861in}{1.782876in}}%
\pgfpathcurveto{\pgfqpoint{1.967048in}{1.775062in}}{\pgfqpoint{1.962658in}{1.764463in}}{\pgfqpoint{1.962658in}{1.753413in}}%
\pgfpathcurveto{\pgfqpoint{1.962658in}{1.742363in}}{\pgfqpoint{1.967048in}{1.731764in}}{\pgfqpoint{1.974861in}{1.723950in}}%
\pgfpathcurveto{\pgfqpoint{1.982675in}{1.716137in}}{\pgfqpoint{1.993274in}{1.711746in}}{\pgfqpoint{2.004324in}{1.711746in}}%
\pgfpathlineto{\pgfqpoint{2.004324in}{1.711746in}}%
\pgfpathclose%
\pgfusepath{stroke,fill}%
\end{pgfscope}%
\begin{pgfscope}%
\pgfpathrectangle{\pgfqpoint{0.633874in}{0.569136in}}{\pgfqpoint{2.177280in}{2.201755in}}%
\pgfusepath{clip}%
\pgfsetbuttcap%
\pgfsetroundjoin%
\definecolor{currentfill}{rgb}{1.000000,0.498039,0.054902}%
\pgfsetfillcolor{currentfill}%
\pgfsetlinewidth{0.481800pt}%
\definecolor{currentstroke}{rgb}{1.000000,1.000000,1.000000}%
\pgfsetstrokecolor{currentstroke}%
\pgfsetdash{}{0pt}%
\pgfpathmoveto{\pgfqpoint{1.764692in}{1.795146in}}%
\pgfpathcurveto{\pgfqpoint{1.775742in}{1.795146in}}{\pgfqpoint{1.786341in}{1.799536in}}{\pgfqpoint{1.794155in}{1.807350in}}%
\pgfpathcurveto{\pgfqpoint{1.801968in}{1.815164in}}{\pgfqpoint{1.806359in}{1.825763in}}{\pgfqpoint{1.806359in}{1.836813in}}%
\pgfpathcurveto{\pgfqpoint{1.806359in}{1.847863in}}{\pgfqpoint{1.801968in}{1.858462in}}{\pgfqpoint{1.794155in}{1.866276in}}%
\pgfpathcurveto{\pgfqpoint{1.786341in}{1.874089in}}{\pgfqpoint{1.775742in}{1.878479in}}{\pgfqpoint{1.764692in}{1.878479in}}%
\pgfpathcurveto{\pgfqpoint{1.753642in}{1.878479in}}{\pgfqpoint{1.743043in}{1.874089in}}{\pgfqpoint{1.735229in}{1.866276in}}%
\pgfpathcurveto{\pgfqpoint{1.727416in}{1.858462in}}{\pgfqpoint{1.723025in}{1.847863in}}{\pgfqpoint{1.723025in}{1.836813in}}%
\pgfpathcurveto{\pgfqpoint{1.723025in}{1.825763in}}{\pgfqpoint{1.727416in}{1.815164in}}{\pgfqpoint{1.735229in}{1.807350in}}%
\pgfpathcurveto{\pgfqpoint{1.743043in}{1.799536in}}{\pgfqpoint{1.753642in}{1.795146in}}{\pgfqpoint{1.764692in}{1.795146in}}%
\pgfpathlineto{\pgfqpoint{1.764692in}{1.795146in}}%
\pgfpathclose%
\pgfusepath{stroke,fill}%
\end{pgfscope}%
\begin{pgfscope}%
\pgfpathrectangle{\pgfqpoint{0.633874in}{0.569136in}}{\pgfqpoint{2.177280in}{2.201755in}}%
\pgfusepath{clip}%
\pgfsetbuttcap%
\pgfsetroundjoin%
\definecolor{currentfill}{rgb}{1.000000,0.498039,0.054902}%
\pgfsetfillcolor{currentfill}%
\pgfsetlinewidth{0.481800pt}%
\definecolor{currentstroke}{rgb}{1.000000,1.000000,1.000000}%
\pgfsetstrokecolor{currentstroke}%
\pgfsetdash{}{0pt}%
\pgfpathmoveto{\pgfqpoint{1.964385in}{1.795146in}}%
\pgfpathcurveto{\pgfqpoint{1.975436in}{1.795146in}}{\pgfqpoint{1.986035in}{1.799536in}}{\pgfqpoint{1.993848in}{1.807350in}}%
\pgfpathcurveto{\pgfqpoint{2.001662in}{1.815164in}}{\pgfqpoint{2.006052in}{1.825763in}}{\pgfqpoint{2.006052in}{1.836813in}}%
\pgfpathcurveto{\pgfqpoint{2.006052in}{1.847863in}}{\pgfqpoint{2.001662in}{1.858462in}}{\pgfqpoint{1.993848in}{1.866276in}}%
\pgfpathcurveto{\pgfqpoint{1.986035in}{1.874089in}}{\pgfqpoint{1.975436in}{1.878479in}}{\pgfqpoint{1.964385in}{1.878479in}}%
\pgfpathcurveto{\pgfqpoint{1.953335in}{1.878479in}}{\pgfqpoint{1.942736in}{1.874089in}}{\pgfqpoint{1.934923in}{1.866276in}}%
\pgfpathcurveto{\pgfqpoint{1.927109in}{1.858462in}}{\pgfqpoint{1.922719in}{1.847863in}}{\pgfqpoint{1.922719in}{1.836813in}}%
\pgfpathcurveto{\pgfqpoint{1.922719in}{1.825763in}}{\pgfqpoint{1.927109in}{1.815164in}}{\pgfqpoint{1.934923in}{1.807350in}}%
\pgfpathcurveto{\pgfqpoint{1.942736in}{1.799536in}}{\pgfqpoint{1.953335in}{1.795146in}}{\pgfqpoint{1.964385in}{1.795146in}}%
\pgfpathlineto{\pgfqpoint{1.964385in}{1.795146in}}%
\pgfpathclose%
\pgfusepath{stroke,fill}%
\end{pgfscope}%
\begin{pgfscope}%
\pgfpathrectangle{\pgfqpoint{0.633874in}{0.569136in}}{\pgfqpoint{2.177280in}{2.201755in}}%
\pgfusepath{clip}%
\pgfsetbuttcap%
\pgfsetroundjoin%
\definecolor{currentfill}{rgb}{1.000000,0.498039,0.054902}%
\pgfsetfillcolor{currentfill}%
\pgfsetlinewidth{0.481800pt}%
\definecolor{currentstroke}{rgb}{1.000000,1.000000,1.000000}%
\pgfsetstrokecolor{currentstroke}%
\pgfsetdash{}{0pt}%
\pgfpathmoveto{\pgfqpoint{1.405244in}{1.628346in}}%
\pgfpathcurveto{\pgfqpoint{1.416294in}{1.628346in}}{\pgfqpoint{1.426893in}{1.632737in}}{\pgfqpoint{1.434706in}{1.640550in}}%
\pgfpathcurveto{\pgfqpoint{1.442520in}{1.648364in}}{\pgfqpoint{1.446910in}{1.658963in}}{\pgfqpoint{1.446910in}{1.670013in}}%
\pgfpathcurveto{\pgfqpoint{1.446910in}{1.681063in}}{\pgfqpoint{1.442520in}{1.691662in}}{\pgfqpoint{1.434706in}{1.699476in}}%
\pgfpathcurveto{\pgfqpoint{1.426893in}{1.707290in}}{\pgfqpoint{1.416294in}{1.711680in}}{\pgfqpoint{1.405244in}{1.711680in}}%
\pgfpathcurveto{\pgfqpoint{1.394193in}{1.711680in}}{\pgfqpoint{1.383594in}{1.707290in}}{\pgfqpoint{1.375781in}{1.699476in}}%
\pgfpathcurveto{\pgfqpoint{1.367967in}{1.691662in}}{\pgfqpoint{1.363577in}{1.681063in}}{\pgfqpoint{1.363577in}{1.670013in}}%
\pgfpathcurveto{\pgfqpoint{1.363577in}{1.658963in}}{\pgfqpoint{1.367967in}{1.648364in}}{\pgfqpoint{1.375781in}{1.640550in}}%
\pgfpathcurveto{\pgfqpoint{1.383594in}{1.632737in}}{\pgfqpoint{1.394193in}{1.628346in}}{\pgfqpoint{1.405244in}{1.628346in}}%
\pgfpathlineto{\pgfqpoint{1.405244in}{1.628346in}}%
\pgfpathclose%
\pgfusepath{stroke,fill}%
\end{pgfscope}%
\begin{pgfscope}%
\pgfpathrectangle{\pgfqpoint{0.633874in}{0.569136in}}{\pgfqpoint{2.177280in}{2.201755in}}%
\pgfusepath{clip}%
\pgfsetbuttcap%
\pgfsetroundjoin%
\definecolor{currentfill}{rgb}{1.000000,0.498039,0.054902}%
\pgfsetfillcolor{currentfill}%
\pgfsetlinewidth{0.481800pt}%
\definecolor{currentstroke}{rgb}{1.000000,1.000000,1.000000}%
\pgfsetstrokecolor{currentstroke}%
\pgfsetdash{}{0pt}%
\pgfpathmoveto{\pgfqpoint{1.804631in}{1.795146in}}%
\pgfpathcurveto{\pgfqpoint{1.815681in}{1.795146in}}{\pgfqpoint{1.826280in}{1.799536in}}{\pgfqpoint{1.834093in}{1.807350in}}%
\pgfpathcurveto{\pgfqpoint{1.841907in}{1.815164in}}{\pgfqpoint{1.846297in}{1.825763in}}{\pgfqpoint{1.846297in}{1.836813in}}%
\pgfpathcurveto{\pgfqpoint{1.846297in}{1.847863in}}{\pgfqpoint{1.841907in}{1.858462in}}{\pgfqpoint{1.834093in}{1.866276in}}%
\pgfpathcurveto{\pgfqpoint{1.826280in}{1.874089in}}{\pgfqpoint{1.815681in}{1.878479in}}{\pgfqpoint{1.804631in}{1.878479in}}%
\pgfpathcurveto{\pgfqpoint{1.793581in}{1.878479in}}{\pgfqpoint{1.782981in}{1.874089in}}{\pgfqpoint{1.775168in}{1.866276in}}%
\pgfpathcurveto{\pgfqpoint{1.767354in}{1.858462in}}{\pgfqpoint{1.762964in}{1.847863in}}{\pgfqpoint{1.762964in}{1.836813in}}%
\pgfpathcurveto{\pgfqpoint{1.762964in}{1.825763in}}{\pgfqpoint{1.767354in}{1.815164in}}{\pgfqpoint{1.775168in}{1.807350in}}%
\pgfpathcurveto{\pgfqpoint{1.782981in}{1.799536in}}{\pgfqpoint{1.793581in}{1.795146in}}{\pgfqpoint{1.804631in}{1.795146in}}%
\pgfpathlineto{\pgfqpoint{1.804631in}{1.795146in}}%
\pgfpathclose%
\pgfusepath{stroke,fill}%
\end{pgfscope}%
\begin{pgfscope}%
\pgfpathrectangle{\pgfqpoint{0.633874in}{0.569136in}}{\pgfqpoint{2.177280in}{2.201755in}}%
\pgfusepath{clip}%
\pgfsetbuttcap%
\pgfsetroundjoin%
\definecolor{currentfill}{rgb}{1.000000,0.498039,0.054902}%
\pgfsetfillcolor{currentfill}%
\pgfsetlinewidth{0.481800pt}%
\definecolor{currentstroke}{rgb}{1.000000,1.000000,1.000000}%
\pgfsetstrokecolor{currentstroke}%
\pgfsetdash{}{0pt}%
\pgfpathmoveto{\pgfqpoint{1.485121in}{1.628346in}}%
\pgfpathcurveto{\pgfqpoint{1.496171in}{1.628346in}}{\pgfqpoint{1.506770in}{1.632737in}}{\pgfqpoint{1.514584in}{1.640550in}}%
\pgfpathcurveto{\pgfqpoint{1.522397in}{1.648364in}}{\pgfqpoint{1.526788in}{1.658963in}}{\pgfqpoint{1.526788in}{1.670013in}}%
\pgfpathcurveto{\pgfqpoint{1.526788in}{1.681063in}}{\pgfqpoint{1.522397in}{1.691662in}}{\pgfqpoint{1.514584in}{1.699476in}}%
\pgfpathcurveto{\pgfqpoint{1.506770in}{1.707290in}}{\pgfqpoint{1.496171in}{1.711680in}}{\pgfqpoint{1.485121in}{1.711680in}}%
\pgfpathcurveto{\pgfqpoint{1.474071in}{1.711680in}}{\pgfqpoint{1.463472in}{1.707290in}}{\pgfqpoint{1.455658in}{1.699476in}}%
\pgfpathcurveto{\pgfqpoint{1.447845in}{1.691662in}}{\pgfqpoint{1.443454in}{1.681063in}}{\pgfqpoint{1.443454in}{1.670013in}}%
\pgfpathcurveto{\pgfqpoint{1.443454in}{1.658963in}}{\pgfqpoint{1.447845in}{1.648364in}}{\pgfqpoint{1.455658in}{1.640550in}}%
\pgfpathcurveto{\pgfqpoint{1.463472in}{1.632737in}}{\pgfqpoint{1.474071in}{1.628346in}}{\pgfqpoint{1.485121in}{1.628346in}}%
\pgfpathlineto{\pgfqpoint{1.485121in}{1.628346in}}%
\pgfpathclose%
\pgfusepath{stroke,fill}%
\end{pgfscope}%
\begin{pgfscope}%
\pgfpathrectangle{\pgfqpoint{0.633874in}{0.569136in}}{\pgfqpoint{2.177280in}{2.201755in}}%
\pgfusepath{clip}%
\pgfsetbuttcap%
\pgfsetroundjoin%
\definecolor{currentfill}{rgb}{1.000000,0.498039,0.054902}%
\pgfsetfillcolor{currentfill}%
\pgfsetlinewidth{0.481800pt}%
\definecolor{currentstroke}{rgb}{1.000000,1.000000,1.000000}%
\pgfsetstrokecolor{currentstroke}%
\pgfsetdash{}{0pt}%
\pgfpathmoveto{\pgfqpoint{1.724753in}{1.878546in}}%
\pgfpathcurveto{\pgfqpoint{1.735803in}{1.878546in}}{\pgfqpoint{1.746402in}{1.882936in}}{\pgfqpoint{1.754216in}{1.890750in}}%
\pgfpathcurveto{\pgfqpoint{1.762030in}{1.898563in}}{\pgfqpoint{1.766420in}{1.909162in}}{\pgfqpoint{1.766420in}{1.920213in}}%
\pgfpathcurveto{\pgfqpoint{1.766420in}{1.931263in}}{\pgfqpoint{1.762030in}{1.941862in}}{\pgfqpoint{1.754216in}{1.949675in}}%
\pgfpathcurveto{\pgfqpoint{1.746402in}{1.957489in}}{\pgfqpoint{1.735803in}{1.961879in}}{\pgfqpoint{1.724753in}{1.961879in}}%
\pgfpathcurveto{\pgfqpoint{1.713703in}{1.961879in}}{\pgfqpoint{1.703104in}{1.957489in}}{\pgfqpoint{1.695290in}{1.949675in}}%
\pgfpathcurveto{\pgfqpoint{1.687477in}{1.941862in}}{\pgfqpoint{1.683087in}{1.931263in}}{\pgfqpoint{1.683087in}{1.920213in}}%
\pgfpathcurveto{\pgfqpoint{1.683087in}{1.909162in}}{\pgfqpoint{1.687477in}{1.898563in}}{\pgfqpoint{1.695290in}{1.890750in}}%
\pgfpathcurveto{\pgfqpoint{1.703104in}{1.882936in}}{\pgfqpoint{1.713703in}{1.878546in}}{\pgfqpoint{1.724753in}{1.878546in}}%
\pgfpathlineto{\pgfqpoint{1.724753in}{1.878546in}}%
\pgfpathclose%
\pgfusepath{stroke,fill}%
\end{pgfscope}%
\begin{pgfscope}%
\pgfpathrectangle{\pgfqpoint{0.633874in}{0.569136in}}{\pgfqpoint{2.177280in}{2.201755in}}%
\pgfusepath{clip}%
\pgfsetbuttcap%
\pgfsetroundjoin%
\definecolor{currentfill}{rgb}{1.000000,0.498039,0.054902}%
\pgfsetfillcolor{currentfill}%
\pgfsetlinewidth{0.481800pt}%
\definecolor{currentstroke}{rgb}{1.000000,1.000000,1.000000}%
\pgfsetstrokecolor{currentstroke}%
\pgfsetdash{}{0pt}%
\pgfpathmoveto{\pgfqpoint{1.165611in}{1.378147in}}%
\pgfpathcurveto{\pgfqpoint{1.176662in}{1.378147in}}{\pgfqpoint{1.187261in}{1.382537in}}{\pgfqpoint{1.195074in}{1.390351in}}%
\pgfpathcurveto{\pgfqpoint{1.202888in}{1.398164in}}{\pgfqpoint{1.207278in}{1.408764in}}{\pgfqpoint{1.207278in}{1.419814in}}%
\pgfpathcurveto{\pgfqpoint{1.207278in}{1.430864in}}{\pgfqpoint{1.202888in}{1.441463in}}{\pgfqpoint{1.195074in}{1.449276in}}%
\pgfpathcurveto{\pgfqpoint{1.187261in}{1.457090in}}{\pgfqpoint{1.176662in}{1.461480in}}{\pgfqpoint{1.165611in}{1.461480in}}%
\pgfpathcurveto{\pgfqpoint{1.154561in}{1.461480in}}{\pgfqpoint{1.143962in}{1.457090in}}{\pgfqpoint{1.136149in}{1.449276in}}%
\pgfpathcurveto{\pgfqpoint{1.128335in}{1.441463in}}{\pgfqpoint{1.123945in}{1.430864in}}{\pgfqpoint{1.123945in}{1.419814in}}%
\pgfpathcurveto{\pgfqpoint{1.123945in}{1.408764in}}{\pgfqpoint{1.128335in}{1.398164in}}{\pgfqpoint{1.136149in}{1.390351in}}%
\pgfpathcurveto{\pgfqpoint{1.143962in}{1.382537in}}{\pgfqpoint{1.154561in}{1.378147in}}{\pgfqpoint{1.165611in}{1.378147in}}%
\pgfpathlineto{\pgfqpoint{1.165611in}{1.378147in}}%
\pgfpathclose%
\pgfusepath{stroke,fill}%
\end{pgfscope}%
\begin{pgfscope}%
\pgfpathrectangle{\pgfqpoint{0.633874in}{0.569136in}}{\pgfqpoint{2.177280in}{2.201755in}}%
\pgfusepath{clip}%
\pgfsetbuttcap%
\pgfsetroundjoin%
\definecolor{currentfill}{rgb}{1.000000,0.498039,0.054902}%
\pgfsetfillcolor{currentfill}%
\pgfsetlinewidth{0.481800pt}%
\definecolor{currentstroke}{rgb}{1.000000,1.000000,1.000000}%
\pgfsetstrokecolor{currentstroke}%
\pgfsetdash{}{0pt}%
\pgfpathmoveto{\pgfqpoint{1.844569in}{1.628346in}}%
\pgfpathcurveto{\pgfqpoint{1.855619in}{1.628346in}}{\pgfqpoint{1.866219in}{1.632737in}}{\pgfqpoint{1.874032in}{1.640550in}}%
\pgfpathcurveto{\pgfqpoint{1.881846in}{1.648364in}}{\pgfqpoint{1.886236in}{1.658963in}}{\pgfqpoint{1.886236in}{1.670013in}}%
\pgfpathcurveto{\pgfqpoint{1.886236in}{1.681063in}}{\pgfqpoint{1.881846in}{1.691662in}}{\pgfqpoint{1.874032in}{1.699476in}}%
\pgfpathcurveto{\pgfqpoint{1.866219in}{1.707290in}}{\pgfqpoint{1.855619in}{1.711680in}}{\pgfqpoint{1.844569in}{1.711680in}}%
\pgfpathcurveto{\pgfqpoint{1.833519in}{1.711680in}}{\pgfqpoint{1.822920in}{1.707290in}}{\pgfqpoint{1.815107in}{1.699476in}}%
\pgfpathcurveto{\pgfqpoint{1.807293in}{1.691662in}}{\pgfqpoint{1.802903in}{1.681063in}}{\pgfqpoint{1.802903in}{1.670013in}}%
\pgfpathcurveto{\pgfqpoint{1.802903in}{1.658963in}}{\pgfqpoint{1.807293in}{1.648364in}}{\pgfqpoint{1.815107in}{1.640550in}}%
\pgfpathcurveto{\pgfqpoint{1.822920in}{1.632737in}}{\pgfqpoint{1.833519in}{1.628346in}}{\pgfqpoint{1.844569in}{1.628346in}}%
\pgfpathlineto{\pgfqpoint{1.844569in}{1.628346in}}%
\pgfpathclose%
\pgfusepath{stroke,fill}%
\end{pgfscope}%
\begin{pgfscope}%
\pgfpathrectangle{\pgfqpoint{0.633874in}{0.569136in}}{\pgfqpoint{2.177280in}{2.201755in}}%
\pgfusepath{clip}%
\pgfsetbuttcap%
\pgfsetroundjoin%
\definecolor{currentfill}{rgb}{1.000000,0.498039,0.054902}%
\pgfsetfillcolor{currentfill}%
\pgfsetlinewidth{0.481800pt}%
\definecolor{currentstroke}{rgb}{1.000000,1.000000,1.000000}%
\pgfsetstrokecolor{currentstroke}%
\pgfsetdash{}{0pt}%
\pgfpathmoveto{\pgfqpoint{1.285428in}{1.711746in}}%
\pgfpathcurveto{\pgfqpoint{1.296478in}{1.711746in}}{\pgfqpoint{1.307077in}{1.716137in}}{\pgfqpoint{1.314890in}{1.723950in}}%
\pgfpathcurveto{\pgfqpoint{1.322704in}{1.731764in}}{\pgfqpoint{1.327094in}{1.742363in}}{\pgfqpoint{1.327094in}{1.753413in}}%
\pgfpathcurveto{\pgfqpoint{1.327094in}{1.764463in}}{\pgfqpoint{1.322704in}{1.775062in}}{\pgfqpoint{1.314890in}{1.782876in}}%
\pgfpathcurveto{\pgfqpoint{1.307077in}{1.790689in}}{\pgfqpoint{1.296478in}{1.795080in}}{\pgfqpoint{1.285428in}{1.795080in}}%
\pgfpathcurveto{\pgfqpoint{1.274377in}{1.795080in}}{\pgfqpoint{1.263778in}{1.790689in}}{\pgfqpoint{1.255965in}{1.782876in}}%
\pgfpathcurveto{\pgfqpoint{1.248151in}{1.775062in}}{\pgfqpoint{1.243761in}{1.764463in}}{\pgfqpoint{1.243761in}{1.753413in}}%
\pgfpathcurveto{\pgfqpoint{1.243761in}{1.742363in}}{\pgfqpoint{1.248151in}{1.731764in}}{\pgfqpoint{1.255965in}{1.723950in}}%
\pgfpathcurveto{\pgfqpoint{1.263778in}{1.716137in}}{\pgfqpoint{1.274377in}{1.711746in}}{\pgfqpoint{1.285428in}{1.711746in}}%
\pgfpathlineto{\pgfqpoint{1.285428in}{1.711746in}}%
\pgfpathclose%
\pgfusepath{stroke,fill}%
\end{pgfscope}%
\begin{pgfscope}%
\pgfpathrectangle{\pgfqpoint{0.633874in}{0.569136in}}{\pgfqpoint{2.177280in}{2.201755in}}%
\pgfusepath{clip}%
\pgfsetbuttcap%
\pgfsetroundjoin%
\definecolor{currentfill}{rgb}{1.000000,0.498039,0.054902}%
\pgfsetfillcolor{currentfill}%
\pgfsetlinewidth{0.481800pt}%
\definecolor{currentstroke}{rgb}{1.000000,1.000000,1.000000}%
\pgfsetstrokecolor{currentstroke}%
\pgfsetdash{}{0pt}%
\pgfpathmoveto{\pgfqpoint{1.205550in}{1.378147in}}%
\pgfpathcurveto{\pgfqpoint{1.216600in}{1.378147in}}{\pgfqpoint{1.227199in}{1.382537in}}{\pgfqpoint{1.235013in}{1.390351in}}%
\pgfpathcurveto{\pgfqpoint{1.242826in}{1.398164in}}{\pgfqpoint{1.247217in}{1.408764in}}{\pgfqpoint{1.247217in}{1.419814in}}%
\pgfpathcurveto{\pgfqpoint{1.247217in}{1.430864in}}{\pgfqpoint{1.242826in}{1.441463in}}{\pgfqpoint{1.235013in}{1.449276in}}%
\pgfpathcurveto{\pgfqpoint{1.227199in}{1.457090in}}{\pgfqpoint{1.216600in}{1.461480in}}{\pgfqpoint{1.205550in}{1.461480in}}%
\pgfpathcurveto{\pgfqpoint{1.194500in}{1.461480in}}{\pgfqpoint{1.183901in}{1.457090in}}{\pgfqpoint{1.176087in}{1.449276in}}%
\pgfpathcurveto{\pgfqpoint{1.168274in}{1.441463in}}{\pgfqpoint{1.163883in}{1.430864in}}{\pgfqpoint{1.163883in}{1.419814in}}%
\pgfpathcurveto{\pgfqpoint{1.163883in}{1.408764in}}{\pgfqpoint{1.168274in}{1.398164in}}{\pgfqpoint{1.176087in}{1.390351in}}%
\pgfpathcurveto{\pgfqpoint{1.183901in}{1.382537in}}{\pgfqpoint{1.194500in}{1.378147in}}{\pgfqpoint{1.205550in}{1.378147in}}%
\pgfpathlineto{\pgfqpoint{1.205550in}{1.378147in}}%
\pgfpathclose%
\pgfusepath{stroke,fill}%
\end{pgfscope}%
\begin{pgfscope}%
\pgfpathrectangle{\pgfqpoint{0.633874in}{0.569136in}}{\pgfqpoint{2.177280in}{2.201755in}}%
\pgfusepath{clip}%
\pgfsetbuttcap%
\pgfsetroundjoin%
\definecolor{currentfill}{rgb}{1.000000,0.498039,0.054902}%
\pgfsetfillcolor{currentfill}%
\pgfsetlinewidth{0.481800pt}%
\definecolor{currentstroke}{rgb}{1.000000,1.000000,1.000000}%
\pgfsetstrokecolor{currentstroke}%
\pgfsetdash{}{0pt}%
\pgfpathmoveto{\pgfqpoint{1.564998in}{1.795146in}}%
\pgfpathcurveto{\pgfqpoint{1.576049in}{1.795146in}}{\pgfqpoint{1.586648in}{1.799536in}}{\pgfqpoint{1.594461in}{1.807350in}}%
\pgfpathcurveto{\pgfqpoint{1.602275in}{1.815164in}}{\pgfqpoint{1.606665in}{1.825763in}}{\pgfqpoint{1.606665in}{1.836813in}}%
\pgfpathcurveto{\pgfqpoint{1.606665in}{1.847863in}}{\pgfqpoint{1.602275in}{1.858462in}}{\pgfqpoint{1.594461in}{1.866276in}}%
\pgfpathcurveto{\pgfqpoint{1.586648in}{1.874089in}}{\pgfqpoint{1.576049in}{1.878479in}}{\pgfqpoint{1.564998in}{1.878479in}}%
\pgfpathcurveto{\pgfqpoint{1.553948in}{1.878479in}}{\pgfqpoint{1.543349in}{1.874089in}}{\pgfqpoint{1.535536in}{1.866276in}}%
\pgfpathcurveto{\pgfqpoint{1.527722in}{1.858462in}}{\pgfqpoint{1.523332in}{1.847863in}}{\pgfqpoint{1.523332in}{1.836813in}}%
\pgfpathcurveto{\pgfqpoint{1.523332in}{1.825763in}}{\pgfqpoint{1.527722in}{1.815164in}}{\pgfqpoint{1.535536in}{1.807350in}}%
\pgfpathcurveto{\pgfqpoint{1.543349in}{1.799536in}}{\pgfqpoint{1.553948in}{1.795146in}}{\pgfqpoint{1.564998in}{1.795146in}}%
\pgfpathlineto{\pgfqpoint{1.564998in}{1.795146in}}%
\pgfpathclose%
\pgfusepath{stroke,fill}%
\end{pgfscope}%
\begin{pgfscope}%
\pgfpathrectangle{\pgfqpoint{0.633874in}{0.569136in}}{\pgfqpoint{2.177280in}{2.201755in}}%
\pgfusepath{clip}%
\pgfsetbuttcap%
\pgfsetroundjoin%
\definecolor{currentfill}{rgb}{1.000000,0.498039,0.054902}%
\pgfsetfillcolor{currentfill}%
\pgfsetlinewidth{0.481800pt}%
\definecolor{currentstroke}{rgb}{1.000000,1.000000,1.000000}%
\pgfsetstrokecolor{currentstroke}%
\pgfsetdash{}{0pt}%
\pgfpathmoveto{\pgfqpoint{1.604937in}{1.378147in}}%
\pgfpathcurveto{\pgfqpoint{1.615987in}{1.378147in}}{\pgfqpoint{1.626586in}{1.382537in}}{\pgfqpoint{1.634400in}{1.390351in}}%
\pgfpathcurveto{\pgfqpoint{1.642214in}{1.398164in}}{\pgfqpoint{1.646604in}{1.408764in}}{\pgfqpoint{1.646604in}{1.419814in}}%
\pgfpathcurveto{\pgfqpoint{1.646604in}{1.430864in}}{\pgfqpoint{1.642214in}{1.441463in}}{\pgfqpoint{1.634400in}{1.449276in}}%
\pgfpathcurveto{\pgfqpoint{1.626586in}{1.457090in}}{\pgfqpoint{1.615987in}{1.461480in}}{\pgfqpoint{1.604937in}{1.461480in}}%
\pgfpathcurveto{\pgfqpoint{1.593887in}{1.461480in}}{\pgfqpoint{1.583288in}{1.457090in}}{\pgfqpoint{1.575474in}{1.449276in}}%
\pgfpathcurveto{\pgfqpoint{1.567661in}{1.441463in}}{\pgfqpoint{1.563270in}{1.430864in}}{\pgfqpoint{1.563270in}{1.419814in}}%
\pgfpathcurveto{\pgfqpoint{1.563270in}{1.408764in}}{\pgfqpoint{1.567661in}{1.398164in}}{\pgfqpoint{1.575474in}{1.390351in}}%
\pgfpathcurveto{\pgfqpoint{1.583288in}{1.382537in}}{\pgfqpoint{1.593887in}{1.378147in}}{\pgfqpoint{1.604937in}{1.378147in}}%
\pgfpathlineto{\pgfqpoint{1.604937in}{1.378147in}}%
\pgfpathclose%
\pgfusepath{stroke,fill}%
\end{pgfscope}%
\begin{pgfscope}%
\pgfpathrectangle{\pgfqpoint{0.633874in}{0.569136in}}{\pgfqpoint{2.177280in}{2.201755in}}%
\pgfusepath{clip}%
\pgfsetbuttcap%
\pgfsetroundjoin%
\definecolor{currentfill}{rgb}{1.000000,0.498039,0.054902}%
\pgfsetfillcolor{currentfill}%
\pgfsetlinewidth{0.481800pt}%
\definecolor{currentstroke}{rgb}{1.000000,1.000000,1.000000}%
\pgfsetstrokecolor{currentstroke}%
\pgfsetdash{}{0pt}%
\pgfpathmoveto{\pgfqpoint{1.644876in}{1.711746in}}%
\pgfpathcurveto{\pgfqpoint{1.655926in}{1.711746in}}{\pgfqpoint{1.666525in}{1.716137in}}{\pgfqpoint{1.674339in}{1.723950in}}%
\pgfpathcurveto{\pgfqpoint{1.682152in}{1.731764in}}{\pgfqpoint{1.686543in}{1.742363in}}{\pgfqpoint{1.686543in}{1.753413in}}%
\pgfpathcurveto{\pgfqpoint{1.686543in}{1.764463in}}{\pgfqpoint{1.682152in}{1.775062in}}{\pgfqpoint{1.674339in}{1.782876in}}%
\pgfpathcurveto{\pgfqpoint{1.666525in}{1.790689in}}{\pgfqpoint{1.655926in}{1.795080in}}{\pgfqpoint{1.644876in}{1.795080in}}%
\pgfpathcurveto{\pgfqpoint{1.633826in}{1.795080in}}{\pgfqpoint{1.623227in}{1.790689in}}{\pgfqpoint{1.615413in}{1.782876in}}%
\pgfpathcurveto{\pgfqpoint{1.607599in}{1.775062in}}{\pgfqpoint{1.603209in}{1.764463in}}{\pgfqpoint{1.603209in}{1.753413in}}%
\pgfpathcurveto{\pgfqpoint{1.603209in}{1.742363in}}{\pgfqpoint{1.607599in}{1.731764in}}{\pgfqpoint{1.615413in}{1.723950in}}%
\pgfpathcurveto{\pgfqpoint{1.623227in}{1.716137in}}{\pgfqpoint{1.633826in}{1.711746in}}{\pgfqpoint{1.644876in}{1.711746in}}%
\pgfpathlineto{\pgfqpoint{1.644876in}{1.711746in}}%
\pgfpathclose%
\pgfusepath{stroke,fill}%
\end{pgfscope}%
\begin{pgfscope}%
\pgfpathrectangle{\pgfqpoint{0.633874in}{0.569136in}}{\pgfqpoint{2.177280in}{2.201755in}}%
\pgfusepath{clip}%
\pgfsetbuttcap%
\pgfsetroundjoin%
\definecolor{currentfill}{rgb}{1.000000,0.498039,0.054902}%
\pgfsetfillcolor{currentfill}%
\pgfsetlinewidth{0.481800pt}%
\definecolor{currentstroke}{rgb}{1.000000,1.000000,1.000000}%
\pgfsetstrokecolor{currentstroke}%
\pgfsetdash{}{0pt}%
\pgfpathmoveto{\pgfqpoint{1.445182in}{1.628346in}}%
\pgfpathcurveto{\pgfqpoint{1.456232in}{1.628346in}}{\pgfqpoint{1.466831in}{1.632737in}}{\pgfqpoint{1.474645in}{1.640550in}}%
\pgfpathcurveto{\pgfqpoint{1.482459in}{1.648364in}}{\pgfqpoint{1.486849in}{1.658963in}}{\pgfqpoint{1.486849in}{1.670013in}}%
\pgfpathcurveto{\pgfqpoint{1.486849in}{1.681063in}}{\pgfqpoint{1.482459in}{1.691662in}}{\pgfqpoint{1.474645in}{1.699476in}}%
\pgfpathcurveto{\pgfqpoint{1.466831in}{1.707290in}}{\pgfqpoint{1.456232in}{1.711680in}}{\pgfqpoint{1.445182in}{1.711680in}}%
\pgfpathcurveto{\pgfqpoint{1.434132in}{1.711680in}}{\pgfqpoint{1.423533in}{1.707290in}}{\pgfqpoint{1.415720in}{1.699476in}}%
\pgfpathcurveto{\pgfqpoint{1.407906in}{1.691662in}}{\pgfqpoint{1.403516in}{1.681063in}}{\pgfqpoint{1.403516in}{1.670013in}}%
\pgfpathcurveto{\pgfqpoint{1.403516in}{1.658963in}}{\pgfqpoint{1.407906in}{1.648364in}}{\pgfqpoint{1.415720in}{1.640550in}}%
\pgfpathcurveto{\pgfqpoint{1.423533in}{1.632737in}}{\pgfqpoint{1.434132in}{1.628346in}}{\pgfqpoint{1.445182in}{1.628346in}}%
\pgfpathlineto{\pgfqpoint{1.445182in}{1.628346in}}%
\pgfpathclose%
\pgfusepath{stroke,fill}%
\end{pgfscope}%
\begin{pgfscope}%
\pgfpathrectangle{\pgfqpoint{0.633874in}{0.569136in}}{\pgfqpoint{2.177280in}{2.201755in}}%
\pgfusepath{clip}%
\pgfsetbuttcap%
\pgfsetroundjoin%
\definecolor{currentfill}{rgb}{1.000000,0.498039,0.054902}%
\pgfsetfillcolor{currentfill}%
\pgfsetlinewidth{0.481800pt}%
\definecolor{currentstroke}{rgb}{1.000000,1.000000,1.000000}%
\pgfsetstrokecolor{currentstroke}%
\pgfsetdash{}{0pt}%
\pgfpathmoveto{\pgfqpoint{1.884508in}{1.711746in}}%
\pgfpathcurveto{\pgfqpoint{1.895558in}{1.711746in}}{\pgfqpoint{1.906157in}{1.716137in}}{\pgfqpoint{1.913971in}{1.723950in}}%
\pgfpathcurveto{\pgfqpoint{1.921784in}{1.731764in}}{\pgfqpoint{1.926175in}{1.742363in}}{\pgfqpoint{1.926175in}{1.753413in}}%
\pgfpathcurveto{\pgfqpoint{1.926175in}{1.764463in}}{\pgfqpoint{1.921784in}{1.775062in}}{\pgfqpoint{1.913971in}{1.782876in}}%
\pgfpathcurveto{\pgfqpoint{1.906157in}{1.790689in}}{\pgfqpoint{1.895558in}{1.795080in}}{\pgfqpoint{1.884508in}{1.795080in}}%
\pgfpathcurveto{\pgfqpoint{1.873458in}{1.795080in}}{\pgfqpoint{1.862859in}{1.790689in}}{\pgfqpoint{1.855045in}{1.782876in}}%
\pgfpathcurveto{\pgfqpoint{1.847232in}{1.775062in}}{\pgfqpoint{1.842841in}{1.764463in}}{\pgfqpoint{1.842841in}{1.753413in}}%
\pgfpathcurveto{\pgfqpoint{1.842841in}{1.742363in}}{\pgfqpoint{1.847232in}{1.731764in}}{\pgfqpoint{1.855045in}{1.723950in}}%
\pgfpathcurveto{\pgfqpoint{1.862859in}{1.716137in}}{\pgfqpoint{1.873458in}{1.711746in}}{\pgfqpoint{1.884508in}{1.711746in}}%
\pgfpathlineto{\pgfqpoint{1.884508in}{1.711746in}}%
\pgfpathclose%
\pgfusepath{stroke,fill}%
\end{pgfscope}%
\begin{pgfscope}%
\pgfpathrectangle{\pgfqpoint{0.633874in}{0.569136in}}{\pgfqpoint{2.177280in}{2.201755in}}%
\pgfusepath{clip}%
\pgfsetbuttcap%
\pgfsetroundjoin%
\definecolor{currentfill}{rgb}{1.000000,0.498039,0.054902}%
\pgfsetfillcolor{currentfill}%
\pgfsetlinewidth{0.481800pt}%
\definecolor{currentstroke}{rgb}{1.000000,1.000000,1.000000}%
\pgfsetstrokecolor{currentstroke}%
\pgfsetdash{}{0pt}%
\pgfpathmoveto{\pgfqpoint{1.445182in}{1.795146in}}%
\pgfpathcurveto{\pgfqpoint{1.456232in}{1.795146in}}{\pgfqpoint{1.466831in}{1.799536in}}{\pgfqpoint{1.474645in}{1.807350in}}%
\pgfpathcurveto{\pgfqpoint{1.482459in}{1.815164in}}{\pgfqpoint{1.486849in}{1.825763in}}{\pgfqpoint{1.486849in}{1.836813in}}%
\pgfpathcurveto{\pgfqpoint{1.486849in}{1.847863in}}{\pgfqpoint{1.482459in}{1.858462in}}{\pgfqpoint{1.474645in}{1.866276in}}%
\pgfpathcurveto{\pgfqpoint{1.466831in}{1.874089in}}{\pgfqpoint{1.456232in}{1.878479in}}{\pgfqpoint{1.445182in}{1.878479in}}%
\pgfpathcurveto{\pgfqpoint{1.434132in}{1.878479in}}{\pgfqpoint{1.423533in}{1.874089in}}{\pgfqpoint{1.415720in}{1.866276in}}%
\pgfpathcurveto{\pgfqpoint{1.407906in}{1.858462in}}{\pgfqpoint{1.403516in}{1.847863in}}{\pgfqpoint{1.403516in}{1.836813in}}%
\pgfpathcurveto{\pgfqpoint{1.403516in}{1.825763in}}{\pgfqpoint{1.407906in}{1.815164in}}{\pgfqpoint{1.415720in}{1.807350in}}%
\pgfpathcurveto{\pgfqpoint{1.423533in}{1.799536in}}{\pgfqpoint{1.434132in}{1.795146in}}{\pgfqpoint{1.445182in}{1.795146in}}%
\pgfpathlineto{\pgfqpoint{1.445182in}{1.795146in}}%
\pgfpathclose%
\pgfusepath{stroke,fill}%
\end{pgfscope}%
\begin{pgfscope}%
\pgfpathrectangle{\pgfqpoint{0.633874in}{0.569136in}}{\pgfqpoint{2.177280in}{2.201755in}}%
\pgfusepath{clip}%
\pgfsetbuttcap%
\pgfsetroundjoin%
\definecolor{currentfill}{rgb}{1.000000,0.498039,0.054902}%
\pgfsetfillcolor{currentfill}%
\pgfsetlinewidth{0.481800pt}%
\definecolor{currentstroke}{rgb}{1.000000,1.000000,1.000000}%
\pgfsetstrokecolor{currentstroke}%
\pgfsetdash{}{0pt}%
\pgfpathmoveto{\pgfqpoint{1.525060in}{1.378147in}}%
\pgfpathcurveto{\pgfqpoint{1.536110in}{1.378147in}}{\pgfqpoint{1.546709in}{1.382537in}}{\pgfqpoint{1.554523in}{1.390351in}}%
\pgfpathcurveto{\pgfqpoint{1.562336in}{1.398164in}}{\pgfqpoint{1.566726in}{1.408764in}}{\pgfqpoint{1.566726in}{1.419814in}}%
\pgfpathcurveto{\pgfqpoint{1.566726in}{1.430864in}}{\pgfqpoint{1.562336in}{1.441463in}}{\pgfqpoint{1.554523in}{1.449276in}}%
\pgfpathcurveto{\pgfqpoint{1.546709in}{1.457090in}}{\pgfqpoint{1.536110in}{1.461480in}}{\pgfqpoint{1.525060in}{1.461480in}}%
\pgfpathcurveto{\pgfqpoint{1.514010in}{1.461480in}}{\pgfqpoint{1.503411in}{1.457090in}}{\pgfqpoint{1.495597in}{1.449276in}}%
\pgfpathcurveto{\pgfqpoint{1.487783in}{1.441463in}}{\pgfqpoint{1.483393in}{1.430864in}}{\pgfqpoint{1.483393in}{1.419814in}}%
\pgfpathcurveto{\pgfqpoint{1.483393in}{1.408764in}}{\pgfqpoint{1.487783in}{1.398164in}}{\pgfqpoint{1.495597in}{1.390351in}}%
\pgfpathcurveto{\pgfqpoint{1.503411in}{1.382537in}}{\pgfqpoint{1.514010in}{1.378147in}}{\pgfqpoint{1.525060in}{1.378147in}}%
\pgfpathlineto{\pgfqpoint{1.525060in}{1.378147in}}%
\pgfpathclose%
\pgfusepath{stroke,fill}%
\end{pgfscope}%
\begin{pgfscope}%
\pgfpathrectangle{\pgfqpoint{0.633874in}{0.569136in}}{\pgfqpoint{2.177280in}{2.201755in}}%
\pgfusepath{clip}%
\pgfsetbuttcap%
\pgfsetroundjoin%
\definecolor{currentfill}{rgb}{1.000000,0.498039,0.054902}%
\pgfsetfillcolor{currentfill}%
\pgfsetlinewidth{0.481800pt}%
\definecolor{currentstroke}{rgb}{1.000000,1.000000,1.000000}%
\pgfsetstrokecolor{currentstroke}%
\pgfsetdash{}{0pt}%
\pgfpathmoveto{\pgfqpoint{1.684815in}{1.795146in}}%
\pgfpathcurveto{\pgfqpoint{1.695865in}{1.795146in}}{\pgfqpoint{1.706464in}{1.799536in}}{\pgfqpoint{1.714277in}{1.807350in}}%
\pgfpathcurveto{\pgfqpoint{1.722091in}{1.815164in}}{\pgfqpoint{1.726481in}{1.825763in}}{\pgfqpoint{1.726481in}{1.836813in}}%
\pgfpathcurveto{\pgfqpoint{1.726481in}{1.847863in}}{\pgfqpoint{1.722091in}{1.858462in}}{\pgfqpoint{1.714277in}{1.866276in}}%
\pgfpathcurveto{\pgfqpoint{1.706464in}{1.874089in}}{\pgfqpoint{1.695865in}{1.878479in}}{\pgfqpoint{1.684815in}{1.878479in}}%
\pgfpathcurveto{\pgfqpoint{1.673764in}{1.878479in}}{\pgfqpoint{1.663165in}{1.874089in}}{\pgfqpoint{1.655352in}{1.866276in}}%
\pgfpathcurveto{\pgfqpoint{1.647538in}{1.858462in}}{\pgfqpoint{1.643148in}{1.847863in}}{\pgfqpoint{1.643148in}{1.836813in}}%
\pgfpathcurveto{\pgfqpoint{1.643148in}{1.825763in}}{\pgfqpoint{1.647538in}{1.815164in}}{\pgfqpoint{1.655352in}{1.807350in}}%
\pgfpathcurveto{\pgfqpoint{1.663165in}{1.799536in}}{\pgfqpoint{1.673764in}{1.795146in}}{\pgfqpoint{1.684815in}{1.795146in}}%
\pgfpathlineto{\pgfqpoint{1.684815in}{1.795146in}}%
\pgfpathclose%
\pgfusepath{stroke,fill}%
\end{pgfscope}%
\begin{pgfscope}%
\pgfpathrectangle{\pgfqpoint{0.633874in}{0.569136in}}{\pgfqpoint{2.177280in}{2.201755in}}%
\pgfusepath{clip}%
\pgfsetbuttcap%
\pgfsetroundjoin%
\definecolor{currentfill}{rgb}{1.000000,0.498039,0.054902}%
\pgfsetfillcolor{currentfill}%
\pgfsetlinewidth{0.481800pt}%
\definecolor{currentstroke}{rgb}{1.000000,1.000000,1.000000}%
\pgfsetstrokecolor{currentstroke}%
\pgfsetdash{}{0pt}%
\pgfpathmoveto{\pgfqpoint{1.445182in}{1.461547in}}%
\pgfpathcurveto{\pgfqpoint{1.456232in}{1.461547in}}{\pgfqpoint{1.466831in}{1.465937in}}{\pgfqpoint{1.474645in}{1.473751in}}%
\pgfpathcurveto{\pgfqpoint{1.482459in}{1.481564in}}{\pgfqpoint{1.486849in}{1.492163in}}{\pgfqpoint{1.486849in}{1.503213in}}%
\pgfpathcurveto{\pgfqpoint{1.486849in}{1.514264in}}{\pgfqpoint{1.482459in}{1.524863in}}{\pgfqpoint{1.474645in}{1.532676in}}%
\pgfpathcurveto{\pgfqpoint{1.466831in}{1.540490in}}{\pgfqpoint{1.456232in}{1.544880in}}{\pgfqpoint{1.445182in}{1.544880in}}%
\pgfpathcurveto{\pgfqpoint{1.434132in}{1.544880in}}{\pgfqpoint{1.423533in}{1.540490in}}{\pgfqpoint{1.415720in}{1.532676in}}%
\pgfpathcurveto{\pgfqpoint{1.407906in}{1.524863in}}{\pgfqpoint{1.403516in}{1.514264in}}{\pgfqpoint{1.403516in}{1.503213in}}%
\pgfpathcurveto{\pgfqpoint{1.403516in}{1.492163in}}{\pgfqpoint{1.407906in}{1.481564in}}{\pgfqpoint{1.415720in}{1.473751in}}%
\pgfpathcurveto{\pgfqpoint{1.423533in}{1.465937in}}{\pgfqpoint{1.434132in}{1.461547in}}{\pgfqpoint{1.445182in}{1.461547in}}%
\pgfpathlineto{\pgfqpoint{1.445182in}{1.461547in}}%
\pgfpathclose%
\pgfusepath{stroke,fill}%
\end{pgfscope}%
\begin{pgfscope}%
\pgfpathrectangle{\pgfqpoint{0.633874in}{0.569136in}}{\pgfqpoint{2.177280in}{2.201755in}}%
\pgfusepath{clip}%
\pgfsetbuttcap%
\pgfsetroundjoin%
\definecolor{currentfill}{rgb}{1.000000,0.498039,0.054902}%
\pgfsetfillcolor{currentfill}%
\pgfsetlinewidth{0.481800pt}%
\definecolor{currentstroke}{rgb}{1.000000,1.000000,1.000000}%
\pgfsetstrokecolor{currentstroke}%
\pgfsetdash{}{0pt}%
\pgfpathmoveto{\pgfqpoint{1.564998in}{2.045346in}}%
\pgfpathcurveto{\pgfqpoint{1.576049in}{2.045346in}}{\pgfqpoint{1.586648in}{2.049736in}}{\pgfqpoint{1.594461in}{2.057549in}}%
\pgfpathcurveto{\pgfqpoint{1.602275in}{2.065363in}}{\pgfqpoint{1.606665in}{2.075962in}}{\pgfqpoint{1.606665in}{2.087012in}}%
\pgfpathcurveto{\pgfqpoint{1.606665in}{2.098062in}}{\pgfqpoint{1.602275in}{2.108661in}}{\pgfqpoint{1.594461in}{2.116475in}}%
\pgfpathcurveto{\pgfqpoint{1.586648in}{2.124289in}}{\pgfqpoint{1.576049in}{2.128679in}}{\pgfqpoint{1.564998in}{2.128679in}}%
\pgfpathcurveto{\pgfqpoint{1.553948in}{2.128679in}}{\pgfqpoint{1.543349in}{2.124289in}}{\pgfqpoint{1.535536in}{2.116475in}}%
\pgfpathcurveto{\pgfqpoint{1.527722in}{2.108661in}}{\pgfqpoint{1.523332in}{2.098062in}}{\pgfqpoint{1.523332in}{2.087012in}}%
\pgfpathcurveto{\pgfqpoint{1.523332in}{2.075962in}}{\pgfqpoint{1.527722in}{2.065363in}}{\pgfqpoint{1.535536in}{2.057549in}}%
\pgfpathcurveto{\pgfqpoint{1.543349in}{2.049736in}}{\pgfqpoint{1.553948in}{2.045346in}}{\pgfqpoint{1.564998in}{2.045346in}}%
\pgfpathlineto{\pgfqpoint{1.564998in}{2.045346in}}%
\pgfpathclose%
\pgfusepath{stroke,fill}%
\end{pgfscope}%
\begin{pgfscope}%
\pgfpathrectangle{\pgfqpoint{0.633874in}{0.569136in}}{\pgfqpoint{2.177280in}{2.201755in}}%
\pgfusepath{clip}%
\pgfsetbuttcap%
\pgfsetroundjoin%
\definecolor{currentfill}{rgb}{1.000000,0.498039,0.054902}%
\pgfsetfillcolor{currentfill}%
\pgfsetlinewidth{0.481800pt}%
\definecolor{currentstroke}{rgb}{1.000000,1.000000,1.000000}%
\pgfsetstrokecolor{currentstroke}%
\pgfsetdash{}{0pt}%
\pgfpathmoveto{\pgfqpoint{1.644876in}{1.628346in}}%
\pgfpathcurveto{\pgfqpoint{1.655926in}{1.628346in}}{\pgfqpoint{1.666525in}{1.632737in}}{\pgfqpoint{1.674339in}{1.640550in}}%
\pgfpathcurveto{\pgfqpoint{1.682152in}{1.648364in}}{\pgfqpoint{1.686543in}{1.658963in}}{\pgfqpoint{1.686543in}{1.670013in}}%
\pgfpathcurveto{\pgfqpoint{1.686543in}{1.681063in}}{\pgfqpoint{1.682152in}{1.691662in}}{\pgfqpoint{1.674339in}{1.699476in}}%
\pgfpathcurveto{\pgfqpoint{1.666525in}{1.707290in}}{\pgfqpoint{1.655926in}{1.711680in}}{\pgfqpoint{1.644876in}{1.711680in}}%
\pgfpathcurveto{\pgfqpoint{1.633826in}{1.711680in}}{\pgfqpoint{1.623227in}{1.707290in}}{\pgfqpoint{1.615413in}{1.699476in}}%
\pgfpathcurveto{\pgfqpoint{1.607599in}{1.691662in}}{\pgfqpoint{1.603209in}{1.681063in}}{\pgfqpoint{1.603209in}{1.670013in}}%
\pgfpathcurveto{\pgfqpoint{1.603209in}{1.658963in}}{\pgfqpoint{1.607599in}{1.648364in}}{\pgfqpoint{1.615413in}{1.640550in}}%
\pgfpathcurveto{\pgfqpoint{1.623227in}{1.632737in}}{\pgfqpoint{1.633826in}{1.628346in}}{\pgfqpoint{1.644876in}{1.628346in}}%
\pgfpathlineto{\pgfqpoint{1.644876in}{1.628346in}}%
\pgfpathclose%
\pgfusepath{stroke,fill}%
\end{pgfscope}%
\begin{pgfscope}%
\pgfpathrectangle{\pgfqpoint{0.633874in}{0.569136in}}{\pgfqpoint{2.177280in}{2.201755in}}%
\pgfusepath{clip}%
\pgfsetbuttcap%
\pgfsetroundjoin%
\definecolor{currentfill}{rgb}{1.000000,0.498039,0.054902}%
\pgfsetfillcolor{currentfill}%
\pgfsetlinewidth{0.481800pt}%
\definecolor{currentstroke}{rgb}{1.000000,1.000000,1.000000}%
\pgfsetstrokecolor{currentstroke}%
\pgfsetdash{}{0pt}%
\pgfpathmoveto{\pgfqpoint{1.724753in}{1.795146in}}%
\pgfpathcurveto{\pgfqpoint{1.735803in}{1.795146in}}{\pgfqpoint{1.746402in}{1.799536in}}{\pgfqpoint{1.754216in}{1.807350in}}%
\pgfpathcurveto{\pgfqpoint{1.762030in}{1.815164in}}{\pgfqpoint{1.766420in}{1.825763in}}{\pgfqpoint{1.766420in}{1.836813in}}%
\pgfpathcurveto{\pgfqpoint{1.766420in}{1.847863in}}{\pgfqpoint{1.762030in}{1.858462in}}{\pgfqpoint{1.754216in}{1.866276in}}%
\pgfpathcurveto{\pgfqpoint{1.746402in}{1.874089in}}{\pgfqpoint{1.735803in}{1.878479in}}{\pgfqpoint{1.724753in}{1.878479in}}%
\pgfpathcurveto{\pgfqpoint{1.713703in}{1.878479in}}{\pgfqpoint{1.703104in}{1.874089in}}{\pgfqpoint{1.695290in}{1.866276in}}%
\pgfpathcurveto{\pgfqpoint{1.687477in}{1.858462in}}{\pgfqpoint{1.683087in}{1.847863in}}{\pgfqpoint{1.683087in}{1.836813in}}%
\pgfpathcurveto{\pgfqpoint{1.683087in}{1.825763in}}{\pgfqpoint{1.687477in}{1.815164in}}{\pgfqpoint{1.695290in}{1.807350in}}%
\pgfpathcurveto{\pgfqpoint{1.703104in}{1.799536in}}{\pgfqpoint{1.713703in}{1.795146in}}{\pgfqpoint{1.724753in}{1.795146in}}%
\pgfpathlineto{\pgfqpoint{1.724753in}{1.795146in}}%
\pgfpathclose%
\pgfusepath{stroke,fill}%
\end{pgfscope}%
\begin{pgfscope}%
\pgfpathrectangle{\pgfqpoint{0.633874in}{0.569136in}}{\pgfqpoint{2.177280in}{2.201755in}}%
\pgfusepath{clip}%
\pgfsetbuttcap%
\pgfsetroundjoin%
\definecolor{currentfill}{rgb}{1.000000,0.498039,0.054902}%
\pgfsetfillcolor{currentfill}%
\pgfsetlinewidth{0.481800pt}%
\definecolor{currentstroke}{rgb}{1.000000,1.000000,1.000000}%
\pgfsetstrokecolor{currentstroke}%
\pgfsetdash{}{0pt}%
\pgfpathmoveto{\pgfqpoint{1.644876in}{1.544947in}}%
\pgfpathcurveto{\pgfqpoint{1.655926in}{1.544947in}}{\pgfqpoint{1.666525in}{1.549337in}}{\pgfqpoint{1.674339in}{1.557151in}}%
\pgfpathcurveto{\pgfqpoint{1.682152in}{1.564964in}}{\pgfqpoint{1.686543in}{1.575563in}}{\pgfqpoint{1.686543in}{1.586613in}}%
\pgfpathcurveto{\pgfqpoint{1.686543in}{1.597663in}}{\pgfqpoint{1.682152in}{1.608262in}}{\pgfqpoint{1.674339in}{1.616076in}}%
\pgfpathcurveto{\pgfqpoint{1.666525in}{1.623890in}}{\pgfqpoint{1.655926in}{1.628280in}}{\pgfqpoint{1.644876in}{1.628280in}}%
\pgfpathcurveto{\pgfqpoint{1.633826in}{1.628280in}}{\pgfqpoint{1.623227in}{1.623890in}}{\pgfqpoint{1.615413in}{1.616076in}}%
\pgfpathcurveto{\pgfqpoint{1.607599in}{1.608262in}}{\pgfqpoint{1.603209in}{1.597663in}}{\pgfqpoint{1.603209in}{1.586613in}}%
\pgfpathcurveto{\pgfqpoint{1.603209in}{1.575563in}}{\pgfqpoint{1.607599in}{1.564964in}}{\pgfqpoint{1.615413in}{1.557151in}}%
\pgfpathcurveto{\pgfqpoint{1.623227in}{1.549337in}}{\pgfqpoint{1.633826in}{1.544947in}}{\pgfqpoint{1.644876in}{1.544947in}}%
\pgfpathlineto{\pgfqpoint{1.644876in}{1.544947in}}%
\pgfpathclose%
\pgfusepath{stroke,fill}%
\end{pgfscope}%
\begin{pgfscope}%
\pgfpathrectangle{\pgfqpoint{0.633874in}{0.569136in}}{\pgfqpoint{2.177280in}{2.201755in}}%
\pgfusepath{clip}%
\pgfsetbuttcap%
\pgfsetroundjoin%
\definecolor{currentfill}{rgb}{1.000000,0.498039,0.054902}%
\pgfsetfillcolor{currentfill}%
\pgfsetlinewidth{0.481800pt}%
\definecolor{currentstroke}{rgb}{1.000000,1.000000,1.000000}%
\pgfsetstrokecolor{currentstroke}%
\pgfsetdash{}{0pt}%
\pgfpathmoveto{\pgfqpoint{1.764692in}{1.628346in}}%
\pgfpathcurveto{\pgfqpoint{1.775742in}{1.628346in}}{\pgfqpoint{1.786341in}{1.632737in}}{\pgfqpoint{1.794155in}{1.640550in}}%
\pgfpathcurveto{\pgfqpoint{1.801968in}{1.648364in}}{\pgfqpoint{1.806359in}{1.658963in}}{\pgfqpoint{1.806359in}{1.670013in}}%
\pgfpathcurveto{\pgfqpoint{1.806359in}{1.681063in}}{\pgfqpoint{1.801968in}{1.691662in}}{\pgfqpoint{1.794155in}{1.699476in}}%
\pgfpathcurveto{\pgfqpoint{1.786341in}{1.707290in}}{\pgfqpoint{1.775742in}{1.711680in}}{\pgfqpoint{1.764692in}{1.711680in}}%
\pgfpathcurveto{\pgfqpoint{1.753642in}{1.711680in}}{\pgfqpoint{1.743043in}{1.707290in}}{\pgfqpoint{1.735229in}{1.699476in}}%
\pgfpathcurveto{\pgfqpoint{1.727416in}{1.691662in}}{\pgfqpoint{1.723025in}{1.681063in}}{\pgfqpoint{1.723025in}{1.670013in}}%
\pgfpathcurveto{\pgfqpoint{1.723025in}{1.658963in}}{\pgfqpoint{1.727416in}{1.648364in}}{\pgfqpoint{1.735229in}{1.640550in}}%
\pgfpathcurveto{\pgfqpoint{1.743043in}{1.632737in}}{\pgfqpoint{1.753642in}{1.628346in}}{\pgfqpoint{1.764692in}{1.628346in}}%
\pgfpathlineto{\pgfqpoint{1.764692in}{1.628346in}}%
\pgfpathclose%
\pgfusepath{stroke,fill}%
\end{pgfscope}%
\begin{pgfscope}%
\pgfpathrectangle{\pgfqpoint{0.633874in}{0.569136in}}{\pgfqpoint{2.177280in}{2.201755in}}%
\pgfusepath{clip}%
\pgfsetbuttcap%
\pgfsetroundjoin%
\definecolor{currentfill}{rgb}{1.000000,0.498039,0.054902}%
\pgfsetfillcolor{currentfill}%
\pgfsetlinewidth{0.481800pt}%
\definecolor{currentstroke}{rgb}{1.000000,1.000000,1.000000}%
\pgfsetstrokecolor{currentstroke}%
\pgfsetdash{}{0pt}%
\pgfpathmoveto{\pgfqpoint{1.844569in}{1.711746in}}%
\pgfpathcurveto{\pgfqpoint{1.855619in}{1.711746in}}{\pgfqpoint{1.866219in}{1.716137in}}{\pgfqpoint{1.874032in}{1.723950in}}%
\pgfpathcurveto{\pgfqpoint{1.881846in}{1.731764in}}{\pgfqpoint{1.886236in}{1.742363in}}{\pgfqpoint{1.886236in}{1.753413in}}%
\pgfpathcurveto{\pgfqpoint{1.886236in}{1.764463in}}{\pgfqpoint{1.881846in}{1.775062in}}{\pgfqpoint{1.874032in}{1.782876in}}%
\pgfpathcurveto{\pgfqpoint{1.866219in}{1.790689in}}{\pgfqpoint{1.855619in}{1.795080in}}{\pgfqpoint{1.844569in}{1.795080in}}%
\pgfpathcurveto{\pgfqpoint{1.833519in}{1.795080in}}{\pgfqpoint{1.822920in}{1.790689in}}{\pgfqpoint{1.815107in}{1.782876in}}%
\pgfpathcurveto{\pgfqpoint{1.807293in}{1.775062in}}{\pgfqpoint{1.802903in}{1.764463in}}{\pgfqpoint{1.802903in}{1.753413in}}%
\pgfpathcurveto{\pgfqpoint{1.802903in}{1.742363in}}{\pgfqpoint{1.807293in}{1.731764in}}{\pgfqpoint{1.815107in}{1.723950in}}%
\pgfpathcurveto{\pgfqpoint{1.822920in}{1.716137in}}{\pgfqpoint{1.833519in}{1.711746in}}{\pgfqpoint{1.844569in}{1.711746in}}%
\pgfpathlineto{\pgfqpoint{1.844569in}{1.711746in}}%
\pgfpathclose%
\pgfusepath{stroke,fill}%
\end{pgfscope}%
\begin{pgfscope}%
\pgfpathrectangle{\pgfqpoint{0.633874in}{0.569136in}}{\pgfqpoint{2.177280in}{2.201755in}}%
\pgfusepath{clip}%
\pgfsetbuttcap%
\pgfsetroundjoin%
\definecolor{currentfill}{rgb}{1.000000,0.498039,0.054902}%
\pgfsetfillcolor{currentfill}%
\pgfsetlinewidth{0.481800pt}%
\definecolor{currentstroke}{rgb}{1.000000,1.000000,1.000000}%
\pgfsetstrokecolor{currentstroke}%
\pgfsetdash{}{0pt}%
\pgfpathmoveto{\pgfqpoint{1.924447in}{1.711746in}}%
\pgfpathcurveto{\pgfqpoint{1.935497in}{1.711746in}}{\pgfqpoint{1.946096in}{1.716137in}}{\pgfqpoint{1.953910in}{1.723950in}}%
\pgfpathcurveto{\pgfqpoint{1.961723in}{1.731764in}}{\pgfqpoint{1.966113in}{1.742363in}}{\pgfqpoint{1.966113in}{1.753413in}}%
\pgfpathcurveto{\pgfqpoint{1.966113in}{1.764463in}}{\pgfqpoint{1.961723in}{1.775062in}}{\pgfqpoint{1.953910in}{1.782876in}}%
\pgfpathcurveto{\pgfqpoint{1.946096in}{1.790689in}}{\pgfqpoint{1.935497in}{1.795080in}}{\pgfqpoint{1.924447in}{1.795080in}}%
\pgfpathcurveto{\pgfqpoint{1.913397in}{1.795080in}}{\pgfqpoint{1.902798in}{1.790689in}}{\pgfqpoint{1.894984in}{1.782876in}}%
\pgfpathcurveto{\pgfqpoint{1.887170in}{1.775062in}}{\pgfqpoint{1.882780in}{1.764463in}}{\pgfqpoint{1.882780in}{1.753413in}}%
\pgfpathcurveto{\pgfqpoint{1.882780in}{1.742363in}}{\pgfqpoint{1.887170in}{1.731764in}}{\pgfqpoint{1.894984in}{1.723950in}}%
\pgfpathcurveto{\pgfqpoint{1.902798in}{1.716137in}}{\pgfqpoint{1.913397in}{1.711746in}}{\pgfqpoint{1.924447in}{1.711746in}}%
\pgfpathlineto{\pgfqpoint{1.924447in}{1.711746in}}%
\pgfpathclose%
\pgfusepath{stroke,fill}%
\end{pgfscope}%
\begin{pgfscope}%
\pgfpathrectangle{\pgfqpoint{0.633874in}{0.569136in}}{\pgfqpoint{2.177280in}{2.201755in}}%
\pgfusepath{clip}%
\pgfsetbuttcap%
\pgfsetroundjoin%
\definecolor{currentfill}{rgb}{1.000000,0.498039,0.054902}%
\pgfsetfillcolor{currentfill}%
\pgfsetlinewidth{0.481800pt}%
\definecolor{currentstroke}{rgb}{1.000000,1.000000,1.000000}%
\pgfsetstrokecolor{currentstroke}%
\pgfsetdash{}{0pt}%
\pgfpathmoveto{\pgfqpoint{1.884508in}{1.961946in}}%
\pgfpathcurveto{\pgfqpoint{1.895558in}{1.961946in}}{\pgfqpoint{1.906157in}{1.966336in}}{\pgfqpoint{1.913971in}{1.974150in}}%
\pgfpathcurveto{\pgfqpoint{1.921784in}{1.981963in}}{\pgfqpoint{1.926175in}{1.992562in}}{\pgfqpoint{1.926175in}{2.003612in}}%
\pgfpathcurveto{\pgfqpoint{1.926175in}{2.014662in}}{\pgfqpoint{1.921784in}{2.025262in}}{\pgfqpoint{1.913971in}{2.033075in}}%
\pgfpathcurveto{\pgfqpoint{1.906157in}{2.040889in}}{\pgfqpoint{1.895558in}{2.045279in}}{\pgfqpoint{1.884508in}{2.045279in}}%
\pgfpathcurveto{\pgfqpoint{1.873458in}{2.045279in}}{\pgfqpoint{1.862859in}{2.040889in}}{\pgfqpoint{1.855045in}{2.033075in}}%
\pgfpathcurveto{\pgfqpoint{1.847232in}{2.025262in}}{\pgfqpoint{1.842841in}{2.014662in}}{\pgfqpoint{1.842841in}{2.003612in}}%
\pgfpathcurveto{\pgfqpoint{1.842841in}{1.992562in}}{\pgfqpoint{1.847232in}{1.981963in}}{\pgfqpoint{1.855045in}{1.974150in}}%
\pgfpathcurveto{\pgfqpoint{1.862859in}{1.966336in}}{\pgfqpoint{1.873458in}{1.961946in}}{\pgfqpoint{1.884508in}{1.961946in}}%
\pgfpathlineto{\pgfqpoint{1.884508in}{1.961946in}}%
\pgfpathclose%
\pgfusepath{stroke,fill}%
\end{pgfscope}%
\begin{pgfscope}%
\pgfpathrectangle{\pgfqpoint{0.633874in}{0.569136in}}{\pgfqpoint{2.177280in}{2.201755in}}%
\pgfusepath{clip}%
\pgfsetbuttcap%
\pgfsetroundjoin%
\definecolor{currentfill}{rgb}{1.000000,0.498039,0.054902}%
\pgfsetfillcolor{currentfill}%
\pgfsetlinewidth{0.481800pt}%
\definecolor{currentstroke}{rgb}{1.000000,1.000000,1.000000}%
\pgfsetstrokecolor{currentstroke}%
\pgfsetdash{}{0pt}%
\pgfpathmoveto{\pgfqpoint{1.604937in}{1.795146in}}%
\pgfpathcurveto{\pgfqpoint{1.615987in}{1.795146in}}{\pgfqpoint{1.626586in}{1.799536in}}{\pgfqpoint{1.634400in}{1.807350in}}%
\pgfpathcurveto{\pgfqpoint{1.642214in}{1.815164in}}{\pgfqpoint{1.646604in}{1.825763in}}{\pgfqpoint{1.646604in}{1.836813in}}%
\pgfpathcurveto{\pgfqpoint{1.646604in}{1.847863in}}{\pgfqpoint{1.642214in}{1.858462in}}{\pgfqpoint{1.634400in}{1.866276in}}%
\pgfpathcurveto{\pgfqpoint{1.626586in}{1.874089in}}{\pgfqpoint{1.615987in}{1.878479in}}{\pgfqpoint{1.604937in}{1.878479in}}%
\pgfpathcurveto{\pgfqpoint{1.593887in}{1.878479in}}{\pgfqpoint{1.583288in}{1.874089in}}{\pgfqpoint{1.575474in}{1.866276in}}%
\pgfpathcurveto{\pgfqpoint{1.567661in}{1.858462in}}{\pgfqpoint{1.563270in}{1.847863in}}{\pgfqpoint{1.563270in}{1.836813in}}%
\pgfpathcurveto{\pgfqpoint{1.563270in}{1.825763in}}{\pgfqpoint{1.567661in}{1.815164in}}{\pgfqpoint{1.575474in}{1.807350in}}%
\pgfpathcurveto{\pgfqpoint{1.583288in}{1.799536in}}{\pgfqpoint{1.593887in}{1.795146in}}{\pgfqpoint{1.604937in}{1.795146in}}%
\pgfpathlineto{\pgfqpoint{1.604937in}{1.795146in}}%
\pgfpathclose%
\pgfusepath{stroke,fill}%
\end{pgfscope}%
\begin{pgfscope}%
\pgfpathrectangle{\pgfqpoint{0.633874in}{0.569136in}}{\pgfqpoint{2.177280in}{2.201755in}}%
\pgfusepath{clip}%
\pgfsetbuttcap%
\pgfsetroundjoin%
\definecolor{currentfill}{rgb}{1.000000,0.498039,0.054902}%
\pgfsetfillcolor{currentfill}%
\pgfsetlinewidth{0.481800pt}%
\definecolor{currentstroke}{rgb}{1.000000,1.000000,1.000000}%
\pgfsetstrokecolor{currentstroke}%
\pgfsetdash{}{0pt}%
\pgfpathmoveto{\pgfqpoint{1.485121in}{1.378147in}}%
\pgfpathcurveto{\pgfqpoint{1.496171in}{1.378147in}}{\pgfqpoint{1.506770in}{1.382537in}}{\pgfqpoint{1.514584in}{1.390351in}}%
\pgfpathcurveto{\pgfqpoint{1.522397in}{1.398164in}}{\pgfqpoint{1.526788in}{1.408764in}}{\pgfqpoint{1.526788in}{1.419814in}}%
\pgfpathcurveto{\pgfqpoint{1.526788in}{1.430864in}}{\pgfqpoint{1.522397in}{1.441463in}}{\pgfqpoint{1.514584in}{1.449276in}}%
\pgfpathcurveto{\pgfqpoint{1.506770in}{1.457090in}}{\pgfqpoint{1.496171in}{1.461480in}}{\pgfqpoint{1.485121in}{1.461480in}}%
\pgfpathcurveto{\pgfqpoint{1.474071in}{1.461480in}}{\pgfqpoint{1.463472in}{1.457090in}}{\pgfqpoint{1.455658in}{1.449276in}}%
\pgfpathcurveto{\pgfqpoint{1.447845in}{1.441463in}}{\pgfqpoint{1.443454in}{1.430864in}}{\pgfqpoint{1.443454in}{1.419814in}}%
\pgfpathcurveto{\pgfqpoint{1.443454in}{1.408764in}}{\pgfqpoint{1.447845in}{1.398164in}}{\pgfqpoint{1.455658in}{1.390351in}}%
\pgfpathcurveto{\pgfqpoint{1.463472in}{1.382537in}}{\pgfqpoint{1.474071in}{1.378147in}}{\pgfqpoint{1.485121in}{1.378147in}}%
\pgfpathlineto{\pgfqpoint{1.485121in}{1.378147in}}%
\pgfpathclose%
\pgfusepath{stroke,fill}%
\end{pgfscope}%
\begin{pgfscope}%
\pgfpathrectangle{\pgfqpoint{0.633874in}{0.569136in}}{\pgfqpoint{2.177280in}{2.201755in}}%
\pgfusepath{clip}%
\pgfsetbuttcap%
\pgfsetroundjoin%
\definecolor{currentfill}{rgb}{1.000000,0.498039,0.054902}%
\pgfsetfillcolor{currentfill}%
\pgfsetlinewidth{0.481800pt}%
\definecolor{currentstroke}{rgb}{1.000000,1.000000,1.000000}%
\pgfsetstrokecolor{currentstroke}%
\pgfsetdash{}{0pt}%
\pgfpathmoveto{\pgfqpoint{1.405244in}{1.461547in}}%
\pgfpathcurveto{\pgfqpoint{1.416294in}{1.461547in}}{\pgfqpoint{1.426893in}{1.465937in}}{\pgfqpoint{1.434706in}{1.473751in}}%
\pgfpathcurveto{\pgfqpoint{1.442520in}{1.481564in}}{\pgfqpoint{1.446910in}{1.492163in}}{\pgfqpoint{1.446910in}{1.503213in}}%
\pgfpathcurveto{\pgfqpoint{1.446910in}{1.514264in}}{\pgfqpoint{1.442520in}{1.524863in}}{\pgfqpoint{1.434706in}{1.532676in}}%
\pgfpathcurveto{\pgfqpoint{1.426893in}{1.540490in}}{\pgfqpoint{1.416294in}{1.544880in}}{\pgfqpoint{1.405244in}{1.544880in}}%
\pgfpathcurveto{\pgfqpoint{1.394193in}{1.544880in}}{\pgfqpoint{1.383594in}{1.540490in}}{\pgfqpoint{1.375781in}{1.532676in}}%
\pgfpathcurveto{\pgfqpoint{1.367967in}{1.524863in}}{\pgfqpoint{1.363577in}{1.514264in}}{\pgfqpoint{1.363577in}{1.503213in}}%
\pgfpathcurveto{\pgfqpoint{1.363577in}{1.492163in}}{\pgfqpoint{1.367967in}{1.481564in}}{\pgfqpoint{1.375781in}{1.473751in}}%
\pgfpathcurveto{\pgfqpoint{1.383594in}{1.465937in}}{\pgfqpoint{1.394193in}{1.461547in}}{\pgfqpoint{1.405244in}{1.461547in}}%
\pgfpathlineto{\pgfqpoint{1.405244in}{1.461547in}}%
\pgfpathclose%
\pgfusepath{stroke,fill}%
\end{pgfscope}%
\begin{pgfscope}%
\pgfpathrectangle{\pgfqpoint{0.633874in}{0.569136in}}{\pgfqpoint{2.177280in}{2.201755in}}%
\pgfusepath{clip}%
\pgfsetbuttcap%
\pgfsetroundjoin%
\definecolor{currentfill}{rgb}{1.000000,0.498039,0.054902}%
\pgfsetfillcolor{currentfill}%
\pgfsetlinewidth{0.481800pt}%
\definecolor{currentstroke}{rgb}{1.000000,1.000000,1.000000}%
\pgfsetstrokecolor{currentstroke}%
\pgfsetdash{}{0pt}%
\pgfpathmoveto{\pgfqpoint{1.405244in}{1.378147in}}%
\pgfpathcurveto{\pgfqpoint{1.416294in}{1.378147in}}{\pgfqpoint{1.426893in}{1.382537in}}{\pgfqpoint{1.434706in}{1.390351in}}%
\pgfpathcurveto{\pgfqpoint{1.442520in}{1.398164in}}{\pgfqpoint{1.446910in}{1.408764in}}{\pgfqpoint{1.446910in}{1.419814in}}%
\pgfpathcurveto{\pgfqpoint{1.446910in}{1.430864in}}{\pgfqpoint{1.442520in}{1.441463in}}{\pgfqpoint{1.434706in}{1.449276in}}%
\pgfpathcurveto{\pgfqpoint{1.426893in}{1.457090in}}{\pgfqpoint{1.416294in}{1.461480in}}{\pgfqpoint{1.405244in}{1.461480in}}%
\pgfpathcurveto{\pgfqpoint{1.394193in}{1.461480in}}{\pgfqpoint{1.383594in}{1.457090in}}{\pgfqpoint{1.375781in}{1.449276in}}%
\pgfpathcurveto{\pgfqpoint{1.367967in}{1.441463in}}{\pgfqpoint{1.363577in}{1.430864in}}{\pgfqpoint{1.363577in}{1.419814in}}%
\pgfpathcurveto{\pgfqpoint{1.363577in}{1.408764in}}{\pgfqpoint{1.367967in}{1.398164in}}{\pgfqpoint{1.375781in}{1.390351in}}%
\pgfpathcurveto{\pgfqpoint{1.383594in}{1.382537in}}{\pgfqpoint{1.394193in}{1.378147in}}{\pgfqpoint{1.405244in}{1.378147in}}%
\pgfpathlineto{\pgfqpoint{1.405244in}{1.378147in}}%
\pgfpathclose%
\pgfusepath{stroke,fill}%
\end{pgfscope}%
\begin{pgfscope}%
\pgfpathrectangle{\pgfqpoint{0.633874in}{0.569136in}}{\pgfqpoint{2.177280in}{2.201755in}}%
\pgfusepath{clip}%
\pgfsetbuttcap%
\pgfsetroundjoin%
\definecolor{currentfill}{rgb}{1.000000,0.498039,0.054902}%
\pgfsetfillcolor{currentfill}%
\pgfsetlinewidth{0.481800pt}%
\definecolor{currentstroke}{rgb}{1.000000,1.000000,1.000000}%
\pgfsetstrokecolor{currentstroke}%
\pgfsetdash{}{0pt}%
\pgfpathmoveto{\pgfqpoint{1.525060in}{1.544947in}}%
\pgfpathcurveto{\pgfqpoint{1.536110in}{1.544947in}}{\pgfqpoint{1.546709in}{1.549337in}}{\pgfqpoint{1.554523in}{1.557151in}}%
\pgfpathcurveto{\pgfqpoint{1.562336in}{1.564964in}}{\pgfqpoint{1.566726in}{1.575563in}}{\pgfqpoint{1.566726in}{1.586613in}}%
\pgfpathcurveto{\pgfqpoint{1.566726in}{1.597663in}}{\pgfqpoint{1.562336in}{1.608262in}}{\pgfqpoint{1.554523in}{1.616076in}}%
\pgfpathcurveto{\pgfqpoint{1.546709in}{1.623890in}}{\pgfqpoint{1.536110in}{1.628280in}}{\pgfqpoint{1.525060in}{1.628280in}}%
\pgfpathcurveto{\pgfqpoint{1.514010in}{1.628280in}}{\pgfqpoint{1.503411in}{1.623890in}}{\pgfqpoint{1.495597in}{1.616076in}}%
\pgfpathcurveto{\pgfqpoint{1.487783in}{1.608262in}}{\pgfqpoint{1.483393in}{1.597663in}}{\pgfqpoint{1.483393in}{1.586613in}}%
\pgfpathcurveto{\pgfqpoint{1.483393in}{1.575563in}}{\pgfqpoint{1.487783in}{1.564964in}}{\pgfqpoint{1.495597in}{1.557151in}}%
\pgfpathcurveto{\pgfqpoint{1.503411in}{1.549337in}}{\pgfqpoint{1.514010in}{1.544947in}}{\pgfqpoint{1.525060in}{1.544947in}}%
\pgfpathlineto{\pgfqpoint{1.525060in}{1.544947in}}%
\pgfpathclose%
\pgfusepath{stroke,fill}%
\end{pgfscope}%
\begin{pgfscope}%
\pgfpathrectangle{\pgfqpoint{0.633874in}{0.569136in}}{\pgfqpoint{2.177280in}{2.201755in}}%
\pgfusepath{clip}%
\pgfsetbuttcap%
\pgfsetroundjoin%
\definecolor{currentfill}{rgb}{1.000000,0.498039,0.054902}%
\pgfsetfillcolor{currentfill}%
\pgfsetlinewidth{0.481800pt}%
\definecolor{currentstroke}{rgb}{1.000000,1.000000,1.000000}%
\pgfsetstrokecolor{currentstroke}%
\pgfsetdash{}{0pt}%
\pgfpathmoveto{\pgfqpoint{1.604937in}{1.878546in}}%
\pgfpathcurveto{\pgfqpoint{1.615987in}{1.878546in}}{\pgfqpoint{1.626586in}{1.882936in}}{\pgfqpoint{1.634400in}{1.890750in}}%
\pgfpathcurveto{\pgfqpoint{1.642214in}{1.898563in}}{\pgfqpoint{1.646604in}{1.909162in}}{\pgfqpoint{1.646604in}{1.920213in}}%
\pgfpathcurveto{\pgfqpoint{1.646604in}{1.931263in}}{\pgfqpoint{1.642214in}{1.941862in}}{\pgfqpoint{1.634400in}{1.949675in}}%
\pgfpathcurveto{\pgfqpoint{1.626586in}{1.957489in}}{\pgfqpoint{1.615987in}{1.961879in}}{\pgfqpoint{1.604937in}{1.961879in}}%
\pgfpathcurveto{\pgfqpoint{1.593887in}{1.961879in}}{\pgfqpoint{1.583288in}{1.957489in}}{\pgfqpoint{1.575474in}{1.949675in}}%
\pgfpathcurveto{\pgfqpoint{1.567661in}{1.941862in}}{\pgfqpoint{1.563270in}{1.931263in}}{\pgfqpoint{1.563270in}{1.920213in}}%
\pgfpathcurveto{\pgfqpoint{1.563270in}{1.909162in}}{\pgfqpoint{1.567661in}{1.898563in}}{\pgfqpoint{1.575474in}{1.890750in}}%
\pgfpathcurveto{\pgfqpoint{1.583288in}{1.882936in}}{\pgfqpoint{1.593887in}{1.878546in}}{\pgfqpoint{1.604937in}{1.878546in}}%
\pgfpathlineto{\pgfqpoint{1.604937in}{1.878546in}}%
\pgfpathclose%
\pgfusepath{stroke,fill}%
\end{pgfscope}%
\begin{pgfscope}%
\pgfpathrectangle{\pgfqpoint{0.633874in}{0.569136in}}{\pgfqpoint{2.177280in}{2.201755in}}%
\pgfusepath{clip}%
\pgfsetbuttcap%
\pgfsetroundjoin%
\definecolor{currentfill}{rgb}{1.000000,0.498039,0.054902}%
\pgfsetfillcolor{currentfill}%
\pgfsetlinewidth{0.481800pt}%
\definecolor{currentstroke}{rgb}{1.000000,1.000000,1.000000}%
\pgfsetstrokecolor{currentstroke}%
\pgfsetdash{}{0pt}%
\pgfpathmoveto{\pgfqpoint{1.365305in}{1.795146in}}%
\pgfpathcurveto{\pgfqpoint{1.376355in}{1.795146in}}{\pgfqpoint{1.386954in}{1.799536in}}{\pgfqpoint{1.394768in}{1.807350in}}%
\pgfpathcurveto{\pgfqpoint{1.402581in}{1.815164in}}{\pgfqpoint{1.406972in}{1.825763in}}{\pgfqpoint{1.406972in}{1.836813in}}%
\pgfpathcurveto{\pgfqpoint{1.406972in}{1.847863in}}{\pgfqpoint{1.402581in}{1.858462in}}{\pgfqpoint{1.394768in}{1.866276in}}%
\pgfpathcurveto{\pgfqpoint{1.386954in}{1.874089in}}{\pgfqpoint{1.376355in}{1.878479in}}{\pgfqpoint{1.365305in}{1.878479in}}%
\pgfpathcurveto{\pgfqpoint{1.354255in}{1.878479in}}{\pgfqpoint{1.343656in}{1.874089in}}{\pgfqpoint{1.335842in}{1.866276in}}%
\pgfpathcurveto{\pgfqpoint{1.328029in}{1.858462in}}{\pgfqpoint{1.323638in}{1.847863in}}{\pgfqpoint{1.323638in}{1.836813in}}%
\pgfpathcurveto{\pgfqpoint{1.323638in}{1.825763in}}{\pgfqpoint{1.328029in}{1.815164in}}{\pgfqpoint{1.335842in}{1.807350in}}%
\pgfpathcurveto{\pgfqpoint{1.343656in}{1.799536in}}{\pgfqpoint{1.354255in}{1.795146in}}{\pgfqpoint{1.365305in}{1.795146in}}%
\pgfpathlineto{\pgfqpoint{1.365305in}{1.795146in}}%
\pgfpathclose%
\pgfusepath{stroke,fill}%
\end{pgfscope}%
\begin{pgfscope}%
\pgfpathrectangle{\pgfqpoint{0.633874in}{0.569136in}}{\pgfqpoint{2.177280in}{2.201755in}}%
\pgfusepath{clip}%
\pgfsetbuttcap%
\pgfsetroundjoin%
\definecolor{currentfill}{rgb}{1.000000,0.498039,0.054902}%
\pgfsetfillcolor{currentfill}%
\pgfsetlinewidth{0.481800pt}%
\definecolor{currentstroke}{rgb}{1.000000,1.000000,1.000000}%
\pgfsetstrokecolor{currentstroke}%
\pgfsetdash{}{0pt}%
\pgfpathmoveto{\pgfqpoint{1.604937in}{1.878546in}}%
\pgfpathcurveto{\pgfqpoint{1.615987in}{1.878546in}}{\pgfqpoint{1.626586in}{1.882936in}}{\pgfqpoint{1.634400in}{1.890750in}}%
\pgfpathcurveto{\pgfqpoint{1.642214in}{1.898563in}}{\pgfqpoint{1.646604in}{1.909162in}}{\pgfqpoint{1.646604in}{1.920213in}}%
\pgfpathcurveto{\pgfqpoint{1.646604in}{1.931263in}}{\pgfqpoint{1.642214in}{1.941862in}}{\pgfqpoint{1.634400in}{1.949675in}}%
\pgfpathcurveto{\pgfqpoint{1.626586in}{1.957489in}}{\pgfqpoint{1.615987in}{1.961879in}}{\pgfqpoint{1.604937in}{1.961879in}}%
\pgfpathcurveto{\pgfqpoint{1.593887in}{1.961879in}}{\pgfqpoint{1.583288in}{1.957489in}}{\pgfqpoint{1.575474in}{1.949675in}}%
\pgfpathcurveto{\pgfqpoint{1.567661in}{1.941862in}}{\pgfqpoint{1.563270in}{1.931263in}}{\pgfqpoint{1.563270in}{1.920213in}}%
\pgfpathcurveto{\pgfqpoint{1.563270in}{1.909162in}}{\pgfqpoint{1.567661in}{1.898563in}}{\pgfqpoint{1.575474in}{1.890750in}}%
\pgfpathcurveto{\pgfqpoint{1.583288in}{1.882936in}}{\pgfqpoint{1.593887in}{1.878546in}}{\pgfqpoint{1.604937in}{1.878546in}}%
\pgfpathlineto{\pgfqpoint{1.604937in}{1.878546in}}%
\pgfpathclose%
\pgfusepath{stroke,fill}%
\end{pgfscope}%
\begin{pgfscope}%
\pgfpathrectangle{\pgfqpoint{0.633874in}{0.569136in}}{\pgfqpoint{2.177280in}{2.201755in}}%
\pgfusepath{clip}%
\pgfsetbuttcap%
\pgfsetroundjoin%
\definecolor{currentfill}{rgb}{1.000000,0.498039,0.054902}%
\pgfsetfillcolor{currentfill}%
\pgfsetlinewidth{0.481800pt}%
\definecolor{currentstroke}{rgb}{1.000000,1.000000,1.000000}%
\pgfsetstrokecolor{currentstroke}%
\pgfsetdash{}{0pt}%
\pgfpathmoveto{\pgfqpoint{1.884508in}{1.795146in}}%
\pgfpathcurveto{\pgfqpoint{1.895558in}{1.795146in}}{\pgfqpoint{1.906157in}{1.799536in}}{\pgfqpoint{1.913971in}{1.807350in}}%
\pgfpathcurveto{\pgfqpoint{1.921784in}{1.815164in}}{\pgfqpoint{1.926175in}{1.825763in}}{\pgfqpoint{1.926175in}{1.836813in}}%
\pgfpathcurveto{\pgfqpoint{1.926175in}{1.847863in}}{\pgfqpoint{1.921784in}{1.858462in}}{\pgfqpoint{1.913971in}{1.866276in}}%
\pgfpathcurveto{\pgfqpoint{1.906157in}{1.874089in}}{\pgfqpoint{1.895558in}{1.878479in}}{\pgfqpoint{1.884508in}{1.878479in}}%
\pgfpathcurveto{\pgfqpoint{1.873458in}{1.878479in}}{\pgfqpoint{1.862859in}{1.874089in}}{\pgfqpoint{1.855045in}{1.866276in}}%
\pgfpathcurveto{\pgfqpoint{1.847232in}{1.858462in}}{\pgfqpoint{1.842841in}{1.847863in}}{\pgfqpoint{1.842841in}{1.836813in}}%
\pgfpathcurveto{\pgfqpoint{1.842841in}{1.825763in}}{\pgfqpoint{1.847232in}{1.815164in}}{\pgfqpoint{1.855045in}{1.807350in}}%
\pgfpathcurveto{\pgfqpoint{1.862859in}{1.799536in}}{\pgfqpoint{1.873458in}{1.795146in}}{\pgfqpoint{1.884508in}{1.795146in}}%
\pgfpathlineto{\pgfqpoint{1.884508in}{1.795146in}}%
\pgfpathclose%
\pgfusepath{stroke,fill}%
\end{pgfscope}%
\begin{pgfscope}%
\pgfpathrectangle{\pgfqpoint{0.633874in}{0.569136in}}{\pgfqpoint{2.177280in}{2.201755in}}%
\pgfusepath{clip}%
\pgfsetbuttcap%
\pgfsetroundjoin%
\definecolor{currentfill}{rgb}{1.000000,0.498039,0.054902}%
\pgfsetfillcolor{currentfill}%
\pgfsetlinewidth{0.481800pt}%
\definecolor{currentstroke}{rgb}{1.000000,1.000000,1.000000}%
\pgfsetstrokecolor{currentstroke}%
\pgfsetdash{}{0pt}%
\pgfpathmoveto{\pgfqpoint{1.724753in}{1.628346in}}%
\pgfpathcurveto{\pgfqpoint{1.735803in}{1.628346in}}{\pgfqpoint{1.746402in}{1.632737in}}{\pgfqpoint{1.754216in}{1.640550in}}%
\pgfpathcurveto{\pgfqpoint{1.762030in}{1.648364in}}{\pgfqpoint{1.766420in}{1.658963in}}{\pgfqpoint{1.766420in}{1.670013in}}%
\pgfpathcurveto{\pgfqpoint{1.766420in}{1.681063in}}{\pgfqpoint{1.762030in}{1.691662in}}{\pgfqpoint{1.754216in}{1.699476in}}%
\pgfpathcurveto{\pgfqpoint{1.746402in}{1.707290in}}{\pgfqpoint{1.735803in}{1.711680in}}{\pgfqpoint{1.724753in}{1.711680in}}%
\pgfpathcurveto{\pgfqpoint{1.713703in}{1.711680in}}{\pgfqpoint{1.703104in}{1.707290in}}{\pgfqpoint{1.695290in}{1.699476in}}%
\pgfpathcurveto{\pgfqpoint{1.687477in}{1.691662in}}{\pgfqpoint{1.683087in}{1.681063in}}{\pgfqpoint{1.683087in}{1.670013in}}%
\pgfpathcurveto{\pgfqpoint{1.683087in}{1.658963in}}{\pgfqpoint{1.687477in}{1.648364in}}{\pgfqpoint{1.695290in}{1.640550in}}%
\pgfpathcurveto{\pgfqpoint{1.703104in}{1.632737in}}{\pgfqpoint{1.713703in}{1.628346in}}{\pgfqpoint{1.724753in}{1.628346in}}%
\pgfpathlineto{\pgfqpoint{1.724753in}{1.628346in}}%
\pgfpathclose%
\pgfusepath{stroke,fill}%
\end{pgfscope}%
\begin{pgfscope}%
\pgfpathrectangle{\pgfqpoint{0.633874in}{0.569136in}}{\pgfqpoint{2.177280in}{2.201755in}}%
\pgfusepath{clip}%
\pgfsetbuttcap%
\pgfsetroundjoin%
\definecolor{currentfill}{rgb}{1.000000,0.498039,0.054902}%
\pgfsetfillcolor{currentfill}%
\pgfsetlinewidth{0.481800pt}%
\definecolor{currentstroke}{rgb}{1.000000,1.000000,1.000000}%
\pgfsetstrokecolor{currentstroke}%
\pgfsetdash{}{0pt}%
\pgfpathmoveto{\pgfqpoint{1.445182in}{1.628346in}}%
\pgfpathcurveto{\pgfqpoint{1.456232in}{1.628346in}}{\pgfqpoint{1.466831in}{1.632737in}}{\pgfqpoint{1.474645in}{1.640550in}}%
\pgfpathcurveto{\pgfqpoint{1.482459in}{1.648364in}}{\pgfqpoint{1.486849in}{1.658963in}}{\pgfqpoint{1.486849in}{1.670013in}}%
\pgfpathcurveto{\pgfqpoint{1.486849in}{1.681063in}}{\pgfqpoint{1.482459in}{1.691662in}}{\pgfqpoint{1.474645in}{1.699476in}}%
\pgfpathcurveto{\pgfqpoint{1.466831in}{1.707290in}}{\pgfqpoint{1.456232in}{1.711680in}}{\pgfqpoint{1.445182in}{1.711680in}}%
\pgfpathcurveto{\pgfqpoint{1.434132in}{1.711680in}}{\pgfqpoint{1.423533in}{1.707290in}}{\pgfqpoint{1.415720in}{1.699476in}}%
\pgfpathcurveto{\pgfqpoint{1.407906in}{1.691662in}}{\pgfqpoint{1.403516in}{1.681063in}}{\pgfqpoint{1.403516in}{1.670013in}}%
\pgfpathcurveto{\pgfqpoint{1.403516in}{1.658963in}}{\pgfqpoint{1.407906in}{1.648364in}}{\pgfqpoint{1.415720in}{1.640550in}}%
\pgfpathcurveto{\pgfqpoint{1.423533in}{1.632737in}}{\pgfqpoint{1.434132in}{1.628346in}}{\pgfqpoint{1.445182in}{1.628346in}}%
\pgfpathlineto{\pgfqpoint{1.445182in}{1.628346in}}%
\pgfpathclose%
\pgfusepath{stroke,fill}%
\end{pgfscope}%
\begin{pgfscope}%
\pgfpathrectangle{\pgfqpoint{0.633874in}{0.569136in}}{\pgfqpoint{2.177280in}{2.201755in}}%
\pgfusepath{clip}%
\pgfsetbuttcap%
\pgfsetroundjoin%
\definecolor{currentfill}{rgb}{1.000000,0.498039,0.054902}%
\pgfsetfillcolor{currentfill}%
\pgfsetlinewidth{0.481800pt}%
\definecolor{currentstroke}{rgb}{1.000000,1.000000,1.000000}%
\pgfsetstrokecolor{currentstroke}%
\pgfsetdash{}{0pt}%
\pgfpathmoveto{\pgfqpoint{1.405244in}{1.628346in}}%
\pgfpathcurveto{\pgfqpoint{1.416294in}{1.628346in}}{\pgfqpoint{1.426893in}{1.632737in}}{\pgfqpoint{1.434706in}{1.640550in}}%
\pgfpathcurveto{\pgfqpoint{1.442520in}{1.648364in}}{\pgfqpoint{1.446910in}{1.658963in}}{\pgfqpoint{1.446910in}{1.670013in}}%
\pgfpathcurveto{\pgfqpoint{1.446910in}{1.681063in}}{\pgfqpoint{1.442520in}{1.691662in}}{\pgfqpoint{1.434706in}{1.699476in}}%
\pgfpathcurveto{\pgfqpoint{1.426893in}{1.707290in}}{\pgfqpoint{1.416294in}{1.711680in}}{\pgfqpoint{1.405244in}{1.711680in}}%
\pgfpathcurveto{\pgfqpoint{1.394193in}{1.711680in}}{\pgfqpoint{1.383594in}{1.707290in}}{\pgfqpoint{1.375781in}{1.699476in}}%
\pgfpathcurveto{\pgfqpoint{1.367967in}{1.691662in}}{\pgfqpoint{1.363577in}{1.681063in}}{\pgfqpoint{1.363577in}{1.670013in}}%
\pgfpathcurveto{\pgfqpoint{1.363577in}{1.658963in}}{\pgfqpoint{1.367967in}{1.648364in}}{\pgfqpoint{1.375781in}{1.640550in}}%
\pgfpathcurveto{\pgfqpoint{1.383594in}{1.632737in}}{\pgfqpoint{1.394193in}{1.628346in}}{\pgfqpoint{1.405244in}{1.628346in}}%
\pgfpathlineto{\pgfqpoint{1.405244in}{1.628346in}}%
\pgfpathclose%
\pgfusepath{stroke,fill}%
\end{pgfscope}%
\begin{pgfscope}%
\pgfpathrectangle{\pgfqpoint{0.633874in}{0.569136in}}{\pgfqpoint{2.177280in}{2.201755in}}%
\pgfusepath{clip}%
\pgfsetbuttcap%
\pgfsetroundjoin%
\definecolor{currentfill}{rgb}{1.000000,0.498039,0.054902}%
\pgfsetfillcolor{currentfill}%
\pgfsetlinewidth{0.481800pt}%
\definecolor{currentstroke}{rgb}{1.000000,1.000000,1.000000}%
\pgfsetstrokecolor{currentstroke}%
\pgfsetdash{}{0pt}%
\pgfpathmoveto{\pgfqpoint{1.405244in}{1.544947in}}%
\pgfpathcurveto{\pgfqpoint{1.416294in}{1.544947in}}{\pgfqpoint{1.426893in}{1.549337in}}{\pgfqpoint{1.434706in}{1.557151in}}%
\pgfpathcurveto{\pgfqpoint{1.442520in}{1.564964in}}{\pgfqpoint{1.446910in}{1.575563in}}{\pgfqpoint{1.446910in}{1.586613in}}%
\pgfpathcurveto{\pgfqpoint{1.446910in}{1.597663in}}{\pgfqpoint{1.442520in}{1.608262in}}{\pgfqpoint{1.434706in}{1.616076in}}%
\pgfpathcurveto{\pgfqpoint{1.426893in}{1.623890in}}{\pgfqpoint{1.416294in}{1.628280in}}{\pgfqpoint{1.405244in}{1.628280in}}%
\pgfpathcurveto{\pgfqpoint{1.394193in}{1.628280in}}{\pgfqpoint{1.383594in}{1.623890in}}{\pgfqpoint{1.375781in}{1.616076in}}%
\pgfpathcurveto{\pgfqpoint{1.367967in}{1.608262in}}{\pgfqpoint{1.363577in}{1.597663in}}{\pgfqpoint{1.363577in}{1.586613in}}%
\pgfpathcurveto{\pgfqpoint{1.363577in}{1.575563in}}{\pgfqpoint{1.367967in}{1.564964in}}{\pgfqpoint{1.375781in}{1.557151in}}%
\pgfpathcurveto{\pgfqpoint{1.383594in}{1.549337in}}{\pgfqpoint{1.394193in}{1.544947in}}{\pgfqpoint{1.405244in}{1.544947in}}%
\pgfpathlineto{\pgfqpoint{1.405244in}{1.544947in}}%
\pgfpathclose%
\pgfusepath{stroke,fill}%
\end{pgfscope}%
\begin{pgfscope}%
\pgfpathrectangle{\pgfqpoint{0.633874in}{0.569136in}}{\pgfqpoint{2.177280in}{2.201755in}}%
\pgfusepath{clip}%
\pgfsetbuttcap%
\pgfsetroundjoin%
\definecolor{currentfill}{rgb}{1.000000,0.498039,0.054902}%
\pgfsetfillcolor{currentfill}%
\pgfsetlinewidth{0.481800pt}%
\definecolor{currentstroke}{rgb}{1.000000,1.000000,1.000000}%
\pgfsetstrokecolor{currentstroke}%
\pgfsetdash{}{0pt}%
\pgfpathmoveto{\pgfqpoint{1.644876in}{1.711746in}}%
\pgfpathcurveto{\pgfqpoint{1.655926in}{1.711746in}}{\pgfqpoint{1.666525in}{1.716137in}}{\pgfqpoint{1.674339in}{1.723950in}}%
\pgfpathcurveto{\pgfqpoint{1.682152in}{1.731764in}}{\pgfqpoint{1.686543in}{1.742363in}}{\pgfqpoint{1.686543in}{1.753413in}}%
\pgfpathcurveto{\pgfqpoint{1.686543in}{1.764463in}}{\pgfqpoint{1.682152in}{1.775062in}}{\pgfqpoint{1.674339in}{1.782876in}}%
\pgfpathcurveto{\pgfqpoint{1.666525in}{1.790689in}}{\pgfqpoint{1.655926in}{1.795080in}}{\pgfqpoint{1.644876in}{1.795080in}}%
\pgfpathcurveto{\pgfqpoint{1.633826in}{1.795080in}}{\pgfqpoint{1.623227in}{1.790689in}}{\pgfqpoint{1.615413in}{1.782876in}}%
\pgfpathcurveto{\pgfqpoint{1.607599in}{1.775062in}}{\pgfqpoint{1.603209in}{1.764463in}}{\pgfqpoint{1.603209in}{1.753413in}}%
\pgfpathcurveto{\pgfqpoint{1.603209in}{1.742363in}}{\pgfqpoint{1.607599in}{1.731764in}}{\pgfqpoint{1.615413in}{1.723950in}}%
\pgfpathcurveto{\pgfqpoint{1.623227in}{1.716137in}}{\pgfqpoint{1.633826in}{1.711746in}}{\pgfqpoint{1.644876in}{1.711746in}}%
\pgfpathlineto{\pgfqpoint{1.644876in}{1.711746in}}%
\pgfpathclose%
\pgfusepath{stroke,fill}%
\end{pgfscope}%
\begin{pgfscope}%
\pgfpathrectangle{\pgfqpoint{0.633874in}{0.569136in}}{\pgfqpoint{2.177280in}{2.201755in}}%
\pgfusepath{clip}%
\pgfsetbuttcap%
\pgfsetroundjoin%
\definecolor{currentfill}{rgb}{1.000000,0.498039,0.054902}%
\pgfsetfillcolor{currentfill}%
\pgfsetlinewidth{0.481800pt}%
\definecolor{currentstroke}{rgb}{1.000000,1.000000,1.000000}%
\pgfsetstrokecolor{currentstroke}%
\pgfsetdash{}{0pt}%
\pgfpathmoveto{\pgfqpoint{1.525060in}{1.544947in}}%
\pgfpathcurveto{\pgfqpoint{1.536110in}{1.544947in}}{\pgfqpoint{1.546709in}{1.549337in}}{\pgfqpoint{1.554523in}{1.557151in}}%
\pgfpathcurveto{\pgfqpoint{1.562336in}{1.564964in}}{\pgfqpoint{1.566726in}{1.575563in}}{\pgfqpoint{1.566726in}{1.586613in}}%
\pgfpathcurveto{\pgfqpoint{1.566726in}{1.597663in}}{\pgfqpoint{1.562336in}{1.608262in}}{\pgfqpoint{1.554523in}{1.616076in}}%
\pgfpathcurveto{\pgfqpoint{1.546709in}{1.623890in}}{\pgfqpoint{1.536110in}{1.628280in}}{\pgfqpoint{1.525060in}{1.628280in}}%
\pgfpathcurveto{\pgfqpoint{1.514010in}{1.628280in}}{\pgfqpoint{1.503411in}{1.623890in}}{\pgfqpoint{1.495597in}{1.616076in}}%
\pgfpathcurveto{\pgfqpoint{1.487783in}{1.608262in}}{\pgfqpoint{1.483393in}{1.597663in}}{\pgfqpoint{1.483393in}{1.586613in}}%
\pgfpathcurveto{\pgfqpoint{1.483393in}{1.575563in}}{\pgfqpoint{1.487783in}{1.564964in}}{\pgfqpoint{1.495597in}{1.557151in}}%
\pgfpathcurveto{\pgfqpoint{1.503411in}{1.549337in}}{\pgfqpoint{1.514010in}{1.544947in}}{\pgfqpoint{1.525060in}{1.544947in}}%
\pgfpathlineto{\pgfqpoint{1.525060in}{1.544947in}}%
\pgfpathclose%
\pgfusepath{stroke,fill}%
\end{pgfscope}%
\begin{pgfscope}%
\pgfpathrectangle{\pgfqpoint{0.633874in}{0.569136in}}{\pgfqpoint{2.177280in}{2.201755in}}%
\pgfusepath{clip}%
\pgfsetbuttcap%
\pgfsetroundjoin%
\definecolor{currentfill}{rgb}{1.000000,0.498039,0.054902}%
\pgfsetfillcolor{currentfill}%
\pgfsetlinewidth{0.481800pt}%
\definecolor{currentstroke}{rgb}{1.000000,1.000000,1.000000}%
\pgfsetstrokecolor{currentstroke}%
\pgfsetdash{}{0pt}%
\pgfpathmoveto{\pgfqpoint{1.205550in}{1.378147in}}%
\pgfpathcurveto{\pgfqpoint{1.216600in}{1.378147in}}{\pgfqpoint{1.227199in}{1.382537in}}{\pgfqpoint{1.235013in}{1.390351in}}%
\pgfpathcurveto{\pgfqpoint{1.242826in}{1.398164in}}{\pgfqpoint{1.247217in}{1.408764in}}{\pgfqpoint{1.247217in}{1.419814in}}%
\pgfpathcurveto{\pgfqpoint{1.247217in}{1.430864in}}{\pgfqpoint{1.242826in}{1.441463in}}{\pgfqpoint{1.235013in}{1.449276in}}%
\pgfpathcurveto{\pgfqpoint{1.227199in}{1.457090in}}{\pgfqpoint{1.216600in}{1.461480in}}{\pgfqpoint{1.205550in}{1.461480in}}%
\pgfpathcurveto{\pgfqpoint{1.194500in}{1.461480in}}{\pgfqpoint{1.183901in}{1.457090in}}{\pgfqpoint{1.176087in}{1.449276in}}%
\pgfpathcurveto{\pgfqpoint{1.168274in}{1.441463in}}{\pgfqpoint{1.163883in}{1.430864in}}{\pgfqpoint{1.163883in}{1.419814in}}%
\pgfpathcurveto{\pgfqpoint{1.163883in}{1.408764in}}{\pgfqpoint{1.168274in}{1.398164in}}{\pgfqpoint{1.176087in}{1.390351in}}%
\pgfpathcurveto{\pgfqpoint{1.183901in}{1.382537in}}{\pgfqpoint{1.194500in}{1.378147in}}{\pgfqpoint{1.205550in}{1.378147in}}%
\pgfpathlineto{\pgfqpoint{1.205550in}{1.378147in}}%
\pgfpathclose%
\pgfusepath{stroke,fill}%
\end{pgfscope}%
\begin{pgfscope}%
\pgfpathrectangle{\pgfqpoint{0.633874in}{0.569136in}}{\pgfqpoint{2.177280in}{2.201755in}}%
\pgfusepath{clip}%
\pgfsetbuttcap%
\pgfsetroundjoin%
\definecolor{currentfill}{rgb}{1.000000,0.498039,0.054902}%
\pgfsetfillcolor{currentfill}%
\pgfsetlinewidth{0.481800pt}%
\definecolor{currentstroke}{rgb}{1.000000,1.000000,1.000000}%
\pgfsetstrokecolor{currentstroke}%
\pgfsetdash{}{0pt}%
\pgfpathmoveto{\pgfqpoint{1.445182in}{1.628346in}}%
\pgfpathcurveto{\pgfqpoint{1.456232in}{1.628346in}}{\pgfqpoint{1.466831in}{1.632737in}}{\pgfqpoint{1.474645in}{1.640550in}}%
\pgfpathcurveto{\pgfqpoint{1.482459in}{1.648364in}}{\pgfqpoint{1.486849in}{1.658963in}}{\pgfqpoint{1.486849in}{1.670013in}}%
\pgfpathcurveto{\pgfqpoint{1.486849in}{1.681063in}}{\pgfqpoint{1.482459in}{1.691662in}}{\pgfqpoint{1.474645in}{1.699476in}}%
\pgfpathcurveto{\pgfqpoint{1.466831in}{1.707290in}}{\pgfqpoint{1.456232in}{1.711680in}}{\pgfqpoint{1.445182in}{1.711680in}}%
\pgfpathcurveto{\pgfqpoint{1.434132in}{1.711680in}}{\pgfqpoint{1.423533in}{1.707290in}}{\pgfqpoint{1.415720in}{1.699476in}}%
\pgfpathcurveto{\pgfqpoint{1.407906in}{1.691662in}}{\pgfqpoint{1.403516in}{1.681063in}}{\pgfqpoint{1.403516in}{1.670013in}}%
\pgfpathcurveto{\pgfqpoint{1.403516in}{1.658963in}}{\pgfqpoint{1.407906in}{1.648364in}}{\pgfqpoint{1.415720in}{1.640550in}}%
\pgfpathcurveto{\pgfqpoint{1.423533in}{1.632737in}}{\pgfqpoint{1.434132in}{1.628346in}}{\pgfqpoint{1.445182in}{1.628346in}}%
\pgfpathlineto{\pgfqpoint{1.445182in}{1.628346in}}%
\pgfpathclose%
\pgfusepath{stroke,fill}%
\end{pgfscope}%
\begin{pgfscope}%
\pgfpathrectangle{\pgfqpoint{0.633874in}{0.569136in}}{\pgfqpoint{2.177280in}{2.201755in}}%
\pgfusepath{clip}%
\pgfsetbuttcap%
\pgfsetroundjoin%
\definecolor{currentfill}{rgb}{1.000000,0.498039,0.054902}%
\pgfsetfillcolor{currentfill}%
\pgfsetlinewidth{0.481800pt}%
\definecolor{currentstroke}{rgb}{1.000000,1.000000,1.000000}%
\pgfsetstrokecolor{currentstroke}%
\pgfsetdash{}{0pt}%
\pgfpathmoveto{\pgfqpoint{1.485121in}{1.544947in}}%
\pgfpathcurveto{\pgfqpoint{1.496171in}{1.544947in}}{\pgfqpoint{1.506770in}{1.549337in}}{\pgfqpoint{1.514584in}{1.557151in}}%
\pgfpathcurveto{\pgfqpoint{1.522397in}{1.564964in}}{\pgfqpoint{1.526788in}{1.575563in}}{\pgfqpoint{1.526788in}{1.586613in}}%
\pgfpathcurveto{\pgfqpoint{1.526788in}{1.597663in}}{\pgfqpoint{1.522397in}{1.608262in}}{\pgfqpoint{1.514584in}{1.616076in}}%
\pgfpathcurveto{\pgfqpoint{1.506770in}{1.623890in}}{\pgfqpoint{1.496171in}{1.628280in}}{\pgfqpoint{1.485121in}{1.628280in}}%
\pgfpathcurveto{\pgfqpoint{1.474071in}{1.628280in}}{\pgfqpoint{1.463472in}{1.623890in}}{\pgfqpoint{1.455658in}{1.616076in}}%
\pgfpathcurveto{\pgfqpoint{1.447845in}{1.608262in}}{\pgfqpoint{1.443454in}{1.597663in}}{\pgfqpoint{1.443454in}{1.586613in}}%
\pgfpathcurveto{\pgfqpoint{1.443454in}{1.575563in}}{\pgfqpoint{1.447845in}{1.564964in}}{\pgfqpoint{1.455658in}{1.557151in}}%
\pgfpathcurveto{\pgfqpoint{1.463472in}{1.549337in}}{\pgfqpoint{1.474071in}{1.544947in}}{\pgfqpoint{1.485121in}{1.544947in}}%
\pgfpathlineto{\pgfqpoint{1.485121in}{1.544947in}}%
\pgfpathclose%
\pgfusepath{stroke,fill}%
\end{pgfscope}%
\begin{pgfscope}%
\pgfpathrectangle{\pgfqpoint{0.633874in}{0.569136in}}{\pgfqpoint{2.177280in}{2.201755in}}%
\pgfusepath{clip}%
\pgfsetbuttcap%
\pgfsetroundjoin%
\definecolor{currentfill}{rgb}{1.000000,0.498039,0.054902}%
\pgfsetfillcolor{currentfill}%
\pgfsetlinewidth{0.481800pt}%
\definecolor{currentstroke}{rgb}{1.000000,1.000000,1.000000}%
\pgfsetstrokecolor{currentstroke}%
\pgfsetdash{}{0pt}%
\pgfpathmoveto{\pgfqpoint{1.485121in}{1.628346in}}%
\pgfpathcurveto{\pgfqpoint{1.496171in}{1.628346in}}{\pgfqpoint{1.506770in}{1.632737in}}{\pgfqpoint{1.514584in}{1.640550in}}%
\pgfpathcurveto{\pgfqpoint{1.522397in}{1.648364in}}{\pgfqpoint{1.526788in}{1.658963in}}{\pgfqpoint{1.526788in}{1.670013in}}%
\pgfpathcurveto{\pgfqpoint{1.526788in}{1.681063in}}{\pgfqpoint{1.522397in}{1.691662in}}{\pgfqpoint{1.514584in}{1.699476in}}%
\pgfpathcurveto{\pgfqpoint{1.506770in}{1.707290in}}{\pgfqpoint{1.496171in}{1.711680in}}{\pgfqpoint{1.485121in}{1.711680in}}%
\pgfpathcurveto{\pgfqpoint{1.474071in}{1.711680in}}{\pgfqpoint{1.463472in}{1.707290in}}{\pgfqpoint{1.455658in}{1.699476in}}%
\pgfpathcurveto{\pgfqpoint{1.447845in}{1.691662in}}{\pgfqpoint{1.443454in}{1.681063in}}{\pgfqpoint{1.443454in}{1.670013in}}%
\pgfpathcurveto{\pgfqpoint{1.443454in}{1.658963in}}{\pgfqpoint{1.447845in}{1.648364in}}{\pgfqpoint{1.455658in}{1.640550in}}%
\pgfpathcurveto{\pgfqpoint{1.463472in}{1.632737in}}{\pgfqpoint{1.474071in}{1.628346in}}{\pgfqpoint{1.485121in}{1.628346in}}%
\pgfpathlineto{\pgfqpoint{1.485121in}{1.628346in}}%
\pgfpathclose%
\pgfusepath{stroke,fill}%
\end{pgfscope}%
\begin{pgfscope}%
\pgfpathrectangle{\pgfqpoint{0.633874in}{0.569136in}}{\pgfqpoint{2.177280in}{2.201755in}}%
\pgfusepath{clip}%
\pgfsetbuttcap%
\pgfsetroundjoin%
\definecolor{currentfill}{rgb}{1.000000,0.498039,0.054902}%
\pgfsetfillcolor{currentfill}%
\pgfsetlinewidth{0.481800pt}%
\definecolor{currentstroke}{rgb}{1.000000,1.000000,1.000000}%
\pgfsetstrokecolor{currentstroke}%
\pgfsetdash{}{0pt}%
\pgfpathmoveto{\pgfqpoint{1.684815in}{1.628346in}}%
\pgfpathcurveto{\pgfqpoint{1.695865in}{1.628346in}}{\pgfqpoint{1.706464in}{1.632737in}}{\pgfqpoint{1.714277in}{1.640550in}}%
\pgfpathcurveto{\pgfqpoint{1.722091in}{1.648364in}}{\pgfqpoint{1.726481in}{1.658963in}}{\pgfqpoint{1.726481in}{1.670013in}}%
\pgfpathcurveto{\pgfqpoint{1.726481in}{1.681063in}}{\pgfqpoint{1.722091in}{1.691662in}}{\pgfqpoint{1.714277in}{1.699476in}}%
\pgfpathcurveto{\pgfqpoint{1.706464in}{1.707290in}}{\pgfqpoint{1.695865in}{1.711680in}}{\pgfqpoint{1.684815in}{1.711680in}}%
\pgfpathcurveto{\pgfqpoint{1.673764in}{1.711680in}}{\pgfqpoint{1.663165in}{1.707290in}}{\pgfqpoint{1.655352in}{1.699476in}}%
\pgfpathcurveto{\pgfqpoint{1.647538in}{1.691662in}}{\pgfqpoint{1.643148in}{1.681063in}}{\pgfqpoint{1.643148in}{1.670013in}}%
\pgfpathcurveto{\pgfqpoint{1.643148in}{1.658963in}}{\pgfqpoint{1.647538in}{1.648364in}}{\pgfqpoint{1.655352in}{1.640550in}}%
\pgfpathcurveto{\pgfqpoint{1.663165in}{1.632737in}}{\pgfqpoint{1.673764in}{1.628346in}}{\pgfqpoint{1.684815in}{1.628346in}}%
\pgfpathlineto{\pgfqpoint{1.684815in}{1.628346in}}%
\pgfpathclose%
\pgfusepath{stroke,fill}%
\end{pgfscope}%
\begin{pgfscope}%
\pgfpathrectangle{\pgfqpoint{0.633874in}{0.569136in}}{\pgfqpoint{2.177280in}{2.201755in}}%
\pgfusepath{clip}%
\pgfsetbuttcap%
\pgfsetroundjoin%
\definecolor{currentfill}{rgb}{1.000000,0.498039,0.054902}%
\pgfsetfillcolor{currentfill}%
\pgfsetlinewidth{0.481800pt}%
\definecolor{currentstroke}{rgb}{1.000000,1.000000,1.000000}%
\pgfsetstrokecolor{currentstroke}%
\pgfsetdash{}{0pt}%
\pgfpathmoveto{\pgfqpoint{1.245489in}{1.461547in}}%
\pgfpathcurveto{\pgfqpoint{1.256539in}{1.461547in}}{\pgfqpoint{1.267138in}{1.465937in}}{\pgfqpoint{1.274952in}{1.473751in}}%
\pgfpathcurveto{\pgfqpoint{1.282765in}{1.481564in}}{\pgfqpoint{1.287155in}{1.492163in}}{\pgfqpoint{1.287155in}{1.503213in}}%
\pgfpathcurveto{\pgfqpoint{1.287155in}{1.514264in}}{\pgfqpoint{1.282765in}{1.524863in}}{\pgfqpoint{1.274952in}{1.532676in}}%
\pgfpathcurveto{\pgfqpoint{1.267138in}{1.540490in}}{\pgfqpoint{1.256539in}{1.544880in}}{\pgfqpoint{1.245489in}{1.544880in}}%
\pgfpathcurveto{\pgfqpoint{1.234439in}{1.544880in}}{\pgfqpoint{1.223840in}{1.540490in}}{\pgfqpoint{1.216026in}{1.532676in}}%
\pgfpathcurveto{\pgfqpoint{1.208212in}{1.524863in}}{\pgfqpoint{1.203822in}{1.514264in}}{\pgfqpoint{1.203822in}{1.503213in}}%
\pgfpathcurveto{\pgfqpoint{1.203822in}{1.492163in}}{\pgfqpoint{1.208212in}{1.481564in}}{\pgfqpoint{1.216026in}{1.473751in}}%
\pgfpathcurveto{\pgfqpoint{1.223840in}{1.465937in}}{\pgfqpoint{1.234439in}{1.461547in}}{\pgfqpoint{1.245489in}{1.461547in}}%
\pgfpathlineto{\pgfqpoint{1.245489in}{1.461547in}}%
\pgfpathclose%
\pgfusepath{stroke,fill}%
\end{pgfscope}%
\begin{pgfscope}%
\pgfpathrectangle{\pgfqpoint{0.633874in}{0.569136in}}{\pgfqpoint{2.177280in}{2.201755in}}%
\pgfusepath{clip}%
\pgfsetbuttcap%
\pgfsetroundjoin%
\definecolor{currentfill}{rgb}{1.000000,0.498039,0.054902}%
\pgfsetfillcolor{currentfill}%
\pgfsetlinewidth{0.481800pt}%
\definecolor{currentstroke}{rgb}{1.000000,1.000000,1.000000}%
\pgfsetstrokecolor{currentstroke}%
\pgfsetdash{}{0pt}%
\pgfpathmoveto{\pgfqpoint{1.485121in}{1.628346in}}%
\pgfpathcurveto{\pgfqpoint{1.496171in}{1.628346in}}{\pgfqpoint{1.506770in}{1.632737in}}{\pgfqpoint{1.514584in}{1.640550in}}%
\pgfpathcurveto{\pgfqpoint{1.522397in}{1.648364in}}{\pgfqpoint{1.526788in}{1.658963in}}{\pgfqpoint{1.526788in}{1.670013in}}%
\pgfpathcurveto{\pgfqpoint{1.526788in}{1.681063in}}{\pgfqpoint{1.522397in}{1.691662in}}{\pgfqpoint{1.514584in}{1.699476in}}%
\pgfpathcurveto{\pgfqpoint{1.506770in}{1.707290in}}{\pgfqpoint{1.496171in}{1.711680in}}{\pgfqpoint{1.485121in}{1.711680in}}%
\pgfpathcurveto{\pgfqpoint{1.474071in}{1.711680in}}{\pgfqpoint{1.463472in}{1.707290in}}{\pgfqpoint{1.455658in}{1.699476in}}%
\pgfpathcurveto{\pgfqpoint{1.447845in}{1.691662in}}{\pgfqpoint{1.443454in}{1.681063in}}{\pgfqpoint{1.443454in}{1.670013in}}%
\pgfpathcurveto{\pgfqpoint{1.443454in}{1.658963in}}{\pgfqpoint{1.447845in}{1.648364in}}{\pgfqpoint{1.455658in}{1.640550in}}%
\pgfpathcurveto{\pgfqpoint{1.463472in}{1.632737in}}{\pgfqpoint{1.474071in}{1.628346in}}{\pgfqpoint{1.485121in}{1.628346in}}%
\pgfpathlineto{\pgfqpoint{1.485121in}{1.628346in}}%
\pgfpathclose%
\pgfusepath{stroke,fill}%
\end{pgfscope}%
\begin{pgfscope}%
\pgfpathrectangle{\pgfqpoint{0.633874in}{0.569136in}}{\pgfqpoint{2.177280in}{2.201755in}}%
\pgfusepath{clip}%
\pgfsetbuttcap%
\pgfsetroundjoin%
\definecolor{currentfill}{rgb}{0.172549,0.627451,0.172549}%
\pgfsetfillcolor{currentfill}%
\pgfsetlinewidth{0.481800pt}%
\definecolor{currentstroke}{rgb}{1.000000,1.000000,1.000000}%
\pgfsetstrokecolor{currentstroke}%
\pgfsetdash{}{0pt}%
\pgfpathmoveto{\pgfqpoint{1.724753in}{2.629144in}}%
\pgfpathcurveto{\pgfqpoint{1.735803in}{2.629144in}}{\pgfqpoint{1.746402in}{2.633534in}}{\pgfqpoint{1.754216in}{2.641348in}}%
\pgfpathcurveto{\pgfqpoint{1.762030in}{2.649162in}}{\pgfqpoint{1.766420in}{2.659761in}}{\pgfqpoint{1.766420in}{2.670811in}}%
\pgfpathcurveto{\pgfqpoint{1.766420in}{2.681861in}}{\pgfqpoint{1.762030in}{2.692460in}}{\pgfqpoint{1.754216in}{2.700274in}}%
\pgfpathcurveto{\pgfqpoint{1.746402in}{2.708087in}}{\pgfqpoint{1.735803in}{2.712478in}}{\pgfqpoint{1.724753in}{2.712478in}}%
\pgfpathcurveto{\pgfqpoint{1.713703in}{2.712478in}}{\pgfqpoint{1.703104in}{2.708087in}}{\pgfqpoint{1.695290in}{2.700274in}}%
\pgfpathcurveto{\pgfqpoint{1.687477in}{2.692460in}}{\pgfqpoint{1.683087in}{2.681861in}}{\pgfqpoint{1.683087in}{2.670811in}}%
\pgfpathcurveto{\pgfqpoint{1.683087in}{2.659761in}}{\pgfqpoint{1.687477in}{2.649162in}}{\pgfqpoint{1.695290in}{2.641348in}}%
\pgfpathcurveto{\pgfqpoint{1.703104in}{2.633534in}}{\pgfqpoint{1.713703in}{2.629144in}}{\pgfqpoint{1.724753in}{2.629144in}}%
\pgfpathlineto{\pgfqpoint{1.724753in}{2.629144in}}%
\pgfpathclose%
\pgfusepath{stroke,fill}%
\end{pgfscope}%
\begin{pgfscope}%
\pgfpathrectangle{\pgfqpoint{0.633874in}{0.569136in}}{\pgfqpoint{2.177280in}{2.201755in}}%
\pgfusepath{clip}%
\pgfsetbuttcap%
\pgfsetroundjoin%
\definecolor{currentfill}{rgb}{0.172549,0.627451,0.172549}%
\pgfsetfillcolor{currentfill}%
\pgfsetlinewidth{0.481800pt}%
\definecolor{currentstroke}{rgb}{1.000000,1.000000,1.000000}%
\pgfsetstrokecolor{currentstroke}%
\pgfsetdash{}{0pt}%
\pgfpathmoveto{\pgfqpoint{1.525060in}{2.128745in}}%
\pgfpathcurveto{\pgfqpoint{1.536110in}{2.128745in}}{\pgfqpoint{1.546709in}{2.133136in}}{\pgfqpoint{1.554523in}{2.140949in}}%
\pgfpathcurveto{\pgfqpoint{1.562336in}{2.148763in}}{\pgfqpoint{1.566726in}{2.159362in}}{\pgfqpoint{1.566726in}{2.170412in}}%
\pgfpathcurveto{\pgfqpoint{1.566726in}{2.181462in}}{\pgfqpoint{1.562336in}{2.192061in}}{\pgfqpoint{1.554523in}{2.199875in}}%
\pgfpathcurveto{\pgfqpoint{1.546709in}{2.207688in}}{\pgfqpoint{1.536110in}{2.212079in}}{\pgfqpoint{1.525060in}{2.212079in}}%
\pgfpathcurveto{\pgfqpoint{1.514010in}{2.212079in}}{\pgfqpoint{1.503411in}{2.207688in}}{\pgfqpoint{1.495597in}{2.199875in}}%
\pgfpathcurveto{\pgfqpoint{1.487783in}{2.192061in}}{\pgfqpoint{1.483393in}{2.181462in}}{\pgfqpoint{1.483393in}{2.170412in}}%
\pgfpathcurveto{\pgfqpoint{1.483393in}{2.159362in}}{\pgfqpoint{1.487783in}{2.148763in}}{\pgfqpoint{1.495597in}{2.140949in}}%
\pgfpathcurveto{\pgfqpoint{1.503411in}{2.133136in}}{\pgfqpoint{1.514010in}{2.128745in}}{\pgfqpoint{1.525060in}{2.128745in}}%
\pgfpathlineto{\pgfqpoint{1.525060in}{2.128745in}}%
\pgfpathclose%
\pgfusepath{stroke,fill}%
\end{pgfscope}%
\begin{pgfscope}%
\pgfpathrectangle{\pgfqpoint{0.633874in}{0.569136in}}{\pgfqpoint{2.177280in}{2.201755in}}%
\pgfusepath{clip}%
\pgfsetbuttcap%
\pgfsetroundjoin%
\definecolor{currentfill}{rgb}{0.172549,0.627451,0.172549}%
\pgfsetfillcolor{currentfill}%
\pgfsetlinewidth{0.481800pt}%
\definecolor{currentstroke}{rgb}{1.000000,1.000000,1.000000}%
\pgfsetstrokecolor{currentstroke}%
\pgfsetdash{}{0pt}%
\pgfpathmoveto{\pgfqpoint{2.044263in}{2.295545in}}%
\pgfpathcurveto{\pgfqpoint{2.055313in}{2.295545in}}{\pgfqpoint{2.065912in}{2.299935in}}{\pgfqpoint{2.073726in}{2.307749in}}%
\pgfpathcurveto{\pgfqpoint{2.081539in}{2.315562in}}{\pgfqpoint{2.085930in}{2.326161in}}{\pgfqpoint{2.085930in}{2.337212in}}%
\pgfpathcurveto{\pgfqpoint{2.085930in}{2.348262in}}{\pgfqpoint{2.081539in}{2.358861in}}{\pgfqpoint{2.073726in}{2.366674in}}%
\pgfpathcurveto{\pgfqpoint{2.065912in}{2.374488in}}{\pgfqpoint{2.055313in}{2.378878in}}{\pgfqpoint{2.044263in}{2.378878in}}%
\pgfpathcurveto{\pgfqpoint{2.033213in}{2.378878in}}{\pgfqpoint{2.022614in}{2.374488in}}{\pgfqpoint{2.014800in}{2.366674in}}%
\pgfpathcurveto{\pgfqpoint{2.006986in}{2.358861in}}{\pgfqpoint{2.002596in}{2.348262in}}{\pgfqpoint{2.002596in}{2.337212in}}%
\pgfpathcurveto{\pgfqpoint{2.002596in}{2.326161in}}{\pgfqpoint{2.006986in}{2.315562in}}{\pgfqpoint{2.014800in}{2.307749in}}%
\pgfpathcurveto{\pgfqpoint{2.022614in}{2.299935in}}{\pgfqpoint{2.033213in}{2.295545in}}{\pgfqpoint{2.044263in}{2.295545in}}%
\pgfpathlineto{\pgfqpoint{2.044263in}{2.295545in}}%
\pgfpathclose%
\pgfusepath{stroke,fill}%
\end{pgfscope}%
\begin{pgfscope}%
\pgfpathrectangle{\pgfqpoint{0.633874in}{0.569136in}}{\pgfqpoint{2.177280in}{2.201755in}}%
\pgfusepath{clip}%
\pgfsetbuttcap%
\pgfsetroundjoin%
\definecolor{currentfill}{rgb}{0.172549,0.627451,0.172549}%
\pgfsetfillcolor{currentfill}%
\pgfsetlinewidth{0.481800pt}%
\definecolor{currentstroke}{rgb}{1.000000,1.000000,1.000000}%
\pgfsetstrokecolor{currentstroke}%
\pgfsetdash{}{0pt}%
\pgfpathmoveto{\pgfqpoint{1.724753in}{2.045346in}}%
\pgfpathcurveto{\pgfqpoint{1.735803in}{2.045346in}}{\pgfqpoint{1.746402in}{2.049736in}}{\pgfqpoint{1.754216in}{2.057549in}}%
\pgfpathcurveto{\pgfqpoint{1.762030in}{2.065363in}}{\pgfqpoint{1.766420in}{2.075962in}}{\pgfqpoint{1.766420in}{2.087012in}}%
\pgfpathcurveto{\pgfqpoint{1.766420in}{2.098062in}}{\pgfqpoint{1.762030in}{2.108661in}}{\pgfqpoint{1.754216in}{2.116475in}}%
\pgfpathcurveto{\pgfqpoint{1.746402in}{2.124289in}}{\pgfqpoint{1.735803in}{2.128679in}}{\pgfqpoint{1.724753in}{2.128679in}}%
\pgfpathcurveto{\pgfqpoint{1.713703in}{2.128679in}}{\pgfqpoint{1.703104in}{2.124289in}}{\pgfqpoint{1.695290in}{2.116475in}}%
\pgfpathcurveto{\pgfqpoint{1.687477in}{2.108661in}}{\pgfqpoint{1.683087in}{2.098062in}}{\pgfqpoint{1.683087in}{2.087012in}}%
\pgfpathcurveto{\pgfqpoint{1.683087in}{2.075962in}}{\pgfqpoint{1.687477in}{2.065363in}}{\pgfqpoint{1.695290in}{2.057549in}}%
\pgfpathcurveto{\pgfqpoint{1.703104in}{2.049736in}}{\pgfqpoint{1.713703in}{2.045346in}}{\pgfqpoint{1.724753in}{2.045346in}}%
\pgfpathlineto{\pgfqpoint{1.724753in}{2.045346in}}%
\pgfpathclose%
\pgfusepath{stroke,fill}%
\end{pgfscope}%
\begin{pgfscope}%
\pgfpathrectangle{\pgfqpoint{0.633874in}{0.569136in}}{\pgfqpoint{2.177280in}{2.201755in}}%
\pgfusepath{clip}%
\pgfsetbuttcap%
\pgfsetroundjoin%
\definecolor{currentfill}{rgb}{0.172549,0.627451,0.172549}%
\pgfsetfillcolor{currentfill}%
\pgfsetlinewidth{0.481800pt}%
\definecolor{currentstroke}{rgb}{1.000000,1.000000,1.000000}%
\pgfsetstrokecolor{currentstroke}%
\pgfsetdash{}{0pt}%
\pgfpathmoveto{\pgfqpoint{1.804631in}{2.378945in}}%
\pgfpathcurveto{\pgfqpoint{1.815681in}{2.378945in}}{\pgfqpoint{1.826280in}{2.383335in}}{\pgfqpoint{1.834093in}{2.391149in}}%
\pgfpathcurveto{\pgfqpoint{1.841907in}{2.398962in}}{\pgfqpoint{1.846297in}{2.409561in}}{\pgfqpoint{1.846297in}{2.420611in}}%
\pgfpathcurveto{\pgfqpoint{1.846297in}{2.431662in}}{\pgfqpoint{1.841907in}{2.442261in}}{\pgfqpoint{1.834093in}{2.450074in}}%
\pgfpathcurveto{\pgfqpoint{1.826280in}{2.457888in}}{\pgfqpoint{1.815681in}{2.462278in}}{\pgfqpoint{1.804631in}{2.462278in}}%
\pgfpathcurveto{\pgfqpoint{1.793581in}{2.462278in}}{\pgfqpoint{1.782981in}{2.457888in}}{\pgfqpoint{1.775168in}{2.450074in}}%
\pgfpathcurveto{\pgfqpoint{1.767354in}{2.442261in}}{\pgfqpoint{1.762964in}{2.431662in}}{\pgfqpoint{1.762964in}{2.420611in}}%
\pgfpathcurveto{\pgfqpoint{1.762964in}{2.409561in}}{\pgfqpoint{1.767354in}{2.398962in}}{\pgfqpoint{1.775168in}{2.391149in}}%
\pgfpathcurveto{\pgfqpoint{1.782981in}{2.383335in}}{\pgfqpoint{1.793581in}{2.378945in}}{\pgfqpoint{1.804631in}{2.378945in}}%
\pgfpathlineto{\pgfqpoint{1.804631in}{2.378945in}}%
\pgfpathclose%
\pgfusepath{stroke,fill}%
\end{pgfscope}%
\begin{pgfscope}%
\pgfpathrectangle{\pgfqpoint{0.633874in}{0.569136in}}{\pgfqpoint{2.177280in}{2.201755in}}%
\pgfusepath{clip}%
\pgfsetbuttcap%
\pgfsetroundjoin%
\definecolor{currentfill}{rgb}{0.172549,0.627451,0.172549}%
\pgfsetfillcolor{currentfill}%
\pgfsetlinewidth{0.481800pt}%
\definecolor{currentstroke}{rgb}{1.000000,1.000000,1.000000}%
\pgfsetstrokecolor{currentstroke}%
\pgfsetdash{}{0pt}%
\pgfpathmoveto{\pgfqpoint{2.243956in}{2.295545in}}%
\pgfpathcurveto{\pgfqpoint{2.255007in}{2.295545in}}{\pgfqpoint{2.265606in}{2.299935in}}{\pgfqpoint{2.273419in}{2.307749in}}%
\pgfpathcurveto{\pgfqpoint{2.281233in}{2.315562in}}{\pgfqpoint{2.285623in}{2.326161in}}{\pgfqpoint{2.285623in}{2.337212in}}%
\pgfpathcurveto{\pgfqpoint{2.285623in}{2.348262in}}{\pgfqpoint{2.281233in}{2.358861in}}{\pgfqpoint{2.273419in}{2.366674in}}%
\pgfpathcurveto{\pgfqpoint{2.265606in}{2.374488in}}{\pgfqpoint{2.255007in}{2.378878in}}{\pgfqpoint{2.243956in}{2.378878in}}%
\pgfpathcurveto{\pgfqpoint{2.232906in}{2.378878in}}{\pgfqpoint{2.222307in}{2.374488in}}{\pgfqpoint{2.214494in}{2.366674in}}%
\pgfpathcurveto{\pgfqpoint{2.206680in}{2.358861in}}{\pgfqpoint{2.202290in}{2.348262in}}{\pgfqpoint{2.202290in}{2.337212in}}%
\pgfpathcurveto{\pgfqpoint{2.202290in}{2.326161in}}{\pgfqpoint{2.206680in}{2.315562in}}{\pgfqpoint{2.214494in}{2.307749in}}%
\pgfpathcurveto{\pgfqpoint{2.222307in}{2.299935in}}{\pgfqpoint{2.232906in}{2.295545in}}{\pgfqpoint{2.243956in}{2.295545in}}%
\pgfpathlineto{\pgfqpoint{2.243956in}{2.295545in}}%
\pgfpathclose%
\pgfusepath{stroke,fill}%
\end{pgfscope}%
\begin{pgfscope}%
\pgfpathrectangle{\pgfqpoint{0.633874in}{0.569136in}}{\pgfqpoint{2.177280in}{2.201755in}}%
\pgfusepath{clip}%
\pgfsetbuttcap%
\pgfsetroundjoin%
\definecolor{currentfill}{rgb}{0.172549,0.627451,0.172549}%
\pgfsetfillcolor{currentfill}%
\pgfsetlinewidth{0.481800pt}%
\definecolor{currentstroke}{rgb}{1.000000,1.000000,1.000000}%
\pgfsetstrokecolor{currentstroke}%
\pgfsetdash{}{0pt}%
\pgfpathmoveto{\pgfqpoint{1.165611in}{1.961946in}}%
\pgfpathcurveto{\pgfqpoint{1.176662in}{1.961946in}}{\pgfqpoint{1.187261in}{1.966336in}}{\pgfqpoint{1.195074in}{1.974150in}}%
\pgfpathcurveto{\pgfqpoint{1.202888in}{1.981963in}}{\pgfqpoint{1.207278in}{1.992562in}}{\pgfqpoint{1.207278in}{2.003612in}}%
\pgfpathcurveto{\pgfqpoint{1.207278in}{2.014662in}}{\pgfqpoint{1.202888in}{2.025262in}}{\pgfqpoint{1.195074in}{2.033075in}}%
\pgfpathcurveto{\pgfqpoint{1.187261in}{2.040889in}}{\pgfqpoint{1.176662in}{2.045279in}}{\pgfqpoint{1.165611in}{2.045279in}}%
\pgfpathcurveto{\pgfqpoint{1.154561in}{2.045279in}}{\pgfqpoint{1.143962in}{2.040889in}}{\pgfqpoint{1.136149in}{2.033075in}}%
\pgfpathcurveto{\pgfqpoint{1.128335in}{2.025262in}}{\pgfqpoint{1.123945in}{2.014662in}}{\pgfqpoint{1.123945in}{2.003612in}}%
\pgfpathcurveto{\pgfqpoint{1.123945in}{1.992562in}}{\pgfqpoint{1.128335in}{1.981963in}}{\pgfqpoint{1.136149in}{1.974150in}}%
\pgfpathcurveto{\pgfqpoint{1.143962in}{1.966336in}}{\pgfqpoint{1.154561in}{1.961946in}}{\pgfqpoint{1.165611in}{1.961946in}}%
\pgfpathlineto{\pgfqpoint{1.165611in}{1.961946in}}%
\pgfpathclose%
\pgfusepath{stroke,fill}%
\end{pgfscope}%
\begin{pgfscope}%
\pgfpathrectangle{\pgfqpoint{0.633874in}{0.569136in}}{\pgfqpoint{2.177280in}{2.201755in}}%
\pgfusepath{clip}%
\pgfsetbuttcap%
\pgfsetroundjoin%
\definecolor{currentfill}{rgb}{0.172549,0.627451,0.172549}%
\pgfsetfillcolor{currentfill}%
\pgfsetlinewidth{0.481800pt}%
\definecolor{currentstroke}{rgb}{1.000000,1.000000,1.000000}%
\pgfsetstrokecolor{currentstroke}%
\pgfsetdash{}{0pt}%
\pgfpathmoveto{\pgfqpoint{2.124140in}{2.045346in}}%
\pgfpathcurveto{\pgfqpoint{2.135190in}{2.045346in}}{\pgfqpoint{2.145789in}{2.049736in}}{\pgfqpoint{2.153603in}{2.057549in}}%
\pgfpathcurveto{\pgfqpoint{2.161417in}{2.065363in}}{\pgfqpoint{2.165807in}{2.075962in}}{\pgfqpoint{2.165807in}{2.087012in}}%
\pgfpathcurveto{\pgfqpoint{2.165807in}{2.098062in}}{\pgfqpoint{2.161417in}{2.108661in}}{\pgfqpoint{2.153603in}{2.116475in}}%
\pgfpathcurveto{\pgfqpoint{2.145789in}{2.124289in}}{\pgfqpoint{2.135190in}{2.128679in}}{\pgfqpoint{2.124140in}{2.128679in}}%
\pgfpathcurveto{\pgfqpoint{2.113090in}{2.128679in}}{\pgfqpoint{2.102491in}{2.124289in}}{\pgfqpoint{2.094677in}{2.116475in}}%
\pgfpathcurveto{\pgfqpoint{2.086864in}{2.108661in}}{\pgfqpoint{2.082474in}{2.098062in}}{\pgfqpoint{2.082474in}{2.087012in}}%
\pgfpathcurveto{\pgfqpoint{2.082474in}{2.075962in}}{\pgfqpoint{2.086864in}{2.065363in}}{\pgfqpoint{2.094677in}{2.057549in}}%
\pgfpathcurveto{\pgfqpoint{2.102491in}{2.049736in}}{\pgfqpoint{2.113090in}{2.045346in}}{\pgfqpoint{2.124140in}{2.045346in}}%
\pgfpathlineto{\pgfqpoint{2.124140in}{2.045346in}}%
\pgfpathclose%
\pgfusepath{stroke,fill}%
\end{pgfscope}%
\begin{pgfscope}%
\pgfpathrectangle{\pgfqpoint{0.633874in}{0.569136in}}{\pgfqpoint{2.177280in}{2.201755in}}%
\pgfusepath{clip}%
\pgfsetbuttcap%
\pgfsetroundjoin%
\definecolor{currentfill}{rgb}{0.172549,0.627451,0.172549}%
\pgfsetfillcolor{currentfill}%
\pgfsetlinewidth{0.481800pt}%
\definecolor{currentstroke}{rgb}{1.000000,1.000000,1.000000}%
\pgfsetstrokecolor{currentstroke}%
\pgfsetdash{}{0pt}%
\pgfpathmoveto{\pgfqpoint{1.884508in}{2.045346in}}%
\pgfpathcurveto{\pgfqpoint{1.895558in}{2.045346in}}{\pgfqpoint{1.906157in}{2.049736in}}{\pgfqpoint{1.913971in}{2.057549in}}%
\pgfpathcurveto{\pgfqpoint{1.921784in}{2.065363in}}{\pgfqpoint{1.926175in}{2.075962in}}{\pgfqpoint{1.926175in}{2.087012in}}%
\pgfpathcurveto{\pgfqpoint{1.926175in}{2.098062in}}{\pgfqpoint{1.921784in}{2.108661in}}{\pgfqpoint{1.913971in}{2.116475in}}%
\pgfpathcurveto{\pgfqpoint{1.906157in}{2.124289in}}{\pgfqpoint{1.895558in}{2.128679in}}{\pgfqpoint{1.884508in}{2.128679in}}%
\pgfpathcurveto{\pgfqpoint{1.873458in}{2.128679in}}{\pgfqpoint{1.862859in}{2.124289in}}{\pgfqpoint{1.855045in}{2.116475in}}%
\pgfpathcurveto{\pgfqpoint{1.847232in}{2.108661in}}{\pgfqpoint{1.842841in}{2.098062in}}{\pgfqpoint{1.842841in}{2.087012in}}%
\pgfpathcurveto{\pgfqpoint{1.842841in}{2.075962in}}{\pgfqpoint{1.847232in}{2.065363in}}{\pgfqpoint{1.855045in}{2.057549in}}%
\pgfpathcurveto{\pgfqpoint{1.862859in}{2.049736in}}{\pgfqpoint{1.873458in}{2.045346in}}{\pgfqpoint{1.884508in}{2.045346in}}%
\pgfpathlineto{\pgfqpoint{1.884508in}{2.045346in}}%
\pgfpathclose%
\pgfusepath{stroke,fill}%
\end{pgfscope}%
\begin{pgfscope}%
\pgfpathrectangle{\pgfqpoint{0.633874in}{0.569136in}}{\pgfqpoint{2.177280in}{2.201755in}}%
\pgfusepath{clip}%
\pgfsetbuttcap%
\pgfsetroundjoin%
\definecolor{currentfill}{rgb}{0.172549,0.627451,0.172549}%
\pgfsetfillcolor{currentfill}%
\pgfsetlinewidth{0.481800pt}%
\definecolor{currentstroke}{rgb}{1.000000,1.000000,1.000000}%
\pgfsetstrokecolor{currentstroke}%
\pgfsetdash{}{0pt}%
\pgfpathmoveto{\pgfqpoint{2.084202in}{2.629144in}}%
\pgfpathcurveto{\pgfqpoint{2.095252in}{2.629144in}}{\pgfqpoint{2.105851in}{2.633534in}}{\pgfqpoint{2.113664in}{2.641348in}}%
\pgfpathcurveto{\pgfqpoint{2.121478in}{2.649162in}}{\pgfqpoint{2.125868in}{2.659761in}}{\pgfqpoint{2.125868in}{2.670811in}}%
\pgfpathcurveto{\pgfqpoint{2.125868in}{2.681861in}}{\pgfqpoint{2.121478in}{2.692460in}}{\pgfqpoint{2.113664in}{2.700274in}}%
\pgfpathcurveto{\pgfqpoint{2.105851in}{2.708087in}}{\pgfqpoint{2.095252in}{2.712478in}}{\pgfqpoint{2.084202in}{2.712478in}}%
\pgfpathcurveto{\pgfqpoint{2.073151in}{2.712478in}}{\pgfqpoint{2.062552in}{2.708087in}}{\pgfqpoint{2.054739in}{2.700274in}}%
\pgfpathcurveto{\pgfqpoint{2.046925in}{2.692460in}}{\pgfqpoint{2.042535in}{2.681861in}}{\pgfqpoint{2.042535in}{2.670811in}}%
\pgfpathcurveto{\pgfqpoint{2.042535in}{2.659761in}}{\pgfqpoint{2.046925in}{2.649162in}}{\pgfqpoint{2.054739in}{2.641348in}}%
\pgfpathcurveto{\pgfqpoint{2.062552in}{2.633534in}}{\pgfqpoint{2.073151in}{2.629144in}}{\pgfqpoint{2.084202in}{2.629144in}}%
\pgfpathlineto{\pgfqpoint{2.084202in}{2.629144in}}%
\pgfpathclose%
\pgfusepath{stroke,fill}%
\end{pgfscope}%
\begin{pgfscope}%
\pgfpathrectangle{\pgfqpoint{0.633874in}{0.569136in}}{\pgfqpoint{2.177280in}{2.201755in}}%
\pgfusepath{clip}%
\pgfsetbuttcap%
\pgfsetroundjoin%
\definecolor{currentfill}{rgb}{0.172549,0.627451,0.172549}%
\pgfsetfillcolor{currentfill}%
\pgfsetlinewidth{0.481800pt}%
\definecolor{currentstroke}{rgb}{1.000000,1.000000,1.000000}%
\pgfsetstrokecolor{currentstroke}%
\pgfsetdash{}{0pt}%
\pgfpathmoveto{\pgfqpoint{1.804631in}{2.212145in}}%
\pgfpathcurveto{\pgfqpoint{1.815681in}{2.212145in}}{\pgfqpoint{1.826280in}{2.216535in}}{\pgfqpoint{1.834093in}{2.224349in}}%
\pgfpathcurveto{\pgfqpoint{1.841907in}{2.232163in}}{\pgfqpoint{1.846297in}{2.242762in}}{\pgfqpoint{1.846297in}{2.253812in}}%
\pgfpathcurveto{\pgfqpoint{1.846297in}{2.264862in}}{\pgfqpoint{1.841907in}{2.275461in}}{\pgfqpoint{1.834093in}{2.283275in}}%
\pgfpathcurveto{\pgfqpoint{1.826280in}{2.291088in}}{\pgfqpoint{1.815681in}{2.295478in}}{\pgfqpoint{1.804631in}{2.295478in}}%
\pgfpathcurveto{\pgfqpoint{1.793581in}{2.295478in}}{\pgfqpoint{1.782981in}{2.291088in}}{\pgfqpoint{1.775168in}{2.283275in}}%
\pgfpathcurveto{\pgfqpoint{1.767354in}{2.275461in}}{\pgfqpoint{1.762964in}{2.264862in}}{\pgfqpoint{1.762964in}{2.253812in}}%
\pgfpathcurveto{\pgfqpoint{1.762964in}{2.242762in}}{\pgfqpoint{1.767354in}{2.232163in}}{\pgfqpoint{1.775168in}{2.224349in}}%
\pgfpathcurveto{\pgfqpoint{1.782981in}{2.216535in}}{\pgfqpoint{1.793581in}{2.212145in}}{\pgfqpoint{1.804631in}{2.212145in}}%
\pgfpathlineto{\pgfqpoint{1.804631in}{2.212145in}}%
\pgfpathclose%
\pgfusepath{stroke,fill}%
\end{pgfscope}%
\begin{pgfscope}%
\pgfpathrectangle{\pgfqpoint{0.633874in}{0.569136in}}{\pgfqpoint{2.177280in}{2.201755in}}%
\pgfusepath{clip}%
\pgfsetbuttcap%
\pgfsetroundjoin%
\definecolor{currentfill}{rgb}{0.172549,0.627451,0.172549}%
\pgfsetfillcolor{currentfill}%
\pgfsetlinewidth{0.481800pt}%
\definecolor{currentstroke}{rgb}{1.000000,1.000000,1.000000}%
\pgfsetstrokecolor{currentstroke}%
\pgfsetdash{}{0pt}%
\pgfpathmoveto{\pgfqpoint{1.764692in}{2.128745in}}%
\pgfpathcurveto{\pgfqpoint{1.775742in}{2.128745in}}{\pgfqpoint{1.786341in}{2.133136in}}{\pgfqpoint{1.794155in}{2.140949in}}%
\pgfpathcurveto{\pgfqpoint{1.801968in}{2.148763in}}{\pgfqpoint{1.806359in}{2.159362in}}{\pgfqpoint{1.806359in}{2.170412in}}%
\pgfpathcurveto{\pgfqpoint{1.806359in}{2.181462in}}{\pgfqpoint{1.801968in}{2.192061in}}{\pgfqpoint{1.794155in}{2.199875in}}%
\pgfpathcurveto{\pgfqpoint{1.786341in}{2.207688in}}{\pgfqpoint{1.775742in}{2.212079in}}{\pgfqpoint{1.764692in}{2.212079in}}%
\pgfpathcurveto{\pgfqpoint{1.753642in}{2.212079in}}{\pgfqpoint{1.743043in}{2.207688in}}{\pgfqpoint{1.735229in}{2.199875in}}%
\pgfpathcurveto{\pgfqpoint{1.727416in}{2.192061in}}{\pgfqpoint{1.723025in}{2.181462in}}{\pgfqpoint{1.723025in}{2.170412in}}%
\pgfpathcurveto{\pgfqpoint{1.723025in}{2.159362in}}{\pgfqpoint{1.727416in}{2.148763in}}{\pgfqpoint{1.735229in}{2.140949in}}%
\pgfpathcurveto{\pgfqpoint{1.743043in}{2.133136in}}{\pgfqpoint{1.753642in}{2.128745in}}{\pgfqpoint{1.764692in}{2.128745in}}%
\pgfpathlineto{\pgfqpoint{1.764692in}{2.128745in}}%
\pgfpathclose%
\pgfusepath{stroke,fill}%
\end{pgfscope}%
\begin{pgfscope}%
\pgfpathrectangle{\pgfqpoint{0.633874in}{0.569136in}}{\pgfqpoint{2.177280in}{2.201755in}}%
\pgfusepath{clip}%
\pgfsetbuttcap%
\pgfsetroundjoin%
\definecolor{currentfill}{rgb}{0.172549,0.627451,0.172549}%
\pgfsetfillcolor{currentfill}%
\pgfsetlinewidth{0.481800pt}%
\definecolor{currentstroke}{rgb}{1.000000,1.000000,1.000000}%
\pgfsetstrokecolor{currentstroke}%
\pgfsetdash{}{0pt}%
\pgfpathmoveto{\pgfqpoint{1.924447in}{2.295545in}}%
\pgfpathcurveto{\pgfqpoint{1.935497in}{2.295545in}}{\pgfqpoint{1.946096in}{2.299935in}}{\pgfqpoint{1.953910in}{2.307749in}}%
\pgfpathcurveto{\pgfqpoint{1.961723in}{2.315562in}}{\pgfqpoint{1.966113in}{2.326161in}}{\pgfqpoint{1.966113in}{2.337212in}}%
\pgfpathcurveto{\pgfqpoint{1.966113in}{2.348262in}}{\pgfqpoint{1.961723in}{2.358861in}}{\pgfqpoint{1.953910in}{2.366674in}}%
\pgfpathcurveto{\pgfqpoint{1.946096in}{2.374488in}}{\pgfqpoint{1.935497in}{2.378878in}}{\pgfqpoint{1.924447in}{2.378878in}}%
\pgfpathcurveto{\pgfqpoint{1.913397in}{2.378878in}}{\pgfqpoint{1.902798in}{2.374488in}}{\pgfqpoint{1.894984in}{2.366674in}}%
\pgfpathcurveto{\pgfqpoint{1.887170in}{2.358861in}}{\pgfqpoint{1.882780in}{2.348262in}}{\pgfqpoint{1.882780in}{2.337212in}}%
\pgfpathcurveto{\pgfqpoint{1.882780in}{2.326161in}}{\pgfqpoint{1.887170in}{2.315562in}}{\pgfqpoint{1.894984in}{2.307749in}}%
\pgfpathcurveto{\pgfqpoint{1.902798in}{2.299935in}}{\pgfqpoint{1.913397in}{2.295545in}}{\pgfqpoint{1.924447in}{2.295545in}}%
\pgfpathlineto{\pgfqpoint{1.924447in}{2.295545in}}%
\pgfpathclose%
\pgfusepath{stroke,fill}%
\end{pgfscope}%
\begin{pgfscope}%
\pgfpathrectangle{\pgfqpoint{0.633874in}{0.569136in}}{\pgfqpoint{2.177280in}{2.201755in}}%
\pgfusepath{clip}%
\pgfsetbuttcap%
\pgfsetroundjoin%
\definecolor{currentfill}{rgb}{0.172549,0.627451,0.172549}%
\pgfsetfillcolor{currentfill}%
\pgfsetlinewidth{0.481800pt}%
\definecolor{currentstroke}{rgb}{1.000000,1.000000,1.000000}%
\pgfsetstrokecolor{currentstroke}%
\pgfsetdash{}{0pt}%
\pgfpathmoveto{\pgfqpoint{1.485121in}{2.212145in}}%
\pgfpathcurveto{\pgfqpoint{1.496171in}{2.212145in}}{\pgfqpoint{1.506770in}{2.216535in}}{\pgfqpoint{1.514584in}{2.224349in}}%
\pgfpathcurveto{\pgfqpoint{1.522397in}{2.232163in}}{\pgfqpoint{1.526788in}{2.242762in}}{\pgfqpoint{1.526788in}{2.253812in}}%
\pgfpathcurveto{\pgfqpoint{1.526788in}{2.264862in}}{\pgfqpoint{1.522397in}{2.275461in}}{\pgfqpoint{1.514584in}{2.283275in}}%
\pgfpathcurveto{\pgfqpoint{1.506770in}{2.291088in}}{\pgfqpoint{1.496171in}{2.295478in}}{\pgfqpoint{1.485121in}{2.295478in}}%
\pgfpathcurveto{\pgfqpoint{1.474071in}{2.295478in}}{\pgfqpoint{1.463472in}{2.291088in}}{\pgfqpoint{1.455658in}{2.283275in}}%
\pgfpathcurveto{\pgfqpoint{1.447845in}{2.275461in}}{\pgfqpoint{1.443454in}{2.264862in}}{\pgfqpoint{1.443454in}{2.253812in}}%
\pgfpathcurveto{\pgfqpoint{1.443454in}{2.242762in}}{\pgfqpoint{1.447845in}{2.232163in}}{\pgfqpoint{1.455658in}{2.224349in}}%
\pgfpathcurveto{\pgfqpoint{1.463472in}{2.216535in}}{\pgfqpoint{1.474071in}{2.212145in}}{\pgfqpoint{1.485121in}{2.212145in}}%
\pgfpathlineto{\pgfqpoint{1.485121in}{2.212145in}}%
\pgfpathclose%
\pgfusepath{stroke,fill}%
\end{pgfscope}%
\begin{pgfscope}%
\pgfpathrectangle{\pgfqpoint{0.633874in}{0.569136in}}{\pgfqpoint{2.177280in}{2.201755in}}%
\pgfusepath{clip}%
\pgfsetbuttcap%
\pgfsetroundjoin%
\definecolor{currentfill}{rgb}{0.172549,0.627451,0.172549}%
\pgfsetfillcolor{currentfill}%
\pgfsetlinewidth{0.481800pt}%
\definecolor{currentstroke}{rgb}{1.000000,1.000000,1.000000}%
\pgfsetstrokecolor{currentstroke}%
\pgfsetdash{}{0pt}%
\pgfpathmoveto{\pgfqpoint{1.525060in}{2.545744in}}%
\pgfpathcurveto{\pgfqpoint{1.536110in}{2.545744in}}{\pgfqpoint{1.546709in}{2.550135in}}{\pgfqpoint{1.554523in}{2.557948in}}%
\pgfpathcurveto{\pgfqpoint{1.562336in}{2.565762in}}{\pgfqpoint{1.566726in}{2.576361in}}{\pgfqpoint{1.566726in}{2.587411in}}%
\pgfpathcurveto{\pgfqpoint{1.566726in}{2.598461in}}{\pgfqpoint{1.562336in}{2.609060in}}{\pgfqpoint{1.554523in}{2.616874in}}%
\pgfpathcurveto{\pgfqpoint{1.546709in}{2.624687in}}{\pgfqpoint{1.536110in}{2.629078in}}{\pgfqpoint{1.525060in}{2.629078in}}%
\pgfpathcurveto{\pgfqpoint{1.514010in}{2.629078in}}{\pgfqpoint{1.503411in}{2.624687in}}{\pgfqpoint{1.495597in}{2.616874in}}%
\pgfpathcurveto{\pgfqpoint{1.487783in}{2.609060in}}{\pgfqpoint{1.483393in}{2.598461in}}{\pgfqpoint{1.483393in}{2.587411in}}%
\pgfpathcurveto{\pgfqpoint{1.483393in}{2.576361in}}{\pgfqpoint{1.487783in}{2.565762in}}{\pgfqpoint{1.495597in}{2.557948in}}%
\pgfpathcurveto{\pgfqpoint{1.503411in}{2.550135in}}{\pgfqpoint{1.514010in}{2.545744in}}{\pgfqpoint{1.525060in}{2.545744in}}%
\pgfpathlineto{\pgfqpoint{1.525060in}{2.545744in}}%
\pgfpathclose%
\pgfusepath{stroke,fill}%
\end{pgfscope}%
\begin{pgfscope}%
\pgfpathrectangle{\pgfqpoint{0.633874in}{0.569136in}}{\pgfqpoint{2.177280in}{2.201755in}}%
\pgfusepath{clip}%
\pgfsetbuttcap%
\pgfsetroundjoin%
\definecolor{currentfill}{rgb}{0.172549,0.627451,0.172549}%
\pgfsetfillcolor{currentfill}%
\pgfsetlinewidth{0.481800pt}%
\definecolor{currentstroke}{rgb}{1.000000,1.000000,1.000000}%
\pgfsetstrokecolor{currentstroke}%
\pgfsetdash{}{0pt}%
\pgfpathmoveto{\pgfqpoint{1.764692in}{2.462345in}}%
\pgfpathcurveto{\pgfqpoint{1.775742in}{2.462345in}}{\pgfqpoint{1.786341in}{2.466735in}}{\pgfqpoint{1.794155in}{2.474548in}}%
\pgfpathcurveto{\pgfqpoint{1.801968in}{2.482362in}}{\pgfqpoint{1.806359in}{2.492961in}}{\pgfqpoint{1.806359in}{2.504011in}}%
\pgfpathcurveto{\pgfqpoint{1.806359in}{2.515061in}}{\pgfqpoint{1.801968in}{2.525660in}}{\pgfqpoint{1.794155in}{2.533474in}}%
\pgfpathcurveto{\pgfqpoint{1.786341in}{2.541288in}}{\pgfqpoint{1.775742in}{2.545678in}}{\pgfqpoint{1.764692in}{2.545678in}}%
\pgfpathcurveto{\pgfqpoint{1.753642in}{2.545678in}}{\pgfqpoint{1.743043in}{2.541288in}}{\pgfqpoint{1.735229in}{2.533474in}}%
\pgfpathcurveto{\pgfqpoint{1.727416in}{2.525660in}}{\pgfqpoint{1.723025in}{2.515061in}}{\pgfqpoint{1.723025in}{2.504011in}}%
\pgfpathcurveto{\pgfqpoint{1.723025in}{2.492961in}}{\pgfqpoint{1.727416in}{2.482362in}}{\pgfqpoint{1.735229in}{2.474548in}}%
\pgfpathcurveto{\pgfqpoint{1.743043in}{2.466735in}}{\pgfqpoint{1.753642in}{2.462345in}}{\pgfqpoint{1.764692in}{2.462345in}}%
\pgfpathlineto{\pgfqpoint{1.764692in}{2.462345in}}%
\pgfpathclose%
\pgfusepath{stroke,fill}%
\end{pgfscope}%
\begin{pgfscope}%
\pgfpathrectangle{\pgfqpoint{0.633874in}{0.569136in}}{\pgfqpoint{2.177280in}{2.201755in}}%
\pgfusepath{clip}%
\pgfsetbuttcap%
\pgfsetroundjoin%
\definecolor{currentfill}{rgb}{0.172549,0.627451,0.172549}%
\pgfsetfillcolor{currentfill}%
\pgfsetlinewidth{0.481800pt}%
\definecolor{currentstroke}{rgb}{1.000000,1.000000,1.000000}%
\pgfsetstrokecolor{currentstroke}%
\pgfsetdash{}{0pt}%
\pgfpathmoveto{\pgfqpoint{1.804631in}{2.045346in}}%
\pgfpathcurveto{\pgfqpoint{1.815681in}{2.045346in}}{\pgfqpoint{1.826280in}{2.049736in}}{\pgfqpoint{1.834093in}{2.057549in}}%
\pgfpathcurveto{\pgfqpoint{1.841907in}{2.065363in}}{\pgfqpoint{1.846297in}{2.075962in}}{\pgfqpoint{1.846297in}{2.087012in}}%
\pgfpathcurveto{\pgfqpoint{1.846297in}{2.098062in}}{\pgfqpoint{1.841907in}{2.108661in}}{\pgfqpoint{1.834093in}{2.116475in}}%
\pgfpathcurveto{\pgfqpoint{1.826280in}{2.124289in}}{\pgfqpoint{1.815681in}{2.128679in}}{\pgfqpoint{1.804631in}{2.128679in}}%
\pgfpathcurveto{\pgfqpoint{1.793581in}{2.128679in}}{\pgfqpoint{1.782981in}{2.124289in}}{\pgfqpoint{1.775168in}{2.116475in}}%
\pgfpathcurveto{\pgfqpoint{1.767354in}{2.108661in}}{\pgfqpoint{1.762964in}{2.098062in}}{\pgfqpoint{1.762964in}{2.087012in}}%
\pgfpathcurveto{\pgfqpoint{1.762964in}{2.075962in}}{\pgfqpoint{1.767354in}{2.065363in}}{\pgfqpoint{1.775168in}{2.057549in}}%
\pgfpathcurveto{\pgfqpoint{1.782981in}{2.049736in}}{\pgfqpoint{1.793581in}{2.045346in}}{\pgfqpoint{1.804631in}{2.045346in}}%
\pgfpathlineto{\pgfqpoint{1.804631in}{2.045346in}}%
\pgfpathclose%
\pgfusepath{stroke,fill}%
\end{pgfscope}%
\begin{pgfscope}%
\pgfpathrectangle{\pgfqpoint{0.633874in}{0.569136in}}{\pgfqpoint{2.177280in}{2.201755in}}%
\pgfusepath{clip}%
\pgfsetbuttcap%
\pgfsetroundjoin%
\definecolor{currentfill}{rgb}{0.172549,0.627451,0.172549}%
\pgfsetfillcolor{currentfill}%
\pgfsetlinewidth{0.481800pt}%
\definecolor{currentstroke}{rgb}{1.000000,1.000000,1.000000}%
\pgfsetstrokecolor{currentstroke}%
\pgfsetdash{}{0pt}%
\pgfpathmoveto{\pgfqpoint{2.283895in}{2.378945in}}%
\pgfpathcurveto{\pgfqpoint{2.294945in}{2.378945in}}{\pgfqpoint{2.305544in}{2.383335in}}{\pgfqpoint{2.313358in}{2.391149in}}%
\pgfpathcurveto{\pgfqpoint{2.321171in}{2.398962in}}{\pgfqpoint{2.325562in}{2.409561in}}{\pgfqpoint{2.325562in}{2.420611in}}%
\pgfpathcurveto{\pgfqpoint{2.325562in}{2.431662in}}{\pgfqpoint{2.321171in}{2.442261in}}{\pgfqpoint{2.313358in}{2.450074in}}%
\pgfpathcurveto{\pgfqpoint{2.305544in}{2.457888in}}{\pgfqpoint{2.294945in}{2.462278in}}{\pgfqpoint{2.283895in}{2.462278in}}%
\pgfpathcurveto{\pgfqpoint{2.272845in}{2.462278in}}{\pgfqpoint{2.262246in}{2.457888in}}{\pgfqpoint{2.254432in}{2.450074in}}%
\pgfpathcurveto{\pgfqpoint{2.246619in}{2.442261in}}{\pgfqpoint{2.242228in}{2.431662in}}{\pgfqpoint{2.242228in}{2.420611in}}%
\pgfpathcurveto{\pgfqpoint{2.242228in}{2.409561in}}{\pgfqpoint{2.246619in}{2.398962in}}{\pgfqpoint{2.254432in}{2.391149in}}%
\pgfpathcurveto{\pgfqpoint{2.262246in}{2.383335in}}{\pgfqpoint{2.272845in}{2.378945in}}{\pgfqpoint{2.283895in}{2.378945in}}%
\pgfpathlineto{\pgfqpoint{2.283895in}{2.378945in}}%
\pgfpathclose%
\pgfusepath{stroke,fill}%
\end{pgfscope}%
\begin{pgfscope}%
\pgfpathrectangle{\pgfqpoint{0.633874in}{0.569136in}}{\pgfqpoint{2.177280in}{2.201755in}}%
\pgfusepath{clip}%
\pgfsetbuttcap%
\pgfsetroundjoin%
\definecolor{currentfill}{rgb}{0.172549,0.627451,0.172549}%
\pgfsetfillcolor{currentfill}%
\pgfsetlinewidth{0.481800pt}%
\definecolor{currentstroke}{rgb}{1.000000,1.000000,1.000000}%
\pgfsetstrokecolor{currentstroke}%
\pgfsetdash{}{0pt}%
\pgfpathmoveto{\pgfqpoint{2.283895in}{2.462345in}}%
\pgfpathcurveto{\pgfqpoint{2.294945in}{2.462345in}}{\pgfqpoint{2.305544in}{2.466735in}}{\pgfqpoint{2.313358in}{2.474548in}}%
\pgfpathcurveto{\pgfqpoint{2.321171in}{2.482362in}}{\pgfqpoint{2.325562in}{2.492961in}}{\pgfqpoint{2.325562in}{2.504011in}}%
\pgfpathcurveto{\pgfqpoint{2.325562in}{2.515061in}}{\pgfqpoint{2.321171in}{2.525660in}}{\pgfqpoint{2.313358in}{2.533474in}}%
\pgfpathcurveto{\pgfqpoint{2.305544in}{2.541288in}}{\pgfqpoint{2.294945in}{2.545678in}}{\pgfqpoint{2.283895in}{2.545678in}}%
\pgfpathcurveto{\pgfqpoint{2.272845in}{2.545678in}}{\pgfqpoint{2.262246in}{2.541288in}}{\pgfqpoint{2.254432in}{2.533474in}}%
\pgfpathcurveto{\pgfqpoint{2.246619in}{2.525660in}}{\pgfqpoint{2.242228in}{2.515061in}}{\pgfqpoint{2.242228in}{2.504011in}}%
\pgfpathcurveto{\pgfqpoint{2.242228in}{2.492961in}}{\pgfqpoint{2.246619in}{2.482362in}}{\pgfqpoint{2.254432in}{2.474548in}}%
\pgfpathcurveto{\pgfqpoint{2.262246in}{2.466735in}}{\pgfqpoint{2.272845in}{2.462345in}}{\pgfqpoint{2.283895in}{2.462345in}}%
\pgfpathlineto{\pgfqpoint{2.283895in}{2.462345in}}%
\pgfpathclose%
\pgfusepath{stroke,fill}%
\end{pgfscope}%
\begin{pgfscope}%
\pgfpathrectangle{\pgfqpoint{0.633874in}{0.569136in}}{\pgfqpoint{2.177280in}{2.201755in}}%
\pgfusepath{clip}%
\pgfsetbuttcap%
\pgfsetroundjoin%
\definecolor{currentfill}{rgb}{0.172549,0.627451,0.172549}%
\pgfsetfillcolor{currentfill}%
\pgfsetlinewidth{0.481800pt}%
\definecolor{currentstroke}{rgb}{1.000000,1.000000,1.000000}%
\pgfsetstrokecolor{currentstroke}%
\pgfsetdash{}{0pt}%
\pgfpathmoveto{\pgfqpoint{1.604937in}{1.795146in}}%
\pgfpathcurveto{\pgfqpoint{1.615987in}{1.795146in}}{\pgfqpoint{1.626586in}{1.799536in}}{\pgfqpoint{1.634400in}{1.807350in}}%
\pgfpathcurveto{\pgfqpoint{1.642214in}{1.815164in}}{\pgfqpoint{1.646604in}{1.825763in}}{\pgfqpoint{1.646604in}{1.836813in}}%
\pgfpathcurveto{\pgfqpoint{1.646604in}{1.847863in}}{\pgfqpoint{1.642214in}{1.858462in}}{\pgfqpoint{1.634400in}{1.866276in}}%
\pgfpathcurveto{\pgfqpoint{1.626586in}{1.874089in}}{\pgfqpoint{1.615987in}{1.878479in}}{\pgfqpoint{1.604937in}{1.878479in}}%
\pgfpathcurveto{\pgfqpoint{1.593887in}{1.878479in}}{\pgfqpoint{1.583288in}{1.874089in}}{\pgfqpoint{1.575474in}{1.866276in}}%
\pgfpathcurveto{\pgfqpoint{1.567661in}{1.858462in}}{\pgfqpoint{1.563270in}{1.847863in}}{\pgfqpoint{1.563270in}{1.836813in}}%
\pgfpathcurveto{\pgfqpoint{1.563270in}{1.825763in}}{\pgfqpoint{1.567661in}{1.815164in}}{\pgfqpoint{1.575474in}{1.807350in}}%
\pgfpathcurveto{\pgfqpoint{1.583288in}{1.799536in}}{\pgfqpoint{1.593887in}{1.795146in}}{\pgfqpoint{1.604937in}{1.795146in}}%
\pgfpathlineto{\pgfqpoint{1.604937in}{1.795146in}}%
\pgfpathclose%
\pgfusepath{stroke,fill}%
\end{pgfscope}%
\begin{pgfscope}%
\pgfpathrectangle{\pgfqpoint{0.633874in}{0.569136in}}{\pgfqpoint{2.177280in}{2.201755in}}%
\pgfusepath{clip}%
\pgfsetbuttcap%
\pgfsetroundjoin%
\definecolor{currentfill}{rgb}{0.172549,0.627451,0.172549}%
\pgfsetfillcolor{currentfill}%
\pgfsetlinewidth{0.481800pt}%
\definecolor{currentstroke}{rgb}{1.000000,1.000000,1.000000}%
\pgfsetstrokecolor{currentstroke}%
\pgfsetdash{}{0pt}%
\pgfpathmoveto{\pgfqpoint{1.964385in}{2.462345in}}%
\pgfpathcurveto{\pgfqpoint{1.975436in}{2.462345in}}{\pgfqpoint{1.986035in}{2.466735in}}{\pgfqpoint{1.993848in}{2.474548in}}%
\pgfpathcurveto{\pgfqpoint{2.001662in}{2.482362in}}{\pgfqpoint{2.006052in}{2.492961in}}{\pgfqpoint{2.006052in}{2.504011in}}%
\pgfpathcurveto{\pgfqpoint{2.006052in}{2.515061in}}{\pgfqpoint{2.001662in}{2.525660in}}{\pgfqpoint{1.993848in}{2.533474in}}%
\pgfpathcurveto{\pgfqpoint{1.986035in}{2.541288in}}{\pgfqpoint{1.975436in}{2.545678in}}{\pgfqpoint{1.964385in}{2.545678in}}%
\pgfpathcurveto{\pgfqpoint{1.953335in}{2.545678in}}{\pgfqpoint{1.942736in}{2.541288in}}{\pgfqpoint{1.934923in}{2.533474in}}%
\pgfpathcurveto{\pgfqpoint{1.927109in}{2.525660in}}{\pgfqpoint{1.922719in}{2.515061in}}{\pgfqpoint{1.922719in}{2.504011in}}%
\pgfpathcurveto{\pgfqpoint{1.922719in}{2.492961in}}{\pgfqpoint{1.927109in}{2.482362in}}{\pgfqpoint{1.934923in}{2.474548in}}%
\pgfpathcurveto{\pgfqpoint{1.942736in}{2.466735in}}{\pgfqpoint{1.953335in}{2.462345in}}{\pgfqpoint{1.964385in}{2.462345in}}%
\pgfpathlineto{\pgfqpoint{1.964385in}{2.462345in}}%
\pgfpathclose%
\pgfusepath{stroke,fill}%
\end{pgfscope}%
\begin{pgfscope}%
\pgfpathrectangle{\pgfqpoint{0.633874in}{0.569136in}}{\pgfqpoint{2.177280in}{2.201755in}}%
\pgfusepath{clip}%
\pgfsetbuttcap%
\pgfsetroundjoin%
\definecolor{currentfill}{rgb}{0.172549,0.627451,0.172549}%
\pgfsetfillcolor{currentfill}%
\pgfsetlinewidth{0.481800pt}%
\definecolor{currentstroke}{rgb}{1.000000,1.000000,1.000000}%
\pgfsetstrokecolor{currentstroke}%
\pgfsetdash{}{0pt}%
\pgfpathmoveto{\pgfqpoint{1.445182in}{2.212145in}}%
\pgfpathcurveto{\pgfqpoint{1.456232in}{2.212145in}}{\pgfqpoint{1.466831in}{2.216535in}}{\pgfqpoint{1.474645in}{2.224349in}}%
\pgfpathcurveto{\pgfqpoint{1.482459in}{2.232163in}}{\pgfqpoint{1.486849in}{2.242762in}}{\pgfqpoint{1.486849in}{2.253812in}}%
\pgfpathcurveto{\pgfqpoint{1.486849in}{2.264862in}}{\pgfqpoint{1.482459in}{2.275461in}}{\pgfqpoint{1.474645in}{2.283275in}}%
\pgfpathcurveto{\pgfqpoint{1.466831in}{2.291088in}}{\pgfqpoint{1.456232in}{2.295478in}}{\pgfqpoint{1.445182in}{2.295478in}}%
\pgfpathcurveto{\pgfqpoint{1.434132in}{2.295478in}}{\pgfqpoint{1.423533in}{2.291088in}}{\pgfqpoint{1.415720in}{2.283275in}}%
\pgfpathcurveto{\pgfqpoint{1.407906in}{2.275461in}}{\pgfqpoint{1.403516in}{2.264862in}}{\pgfqpoint{1.403516in}{2.253812in}}%
\pgfpathcurveto{\pgfqpoint{1.403516in}{2.242762in}}{\pgfqpoint{1.407906in}{2.232163in}}{\pgfqpoint{1.415720in}{2.224349in}}%
\pgfpathcurveto{\pgfqpoint{1.423533in}{2.216535in}}{\pgfqpoint{1.434132in}{2.212145in}}{\pgfqpoint{1.445182in}{2.212145in}}%
\pgfpathlineto{\pgfqpoint{1.445182in}{2.212145in}}%
\pgfpathclose%
\pgfusepath{stroke,fill}%
\end{pgfscope}%
\begin{pgfscope}%
\pgfpathrectangle{\pgfqpoint{0.633874in}{0.569136in}}{\pgfqpoint{2.177280in}{2.201755in}}%
\pgfusepath{clip}%
\pgfsetbuttcap%
\pgfsetroundjoin%
\definecolor{currentfill}{rgb}{0.172549,0.627451,0.172549}%
\pgfsetfillcolor{currentfill}%
\pgfsetlinewidth{0.481800pt}%
\definecolor{currentstroke}{rgb}{1.000000,1.000000,1.000000}%
\pgfsetstrokecolor{currentstroke}%
\pgfsetdash{}{0pt}%
\pgfpathmoveto{\pgfqpoint{2.283895in}{2.212145in}}%
\pgfpathcurveto{\pgfqpoint{2.294945in}{2.212145in}}{\pgfqpoint{2.305544in}{2.216535in}}{\pgfqpoint{2.313358in}{2.224349in}}%
\pgfpathcurveto{\pgfqpoint{2.321171in}{2.232163in}}{\pgfqpoint{2.325562in}{2.242762in}}{\pgfqpoint{2.325562in}{2.253812in}}%
\pgfpathcurveto{\pgfqpoint{2.325562in}{2.264862in}}{\pgfqpoint{2.321171in}{2.275461in}}{\pgfqpoint{2.313358in}{2.283275in}}%
\pgfpathcurveto{\pgfqpoint{2.305544in}{2.291088in}}{\pgfqpoint{2.294945in}{2.295478in}}{\pgfqpoint{2.283895in}{2.295478in}}%
\pgfpathcurveto{\pgfqpoint{2.272845in}{2.295478in}}{\pgfqpoint{2.262246in}{2.291088in}}{\pgfqpoint{2.254432in}{2.283275in}}%
\pgfpathcurveto{\pgfqpoint{2.246619in}{2.275461in}}{\pgfqpoint{2.242228in}{2.264862in}}{\pgfqpoint{2.242228in}{2.253812in}}%
\pgfpathcurveto{\pgfqpoint{2.242228in}{2.242762in}}{\pgfqpoint{2.246619in}{2.232163in}}{\pgfqpoint{2.254432in}{2.224349in}}%
\pgfpathcurveto{\pgfqpoint{2.262246in}{2.216535in}}{\pgfqpoint{2.272845in}{2.212145in}}{\pgfqpoint{2.283895in}{2.212145in}}%
\pgfpathlineto{\pgfqpoint{2.283895in}{2.212145in}}%
\pgfpathclose%
\pgfusepath{stroke,fill}%
\end{pgfscope}%
\begin{pgfscope}%
\pgfpathrectangle{\pgfqpoint{0.633874in}{0.569136in}}{\pgfqpoint{2.177280in}{2.201755in}}%
\pgfusepath{clip}%
\pgfsetbuttcap%
\pgfsetroundjoin%
\definecolor{currentfill}{rgb}{0.172549,0.627451,0.172549}%
\pgfsetfillcolor{currentfill}%
\pgfsetlinewidth{0.481800pt}%
\definecolor{currentstroke}{rgb}{1.000000,1.000000,1.000000}%
\pgfsetstrokecolor{currentstroke}%
\pgfsetdash{}{0pt}%
\pgfpathmoveto{\pgfqpoint{1.724753in}{2.045346in}}%
\pgfpathcurveto{\pgfqpoint{1.735803in}{2.045346in}}{\pgfqpoint{1.746402in}{2.049736in}}{\pgfqpoint{1.754216in}{2.057549in}}%
\pgfpathcurveto{\pgfqpoint{1.762030in}{2.065363in}}{\pgfqpoint{1.766420in}{2.075962in}}{\pgfqpoint{1.766420in}{2.087012in}}%
\pgfpathcurveto{\pgfqpoint{1.766420in}{2.098062in}}{\pgfqpoint{1.762030in}{2.108661in}}{\pgfqpoint{1.754216in}{2.116475in}}%
\pgfpathcurveto{\pgfqpoint{1.746402in}{2.124289in}}{\pgfqpoint{1.735803in}{2.128679in}}{\pgfqpoint{1.724753in}{2.128679in}}%
\pgfpathcurveto{\pgfqpoint{1.713703in}{2.128679in}}{\pgfqpoint{1.703104in}{2.124289in}}{\pgfqpoint{1.695290in}{2.116475in}}%
\pgfpathcurveto{\pgfqpoint{1.687477in}{2.108661in}}{\pgfqpoint{1.683087in}{2.098062in}}{\pgfqpoint{1.683087in}{2.087012in}}%
\pgfpathcurveto{\pgfqpoint{1.683087in}{2.075962in}}{\pgfqpoint{1.687477in}{2.065363in}}{\pgfqpoint{1.695290in}{2.057549in}}%
\pgfpathcurveto{\pgfqpoint{1.703104in}{2.049736in}}{\pgfqpoint{1.713703in}{2.045346in}}{\pgfqpoint{1.724753in}{2.045346in}}%
\pgfpathlineto{\pgfqpoint{1.724753in}{2.045346in}}%
\pgfpathclose%
\pgfusepath{stroke,fill}%
\end{pgfscope}%
\begin{pgfscope}%
\pgfpathrectangle{\pgfqpoint{0.633874in}{0.569136in}}{\pgfqpoint{2.177280in}{2.201755in}}%
\pgfusepath{clip}%
\pgfsetbuttcap%
\pgfsetroundjoin%
\definecolor{currentfill}{rgb}{0.172549,0.627451,0.172549}%
\pgfsetfillcolor{currentfill}%
\pgfsetlinewidth{0.481800pt}%
\definecolor{currentstroke}{rgb}{1.000000,1.000000,1.000000}%
\pgfsetstrokecolor{currentstroke}%
\pgfsetdash{}{0pt}%
\pgfpathmoveto{\pgfqpoint{1.884508in}{2.295545in}}%
\pgfpathcurveto{\pgfqpoint{1.895558in}{2.295545in}}{\pgfqpoint{1.906157in}{2.299935in}}{\pgfqpoint{1.913971in}{2.307749in}}%
\pgfpathcurveto{\pgfqpoint{1.921784in}{2.315562in}}{\pgfqpoint{1.926175in}{2.326161in}}{\pgfqpoint{1.926175in}{2.337212in}}%
\pgfpathcurveto{\pgfqpoint{1.926175in}{2.348262in}}{\pgfqpoint{1.921784in}{2.358861in}}{\pgfqpoint{1.913971in}{2.366674in}}%
\pgfpathcurveto{\pgfqpoint{1.906157in}{2.374488in}}{\pgfqpoint{1.895558in}{2.378878in}}{\pgfqpoint{1.884508in}{2.378878in}}%
\pgfpathcurveto{\pgfqpoint{1.873458in}{2.378878in}}{\pgfqpoint{1.862859in}{2.374488in}}{\pgfqpoint{1.855045in}{2.366674in}}%
\pgfpathcurveto{\pgfqpoint{1.847232in}{2.358861in}}{\pgfqpoint{1.842841in}{2.348262in}}{\pgfqpoint{1.842841in}{2.337212in}}%
\pgfpathcurveto{\pgfqpoint{1.842841in}{2.326161in}}{\pgfqpoint{1.847232in}{2.315562in}}{\pgfqpoint{1.855045in}{2.307749in}}%
\pgfpathcurveto{\pgfqpoint{1.862859in}{2.299935in}}{\pgfqpoint{1.873458in}{2.295545in}}{\pgfqpoint{1.884508in}{2.295545in}}%
\pgfpathlineto{\pgfqpoint{1.884508in}{2.295545in}}%
\pgfpathclose%
\pgfusepath{stroke,fill}%
\end{pgfscope}%
\begin{pgfscope}%
\pgfpathrectangle{\pgfqpoint{0.633874in}{0.569136in}}{\pgfqpoint{2.177280in}{2.201755in}}%
\pgfusepath{clip}%
\pgfsetbuttcap%
\pgfsetroundjoin%
\definecolor{currentfill}{rgb}{0.172549,0.627451,0.172549}%
\pgfsetfillcolor{currentfill}%
\pgfsetlinewidth{0.481800pt}%
\definecolor{currentstroke}{rgb}{1.000000,1.000000,1.000000}%
\pgfsetstrokecolor{currentstroke}%
\pgfsetdash{}{0pt}%
\pgfpathmoveto{\pgfqpoint{2.084202in}{2.045346in}}%
\pgfpathcurveto{\pgfqpoint{2.095252in}{2.045346in}}{\pgfqpoint{2.105851in}{2.049736in}}{\pgfqpoint{2.113664in}{2.057549in}}%
\pgfpathcurveto{\pgfqpoint{2.121478in}{2.065363in}}{\pgfqpoint{2.125868in}{2.075962in}}{\pgfqpoint{2.125868in}{2.087012in}}%
\pgfpathcurveto{\pgfqpoint{2.125868in}{2.098062in}}{\pgfqpoint{2.121478in}{2.108661in}}{\pgfqpoint{2.113664in}{2.116475in}}%
\pgfpathcurveto{\pgfqpoint{2.105851in}{2.124289in}}{\pgfqpoint{2.095252in}{2.128679in}}{\pgfqpoint{2.084202in}{2.128679in}}%
\pgfpathcurveto{\pgfqpoint{2.073151in}{2.128679in}}{\pgfqpoint{2.062552in}{2.124289in}}{\pgfqpoint{2.054739in}{2.116475in}}%
\pgfpathcurveto{\pgfqpoint{2.046925in}{2.108661in}}{\pgfqpoint{2.042535in}{2.098062in}}{\pgfqpoint{2.042535in}{2.087012in}}%
\pgfpathcurveto{\pgfqpoint{2.042535in}{2.075962in}}{\pgfqpoint{2.046925in}{2.065363in}}{\pgfqpoint{2.054739in}{2.057549in}}%
\pgfpathcurveto{\pgfqpoint{2.062552in}{2.049736in}}{\pgfqpoint{2.073151in}{2.045346in}}{\pgfqpoint{2.084202in}{2.045346in}}%
\pgfpathlineto{\pgfqpoint{2.084202in}{2.045346in}}%
\pgfpathclose%
\pgfusepath{stroke,fill}%
\end{pgfscope}%
\begin{pgfscope}%
\pgfpathrectangle{\pgfqpoint{0.633874in}{0.569136in}}{\pgfqpoint{2.177280in}{2.201755in}}%
\pgfusepath{clip}%
\pgfsetbuttcap%
\pgfsetroundjoin%
\definecolor{currentfill}{rgb}{0.172549,0.627451,0.172549}%
\pgfsetfillcolor{currentfill}%
\pgfsetlinewidth{0.481800pt}%
\definecolor{currentstroke}{rgb}{1.000000,1.000000,1.000000}%
\pgfsetstrokecolor{currentstroke}%
\pgfsetdash{}{0pt}%
\pgfpathmoveto{\pgfqpoint{1.684815in}{2.045346in}}%
\pgfpathcurveto{\pgfqpoint{1.695865in}{2.045346in}}{\pgfqpoint{1.706464in}{2.049736in}}{\pgfqpoint{1.714277in}{2.057549in}}%
\pgfpathcurveto{\pgfqpoint{1.722091in}{2.065363in}}{\pgfqpoint{1.726481in}{2.075962in}}{\pgfqpoint{1.726481in}{2.087012in}}%
\pgfpathcurveto{\pgfqpoint{1.726481in}{2.098062in}}{\pgfqpoint{1.722091in}{2.108661in}}{\pgfqpoint{1.714277in}{2.116475in}}%
\pgfpathcurveto{\pgfqpoint{1.706464in}{2.124289in}}{\pgfqpoint{1.695865in}{2.128679in}}{\pgfqpoint{1.684815in}{2.128679in}}%
\pgfpathcurveto{\pgfqpoint{1.673764in}{2.128679in}}{\pgfqpoint{1.663165in}{2.124289in}}{\pgfqpoint{1.655352in}{2.116475in}}%
\pgfpathcurveto{\pgfqpoint{1.647538in}{2.108661in}}{\pgfqpoint{1.643148in}{2.098062in}}{\pgfqpoint{1.643148in}{2.087012in}}%
\pgfpathcurveto{\pgfqpoint{1.643148in}{2.075962in}}{\pgfqpoint{1.647538in}{2.065363in}}{\pgfqpoint{1.655352in}{2.057549in}}%
\pgfpathcurveto{\pgfqpoint{1.663165in}{2.049736in}}{\pgfqpoint{1.673764in}{2.045346in}}{\pgfqpoint{1.684815in}{2.045346in}}%
\pgfpathlineto{\pgfqpoint{1.684815in}{2.045346in}}%
\pgfpathclose%
\pgfusepath{stroke,fill}%
\end{pgfscope}%
\begin{pgfscope}%
\pgfpathrectangle{\pgfqpoint{0.633874in}{0.569136in}}{\pgfqpoint{2.177280in}{2.201755in}}%
\pgfusepath{clip}%
\pgfsetbuttcap%
\pgfsetroundjoin%
\definecolor{currentfill}{rgb}{0.172549,0.627451,0.172549}%
\pgfsetfillcolor{currentfill}%
\pgfsetlinewidth{0.481800pt}%
\definecolor{currentstroke}{rgb}{1.000000,1.000000,1.000000}%
\pgfsetstrokecolor{currentstroke}%
\pgfsetdash{}{0pt}%
\pgfpathmoveto{\pgfqpoint{1.644876in}{2.045346in}}%
\pgfpathcurveto{\pgfqpoint{1.655926in}{2.045346in}}{\pgfqpoint{1.666525in}{2.049736in}}{\pgfqpoint{1.674339in}{2.057549in}}%
\pgfpathcurveto{\pgfqpoint{1.682152in}{2.065363in}}{\pgfqpoint{1.686543in}{2.075962in}}{\pgfqpoint{1.686543in}{2.087012in}}%
\pgfpathcurveto{\pgfqpoint{1.686543in}{2.098062in}}{\pgfqpoint{1.682152in}{2.108661in}}{\pgfqpoint{1.674339in}{2.116475in}}%
\pgfpathcurveto{\pgfqpoint{1.666525in}{2.124289in}}{\pgfqpoint{1.655926in}{2.128679in}}{\pgfqpoint{1.644876in}{2.128679in}}%
\pgfpathcurveto{\pgfqpoint{1.633826in}{2.128679in}}{\pgfqpoint{1.623227in}{2.124289in}}{\pgfqpoint{1.615413in}{2.116475in}}%
\pgfpathcurveto{\pgfqpoint{1.607599in}{2.108661in}}{\pgfqpoint{1.603209in}{2.098062in}}{\pgfqpoint{1.603209in}{2.087012in}}%
\pgfpathcurveto{\pgfqpoint{1.603209in}{2.075962in}}{\pgfqpoint{1.607599in}{2.065363in}}{\pgfqpoint{1.615413in}{2.057549in}}%
\pgfpathcurveto{\pgfqpoint{1.623227in}{2.049736in}}{\pgfqpoint{1.633826in}{2.045346in}}{\pgfqpoint{1.644876in}{2.045346in}}%
\pgfpathlineto{\pgfqpoint{1.644876in}{2.045346in}}%
\pgfpathclose%
\pgfusepath{stroke,fill}%
\end{pgfscope}%
\begin{pgfscope}%
\pgfpathrectangle{\pgfqpoint{0.633874in}{0.569136in}}{\pgfqpoint{2.177280in}{2.201755in}}%
\pgfusepath{clip}%
\pgfsetbuttcap%
\pgfsetroundjoin%
\definecolor{currentfill}{rgb}{0.172549,0.627451,0.172549}%
\pgfsetfillcolor{currentfill}%
\pgfsetlinewidth{0.481800pt}%
\definecolor{currentstroke}{rgb}{1.000000,1.000000,1.000000}%
\pgfsetstrokecolor{currentstroke}%
\pgfsetdash{}{0pt}%
\pgfpathmoveto{\pgfqpoint{1.764692in}{2.295545in}}%
\pgfpathcurveto{\pgfqpoint{1.775742in}{2.295545in}}{\pgfqpoint{1.786341in}{2.299935in}}{\pgfqpoint{1.794155in}{2.307749in}}%
\pgfpathcurveto{\pgfqpoint{1.801968in}{2.315562in}}{\pgfqpoint{1.806359in}{2.326161in}}{\pgfqpoint{1.806359in}{2.337212in}}%
\pgfpathcurveto{\pgfqpoint{1.806359in}{2.348262in}}{\pgfqpoint{1.801968in}{2.358861in}}{\pgfqpoint{1.794155in}{2.366674in}}%
\pgfpathcurveto{\pgfqpoint{1.786341in}{2.374488in}}{\pgfqpoint{1.775742in}{2.378878in}}{\pgfqpoint{1.764692in}{2.378878in}}%
\pgfpathcurveto{\pgfqpoint{1.753642in}{2.378878in}}{\pgfqpoint{1.743043in}{2.374488in}}{\pgfqpoint{1.735229in}{2.366674in}}%
\pgfpathcurveto{\pgfqpoint{1.727416in}{2.358861in}}{\pgfqpoint{1.723025in}{2.348262in}}{\pgfqpoint{1.723025in}{2.337212in}}%
\pgfpathcurveto{\pgfqpoint{1.723025in}{2.326161in}}{\pgfqpoint{1.727416in}{2.315562in}}{\pgfqpoint{1.735229in}{2.307749in}}%
\pgfpathcurveto{\pgfqpoint{1.743043in}{2.299935in}}{\pgfqpoint{1.753642in}{2.295545in}}{\pgfqpoint{1.764692in}{2.295545in}}%
\pgfpathlineto{\pgfqpoint{1.764692in}{2.295545in}}%
\pgfpathclose%
\pgfusepath{stroke,fill}%
\end{pgfscope}%
\begin{pgfscope}%
\pgfpathrectangle{\pgfqpoint{0.633874in}{0.569136in}}{\pgfqpoint{2.177280in}{2.201755in}}%
\pgfusepath{clip}%
\pgfsetbuttcap%
\pgfsetroundjoin%
\definecolor{currentfill}{rgb}{0.172549,0.627451,0.172549}%
\pgfsetfillcolor{currentfill}%
\pgfsetlinewidth{0.481800pt}%
\definecolor{currentstroke}{rgb}{1.000000,1.000000,1.000000}%
\pgfsetstrokecolor{currentstroke}%
\pgfsetdash{}{0pt}%
\pgfpathmoveto{\pgfqpoint{2.084202in}{1.878546in}}%
\pgfpathcurveto{\pgfqpoint{2.095252in}{1.878546in}}{\pgfqpoint{2.105851in}{1.882936in}}{\pgfqpoint{2.113664in}{1.890750in}}%
\pgfpathcurveto{\pgfqpoint{2.121478in}{1.898563in}}{\pgfqpoint{2.125868in}{1.909162in}}{\pgfqpoint{2.125868in}{1.920213in}}%
\pgfpathcurveto{\pgfqpoint{2.125868in}{1.931263in}}{\pgfqpoint{2.121478in}{1.941862in}}{\pgfqpoint{2.113664in}{1.949675in}}%
\pgfpathcurveto{\pgfqpoint{2.105851in}{1.957489in}}{\pgfqpoint{2.095252in}{1.961879in}}{\pgfqpoint{2.084202in}{1.961879in}}%
\pgfpathcurveto{\pgfqpoint{2.073151in}{1.961879in}}{\pgfqpoint{2.062552in}{1.957489in}}{\pgfqpoint{2.054739in}{1.949675in}}%
\pgfpathcurveto{\pgfqpoint{2.046925in}{1.941862in}}{\pgfqpoint{2.042535in}{1.931263in}}{\pgfqpoint{2.042535in}{1.920213in}}%
\pgfpathcurveto{\pgfqpoint{2.042535in}{1.909162in}}{\pgfqpoint{2.046925in}{1.898563in}}{\pgfqpoint{2.054739in}{1.890750in}}%
\pgfpathcurveto{\pgfqpoint{2.062552in}{1.882936in}}{\pgfqpoint{2.073151in}{1.878546in}}{\pgfqpoint{2.084202in}{1.878546in}}%
\pgfpathlineto{\pgfqpoint{2.084202in}{1.878546in}}%
\pgfpathclose%
\pgfusepath{stroke,fill}%
\end{pgfscope}%
\begin{pgfscope}%
\pgfpathrectangle{\pgfqpoint{0.633874in}{0.569136in}}{\pgfqpoint{2.177280in}{2.201755in}}%
\pgfusepath{clip}%
\pgfsetbuttcap%
\pgfsetroundjoin%
\definecolor{currentfill}{rgb}{0.172549,0.627451,0.172549}%
\pgfsetfillcolor{currentfill}%
\pgfsetlinewidth{0.481800pt}%
\definecolor{currentstroke}{rgb}{1.000000,1.000000,1.000000}%
\pgfsetstrokecolor{currentstroke}%
\pgfsetdash{}{0pt}%
\pgfpathmoveto{\pgfqpoint{2.164079in}{2.128745in}}%
\pgfpathcurveto{\pgfqpoint{2.175129in}{2.128745in}}{\pgfqpoint{2.185728in}{2.133136in}}{\pgfqpoint{2.193542in}{2.140949in}}%
\pgfpathcurveto{\pgfqpoint{2.201355in}{2.148763in}}{\pgfqpoint{2.205746in}{2.159362in}}{\pgfqpoint{2.205746in}{2.170412in}}%
\pgfpathcurveto{\pgfqpoint{2.205746in}{2.181462in}}{\pgfqpoint{2.201355in}{2.192061in}}{\pgfqpoint{2.193542in}{2.199875in}}%
\pgfpathcurveto{\pgfqpoint{2.185728in}{2.207688in}}{\pgfqpoint{2.175129in}{2.212079in}}{\pgfqpoint{2.164079in}{2.212079in}}%
\pgfpathcurveto{\pgfqpoint{2.153029in}{2.212079in}}{\pgfqpoint{2.142430in}{2.207688in}}{\pgfqpoint{2.134616in}{2.199875in}}%
\pgfpathcurveto{\pgfqpoint{2.126803in}{2.192061in}}{\pgfqpoint{2.122412in}{2.181462in}}{\pgfqpoint{2.122412in}{2.170412in}}%
\pgfpathcurveto{\pgfqpoint{2.122412in}{2.159362in}}{\pgfqpoint{2.126803in}{2.148763in}}{\pgfqpoint{2.134616in}{2.140949in}}%
\pgfpathcurveto{\pgfqpoint{2.142430in}{2.133136in}}{\pgfqpoint{2.153029in}{2.128745in}}{\pgfqpoint{2.164079in}{2.128745in}}%
\pgfpathlineto{\pgfqpoint{2.164079in}{2.128745in}}%
\pgfpathclose%
\pgfusepath{stroke,fill}%
\end{pgfscope}%
\begin{pgfscope}%
\pgfpathrectangle{\pgfqpoint{0.633874in}{0.569136in}}{\pgfqpoint{2.177280in}{2.201755in}}%
\pgfusepath{clip}%
\pgfsetbuttcap%
\pgfsetroundjoin%
\definecolor{currentfill}{rgb}{0.172549,0.627451,0.172549}%
\pgfsetfillcolor{currentfill}%
\pgfsetlinewidth{0.481800pt}%
\definecolor{currentstroke}{rgb}{1.000000,1.000000,1.000000}%
\pgfsetstrokecolor{currentstroke}%
\pgfsetdash{}{0pt}%
\pgfpathmoveto{\pgfqpoint{2.363772in}{2.212145in}}%
\pgfpathcurveto{\pgfqpoint{2.374823in}{2.212145in}}{\pgfqpoint{2.385422in}{2.216535in}}{\pgfqpoint{2.393235in}{2.224349in}}%
\pgfpathcurveto{\pgfqpoint{2.401049in}{2.232163in}}{\pgfqpoint{2.405439in}{2.242762in}}{\pgfqpoint{2.405439in}{2.253812in}}%
\pgfpathcurveto{\pgfqpoint{2.405439in}{2.264862in}}{\pgfqpoint{2.401049in}{2.275461in}}{\pgfqpoint{2.393235in}{2.283275in}}%
\pgfpathcurveto{\pgfqpoint{2.385422in}{2.291088in}}{\pgfqpoint{2.374823in}{2.295478in}}{\pgfqpoint{2.363772in}{2.295478in}}%
\pgfpathcurveto{\pgfqpoint{2.352722in}{2.295478in}}{\pgfqpoint{2.342123in}{2.291088in}}{\pgfqpoint{2.334310in}{2.283275in}}%
\pgfpathcurveto{\pgfqpoint{2.326496in}{2.275461in}}{\pgfqpoint{2.322106in}{2.264862in}}{\pgfqpoint{2.322106in}{2.253812in}}%
\pgfpathcurveto{\pgfqpoint{2.322106in}{2.242762in}}{\pgfqpoint{2.326496in}{2.232163in}}{\pgfqpoint{2.334310in}{2.224349in}}%
\pgfpathcurveto{\pgfqpoint{2.342123in}{2.216535in}}{\pgfqpoint{2.352722in}{2.212145in}}{\pgfqpoint{2.363772in}{2.212145in}}%
\pgfpathlineto{\pgfqpoint{2.363772in}{2.212145in}}%
\pgfpathclose%
\pgfusepath{stroke,fill}%
\end{pgfscope}%
\begin{pgfscope}%
\pgfpathrectangle{\pgfqpoint{0.633874in}{0.569136in}}{\pgfqpoint{2.177280in}{2.201755in}}%
\pgfusepath{clip}%
\pgfsetbuttcap%
\pgfsetroundjoin%
\definecolor{currentfill}{rgb}{0.172549,0.627451,0.172549}%
\pgfsetfillcolor{currentfill}%
\pgfsetlinewidth{0.481800pt}%
\definecolor{currentstroke}{rgb}{1.000000,1.000000,1.000000}%
\pgfsetstrokecolor{currentstroke}%
\pgfsetdash{}{0pt}%
\pgfpathmoveto{\pgfqpoint{1.764692in}{2.378945in}}%
\pgfpathcurveto{\pgfqpoint{1.775742in}{2.378945in}}{\pgfqpoint{1.786341in}{2.383335in}}{\pgfqpoint{1.794155in}{2.391149in}}%
\pgfpathcurveto{\pgfqpoint{1.801968in}{2.398962in}}{\pgfqpoint{1.806359in}{2.409561in}}{\pgfqpoint{1.806359in}{2.420611in}}%
\pgfpathcurveto{\pgfqpoint{1.806359in}{2.431662in}}{\pgfqpoint{1.801968in}{2.442261in}}{\pgfqpoint{1.794155in}{2.450074in}}%
\pgfpathcurveto{\pgfqpoint{1.786341in}{2.457888in}}{\pgfqpoint{1.775742in}{2.462278in}}{\pgfqpoint{1.764692in}{2.462278in}}%
\pgfpathcurveto{\pgfqpoint{1.753642in}{2.462278in}}{\pgfqpoint{1.743043in}{2.457888in}}{\pgfqpoint{1.735229in}{2.450074in}}%
\pgfpathcurveto{\pgfqpoint{1.727416in}{2.442261in}}{\pgfqpoint{1.723025in}{2.431662in}}{\pgfqpoint{1.723025in}{2.420611in}}%
\pgfpathcurveto{\pgfqpoint{1.723025in}{2.409561in}}{\pgfqpoint{1.727416in}{2.398962in}}{\pgfqpoint{1.735229in}{2.391149in}}%
\pgfpathcurveto{\pgfqpoint{1.743043in}{2.383335in}}{\pgfqpoint{1.753642in}{2.378945in}}{\pgfqpoint{1.764692in}{2.378945in}}%
\pgfpathlineto{\pgfqpoint{1.764692in}{2.378945in}}%
\pgfpathclose%
\pgfusepath{stroke,fill}%
\end{pgfscope}%
\begin{pgfscope}%
\pgfpathrectangle{\pgfqpoint{0.633874in}{0.569136in}}{\pgfqpoint{2.177280in}{2.201755in}}%
\pgfusepath{clip}%
\pgfsetbuttcap%
\pgfsetroundjoin%
\definecolor{currentfill}{rgb}{0.172549,0.627451,0.172549}%
\pgfsetfillcolor{currentfill}%
\pgfsetlinewidth{0.481800pt}%
\definecolor{currentstroke}{rgb}{1.000000,1.000000,1.000000}%
\pgfsetstrokecolor{currentstroke}%
\pgfsetdash{}{0pt}%
\pgfpathmoveto{\pgfqpoint{1.724753in}{1.795146in}}%
\pgfpathcurveto{\pgfqpoint{1.735803in}{1.795146in}}{\pgfqpoint{1.746402in}{1.799536in}}{\pgfqpoint{1.754216in}{1.807350in}}%
\pgfpathcurveto{\pgfqpoint{1.762030in}{1.815164in}}{\pgfqpoint{1.766420in}{1.825763in}}{\pgfqpoint{1.766420in}{1.836813in}}%
\pgfpathcurveto{\pgfqpoint{1.766420in}{1.847863in}}{\pgfqpoint{1.762030in}{1.858462in}}{\pgfqpoint{1.754216in}{1.866276in}}%
\pgfpathcurveto{\pgfqpoint{1.746402in}{1.874089in}}{\pgfqpoint{1.735803in}{1.878479in}}{\pgfqpoint{1.724753in}{1.878479in}}%
\pgfpathcurveto{\pgfqpoint{1.713703in}{1.878479in}}{\pgfqpoint{1.703104in}{1.874089in}}{\pgfqpoint{1.695290in}{1.866276in}}%
\pgfpathcurveto{\pgfqpoint{1.687477in}{1.858462in}}{\pgfqpoint{1.683087in}{1.847863in}}{\pgfqpoint{1.683087in}{1.836813in}}%
\pgfpathcurveto{\pgfqpoint{1.683087in}{1.825763in}}{\pgfqpoint{1.687477in}{1.815164in}}{\pgfqpoint{1.695290in}{1.807350in}}%
\pgfpathcurveto{\pgfqpoint{1.703104in}{1.799536in}}{\pgfqpoint{1.713703in}{1.795146in}}{\pgfqpoint{1.724753in}{1.795146in}}%
\pgfpathlineto{\pgfqpoint{1.724753in}{1.795146in}}%
\pgfpathclose%
\pgfusepath{stroke,fill}%
\end{pgfscope}%
\begin{pgfscope}%
\pgfpathrectangle{\pgfqpoint{0.633874in}{0.569136in}}{\pgfqpoint{2.177280in}{2.201755in}}%
\pgfusepath{clip}%
\pgfsetbuttcap%
\pgfsetroundjoin%
\definecolor{currentfill}{rgb}{0.172549,0.627451,0.172549}%
\pgfsetfillcolor{currentfill}%
\pgfsetlinewidth{0.481800pt}%
\definecolor{currentstroke}{rgb}{1.000000,1.000000,1.000000}%
\pgfsetstrokecolor{currentstroke}%
\pgfsetdash{}{0pt}%
\pgfpathmoveto{\pgfqpoint{1.644876in}{1.711746in}}%
\pgfpathcurveto{\pgfqpoint{1.655926in}{1.711746in}}{\pgfqpoint{1.666525in}{1.716137in}}{\pgfqpoint{1.674339in}{1.723950in}}%
\pgfpathcurveto{\pgfqpoint{1.682152in}{1.731764in}}{\pgfqpoint{1.686543in}{1.742363in}}{\pgfqpoint{1.686543in}{1.753413in}}%
\pgfpathcurveto{\pgfqpoint{1.686543in}{1.764463in}}{\pgfqpoint{1.682152in}{1.775062in}}{\pgfqpoint{1.674339in}{1.782876in}}%
\pgfpathcurveto{\pgfqpoint{1.666525in}{1.790689in}}{\pgfqpoint{1.655926in}{1.795080in}}{\pgfqpoint{1.644876in}{1.795080in}}%
\pgfpathcurveto{\pgfqpoint{1.633826in}{1.795080in}}{\pgfqpoint{1.623227in}{1.790689in}}{\pgfqpoint{1.615413in}{1.782876in}}%
\pgfpathcurveto{\pgfqpoint{1.607599in}{1.775062in}}{\pgfqpoint{1.603209in}{1.764463in}}{\pgfqpoint{1.603209in}{1.753413in}}%
\pgfpathcurveto{\pgfqpoint{1.603209in}{1.742363in}}{\pgfqpoint{1.607599in}{1.731764in}}{\pgfqpoint{1.615413in}{1.723950in}}%
\pgfpathcurveto{\pgfqpoint{1.623227in}{1.716137in}}{\pgfqpoint{1.633826in}{1.711746in}}{\pgfqpoint{1.644876in}{1.711746in}}%
\pgfpathlineto{\pgfqpoint{1.644876in}{1.711746in}}%
\pgfpathclose%
\pgfusepath{stroke,fill}%
\end{pgfscope}%
\begin{pgfscope}%
\pgfpathrectangle{\pgfqpoint{0.633874in}{0.569136in}}{\pgfqpoint{2.177280in}{2.201755in}}%
\pgfusepath{clip}%
\pgfsetbuttcap%
\pgfsetroundjoin%
\definecolor{currentfill}{rgb}{0.172549,0.627451,0.172549}%
\pgfsetfillcolor{currentfill}%
\pgfsetlinewidth{0.481800pt}%
\definecolor{currentstroke}{rgb}{1.000000,1.000000,1.000000}%
\pgfsetstrokecolor{currentstroke}%
\pgfsetdash{}{0pt}%
\pgfpathmoveto{\pgfqpoint{2.283895in}{2.462345in}}%
\pgfpathcurveto{\pgfqpoint{2.294945in}{2.462345in}}{\pgfqpoint{2.305544in}{2.466735in}}{\pgfqpoint{2.313358in}{2.474548in}}%
\pgfpathcurveto{\pgfqpoint{2.321171in}{2.482362in}}{\pgfqpoint{2.325562in}{2.492961in}}{\pgfqpoint{2.325562in}{2.504011in}}%
\pgfpathcurveto{\pgfqpoint{2.325562in}{2.515061in}}{\pgfqpoint{2.321171in}{2.525660in}}{\pgfqpoint{2.313358in}{2.533474in}}%
\pgfpathcurveto{\pgfqpoint{2.305544in}{2.541288in}}{\pgfqpoint{2.294945in}{2.545678in}}{\pgfqpoint{2.283895in}{2.545678in}}%
\pgfpathcurveto{\pgfqpoint{2.272845in}{2.545678in}}{\pgfqpoint{2.262246in}{2.541288in}}{\pgfqpoint{2.254432in}{2.533474in}}%
\pgfpathcurveto{\pgfqpoint{2.246619in}{2.525660in}}{\pgfqpoint{2.242228in}{2.515061in}}{\pgfqpoint{2.242228in}{2.504011in}}%
\pgfpathcurveto{\pgfqpoint{2.242228in}{2.492961in}}{\pgfqpoint{2.246619in}{2.482362in}}{\pgfqpoint{2.254432in}{2.474548in}}%
\pgfpathcurveto{\pgfqpoint{2.262246in}{2.466735in}}{\pgfqpoint{2.272845in}{2.462345in}}{\pgfqpoint{2.283895in}{2.462345in}}%
\pgfpathlineto{\pgfqpoint{2.283895in}{2.462345in}}%
\pgfpathclose%
\pgfusepath{stroke,fill}%
\end{pgfscope}%
\begin{pgfscope}%
\pgfpathrectangle{\pgfqpoint{0.633874in}{0.569136in}}{\pgfqpoint{2.177280in}{2.201755in}}%
\pgfusepath{clip}%
\pgfsetbuttcap%
\pgfsetroundjoin%
\definecolor{currentfill}{rgb}{0.172549,0.627451,0.172549}%
\pgfsetfillcolor{currentfill}%
\pgfsetlinewidth{0.481800pt}%
\definecolor{currentstroke}{rgb}{1.000000,1.000000,1.000000}%
\pgfsetstrokecolor{currentstroke}%
\pgfsetdash{}{0pt}%
\pgfpathmoveto{\pgfqpoint{1.724753in}{2.545744in}}%
\pgfpathcurveto{\pgfqpoint{1.735803in}{2.545744in}}{\pgfqpoint{1.746402in}{2.550135in}}{\pgfqpoint{1.754216in}{2.557948in}}%
\pgfpathcurveto{\pgfqpoint{1.762030in}{2.565762in}}{\pgfqpoint{1.766420in}{2.576361in}}{\pgfqpoint{1.766420in}{2.587411in}}%
\pgfpathcurveto{\pgfqpoint{1.766420in}{2.598461in}}{\pgfqpoint{1.762030in}{2.609060in}}{\pgfqpoint{1.754216in}{2.616874in}}%
\pgfpathcurveto{\pgfqpoint{1.746402in}{2.624687in}}{\pgfqpoint{1.735803in}{2.629078in}}{\pgfqpoint{1.724753in}{2.629078in}}%
\pgfpathcurveto{\pgfqpoint{1.713703in}{2.629078in}}{\pgfqpoint{1.703104in}{2.624687in}}{\pgfqpoint{1.695290in}{2.616874in}}%
\pgfpathcurveto{\pgfqpoint{1.687477in}{2.609060in}}{\pgfqpoint{1.683087in}{2.598461in}}{\pgfqpoint{1.683087in}{2.587411in}}%
\pgfpathcurveto{\pgfqpoint{1.683087in}{2.576361in}}{\pgfqpoint{1.687477in}{2.565762in}}{\pgfqpoint{1.695290in}{2.557948in}}%
\pgfpathcurveto{\pgfqpoint{1.703104in}{2.550135in}}{\pgfqpoint{1.713703in}{2.545744in}}{\pgfqpoint{1.724753in}{2.545744in}}%
\pgfpathlineto{\pgfqpoint{1.724753in}{2.545744in}}%
\pgfpathclose%
\pgfusepath{stroke,fill}%
\end{pgfscope}%
\begin{pgfscope}%
\pgfpathrectangle{\pgfqpoint{0.633874in}{0.569136in}}{\pgfqpoint{2.177280in}{2.201755in}}%
\pgfusepath{clip}%
\pgfsetbuttcap%
\pgfsetroundjoin%
\definecolor{currentfill}{rgb}{0.172549,0.627451,0.172549}%
\pgfsetfillcolor{currentfill}%
\pgfsetlinewidth{0.481800pt}%
\definecolor{currentstroke}{rgb}{1.000000,1.000000,1.000000}%
\pgfsetstrokecolor{currentstroke}%
\pgfsetdash{}{0pt}%
\pgfpathmoveto{\pgfqpoint{1.764692in}{2.045346in}}%
\pgfpathcurveto{\pgfqpoint{1.775742in}{2.045346in}}{\pgfqpoint{1.786341in}{2.049736in}}{\pgfqpoint{1.794155in}{2.057549in}}%
\pgfpathcurveto{\pgfqpoint{1.801968in}{2.065363in}}{\pgfqpoint{1.806359in}{2.075962in}}{\pgfqpoint{1.806359in}{2.087012in}}%
\pgfpathcurveto{\pgfqpoint{1.806359in}{2.098062in}}{\pgfqpoint{1.801968in}{2.108661in}}{\pgfqpoint{1.794155in}{2.116475in}}%
\pgfpathcurveto{\pgfqpoint{1.786341in}{2.124289in}}{\pgfqpoint{1.775742in}{2.128679in}}{\pgfqpoint{1.764692in}{2.128679in}}%
\pgfpathcurveto{\pgfqpoint{1.753642in}{2.128679in}}{\pgfqpoint{1.743043in}{2.124289in}}{\pgfqpoint{1.735229in}{2.116475in}}%
\pgfpathcurveto{\pgfqpoint{1.727416in}{2.108661in}}{\pgfqpoint{1.723025in}{2.098062in}}{\pgfqpoint{1.723025in}{2.087012in}}%
\pgfpathcurveto{\pgfqpoint{1.723025in}{2.075962in}}{\pgfqpoint{1.727416in}{2.065363in}}{\pgfqpoint{1.735229in}{2.057549in}}%
\pgfpathcurveto{\pgfqpoint{1.743043in}{2.049736in}}{\pgfqpoint{1.753642in}{2.045346in}}{\pgfqpoint{1.764692in}{2.045346in}}%
\pgfpathlineto{\pgfqpoint{1.764692in}{2.045346in}}%
\pgfpathclose%
\pgfusepath{stroke,fill}%
\end{pgfscope}%
\begin{pgfscope}%
\pgfpathrectangle{\pgfqpoint{0.633874in}{0.569136in}}{\pgfqpoint{2.177280in}{2.201755in}}%
\pgfusepath{clip}%
\pgfsetbuttcap%
\pgfsetroundjoin%
\definecolor{currentfill}{rgb}{0.172549,0.627451,0.172549}%
\pgfsetfillcolor{currentfill}%
\pgfsetlinewidth{0.481800pt}%
\definecolor{currentstroke}{rgb}{1.000000,1.000000,1.000000}%
\pgfsetstrokecolor{currentstroke}%
\pgfsetdash{}{0pt}%
\pgfpathmoveto{\pgfqpoint{1.604937in}{2.045346in}}%
\pgfpathcurveto{\pgfqpoint{1.615987in}{2.045346in}}{\pgfqpoint{1.626586in}{2.049736in}}{\pgfqpoint{1.634400in}{2.057549in}}%
\pgfpathcurveto{\pgfqpoint{1.642214in}{2.065363in}}{\pgfqpoint{1.646604in}{2.075962in}}{\pgfqpoint{1.646604in}{2.087012in}}%
\pgfpathcurveto{\pgfqpoint{1.646604in}{2.098062in}}{\pgfqpoint{1.642214in}{2.108661in}}{\pgfqpoint{1.634400in}{2.116475in}}%
\pgfpathcurveto{\pgfqpoint{1.626586in}{2.124289in}}{\pgfqpoint{1.615987in}{2.128679in}}{\pgfqpoint{1.604937in}{2.128679in}}%
\pgfpathcurveto{\pgfqpoint{1.593887in}{2.128679in}}{\pgfqpoint{1.583288in}{2.124289in}}{\pgfqpoint{1.575474in}{2.116475in}}%
\pgfpathcurveto{\pgfqpoint{1.567661in}{2.108661in}}{\pgfqpoint{1.563270in}{2.098062in}}{\pgfqpoint{1.563270in}{2.087012in}}%
\pgfpathcurveto{\pgfqpoint{1.563270in}{2.075962in}}{\pgfqpoint{1.567661in}{2.065363in}}{\pgfqpoint{1.575474in}{2.057549in}}%
\pgfpathcurveto{\pgfqpoint{1.583288in}{2.049736in}}{\pgfqpoint{1.593887in}{2.045346in}}{\pgfqpoint{1.604937in}{2.045346in}}%
\pgfpathlineto{\pgfqpoint{1.604937in}{2.045346in}}%
\pgfpathclose%
\pgfusepath{stroke,fill}%
\end{pgfscope}%
\begin{pgfscope}%
\pgfpathrectangle{\pgfqpoint{0.633874in}{0.569136in}}{\pgfqpoint{2.177280in}{2.201755in}}%
\pgfusepath{clip}%
\pgfsetbuttcap%
\pgfsetroundjoin%
\definecolor{currentfill}{rgb}{0.172549,0.627451,0.172549}%
\pgfsetfillcolor{currentfill}%
\pgfsetlinewidth{0.481800pt}%
\definecolor{currentstroke}{rgb}{1.000000,1.000000,1.000000}%
\pgfsetstrokecolor{currentstroke}%
\pgfsetdash{}{0pt}%
\pgfpathmoveto{\pgfqpoint{1.964385in}{2.295545in}}%
\pgfpathcurveto{\pgfqpoint{1.975436in}{2.295545in}}{\pgfqpoint{1.986035in}{2.299935in}}{\pgfqpoint{1.993848in}{2.307749in}}%
\pgfpathcurveto{\pgfqpoint{2.001662in}{2.315562in}}{\pgfqpoint{2.006052in}{2.326161in}}{\pgfqpoint{2.006052in}{2.337212in}}%
\pgfpathcurveto{\pgfqpoint{2.006052in}{2.348262in}}{\pgfqpoint{2.001662in}{2.358861in}}{\pgfqpoint{1.993848in}{2.366674in}}%
\pgfpathcurveto{\pgfqpoint{1.986035in}{2.374488in}}{\pgfqpoint{1.975436in}{2.378878in}}{\pgfqpoint{1.964385in}{2.378878in}}%
\pgfpathcurveto{\pgfqpoint{1.953335in}{2.378878in}}{\pgfqpoint{1.942736in}{2.374488in}}{\pgfqpoint{1.934923in}{2.366674in}}%
\pgfpathcurveto{\pgfqpoint{1.927109in}{2.358861in}}{\pgfqpoint{1.922719in}{2.348262in}}{\pgfqpoint{1.922719in}{2.337212in}}%
\pgfpathcurveto{\pgfqpoint{1.922719in}{2.326161in}}{\pgfqpoint{1.927109in}{2.315562in}}{\pgfqpoint{1.934923in}{2.307749in}}%
\pgfpathcurveto{\pgfqpoint{1.942736in}{2.299935in}}{\pgfqpoint{1.953335in}{2.295545in}}{\pgfqpoint{1.964385in}{2.295545in}}%
\pgfpathlineto{\pgfqpoint{1.964385in}{2.295545in}}%
\pgfpathclose%
\pgfusepath{stroke,fill}%
\end{pgfscope}%
\begin{pgfscope}%
\pgfpathrectangle{\pgfqpoint{0.633874in}{0.569136in}}{\pgfqpoint{2.177280in}{2.201755in}}%
\pgfusepath{clip}%
\pgfsetbuttcap%
\pgfsetroundjoin%
\definecolor{currentfill}{rgb}{0.172549,0.627451,0.172549}%
\pgfsetfillcolor{currentfill}%
\pgfsetlinewidth{0.481800pt}%
\definecolor{currentstroke}{rgb}{1.000000,1.000000,1.000000}%
\pgfsetstrokecolor{currentstroke}%
\pgfsetdash{}{0pt}%
\pgfpathmoveto{\pgfqpoint{1.884508in}{2.545744in}}%
\pgfpathcurveto{\pgfqpoint{1.895558in}{2.545744in}}{\pgfqpoint{1.906157in}{2.550135in}}{\pgfqpoint{1.913971in}{2.557948in}}%
\pgfpathcurveto{\pgfqpoint{1.921784in}{2.565762in}}{\pgfqpoint{1.926175in}{2.576361in}}{\pgfqpoint{1.926175in}{2.587411in}}%
\pgfpathcurveto{\pgfqpoint{1.926175in}{2.598461in}}{\pgfqpoint{1.921784in}{2.609060in}}{\pgfqpoint{1.913971in}{2.616874in}}%
\pgfpathcurveto{\pgfqpoint{1.906157in}{2.624687in}}{\pgfqpoint{1.895558in}{2.629078in}}{\pgfqpoint{1.884508in}{2.629078in}}%
\pgfpathcurveto{\pgfqpoint{1.873458in}{2.629078in}}{\pgfqpoint{1.862859in}{2.624687in}}{\pgfqpoint{1.855045in}{2.616874in}}%
\pgfpathcurveto{\pgfqpoint{1.847232in}{2.609060in}}{\pgfqpoint{1.842841in}{2.598461in}}{\pgfqpoint{1.842841in}{2.587411in}}%
\pgfpathcurveto{\pgfqpoint{1.842841in}{2.576361in}}{\pgfqpoint{1.847232in}{2.565762in}}{\pgfqpoint{1.855045in}{2.557948in}}%
\pgfpathcurveto{\pgfqpoint{1.862859in}{2.550135in}}{\pgfqpoint{1.873458in}{2.545744in}}{\pgfqpoint{1.884508in}{2.545744in}}%
\pgfpathlineto{\pgfqpoint{1.884508in}{2.545744in}}%
\pgfpathclose%
\pgfusepath{stroke,fill}%
\end{pgfscope}%
\begin{pgfscope}%
\pgfpathrectangle{\pgfqpoint{0.633874in}{0.569136in}}{\pgfqpoint{2.177280in}{2.201755in}}%
\pgfusepath{clip}%
\pgfsetbuttcap%
\pgfsetroundjoin%
\definecolor{currentfill}{rgb}{0.172549,0.627451,0.172549}%
\pgfsetfillcolor{currentfill}%
\pgfsetlinewidth{0.481800pt}%
\definecolor{currentstroke}{rgb}{1.000000,1.000000,1.000000}%
\pgfsetstrokecolor{currentstroke}%
\pgfsetdash{}{0pt}%
\pgfpathmoveto{\pgfqpoint{1.964385in}{2.462345in}}%
\pgfpathcurveto{\pgfqpoint{1.975436in}{2.462345in}}{\pgfqpoint{1.986035in}{2.466735in}}{\pgfqpoint{1.993848in}{2.474548in}}%
\pgfpathcurveto{\pgfqpoint{2.001662in}{2.482362in}}{\pgfqpoint{2.006052in}{2.492961in}}{\pgfqpoint{2.006052in}{2.504011in}}%
\pgfpathcurveto{\pgfqpoint{2.006052in}{2.515061in}}{\pgfqpoint{2.001662in}{2.525660in}}{\pgfqpoint{1.993848in}{2.533474in}}%
\pgfpathcurveto{\pgfqpoint{1.986035in}{2.541288in}}{\pgfqpoint{1.975436in}{2.545678in}}{\pgfqpoint{1.964385in}{2.545678in}}%
\pgfpathcurveto{\pgfqpoint{1.953335in}{2.545678in}}{\pgfqpoint{1.942736in}{2.541288in}}{\pgfqpoint{1.934923in}{2.533474in}}%
\pgfpathcurveto{\pgfqpoint{1.927109in}{2.525660in}}{\pgfqpoint{1.922719in}{2.515061in}}{\pgfqpoint{1.922719in}{2.504011in}}%
\pgfpathcurveto{\pgfqpoint{1.922719in}{2.492961in}}{\pgfqpoint{1.927109in}{2.482362in}}{\pgfqpoint{1.934923in}{2.474548in}}%
\pgfpathcurveto{\pgfqpoint{1.942736in}{2.466735in}}{\pgfqpoint{1.953335in}{2.462345in}}{\pgfqpoint{1.964385in}{2.462345in}}%
\pgfpathlineto{\pgfqpoint{1.964385in}{2.462345in}}%
\pgfpathclose%
\pgfusepath{stroke,fill}%
\end{pgfscope}%
\begin{pgfscope}%
\pgfpathrectangle{\pgfqpoint{0.633874in}{0.569136in}}{\pgfqpoint{2.177280in}{2.201755in}}%
\pgfusepath{clip}%
\pgfsetbuttcap%
\pgfsetroundjoin%
\definecolor{currentfill}{rgb}{0.172549,0.627451,0.172549}%
\pgfsetfillcolor{currentfill}%
\pgfsetlinewidth{0.481800pt}%
\definecolor{currentstroke}{rgb}{1.000000,1.000000,1.000000}%
\pgfsetstrokecolor{currentstroke}%
\pgfsetdash{}{0pt}%
\pgfpathmoveto{\pgfqpoint{1.525060in}{2.128745in}}%
\pgfpathcurveto{\pgfqpoint{1.536110in}{2.128745in}}{\pgfqpoint{1.546709in}{2.133136in}}{\pgfqpoint{1.554523in}{2.140949in}}%
\pgfpathcurveto{\pgfqpoint{1.562336in}{2.148763in}}{\pgfqpoint{1.566726in}{2.159362in}}{\pgfqpoint{1.566726in}{2.170412in}}%
\pgfpathcurveto{\pgfqpoint{1.566726in}{2.181462in}}{\pgfqpoint{1.562336in}{2.192061in}}{\pgfqpoint{1.554523in}{2.199875in}}%
\pgfpathcurveto{\pgfqpoint{1.546709in}{2.207688in}}{\pgfqpoint{1.536110in}{2.212079in}}{\pgfqpoint{1.525060in}{2.212079in}}%
\pgfpathcurveto{\pgfqpoint{1.514010in}{2.212079in}}{\pgfqpoint{1.503411in}{2.207688in}}{\pgfqpoint{1.495597in}{2.199875in}}%
\pgfpathcurveto{\pgfqpoint{1.487783in}{2.192061in}}{\pgfqpoint{1.483393in}{2.181462in}}{\pgfqpoint{1.483393in}{2.170412in}}%
\pgfpathcurveto{\pgfqpoint{1.483393in}{2.159362in}}{\pgfqpoint{1.487783in}{2.148763in}}{\pgfqpoint{1.495597in}{2.140949in}}%
\pgfpathcurveto{\pgfqpoint{1.503411in}{2.133136in}}{\pgfqpoint{1.514010in}{2.128745in}}{\pgfqpoint{1.525060in}{2.128745in}}%
\pgfpathlineto{\pgfqpoint{1.525060in}{2.128745in}}%
\pgfpathclose%
\pgfusepath{stroke,fill}%
\end{pgfscope}%
\begin{pgfscope}%
\pgfpathrectangle{\pgfqpoint{0.633874in}{0.569136in}}{\pgfqpoint{2.177280in}{2.201755in}}%
\pgfusepath{clip}%
\pgfsetbuttcap%
\pgfsetroundjoin%
\definecolor{currentfill}{rgb}{0.172549,0.627451,0.172549}%
\pgfsetfillcolor{currentfill}%
\pgfsetlinewidth{0.481800pt}%
\definecolor{currentstroke}{rgb}{1.000000,1.000000,1.000000}%
\pgfsetstrokecolor{currentstroke}%
\pgfsetdash{}{0pt}%
\pgfpathmoveto{\pgfqpoint{1.924447in}{2.462345in}}%
\pgfpathcurveto{\pgfqpoint{1.935497in}{2.462345in}}{\pgfqpoint{1.946096in}{2.466735in}}{\pgfqpoint{1.953910in}{2.474548in}}%
\pgfpathcurveto{\pgfqpoint{1.961723in}{2.482362in}}{\pgfqpoint{1.966113in}{2.492961in}}{\pgfqpoint{1.966113in}{2.504011in}}%
\pgfpathcurveto{\pgfqpoint{1.966113in}{2.515061in}}{\pgfqpoint{1.961723in}{2.525660in}}{\pgfqpoint{1.953910in}{2.533474in}}%
\pgfpathcurveto{\pgfqpoint{1.946096in}{2.541288in}}{\pgfqpoint{1.935497in}{2.545678in}}{\pgfqpoint{1.924447in}{2.545678in}}%
\pgfpathcurveto{\pgfqpoint{1.913397in}{2.545678in}}{\pgfqpoint{1.902798in}{2.541288in}}{\pgfqpoint{1.894984in}{2.533474in}}%
\pgfpathcurveto{\pgfqpoint{1.887170in}{2.525660in}}{\pgfqpoint{1.882780in}{2.515061in}}{\pgfqpoint{1.882780in}{2.504011in}}%
\pgfpathcurveto{\pgfqpoint{1.882780in}{2.492961in}}{\pgfqpoint{1.887170in}{2.482362in}}{\pgfqpoint{1.894984in}{2.474548in}}%
\pgfpathcurveto{\pgfqpoint{1.902798in}{2.466735in}}{\pgfqpoint{1.913397in}{2.462345in}}{\pgfqpoint{1.924447in}{2.462345in}}%
\pgfpathlineto{\pgfqpoint{1.924447in}{2.462345in}}%
\pgfpathclose%
\pgfusepath{stroke,fill}%
\end{pgfscope}%
\begin{pgfscope}%
\pgfpathrectangle{\pgfqpoint{0.633874in}{0.569136in}}{\pgfqpoint{2.177280in}{2.201755in}}%
\pgfusepath{clip}%
\pgfsetbuttcap%
\pgfsetroundjoin%
\definecolor{currentfill}{rgb}{0.172549,0.627451,0.172549}%
\pgfsetfillcolor{currentfill}%
\pgfsetlinewidth{0.481800pt}%
\definecolor{currentstroke}{rgb}{1.000000,1.000000,1.000000}%
\pgfsetstrokecolor{currentstroke}%
\pgfsetdash{}{0pt}%
\pgfpathmoveto{\pgfqpoint{1.884508in}{2.629144in}}%
\pgfpathcurveto{\pgfqpoint{1.895558in}{2.629144in}}{\pgfqpoint{1.906157in}{2.633534in}}{\pgfqpoint{1.913971in}{2.641348in}}%
\pgfpathcurveto{\pgfqpoint{1.921784in}{2.649162in}}{\pgfqpoint{1.926175in}{2.659761in}}{\pgfqpoint{1.926175in}{2.670811in}}%
\pgfpathcurveto{\pgfqpoint{1.926175in}{2.681861in}}{\pgfqpoint{1.921784in}{2.692460in}}{\pgfqpoint{1.913971in}{2.700274in}}%
\pgfpathcurveto{\pgfqpoint{1.906157in}{2.708087in}}{\pgfqpoint{1.895558in}{2.712478in}}{\pgfqpoint{1.884508in}{2.712478in}}%
\pgfpathcurveto{\pgfqpoint{1.873458in}{2.712478in}}{\pgfqpoint{1.862859in}{2.708087in}}{\pgfqpoint{1.855045in}{2.700274in}}%
\pgfpathcurveto{\pgfqpoint{1.847232in}{2.692460in}}{\pgfqpoint{1.842841in}{2.681861in}}{\pgfqpoint{1.842841in}{2.670811in}}%
\pgfpathcurveto{\pgfqpoint{1.842841in}{2.659761in}}{\pgfqpoint{1.847232in}{2.649162in}}{\pgfqpoint{1.855045in}{2.641348in}}%
\pgfpathcurveto{\pgfqpoint{1.862859in}{2.633534in}}{\pgfqpoint{1.873458in}{2.629144in}}{\pgfqpoint{1.884508in}{2.629144in}}%
\pgfpathlineto{\pgfqpoint{1.884508in}{2.629144in}}%
\pgfpathclose%
\pgfusepath{stroke,fill}%
\end{pgfscope}%
\begin{pgfscope}%
\pgfpathrectangle{\pgfqpoint{0.633874in}{0.569136in}}{\pgfqpoint{2.177280in}{2.201755in}}%
\pgfusepath{clip}%
\pgfsetbuttcap%
\pgfsetroundjoin%
\definecolor{currentfill}{rgb}{0.172549,0.627451,0.172549}%
\pgfsetfillcolor{currentfill}%
\pgfsetlinewidth{0.481800pt}%
\definecolor{currentstroke}{rgb}{1.000000,1.000000,1.000000}%
\pgfsetstrokecolor{currentstroke}%
\pgfsetdash{}{0pt}%
\pgfpathmoveto{\pgfqpoint{1.884508in}{2.462345in}}%
\pgfpathcurveto{\pgfqpoint{1.895558in}{2.462345in}}{\pgfqpoint{1.906157in}{2.466735in}}{\pgfqpoint{1.913971in}{2.474548in}}%
\pgfpathcurveto{\pgfqpoint{1.921784in}{2.482362in}}{\pgfqpoint{1.926175in}{2.492961in}}{\pgfqpoint{1.926175in}{2.504011in}}%
\pgfpathcurveto{\pgfqpoint{1.926175in}{2.515061in}}{\pgfqpoint{1.921784in}{2.525660in}}{\pgfqpoint{1.913971in}{2.533474in}}%
\pgfpathcurveto{\pgfqpoint{1.906157in}{2.541288in}}{\pgfqpoint{1.895558in}{2.545678in}}{\pgfqpoint{1.884508in}{2.545678in}}%
\pgfpathcurveto{\pgfqpoint{1.873458in}{2.545678in}}{\pgfqpoint{1.862859in}{2.541288in}}{\pgfqpoint{1.855045in}{2.533474in}}%
\pgfpathcurveto{\pgfqpoint{1.847232in}{2.525660in}}{\pgfqpoint{1.842841in}{2.515061in}}{\pgfqpoint{1.842841in}{2.504011in}}%
\pgfpathcurveto{\pgfqpoint{1.842841in}{2.492961in}}{\pgfqpoint{1.847232in}{2.482362in}}{\pgfqpoint{1.855045in}{2.474548in}}%
\pgfpathcurveto{\pgfqpoint{1.862859in}{2.466735in}}{\pgfqpoint{1.873458in}{2.462345in}}{\pgfqpoint{1.884508in}{2.462345in}}%
\pgfpathlineto{\pgfqpoint{1.884508in}{2.462345in}}%
\pgfpathclose%
\pgfusepath{stroke,fill}%
\end{pgfscope}%
\begin{pgfscope}%
\pgfpathrectangle{\pgfqpoint{0.633874in}{0.569136in}}{\pgfqpoint{2.177280in}{2.201755in}}%
\pgfusepath{clip}%
\pgfsetbuttcap%
\pgfsetroundjoin%
\definecolor{currentfill}{rgb}{0.172549,0.627451,0.172549}%
\pgfsetfillcolor{currentfill}%
\pgfsetlinewidth{0.481800pt}%
\definecolor{currentstroke}{rgb}{1.000000,1.000000,1.000000}%
\pgfsetstrokecolor{currentstroke}%
\pgfsetdash{}{0pt}%
\pgfpathmoveto{\pgfqpoint{1.724753in}{2.128745in}}%
\pgfpathcurveto{\pgfqpoint{1.735803in}{2.128745in}}{\pgfqpoint{1.746402in}{2.133136in}}{\pgfqpoint{1.754216in}{2.140949in}}%
\pgfpathcurveto{\pgfqpoint{1.762030in}{2.148763in}}{\pgfqpoint{1.766420in}{2.159362in}}{\pgfqpoint{1.766420in}{2.170412in}}%
\pgfpathcurveto{\pgfqpoint{1.766420in}{2.181462in}}{\pgfqpoint{1.762030in}{2.192061in}}{\pgfqpoint{1.754216in}{2.199875in}}%
\pgfpathcurveto{\pgfqpoint{1.746402in}{2.207688in}}{\pgfqpoint{1.735803in}{2.212079in}}{\pgfqpoint{1.724753in}{2.212079in}}%
\pgfpathcurveto{\pgfqpoint{1.713703in}{2.212079in}}{\pgfqpoint{1.703104in}{2.207688in}}{\pgfqpoint{1.695290in}{2.199875in}}%
\pgfpathcurveto{\pgfqpoint{1.687477in}{2.192061in}}{\pgfqpoint{1.683087in}{2.181462in}}{\pgfqpoint{1.683087in}{2.170412in}}%
\pgfpathcurveto{\pgfqpoint{1.683087in}{2.159362in}}{\pgfqpoint{1.687477in}{2.148763in}}{\pgfqpoint{1.695290in}{2.140949in}}%
\pgfpathcurveto{\pgfqpoint{1.703104in}{2.133136in}}{\pgfqpoint{1.713703in}{2.128745in}}{\pgfqpoint{1.724753in}{2.128745in}}%
\pgfpathlineto{\pgfqpoint{1.724753in}{2.128745in}}%
\pgfpathclose%
\pgfusepath{stroke,fill}%
\end{pgfscope}%
\begin{pgfscope}%
\pgfpathrectangle{\pgfqpoint{0.633874in}{0.569136in}}{\pgfqpoint{2.177280in}{2.201755in}}%
\pgfusepath{clip}%
\pgfsetbuttcap%
\pgfsetroundjoin%
\definecolor{currentfill}{rgb}{0.172549,0.627451,0.172549}%
\pgfsetfillcolor{currentfill}%
\pgfsetlinewidth{0.481800pt}%
\definecolor{currentstroke}{rgb}{1.000000,1.000000,1.000000}%
\pgfsetstrokecolor{currentstroke}%
\pgfsetdash{}{0pt}%
\pgfpathmoveto{\pgfqpoint{1.804631in}{2.212145in}}%
\pgfpathcurveto{\pgfqpoint{1.815681in}{2.212145in}}{\pgfqpoint{1.826280in}{2.216535in}}{\pgfqpoint{1.834093in}{2.224349in}}%
\pgfpathcurveto{\pgfqpoint{1.841907in}{2.232163in}}{\pgfqpoint{1.846297in}{2.242762in}}{\pgfqpoint{1.846297in}{2.253812in}}%
\pgfpathcurveto{\pgfqpoint{1.846297in}{2.264862in}}{\pgfqpoint{1.841907in}{2.275461in}}{\pgfqpoint{1.834093in}{2.283275in}}%
\pgfpathcurveto{\pgfqpoint{1.826280in}{2.291088in}}{\pgfqpoint{1.815681in}{2.295478in}}{\pgfqpoint{1.804631in}{2.295478in}}%
\pgfpathcurveto{\pgfqpoint{1.793581in}{2.295478in}}{\pgfqpoint{1.782981in}{2.291088in}}{\pgfqpoint{1.775168in}{2.283275in}}%
\pgfpathcurveto{\pgfqpoint{1.767354in}{2.275461in}}{\pgfqpoint{1.762964in}{2.264862in}}{\pgfqpoint{1.762964in}{2.253812in}}%
\pgfpathcurveto{\pgfqpoint{1.762964in}{2.242762in}}{\pgfqpoint{1.767354in}{2.232163in}}{\pgfqpoint{1.775168in}{2.224349in}}%
\pgfpathcurveto{\pgfqpoint{1.782981in}{2.216535in}}{\pgfqpoint{1.793581in}{2.212145in}}{\pgfqpoint{1.804631in}{2.212145in}}%
\pgfpathlineto{\pgfqpoint{1.804631in}{2.212145in}}%
\pgfpathclose%
\pgfusepath{stroke,fill}%
\end{pgfscope}%
\begin{pgfscope}%
\pgfpathrectangle{\pgfqpoint{0.633874in}{0.569136in}}{\pgfqpoint{2.177280in}{2.201755in}}%
\pgfusepath{clip}%
\pgfsetbuttcap%
\pgfsetroundjoin%
\definecolor{currentfill}{rgb}{0.172549,0.627451,0.172549}%
\pgfsetfillcolor{currentfill}%
\pgfsetlinewidth{0.481800pt}%
\definecolor{currentstroke}{rgb}{1.000000,1.000000,1.000000}%
\pgfsetstrokecolor{currentstroke}%
\pgfsetdash{}{0pt}%
\pgfpathmoveto{\pgfqpoint{1.684815in}{2.462345in}}%
\pgfpathcurveto{\pgfqpoint{1.695865in}{2.462345in}}{\pgfqpoint{1.706464in}{2.466735in}}{\pgfqpoint{1.714277in}{2.474548in}}%
\pgfpathcurveto{\pgfqpoint{1.722091in}{2.482362in}}{\pgfqpoint{1.726481in}{2.492961in}}{\pgfqpoint{1.726481in}{2.504011in}}%
\pgfpathcurveto{\pgfqpoint{1.726481in}{2.515061in}}{\pgfqpoint{1.722091in}{2.525660in}}{\pgfqpoint{1.714277in}{2.533474in}}%
\pgfpathcurveto{\pgfqpoint{1.706464in}{2.541288in}}{\pgfqpoint{1.695865in}{2.545678in}}{\pgfqpoint{1.684815in}{2.545678in}}%
\pgfpathcurveto{\pgfqpoint{1.673764in}{2.545678in}}{\pgfqpoint{1.663165in}{2.541288in}}{\pgfqpoint{1.655352in}{2.533474in}}%
\pgfpathcurveto{\pgfqpoint{1.647538in}{2.525660in}}{\pgfqpoint{1.643148in}{2.515061in}}{\pgfqpoint{1.643148in}{2.504011in}}%
\pgfpathcurveto{\pgfqpoint{1.643148in}{2.492961in}}{\pgfqpoint{1.647538in}{2.482362in}}{\pgfqpoint{1.655352in}{2.474548in}}%
\pgfpathcurveto{\pgfqpoint{1.663165in}{2.466735in}}{\pgfqpoint{1.673764in}{2.462345in}}{\pgfqpoint{1.684815in}{2.462345in}}%
\pgfpathlineto{\pgfqpoint{1.684815in}{2.462345in}}%
\pgfpathclose%
\pgfusepath{stroke,fill}%
\end{pgfscope}%
\begin{pgfscope}%
\pgfpathrectangle{\pgfqpoint{0.633874in}{0.569136in}}{\pgfqpoint{2.177280in}{2.201755in}}%
\pgfusepath{clip}%
\pgfsetbuttcap%
\pgfsetroundjoin%
\definecolor{currentfill}{rgb}{0.172549,0.627451,0.172549}%
\pgfsetfillcolor{currentfill}%
\pgfsetlinewidth{0.481800pt}%
\definecolor{currentstroke}{rgb}{1.000000,1.000000,1.000000}%
\pgfsetstrokecolor{currentstroke}%
\pgfsetdash{}{0pt}%
\pgfpathmoveto{\pgfqpoint{1.564998in}{2.045346in}}%
\pgfpathcurveto{\pgfqpoint{1.576049in}{2.045346in}}{\pgfqpoint{1.586648in}{2.049736in}}{\pgfqpoint{1.594461in}{2.057549in}}%
\pgfpathcurveto{\pgfqpoint{1.602275in}{2.065363in}}{\pgfqpoint{1.606665in}{2.075962in}}{\pgfqpoint{1.606665in}{2.087012in}}%
\pgfpathcurveto{\pgfqpoint{1.606665in}{2.098062in}}{\pgfqpoint{1.602275in}{2.108661in}}{\pgfqpoint{1.594461in}{2.116475in}}%
\pgfpathcurveto{\pgfqpoint{1.586648in}{2.124289in}}{\pgfqpoint{1.576049in}{2.128679in}}{\pgfqpoint{1.564998in}{2.128679in}}%
\pgfpathcurveto{\pgfqpoint{1.553948in}{2.128679in}}{\pgfqpoint{1.543349in}{2.124289in}}{\pgfqpoint{1.535536in}{2.116475in}}%
\pgfpathcurveto{\pgfqpoint{1.527722in}{2.108661in}}{\pgfqpoint{1.523332in}{2.098062in}}{\pgfqpoint{1.523332in}{2.087012in}}%
\pgfpathcurveto{\pgfqpoint{1.523332in}{2.075962in}}{\pgfqpoint{1.527722in}{2.065363in}}{\pgfqpoint{1.535536in}{2.057549in}}%
\pgfpathcurveto{\pgfqpoint{1.543349in}{2.049736in}}{\pgfqpoint{1.553948in}{2.045346in}}{\pgfqpoint{1.564998in}{2.045346in}}%
\pgfpathlineto{\pgfqpoint{1.564998in}{2.045346in}}%
\pgfpathclose%
\pgfusepath{stroke,fill}%
\end{pgfscope}%
\begin{pgfscope}%
\pgfpathrectangle{\pgfqpoint{0.633874in}{0.569136in}}{\pgfqpoint{2.177280in}{2.201755in}}%
\pgfusepath{clip}%
\pgfsetbuttcap%
\pgfsetroundjoin%
\definecolor{currentfill}{rgb}{0.121569,0.466667,0.705882}%
\pgfsetfillcolor{currentfill}%
\pgfsetlinewidth{1.003750pt}%
\definecolor{currentstroke}{rgb}{0.121569,0.466667,0.705882}%
\pgfsetstrokecolor{currentstroke}%
\pgfsetdash{}{0pt}%
\pgfsys@defobject{currentmarker}{\pgfqpoint{-0.041667in}{-0.041667in}}{\pgfqpoint{0.041667in}{0.041667in}}{%
\pgfpathmoveto{\pgfqpoint{0.000000in}{-0.041667in}}%
\pgfpathcurveto{\pgfqpoint{0.011050in}{-0.041667in}}{\pgfqpoint{0.021649in}{-0.037276in}}{\pgfqpoint{0.029463in}{-0.029463in}}%
\pgfpathcurveto{\pgfqpoint{0.037276in}{-0.021649in}}{\pgfqpoint{0.041667in}{-0.011050in}}{\pgfqpoint{0.041667in}{0.000000in}}%
\pgfpathcurveto{\pgfqpoint{0.041667in}{0.011050in}}{\pgfqpoint{0.037276in}{0.021649in}}{\pgfqpoint{0.029463in}{0.029463in}}%
\pgfpathcurveto{\pgfqpoint{0.021649in}{0.037276in}}{\pgfqpoint{0.011050in}{0.041667in}}{\pgfqpoint{0.000000in}{0.041667in}}%
\pgfpathcurveto{\pgfqpoint{-0.011050in}{0.041667in}}{\pgfqpoint{-0.021649in}{0.037276in}}{\pgfqpoint{-0.029463in}{0.029463in}}%
\pgfpathcurveto{\pgfqpoint{-0.037276in}{0.021649in}}{\pgfqpoint{-0.041667in}{0.011050in}}{\pgfqpoint{-0.041667in}{0.000000in}}%
\pgfpathcurveto{\pgfqpoint{-0.041667in}{-0.011050in}}{\pgfqpoint{-0.037276in}{-0.021649in}}{\pgfqpoint{-0.029463in}{-0.029463in}}%
\pgfpathcurveto{\pgfqpoint{-0.021649in}{-0.037276in}}{\pgfqpoint{-0.011050in}{-0.041667in}}{\pgfqpoint{0.000000in}{-0.041667in}}%
\pgfpathlineto{\pgfqpoint{0.000000in}{-0.041667in}}%
\pgfpathclose%
\pgfusepath{stroke,fill}%
}%
\end{pgfscope}%
\begin{pgfscope}%
\pgfpathrectangle{\pgfqpoint{0.633874in}{0.569136in}}{\pgfqpoint{2.177280in}{2.201755in}}%
\pgfusepath{clip}%
\pgfsetbuttcap%
\pgfsetroundjoin%
\definecolor{currentfill}{rgb}{1.000000,0.498039,0.054902}%
\pgfsetfillcolor{currentfill}%
\pgfsetlinewidth{1.003750pt}%
\definecolor{currentstroke}{rgb}{1.000000,0.498039,0.054902}%
\pgfsetstrokecolor{currentstroke}%
\pgfsetdash{}{0pt}%
\pgfsys@defobject{currentmarker}{\pgfqpoint{-0.041667in}{-0.041667in}}{\pgfqpoint{0.041667in}{0.041667in}}{%
\pgfpathmoveto{\pgfqpoint{0.000000in}{-0.041667in}}%
\pgfpathcurveto{\pgfqpoint{0.011050in}{-0.041667in}}{\pgfqpoint{0.021649in}{-0.037276in}}{\pgfqpoint{0.029463in}{-0.029463in}}%
\pgfpathcurveto{\pgfqpoint{0.037276in}{-0.021649in}}{\pgfqpoint{0.041667in}{-0.011050in}}{\pgfqpoint{0.041667in}{0.000000in}}%
\pgfpathcurveto{\pgfqpoint{0.041667in}{0.011050in}}{\pgfqpoint{0.037276in}{0.021649in}}{\pgfqpoint{0.029463in}{0.029463in}}%
\pgfpathcurveto{\pgfqpoint{0.021649in}{0.037276in}}{\pgfqpoint{0.011050in}{0.041667in}}{\pgfqpoint{0.000000in}{0.041667in}}%
\pgfpathcurveto{\pgfqpoint{-0.011050in}{0.041667in}}{\pgfqpoint{-0.021649in}{0.037276in}}{\pgfqpoint{-0.029463in}{0.029463in}}%
\pgfpathcurveto{\pgfqpoint{-0.037276in}{0.021649in}}{\pgfqpoint{-0.041667in}{0.011050in}}{\pgfqpoint{-0.041667in}{0.000000in}}%
\pgfpathcurveto{\pgfqpoint{-0.041667in}{-0.011050in}}{\pgfqpoint{-0.037276in}{-0.021649in}}{\pgfqpoint{-0.029463in}{-0.029463in}}%
\pgfpathcurveto{\pgfqpoint{-0.021649in}{-0.037276in}}{\pgfqpoint{-0.011050in}{-0.041667in}}{\pgfqpoint{0.000000in}{-0.041667in}}%
\pgfpathlineto{\pgfqpoint{0.000000in}{-0.041667in}}%
\pgfpathclose%
\pgfusepath{stroke,fill}%
}%
\end{pgfscope}%
\begin{pgfscope}%
\pgfpathrectangle{\pgfqpoint{0.633874in}{0.569136in}}{\pgfqpoint{2.177280in}{2.201755in}}%
\pgfusepath{clip}%
\pgfsetbuttcap%
\pgfsetroundjoin%
\definecolor{currentfill}{rgb}{0.172549,0.627451,0.172549}%
\pgfsetfillcolor{currentfill}%
\pgfsetlinewidth{1.003750pt}%
\definecolor{currentstroke}{rgb}{0.172549,0.627451,0.172549}%
\pgfsetstrokecolor{currentstroke}%
\pgfsetdash{}{0pt}%
\pgfsys@defobject{currentmarker}{\pgfqpoint{-0.041667in}{-0.041667in}}{\pgfqpoint{0.041667in}{0.041667in}}{%
\pgfpathmoveto{\pgfqpoint{0.000000in}{-0.041667in}}%
\pgfpathcurveto{\pgfqpoint{0.011050in}{-0.041667in}}{\pgfqpoint{0.021649in}{-0.037276in}}{\pgfqpoint{0.029463in}{-0.029463in}}%
\pgfpathcurveto{\pgfqpoint{0.037276in}{-0.021649in}}{\pgfqpoint{0.041667in}{-0.011050in}}{\pgfqpoint{0.041667in}{0.000000in}}%
\pgfpathcurveto{\pgfqpoint{0.041667in}{0.011050in}}{\pgfqpoint{0.037276in}{0.021649in}}{\pgfqpoint{0.029463in}{0.029463in}}%
\pgfpathcurveto{\pgfqpoint{0.021649in}{0.037276in}}{\pgfqpoint{0.011050in}{0.041667in}}{\pgfqpoint{0.000000in}{0.041667in}}%
\pgfpathcurveto{\pgfqpoint{-0.011050in}{0.041667in}}{\pgfqpoint{-0.021649in}{0.037276in}}{\pgfqpoint{-0.029463in}{0.029463in}}%
\pgfpathcurveto{\pgfqpoint{-0.037276in}{0.021649in}}{\pgfqpoint{-0.041667in}{0.011050in}}{\pgfqpoint{-0.041667in}{0.000000in}}%
\pgfpathcurveto{\pgfqpoint{-0.041667in}{-0.011050in}}{\pgfqpoint{-0.037276in}{-0.021649in}}{\pgfqpoint{-0.029463in}{-0.029463in}}%
\pgfpathcurveto{\pgfqpoint{-0.021649in}{-0.037276in}}{\pgfqpoint{-0.011050in}{-0.041667in}}{\pgfqpoint{0.000000in}{-0.041667in}}%
\pgfpathlineto{\pgfqpoint{0.000000in}{-0.041667in}}%
\pgfpathclose%
\pgfusepath{stroke,fill}%
}%
\end{pgfscope}%
\begin{pgfscope}%
\pgfsetbuttcap%
\pgfsetroundjoin%
\definecolor{currentfill}{rgb}{0.000000,0.000000,0.000000}%
\pgfsetfillcolor{currentfill}%
\pgfsetlinewidth{0.803000pt}%
\definecolor{currentstroke}{rgb}{0.000000,0.000000,0.000000}%
\pgfsetstrokecolor{currentstroke}%
\pgfsetdash{}{0pt}%
\pgfsys@defobject{currentmarker}{\pgfqpoint{0.000000in}{-0.048611in}}{\pgfqpoint{0.000000in}{0.000000in}}{%
\pgfpathmoveto{\pgfqpoint{0.000000in}{0.000000in}}%
\pgfpathlineto{\pgfqpoint{0.000000in}{-0.048611in}}%
\pgfusepath{stroke,fill}%
}%
\begin{pgfscope}%
\pgfsys@transformshift{0.806163in}{0.569136in}%
\pgfsys@useobject{currentmarker}{}%
\end{pgfscope}%
\end{pgfscope}%
\begin{pgfscope}%
\definecolor{textcolor}{rgb}{0.000000,0.000000,0.000000}%
\pgfsetstrokecolor{textcolor}%
\pgfsetfillcolor{textcolor}%
\pgftext[x=0.806163in,y=0.471913in,,top]{\color{textcolor}\rmfamily\fontsize{10.000000}{12.000000}\selectfont \(\displaystyle {4}\)}%
\end{pgfscope}%
\begin{pgfscope}%
\pgfsetbuttcap%
\pgfsetroundjoin%
\definecolor{currentfill}{rgb}{0.000000,0.000000,0.000000}%
\pgfsetfillcolor{currentfill}%
\pgfsetlinewidth{0.803000pt}%
\definecolor{currentstroke}{rgb}{0.000000,0.000000,0.000000}%
\pgfsetstrokecolor{currentstroke}%
\pgfsetdash{}{0pt}%
\pgfsys@defobject{currentmarker}{\pgfqpoint{0.000000in}{-0.048611in}}{\pgfqpoint{0.000000in}{0.000000in}}{%
\pgfpathmoveto{\pgfqpoint{0.000000in}{0.000000in}}%
\pgfpathlineto{\pgfqpoint{0.000000in}{-0.048611in}}%
\pgfusepath{stroke,fill}%
}%
\begin{pgfscope}%
\pgfsys@transformshift{1.604937in}{0.569136in}%
\pgfsys@useobject{currentmarker}{}%
\end{pgfscope}%
\end{pgfscope}%
\begin{pgfscope}%
\definecolor{textcolor}{rgb}{0.000000,0.000000,0.000000}%
\pgfsetstrokecolor{textcolor}%
\pgfsetfillcolor{textcolor}%
\pgftext[x=1.604937in,y=0.471913in,,top]{\color{textcolor}\rmfamily\fontsize{10.000000}{12.000000}\selectfont \(\displaystyle {6}\)}%
\end{pgfscope}%
\begin{pgfscope}%
\pgfsetbuttcap%
\pgfsetroundjoin%
\definecolor{currentfill}{rgb}{0.000000,0.000000,0.000000}%
\pgfsetfillcolor{currentfill}%
\pgfsetlinewidth{0.803000pt}%
\definecolor{currentstroke}{rgb}{0.000000,0.000000,0.000000}%
\pgfsetstrokecolor{currentstroke}%
\pgfsetdash{}{0pt}%
\pgfsys@defobject{currentmarker}{\pgfqpoint{0.000000in}{-0.048611in}}{\pgfqpoint{0.000000in}{0.000000in}}{%
\pgfpathmoveto{\pgfqpoint{0.000000in}{0.000000in}}%
\pgfpathlineto{\pgfqpoint{0.000000in}{-0.048611in}}%
\pgfusepath{stroke,fill}%
}%
\begin{pgfscope}%
\pgfsys@transformshift{2.403711in}{0.569136in}%
\pgfsys@useobject{currentmarker}{}%
\end{pgfscope}%
\end{pgfscope}%
\begin{pgfscope}%
\definecolor{textcolor}{rgb}{0.000000,0.000000,0.000000}%
\pgfsetstrokecolor{textcolor}%
\pgfsetfillcolor{textcolor}%
\pgftext[x=2.403711in,y=0.471913in,,top]{\color{textcolor}\rmfamily\fontsize{10.000000}{12.000000}\selectfont \(\displaystyle {8}\)}%
\end{pgfscope}%
\begin{pgfscope}%
\definecolor{textcolor}{rgb}{0.000000,0.000000,0.000000}%
\pgfsetstrokecolor{textcolor}%
\pgfsetfillcolor{textcolor}%
\pgftext[x=1.722514in,y=0.292901in,,top]{\color{textcolor}\rmfamily\fontsize{10.000000}{12.000000}\selectfont sepal\_length}%
\end{pgfscope}%
\begin{pgfscope}%
\pgfsetbuttcap%
\pgfsetroundjoin%
\definecolor{currentfill}{rgb}{0.000000,0.000000,0.000000}%
\pgfsetfillcolor{currentfill}%
\pgfsetlinewidth{0.803000pt}%
\definecolor{currentstroke}{rgb}{0.000000,0.000000,0.000000}%
\pgfsetstrokecolor{currentstroke}%
\pgfsetdash{}{0pt}%
\pgfsys@defobject{currentmarker}{\pgfqpoint{-0.048611in}{0.000000in}}{\pgfqpoint{-0.000000in}{0.000000in}}{%
\pgfpathmoveto{\pgfqpoint{-0.000000in}{0.000000in}}%
\pgfpathlineto{\pgfqpoint{-0.048611in}{0.000000in}}%
\pgfusepath{stroke,fill}%
}%
\begin{pgfscope}%
\pgfsys@transformshift{0.633874in}{0.585816in}%
\pgfsys@useobject{currentmarker}{}%
\end{pgfscope}%
\end{pgfscope}%
\begin{pgfscope}%
\definecolor{textcolor}{rgb}{0.000000,0.000000,0.000000}%
\pgfsetstrokecolor{textcolor}%
\pgfsetfillcolor{textcolor}%
\pgftext[x=0.359182in, y=0.537590in, left, base]{\color{textcolor}\rmfamily\fontsize{10.000000}{12.000000}\selectfont \(\displaystyle {0.0}\)}%
\end{pgfscope}%
\begin{pgfscope}%
\pgfsetbuttcap%
\pgfsetroundjoin%
\definecolor{currentfill}{rgb}{0.000000,0.000000,0.000000}%
\pgfsetfillcolor{currentfill}%
\pgfsetlinewidth{0.803000pt}%
\definecolor{currentstroke}{rgb}{0.000000,0.000000,0.000000}%
\pgfsetstrokecolor{currentstroke}%
\pgfsetdash{}{0pt}%
\pgfsys@defobject{currentmarker}{\pgfqpoint{-0.048611in}{0.000000in}}{\pgfqpoint{-0.000000in}{0.000000in}}{%
\pgfpathmoveto{\pgfqpoint{-0.000000in}{0.000000in}}%
\pgfpathlineto{\pgfqpoint{-0.048611in}{0.000000in}}%
\pgfusepath{stroke,fill}%
}%
\begin{pgfscope}%
\pgfsys@transformshift{0.633874in}{1.002815in}%
\pgfsys@useobject{currentmarker}{}%
\end{pgfscope}%
\end{pgfscope}%
\begin{pgfscope}%
\definecolor{textcolor}{rgb}{0.000000,0.000000,0.000000}%
\pgfsetstrokecolor{textcolor}%
\pgfsetfillcolor{textcolor}%
\pgftext[x=0.359182in, y=0.954589in, left, base]{\color{textcolor}\rmfamily\fontsize{10.000000}{12.000000}\selectfont \(\displaystyle {0.5}\)}%
\end{pgfscope}%
\begin{pgfscope}%
\pgfsetbuttcap%
\pgfsetroundjoin%
\definecolor{currentfill}{rgb}{0.000000,0.000000,0.000000}%
\pgfsetfillcolor{currentfill}%
\pgfsetlinewidth{0.803000pt}%
\definecolor{currentstroke}{rgb}{0.000000,0.000000,0.000000}%
\pgfsetstrokecolor{currentstroke}%
\pgfsetdash{}{0pt}%
\pgfsys@defobject{currentmarker}{\pgfqpoint{-0.048611in}{0.000000in}}{\pgfqpoint{-0.000000in}{0.000000in}}{%
\pgfpathmoveto{\pgfqpoint{-0.000000in}{0.000000in}}%
\pgfpathlineto{\pgfqpoint{-0.048611in}{0.000000in}}%
\pgfusepath{stroke,fill}%
}%
\begin{pgfscope}%
\pgfsys@transformshift{0.633874in}{1.419814in}%
\pgfsys@useobject{currentmarker}{}%
\end{pgfscope}%
\end{pgfscope}%
\begin{pgfscope}%
\definecolor{textcolor}{rgb}{0.000000,0.000000,0.000000}%
\pgfsetstrokecolor{textcolor}%
\pgfsetfillcolor{textcolor}%
\pgftext[x=0.359182in, y=1.371588in, left, base]{\color{textcolor}\rmfamily\fontsize{10.000000}{12.000000}\selectfont \(\displaystyle {1.0}\)}%
\end{pgfscope}%
\begin{pgfscope}%
\pgfsetbuttcap%
\pgfsetroundjoin%
\definecolor{currentfill}{rgb}{0.000000,0.000000,0.000000}%
\pgfsetfillcolor{currentfill}%
\pgfsetlinewidth{0.803000pt}%
\definecolor{currentstroke}{rgb}{0.000000,0.000000,0.000000}%
\pgfsetstrokecolor{currentstroke}%
\pgfsetdash{}{0pt}%
\pgfsys@defobject{currentmarker}{\pgfqpoint{-0.048611in}{0.000000in}}{\pgfqpoint{-0.000000in}{0.000000in}}{%
\pgfpathmoveto{\pgfqpoint{-0.000000in}{0.000000in}}%
\pgfpathlineto{\pgfqpoint{-0.048611in}{0.000000in}}%
\pgfusepath{stroke,fill}%
}%
\begin{pgfscope}%
\pgfsys@transformshift{0.633874in}{1.836813in}%
\pgfsys@useobject{currentmarker}{}%
\end{pgfscope}%
\end{pgfscope}%
\begin{pgfscope}%
\definecolor{textcolor}{rgb}{0.000000,0.000000,0.000000}%
\pgfsetstrokecolor{textcolor}%
\pgfsetfillcolor{textcolor}%
\pgftext[x=0.359182in, y=1.788587in, left, base]{\color{textcolor}\rmfamily\fontsize{10.000000}{12.000000}\selectfont \(\displaystyle {1.5}\)}%
\end{pgfscope}%
\begin{pgfscope}%
\pgfsetbuttcap%
\pgfsetroundjoin%
\definecolor{currentfill}{rgb}{0.000000,0.000000,0.000000}%
\pgfsetfillcolor{currentfill}%
\pgfsetlinewidth{0.803000pt}%
\definecolor{currentstroke}{rgb}{0.000000,0.000000,0.000000}%
\pgfsetstrokecolor{currentstroke}%
\pgfsetdash{}{0pt}%
\pgfsys@defobject{currentmarker}{\pgfqpoint{-0.048611in}{0.000000in}}{\pgfqpoint{-0.000000in}{0.000000in}}{%
\pgfpathmoveto{\pgfqpoint{-0.000000in}{0.000000in}}%
\pgfpathlineto{\pgfqpoint{-0.048611in}{0.000000in}}%
\pgfusepath{stroke,fill}%
}%
\begin{pgfscope}%
\pgfsys@transformshift{0.633874in}{2.253812in}%
\pgfsys@useobject{currentmarker}{}%
\end{pgfscope}%
\end{pgfscope}%
\begin{pgfscope}%
\definecolor{textcolor}{rgb}{0.000000,0.000000,0.000000}%
\pgfsetstrokecolor{textcolor}%
\pgfsetfillcolor{textcolor}%
\pgftext[x=0.359182in, y=2.205587in, left, base]{\color{textcolor}\rmfamily\fontsize{10.000000}{12.000000}\selectfont \(\displaystyle {2.0}\)}%
\end{pgfscope}%
\begin{pgfscope}%
\pgfsetbuttcap%
\pgfsetroundjoin%
\definecolor{currentfill}{rgb}{0.000000,0.000000,0.000000}%
\pgfsetfillcolor{currentfill}%
\pgfsetlinewidth{0.803000pt}%
\definecolor{currentstroke}{rgb}{0.000000,0.000000,0.000000}%
\pgfsetstrokecolor{currentstroke}%
\pgfsetdash{}{0pt}%
\pgfsys@defobject{currentmarker}{\pgfqpoint{-0.048611in}{0.000000in}}{\pgfqpoint{-0.000000in}{0.000000in}}{%
\pgfpathmoveto{\pgfqpoint{-0.000000in}{0.000000in}}%
\pgfpathlineto{\pgfqpoint{-0.048611in}{0.000000in}}%
\pgfusepath{stroke,fill}%
}%
\begin{pgfscope}%
\pgfsys@transformshift{0.633874in}{2.670811in}%
\pgfsys@useobject{currentmarker}{}%
\end{pgfscope}%
\end{pgfscope}%
\begin{pgfscope}%
\definecolor{textcolor}{rgb}{0.000000,0.000000,0.000000}%
\pgfsetstrokecolor{textcolor}%
\pgfsetfillcolor{textcolor}%
\pgftext[x=0.359182in, y=2.622586in, left, base]{\color{textcolor}\rmfamily\fontsize{10.000000}{12.000000}\selectfont \(\displaystyle {2.5}\)}%
\end{pgfscope}%
\begin{pgfscope}%
\definecolor{textcolor}{rgb}{0.000000,0.000000,0.000000}%
\pgfsetstrokecolor{textcolor}%
\pgfsetfillcolor{textcolor}%
\pgftext[x=0.303626in,y=1.670013in,,bottom,rotate=90.000000]{\color{textcolor}\rmfamily\fontsize{10.000000}{12.000000}\selectfont petal\_width}%
\end{pgfscope}%
\begin{pgfscope}%
\pgfsetrectcap%
\pgfsetmiterjoin%
\pgfsetlinewidth{0.803000pt}%
\definecolor{currentstroke}{rgb}{0.000000,0.000000,0.000000}%
\pgfsetstrokecolor{currentstroke}%
\pgfsetdash{}{0pt}%
\pgfpathmoveto{\pgfqpoint{0.633874in}{0.569136in}}%
\pgfpathlineto{\pgfqpoint{0.633874in}{2.770891in}}%
\pgfusepath{stroke}%
\end{pgfscope}%
\begin{pgfscope}%
\pgfsetrectcap%
\pgfsetmiterjoin%
\pgfsetlinewidth{0.803000pt}%
\definecolor{currentstroke}{rgb}{0.000000,0.000000,0.000000}%
\pgfsetstrokecolor{currentstroke}%
\pgfsetdash{}{0pt}%
\pgfpathmoveto{\pgfqpoint{0.633874in}{0.569136in}}%
\pgfpathlineto{\pgfqpoint{2.811154in}{0.569136in}}%
\pgfusepath{stroke}%
\end{pgfscope}%
\begin{pgfscope}%
\pgfsetbuttcap%
\pgfsetmiterjoin%
\definecolor{currentfill}{rgb}{1.000000,1.000000,1.000000}%
\pgfsetfillcolor{currentfill}%
\pgfsetlinewidth{0.000000pt}%
\definecolor{currentstroke}{rgb}{0.000000,0.000000,0.000000}%
\pgfsetstrokecolor{currentstroke}%
\pgfsetstrokeopacity{0.000000}%
\pgfsetdash{}{0pt}%
\pgfpathmoveto{\pgfqpoint{2.963410in}{0.569136in}}%
\pgfpathlineto{\pgfqpoint{5.140690in}{0.569136in}}%
\pgfpathlineto{\pgfqpoint{5.140690in}{2.770891in}}%
\pgfpathlineto{\pgfqpoint{2.963410in}{2.770891in}}%
\pgfpathlineto{\pgfqpoint{2.963410in}{0.569136in}}%
\pgfpathclose%
\pgfusepath{fill}%
\end{pgfscope}%
\begin{pgfscope}%
\pgfpathrectangle{\pgfqpoint{2.963410in}{0.569136in}}{\pgfqpoint{2.177280in}{2.201755in}}%
\pgfusepath{clip}%
\pgfsetbuttcap%
\pgfsetroundjoin%
\definecolor{currentfill}{rgb}{0.121569,0.466667,0.705882}%
\pgfsetfillcolor{currentfill}%
\pgfsetlinewidth{0.481800pt}%
\definecolor{currentstroke}{rgb}{1.000000,1.000000,1.000000}%
\pgfsetstrokecolor{currentstroke}%
\pgfsetdash{}{0pt}%
\pgfpathmoveto{\pgfqpoint{4.202828in}{0.710948in}}%
\pgfpathcurveto{\pgfqpoint{4.213878in}{0.710948in}}{\pgfqpoint{4.224477in}{0.715339in}}{\pgfqpoint{4.232291in}{0.723152in}}%
\pgfpathcurveto{\pgfqpoint{4.240104in}{0.730966in}}{\pgfqpoint{4.244495in}{0.741565in}}{\pgfqpoint{4.244495in}{0.752615in}}%
\pgfpathcurveto{\pgfqpoint{4.244495in}{0.763665in}}{\pgfqpoint{4.240104in}{0.774264in}}{\pgfqpoint{4.232291in}{0.782078in}}%
\pgfpathcurveto{\pgfqpoint{4.224477in}{0.789892in}}{\pgfqpoint{4.213878in}{0.794282in}}{\pgfqpoint{4.202828in}{0.794282in}}%
\pgfpathcurveto{\pgfqpoint{4.191778in}{0.794282in}}{\pgfqpoint{4.181179in}{0.789892in}}{\pgfqpoint{4.173365in}{0.782078in}}%
\pgfpathcurveto{\pgfqpoint{4.165552in}{0.774264in}}{\pgfqpoint{4.161161in}{0.763665in}}{\pgfqpoint{4.161161in}{0.752615in}}%
\pgfpathcurveto{\pgfqpoint{4.161161in}{0.741565in}}{\pgfqpoint{4.165552in}{0.730966in}}{\pgfqpoint{4.173365in}{0.723152in}}%
\pgfpathcurveto{\pgfqpoint{4.181179in}{0.715339in}}{\pgfqpoint{4.191778in}{0.710948in}}{\pgfqpoint{4.202828in}{0.710948in}}%
\pgfpathlineto{\pgfqpoint{4.202828in}{0.710948in}}%
\pgfpathclose%
\pgfusepath{stroke,fill}%
\end{pgfscope}%
\begin{pgfscope}%
\pgfpathrectangle{\pgfqpoint{2.963410in}{0.569136in}}{\pgfqpoint{2.177280in}{2.201755in}}%
\pgfusepath{clip}%
\pgfsetbuttcap%
\pgfsetroundjoin%
\definecolor{currentfill}{rgb}{0.121569,0.466667,0.705882}%
\pgfsetfillcolor{currentfill}%
\pgfsetlinewidth{0.481800pt}%
\definecolor{currentstroke}{rgb}{1.000000,1.000000,1.000000}%
\pgfsetstrokecolor{currentstroke}%
\pgfsetdash{}{0pt}%
\pgfpathmoveto{\pgfqpoint{3.907452in}{0.710948in}}%
\pgfpathcurveto{\pgfqpoint{3.918502in}{0.710948in}}{\pgfqpoint{3.929101in}{0.715339in}}{\pgfqpoint{3.936915in}{0.723152in}}%
\pgfpathcurveto{\pgfqpoint{3.944728in}{0.730966in}}{\pgfqpoint{3.949118in}{0.741565in}}{\pgfqpoint{3.949118in}{0.752615in}}%
\pgfpathcurveto{\pgfqpoint{3.949118in}{0.763665in}}{\pgfqpoint{3.944728in}{0.774264in}}{\pgfqpoint{3.936915in}{0.782078in}}%
\pgfpathcurveto{\pgfqpoint{3.929101in}{0.789892in}}{\pgfqpoint{3.918502in}{0.794282in}}{\pgfqpoint{3.907452in}{0.794282in}}%
\pgfpathcurveto{\pgfqpoint{3.896402in}{0.794282in}}{\pgfqpoint{3.885803in}{0.789892in}}{\pgfqpoint{3.877989in}{0.782078in}}%
\pgfpathcurveto{\pgfqpoint{3.870175in}{0.774264in}}{\pgfqpoint{3.865785in}{0.763665in}}{\pgfqpoint{3.865785in}{0.752615in}}%
\pgfpathcurveto{\pgfqpoint{3.865785in}{0.741565in}}{\pgfqpoint{3.870175in}{0.730966in}}{\pgfqpoint{3.877989in}{0.723152in}}%
\pgfpathcurveto{\pgfqpoint{3.885803in}{0.715339in}}{\pgfqpoint{3.896402in}{0.710948in}}{\pgfqpoint{3.907452in}{0.710948in}}%
\pgfpathlineto{\pgfqpoint{3.907452in}{0.710948in}}%
\pgfpathclose%
\pgfusepath{stroke,fill}%
\end{pgfscope}%
\begin{pgfscope}%
\pgfpathrectangle{\pgfqpoint{2.963410in}{0.569136in}}{\pgfqpoint{2.177280in}{2.201755in}}%
\pgfusepath{clip}%
\pgfsetbuttcap%
\pgfsetroundjoin%
\definecolor{currentfill}{rgb}{0.121569,0.466667,0.705882}%
\pgfsetfillcolor{currentfill}%
\pgfsetlinewidth{0.481800pt}%
\definecolor{currentstroke}{rgb}{1.000000,1.000000,1.000000}%
\pgfsetstrokecolor{currentstroke}%
\pgfsetdash{}{0pt}%
\pgfpathmoveto{\pgfqpoint{4.025602in}{0.710948in}}%
\pgfpathcurveto{\pgfqpoint{4.036652in}{0.710948in}}{\pgfqpoint{4.047251in}{0.715339in}}{\pgfqpoint{4.055065in}{0.723152in}}%
\pgfpathcurveto{\pgfqpoint{4.062879in}{0.730966in}}{\pgfqpoint{4.067269in}{0.741565in}}{\pgfqpoint{4.067269in}{0.752615in}}%
\pgfpathcurveto{\pgfqpoint{4.067269in}{0.763665in}}{\pgfqpoint{4.062879in}{0.774264in}}{\pgfqpoint{4.055065in}{0.782078in}}%
\pgfpathcurveto{\pgfqpoint{4.047251in}{0.789892in}}{\pgfqpoint{4.036652in}{0.794282in}}{\pgfqpoint{4.025602in}{0.794282in}}%
\pgfpathcurveto{\pgfqpoint{4.014552in}{0.794282in}}{\pgfqpoint{4.003953in}{0.789892in}}{\pgfqpoint{3.996139in}{0.782078in}}%
\pgfpathcurveto{\pgfqpoint{3.988326in}{0.774264in}}{\pgfqpoint{3.983936in}{0.763665in}}{\pgfqpoint{3.983936in}{0.752615in}}%
\pgfpathcurveto{\pgfqpoint{3.983936in}{0.741565in}}{\pgfqpoint{3.988326in}{0.730966in}}{\pgfqpoint{3.996139in}{0.723152in}}%
\pgfpathcurveto{\pgfqpoint{4.003953in}{0.715339in}}{\pgfqpoint{4.014552in}{0.710948in}}{\pgfqpoint{4.025602in}{0.710948in}}%
\pgfpathlineto{\pgfqpoint{4.025602in}{0.710948in}}%
\pgfpathclose%
\pgfusepath{stroke,fill}%
\end{pgfscope}%
\begin{pgfscope}%
\pgfpathrectangle{\pgfqpoint{2.963410in}{0.569136in}}{\pgfqpoint{2.177280in}{2.201755in}}%
\pgfusepath{clip}%
\pgfsetbuttcap%
\pgfsetroundjoin%
\definecolor{currentfill}{rgb}{0.121569,0.466667,0.705882}%
\pgfsetfillcolor{currentfill}%
\pgfsetlinewidth{0.481800pt}%
\definecolor{currentstroke}{rgb}{1.000000,1.000000,1.000000}%
\pgfsetstrokecolor{currentstroke}%
\pgfsetdash{}{0pt}%
\pgfpathmoveto{\pgfqpoint{3.966527in}{0.710948in}}%
\pgfpathcurveto{\pgfqpoint{3.977577in}{0.710948in}}{\pgfqpoint{3.988176in}{0.715339in}}{\pgfqpoint{3.995990in}{0.723152in}}%
\pgfpathcurveto{\pgfqpoint{4.003803in}{0.730966in}}{\pgfqpoint{4.008194in}{0.741565in}}{\pgfqpoint{4.008194in}{0.752615in}}%
\pgfpathcurveto{\pgfqpoint{4.008194in}{0.763665in}}{\pgfqpoint{4.003803in}{0.774264in}}{\pgfqpoint{3.995990in}{0.782078in}}%
\pgfpathcurveto{\pgfqpoint{3.988176in}{0.789892in}}{\pgfqpoint{3.977577in}{0.794282in}}{\pgfqpoint{3.966527in}{0.794282in}}%
\pgfpathcurveto{\pgfqpoint{3.955477in}{0.794282in}}{\pgfqpoint{3.944878in}{0.789892in}}{\pgfqpoint{3.937064in}{0.782078in}}%
\pgfpathcurveto{\pgfqpoint{3.929251in}{0.774264in}}{\pgfqpoint{3.924860in}{0.763665in}}{\pgfqpoint{3.924860in}{0.752615in}}%
\pgfpathcurveto{\pgfqpoint{3.924860in}{0.741565in}}{\pgfqpoint{3.929251in}{0.730966in}}{\pgfqpoint{3.937064in}{0.723152in}}%
\pgfpathcurveto{\pgfqpoint{3.944878in}{0.715339in}}{\pgfqpoint{3.955477in}{0.710948in}}{\pgfqpoint{3.966527in}{0.710948in}}%
\pgfpathlineto{\pgfqpoint{3.966527in}{0.710948in}}%
\pgfpathclose%
\pgfusepath{stroke,fill}%
\end{pgfscope}%
\begin{pgfscope}%
\pgfpathrectangle{\pgfqpoint{2.963410in}{0.569136in}}{\pgfqpoint{2.177280in}{2.201755in}}%
\pgfusepath{clip}%
\pgfsetbuttcap%
\pgfsetroundjoin%
\definecolor{currentfill}{rgb}{0.121569,0.466667,0.705882}%
\pgfsetfillcolor{currentfill}%
\pgfsetlinewidth{0.481800pt}%
\definecolor{currentstroke}{rgb}{1.000000,1.000000,1.000000}%
\pgfsetstrokecolor{currentstroke}%
\pgfsetdash{}{0pt}%
\pgfpathmoveto{\pgfqpoint{4.261903in}{0.710948in}}%
\pgfpathcurveto{\pgfqpoint{4.272953in}{0.710948in}}{\pgfqpoint{4.283552in}{0.715339in}}{\pgfqpoint{4.291366in}{0.723152in}}%
\pgfpathcurveto{\pgfqpoint{4.299180in}{0.730966in}}{\pgfqpoint{4.303570in}{0.741565in}}{\pgfqpoint{4.303570in}{0.752615in}}%
\pgfpathcurveto{\pgfqpoint{4.303570in}{0.763665in}}{\pgfqpoint{4.299180in}{0.774264in}}{\pgfqpoint{4.291366in}{0.782078in}}%
\pgfpathcurveto{\pgfqpoint{4.283552in}{0.789892in}}{\pgfqpoint{4.272953in}{0.794282in}}{\pgfqpoint{4.261903in}{0.794282in}}%
\pgfpathcurveto{\pgfqpoint{4.250853in}{0.794282in}}{\pgfqpoint{4.240254in}{0.789892in}}{\pgfqpoint{4.232441in}{0.782078in}}%
\pgfpathcurveto{\pgfqpoint{4.224627in}{0.774264in}}{\pgfqpoint{4.220237in}{0.763665in}}{\pgfqpoint{4.220237in}{0.752615in}}%
\pgfpathcurveto{\pgfqpoint{4.220237in}{0.741565in}}{\pgfqpoint{4.224627in}{0.730966in}}{\pgfqpoint{4.232441in}{0.723152in}}%
\pgfpathcurveto{\pgfqpoint{4.240254in}{0.715339in}}{\pgfqpoint{4.250853in}{0.710948in}}{\pgfqpoint{4.261903in}{0.710948in}}%
\pgfpathlineto{\pgfqpoint{4.261903in}{0.710948in}}%
\pgfpathclose%
\pgfusepath{stroke,fill}%
\end{pgfscope}%
\begin{pgfscope}%
\pgfpathrectangle{\pgfqpoint{2.963410in}{0.569136in}}{\pgfqpoint{2.177280in}{2.201755in}}%
\pgfusepath{clip}%
\pgfsetbuttcap%
\pgfsetroundjoin%
\definecolor{currentfill}{rgb}{0.121569,0.466667,0.705882}%
\pgfsetfillcolor{currentfill}%
\pgfsetlinewidth{0.481800pt}%
\definecolor{currentstroke}{rgb}{1.000000,1.000000,1.000000}%
\pgfsetstrokecolor{currentstroke}%
\pgfsetdash{}{0pt}%
\pgfpathmoveto{\pgfqpoint{4.439129in}{0.877748in}}%
\pgfpathcurveto{\pgfqpoint{4.450179in}{0.877748in}}{\pgfqpoint{4.460778in}{0.882138in}}{\pgfqpoint{4.468592in}{0.889952in}}%
\pgfpathcurveto{\pgfqpoint{4.476406in}{0.897766in}}{\pgfqpoint{4.480796in}{0.908365in}}{\pgfqpoint{4.480796in}{0.919415in}}%
\pgfpathcurveto{\pgfqpoint{4.480796in}{0.930465in}}{\pgfqpoint{4.476406in}{0.941064in}}{\pgfqpoint{4.468592in}{0.948878in}}%
\pgfpathcurveto{\pgfqpoint{4.460778in}{0.956691in}}{\pgfqpoint{4.450179in}{0.961081in}}{\pgfqpoint{4.439129in}{0.961081in}}%
\pgfpathcurveto{\pgfqpoint{4.428079in}{0.961081in}}{\pgfqpoint{4.417480in}{0.956691in}}{\pgfqpoint{4.409666in}{0.948878in}}%
\pgfpathcurveto{\pgfqpoint{4.401853in}{0.941064in}}{\pgfqpoint{4.397462in}{0.930465in}}{\pgfqpoint{4.397462in}{0.919415in}}%
\pgfpathcurveto{\pgfqpoint{4.397462in}{0.908365in}}{\pgfqpoint{4.401853in}{0.897766in}}{\pgfqpoint{4.409666in}{0.889952in}}%
\pgfpathcurveto{\pgfqpoint{4.417480in}{0.882138in}}{\pgfqpoint{4.428079in}{0.877748in}}{\pgfqpoint{4.439129in}{0.877748in}}%
\pgfpathlineto{\pgfqpoint{4.439129in}{0.877748in}}%
\pgfpathclose%
\pgfusepath{stroke,fill}%
\end{pgfscope}%
\begin{pgfscope}%
\pgfpathrectangle{\pgfqpoint{2.963410in}{0.569136in}}{\pgfqpoint{2.177280in}{2.201755in}}%
\pgfusepath{clip}%
\pgfsetbuttcap%
\pgfsetroundjoin%
\definecolor{currentfill}{rgb}{0.121569,0.466667,0.705882}%
\pgfsetfillcolor{currentfill}%
\pgfsetlinewidth{0.481800pt}%
\definecolor{currentstroke}{rgb}{1.000000,1.000000,1.000000}%
\pgfsetstrokecolor{currentstroke}%
\pgfsetdash{}{0pt}%
\pgfpathmoveto{\pgfqpoint{4.143753in}{0.794348in}}%
\pgfpathcurveto{\pgfqpoint{4.154803in}{0.794348in}}{\pgfqpoint{4.165402in}{0.798739in}}{\pgfqpoint{4.173216in}{0.806552in}}%
\pgfpathcurveto{\pgfqpoint{4.181029in}{0.814366in}}{\pgfqpoint{4.185419in}{0.824965in}}{\pgfqpoint{4.185419in}{0.836015in}}%
\pgfpathcurveto{\pgfqpoint{4.185419in}{0.847065in}}{\pgfqpoint{4.181029in}{0.857664in}}{\pgfqpoint{4.173216in}{0.865478in}}%
\pgfpathcurveto{\pgfqpoint{4.165402in}{0.873291in}}{\pgfqpoint{4.154803in}{0.877682in}}{\pgfqpoint{4.143753in}{0.877682in}}%
\pgfpathcurveto{\pgfqpoint{4.132703in}{0.877682in}}{\pgfqpoint{4.122104in}{0.873291in}}{\pgfqpoint{4.114290in}{0.865478in}}%
\pgfpathcurveto{\pgfqpoint{4.106476in}{0.857664in}}{\pgfqpoint{4.102086in}{0.847065in}}{\pgfqpoint{4.102086in}{0.836015in}}%
\pgfpathcurveto{\pgfqpoint{4.102086in}{0.824965in}}{\pgfqpoint{4.106476in}{0.814366in}}{\pgfqpoint{4.114290in}{0.806552in}}%
\pgfpathcurveto{\pgfqpoint{4.122104in}{0.798739in}}{\pgfqpoint{4.132703in}{0.794348in}}{\pgfqpoint{4.143753in}{0.794348in}}%
\pgfpathlineto{\pgfqpoint{4.143753in}{0.794348in}}%
\pgfpathclose%
\pgfusepath{stroke,fill}%
\end{pgfscope}%
\begin{pgfscope}%
\pgfpathrectangle{\pgfqpoint{2.963410in}{0.569136in}}{\pgfqpoint{2.177280in}{2.201755in}}%
\pgfusepath{clip}%
\pgfsetbuttcap%
\pgfsetroundjoin%
\definecolor{currentfill}{rgb}{0.121569,0.466667,0.705882}%
\pgfsetfillcolor{currentfill}%
\pgfsetlinewidth{0.481800pt}%
\definecolor{currentstroke}{rgb}{1.000000,1.000000,1.000000}%
\pgfsetstrokecolor{currentstroke}%
\pgfsetdash{}{0pt}%
\pgfpathmoveto{\pgfqpoint{4.143753in}{0.710948in}}%
\pgfpathcurveto{\pgfqpoint{4.154803in}{0.710948in}}{\pgfqpoint{4.165402in}{0.715339in}}{\pgfqpoint{4.173216in}{0.723152in}}%
\pgfpathcurveto{\pgfqpoint{4.181029in}{0.730966in}}{\pgfqpoint{4.185419in}{0.741565in}}{\pgfqpoint{4.185419in}{0.752615in}}%
\pgfpathcurveto{\pgfqpoint{4.185419in}{0.763665in}}{\pgfqpoint{4.181029in}{0.774264in}}{\pgfqpoint{4.173216in}{0.782078in}}%
\pgfpathcurveto{\pgfqpoint{4.165402in}{0.789892in}}{\pgfqpoint{4.154803in}{0.794282in}}{\pgfqpoint{4.143753in}{0.794282in}}%
\pgfpathcurveto{\pgfqpoint{4.132703in}{0.794282in}}{\pgfqpoint{4.122104in}{0.789892in}}{\pgfqpoint{4.114290in}{0.782078in}}%
\pgfpathcurveto{\pgfqpoint{4.106476in}{0.774264in}}{\pgfqpoint{4.102086in}{0.763665in}}{\pgfqpoint{4.102086in}{0.752615in}}%
\pgfpathcurveto{\pgfqpoint{4.102086in}{0.741565in}}{\pgfqpoint{4.106476in}{0.730966in}}{\pgfqpoint{4.114290in}{0.723152in}}%
\pgfpathcurveto{\pgfqpoint{4.122104in}{0.715339in}}{\pgfqpoint{4.132703in}{0.710948in}}{\pgfqpoint{4.143753in}{0.710948in}}%
\pgfpathlineto{\pgfqpoint{4.143753in}{0.710948in}}%
\pgfpathclose%
\pgfusepath{stroke,fill}%
\end{pgfscope}%
\begin{pgfscope}%
\pgfpathrectangle{\pgfqpoint{2.963410in}{0.569136in}}{\pgfqpoint{2.177280in}{2.201755in}}%
\pgfusepath{clip}%
\pgfsetbuttcap%
\pgfsetroundjoin%
\definecolor{currentfill}{rgb}{0.121569,0.466667,0.705882}%
\pgfsetfillcolor{currentfill}%
\pgfsetlinewidth{0.481800pt}%
\definecolor{currentstroke}{rgb}{1.000000,1.000000,1.000000}%
\pgfsetstrokecolor{currentstroke}%
\pgfsetdash{}{0pt}%
\pgfpathmoveto{\pgfqpoint{3.848376in}{0.710948in}}%
\pgfpathcurveto{\pgfqpoint{3.859427in}{0.710948in}}{\pgfqpoint{3.870026in}{0.715339in}}{\pgfqpoint{3.877839in}{0.723152in}}%
\pgfpathcurveto{\pgfqpoint{3.885653in}{0.730966in}}{\pgfqpoint{3.890043in}{0.741565in}}{\pgfqpoint{3.890043in}{0.752615in}}%
\pgfpathcurveto{\pgfqpoint{3.890043in}{0.763665in}}{\pgfqpoint{3.885653in}{0.774264in}}{\pgfqpoint{3.877839in}{0.782078in}}%
\pgfpathcurveto{\pgfqpoint{3.870026in}{0.789892in}}{\pgfqpoint{3.859427in}{0.794282in}}{\pgfqpoint{3.848376in}{0.794282in}}%
\pgfpathcurveto{\pgfqpoint{3.837326in}{0.794282in}}{\pgfqpoint{3.826727in}{0.789892in}}{\pgfqpoint{3.818914in}{0.782078in}}%
\pgfpathcurveto{\pgfqpoint{3.811100in}{0.774264in}}{\pgfqpoint{3.806710in}{0.763665in}}{\pgfqpoint{3.806710in}{0.752615in}}%
\pgfpathcurveto{\pgfqpoint{3.806710in}{0.741565in}}{\pgfqpoint{3.811100in}{0.730966in}}{\pgfqpoint{3.818914in}{0.723152in}}%
\pgfpathcurveto{\pgfqpoint{3.826727in}{0.715339in}}{\pgfqpoint{3.837326in}{0.710948in}}{\pgfqpoint{3.848376in}{0.710948in}}%
\pgfpathlineto{\pgfqpoint{3.848376in}{0.710948in}}%
\pgfpathclose%
\pgfusepath{stroke,fill}%
\end{pgfscope}%
\begin{pgfscope}%
\pgfpathrectangle{\pgfqpoint{2.963410in}{0.569136in}}{\pgfqpoint{2.177280in}{2.201755in}}%
\pgfusepath{clip}%
\pgfsetbuttcap%
\pgfsetroundjoin%
\definecolor{currentfill}{rgb}{0.121569,0.466667,0.705882}%
\pgfsetfillcolor{currentfill}%
\pgfsetlinewidth{0.481800pt}%
\definecolor{currentstroke}{rgb}{1.000000,1.000000,1.000000}%
\pgfsetstrokecolor{currentstroke}%
\pgfsetdash{}{0pt}%
\pgfpathmoveto{\pgfqpoint{3.966527in}{0.627549in}}%
\pgfpathcurveto{\pgfqpoint{3.977577in}{0.627549in}}{\pgfqpoint{3.988176in}{0.631939in}}{\pgfqpoint{3.995990in}{0.639753in}}%
\pgfpathcurveto{\pgfqpoint{4.003803in}{0.647566in}}{\pgfqpoint{4.008194in}{0.658165in}}{\pgfqpoint{4.008194in}{0.669215in}}%
\pgfpathcurveto{\pgfqpoint{4.008194in}{0.680265in}}{\pgfqpoint{4.003803in}{0.690864in}}{\pgfqpoint{3.995990in}{0.698678in}}%
\pgfpathcurveto{\pgfqpoint{3.988176in}{0.706492in}}{\pgfqpoint{3.977577in}{0.710882in}}{\pgfqpoint{3.966527in}{0.710882in}}%
\pgfpathcurveto{\pgfqpoint{3.955477in}{0.710882in}}{\pgfqpoint{3.944878in}{0.706492in}}{\pgfqpoint{3.937064in}{0.698678in}}%
\pgfpathcurveto{\pgfqpoint{3.929251in}{0.690864in}}{\pgfqpoint{3.924860in}{0.680265in}}{\pgfqpoint{3.924860in}{0.669215in}}%
\pgfpathcurveto{\pgfqpoint{3.924860in}{0.658165in}}{\pgfqpoint{3.929251in}{0.647566in}}{\pgfqpoint{3.937064in}{0.639753in}}%
\pgfpathcurveto{\pgfqpoint{3.944878in}{0.631939in}}{\pgfqpoint{3.955477in}{0.627549in}}{\pgfqpoint{3.966527in}{0.627549in}}%
\pgfpathlineto{\pgfqpoint{3.966527in}{0.627549in}}%
\pgfpathclose%
\pgfusepath{stroke,fill}%
\end{pgfscope}%
\begin{pgfscope}%
\pgfpathrectangle{\pgfqpoint{2.963410in}{0.569136in}}{\pgfqpoint{2.177280in}{2.201755in}}%
\pgfusepath{clip}%
\pgfsetbuttcap%
\pgfsetroundjoin%
\definecolor{currentfill}{rgb}{0.121569,0.466667,0.705882}%
\pgfsetfillcolor{currentfill}%
\pgfsetlinewidth{0.481800pt}%
\definecolor{currentstroke}{rgb}{1.000000,1.000000,1.000000}%
\pgfsetstrokecolor{currentstroke}%
\pgfsetdash{}{0pt}%
\pgfpathmoveto{\pgfqpoint{4.320979in}{0.710948in}}%
\pgfpathcurveto{\pgfqpoint{4.332029in}{0.710948in}}{\pgfqpoint{4.342628in}{0.715339in}}{\pgfqpoint{4.350441in}{0.723152in}}%
\pgfpathcurveto{\pgfqpoint{4.358255in}{0.730966in}}{\pgfqpoint{4.362645in}{0.741565in}}{\pgfqpoint{4.362645in}{0.752615in}}%
\pgfpathcurveto{\pgfqpoint{4.362645in}{0.763665in}}{\pgfqpoint{4.358255in}{0.774264in}}{\pgfqpoint{4.350441in}{0.782078in}}%
\pgfpathcurveto{\pgfqpoint{4.342628in}{0.789892in}}{\pgfqpoint{4.332029in}{0.794282in}}{\pgfqpoint{4.320979in}{0.794282in}}%
\pgfpathcurveto{\pgfqpoint{4.309928in}{0.794282in}}{\pgfqpoint{4.299329in}{0.789892in}}{\pgfqpoint{4.291516in}{0.782078in}}%
\pgfpathcurveto{\pgfqpoint{4.283702in}{0.774264in}}{\pgfqpoint{4.279312in}{0.763665in}}{\pgfqpoint{4.279312in}{0.752615in}}%
\pgfpathcurveto{\pgfqpoint{4.279312in}{0.741565in}}{\pgfqpoint{4.283702in}{0.730966in}}{\pgfqpoint{4.291516in}{0.723152in}}%
\pgfpathcurveto{\pgfqpoint{4.299329in}{0.715339in}}{\pgfqpoint{4.309928in}{0.710948in}}{\pgfqpoint{4.320979in}{0.710948in}}%
\pgfpathlineto{\pgfqpoint{4.320979in}{0.710948in}}%
\pgfpathclose%
\pgfusepath{stroke,fill}%
\end{pgfscope}%
\begin{pgfscope}%
\pgfpathrectangle{\pgfqpoint{2.963410in}{0.569136in}}{\pgfqpoint{2.177280in}{2.201755in}}%
\pgfusepath{clip}%
\pgfsetbuttcap%
\pgfsetroundjoin%
\definecolor{currentfill}{rgb}{0.121569,0.466667,0.705882}%
\pgfsetfillcolor{currentfill}%
\pgfsetlinewidth{0.481800pt}%
\definecolor{currentstroke}{rgb}{1.000000,1.000000,1.000000}%
\pgfsetstrokecolor{currentstroke}%
\pgfsetdash{}{0pt}%
\pgfpathmoveto{\pgfqpoint{4.143753in}{0.710948in}}%
\pgfpathcurveto{\pgfqpoint{4.154803in}{0.710948in}}{\pgfqpoint{4.165402in}{0.715339in}}{\pgfqpoint{4.173216in}{0.723152in}}%
\pgfpathcurveto{\pgfqpoint{4.181029in}{0.730966in}}{\pgfqpoint{4.185419in}{0.741565in}}{\pgfqpoint{4.185419in}{0.752615in}}%
\pgfpathcurveto{\pgfqpoint{4.185419in}{0.763665in}}{\pgfqpoint{4.181029in}{0.774264in}}{\pgfqpoint{4.173216in}{0.782078in}}%
\pgfpathcurveto{\pgfqpoint{4.165402in}{0.789892in}}{\pgfqpoint{4.154803in}{0.794282in}}{\pgfqpoint{4.143753in}{0.794282in}}%
\pgfpathcurveto{\pgfqpoint{4.132703in}{0.794282in}}{\pgfqpoint{4.122104in}{0.789892in}}{\pgfqpoint{4.114290in}{0.782078in}}%
\pgfpathcurveto{\pgfqpoint{4.106476in}{0.774264in}}{\pgfqpoint{4.102086in}{0.763665in}}{\pgfqpoint{4.102086in}{0.752615in}}%
\pgfpathcurveto{\pgfqpoint{4.102086in}{0.741565in}}{\pgfqpoint{4.106476in}{0.730966in}}{\pgfqpoint{4.114290in}{0.723152in}}%
\pgfpathcurveto{\pgfqpoint{4.122104in}{0.715339in}}{\pgfqpoint{4.132703in}{0.710948in}}{\pgfqpoint{4.143753in}{0.710948in}}%
\pgfpathlineto{\pgfqpoint{4.143753in}{0.710948in}}%
\pgfpathclose%
\pgfusepath{stroke,fill}%
\end{pgfscope}%
\begin{pgfscope}%
\pgfpathrectangle{\pgfqpoint{2.963410in}{0.569136in}}{\pgfqpoint{2.177280in}{2.201755in}}%
\pgfusepath{clip}%
\pgfsetbuttcap%
\pgfsetroundjoin%
\definecolor{currentfill}{rgb}{0.121569,0.466667,0.705882}%
\pgfsetfillcolor{currentfill}%
\pgfsetlinewidth{0.481800pt}%
\definecolor{currentstroke}{rgb}{1.000000,1.000000,1.000000}%
\pgfsetstrokecolor{currentstroke}%
\pgfsetdash{}{0pt}%
\pgfpathmoveto{\pgfqpoint{3.907452in}{0.627549in}}%
\pgfpathcurveto{\pgfqpoint{3.918502in}{0.627549in}}{\pgfqpoint{3.929101in}{0.631939in}}{\pgfqpoint{3.936915in}{0.639753in}}%
\pgfpathcurveto{\pgfqpoint{3.944728in}{0.647566in}}{\pgfqpoint{3.949118in}{0.658165in}}{\pgfqpoint{3.949118in}{0.669215in}}%
\pgfpathcurveto{\pgfqpoint{3.949118in}{0.680265in}}{\pgfqpoint{3.944728in}{0.690864in}}{\pgfqpoint{3.936915in}{0.698678in}}%
\pgfpathcurveto{\pgfqpoint{3.929101in}{0.706492in}}{\pgfqpoint{3.918502in}{0.710882in}}{\pgfqpoint{3.907452in}{0.710882in}}%
\pgfpathcurveto{\pgfqpoint{3.896402in}{0.710882in}}{\pgfqpoint{3.885803in}{0.706492in}}{\pgfqpoint{3.877989in}{0.698678in}}%
\pgfpathcurveto{\pgfqpoint{3.870175in}{0.690864in}}{\pgfqpoint{3.865785in}{0.680265in}}{\pgfqpoint{3.865785in}{0.669215in}}%
\pgfpathcurveto{\pgfqpoint{3.865785in}{0.658165in}}{\pgfqpoint{3.870175in}{0.647566in}}{\pgfqpoint{3.877989in}{0.639753in}}%
\pgfpathcurveto{\pgfqpoint{3.885803in}{0.631939in}}{\pgfqpoint{3.896402in}{0.627549in}}{\pgfqpoint{3.907452in}{0.627549in}}%
\pgfpathlineto{\pgfqpoint{3.907452in}{0.627549in}}%
\pgfpathclose%
\pgfusepath{stroke,fill}%
\end{pgfscope}%
\begin{pgfscope}%
\pgfpathrectangle{\pgfqpoint{2.963410in}{0.569136in}}{\pgfqpoint{2.177280in}{2.201755in}}%
\pgfusepath{clip}%
\pgfsetbuttcap%
\pgfsetroundjoin%
\definecolor{currentfill}{rgb}{0.121569,0.466667,0.705882}%
\pgfsetfillcolor{currentfill}%
\pgfsetlinewidth{0.481800pt}%
\definecolor{currentstroke}{rgb}{1.000000,1.000000,1.000000}%
\pgfsetstrokecolor{currentstroke}%
\pgfsetdash{}{0pt}%
\pgfpathmoveto{\pgfqpoint{3.907452in}{0.627549in}}%
\pgfpathcurveto{\pgfqpoint{3.918502in}{0.627549in}}{\pgfqpoint{3.929101in}{0.631939in}}{\pgfqpoint{3.936915in}{0.639753in}}%
\pgfpathcurveto{\pgfqpoint{3.944728in}{0.647566in}}{\pgfqpoint{3.949118in}{0.658165in}}{\pgfqpoint{3.949118in}{0.669215in}}%
\pgfpathcurveto{\pgfqpoint{3.949118in}{0.680265in}}{\pgfqpoint{3.944728in}{0.690864in}}{\pgfqpoint{3.936915in}{0.698678in}}%
\pgfpathcurveto{\pgfqpoint{3.929101in}{0.706492in}}{\pgfqpoint{3.918502in}{0.710882in}}{\pgfqpoint{3.907452in}{0.710882in}}%
\pgfpathcurveto{\pgfqpoint{3.896402in}{0.710882in}}{\pgfqpoint{3.885803in}{0.706492in}}{\pgfqpoint{3.877989in}{0.698678in}}%
\pgfpathcurveto{\pgfqpoint{3.870175in}{0.690864in}}{\pgfqpoint{3.865785in}{0.680265in}}{\pgfqpoint{3.865785in}{0.669215in}}%
\pgfpathcurveto{\pgfqpoint{3.865785in}{0.658165in}}{\pgfqpoint{3.870175in}{0.647566in}}{\pgfqpoint{3.877989in}{0.639753in}}%
\pgfpathcurveto{\pgfqpoint{3.885803in}{0.631939in}}{\pgfqpoint{3.896402in}{0.627549in}}{\pgfqpoint{3.907452in}{0.627549in}}%
\pgfpathlineto{\pgfqpoint{3.907452in}{0.627549in}}%
\pgfpathclose%
\pgfusepath{stroke,fill}%
\end{pgfscope}%
\begin{pgfscope}%
\pgfpathrectangle{\pgfqpoint{2.963410in}{0.569136in}}{\pgfqpoint{2.177280in}{2.201755in}}%
\pgfusepath{clip}%
\pgfsetbuttcap%
\pgfsetroundjoin%
\definecolor{currentfill}{rgb}{0.121569,0.466667,0.705882}%
\pgfsetfillcolor{currentfill}%
\pgfsetlinewidth{0.481800pt}%
\definecolor{currentstroke}{rgb}{1.000000,1.000000,1.000000}%
\pgfsetstrokecolor{currentstroke}%
\pgfsetdash{}{0pt}%
\pgfpathmoveto{\pgfqpoint{4.498204in}{0.710948in}}%
\pgfpathcurveto{\pgfqpoint{4.509255in}{0.710948in}}{\pgfqpoint{4.519854in}{0.715339in}}{\pgfqpoint{4.527667in}{0.723152in}}%
\pgfpathcurveto{\pgfqpoint{4.535481in}{0.730966in}}{\pgfqpoint{4.539871in}{0.741565in}}{\pgfqpoint{4.539871in}{0.752615in}}%
\pgfpathcurveto{\pgfqpoint{4.539871in}{0.763665in}}{\pgfqpoint{4.535481in}{0.774264in}}{\pgfqpoint{4.527667in}{0.782078in}}%
\pgfpathcurveto{\pgfqpoint{4.519854in}{0.789892in}}{\pgfqpoint{4.509255in}{0.794282in}}{\pgfqpoint{4.498204in}{0.794282in}}%
\pgfpathcurveto{\pgfqpoint{4.487154in}{0.794282in}}{\pgfqpoint{4.476555in}{0.789892in}}{\pgfqpoint{4.468742in}{0.782078in}}%
\pgfpathcurveto{\pgfqpoint{4.460928in}{0.774264in}}{\pgfqpoint{4.456538in}{0.763665in}}{\pgfqpoint{4.456538in}{0.752615in}}%
\pgfpathcurveto{\pgfqpoint{4.456538in}{0.741565in}}{\pgfqpoint{4.460928in}{0.730966in}}{\pgfqpoint{4.468742in}{0.723152in}}%
\pgfpathcurveto{\pgfqpoint{4.476555in}{0.715339in}}{\pgfqpoint{4.487154in}{0.710948in}}{\pgfqpoint{4.498204in}{0.710948in}}%
\pgfpathlineto{\pgfqpoint{4.498204in}{0.710948in}}%
\pgfpathclose%
\pgfusepath{stroke,fill}%
\end{pgfscope}%
\begin{pgfscope}%
\pgfpathrectangle{\pgfqpoint{2.963410in}{0.569136in}}{\pgfqpoint{2.177280in}{2.201755in}}%
\pgfusepath{clip}%
\pgfsetbuttcap%
\pgfsetroundjoin%
\definecolor{currentfill}{rgb}{0.121569,0.466667,0.705882}%
\pgfsetfillcolor{currentfill}%
\pgfsetlinewidth{0.481800pt}%
\definecolor{currentstroke}{rgb}{1.000000,1.000000,1.000000}%
\pgfsetstrokecolor{currentstroke}%
\pgfsetdash{}{0pt}%
\pgfpathmoveto{\pgfqpoint{4.734505in}{0.877748in}}%
\pgfpathcurveto{\pgfqpoint{4.745556in}{0.877748in}}{\pgfqpoint{4.756155in}{0.882138in}}{\pgfqpoint{4.763968in}{0.889952in}}%
\pgfpathcurveto{\pgfqpoint{4.771782in}{0.897766in}}{\pgfqpoint{4.776172in}{0.908365in}}{\pgfqpoint{4.776172in}{0.919415in}}%
\pgfpathcurveto{\pgfqpoint{4.776172in}{0.930465in}}{\pgfqpoint{4.771782in}{0.941064in}}{\pgfqpoint{4.763968in}{0.948878in}}%
\pgfpathcurveto{\pgfqpoint{4.756155in}{0.956691in}}{\pgfqpoint{4.745556in}{0.961081in}}{\pgfqpoint{4.734505in}{0.961081in}}%
\pgfpathcurveto{\pgfqpoint{4.723455in}{0.961081in}}{\pgfqpoint{4.712856in}{0.956691in}}{\pgfqpoint{4.705043in}{0.948878in}}%
\pgfpathcurveto{\pgfqpoint{4.697229in}{0.941064in}}{\pgfqpoint{4.692839in}{0.930465in}}{\pgfqpoint{4.692839in}{0.919415in}}%
\pgfpathcurveto{\pgfqpoint{4.692839in}{0.908365in}}{\pgfqpoint{4.697229in}{0.897766in}}{\pgfqpoint{4.705043in}{0.889952in}}%
\pgfpathcurveto{\pgfqpoint{4.712856in}{0.882138in}}{\pgfqpoint{4.723455in}{0.877748in}}{\pgfqpoint{4.734505in}{0.877748in}}%
\pgfpathlineto{\pgfqpoint{4.734505in}{0.877748in}}%
\pgfpathclose%
\pgfusepath{stroke,fill}%
\end{pgfscope}%
\begin{pgfscope}%
\pgfpathrectangle{\pgfqpoint{2.963410in}{0.569136in}}{\pgfqpoint{2.177280in}{2.201755in}}%
\pgfusepath{clip}%
\pgfsetbuttcap%
\pgfsetroundjoin%
\definecolor{currentfill}{rgb}{0.121569,0.466667,0.705882}%
\pgfsetfillcolor{currentfill}%
\pgfsetlinewidth{0.481800pt}%
\definecolor{currentstroke}{rgb}{1.000000,1.000000,1.000000}%
\pgfsetstrokecolor{currentstroke}%
\pgfsetdash{}{0pt}%
\pgfpathmoveto{\pgfqpoint{4.439129in}{0.877748in}}%
\pgfpathcurveto{\pgfqpoint{4.450179in}{0.877748in}}{\pgfqpoint{4.460778in}{0.882138in}}{\pgfqpoint{4.468592in}{0.889952in}}%
\pgfpathcurveto{\pgfqpoint{4.476406in}{0.897766in}}{\pgfqpoint{4.480796in}{0.908365in}}{\pgfqpoint{4.480796in}{0.919415in}}%
\pgfpathcurveto{\pgfqpoint{4.480796in}{0.930465in}}{\pgfqpoint{4.476406in}{0.941064in}}{\pgfqpoint{4.468592in}{0.948878in}}%
\pgfpathcurveto{\pgfqpoint{4.460778in}{0.956691in}}{\pgfqpoint{4.450179in}{0.961081in}}{\pgfqpoint{4.439129in}{0.961081in}}%
\pgfpathcurveto{\pgfqpoint{4.428079in}{0.961081in}}{\pgfqpoint{4.417480in}{0.956691in}}{\pgfqpoint{4.409666in}{0.948878in}}%
\pgfpathcurveto{\pgfqpoint{4.401853in}{0.941064in}}{\pgfqpoint{4.397462in}{0.930465in}}{\pgfqpoint{4.397462in}{0.919415in}}%
\pgfpathcurveto{\pgfqpoint{4.397462in}{0.908365in}}{\pgfqpoint{4.401853in}{0.897766in}}{\pgfqpoint{4.409666in}{0.889952in}}%
\pgfpathcurveto{\pgfqpoint{4.417480in}{0.882138in}}{\pgfqpoint{4.428079in}{0.877748in}}{\pgfqpoint{4.439129in}{0.877748in}}%
\pgfpathlineto{\pgfqpoint{4.439129in}{0.877748in}}%
\pgfpathclose%
\pgfusepath{stroke,fill}%
\end{pgfscope}%
\begin{pgfscope}%
\pgfpathrectangle{\pgfqpoint{2.963410in}{0.569136in}}{\pgfqpoint{2.177280in}{2.201755in}}%
\pgfusepath{clip}%
\pgfsetbuttcap%
\pgfsetroundjoin%
\definecolor{currentfill}{rgb}{0.121569,0.466667,0.705882}%
\pgfsetfillcolor{currentfill}%
\pgfsetlinewidth{0.481800pt}%
\definecolor{currentstroke}{rgb}{1.000000,1.000000,1.000000}%
\pgfsetstrokecolor{currentstroke}%
\pgfsetdash{}{0pt}%
\pgfpathmoveto{\pgfqpoint{4.202828in}{0.794348in}}%
\pgfpathcurveto{\pgfqpoint{4.213878in}{0.794348in}}{\pgfqpoint{4.224477in}{0.798739in}}{\pgfqpoint{4.232291in}{0.806552in}}%
\pgfpathcurveto{\pgfqpoint{4.240104in}{0.814366in}}{\pgfqpoint{4.244495in}{0.824965in}}{\pgfqpoint{4.244495in}{0.836015in}}%
\pgfpathcurveto{\pgfqpoint{4.244495in}{0.847065in}}{\pgfqpoint{4.240104in}{0.857664in}}{\pgfqpoint{4.232291in}{0.865478in}}%
\pgfpathcurveto{\pgfqpoint{4.224477in}{0.873291in}}{\pgfqpoint{4.213878in}{0.877682in}}{\pgfqpoint{4.202828in}{0.877682in}}%
\pgfpathcurveto{\pgfqpoint{4.191778in}{0.877682in}}{\pgfqpoint{4.181179in}{0.873291in}}{\pgfqpoint{4.173365in}{0.865478in}}%
\pgfpathcurveto{\pgfqpoint{4.165552in}{0.857664in}}{\pgfqpoint{4.161161in}{0.847065in}}{\pgfqpoint{4.161161in}{0.836015in}}%
\pgfpathcurveto{\pgfqpoint{4.161161in}{0.824965in}}{\pgfqpoint{4.165552in}{0.814366in}}{\pgfqpoint{4.173365in}{0.806552in}}%
\pgfpathcurveto{\pgfqpoint{4.181179in}{0.798739in}}{\pgfqpoint{4.191778in}{0.794348in}}{\pgfqpoint{4.202828in}{0.794348in}}%
\pgfpathlineto{\pgfqpoint{4.202828in}{0.794348in}}%
\pgfpathclose%
\pgfusepath{stroke,fill}%
\end{pgfscope}%
\begin{pgfscope}%
\pgfpathrectangle{\pgfqpoint{2.963410in}{0.569136in}}{\pgfqpoint{2.177280in}{2.201755in}}%
\pgfusepath{clip}%
\pgfsetbuttcap%
\pgfsetroundjoin%
\definecolor{currentfill}{rgb}{0.121569,0.466667,0.705882}%
\pgfsetfillcolor{currentfill}%
\pgfsetlinewidth{0.481800pt}%
\definecolor{currentstroke}{rgb}{1.000000,1.000000,1.000000}%
\pgfsetstrokecolor{currentstroke}%
\pgfsetdash{}{0pt}%
\pgfpathmoveto{\pgfqpoint{4.380054in}{0.794348in}}%
\pgfpathcurveto{\pgfqpoint{4.391104in}{0.794348in}}{\pgfqpoint{4.401703in}{0.798739in}}{\pgfqpoint{4.409517in}{0.806552in}}%
\pgfpathcurveto{\pgfqpoint{4.417330in}{0.814366in}}{\pgfqpoint{4.421721in}{0.824965in}}{\pgfqpoint{4.421721in}{0.836015in}}%
\pgfpathcurveto{\pgfqpoint{4.421721in}{0.847065in}}{\pgfqpoint{4.417330in}{0.857664in}}{\pgfqpoint{4.409517in}{0.865478in}}%
\pgfpathcurveto{\pgfqpoint{4.401703in}{0.873291in}}{\pgfqpoint{4.391104in}{0.877682in}}{\pgfqpoint{4.380054in}{0.877682in}}%
\pgfpathcurveto{\pgfqpoint{4.369004in}{0.877682in}}{\pgfqpoint{4.358405in}{0.873291in}}{\pgfqpoint{4.350591in}{0.865478in}}%
\pgfpathcurveto{\pgfqpoint{4.342777in}{0.857664in}}{\pgfqpoint{4.338387in}{0.847065in}}{\pgfqpoint{4.338387in}{0.836015in}}%
\pgfpathcurveto{\pgfqpoint{4.338387in}{0.824965in}}{\pgfqpoint{4.342777in}{0.814366in}}{\pgfqpoint{4.350591in}{0.806552in}}%
\pgfpathcurveto{\pgfqpoint{4.358405in}{0.798739in}}{\pgfqpoint{4.369004in}{0.794348in}}{\pgfqpoint{4.380054in}{0.794348in}}%
\pgfpathlineto{\pgfqpoint{4.380054in}{0.794348in}}%
\pgfpathclose%
\pgfusepath{stroke,fill}%
\end{pgfscope}%
\begin{pgfscope}%
\pgfpathrectangle{\pgfqpoint{2.963410in}{0.569136in}}{\pgfqpoint{2.177280in}{2.201755in}}%
\pgfusepath{clip}%
\pgfsetbuttcap%
\pgfsetroundjoin%
\definecolor{currentfill}{rgb}{0.121569,0.466667,0.705882}%
\pgfsetfillcolor{currentfill}%
\pgfsetlinewidth{0.481800pt}%
\definecolor{currentstroke}{rgb}{1.000000,1.000000,1.000000}%
\pgfsetstrokecolor{currentstroke}%
\pgfsetdash{}{0pt}%
\pgfpathmoveto{\pgfqpoint{4.380054in}{0.794348in}}%
\pgfpathcurveto{\pgfqpoint{4.391104in}{0.794348in}}{\pgfqpoint{4.401703in}{0.798739in}}{\pgfqpoint{4.409517in}{0.806552in}}%
\pgfpathcurveto{\pgfqpoint{4.417330in}{0.814366in}}{\pgfqpoint{4.421721in}{0.824965in}}{\pgfqpoint{4.421721in}{0.836015in}}%
\pgfpathcurveto{\pgfqpoint{4.421721in}{0.847065in}}{\pgfqpoint{4.417330in}{0.857664in}}{\pgfqpoint{4.409517in}{0.865478in}}%
\pgfpathcurveto{\pgfqpoint{4.401703in}{0.873291in}}{\pgfqpoint{4.391104in}{0.877682in}}{\pgfqpoint{4.380054in}{0.877682in}}%
\pgfpathcurveto{\pgfqpoint{4.369004in}{0.877682in}}{\pgfqpoint{4.358405in}{0.873291in}}{\pgfqpoint{4.350591in}{0.865478in}}%
\pgfpathcurveto{\pgfqpoint{4.342777in}{0.857664in}}{\pgfqpoint{4.338387in}{0.847065in}}{\pgfqpoint{4.338387in}{0.836015in}}%
\pgfpathcurveto{\pgfqpoint{4.338387in}{0.824965in}}{\pgfqpoint{4.342777in}{0.814366in}}{\pgfqpoint{4.350591in}{0.806552in}}%
\pgfpathcurveto{\pgfqpoint{4.358405in}{0.798739in}}{\pgfqpoint{4.369004in}{0.794348in}}{\pgfqpoint{4.380054in}{0.794348in}}%
\pgfpathlineto{\pgfqpoint{4.380054in}{0.794348in}}%
\pgfpathclose%
\pgfusepath{stroke,fill}%
\end{pgfscope}%
\begin{pgfscope}%
\pgfpathrectangle{\pgfqpoint{2.963410in}{0.569136in}}{\pgfqpoint{2.177280in}{2.201755in}}%
\pgfusepath{clip}%
\pgfsetbuttcap%
\pgfsetroundjoin%
\definecolor{currentfill}{rgb}{0.121569,0.466667,0.705882}%
\pgfsetfillcolor{currentfill}%
\pgfsetlinewidth{0.481800pt}%
\definecolor{currentstroke}{rgb}{1.000000,1.000000,1.000000}%
\pgfsetstrokecolor{currentstroke}%
\pgfsetdash{}{0pt}%
\pgfpathmoveto{\pgfqpoint{4.143753in}{0.710948in}}%
\pgfpathcurveto{\pgfqpoint{4.154803in}{0.710948in}}{\pgfqpoint{4.165402in}{0.715339in}}{\pgfqpoint{4.173216in}{0.723152in}}%
\pgfpathcurveto{\pgfqpoint{4.181029in}{0.730966in}}{\pgfqpoint{4.185419in}{0.741565in}}{\pgfqpoint{4.185419in}{0.752615in}}%
\pgfpathcurveto{\pgfqpoint{4.185419in}{0.763665in}}{\pgfqpoint{4.181029in}{0.774264in}}{\pgfqpoint{4.173216in}{0.782078in}}%
\pgfpathcurveto{\pgfqpoint{4.165402in}{0.789892in}}{\pgfqpoint{4.154803in}{0.794282in}}{\pgfqpoint{4.143753in}{0.794282in}}%
\pgfpathcurveto{\pgfqpoint{4.132703in}{0.794282in}}{\pgfqpoint{4.122104in}{0.789892in}}{\pgfqpoint{4.114290in}{0.782078in}}%
\pgfpathcurveto{\pgfqpoint{4.106476in}{0.774264in}}{\pgfqpoint{4.102086in}{0.763665in}}{\pgfqpoint{4.102086in}{0.752615in}}%
\pgfpathcurveto{\pgfqpoint{4.102086in}{0.741565in}}{\pgfqpoint{4.106476in}{0.730966in}}{\pgfqpoint{4.114290in}{0.723152in}}%
\pgfpathcurveto{\pgfqpoint{4.122104in}{0.715339in}}{\pgfqpoint{4.132703in}{0.710948in}}{\pgfqpoint{4.143753in}{0.710948in}}%
\pgfpathlineto{\pgfqpoint{4.143753in}{0.710948in}}%
\pgfpathclose%
\pgfusepath{stroke,fill}%
\end{pgfscope}%
\begin{pgfscope}%
\pgfpathrectangle{\pgfqpoint{2.963410in}{0.569136in}}{\pgfqpoint{2.177280in}{2.201755in}}%
\pgfusepath{clip}%
\pgfsetbuttcap%
\pgfsetroundjoin%
\definecolor{currentfill}{rgb}{0.121569,0.466667,0.705882}%
\pgfsetfillcolor{currentfill}%
\pgfsetlinewidth{0.481800pt}%
\definecolor{currentstroke}{rgb}{1.000000,1.000000,1.000000}%
\pgfsetstrokecolor{currentstroke}%
\pgfsetdash{}{0pt}%
\pgfpathmoveto{\pgfqpoint{4.320979in}{0.877748in}}%
\pgfpathcurveto{\pgfqpoint{4.332029in}{0.877748in}}{\pgfqpoint{4.342628in}{0.882138in}}{\pgfqpoint{4.350441in}{0.889952in}}%
\pgfpathcurveto{\pgfqpoint{4.358255in}{0.897766in}}{\pgfqpoint{4.362645in}{0.908365in}}{\pgfqpoint{4.362645in}{0.919415in}}%
\pgfpathcurveto{\pgfqpoint{4.362645in}{0.930465in}}{\pgfqpoint{4.358255in}{0.941064in}}{\pgfqpoint{4.350441in}{0.948878in}}%
\pgfpathcurveto{\pgfqpoint{4.342628in}{0.956691in}}{\pgfqpoint{4.332029in}{0.961081in}}{\pgfqpoint{4.320979in}{0.961081in}}%
\pgfpathcurveto{\pgfqpoint{4.309928in}{0.961081in}}{\pgfqpoint{4.299329in}{0.956691in}}{\pgfqpoint{4.291516in}{0.948878in}}%
\pgfpathcurveto{\pgfqpoint{4.283702in}{0.941064in}}{\pgfqpoint{4.279312in}{0.930465in}}{\pgfqpoint{4.279312in}{0.919415in}}%
\pgfpathcurveto{\pgfqpoint{4.279312in}{0.908365in}}{\pgfqpoint{4.283702in}{0.897766in}}{\pgfqpoint{4.291516in}{0.889952in}}%
\pgfpathcurveto{\pgfqpoint{4.299329in}{0.882138in}}{\pgfqpoint{4.309928in}{0.877748in}}{\pgfqpoint{4.320979in}{0.877748in}}%
\pgfpathlineto{\pgfqpoint{4.320979in}{0.877748in}}%
\pgfpathclose%
\pgfusepath{stroke,fill}%
\end{pgfscope}%
\begin{pgfscope}%
\pgfpathrectangle{\pgfqpoint{2.963410in}{0.569136in}}{\pgfqpoint{2.177280in}{2.201755in}}%
\pgfusepath{clip}%
\pgfsetbuttcap%
\pgfsetroundjoin%
\definecolor{currentfill}{rgb}{0.121569,0.466667,0.705882}%
\pgfsetfillcolor{currentfill}%
\pgfsetlinewidth{0.481800pt}%
\definecolor{currentstroke}{rgb}{1.000000,1.000000,1.000000}%
\pgfsetstrokecolor{currentstroke}%
\pgfsetdash{}{0pt}%
\pgfpathmoveto{\pgfqpoint{4.261903in}{0.710948in}}%
\pgfpathcurveto{\pgfqpoint{4.272953in}{0.710948in}}{\pgfqpoint{4.283552in}{0.715339in}}{\pgfqpoint{4.291366in}{0.723152in}}%
\pgfpathcurveto{\pgfqpoint{4.299180in}{0.730966in}}{\pgfqpoint{4.303570in}{0.741565in}}{\pgfqpoint{4.303570in}{0.752615in}}%
\pgfpathcurveto{\pgfqpoint{4.303570in}{0.763665in}}{\pgfqpoint{4.299180in}{0.774264in}}{\pgfqpoint{4.291366in}{0.782078in}}%
\pgfpathcurveto{\pgfqpoint{4.283552in}{0.789892in}}{\pgfqpoint{4.272953in}{0.794282in}}{\pgfqpoint{4.261903in}{0.794282in}}%
\pgfpathcurveto{\pgfqpoint{4.250853in}{0.794282in}}{\pgfqpoint{4.240254in}{0.789892in}}{\pgfqpoint{4.232441in}{0.782078in}}%
\pgfpathcurveto{\pgfqpoint{4.224627in}{0.774264in}}{\pgfqpoint{4.220237in}{0.763665in}}{\pgfqpoint{4.220237in}{0.752615in}}%
\pgfpathcurveto{\pgfqpoint{4.220237in}{0.741565in}}{\pgfqpoint{4.224627in}{0.730966in}}{\pgfqpoint{4.232441in}{0.723152in}}%
\pgfpathcurveto{\pgfqpoint{4.240254in}{0.715339in}}{\pgfqpoint{4.250853in}{0.710948in}}{\pgfqpoint{4.261903in}{0.710948in}}%
\pgfpathlineto{\pgfqpoint{4.261903in}{0.710948in}}%
\pgfpathclose%
\pgfusepath{stroke,fill}%
\end{pgfscope}%
\begin{pgfscope}%
\pgfpathrectangle{\pgfqpoint{2.963410in}{0.569136in}}{\pgfqpoint{2.177280in}{2.201755in}}%
\pgfusepath{clip}%
\pgfsetbuttcap%
\pgfsetroundjoin%
\definecolor{currentfill}{rgb}{0.121569,0.466667,0.705882}%
\pgfsetfillcolor{currentfill}%
\pgfsetlinewidth{0.481800pt}%
\definecolor{currentstroke}{rgb}{1.000000,1.000000,1.000000}%
\pgfsetstrokecolor{currentstroke}%
\pgfsetdash{}{0pt}%
\pgfpathmoveto{\pgfqpoint{4.084678in}{0.961148in}}%
\pgfpathcurveto{\pgfqpoint{4.095728in}{0.961148in}}{\pgfqpoint{4.106327in}{0.965538in}}{\pgfqpoint{4.114140in}{0.973352in}}%
\pgfpathcurveto{\pgfqpoint{4.121954in}{0.981165in}}{\pgfqpoint{4.126344in}{0.991764in}}{\pgfqpoint{4.126344in}{1.002815in}}%
\pgfpathcurveto{\pgfqpoint{4.126344in}{1.013865in}}{\pgfqpoint{4.121954in}{1.024464in}}{\pgfqpoint{4.114140in}{1.032277in}}%
\pgfpathcurveto{\pgfqpoint{4.106327in}{1.040091in}}{\pgfqpoint{4.095728in}{1.044481in}}{\pgfqpoint{4.084678in}{1.044481in}}%
\pgfpathcurveto{\pgfqpoint{4.073627in}{1.044481in}}{\pgfqpoint{4.063028in}{1.040091in}}{\pgfqpoint{4.055215in}{1.032277in}}%
\pgfpathcurveto{\pgfqpoint{4.047401in}{1.024464in}}{\pgfqpoint{4.043011in}{1.013865in}}{\pgfqpoint{4.043011in}{1.002815in}}%
\pgfpathcurveto{\pgfqpoint{4.043011in}{0.991764in}}{\pgfqpoint{4.047401in}{0.981165in}}{\pgfqpoint{4.055215in}{0.973352in}}%
\pgfpathcurveto{\pgfqpoint{4.063028in}{0.965538in}}{\pgfqpoint{4.073627in}{0.961148in}}{\pgfqpoint{4.084678in}{0.961148in}}%
\pgfpathlineto{\pgfqpoint{4.084678in}{0.961148in}}%
\pgfpathclose%
\pgfusepath{stroke,fill}%
\end{pgfscope}%
\begin{pgfscope}%
\pgfpathrectangle{\pgfqpoint{2.963410in}{0.569136in}}{\pgfqpoint{2.177280in}{2.201755in}}%
\pgfusepath{clip}%
\pgfsetbuttcap%
\pgfsetroundjoin%
\definecolor{currentfill}{rgb}{0.121569,0.466667,0.705882}%
\pgfsetfillcolor{currentfill}%
\pgfsetlinewidth{0.481800pt}%
\definecolor{currentstroke}{rgb}{1.000000,1.000000,1.000000}%
\pgfsetstrokecolor{currentstroke}%
\pgfsetdash{}{0pt}%
\pgfpathmoveto{\pgfqpoint{4.143753in}{0.710948in}}%
\pgfpathcurveto{\pgfqpoint{4.154803in}{0.710948in}}{\pgfqpoint{4.165402in}{0.715339in}}{\pgfqpoint{4.173216in}{0.723152in}}%
\pgfpathcurveto{\pgfqpoint{4.181029in}{0.730966in}}{\pgfqpoint{4.185419in}{0.741565in}}{\pgfqpoint{4.185419in}{0.752615in}}%
\pgfpathcurveto{\pgfqpoint{4.185419in}{0.763665in}}{\pgfqpoint{4.181029in}{0.774264in}}{\pgfqpoint{4.173216in}{0.782078in}}%
\pgfpathcurveto{\pgfqpoint{4.165402in}{0.789892in}}{\pgfqpoint{4.154803in}{0.794282in}}{\pgfqpoint{4.143753in}{0.794282in}}%
\pgfpathcurveto{\pgfqpoint{4.132703in}{0.794282in}}{\pgfqpoint{4.122104in}{0.789892in}}{\pgfqpoint{4.114290in}{0.782078in}}%
\pgfpathcurveto{\pgfqpoint{4.106476in}{0.774264in}}{\pgfqpoint{4.102086in}{0.763665in}}{\pgfqpoint{4.102086in}{0.752615in}}%
\pgfpathcurveto{\pgfqpoint{4.102086in}{0.741565in}}{\pgfqpoint{4.106476in}{0.730966in}}{\pgfqpoint{4.114290in}{0.723152in}}%
\pgfpathcurveto{\pgfqpoint{4.122104in}{0.715339in}}{\pgfqpoint{4.132703in}{0.710948in}}{\pgfqpoint{4.143753in}{0.710948in}}%
\pgfpathlineto{\pgfqpoint{4.143753in}{0.710948in}}%
\pgfpathclose%
\pgfusepath{stroke,fill}%
\end{pgfscope}%
\begin{pgfscope}%
\pgfpathrectangle{\pgfqpoint{2.963410in}{0.569136in}}{\pgfqpoint{2.177280in}{2.201755in}}%
\pgfusepath{clip}%
\pgfsetbuttcap%
\pgfsetroundjoin%
\definecolor{currentfill}{rgb}{0.121569,0.466667,0.705882}%
\pgfsetfillcolor{currentfill}%
\pgfsetlinewidth{0.481800pt}%
\definecolor{currentstroke}{rgb}{1.000000,1.000000,1.000000}%
\pgfsetstrokecolor{currentstroke}%
\pgfsetdash{}{0pt}%
\pgfpathmoveto{\pgfqpoint{3.907452in}{0.710948in}}%
\pgfpathcurveto{\pgfqpoint{3.918502in}{0.710948in}}{\pgfqpoint{3.929101in}{0.715339in}}{\pgfqpoint{3.936915in}{0.723152in}}%
\pgfpathcurveto{\pgfqpoint{3.944728in}{0.730966in}}{\pgfqpoint{3.949118in}{0.741565in}}{\pgfqpoint{3.949118in}{0.752615in}}%
\pgfpathcurveto{\pgfqpoint{3.949118in}{0.763665in}}{\pgfqpoint{3.944728in}{0.774264in}}{\pgfqpoint{3.936915in}{0.782078in}}%
\pgfpathcurveto{\pgfqpoint{3.929101in}{0.789892in}}{\pgfqpoint{3.918502in}{0.794282in}}{\pgfqpoint{3.907452in}{0.794282in}}%
\pgfpathcurveto{\pgfqpoint{3.896402in}{0.794282in}}{\pgfqpoint{3.885803in}{0.789892in}}{\pgfqpoint{3.877989in}{0.782078in}}%
\pgfpathcurveto{\pgfqpoint{3.870175in}{0.774264in}}{\pgfqpoint{3.865785in}{0.763665in}}{\pgfqpoint{3.865785in}{0.752615in}}%
\pgfpathcurveto{\pgfqpoint{3.865785in}{0.741565in}}{\pgfqpoint{3.870175in}{0.730966in}}{\pgfqpoint{3.877989in}{0.723152in}}%
\pgfpathcurveto{\pgfqpoint{3.885803in}{0.715339in}}{\pgfqpoint{3.896402in}{0.710948in}}{\pgfqpoint{3.907452in}{0.710948in}}%
\pgfpathlineto{\pgfqpoint{3.907452in}{0.710948in}}%
\pgfpathclose%
\pgfusepath{stroke,fill}%
\end{pgfscope}%
\begin{pgfscope}%
\pgfpathrectangle{\pgfqpoint{2.963410in}{0.569136in}}{\pgfqpoint{2.177280in}{2.201755in}}%
\pgfusepath{clip}%
\pgfsetbuttcap%
\pgfsetroundjoin%
\definecolor{currentfill}{rgb}{0.121569,0.466667,0.705882}%
\pgfsetfillcolor{currentfill}%
\pgfsetlinewidth{0.481800pt}%
\definecolor{currentstroke}{rgb}{1.000000,1.000000,1.000000}%
\pgfsetstrokecolor{currentstroke}%
\pgfsetdash{}{0pt}%
\pgfpathmoveto{\pgfqpoint{4.143753in}{0.877748in}}%
\pgfpathcurveto{\pgfqpoint{4.154803in}{0.877748in}}{\pgfqpoint{4.165402in}{0.882138in}}{\pgfqpoint{4.173216in}{0.889952in}}%
\pgfpathcurveto{\pgfqpoint{4.181029in}{0.897766in}}{\pgfqpoint{4.185419in}{0.908365in}}{\pgfqpoint{4.185419in}{0.919415in}}%
\pgfpathcurveto{\pgfqpoint{4.185419in}{0.930465in}}{\pgfqpoint{4.181029in}{0.941064in}}{\pgfqpoint{4.173216in}{0.948878in}}%
\pgfpathcurveto{\pgfqpoint{4.165402in}{0.956691in}}{\pgfqpoint{4.154803in}{0.961081in}}{\pgfqpoint{4.143753in}{0.961081in}}%
\pgfpathcurveto{\pgfqpoint{4.132703in}{0.961081in}}{\pgfqpoint{4.122104in}{0.956691in}}{\pgfqpoint{4.114290in}{0.948878in}}%
\pgfpathcurveto{\pgfqpoint{4.106476in}{0.941064in}}{\pgfqpoint{4.102086in}{0.930465in}}{\pgfqpoint{4.102086in}{0.919415in}}%
\pgfpathcurveto{\pgfqpoint{4.102086in}{0.908365in}}{\pgfqpoint{4.106476in}{0.897766in}}{\pgfqpoint{4.114290in}{0.889952in}}%
\pgfpathcurveto{\pgfqpoint{4.122104in}{0.882138in}}{\pgfqpoint{4.132703in}{0.877748in}}{\pgfqpoint{4.143753in}{0.877748in}}%
\pgfpathlineto{\pgfqpoint{4.143753in}{0.877748in}}%
\pgfpathclose%
\pgfusepath{stroke,fill}%
\end{pgfscope}%
\begin{pgfscope}%
\pgfpathrectangle{\pgfqpoint{2.963410in}{0.569136in}}{\pgfqpoint{2.177280in}{2.201755in}}%
\pgfusepath{clip}%
\pgfsetbuttcap%
\pgfsetroundjoin%
\definecolor{currentfill}{rgb}{0.121569,0.466667,0.705882}%
\pgfsetfillcolor{currentfill}%
\pgfsetlinewidth{0.481800pt}%
\definecolor{currentstroke}{rgb}{1.000000,1.000000,1.000000}%
\pgfsetstrokecolor{currentstroke}%
\pgfsetdash{}{0pt}%
\pgfpathmoveto{\pgfqpoint{4.202828in}{0.710948in}}%
\pgfpathcurveto{\pgfqpoint{4.213878in}{0.710948in}}{\pgfqpoint{4.224477in}{0.715339in}}{\pgfqpoint{4.232291in}{0.723152in}}%
\pgfpathcurveto{\pgfqpoint{4.240104in}{0.730966in}}{\pgfqpoint{4.244495in}{0.741565in}}{\pgfqpoint{4.244495in}{0.752615in}}%
\pgfpathcurveto{\pgfqpoint{4.244495in}{0.763665in}}{\pgfqpoint{4.240104in}{0.774264in}}{\pgfqpoint{4.232291in}{0.782078in}}%
\pgfpathcurveto{\pgfqpoint{4.224477in}{0.789892in}}{\pgfqpoint{4.213878in}{0.794282in}}{\pgfqpoint{4.202828in}{0.794282in}}%
\pgfpathcurveto{\pgfqpoint{4.191778in}{0.794282in}}{\pgfqpoint{4.181179in}{0.789892in}}{\pgfqpoint{4.173365in}{0.782078in}}%
\pgfpathcurveto{\pgfqpoint{4.165552in}{0.774264in}}{\pgfqpoint{4.161161in}{0.763665in}}{\pgfqpoint{4.161161in}{0.752615in}}%
\pgfpathcurveto{\pgfqpoint{4.161161in}{0.741565in}}{\pgfqpoint{4.165552in}{0.730966in}}{\pgfqpoint{4.173365in}{0.723152in}}%
\pgfpathcurveto{\pgfqpoint{4.181179in}{0.715339in}}{\pgfqpoint{4.191778in}{0.710948in}}{\pgfqpoint{4.202828in}{0.710948in}}%
\pgfpathlineto{\pgfqpoint{4.202828in}{0.710948in}}%
\pgfpathclose%
\pgfusepath{stroke,fill}%
\end{pgfscope}%
\begin{pgfscope}%
\pgfpathrectangle{\pgfqpoint{2.963410in}{0.569136in}}{\pgfqpoint{2.177280in}{2.201755in}}%
\pgfusepath{clip}%
\pgfsetbuttcap%
\pgfsetroundjoin%
\definecolor{currentfill}{rgb}{0.121569,0.466667,0.705882}%
\pgfsetfillcolor{currentfill}%
\pgfsetlinewidth{0.481800pt}%
\definecolor{currentstroke}{rgb}{1.000000,1.000000,1.000000}%
\pgfsetstrokecolor{currentstroke}%
\pgfsetdash{}{0pt}%
\pgfpathmoveto{\pgfqpoint{4.143753in}{0.710948in}}%
\pgfpathcurveto{\pgfqpoint{4.154803in}{0.710948in}}{\pgfqpoint{4.165402in}{0.715339in}}{\pgfqpoint{4.173216in}{0.723152in}}%
\pgfpathcurveto{\pgfqpoint{4.181029in}{0.730966in}}{\pgfqpoint{4.185419in}{0.741565in}}{\pgfqpoint{4.185419in}{0.752615in}}%
\pgfpathcurveto{\pgfqpoint{4.185419in}{0.763665in}}{\pgfqpoint{4.181029in}{0.774264in}}{\pgfqpoint{4.173216in}{0.782078in}}%
\pgfpathcurveto{\pgfqpoint{4.165402in}{0.789892in}}{\pgfqpoint{4.154803in}{0.794282in}}{\pgfqpoint{4.143753in}{0.794282in}}%
\pgfpathcurveto{\pgfqpoint{4.132703in}{0.794282in}}{\pgfqpoint{4.122104in}{0.789892in}}{\pgfqpoint{4.114290in}{0.782078in}}%
\pgfpathcurveto{\pgfqpoint{4.106476in}{0.774264in}}{\pgfqpoint{4.102086in}{0.763665in}}{\pgfqpoint{4.102086in}{0.752615in}}%
\pgfpathcurveto{\pgfqpoint{4.102086in}{0.741565in}}{\pgfqpoint{4.106476in}{0.730966in}}{\pgfqpoint{4.114290in}{0.723152in}}%
\pgfpathcurveto{\pgfqpoint{4.122104in}{0.715339in}}{\pgfqpoint{4.132703in}{0.710948in}}{\pgfqpoint{4.143753in}{0.710948in}}%
\pgfpathlineto{\pgfqpoint{4.143753in}{0.710948in}}%
\pgfpathclose%
\pgfusepath{stroke,fill}%
\end{pgfscope}%
\begin{pgfscope}%
\pgfpathrectangle{\pgfqpoint{2.963410in}{0.569136in}}{\pgfqpoint{2.177280in}{2.201755in}}%
\pgfusepath{clip}%
\pgfsetbuttcap%
\pgfsetroundjoin%
\definecolor{currentfill}{rgb}{0.121569,0.466667,0.705882}%
\pgfsetfillcolor{currentfill}%
\pgfsetlinewidth{0.481800pt}%
\definecolor{currentstroke}{rgb}{1.000000,1.000000,1.000000}%
\pgfsetstrokecolor{currentstroke}%
\pgfsetdash{}{0pt}%
\pgfpathmoveto{\pgfqpoint{4.025602in}{0.710948in}}%
\pgfpathcurveto{\pgfqpoint{4.036652in}{0.710948in}}{\pgfqpoint{4.047251in}{0.715339in}}{\pgfqpoint{4.055065in}{0.723152in}}%
\pgfpathcurveto{\pgfqpoint{4.062879in}{0.730966in}}{\pgfqpoint{4.067269in}{0.741565in}}{\pgfqpoint{4.067269in}{0.752615in}}%
\pgfpathcurveto{\pgfqpoint{4.067269in}{0.763665in}}{\pgfqpoint{4.062879in}{0.774264in}}{\pgfqpoint{4.055065in}{0.782078in}}%
\pgfpathcurveto{\pgfqpoint{4.047251in}{0.789892in}}{\pgfqpoint{4.036652in}{0.794282in}}{\pgfqpoint{4.025602in}{0.794282in}}%
\pgfpathcurveto{\pgfqpoint{4.014552in}{0.794282in}}{\pgfqpoint{4.003953in}{0.789892in}}{\pgfqpoint{3.996139in}{0.782078in}}%
\pgfpathcurveto{\pgfqpoint{3.988326in}{0.774264in}}{\pgfqpoint{3.983936in}{0.763665in}}{\pgfqpoint{3.983936in}{0.752615in}}%
\pgfpathcurveto{\pgfqpoint{3.983936in}{0.741565in}}{\pgfqpoint{3.988326in}{0.730966in}}{\pgfqpoint{3.996139in}{0.723152in}}%
\pgfpathcurveto{\pgfqpoint{4.003953in}{0.715339in}}{\pgfqpoint{4.014552in}{0.710948in}}{\pgfqpoint{4.025602in}{0.710948in}}%
\pgfpathlineto{\pgfqpoint{4.025602in}{0.710948in}}%
\pgfpathclose%
\pgfusepath{stroke,fill}%
\end{pgfscope}%
\begin{pgfscope}%
\pgfpathrectangle{\pgfqpoint{2.963410in}{0.569136in}}{\pgfqpoint{2.177280in}{2.201755in}}%
\pgfusepath{clip}%
\pgfsetbuttcap%
\pgfsetroundjoin%
\definecolor{currentfill}{rgb}{0.121569,0.466667,0.705882}%
\pgfsetfillcolor{currentfill}%
\pgfsetlinewidth{0.481800pt}%
\definecolor{currentstroke}{rgb}{1.000000,1.000000,1.000000}%
\pgfsetstrokecolor{currentstroke}%
\pgfsetdash{}{0pt}%
\pgfpathmoveto{\pgfqpoint{3.966527in}{0.710948in}}%
\pgfpathcurveto{\pgfqpoint{3.977577in}{0.710948in}}{\pgfqpoint{3.988176in}{0.715339in}}{\pgfqpoint{3.995990in}{0.723152in}}%
\pgfpathcurveto{\pgfqpoint{4.003803in}{0.730966in}}{\pgfqpoint{4.008194in}{0.741565in}}{\pgfqpoint{4.008194in}{0.752615in}}%
\pgfpathcurveto{\pgfqpoint{4.008194in}{0.763665in}}{\pgfqpoint{4.003803in}{0.774264in}}{\pgfqpoint{3.995990in}{0.782078in}}%
\pgfpathcurveto{\pgfqpoint{3.988176in}{0.789892in}}{\pgfqpoint{3.977577in}{0.794282in}}{\pgfqpoint{3.966527in}{0.794282in}}%
\pgfpathcurveto{\pgfqpoint{3.955477in}{0.794282in}}{\pgfqpoint{3.944878in}{0.789892in}}{\pgfqpoint{3.937064in}{0.782078in}}%
\pgfpathcurveto{\pgfqpoint{3.929251in}{0.774264in}}{\pgfqpoint{3.924860in}{0.763665in}}{\pgfqpoint{3.924860in}{0.752615in}}%
\pgfpathcurveto{\pgfqpoint{3.924860in}{0.741565in}}{\pgfqpoint{3.929251in}{0.730966in}}{\pgfqpoint{3.937064in}{0.723152in}}%
\pgfpathcurveto{\pgfqpoint{3.944878in}{0.715339in}}{\pgfqpoint{3.955477in}{0.710948in}}{\pgfqpoint{3.966527in}{0.710948in}}%
\pgfpathlineto{\pgfqpoint{3.966527in}{0.710948in}}%
\pgfpathclose%
\pgfusepath{stroke,fill}%
\end{pgfscope}%
\begin{pgfscope}%
\pgfpathrectangle{\pgfqpoint{2.963410in}{0.569136in}}{\pgfqpoint{2.177280in}{2.201755in}}%
\pgfusepath{clip}%
\pgfsetbuttcap%
\pgfsetroundjoin%
\definecolor{currentfill}{rgb}{0.121569,0.466667,0.705882}%
\pgfsetfillcolor{currentfill}%
\pgfsetlinewidth{0.481800pt}%
\definecolor{currentstroke}{rgb}{1.000000,1.000000,1.000000}%
\pgfsetstrokecolor{currentstroke}%
\pgfsetdash{}{0pt}%
\pgfpathmoveto{\pgfqpoint{4.143753in}{0.877748in}}%
\pgfpathcurveto{\pgfqpoint{4.154803in}{0.877748in}}{\pgfqpoint{4.165402in}{0.882138in}}{\pgfqpoint{4.173216in}{0.889952in}}%
\pgfpathcurveto{\pgfqpoint{4.181029in}{0.897766in}}{\pgfqpoint{4.185419in}{0.908365in}}{\pgfqpoint{4.185419in}{0.919415in}}%
\pgfpathcurveto{\pgfqpoint{4.185419in}{0.930465in}}{\pgfqpoint{4.181029in}{0.941064in}}{\pgfqpoint{4.173216in}{0.948878in}}%
\pgfpathcurveto{\pgfqpoint{4.165402in}{0.956691in}}{\pgfqpoint{4.154803in}{0.961081in}}{\pgfqpoint{4.143753in}{0.961081in}}%
\pgfpathcurveto{\pgfqpoint{4.132703in}{0.961081in}}{\pgfqpoint{4.122104in}{0.956691in}}{\pgfqpoint{4.114290in}{0.948878in}}%
\pgfpathcurveto{\pgfqpoint{4.106476in}{0.941064in}}{\pgfqpoint{4.102086in}{0.930465in}}{\pgfqpoint{4.102086in}{0.919415in}}%
\pgfpathcurveto{\pgfqpoint{4.102086in}{0.908365in}}{\pgfqpoint{4.106476in}{0.897766in}}{\pgfqpoint{4.114290in}{0.889952in}}%
\pgfpathcurveto{\pgfqpoint{4.122104in}{0.882138in}}{\pgfqpoint{4.132703in}{0.877748in}}{\pgfqpoint{4.143753in}{0.877748in}}%
\pgfpathlineto{\pgfqpoint{4.143753in}{0.877748in}}%
\pgfpathclose%
\pgfusepath{stroke,fill}%
\end{pgfscope}%
\begin{pgfscope}%
\pgfpathrectangle{\pgfqpoint{2.963410in}{0.569136in}}{\pgfqpoint{2.177280in}{2.201755in}}%
\pgfusepath{clip}%
\pgfsetbuttcap%
\pgfsetroundjoin%
\definecolor{currentfill}{rgb}{0.121569,0.466667,0.705882}%
\pgfsetfillcolor{currentfill}%
\pgfsetlinewidth{0.481800pt}%
\definecolor{currentstroke}{rgb}{1.000000,1.000000,1.000000}%
\pgfsetstrokecolor{currentstroke}%
\pgfsetdash{}{0pt}%
\pgfpathmoveto{\pgfqpoint{4.557280in}{0.627549in}}%
\pgfpathcurveto{\pgfqpoint{4.568330in}{0.627549in}}{\pgfqpoint{4.578929in}{0.631939in}}{\pgfqpoint{4.586742in}{0.639753in}}%
\pgfpathcurveto{\pgfqpoint{4.594556in}{0.647566in}}{\pgfqpoint{4.598946in}{0.658165in}}{\pgfqpoint{4.598946in}{0.669215in}}%
\pgfpathcurveto{\pgfqpoint{4.598946in}{0.680265in}}{\pgfqpoint{4.594556in}{0.690864in}}{\pgfqpoint{4.586742in}{0.698678in}}%
\pgfpathcurveto{\pgfqpoint{4.578929in}{0.706492in}}{\pgfqpoint{4.568330in}{0.710882in}}{\pgfqpoint{4.557280in}{0.710882in}}%
\pgfpathcurveto{\pgfqpoint{4.546230in}{0.710882in}}{\pgfqpoint{4.535631in}{0.706492in}}{\pgfqpoint{4.527817in}{0.698678in}}%
\pgfpathcurveto{\pgfqpoint{4.520003in}{0.690864in}}{\pgfqpoint{4.515613in}{0.680265in}}{\pgfqpoint{4.515613in}{0.669215in}}%
\pgfpathcurveto{\pgfqpoint{4.515613in}{0.658165in}}{\pgfqpoint{4.520003in}{0.647566in}}{\pgfqpoint{4.527817in}{0.639753in}}%
\pgfpathcurveto{\pgfqpoint{4.535631in}{0.631939in}}{\pgfqpoint{4.546230in}{0.627549in}}{\pgfqpoint{4.557280in}{0.627549in}}%
\pgfpathlineto{\pgfqpoint{4.557280in}{0.627549in}}%
\pgfpathclose%
\pgfusepath{stroke,fill}%
\end{pgfscope}%
\begin{pgfscope}%
\pgfpathrectangle{\pgfqpoint{2.963410in}{0.569136in}}{\pgfqpoint{2.177280in}{2.201755in}}%
\pgfusepath{clip}%
\pgfsetbuttcap%
\pgfsetroundjoin%
\definecolor{currentfill}{rgb}{0.121569,0.466667,0.705882}%
\pgfsetfillcolor{currentfill}%
\pgfsetlinewidth{0.481800pt}%
\definecolor{currentstroke}{rgb}{1.000000,1.000000,1.000000}%
\pgfsetstrokecolor{currentstroke}%
\pgfsetdash{}{0pt}%
\pgfpathmoveto{\pgfqpoint{4.616355in}{0.710948in}}%
\pgfpathcurveto{\pgfqpoint{4.627405in}{0.710948in}}{\pgfqpoint{4.638004in}{0.715339in}}{\pgfqpoint{4.645818in}{0.723152in}}%
\pgfpathcurveto{\pgfqpoint{4.653631in}{0.730966in}}{\pgfqpoint{4.658022in}{0.741565in}}{\pgfqpoint{4.658022in}{0.752615in}}%
\pgfpathcurveto{\pgfqpoint{4.658022in}{0.763665in}}{\pgfqpoint{4.653631in}{0.774264in}}{\pgfqpoint{4.645818in}{0.782078in}}%
\pgfpathcurveto{\pgfqpoint{4.638004in}{0.789892in}}{\pgfqpoint{4.627405in}{0.794282in}}{\pgfqpoint{4.616355in}{0.794282in}}%
\pgfpathcurveto{\pgfqpoint{4.605305in}{0.794282in}}{\pgfqpoint{4.594706in}{0.789892in}}{\pgfqpoint{4.586892in}{0.782078in}}%
\pgfpathcurveto{\pgfqpoint{4.579079in}{0.774264in}}{\pgfqpoint{4.574688in}{0.763665in}}{\pgfqpoint{4.574688in}{0.752615in}}%
\pgfpathcurveto{\pgfqpoint{4.574688in}{0.741565in}}{\pgfqpoint{4.579079in}{0.730966in}}{\pgfqpoint{4.586892in}{0.723152in}}%
\pgfpathcurveto{\pgfqpoint{4.594706in}{0.715339in}}{\pgfqpoint{4.605305in}{0.710948in}}{\pgfqpoint{4.616355in}{0.710948in}}%
\pgfpathlineto{\pgfqpoint{4.616355in}{0.710948in}}%
\pgfpathclose%
\pgfusepath{stroke,fill}%
\end{pgfscope}%
\begin{pgfscope}%
\pgfpathrectangle{\pgfqpoint{2.963410in}{0.569136in}}{\pgfqpoint{2.177280in}{2.201755in}}%
\pgfusepath{clip}%
\pgfsetbuttcap%
\pgfsetroundjoin%
\definecolor{currentfill}{rgb}{0.121569,0.466667,0.705882}%
\pgfsetfillcolor{currentfill}%
\pgfsetlinewidth{0.481800pt}%
\definecolor{currentstroke}{rgb}{1.000000,1.000000,1.000000}%
\pgfsetstrokecolor{currentstroke}%
\pgfsetdash{}{0pt}%
\pgfpathmoveto{\pgfqpoint{3.966527in}{0.710948in}}%
\pgfpathcurveto{\pgfqpoint{3.977577in}{0.710948in}}{\pgfqpoint{3.988176in}{0.715339in}}{\pgfqpoint{3.995990in}{0.723152in}}%
\pgfpathcurveto{\pgfqpoint{4.003803in}{0.730966in}}{\pgfqpoint{4.008194in}{0.741565in}}{\pgfqpoint{4.008194in}{0.752615in}}%
\pgfpathcurveto{\pgfqpoint{4.008194in}{0.763665in}}{\pgfqpoint{4.003803in}{0.774264in}}{\pgfqpoint{3.995990in}{0.782078in}}%
\pgfpathcurveto{\pgfqpoint{3.988176in}{0.789892in}}{\pgfqpoint{3.977577in}{0.794282in}}{\pgfqpoint{3.966527in}{0.794282in}}%
\pgfpathcurveto{\pgfqpoint{3.955477in}{0.794282in}}{\pgfqpoint{3.944878in}{0.789892in}}{\pgfqpoint{3.937064in}{0.782078in}}%
\pgfpathcurveto{\pgfqpoint{3.929251in}{0.774264in}}{\pgfqpoint{3.924860in}{0.763665in}}{\pgfqpoint{3.924860in}{0.752615in}}%
\pgfpathcurveto{\pgfqpoint{3.924860in}{0.741565in}}{\pgfqpoint{3.929251in}{0.730966in}}{\pgfqpoint{3.937064in}{0.723152in}}%
\pgfpathcurveto{\pgfqpoint{3.944878in}{0.715339in}}{\pgfqpoint{3.955477in}{0.710948in}}{\pgfqpoint{3.966527in}{0.710948in}}%
\pgfpathlineto{\pgfqpoint{3.966527in}{0.710948in}}%
\pgfpathclose%
\pgfusepath{stroke,fill}%
\end{pgfscope}%
\begin{pgfscope}%
\pgfpathrectangle{\pgfqpoint{2.963410in}{0.569136in}}{\pgfqpoint{2.177280in}{2.201755in}}%
\pgfusepath{clip}%
\pgfsetbuttcap%
\pgfsetroundjoin%
\definecolor{currentfill}{rgb}{0.121569,0.466667,0.705882}%
\pgfsetfillcolor{currentfill}%
\pgfsetlinewidth{0.481800pt}%
\definecolor{currentstroke}{rgb}{1.000000,1.000000,1.000000}%
\pgfsetstrokecolor{currentstroke}%
\pgfsetdash{}{0pt}%
\pgfpathmoveto{\pgfqpoint{4.025602in}{0.710948in}}%
\pgfpathcurveto{\pgfqpoint{4.036652in}{0.710948in}}{\pgfqpoint{4.047251in}{0.715339in}}{\pgfqpoint{4.055065in}{0.723152in}}%
\pgfpathcurveto{\pgfqpoint{4.062879in}{0.730966in}}{\pgfqpoint{4.067269in}{0.741565in}}{\pgfqpoint{4.067269in}{0.752615in}}%
\pgfpathcurveto{\pgfqpoint{4.067269in}{0.763665in}}{\pgfqpoint{4.062879in}{0.774264in}}{\pgfqpoint{4.055065in}{0.782078in}}%
\pgfpathcurveto{\pgfqpoint{4.047251in}{0.789892in}}{\pgfqpoint{4.036652in}{0.794282in}}{\pgfqpoint{4.025602in}{0.794282in}}%
\pgfpathcurveto{\pgfqpoint{4.014552in}{0.794282in}}{\pgfqpoint{4.003953in}{0.789892in}}{\pgfqpoint{3.996139in}{0.782078in}}%
\pgfpathcurveto{\pgfqpoint{3.988326in}{0.774264in}}{\pgfqpoint{3.983936in}{0.763665in}}{\pgfqpoint{3.983936in}{0.752615in}}%
\pgfpathcurveto{\pgfqpoint{3.983936in}{0.741565in}}{\pgfqpoint{3.988326in}{0.730966in}}{\pgfqpoint{3.996139in}{0.723152in}}%
\pgfpathcurveto{\pgfqpoint{4.003953in}{0.715339in}}{\pgfqpoint{4.014552in}{0.710948in}}{\pgfqpoint{4.025602in}{0.710948in}}%
\pgfpathlineto{\pgfqpoint{4.025602in}{0.710948in}}%
\pgfpathclose%
\pgfusepath{stroke,fill}%
\end{pgfscope}%
\begin{pgfscope}%
\pgfpathrectangle{\pgfqpoint{2.963410in}{0.569136in}}{\pgfqpoint{2.177280in}{2.201755in}}%
\pgfusepath{clip}%
\pgfsetbuttcap%
\pgfsetroundjoin%
\definecolor{currentfill}{rgb}{0.121569,0.466667,0.705882}%
\pgfsetfillcolor{currentfill}%
\pgfsetlinewidth{0.481800pt}%
\definecolor{currentstroke}{rgb}{1.000000,1.000000,1.000000}%
\pgfsetstrokecolor{currentstroke}%
\pgfsetdash{}{0pt}%
\pgfpathmoveto{\pgfqpoint{4.202828in}{0.710948in}}%
\pgfpathcurveto{\pgfqpoint{4.213878in}{0.710948in}}{\pgfqpoint{4.224477in}{0.715339in}}{\pgfqpoint{4.232291in}{0.723152in}}%
\pgfpathcurveto{\pgfqpoint{4.240104in}{0.730966in}}{\pgfqpoint{4.244495in}{0.741565in}}{\pgfqpoint{4.244495in}{0.752615in}}%
\pgfpathcurveto{\pgfqpoint{4.244495in}{0.763665in}}{\pgfqpoint{4.240104in}{0.774264in}}{\pgfqpoint{4.232291in}{0.782078in}}%
\pgfpathcurveto{\pgfqpoint{4.224477in}{0.789892in}}{\pgfqpoint{4.213878in}{0.794282in}}{\pgfqpoint{4.202828in}{0.794282in}}%
\pgfpathcurveto{\pgfqpoint{4.191778in}{0.794282in}}{\pgfqpoint{4.181179in}{0.789892in}}{\pgfqpoint{4.173365in}{0.782078in}}%
\pgfpathcurveto{\pgfqpoint{4.165552in}{0.774264in}}{\pgfqpoint{4.161161in}{0.763665in}}{\pgfqpoint{4.161161in}{0.752615in}}%
\pgfpathcurveto{\pgfqpoint{4.161161in}{0.741565in}}{\pgfqpoint{4.165552in}{0.730966in}}{\pgfqpoint{4.173365in}{0.723152in}}%
\pgfpathcurveto{\pgfqpoint{4.181179in}{0.715339in}}{\pgfqpoint{4.191778in}{0.710948in}}{\pgfqpoint{4.202828in}{0.710948in}}%
\pgfpathlineto{\pgfqpoint{4.202828in}{0.710948in}}%
\pgfpathclose%
\pgfusepath{stroke,fill}%
\end{pgfscope}%
\begin{pgfscope}%
\pgfpathrectangle{\pgfqpoint{2.963410in}{0.569136in}}{\pgfqpoint{2.177280in}{2.201755in}}%
\pgfusepath{clip}%
\pgfsetbuttcap%
\pgfsetroundjoin%
\definecolor{currentfill}{rgb}{0.121569,0.466667,0.705882}%
\pgfsetfillcolor{currentfill}%
\pgfsetlinewidth{0.481800pt}%
\definecolor{currentstroke}{rgb}{1.000000,1.000000,1.000000}%
\pgfsetstrokecolor{currentstroke}%
\pgfsetdash{}{0pt}%
\pgfpathmoveto{\pgfqpoint{4.261903in}{0.627549in}}%
\pgfpathcurveto{\pgfqpoint{4.272953in}{0.627549in}}{\pgfqpoint{4.283552in}{0.631939in}}{\pgfqpoint{4.291366in}{0.639753in}}%
\pgfpathcurveto{\pgfqpoint{4.299180in}{0.647566in}}{\pgfqpoint{4.303570in}{0.658165in}}{\pgfqpoint{4.303570in}{0.669215in}}%
\pgfpathcurveto{\pgfqpoint{4.303570in}{0.680265in}}{\pgfqpoint{4.299180in}{0.690864in}}{\pgfqpoint{4.291366in}{0.698678in}}%
\pgfpathcurveto{\pgfqpoint{4.283552in}{0.706492in}}{\pgfqpoint{4.272953in}{0.710882in}}{\pgfqpoint{4.261903in}{0.710882in}}%
\pgfpathcurveto{\pgfqpoint{4.250853in}{0.710882in}}{\pgfqpoint{4.240254in}{0.706492in}}{\pgfqpoint{4.232441in}{0.698678in}}%
\pgfpathcurveto{\pgfqpoint{4.224627in}{0.690864in}}{\pgfqpoint{4.220237in}{0.680265in}}{\pgfqpoint{4.220237in}{0.669215in}}%
\pgfpathcurveto{\pgfqpoint{4.220237in}{0.658165in}}{\pgfqpoint{4.224627in}{0.647566in}}{\pgfqpoint{4.232441in}{0.639753in}}%
\pgfpathcurveto{\pgfqpoint{4.240254in}{0.631939in}}{\pgfqpoint{4.250853in}{0.627549in}}{\pgfqpoint{4.261903in}{0.627549in}}%
\pgfpathlineto{\pgfqpoint{4.261903in}{0.627549in}}%
\pgfpathclose%
\pgfusepath{stroke,fill}%
\end{pgfscope}%
\begin{pgfscope}%
\pgfpathrectangle{\pgfqpoint{2.963410in}{0.569136in}}{\pgfqpoint{2.177280in}{2.201755in}}%
\pgfusepath{clip}%
\pgfsetbuttcap%
\pgfsetroundjoin%
\definecolor{currentfill}{rgb}{0.121569,0.466667,0.705882}%
\pgfsetfillcolor{currentfill}%
\pgfsetlinewidth{0.481800pt}%
\definecolor{currentstroke}{rgb}{1.000000,1.000000,1.000000}%
\pgfsetstrokecolor{currentstroke}%
\pgfsetdash{}{0pt}%
\pgfpathmoveto{\pgfqpoint{3.907452in}{0.710948in}}%
\pgfpathcurveto{\pgfqpoint{3.918502in}{0.710948in}}{\pgfqpoint{3.929101in}{0.715339in}}{\pgfqpoint{3.936915in}{0.723152in}}%
\pgfpathcurveto{\pgfqpoint{3.944728in}{0.730966in}}{\pgfqpoint{3.949118in}{0.741565in}}{\pgfqpoint{3.949118in}{0.752615in}}%
\pgfpathcurveto{\pgfqpoint{3.949118in}{0.763665in}}{\pgfqpoint{3.944728in}{0.774264in}}{\pgfqpoint{3.936915in}{0.782078in}}%
\pgfpathcurveto{\pgfqpoint{3.929101in}{0.789892in}}{\pgfqpoint{3.918502in}{0.794282in}}{\pgfqpoint{3.907452in}{0.794282in}}%
\pgfpathcurveto{\pgfqpoint{3.896402in}{0.794282in}}{\pgfqpoint{3.885803in}{0.789892in}}{\pgfqpoint{3.877989in}{0.782078in}}%
\pgfpathcurveto{\pgfqpoint{3.870175in}{0.774264in}}{\pgfqpoint{3.865785in}{0.763665in}}{\pgfqpoint{3.865785in}{0.752615in}}%
\pgfpathcurveto{\pgfqpoint{3.865785in}{0.741565in}}{\pgfqpoint{3.870175in}{0.730966in}}{\pgfqpoint{3.877989in}{0.723152in}}%
\pgfpathcurveto{\pgfqpoint{3.885803in}{0.715339in}}{\pgfqpoint{3.896402in}{0.710948in}}{\pgfqpoint{3.907452in}{0.710948in}}%
\pgfpathlineto{\pgfqpoint{3.907452in}{0.710948in}}%
\pgfpathclose%
\pgfusepath{stroke,fill}%
\end{pgfscope}%
\begin{pgfscope}%
\pgfpathrectangle{\pgfqpoint{2.963410in}{0.569136in}}{\pgfqpoint{2.177280in}{2.201755in}}%
\pgfusepath{clip}%
\pgfsetbuttcap%
\pgfsetroundjoin%
\definecolor{currentfill}{rgb}{0.121569,0.466667,0.705882}%
\pgfsetfillcolor{currentfill}%
\pgfsetlinewidth{0.481800pt}%
\definecolor{currentstroke}{rgb}{1.000000,1.000000,1.000000}%
\pgfsetstrokecolor{currentstroke}%
\pgfsetdash{}{0pt}%
\pgfpathmoveto{\pgfqpoint{4.143753in}{0.710948in}}%
\pgfpathcurveto{\pgfqpoint{4.154803in}{0.710948in}}{\pgfqpoint{4.165402in}{0.715339in}}{\pgfqpoint{4.173216in}{0.723152in}}%
\pgfpathcurveto{\pgfqpoint{4.181029in}{0.730966in}}{\pgfqpoint{4.185419in}{0.741565in}}{\pgfqpoint{4.185419in}{0.752615in}}%
\pgfpathcurveto{\pgfqpoint{4.185419in}{0.763665in}}{\pgfqpoint{4.181029in}{0.774264in}}{\pgfqpoint{4.173216in}{0.782078in}}%
\pgfpathcurveto{\pgfqpoint{4.165402in}{0.789892in}}{\pgfqpoint{4.154803in}{0.794282in}}{\pgfqpoint{4.143753in}{0.794282in}}%
\pgfpathcurveto{\pgfqpoint{4.132703in}{0.794282in}}{\pgfqpoint{4.122104in}{0.789892in}}{\pgfqpoint{4.114290in}{0.782078in}}%
\pgfpathcurveto{\pgfqpoint{4.106476in}{0.774264in}}{\pgfqpoint{4.102086in}{0.763665in}}{\pgfqpoint{4.102086in}{0.752615in}}%
\pgfpathcurveto{\pgfqpoint{4.102086in}{0.741565in}}{\pgfqpoint{4.106476in}{0.730966in}}{\pgfqpoint{4.114290in}{0.723152in}}%
\pgfpathcurveto{\pgfqpoint{4.122104in}{0.715339in}}{\pgfqpoint{4.132703in}{0.710948in}}{\pgfqpoint{4.143753in}{0.710948in}}%
\pgfpathlineto{\pgfqpoint{4.143753in}{0.710948in}}%
\pgfpathclose%
\pgfusepath{stroke,fill}%
\end{pgfscope}%
\begin{pgfscope}%
\pgfpathrectangle{\pgfqpoint{2.963410in}{0.569136in}}{\pgfqpoint{2.177280in}{2.201755in}}%
\pgfusepath{clip}%
\pgfsetbuttcap%
\pgfsetroundjoin%
\definecolor{currentfill}{rgb}{0.121569,0.466667,0.705882}%
\pgfsetfillcolor{currentfill}%
\pgfsetlinewidth{0.481800pt}%
\definecolor{currentstroke}{rgb}{1.000000,1.000000,1.000000}%
\pgfsetstrokecolor{currentstroke}%
\pgfsetdash{}{0pt}%
\pgfpathmoveto{\pgfqpoint{4.202828in}{0.794348in}}%
\pgfpathcurveto{\pgfqpoint{4.213878in}{0.794348in}}{\pgfqpoint{4.224477in}{0.798739in}}{\pgfqpoint{4.232291in}{0.806552in}}%
\pgfpathcurveto{\pgfqpoint{4.240104in}{0.814366in}}{\pgfqpoint{4.244495in}{0.824965in}}{\pgfqpoint{4.244495in}{0.836015in}}%
\pgfpathcurveto{\pgfqpoint{4.244495in}{0.847065in}}{\pgfqpoint{4.240104in}{0.857664in}}{\pgfqpoint{4.232291in}{0.865478in}}%
\pgfpathcurveto{\pgfqpoint{4.224477in}{0.873291in}}{\pgfqpoint{4.213878in}{0.877682in}}{\pgfqpoint{4.202828in}{0.877682in}}%
\pgfpathcurveto{\pgfqpoint{4.191778in}{0.877682in}}{\pgfqpoint{4.181179in}{0.873291in}}{\pgfqpoint{4.173365in}{0.865478in}}%
\pgfpathcurveto{\pgfqpoint{4.165552in}{0.857664in}}{\pgfqpoint{4.161161in}{0.847065in}}{\pgfqpoint{4.161161in}{0.836015in}}%
\pgfpathcurveto{\pgfqpoint{4.161161in}{0.824965in}}{\pgfqpoint{4.165552in}{0.814366in}}{\pgfqpoint{4.173365in}{0.806552in}}%
\pgfpathcurveto{\pgfqpoint{4.181179in}{0.798739in}}{\pgfqpoint{4.191778in}{0.794348in}}{\pgfqpoint{4.202828in}{0.794348in}}%
\pgfpathlineto{\pgfqpoint{4.202828in}{0.794348in}}%
\pgfpathclose%
\pgfusepath{stroke,fill}%
\end{pgfscope}%
\begin{pgfscope}%
\pgfpathrectangle{\pgfqpoint{2.963410in}{0.569136in}}{\pgfqpoint{2.177280in}{2.201755in}}%
\pgfusepath{clip}%
\pgfsetbuttcap%
\pgfsetroundjoin%
\definecolor{currentfill}{rgb}{0.121569,0.466667,0.705882}%
\pgfsetfillcolor{currentfill}%
\pgfsetlinewidth{0.481800pt}%
\definecolor{currentstroke}{rgb}{1.000000,1.000000,1.000000}%
\pgfsetstrokecolor{currentstroke}%
\pgfsetdash{}{0pt}%
\pgfpathmoveto{\pgfqpoint{3.493925in}{0.794348in}}%
\pgfpathcurveto{\pgfqpoint{3.504975in}{0.794348in}}{\pgfqpoint{3.515574in}{0.798739in}}{\pgfqpoint{3.523388in}{0.806552in}}%
\pgfpathcurveto{\pgfqpoint{3.531201in}{0.814366in}}{\pgfqpoint{3.535592in}{0.824965in}}{\pgfqpoint{3.535592in}{0.836015in}}%
\pgfpathcurveto{\pgfqpoint{3.535592in}{0.847065in}}{\pgfqpoint{3.531201in}{0.857664in}}{\pgfqpoint{3.523388in}{0.865478in}}%
\pgfpathcurveto{\pgfqpoint{3.515574in}{0.873291in}}{\pgfqpoint{3.504975in}{0.877682in}}{\pgfqpoint{3.493925in}{0.877682in}}%
\pgfpathcurveto{\pgfqpoint{3.482875in}{0.877682in}}{\pgfqpoint{3.472276in}{0.873291in}}{\pgfqpoint{3.464462in}{0.865478in}}%
\pgfpathcurveto{\pgfqpoint{3.456648in}{0.857664in}}{\pgfqpoint{3.452258in}{0.847065in}}{\pgfqpoint{3.452258in}{0.836015in}}%
\pgfpathcurveto{\pgfqpoint{3.452258in}{0.824965in}}{\pgfqpoint{3.456648in}{0.814366in}}{\pgfqpoint{3.464462in}{0.806552in}}%
\pgfpathcurveto{\pgfqpoint{3.472276in}{0.798739in}}{\pgfqpoint{3.482875in}{0.794348in}}{\pgfqpoint{3.493925in}{0.794348in}}%
\pgfpathlineto{\pgfqpoint{3.493925in}{0.794348in}}%
\pgfpathclose%
\pgfusepath{stroke,fill}%
\end{pgfscope}%
\begin{pgfscope}%
\pgfpathrectangle{\pgfqpoint{2.963410in}{0.569136in}}{\pgfqpoint{2.177280in}{2.201755in}}%
\pgfusepath{clip}%
\pgfsetbuttcap%
\pgfsetroundjoin%
\definecolor{currentfill}{rgb}{0.121569,0.466667,0.705882}%
\pgfsetfillcolor{currentfill}%
\pgfsetlinewidth{0.481800pt}%
\definecolor{currentstroke}{rgb}{1.000000,1.000000,1.000000}%
\pgfsetstrokecolor{currentstroke}%
\pgfsetdash{}{0pt}%
\pgfpathmoveto{\pgfqpoint{4.025602in}{0.710948in}}%
\pgfpathcurveto{\pgfqpoint{4.036652in}{0.710948in}}{\pgfqpoint{4.047251in}{0.715339in}}{\pgfqpoint{4.055065in}{0.723152in}}%
\pgfpathcurveto{\pgfqpoint{4.062879in}{0.730966in}}{\pgfqpoint{4.067269in}{0.741565in}}{\pgfqpoint{4.067269in}{0.752615in}}%
\pgfpathcurveto{\pgfqpoint{4.067269in}{0.763665in}}{\pgfqpoint{4.062879in}{0.774264in}}{\pgfqpoint{4.055065in}{0.782078in}}%
\pgfpathcurveto{\pgfqpoint{4.047251in}{0.789892in}}{\pgfqpoint{4.036652in}{0.794282in}}{\pgfqpoint{4.025602in}{0.794282in}}%
\pgfpathcurveto{\pgfqpoint{4.014552in}{0.794282in}}{\pgfqpoint{4.003953in}{0.789892in}}{\pgfqpoint{3.996139in}{0.782078in}}%
\pgfpathcurveto{\pgfqpoint{3.988326in}{0.774264in}}{\pgfqpoint{3.983936in}{0.763665in}}{\pgfqpoint{3.983936in}{0.752615in}}%
\pgfpathcurveto{\pgfqpoint{3.983936in}{0.741565in}}{\pgfqpoint{3.988326in}{0.730966in}}{\pgfqpoint{3.996139in}{0.723152in}}%
\pgfpathcurveto{\pgfqpoint{4.003953in}{0.715339in}}{\pgfqpoint{4.014552in}{0.710948in}}{\pgfqpoint{4.025602in}{0.710948in}}%
\pgfpathlineto{\pgfqpoint{4.025602in}{0.710948in}}%
\pgfpathclose%
\pgfusepath{stroke,fill}%
\end{pgfscope}%
\begin{pgfscope}%
\pgfpathrectangle{\pgfqpoint{2.963410in}{0.569136in}}{\pgfqpoint{2.177280in}{2.201755in}}%
\pgfusepath{clip}%
\pgfsetbuttcap%
\pgfsetroundjoin%
\definecolor{currentfill}{rgb}{0.121569,0.466667,0.705882}%
\pgfsetfillcolor{currentfill}%
\pgfsetlinewidth{0.481800pt}%
\definecolor{currentstroke}{rgb}{1.000000,1.000000,1.000000}%
\pgfsetstrokecolor{currentstroke}%
\pgfsetdash{}{0pt}%
\pgfpathmoveto{\pgfqpoint{4.202828in}{1.044548in}}%
\pgfpathcurveto{\pgfqpoint{4.213878in}{1.044548in}}{\pgfqpoint{4.224477in}{1.048938in}}{\pgfqpoint{4.232291in}{1.056752in}}%
\pgfpathcurveto{\pgfqpoint{4.240104in}{1.064565in}}{\pgfqpoint{4.244495in}{1.075164in}}{\pgfqpoint{4.244495in}{1.086214in}}%
\pgfpathcurveto{\pgfqpoint{4.244495in}{1.097265in}}{\pgfqpoint{4.240104in}{1.107864in}}{\pgfqpoint{4.232291in}{1.115677in}}%
\pgfpathcurveto{\pgfqpoint{4.224477in}{1.123491in}}{\pgfqpoint{4.213878in}{1.127881in}}{\pgfqpoint{4.202828in}{1.127881in}}%
\pgfpathcurveto{\pgfqpoint{4.191778in}{1.127881in}}{\pgfqpoint{4.181179in}{1.123491in}}{\pgfqpoint{4.173365in}{1.115677in}}%
\pgfpathcurveto{\pgfqpoint{4.165552in}{1.107864in}}{\pgfqpoint{4.161161in}{1.097265in}}{\pgfqpoint{4.161161in}{1.086214in}}%
\pgfpathcurveto{\pgfqpoint{4.161161in}{1.075164in}}{\pgfqpoint{4.165552in}{1.064565in}}{\pgfqpoint{4.173365in}{1.056752in}}%
\pgfpathcurveto{\pgfqpoint{4.181179in}{1.048938in}}{\pgfqpoint{4.191778in}{1.044548in}}{\pgfqpoint{4.202828in}{1.044548in}}%
\pgfpathlineto{\pgfqpoint{4.202828in}{1.044548in}}%
\pgfpathclose%
\pgfusepath{stroke,fill}%
\end{pgfscope}%
\begin{pgfscope}%
\pgfpathrectangle{\pgfqpoint{2.963410in}{0.569136in}}{\pgfqpoint{2.177280in}{2.201755in}}%
\pgfusepath{clip}%
\pgfsetbuttcap%
\pgfsetroundjoin%
\definecolor{currentfill}{rgb}{0.121569,0.466667,0.705882}%
\pgfsetfillcolor{currentfill}%
\pgfsetlinewidth{0.481800pt}%
\definecolor{currentstroke}{rgb}{1.000000,1.000000,1.000000}%
\pgfsetstrokecolor{currentstroke}%
\pgfsetdash{}{0pt}%
\pgfpathmoveto{\pgfqpoint{4.380054in}{0.877748in}}%
\pgfpathcurveto{\pgfqpoint{4.391104in}{0.877748in}}{\pgfqpoint{4.401703in}{0.882138in}}{\pgfqpoint{4.409517in}{0.889952in}}%
\pgfpathcurveto{\pgfqpoint{4.417330in}{0.897766in}}{\pgfqpoint{4.421721in}{0.908365in}}{\pgfqpoint{4.421721in}{0.919415in}}%
\pgfpathcurveto{\pgfqpoint{4.421721in}{0.930465in}}{\pgfqpoint{4.417330in}{0.941064in}}{\pgfqpoint{4.409517in}{0.948878in}}%
\pgfpathcurveto{\pgfqpoint{4.401703in}{0.956691in}}{\pgfqpoint{4.391104in}{0.961081in}}{\pgfqpoint{4.380054in}{0.961081in}}%
\pgfpathcurveto{\pgfqpoint{4.369004in}{0.961081in}}{\pgfqpoint{4.358405in}{0.956691in}}{\pgfqpoint{4.350591in}{0.948878in}}%
\pgfpathcurveto{\pgfqpoint{4.342777in}{0.941064in}}{\pgfqpoint{4.338387in}{0.930465in}}{\pgfqpoint{4.338387in}{0.919415in}}%
\pgfpathcurveto{\pgfqpoint{4.338387in}{0.908365in}}{\pgfqpoint{4.342777in}{0.897766in}}{\pgfqpoint{4.350591in}{0.889952in}}%
\pgfpathcurveto{\pgfqpoint{4.358405in}{0.882138in}}{\pgfqpoint{4.369004in}{0.877748in}}{\pgfqpoint{4.380054in}{0.877748in}}%
\pgfpathlineto{\pgfqpoint{4.380054in}{0.877748in}}%
\pgfpathclose%
\pgfusepath{stroke,fill}%
\end{pgfscope}%
\begin{pgfscope}%
\pgfpathrectangle{\pgfqpoint{2.963410in}{0.569136in}}{\pgfqpoint{2.177280in}{2.201755in}}%
\pgfusepath{clip}%
\pgfsetbuttcap%
\pgfsetroundjoin%
\definecolor{currentfill}{rgb}{0.121569,0.466667,0.705882}%
\pgfsetfillcolor{currentfill}%
\pgfsetlinewidth{0.481800pt}%
\definecolor{currentstroke}{rgb}{1.000000,1.000000,1.000000}%
\pgfsetstrokecolor{currentstroke}%
\pgfsetdash{}{0pt}%
\pgfpathmoveto{\pgfqpoint{3.907452in}{0.794348in}}%
\pgfpathcurveto{\pgfqpoint{3.918502in}{0.794348in}}{\pgfqpoint{3.929101in}{0.798739in}}{\pgfqpoint{3.936915in}{0.806552in}}%
\pgfpathcurveto{\pgfqpoint{3.944728in}{0.814366in}}{\pgfqpoint{3.949118in}{0.824965in}}{\pgfqpoint{3.949118in}{0.836015in}}%
\pgfpathcurveto{\pgfqpoint{3.949118in}{0.847065in}}{\pgfqpoint{3.944728in}{0.857664in}}{\pgfqpoint{3.936915in}{0.865478in}}%
\pgfpathcurveto{\pgfqpoint{3.929101in}{0.873291in}}{\pgfqpoint{3.918502in}{0.877682in}}{\pgfqpoint{3.907452in}{0.877682in}}%
\pgfpathcurveto{\pgfqpoint{3.896402in}{0.877682in}}{\pgfqpoint{3.885803in}{0.873291in}}{\pgfqpoint{3.877989in}{0.865478in}}%
\pgfpathcurveto{\pgfqpoint{3.870175in}{0.857664in}}{\pgfqpoint{3.865785in}{0.847065in}}{\pgfqpoint{3.865785in}{0.836015in}}%
\pgfpathcurveto{\pgfqpoint{3.865785in}{0.824965in}}{\pgfqpoint{3.870175in}{0.814366in}}{\pgfqpoint{3.877989in}{0.806552in}}%
\pgfpathcurveto{\pgfqpoint{3.885803in}{0.798739in}}{\pgfqpoint{3.896402in}{0.794348in}}{\pgfqpoint{3.907452in}{0.794348in}}%
\pgfpathlineto{\pgfqpoint{3.907452in}{0.794348in}}%
\pgfpathclose%
\pgfusepath{stroke,fill}%
\end{pgfscope}%
\begin{pgfscope}%
\pgfpathrectangle{\pgfqpoint{2.963410in}{0.569136in}}{\pgfqpoint{2.177280in}{2.201755in}}%
\pgfusepath{clip}%
\pgfsetbuttcap%
\pgfsetroundjoin%
\definecolor{currentfill}{rgb}{0.121569,0.466667,0.705882}%
\pgfsetfillcolor{currentfill}%
\pgfsetlinewidth{0.481800pt}%
\definecolor{currentstroke}{rgb}{1.000000,1.000000,1.000000}%
\pgfsetstrokecolor{currentstroke}%
\pgfsetdash{}{0pt}%
\pgfpathmoveto{\pgfqpoint{4.380054in}{0.710948in}}%
\pgfpathcurveto{\pgfqpoint{4.391104in}{0.710948in}}{\pgfqpoint{4.401703in}{0.715339in}}{\pgfqpoint{4.409517in}{0.723152in}}%
\pgfpathcurveto{\pgfqpoint{4.417330in}{0.730966in}}{\pgfqpoint{4.421721in}{0.741565in}}{\pgfqpoint{4.421721in}{0.752615in}}%
\pgfpathcurveto{\pgfqpoint{4.421721in}{0.763665in}}{\pgfqpoint{4.417330in}{0.774264in}}{\pgfqpoint{4.409517in}{0.782078in}}%
\pgfpathcurveto{\pgfqpoint{4.401703in}{0.789892in}}{\pgfqpoint{4.391104in}{0.794282in}}{\pgfqpoint{4.380054in}{0.794282in}}%
\pgfpathcurveto{\pgfqpoint{4.369004in}{0.794282in}}{\pgfqpoint{4.358405in}{0.789892in}}{\pgfqpoint{4.350591in}{0.782078in}}%
\pgfpathcurveto{\pgfqpoint{4.342777in}{0.774264in}}{\pgfqpoint{4.338387in}{0.763665in}}{\pgfqpoint{4.338387in}{0.752615in}}%
\pgfpathcurveto{\pgfqpoint{4.338387in}{0.741565in}}{\pgfqpoint{4.342777in}{0.730966in}}{\pgfqpoint{4.350591in}{0.723152in}}%
\pgfpathcurveto{\pgfqpoint{4.358405in}{0.715339in}}{\pgfqpoint{4.369004in}{0.710948in}}{\pgfqpoint{4.380054in}{0.710948in}}%
\pgfpathlineto{\pgfqpoint{4.380054in}{0.710948in}}%
\pgfpathclose%
\pgfusepath{stroke,fill}%
\end{pgfscope}%
\begin{pgfscope}%
\pgfpathrectangle{\pgfqpoint{2.963410in}{0.569136in}}{\pgfqpoint{2.177280in}{2.201755in}}%
\pgfusepath{clip}%
\pgfsetbuttcap%
\pgfsetroundjoin%
\definecolor{currentfill}{rgb}{0.121569,0.466667,0.705882}%
\pgfsetfillcolor{currentfill}%
\pgfsetlinewidth{0.481800pt}%
\definecolor{currentstroke}{rgb}{1.000000,1.000000,1.000000}%
\pgfsetstrokecolor{currentstroke}%
\pgfsetdash{}{0pt}%
\pgfpathmoveto{\pgfqpoint{4.025602in}{0.710948in}}%
\pgfpathcurveto{\pgfqpoint{4.036652in}{0.710948in}}{\pgfqpoint{4.047251in}{0.715339in}}{\pgfqpoint{4.055065in}{0.723152in}}%
\pgfpathcurveto{\pgfqpoint{4.062879in}{0.730966in}}{\pgfqpoint{4.067269in}{0.741565in}}{\pgfqpoint{4.067269in}{0.752615in}}%
\pgfpathcurveto{\pgfqpoint{4.067269in}{0.763665in}}{\pgfqpoint{4.062879in}{0.774264in}}{\pgfqpoint{4.055065in}{0.782078in}}%
\pgfpathcurveto{\pgfqpoint{4.047251in}{0.789892in}}{\pgfqpoint{4.036652in}{0.794282in}}{\pgfqpoint{4.025602in}{0.794282in}}%
\pgfpathcurveto{\pgfqpoint{4.014552in}{0.794282in}}{\pgfqpoint{4.003953in}{0.789892in}}{\pgfqpoint{3.996139in}{0.782078in}}%
\pgfpathcurveto{\pgfqpoint{3.988326in}{0.774264in}}{\pgfqpoint{3.983936in}{0.763665in}}{\pgfqpoint{3.983936in}{0.752615in}}%
\pgfpathcurveto{\pgfqpoint{3.983936in}{0.741565in}}{\pgfqpoint{3.988326in}{0.730966in}}{\pgfqpoint{3.996139in}{0.723152in}}%
\pgfpathcurveto{\pgfqpoint{4.003953in}{0.715339in}}{\pgfqpoint{4.014552in}{0.710948in}}{\pgfqpoint{4.025602in}{0.710948in}}%
\pgfpathlineto{\pgfqpoint{4.025602in}{0.710948in}}%
\pgfpathclose%
\pgfusepath{stroke,fill}%
\end{pgfscope}%
\begin{pgfscope}%
\pgfpathrectangle{\pgfqpoint{2.963410in}{0.569136in}}{\pgfqpoint{2.177280in}{2.201755in}}%
\pgfusepath{clip}%
\pgfsetbuttcap%
\pgfsetroundjoin%
\definecolor{currentfill}{rgb}{0.121569,0.466667,0.705882}%
\pgfsetfillcolor{currentfill}%
\pgfsetlinewidth{0.481800pt}%
\definecolor{currentstroke}{rgb}{1.000000,1.000000,1.000000}%
\pgfsetstrokecolor{currentstroke}%
\pgfsetdash{}{0pt}%
\pgfpathmoveto{\pgfqpoint{4.320979in}{0.710948in}}%
\pgfpathcurveto{\pgfqpoint{4.332029in}{0.710948in}}{\pgfqpoint{4.342628in}{0.715339in}}{\pgfqpoint{4.350441in}{0.723152in}}%
\pgfpathcurveto{\pgfqpoint{4.358255in}{0.730966in}}{\pgfqpoint{4.362645in}{0.741565in}}{\pgfqpoint{4.362645in}{0.752615in}}%
\pgfpathcurveto{\pgfqpoint{4.362645in}{0.763665in}}{\pgfqpoint{4.358255in}{0.774264in}}{\pgfqpoint{4.350441in}{0.782078in}}%
\pgfpathcurveto{\pgfqpoint{4.342628in}{0.789892in}}{\pgfqpoint{4.332029in}{0.794282in}}{\pgfqpoint{4.320979in}{0.794282in}}%
\pgfpathcurveto{\pgfqpoint{4.309928in}{0.794282in}}{\pgfqpoint{4.299329in}{0.789892in}}{\pgfqpoint{4.291516in}{0.782078in}}%
\pgfpathcurveto{\pgfqpoint{4.283702in}{0.774264in}}{\pgfqpoint{4.279312in}{0.763665in}}{\pgfqpoint{4.279312in}{0.752615in}}%
\pgfpathcurveto{\pgfqpoint{4.279312in}{0.741565in}}{\pgfqpoint{4.283702in}{0.730966in}}{\pgfqpoint{4.291516in}{0.723152in}}%
\pgfpathcurveto{\pgfqpoint{4.299329in}{0.715339in}}{\pgfqpoint{4.309928in}{0.710948in}}{\pgfqpoint{4.320979in}{0.710948in}}%
\pgfpathlineto{\pgfqpoint{4.320979in}{0.710948in}}%
\pgfpathclose%
\pgfusepath{stroke,fill}%
\end{pgfscope}%
\begin{pgfscope}%
\pgfpathrectangle{\pgfqpoint{2.963410in}{0.569136in}}{\pgfqpoint{2.177280in}{2.201755in}}%
\pgfusepath{clip}%
\pgfsetbuttcap%
\pgfsetroundjoin%
\definecolor{currentfill}{rgb}{0.121569,0.466667,0.705882}%
\pgfsetfillcolor{currentfill}%
\pgfsetlinewidth{0.481800pt}%
\definecolor{currentstroke}{rgb}{1.000000,1.000000,1.000000}%
\pgfsetstrokecolor{currentstroke}%
\pgfsetdash{}{0pt}%
\pgfpathmoveto{\pgfqpoint{4.084678in}{0.710948in}}%
\pgfpathcurveto{\pgfqpoint{4.095728in}{0.710948in}}{\pgfqpoint{4.106327in}{0.715339in}}{\pgfqpoint{4.114140in}{0.723152in}}%
\pgfpathcurveto{\pgfqpoint{4.121954in}{0.730966in}}{\pgfqpoint{4.126344in}{0.741565in}}{\pgfqpoint{4.126344in}{0.752615in}}%
\pgfpathcurveto{\pgfqpoint{4.126344in}{0.763665in}}{\pgfqpoint{4.121954in}{0.774264in}}{\pgfqpoint{4.114140in}{0.782078in}}%
\pgfpathcurveto{\pgfqpoint{4.106327in}{0.789892in}}{\pgfqpoint{4.095728in}{0.794282in}}{\pgfqpoint{4.084678in}{0.794282in}}%
\pgfpathcurveto{\pgfqpoint{4.073627in}{0.794282in}}{\pgfqpoint{4.063028in}{0.789892in}}{\pgfqpoint{4.055215in}{0.782078in}}%
\pgfpathcurveto{\pgfqpoint{4.047401in}{0.774264in}}{\pgfqpoint{4.043011in}{0.763665in}}{\pgfqpoint{4.043011in}{0.752615in}}%
\pgfpathcurveto{\pgfqpoint{4.043011in}{0.741565in}}{\pgfqpoint{4.047401in}{0.730966in}}{\pgfqpoint{4.055215in}{0.723152in}}%
\pgfpathcurveto{\pgfqpoint{4.063028in}{0.715339in}}{\pgfqpoint{4.073627in}{0.710948in}}{\pgfqpoint{4.084678in}{0.710948in}}%
\pgfpathlineto{\pgfqpoint{4.084678in}{0.710948in}}%
\pgfpathclose%
\pgfusepath{stroke,fill}%
\end{pgfscope}%
\begin{pgfscope}%
\pgfpathrectangle{\pgfqpoint{2.963410in}{0.569136in}}{\pgfqpoint{2.177280in}{2.201755in}}%
\pgfusepath{clip}%
\pgfsetbuttcap%
\pgfsetroundjoin%
\definecolor{currentfill}{rgb}{1.000000,0.498039,0.054902}%
\pgfsetfillcolor{currentfill}%
\pgfsetlinewidth{0.481800pt}%
\definecolor{currentstroke}{rgb}{1.000000,1.000000,1.000000}%
\pgfsetstrokecolor{currentstroke}%
\pgfsetdash{}{0pt}%
\pgfpathmoveto{\pgfqpoint{4.025602in}{1.711746in}}%
\pgfpathcurveto{\pgfqpoint{4.036652in}{1.711746in}}{\pgfqpoint{4.047251in}{1.716137in}}{\pgfqpoint{4.055065in}{1.723950in}}%
\pgfpathcurveto{\pgfqpoint{4.062879in}{1.731764in}}{\pgfqpoint{4.067269in}{1.742363in}}{\pgfqpoint{4.067269in}{1.753413in}}%
\pgfpathcurveto{\pgfqpoint{4.067269in}{1.764463in}}{\pgfqpoint{4.062879in}{1.775062in}}{\pgfqpoint{4.055065in}{1.782876in}}%
\pgfpathcurveto{\pgfqpoint{4.047251in}{1.790689in}}{\pgfqpoint{4.036652in}{1.795080in}}{\pgfqpoint{4.025602in}{1.795080in}}%
\pgfpathcurveto{\pgfqpoint{4.014552in}{1.795080in}}{\pgfqpoint{4.003953in}{1.790689in}}{\pgfqpoint{3.996139in}{1.782876in}}%
\pgfpathcurveto{\pgfqpoint{3.988326in}{1.775062in}}{\pgfqpoint{3.983936in}{1.764463in}}{\pgfqpoint{3.983936in}{1.753413in}}%
\pgfpathcurveto{\pgfqpoint{3.983936in}{1.742363in}}{\pgfqpoint{3.988326in}{1.731764in}}{\pgfqpoint{3.996139in}{1.723950in}}%
\pgfpathcurveto{\pgfqpoint{4.003953in}{1.716137in}}{\pgfqpoint{4.014552in}{1.711746in}}{\pgfqpoint{4.025602in}{1.711746in}}%
\pgfpathlineto{\pgfqpoint{4.025602in}{1.711746in}}%
\pgfpathclose%
\pgfusepath{stroke,fill}%
\end{pgfscope}%
\begin{pgfscope}%
\pgfpathrectangle{\pgfqpoint{2.963410in}{0.569136in}}{\pgfqpoint{2.177280in}{2.201755in}}%
\pgfusepath{clip}%
\pgfsetbuttcap%
\pgfsetroundjoin%
\definecolor{currentfill}{rgb}{1.000000,0.498039,0.054902}%
\pgfsetfillcolor{currentfill}%
\pgfsetlinewidth{0.481800pt}%
\definecolor{currentstroke}{rgb}{1.000000,1.000000,1.000000}%
\pgfsetstrokecolor{currentstroke}%
\pgfsetdash{}{0pt}%
\pgfpathmoveto{\pgfqpoint{4.025602in}{1.795146in}}%
\pgfpathcurveto{\pgfqpoint{4.036652in}{1.795146in}}{\pgfqpoint{4.047251in}{1.799536in}}{\pgfqpoint{4.055065in}{1.807350in}}%
\pgfpathcurveto{\pgfqpoint{4.062879in}{1.815164in}}{\pgfqpoint{4.067269in}{1.825763in}}{\pgfqpoint{4.067269in}{1.836813in}}%
\pgfpathcurveto{\pgfqpoint{4.067269in}{1.847863in}}{\pgfqpoint{4.062879in}{1.858462in}}{\pgfqpoint{4.055065in}{1.866276in}}%
\pgfpathcurveto{\pgfqpoint{4.047251in}{1.874089in}}{\pgfqpoint{4.036652in}{1.878479in}}{\pgfqpoint{4.025602in}{1.878479in}}%
\pgfpathcurveto{\pgfqpoint{4.014552in}{1.878479in}}{\pgfqpoint{4.003953in}{1.874089in}}{\pgfqpoint{3.996139in}{1.866276in}}%
\pgfpathcurveto{\pgfqpoint{3.988326in}{1.858462in}}{\pgfqpoint{3.983936in}{1.847863in}}{\pgfqpoint{3.983936in}{1.836813in}}%
\pgfpathcurveto{\pgfqpoint{3.983936in}{1.825763in}}{\pgfqpoint{3.988326in}{1.815164in}}{\pgfqpoint{3.996139in}{1.807350in}}%
\pgfpathcurveto{\pgfqpoint{4.003953in}{1.799536in}}{\pgfqpoint{4.014552in}{1.795146in}}{\pgfqpoint{4.025602in}{1.795146in}}%
\pgfpathlineto{\pgfqpoint{4.025602in}{1.795146in}}%
\pgfpathclose%
\pgfusepath{stroke,fill}%
\end{pgfscope}%
\begin{pgfscope}%
\pgfpathrectangle{\pgfqpoint{2.963410in}{0.569136in}}{\pgfqpoint{2.177280in}{2.201755in}}%
\pgfusepath{clip}%
\pgfsetbuttcap%
\pgfsetroundjoin%
\definecolor{currentfill}{rgb}{1.000000,0.498039,0.054902}%
\pgfsetfillcolor{currentfill}%
\pgfsetlinewidth{0.481800pt}%
\definecolor{currentstroke}{rgb}{1.000000,1.000000,1.000000}%
\pgfsetstrokecolor{currentstroke}%
\pgfsetdash{}{0pt}%
\pgfpathmoveto{\pgfqpoint{3.966527in}{1.795146in}}%
\pgfpathcurveto{\pgfqpoint{3.977577in}{1.795146in}}{\pgfqpoint{3.988176in}{1.799536in}}{\pgfqpoint{3.995990in}{1.807350in}}%
\pgfpathcurveto{\pgfqpoint{4.003803in}{1.815164in}}{\pgfqpoint{4.008194in}{1.825763in}}{\pgfqpoint{4.008194in}{1.836813in}}%
\pgfpathcurveto{\pgfqpoint{4.008194in}{1.847863in}}{\pgfqpoint{4.003803in}{1.858462in}}{\pgfqpoint{3.995990in}{1.866276in}}%
\pgfpathcurveto{\pgfqpoint{3.988176in}{1.874089in}}{\pgfqpoint{3.977577in}{1.878479in}}{\pgfqpoint{3.966527in}{1.878479in}}%
\pgfpathcurveto{\pgfqpoint{3.955477in}{1.878479in}}{\pgfqpoint{3.944878in}{1.874089in}}{\pgfqpoint{3.937064in}{1.866276in}}%
\pgfpathcurveto{\pgfqpoint{3.929251in}{1.858462in}}{\pgfqpoint{3.924860in}{1.847863in}}{\pgfqpoint{3.924860in}{1.836813in}}%
\pgfpathcurveto{\pgfqpoint{3.924860in}{1.825763in}}{\pgfqpoint{3.929251in}{1.815164in}}{\pgfqpoint{3.937064in}{1.807350in}}%
\pgfpathcurveto{\pgfqpoint{3.944878in}{1.799536in}}{\pgfqpoint{3.955477in}{1.795146in}}{\pgfqpoint{3.966527in}{1.795146in}}%
\pgfpathlineto{\pgfqpoint{3.966527in}{1.795146in}}%
\pgfpathclose%
\pgfusepath{stroke,fill}%
\end{pgfscope}%
\begin{pgfscope}%
\pgfpathrectangle{\pgfqpoint{2.963410in}{0.569136in}}{\pgfqpoint{2.177280in}{2.201755in}}%
\pgfusepath{clip}%
\pgfsetbuttcap%
\pgfsetroundjoin%
\definecolor{currentfill}{rgb}{1.000000,0.498039,0.054902}%
\pgfsetfillcolor{currentfill}%
\pgfsetlinewidth{0.481800pt}%
\definecolor{currentstroke}{rgb}{1.000000,1.000000,1.000000}%
\pgfsetstrokecolor{currentstroke}%
\pgfsetdash{}{0pt}%
\pgfpathmoveto{\pgfqpoint{3.493925in}{1.628346in}}%
\pgfpathcurveto{\pgfqpoint{3.504975in}{1.628346in}}{\pgfqpoint{3.515574in}{1.632737in}}{\pgfqpoint{3.523388in}{1.640550in}}%
\pgfpathcurveto{\pgfqpoint{3.531201in}{1.648364in}}{\pgfqpoint{3.535592in}{1.658963in}}{\pgfqpoint{3.535592in}{1.670013in}}%
\pgfpathcurveto{\pgfqpoint{3.535592in}{1.681063in}}{\pgfqpoint{3.531201in}{1.691662in}}{\pgfqpoint{3.523388in}{1.699476in}}%
\pgfpathcurveto{\pgfqpoint{3.515574in}{1.707290in}}{\pgfqpoint{3.504975in}{1.711680in}}{\pgfqpoint{3.493925in}{1.711680in}}%
\pgfpathcurveto{\pgfqpoint{3.482875in}{1.711680in}}{\pgfqpoint{3.472276in}{1.707290in}}{\pgfqpoint{3.464462in}{1.699476in}}%
\pgfpathcurveto{\pgfqpoint{3.456648in}{1.691662in}}{\pgfqpoint{3.452258in}{1.681063in}}{\pgfqpoint{3.452258in}{1.670013in}}%
\pgfpathcurveto{\pgfqpoint{3.452258in}{1.658963in}}{\pgfqpoint{3.456648in}{1.648364in}}{\pgfqpoint{3.464462in}{1.640550in}}%
\pgfpathcurveto{\pgfqpoint{3.472276in}{1.632737in}}{\pgfqpoint{3.482875in}{1.628346in}}{\pgfqpoint{3.493925in}{1.628346in}}%
\pgfpathlineto{\pgfqpoint{3.493925in}{1.628346in}}%
\pgfpathclose%
\pgfusepath{stroke,fill}%
\end{pgfscope}%
\begin{pgfscope}%
\pgfpathrectangle{\pgfqpoint{2.963410in}{0.569136in}}{\pgfqpoint{2.177280in}{2.201755in}}%
\pgfusepath{clip}%
\pgfsetbuttcap%
\pgfsetroundjoin%
\definecolor{currentfill}{rgb}{1.000000,0.498039,0.054902}%
\pgfsetfillcolor{currentfill}%
\pgfsetlinewidth{0.481800pt}%
\definecolor{currentstroke}{rgb}{1.000000,1.000000,1.000000}%
\pgfsetstrokecolor{currentstroke}%
\pgfsetdash{}{0pt}%
\pgfpathmoveto{\pgfqpoint{3.789301in}{1.795146in}}%
\pgfpathcurveto{\pgfqpoint{3.800351in}{1.795146in}}{\pgfqpoint{3.810950in}{1.799536in}}{\pgfqpoint{3.818764in}{1.807350in}}%
\pgfpathcurveto{\pgfqpoint{3.826578in}{1.815164in}}{\pgfqpoint{3.830968in}{1.825763in}}{\pgfqpoint{3.830968in}{1.836813in}}%
\pgfpathcurveto{\pgfqpoint{3.830968in}{1.847863in}}{\pgfqpoint{3.826578in}{1.858462in}}{\pgfqpoint{3.818764in}{1.866276in}}%
\pgfpathcurveto{\pgfqpoint{3.810950in}{1.874089in}}{\pgfqpoint{3.800351in}{1.878479in}}{\pgfqpoint{3.789301in}{1.878479in}}%
\pgfpathcurveto{\pgfqpoint{3.778251in}{1.878479in}}{\pgfqpoint{3.767652in}{1.874089in}}{\pgfqpoint{3.759838in}{1.866276in}}%
\pgfpathcurveto{\pgfqpoint{3.752025in}{1.858462in}}{\pgfqpoint{3.747635in}{1.847863in}}{\pgfqpoint{3.747635in}{1.836813in}}%
\pgfpathcurveto{\pgfqpoint{3.747635in}{1.825763in}}{\pgfqpoint{3.752025in}{1.815164in}}{\pgfqpoint{3.759838in}{1.807350in}}%
\pgfpathcurveto{\pgfqpoint{3.767652in}{1.799536in}}{\pgfqpoint{3.778251in}{1.795146in}}{\pgfqpoint{3.789301in}{1.795146in}}%
\pgfpathlineto{\pgfqpoint{3.789301in}{1.795146in}}%
\pgfpathclose%
\pgfusepath{stroke,fill}%
\end{pgfscope}%
\begin{pgfscope}%
\pgfpathrectangle{\pgfqpoint{2.963410in}{0.569136in}}{\pgfqpoint{2.177280in}{2.201755in}}%
\pgfusepath{clip}%
\pgfsetbuttcap%
\pgfsetroundjoin%
\definecolor{currentfill}{rgb}{1.000000,0.498039,0.054902}%
\pgfsetfillcolor{currentfill}%
\pgfsetlinewidth{0.481800pt}%
\definecolor{currentstroke}{rgb}{1.000000,1.000000,1.000000}%
\pgfsetstrokecolor{currentstroke}%
\pgfsetdash{}{0pt}%
\pgfpathmoveto{\pgfqpoint{3.789301in}{1.628346in}}%
\pgfpathcurveto{\pgfqpoint{3.800351in}{1.628346in}}{\pgfqpoint{3.810950in}{1.632737in}}{\pgfqpoint{3.818764in}{1.640550in}}%
\pgfpathcurveto{\pgfqpoint{3.826578in}{1.648364in}}{\pgfqpoint{3.830968in}{1.658963in}}{\pgfqpoint{3.830968in}{1.670013in}}%
\pgfpathcurveto{\pgfqpoint{3.830968in}{1.681063in}}{\pgfqpoint{3.826578in}{1.691662in}}{\pgfqpoint{3.818764in}{1.699476in}}%
\pgfpathcurveto{\pgfqpoint{3.810950in}{1.707290in}}{\pgfqpoint{3.800351in}{1.711680in}}{\pgfqpoint{3.789301in}{1.711680in}}%
\pgfpathcurveto{\pgfqpoint{3.778251in}{1.711680in}}{\pgfqpoint{3.767652in}{1.707290in}}{\pgfqpoint{3.759838in}{1.699476in}}%
\pgfpathcurveto{\pgfqpoint{3.752025in}{1.691662in}}{\pgfqpoint{3.747635in}{1.681063in}}{\pgfqpoint{3.747635in}{1.670013in}}%
\pgfpathcurveto{\pgfqpoint{3.747635in}{1.658963in}}{\pgfqpoint{3.752025in}{1.648364in}}{\pgfqpoint{3.759838in}{1.640550in}}%
\pgfpathcurveto{\pgfqpoint{3.767652in}{1.632737in}}{\pgfqpoint{3.778251in}{1.628346in}}{\pgfqpoint{3.789301in}{1.628346in}}%
\pgfpathlineto{\pgfqpoint{3.789301in}{1.628346in}}%
\pgfpathclose%
\pgfusepath{stroke,fill}%
\end{pgfscope}%
\begin{pgfscope}%
\pgfpathrectangle{\pgfqpoint{2.963410in}{0.569136in}}{\pgfqpoint{2.177280in}{2.201755in}}%
\pgfusepath{clip}%
\pgfsetbuttcap%
\pgfsetroundjoin%
\definecolor{currentfill}{rgb}{1.000000,0.498039,0.054902}%
\pgfsetfillcolor{currentfill}%
\pgfsetlinewidth{0.481800pt}%
\definecolor{currentstroke}{rgb}{1.000000,1.000000,1.000000}%
\pgfsetstrokecolor{currentstroke}%
\pgfsetdash{}{0pt}%
\pgfpathmoveto{\pgfqpoint{4.084678in}{1.878546in}}%
\pgfpathcurveto{\pgfqpoint{4.095728in}{1.878546in}}{\pgfqpoint{4.106327in}{1.882936in}}{\pgfqpoint{4.114140in}{1.890750in}}%
\pgfpathcurveto{\pgfqpoint{4.121954in}{1.898563in}}{\pgfqpoint{4.126344in}{1.909162in}}{\pgfqpoint{4.126344in}{1.920213in}}%
\pgfpathcurveto{\pgfqpoint{4.126344in}{1.931263in}}{\pgfqpoint{4.121954in}{1.941862in}}{\pgfqpoint{4.114140in}{1.949675in}}%
\pgfpathcurveto{\pgfqpoint{4.106327in}{1.957489in}}{\pgfqpoint{4.095728in}{1.961879in}}{\pgfqpoint{4.084678in}{1.961879in}}%
\pgfpathcurveto{\pgfqpoint{4.073627in}{1.961879in}}{\pgfqpoint{4.063028in}{1.957489in}}{\pgfqpoint{4.055215in}{1.949675in}}%
\pgfpathcurveto{\pgfqpoint{4.047401in}{1.941862in}}{\pgfqpoint{4.043011in}{1.931263in}}{\pgfqpoint{4.043011in}{1.920213in}}%
\pgfpathcurveto{\pgfqpoint{4.043011in}{1.909162in}}{\pgfqpoint{4.047401in}{1.898563in}}{\pgfqpoint{4.055215in}{1.890750in}}%
\pgfpathcurveto{\pgfqpoint{4.063028in}{1.882936in}}{\pgfqpoint{4.073627in}{1.878546in}}{\pgfqpoint{4.084678in}{1.878546in}}%
\pgfpathlineto{\pgfqpoint{4.084678in}{1.878546in}}%
\pgfpathclose%
\pgfusepath{stroke,fill}%
\end{pgfscope}%
\begin{pgfscope}%
\pgfpathrectangle{\pgfqpoint{2.963410in}{0.569136in}}{\pgfqpoint{2.177280in}{2.201755in}}%
\pgfusepath{clip}%
\pgfsetbuttcap%
\pgfsetroundjoin%
\definecolor{currentfill}{rgb}{1.000000,0.498039,0.054902}%
\pgfsetfillcolor{currentfill}%
\pgfsetlinewidth{0.481800pt}%
\definecolor{currentstroke}{rgb}{1.000000,1.000000,1.000000}%
\pgfsetstrokecolor{currentstroke}%
\pgfsetdash{}{0pt}%
\pgfpathmoveto{\pgfqpoint{3.553000in}{1.378147in}}%
\pgfpathcurveto{\pgfqpoint{3.564050in}{1.378147in}}{\pgfqpoint{3.574649in}{1.382537in}}{\pgfqpoint{3.582463in}{1.390351in}}%
\pgfpathcurveto{\pgfqpoint{3.590277in}{1.398164in}}{\pgfqpoint{3.594667in}{1.408764in}}{\pgfqpoint{3.594667in}{1.419814in}}%
\pgfpathcurveto{\pgfqpoint{3.594667in}{1.430864in}}{\pgfqpoint{3.590277in}{1.441463in}}{\pgfqpoint{3.582463in}{1.449276in}}%
\pgfpathcurveto{\pgfqpoint{3.574649in}{1.457090in}}{\pgfqpoint{3.564050in}{1.461480in}}{\pgfqpoint{3.553000in}{1.461480in}}%
\pgfpathcurveto{\pgfqpoint{3.541950in}{1.461480in}}{\pgfqpoint{3.531351in}{1.457090in}}{\pgfqpoint{3.523537in}{1.449276in}}%
\pgfpathcurveto{\pgfqpoint{3.515724in}{1.441463in}}{\pgfqpoint{3.511333in}{1.430864in}}{\pgfqpoint{3.511333in}{1.419814in}}%
\pgfpathcurveto{\pgfqpoint{3.511333in}{1.408764in}}{\pgfqpoint{3.515724in}{1.398164in}}{\pgfqpoint{3.523537in}{1.390351in}}%
\pgfpathcurveto{\pgfqpoint{3.531351in}{1.382537in}}{\pgfqpoint{3.541950in}{1.378147in}}{\pgfqpoint{3.553000in}{1.378147in}}%
\pgfpathlineto{\pgfqpoint{3.553000in}{1.378147in}}%
\pgfpathclose%
\pgfusepath{stroke,fill}%
\end{pgfscope}%
\begin{pgfscope}%
\pgfpathrectangle{\pgfqpoint{2.963410in}{0.569136in}}{\pgfqpoint{2.177280in}{2.201755in}}%
\pgfusepath{clip}%
\pgfsetbuttcap%
\pgfsetroundjoin%
\definecolor{currentfill}{rgb}{1.000000,0.498039,0.054902}%
\pgfsetfillcolor{currentfill}%
\pgfsetlinewidth{0.481800pt}%
\definecolor{currentstroke}{rgb}{1.000000,1.000000,1.000000}%
\pgfsetstrokecolor{currentstroke}%
\pgfsetdash{}{0pt}%
\pgfpathmoveto{\pgfqpoint{3.848376in}{1.628346in}}%
\pgfpathcurveto{\pgfqpoint{3.859427in}{1.628346in}}{\pgfqpoint{3.870026in}{1.632737in}}{\pgfqpoint{3.877839in}{1.640550in}}%
\pgfpathcurveto{\pgfqpoint{3.885653in}{1.648364in}}{\pgfqpoint{3.890043in}{1.658963in}}{\pgfqpoint{3.890043in}{1.670013in}}%
\pgfpathcurveto{\pgfqpoint{3.890043in}{1.681063in}}{\pgfqpoint{3.885653in}{1.691662in}}{\pgfqpoint{3.877839in}{1.699476in}}%
\pgfpathcurveto{\pgfqpoint{3.870026in}{1.707290in}}{\pgfqpoint{3.859427in}{1.711680in}}{\pgfqpoint{3.848376in}{1.711680in}}%
\pgfpathcurveto{\pgfqpoint{3.837326in}{1.711680in}}{\pgfqpoint{3.826727in}{1.707290in}}{\pgfqpoint{3.818914in}{1.699476in}}%
\pgfpathcurveto{\pgfqpoint{3.811100in}{1.691662in}}{\pgfqpoint{3.806710in}{1.681063in}}{\pgfqpoint{3.806710in}{1.670013in}}%
\pgfpathcurveto{\pgfqpoint{3.806710in}{1.658963in}}{\pgfqpoint{3.811100in}{1.648364in}}{\pgfqpoint{3.818914in}{1.640550in}}%
\pgfpathcurveto{\pgfqpoint{3.826727in}{1.632737in}}{\pgfqpoint{3.837326in}{1.628346in}}{\pgfqpoint{3.848376in}{1.628346in}}%
\pgfpathlineto{\pgfqpoint{3.848376in}{1.628346in}}%
\pgfpathclose%
\pgfusepath{stroke,fill}%
\end{pgfscope}%
\begin{pgfscope}%
\pgfpathrectangle{\pgfqpoint{2.963410in}{0.569136in}}{\pgfqpoint{2.177280in}{2.201755in}}%
\pgfusepath{clip}%
\pgfsetbuttcap%
\pgfsetroundjoin%
\definecolor{currentfill}{rgb}{1.000000,0.498039,0.054902}%
\pgfsetfillcolor{currentfill}%
\pgfsetlinewidth{0.481800pt}%
\definecolor{currentstroke}{rgb}{1.000000,1.000000,1.000000}%
\pgfsetstrokecolor{currentstroke}%
\pgfsetdash{}{0pt}%
\pgfpathmoveto{\pgfqpoint{3.730226in}{1.711746in}}%
\pgfpathcurveto{\pgfqpoint{3.741276in}{1.711746in}}{\pgfqpoint{3.751875in}{1.716137in}}{\pgfqpoint{3.759689in}{1.723950in}}%
\pgfpathcurveto{\pgfqpoint{3.767502in}{1.731764in}}{\pgfqpoint{3.771893in}{1.742363in}}{\pgfqpoint{3.771893in}{1.753413in}}%
\pgfpathcurveto{\pgfqpoint{3.771893in}{1.764463in}}{\pgfqpoint{3.767502in}{1.775062in}}{\pgfqpoint{3.759689in}{1.782876in}}%
\pgfpathcurveto{\pgfqpoint{3.751875in}{1.790689in}}{\pgfqpoint{3.741276in}{1.795080in}}{\pgfqpoint{3.730226in}{1.795080in}}%
\pgfpathcurveto{\pgfqpoint{3.719176in}{1.795080in}}{\pgfqpoint{3.708577in}{1.790689in}}{\pgfqpoint{3.700763in}{1.782876in}}%
\pgfpathcurveto{\pgfqpoint{3.692950in}{1.775062in}}{\pgfqpoint{3.688559in}{1.764463in}}{\pgfqpoint{3.688559in}{1.753413in}}%
\pgfpathcurveto{\pgfqpoint{3.688559in}{1.742363in}}{\pgfqpoint{3.692950in}{1.731764in}}{\pgfqpoint{3.700763in}{1.723950in}}%
\pgfpathcurveto{\pgfqpoint{3.708577in}{1.716137in}}{\pgfqpoint{3.719176in}{1.711746in}}{\pgfqpoint{3.730226in}{1.711746in}}%
\pgfpathlineto{\pgfqpoint{3.730226in}{1.711746in}}%
\pgfpathclose%
\pgfusepath{stroke,fill}%
\end{pgfscope}%
\begin{pgfscope}%
\pgfpathrectangle{\pgfqpoint{2.963410in}{0.569136in}}{\pgfqpoint{2.177280in}{2.201755in}}%
\pgfusepath{clip}%
\pgfsetbuttcap%
\pgfsetroundjoin%
\definecolor{currentfill}{rgb}{1.000000,0.498039,0.054902}%
\pgfsetfillcolor{currentfill}%
\pgfsetlinewidth{0.481800pt}%
\definecolor{currentstroke}{rgb}{1.000000,1.000000,1.000000}%
\pgfsetstrokecolor{currentstroke}%
\pgfsetdash{}{0pt}%
\pgfpathmoveto{\pgfqpoint{3.316699in}{1.378147in}}%
\pgfpathcurveto{\pgfqpoint{3.327749in}{1.378147in}}{\pgfqpoint{3.338348in}{1.382537in}}{\pgfqpoint{3.346162in}{1.390351in}}%
\pgfpathcurveto{\pgfqpoint{3.353975in}{1.398164in}}{\pgfqpoint{3.358366in}{1.408764in}}{\pgfqpoint{3.358366in}{1.419814in}}%
\pgfpathcurveto{\pgfqpoint{3.358366in}{1.430864in}}{\pgfqpoint{3.353975in}{1.441463in}}{\pgfqpoint{3.346162in}{1.449276in}}%
\pgfpathcurveto{\pgfqpoint{3.338348in}{1.457090in}}{\pgfqpoint{3.327749in}{1.461480in}}{\pgfqpoint{3.316699in}{1.461480in}}%
\pgfpathcurveto{\pgfqpoint{3.305649in}{1.461480in}}{\pgfqpoint{3.295050in}{1.457090in}}{\pgfqpoint{3.287236in}{1.449276in}}%
\pgfpathcurveto{\pgfqpoint{3.279423in}{1.441463in}}{\pgfqpoint{3.275032in}{1.430864in}}{\pgfqpoint{3.275032in}{1.419814in}}%
\pgfpathcurveto{\pgfqpoint{3.275032in}{1.408764in}}{\pgfqpoint{3.279423in}{1.398164in}}{\pgfqpoint{3.287236in}{1.390351in}}%
\pgfpathcurveto{\pgfqpoint{3.295050in}{1.382537in}}{\pgfqpoint{3.305649in}{1.378147in}}{\pgfqpoint{3.316699in}{1.378147in}}%
\pgfpathlineto{\pgfqpoint{3.316699in}{1.378147in}}%
\pgfpathclose%
\pgfusepath{stroke,fill}%
\end{pgfscope}%
\begin{pgfscope}%
\pgfpathrectangle{\pgfqpoint{2.963410in}{0.569136in}}{\pgfqpoint{2.177280in}{2.201755in}}%
\pgfusepath{clip}%
\pgfsetbuttcap%
\pgfsetroundjoin%
\definecolor{currentfill}{rgb}{1.000000,0.498039,0.054902}%
\pgfsetfillcolor{currentfill}%
\pgfsetlinewidth{0.481800pt}%
\definecolor{currentstroke}{rgb}{1.000000,1.000000,1.000000}%
\pgfsetstrokecolor{currentstroke}%
\pgfsetdash{}{0pt}%
\pgfpathmoveto{\pgfqpoint{3.907452in}{1.795146in}}%
\pgfpathcurveto{\pgfqpoint{3.918502in}{1.795146in}}{\pgfqpoint{3.929101in}{1.799536in}}{\pgfqpoint{3.936915in}{1.807350in}}%
\pgfpathcurveto{\pgfqpoint{3.944728in}{1.815164in}}{\pgfqpoint{3.949118in}{1.825763in}}{\pgfqpoint{3.949118in}{1.836813in}}%
\pgfpathcurveto{\pgfqpoint{3.949118in}{1.847863in}}{\pgfqpoint{3.944728in}{1.858462in}}{\pgfqpoint{3.936915in}{1.866276in}}%
\pgfpathcurveto{\pgfqpoint{3.929101in}{1.874089in}}{\pgfqpoint{3.918502in}{1.878479in}}{\pgfqpoint{3.907452in}{1.878479in}}%
\pgfpathcurveto{\pgfqpoint{3.896402in}{1.878479in}}{\pgfqpoint{3.885803in}{1.874089in}}{\pgfqpoint{3.877989in}{1.866276in}}%
\pgfpathcurveto{\pgfqpoint{3.870175in}{1.858462in}}{\pgfqpoint{3.865785in}{1.847863in}}{\pgfqpoint{3.865785in}{1.836813in}}%
\pgfpathcurveto{\pgfqpoint{3.865785in}{1.825763in}}{\pgfqpoint{3.870175in}{1.815164in}}{\pgfqpoint{3.877989in}{1.807350in}}%
\pgfpathcurveto{\pgfqpoint{3.885803in}{1.799536in}}{\pgfqpoint{3.896402in}{1.795146in}}{\pgfqpoint{3.907452in}{1.795146in}}%
\pgfpathlineto{\pgfqpoint{3.907452in}{1.795146in}}%
\pgfpathclose%
\pgfusepath{stroke,fill}%
\end{pgfscope}%
\begin{pgfscope}%
\pgfpathrectangle{\pgfqpoint{2.963410in}{0.569136in}}{\pgfqpoint{2.177280in}{2.201755in}}%
\pgfusepath{clip}%
\pgfsetbuttcap%
\pgfsetroundjoin%
\definecolor{currentfill}{rgb}{1.000000,0.498039,0.054902}%
\pgfsetfillcolor{currentfill}%
\pgfsetlinewidth{0.481800pt}%
\definecolor{currentstroke}{rgb}{1.000000,1.000000,1.000000}%
\pgfsetstrokecolor{currentstroke}%
\pgfsetdash{}{0pt}%
\pgfpathmoveto{\pgfqpoint{3.434850in}{1.378147in}}%
\pgfpathcurveto{\pgfqpoint{3.445900in}{1.378147in}}{\pgfqpoint{3.456499in}{1.382537in}}{\pgfqpoint{3.464312in}{1.390351in}}%
\pgfpathcurveto{\pgfqpoint{3.472126in}{1.398164in}}{\pgfqpoint{3.476516in}{1.408764in}}{\pgfqpoint{3.476516in}{1.419814in}}%
\pgfpathcurveto{\pgfqpoint{3.476516in}{1.430864in}}{\pgfqpoint{3.472126in}{1.441463in}}{\pgfqpoint{3.464312in}{1.449276in}}%
\pgfpathcurveto{\pgfqpoint{3.456499in}{1.457090in}}{\pgfqpoint{3.445900in}{1.461480in}}{\pgfqpoint{3.434850in}{1.461480in}}%
\pgfpathcurveto{\pgfqpoint{3.423799in}{1.461480in}}{\pgfqpoint{3.413200in}{1.457090in}}{\pgfqpoint{3.405387in}{1.449276in}}%
\pgfpathcurveto{\pgfqpoint{3.397573in}{1.441463in}}{\pgfqpoint{3.393183in}{1.430864in}}{\pgfqpoint{3.393183in}{1.419814in}}%
\pgfpathcurveto{\pgfqpoint{3.393183in}{1.408764in}}{\pgfqpoint{3.397573in}{1.398164in}}{\pgfqpoint{3.405387in}{1.390351in}}%
\pgfpathcurveto{\pgfqpoint{3.413200in}{1.382537in}}{\pgfqpoint{3.423799in}{1.378147in}}{\pgfqpoint{3.434850in}{1.378147in}}%
\pgfpathlineto{\pgfqpoint{3.434850in}{1.378147in}}%
\pgfpathclose%
\pgfusepath{stroke,fill}%
\end{pgfscope}%
\begin{pgfscope}%
\pgfpathrectangle{\pgfqpoint{2.963410in}{0.569136in}}{\pgfqpoint{2.177280in}{2.201755in}}%
\pgfusepath{clip}%
\pgfsetbuttcap%
\pgfsetroundjoin%
\definecolor{currentfill}{rgb}{1.000000,0.498039,0.054902}%
\pgfsetfillcolor{currentfill}%
\pgfsetlinewidth{0.481800pt}%
\definecolor{currentstroke}{rgb}{1.000000,1.000000,1.000000}%
\pgfsetstrokecolor{currentstroke}%
\pgfsetdash{}{0pt}%
\pgfpathmoveto{\pgfqpoint{3.848376in}{1.711746in}}%
\pgfpathcurveto{\pgfqpoint{3.859427in}{1.711746in}}{\pgfqpoint{3.870026in}{1.716137in}}{\pgfqpoint{3.877839in}{1.723950in}}%
\pgfpathcurveto{\pgfqpoint{3.885653in}{1.731764in}}{\pgfqpoint{3.890043in}{1.742363in}}{\pgfqpoint{3.890043in}{1.753413in}}%
\pgfpathcurveto{\pgfqpoint{3.890043in}{1.764463in}}{\pgfqpoint{3.885653in}{1.775062in}}{\pgfqpoint{3.877839in}{1.782876in}}%
\pgfpathcurveto{\pgfqpoint{3.870026in}{1.790689in}}{\pgfqpoint{3.859427in}{1.795080in}}{\pgfqpoint{3.848376in}{1.795080in}}%
\pgfpathcurveto{\pgfqpoint{3.837326in}{1.795080in}}{\pgfqpoint{3.826727in}{1.790689in}}{\pgfqpoint{3.818914in}{1.782876in}}%
\pgfpathcurveto{\pgfqpoint{3.811100in}{1.775062in}}{\pgfqpoint{3.806710in}{1.764463in}}{\pgfqpoint{3.806710in}{1.753413in}}%
\pgfpathcurveto{\pgfqpoint{3.806710in}{1.742363in}}{\pgfqpoint{3.811100in}{1.731764in}}{\pgfqpoint{3.818914in}{1.723950in}}%
\pgfpathcurveto{\pgfqpoint{3.826727in}{1.716137in}}{\pgfqpoint{3.837326in}{1.711746in}}{\pgfqpoint{3.848376in}{1.711746in}}%
\pgfpathlineto{\pgfqpoint{3.848376in}{1.711746in}}%
\pgfpathclose%
\pgfusepath{stroke,fill}%
\end{pgfscope}%
\begin{pgfscope}%
\pgfpathrectangle{\pgfqpoint{2.963410in}{0.569136in}}{\pgfqpoint{2.177280in}{2.201755in}}%
\pgfusepath{clip}%
\pgfsetbuttcap%
\pgfsetroundjoin%
\definecolor{currentfill}{rgb}{1.000000,0.498039,0.054902}%
\pgfsetfillcolor{currentfill}%
\pgfsetlinewidth{0.481800pt}%
\definecolor{currentstroke}{rgb}{1.000000,1.000000,1.000000}%
\pgfsetstrokecolor{currentstroke}%
\pgfsetdash{}{0pt}%
\pgfpathmoveto{\pgfqpoint{3.848376in}{1.628346in}}%
\pgfpathcurveto{\pgfqpoint{3.859427in}{1.628346in}}{\pgfqpoint{3.870026in}{1.632737in}}{\pgfqpoint{3.877839in}{1.640550in}}%
\pgfpathcurveto{\pgfqpoint{3.885653in}{1.648364in}}{\pgfqpoint{3.890043in}{1.658963in}}{\pgfqpoint{3.890043in}{1.670013in}}%
\pgfpathcurveto{\pgfqpoint{3.890043in}{1.681063in}}{\pgfqpoint{3.885653in}{1.691662in}}{\pgfqpoint{3.877839in}{1.699476in}}%
\pgfpathcurveto{\pgfqpoint{3.870026in}{1.707290in}}{\pgfqpoint{3.859427in}{1.711680in}}{\pgfqpoint{3.848376in}{1.711680in}}%
\pgfpathcurveto{\pgfqpoint{3.837326in}{1.711680in}}{\pgfqpoint{3.826727in}{1.707290in}}{\pgfqpoint{3.818914in}{1.699476in}}%
\pgfpathcurveto{\pgfqpoint{3.811100in}{1.691662in}}{\pgfqpoint{3.806710in}{1.681063in}}{\pgfqpoint{3.806710in}{1.670013in}}%
\pgfpathcurveto{\pgfqpoint{3.806710in}{1.658963in}}{\pgfqpoint{3.811100in}{1.648364in}}{\pgfqpoint{3.818914in}{1.640550in}}%
\pgfpathcurveto{\pgfqpoint{3.826727in}{1.632737in}}{\pgfqpoint{3.837326in}{1.628346in}}{\pgfqpoint{3.848376in}{1.628346in}}%
\pgfpathlineto{\pgfqpoint{3.848376in}{1.628346in}}%
\pgfpathclose%
\pgfusepath{stroke,fill}%
\end{pgfscope}%
\begin{pgfscope}%
\pgfpathrectangle{\pgfqpoint{2.963410in}{0.569136in}}{\pgfqpoint{2.177280in}{2.201755in}}%
\pgfusepath{clip}%
\pgfsetbuttcap%
\pgfsetroundjoin%
\definecolor{currentfill}{rgb}{1.000000,0.498039,0.054902}%
\pgfsetfillcolor{currentfill}%
\pgfsetlinewidth{0.481800pt}%
\definecolor{currentstroke}{rgb}{1.000000,1.000000,1.000000}%
\pgfsetstrokecolor{currentstroke}%
\pgfsetdash{}{0pt}%
\pgfpathmoveto{\pgfqpoint{3.966527in}{1.711746in}}%
\pgfpathcurveto{\pgfqpoint{3.977577in}{1.711746in}}{\pgfqpoint{3.988176in}{1.716137in}}{\pgfqpoint{3.995990in}{1.723950in}}%
\pgfpathcurveto{\pgfqpoint{4.003803in}{1.731764in}}{\pgfqpoint{4.008194in}{1.742363in}}{\pgfqpoint{4.008194in}{1.753413in}}%
\pgfpathcurveto{\pgfqpoint{4.008194in}{1.764463in}}{\pgfqpoint{4.003803in}{1.775062in}}{\pgfqpoint{3.995990in}{1.782876in}}%
\pgfpathcurveto{\pgfqpoint{3.988176in}{1.790689in}}{\pgfqpoint{3.977577in}{1.795080in}}{\pgfqpoint{3.966527in}{1.795080in}}%
\pgfpathcurveto{\pgfqpoint{3.955477in}{1.795080in}}{\pgfqpoint{3.944878in}{1.790689in}}{\pgfqpoint{3.937064in}{1.782876in}}%
\pgfpathcurveto{\pgfqpoint{3.929251in}{1.775062in}}{\pgfqpoint{3.924860in}{1.764463in}}{\pgfqpoint{3.924860in}{1.753413in}}%
\pgfpathcurveto{\pgfqpoint{3.924860in}{1.742363in}}{\pgfqpoint{3.929251in}{1.731764in}}{\pgfqpoint{3.937064in}{1.723950in}}%
\pgfpathcurveto{\pgfqpoint{3.944878in}{1.716137in}}{\pgfqpoint{3.955477in}{1.711746in}}{\pgfqpoint{3.966527in}{1.711746in}}%
\pgfpathlineto{\pgfqpoint{3.966527in}{1.711746in}}%
\pgfpathclose%
\pgfusepath{stroke,fill}%
\end{pgfscope}%
\begin{pgfscope}%
\pgfpathrectangle{\pgfqpoint{2.963410in}{0.569136in}}{\pgfqpoint{2.177280in}{2.201755in}}%
\pgfusepath{clip}%
\pgfsetbuttcap%
\pgfsetroundjoin%
\definecolor{currentfill}{rgb}{1.000000,0.498039,0.054902}%
\pgfsetfillcolor{currentfill}%
\pgfsetlinewidth{0.481800pt}%
\definecolor{currentstroke}{rgb}{1.000000,1.000000,1.000000}%
\pgfsetstrokecolor{currentstroke}%
\pgfsetdash{}{0pt}%
\pgfpathmoveto{\pgfqpoint{3.907452in}{1.795146in}}%
\pgfpathcurveto{\pgfqpoint{3.918502in}{1.795146in}}{\pgfqpoint{3.929101in}{1.799536in}}{\pgfqpoint{3.936915in}{1.807350in}}%
\pgfpathcurveto{\pgfqpoint{3.944728in}{1.815164in}}{\pgfqpoint{3.949118in}{1.825763in}}{\pgfqpoint{3.949118in}{1.836813in}}%
\pgfpathcurveto{\pgfqpoint{3.949118in}{1.847863in}}{\pgfqpoint{3.944728in}{1.858462in}}{\pgfqpoint{3.936915in}{1.866276in}}%
\pgfpathcurveto{\pgfqpoint{3.929101in}{1.874089in}}{\pgfqpoint{3.918502in}{1.878479in}}{\pgfqpoint{3.907452in}{1.878479in}}%
\pgfpathcurveto{\pgfqpoint{3.896402in}{1.878479in}}{\pgfqpoint{3.885803in}{1.874089in}}{\pgfqpoint{3.877989in}{1.866276in}}%
\pgfpathcurveto{\pgfqpoint{3.870175in}{1.858462in}}{\pgfqpoint{3.865785in}{1.847863in}}{\pgfqpoint{3.865785in}{1.836813in}}%
\pgfpathcurveto{\pgfqpoint{3.865785in}{1.825763in}}{\pgfqpoint{3.870175in}{1.815164in}}{\pgfqpoint{3.877989in}{1.807350in}}%
\pgfpathcurveto{\pgfqpoint{3.885803in}{1.799536in}}{\pgfqpoint{3.896402in}{1.795146in}}{\pgfqpoint{3.907452in}{1.795146in}}%
\pgfpathlineto{\pgfqpoint{3.907452in}{1.795146in}}%
\pgfpathclose%
\pgfusepath{stroke,fill}%
\end{pgfscope}%
\begin{pgfscope}%
\pgfpathrectangle{\pgfqpoint{2.963410in}{0.569136in}}{\pgfqpoint{2.177280in}{2.201755in}}%
\pgfusepath{clip}%
\pgfsetbuttcap%
\pgfsetroundjoin%
\definecolor{currentfill}{rgb}{1.000000,0.498039,0.054902}%
\pgfsetfillcolor{currentfill}%
\pgfsetlinewidth{0.481800pt}%
\definecolor{currentstroke}{rgb}{1.000000,1.000000,1.000000}%
\pgfsetstrokecolor{currentstroke}%
\pgfsetdash{}{0pt}%
\pgfpathmoveto{\pgfqpoint{3.730226in}{1.378147in}}%
\pgfpathcurveto{\pgfqpoint{3.741276in}{1.378147in}}{\pgfqpoint{3.751875in}{1.382537in}}{\pgfqpoint{3.759689in}{1.390351in}}%
\pgfpathcurveto{\pgfqpoint{3.767502in}{1.398164in}}{\pgfqpoint{3.771893in}{1.408764in}}{\pgfqpoint{3.771893in}{1.419814in}}%
\pgfpathcurveto{\pgfqpoint{3.771893in}{1.430864in}}{\pgfqpoint{3.767502in}{1.441463in}}{\pgfqpoint{3.759689in}{1.449276in}}%
\pgfpathcurveto{\pgfqpoint{3.751875in}{1.457090in}}{\pgfqpoint{3.741276in}{1.461480in}}{\pgfqpoint{3.730226in}{1.461480in}}%
\pgfpathcurveto{\pgfqpoint{3.719176in}{1.461480in}}{\pgfqpoint{3.708577in}{1.457090in}}{\pgfqpoint{3.700763in}{1.449276in}}%
\pgfpathcurveto{\pgfqpoint{3.692950in}{1.441463in}}{\pgfqpoint{3.688559in}{1.430864in}}{\pgfqpoint{3.688559in}{1.419814in}}%
\pgfpathcurveto{\pgfqpoint{3.688559in}{1.408764in}}{\pgfqpoint{3.692950in}{1.398164in}}{\pgfqpoint{3.700763in}{1.390351in}}%
\pgfpathcurveto{\pgfqpoint{3.708577in}{1.382537in}}{\pgfqpoint{3.719176in}{1.378147in}}{\pgfqpoint{3.730226in}{1.378147in}}%
\pgfpathlineto{\pgfqpoint{3.730226in}{1.378147in}}%
\pgfpathclose%
\pgfusepath{stroke,fill}%
\end{pgfscope}%
\begin{pgfscope}%
\pgfpathrectangle{\pgfqpoint{2.963410in}{0.569136in}}{\pgfqpoint{2.177280in}{2.201755in}}%
\pgfusepath{clip}%
\pgfsetbuttcap%
\pgfsetroundjoin%
\definecolor{currentfill}{rgb}{1.000000,0.498039,0.054902}%
\pgfsetfillcolor{currentfill}%
\pgfsetlinewidth{0.481800pt}%
\definecolor{currentstroke}{rgb}{1.000000,1.000000,1.000000}%
\pgfsetstrokecolor{currentstroke}%
\pgfsetdash{}{0pt}%
\pgfpathmoveto{\pgfqpoint{3.434850in}{1.795146in}}%
\pgfpathcurveto{\pgfqpoint{3.445900in}{1.795146in}}{\pgfqpoint{3.456499in}{1.799536in}}{\pgfqpoint{3.464312in}{1.807350in}}%
\pgfpathcurveto{\pgfqpoint{3.472126in}{1.815164in}}{\pgfqpoint{3.476516in}{1.825763in}}{\pgfqpoint{3.476516in}{1.836813in}}%
\pgfpathcurveto{\pgfqpoint{3.476516in}{1.847863in}}{\pgfqpoint{3.472126in}{1.858462in}}{\pgfqpoint{3.464312in}{1.866276in}}%
\pgfpathcurveto{\pgfqpoint{3.456499in}{1.874089in}}{\pgfqpoint{3.445900in}{1.878479in}}{\pgfqpoint{3.434850in}{1.878479in}}%
\pgfpathcurveto{\pgfqpoint{3.423799in}{1.878479in}}{\pgfqpoint{3.413200in}{1.874089in}}{\pgfqpoint{3.405387in}{1.866276in}}%
\pgfpathcurveto{\pgfqpoint{3.397573in}{1.858462in}}{\pgfqpoint{3.393183in}{1.847863in}}{\pgfqpoint{3.393183in}{1.836813in}}%
\pgfpathcurveto{\pgfqpoint{3.393183in}{1.825763in}}{\pgfqpoint{3.397573in}{1.815164in}}{\pgfqpoint{3.405387in}{1.807350in}}%
\pgfpathcurveto{\pgfqpoint{3.413200in}{1.799536in}}{\pgfqpoint{3.423799in}{1.795146in}}{\pgfqpoint{3.434850in}{1.795146in}}%
\pgfpathlineto{\pgfqpoint{3.434850in}{1.795146in}}%
\pgfpathclose%
\pgfusepath{stroke,fill}%
\end{pgfscope}%
\begin{pgfscope}%
\pgfpathrectangle{\pgfqpoint{2.963410in}{0.569136in}}{\pgfqpoint{2.177280in}{2.201755in}}%
\pgfusepath{clip}%
\pgfsetbuttcap%
\pgfsetroundjoin%
\definecolor{currentfill}{rgb}{1.000000,0.498039,0.054902}%
\pgfsetfillcolor{currentfill}%
\pgfsetlinewidth{0.481800pt}%
\definecolor{currentstroke}{rgb}{1.000000,1.000000,1.000000}%
\pgfsetstrokecolor{currentstroke}%
\pgfsetdash{}{0pt}%
\pgfpathmoveto{\pgfqpoint{3.612075in}{1.461547in}}%
\pgfpathcurveto{\pgfqpoint{3.623126in}{1.461547in}}{\pgfqpoint{3.633725in}{1.465937in}}{\pgfqpoint{3.641538in}{1.473751in}}%
\pgfpathcurveto{\pgfqpoint{3.649352in}{1.481564in}}{\pgfqpoint{3.653742in}{1.492163in}}{\pgfqpoint{3.653742in}{1.503213in}}%
\pgfpathcurveto{\pgfqpoint{3.653742in}{1.514264in}}{\pgfqpoint{3.649352in}{1.524863in}}{\pgfqpoint{3.641538in}{1.532676in}}%
\pgfpathcurveto{\pgfqpoint{3.633725in}{1.540490in}}{\pgfqpoint{3.623126in}{1.544880in}}{\pgfqpoint{3.612075in}{1.544880in}}%
\pgfpathcurveto{\pgfqpoint{3.601025in}{1.544880in}}{\pgfqpoint{3.590426in}{1.540490in}}{\pgfqpoint{3.582613in}{1.532676in}}%
\pgfpathcurveto{\pgfqpoint{3.574799in}{1.524863in}}{\pgfqpoint{3.570409in}{1.514264in}}{\pgfqpoint{3.570409in}{1.503213in}}%
\pgfpathcurveto{\pgfqpoint{3.570409in}{1.492163in}}{\pgfqpoint{3.574799in}{1.481564in}}{\pgfqpoint{3.582613in}{1.473751in}}%
\pgfpathcurveto{\pgfqpoint{3.590426in}{1.465937in}}{\pgfqpoint{3.601025in}{1.461547in}}{\pgfqpoint{3.612075in}{1.461547in}}%
\pgfpathlineto{\pgfqpoint{3.612075in}{1.461547in}}%
\pgfpathclose%
\pgfusepath{stroke,fill}%
\end{pgfscope}%
\begin{pgfscope}%
\pgfpathrectangle{\pgfqpoint{2.963410in}{0.569136in}}{\pgfqpoint{2.177280in}{2.201755in}}%
\pgfusepath{clip}%
\pgfsetbuttcap%
\pgfsetroundjoin%
\definecolor{currentfill}{rgb}{1.000000,0.498039,0.054902}%
\pgfsetfillcolor{currentfill}%
\pgfsetlinewidth{0.481800pt}%
\definecolor{currentstroke}{rgb}{1.000000,1.000000,1.000000}%
\pgfsetstrokecolor{currentstroke}%
\pgfsetdash{}{0pt}%
\pgfpathmoveto{\pgfqpoint{4.025602in}{2.045346in}}%
\pgfpathcurveto{\pgfqpoint{4.036652in}{2.045346in}}{\pgfqpoint{4.047251in}{2.049736in}}{\pgfqpoint{4.055065in}{2.057549in}}%
\pgfpathcurveto{\pgfqpoint{4.062879in}{2.065363in}}{\pgfqpoint{4.067269in}{2.075962in}}{\pgfqpoint{4.067269in}{2.087012in}}%
\pgfpathcurveto{\pgfqpoint{4.067269in}{2.098062in}}{\pgfqpoint{4.062879in}{2.108661in}}{\pgfqpoint{4.055065in}{2.116475in}}%
\pgfpathcurveto{\pgfqpoint{4.047251in}{2.124289in}}{\pgfqpoint{4.036652in}{2.128679in}}{\pgfqpoint{4.025602in}{2.128679in}}%
\pgfpathcurveto{\pgfqpoint{4.014552in}{2.128679in}}{\pgfqpoint{4.003953in}{2.124289in}}{\pgfqpoint{3.996139in}{2.116475in}}%
\pgfpathcurveto{\pgfqpoint{3.988326in}{2.108661in}}{\pgfqpoint{3.983936in}{2.098062in}}{\pgfqpoint{3.983936in}{2.087012in}}%
\pgfpathcurveto{\pgfqpoint{3.983936in}{2.075962in}}{\pgfqpoint{3.988326in}{2.065363in}}{\pgfqpoint{3.996139in}{2.057549in}}%
\pgfpathcurveto{\pgfqpoint{4.003953in}{2.049736in}}{\pgfqpoint{4.014552in}{2.045346in}}{\pgfqpoint{4.025602in}{2.045346in}}%
\pgfpathlineto{\pgfqpoint{4.025602in}{2.045346in}}%
\pgfpathclose%
\pgfusepath{stroke,fill}%
\end{pgfscope}%
\begin{pgfscope}%
\pgfpathrectangle{\pgfqpoint{2.963410in}{0.569136in}}{\pgfqpoint{2.177280in}{2.201755in}}%
\pgfusepath{clip}%
\pgfsetbuttcap%
\pgfsetroundjoin%
\definecolor{currentfill}{rgb}{1.000000,0.498039,0.054902}%
\pgfsetfillcolor{currentfill}%
\pgfsetlinewidth{0.481800pt}%
\definecolor{currentstroke}{rgb}{1.000000,1.000000,1.000000}%
\pgfsetstrokecolor{currentstroke}%
\pgfsetdash{}{0pt}%
\pgfpathmoveto{\pgfqpoint{3.789301in}{1.628346in}}%
\pgfpathcurveto{\pgfqpoint{3.800351in}{1.628346in}}{\pgfqpoint{3.810950in}{1.632737in}}{\pgfqpoint{3.818764in}{1.640550in}}%
\pgfpathcurveto{\pgfqpoint{3.826578in}{1.648364in}}{\pgfqpoint{3.830968in}{1.658963in}}{\pgfqpoint{3.830968in}{1.670013in}}%
\pgfpathcurveto{\pgfqpoint{3.830968in}{1.681063in}}{\pgfqpoint{3.826578in}{1.691662in}}{\pgfqpoint{3.818764in}{1.699476in}}%
\pgfpathcurveto{\pgfqpoint{3.810950in}{1.707290in}}{\pgfqpoint{3.800351in}{1.711680in}}{\pgfqpoint{3.789301in}{1.711680in}}%
\pgfpathcurveto{\pgfqpoint{3.778251in}{1.711680in}}{\pgfqpoint{3.767652in}{1.707290in}}{\pgfqpoint{3.759838in}{1.699476in}}%
\pgfpathcurveto{\pgfqpoint{3.752025in}{1.691662in}}{\pgfqpoint{3.747635in}{1.681063in}}{\pgfqpoint{3.747635in}{1.670013in}}%
\pgfpathcurveto{\pgfqpoint{3.747635in}{1.658963in}}{\pgfqpoint{3.752025in}{1.648364in}}{\pgfqpoint{3.759838in}{1.640550in}}%
\pgfpathcurveto{\pgfqpoint{3.767652in}{1.632737in}}{\pgfqpoint{3.778251in}{1.628346in}}{\pgfqpoint{3.789301in}{1.628346in}}%
\pgfpathlineto{\pgfqpoint{3.789301in}{1.628346in}}%
\pgfpathclose%
\pgfusepath{stroke,fill}%
\end{pgfscope}%
\begin{pgfscope}%
\pgfpathrectangle{\pgfqpoint{2.963410in}{0.569136in}}{\pgfqpoint{2.177280in}{2.201755in}}%
\pgfusepath{clip}%
\pgfsetbuttcap%
\pgfsetroundjoin%
\definecolor{currentfill}{rgb}{1.000000,0.498039,0.054902}%
\pgfsetfillcolor{currentfill}%
\pgfsetlinewidth{0.481800pt}%
\definecolor{currentstroke}{rgb}{1.000000,1.000000,1.000000}%
\pgfsetstrokecolor{currentstroke}%
\pgfsetdash{}{0pt}%
\pgfpathmoveto{\pgfqpoint{3.612075in}{1.795146in}}%
\pgfpathcurveto{\pgfqpoint{3.623126in}{1.795146in}}{\pgfqpoint{3.633725in}{1.799536in}}{\pgfqpoint{3.641538in}{1.807350in}}%
\pgfpathcurveto{\pgfqpoint{3.649352in}{1.815164in}}{\pgfqpoint{3.653742in}{1.825763in}}{\pgfqpoint{3.653742in}{1.836813in}}%
\pgfpathcurveto{\pgfqpoint{3.653742in}{1.847863in}}{\pgfqpoint{3.649352in}{1.858462in}}{\pgfqpoint{3.641538in}{1.866276in}}%
\pgfpathcurveto{\pgfqpoint{3.633725in}{1.874089in}}{\pgfqpoint{3.623126in}{1.878479in}}{\pgfqpoint{3.612075in}{1.878479in}}%
\pgfpathcurveto{\pgfqpoint{3.601025in}{1.878479in}}{\pgfqpoint{3.590426in}{1.874089in}}{\pgfqpoint{3.582613in}{1.866276in}}%
\pgfpathcurveto{\pgfqpoint{3.574799in}{1.858462in}}{\pgfqpoint{3.570409in}{1.847863in}}{\pgfqpoint{3.570409in}{1.836813in}}%
\pgfpathcurveto{\pgfqpoint{3.570409in}{1.825763in}}{\pgfqpoint{3.574799in}{1.815164in}}{\pgfqpoint{3.582613in}{1.807350in}}%
\pgfpathcurveto{\pgfqpoint{3.590426in}{1.799536in}}{\pgfqpoint{3.601025in}{1.795146in}}{\pgfqpoint{3.612075in}{1.795146in}}%
\pgfpathlineto{\pgfqpoint{3.612075in}{1.795146in}}%
\pgfpathclose%
\pgfusepath{stroke,fill}%
\end{pgfscope}%
\begin{pgfscope}%
\pgfpathrectangle{\pgfqpoint{2.963410in}{0.569136in}}{\pgfqpoint{2.177280in}{2.201755in}}%
\pgfusepath{clip}%
\pgfsetbuttcap%
\pgfsetroundjoin%
\definecolor{currentfill}{rgb}{1.000000,0.498039,0.054902}%
\pgfsetfillcolor{currentfill}%
\pgfsetlinewidth{0.481800pt}%
\definecolor{currentstroke}{rgb}{1.000000,1.000000,1.000000}%
\pgfsetstrokecolor{currentstroke}%
\pgfsetdash{}{0pt}%
\pgfpathmoveto{\pgfqpoint{3.789301in}{1.544947in}}%
\pgfpathcurveto{\pgfqpoint{3.800351in}{1.544947in}}{\pgfqpoint{3.810950in}{1.549337in}}{\pgfqpoint{3.818764in}{1.557151in}}%
\pgfpathcurveto{\pgfqpoint{3.826578in}{1.564964in}}{\pgfqpoint{3.830968in}{1.575563in}}{\pgfqpoint{3.830968in}{1.586613in}}%
\pgfpathcurveto{\pgfqpoint{3.830968in}{1.597663in}}{\pgfqpoint{3.826578in}{1.608262in}}{\pgfqpoint{3.818764in}{1.616076in}}%
\pgfpathcurveto{\pgfqpoint{3.810950in}{1.623890in}}{\pgfqpoint{3.800351in}{1.628280in}}{\pgfqpoint{3.789301in}{1.628280in}}%
\pgfpathcurveto{\pgfqpoint{3.778251in}{1.628280in}}{\pgfqpoint{3.767652in}{1.623890in}}{\pgfqpoint{3.759838in}{1.616076in}}%
\pgfpathcurveto{\pgfqpoint{3.752025in}{1.608262in}}{\pgfqpoint{3.747635in}{1.597663in}}{\pgfqpoint{3.747635in}{1.586613in}}%
\pgfpathcurveto{\pgfqpoint{3.747635in}{1.575563in}}{\pgfqpoint{3.752025in}{1.564964in}}{\pgfqpoint{3.759838in}{1.557151in}}%
\pgfpathcurveto{\pgfqpoint{3.767652in}{1.549337in}}{\pgfqpoint{3.778251in}{1.544947in}}{\pgfqpoint{3.789301in}{1.544947in}}%
\pgfpathlineto{\pgfqpoint{3.789301in}{1.544947in}}%
\pgfpathclose%
\pgfusepath{stroke,fill}%
\end{pgfscope}%
\begin{pgfscope}%
\pgfpathrectangle{\pgfqpoint{2.963410in}{0.569136in}}{\pgfqpoint{2.177280in}{2.201755in}}%
\pgfusepath{clip}%
\pgfsetbuttcap%
\pgfsetroundjoin%
\definecolor{currentfill}{rgb}{1.000000,0.498039,0.054902}%
\pgfsetfillcolor{currentfill}%
\pgfsetlinewidth{0.481800pt}%
\definecolor{currentstroke}{rgb}{1.000000,1.000000,1.000000}%
\pgfsetstrokecolor{currentstroke}%
\pgfsetdash{}{0pt}%
\pgfpathmoveto{\pgfqpoint{3.848376in}{1.628346in}}%
\pgfpathcurveto{\pgfqpoint{3.859427in}{1.628346in}}{\pgfqpoint{3.870026in}{1.632737in}}{\pgfqpoint{3.877839in}{1.640550in}}%
\pgfpathcurveto{\pgfqpoint{3.885653in}{1.648364in}}{\pgfqpoint{3.890043in}{1.658963in}}{\pgfqpoint{3.890043in}{1.670013in}}%
\pgfpathcurveto{\pgfqpoint{3.890043in}{1.681063in}}{\pgfqpoint{3.885653in}{1.691662in}}{\pgfqpoint{3.877839in}{1.699476in}}%
\pgfpathcurveto{\pgfqpoint{3.870026in}{1.707290in}}{\pgfqpoint{3.859427in}{1.711680in}}{\pgfqpoint{3.848376in}{1.711680in}}%
\pgfpathcurveto{\pgfqpoint{3.837326in}{1.711680in}}{\pgfqpoint{3.826727in}{1.707290in}}{\pgfqpoint{3.818914in}{1.699476in}}%
\pgfpathcurveto{\pgfqpoint{3.811100in}{1.691662in}}{\pgfqpoint{3.806710in}{1.681063in}}{\pgfqpoint{3.806710in}{1.670013in}}%
\pgfpathcurveto{\pgfqpoint{3.806710in}{1.658963in}}{\pgfqpoint{3.811100in}{1.648364in}}{\pgfqpoint{3.818914in}{1.640550in}}%
\pgfpathcurveto{\pgfqpoint{3.826727in}{1.632737in}}{\pgfqpoint{3.837326in}{1.628346in}}{\pgfqpoint{3.848376in}{1.628346in}}%
\pgfpathlineto{\pgfqpoint{3.848376in}{1.628346in}}%
\pgfpathclose%
\pgfusepath{stroke,fill}%
\end{pgfscope}%
\begin{pgfscope}%
\pgfpathrectangle{\pgfqpoint{2.963410in}{0.569136in}}{\pgfqpoint{2.177280in}{2.201755in}}%
\pgfusepath{clip}%
\pgfsetbuttcap%
\pgfsetroundjoin%
\definecolor{currentfill}{rgb}{1.000000,0.498039,0.054902}%
\pgfsetfillcolor{currentfill}%
\pgfsetlinewidth{0.481800pt}%
\definecolor{currentstroke}{rgb}{1.000000,1.000000,1.000000}%
\pgfsetstrokecolor{currentstroke}%
\pgfsetdash{}{0pt}%
\pgfpathmoveto{\pgfqpoint{3.907452in}{1.711746in}}%
\pgfpathcurveto{\pgfqpoint{3.918502in}{1.711746in}}{\pgfqpoint{3.929101in}{1.716137in}}{\pgfqpoint{3.936915in}{1.723950in}}%
\pgfpathcurveto{\pgfqpoint{3.944728in}{1.731764in}}{\pgfqpoint{3.949118in}{1.742363in}}{\pgfqpoint{3.949118in}{1.753413in}}%
\pgfpathcurveto{\pgfqpoint{3.949118in}{1.764463in}}{\pgfqpoint{3.944728in}{1.775062in}}{\pgfqpoint{3.936915in}{1.782876in}}%
\pgfpathcurveto{\pgfqpoint{3.929101in}{1.790689in}}{\pgfqpoint{3.918502in}{1.795080in}}{\pgfqpoint{3.907452in}{1.795080in}}%
\pgfpathcurveto{\pgfqpoint{3.896402in}{1.795080in}}{\pgfqpoint{3.885803in}{1.790689in}}{\pgfqpoint{3.877989in}{1.782876in}}%
\pgfpathcurveto{\pgfqpoint{3.870175in}{1.775062in}}{\pgfqpoint{3.865785in}{1.764463in}}{\pgfqpoint{3.865785in}{1.753413in}}%
\pgfpathcurveto{\pgfqpoint{3.865785in}{1.742363in}}{\pgfqpoint{3.870175in}{1.731764in}}{\pgfqpoint{3.877989in}{1.723950in}}%
\pgfpathcurveto{\pgfqpoint{3.885803in}{1.716137in}}{\pgfqpoint{3.896402in}{1.711746in}}{\pgfqpoint{3.907452in}{1.711746in}}%
\pgfpathlineto{\pgfqpoint{3.907452in}{1.711746in}}%
\pgfpathclose%
\pgfusepath{stroke,fill}%
\end{pgfscope}%
\begin{pgfscope}%
\pgfpathrectangle{\pgfqpoint{2.963410in}{0.569136in}}{\pgfqpoint{2.177280in}{2.201755in}}%
\pgfusepath{clip}%
\pgfsetbuttcap%
\pgfsetroundjoin%
\definecolor{currentfill}{rgb}{1.000000,0.498039,0.054902}%
\pgfsetfillcolor{currentfill}%
\pgfsetlinewidth{0.481800pt}%
\definecolor{currentstroke}{rgb}{1.000000,1.000000,1.000000}%
\pgfsetstrokecolor{currentstroke}%
\pgfsetdash{}{0pt}%
\pgfpathmoveto{\pgfqpoint{3.789301in}{1.711746in}}%
\pgfpathcurveto{\pgfqpoint{3.800351in}{1.711746in}}{\pgfqpoint{3.810950in}{1.716137in}}{\pgfqpoint{3.818764in}{1.723950in}}%
\pgfpathcurveto{\pgfqpoint{3.826578in}{1.731764in}}{\pgfqpoint{3.830968in}{1.742363in}}{\pgfqpoint{3.830968in}{1.753413in}}%
\pgfpathcurveto{\pgfqpoint{3.830968in}{1.764463in}}{\pgfqpoint{3.826578in}{1.775062in}}{\pgfqpoint{3.818764in}{1.782876in}}%
\pgfpathcurveto{\pgfqpoint{3.810950in}{1.790689in}}{\pgfqpoint{3.800351in}{1.795080in}}{\pgfqpoint{3.789301in}{1.795080in}}%
\pgfpathcurveto{\pgfqpoint{3.778251in}{1.795080in}}{\pgfqpoint{3.767652in}{1.790689in}}{\pgfqpoint{3.759838in}{1.782876in}}%
\pgfpathcurveto{\pgfqpoint{3.752025in}{1.775062in}}{\pgfqpoint{3.747635in}{1.764463in}}{\pgfqpoint{3.747635in}{1.753413in}}%
\pgfpathcurveto{\pgfqpoint{3.747635in}{1.742363in}}{\pgfqpoint{3.752025in}{1.731764in}}{\pgfqpoint{3.759838in}{1.723950in}}%
\pgfpathcurveto{\pgfqpoint{3.767652in}{1.716137in}}{\pgfqpoint{3.778251in}{1.711746in}}{\pgfqpoint{3.789301in}{1.711746in}}%
\pgfpathlineto{\pgfqpoint{3.789301in}{1.711746in}}%
\pgfpathclose%
\pgfusepath{stroke,fill}%
\end{pgfscope}%
\begin{pgfscope}%
\pgfpathrectangle{\pgfqpoint{2.963410in}{0.569136in}}{\pgfqpoint{2.177280in}{2.201755in}}%
\pgfusepath{clip}%
\pgfsetbuttcap%
\pgfsetroundjoin%
\definecolor{currentfill}{rgb}{1.000000,0.498039,0.054902}%
\pgfsetfillcolor{currentfill}%
\pgfsetlinewidth{0.481800pt}%
\definecolor{currentstroke}{rgb}{1.000000,1.000000,1.000000}%
\pgfsetstrokecolor{currentstroke}%
\pgfsetdash{}{0pt}%
\pgfpathmoveto{\pgfqpoint{3.907452in}{1.961946in}}%
\pgfpathcurveto{\pgfqpoint{3.918502in}{1.961946in}}{\pgfqpoint{3.929101in}{1.966336in}}{\pgfqpoint{3.936915in}{1.974150in}}%
\pgfpathcurveto{\pgfqpoint{3.944728in}{1.981963in}}{\pgfqpoint{3.949118in}{1.992562in}}{\pgfqpoint{3.949118in}{2.003612in}}%
\pgfpathcurveto{\pgfqpoint{3.949118in}{2.014662in}}{\pgfqpoint{3.944728in}{2.025262in}}{\pgfqpoint{3.936915in}{2.033075in}}%
\pgfpathcurveto{\pgfqpoint{3.929101in}{2.040889in}}{\pgfqpoint{3.918502in}{2.045279in}}{\pgfqpoint{3.907452in}{2.045279in}}%
\pgfpathcurveto{\pgfqpoint{3.896402in}{2.045279in}}{\pgfqpoint{3.885803in}{2.040889in}}{\pgfqpoint{3.877989in}{2.033075in}}%
\pgfpathcurveto{\pgfqpoint{3.870175in}{2.025262in}}{\pgfqpoint{3.865785in}{2.014662in}}{\pgfqpoint{3.865785in}{2.003612in}}%
\pgfpathcurveto{\pgfqpoint{3.865785in}{1.992562in}}{\pgfqpoint{3.870175in}{1.981963in}}{\pgfqpoint{3.877989in}{1.974150in}}%
\pgfpathcurveto{\pgfqpoint{3.885803in}{1.966336in}}{\pgfqpoint{3.896402in}{1.961946in}}{\pgfqpoint{3.907452in}{1.961946in}}%
\pgfpathlineto{\pgfqpoint{3.907452in}{1.961946in}}%
\pgfpathclose%
\pgfusepath{stroke,fill}%
\end{pgfscope}%
\begin{pgfscope}%
\pgfpathrectangle{\pgfqpoint{2.963410in}{0.569136in}}{\pgfqpoint{2.177280in}{2.201755in}}%
\pgfusepath{clip}%
\pgfsetbuttcap%
\pgfsetroundjoin%
\definecolor{currentfill}{rgb}{1.000000,0.498039,0.054902}%
\pgfsetfillcolor{currentfill}%
\pgfsetlinewidth{0.481800pt}%
\definecolor{currentstroke}{rgb}{1.000000,1.000000,1.000000}%
\pgfsetstrokecolor{currentstroke}%
\pgfsetdash{}{0pt}%
\pgfpathmoveto{\pgfqpoint{3.848376in}{1.795146in}}%
\pgfpathcurveto{\pgfqpoint{3.859427in}{1.795146in}}{\pgfqpoint{3.870026in}{1.799536in}}{\pgfqpoint{3.877839in}{1.807350in}}%
\pgfpathcurveto{\pgfqpoint{3.885653in}{1.815164in}}{\pgfqpoint{3.890043in}{1.825763in}}{\pgfqpoint{3.890043in}{1.836813in}}%
\pgfpathcurveto{\pgfqpoint{3.890043in}{1.847863in}}{\pgfqpoint{3.885653in}{1.858462in}}{\pgfqpoint{3.877839in}{1.866276in}}%
\pgfpathcurveto{\pgfqpoint{3.870026in}{1.874089in}}{\pgfqpoint{3.859427in}{1.878479in}}{\pgfqpoint{3.848376in}{1.878479in}}%
\pgfpathcurveto{\pgfqpoint{3.837326in}{1.878479in}}{\pgfqpoint{3.826727in}{1.874089in}}{\pgfqpoint{3.818914in}{1.866276in}}%
\pgfpathcurveto{\pgfqpoint{3.811100in}{1.858462in}}{\pgfqpoint{3.806710in}{1.847863in}}{\pgfqpoint{3.806710in}{1.836813in}}%
\pgfpathcurveto{\pgfqpoint{3.806710in}{1.825763in}}{\pgfqpoint{3.811100in}{1.815164in}}{\pgfqpoint{3.818914in}{1.807350in}}%
\pgfpathcurveto{\pgfqpoint{3.826727in}{1.799536in}}{\pgfqpoint{3.837326in}{1.795146in}}{\pgfqpoint{3.848376in}{1.795146in}}%
\pgfpathlineto{\pgfqpoint{3.848376in}{1.795146in}}%
\pgfpathclose%
\pgfusepath{stroke,fill}%
\end{pgfscope}%
\begin{pgfscope}%
\pgfpathrectangle{\pgfqpoint{2.963410in}{0.569136in}}{\pgfqpoint{2.177280in}{2.201755in}}%
\pgfusepath{clip}%
\pgfsetbuttcap%
\pgfsetroundjoin%
\definecolor{currentfill}{rgb}{1.000000,0.498039,0.054902}%
\pgfsetfillcolor{currentfill}%
\pgfsetlinewidth{0.481800pt}%
\definecolor{currentstroke}{rgb}{1.000000,1.000000,1.000000}%
\pgfsetstrokecolor{currentstroke}%
\pgfsetdash{}{0pt}%
\pgfpathmoveto{\pgfqpoint{3.671151in}{1.378147in}}%
\pgfpathcurveto{\pgfqpoint{3.682201in}{1.378147in}}{\pgfqpoint{3.692800in}{1.382537in}}{\pgfqpoint{3.700613in}{1.390351in}}%
\pgfpathcurveto{\pgfqpoint{3.708427in}{1.398164in}}{\pgfqpoint{3.712817in}{1.408764in}}{\pgfqpoint{3.712817in}{1.419814in}}%
\pgfpathcurveto{\pgfqpoint{3.712817in}{1.430864in}}{\pgfqpoint{3.708427in}{1.441463in}}{\pgfqpoint{3.700613in}{1.449276in}}%
\pgfpathcurveto{\pgfqpoint{3.692800in}{1.457090in}}{\pgfqpoint{3.682201in}{1.461480in}}{\pgfqpoint{3.671151in}{1.461480in}}%
\pgfpathcurveto{\pgfqpoint{3.660101in}{1.461480in}}{\pgfqpoint{3.649501in}{1.457090in}}{\pgfqpoint{3.641688in}{1.449276in}}%
\pgfpathcurveto{\pgfqpoint{3.633874in}{1.441463in}}{\pgfqpoint{3.629484in}{1.430864in}}{\pgfqpoint{3.629484in}{1.419814in}}%
\pgfpathcurveto{\pgfqpoint{3.629484in}{1.408764in}}{\pgfqpoint{3.633874in}{1.398164in}}{\pgfqpoint{3.641688in}{1.390351in}}%
\pgfpathcurveto{\pgfqpoint{3.649501in}{1.382537in}}{\pgfqpoint{3.660101in}{1.378147in}}{\pgfqpoint{3.671151in}{1.378147in}}%
\pgfpathlineto{\pgfqpoint{3.671151in}{1.378147in}}%
\pgfpathclose%
\pgfusepath{stroke,fill}%
\end{pgfscope}%
\begin{pgfscope}%
\pgfpathrectangle{\pgfqpoint{2.963410in}{0.569136in}}{\pgfqpoint{2.177280in}{2.201755in}}%
\pgfusepath{clip}%
\pgfsetbuttcap%
\pgfsetroundjoin%
\definecolor{currentfill}{rgb}{1.000000,0.498039,0.054902}%
\pgfsetfillcolor{currentfill}%
\pgfsetlinewidth{0.481800pt}%
\definecolor{currentstroke}{rgb}{1.000000,1.000000,1.000000}%
\pgfsetstrokecolor{currentstroke}%
\pgfsetdash{}{0pt}%
\pgfpathmoveto{\pgfqpoint{3.553000in}{1.461547in}}%
\pgfpathcurveto{\pgfqpoint{3.564050in}{1.461547in}}{\pgfqpoint{3.574649in}{1.465937in}}{\pgfqpoint{3.582463in}{1.473751in}}%
\pgfpathcurveto{\pgfqpoint{3.590277in}{1.481564in}}{\pgfqpoint{3.594667in}{1.492163in}}{\pgfqpoint{3.594667in}{1.503213in}}%
\pgfpathcurveto{\pgfqpoint{3.594667in}{1.514264in}}{\pgfqpoint{3.590277in}{1.524863in}}{\pgfqpoint{3.582463in}{1.532676in}}%
\pgfpathcurveto{\pgfqpoint{3.574649in}{1.540490in}}{\pgfqpoint{3.564050in}{1.544880in}}{\pgfqpoint{3.553000in}{1.544880in}}%
\pgfpathcurveto{\pgfqpoint{3.541950in}{1.544880in}}{\pgfqpoint{3.531351in}{1.540490in}}{\pgfqpoint{3.523537in}{1.532676in}}%
\pgfpathcurveto{\pgfqpoint{3.515724in}{1.524863in}}{\pgfqpoint{3.511333in}{1.514264in}}{\pgfqpoint{3.511333in}{1.503213in}}%
\pgfpathcurveto{\pgfqpoint{3.511333in}{1.492163in}}{\pgfqpoint{3.515724in}{1.481564in}}{\pgfqpoint{3.523537in}{1.473751in}}%
\pgfpathcurveto{\pgfqpoint{3.531351in}{1.465937in}}{\pgfqpoint{3.541950in}{1.461547in}}{\pgfqpoint{3.553000in}{1.461547in}}%
\pgfpathlineto{\pgfqpoint{3.553000in}{1.461547in}}%
\pgfpathclose%
\pgfusepath{stroke,fill}%
\end{pgfscope}%
\begin{pgfscope}%
\pgfpathrectangle{\pgfqpoint{2.963410in}{0.569136in}}{\pgfqpoint{2.177280in}{2.201755in}}%
\pgfusepath{clip}%
\pgfsetbuttcap%
\pgfsetroundjoin%
\definecolor{currentfill}{rgb}{1.000000,0.498039,0.054902}%
\pgfsetfillcolor{currentfill}%
\pgfsetlinewidth{0.481800pt}%
\definecolor{currentstroke}{rgb}{1.000000,1.000000,1.000000}%
\pgfsetstrokecolor{currentstroke}%
\pgfsetdash{}{0pt}%
\pgfpathmoveto{\pgfqpoint{3.553000in}{1.378147in}}%
\pgfpathcurveto{\pgfqpoint{3.564050in}{1.378147in}}{\pgfqpoint{3.574649in}{1.382537in}}{\pgfqpoint{3.582463in}{1.390351in}}%
\pgfpathcurveto{\pgfqpoint{3.590277in}{1.398164in}}{\pgfqpoint{3.594667in}{1.408764in}}{\pgfqpoint{3.594667in}{1.419814in}}%
\pgfpathcurveto{\pgfqpoint{3.594667in}{1.430864in}}{\pgfqpoint{3.590277in}{1.441463in}}{\pgfqpoint{3.582463in}{1.449276in}}%
\pgfpathcurveto{\pgfqpoint{3.574649in}{1.457090in}}{\pgfqpoint{3.564050in}{1.461480in}}{\pgfqpoint{3.553000in}{1.461480in}}%
\pgfpathcurveto{\pgfqpoint{3.541950in}{1.461480in}}{\pgfqpoint{3.531351in}{1.457090in}}{\pgfqpoint{3.523537in}{1.449276in}}%
\pgfpathcurveto{\pgfqpoint{3.515724in}{1.441463in}}{\pgfqpoint{3.511333in}{1.430864in}}{\pgfqpoint{3.511333in}{1.419814in}}%
\pgfpathcurveto{\pgfqpoint{3.511333in}{1.408764in}}{\pgfqpoint{3.515724in}{1.398164in}}{\pgfqpoint{3.523537in}{1.390351in}}%
\pgfpathcurveto{\pgfqpoint{3.531351in}{1.382537in}}{\pgfqpoint{3.541950in}{1.378147in}}{\pgfqpoint{3.553000in}{1.378147in}}%
\pgfpathlineto{\pgfqpoint{3.553000in}{1.378147in}}%
\pgfpathclose%
\pgfusepath{stroke,fill}%
\end{pgfscope}%
\begin{pgfscope}%
\pgfpathrectangle{\pgfqpoint{2.963410in}{0.569136in}}{\pgfqpoint{2.177280in}{2.201755in}}%
\pgfusepath{clip}%
\pgfsetbuttcap%
\pgfsetroundjoin%
\definecolor{currentfill}{rgb}{1.000000,0.498039,0.054902}%
\pgfsetfillcolor{currentfill}%
\pgfsetlinewidth{0.481800pt}%
\definecolor{currentstroke}{rgb}{1.000000,1.000000,1.000000}%
\pgfsetstrokecolor{currentstroke}%
\pgfsetdash{}{0pt}%
\pgfpathmoveto{\pgfqpoint{3.730226in}{1.544947in}}%
\pgfpathcurveto{\pgfqpoint{3.741276in}{1.544947in}}{\pgfqpoint{3.751875in}{1.549337in}}{\pgfqpoint{3.759689in}{1.557151in}}%
\pgfpathcurveto{\pgfqpoint{3.767502in}{1.564964in}}{\pgfqpoint{3.771893in}{1.575563in}}{\pgfqpoint{3.771893in}{1.586613in}}%
\pgfpathcurveto{\pgfqpoint{3.771893in}{1.597663in}}{\pgfqpoint{3.767502in}{1.608262in}}{\pgfqpoint{3.759689in}{1.616076in}}%
\pgfpathcurveto{\pgfqpoint{3.751875in}{1.623890in}}{\pgfqpoint{3.741276in}{1.628280in}}{\pgfqpoint{3.730226in}{1.628280in}}%
\pgfpathcurveto{\pgfqpoint{3.719176in}{1.628280in}}{\pgfqpoint{3.708577in}{1.623890in}}{\pgfqpoint{3.700763in}{1.616076in}}%
\pgfpathcurveto{\pgfqpoint{3.692950in}{1.608262in}}{\pgfqpoint{3.688559in}{1.597663in}}{\pgfqpoint{3.688559in}{1.586613in}}%
\pgfpathcurveto{\pgfqpoint{3.688559in}{1.575563in}}{\pgfqpoint{3.692950in}{1.564964in}}{\pgfqpoint{3.700763in}{1.557151in}}%
\pgfpathcurveto{\pgfqpoint{3.708577in}{1.549337in}}{\pgfqpoint{3.719176in}{1.544947in}}{\pgfqpoint{3.730226in}{1.544947in}}%
\pgfpathlineto{\pgfqpoint{3.730226in}{1.544947in}}%
\pgfpathclose%
\pgfusepath{stroke,fill}%
\end{pgfscope}%
\begin{pgfscope}%
\pgfpathrectangle{\pgfqpoint{2.963410in}{0.569136in}}{\pgfqpoint{2.177280in}{2.201755in}}%
\pgfusepath{clip}%
\pgfsetbuttcap%
\pgfsetroundjoin%
\definecolor{currentfill}{rgb}{1.000000,0.498039,0.054902}%
\pgfsetfillcolor{currentfill}%
\pgfsetlinewidth{0.481800pt}%
\definecolor{currentstroke}{rgb}{1.000000,1.000000,1.000000}%
\pgfsetstrokecolor{currentstroke}%
\pgfsetdash{}{0pt}%
\pgfpathmoveto{\pgfqpoint{3.730226in}{1.878546in}}%
\pgfpathcurveto{\pgfqpoint{3.741276in}{1.878546in}}{\pgfqpoint{3.751875in}{1.882936in}}{\pgfqpoint{3.759689in}{1.890750in}}%
\pgfpathcurveto{\pgfqpoint{3.767502in}{1.898563in}}{\pgfqpoint{3.771893in}{1.909162in}}{\pgfqpoint{3.771893in}{1.920213in}}%
\pgfpathcurveto{\pgfqpoint{3.771893in}{1.931263in}}{\pgfqpoint{3.767502in}{1.941862in}}{\pgfqpoint{3.759689in}{1.949675in}}%
\pgfpathcurveto{\pgfqpoint{3.751875in}{1.957489in}}{\pgfqpoint{3.741276in}{1.961879in}}{\pgfqpoint{3.730226in}{1.961879in}}%
\pgfpathcurveto{\pgfqpoint{3.719176in}{1.961879in}}{\pgfqpoint{3.708577in}{1.957489in}}{\pgfqpoint{3.700763in}{1.949675in}}%
\pgfpathcurveto{\pgfqpoint{3.692950in}{1.941862in}}{\pgfqpoint{3.688559in}{1.931263in}}{\pgfqpoint{3.688559in}{1.920213in}}%
\pgfpathcurveto{\pgfqpoint{3.688559in}{1.909162in}}{\pgfqpoint{3.692950in}{1.898563in}}{\pgfqpoint{3.700763in}{1.890750in}}%
\pgfpathcurveto{\pgfqpoint{3.708577in}{1.882936in}}{\pgfqpoint{3.719176in}{1.878546in}}{\pgfqpoint{3.730226in}{1.878546in}}%
\pgfpathlineto{\pgfqpoint{3.730226in}{1.878546in}}%
\pgfpathclose%
\pgfusepath{stroke,fill}%
\end{pgfscope}%
\begin{pgfscope}%
\pgfpathrectangle{\pgfqpoint{2.963410in}{0.569136in}}{\pgfqpoint{2.177280in}{2.201755in}}%
\pgfusepath{clip}%
\pgfsetbuttcap%
\pgfsetroundjoin%
\definecolor{currentfill}{rgb}{1.000000,0.498039,0.054902}%
\pgfsetfillcolor{currentfill}%
\pgfsetlinewidth{0.481800pt}%
\definecolor{currentstroke}{rgb}{1.000000,1.000000,1.000000}%
\pgfsetstrokecolor{currentstroke}%
\pgfsetdash{}{0pt}%
\pgfpathmoveto{\pgfqpoint{3.907452in}{1.795146in}}%
\pgfpathcurveto{\pgfqpoint{3.918502in}{1.795146in}}{\pgfqpoint{3.929101in}{1.799536in}}{\pgfqpoint{3.936915in}{1.807350in}}%
\pgfpathcurveto{\pgfqpoint{3.944728in}{1.815164in}}{\pgfqpoint{3.949118in}{1.825763in}}{\pgfqpoint{3.949118in}{1.836813in}}%
\pgfpathcurveto{\pgfqpoint{3.949118in}{1.847863in}}{\pgfqpoint{3.944728in}{1.858462in}}{\pgfqpoint{3.936915in}{1.866276in}}%
\pgfpathcurveto{\pgfqpoint{3.929101in}{1.874089in}}{\pgfqpoint{3.918502in}{1.878479in}}{\pgfqpoint{3.907452in}{1.878479in}}%
\pgfpathcurveto{\pgfqpoint{3.896402in}{1.878479in}}{\pgfqpoint{3.885803in}{1.874089in}}{\pgfqpoint{3.877989in}{1.866276in}}%
\pgfpathcurveto{\pgfqpoint{3.870175in}{1.858462in}}{\pgfqpoint{3.865785in}{1.847863in}}{\pgfqpoint{3.865785in}{1.836813in}}%
\pgfpathcurveto{\pgfqpoint{3.865785in}{1.825763in}}{\pgfqpoint{3.870175in}{1.815164in}}{\pgfqpoint{3.877989in}{1.807350in}}%
\pgfpathcurveto{\pgfqpoint{3.885803in}{1.799536in}}{\pgfqpoint{3.896402in}{1.795146in}}{\pgfqpoint{3.907452in}{1.795146in}}%
\pgfpathlineto{\pgfqpoint{3.907452in}{1.795146in}}%
\pgfpathclose%
\pgfusepath{stroke,fill}%
\end{pgfscope}%
\begin{pgfscope}%
\pgfpathrectangle{\pgfqpoint{2.963410in}{0.569136in}}{\pgfqpoint{2.177280in}{2.201755in}}%
\pgfusepath{clip}%
\pgfsetbuttcap%
\pgfsetroundjoin%
\definecolor{currentfill}{rgb}{1.000000,0.498039,0.054902}%
\pgfsetfillcolor{currentfill}%
\pgfsetlinewidth{0.481800pt}%
\definecolor{currentstroke}{rgb}{1.000000,1.000000,1.000000}%
\pgfsetstrokecolor{currentstroke}%
\pgfsetdash{}{0pt}%
\pgfpathmoveto{\pgfqpoint{4.143753in}{1.878546in}}%
\pgfpathcurveto{\pgfqpoint{4.154803in}{1.878546in}}{\pgfqpoint{4.165402in}{1.882936in}}{\pgfqpoint{4.173216in}{1.890750in}}%
\pgfpathcurveto{\pgfqpoint{4.181029in}{1.898563in}}{\pgfqpoint{4.185419in}{1.909162in}}{\pgfqpoint{4.185419in}{1.920213in}}%
\pgfpathcurveto{\pgfqpoint{4.185419in}{1.931263in}}{\pgfqpoint{4.181029in}{1.941862in}}{\pgfqpoint{4.173216in}{1.949675in}}%
\pgfpathcurveto{\pgfqpoint{4.165402in}{1.957489in}}{\pgfqpoint{4.154803in}{1.961879in}}{\pgfqpoint{4.143753in}{1.961879in}}%
\pgfpathcurveto{\pgfqpoint{4.132703in}{1.961879in}}{\pgfqpoint{4.122104in}{1.957489in}}{\pgfqpoint{4.114290in}{1.949675in}}%
\pgfpathcurveto{\pgfqpoint{4.106476in}{1.941862in}}{\pgfqpoint{4.102086in}{1.931263in}}{\pgfqpoint{4.102086in}{1.920213in}}%
\pgfpathcurveto{\pgfqpoint{4.102086in}{1.909162in}}{\pgfqpoint{4.106476in}{1.898563in}}{\pgfqpoint{4.114290in}{1.890750in}}%
\pgfpathcurveto{\pgfqpoint{4.122104in}{1.882936in}}{\pgfqpoint{4.132703in}{1.878546in}}{\pgfqpoint{4.143753in}{1.878546in}}%
\pgfpathlineto{\pgfqpoint{4.143753in}{1.878546in}}%
\pgfpathclose%
\pgfusepath{stroke,fill}%
\end{pgfscope}%
\begin{pgfscope}%
\pgfpathrectangle{\pgfqpoint{2.963410in}{0.569136in}}{\pgfqpoint{2.177280in}{2.201755in}}%
\pgfusepath{clip}%
\pgfsetbuttcap%
\pgfsetroundjoin%
\definecolor{currentfill}{rgb}{1.000000,0.498039,0.054902}%
\pgfsetfillcolor{currentfill}%
\pgfsetlinewidth{0.481800pt}%
\definecolor{currentstroke}{rgb}{1.000000,1.000000,1.000000}%
\pgfsetstrokecolor{currentstroke}%
\pgfsetdash{}{0pt}%
\pgfpathmoveto{\pgfqpoint{3.966527in}{1.795146in}}%
\pgfpathcurveto{\pgfqpoint{3.977577in}{1.795146in}}{\pgfqpoint{3.988176in}{1.799536in}}{\pgfqpoint{3.995990in}{1.807350in}}%
\pgfpathcurveto{\pgfqpoint{4.003803in}{1.815164in}}{\pgfqpoint{4.008194in}{1.825763in}}{\pgfqpoint{4.008194in}{1.836813in}}%
\pgfpathcurveto{\pgfqpoint{4.008194in}{1.847863in}}{\pgfqpoint{4.003803in}{1.858462in}}{\pgfqpoint{3.995990in}{1.866276in}}%
\pgfpathcurveto{\pgfqpoint{3.988176in}{1.874089in}}{\pgfqpoint{3.977577in}{1.878479in}}{\pgfqpoint{3.966527in}{1.878479in}}%
\pgfpathcurveto{\pgfqpoint{3.955477in}{1.878479in}}{\pgfqpoint{3.944878in}{1.874089in}}{\pgfqpoint{3.937064in}{1.866276in}}%
\pgfpathcurveto{\pgfqpoint{3.929251in}{1.858462in}}{\pgfqpoint{3.924860in}{1.847863in}}{\pgfqpoint{3.924860in}{1.836813in}}%
\pgfpathcurveto{\pgfqpoint{3.924860in}{1.825763in}}{\pgfqpoint{3.929251in}{1.815164in}}{\pgfqpoint{3.937064in}{1.807350in}}%
\pgfpathcurveto{\pgfqpoint{3.944878in}{1.799536in}}{\pgfqpoint{3.955477in}{1.795146in}}{\pgfqpoint{3.966527in}{1.795146in}}%
\pgfpathlineto{\pgfqpoint{3.966527in}{1.795146in}}%
\pgfpathclose%
\pgfusepath{stroke,fill}%
\end{pgfscope}%
\begin{pgfscope}%
\pgfpathrectangle{\pgfqpoint{2.963410in}{0.569136in}}{\pgfqpoint{2.177280in}{2.201755in}}%
\pgfusepath{clip}%
\pgfsetbuttcap%
\pgfsetroundjoin%
\definecolor{currentfill}{rgb}{1.000000,0.498039,0.054902}%
\pgfsetfillcolor{currentfill}%
\pgfsetlinewidth{0.481800pt}%
\definecolor{currentstroke}{rgb}{1.000000,1.000000,1.000000}%
\pgfsetstrokecolor{currentstroke}%
\pgfsetdash{}{0pt}%
\pgfpathmoveto{\pgfqpoint{3.493925in}{1.628346in}}%
\pgfpathcurveto{\pgfqpoint{3.504975in}{1.628346in}}{\pgfqpoint{3.515574in}{1.632737in}}{\pgfqpoint{3.523388in}{1.640550in}}%
\pgfpathcurveto{\pgfqpoint{3.531201in}{1.648364in}}{\pgfqpoint{3.535592in}{1.658963in}}{\pgfqpoint{3.535592in}{1.670013in}}%
\pgfpathcurveto{\pgfqpoint{3.535592in}{1.681063in}}{\pgfqpoint{3.531201in}{1.691662in}}{\pgfqpoint{3.523388in}{1.699476in}}%
\pgfpathcurveto{\pgfqpoint{3.515574in}{1.707290in}}{\pgfqpoint{3.504975in}{1.711680in}}{\pgfqpoint{3.493925in}{1.711680in}}%
\pgfpathcurveto{\pgfqpoint{3.482875in}{1.711680in}}{\pgfqpoint{3.472276in}{1.707290in}}{\pgfqpoint{3.464462in}{1.699476in}}%
\pgfpathcurveto{\pgfqpoint{3.456648in}{1.691662in}}{\pgfqpoint{3.452258in}{1.681063in}}{\pgfqpoint{3.452258in}{1.670013in}}%
\pgfpathcurveto{\pgfqpoint{3.452258in}{1.658963in}}{\pgfqpoint{3.456648in}{1.648364in}}{\pgfqpoint{3.464462in}{1.640550in}}%
\pgfpathcurveto{\pgfqpoint{3.472276in}{1.632737in}}{\pgfqpoint{3.482875in}{1.628346in}}{\pgfqpoint{3.493925in}{1.628346in}}%
\pgfpathlineto{\pgfqpoint{3.493925in}{1.628346in}}%
\pgfpathclose%
\pgfusepath{stroke,fill}%
\end{pgfscope}%
\begin{pgfscope}%
\pgfpathrectangle{\pgfqpoint{2.963410in}{0.569136in}}{\pgfqpoint{2.177280in}{2.201755in}}%
\pgfusepath{clip}%
\pgfsetbuttcap%
\pgfsetroundjoin%
\definecolor{currentfill}{rgb}{1.000000,0.498039,0.054902}%
\pgfsetfillcolor{currentfill}%
\pgfsetlinewidth{0.481800pt}%
\definecolor{currentstroke}{rgb}{1.000000,1.000000,1.000000}%
\pgfsetstrokecolor{currentstroke}%
\pgfsetdash{}{0pt}%
\pgfpathmoveto{\pgfqpoint{3.907452in}{1.628346in}}%
\pgfpathcurveto{\pgfqpoint{3.918502in}{1.628346in}}{\pgfqpoint{3.929101in}{1.632737in}}{\pgfqpoint{3.936915in}{1.640550in}}%
\pgfpathcurveto{\pgfqpoint{3.944728in}{1.648364in}}{\pgfqpoint{3.949118in}{1.658963in}}{\pgfqpoint{3.949118in}{1.670013in}}%
\pgfpathcurveto{\pgfqpoint{3.949118in}{1.681063in}}{\pgfqpoint{3.944728in}{1.691662in}}{\pgfqpoint{3.936915in}{1.699476in}}%
\pgfpathcurveto{\pgfqpoint{3.929101in}{1.707290in}}{\pgfqpoint{3.918502in}{1.711680in}}{\pgfqpoint{3.907452in}{1.711680in}}%
\pgfpathcurveto{\pgfqpoint{3.896402in}{1.711680in}}{\pgfqpoint{3.885803in}{1.707290in}}{\pgfqpoint{3.877989in}{1.699476in}}%
\pgfpathcurveto{\pgfqpoint{3.870175in}{1.691662in}}{\pgfqpoint{3.865785in}{1.681063in}}{\pgfqpoint{3.865785in}{1.670013in}}%
\pgfpathcurveto{\pgfqpoint{3.865785in}{1.658963in}}{\pgfqpoint{3.870175in}{1.648364in}}{\pgfqpoint{3.877989in}{1.640550in}}%
\pgfpathcurveto{\pgfqpoint{3.885803in}{1.632737in}}{\pgfqpoint{3.896402in}{1.628346in}}{\pgfqpoint{3.907452in}{1.628346in}}%
\pgfpathlineto{\pgfqpoint{3.907452in}{1.628346in}}%
\pgfpathclose%
\pgfusepath{stroke,fill}%
\end{pgfscope}%
\begin{pgfscope}%
\pgfpathrectangle{\pgfqpoint{2.963410in}{0.569136in}}{\pgfqpoint{2.177280in}{2.201755in}}%
\pgfusepath{clip}%
\pgfsetbuttcap%
\pgfsetroundjoin%
\definecolor{currentfill}{rgb}{1.000000,0.498039,0.054902}%
\pgfsetfillcolor{currentfill}%
\pgfsetlinewidth{0.481800pt}%
\definecolor{currentstroke}{rgb}{1.000000,1.000000,1.000000}%
\pgfsetstrokecolor{currentstroke}%
\pgfsetdash{}{0pt}%
\pgfpathmoveto{\pgfqpoint{3.612075in}{1.628346in}}%
\pgfpathcurveto{\pgfqpoint{3.623126in}{1.628346in}}{\pgfqpoint{3.633725in}{1.632737in}}{\pgfqpoint{3.641538in}{1.640550in}}%
\pgfpathcurveto{\pgfqpoint{3.649352in}{1.648364in}}{\pgfqpoint{3.653742in}{1.658963in}}{\pgfqpoint{3.653742in}{1.670013in}}%
\pgfpathcurveto{\pgfqpoint{3.653742in}{1.681063in}}{\pgfqpoint{3.649352in}{1.691662in}}{\pgfqpoint{3.641538in}{1.699476in}}%
\pgfpathcurveto{\pgfqpoint{3.633725in}{1.707290in}}{\pgfqpoint{3.623126in}{1.711680in}}{\pgfqpoint{3.612075in}{1.711680in}}%
\pgfpathcurveto{\pgfqpoint{3.601025in}{1.711680in}}{\pgfqpoint{3.590426in}{1.707290in}}{\pgfqpoint{3.582613in}{1.699476in}}%
\pgfpathcurveto{\pgfqpoint{3.574799in}{1.691662in}}{\pgfqpoint{3.570409in}{1.681063in}}{\pgfqpoint{3.570409in}{1.670013in}}%
\pgfpathcurveto{\pgfqpoint{3.570409in}{1.658963in}}{\pgfqpoint{3.574799in}{1.648364in}}{\pgfqpoint{3.582613in}{1.640550in}}%
\pgfpathcurveto{\pgfqpoint{3.590426in}{1.632737in}}{\pgfqpoint{3.601025in}{1.628346in}}{\pgfqpoint{3.612075in}{1.628346in}}%
\pgfpathlineto{\pgfqpoint{3.612075in}{1.628346in}}%
\pgfpathclose%
\pgfusepath{stroke,fill}%
\end{pgfscope}%
\begin{pgfscope}%
\pgfpathrectangle{\pgfqpoint{2.963410in}{0.569136in}}{\pgfqpoint{2.177280in}{2.201755in}}%
\pgfusepath{clip}%
\pgfsetbuttcap%
\pgfsetroundjoin%
\definecolor{currentfill}{rgb}{1.000000,0.498039,0.054902}%
\pgfsetfillcolor{currentfill}%
\pgfsetlinewidth{0.481800pt}%
\definecolor{currentstroke}{rgb}{1.000000,1.000000,1.000000}%
\pgfsetstrokecolor{currentstroke}%
\pgfsetdash{}{0pt}%
\pgfpathmoveto{\pgfqpoint{3.671151in}{1.544947in}}%
\pgfpathcurveto{\pgfqpoint{3.682201in}{1.544947in}}{\pgfqpoint{3.692800in}{1.549337in}}{\pgfqpoint{3.700613in}{1.557151in}}%
\pgfpathcurveto{\pgfqpoint{3.708427in}{1.564964in}}{\pgfqpoint{3.712817in}{1.575563in}}{\pgfqpoint{3.712817in}{1.586613in}}%
\pgfpathcurveto{\pgfqpoint{3.712817in}{1.597663in}}{\pgfqpoint{3.708427in}{1.608262in}}{\pgfqpoint{3.700613in}{1.616076in}}%
\pgfpathcurveto{\pgfqpoint{3.692800in}{1.623890in}}{\pgfqpoint{3.682201in}{1.628280in}}{\pgfqpoint{3.671151in}{1.628280in}}%
\pgfpathcurveto{\pgfqpoint{3.660101in}{1.628280in}}{\pgfqpoint{3.649501in}{1.623890in}}{\pgfqpoint{3.641688in}{1.616076in}}%
\pgfpathcurveto{\pgfqpoint{3.633874in}{1.608262in}}{\pgfqpoint{3.629484in}{1.597663in}}{\pgfqpoint{3.629484in}{1.586613in}}%
\pgfpathcurveto{\pgfqpoint{3.629484in}{1.575563in}}{\pgfqpoint{3.633874in}{1.564964in}}{\pgfqpoint{3.641688in}{1.557151in}}%
\pgfpathcurveto{\pgfqpoint{3.649501in}{1.549337in}}{\pgfqpoint{3.660101in}{1.544947in}}{\pgfqpoint{3.671151in}{1.544947in}}%
\pgfpathlineto{\pgfqpoint{3.671151in}{1.544947in}}%
\pgfpathclose%
\pgfusepath{stroke,fill}%
\end{pgfscope}%
\begin{pgfscope}%
\pgfpathrectangle{\pgfqpoint{2.963410in}{0.569136in}}{\pgfqpoint{2.177280in}{2.201755in}}%
\pgfusepath{clip}%
\pgfsetbuttcap%
\pgfsetroundjoin%
\definecolor{currentfill}{rgb}{1.000000,0.498039,0.054902}%
\pgfsetfillcolor{currentfill}%
\pgfsetlinewidth{0.481800pt}%
\definecolor{currentstroke}{rgb}{1.000000,1.000000,1.000000}%
\pgfsetstrokecolor{currentstroke}%
\pgfsetdash{}{0pt}%
\pgfpathmoveto{\pgfqpoint{3.907452in}{1.711746in}}%
\pgfpathcurveto{\pgfqpoint{3.918502in}{1.711746in}}{\pgfqpoint{3.929101in}{1.716137in}}{\pgfqpoint{3.936915in}{1.723950in}}%
\pgfpathcurveto{\pgfqpoint{3.944728in}{1.731764in}}{\pgfqpoint{3.949118in}{1.742363in}}{\pgfqpoint{3.949118in}{1.753413in}}%
\pgfpathcurveto{\pgfqpoint{3.949118in}{1.764463in}}{\pgfqpoint{3.944728in}{1.775062in}}{\pgfqpoint{3.936915in}{1.782876in}}%
\pgfpathcurveto{\pgfqpoint{3.929101in}{1.790689in}}{\pgfqpoint{3.918502in}{1.795080in}}{\pgfqpoint{3.907452in}{1.795080in}}%
\pgfpathcurveto{\pgfqpoint{3.896402in}{1.795080in}}{\pgfqpoint{3.885803in}{1.790689in}}{\pgfqpoint{3.877989in}{1.782876in}}%
\pgfpathcurveto{\pgfqpoint{3.870175in}{1.775062in}}{\pgfqpoint{3.865785in}{1.764463in}}{\pgfqpoint{3.865785in}{1.753413in}}%
\pgfpathcurveto{\pgfqpoint{3.865785in}{1.742363in}}{\pgfqpoint{3.870175in}{1.731764in}}{\pgfqpoint{3.877989in}{1.723950in}}%
\pgfpathcurveto{\pgfqpoint{3.885803in}{1.716137in}}{\pgfqpoint{3.896402in}{1.711746in}}{\pgfqpoint{3.907452in}{1.711746in}}%
\pgfpathlineto{\pgfqpoint{3.907452in}{1.711746in}}%
\pgfpathclose%
\pgfusepath{stroke,fill}%
\end{pgfscope}%
\begin{pgfscope}%
\pgfpathrectangle{\pgfqpoint{2.963410in}{0.569136in}}{\pgfqpoint{2.177280in}{2.201755in}}%
\pgfusepath{clip}%
\pgfsetbuttcap%
\pgfsetroundjoin%
\definecolor{currentfill}{rgb}{1.000000,0.498039,0.054902}%
\pgfsetfillcolor{currentfill}%
\pgfsetlinewidth{0.481800pt}%
\definecolor{currentstroke}{rgb}{1.000000,1.000000,1.000000}%
\pgfsetstrokecolor{currentstroke}%
\pgfsetdash{}{0pt}%
\pgfpathmoveto{\pgfqpoint{3.671151in}{1.544947in}}%
\pgfpathcurveto{\pgfqpoint{3.682201in}{1.544947in}}{\pgfqpoint{3.692800in}{1.549337in}}{\pgfqpoint{3.700613in}{1.557151in}}%
\pgfpathcurveto{\pgfqpoint{3.708427in}{1.564964in}}{\pgfqpoint{3.712817in}{1.575563in}}{\pgfqpoint{3.712817in}{1.586613in}}%
\pgfpathcurveto{\pgfqpoint{3.712817in}{1.597663in}}{\pgfqpoint{3.708427in}{1.608262in}}{\pgfqpoint{3.700613in}{1.616076in}}%
\pgfpathcurveto{\pgfqpoint{3.692800in}{1.623890in}}{\pgfqpoint{3.682201in}{1.628280in}}{\pgfqpoint{3.671151in}{1.628280in}}%
\pgfpathcurveto{\pgfqpoint{3.660101in}{1.628280in}}{\pgfqpoint{3.649501in}{1.623890in}}{\pgfqpoint{3.641688in}{1.616076in}}%
\pgfpathcurveto{\pgfqpoint{3.633874in}{1.608262in}}{\pgfqpoint{3.629484in}{1.597663in}}{\pgfqpoint{3.629484in}{1.586613in}}%
\pgfpathcurveto{\pgfqpoint{3.629484in}{1.575563in}}{\pgfqpoint{3.633874in}{1.564964in}}{\pgfqpoint{3.641688in}{1.557151in}}%
\pgfpathcurveto{\pgfqpoint{3.649501in}{1.549337in}}{\pgfqpoint{3.660101in}{1.544947in}}{\pgfqpoint{3.671151in}{1.544947in}}%
\pgfpathlineto{\pgfqpoint{3.671151in}{1.544947in}}%
\pgfpathclose%
\pgfusepath{stroke,fill}%
\end{pgfscope}%
\begin{pgfscope}%
\pgfpathrectangle{\pgfqpoint{2.963410in}{0.569136in}}{\pgfqpoint{2.177280in}{2.201755in}}%
\pgfusepath{clip}%
\pgfsetbuttcap%
\pgfsetroundjoin%
\definecolor{currentfill}{rgb}{1.000000,0.498039,0.054902}%
\pgfsetfillcolor{currentfill}%
\pgfsetlinewidth{0.481800pt}%
\definecolor{currentstroke}{rgb}{1.000000,1.000000,1.000000}%
\pgfsetstrokecolor{currentstroke}%
\pgfsetdash{}{0pt}%
\pgfpathmoveto{\pgfqpoint{3.493925in}{1.378147in}}%
\pgfpathcurveto{\pgfqpoint{3.504975in}{1.378147in}}{\pgfqpoint{3.515574in}{1.382537in}}{\pgfqpoint{3.523388in}{1.390351in}}%
\pgfpathcurveto{\pgfqpoint{3.531201in}{1.398164in}}{\pgfqpoint{3.535592in}{1.408764in}}{\pgfqpoint{3.535592in}{1.419814in}}%
\pgfpathcurveto{\pgfqpoint{3.535592in}{1.430864in}}{\pgfqpoint{3.531201in}{1.441463in}}{\pgfqpoint{3.523388in}{1.449276in}}%
\pgfpathcurveto{\pgfqpoint{3.515574in}{1.457090in}}{\pgfqpoint{3.504975in}{1.461480in}}{\pgfqpoint{3.493925in}{1.461480in}}%
\pgfpathcurveto{\pgfqpoint{3.482875in}{1.461480in}}{\pgfqpoint{3.472276in}{1.457090in}}{\pgfqpoint{3.464462in}{1.449276in}}%
\pgfpathcurveto{\pgfqpoint{3.456648in}{1.441463in}}{\pgfqpoint{3.452258in}{1.430864in}}{\pgfqpoint{3.452258in}{1.419814in}}%
\pgfpathcurveto{\pgfqpoint{3.452258in}{1.408764in}}{\pgfqpoint{3.456648in}{1.398164in}}{\pgfqpoint{3.464462in}{1.390351in}}%
\pgfpathcurveto{\pgfqpoint{3.472276in}{1.382537in}}{\pgfqpoint{3.482875in}{1.378147in}}{\pgfqpoint{3.493925in}{1.378147in}}%
\pgfpathlineto{\pgfqpoint{3.493925in}{1.378147in}}%
\pgfpathclose%
\pgfusepath{stroke,fill}%
\end{pgfscope}%
\begin{pgfscope}%
\pgfpathrectangle{\pgfqpoint{2.963410in}{0.569136in}}{\pgfqpoint{2.177280in}{2.201755in}}%
\pgfusepath{clip}%
\pgfsetbuttcap%
\pgfsetroundjoin%
\definecolor{currentfill}{rgb}{1.000000,0.498039,0.054902}%
\pgfsetfillcolor{currentfill}%
\pgfsetlinewidth{0.481800pt}%
\definecolor{currentstroke}{rgb}{1.000000,1.000000,1.000000}%
\pgfsetstrokecolor{currentstroke}%
\pgfsetdash{}{0pt}%
\pgfpathmoveto{\pgfqpoint{3.730226in}{1.628346in}}%
\pgfpathcurveto{\pgfqpoint{3.741276in}{1.628346in}}{\pgfqpoint{3.751875in}{1.632737in}}{\pgfqpoint{3.759689in}{1.640550in}}%
\pgfpathcurveto{\pgfqpoint{3.767502in}{1.648364in}}{\pgfqpoint{3.771893in}{1.658963in}}{\pgfqpoint{3.771893in}{1.670013in}}%
\pgfpathcurveto{\pgfqpoint{3.771893in}{1.681063in}}{\pgfqpoint{3.767502in}{1.691662in}}{\pgfqpoint{3.759689in}{1.699476in}}%
\pgfpathcurveto{\pgfqpoint{3.751875in}{1.707290in}}{\pgfqpoint{3.741276in}{1.711680in}}{\pgfqpoint{3.730226in}{1.711680in}}%
\pgfpathcurveto{\pgfqpoint{3.719176in}{1.711680in}}{\pgfqpoint{3.708577in}{1.707290in}}{\pgfqpoint{3.700763in}{1.699476in}}%
\pgfpathcurveto{\pgfqpoint{3.692950in}{1.691662in}}{\pgfqpoint{3.688559in}{1.681063in}}{\pgfqpoint{3.688559in}{1.670013in}}%
\pgfpathcurveto{\pgfqpoint{3.688559in}{1.658963in}}{\pgfqpoint{3.692950in}{1.648364in}}{\pgfqpoint{3.700763in}{1.640550in}}%
\pgfpathcurveto{\pgfqpoint{3.708577in}{1.632737in}}{\pgfqpoint{3.719176in}{1.628346in}}{\pgfqpoint{3.730226in}{1.628346in}}%
\pgfpathlineto{\pgfqpoint{3.730226in}{1.628346in}}%
\pgfpathclose%
\pgfusepath{stroke,fill}%
\end{pgfscope}%
\begin{pgfscope}%
\pgfpathrectangle{\pgfqpoint{2.963410in}{0.569136in}}{\pgfqpoint{2.177280in}{2.201755in}}%
\pgfusepath{clip}%
\pgfsetbuttcap%
\pgfsetroundjoin%
\definecolor{currentfill}{rgb}{1.000000,0.498039,0.054902}%
\pgfsetfillcolor{currentfill}%
\pgfsetlinewidth{0.481800pt}%
\definecolor{currentstroke}{rgb}{1.000000,1.000000,1.000000}%
\pgfsetstrokecolor{currentstroke}%
\pgfsetdash{}{0pt}%
\pgfpathmoveto{\pgfqpoint{3.907452in}{1.544947in}}%
\pgfpathcurveto{\pgfqpoint{3.918502in}{1.544947in}}{\pgfqpoint{3.929101in}{1.549337in}}{\pgfqpoint{3.936915in}{1.557151in}}%
\pgfpathcurveto{\pgfqpoint{3.944728in}{1.564964in}}{\pgfqpoint{3.949118in}{1.575563in}}{\pgfqpoint{3.949118in}{1.586613in}}%
\pgfpathcurveto{\pgfqpoint{3.949118in}{1.597663in}}{\pgfqpoint{3.944728in}{1.608262in}}{\pgfqpoint{3.936915in}{1.616076in}}%
\pgfpathcurveto{\pgfqpoint{3.929101in}{1.623890in}}{\pgfqpoint{3.918502in}{1.628280in}}{\pgfqpoint{3.907452in}{1.628280in}}%
\pgfpathcurveto{\pgfqpoint{3.896402in}{1.628280in}}{\pgfqpoint{3.885803in}{1.623890in}}{\pgfqpoint{3.877989in}{1.616076in}}%
\pgfpathcurveto{\pgfqpoint{3.870175in}{1.608262in}}{\pgfqpoint{3.865785in}{1.597663in}}{\pgfqpoint{3.865785in}{1.586613in}}%
\pgfpathcurveto{\pgfqpoint{3.865785in}{1.575563in}}{\pgfqpoint{3.870175in}{1.564964in}}{\pgfqpoint{3.877989in}{1.557151in}}%
\pgfpathcurveto{\pgfqpoint{3.885803in}{1.549337in}}{\pgfqpoint{3.896402in}{1.544947in}}{\pgfqpoint{3.907452in}{1.544947in}}%
\pgfpathlineto{\pgfqpoint{3.907452in}{1.544947in}}%
\pgfpathclose%
\pgfusepath{stroke,fill}%
\end{pgfscope}%
\begin{pgfscope}%
\pgfpathrectangle{\pgfqpoint{2.963410in}{0.569136in}}{\pgfqpoint{2.177280in}{2.201755in}}%
\pgfusepath{clip}%
\pgfsetbuttcap%
\pgfsetroundjoin%
\definecolor{currentfill}{rgb}{1.000000,0.498039,0.054902}%
\pgfsetfillcolor{currentfill}%
\pgfsetlinewidth{0.481800pt}%
\definecolor{currentstroke}{rgb}{1.000000,1.000000,1.000000}%
\pgfsetstrokecolor{currentstroke}%
\pgfsetdash{}{0pt}%
\pgfpathmoveto{\pgfqpoint{3.848376in}{1.628346in}}%
\pgfpathcurveto{\pgfqpoint{3.859427in}{1.628346in}}{\pgfqpoint{3.870026in}{1.632737in}}{\pgfqpoint{3.877839in}{1.640550in}}%
\pgfpathcurveto{\pgfqpoint{3.885653in}{1.648364in}}{\pgfqpoint{3.890043in}{1.658963in}}{\pgfqpoint{3.890043in}{1.670013in}}%
\pgfpathcurveto{\pgfqpoint{3.890043in}{1.681063in}}{\pgfqpoint{3.885653in}{1.691662in}}{\pgfqpoint{3.877839in}{1.699476in}}%
\pgfpathcurveto{\pgfqpoint{3.870026in}{1.707290in}}{\pgfqpoint{3.859427in}{1.711680in}}{\pgfqpoint{3.848376in}{1.711680in}}%
\pgfpathcurveto{\pgfqpoint{3.837326in}{1.711680in}}{\pgfqpoint{3.826727in}{1.707290in}}{\pgfqpoint{3.818914in}{1.699476in}}%
\pgfpathcurveto{\pgfqpoint{3.811100in}{1.691662in}}{\pgfqpoint{3.806710in}{1.681063in}}{\pgfqpoint{3.806710in}{1.670013in}}%
\pgfpathcurveto{\pgfqpoint{3.806710in}{1.658963in}}{\pgfqpoint{3.811100in}{1.648364in}}{\pgfqpoint{3.818914in}{1.640550in}}%
\pgfpathcurveto{\pgfqpoint{3.826727in}{1.632737in}}{\pgfqpoint{3.837326in}{1.628346in}}{\pgfqpoint{3.848376in}{1.628346in}}%
\pgfpathlineto{\pgfqpoint{3.848376in}{1.628346in}}%
\pgfpathclose%
\pgfusepath{stroke,fill}%
\end{pgfscope}%
\begin{pgfscope}%
\pgfpathrectangle{\pgfqpoint{2.963410in}{0.569136in}}{\pgfqpoint{2.177280in}{2.201755in}}%
\pgfusepath{clip}%
\pgfsetbuttcap%
\pgfsetroundjoin%
\definecolor{currentfill}{rgb}{1.000000,0.498039,0.054902}%
\pgfsetfillcolor{currentfill}%
\pgfsetlinewidth{0.481800pt}%
\definecolor{currentstroke}{rgb}{1.000000,1.000000,1.000000}%
\pgfsetstrokecolor{currentstroke}%
\pgfsetdash{}{0pt}%
\pgfpathmoveto{\pgfqpoint{3.848376in}{1.628346in}}%
\pgfpathcurveto{\pgfqpoint{3.859427in}{1.628346in}}{\pgfqpoint{3.870026in}{1.632737in}}{\pgfqpoint{3.877839in}{1.640550in}}%
\pgfpathcurveto{\pgfqpoint{3.885653in}{1.648364in}}{\pgfqpoint{3.890043in}{1.658963in}}{\pgfqpoint{3.890043in}{1.670013in}}%
\pgfpathcurveto{\pgfqpoint{3.890043in}{1.681063in}}{\pgfqpoint{3.885653in}{1.691662in}}{\pgfqpoint{3.877839in}{1.699476in}}%
\pgfpathcurveto{\pgfqpoint{3.870026in}{1.707290in}}{\pgfqpoint{3.859427in}{1.711680in}}{\pgfqpoint{3.848376in}{1.711680in}}%
\pgfpathcurveto{\pgfqpoint{3.837326in}{1.711680in}}{\pgfqpoint{3.826727in}{1.707290in}}{\pgfqpoint{3.818914in}{1.699476in}}%
\pgfpathcurveto{\pgfqpoint{3.811100in}{1.691662in}}{\pgfqpoint{3.806710in}{1.681063in}}{\pgfqpoint{3.806710in}{1.670013in}}%
\pgfpathcurveto{\pgfqpoint{3.806710in}{1.658963in}}{\pgfqpoint{3.811100in}{1.648364in}}{\pgfqpoint{3.818914in}{1.640550in}}%
\pgfpathcurveto{\pgfqpoint{3.826727in}{1.632737in}}{\pgfqpoint{3.837326in}{1.628346in}}{\pgfqpoint{3.848376in}{1.628346in}}%
\pgfpathlineto{\pgfqpoint{3.848376in}{1.628346in}}%
\pgfpathclose%
\pgfusepath{stroke,fill}%
\end{pgfscope}%
\begin{pgfscope}%
\pgfpathrectangle{\pgfqpoint{2.963410in}{0.569136in}}{\pgfqpoint{2.177280in}{2.201755in}}%
\pgfusepath{clip}%
\pgfsetbuttcap%
\pgfsetroundjoin%
\definecolor{currentfill}{rgb}{1.000000,0.498039,0.054902}%
\pgfsetfillcolor{currentfill}%
\pgfsetlinewidth{0.481800pt}%
\definecolor{currentstroke}{rgb}{1.000000,1.000000,1.000000}%
\pgfsetstrokecolor{currentstroke}%
\pgfsetdash{}{0pt}%
\pgfpathmoveto{\pgfqpoint{3.612075in}{1.461547in}}%
\pgfpathcurveto{\pgfqpoint{3.623126in}{1.461547in}}{\pgfqpoint{3.633725in}{1.465937in}}{\pgfqpoint{3.641538in}{1.473751in}}%
\pgfpathcurveto{\pgfqpoint{3.649352in}{1.481564in}}{\pgfqpoint{3.653742in}{1.492163in}}{\pgfqpoint{3.653742in}{1.503213in}}%
\pgfpathcurveto{\pgfqpoint{3.653742in}{1.514264in}}{\pgfqpoint{3.649352in}{1.524863in}}{\pgfqpoint{3.641538in}{1.532676in}}%
\pgfpathcurveto{\pgfqpoint{3.633725in}{1.540490in}}{\pgfqpoint{3.623126in}{1.544880in}}{\pgfqpoint{3.612075in}{1.544880in}}%
\pgfpathcurveto{\pgfqpoint{3.601025in}{1.544880in}}{\pgfqpoint{3.590426in}{1.540490in}}{\pgfqpoint{3.582613in}{1.532676in}}%
\pgfpathcurveto{\pgfqpoint{3.574799in}{1.524863in}}{\pgfqpoint{3.570409in}{1.514264in}}{\pgfqpoint{3.570409in}{1.503213in}}%
\pgfpathcurveto{\pgfqpoint{3.570409in}{1.492163in}}{\pgfqpoint{3.574799in}{1.481564in}}{\pgfqpoint{3.582613in}{1.473751in}}%
\pgfpathcurveto{\pgfqpoint{3.590426in}{1.465937in}}{\pgfqpoint{3.601025in}{1.461547in}}{\pgfqpoint{3.612075in}{1.461547in}}%
\pgfpathlineto{\pgfqpoint{3.612075in}{1.461547in}}%
\pgfpathclose%
\pgfusepath{stroke,fill}%
\end{pgfscope}%
\begin{pgfscope}%
\pgfpathrectangle{\pgfqpoint{2.963410in}{0.569136in}}{\pgfqpoint{2.177280in}{2.201755in}}%
\pgfusepath{clip}%
\pgfsetbuttcap%
\pgfsetroundjoin%
\definecolor{currentfill}{rgb}{1.000000,0.498039,0.054902}%
\pgfsetfillcolor{currentfill}%
\pgfsetlinewidth{0.481800pt}%
\definecolor{currentstroke}{rgb}{1.000000,1.000000,1.000000}%
\pgfsetstrokecolor{currentstroke}%
\pgfsetdash{}{0pt}%
\pgfpathmoveto{\pgfqpoint{3.789301in}{1.628346in}}%
\pgfpathcurveto{\pgfqpoint{3.800351in}{1.628346in}}{\pgfqpoint{3.810950in}{1.632737in}}{\pgfqpoint{3.818764in}{1.640550in}}%
\pgfpathcurveto{\pgfqpoint{3.826578in}{1.648364in}}{\pgfqpoint{3.830968in}{1.658963in}}{\pgfqpoint{3.830968in}{1.670013in}}%
\pgfpathcurveto{\pgfqpoint{3.830968in}{1.681063in}}{\pgfqpoint{3.826578in}{1.691662in}}{\pgfqpoint{3.818764in}{1.699476in}}%
\pgfpathcurveto{\pgfqpoint{3.810950in}{1.707290in}}{\pgfqpoint{3.800351in}{1.711680in}}{\pgfqpoint{3.789301in}{1.711680in}}%
\pgfpathcurveto{\pgfqpoint{3.778251in}{1.711680in}}{\pgfqpoint{3.767652in}{1.707290in}}{\pgfqpoint{3.759838in}{1.699476in}}%
\pgfpathcurveto{\pgfqpoint{3.752025in}{1.691662in}}{\pgfqpoint{3.747635in}{1.681063in}}{\pgfqpoint{3.747635in}{1.670013in}}%
\pgfpathcurveto{\pgfqpoint{3.747635in}{1.658963in}}{\pgfqpoint{3.752025in}{1.648364in}}{\pgfqpoint{3.759838in}{1.640550in}}%
\pgfpathcurveto{\pgfqpoint{3.767652in}{1.632737in}}{\pgfqpoint{3.778251in}{1.628346in}}{\pgfqpoint{3.789301in}{1.628346in}}%
\pgfpathlineto{\pgfqpoint{3.789301in}{1.628346in}}%
\pgfpathclose%
\pgfusepath{stroke,fill}%
\end{pgfscope}%
\begin{pgfscope}%
\pgfpathrectangle{\pgfqpoint{2.963410in}{0.569136in}}{\pgfqpoint{2.177280in}{2.201755in}}%
\pgfusepath{clip}%
\pgfsetbuttcap%
\pgfsetroundjoin%
\definecolor{currentfill}{rgb}{0.172549,0.627451,0.172549}%
\pgfsetfillcolor{currentfill}%
\pgfsetlinewidth{0.481800pt}%
\definecolor{currentstroke}{rgb}{1.000000,1.000000,1.000000}%
\pgfsetstrokecolor{currentstroke}%
\pgfsetdash{}{0pt}%
\pgfpathmoveto{\pgfqpoint{4.084678in}{2.629144in}}%
\pgfpathcurveto{\pgfqpoint{4.095728in}{2.629144in}}{\pgfqpoint{4.106327in}{2.633534in}}{\pgfqpoint{4.114140in}{2.641348in}}%
\pgfpathcurveto{\pgfqpoint{4.121954in}{2.649162in}}{\pgfqpoint{4.126344in}{2.659761in}}{\pgfqpoint{4.126344in}{2.670811in}}%
\pgfpathcurveto{\pgfqpoint{4.126344in}{2.681861in}}{\pgfqpoint{4.121954in}{2.692460in}}{\pgfqpoint{4.114140in}{2.700274in}}%
\pgfpathcurveto{\pgfqpoint{4.106327in}{2.708087in}}{\pgfqpoint{4.095728in}{2.712478in}}{\pgfqpoint{4.084678in}{2.712478in}}%
\pgfpathcurveto{\pgfqpoint{4.073627in}{2.712478in}}{\pgfqpoint{4.063028in}{2.708087in}}{\pgfqpoint{4.055215in}{2.700274in}}%
\pgfpathcurveto{\pgfqpoint{4.047401in}{2.692460in}}{\pgfqpoint{4.043011in}{2.681861in}}{\pgfqpoint{4.043011in}{2.670811in}}%
\pgfpathcurveto{\pgfqpoint{4.043011in}{2.659761in}}{\pgfqpoint{4.047401in}{2.649162in}}{\pgfqpoint{4.055215in}{2.641348in}}%
\pgfpathcurveto{\pgfqpoint{4.063028in}{2.633534in}}{\pgfqpoint{4.073627in}{2.629144in}}{\pgfqpoint{4.084678in}{2.629144in}}%
\pgfpathlineto{\pgfqpoint{4.084678in}{2.629144in}}%
\pgfpathclose%
\pgfusepath{stroke,fill}%
\end{pgfscope}%
\begin{pgfscope}%
\pgfpathrectangle{\pgfqpoint{2.963410in}{0.569136in}}{\pgfqpoint{2.177280in}{2.201755in}}%
\pgfusepath{clip}%
\pgfsetbuttcap%
\pgfsetroundjoin%
\definecolor{currentfill}{rgb}{0.172549,0.627451,0.172549}%
\pgfsetfillcolor{currentfill}%
\pgfsetlinewidth{0.481800pt}%
\definecolor{currentstroke}{rgb}{1.000000,1.000000,1.000000}%
\pgfsetstrokecolor{currentstroke}%
\pgfsetdash{}{0pt}%
\pgfpathmoveto{\pgfqpoint{3.730226in}{2.128745in}}%
\pgfpathcurveto{\pgfqpoint{3.741276in}{2.128745in}}{\pgfqpoint{3.751875in}{2.133136in}}{\pgfqpoint{3.759689in}{2.140949in}}%
\pgfpathcurveto{\pgfqpoint{3.767502in}{2.148763in}}{\pgfqpoint{3.771893in}{2.159362in}}{\pgfqpoint{3.771893in}{2.170412in}}%
\pgfpathcurveto{\pgfqpoint{3.771893in}{2.181462in}}{\pgfqpoint{3.767502in}{2.192061in}}{\pgfqpoint{3.759689in}{2.199875in}}%
\pgfpathcurveto{\pgfqpoint{3.751875in}{2.207688in}}{\pgfqpoint{3.741276in}{2.212079in}}{\pgfqpoint{3.730226in}{2.212079in}}%
\pgfpathcurveto{\pgfqpoint{3.719176in}{2.212079in}}{\pgfqpoint{3.708577in}{2.207688in}}{\pgfqpoint{3.700763in}{2.199875in}}%
\pgfpathcurveto{\pgfqpoint{3.692950in}{2.192061in}}{\pgfqpoint{3.688559in}{2.181462in}}{\pgfqpoint{3.688559in}{2.170412in}}%
\pgfpathcurveto{\pgfqpoint{3.688559in}{2.159362in}}{\pgfqpoint{3.692950in}{2.148763in}}{\pgfqpoint{3.700763in}{2.140949in}}%
\pgfpathcurveto{\pgfqpoint{3.708577in}{2.133136in}}{\pgfqpoint{3.719176in}{2.128745in}}{\pgfqpoint{3.730226in}{2.128745in}}%
\pgfpathlineto{\pgfqpoint{3.730226in}{2.128745in}}%
\pgfpathclose%
\pgfusepath{stroke,fill}%
\end{pgfscope}%
\begin{pgfscope}%
\pgfpathrectangle{\pgfqpoint{2.963410in}{0.569136in}}{\pgfqpoint{2.177280in}{2.201755in}}%
\pgfusepath{clip}%
\pgfsetbuttcap%
\pgfsetroundjoin%
\definecolor{currentfill}{rgb}{0.172549,0.627451,0.172549}%
\pgfsetfillcolor{currentfill}%
\pgfsetlinewidth{0.481800pt}%
\definecolor{currentstroke}{rgb}{1.000000,1.000000,1.000000}%
\pgfsetstrokecolor{currentstroke}%
\pgfsetdash{}{0pt}%
\pgfpathmoveto{\pgfqpoint{3.907452in}{2.295545in}}%
\pgfpathcurveto{\pgfqpoint{3.918502in}{2.295545in}}{\pgfqpoint{3.929101in}{2.299935in}}{\pgfqpoint{3.936915in}{2.307749in}}%
\pgfpathcurveto{\pgfqpoint{3.944728in}{2.315562in}}{\pgfqpoint{3.949118in}{2.326161in}}{\pgfqpoint{3.949118in}{2.337212in}}%
\pgfpathcurveto{\pgfqpoint{3.949118in}{2.348262in}}{\pgfqpoint{3.944728in}{2.358861in}}{\pgfqpoint{3.936915in}{2.366674in}}%
\pgfpathcurveto{\pgfqpoint{3.929101in}{2.374488in}}{\pgfqpoint{3.918502in}{2.378878in}}{\pgfqpoint{3.907452in}{2.378878in}}%
\pgfpathcurveto{\pgfqpoint{3.896402in}{2.378878in}}{\pgfqpoint{3.885803in}{2.374488in}}{\pgfqpoint{3.877989in}{2.366674in}}%
\pgfpathcurveto{\pgfqpoint{3.870175in}{2.358861in}}{\pgfqpoint{3.865785in}{2.348262in}}{\pgfqpoint{3.865785in}{2.337212in}}%
\pgfpathcurveto{\pgfqpoint{3.865785in}{2.326161in}}{\pgfqpoint{3.870175in}{2.315562in}}{\pgfqpoint{3.877989in}{2.307749in}}%
\pgfpathcurveto{\pgfqpoint{3.885803in}{2.299935in}}{\pgfqpoint{3.896402in}{2.295545in}}{\pgfqpoint{3.907452in}{2.295545in}}%
\pgfpathlineto{\pgfqpoint{3.907452in}{2.295545in}}%
\pgfpathclose%
\pgfusepath{stroke,fill}%
\end{pgfscope}%
\begin{pgfscope}%
\pgfpathrectangle{\pgfqpoint{2.963410in}{0.569136in}}{\pgfqpoint{2.177280in}{2.201755in}}%
\pgfusepath{clip}%
\pgfsetbuttcap%
\pgfsetroundjoin%
\definecolor{currentfill}{rgb}{0.172549,0.627451,0.172549}%
\pgfsetfillcolor{currentfill}%
\pgfsetlinewidth{0.481800pt}%
\definecolor{currentstroke}{rgb}{1.000000,1.000000,1.000000}%
\pgfsetstrokecolor{currentstroke}%
\pgfsetdash{}{0pt}%
\pgfpathmoveto{\pgfqpoint{3.848376in}{2.045346in}}%
\pgfpathcurveto{\pgfqpoint{3.859427in}{2.045346in}}{\pgfqpoint{3.870026in}{2.049736in}}{\pgfqpoint{3.877839in}{2.057549in}}%
\pgfpathcurveto{\pgfqpoint{3.885653in}{2.065363in}}{\pgfqpoint{3.890043in}{2.075962in}}{\pgfqpoint{3.890043in}{2.087012in}}%
\pgfpathcurveto{\pgfqpoint{3.890043in}{2.098062in}}{\pgfqpoint{3.885653in}{2.108661in}}{\pgfqpoint{3.877839in}{2.116475in}}%
\pgfpathcurveto{\pgfqpoint{3.870026in}{2.124289in}}{\pgfqpoint{3.859427in}{2.128679in}}{\pgfqpoint{3.848376in}{2.128679in}}%
\pgfpathcurveto{\pgfqpoint{3.837326in}{2.128679in}}{\pgfqpoint{3.826727in}{2.124289in}}{\pgfqpoint{3.818914in}{2.116475in}}%
\pgfpathcurveto{\pgfqpoint{3.811100in}{2.108661in}}{\pgfqpoint{3.806710in}{2.098062in}}{\pgfqpoint{3.806710in}{2.087012in}}%
\pgfpathcurveto{\pgfqpoint{3.806710in}{2.075962in}}{\pgfqpoint{3.811100in}{2.065363in}}{\pgfqpoint{3.818914in}{2.057549in}}%
\pgfpathcurveto{\pgfqpoint{3.826727in}{2.049736in}}{\pgfqpoint{3.837326in}{2.045346in}}{\pgfqpoint{3.848376in}{2.045346in}}%
\pgfpathlineto{\pgfqpoint{3.848376in}{2.045346in}}%
\pgfpathclose%
\pgfusepath{stroke,fill}%
\end{pgfscope}%
\begin{pgfscope}%
\pgfpathrectangle{\pgfqpoint{2.963410in}{0.569136in}}{\pgfqpoint{2.177280in}{2.201755in}}%
\pgfusepath{clip}%
\pgfsetbuttcap%
\pgfsetroundjoin%
\definecolor{currentfill}{rgb}{0.172549,0.627451,0.172549}%
\pgfsetfillcolor{currentfill}%
\pgfsetlinewidth{0.481800pt}%
\definecolor{currentstroke}{rgb}{1.000000,1.000000,1.000000}%
\pgfsetstrokecolor{currentstroke}%
\pgfsetdash{}{0pt}%
\pgfpathmoveto{\pgfqpoint{3.907452in}{2.378945in}}%
\pgfpathcurveto{\pgfqpoint{3.918502in}{2.378945in}}{\pgfqpoint{3.929101in}{2.383335in}}{\pgfqpoint{3.936915in}{2.391149in}}%
\pgfpathcurveto{\pgfqpoint{3.944728in}{2.398962in}}{\pgfqpoint{3.949118in}{2.409561in}}{\pgfqpoint{3.949118in}{2.420611in}}%
\pgfpathcurveto{\pgfqpoint{3.949118in}{2.431662in}}{\pgfqpoint{3.944728in}{2.442261in}}{\pgfqpoint{3.936915in}{2.450074in}}%
\pgfpathcurveto{\pgfqpoint{3.929101in}{2.457888in}}{\pgfqpoint{3.918502in}{2.462278in}}{\pgfqpoint{3.907452in}{2.462278in}}%
\pgfpathcurveto{\pgfqpoint{3.896402in}{2.462278in}}{\pgfqpoint{3.885803in}{2.457888in}}{\pgfqpoint{3.877989in}{2.450074in}}%
\pgfpathcurveto{\pgfqpoint{3.870175in}{2.442261in}}{\pgfqpoint{3.865785in}{2.431662in}}{\pgfqpoint{3.865785in}{2.420611in}}%
\pgfpathcurveto{\pgfqpoint{3.865785in}{2.409561in}}{\pgfqpoint{3.870175in}{2.398962in}}{\pgfqpoint{3.877989in}{2.391149in}}%
\pgfpathcurveto{\pgfqpoint{3.885803in}{2.383335in}}{\pgfqpoint{3.896402in}{2.378945in}}{\pgfqpoint{3.907452in}{2.378945in}}%
\pgfpathlineto{\pgfqpoint{3.907452in}{2.378945in}}%
\pgfpathclose%
\pgfusepath{stroke,fill}%
\end{pgfscope}%
\begin{pgfscope}%
\pgfpathrectangle{\pgfqpoint{2.963410in}{0.569136in}}{\pgfqpoint{2.177280in}{2.201755in}}%
\pgfusepath{clip}%
\pgfsetbuttcap%
\pgfsetroundjoin%
\definecolor{currentfill}{rgb}{0.172549,0.627451,0.172549}%
\pgfsetfillcolor{currentfill}%
\pgfsetlinewidth{0.481800pt}%
\definecolor{currentstroke}{rgb}{1.000000,1.000000,1.000000}%
\pgfsetstrokecolor{currentstroke}%
\pgfsetdash{}{0pt}%
\pgfpathmoveto{\pgfqpoint{3.907452in}{2.295545in}}%
\pgfpathcurveto{\pgfqpoint{3.918502in}{2.295545in}}{\pgfqpoint{3.929101in}{2.299935in}}{\pgfqpoint{3.936915in}{2.307749in}}%
\pgfpathcurveto{\pgfqpoint{3.944728in}{2.315562in}}{\pgfqpoint{3.949118in}{2.326161in}}{\pgfqpoint{3.949118in}{2.337212in}}%
\pgfpathcurveto{\pgfqpoint{3.949118in}{2.348262in}}{\pgfqpoint{3.944728in}{2.358861in}}{\pgfqpoint{3.936915in}{2.366674in}}%
\pgfpathcurveto{\pgfqpoint{3.929101in}{2.374488in}}{\pgfqpoint{3.918502in}{2.378878in}}{\pgfqpoint{3.907452in}{2.378878in}}%
\pgfpathcurveto{\pgfqpoint{3.896402in}{2.378878in}}{\pgfqpoint{3.885803in}{2.374488in}}{\pgfqpoint{3.877989in}{2.366674in}}%
\pgfpathcurveto{\pgfqpoint{3.870175in}{2.358861in}}{\pgfqpoint{3.865785in}{2.348262in}}{\pgfqpoint{3.865785in}{2.337212in}}%
\pgfpathcurveto{\pgfqpoint{3.865785in}{2.326161in}}{\pgfqpoint{3.870175in}{2.315562in}}{\pgfqpoint{3.877989in}{2.307749in}}%
\pgfpathcurveto{\pgfqpoint{3.885803in}{2.299935in}}{\pgfqpoint{3.896402in}{2.295545in}}{\pgfqpoint{3.907452in}{2.295545in}}%
\pgfpathlineto{\pgfqpoint{3.907452in}{2.295545in}}%
\pgfpathclose%
\pgfusepath{stroke,fill}%
\end{pgfscope}%
\begin{pgfscope}%
\pgfpathrectangle{\pgfqpoint{2.963410in}{0.569136in}}{\pgfqpoint{2.177280in}{2.201755in}}%
\pgfusepath{clip}%
\pgfsetbuttcap%
\pgfsetroundjoin%
\definecolor{currentfill}{rgb}{0.172549,0.627451,0.172549}%
\pgfsetfillcolor{currentfill}%
\pgfsetlinewidth{0.481800pt}%
\definecolor{currentstroke}{rgb}{1.000000,1.000000,1.000000}%
\pgfsetstrokecolor{currentstroke}%
\pgfsetdash{}{0pt}%
\pgfpathmoveto{\pgfqpoint{3.612075in}{1.961946in}}%
\pgfpathcurveto{\pgfqpoint{3.623126in}{1.961946in}}{\pgfqpoint{3.633725in}{1.966336in}}{\pgfqpoint{3.641538in}{1.974150in}}%
\pgfpathcurveto{\pgfqpoint{3.649352in}{1.981963in}}{\pgfqpoint{3.653742in}{1.992562in}}{\pgfqpoint{3.653742in}{2.003612in}}%
\pgfpathcurveto{\pgfqpoint{3.653742in}{2.014662in}}{\pgfqpoint{3.649352in}{2.025262in}}{\pgfqpoint{3.641538in}{2.033075in}}%
\pgfpathcurveto{\pgfqpoint{3.633725in}{2.040889in}}{\pgfqpoint{3.623126in}{2.045279in}}{\pgfqpoint{3.612075in}{2.045279in}}%
\pgfpathcurveto{\pgfqpoint{3.601025in}{2.045279in}}{\pgfqpoint{3.590426in}{2.040889in}}{\pgfqpoint{3.582613in}{2.033075in}}%
\pgfpathcurveto{\pgfqpoint{3.574799in}{2.025262in}}{\pgfqpoint{3.570409in}{2.014662in}}{\pgfqpoint{3.570409in}{2.003612in}}%
\pgfpathcurveto{\pgfqpoint{3.570409in}{1.992562in}}{\pgfqpoint{3.574799in}{1.981963in}}{\pgfqpoint{3.582613in}{1.974150in}}%
\pgfpathcurveto{\pgfqpoint{3.590426in}{1.966336in}}{\pgfqpoint{3.601025in}{1.961946in}}{\pgfqpoint{3.612075in}{1.961946in}}%
\pgfpathlineto{\pgfqpoint{3.612075in}{1.961946in}}%
\pgfpathclose%
\pgfusepath{stroke,fill}%
\end{pgfscope}%
\begin{pgfscope}%
\pgfpathrectangle{\pgfqpoint{2.963410in}{0.569136in}}{\pgfqpoint{2.177280in}{2.201755in}}%
\pgfusepath{clip}%
\pgfsetbuttcap%
\pgfsetroundjoin%
\definecolor{currentfill}{rgb}{0.172549,0.627451,0.172549}%
\pgfsetfillcolor{currentfill}%
\pgfsetlinewidth{0.481800pt}%
\definecolor{currentstroke}{rgb}{1.000000,1.000000,1.000000}%
\pgfsetstrokecolor{currentstroke}%
\pgfsetdash{}{0pt}%
\pgfpathmoveto{\pgfqpoint{3.848376in}{2.045346in}}%
\pgfpathcurveto{\pgfqpoint{3.859427in}{2.045346in}}{\pgfqpoint{3.870026in}{2.049736in}}{\pgfqpoint{3.877839in}{2.057549in}}%
\pgfpathcurveto{\pgfqpoint{3.885653in}{2.065363in}}{\pgfqpoint{3.890043in}{2.075962in}}{\pgfqpoint{3.890043in}{2.087012in}}%
\pgfpathcurveto{\pgfqpoint{3.890043in}{2.098062in}}{\pgfqpoint{3.885653in}{2.108661in}}{\pgfqpoint{3.877839in}{2.116475in}}%
\pgfpathcurveto{\pgfqpoint{3.870026in}{2.124289in}}{\pgfqpoint{3.859427in}{2.128679in}}{\pgfqpoint{3.848376in}{2.128679in}}%
\pgfpathcurveto{\pgfqpoint{3.837326in}{2.128679in}}{\pgfqpoint{3.826727in}{2.124289in}}{\pgfqpoint{3.818914in}{2.116475in}}%
\pgfpathcurveto{\pgfqpoint{3.811100in}{2.108661in}}{\pgfqpoint{3.806710in}{2.098062in}}{\pgfqpoint{3.806710in}{2.087012in}}%
\pgfpathcurveto{\pgfqpoint{3.806710in}{2.075962in}}{\pgfqpoint{3.811100in}{2.065363in}}{\pgfqpoint{3.818914in}{2.057549in}}%
\pgfpathcurveto{\pgfqpoint{3.826727in}{2.049736in}}{\pgfqpoint{3.837326in}{2.045346in}}{\pgfqpoint{3.848376in}{2.045346in}}%
\pgfpathlineto{\pgfqpoint{3.848376in}{2.045346in}}%
\pgfpathclose%
\pgfusepath{stroke,fill}%
\end{pgfscope}%
\begin{pgfscope}%
\pgfpathrectangle{\pgfqpoint{2.963410in}{0.569136in}}{\pgfqpoint{2.177280in}{2.201755in}}%
\pgfusepath{clip}%
\pgfsetbuttcap%
\pgfsetroundjoin%
\definecolor{currentfill}{rgb}{0.172549,0.627451,0.172549}%
\pgfsetfillcolor{currentfill}%
\pgfsetlinewidth{0.481800pt}%
\definecolor{currentstroke}{rgb}{1.000000,1.000000,1.000000}%
\pgfsetstrokecolor{currentstroke}%
\pgfsetdash{}{0pt}%
\pgfpathmoveto{\pgfqpoint{3.612075in}{2.045346in}}%
\pgfpathcurveto{\pgfqpoint{3.623126in}{2.045346in}}{\pgfqpoint{3.633725in}{2.049736in}}{\pgfqpoint{3.641538in}{2.057549in}}%
\pgfpathcurveto{\pgfqpoint{3.649352in}{2.065363in}}{\pgfqpoint{3.653742in}{2.075962in}}{\pgfqpoint{3.653742in}{2.087012in}}%
\pgfpathcurveto{\pgfqpoint{3.653742in}{2.098062in}}{\pgfqpoint{3.649352in}{2.108661in}}{\pgfqpoint{3.641538in}{2.116475in}}%
\pgfpathcurveto{\pgfqpoint{3.633725in}{2.124289in}}{\pgfqpoint{3.623126in}{2.128679in}}{\pgfqpoint{3.612075in}{2.128679in}}%
\pgfpathcurveto{\pgfqpoint{3.601025in}{2.128679in}}{\pgfqpoint{3.590426in}{2.124289in}}{\pgfqpoint{3.582613in}{2.116475in}}%
\pgfpathcurveto{\pgfqpoint{3.574799in}{2.108661in}}{\pgfqpoint{3.570409in}{2.098062in}}{\pgfqpoint{3.570409in}{2.087012in}}%
\pgfpathcurveto{\pgfqpoint{3.570409in}{2.075962in}}{\pgfqpoint{3.574799in}{2.065363in}}{\pgfqpoint{3.582613in}{2.057549in}}%
\pgfpathcurveto{\pgfqpoint{3.590426in}{2.049736in}}{\pgfqpoint{3.601025in}{2.045346in}}{\pgfqpoint{3.612075in}{2.045346in}}%
\pgfpathlineto{\pgfqpoint{3.612075in}{2.045346in}}%
\pgfpathclose%
\pgfusepath{stroke,fill}%
\end{pgfscope}%
\begin{pgfscope}%
\pgfpathrectangle{\pgfqpoint{2.963410in}{0.569136in}}{\pgfqpoint{2.177280in}{2.201755in}}%
\pgfusepath{clip}%
\pgfsetbuttcap%
\pgfsetroundjoin%
\definecolor{currentfill}{rgb}{0.172549,0.627451,0.172549}%
\pgfsetfillcolor{currentfill}%
\pgfsetlinewidth{0.481800pt}%
\definecolor{currentstroke}{rgb}{1.000000,1.000000,1.000000}%
\pgfsetstrokecolor{currentstroke}%
\pgfsetdash{}{0pt}%
\pgfpathmoveto{\pgfqpoint{4.261903in}{2.629144in}}%
\pgfpathcurveto{\pgfqpoint{4.272953in}{2.629144in}}{\pgfqpoint{4.283552in}{2.633534in}}{\pgfqpoint{4.291366in}{2.641348in}}%
\pgfpathcurveto{\pgfqpoint{4.299180in}{2.649162in}}{\pgfqpoint{4.303570in}{2.659761in}}{\pgfqpoint{4.303570in}{2.670811in}}%
\pgfpathcurveto{\pgfqpoint{4.303570in}{2.681861in}}{\pgfqpoint{4.299180in}{2.692460in}}{\pgfqpoint{4.291366in}{2.700274in}}%
\pgfpathcurveto{\pgfqpoint{4.283552in}{2.708087in}}{\pgfqpoint{4.272953in}{2.712478in}}{\pgfqpoint{4.261903in}{2.712478in}}%
\pgfpathcurveto{\pgfqpoint{4.250853in}{2.712478in}}{\pgfqpoint{4.240254in}{2.708087in}}{\pgfqpoint{4.232441in}{2.700274in}}%
\pgfpathcurveto{\pgfqpoint{4.224627in}{2.692460in}}{\pgfqpoint{4.220237in}{2.681861in}}{\pgfqpoint{4.220237in}{2.670811in}}%
\pgfpathcurveto{\pgfqpoint{4.220237in}{2.659761in}}{\pgfqpoint{4.224627in}{2.649162in}}{\pgfqpoint{4.232441in}{2.641348in}}%
\pgfpathcurveto{\pgfqpoint{4.240254in}{2.633534in}}{\pgfqpoint{4.250853in}{2.629144in}}{\pgfqpoint{4.261903in}{2.629144in}}%
\pgfpathlineto{\pgfqpoint{4.261903in}{2.629144in}}%
\pgfpathclose%
\pgfusepath{stroke,fill}%
\end{pgfscope}%
\begin{pgfscope}%
\pgfpathrectangle{\pgfqpoint{2.963410in}{0.569136in}}{\pgfqpoint{2.177280in}{2.201755in}}%
\pgfusepath{clip}%
\pgfsetbuttcap%
\pgfsetroundjoin%
\definecolor{currentfill}{rgb}{0.172549,0.627451,0.172549}%
\pgfsetfillcolor{currentfill}%
\pgfsetlinewidth{0.481800pt}%
\definecolor{currentstroke}{rgb}{1.000000,1.000000,1.000000}%
\pgfsetstrokecolor{currentstroke}%
\pgfsetdash{}{0pt}%
\pgfpathmoveto{\pgfqpoint{4.025602in}{2.212145in}}%
\pgfpathcurveto{\pgfqpoint{4.036652in}{2.212145in}}{\pgfqpoint{4.047251in}{2.216535in}}{\pgfqpoint{4.055065in}{2.224349in}}%
\pgfpathcurveto{\pgfqpoint{4.062879in}{2.232163in}}{\pgfqpoint{4.067269in}{2.242762in}}{\pgfqpoint{4.067269in}{2.253812in}}%
\pgfpathcurveto{\pgfqpoint{4.067269in}{2.264862in}}{\pgfqpoint{4.062879in}{2.275461in}}{\pgfqpoint{4.055065in}{2.283275in}}%
\pgfpathcurveto{\pgfqpoint{4.047251in}{2.291088in}}{\pgfqpoint{4.036652in}{2.295478in}}{\pgfqpoint{4.025602in}{2.295478in}}%
\pgfpathcurveto{\pgfqpoint{4.014552in}{2.295478in}}{\pgfqpoint{4.003953in}{2.291088in}}{\pgfqpoint{3.996139in}{2.283275in}}%
\pgfpathcurveto{\pgfqpoint{3.988326in}{2.275461in}}{\pgfqpoint{3.983936in}{2.264862in}}{\pgfqpoint{3.983936in}{2.253812in}}%
\pgfpathcurveto{\pgfqpoint{3.983936in}{2.242762in}}{\pgfqpoint{3.988326in}{2.232163in}}{\pgfqpoint{3.996139in}{2.224349in}}%
\pgfpathcurveto{\pgfqpoint{4.003953in}{2.216535in}}{\pgfqpoint{4.014552in}{2.212145in}}{\pgfqpoint{4.025602in}{2.212145in}}%
\pgfpathlineto{\pgfqpoint{4.025602in}{2.212145in}}%
\pgfpathclose%
\pgfusepath{stroke,fill}%
\end{pgfscope}%
\begin{pgfscope}%
\pgfpathrectangle{\pgfqpoint{2.963410in}{0.569136in}}{\pgfqpoint{2.177280in}{2.201755in}}%
\pgfusepath{clip}%
\pgfsetbuttcap%
\pgfsetroundjoin%
\definecolor{currentfill}{rgb}{0.172549,0.627451,0.172549}%
\pgfsetfillcolor{currentfill}%
\pgfsetlinewidth{0.481800pt}%
\definecolor{currentstroke}{rgb}{1.000000,1.000000,1.000000}%
\pgfsetstrokecolor{currentstroke}%
\pgfsetdash{}{0pt}%
\pgfpathmoveto{\pgfqpoint{3.730226in}{2.128745in}}%
\pgfpathcurveto{\pgfqpoint{3.741276in}{2.128745in}}{\pgfqpoint{3.751875in}{2.133136in}}{\pgfqpoint{3.759689in}{2.140949in}}%
\pgfpathcurveto{\pgfqpoint{3.767502in}{2.148763in}}{\pgfqpoint{3.771893in}{2.159362in}}{\pgfqpoint{3.771893in}{2.170412in}}%
\pgfpathcurveto{\pgfqpoint{3.771893in}{2.181462in}}{\pgfqpoint{3.767502in}{2.192061in}}{\pgfqpoint{3.759689in}{2.199875in}}%
\pgfpathcurveto{\pgfqpoint{3.751875in}{2.207688in}}{\pgfqpoint{3.741276in}{2.212079in}}{\pgfqpoint{3.730226in}{2.212079in}}%
\pgfpathcurveto{\pgfqpoint{3.719176in}{2.212079in}}{\pgfqpoint{3.708577in}{2.207688in}}{\pgfqpoint{3.700763in}{2.199875in}}%
\pgfpathcurveto{\pgfqpoint{3.692950in}{2.192061in}}{\pgfqpoint{3.688559in}{2.181462in}}{\pgfqpoint{3.688559in}{2.170412in}}%
\pgfpathcurveto{\pgfqpoint{3.688559in}{2.159362in}}{\pgfqpoint{3.692950in}{2.148763in}}{\pgfqpoint{3.700763in}{2.140949in}}%
\pgfpathcurveto{\pgfqpoint{3.708577in}{2.133136in}}{\pgfqpoint{3.719176in}{2.128745in}}{\pgfqpoint{3.730226in}{2.128745in}}%
\pgfpathlineto{\pgfqpoint{3.730226in}{2.128745in}}%
\pgfpathclose%
\pgfusepath{stroke,fill}%
\end{pgfscope}%
\begin{pgfscope}%
\pgfpathrectangle{\pgfqpoint{2.963410in}{0.569136in}}{\pgfqpoint{2.177280in}{2.201755in}}%
\pgfusepath{clip}%
\pgfsetbuttcap%
\pgfsetroundjoin%
\definecolor{currentfill}{rgb}{0.172549,0.627451,0.172549}%
\pgfsetfillcolor{currentfill}%
\pgfsetlinewidth{0.481800pt}%
\definecolor{currentstroke}{rgb}{1.000000,1.000000,1.000000}%
\pgfsetstrokecolor{currentstroke}%
\pgfsetdash{}{0pt}%
\pgfpathmoveto{\pgfqpoint{3.907452in}{2.295545in}}%
\pgfpathcurveto{\pgfqpoint{3.918502in}{2.295545in}}{\pgfqpoint{3.929101in}{2.299935in}}{\pgfqpoint{3.936915in}{2.307749in}}%
\pgfpathcurveto{\pgfqpoint{3.944728in}{2.315562in}}{\pgfqpoint{3.949118in}{2.326161in}}{\pgfqpoint{3.949118in}{2.337212in}}%
\pgfpathcurveto{\pgfqpoint{3.949118in}{2.348262in}}{\pgfqpoint{3.944728in}{2.358861in}}{\pgfqpoint{3.936915in}{2.366674in}}%
\pgfpathcurveto{\pgfqpoint{3.929101in}{2.374488in}}{\pgfqpoint{3.918502in}{2.378878in}}{\pgfqpoint{3.907452in}{2.378878in}}%
\pgfpathcurveto{\pgfqpoint{3.896402in}{2.378878in}}{\pgfqpoint{3.885803in}{2.374488in}}{\pgfqpoint{3.877989in}{2.366674in}}%
\pgfpathcurveto{\pgfqpoint{3.870175in}{2.358861in}}{\pgfqpoint{3.865785in}{2.348262in}}{\pgfqpoint{3.865785in}{2.337212in}}%
\pgfpathcurveto{\pgfqpoint{3.865785in}{2.326161in}}{\pgfqpoint{3.870175in}{2.315562in}}{\pgfqpoint{3.877989in}{2.307749in}}%
\pgfpathcurveto{\pgfqpoint{3.885803in}{2.299935in}}{\pgfqpoint{3.896402in}{2.295545in}}{\pgfqpoint{3.907452in}{2.295545in}}%
\pgfpathlineto{\pgfqpoint{3.907452in}{2.295545in}}%
\pgfpathclose%
\pgfusepath{stroke,fill}%
\end{pgfscope}%
\begin{pgfscope}%
\pgfpathrectangle{\pgfqpoint{2.963410in}{0.569136in}}{\pgfqpoint{2.177280in}{2.201755in}}%
\pgfusepath{clip}%
\pgfsetbuttcap%
\pgfsetroundjoin%
\definecolor{currentfill}{rgb}{0.172549,0.627451,0.172549}%
\pgfsetfillcolor{currentfill}%
\pgfsetlinewidth{0.481800pt}%
\definecolor{currentstroke}{rgb}{1.000000,1.000000,1.000000}%
\pgfsetstrokecolor{currentstroke}%
\pgfsetdash{}{0pt}%
\pgfpathmoveto{\pgfqpoint{3.612075in}{2.212145in}}%
\pgfpathcurveto{\pgfqpoint{3.623126in}{2.212145in}}{\pgfqpoint{3.633725in}{2.216535in}}{\pgfqpoint{3.641538in}{2.224349in}}%
\pgfpathcurveto{\pgfqpoint{3.649352in}{2.232163in}}{\pgfqpoint{3.653742in}{2.242762in}}{\pgfqpoint{3.653742in}{2.253812in}}%
\pgfpathcurveto{\pgfqpoint{3.653742in}{2.264862in}}{\pgfqpoint{3.649352in}{2.275461in}}{\pgfqpoint{3.641538in}{2.283275in}}%
\pgfpathcurveto{\pgfqpoint{3.633725in}{2.291088in}}{\pgfqpoint{3.623126in}{2.295478in}}{\pgfqpoint{3.612075in}{2.295478in}}%
\pgfpathcurveto{\pgfqpoint{3.601025in}{2.295478in}}{\pgfqpoint{3.590426in}{2.291088in}}{\pgfqpoint{3.582613in}{2.283275in}}%
\pgfpathcurveto{\pgfqpoint{3.574799in}{2.275461in}}{\pgfqpoint{3.570409in}{2.264862in}}{\pgfqpoint{3.570409in}{2.253812in}}%
\pgfpathcurveto{\pgfqpoint{3.570409in}{2.242762in}}{\pgfqpoint{3.574799in}{2.232163in}}{\pgfqpoint{3.582613in}{2.224349in}}%
\pgfpathcurveto{\pgfqpoint{3.590426in}{2.216535in}}{\pgfqpoint{3.601025in}{2.212145in}}{\pgfqpoint{3.612075in}{2.212145in}}%
\pgfpathlineto{\pgfqpoint{3.612075in}{2.212145in}}%
\pgfpathclose%
\pgfusepath{stroke,fill}%
\end{pgfscope}%
\begin{pgfscope}%
\pgfpathrectangle{\pgfqpoint{2.963410in}{0.569136in}}{\pgfqpoint{2.177280in}{2.201755in}}%
\pgfusepath{clip}%
\pgfsetbuttcap%
\pgfsetroundjoin%
\definecolor{currentfill}{rgb}{0.172549,0.627451,0.172549}%
\pgfsetfillcolor{currentfill}%
\pgfsetlinewidth{0.481800pt}%
\definecolor{currentstroke}{rgb}{1.000000,1.000000,1.000000}%
\pgfsetstrokecolor{currentstroke}%
\pgfsetdash{}{0pt}%
\pgfpathmoveto{\pgfqpoint{3.789301in}{2.545744in}}%
\pgfpathcurveto{\pgfqpoint{3.800351in}{2.545744in}}{\pgfqpoint{3.810950in}{2.550135in}}{\pgfqpoint{3.818764in}{2.557948in}}%
\pgfpathcurveto{\pgfqpoint{3.826578in}{2.565762in}}{\pgfqpoint{3.830968in}{2.576361in}}{\pgfqpoint{3.830968in}{2.587411in}}%
\pgfpathcurveto{\pgfqpoint{3.830968in}{2.598461in}}{\pgfqpoint{3.826578in}{2.609060in}}{\pgfqpoint{3.818764in}{2.616874in}}%
\pgfpathcurveto{\pgfqpoint{3.810950in}{2.624687in}}{\pgfqpoint{3.800351in}{2.629078in}}{\pgfqpoint{3.789301in}{2.629078in}}%
\pgfpathcurveto{\pgfqpoint{3.778251in}{2.629078in}}{\pgfqpoint{3.767652in}{2.624687in}}{\pgfqpoint{3.759838in}{2.616874in}}%
\pgfpathcurveto{\pgfqpoint{3.752025in}{2.609060in}}{\pgfqpoint{3.747635in}{2.598461in}}{\pgfqpoint{3.747635in}{2.587411in}}%
\pgfpathcurveto{\pgfqpoint{3.747635in}{2.576361in}}{\pgfqpoint{3.752025in}{2.565762in}}{\pgfqpoint{3.759838in}{2.557948in}}%
\pgfpathcurveto{\pgfqpoint{3.767652in}{2.550135in}}{\pgfqpoint{3.778251in}{2.545744in}}{\pgfqpoint{3.789301in}{2.545744in}}%
\pgfpathlineto{\pgfqpoint{3.789301in}{2.545744in}}%
\pgfpathclose%
\pgfusepath{stroke,fill}%
\end{pgfscope}%
\begin{pgfscope}%
\pgfpathrectangle{\pgfqpoint{2.963410in}{0.569136in}}{\pgfqpoint{2.177280in}{2.201755in}}%
\pgfusepath{clip}%
\pgfsetbuttcap%
\pgfsetroundjoin%
\definecolor{currentfill}{rgb}{0.172549,0.627451,0.172549}%
\pgfsetfillcolor{currentfill}%
\pgfsetlinewidth{0.481800pt}%
\definecolor{currentstroke}{rgb}{1.000000,1.000000,1.000000}%
\pgfsetstrokecolor{currentstroke}%
\pgfsetdash{}{0pt}%
\pgfpathmoveto{\pgfqpoint{4.025602in}{2.462345in}}%
\pgfpathcurveto{\pgfqpoint{4.036652in}{2.462345in}}{\pgfqpoint{4.047251in}{2.466735in}}{\pgfqpoint{4.055065in}{2.474548in}}%
\pgfpathcurveto{\pgfqpoint{4.062879in}{2.482362in}}{\pgfqpoint{4.067269in}{2.492961in}}{\pgfqpoint{4.067269in}{2.504011in}}%
\pgfpathcurveto{\pgfqpoint{4.067269in}{2.515061in}}{\pgfqpoint{4.062879in}{2.525660in}}{\pgfqpoint{4.055065in}{2.533474in}}%
\pgfpathcurveto{\pgfqpoint{4.047251in}{2.541288in}}{\pgfqpoint{4.036652in}{2.545678in}}{\pgfqpoint{4.025602in}{2.545678in}}%
\pgfpathcurveto{\pgfqpoint{4.014552in}{2.545678in}}{\pgfqpoint{4.003953in}{2.541288in}}{\pgfqpoint{3.996139in}{2.533474in}}%
\pgfpathcurveto{\pgfqpoint{3.988326in}{2.525660in}}{\pgfqpoint{3.983936in}{2.515061in}}{\pgfqpoint{3.983936in}{2.504011in}}%
\pgfpathcurveto{\pgfqpoint{3.983936in}{2.492961in}}{\pgfqpoint{3.988326in}{2.482362in}}{\pgfqpoint{3.996139in}{2.474548in}}%
\pgfpathcurveto{\pgfqpoint{4.003953in}{2.466735in}}{\pgfqpoint{4.014552in}{2.462345in}}{\pgfqpoint{4.025602in}{2.462345in}}%
\pgfpathlineto{\pgfqpoint{4.025602in}{2.462345in}}%
\pgfpathclose%
\pgfusepath{stroke,fill}%
\end{pgfscope}%
\begin{pgfscope}%
\pgfpathrectangle{\pgfqpoint{2.963410in}{0.569136in}}{\pgfqpoint{2.177280in}{2.201755in}}%
\pgfusepath{clip}%
\pgfsetbuttcap%
\pgfsetroundjoin%
\definecolor{currentfill}{rgb}{0.172549,0.627451,0.172549}%
\pgfsetfillcolor{currentfill}%
\pgfsetlinewidth{0.481800pt}%
\definecolor{currentstroke}{rgb}{1.000000,1.000000,1.000000}%
\pgfsetstrokecolor{currentstroke}%
\pgfsetdash{}{0pt}%
\pgfpathmoveto{\pgfqpoint{3.907452in}{2.045346in}}%
\pgfpathcurveto{\pgfqpoint{3.918502in}{2.045346in}}{\pgfqpoint{3.929101in}{2.049736in}}{\pgfqpoint{3.936915in}{2.057549in}}%
\pgfpathcurveto{\pgfqpoint{3.944728in}{2.065363in}}{\pgfqpoint{3.949118in}{2.075962in}}{\pgfqpoint{3.949118in}{2.087012in}}%
\pgfpathcurveto{\pgfqpoint{3.949118in}{2.098062in}}{\pgfqpoint{3.944728in}{2.108661in}}{\pgfqpoint{3.936915in}{2.116475in}}%
\pgfpathcurveto{\pgfqpoint{3.929101in}{2.124289in}}{\pgfqpoint{3.918502in}{2.128679in}}{\pgfqpoint{3.907452in}{2.128679in}}%
\pgfpathcurveto{\pgfqpoint{3.896402in}{2.128679in}}{\pgfqpoint{3.885803in}{2.124289in}}{\pgfqpoint{3.877989in}{2.116475in}}%
\pgfpathcurveto{\pgfqpoint{3.870175in}{2.108661in}}{\pgfqpoint{3.865785in}{2.098062in}}{\pgfqpoint{3.865785in}{2.087012in}}%
\pgfpathcurveto{\pgfqpoint{3.865785in}{2.075962in}}{\pgfqpoint{3.870175in}{2.065363in}}{\pgfqpoint{3.877989in}{2.057549in}}%
\pgfpathcurveto{\pgfqpoint{3.885803in}{2.049736in}}{\pgfqpoint{3.896402in}{2.045346in}}{\pgfqpoint{3.907452in}{2.045346in}}%
\pgfpathlineto{\pgfqpoint{3.907452in}{2.045346in}}%
\pgfpathclose%
\pgfusepath{stroke,fill}%
\end{pgfscope}%
\begin{pgfscope}%
\pgfpathrectangle{\pgfqpoint{2.963410in}{0.569136in}}{\pgfqpoint{2.177280in}{2.201755in}}%
\pgfusepath{clip}%
\pgfsetbuttcap%
\pgfsetroundjoin%
\definecolor{currentfill}{rgb}{0.172549,0.627451,0.172549}%
\pgfsetfillcolor{currentfill}%
\pgfsetlinewidth{0.481800pt}%
\definecolor{currentstroke}{rgb}{1.000000,1.000000,1.000000}%
\pgfsetstrokecolor{currentstroke}%
\pgfsetdash{}{0pt}%
\pgfpathmoveto{\pgfqpoint{4.380054in}{2.378945in}}%
\pgfpathcurveto{\pgfqpoint{4.391104in}{2.378945in}}{\pgfqpoint{4.401703in}{2.383335in}}{\pgfqpoint{4.409517in}{2.391149in}}%
\pgfpathcurveto{\pgfqpoint{4.417330in}{2.398962in}}{\pgfqpoint{4.421721in}{2.409561in}}{\pgfqpoint{4.421721in}{2.420611in}}%
\pgfpathcurveto{\pgfqpoint{4.421721in}{2.431662in}}{\pgfqpoint{4.417330in}{2.442261in}}{\pgfqpoint{4.409517in}{2.450074in}}%
\pgfpathcurveto{\pgfqpoint{4.401703in}{2.457888in}}{\pgfqpoint{4.391104in}{2.462278in}}{\pgfqpoint{4.380054in}{2.462278in}}%
\pgfpathcurveto{\pgfqpoint{4.369004in}{2.462278in}}{\pgfqpoint{4.358405in}{2.457888in}}{\pgfqpoint{4.350591in}{2.450074in}}%
\pgfpathcurveto{\pgfqpoint{4.342777in}{2.442261in}}{\pgfqpoint{4.338387in}{2.431662in}}{\pgfqpoint{4.338387in}{2.420611in}}%
\pgfpathcurveto{\pgfqpoint{4.338387in}{2.409561in}}{\pgfqpoint{4.342777in}{2.398962in}}{\pgfqpoint{4.350591in}{2.391149in}}%
\pgfpathcurveto{\pgfqpoint{4.358405in}{2.383335in}}{\pgfqpoint{4.369004in}{2.378945in}}{\pgfqpoint{4.380054in}{2.378945in}}%
\pgfpathlineto{\pgfqpoint{4.380054in}{2.378945in}}%
\pgfpathclose%
\pgfusepath{stroke,fill}%
\end{pgfscope}%
\begin{pgfscope}%
\pgfpathrectangle{\pgfqpoint{2.963410in}{0.569136in}}{\pgfqpoint{2.177280in}{2.201755in}}%
\pgfusepath{clip}%
\pgfsetbuttcap%
\pgfsetroundjoin%
\definecolor{currentfill}{rgb}{0.172549,0.627451,0.172549}%
\pgfsetfillcolor{currentfill}%
\pgfsetlinewidth{0.481800pt}%
\definecolor{currentstroke}{rgb}{1.000000,1.000000,1.000000}%
\pgfsetstrokecolor{currentstroke}%
\pgfsetdash{}{0pt}%
\pgfpathmoveto{\pgfqpoint{3.671151in}{2.462345in}}%
\pgfpathcurveto{\pgfqpoint{3.682201in}{2.462345in}}{\pgfqpoint{3.692800in}{2.466735in}}{\pgfqpoint{3.700613in}{2.474548in}}%
\pgfpathcurveto{\pgfqpoint{3.708427in}{2.482362in}}{\pgfqpoint{3.712817in}{2.492961in}}{\pgfqpoint{3.712817in}{2.504011in}}%
\pgfpathcurveto{\pgfqpoint{3.712817in}{2.515061in}}{\pgfqpoint{3.708427in}{2.525660in}}{\pgfqpoint{3.700613in}{2.533474in}}%
\pgfpathcurveto{\pgfqpoint{3.692800in}{2.541288in}}{\pgfqpoint{3.682201in}{2.545678in}}{\pgfqpoint{3.671151in}{2.545678in}}%
\pgfpathcurveto{\pgfqpoint{3.660101in}{2.545678in}}{\pgfqpoint{3.649501in}{2.541288in}}{\pgfqpoint{3.641688in}{2.533474in}}%
\pgfpathcurveto{\pgfqpoint{3.633874in}{2.525660in}}{\pgfqpoint{3.629484in}{2.515061in}}{\pgfqpoint{3.629484in}{2.504011in}}%
\pgfpathcurveto{\pgfqpoint{3.629484in}{2.492961in}}{\pgfqpoint{3.633874in}{2.482362in}}{\pgfqpoint{3.641688in}{2.474548in}}%
\pgfpathcurveto{\pgfqpoint{3.649501in}{2.466735in}}{\pgfqpoint{3.660101in}{2.462345in}}{\pgfqpoint{3.671151in}{2.462345in}}%
\pgfpathlineto{\pgfqpoint{3.671151in}{2.462345in}}%
\pgfpathclose%
\pgfusepath{stroke,fill}%
\end{pgfscope}%
\begin{pgfscope}%
\pgfpathrectangle{\pgfqpoint{2.963410in}{0.569136in}}{\pgfqpoint{2.177280in}{2.201755in}}%
\pgfusepath{clip}%
\pgfsetbuttcap%
\pgfsetroundjoin%
\definecolor{currentfill}{rgb}{0.172549,0.627451,0.172549}%
\pgfsetfillcolor{currentfill}%
\pgfsetlinewidth{0.481800pt}%
\definecolor{currentstroke}{rgb}{1.000000,1.000000,1.000000}%
\pgfsetstrokecolor{currentstroke}%
\pgfsetdash{}{0pt}%
\pgfpathmoveto{\pgfqpoint{3.434850in}{1.795146in}}%
\pgfpathcurveto{\pgfqpoint{3.445900in}{1.795146in}}{\pgfqpoint{3.456499in}{1.799536in}}{\pgfqpoint{3.464312in}{1.807350in}}%
\pgfpathcurveto{\pgfqpoint{3.472126in}{1.815164in}}{\pgfqpoint{3.476516in}{1.825763in}}{\pgfqpoint{3.476516in}{1.836813in}}%
\pgfpathcurveto{\pgfqpoint{3.476516in}{1.847863in}}{\pgfqpoint{3.472126in}{1.858462in}}{\pgfqpoint{3.464312in}{1.866276in}}%
\pgfpathcurveto{\pgfqpoint{3.456499in}{1.874089in}}{\pgfqpoint{3.445900in}{1.878479in}}{\pgfqpoint{3.434850in}{1.878479in}}%
\pgfpathcurveto{\pgfqpoint{3.423799in}{1.878479in}}{\pgfqpoint{3.413200in}{1.874089in}}{\pgfqpoint{3.405387in}{1.866276in}}%
\pgfpathcurveto{\pgfqpoint{3.397573in}{1.858462in}}{\pgfqpoint{3.393183in}{1.847863in}}{\pgfqpoint{3.393183in}{1.836813in}}%
\pgfpathcurveto{\pgfqpoint{3.393183in}{1.825763in}}{\pgfqpoint{3.397573in}{1.815164in}}{\pgfqpoint{3.405387in}{1.807350in}}%
\pgfpathcurveto{\pgfqpoint{3.413200in}{1.799536in}}{\pgfqpoint{3.423799in}{1.795146in}}{\pgfqpoint{3.434850in}{1.795146in}}%
\pgfpathlineto{\pgfqpoint{3.434850in}{1.795146in}}%
\pgfpathclose%
\pgfusepath{stroke,fill}%
\end{pgfscope}%
\begin{pgfscope}%
\pgfpathrectangle{\pgfqpoint{2.963410in}{0.569136in}}{\pgfqpoint{2.177280in}{2.201755in}}%
\pgfusepath{clip}%
\pgfsetbuttcap%
\pgfsetroundjoin%
\definecolor{currentfill}{rgb}{0.172549,0.627451,0.172549}%
\pgfsetfillcolor{currentfill}%
\pgfsetlinewidth{0.481800pt}%
\definecolor{currentstroke}{rgb}{1.000000,1.000000,1.000000}%
\pgfsetstrokecolor{currentstroke}%
\pgfsetdash{}{0pt}%
\pgfpathmoveto{\pgfqpoint{4.025602in}{2.462345in}}%
\pgfpathcurveto{\pgfqpoint{4.036652in}{2.462345in}}{\pgfqpoint{4.047251in}{2.466735in}}{\pgfqpoint{4.055065in}{2.474548in}}%
\pgfpathcurveto{\pgfqpoint{4.062879in}{2.482362in}}{\pgfqpoint{4.067269in}{2.492961in}}{\pgfqpoint{4.067269in}{2.504011in}}%
\pgfpathcurveto{\pgfqpoint{4.067269in}{2.515061in}}{\pgfqpoint{4.062879in}{2.525660in}}{\pgfqpoint{4.055065in}{2.533474in}}%
\pgfpathcurveto{\pgfqpoint{4.047251in}{2.541288in}}{\pgfqpoint{4.036652in}{2.545678in}}{\pgfqpoint{4.025602in}{2.545678in}}%
\pgfpathcurveto{\pgfqpoint{4.014552in}{2.545678in}}{\pgfqpoint{4.003953in}{2.541288in}}{\pgfqpoint{3.996139in}{2.533474in}}%
\pgfpathcurveto{\pgfqpoint{3.988326in}{2.525660in}}{\pgfqpoint{3.983936in}{2.515061in}}{\pgfqpoint{3.983936in}{2.504011in}}%
\pgfpathcurveto{\pgfqpoint{3.983936in}{2.492961in}}{\pgfqpoint{3.988326in}{2.482362in}}{\pgfqpoint{3.996139in}{2.474548in}}%
\pgfpathcurveto{\pgfqpoint{4.003953in}{2.466735in}}{\pgfqpoint{4.014552in}{2.462345in}}{\pgfqpoint{4.025602in}{2.462345in}}%
\pgfpathlineto{\pgfqpoint{4.025602in}{2.462345in}}%
\pgfpathclose%
\pgfusepath{stroke,fill}%
\end{pgfscope}%
\begin{pgfscope}%
\pgfpathrectangle{\pgfqpoint{2.963410in}{0.569136in}}{\pgfqpoint{2.177280in}{2.201755in}}%
\pgfusepath{clip}%
\pgfsetbuttcap%
\pgfsetroundjoin%
\definecolor{currentfill}{rgb}{0.172549,0.627451,0.172549}%
\pgfsetfillcolor{currentfill}%
\pgfsetlinewidth{0.481800pt}%
\definecolor{currentstroke}{rgb}{1.000000,1.000000,1.000000}%
\pgfsetstrokecolor{currentstroke}%
\pgfsetdash{}{0pt}%
\pgfpathmoveto{\pgfqpoint{3.789301in}{2.212145in}}%
\pgfpathcurveto{\pgfqpoint{3.800351in}{2.212145in}}{\pgfqpoint{3.810950in}{2.216535in}}{\pgfqpoint{3.818764in}{2.224349in}}%
\pgfpathcurveto{\pgfqpoint{3.826578in}{2.232163in}}{\pgfqpoint{3.830968in}{2.242762in}}{\pgfqpoint{3.830968in}{2.253812in}}%
\pgfpathcurveto{\pgfqpoint{3.830968in}{2.264862in}}{\pgfqpoint{3.826578in}{2.275461in}}{\pgfqpoint{3.818764in}{2.283275in}}%
\pgfpathcurveto{\pgfqpoint{3.810950in}{2.291088in}}{\pgfqpoint{3.800351in}{2.295478in}}{\pgfqpoint{3.789301in}{2.295478in}}%
\pgfpathcurveto{\pgfqpoint{3.778251in}{2.295478in}}{\pgfqpoint{3.767652in}{2.291088in}}{\pgfqpoint{3.759838in}{2.283275in}}%
\pgfpathcurveto{\pgfqpoint{3.752025in}{2.275461in}}{\pgfqpoint{3.747635in}{2.264862in}}{\pgfqpoint{3.747635in}{2.253812in}}%
\pgfpathcurveto{\pgfqpoint{3.747635in}{2.242762in}}{\pgfqpoint{3.752025in}{2.232163in}}{\pgfqpoint{3.759838in}{2.224349in}}%
\pgfpathcurveto{\pgfqpoint{3.767652in}{2.216535in}}{\pgfqpoint{3.778251in}{2.212145in}}{\pgfqpoint{3.789301in}{2.212145in}}%
\pgfpathlineto{\pgfqpoint{3.789301in}{2.212145in}}%
\pgfpathclose%
\pgfusepath{stroke,fill}%
\end{pgfscope}%
\begin{pgfscope}%
\pgfpathrectangle{\pgfqpoint{2.963410in}{0.569136in}}{\pgfqpoint{2.177280in}{2.201755in}}%
\pgfusepath{clip}%
\pgfsetbuttcap%
\pgfsetroundjoin%
\definecolor{currentfill}{rgb}{0.172549,0.627451,0.172549}%
\pgfsetfillcolor{currentfill}%
\pgfsetlinewidth{0.481800pt}%
\definecolor{currentstroke}{rgb}{1.000000,1.000000,1.000000}%
\pgfsetstrokecolor{currentstroke}%
\pgfsetdash{}{0pt}%
\pgfpathmoveto{\pgfqpoint{3.789301in}{2.212145in}}%
\pgfpathcurveto{\pgfqpoint{3.800351in}{2.212145in}}{\pgfqpoint{3.810950in}{2.216535in}}{\pgfqpoint{3.818764in}{2.224349in}}%
\pgfpathcurveto{\pgfqpoint{3.826578in}{2.232163in}}{\pgfqpoint{3.830968in}{2.242762in}}{\pgfqpoint{3.830968in}{2.253812in}}%
\pgfpathcurveto{\pgfqpoint{3.830968in}{2.264862in}}{\pgfqpoint{3.826578in}{2.275461in}}{\pgfqpoint{3.818764in}{2.283275in}}%
\pgfpathcurveto{\pgfqpoint{3.810950in}{2.291088in}}{\pgfqpoint{3.800351in}{2.295478in}}{\pgfqpoint{3.789301in}{2.295478in}}%
\pgfpathcurveto{\pgfqpoint{3.778251in}{2.295478in}}{\pgfqpoint{3.767652in}{2.291088in}}{\pgfqpoint{3.759838in}{2.283275in}}%
\pgfpathcurveto{\pgfqpoint{3.752025in}{2.275461in}}{\pgfqpoint{3.747635in}{2.264862in}}{\pgfqpoint{3.747635in}{2.253812in}}%
\pgfpathcurveto{\pgfqpoint{3.747635in}{2.242762in}}{\pgfqpoint{3.752025in}{2.232163in}}{\pgfqpoint{3.759838in}{2.224349in}}%
\pgfpathcurveto{\pgfqpoint{3.767652in}{2.216535in}}{\pgfqpoint{3.778251in}{2.212145in}}{\pgfqpoint{3.789301in}{2.212145in}}%
\pgfpathlineto{\pgfqpoint{3.789301in}{2.212145in}}%
\pgfpathclose%
\pgfusepath{stroke,fill}%
\end{pgfscope}%
\begin{pgfscope}%
\pgfpathrectangle{\pgfqpoint{2.963410in}{0.569136in}}{\pgfqpoint{2.177280in}{2.201755in}}%
\pgfusepath{clip}%
\pgfsetbuttcap%
\pgfsetroundjoin%
\definecolor{currentfill}{rgb}{0.172549,0.627451,0.172549}%
\pgfsetfillcolor{currentfill}%
\pgfsetlinewidth{0.481800pt}%
\definecolor{currentstroke}{rgb}{1.000000,1.000000,1.000000}%
\pgfsetstrokecolor{currentstroke}%
\pgfsetdash{}{0pt}%
\pgfpathmoveto{\pgfqpoint{3.730226in}{2.045346in}}%
\pgfpathcurveto{\pgfqpoint{3.741276in}{2.045346in}}{\pgfqpoint{3.751875in}{2.049736in}}{\pgfqpoint{3.759689in}{2.057549in}}%
\pgfpathcurveto{\pgfqpoint{3.767502in}{2.065363in}}{\pgfqpoint{3.771893in}{2.075962in}}{\pgfqpoint{3.771893in}{2.087012in}}%
\pgfpathcurveto{\pgfqpoint{3.771893in}{2.098062in}}{\pgfqpoint{3.767502in}{2.108661in}}{\pgfqpoint{3.759689in}{2.116475in}}%
\pgfpathcurveto{\pgfqpoint{3.751875in}{2.124289in}}{\pgfqpoint{3.741276in}{2.128679in}}{\pgfqpoint{3.730226in}{2.128679in}}%
\pgfpathcurveto{\pgfqpoint{3.719176in}{2.128679in}}{\pgfqpoint{3.708577in}{2.124289in}}{\pgfqpoint{3.700763in}{2.116475in}}%
\pgfpathcurveto{\pgfqpoint{3.692950in}{2.108661in}}{\pgfqpoint{3.688559in}{2.098062in}}{\pgfqpoint{3.688559in}{2.087012in}}%
\pgfpathcurveto{\pgfqpoint{3.688559in}{2.075962in}}{\pgfqpoint{3.692950in}{2.065363in}}{\pgfqpoint{3.700763in}{2.057549in}}%
\pgfpathcurveto{\pgfqpoint{3.708577in}{2.049736in}}{\pgfqpoint{3.719176in}{2.045346in}}{\pgfqpoint{3.730226in}{2.045346in}}%
\pgfpathlineto{\pgfqpoint{3.730226in}{2.045346in}}%
\pgfpathclose%
\pgfusepath{stroke,fill}%
\end{pgfscope}%
\begin{pgfscope}%
\pgfpathrectangle{\pgfqpoint{2.963410in}{0.569136in}}{\pgfqpoint{2.177280in}{2.201755in}}%
\pgfusepath{clip}%
\pgfsetbuttcap%
\pgfsetroundjoin%
\definecolor{currentfill}{rgb}{0.172549,0.627451,0.172549}%
\pgfsetfillcolor{currentfill}%
\pgfsetlinewidth{0.481800pt}%
\definecolor{currentstroke}{rgb}{1.000000,1.000000,1.000000}%
\pgfsetstrokecolor{currentstroke}%
\pgfsetdash{}{0pt}%
\pgfpathmoveto{\pgfqpoint{4.084678in}{2.295545in}}%
\pgfpathcurveto{\pgfqpoint{4.095728in}{2.295545in}}{\pgfqpoint{4.106327in}{2.299935in}}{\pgfqpoint{4.114140in}{2.307749in}}%
\pgfpathcurveto{\pgfqpoint{4.121954in}{2.315562in}}{\pgfqpoint{4.126344in}{2.326161in}}{\pgfqpoint{4.126344in}{2.337212in}}%
\pgfpathcurveto{\pgfqpoint{4.126344in}{2.348262in}}{\pgfqpoint{4.121954in}{2.358861in}}{\pgfqpoint{4.114140in}{2.366674in}}%
\pgfpathcurveto{\pgfqpoint{4.106327in}{2.374488in}}{\pgfqpoint{4.095728in}{2.378878in}}{\pgfqpoint{4.084678in}{2.378878in}}%
\pgfpathcurveto{\pgfqpoint{4.073627in}{2.378878in}}{\pgfqpoint{4.063028in}{2.374488in}}{\pgfqpoint{4.055215in}{2.366674in}}%
\pgfpathcurveto{\pgfqpoint{4.047401in}{2.358861in}}{\pgfqpoint{4.043011in}{2.348262in}}{\pgfqpoint{4.043011in}{2.337212in}}%
\pgfpathcurveto{\pgfqpoint{4.043011in}{2.326161in}}{\pgfqpoint{4.047401in}{2.315562in}}{\pgfqpoint{4.055215in}{2.307749in}}%
\pgfpathcurveto{\pgfqpoint{4.063028in}{2.299935in}}{\pgfqpoint{4.073627in}{2.295545in}}{\pgfqpoint{4.084678in}{2.295545in}}%
\pgfpathlineto{\pgfqpoint{4.084678in}{2.295545in}}%
\pgfpathclose%
\pgfusepath{stroke,fill}%
\end{pgfscope}%
\begin{pgfscope}%
\pgfpathrectangle{\pgfqpoint{2.963410in}{0.569136in}}{\pgfqpoint{2.177280in}{2.201755in}}%
\pgfusepath{clip}%
\pgfsetbuttcap%
\pgfsetroundjoin%
\definecolor{currentfill}{rgb}{0.172549,0.627451,0.172549}%
\pgfsetfillcolor{currentfill}%
\pgfsetlinewidth{0.481800pt}%
\definecolor{currentstroke}{rgb}{1.000000,1.000000,1.000000}%
\pgfsetstrokecolor{currentstroke}%
\pgfsetdash{}{0pt}%
\pgfpathmoveto{\pgfqpoint{4.025602in}{2.045346in}}%
\pgfpathcurveto{\pgfqpoint{4.036652in}{2.045346in}}{\pgfqpoint{4.047251in}{2.049736in}}{\pgfqpoint{4.055065in}{2.057549in}}%
\pgfpathcurveto{\pgfqpoint{4.062879in}{2.065363in}}{\pgfqpoint{4.067269in}{2.075962in}}{\pgfqpoint{4.067269in}{2.087012in}}%
\pgfpathcurveto{\pgfqpoint{4.067269in}{2.098062in}}{\pgfqpoint{4.062879in}{2.108661in}}{\pgfqpoint{4.055065in}{2.116475in}}%
\pgfpathcurveto{\pgfqpoint{4.047251in}{2.124289in}}{\pgfqpoint{4.036652in}{2.128679in}}{\pgfqpoint{4.025602in}{2.128679in}}%
\pgfpathcurveto{\pgfqpoint{4.014552in}{2.128679in}}{\pgfqpoint{4.003953in}{2.124289in}}{\pgfqpoint{3.996139in}{2.116475in}}%
\pgfpathcurveto{\pgfqpoint{3.988326in}{2.108661in}}{\pgfqpoint{3.983936in}{2.098062in}}{\pgfqpoint{3.983936in}{2.087012in}}%
\pgfpathcurveto{\pgfqpoint{3.983936in}{2.075962in}}{\pgfqpoint{3.988326in}{2.065363in}}{\pgfqpoint{3.996139in}{2.057549in}}%
\pgfpathcurveto{\pgfqpoint{4.003953in}{2.049736in}}{\pgfqpoint{4.014552in}{2.045346in}}{\pgfqpoint{4.025602in}{2.045346in}}%
\pgfpathlineto{\pgfqpoint{4.025602in}{2.045346in}}%
\pgfpathclose%
\pgfusepath{stroke,fill}%
\end{pgfscope}%
\begin{pgfscope}%
\pgfpathrectangle{\pgfqpoint{2.963410in}{0.569136in}}{\pgfqpoint{2.177280in}{2.201755in}}%
\pgfusepath{clip}%
\pgfsetbuttcap%
\pgfsetroundjoin%
\definecolor{currentfill}{rgb}{0.172549,0.627451,0.172549}%
\pgfsetfillcolor{currentfill}%
\pgfsetlinewidth{0.481800pt}%
\definecolor{currentstroke}{rgb}{1.000000,1.000000,1.000000}%
\pgfsetstrokecolor{currentstroke}%
\pgfsetdash{}{0pt}%
\pgfpathmoveto{\pgfqpoint{3.789301in}{2.045346in}}%
\pgfpathcurveto{\pgfqpoint{3.800351in}{2.045346in}}{\pgfqpoint{3.810950in}{2.049736in}}{\pgfqpoint{3.818764in}{2.057549in}}%
\pgfpathcurveto{\pgfqpoint{3.826578in}{2.065363in}}{\pgfqpoint{3.830968in}{2.075962in}}{\pgfqpoint{3.830968in}{2.087012in}}%
\pgfpathcurveto{\pgfqpoint{3.830968in}{2.098062in}}{\pgfqpoint{3.826578in}{2.108661in}}{\pgfqpoint{3.818764in}{2.116475in}}%
\pgfpathcurveto{\pgfqpoint{3.810950in}{2.124289in}}{\pgfqpoint{3.800351in}{2.128679in}}{\pgfqpoint{3.789301in}{2.128679in}}%
\pgfpathcurveto{\pgfqpoint{3.778251in}{2.128679in}}{\pgfqpoint{3.767652in}{2.124289in}}{\pgfqpoint{3.759838in}{2.116475in}}%
\pgfpathcurveto{\pgfqpoint{3.752025in}{2.108661in}}{\pgfqpoint{3.747635in}{2.098062in}}{\pgfqpoint{3.747635in}{2.087012in}}%
\pgfpathcurveto{\pgfqpoint{3.747635in}{2.075962in}}{\pgfqpoint{3.752025in}{2.065363in}}{\pgfqpoint{3.759838in}{2.057549in}}%
\pgfpathcurveto{\pgfqpoint{3.767652in}{2.049736in}}{\pgfqpoint{3.778251in}{2.045346in}}{\pgfqpoint{3.789301in}{2.045346in}}%
\pgfpathlineto{\pgfqpoint{3.789301in}{2.045346in}}%
\pgfpathclose%
\pgfusepath{stroke,fill}%
\end{pgfscope}%
\begin{pgfscope}%
\pgfpathrectangle{\pgfqpoint{2.963410in}{0.569136in}}{\pgfqpoint{2.177280in}{2.201755in}}%
\pgfusepath{clip}%
\pgfsetbuttcap%
\pgfsetroundjoin%
\definecolor{currentfill}{rgb}{0.172549,0.627451,0.172549}%
\pgfsetfillcolor{currentfill}%
\pgfsetlinewidth{0.481800pt}%
\definecolor{currentstroke}{rgb}{1.000000,1.000000,1.000000}%
\pgfsetstrokecolor{currentstroke}%
\pgfsetdash{}{0pt}%
\pgfpathmoveto{\pgfqpoint{3.907452in}{2.045346in}}%
\pgfpathcurveto{\pgfqpoint{3.918502in}{2.045346in}}{\pgfqpoint{3.929101in}{2.049736in}}{\pgfqpoint{3.936915in}{2.057549in}}%
\pgfpathcurveto{\pgfqpoint{3.944728in}{2.065363in}}{\pgfqpoint{3.949118in}{2.075962in}}{\pgfqpoint{3.949118in}{2.087012in}}%
\pgfpathcurveto{\pgfqpoint{3.949118in}{2.098062in}}{\pgfqpoint{3.944728in}{2.108661in}}{\pgfqpoint{3.936915in}{2.116475in}}%
\pgfpathcurveto{\pgfqpoint{3.929101in}{2.124289in}}{\pgfqpoint{3.918502in}{2.128679in}}{\pgfqpoint{3.907452in}{2.128679in}}%
\pgfpathcurveto{\pgfqpoint{3.896402in}{2.128679in}}{\pgfqpoint{3.885803in}{2.124289in}}{\pgfqpoint{3.877989in}{2.116475in}}%
\pgfpathcurveto{\pgfqpoint{3.870175in}{2.108661in}}{\pgfqpoint{3.865785in}{2.098062in}}{\pgfqpoint{3.865785in}{2.087012in}}%
\pgfpathcurveto{\pgfqpoint{3.865785in}{2.075962in}}{\pgfqpoint{3.870175in}{2.065363in}}{\pgfqpoint{3.877989in}{2.057549in}}%
\pgfpathcurveto{\pgfqpoint{3.885803in}{2.049736in}}{\pgfqpoint{3.896402in}{2.045346in}}{\pgfqpoint{3.907452in}{2.045346in}}%
\pgfpathlineto{\pgfqpoint{3.907452in}{2.045346in}}%
\pgfpathclose%
\pgfusepath{stroke,fill}%
\end{pgfscope}%
\begin{pgfscope}%
\pgfpathrectangle{\pgfqpoint{2.963410in}{0.569136in}}{\pgfqpoint{2.177280in}{2.201755in}}%
\pgfusepath{clip}%
\pgfsetbuttcap%
\pgfsetroundjoin%
\definecolor{currentfill}{rgb}{0.172549,0.627451,0.172549}%
\pgfsetfillcolor{currentfill}%
\pgfsetlinewidth{0.481800pt}%
\definecolor{currentstroke}{rgb}{1.000000,1.000000,1.000000}%
\pgfsetstrokecolor{currentstroke}%
\pgfsetdash{}{0pt}%
\pgfpathmoveto{\pgfqpoint{3.789301in}{2.295545in}}%
\pgfpathcurveto{\pgfqpoint{3.800351in}{2.295545in}}{\pgfqpoint{3.810950in}{2.299935in}}{\pgfqpoint{3.818764in}{2.307749in}}%
\pgfpathcurveto{\pgfqpoint{3.826578in}{2.315562in}}{\pgfqpoint{3.830968in}{2.326161in}}{\pgfqpoint{3.830968in}{2.337212in}}%
\pgfpathcurveto{\pgfqpoint{3.830968in}{2.348262in}}{\pgfqpoint{3.826578in}{2.358861in}}{\pgfqpoint{3.818764in}{2.366674in}}%
\pgfpathcurveto{\pgfqpoint{3.810950in}{2.374488in}}{\pgfqpoint{3.800351in}{2.378878in}}{\pgfqpoint{3.789301in}{2.378878in}}%
\pgfpathcurveto{\pgfqpoint{3.778251in}{2.378878in}}{\pgfqpoint{3.767652in}{2.374488in}}{\pgfqpoint{3.759838in}{2.366674in}}%
\pgfpathcurveto{\pgfqpoint{3.752025in}{2.358861in}}{\pgfqpoint{3.747635in}{2.348262in}}{\pgfqpoint{3.747635in}{2.337212in}}%
\pgfpathcurveto{\pgfqpoint{3.747635in}{2.326161in}}{\pgfqpoint{3.752025in}{2.315562in}}{\pgfqpoint{3.759838in}{2.307749in}}%
\pgfpathcurveto{\pgfqpoint{3.767652in}{2.299935in}}{\pgfqpoint{3.778251in}{2.295545in}}{\pgfqpoint{3.789301in}{2.295545in}}%
\pgfpathlineto{\pgfqpoint{3.789301in}{2.295545in}}%
\pgfpathclose%
\pgfusepath{stroke,fill}%
\end{pgfscope}%
\begin{pgfscope}%
\pgfpathrectangle{\pgfqpoint{2.963410in}{0.569136in}}{\pgfqpoint{2.177280in}{2.201755in}}%
\pgfusepath{clip}%
\pgfsetbuttcap%
\pgfsetroundjoin%
\definecolor{currentfill}{rgb}{0.172549,0.627451,0.172549}%
\pgfsetfillcolor{currentfill}%
\pgfsetlinewidth{0.481800pt}%
\definecolor{currentstroke}{rgb}{1.000000,1.000000,1.000000}%
\pgfsetstrokecolor{currentstroke}%
\pgfsetdash{}{0pt}%
\pgfpathmoveto{\pgfqpoint{3.907452in}{1.878546in}}%
\pgfpathcurveto{\pgfqpoint{3.918502in}{1.878546in}}{\pgfqpoint{3.929101in}{1.882936in}}{\pgfqpoint{3.936915in}{1.890750in}}%
\pgfpathcurveto{\pgfqpoint{3.944728in}{1.898563in}}{\pgfqpoint{3.949118in}{1.909162in}}{\pgfqpoint{3.949118in}{1.920213in}}%
\pgfpathcurveto{\pgfqpoint{3.949118in}{1.931263in}}{\pgfqpoint{3.944728in}{1.941862in}}{\pgfqpoint{3.936915in}{1.949675in}}%
\pgfpathcurveto{\pgfqpoint{3.929101in}{1.957489in}}{\pgfqpoint{3.918502in}{1.961879in}}{\pgfqpoint{3.907452in}{1.961879in}}%
\pgfpathcurveto{\pgfqpoint{3.896402in}{1.961879in}}{\pgfqpoint{3.885803in}{1.957489in}}{\pgfqpoint{3.877989in}{1.949675in}}%
\pgfpathcurveto{\pgfqpoint{3.870175in}{1.941862in}}{\pgfqpoint{3.865785in}{1.931263in}}{\pgfqpoint{3.865785in}{1.920213in}}%
\pgfpathcurveto{\pgfqpoint{3.865785in}{1.909162in}}{\pgfqpoint{3.870175in}{1.898563in}}{\pgfqpoint{3.877989in}{1.890750in}}%
\pgfpathcurveto{\pgfqpoint{3.885803in}{1.882936in}}{\pgfqpoint{3.896402in}{1.878546in}}{\pgfqpoint{3.907452in}{1.878546in}}%
\pgfpathlineto{\pgfqpoint{3.907452in}{1.878546in}}%
\pgfpathclose%
\pgfusepath{stroke,fill}%
\end{pgfscope}%
\begin{pgfscope}%
\pgfpathrectangle{\pgfqpoint{2.963410in}{0.569136in}}{\pgfqpoint{2.177280in}{2.201755in}}%
\pgfusepath{clip}%
\pgfsetbuttcap%
\pgfsetroundjoin%
\definecolor{currentfill}{rgb}{0.172549,0.627451,0.172549}%
\pgfsetfillcolor{currentfill}%
\pgfsetlinewidth{0.481800pt}%
\definecolor{currentstroke}{rgb}{1.000000,1.000000,1.000000}%
\pgfsetstrokecolor{currentstroke}%
\pgfsetdash{}{0pt}%
\pgfpathmoveto{\pgfqpoint{3.789301in}{2.128745in}}%
\pgfpathcurveto{\pgfqpoint{3.800351in}{2.128745in}}{\pgfqpoint{3.810950in}{2.133136in}}{\pgfqpoint{3.818764in}{2.140949in}}%
\pgfpathcurveto{\pgfqpoint{3.826578in}{2.148763in}}{\pgfqpoint{3.830968in}{2.159362in}}{\pgfqpoint{3.830968in}{2.170412in}}%
\pgfpathcurveto{\pgfqpoint{3.830968in}{2.181462in}}{\pgfqpoint{3.826578in}{2.192061in}}{\pgfqpoint{3.818764in}{2.199875in}}%
\pgfpathcurveto{\pgfqpoint{3.810950in}{2.207688in}}{\pgfqpoint{3.800351in}{2.212079in}}{\pgfqpoint{3.789301in}{2.212079in}}%
\pgfpathcurveto{\pgfqpoint{3.778251in}{2.212079in}}{\pgfqpoint{3.767652in}{2.207688in}}{\pgfqpoint{3.759838in}{2.199875in}}%
\pgfpathcurveto{\pgfqpoint{3.752025in}{2.192061in}}{\pgfqpoint{3.747635in}{2.181462in}}{\pgfqpoint{3.747635in}{2.170412in}}%
\pgfpathcurveto{\pgfqpoint{3.747635in}{2.159362in}}{\pgfqpoint{3.752025in}{2.148763in}}{\pgfqpoint{3.759838in}{2.140949in}}%
\pgfpathcurveto{\pgfqpoint{3.767652in}{2.133136in}}{\pgfqpoint{3.778251in}{2.128745in}}{\pgfqpoint{3.789301in}{2.128745in}}%
\pgfpathlineto{\pgfqpoint{3.789301in}{2.128745in}}%
\pgfpathclose%
\pgfusepath{stroke,fill}%
\end{pgfscope}%
\begin{pgfscope}%
\pgfpathrectangle{\pgfqpoint{2.963410in}{0.569136in}}{\pgfqpoint{2.177280in}{2.201755in}}%
\pgfusepath{clip}%
\pgfsetbuttcap%
\pgfsetroundjoin%
\definecolor{currentfill}{rgb}{0.172549,0.627451,0.172549}%
\pgfsetfillcolor{currentfill}%
\pgfsetlinewidth{0.481800pt}%
\definecolor{currentstroke}{rgb}{1.000000,1.000000,1.000000}%
\pgfsetstrokecolor{currentstroke}%
\pgfsetdash{}{0pt}%
\pgfpathmoveto{\pgfqpoint{4.380054in}{2.212145in}}%
\pgfpathcurveto{\pgfqpoint{4.391104in}{2.212145in}}{\pgfqpoint{4.401703in}{2.216535in}}{\pgfqpoint{4.409517in}{2.224349in}}%
\pgfpathcurveto{\pgfqpoint{4.417330in}{2.232163in}}{\pgfqpoint{4.421721in}{2.242762in}}{\pgfqpoint{4.421721in}{2.253812in}}%
\pgfpathcurveto{\pgfqpoint{4.421721in}{2.264862in}}{\pgfqpoint{4.417330in}{2.275461in}}{\pgfqpoint{4.409517in}{2.283275in}}%
\pgfpathcurveto{\pgfqpoint{4.401703in}{2.291088in}}{\pgfqpoint{4.391104in}{2.295478in}}{\pgfqpoint{4.380054in}{2.295478in}}%
\pgfpathcurveto{\pgfqpoint{4.369004in}{2.295478in}}{\pgfqpoint{4.358405in}{2.291088in}}{\pgfqpoint{4.350591in}{2.283275in}}%
\pgfpathcurveto{\pgfqpoint{4.342777in}{2.275461in}}{\pgfqpoint{4.338387in}{2.264862in}}{\pgfqpoint{4.338387in}{2.253812in}}%
\pgfpathcurveto{\pgfqpoint{4.338387in}{2.242762in}}{\pgfqpoint{4.342777in}{2.232163in}}{\pgfqpoint{4.350591in}{2.224349in}}%
\pgfpathcurveto{\pgfqpoint{4.358405in}{2.216535in}}{\pgfqpoint{4.369004in}{2.212145in}}{\pgfqpoint{4.380054in}{2.212145in}}%
\pgfpathlineto{\pgfqpoint{4.380054in}{2.212145in}}%
\pgfpathclose%
\pgfusepath{stroke,fill}%
\end{pgfscope}%
\begin{pgfscope}%
\pgfpathrectangle{\pgfqpoint{2.963410in}{0.569136in}}{\pgfqpoint{2.177280in}{2.201755in}}%
\pgfusepath{clip}%
\pgfsetbuttcap%
\pgfsetroundjoin%
\definecolor{currentfill}{rgb}{0.172549,0.627451,0.172549}%
\pgfsetfillcolor{currentfill}%
\pgfsetlinewidth{0.481800pt}%
\definecolor{currentstroke}{rgb}{1.000000,1.000000,1.000000}%
\pgfsetstrokecolor{currentstroke}%
\pgfsetdash{}{0pt}%
\pgfpathmoveto{\pgfqpoint{3.789301in}{2.378945in}}%
\pgfpathcurveto{\pgfqpoint{3.800351in}{2.378945in}}{\pgfqpoint{3.810950in}{2.383335in}}{\pgfqpoint{3.818764in}{2.391149in}}%
\pgfpathcurveto{\pgfqpoint{3.826578in}{2.398962in}}{\pgfqpoint{3.830968in}{2.409561in}}{\pgfqpoint{3.830968in}{2.420611in}}%
\pgfpathcurveto{\pgfqpoint{3.830968in}{2.431662in}}{\pgfqpoint{3.826578in}{2.442261in}}{\pgfqpoint{3.818764in}{2.450074in}}%
\pgfpathcurveto{\pgfqpoint{3.810950in}{2.457888in}}{\pgfqpoint{3.800351in}{2.462278in}}{\pgfqpoint{3.789301in}{2.462278in}}%
\pgfpathcurveto{\pgfqpoint{3.778251in}{2.462278in}}{\pgfqpoint{3.767652in}{2.457888in}}{\pgfqpoint{3.759838in}{2.450074in}}%
\pgfpathcurveto{\pgfqpoint{3.752025in}{2.442261in}}{\pgfqpoint{3.747635in}{2.431662in}}{\pgfqpoint{3.747635in}{2.420611in}}%
\pgfpathcurveto{\pgfqpoint{3.747635in}{2.409561in}}{\pgfqpoint{3.752025in}{2.398962in}}{\pgfqpoint{3.759838in}{2.391149in}}%
\pgfpathcurveto{\pgfqpoint{3.767652in}{2.383335in}}{\pgfqpoint{3.778251in}{2.378945in}}{\pgfqpoint{3.789301in}{2.378945in}}%
\pgfpathlineto{\pgfqpoint{3.789301in}{2.378945in}}%
\pgfpathclose%
\pgfusepath{stroke,fill}%
\end{pgfscope}%
\begin{pgfscope}%
\pgfpathrectangle{\pgfqpoint{2.963410in}{0.569136in}}{\pgfqpoint{2.177280in}{2.201755in}}%
\pgfusepath{clip}%
\pgfsetbuttcap%
\pgfsetroundjoin%
\definecolor{currentfill}{rgb}{0.172549,0.627451,0.172549}%
\pgfsetfillcolor{currentfill}%
\pgfsetlinewidth{0.481800pt}%
\definecolor{currentstroke}{rgb}{1.000000,1.000000,1.000000}%
\pgfsetstrokecolor{currentstroke}%
\pgfsetdash{}{0pt}%
\pgfpathmoveto{\pgfqpoint{3.789301in}{1.795146in}}%
\pgfpathcurveto{\pgfqpoint{3.800351in}{1.795146in}}{\pgfqpoint{3.810950in}{1.799536in}}{\pgfqpoint{3.818764in}{1.807350in}}%
\pgfpathcurveto{\pgfqpoint{3.826578in}{1.815164in}}{\pgfqpoint{3.830968in}{1.825763in}}{\pgfqpoint{3.830968in}{1.836813in}}%
\pgfpathcurveto{\pgfqpoint{3.830968in}{1.847863in}}{\pgfqpoint{3.826578in}{1.858462in}}{\pgfqpoint{3.818764in}{1.866276in}}%
\pgfpathcurveto{\pgfqpoint{3.810950in}{1.874089in}}{\pgfqpoint{3.800351in}{1.878479in}}{\pgfqpoint{3.789301in}{1.878479in}}%
\pgfpathcurveto{\pgfqpoint{3.778251in}{1.878479in}}{\pgfqpoint{3.767652in}{1.874089in}}{\pgfqpoint{3.759838in}{1.866276in}}%
\pgfpathcurveto{\pgfqpoint{3.752025in}{1.858462in}}{\pgfqpoint{3.747635in}{1.847863in}}{\pgfqpoint{3.747635in}{1.836813in}}%
\pgfpathcurveto{\pgfqpoint{3.747635in}{1.825763in}}{\pgfqpoint{3.752025in}{1.815164in}}{\pgfqpoint{3.759838in}{1.807350in}}%
\pgfpathcurveto{\pgfqpoint{3.767652in}{1.799536in}}{\pgfqpoint{3.778251in}{1.795146in}}{\pgfqpoint{3.789301in}{1.795146in}}%
\pgfpathlineto{\pgfqpoint{3.789301in}{1.795146in}}%
\pgfpathclose%
\pgfusepath{stroke,fill}%
\end{pgfscope}%
\begin{pgfscope}%
\pgfpathrectangle{\pgfqpoint{2.963410in}{0.569136in}}{\pgfqpoint{2.177280in}{2.201755in}}%
\pgfusepath{clip}%
\pgfsetbuttcap%
\pgfsetroundjoin%
\definecolor{currentfill}{rgb}{0.172549,0.627451,0.172549}%
\pgfsetfillcolor{currentfill}%
\pgfsetlinewidth{0.481800pt}%
\definecolor{currentstroke}{rgb}{1.000000,1.000000,1.000000}%
\pgfsetstrokecolor{currentstroke}%
\pgfsetdash{}{0pt}%
\pgfpathmoveto{\pgfqpoint{3.671151in}{1.711746in}}%
\pgfpathcurveto{\pgfqpoint{3.682201in}{1.711746in}}{\pgfqpoint{3.692800in}{1.716137in}}{\pgfqpoint{3.700613in}{1.723950in}}%
\pgfpathcurveto{\pgfqpoint{3.708427in}{1.731764in}}{\pgfqpoint{3.712817in}{1.742363in}}{\pgfqpoint{3.712817in}{1.753413in}}%
\pgfpathcurveto{\pgfqpoint{3.712817in}{1.764463in}}{\pgfqpoint{3.708427in}{1.775062in}}{\pgfqpoint{3.700613in}{1.782876in}}%
\pgfpathcurveto{\pgfqpoint{3.692800in}{1.790689in}}{\pgfqpoint{3.682201in}{1.795080in}}{\pgfqpoint{3.671151in}{1.795080in}}%
\pgfpathcurveto{\pgfqpoint{3.660101in}{1.795080in}}{\pgfqpoint{3.649501in}{1.790689in}}{\pgfqpoint{3.641688in}{1.782876in}}%
\pgfpathcurveto{\pgfqpoint{3.633874in}{1.775062in}}{\pgfqpoint{3.629484in}{1.764463in}}{\pgfqpoint{3.629484in}{1.753413in}}%
\pgfpathcurveto{\pgfqpoint{3.629484in}{1.742363in}}{\pgfqpoint{3.633874in}{1.731764in}}{\pgfqpoint{3.641688in}{1.723950in}}%
\pgfpathcurveto{\pgfqpoint{3.649501in}{1.716137in}}{\pgfqpoint{3.660101in}{1.711746in}}{\pgfqpoint{3.671151in}{1.711746in}}%
\pgfpathlineto{\pgfqpoint{3.671151in}{1.711746in}}%
\pgfpathclose%
\pgfusepath{stroke,fill}%
\end{pgfscope}%
\begin{pgfscope}%
\pgfpathrectangle{\pgfqpoint{2.963410in}{0.569136in}}{\pgfqpoint{2.177280in}{2.201755in}}%
\pgfusepath{clip}%
\pgfsetbuttcap%
\pgfsetroundjoin%
\definecolor{currentfill}{rgb}{0.172549,0.627451,0.172549}%
\pgfsetfillcolor{currentfill}%
\pgfsetlinewidth{0.481800pt}%
\definecolor{currentstroke}{rgb}{1.000000,1.000000,1.000000}%
\pgfsetstrokecolor{currentstroke}%
\pgfsetdash{}{0pt}%
\pgfpathmoveto{\pgfqpoint{3.907452in}{2.462345in}}%
\pgfpathcurveto{\pgfqpoint{3.918502in}{2.462345in}}{\pgfqpoint{3.929101in}{2.466735in}}{\pgfqpoint{3.936915in}{2.474548in}}%
\pgfpathcurveto{\pgfqpoint{3.944728in}{2.482362in}}{\pgfqpoint{3.949118in}{2.492961in}}{\pgfqpoint{3.949118in}{2.504011in}}%
\pgfpathcurveto{\pgfqpoint{3.949118in}{2.515061in}}{\pgfqpoint{3.944728in}{2.525660in}}{\pgfqpoint{3.936915in}{2.533474in}}%
\pgfpathcurveto{\pgfqpoint{3.929101in}{2.541288in}}{\pgfqpoint{3.918502in}{2.545678in}}{\pgfqpoint{3.907452in}{2.545678in}}%
\pgfpathcurveto{\pgfqpoint{3.896402in}{2.545678in}}{\pgfqpoint{3.885803in}{2.541288in}}{\pgfqpoint{3.877989in}{2.533474in}}%
\pgfpathcurveto{\pgfqpoint{3.870175in}{2.525660in}}{\pgfqpoint{3.865785in}{2.515061in}}{\pgfqpoint{3.865785in}{2.504011in}}%
\pgfpathcurveto{\pgfqpoint{3.865785in}{2.492961in}}{\pgfqpoint{3.870175in}{2.482362in}}{\pgfqpoint{3.877989in}{2.474548in}}%
\pgfpathcurveto{\pgfqpoint{3.885803in}{2.466735in}}{\pgfqpoint{3.896402in}{2.462345in}}{\pgfqpoint{3.907452in}{2.462345in}}%
\pgfpathlineto{\pgfqpoint{3.907452in}{2.462345in}}%
\pgfpathclose%
\pgfusepath{stroke,fill}%
\end{pgfscope}%
\begin{pgfscope}%
\pgfpathrectangle{\pgfqpoint{2.963410in}{0.569136in}}{\pgfqpoint{2.177280in}{2.201755in}}%
\pgfusepath{clip}%
\pgfsetbuttcap%
\pgfsetroundjoin%
\definecolor{currentfill}{rgb}{0.172549,0.627451,0.172549}%
\pgfsetfillcolor{currentfill}%
\pgfsetlinewidth{0.481800pt}%
\definecolor{currentstroke}{rgb}{1.000000,1.000000,1.000000}%
\pgfsetstrokecolor{currentstroke}%
\pgfsetdash{}{0pt}%
\pgfpathmoveto{\pgfqpoint{4.143753in}{2.545744in}}%
\pgfpathcurveto{\pgfqpoint{4.154803in}{2.545744in}}{\pgfqpoint{4.165402in}{2.550135in}}{\pgfqpoint{4.173216in}{2.557948in}}%
\pgfpathcurveto{\pgfqpoint{4.181029in}{2.565762in}}{\pgfqpoint{4.185419in}{2.576361in}}{\pgfqpoint{4.185419in}{2.587411in}}%
\pgfpathcurveto{\pgfqpoint{4.185419in}{2.598461in}}{\pgfqpoint{4.181029in}{2.609060in}}{\pgfqpoint{4.173216in}{2.616874in}}%
\pgfpathcurveto{\pgfqpoint{4.165402in}{2.624687in}}{\pgfqpoint{4.154803in}{2.629078in}}{\pgfqpoint{4.143753in}{2.629078in}}%
\pgfpathcurveto{\pgfqpoint{4.132703in}{2.629078in}}{\pgfqpoint{4.122104in}{2.624687in}}{\pgfqpoint{4.114290in}{2.616874in}}%
\pgfpathcurveto{\pgfqpoint{4.106476in}{2.609060in}}{\pgfqpoint{4.102086in}{2.598461in}}{\pgfqpoint{4.102086in}{2.587411in}}%
\pgfpathcurveto{\pgfqpoint{4.102086in}{2.576361in}}{\pgfqpoint{4.106476in}{2.565762in}}{\pgfqpoint{4.114290in}{2.557948in}}%
\pgfpathcurveto{\pgfqpoint{4.122104in}{2.550135in}}{\pgfqpoint{4.132703in}{2.545744in}}{\pgfqpoint{4.143753in}{2.545744in}}%
\pgfpathlineto{\pgfqpoint{4.143753in}{2.545744in}}%
\pgfpathclose%
\pgfusepath{stroke,fill}%
\end{pgfscope}%
\begin{pgfscope}%
\pgfpathrectangle{\pgfqpoint{2.963410in}{0.569136in}}{\pgfqpoint{2.177280in}{2.201755in}}%
\pgfusepath{clip}%
\pgfsetbuttcap%
\pgfsetroundjoin%
\definecolor{currentfill}{rgb}{0.172549,0.627451,0.172549}%
\pgfsetfillcolor{currentfill}%
\pgfsetlinewidth{0.481800pt}%
\definecolor{currentstroke}{rgb}{1.000000,1.000000,1.000000}%
\pgfsetstrokecolor{currentstroke}%
\pgfsetdash{}{0pt}%
\pgfpathmoveto{\pgfqpoint{3.966527in}{2.045346in}}%
\pgfpathcurveto{\pgfqpoint{3.977577in}{2.045346in}}{\pgfqpoint{3.988176in}{2.049736in}}{\pgfqpoint{3.995990in}{2.057549in}}%
\pgfpathcurveto{\pgfqpoint{4.003803in}{2.065363in}}{\pgfqpoint{4.008194in}{2.075962in}}{\pgfqpoint{4.008194in}{2.087012in}}%
\pgfpathcurveto{\pgfqpoint{4.008194in}{2.098062in}}{\pgfqpoint{4.003803in}{2.108661in}}{\pgfqpoint{3.995990in}{2.116475in}}%
\pgfpathcurveto{\pgfqpoint{3.988176in}{2.124289in}}{\pgfqpoint{3.977577in}{2.128679in}}{\pgfqpoint{3.966527in}{2.128679in}}%
\pgfpathcurveto{\pgfqpoint{3.955477in}{2.128679in}}{\pgfqpoint{3.944878in}{2.124289in}}{\pgfqpoint{3.937064in}{2.116475in}}%
\pgfpathcurveto{\pgfqpoint{3.929251in}{2.108661in}}{\pgfqpoint{3.924860in}{2.098062in}}{\pgfqpoint{3.924860in}{2.087012in}}%
\pgfpathcurveto{\pgfqpoint{3.924860in}{2.075962in}}{\pgfqpoint{3.929251in}{2.065363in}}{\pgfqpoint{3.937064in}{2.057549in}}%
\pgfpathcurveto{\pgfqpoint{3.944878in}{2.049736in}}{\pgfqpoint{3.955477in}{2.045346in}}{\pgfqpoint{3.966527in}{2.045346in}}%
\pgfpathlineto{\pgfqpoint{3.966527in}{2.045346in}}%
\pgfpathclose%
\pgfusepath{stroke,fill}%
\end{pgfscope}%
\begin{pgfscope}%
\pgfpathrectangle{\pgfqpoint{2.963410in}{0.569136in}}{\pgfqpoint{2.177280in}{2.201755in}}%
\pgfusepath{clip}%
\pgfsetbuttcap%
\pgfsetroundjoin%
\definecolor{currentfill}{rgb}{0.172549,0.627451,0.172549}%
\pgfsetfillcolor{currentfill}%
\pgfsetlinewidth{0.481800pt}%
\definecolor{currentstroke}{rgb}{1.000000,1.000000,1.000000}%
\pgfsetstrokecolor{currentstroke}%
\pgfsetdash{}{0pt}%
\pgfpathmoveto{\pgfqpoint{3.907452in}{2.045346in}}%
\pgfpathcurveto{\pgfqpoint{3.918502in}{2.045346in}}{\pgfqpoint{3.929101in}{2.049736in}}{\pgfqpoint{3.936915in}{2.057549in}}%
\pgfpathcurveto{\pgfqpoint{3.944728in}{2.065363in}}{\pgfqpoint{3.949118in}{2.075962in}}{\pgfqpoint{3.949118in}{2.087012in}}%
\pgfpathcurveto{\pgfqpoint{3.949118in}{2.098062in}}{\pgfqpoint{3.944728in}{2.108661in}}{\pgfqpoint{3.936915in}{2.116475in}}%
\pgfpathcurveto{\pgfqpoint{3.929101in}{2.124289in}}{\pgfqpoint{3.918502in}{2.128679in}}{\pgfqpoint{3.907452in}{2.128679in}}%
\pgfpathcurveto{\pgfqpoint{3.896402in}{2.128679in}}{\pgfqpoint{3.885803in}{2.124289in}}{\pgfqpoint{3.877989in}{2.116475in}}%
\pgfpathcurveto{\pgfqpoint{3.870175in}{2.108661in}}{\pgfqpoint{3.865785in}{2.098062in}}{\pgfqpoint{3.865785in}{2.087012in}}%
\pgfpathcurveto{\pgfqpoint{3.865785in}{2.075962in}}{\pgfqpoint{3.870175in}{2.065363in}}{\pgfqpoint{3.877989in}{2.057549in}}%
\pgfpathcurveto{\pgfqpoint{3.885803in}{2.049736in}}{\pgfqpoint{3.896402in}{2.045346in}}{\pgfqpoint{3.907452in}{2.045346in}}%
\pgfpathlineto{\pgfqpoint{3.907452in}{2.045346in}}%
\pgfpathclose%
\pgfusepath{stroke,fill}%
\end{pgfscope}%
\begin{pgfscope}%
\pgfpathrectangle{\pgfqpoint{2.963410in}{0.569136in}}{\pgfqpoint{2.177280in}{2.201755in}}%
\pgfusepath{clip}%
\pgfsetbuttcap%
\pgfsetroundjoin%
\definecolor{currentfill}{rgb}{0.172549,0.627451,0.172549}%
\pgfsetfillcolor{currentfill}%
\pgfsetlinewidth{0.481800pt}%
\definecolor{currentstroke}{rgb}{1.000000,1.000000,1.000000}%
\pgfsetstrokecolor{currentstroke}%
\pgfsetdash{}{0pt}%
\pgfpathmoveto{\pgfqpoint{3.966527in}{2.295545in}}%
\pgfpathcurveto{\pgfqpoint{3.977577in}{2.295545in}}{\pgfqpoint{3.988176in}{2.299935in}}{\pgfqpoint{3.995990in}{2.307749in}}%
\pgfpathcurveto{\pgfqpoint{4.003803in}{2.315562in}}{\pgfqpoint{4.008194in}{2.326161in}}{\pgfqpoint{4.008194in}{2.337212in}}%
\pgfpathcurveto{\pgfqpoint{4.008194in}{2.348262in}}{\pgfqpoint{4.003803in}{2.358861in}}{\pgfqpoint{3.995990in}{2.366674in}}%
\pgfpathcurveto{\pgfqpoint{3.988176in}{2.374488in}}{\pgfqpoint{3.977577in}{2.378878in}}{\pgfqpoint{3.966527in}{2.378878in}}%
\pgfpathcurveto{\pgfqpoint{3.955477in}{2.378878in}}{\pgfqpoint{3.944878in}{2.374488in}}{\pgfqpoint{3.937064in}{2.366674in}}%
\pgfpathcurveto{\pgfqpoint{3.929251in}{2.358861in}}{\pgfqpoint{3.924860in}{2.348262in}}{\pgfqpoint{3.924860in}{2.337212in}}%
\pgfpathcurveto{\pgfqpoint{3.924860in}{2.326161in}}{\pgfqpoint{3.929251in}{2.315562in}}{\pgfqpoint{3.937064in}{2.307749in}}%
\pgfpathcurveto{\pgfqpoint{3.944878in}{2.299935in}}{\pgfqpoint{3.955477in}{2.295545in}}{\pgfqpoint{3.966527in}{2.295545in}}%
\pgfpathlineto{\pgfqpoint{3.966527in}{2.295545in}}%
\pgfpathclose%
\pgfusepath{stroke,fill}%
\end{pgfscope}%
\begin{pgfscope}%
\pgfpathrectangle{\pgfqpoint{2.963410in}{0.569136in}}{\pgfqpoint{2.177280in}{2.201755in}}%
\pgfusepath{clip}%
\pgfsetbuttcap%
\pgfsetroundjoin%
\definecolor{currentfill}{rgb}{0.172549,0.627451,0.172549}%
\pgfsetfillcolor{currentfill}%
\pgfsetlinewidth{0.481800pt}%
\definecolor{currentstroke}{rgb}{1.000000,1.000000,1.000000}%
\pgfsetstrokecolor{currentstroke}%
\pgfsetdash{}{0pt}%
\pgfpathmoveto{\pgfqpoint{3.966527in}{2.545744in}}%
\pgfpathcurveto{\pgfqpoint{3.977577in}{2.545744in}}{\pgfqpoint{3.988176in}{2.550135in}}{\pgfqpoint{3.995990in}{2.557948in}}%
\pgfpathcurveto{\pgfqpoint{4.003803in}{2.565762in}}{\pgfqpoint{4.008194in}{2.576361in}}{\pgfqpoint{4.008194in}{2.587411in}}%
\pgfpathcurveto{\pgfqpoint{4.008194in}{2.598461in}}{\pgfqpoint{4.003803in}{2.609060in}}{\pgfqpoint{3.995990in}{2.616874in}}%
\pgfpathcurveto{\pgfqpoint{3.988176in}{2.624687in}}{\pgfqpoint{3.977577in}{2.629078in}}{\pgfqpoint{3.966527in}{2.629078in}}%
\pgfpathcurveto{\pgfqpoint{3.955477in}{2.629078in}}{\pgfqpoint{3.944878in}{2.624687in}}{\pgfqpoint{3.937064in}{2.616874in}}%
\pgfpathcurveto{\pgfqpoint{3.929251in}{2.609060in}}{\pgfqpoint{3.924860in}{2.598461in}}{\pgfqpoint{3.924860in}{2.587411in}}%
\pgfpathcurveto{\pgfqpoint{3.924860in}{2.576361in}}{\pgfqpoint{3.929251in}{2.565762in}}{\pgfqpoint{3.937064in}{2.557948in}}%
\pgfpathcurveto{\pgfqpoint{3.944878in}{2.550135in}}{\pgfqpoint{3.955477in}{2.545744in}}{\pgfqpoint{3.966527in}{2.545744in}}%
\pgfpathlineto{\pgfqpoint{3.966527in}{2.545744in}}%
\pgfpathclose%
\pgfusepath{stroke,fill}%
\end{pgfscope}%
\begin{pgfscope}%
\pgfpathrectangle{\pgfqpoint{2.963410in}{0.569136in}}{\pgfqpoint{2.177280in}{2.201755in}}%
\pgfusepath{clip}%
\pgfsetbuttcap%
\pgfsetroundjoin%
\definecolor{currentfill}{rgb}{0.172549,0.627451,0.172549}%
\pgfsetfillcolor{currentfill}%
\pgfsetlinewidth{0.481800pt}%
\definecolor{currentstroke}{rgb}{1.000000,1.000000,1.000000}%
\pgfsetstrokecolor{currentstroke}%
\pgfsetdash{}{0pt}%
\pgfpathmoveto{\pgfqpoint{3.966527in}{2.462345in}}%
\pgfpathcurveto{\pgfqpoint{3.977577in}{2.462345in}}{\pgfqpoint{3.988176in}{2.466735in}}{\pgfqpoint{3.995990in}{2.474548in}}%
\pgfpathcurveto{\pgfqpoint{4.003803in}{2.482362in}}{\pgfqpoint{4.008194in}{2.492961in}}{\pgfqpoint{4.008194in}{2.504011in}}%
\pgfpathcurveto{\pgfqpoint{4.008194in}{2.515061in}}{\pgfqpoint{4.003803in}{2.525660in}}{\pgfqpoint{3.995990in}{2.533474in}}%
\pgfpathcurveto{\pgfqpoint{3.988176in}{2.541288in}}{\pgfqpoint{3.977577in}{2.545678in}}{\pgfqpoint{3.966527in}{2.545678in}}%
\pgfpathcurveto{\pgfqpoint{3.955477in}{2.545678in}}{\pgfqpoint{3.944878in}{2.541288in}}{\pgfqpoint{3.937064in}{2.533474in}}%
\pgfpathcurveto{\pgfqpoint{3.929251in}{2.525660in}}{\pgfqpoint{3.924860in}{2.515061in}}{\pgfqpoint{3.924860in}{2.504011in}}%
\pgfpathcurveto{\pgfqpoint{3.924860in}{2.492961in}}{\pgfqpoint{3.929251in}{2.482362in}}{\pgfqpoint{3.937064in}{2.474548in}}%
\pgfpathcurveto{\pgfqpoint{3.944878in}{2.466735in}}{\pgfqpoint{3.955477in}{2.462345in}}{\pgfqpoint{3.966527in}{2.462345in}}%
\pgfpathlineto{\pgfqpoint{3.966527in}{2.462345in}}%
\pgfpathclose%
\pgfusepath{stroke,fill}%
\end{pgfscope}%
\begin{pgfscope}%
\pgfpathrectangle{\pgfqpoint{2.963410in}{0.569136in}}{\pgfqpoint{2.177280in}{2.201755in}}%
\pgfusepath{clip}%
\pgfsetbuttcap%
\pgfsetroundjoin%
\definecolor{currentfill}{rgb}{0.172549,0.627451,0.172549}%
\pgfsetfillcolor{currentfill}%
\pgfsetlinewidth{0.481800pt}%
\definecolor{currentstroke}{rgb}{1.000000,1.000000,1.000000}%
\pgfsetstrokecolor{currentstroke}%
\pgfsetdash{}{0pt}%
\pgfpathmoveto{\pgfqpoint{3.730226in}{2.128745in}}%
\pgfpathcurveto{\pgfqpoint{3.741276in}{2.128745in}}{\pgfqpoint{3.751875in}{2.133136in}}{\pgfqpoint{3.759689in}{2.140949in}}%
\pgfpathcurveto{\pgfqpoint{3.767502in}{2.148763in}}{\pgfqpoint{3.771893in}{2.159362in}}{\pgfqpoint{3.771893in}{2.170412in}}%
\pgfpathcurveto{\pgfqpoint{3.771893in}{2.181462in}}{\pgfqpoint{3.767502in}{2.192061in}}{\pgfqpoint{3.759689in}{2.199875in}}%
\pgfpathcurveto{\pgfqpoint{3.751875in}{2.207688in}}{\pgfqpoint{3.741276in}{2.212079in}}{\pgfqpoint{3.730226in}{2.212079in}}%
\pgfpathcurveto{\pgfqpoint{3.719176in}{2.212079in}}{\pgfqpoint{3.708577in}{2.207688in}}{\pgfqpoint{3.700763in}{2.199875in}}%
\pgfpathcurveto{\pgfqpoint{3.692950in}{2.192061in}}{\pgfqpoint{3.688559in}{2.181462in}}{\pgfqpoint{3.688559in}{2.170412in}}%
\pgfpathcurveto{\pgfqpoint{3.688559in}{2.159362in}}{\pgfqpoint{3.692950in}{2.148763in}}{\pgfqpoint{3.700763in}{2.140949in}}%
\pgfpathcurveto{\pgfqpoint{3.708577in}{2.133136in}}{\pgfqpoint{3.719176in}{2.128745in}}{\pgfqpoint{3.730226in}{2.128745in}}%
\pgfpathlineto{\pgfqpoint{3.730226in}{2.128745in}}%
\pgfpathclose%
\pgfusepath{stroke,fill}%
\end{pgfscope}%
\begin{pgfscope}%
\pgfpathrectangle{\pgfqpoint{2.963410in}{0.569136in}}{\pgfqpoint{2.177280in}{2.201755in}}%
\pgfusepath{clip}%
\pgfsetbuttcap%
\pgfsetroundjoin%
\definecolor{currentfill}{rgb}{0.172549,0.627451,0.172549}%
\pgfsetfillcolor{currentfill}%
\pgfsetlinewidth{0.481800pt}%
\definecolor{currentstroke}{rgb}{1.000000,1.000000,1.000000}%
\pgfsetstrokecolor{currentstroke}%
\pgfsetdash{}{0pt}%
\pgfpathmoveto{\pgfqpoint{4.025602in}{2.462345in}}%
\pgfpathcurveto{\pgfqpoint{4.036652in}{2.462345in}}{\pgfqpoint{4.047251in}{2.466735in}}{\pgfqpoint{4.055065in}{2.474548in}}%
\pgfpathcurveto{\pgfqpoint{4.062879in}{2.482362in}}{\pgfqpoint{4.067269in}{2.492961in}}{\pgfqpoint{4.067269in}{2.504011in}}%
\pgfpathcurveto{\pgfqpoint{4.067269in}{2.515061in}}{\pgfqpoint{4.062879in}{2.525660in}}{\pgfqpoint{4.055065in}{2.533474in}}%
\pgfpathcurveto{\pgfqpoint{4.047251in}{2.541288in}}{\pgfqpoint{4.036652in}{2.545678in}}{\pgfqpoint{4.025602in}{2.545678in}}%
\pgfpathcurveto{\pgfqpoint{4.014552in}{2.545678in}}{\pgfqpoint{4.003953in}{2.541288in}}{\pgfqpoint{3.996139in}{2.533474in}}%
\pgfpathcurveto{\pgfqpoint{3.988326in}{2.525660in}}{\pgfqpoint{3.983936in}{2.515061in}}{\pgfqpoint{3.983936in}{2.504011in}}%
\pgfpathcurveto{\pgfqpoint{3.983936in}{2.492961in}}{\pgfqpoint{3.988326in}{2.482362in}}{\pgfqpoint{3.996139in}{2.474548in}}%
\pgfpathcurveto{\pgfqpoint{4.003953in}{2.466735in}}{\pgfqpoint{4.014552in}{2.462345in}}{\pgfqpoint{4.025602in}{2.462345in}}%
\pgfpathlineto{\pgfqpoint{4.025602in}{2.462345in}}%
\pgfpathclose%
\pgfusepath{stroke,fill}%
\end{pgfscope}%
\begin{pgfscope}%
\pgfpathrectangle{\pgfqpoint{2.963410in}{0.569136in}}{\pgfqpoint{2.177280in}{2.201755in}}%
\pgfusepath{clip}%
\pgfsetbuttcap%
\pgfsetroundjoin%
\definecolor{currentfill}{rgb}{0.172549,0.627451,0.172549}%
\pgfsetfillcolor{currentfill}%
\pgfsetlinewidth{0.481800pt}%
\definecolor{currentstroke}{rgb}{1.000000,1.000000,1.000000}%
\pgfsetstrokecolor{currentstroke}%
\pgfsetdash{}{0pt}%
\pgfpathmoveto{\pgfqpoint{4.084678in}{2.629144in}}%
\pgfpathcurveto{\pgfqpoint{4.095728in}{2.629144in}}{\pgfqpoint{4.106327in}{2.633534in}}{\pgfqpoint{4.114140in}{2.641348in}}%
\pgfpathcurveto{\pgfqpoint{4.121954in}{2.649162in}}{\pgfqpoint{4.126344in}{2.659761in}}{\pgfqpoint{4.126344in}{2.670811in}}%
\pgfpathcurveto{\pgfqpoint{4.126344in}{2.681861in}}{\pgfqpoint{4.121954in}{2.692460in}}{\pgfqpoint{4.114140in}{2.700274in}}%
\pgfpathcurveto{\pgfqpoint{4.106327in}{2.708087in}}{\pgfqpoint{4.095728in}{2.712478in}}{\pgfqpoint{4.084678in}{2.712478in}}%
\pgfpathcurveto{\pgfqpoint{4.073627in}{2.712478in}}{\pgfqpoint{4.063028in}{2.708087in}}{\pgfqpoint{4.055215in}{2.700274in}}%
\pgfpathcurveto{\pgfqpoint{4.047401in}{2.692460in}}{\pgfqpoint{4.043011in}{2.681861in}}{\pgfqpoint{4.043011in}{2.670811in}}%
\pgfpathcurveto{\pgfqpoint{4.043011in}{2.659761in}}{\pgfqpoint{4.047401in}{2.649162in}}{\pgfqpoint{4.055215in}{2.641348in}}%
\pgfpathcurveto{\pgfqpoint{4.063028in}{2.633534in}}{\pgfqpoint{4.073627in}{2.629144in}}{\pgfqpoint{4.084678in}{2.629144in}}%
\pgfpathlineto{\pgfqpoint{4.084678in}{2.629144in}}%
\pgfpathclose%
\pgfusepath{stroke,fill}%
\end{pgfscope}%
\begin{pgfscope}%
\pgfpathrectangle{\pgfqpoint{2.963410in}{0.569136in}}{\pgfqpoint{2.177280in}{2.201755in}}%
\pgfusepath{clip}%
\pgfsetbuttcap%
\pgfsetroundjoin%
\definecolor{currentfill}{rgb}{0.172549,0.627451,0.172549}%
\pgfsetfillcolor{currentfill}%
\pgfsetlinewidth{0.481800pt}%
\definecolor{currentstroke}{rgb}{1.000000,1.000000,1.000000}%
\pgfsetstrokecolor{currentstroke}%
\pgfsetdash{}{0pt}%
\pgfpathmoveto{\pgfqpoint{3.907452in}{2.462345in}}%
\pgfpathcurveto{\pgfqpoint{3.918502in}{2.462345in}}{\pgfqpoint{3.929101in}{2.466735in}}{\pgfqpoint{3.936915in}{2.474548in}}%
\pgfpathcurveto{\pgfqpoint{3.944728in}{2.482362in}}{\pgfqpoint{3.949118in}{2.492961in}}{\pgfqpoint{3.949118in}{2.504011in}}%
\pgfpathcurveto{\pgfqpoint{3.949118in}{2.515061in}}{\pgfqpoint{3.944728in}{2.525660in}}{\pgfqpoint{3.936915in}{2.533474in}}%
\pgfpathcurveto{\pgfqpoint{3.929101in}{2.541288in}}{\pgfqpoint{3.918502in}{2.545678in}}{\pgfqpoint{3.907452in}{2.545678in}}%
\pgfpathcurveto{\pgfqpoint{3.896402in}{2.545678in}}{\pgfqpoint{3.885803in}{2.541288in}}{\pgfqpoint{3.877989in}{2.533474in}}%
\pgfpathcurveto{\pgfqpoint{3.870175in}{2.525660in}}{\pgfqpoint{3.865785in}{2.515061in}}{\pgfqpoint{3.865785in}{2.504011in}}%
\pgfpathcurveto{\pgfqpoint{3.865785in}{2.492961in}}{\pgfqpoint{3.870175in}{2.482362in}}{\pgfqpoint{3.877989in}{2.474548in}}%
\pgfpathcurveto{\pgfqpoint{3.885803in}{2.466735in}}{\pgfqpoint{3.896402in}{2.462345in}}{\pgfqpoint{3.907452in}{2.462345in}}%
\pgfpathlineto{\pgfqpoint{3.907452in}{2.462345in}}%
\pgfpathclose%
\pgfusepath{stroke,fill}%
\end{pgfscope}%
\begin{pgfscope}%
\pgfpathrectangle{\pgfqpoint{2.963410in}{0.569136in}}{\pgfqpoint{2.177280in}{2.201755in}}%
\pgfusepath{clip}%
\pgfsetbuttcap%
\pgfsetroundjoin%
\definecolor{currentfill}{rgb}{0.172549,0.627451,0.172549}%
\pgfsetfillcolor{currentfill}%
\pgfsetlinewidth{0.481800pt}%
\definecolor{currentstroke}{rgb}{1.000000,1.000000,1.000000}%
\pgfsetstrokecolor{currentstroke}%
\pgfsetdash{}{0pt}%
\pgfpathmoveto{\pgfqpoint{3.612075in}{2.128745in}}%
\pgfpathcurveto{\pgfqpoint{3.623126in}{2.128745in}}{\pgfqpoint{3.633725in}{2.133136in}}{\pgfqpoint{3.641538in}{2.140949in}}%
\pgfpathcurveto{\pgfqpoint{3.649352in}{2.148763in}}{\pgfqpoint{3.653742in}{2.159362in}}{\pgfqpoint{3.653742in}{2.170412in}}%
\pgfpathcurveto{\pgfqpoint{3.653742in}{2.181462in}}{\pgfqpoint{3.649352in}{2.192061in}}{\pgfqpoint{3.641538in}{2.199875in}}%
\pgfpathcurveto{\pgfqpoint{3.633725in}{2.207688in}}{\pgfqpoint{3.623126in}{2.212079in}}{\pgfqpoint{3.612075in}{2.212079in}}%
\pgfpathcurveto{\pgfqpoint{3.601025in}{2.212079in}}{\pgfqpoint{3.590426in}{2.207688in}}{\pgfqpoint{3.582613in}{2.199875in}}%
\pgfpathcurveto{\pgfqpoint{3.574799in}{2.192061in}}{\pgfqpoint{3.570409in}{2.181462in}}{\pgfqpoint{3.570409in}{2.170412in}}%
\pgfpathcurveto{\pgfqpoint{3.570409in}{2.159362in}}{\pgfqpoint{3.574799in}{2.148763in}}{\pgfqpoint{3.582613in}{2.140949in}}%
\pgfpathcurveto{\pgfqpoint{3.590426in}{2.133136in}}{\pgfqpoint{3.601025in}{2.128745in}}{\pgfqpoint{3.612075in}{2.128745in}}%
\pgfpathlineto{\pgfqpoint{3.612075in}{2.128745in}}%
\pgfpathclose%
\pgfusepath{stroke,fill}%
\end{pgfscope}%
\begin{pgfscope}%
\pgfpathrectangle{\pgfqpoint{2.963410in}{0.569136in}}{\pgfqpoint{2.177280in}{2.201755in}}%
\pgfusepath{clip}%
\pgfsetbuttcap%
\pgfsetroundjoin%
\definecolor{currentfill}{rgb}{0.172549,0.627451,0.172549}%
\pgfsetfillcolor{currentfill}%
\pgfsetlinewidth{0.481800pt}%
\definecolor{currentstroke}{rgb}{1.000000,1.000000,1.000000}%
\pgfsetstrokecolor{currentstroke}%
\pgfsetdash{}{0pt}%
\pgfpathmoveto{\pgfqpoint{3.907452in}{2.212145in}}%
\pgfpathcurveto{\pgfqpoint{3.918502in}{2.212145in}}{\pgfqpoint{3.929101in}{2.216535in}}{\pgfqpoint{3.936915in}{2.224349in}}%
\pgfpathcurveto{\pgfqpoint{3.944728in}{2.232163in}}{\pgfqpoint{3.949118in}{2.242762in}}{\pgfqpoint{3.949118in}{2.253812in}}%
\pgfpathcurveto{\pgfqpoint{3.949118in}{2.264862in}}{\pgfqpoint{3.944728in}{2.275461in}}{\pgfqpoint{3.936915in}{2.283275in}}%
\pgfpathcurveto{\pgfqpoint{3.929101in}{2.291088in}}{\pgfqpoint{3.918502in}{2.295478in}}{\pgfqpoint{3.907452in}{2.295478in}}%
\pgfpathcurveto{\pgfqpoint{3.896402in}{2.295478in}}{\pgfqpoint{3.885803in}{2.291088in}}{\pgfqpoint{3.877989in}{2.283275in}}%
\pgfpathcurveto{\pgfqpoint{3.870175in}{2.275461in}}{\pgfqpoint{3.865785in}{2.264862in}}{\pgfqpoint{3.865785in}{2.253812in}}%
\pgfpathcurveto{\pgfqpoint{3.865785in}{2.242762in}}{\pgfqpoint{3.870175in}{2.232163in}}{\pgfqpoint{3.877989in}{2.224349in}}%
\pgfpathcurveto{\pgfqpoint{3.885803in}{2.216535in}}{\pgfqpoint{3.896402in}{2.212145in}}{\pgfqpoint{3.907452in}{2.212145in}}%
\pgfpathlineto{\pgfqpoint{3.907452in}{2.212145in}}%
\pgfpathclose%
\pgfusepath{stroke,fill}%
\end{pgfscope}%
\begin{pgfscope}%
\pgfpathrectangle{\pgfqpoint{2.963410in}{0.569136in}}{\pgfqpoint{2.177280in}{2.201755in}}%
\pgfusepath{clip}%
\pgfsetbuttcap%
\pgfsetroundjoin%
\definecolor{currentfill}{rgb}{0.172549,0.627451,0.172549}%
\pgfsetfillcolor{currentfill}%
\pgfsetlinewidth{0.481800pt}%
\definecolor{currentstroke}{rgb}{1.000000,1.000000,1.000000}%
\pgfsetstrokecolor{currentstroke}%
\pgfsetdash{}{0pt}%
\pgfpathmoveto{\pgfqpoint{4.143753in}{2.462345in}}%
\pgfpathcurveto{\pgfqpoint{4.154803in}{2.462345in}}{\pgfqpoint{4.165402in}{2.466735in}}{\pgfqpoint{4.173216in}{2.474548in}}%
\pgfpathcurveto{\pgfqpoint{4.181029in}{2.482362in}}{\pgfqpoint{4.185419in}{2.492961in}}{\pgfqpoint{4.185419in}{2.504011in}}%
\pgfpathcurveto{\pgfqpoint{4.185419in}{2.515061in}}{\pgfqpoint{4.181029in}{2.525660in}}{\pgfqpoint{4.173216in}{2.533474in}}%
\pgfpathcurveto{\pgfqpoint{4.165402in}{2.541288in}}{\pgfqpoint{4.154803in}{2.545678in}}{\pgfqpoint{4.143753in}{2.545678in}}%
\pgfpathcurveto{\pgfqpoint{4.132703in}{2.545678in}}{\pgfqpoint{4.122104in}{2.541288in}}{\pgfqpoint{4.114290in}{2.533474in}}%
\pgfpathcurveto{\pgfqpoint{4.106476in}{2.525660in}}{\pgfqpoint{4.102086in}{2.515061in}}{\pgfqpoint{4.102086in}{2.504011in}}%
\pgfpathcurveto{\pgfqpoint{4.102086in}{2.492961in}}{\pgfqpoint{4.106476in}{2.482362in}}{\pgfqpoint{4.114290in}{2.474548in}}%
\pgfpathcurveto{\pgfqpoint{4.122104in}{2.466735in}}{\pgfqpoint{4.132703in}{2.462345in}}{\pgfqpoint{4.143753in}{2.462345in}}%
\pgfpathlineto{\pgfqpoint{4.143753in}{2.462345in}}%
\pgfpathclose%
\pgfusepath{stroke,fill}%
\end{pgfscope}%
\begin{pgfscope}%
\pgfpathrectangle{\pgfqpoint{2.963410in}{0.569136in}}{\pgfqpoint{2.177280in}{2.201755in}}%
\pgfusepath{clip}%
\pgfsetbuttcap%
\pgfsetroundjoin%
\definecolor{currentfill}{rgb}{0.172549,0.627451,0.172549}%
\pgfsetfillcolor{currentfill}%
\pgfsetlinewidth{0.481800pt}%
\definecolor{currentstroke}{rgb}{1.000000,1.000000,1.000000}%
\pgfsetstrokecolor{currentstroke}%
\pgfsetdash{}{0pt}%
\pgfpathmoveto{\pgfqpoint{3.907452in}{2.045346in}}%
\pgfpathcurveto{\pgfqpoint{3.918502in}{2.045346in}}{\pgfqpoint{3.929101in}{2.049736in}}{\pgfqpoint{3.936915in}{2.057549in}}%
\pgfpathcurveto{\pgfqpoint{3.944728in}{2.065363in}}{\pgfqpoint{3.949118in}{2.075962in}}{\pgfqpoint{3.949118in}{2.087012in}}%
\pgfpathcurveto{\pgfqpoint{3.949118in}{2.098062in}}{\pgfqpoint{3.944728in}{2.108661in}}{\pgfqpoint{3.936915in}{2.116475in}}%
\pgfpathcurveto{\pgfqpoint{3.929101in}{2.124289in}}{\pgfqpoint{3.918502in}{2.128679in}}{\pgfqpoint{3.907452in}{2.128679in}}%
\pgfpathcurveto{\pgfqpoint{3.896402in}{2.128679in}}{\pgfqpoint{3.885803in}{2.124289in}}{\pgfqpoint{3.877989in}{2.116475in}}%
\pgfpathcurveto{\pgfqpoint{3.870175in}{2.108661in}}{\pgfqpoint{3.865785in}{2.098062in}}{\pgfqpoint{3.865785in}{2.087012in}}%
\pgfpathcurveto{\pgfqpoint{3.865785in}{2.075962in}}{\pgfqpoint{3.870175in}{2.065363in}}{\pgfqpoint{3.877989in}{2.057549in}}%
\pgfpathcurveto{\pgfqpoint{3.885803in}{2.049736in}}{\pgfqpoint{3.896402in}{2.045346in}}{\pgfqpoint{3.907452in}{2.045346in}}%
\pgfpathlineto{\pgfqpoint{3.907452in}{2.045346in}}%
\pgfpathclose%
\pgfusepath{stroke,fill}%
\end{pgfscope}%
\begin{pgfscope}%
\pgfpathrectangle{\pgfqpoint{2.963410in}{0.569136in}}{\pgfqpoint{2.177280in}{2.201755in}}%
\pgfusepath{clip}%
\pgfsetbuttcap%
\pgfsetroundjoin%
\definecolor{currentfill}{rgb}{0.121569,0.466667,0.705882}%
\pgfsetfillcolor{currentfill}%
\pgfsetlinewidth{1.003750pt}%
\definecolor{currentstroke}{rgb}{0.121569,0.466667,0.705882}%
\pgfsetstrokecolor{currentstroke}%
\pgfsetdash{}{0pt}%
\pgfsys@defobject{currentmarker}{\pgfqpoint{-0.041667in}{-0.041667in}}{\pgfqpoint{0.041667in}{0.041667in}}{%
\pgfpathmoveto{\pgfqpoint{0.000000in}{-0.041667in}}%
\pgfpathcurveto{\pgfqpoint{0.011050in}{-0.041667in}}{\pgfqpoint{0.021649in}{-0.037276in}}{\pgfqpoint{0.029463in}{-0.029463in}}%
\pgfpathcurveto{\pgfqpoint{0.037276in}{-0.021649in}}{\pgfqpoint{0.041667in}{-0.011050in}}{\pgfqpoint{0.041667in}{0.000000in}}%
\pgfpathcurveto{\pgfqpoint{0.041667in}{0.011050in}}{\pgfqpoint{0.037276in}{0.021649in}}{\pgfqpoint{0.029463in}{0.029463in}}%
\pgfpathcurveto{\pgfqpoint{0.021649in}{0.037276in}}{\pgfqpoint{0.011050in}{0.041667in}}{\pgfqpoint{0.000000in}{0.041667in}}%
\pgfpathcurveto{\pgfqpoint{-0.011050in}{0.041667in}}{\pgfqpoint{-0.021649in}{0.037276in}}{\pgfqpoint{-0.029463in}{0.029463in}}%
\pgfpathcurveto{\pgfqpoint{-0.037276in}{0.021649in}}{\pgfqpoint{-0.041667in}{0.011050in}}{\pgfqpoint{-0.041667in}{0.000000in}}%
\pgfpathcurveto{\pgfqpoint{-0.041667in}{-0.011050in}}{\pgfqpoint{-0.037276in}{-0.021649in}}{\pgfqpoint{-0.029463in}{-0.029463in}}%
\pgfpathcurveto{\pgfqpoint{-0.021649in}{-0.037276in}}{\pgfqpoint{-0.011050in}{-0.041667in}}{\pgfqpoint{0.000000in}{-0.041667in}}%
\pgfpathlineto{\pgfqpoint{0.000000in}{-0.041667in}}%
\pgfpathclose%
\pgfusepath{stroke,fill}%
}%
\end{pgfscope}%
\begin{pgfscope}%
\pgfpathrectangle{\pgfqpoint{2.963410in}{0.569136in}}{\pgfqpoint{2.177280in}{2.201755in}}%
\pgfusepath{clip}%
\pgfsetbuttcap%
\pgfsetroundjoin%
\definecolor{currentfill}{rgb}{1.000000,0.498039,0.054902}%
\pgfsetfillcolor{currentfill}%
\pgfsetlinewidth{1.003750pt}%
\definecolor{currentstroke}{rgb}{1.000000,0.498039,0.054902}%
\pgfsetstrokecolor{currentstroke}%
\pgfsetdash{}{0pt}%
\pgfsys@defobject{currentmarker}{\pgfqpoint{-0.041667in}{-0.041667in}}{\pgfqpoint{0.041667in}{0.041667in}}{%
\pgfpathmoveto{\pgfqpoint{0.000000in}{-0.041667in}}%
\pgfpathcurveto{\pgfqpoint{0.011050in}{-0.041667in}}{\pgfqpoint{0.021649in}{-0.037276in}}{\pgfqpoint{0.029463in}{-0.029463in}}%
\pgfpathcurveto{\pgfqpoint{0.037276in}{-0.021649in}}{\pgfqpoint{0.041667in}{-0.011050in}}{\pgfqpoint{0.041667in}{0.000000in}}%
\pgfpathcurveto{\pgfqpoint{0.041667in}{0.011050in}}{\pgfqpoint{0.037276in}{0.021649in}}{\pgfqpoint{0.029463in}{0.029463in}}%
\pgfpathcurveto{\pgfqpoint{0.021649in}{0.037276in}}{\pgfqpoint{0.011050in}{0.041667in}}{\pgfqpoint{0.000000in}{0.041667in}}%
\pgfpathcurveto{\pgfqpoint{-0.011050in}{0.041667in}}{\pgfqpoint{-0.021649in}{0.037276in}}{\pgfqpoint{-0.029463in}{0.029463in}}%
\pgfpathcurveto{\pgfqpoint{-0.037276in}{0.021649in}}{\pgfqpoint{-0.041667in}{0.011050in}}{\pgfqpoint{-0.041667in}{0.000000in}}%
\pgfpathcurveto{\pgfqpoint{-0.041667in}{-0.011050in}}{\pgfqpoint{-0.037276in}{-0.021649in}}{\pgfqpoint{-0.029463in}{-0.029463in}}%
\pgfpathcurveto{\pgfqpoint{-0.021649in}{-0.037276in}}{\pgfqpoint{-0.011050in}{-0.041667in}}{\pgfqpoint{0.000000in}{-0.041667in}}%
\pgfpathlineto{\pgfqpoint{0.000000in}{-0.041667in}}%
\pgfpathclose%
\pgfusepath{stroke,fill}%
}%
\end{pgfscope}%
\begin{pgfscope}%
\pgfpathrectangle{\pgfqpoint{2.963410in}{0.569136in}}{\pgfqpoint{2.177280in}{2.201755in}}%
\pgfusepath{clip}%
\pgfsetbuttcap%
\pgfsetroundjoin%
\definecolor{currentfill}{rgb}{0.172549,0.627451,0.172549}%
\pgfsetfillcolor{currentfill}%
\pgfsetlinewidth{1.003750pt}%
\definecolor{currentstroke}{rgb}{0.172549,0.627451,0.172549}%
\pgfsetstrokecolor{currentstroke}%
\pgfsetdash{}{0pt}%
\pgfsys@defobject{currentmarker}{\pgfqpoint{-0.041667in}{-0.041667in}}{\pgfqpoint{0.041667in}{0.041667in}}{%
\pgfpathmoveto{\pgfqpoint{0.000000in}{-0.041667in}}%
\pgfpathcurveto{\pgfqpoint{0.011050in}{-0.041667in}}{\pgfqpoint{0.021649in}{-0.037276in}}{\pgfqpoint{0.029463in}{-0.029463in}}%
\pgfpathcurveto{\pgfqpoint{0.037276in}{-0.021649in}}{\pgfqpoint{0.041667in}{-0.011050in}}{\pgfqpoint{0.041667in}{0.000000in}}%
\pgfpathcurveto{\pgfqpoint{0.041667in}{0.011050in}}{\pgfqpoint{0.037276in}{0.021649in}}{\pgfqpoint{0.029463in}{0.029463in}}%
\pgfpathcurveto{\pgfqpoint{0.021649in}{0.037276in}}{\pgfqpoint{0.011050in}{0.041667in}}{\pgfqpoint{0.000000in}{0.041667in}}%
\pgfpathcurveto{\pgfqpoint{-0.011050in}{0.041667in}}{\pgfqpoint{-0.021649in}{0.037276in}}{\pgfqpoint{-0.029463in}{0.029463in}}%
\pgfpathcurveto{\pgfqpoint{-0.037276in}{0.021649in}}{\pgfqpoint{-0.041667in}{0.011050in}}{\pgfqpoint{-0.041667in}{0.000000in}}%
\pgfpathcurveto{\pgfqpoint{-0.041667in}{-0.011050in}}{\pgfqpoint{-0.037276in}{-0.021649in}}{\pgfqpoint{-0.029463in}{-0.029463in}}%
\pgfpathcurveto{\pgfqpoint{-0.021649in}{-0.037276in}}{\pgfqpoint{-0.011050in}{-0.041667in}}{\pgfqpoint{0.000000in}{-0.041667in}}%
\pgfpathlineto{\pgfqpoint{0.000000in}{-0.041667in}}%
\pgfpathclose%
\pgfusepath{stroke,fill}%
}%
\end{pgfscope}%
\begin{pgfscope}%
\pgfsetbuttcap%
\pgfsetroundjoin%
\definecolor{currentfill}{rgb}{0.000000,0.000000,0.000000}%
\pgfsetfillcolor{currentfill}%
\pgfsetlinewidth{0.803000pt}%
\definecolor{currentstroke}{rgb}{0.000000,0.000000,0.000000}%
\pgfsetstrokecolor{currentstroke}%
\pgfsetdash{}{0pt}%
\pgfsys@defobject{currentmarker}{\pgfqpoint{0.000000in}{-0.048611in}}{\pgfqpoint{0.000000in}{0.000000in}}{%
\pgfpathmoveto{\pgfqpoint{0.000000in}{0.000000in}}%
\pgfpathlineto{\pgfqpoint{0.000000in}{-0.048611in}}%
\pgfusepath{stroke,fill}%
}%
\begin{pgfscope}%
\pgfsys@transformshift{3.316699in}{0.569136in}%
\pgfsys@useobject{currentmarker}{}%
\end{pgfscope}%
\end{pgfscope}%
\begin{pgfscope}%
\definecolor{textcolor}{rgb}{0.000000,0.000000,0.000000}%
\pgfsetstrokecolor{textcolor}%
\pgfsetfillcolor{textcolor}%
\pgftext[x=3.316699in,y=0.471913in,,top]{\color{textcolor}\rmfamily\fontsize{10.000000}{12.000000}\selectfont \(\displaystyle {2}\)}%
\end{pgfscope}%
\begin{pgfscope}%
\pgfsetbuttcap%
\pgfsetroundjoin%
\definecolor{currentfill}{rgb}{0.000000,0.000000,0.000000}%
\pgfsetfillcolor{currentfill}%
\pgfsetlinewidth{0.803000pt}%
\definecolor{currentstroke}{rgb}{0.000000,0.000000,0.000000}%
\pgfsetstrokecolor{currentstroke}%
\pgfsetdash{}{0pt}%
\pgfsys@defobject{currentmarker}{\pgfqpoint{0.000000in}{-0.048611in}}{\pgfqpoint{0.000000in}{0.000000in}}{%
\pgfpathmoveto{\pgfqpoint{0.000000in}{0.000000in}}%
\pgfpathlineto{\pgfqpoint{0.000000in}{-0.048611in}}%
\pgfusepath{stroke,fill}%
}%
\begin{pgfscope}%
\pgfsys@transformshift{3.907452in}{0.569136in}%
\pgfsys@useobject{currentmarker}{}%
\end{pgfscope}%
\end{pgfscope}%
\begin{pgfscope}%
\definecolor{textcolor}{rgb}{0.000000,0.000000,0.000000}%
\pgfsetstrokecolor{textcolor}%
\pgfsetfillcolor{textcolor}%
\pgftext[x=3.907452in,y=0.471913in,,top]{\color{textcolor}\rmfamily\fontsize{10.000000}{12.000000}\selectfont \(\displaystyle {3}\)}%
\end{pgfscope}%
\begin{pgfscope}%
\pgfsetbuttcap%
\pgfsetroundjoin%
\definecolor{currentfill}{rgb}{0.000000,0.000000,0.000000}%
\pgfsetfillcolor{currentfill}%
\pgfsetlinewidth{0.803000pt}%
\definecolor{currentstroke}{rgb}{0.000000,0.000000,0.000000}%
\pgfsetstrokecolor{currentstroke}%
\pgfsetdash{}{0pt}%
\pgfsys@defobject{currentmarker}{\pgfqpoint{0.000000in}{-0.048611in}}{\pgfqpoint{0.000000in}{0.000000in}}{%
\pgfpathmoveto{\pgfqpoint{0.000000in}{0.000000in}}%
\pgfpathlineto{\pgfqpoint{0.000000in}{-0.048611in}}%
\pgfusepath{stroke,fill}%
}%
\begin{pgfscope}%
\pgfsys@transformshift{4.498204in}{0.569136in}%
\pgfsys@useobject{currentmarker}{}%
\end{pgfscope}%
\end{pgfscope}%
\begin{pgfscope}%
\definecolor{textcolor}{rgb}{0.000000,0.000000,0.000000}%
\pgfsetstrokecolor{textcolor}%
\pgfsetfillcolor{textcolor}%
\pgftext[x=4.498204in,y=0.471913in,,top]{\color{textcolor}\rmfamily\fontsize{10.000000}{12.000000}\selectfont \(\displaystyle {4}\)}%
\end{pgfscope}%
\begin{pgfscope}%
\pgfsetbuttcap%
\pgfsetroundjoin%
\definecolor{currentfill}{rgb}{0.000000,0.000000,0.000000}%
\pgfsetfillcolor{currentfill}%
\pgfsetlinewidth{0.803000pt}%
\definecolor{currentstroke}{rgb}{0.000000,0.000000,0.000000}%
\pgfsetstrokecolor{currentstroke}%
\pgfsetdash{}{0pt}%
\pgfsys@defobject{currentmarker}{\pgfqpoint{0.000000in}{-0.048611in}}{\pgfqpoint{0.000000in}{0.000000in}}{%
\pgfpathmoveto{\pgfqpoint{0.000000in}{0.000000in}}%
\pgfpathlineto{\pgfqpoint{0.000000in}{-0.048611in}}%
\pgfusepath{stroke,fill}%
}%
\begin{pgfscope}%
\pgfsys@transformshift{5.088957in}{0.569136in}%
\pgfsys@useobject{currentmarker}{}%
\end{pgfscope}%
\end{pgfscope}%
\begin{pgfscope}%
\definecolor{textcolor}{rgb}{0.000000,0.000000,0.000000}%
\pgfsetstrokecolor{textcolor}%
\pgfsetfillcolor{textcolor}%
\pgftext[x=5.088957in,y=0.471913in,,top]{\color{textcolor}\rmfamily\fontsize{10.000000}{12.000000}\selectfont \(\displaystyle {5}\)}%
\end{pgfscope}%
\begin{pgfscope}%
\definecolor{textcolor}{rgb}{0.000000,0.000000,0.000000}%
\pgfsetstrokecolor{textcolor}%
\pgfsetfillcolor{textcolor}%
\pgftext[x=4.052050in,y=0.292901in,,top]{\color{textcolor}\rmfamily\fontsize{10.000000}{12.000000}\selectfont sepal\_width}%
\end{pgfscope}%
\begin{pgfscope}%
\pgfsetbuttcap%
\pgfsetroundjoin%
\definecolor{currentfill}{rgb}{0.000000,0.000000,0.000000}%
\pgfsetfillcolor{currentfill}%
\pgfsetlinewidth{0.803000pt}%
\definecolor{currentstroke}{rgb}{0.000000,0.000000,0.000000}%
\pgfsetstrokecolor{currentstroke}%
\pgfsetdash{}{0pt}%
\pgfsys@defobject{currentmarker}{\pgfqpoint{-0.048611in}{0.000000in}}{\pgfqpoint{-0.000000in}{0.000000in}}{%
\pgfpathmoveto{\pgfqpoint{-0.000000in}{0.000000in}}%
\pgfpathlineto{\pgfqpoint{-0.048611in}{0.000000in}}%
\pgfusepath{stroke,fill}%
}%
\begin{pgfscope}%
\pgfsys@transformshift{2.963410in}{0.585816in}%
\pgfsys@useobject{currentmarker}{}%
\end{pgfscope}%
\end{pgfscope}%
\begin{pgfscope}%
\pgfsetbuttcap%
\pgfsetroundjoin%
\definecolor{currentfill}{rgb}{0.000000,0.000000,0.000000}%
\pgfsetfillcolor{currentfill}%
\pgfsetlinewidth{0.803000pt}%
\definecolor{currentstroke}{rgb}{0.000000,0.000000,0.000000}%
\pgfsetstrokecolor{currentstroke}%
\pgfsetdash{}{0pt}%
\pgfsys@defobject{currentmarker}{\pgfqpoint{-0.048611in}{0.000000in}}{\pgfqpoint{-0.000000in}{0.000000in}}{%
\pgfpathmoveto{\pgfqpoint{-0.000000in}{0.000000in}}%
\pgfpathlineto{\pgfqpoint{-0.048611in}{0.000000in}}%
\pgfusepath{stroke,fill}%
}%
\begin{pgfscope}%
\pgfsys@transformshift{2.963410in}{1.002815in}%
\pgfsys@useobject{currentmarker}{}%
\end{pgfscope}%
\end{pgfscope}%
\begin{pgfscope}%
\pgfsetbuttcap%
\pgfsetroundjoin%
\definecolor{currentfill}{rgb}{0.000000,0.000000,0.000000}%
\pgfsetfillcolor{currentfill}%
\pgfsetlinewidth{0.803000pt}%
\definecolor{currentstroke}{rgb}{0.000000,0.000000,0.000000}%
\pgfsetstrokecolor{currentstroke}%
\pgfsetdash{}{0pt}%
\pgfsys@defobject{currentmarker}{\pgfqpoint{-0.048611in}{0.000000in}}{\pgfqpoint{-0.000000in}{0.000000in}}{%
\pgfpathmoveto{\pgfqpoint{-0.000000in}{0.000000in}}%
\pgfpathlineto{\pgfqpoint{-0.048611in}{0.000000in}}%
\pgfusepath{stroke,fill}%
}%
\begin{pgfscope}%
\pgfsys@transformshift{2.963410in}{1.419814in}%
\pgfsys@useobject{currentmarker}{}%
\end{pgfscope}%
\end{pgfscope}%
\begin{pgfscope}%
\pgfsetbuttcap%
\pgfsetroundjoin%
\definecolor{currentfill}{rgb}{0.000000,0.000000,0.000000}%
\pgfsetfillcolor{currentfill}%
\pgfsetlinewidth{0.803000pt}%
\definecolor{currentstroke}{rgb}{0.000000,0.000000,0.000000}%
\pgfsetstrokecolor{currentstroke}%
\pgfsetdash{}{0pt}%
\pgfsys@defobject{currentmarker}{\pgfqpoint{-0.048611in}{0.000000in}}{\pgfqpoint{-0.000000in}{0.000000in}}{%
\pgfpathmoveto{\pgfqpoint{-0.000000in}{0.000000in}}%
\pgfpathlineto{\pgfqpoint{-0.048611in}{0.000000in}}%
\pgfusepath{stroke,fill}%
}%
\begin{pgfscope}%
\pgfsys@transformshift{2.963410in}{1.836813in}%
\pgfsys@useobject{currentmarker}{}%
\end{pgfscope}%
\end{pgfscope}%
\begin{pgfscope}%
\pgfsetbuttcap%
\pgfsetroundjoin%
\definecolor{currentfill}{rgb}{0.000000,0.000000,0.000000}%
\pgfsetfillcolor{currentfill}%
\pgfsetlinewidth{0.803000pt}%
\definecolor{currentstroke}{rgb}{0.000000,0.000000,0.000000}%
\pgfsetstrokecolor{currentstroke}%
\pgfsetdash{}{0pt}%
\pgfsys@defobject{currentmarker}{\pgfqpoint{-0.048611in}{0.000000in}}{\pgfqpoint{-0.000000in}{0.000000in}}{%
\pgfpathmoveto{\pgfqpoint{-0.000000in}{0.000000in}}%
\pgfpathlineto{\pgfqpoint{-0.048611in}{0.000000in}}%
\pgfusepath{stroke,fill}%
}%
\begin{pgfscope}%
\pgfsys@transformshift{2.963410in}{2.253812in}%
\pgfsys@useobject{currentmarker}{}%
\end{pgfscope}%
\end{pgfscope}%
\begin{pgfscope}%
\pgfsetbuttcap%
\pgfsetroundjoin%
\definecolor{currentfill}{rgb}{0.000000,0.000000,0.000000}%
\pgfsetfillcolor{currentfill}%
\pgfsetlinewidth{0.803000pt}%
\definecolor{currentstroke}{rgb}{0.000000,0.000000,0.000000}%
\pgfsetstrokecolor{currentstroke}%
\pgfsetdash{}{0pt}%
\pgfsys@defobject{currentmarker}{\pgfqpoint{-0.048611in}{0.000000in}}{\pgfqpoint{-0.000000in}{0.000000in}}{%
\pgfpathmoveto{\pgfqpoint{-0.000000in}{0.000000in}}%
\pgfpathlineto{\pgfqpoint{-0.048611in}{0.000000in}}%
\pgfusepath{stroke,fill}%
}%
\begin{pgfscope}%
\pgfsys@transformshift{2.963410in}{2.670811in}%
\pgfsys@useobject{currentmarker}{}%
\end{pgfscope}%
\end{pgfscope}%
\begin{pgfscope}%
\pgfsetrectcap%
\pgfsetmiterjoin%
\pgfsetlinewidth{0.803000pt}%
\definecolor{currentstroke}{rgb}{0.000000,0.000000,0.000000}%
\pgfsetstrokecolor{currentstroke}%
\pgfsetdash{}{0pt}%
\pgfpathmoveto{\pgfqpoint{2.963410in}{0.569136in}}%
\pgfpathlineto{\pgfqpoint{2.963410in}{2.770891in}}%
\pgfusepath{stroke}%
\end{pgfscope}%
\begin{pgfscope}%
\pgfsetrectcap%
\pgfsetmiterjoin%
\pgfsetlinewidth{0.803000pt}%
\definecolor{currentstroke}{rgb}{0.000000,0.000000,0.000000}%
\pgfsetstrokecolor{currentstroke}%
\pgfsetdash{}{0pt}%
\pgfpathmoveto{\pgfqpoint{2.963410in}{0.569136in}}%
\pgfpathlineto{\pgfqpoint{5.140690in}{0.569136in}}%
\pgfusepath{stroke}%
\end{pgfscope}%
\begin{pgfscope}%
\pgfsetbuttcap%
\pgfsetmiterjoin%
\definecolor{currentfill}{rgb}{1.000000,1.000000,1.000000}%
\pgfsetfillcolor{currentfill}%
\pgfsetlinewidth{0.000000pt}%
\definecolor{currentstroke}{rgb}{0.000000,0.000000,0.000000}%
\pgfsetstrokecolor{currentstroke}%
\pgfsetstrokeopacity{0.000000}%
\pgfsetdash{}{0pt}%
\pgfpathmoveto{\pgfqpoint{5.292946in}{0.569136in}}%
\pgfpathlineto{\pgfqpoint{7.470226in}{0.569136in}}%
\pgfpathlineto{\pgfqpoint{7.470226in}{2.770891in}}%
\pgfpathlineto{\pgfqpoint{5.292946in}{2.770891in}}%
\pgfpathlineto{\pgfqpoint{5.292946in}{0.569136in}}%
\pgfpathclose%
\pgfusepath{fill}%
\end{pgfscope}%
\begin{pgfscope}%
\pgfpathrectangle{\pgfqpoint{5.292946in}{0.569136in}}{\pgfqpoint{2.177280in}{2.201755in}}%
\pgfusepath{clip}%
\pgfsetbuttcap%
\pgfsetroundjoin%
\definecolor{currentfill}{rgb}{0.121569,0.466667,0.705882}%
\pgfsetfillcolor{currentfill}%
\pgfsetlinewidth{0.481800pt}%
\definecolor{currentstroke}{rgb}{1.000000,1.000000,1.000000}%
\pgfsetstrokecolor{currentstroke}%
\pgfsetdash{}{0pt}%
\pgfpathmoveto{\pgfqpoint{5.575125in}{0.710948in}}%
\pgfpathcurveto{\pgfqpoint{5.586176in}{0.710948in}}{\pgfqpoint{5.596775in}{0.715339in}}{\pgfqpoint{5.604588in}{0.723152in}}%
\pgfpathcurveto{\pgfqpoint{5.612402in}{0.730966in}}{\pgfqpoint{5.616792in}{0.741565in}}{\pgfqpoint{5.616792in}{0.752615in}}%
\pgfpathcurveto{\pgfqpoint{5.616792in}{0.763665in}}{\pgfqpoint{5.612402in}{0.774264in}}{\pgfqpoint{5.604588in}{0.782078in}}%
\pgfpathcurveto{\pgfqpoint{5.596775in}{0.789892in}}{\pgfqpoint{5.586176in}{0.794282in}}{\pgfqpoint{5.575125in}{0.794282in}}%
\pgfpathcurveto{\pgfqpoint{5.564075in}{0.794282in}}{\pgfqpoint{5.553476in}{0.789892in}}{\pgfqpoint{5.545663in}{0.782078in}}%
\pgfpathcurveto{\pgfqpoint{5.537849in}{0.774264in}}{\pgfqpoint{5.533459in}{0.763665in}}{\pgfqpoint{5.533459in}{0.752615in}}%
\pgfpathcurveto{\pgfqpoint{5.533459in}{0.741565in}}{\pgfqpoint{5.537849in}{0.730966in}}{\pgfqpoint{5.545663in}{0.723152in}}%
\pgfpathcurveto{\pgfqpoint{5.553476in}{0.715339in}}{\pgfqpoint{5.564075in}{0.710948in}}{\pgfqpoint{5.575125in}{0.710948in}}%
\pgfpathlineto{\pgfqpoint{5.575125in}{0.710948in}}%
\pgfpathclose%
\pgfusepath{stroke,fill}%
\end{pgfscope}%
\begin{pgfscope}%
\pgfpathrectangle{\pgfqpoint{5.292946in}{0.569136in}}{\pgfqpoint{2.177280in}{2.201755in}}%
\pgfusepath{clip}%
\pgfsetbuttcap%
\pgfsetroundjoin%
\definecolor{currentfill}{rgb}{0.121569,0.466667,0.705882}%
\pgfsetfillcolor{currentfill}%
\pgfsetlinewidth{0.481800pt}%
\definecolor{currentstroke}{rgb}{1.000000,1.000000,1.000000}%
\pgfsetstrokecolor{currentstroke}%
\pgfsetdash{}{0pt}%
\pgfpathmoveto{\pgfqpoint{5.575125in}{0.710948in}}%
\pgfpathcurveto{\pgfqpoint{5.586176in}{0.710948in}}{\pgfqpoint{5.596775in}{0.715339in}}{\pgfqpoint{5.604588in}{0.723152in}}%
\pgfpathcurveto{\pgfqpoint{5.612402in}{0.730966in}}{\pgfqpoint{5.616792in}{0.741565in}}{\pgfqpoint{5.616792in}{0.752615in}}%
\pgfpathcurveto{\pgfqpoint{5.616792in}{0.763665in}}{\pgfqpoint{5.612402in}{0.774264in}}{\pgfqpoint{5.604588in}{0.782078in}}%
\pgfpathcurveto{\pgfqpoint{5.596775in}{0.789892in}}{\pgfqpoint{5.586176in}{0.794282in}}{\pgfqpoint{5.575125in}{0.794282in}}%
\pgfpathcurveto{\pgfqpoint{5.564075in}{0.794282in}}{\pgfqpoint{5.553476in}{0.789892in}}{\pgfqpoint{5.545663in}{0.782078in}}%
\pgfpathcurveto{\pgfqpoint{5.537849in}{0.774264in}}{\pgfqpoint{5.533459in}{0.763665in}}{\pgfqpoint{5.533459in}{0.752615in}}%
\pgfpathcurveto{\pgfqpoint{5.533459in}{0.741565in}}{\pgfqpoint{5.537849in}{0.730966in}}{\pgfqpoint{5.545663in}{0.723152in}}%
\pgfpathcurveto{\pgfqpoint{5.553476in}{0.715339in}}{\pgfqpoint{5.564075in}{0.710948in}}{\pgfqpoint{5.575125in}{0.710948in}}%
\pgfpathlineto{\pgfqpoint{5.575125in}{0.710948in}}%
\pgfpathclose%
\pgfusepath{stroke,fill}%
\end{pgfscope}%
\begin{pgfscope}%
\pgfpathrectangle{\pgfqpoint{5.292946in}{0.569136in}}{\pgfqpoint{2.177280in}{2.201755in}}%
\pgfusepath{clip}%
\pgfsetbuttcap%
\pgfsetroundjoin%
\definecolor{currentfill}{rgb}{0.121569,0.466667,0.705882}%
\pgfsetfillcolor{currentfill}%
\pgfsetlinewidth{0.481800pt}%
\definecolor{currentstroke}{rgb}{1.000000,1.000000,1.000000}%
\pgfsetstrokecolor{currentstroke}%
\pgfsetdash{}{0pt}%
\pgfpathmoveto{\pgfqpoint{5.546420in}{0.710948in}}%
\pgfpathcurveto{\pgfqpoint{5.557470in}{0.710948in}}{\pgfqpoint{5.568069in}{0.715339in}}{\pgfqpoint{5.575883in}{0.723152in}}%
\pgfpathcurveto{\pgfqpoint{5.583697in}{0.730966in}}{\pgfqpoint{5.588087in}{0.741565in}}{\pgfqpoint{5.588087in}{0.752615in}}%
\pgfpathcurveto{\pgfqpoint{5.588087in}{0.763665in}}{\pgfqpoint{5.583697in}{0.774264in}}{\pgfqpoint{5.575883in}{0.782078in}}%
\pgfpathcurveto{\pgfqpoint{5.568069in}{0.789892in}}{\pgfqpoint{5.557470in}{0.794282in}}{\pgfqpoint{5.546420in}{0.794282in}}%
\pgfpathcurveto{\pgfqpoint{5.535370in}{0.794282in}}{\pgfqpoint{5.524771in}{0.789892in}}{\pgfqpoint{5.516957in}{0.782078in}}%
\pgfpathcurveto{\pgfqpoint{5.509144in}{0.774264in}}{\pgfqpoint{5.504753in}{0.763665in}}{\pgfqpoint{5.504753in}{0.752615in}}%
\pgfpathcurveto{\pgfqpoint{5.504753in}{0.741565in}}{\pgfqpoint{5.509144in}{0.730966in}}{\pgfqpoint{5.516957in}{0.723152in}}%
\pgfpathcurveto{\pgfqpoint{5.524771in}{0.715339in}}{\pgfqpoint{5.535370in}{0.710948in}}{\pgfqpoint{5.546420in}{0.710948in}}%
\pgfpathlineto{\pgfqpoint{5.546420in}{0.710948in}}%
\pgfpathclose%
\pgfusepath{stroke,fill}%
\end{pgfscope}%
\begin{pgfscope}%
\pgfpathrectangle{\pgfqpoint{5.292946in}{0.569136in}}{\pgfqpoint{2.177280in}{2.201755in}}%
\pgfusepath{clip}%
\pgfsetbuttcap%
\pgfsetroundjoin%
\definecolor{currentfill}{rgb}{0.121569,0.466667,0.705882}%
\pgfsetfillcolor{currentfill}%
\pgfsetlinewidth{0.481800pt}%
\definecolor{currentstroke}{rgb}{1.000000,1.000000,1.000000}%
\pgfsetstrokecolor{currentstroke}%
\pgfsetdash{}{0pt}%
\pgfpathmoveto{\pgfqpoint{5.603831in}{0.710948in}}%
\pgfpathcurveto{\pgfqpoint{5.614881in}{0.710948in}}{\pgfqpoint{5.625480in}{0.715339in}}{\pgfqpoint{5.633294in}{0.723152in}}%
\pgfpathcurveto{\pgfqpoint{5.641107in}{0.730966in}}{\pgfqpoint{5.645497in}{0.741565in}}{\pgfqpoint{5.645497in}{0.752615in}}%
\pgfpathcurveto{\pgfqpoint{5.645497in}{0.763665in}}{\pgfqpoint{5.641107in}{0.774264in}}{\pgfqpoint{5.633294in}{0.782078in}}%
\pgfpathcurveto{\pgfqpoint{5.625480in}{0.789892in}}{\pgfqpoint{5.614881in}{0.794282in}}{\pgfqpoint{5.603831in}{0.794282in}}%
\pgfpathcurveto{\pgfqpoint{5.592781in}{0.794282in}}{\pgfqpoint{5.582182in}{0.789892in}}{\pgfqpoint{5.574368in}{0.782078in}}%
\pgfpathcurveto{\pgfqpoint{5.566554in}{0.774264in}}{\pgfqpoint{5.562164in}{0.763665in}}{\pgfqpoint{5.562164in}{0.752615in}}%
\pgfpathcurveto{\pgfqpoint{5.562164in}{0.741565in}}{\pgfqpoint{5.566554in}{0.730966in}}{\pgfqpoint{5.574368in}{0.723152in}}%
\pgfpathcurveto{\pgfqpoint{5.582182in}{0.715339in}}{\pgfqpoint{5.592781in}{0.710948in}}{\pgfqpoint{5.603831in}{0.710948in}}%
\pgfpathlineto{\pgfqpoint{5.603831in}{0.710948in}}%
\pgfpathclose%
\pgfusepath{stroke,fill}%
\end{pgfscope}%
\begin{pgfscope}%
\pgfpathrectangle{\pgfqpoint{5.292946in}{0.569136in}}{\pgfqpoint{2.177280in}{2.201755in}}%
\pgfusepath{clip}%
\pgfsetbuttcap%
\pgfsetroundjoin%
\definecolor{currentfill}{rgb}{0.121569,0.466667,0.705882}%
\pgfsetfillcolor{currentfill}%
\pgfsetlinewidth{0.481800pt}%
\definecolor{currentstroke}{rgb}{1.000000,1.000000,1.000000}%
\pgfsetstrokecolor{currentstroke}%
\pgfsetdash{}{0pt}%
\pgfpathmoveto{\pgfqpoint{5.575125in}{0.710948in}}%
\pgfpathcurveto{\pgfqpoint{5.586176in}{0.710948in}}{\pgfqpoint{5.596775in}{0.715339in}}{\pgfqpoint{5.604588in}{0.723152in}}%
\pgfpathcurveto{\pgfqpoint{5.612402in}{0.730966in}}{\pgfqpoint{5.616792in}{0.741565in}}{\pgfqpoint{5.616792in}{0.752615in}}%
\pgfpathcurveto{\pgfqpoint{5.616792in}{0.763665in}}{\pgfqpoint{5.612402in}{0.774264in}}{\pgfqpoint{5.604588in}{0.782078in}}%
\pgfpathcurveto{\pgfqpoint{5.596775in}{0.789892in}}{\pgfqpoint{5.586176in}{0.794282in}}{\pgfqpoint{5.575125in}{0.794282in}}%
\pgfpathcurveto{\pgfqpoint{5.564075in}{0.794282in}}{\pgfqpoint{5.553476in}{0.789892in}}{\pgfqpoint{5.545663in}{0.782078in}}%
\pgfpathcurveto{\pgfqpoint{5.537849in}{0.774264in}}{\pgfqpoint{5.533459in}{0.763665in}}{\pgfqpoint{5.533459in}{0.752615in}}%
\pgfpathcurveto{\pgfqpoint{5.533459in}{0.741565in}}{\pgfqpoint{5.537849in}{0.730966in}}{\pgfqpoint{5.545663in}{0.723152in}}%
\pgfpathcurveto{\pgfqpoint{5.553476in}{0.715339in}}{\pgfqpoint{5.564075in}{0.710948in}}{\pgfqpoint{5.575125in}{0.710948in}}%
\pgfpathlineto{\pgfqpoint{5.575125in}{0.710948in}}%
\pgfpathclose%
\pgfusepath{stroke,fill}%
\end{pgfscope}%
\begin{pgfscope}%
\pgfpathrectangle{\pgfqpoint{5.292946in}{0.569136in}}{\pgfqpoint{2.177280in}{2.201755in}}%
\pgfusepath{clip}%
\pgfsetbuttcap%
\pgfsetroundjoin%
\definecolor{currentfill}{rgb}{0.121569,0.466667,0.705882}%
\pgfsetfillcolor{currentfill}%
\pgfsetlinewidth{0.481800pt}%
\definecolor{currentstroke}{rgb}{1.000000,1.000000,1.000000}%
\pgfsetstrokecolor{currentstroke}%
\pgfsetdash{}{0pt}%
\pgfpathmoveto{\pgfqpoint{5.661241in}{0.877748in}}%
\pgfpathcurveto{\pgfqpoint{5.672291in}{0.877748in}}{\pgfqpoint{5.682890in}{0.882138in}}{\pgfqpoint{5.690704in}{0.889952in}}%
\pgfpathcurveto{\pgfqpoint{5.698518in}{0.897766in}}{\pgfqpoint{5.702908in}{0.908365in}}{\pgfqpoint{5.702908in}{0.919415in}}%
\pgfpathcurveto{\pgfqpoint{5.702908in}{0.930465in}}{\pgfqpoint{5.698518in}{0.941064in}}{\pgfqpoint{5.690704in}{0.948878in}}%
\pgfpathcurveto{\pgfqpoint{5.682890in}{0.956691in}}{\pgfqpoint{5.672291in}{0.961081in}}{\pgfqpoint{5.661241in}{0.961081in}}%
\pgfpathcurveto{\pgfqpoint{5.650191in}{0.961081in}}{\pgfqpoint{5.639592in}{0.956691in}}{\pgfqpoint{5.631779in}{0.948878in}}%
\pgfpathcurveto{\pgfqpoint{5.623965in}{0.941064in}}{\pgfqpoint{5.619575in}{0.930465in}}{\pgfqpoint{5.619575in}{0.919415in}}%
\pgfpathcurveto{\pgfqpoint{5.619575in}{0.908365in}}{\pgfqpoint{5.623965in}{0.897766in}}{\pgfqpoint{5.631779in}{0.889952in}}%
\pgfpathcurveto{\pgfqpoint{5.639592in}{0.882138in}}{\pgfqpoint{5.650191in}{0.877748in}}{\pgfqpoint{5.661241in}{0.877748in}}%
\pgfpathlineto{\pgfqpoint{5.661241in}{0.877748in}}%
\pgfpathclose%
\pgfusepath{stroke,fill}%
\end{pgfscope}%
\begin{pgfscope}%
\pgfpathrectangle{\pgfqpoint{5.292946in}{0.569136in}}{\pgfqpoint{2.177280in}{2.201755in}}%
\pgfusepath{clip}%
\pgfsetbuttcap%
\pgfsetroundjoin%
\definecolor{currentfill}{rgb}{0.121569,0.466667,0.705882}%
\pgfsetfillcolor{currentfill}%
\pgfsetlinewidth{0.481800pt}%
\definecolor{currentstroke}{rgb}{1.000000,1.000000,1.000000}%
\pgfsetstrokecolor{currentstroke}%
\pgfsetdash{}{0pt}%
\pgfpathmoveto{\pgfqpoint{5.575125in}{0.794348in}}%
\pgfpathcurveto{\pgfqpoint{5.586176in}{0.794348in}}{\pgfqpoint{5.596775in}{0.798739in}}{\pgfqpoint{5.604588in}{0.806552in}}%
\pgfpathcurveto{\pgfqpoint{5.612402in}{0.814366in}}{\pgfqpoint{5.616792in}{0.824965in}}{\pgfqpoint{5.616792in}{0.836015in}}%
\pgfpathcurveto{\pgfqpoint{5.616792in}{0.847065in}}{\pgfqpoint{5.612402in}{0.857664in}}{\pgfqpoint{5.604588in}{0.865478in}}%
\pgfpathcurveto{\pgfqpoint{5.596775in}{0.873291in}}{\pgfqpoint{5.586176in}{0.877682in}}{\pgfqpoint{5.575125in}{0.877682in}}%
\pgfpathcurveto{\pgfqpoint{5.564075in}{0.877682in}}{\pgfqpoint{5.553476in}{0.873291in}}{\pgfqpoint{5.545663in}{0.865478in}}%
\pgfpathcurveto{\pgfqpoint{5.537849in}{0.857664in}}{\pgfqpoint{5.533459in}{0.847065in}}{\pgfqpoint{5.533459in}{0.836015in}}%
\pgfpathcurveto{\pgfqpoint{5.533459in}{0.824965in}}{\pgfqpoint{5.537849in}{0.814366in}}{\pgfqpoint{5.545663in}{0.806552in}}%
\pgfpathcurveto{\pgfqpoint{5.553476in}{0.798739in}}{\pgfqpoint{5.564075in}{0.794348in}}{\pgfqpoint{5.575125in}{0.794348in}}%
\pgfpathlineto{\pgfqpoint{5.575125in}{0.794348in}}%
\pgfpathclose%
\pgfusepath{stroke,fill}%
\end{pgfscope}%
\begin{pgfscope}%
\pgfpathrectangle{\pgfqpoint{5.292946in}{0.569136in}}{\pgfqpoint{2.177280in}{2.201755in}}%
\pgfusepath{clip}%
\pgfsetbuttcap%
\pgfsetroundjoin%
\definecolor{currentfill}{rgb}{0.121569,0.466667,0.705882}%
\pgfsetfillcolor{currentfill}%
\pgfsetlinewidth{0.481800pt}%
\definecolor{currentstroke}{rgb}{1.000000,1.000000,1.000000}%
\pgfsetstrokecolor{currentstroke}%
\pgfsetdash{}{0pt}%
\pgfpathmoveto{\pgfqpoint{5.603831in}{0.710948in}}%
\pgfpathcurveto{\pgfqpoint{5.614881in}{0.710948in}}{\pgfqpoint{5.625480in}{0.715339in}}{\pgfqpoint{5.633294in}{0.723152in}}%
\pgfpathcurveto{\pgfqpoint{5.641107in}{0.730966in}}{\pgfqpoint{5.645497in}{0.741565in}}{\pgfqpoint{5.645497in}{0.752615in}}%
\pgfpathcurveto{\pgfqpoint{5.645497in}{0.763665in}}{\pgfqpoint{5.641107in}{0.774264in}}{\pgfqpoint{5.633294in}{0.782078in}}%
\pgfpathcurveto{\pgfqpoint{5.625480in}{0.789892in}}{\pgfqpoint{5.614881in}{0.794282in}}{\pgfqpoint{5.603831in}{0.794282in}}%
\pgfpathcurveto{\pgfqpoint{5.592781in}{0.794282in}}{\pgfqpoint{5.582182in}{0.789892in}}{\pgfqpoint{5.574368in}{0.782078in}}%
\pgfpathcurveto{\pgfqpoint{5.566554in}{0.774264in}}{\pgfqpoint{5.562164in}{0.763665in}}{\pgfqpoint{5.562164in}{0.752615in}}%
\pgfpathcurveto{\pgfqpoint{5.562164in}{0.741565in}}{\pgfqpoint{5.566554in}{0.730966in}}{\pgfqpoint{5.574368in}{0.723152in}}%
\pgfpathcurveto{\pgfqpoint{5.582182in}{0.715339in}}{\pgfqpoint{5.592781in}{0.710948in}}{\pgfqpoint{5.603831in}{0.710948in}}%
\pgfpathlineto{\pgfqpoint{5.603831in}{0.710948in}}%
\pgfpathclose%
\pgfusepath{stroke,fill}%
\end{pgfscope}%
\begin{pgfscope}%
\pgfpathrectangle{\pgfqpoint{5.292946in}{0.569136in}}{\pgfqpoint{2.177280in}{2.201755in}}%
\pgfusepath{clip}%
\pgfsetbuttcap%
\pgfsetroundjoin%
\definecolor{currentfill}{rgb}{0.121569,0.466667,0.705882}%
\pgfsetfillcolor{currentfill}%
\pgfsetlinewidth{0.481800pt}%
\definecolor{currentstroke}{rgb}{1.000000,1.000000,1.000000}%
\pgfsetstrokecolor{currentstroke}%
\pgfsetdash{}{0pt}%
\pgfpathmoveto{\pgfqpoint{5.575125in}{0.710948in}}%
\pgfpathcurveto{\pgfqpoint{5.586176in}{0.710948in}}{\pgfqpoint{5.596775in}{0.715339in}}{\pgfqpoint{5.604588in}{0.723152in}}%
\pgfpathcurveto{\pgfqpoint{5.612402in}{0.730966in}}{\pgfqpoint{5.616792in}{0.741565in}}{\pgfqpoint{5.616792in}{0.752615in}}%
\pgfpathcurveto{\pgfqpoint{5.616792in}{0.763665in}}{\pgfqpoint{5.612402in}{0.774264in}}{\pgfqpoint{5.604588in}{0.782078in}}%
\pgfpathcurveto{\pgfqpoint{5.596775in}{0.789892in}}{\pgfqpoint{5.586176in}{0.794282in}}{\pgfqpoint{5.575125in}{0.794282in}}%
\pgfpathcurveto{\pgfqpoint{5.564075in}{0.794282in}}{\pgfqpoint{5.553476in}{0.789892in}}{\pgfqpoint{5.545663in}{0.782078in}}%
\pgfpathcurveto{\pgfqpoint{5.537849in}{0.774264in}}{\pgfqpoint{5.533459in}{0.763665in}}{\pgfqpoint{5.533459in}{0.752615in}}%
\pgfpathcurveto{\pgfqpoint{5.533459in}{0.741565in}}{\pgfqpoint{5.537849in}{0.730966in}}{\pgfqpoint{5.545663in}{0.723152in}}%
\pgfpathcurveto{\pgfqpoint{5.553476in}{0.715339in}}{\pgfqpoint{5.564075in}{0.710948in}}{\pgfqpoint{5.575125in}{0.710948in}}%
\pgfpathlineto{\pgfqpoint{5.575125in}{0.710948in}}%
\pgfpathclose%
\pgfusepath{stroke,fill}%
\end{pgfscope}%
\begin{pgfscope}%
\pgfpathrectangle{\pgfqpoint{5.292946in}{0.569136in}}{\pgfqpoint{2.177280in}{2.201755in}}%
\pgfusepath{clip}%
\pgfsetbuttcap%
\pgfsetroundjoin%
\definecolor{currentfill}{rgb}{0.121569,0.466667,0.705882}%
\pgfsetfillcolor{currentfill}%
\pgfsetlinewidth{0.481800pt}%
\definecolor{currentstroke}{rgb}{1.000000,1.000000,1.000000}%
\pgfsetstrokecolor{currentstroke}%
\pgfsetdash{}{0pt}%
\pgfpathmoveto{\pgfqpoint{5.603831in}{0.627549in}}%
\pgfpathcurveto{\pgfqpoint{5.614881in}{0.627549in}}{\pgfqpoint{5.625480in}{0.631939in}}{\pgfqpoint{5.633294in}{0.639753in}}%
\pgfpathcurveto{\pgfqpoint{5.641107in}{0.647566in}}{\pgfqpoint{5.645497in}{0.658165in}}{\pgfqpoint{5.645497in}{0.669215in}}%
\pgfpathcurveto{\pgfqpoint{5.645497in}{0.680265in}}{\pgfqpoint{5.641107in}{0.690864in}}{\pgfqpoint{5.633294in}{0.698678in}}%
\pgfpathcurveto{\pgfqpoint{5.625480in}{0.706492in}}{\pgfqpoint{5.614881in}{0.710882in}}{\pgfqpoint{5.603831in}{0.710882in}}%
\pgfpathcurveto{\pgfqpoint{5.592781in}{0.710882in}}{\pgfqpoint{5.582182in}{0.706492in}}{\pgfqpoint{5.574368in}{0.698678in}}%
\pgfpathcurveto{\pgfqpoint{5.566554in}{0.690864in}}{\pgfqpoint{5.562164in}{0.680265in}}{\pgfqpoint{5.562164in}{0.669215in}}%
\pgfpathcurveto{\pgfqpoint{5.562164in}{0.658165in}}{\pgfqpoint{5.566554in}{0.647566in}}{\pgfqpoint{5.574368in}{0.639753in}}%
\pgfpathcurveto{\pgfqpoint{5.582182in}{0.631939in}}{\pgfqpoint{5.592781in}{0.627549in}}{\pgfqpoint{5.603831in}{0.627549in}}%
\pgfpathlineto{\pgfqpoint{5.603831in}{0.627549in}}%
\pgfpathclose%
\pgfusepath{stroke,fill}%
\end{pgfscope}%
\begin{pgfscope}%
\pgfpathrectangle{\pgfqpoint{5.292946in}{0.569136in}}{\pgfqpoint{2.177280in}{2.201755in}}%
\pgfusepath{clip}%
\pgfsetbuttcap%
\pgfsetroundjoin%
\definecolor{currentfill}{rgb}{0.121569,0.466667,0.705882}%
\pgfsetfillcolor{currentfill}%
\pgfsetlinewidth{0.481800pt}%
\definecolor{currentstroke}{rgb}{1.000000,1.000000,1.000000}%
\pgfsetstrokecolor{currentstroke}%
\pgfsetdash{}{0pt}%
\pgfpathmoveto{\pgfqpoint{5.603831in}{0.710948in}}%
\pgfpathcurveto{\pgfqpoint{5.614881in}{0.710948in}}{\pgfqpoint{5.625480in}{0.715339in}}{\pgfqpoint{5.633294in}{0.723152in}}%
\pgfpathcurveto{\pgfqpoint{5.641107in}{0.730966in}}{\pgfqpoint{5.645497in}{0.741565in}}{\pgfqpoint{5.645497in}{0.752615in}}%
\pgfpathcurveto{\pgfqpoint{5.645497in}{0.763665in}}{\pgfqpoint{5.641107in}{0.774264in}}{\pgfqpoint{5.633294in}{0.782078in}}%
\pgfpathcurveto{\pgfqpoint{5.625480in}{0.789892in}}{\pgfqpoint{5.614881in}{0.794282in}}{\pgfqpoint{5.603831in}{0.794282in}}%
\pgfpathcurveto{\pgfqpoint{5.592781in}{0.794282in}}{\pgfqpoint{5.582182in}{0.789892in}}{\pgfqpoint{5.574368in}{0.782078in}}%
\pgfpathcurveto{\pgfqpoint{5.566554in}{0.774264in}}{\pgfqpoint{5.562164in}{0.763665in}}{\pgfqpoint{5.562164in}{0.752615in}}%
\pgfpathcurveto{\pgfqpoint{5.562164in}{0.741565in}}{\pgfqpoint{5.566554in}{0.730966in}}{\pgfqpoint{5.574368in}{0.723152in}}%
\pgfpathcurveto{\pgfqpoint{5.582182in}{0.715339in}}{\pgfqpoint{5.592781in}{0.710948in}}{\pgfqpoint{5.603831in}{0.710948in}}%
\pgfpathlineto{\pgfqpoint{5.603831in}{0.710948in}}%
\pgfpathclose%
\pgfusepath{stroke,fill}%
\end{pgfscope}%
\begin{pgfscope}%
\pgfpathrectangle{\pgfqpoint{5.292946in}{0.569136in}}{\pgfqpoint{2.177280in}{2.201755in}}%
\pgfusepath{clip}%
\pgfsetbuttcap%
\pgfsetroundjoin%
\definecolor{currentfill}{rgb}{0.121569,0.466667,0.705882}%
\pgfsetfillcolor{currentfill}%
\pgfsetlinewidth{0.481800pt}%
\definecolor{currentstroke}{rgb}{1.000000,1.000000,1.000000}%
\pgfsetstrokecolor{currentstroke}%
\pgfsetdash{}{0pt}%
\pgfpathmoveto{\pgfqpoint{5.632536in}{0.710948in}}%
\pgfpathcurveto{\pgfqpoint{5.643586in}{0.710948in}}{\pgfqpoint{5.654185in}{0.715339in}}{\pgfqpoint{5.661999in}{0.723152in}}%
\pgfpathcurveto{\pgfqpoint{5.669812in}{0.730966in}}{\pgfqpoint{5.674203in}{0.741565in}}{\pgfqpoint{5.674203in}{0.752615in}}%
\pgfpathcurveto{\pgfqpoint{5.674203in}{0.763665in}}{\pgfqpoint{5.669812in}{0.774264in}}{\pgfqpoint{5.661999in}{0.782078in}}%
\pgfpathcurveto{\pgfqpoint{5.654185in}{0.789892in}}{\pgfqpoint{5.643586in}{0.794282in}}{\pgfqpoint{5.632536in}{0.794282in}}%
\pgfpathcurveto{\pgfqpoint{5.621486in}{0.794282in}}{\pgfqpoint{5.610887in}{0.789892in}}{\pgfqpoint{5.603073in}{0.782078in}}%
\pgfpathcurveto{\pgfqpoint{5.595260in}{0.774264in}}{\pgfqpoint{5.590869in}{0.763665in}}{\pgfqpoint{5.590869in}{0.752615in}}%
\pgfpathcurveto{\pgfqpoint{5.590869in}{0.741565in}}{\pgfqpoint{5.595260in}{0.730966in}}{\pgfqpoint{5.603073in}{0.723152in}}%
\pgfpathcurveto{\pgfqpoint{5.610887in}{0.715339in}}{\pgfqpoint{5.621486in}{0.710948in}}{\pgfqpoint{5.632536in}{0.710948in}}%
\pgfpathlineto{\pgfqpoint{5.632536in}{0.710948in}}%
\pgfpathclose%
\pgfusepath{stroke,fill}%
\end{pgfscope}%
\begin{pgfscope}%
\pgfpathrectangle{\pgfqpoint{5.292946in}{0.569136in}}{\pgfqpoint{2.177280in}{2.201755in}}%
\pgfusepath{clip}%
\pgfsetbuttcap%
\pgfsetroundjoin%
\definecolor{currentfill}{rgb}{0.121569,0.466667,0.705882}%
\pgfsetfillcolor{currentfill}%
\pgfsetlinewidth{0.481800pt}%
\definecolor{currentstroke}{rgb}{1.000000,1.000000,1.000000}%
\pgfsetstrokecolor{currentstroke}%
\pgfsetdash{}{0pt}%
\pgfpathmoveto{\pgfqpoint{5.575125in}{0.627549in}}%
\pgfpathcurveto{\pgfqpoint{5.586176in}{0.627549in}}{\pgfqpoint{5.596775in}{0.631939in}}{\pgfqpoint{5.604588in}{0.639753in}}%
\pgfpathcurveto{\pgfqpoint{5.612402in}{0.647566in}}{\pgfqpoint{5.616792in}{0.658165in}}{\pgfqpoint{5.616792in}{0.669215in}}%
\pgfpathcurveto{\pgfqpoint{5.616792in}{0.680265in}}{\pgfqpoint{5.612402in}{0.690864in}}{\pgfqpoint{5.604588in}{0.698678in}}%
\pgfpathcurveto{\pgfqpoint{5.596775in}{0.706492in}}{\pgfqpoint{5.586176in}{0.710882in}}{\pgfqpoint{5.575125in}{0.710882in}}%
\pgfpathcurveto{\pgfqpoint{5.564075in}{0.710882in}}{\pgfqpoint{5.553476in}{0.706492in}}{\pgfqpoint{5.545663in}{0.698678in}}%
\pgfpathcurveto{\pgfqpoint{5.537849in}{0.690864in}}{\pgfqpoint{5.533459in}{0.680265in}}{\pgfqpoint{5.533459in}{0.669215in}}%
\pgfpathcurveto{\pgfqpoint{5.533459in}{0.658165in}}{\pgfqpoint{5.537849in}{0.647566in}}{\pgfqpoint{5.545663in}{0.639753in}}%
\pgfpathcurveto{\pgfqpoint{5.553476in}{0.631939in}}{\pgfqpoint{5.564075in}{0.627549in}}{\pgfqpoint{5.575125in}{0.627549in}}%
\pgfpathlineto{\pgfqpoint{5.575125in}{0.627549in}}%
\pgfpathclose%
\pgfusepath{stroke,fill}%
\end{pgfscope}%
\begin{pgfscope}%
\pgfpathrectangle{\pgfqpoint{5.292946in}{0.569136in}}{\pgfqpoint{2.177280in}{2.201755in}}%
\pgfusepath{clip}%
\pgfsetbuttcap%
\pgfsetroundjoin%
\definecolor{currentfill}{rgb}{0.121569,0.466667,0.705882}%
\pgfsetfillcolor{currentfill}%
\pgfsetlinewidth{0.481800pt}%
\definecolor{currentstroke}{rgb}{1.000000,1.000000,1.000000}%
\pgfsetstrokecolor{currentstroke}%
\pgfsetdash{}{0pt}%
\pgfpathmoveto{\pgfqpoint{5.489010in}{0.627549in}}%
\pgfpathcurveto{\pgfqpoint{5.500060in}{0.627549in}}{\pgfqpoint{5.510659in}{0.631939in}}{\pgfqpoint{5.518472in}{0.639753in}}%
\pgfpathcurveto{\pgfqpoint{5.526286in}{0.647566in}}{\pgfqpoint{5.530676in}{0.658165in}}{\pgfqpoint{5.530676in}{0.669215in}}%
\pgfpathcurveto{\pgfqpoint{5.530676in}{0.680265in}}{\pgfqpoint{5.526286in}{0.690864in}}{\pgfqpoint{5.518472in}{0.698678in}}%
\pgfpathcurveto{\pgfqpoint{5.510659in}{0.706492in}}{\pgfqpoint{5.500060in}{0.710882in}}{\pgfqpoint{5.489010in}{0.710882in}}%
\pgfpathcurveto{\pgfqpoint{5.477959in}{0.710882in}}{\pgfqpoint{5.467360in}{0.706492in}}{\pgfqpoint{5.459547in}{0.698678in}}%
\pgfpathcurveto{\pgfqpoint{5.451733in}{0.690864in}}{\pgfqpoint{5.447343in}{0.680265in}}{\pgfqpoint{5.447343in}{0.669215in}}%
\pgfpathcurveto{\pgfqpoint{5.447343in}{0.658165in}}{\pgfqpoint{5.451733in}{0.647566in}}{\pgfqpoint{5.459547in}{0.639753in}}%
\pgfpathcurveto{\pgfqpoint{5.467360in}{0.631939in}}{\pgfqpoint{5.477959in}{0.627549in}}{\pgfqpoint{5.489010in}{0.627549in}}%
\pgfpathlineto{\pgfqpoint{5.489010in}{0.627549in}}%
\pgfpathclose%
\pgfusepath{stroke,fill}%
\end{pgfscope}%
\begin{pgfscope}%
\pgfpathrectangle{\pgfqpoint{5.292946in}{0.569136in}}{\pgfqpoint{2.177280in}{2.201755in}}%
\pgfusepath{clip}%
\pgfsetbuttcap%
\pgfsetroundjoin%
\definecolor{currentfill}{rgb}{0.121569,0.466667,0.705882}%
\pgfsetfillcolor{currentfill}%
\pgfsetlinewidth{0.481800pt}%
\definecolor{currentstroke}{rgb}{1.000000,1.000000,1.000000}%
\pgfsetstrokecolor{currentstroke}%
\pgfsetdash{}{0pt}%
\pgfpathmoveto{\pgfqpoint{5.517715in}{0.710948in}}%
\pgfpathcurveto{\pgfqpoint{5.528765in}{0.710948in}}{\pgfqpoint{5.539364in}{0.715339in}}{\pgfqpoint{5.547178in}{0.723152in}}%
\pgfpathcurveto{\pgfqpoint{5.554991in}{0.730966in}}{\pgfqpoint{5.559382in}{0.741565in}}{\pgfqpoint{5.559382in}{0.752615in}}%
\pgfpathcurveto{\pgfqpoint{5.559382in}{0.763665in}}{\pgfqpoint{5.554991in}{0.774264in}}{\pgfqpoint{5.547178in}{0.782078in}}%
\pgfpathcurveto{\pgfqpoint{5.539364in}{0.789892in}}{\pgfqpoint{5.528765in}{0.794282in}}{\pgfqpoint{5.517715in}{0.794282in}}%
\pgfpathcurveto{\pgfqpoint{5.506665in}{0.794282in}}{\pgfqpoint{5.496066in}{0.789892in}}{\pgfqpoint{5.488252in}{0.782078in}}%
\pgfpathcurveto{\pgfqpoint{5.480438in}{0.774264in}}{\pgfqpoint{5.476048in}{0.763665in}}{\pgfqpoint{5.476048in}{0.752615in}}%
\pgfpathcurveto{\pgfqpoint{5.476048in}{0.741565in}}{\pgfqpoint{5.480438in}{0.730966in}}{\pgfqpoint{5.488252in}{0.723152in}}%
\pgfpathcurveto{\pgfqpoint{5.496066in}{0.715339in}}{\pgfqpoint{5.506665in}{0.710948in}}{\pgfqpoint{5.517715in}{0.710948in}}%
\pgfpathlineto{\pgfqpoint{5.517715in}{0.710948in}}%
\pgfpathclose%
\pgfusepath{stroke,fill}%
\end{pgfscope}%
\begin{pgfscope}%
\pgfpathrectangle{\pgfqpoint{5.292946in}{0.569136in}}{\pgfqpoint{2.177280in}{2.201755in}}%
\pgfusepath{clip}%
\pgfsetbuttcap%
\pgfsetroundjoin%
\definecolor{currentfill}{rgb}{0.121569,0.466667,0.705882}%
\pgfsetfillcolor{currentfill}%
\pgfsetlinewidth{0.481800pt}%
\definecolor{currentstroke}{rgb}{1.000000,1.000000,1.000000}%
\pgfsetstrokecolor{currentstroke}%
\pgfsetdash{}{0pt}%
\pgfpathmoveto{\pgfqpoint{5.603831in}{0.877748in}}%
\pgfpathcurveto{\pgfqpoint{5.614881in}{0.877748in}}{\pgfqpoint{5.625480in}{0.882138in}}{\pgfqpoint{5.633294in}{0.889952in}}%
\pgfpathcurveto{\pgfqpoint{5.641107in}{0.897766in}}{\pgfqpoint{5.645497in}{0.908365in}}{\pgfqpoint{5.645497in}{0.919415in}}%
\pgfpathcurveto{\pgfqpoint{5.645497in}{0.930465in}}{\pgfqpoint{5.641107in}{0.941064in}}{\pgfqpoint{5.633294in}{0.948878in}}%
\pgfpathcurveto{\pgfqpoint{5.625480in}{0.956691in}}{\pgfqpoint{5.614881in}{0.961081in}}{\pgfqpoint{5.603831in}{0.961081in}}%
\pgfpathcurveto{\pgfqpoint{5.592781in}{0.961081in}}{\pgfqpoint{5.582182in}{0.956691in}}{\pgfqpoint{5.574368in}{0.948878in}}%
\pgfpathcurveto{\pgfqpoint{5.566554in}{0.941064in}}{\pgfqpoint{5.562164in}{0.930465in}}{\pgfqpoint{5.562164in}{0.919415in}}%
\pgfpathcurveto{\pgfqpoint{5.562164in}{0.908365in}}{\pgfqpoint{5.566554in}{0.897766in}}{\pgfqpoint{5.574368in}{0.889952in}}%
\pgfpathcurveto{\pgfqpoint{5.582182in}{0.882138in}}{\pgfqpoint{5.592781in}{0.877748in}}{\pgfqpoint{5.603831in}{0.877748in}}%
\pgfpathlineto{\pgfqpoint{5.603831in}{0.877748in}}%
\pgfpathclose%
\pgfusepath{stroke,fill}%
\end{pgfscope}%
\begin{pgfscope}%
\pgfpathrectangle{\pgfqpoint{5.292946in}{0.569136in}}{\pgfqpoint{2.177280in}{2.201755in}}%
\pgfusepath{clip}%
\pgfsetbuttcap%
\pgfsetroundjoin%
\definecolor{currentfill}{rgb}{0.121569,0.466667,0.705882}%
\pgfsetfillcolor{currentfill}%
\pgfsetlinewidth{0.481800pt}%
\definecolor{currentstroke}{rgb}{1.000000,1.000000,1.000000}%
\pgfsetstrokecolor{currentstroke}%
\pgfsetdash{}{0pt}%
\pgfpathmoveto{\pgfqpoint{5.546420in}{0.877748in}}%
\pgfpathcurveto{\pgfqpoint{5.557470in}{0.877748in}}{\pgfqpoint{5.568069in}{0.882138in}}{\pgfqpoint{5.575883in}{0.889952in}}%
\pgfpathcurveto{\pgfqpoint{5.583697in}{0.897766in}}{\pgfqpoint{5.588087in}{0.908365in}}{\pgfqpoint{5.588087in}{0.919415in}}%
\pgfpathcurveto{\pgfqpoint{5.588087in}{0.930465in}}{\pgfqpoint{5.583697in}{0.941064in}}{\pgfqpoint{5.575883in}{0.948878in}}%
\pgfpathcurveto{\pgfqpoint{5.568069in}{0.956691in}}{\pgfqpoint{5.557470in}{0.961081in}}{\pgfqpoint{5.546420in}{0.961081in}}%
\pgfpathcurveto{\pgfqpoint{5.535370in}{0.961081in}}{\pgfqpoint{5.524771in}{0.956691in}}{\pgfqpoint{5.516957in}{0.948878in}}%
\pgfpathcurveto{\pgfqpoint{5.509144in}{0.941064in}}{\pgfqpoint{5.504753in}{0.930465in}}{\pgfqpoint{5.504753in}{0.919415in}}%
\pgfpathcurveto{\pgfqpoint{5.504753in}{0.908365in}}{\pgfqpoint{5.509144in}{0.897766in}}{\pgfqpoint{5.516957in}{0.889952in}}%
\pgfpathcurveto{\pgfqpoint{5.524771in}{0.882138in}}{\pgfqpoint{5.535370in}{0.877748in}}{\pgfqpoint{5.546420in}{0.877748in}}%
\pgfpathlineto{\pgfqpoint{5.546420in}{0.877748in}}%
\pgfpathclose%
\pgfusepath{stroke,fill}%
\end{pgfscope}%
\begin{pgfscope}%
\pgfpathrectangle{\pgfqpoint{5.292946in}{0.569136in}}{\pgfqpoint{2.177280in}{2.201755in}}%
\pgfusepath{clip}%
\pgfsetbuttcap%
\pgfsetroundjoin%
\definecolor{currentfill}{rgb}{0.121569,0.466667,0.705882}%
\pgfsetfillcolor{currentfill}%
\pgfsetlinewidth{0.481800pt}%
\definecolor{currentstroke}{rgb}{1.000000,1.000000,1.000000}%
\pgfsetstrokecolor{currentstroke}%
\pgfsetdash{}{0pt}%
\pgfpathmoveto{\pgfqpoint{5.575125in}{0.794348in}}%
\pgfpathcurveto{\pgfqpoint{5.586176in}{0.794348in}}{\pgfqpoint{5.596775in}{0.798739in}}{\pgfqpoint{5.604588in}{0.806552in}}%
\pgfpathcurveto{\pgfqpoint{5.612402in}{0.814366in}}{\pgfqpoint{5.616792in}{0.824965in}}{\pgfqpoint{5.616792in}{0.836015in}}%
\pgfpathcurveto{\pgfqpoint{5.616792in}{0.847065in}}{\pgfqpoint{5.612402in}{0.857664in}}{\pgfqpoint{5.604588in}{0.865478in}}%
\pgfpathcurveto{\pgfqpoint{5.596775in}{0.873291in}}{\pgfqpoint{5.586176in}{0.877682in}}{\pgfqpoint{5.575125in}{0.877682in}}%
\pgfpathcurveto{\pgfqpoint{5.564075in}{0.877682in}}{\pgfqpoint{5.553476in}{0.873291in}}{\pgfqpoint{5.545663in}{0.865478in}}%
\pgfpathcurveto{\pgfqpoint{5.537849in}{0.857664in}}{\pgfqpoint{5.533459in}{0.847065in}}{\pgfqpoint{5.533459in}{0.836015in}}%
\pgfpathcurveto{\pgfqpoint{5.533459in}{0.824965in}}{\pgfqpoint{5.537849in}{0.814366in}}{\pgfqpoint{5.545663in}{0.806552in}}%
\pgfpathcurveto{\pgfqpoint{5.553476in}{0.798739in}}{\pgfqpoint{5.564075in}{0.794348in}}{\pgfqpoint{5.575125in}{0.794348in}}%
\pgfpathlineto{\pgfqpoint{5.575125in}{0.794348in}}%
\pgfpathclose%
\pgfusepath{stroke,fill}%
\end{pgfscope}%
\begin{pgfscope}%
\pgfpathrectangle{\pgfqpoint{5.292946in}{0.569136in}}{\pgfqpoint{2.177280in}{2.201755in}}%
\pgfusepath{clip}%
\pgfsetbuttcap%
\pgfsetroundjoin%
\definecolor{currentfill}{rgb}{0.121569,0.466667,0.705882}%
\pgfsetfillcolor{currentfill}%
\pgfsetlinewidth{0.481800pt}%
\definecolor{currentstroke}{rgb}{1.000000,1.000000,1.000000}%
\pgfsetstrokecolor{currentstroke}%
\pgfsetdash{}{0pt}%
\pgfpathmoveto{\pgfqpoint{5.661241in}{0.794348in}}%
\pgfpathcurveto{\pgfqpoint{5.672291in}{0.794348in}}{\pgfqpoint{5.682890in}{0.798739in}}{\pgfqpoint{5.690704in}{0.806552in}}%
\pgfpathcurveto{\pgfqpoint{5.698518in}{0.814366in}}{\pgfqpoint{5.702908in}{0.824965in}}{\pgfqpoint{5.702908in}{0.836015in}}%
\pgfpathcurveto{\pgfqpoint{5.702908in}{0.847065in}}{\pgfqpoint{5.698518in}{0.857664in}}{\pgfqpoint{5.690704in}{0.865478in}}%
\pgfpathcurveto{\pgfqpoint{5.682890in}{0.873291in}}{\pgfqpoint{5.672291in}{0.877682in}}{\pgfqpoint{5.661241in}{0.877682in}}%
\pgfpathcurveto{\pgfqpoint{5.650191in}{0.877682in}}{\pgfqpoint{5.639592in}{0.873291in}}{\pgfqpoint{5.631779in}{0.865478in}}%
\pgfpathcurveto{\pgfqpoint{5.623965in}{0.857664in}}{\pgfqpoint{5.619575in}{0.847065in}}{\pgfqpoint{5.619575in}{0.836015in}}%
\pgfpathcurveto{\pgfqpoint{5.619575in}{0.824965in}}{\pgfqpoint{5.623965in}{0.814366in}}{\pgfqpoint{5.631779in}{0.806552in}}%
\pgfpathcurveto{\pgfqpoint{5.639592in}{0.798739in}}{\pgfqpoint{5.650191in}{0.794348in}}{\pgfqpoint{5.661241in}{0.794348in}}%
\pgfpathlineto{\pgfqpoint{5.661241in}{0.794348in}}%
\pgfpathclose%
\pgfusepath{stroke,fill}%
\end{pgfscope}%
\begin{pgfscope}%
\pgfpathrectangle{\pgfqpoint{5.292946in}{0.569136in}}{\pgfqpoint{2.177280in}{2.201755in}}%
\pgfusepath{clip}%
\pgfsetbuttcap%
\pgfsetroundjoin%
\definecolor{currentfill}{rgb}{0.121569,0.466667,0.705882}%
\pgfsetfillcolor{currentfill}%
\pgfsetlinewidth{0.481800pt}%
\definecolor{currentstroke}{rgb}{1.000000,1.000000,1.000000}%
\pgfsetstrokecolor{currentstroke}%
\pgfsetdash{}{0pt}%
\pgfpathmoveto{\pgfqpoint{5.603831in}{0.794348in}}%
\pgfpathcurveto{\pgfqpoint{5.614881in}{0.794348in}}{\pgfqpoint{5.625480in}{0.798739in}}{\pgfqpoint{5.633294in}{0.806552in}}%
\pgfpathcurveto{\pgfqpoint{5.641107in}{0.814366in}}{\pgfqpoint{5.645497in}{0.824965in}}{\pgfqpoint{5.645497in}{0.836015in}}%
\pgfpathcurveto{\pgfqpoint{5.645497in}{0.847065in}}{\pgfqpoint{5.641107in}{0.857664in}}{\pgfqpoint{5.633294in}{0.865478in}}%
\pgfpathcurveto{\pgfqpoint{5.625480in}{0.873291in}}{\pgfqpoint{5.614881in}{0.877682in}}{\pgfqpoint{5.603831in}{0.877682in}}%
\pgfpathcurveto{\pgfqpoint{5.592781in}{0.877682in}}{\pgfqpoint{5.582182in}{0.873291in}}{\pgfqpoint{5.574368in}{0.865478in}}%
\pgfpathcurveto{\pgfqpoint{5.566554in}{0.857664in}}{\pgfqpoint{5.562164in}{0.847065in}}{\pgfqpoint{5.562164in}{0.836015in}}%
\pgfpathcurveto{\pgfqpoint{5.562164in}{0.824965in}}{\pgfqpoint{5.566554in}{0.814366in}}{\pgfqpoint{5.574368in}{0.806552in}}%
\pgfpathcurveto{\pgfqpoint{5.582182in}{0.798739in}}{\pgfqpoint{5.592781in}{0.794348in}}{\pgfqpoint{5.603831in}{0.794348in}}%
\pgfpathlineto{\pgfqpoint{5.603831in}{0.794348in}}%
\pgfpathclose%
\pgfusepath{stroke,fill}%
\end{pgfscope}%
\begin{pgfscope}%
\pgfpathrectangle{\pgfqpoint{5.292946in}{0.569136in}}{\pgfqpoint{2.177280in}{2.201755in}}%
\pgfusepath{clip}%
\pgfsetbuttcap%
\pgfsetroundjoin%
\definecolor{currentfill}{rgb}{0.121569,0.466667,0.705882}%
\pgfsetfillcolor{currentfill}%
\pgfsetlinewidth{0.481800pt}%
\definecolor{currentstroke}{rgb}{1.000000,1.000000,1.000000}%
\pgfsetstrokecolor{currentstroke}%
\pgfsetdash{}{0pt}%
\pgfpathmoveto{\pgfqpoint{5.661241in}{0.710948in}}%
\pgfpathcurveto{\pgfqpoint{5.672291in}{0.710948in}}{\pgfqpoint{5.682890in}{0.715339in}}{\pgfqpoint{5.690704in}{0.723152in}}%
\pgfpathcurveto{\pgfqpoint{5.698518in}{0.730966in}}{\pgfqpoint{5.702908in}{0.741565in}}{\pgfqpoint{5.702908in}{0.752615in}}%
\pgfpathcurveto{\pgfqpoint{5.702908in}{0.763665in}}{\pgfqpoint{5.698518in}{0.774264in}}{\pgfqpoint{5.690704in}{0.782078in}}%
\pgfpathcurveto{\pgfqpoint{5.682890in}{0.789892in}}{\pgfqpoint{5.672291in}{0.794282in}}{\pgfqpoint{5.661241in}{0.794282in}}%
\pgfpathcurveto{\pgfqpoint{5.650191in}{0.794282in}}{\pgfqpoint{5.639592in}{0.789892in}}{\pgfqpoint{5.631779in}{0.782078in}}%
\pgfpathcurveto{\pgfqpoint{5.623965in}{0.774264in}}{\pgfqpoint{5.619575in}{0.763665in}}{\pgfqpoint{5.619575in}{0.752615in}}%
\pgfpathcurveto{\pgfqpoint{5.619575in}{0.741565in}}{\pgfqpoint{5.623965in}{0.730966in}}{\pgfqpoint{5.631779in}{0.723152in}}%
\pgfpathcurveto{\pgfqpoint{5.639592in}{0.715339in}}{\pgfqpoint{5.650191in}{0.710948in}}{\pgfqpoint{5.661241in}{0.710948in}}%
\pgfpathlineto{\pgfqpoint{5.661241in}{0.710948in}}%
\pgfpathclose%
\pgfusepath{stroke,fill}%
\end{pgfscope}%
\begin{pgfscope}%
\pgfpathrectangle{\pgfqpoint{5.292946in}{0.569136in}}{\pgfqpoint{2.177280in}{2.201755in}}%
\pgfusepath{clip}%
\pgfsetbuttcap%
\pgfsetroundjoin%
\definecolor{currentfill}{rgb}{0.121569,0.466667,0.705882}%
\pgfsetfillcolor{currentfill}%
\pgfsetlinewidth{0.481800pt}%
\definecolor{currentstroke}{rgb}{1.000000,1.000000,1.000000}%
\pgfsetstrokecolor{currentstroke}%
\pgfsetdash{}{0pt}%
\pgfpathmoveto{\pgfqpoint{5.603831in}{0.877748in}}%
\pgfpathcurveto{\pgfqpoint{5.614881in}{0.877748in}}{\pgfqpoint{5.625480in}{0.882138in}}{\pgfqpoint{5.633294in}{0.889952in}}%
\pgfpathcurveto{\pgfqpoint{5.641107in}{0.897766in}}{\pgfqpoint{5.645497in}{0.908365in}}{\pgfqpoint{5.645497in}{0.919415in}}%
\pgfpathcurveto{\pgfqpoint{5.645497in}{0.930465in}}{\pgfqpoint{5.641107in}{0.941064in}}{\pgfqpoint{5.633294in}{0.948878in}}%
\pgfpathcurveto{\pgfqpoint{5.625480in}{0.956691in}}{\pgfqpoint{5.614881in}{0.961081in}}{\pgfqpoint{5.603831in}{0.961081in}}%
\pgfpathcurveto{\pgfqpoint{5.592781in}{0.961081in}}{\pgfqpoint{5.582182in}{0.956691in}}{\pgfqpoint{5.574368in}{0.948878in}}%
\pgfpathcurveto{\pgfqpoint{5.566554in}{0.941064in}}{\pgfqpoint{5.562164in}{0.930465in}}{\pgfqpoint{5.562164in}{0.919415in}}%
\pgfpathcurveto{\pgfqpoint{5.562164in}{0.908365in}}{\pgfqpoint{5.566554in}{0.897766in}}{\pgfqpoint{5.574368in}{0.889952in}}%
\pgfpathcurveto{\pgfqpoint{5.582182in}{0.882138in}}{\pgfqpoint{5.592781in}{0.877748in}}{\pgfqpoint{5.603831in}{0.877748in}}%
\pgfpathlineto{\pgfqpoint{5.603831in}{0.877748in}}%
\pgfpathclose%
\pgfusepath{stroke,fill}%
\end{pgfscope}%
\begin{pgfscope}%
\pgfpathrectangle{\pgfqpoint{5.292946in}{0.569136in}}{\pgfqpoint{2.177280in}{2.201755in}}%
\pgfusepath{clip}%
\pgfsetbuttcap%
\pgfsetroundjoin%
\definecolor{currentfill}{rgb}{0.121569,0.466667,0.705882}%
\pgfsetfillcolor{currentfill}%
\pgfsetlinewidth{0.481800pt}%
\definecolor{currentstroke}{rgb}{1.000000,1.000000,1.000000}%
\pgfsetstrokecolor{currentstroke}%
\pgfsetdash{}{0pt}%
\pgfpathmoveto{\pgfqpoint{5.460304in}{0.710948in}}%
\pgfpathcurveto{\pgfqpoint{5.471354in}{0.710948in}}{\pgfqpoint{5.481953in}{0.715339in}}{\pgfqpoint{5.489767in}{0.723152in}}%
\pgfpathcurveto{\pgfqpoint{5.497581in}{0.730966in}}{\pgfqpoint{5.501971in}{0.741565in}}{\pgfqpoint{5.501971in}{0.752615in}}%
\pgfpathcurveto{\pgfqpoint{5.501971in}{0.763665in}}{\pgfqpoint{5.497581in}{0.774264in}}{\pgfqpoint{5.489767in}{0.782078in}}%
\pgfpathcurveto{\pgfqpoint{5.481953in}{0.789892in}}{\pgfqpoint{5.471354in}{0.794282in}}{\pgfqpoint{5.460304in}{0.794282in}}%
\pgfpathcurveto{\pgfqpoint{5.449254in}{0.794282in}}{\pgfqpoint{5.438655in}{0.789892in}}{\pgfqpoint{5.430842in}{0.782078in}}%
\pgfpathcurveto{\pgfqpoint{5.423028in}{0.774264in}}{\pgfqpoint{5.418638in}{0.763665in}}{\pgfqpoint{5.418638in}{0.752615in}}%
\pgfpathcurveto{\pgfqpoint{5.418638in}{0.741565in}}{\pgfqpoint{5.423028in}{0.730966in}}{\pgfqpoint{5.430842in}{0.723152in}}%
\pgfpathcurveto{\pgfqpoint{5.438655in}{0.715339in}}{\pgfqpoint{5.449254in}{0.710948in}}{\pgfqpoint{5.460304in}{0.710948in}}%
\pgfpathlineto{\pgfqpoint{5.460304in}{0.710948in}}%
\pgfpathclose%
\pgfusepath{stroke,fill}%
\end{pgfscope}%
\begin{pgfscope}%
\pgfpathrectangle{\pgfqpoint{5.292946in}{0.569136in}}{\pgfqpoint{2.177280in}{2.201755in}}%
\pgfusepath{clip}%
\pgfsetbuttcap%
\pgfsetroundjoin%
\definecolor{currentfill}{rgb}{0.121569,0.466667,0.705882}%
\pgfsetfillcolor{currentfill}%
\pgfsetlinewidth{0.481800pt}%
\definecolor{currentstroke}{rgb}{1.000000,1.000000,1.000000}%
\pgfsetstrokecolor{currentstroke}%
\pgfsetdash{}{0pt}%
\pgfpathmoveto{\pgfqpoint{5.661241in}{0.961148in}}%
\pgfpathcurveto{\pgfqpoint{5.672291in}{0.961148in}}{\pgfqpoint{5.682890in}{0.965538in}}{\pgfqpoint{5.690704in}{0.973352in}}%
\pgfpathcurveto{\pgfqpoint{5.698518in}{0.981165in}}{\pgfqpoint{5.702908in}{0.991764in}}{\pgfqpoint{5.702908in}{1.002815in}}%
\pgfpathcurveto{\pgfqpoint{5.702908in}{1.013865in}}{\pgfqpoint{5.698518in}{1.024464in}}{\pgfqpoint{5.690704in}{1.032277in}}%
\pgfpathcurveto{\pgfqpoint{5.682890in}{1.040091in}}{\pgfqpoint{5.672291in}{1.044481in}}{\pgfqpoint{5.661241in}{1.044481in}}%
\pgfpathcurveto{\pgfqpoint{5.650191in}{1.044481in}}{\pgfqpoint{5.639592in}{1.040091in}}{\pgfqpoint{5.631779in}{1.032277in}}%
\pgfpathcurveto{\pgfqpoint{5.623965in}{1.024464in}}{\pgfqpoint{5.619575in}{1.013865in}}{\pgfqpoint{5.619575in}{1.002815in}}%
\pgfpathcurveto{\pgfqpoint{5.619575in}{0.991764in}}{\pgfqpoint{5.623965in}{0.981165in}}{\pgfqpoint{5.631779in}{0.973352in}}%
\pgfpathcurveto{\pgfqpoint{5.639592in}{0.965538in}}{\pgfqpoint{5.650191in}{0.961148in}}{\pgfqpoint{5.661241in}{0.961148in}}%
\pgfpathlineto{\pgfqpoint{5.661241in}{0.961148in}}%
\pgfpathclose%
\pgfusepath{stroke,fill}%
\end{pgfscope}%
\begin{pgfscope}%
\pgfpathrectangle{\pgfqpoint{5.292946in}{0.569136in}}{\pgfqpoint{2.177280in}{2.201755in}}%
\pgfusepath{clip}%
\pgfsetbuttcap%
\pgfsetroundjoin%
\definecolor{currentfill}{rgb}{0.121569,0.466667,0.705882}%
\pgfsetfillcolor{currentfill}%
\pgfsetlinewidth{0.481800pt}%
\definecolor{currentstroke}{rgb}{1.000000,1.000000,1.000000}%
\pgfsetstrokecolor{currentstroke}%
\pgfsetdash{}{0pt}%
\pgfpathmoveto{\pgfqpoint{5.718652in}{0.710948in}}%
\pgfpathcurveto{\pgfqpoint{5.729702in}{0.710948in}}{\pgfqpoint{5.740301in}{0.715339in}}{\pgfqpoint{5.748115in}{0.723152in}}%
\pgfpathcurveto{\pgfqpoint{5.755928in}{0.730966in}}{\pgfqpoint{5.760319in}{0.741565in}}{\pgfqpoint{5.760319in}{0.752615in}}%
\pgfpathcurveto{\pgfqpoint{5.760319in}{0.763665in}}{\pgfqpoint{5.755928in}{0.774264in}}{\pgfqpoint{5.748115in}{0.782078in}}%
\pgfpathcurveto{\pgfqpoint{5.740301in}{0.789892in}}{\pgfqpoint{5.729702in}{0.794282in}}{\pgfqpoint{5.718652in}{0.794282in}}%
\pgfpathcurveto{\pgfqpoint{5.707602in}{0.794282in}}{\pgfqpoint{5.697003in}{0.789892in}}{\pgfqpoint{5.689189in}{0.782078in}}%
\pgfpathcurveto{\pgfqpoint{5.681375in}{0.774264in}}{\pgfqpoint{5.676985in}{0.763665in}}{\pgfqpoint{5.676985in}{0.752615in}}%
\pgfpathcurveto{\pgfqpoint{5.676985in}{0.741565in}}{\pgfqpoint{5.681375in}{0.730966in}}{\pgfqpoint{5.689189in}{0.723152in}}%
\pgfpathcurveto{\pgfqpoint{5.697003in}{0.715339in}}{\pgfqpoint{5.707602in}{0.710948in}}{\pgfqpoint{5.718652in}{0.710948in}}%
\pgfpathlineto{\pgfqpoint{5.718652in}{0.710948in}}%
\pgfpathclose%
\pgfusepath{stroke,fill}%
\end{pgfscope}%
\begin{pgfscope}%
\pgfpathrectangle{\pgfqpoint{5.292946in}{0.569136in}}{\pgfqpoint{2.177280in}{2.201755in}}%
\pgfusepath{clip}%
\pgfsetbuttcap%
\pgfsetroundjoin%
\definecolor{currentfill}{rgb}{0.121569,0.466667,0.705882}%
\pgfsetfillcolor{currentfill}%
\pgfsetlinewidth{0.481800pt}%
\definecolor{currentstroke}{rgb}{1.000000,1.000000,1.000000}%
\pgfsetstrokecolor{currentstroke}%
\pgfsetdash{}{0pt}%
\pgfpathmoveto{\pgfqpoint{5.632536in}{0.710948in}}%
\pgfpathcurveto{\pgfqpoint{5.643586in}{0.710948in}}{\pgfqpoint{5.654185in}{0.715339in}}{\pgfqpoint{5.661999in}{0.723152in}}%
\pgfpathcurveto{\pgfqpoint{5.669812in}{0.730966in}}{\pgfqpoint{5.674203in}{0.741565in}}{\pgfqpoint{5.674203in}{0.752615in}}%
\pgfpathcurveto{\pgfqpoint{5.674203in}{0.763665in}}{\pgfqpoint{5.669812in}{0.774264in}}{\pgfqpoint{5.661999in}{0.782078in}}%
\pgfpathcurveto{\pgfqpoint{5.654185in}{0.789892in}}{\pgfqpoint{5.643586in}{0.794282in}}{\pgfqpoint{5.632536in}{0.794282in}}%
\pgfpathcurveto{\pgfqpoint{5.621486in}{0.794282in}}{\pgfqpoint{5.610887in}{0.789892in}}{\pgfqpoint{5.603073in}{0.782078in}}%
\pgfpathcurveto{\pgfqpoint{5.595260in}{0.774264in}}{\pgfqpoint{5.590869in}{0.763665in}}{\pgfqpoint{5.590869in}{0.752615in}}%
\pgfpathcurveto{\pgfqpoint{5.590869in}{0.741565in}}{\pgfqpoint{5.595260in}{0.730966in}}{\pgfqpoint{5.603073in}{0.723152in}}%
\pgfpathcurveto{\pgfqpoint{5.610887in}{0.715339in}}{\pgfqpoint{5.621486in}{0.710948in}}{\pgfqpoint{5.632536in}{0.710948in}}%
\pgfpathlineto{\pgfqpoint{5.632536in}{0.710948in}}%
\pgfpathclose%
\pgfusepath{stroke,fill}%
\end{pgfscope}%
\begin{pgfscope}%
\pgfpathrectangle{\pgfqpoint{5.292946in}{0.569136in}}{\pgfqpoint{2.177280in}{2.201755in}}%
\pgfusepath{clip}%
\pgfsetbuttcap%
\pgfsetroundjoin%
\definecolor{currentfill}{rgb}{0.121569,0.466667,0.705882}%
\pgfsetfillcolor{currentfill}%
\pgfsetlinewidth{0.481800pt}%
\definecolor{currentstroke}{rgb}{1.000000,1.000000,1.000000}%
\pgfsetstrokecolor{currentstroke}%
\pgfsetdash{}{0pt}%
\pgfpathmoveto{\pgfqpoint{5.632536in}{0.877748in}}%
\pgfpathcurveto{\pgfqpoint{5.643586in}{0.877748in}}{\pgfqpoint{5.654185in}{0.882138in}}{\pgfqpoint{5.661999in}{0.889952in}}%
\pgfpathcurveto{\pgfqpoint{5.669812in}{0.897766in}}{\pgfqpoint{5.674203in}{0.908365in}}{\pgfqpoint{5.674203in}{0.919415in}}%
\pgfpathcurveto{\pgfqpoint{5.674203in}{0.930465in}}{\pgfqpoint{5.669812in}{0.941064in}}{\pgfqpoint{5.661999in}{0.948878in}}%
\pgfpathcurveto{\pgfqpoint{5.654185in}{0.956691in}}{\pgfqpoint{5.643586in}{0.961081in}}{\pgfqpoint{5.632536in}{0.961081in}}%
\pgfpathcurveto{\pgfqpoint{5.621486in}{0.961081in}}{\pgfqpoint{5.610887in}{0.956691in}}{\pgfqpoint{5.603073in}{0.948878in}}%
\pgfpathcurveto{\pgfqpoint{5.595260in}{0.941064in}}{\pgfqpoint{5.590869in}{0.930465in}}{\pgfqpoint{5.590869in}{0.919415in}}%
\pgfpathcurveto{\pgfqpoint{5.590869in}{0.908365in}}{\pgfqpoint{5.595260in}{0.897766in}}{\pgfqpoint{5.603073in}{0.889952in}}%
\pgfpathcurveto{\pgfqpoint{5.610887in}{0.882138in}}{\pgfqpoint{5.621486in}{0.877748in}}{\pgfqpoint{5.632536in}{0.877748in}}%
\pgfpathlineto{\pgfqpoint{5.632536in}{0.877748in}}%
\pgfpathclose%
\pgfusepath{stroke,fill}%
\end{pgfscope}%
\begin{pgfscope}%
\pgfpathrectangle{\pgfqpoint{5.292946in}{0.569136in}}{\pgfqpoint{2.177280in}{2.201755in}}%
\pgfusepath{clip}%
\pgfsetbuttcap%
\pgfsetroundjoin%
\definecolor{currentfill}{rgb}{0.121569,0.466667,0.705882}%
\pgfsetfillcolor{currentfill}%
\pgfsetlinewidth{0.481800pt}%
\definecolor{currentstroke}{rgb}{1.000000,1.000000,1.000000}%
\pgfsetstrokecolor{currentstroke}%
\pgfsetdash{}{0pt}%
\pgfpathmoveto{\pgfqpoint{5.603831in}{0.710948in}}%
\pgfpathcurveto{\pgfqpoint{5.614881in}{0.710948in}}{\pgfqpoint{5.625480in}{0.715339in}}{\pgfqpoint{5.633294in}{0.723152in}}%
\pgfpathcurveto{\pgfqpoint{5.641107in}{0.730966in}}{\pgfqpoint{5.645497in}{0.741565in}}{\pgfqpoint{5.645497in}{0.752615in}}%
\pgfpathcurveto{\pgfqpoint{5.645497in}{0.763665in}}{\pgfqpoint{5.641107in}{0.774264in}}{\pgfqpoint{5.633294in}{0.782078in}}%
\pgfpathcurveto{\pgfqpoint{5.625480in}{0.789892in}}{\pgfqpoint{5.614881in}{0.794282in}}{\pgfqpoint{5.603831in}{0.794282in}}%
\pgfpathcurveto{\pgfqpoint{5.592781in}{0.794282in}}{\pgfqpoint{5.582182in}{0.789892in}}{\pgfqpoint{5.574368in}{0.782078in}}%
\pgfpathcurveto{\pgfqpoint{5.566554in}{0.774264in}}{\pgfqpoint{5.562164in}{0.763665in}}{\pgfqpoint{5.562164in}{0.752615in}}%
\pgfpathcurveto{\pgfqpoint{5.562164in}{0.741565in}}{\pgfqpoint{5.566554in}{0.730966in}}{\pgfqpoint{5.574368in}{0.723152in}}%
\pgfpathcurveto{\pgfqpoint{5.582182in}{0.715339in}}{\pgfqpoint{5.592781in}{0.710948in}}{\pgfqpoint{5.603831in}{0.710948in}}%
\pgfpathlineto{\pgfqpoint{5.603831in}{0.710948in}}%
\pgfpathclose%
\pgfusepath{stroke,fill}%
\end{pgfscope}%
\begin{pgfscope}%
\pgfpathrectangle{\pgfqpoint{5.292946in}{0.569136in}}{\pgfqpoint{2.177280in}{2.201755in}}%
\pgfusepath{clip}%
\pgfsetbuttcap%
\pgfsetroundjoin%
\definecolor{currentfill}{rgb}{0.121569,0.466667,0.705882}%
\pgfsetfillcolor{currentfill}%
\pgfsetlinewidth{0.481800pt}%
\definecolor{currentstroke}{rgb}{1.000000,1.000000,1.000000}%
\pgfsetstrokecolor{currentstroke}%
\pgfsetdash{}{0pt}%
\pgfpathmoveto{\pgfqpoint{5.575125in}{0.710948in}}%
\pgfpathcurveto{\pgfqpoint{5.586176in}{0.710948in}}{\pgfqpoint{5.596775in}{0.715339in}}{\pgfqpoint{5.604588in}{0.723152in}}%
\pgfpathcurveto{\pgfqpoint{5.612402in}{0.730966in}}{\pgfqpoint{5.616792in}{0.741565in}}{\pgfqpoint{5.616792in}{0.752615in}}%
\pgfpathcurveto{\pgfqpoint{5.616792in}{0.763665in}}{\pgfqpoint{5.612402in}{0.774264in}}{\pgfqpoint{5.604588in}{0.782078in}}%
\pgfpathcurveto{\pgfqpoint{5.596775in}{0.789892in}}{\pgfqpoint{5.586176in}{0.794282in}}{\pgfqpoint{5.575125in}{0.794282in}}%
\pgfpathcurveto{\pgfqpoint{5.564075in}{0.794282in}}{\pgfqpoint{5.553476in}{0.789892in}}{\pgfqpoint{5.545663in}{0.782078in}}%
\pgfpathcurveto{\pgfqpoint{5.537849in}{0.774264in}}{\pgfqpoint{5.533459in}{0.763665in}}{\pgfqpoint{5.533459in}{0.752615in}}%
\pgfpathcurveto{\pgfqpoint{5.533459in}{0.741565in}}{\pgfqpoint{5.537849in}{0.730966in}}{\pgfqpoint{5.545663in}{0.723152in}}%
\pgfpathcurveto{\pgfqpoint{5.553476in}{0.715339in}}{\pgfqpoint{5.564075in}{0.710948in}}{\pgfqpoint{5.575125in}{0.710948in}}%
\pgfpathlineto{\pgfqpoint{5.575125in}{0.710948in}}%
\pgfpathclose%
\pgfusepath{stroke,fill}%
\end{pgfscope}%
\begin{pgfscope}%
\pgfpathrectangle{\pgfqpoint{5.292946in}{0.569136in}}{\pgfqpoint{2.177280in}{2.201755in}}%
\pgfusepath{clip}%
\pgfsetbuttcap%
\pgfsetroundjoin%
\definecolor{currentfill}{rgb}{0.121569,0.466667,0.705882}%
\pgfsetfillcolor{currentfill}%
\pgfsetlinewidth{0.481800pt}%
\definecolor{currentstroke}{rgb}{1.000000,1.000000,1.000000}%
\pgfsetstrokecolor{currentstroke}%
\pgfsetdash{}{0pt}%
\pgfpathmoveto{\pgfqpoint{5.632536in}{0.710948in}}%
\pgfpathcurveto{\pgfqpoint{5.643586in}{0.710948in}}{\pgfqpoint{5.654185in}{0.715339in}}{\pgfqpoint{5.661999in}{0.723152in}}%
\pgfpathcurveto{\pgfqpoint{5.669812in}{0.730966in}}{\pgfqpoint{5.674203in}{0.741565in}}{\pgfqpoint{5.674203in}{0.752615in}}%
\pgfpathcurveto{\pgfqpoint{5.674203in}{0.763665in}}{\pgfqpoint{5.669812in}{0.774264in}}{\pgfqpoint{5.661999in}{0.782078in}}%
\pgfpathcurveto{\pgfqpoint{5.654185in}{0.789892in}}{\pgfqpoint{5.643586in}{0.794282in}}{\pgfqpoint{5.632536in}{0.794282in}}%
\pgfpathcurveto{\pgfqpoint{5.621486in}{0.794282in}}{\pgfqpoint{5.610887in}{0.789892in}}{\pgfqpoint{5.603073in}{0.782078in}}%
\pgfpathcurveto{\pgfqpoint{5.595260in}{0.774264in}}{\pgfqpoint{5.590869in}{0.763665in}}{\pgfqpoint{5.590869in}{0.752615in}}%
\pgfpathcurveto{\pgfqpoint{5.590869in}{0.741565in}}{\pgfqpoint{5.595260in}{0.730966in}}{\pgfqpoint{5.603073in}{0.723152in}}%
\pgfpathcurveto{\pgfqpoint{5.610887in}{0.715339in}}{\pgfqpoint{5.621486in}{0.710948in}}{\pgfqpoint{5.632536in}{0.710948in}}%
\pgfpathlineto{\pgfqpoint{5.632536in}{0.710948in}}%
\pgfpathclose%
\pgfusepath{stroke,fill}%
\end{pgfscope}%
\begin{pgfscope}%
\pgfpathrectangle{\pgfqpoint{5.292946in}{0.569136in}}{\pgfqpoint{2.177280in}{2.201755in}}%
\pgfusepath{clip}%
\pgfsetbuttcap%
\pgfsetroundjoin%
\definecolor{currentfill}{rgb}{0.121569,0.466667,0.705882}%
\pgfsetfillcolor{currentfill}%
\pgfsetlinewidth{0.481800pt}%
\definecolor{currentstroke}{rgb}{1.000000,1.000000,1.000000}%
\pgfsetstrokecolor{currentstroke}%
\pgfsetdash{}{0pt}%
\pgfpathmoveto{\pgfqpoint{5.632536in}{0.710948in}}%
\pgfpathcurveto{\pgfqpoint{5.643586in}{0.710948in}}{\pgfqpoint{5.654185in}{0.715339in}}{\pgfqpoint{5.661999in}{0.723152in}}%
\pgfpathcurveto{\pgfqpoint{5.669812in}{0.730966in}}{\pgfqpoint{5.674203in}{0.741565in}}{\pgfqpoint{5.674203in}{0.752615in}}%
\pgfpathcurveto{\pgfqpoint{5.674203in}{0.763665in}}{\pgfqpoint{5.669812in}{0.774264in}}{\pgfqpoint{5.661999in}{0.782078in}}%
\pgfpathcurveto{\pgfqpoint{5.654185in}{0.789892in}}{\pgfqpoint{5.643586in}{0.794282in}}{\pgfqpoint{5.632536in}{0.794282in}}%
\pgfpathcurveto{\pgfqpoint{5.621486in}{0.794282in}}{\pgfqpoint{5.610887in}{0.789892in}}{\pgfqpoint{5.603073in}{0.782078in}}%
\pgfpathcurveto{\pgfqpoint{5.595260in}{0.774264in}}{\pgfqpoint{5.590869in}{0.763665in}}{\pgfqpoint{5.590869in}{0.752615in}}%
\pgfpathcurveto{\pgfqpoint{5.590869in}{0.741565in}}{\pgfqpoint{5.595260in}{0.730966in}}{\pgfqpoint{5.603073in}{0.723152in}}%
\pgfpathcurveto{\pgfqpoint{5.610887in}{0.715339in}}{\pgfqpoint{5.621486in}{0.710948in}}{\pgfqpoint{5.632536in}{0.710948in}}%
\pgfpathlineto{\pgfqpoint{5.632536in}{0.710948in}}%
\pgfpathclose%
\pgfusepath{stroke,fill}%
\end{pgfscope}%
\begin{pgfscope}%
\pgfpathrectangle{\pgfqpoint{5.292946in}{0.569136in}}{\pgfqpoint{2.177280in}{2.201755in}}%
\pgfusepath{clip}%
\pgfsetbuttcap%
\pgfsetroundjoin%
\definecolor{currentfill}{rgb}{0.121569,0.466667,0.705882}%
\pgfsetfillcolor{currentfill}%
\pgfsetlinewidth{0.481800pt}%
\definecolor{currentstroke}{rgb}{1.000000,1.000000,1.000000}%
\pgfsetstrokecolor{currentstroke}%
\pgfsetdash{}{0pt}%
\pgfpathmoveto{\pgfqpoint{5.603831in}{0.877748in}}%
\pgfpathcurveto{\pgfqpoint{5.614881in}{0.877748in}}{\pgfqpoint{5.625480in}{0.882138in}}{\pgfqpoint{5.633294in}{0.889952in}}%
\pgfpathcurveto{\pgfqpoint{5.641107in}{0.897766in}}{\pgfqpoint{5.645497in}{0.908365in}}{\pgfqpoint{5.645497in}{0.919415in}}%
\pgfpathcurveto{\pgfqpoint{5.645497in}{0.930465in}}{\pgfqpoint{5.641107in}{0.941064in}}{\pgfqpoint{5.633294in}{0.948878in}}%
\pgfpathcurveto{\pgfqpoint{5.625480in}{0.956691in}}{\pgfqpoint{5.614881in}{0.961081in}}{\pgfqpoint{5.603831in}{0.961081in}}%
\pgfpathcurveto{\pgfqpoint{5.592781in}{0.961081in}}{\pgfqpoint{5.582182in}{0.956691in}}{\pgfqpoint{5.574368in}{0.948878in}}%
\pgfpathcurveto{\pgfqpoint{5.566554in}{0.941064in}}{\pgfqpoint{5.562164in}{0.930465in}}{\pgfqpoint{5.562164in}{0.919415in}}%
\pgfpathcurveto{\pgfqpoint{5.562164in}{0.908365in}}{\pgfqpoint{5.566554in}{0.897766in}}{\pgfqpoint{5.574368in}{0.889952in}}%
\pgfpathcurveto{\pgfqpoint{5.582182in}{0.882138in}}{\pgfqpoint{5.592781in}{0.877748in}}{\pgfqpoint{5.603831in}{0.877748in}}%
\pgfpathlineto{\pgfqpoint{5.603831in}{0.877748in}}%
\pgfpathclose%
\pgfusepath{stroke,fill}%
\end{pgfscope}%
\begin{pgfscope}%
\pgfpathrectangle{\pgfqpoint{5.292946in}{0.569136in}}{\pgfqpoint{2.177280in}{2.201755in}}%
\pgfusepath{clip}%
\pgfsetbuttcap%
\pgfsetroundjoin%
\definecolor{currentfill}{rgb}{0.121569,0.466667,0.705882}%
\pgfsetfillcolor{currentfill}%
\pgfsetlinewidth{0.481800pt}%
\definecolor{currentstroke}{rgb}{1.000000,1.000000,1.000000}%
\pgfsetstrokecolor{currentstroke}%
\pgfsetdash{}{0pt}%
\pgfpathmoveto{\pgfqpoint{5.603831in}{0.627549in}}%
\pgfpathcurveto{\pgfqpoint{5.614881in}{0.627549in}}{\pgfqpoint{5.625480in}{0.631939in}}{\pgfqpoint{5.633294in}{0.639753in}}%
\pgfpathcurveto{\pgfqpoint{5.641107in}{0.647566in}}{\pgfqpoint{5.645497in}{0.658165in}}{\pgfqpoint{5.645497in}{0.669215in}}%
\pgfpathcurveto{\pgfqpoint{5.645497in}{0.680265in}}{\pgfqpoint{5.641107in}{0.690864in}}{\pgfqpoint{5.633294in}{0.698678in}}%
\pgfpathcurveto{\pgfqpoint{5.625480in}{0.706492in}}{\pgfqpoint{5.614881in}{0.710882in}}{\pgfqpoint{5.603831in}{0.710882in}}%
\pgfpathcurveto{\pgfqpoint{5.592781in}{0.710882in}}{\pgfqpoint{5.582182in}{0.706492in}}{\pgfqpoint{5.574368in}{0.698678in}}%
\pgfpathcurveto{\pgfqpoint{5.566554in}{0.690864in}}{\pgfqpoint{5.562164in}{0.680265in}}{\pgfqpoint{5.562164in}{0.669215in}}%
\pgfpathcurveto{\pgfqpoint{5.562164in}{0.658165in}}{\pgfqpoint{5.566554in}{0.647566in}}{\pgfqpoint{5.574368in}{0.639753in}}%
\pgfpathcurveto{\pgfqpoint{5.582182in}{0.631939in}}{\pgfqpoint{5.592781in}{0.627549in}}{\pgfqpoint{5.603831in}{0.627549in}}%
\pgfpathlineto{\pgfqpoint{5.603831in}{0.627549in}}%
\pgfpathclose%
\pgfusepath{stroke,fill}%
\end{pgfscope}%
\begin{pgfscope}%
\pgfpathrectangle{\pgfqpoint{5.292946in}{0.569136in}}{\pgfqpoint{2.177280in}{2.201755in}}%
\pgfusepath{clip}%
\pgfsetbuttcap%
\pgfsetroundjoin%
\definecolor{currentfill}{rgb}{0.121569,0.466667,0.705882}%
\pgfsetfillcolor{currentfill}%
\pgfsetlinewidth{0.481800pt}%
\definecolor{currentstroke}{rgb}{1.000000,1.000000,1.000000}%
\pgfsetstrokecolor{currentstroke}%
\pgfsetdash{}{0pt}%
\pgfpathmoveto{\pgfqpoint{5.575125in}{0.710948in}}%
\pgfpathcurveto{\pgfqpoint{5.586176in}{0.710948in}}{\pgfqpoint{5.596775in}{0.715339in}}{\pgfqpoint{5.604588in}{0.723152in}}%
\pgfpathcurveto{\pgfqpoint{5.612402in}{0.730966in}}{\pgfqpoint{5.616792in}{0.741565in}}{\pgfqpoint{5.616792in}{0.752615in}}%
\pgfpathcurveto{\pgfqpoint{5.616792in}{0.763665in}}{\pgfqpoint{5.612402in}{0.774264in}}{\pgfqpoint{5.604588in}{0.782078in}}%
\pgfpathcurveto{\pgfqpoint{5.596775in}{0.789892in}}{\pgfqpoint{5.586176in}{0.794282in}}{\pgfqpoint{5.575125in}{0.794282in}}%
\pgfpathcurveto{\pgfqpoint{5.564075in}{0.794282in}}{\pgfqpoint{5.553476in}{0.789892in}}{\pgfqpoint{5.545663in}{0.782078in}}%
\pgfpathcurveto{\pgfqpoint{5.537849in}{0.774264in}}{\pgfqpoint{5.533459in}{0.763665in}}{\pgfqpoint{5.533459in}{0.752615in}}%
\pgfpathcurveto{\pgfqpoint{5.533459in}{0.741565in}}{\pgfqpoint{5.537849in}{0.730966in}}{\pgfqpoint{5.545663in}{0.723152in}}%
\pgfpathcurveto{\pgfqpoint{5.553476in}{0.715339in}}{\pgfqpoint{5.564075in}{0.710948in}}{\pgfqpoint{5.575125in}{0.710948in}}%
\pgfpathlineto{\pgfqpoint{5.575125in}{0.710948in}}%
\pgfpathclose%
\pgfusepath{stroke,fill}%
\end{pgfscope}%
\begin{pgfscope}%
\pgfpathrectangle{\pgfqpoint{5.292946in}{0.569136in}}{\pgfqpoint{2.177280in}{2.201755in}}%
\pgfusepath{clip}%
\pgfsetbuttcap%
\pgfsetroundjoin%
\definecolor{currentfill}{rgb}{0.121569,0.466667,0.705882}%
\pgfsetfillcolor{currentfill}%
\pgfsetlinewidth{0.481800pt}%
\definecolor{currentstroke}{rgb}{1.000000,1.000000,1.000000}%
\pgfsetstrokecolor{currentstroke}%
\pgfsetdash{}{0pt}%
\pgfpathmoveto{\pgfqpoint{5.603831in}{0.710948in}}%
\pgfpathcurveto{\pgfqpoint{5.614881in}{0.710948in}}{\pgfqpoint{5.625480in}{0.715339in}}{\pgfqpoint{5.633294in}{0.723152in}}%
\pgfpathcurveto{\pgfqpoint{5.641107in}{0.730966in}}{\pgfqpoint{5.645497in}{0.741565in}}{\pgfqpoint{5.645497in}{0.752615in}}%
\pgfpathcurveto{\pgfqpoint{5.645497in}{0.763665in}}{\pgfqpoint{5.641107in}{0.774264in}}{\pgfqpoint{5.633294in}{0.782078in}}%
\pgfpathcurveto{\pgfqpoint{5.625480in}{0.789892in}}{\pgfqpoint{5.614881in}{0.794282in}}{\pgfqpoint{5.603831in}{0.794282in}}%
\pgfpathcurveto{\pgfqpoint{5.592781in}{0.794282in}}{\pgfqpoint{5.582182in}{0.789892in}}{\pgfqpoint{5.574368in}{0.782078in}}%
\pgfpathcurveto{\pgfqpoint{5.566554in}{0.774264in}}{\pgfqpoint{5.562164in}{0.763665in}}{\pgfqpoint{5.562164in}{0.752615in}}%
\pgfpathcurveto{\pgfqpoint{5.562164in}{0.741565in}}{\pgfqpoint{5.566554in}{0.730966in}}{\pgfqpoint{5.574368in}{0.723152in}}%
\pgfpathcurveto{\pgfqpoint{5.582182in}{0.715339in}}{\pgfqpoint{5.592781in}{0.710948in}}{\pgfqpoint{5.603831in}{0.710948in}}%
\pgfpathlineto{\pgfqpoint{5.603831in}{0.710948in}}%
\pgfpathclose%
\pgfusepath{stroke,fill}%
\end{pgfscope}%
\begin{pgfscope}%
\pgfpathrectangle{\pgfqpoint{5.292946in}{0.569136in}}{\pgfqpoint{2.177280in}{2.201755in}}%
\pgfusepath{clip}%
\pgfsetbuttcap%
\pgfsetroundjoin%
\definecolor{currentfill}{rgb}{0.121569,0.466667,0.705882}%
\pgfsetfillcolor{currentfill}%
\pgfsetlinewidth{0.481800pt}%
\definecolor{currentstroke}{rgb}{1.000000,1.000000,1.000000}%
\pgfsetstrokecolor{currentstroke}%
\pgfsetdash{}{0pt}%
\pgfpathmoveto{\pgfqpoint{5.517715in}{0.710948in}}%
\pgfpathcurveto{\pgfqpoint{5.528765in}{0.710948in}}{\pgfqpoint{5.539364in}{0.715339in}}{\pgfqpoint{5.547178in}{0.723152in}}%
\pgfpathcurveto{\pgfqpoint{5.554991in}{0.730966in}}{\pgfqpoint{5.559382in}{0.741565in}}{\pgfqpoint{5.559382in}{0.752615in}}%
\pgfpathcurveto{\pgfqpoint{5.559382in}{0.763665in}}{\pgfqpoint{5.554991in}{0.774264in}}{\pgfqpoint{5.547178in}{0.782078in}}%
\pgfpathcurveto{\pgfqpoint{5.539364in}{0.789892in}}{\pgfqpoint{5.528765in}{0.794282in}}{\pgfqpoint{5.517715in}{0.794282in}}%
\pgfpathcurveto{\pgfqpoint{5.506665in}{0.794282in}}{\pgfqpoint{5.496066in}{0.789892in}}{\pgfqpoint{5.488252in}{0.782078in}}%
\pgfpathcurveto{\pgfqpoint{5.480438in}{0.774264in}}{\pgfqpoint{5.476048in}{0.763665in}}{\pgfqpoint{5.476048in}{0.752615in}}%
\pgfpathcurveto{\pgfqpoint{5.476048in}{0.741565in}}{\pgfqpoint{5.480438in}{0.730966in}}{\pgfqpoint{5.488252in}{0.723152in}}%
\pgfpathcurveto{\pgfqpoint{5.496066in}{0.715339in}}{\pgfqpoint{5.506665in}{0.710948in}}{\pgfqpoint{5.517715in}{0.710948in}}%
\pgfpathlineto{\pgfqpoint{5.517715in}{0.710948in}}%
\pgfpathclose%
\pgfusepath{stroke,fill}%
\end{pgfscope}%
\begin{pgfscope}%
\pgfpathrectangle{\pgfqpoint{5.292946in}{0.569136in}}{\pgfqpoint{2.177280in}{2.201755in}}%
\pgfusepath{clip}%
\pgfsetbuttcap%
\pgfsetroundjoin%
\definecolor{currentfill}{rgb}{0.121569,0.466667,0.705882}%
\pgfsetfillcolor{currentfill}%
\pgfsetlinewidth{0.481800pt}%
\definecolor{currentstroke}{rgb}{1.000000,1.000000,1.000000}%
\pgfsetstrokecolor{currentstroke}%
\pgfsetdash{}{0pt}%
\pgfpathmoveto{\pgfqpoint{5.546420in}{0.710948in}}%
\pgfpathcurveto{\pgfqpoint{5.557470in}{0.710948in}}{\pgfqpoint{5.568069in}{0.715339in}}{\pgfqpoint{5.575883in}{0.723152in}}%
\pgfpathcurveto{\pgfqpoint{5.583697in}{0.730966in}}{\pgfqpoint{5.588087in}{0.741565in}}{\pgfqpoint{5.588087in}{0.752615in}}%
\pgfpathcurveto{\pgfqpoint{5.588087in}{0.763665in}}{\pgfqpoint{5.583697in}{0.774264in}}{\pgfqpoint{5.575883in}{0.782078in}}%
\pgfpathcurveto{\pgfqpoint{5.568069in}{0.789892in}}{\pgfqpoint{5.557470in}{0.794282in}}{\pgfqpoint{5.546420in}{0.794282in}}%
\pgfpathcurveto{\pgfqpoint{5.535370in}{0.794282in}}{\pgfqpoint{5.524771in}{0.789892in}}{\pgfqpoint{5.516957in}{0.782078in}}%
\pgfpathcurveto{\pgfqpoint{5.509144in}{0.774264in}}{\pgfqpoint{5.504753in}{0.763665in}}{\pgfqpoint{5.504753in}{0.752615in}}%
\pgfpathcurveto{\pgfqpoint{5.504753in}{0.741565in}}{\pgfqpoint{5.509144in}{0.730966in}}{\pgfqpoint{5.516957in}{0.723152in}}%
\pgfpathcurveto{\pgfqpoint{5.524771in}{0.715339in}}{\pgfqpoint{5.535370in}{0.710948in}}{\pgfqpoint{5.546420in}{0.710948in}}%
\pgfpathlineto{\pgfqpoint{5.546420in}{0.710948in}}%
\pgfpathclose%
\pgfusepath{stroke,fill}%
\end{pgfscope}%
\begin{pgfscope}%
\pgfpathrectangle{\pgfqpoint{5.292946in}{0.569136in}}{\pgfqpoint{2.177280in}{2.201755in}}%
\pgfusepath{clip}%
\pgfsetbuttcap%
\pgfsetroundjoin%
\definecolor{currentfill}{rgb}{0.121569,0.466667,0.705882}%
\pgfsetfillcolor{currentfill}%
\pgfsetlinewidth{0.481800pt}%
\definecolor{currentstroke}{rgb}{1.000000,1.000000,1.000000}%
\pgfsetstrokecolor{currentstroke}%
\pgfsetdash{}{0pt}%
\pgfpathmoveto{\pgfqpoint{5.575125in}{0.627549in}}%
\pgfpathcurveto{\pgfqpoint{5.586176in}{0.627549in}}{\pgfqpoint{5.596775in}{0.631939in}}{\pgfqpoint{5.604588in}{0.639753in}}%
\pgfpathcurveto{\pgfqpoint{5.612402in}{0.647566in}}{\pgfqpoint{5.616792in}{0.658165in}}{\pgfqpoint{5.616792in}{0.669215in}}%
\pgfpathcurveto{\pgfqpoint{5.616792in}{0.680265in}}{\pgfqpoint{5.612402in}{0.690864in}}{\pgfqpoint{5.604588in}{0.698678in}}%
\pgfpathcurveto{\pgfqpoint{5.596775in}{0.706492in}}{\pgfqpoint{5.586176in}{0.710882in}}{\pgfqpoint{5.575125in}{0.710882in}}%
\pgfpathcurveto{\pgfqpoint{5.564075in}{0.710882in}}{\pgfqpoint{5.553476in}{0.706492in}}{\pgfqpoint{5.545663in}{0.698678in}}%
\pgfpathcurveto{\pgfqpoint{5.537849in}{0.690864in}}{\pgfqpoint{5.533459in}{0.680265in}}{\pgfqpoint{5.533459in}{0.669215in}}%
\pgfpathcurveto{\pgfqpoint{5.533459in}{0.658165in}}{\pgfqpoint{5.537849in}{0.647566in}}{\pgfqpoint{5.545663in}{0.639753in}}%
\pgfpathcurveto{\pgfqpoint{5.553476in}{0.631939in}}{\pgfqpoint{5.564075in}{0.627549in}}{\pgfqpoint{5.575125in}{0.627549in}}%
\pgfpathlineto{\pgfqpoint{5.575125in}{0.627549in}}%
\pgfpathclose%
\pgfusepath{stroke,fill}%
\end{pgfscope}%
\begin{pgfscope}%
\pgfpathrectangle{\pgfqpoint{5.292946in}{0.569136in}}{\pgfqpoint{2.177280in}{2.201755in}}%
\pgfusepath{clip}%
\pgfsetbuttcap%
\pgfsetroundjoin%
\definecolor{currentfill}{rgb}{0.121569,0.466667,0.705882}%
\pgfsetfillcolor{currentfill}%
\pgfsetlinewidth{0.481800pt}%
\definecolor{currentstroke}{rgb}{1.000000,1.000000,1.000000}%
\pgfsetstrokecolor{currentstroke}%
\pgfsetdash{}{0pt}%
\pgfpathmoveto{\pgfqpoint{5.546420in}{0.710948in}}%
\pgfpathcurveto{\pgfqpoint{5.557470in}{0.710948in}}{\pgfqpoint{5.568069in}{0.715339in}}{\pgfqpoint{5.575883in}{0.723152in}}%
\pgfpathcurveto{\pgfqpoint{5.583697in}{0.730966in}}{\pgfqpoint{5.588087in}{0.741565in}}{\pgfqpoint{5.588087in}{0.752615in}}%
\pgfpathcurveto{\pgfqpoint{5.588087in}{0.763665in}}{\pgfqpoint{5.583697in}{0.774264in}}{\pgfqpoint{5.575883in}{0.782078in}}%
\pgfpathcurveto{\pgfqpoint{5.568069in}{0.789892in}}{\pgfqpoint{5.557470in}{0.794282in}}{\pgfqpoint{5.546420in}{0.794282in}}%
\pgfpathcurveto{\pgfqpoint{5.535370in}{0.794282in}}{\pgfqpoint{5.524771in}{0.789892in}}{\pgfqpoint{5.516957in}{0.782078in}}%
\pgfpathcurveto{\pgfqpoint{5.509144in}{0.774264in}}{\pgfqpoint{5.504753in}{0.763665in}}{\pgfqpoint{5.504753in}{0.752615in}}%
\pgfpathcurveto{\pgfqpoint{5.504753in}{0.741565in}}{\pgfqpoint{5.509144in}{0.730966in}}{\pgfqpoint{5.516957in}{0.723152in}}%
\pgfpathcurveto{\pgfqpoint{5.524771in}{0.715339in}}{\pgfqpoint{5.535370in}{0.710948in}}{\pgfqpoint{5.546420in}{0.710948in}}%
\pgfpathlineto{\pgfqpoint{5.546420in}{0.710948in}}%
\pgfpathclose%
\pgfusepath{stroke,fill}%
\end{pgfscope}%
\begin{pgfscope}%
\pgfpathrectangle{\pgfqpoint{5.292946in}{0.569136in}}{\pgfqpoint{2.177280in}{2.201755in}}%
\pgfusepath{clip}%
\pgfsetbuttcap%
\pgfsetroundjoin%
\definecolor{currentfill}{rgb}{0.121569,0.466667,0.705882}%
\pgfsetfillcolor{currentfill}%
\pgfsetlinewidth{0.481800pt}%
\definecolor{currentstroke}{rgb}{1.000000,1.000000,1.000000}%
\pgfsetstrokecolor{currentstroke}%
\pgfsetdash{}{0pt}%
\pgfpathmoveto{\pgfqpoint{5.603831in}{0.710948in}}%
\pgfpathcurveto{\pgfqpoint{5.614881in}{0.710948in}}{\pgfqpoint{5.625480in}{0.715339in}}{\pgfqpoint{5.633294in}{0.723152in}}%
\pgfpathcurveto{\pgfqpoint{5.641107in}{0.730966in}}{\pgfqpoint{5.645497in}{0.741565in}}{\pgfqpoint{5.645497in}{0.752615in}}%
\pgfpathcurveto{\pgfqpoint{5.645497in}{0.763665in}}{\pgfqpoint{5.641107in}{0.774264in}}{\pgfqpoint{5.633294in}{0.782078in}}%
\pgfpathcurveto{\pgfqpoint{5.625480in}{0.789892in}}{\pgfqpoint{5.614881in}{0.794282in}}{\pgfqpoint{5.603831in}{0.794282in}}%
\pgfpathcurveto{\pgfqpoint{5.592781in}{0.794282in}}{\pgfqpoint{5.582182in}{0.789892in}}{\pgfqpoint{5.574368in}{0.782078in}}%
\pgfpathcurveto{\pgfqpoint{5.566554in}{0.774264in}}{\pgfqpoint{5.562164in}{0.763665in}}{\pgfqpoint{5.562164in}{0.752615in}}%
\pgfpathcurveto{\pgfqpoint{5.562164in}{0.741565in}}{\pgfqpoint{5.566554in}{0.730966in}}{\pgfqpoint{5.574368in}{0.723152in}}%
\pgfpathcurveto{\pgfqpoint{5.582182in}{0.715339in}}{\pgfqpoint{5.592781in}{0.710948in}}{\pgfqpoint{5.603831in}{0.710948in}}%
\pgfpathlineto{\pgfqpoint{5.603831in}{0.710948in}}%
\pgfpathclose%
\pgfusepath{stroke,fill}%
\end{pgfscope}%
\begin{pgfscope}%
\pgfpathrectangle{\pgfqpoint{5.292946in}{0.569136in}}{\pgfqpoint{2.177280in}{2.201755in}}%
\pgfusepath{clip}%
\pgfsetbuttcap%
\pgfsetroundjoin%
\definecolor{currentfill}{rgb}{0.121569,0.466667,0.705882}%
\pgfsetfillcolor{currentfill}%
\pgfsetlinewidth{0.481800pt}%
\definecolor{currentstroke}{rgb}{1.000000,1.000000,1.000000}%
\pgfsetstrokecolor{currentstroke}%
\pgfsetdash{}{0pt}%
\pgfpathmoveto{\pgfqpoint{5.546420in}{0.794348in}}%
\pgfpathcurveto{\pgfqpoint{5.557470in}{0.794348in}}{\pgfqpoint{5.568069in}{0.798739in}}{\pgfqpoint{5.575883in}{0.806552in}}%
\pgfpathcurveto{\pgfqpoint{5.583697in}{0.814366in}}{\pgfqpoint{5.588087in}{0.824965in}}{\pgfqpoint{5.588087in}{0.836015in}}%
\pgfpathcurveto{\pgfqpoint{5.588087in}{0.847065in}}{\pgfqpoint{5.583697in}{0.857664in}}{\pgfqpoint{5.575883in}{0.865478in}}%
\pgfpathcurveto{\pgfqpoint{5.568069in}{0.873291in}}{\pgfqpoint{5.557470in}{0.877682in}}{\pgfqpoint{5.546420in}{0.877682in}}%
\pgfpathcurveto{\pgfqpoint{5.535370in}{0.877682in}}{\pgfqpoint{5.524771in}{0.873291in}}{\pgfqpoint{5.516957in}{0.865478in}}%
\pgfpathcurveto{\pgfqpoint{5.509144in}{0.857664in}}{\pgfqpoint{5.504753in}{0.847065in}}{\pgfqpoint{5.504753in}{0.836015in}}%
\pgfpathcurveto{\pgfqpoint{5.504753in}{0.824965in}}{\pgfqpoint{5.509144in}{0.814366in}}{\pgfqpoint{5.516957in}{0.806552in}}%
\pgfpathcurveto{\pgfqpoint{5.524771in}{0.798739in}}{\pgfqpoint{5.535370in}{0.794348in}}{\pgfqpoint{5.546420in}{0.794348in}}%
\pgfpathlineto{\pgfqpoint{5.546420in}{0.794348in}}%
\pgfpathclose%
\pgfusepath{stroke,fill}%
\end{pgfscope}%
\begin{pgfscope}%
\pgfpathrectangle{\pgfqpoint{5.292946in}{0.569136in}}{\pgfqpoint{2.177280in}{2.201755in}}%
\pgfusepath{clip}%
\pgfsetbuttcap%
\pgfsetroundjoin%
\definecolor{currentfill}{rgb}{0.121569,0.466667,0.705882}%
\pgfsetfillcolor{currentfill}%
\pgfsetlinewidth{0.481800pt}%
\definecolor{currentstroke}{rgb}{1.000000,1.000000,1.000000}%
\pgfsetstrokecolor{currentstroke}%
\pgfsetdash{}{0pt}%
\pgfpathmoveto{\pgfqpoint{5.546420in}{0.794348in}}%
\pgfpathcurveto{\pgfqpoint{5.557470in}{0.794348in}}{\pgfqpoint{5.568069in}{0.798739in}}{\pgfqpoint{5.575883in}{0.806552in}}%
\pgfpathcurveto{\pgfqpoint{5.583697in}{0.814366in}}{\pgfqpoint{5.588087in}{0.824965in}}{\pgfqpoint{5.588087in}{0.836015in}}%
\pgfpathcurveto{\pgfqpoint{5.588087in}{0.847065in}}{\pgfqpoint{5.583697in}{0.857664in}}{\pgfqpoint{5.575883in}{0.865478in}}%
\pgfpathcurveto{\pgfqpoint{5.568069in}{0.873291in}}{\pgfqpoint{5.557470in}{0.877682in}}{\pgfqpoint{5.546420in}{0.877682in}}%
\pgfpathcurveto{\pgfqpoint{5.535370in}{0.877682in}}{\pgfqpoint{5.524771in}{0.873291in}}{\pgfqpoint{5.516957in}{0.865478in}}%
\pgfpathcurveto{\pgfqpoint{5.509144in}{0.857664in}}{\pgfqpoint{5.504753in}{0.847065in}}{\pgfqpoint{5.504753in}{0.836015in}}%
\pgfpathcurveto{\pgfqpoint{5.504753in}{0.824965in}}{\pgfqpoint{5.509144in}{0.814366in}}{\pgfqpoint{5.516957in}{0.806552in}}%
\pgfpathcurveto{\pgfqpoint{5.524771in}{0.798739in}}{\pgfqpoint{5.535370in}{0.794348in}}{\pgfqpoint{5.546420in}{0.794348in}}%
\pgfpathlineto{\pgfqpoint{5.546420in}{0.794348in}}%
\pgfpathclose%
\pgfusepath{stroke,fill}%
\end{pgfscope}%
\begin{pgfscope}%
\pgfpathrectangle{\pgfqpoint{5.292946in}{0.569136in}}{\pgfqpoint{2.177280in}{2.201755in}}%
\pgfusepath{clip}%
\pgfsetbuttcap%
\pgfsetroundjoin%
\definecolor{currentfill}{rgb}{0.121569,0.466667,0.705882}%
\pgfsetfillcolor{currentfill}%
\pgfsetlinewidth{0.481800pt}%
\definecolor{currentstroke}{rgb}{1.000000,1.000000,1.000000}%
\pgfsetstrokecolor{currentstroke}%
\pgfsetdash{}{0pt}%
\pgfpathmoveto{\pgfqpoint{5.546420in}{0.710948in}}%
\pgfpathcurveto{\pgfqpoint{5.557470in}{0.710948in}}{\pgfqpoint{5.568069in}{0.715339in}}{\pgfqpoint{5.575883in}{0.723152in}}%
\pgfpathcurveto{\pgfqpoint{5.583697in}{0.730966in}}{\pgfqpoint{5.588087in}{0.741565in}}{\pgfqpoint{5.588087in}{0.752615in}}%
\pgfpathcurveto{\pgfqpoint{5.588087in}{0.763665in}}{\pgfqpoint{5.583697in}{0.774264in}}{\pgfqpoint{5.575883in}{0.782078in}}%
\pgfpathcurveto{\pgfqpoint{5.568069in}{0.789892in}}{\pgfqpoint{5.557470in}{0.794282in}}{\pgfqpoint{5.546420in}{0.794282in}}%
\pgfpathcurveto{\pgfqpoint{5.535370in}{0.794282in}}{\pgfqpoint{5.524771in}{0.789892in}}{\pgfqpoint{5.516957in}{0.782078in}}%
\pgfpathcurveto{\pgfqpoint{5.509144in}{0.774264in}}{\pgfqpoint{5.504753in}{0.763665in}}{\pgfqpoint{5.504753in}{0.752615in}}%
\pgfpathcurveto{\pgfqpoint{5.504753in}{0.741565in}}{\pgfqpoint{5.509144in}{0.730966in}}{\pgfqpoint{5.516957in}{0.723152in}}%
\pgfpathcurveto{\pgfqpoint{5.524771in}{0.715339in}}{\pgfqpoint{5.535370in}{0.710948in}}{\pgfqpoint{5.546420in}{0.710948in}}%
\pgfpathlineto{\pgfqpoint{5.546420in}{0.710948in}}%
\pgfpathclose%
\pgfusepath{stroke,fill}%
\end{pgfscope}%
\begin{pgfscope}%
\pgfpathrectangle{\pgfqpoint{5.292946in}{0.569136in}}{\pgfqpoint{2.177280in}{2.201755in}}%
\pgfusepath{clip}%
\pgfsetbuttcap%
\pgfsetroundjoin%
\definecolor{currentfill}{rgb}{0.121569,0.466667,0.705882}%
\pgfsetfillcolor{currentfill}%
\pgfsetlinewidth{0.481800pt}%
\definecolor{currentstroke}{rgb}{1.000000,1.000000,1.000000}%
\pgfsetstrokecolor{currentstroke}%
\pgfsetdash{}{0pt}%
\pgfpathmoveto{\pgfqpoint{5.632536in}{1.044548in}}%
\pgfpathcurveto{\pgfqpoint{5.643586in}{1.044548in}}{\pgfqpoint{5.654185in}{1.048938in}}{\pgfqpoint{5.661999in}{1.056752in}}%
\pgfpathcurveto{\pgfqpoint{5.669812in}{1.064565in}}{\pgfqpoint{5.674203in}{1.075164in}}{\pgfqpoint{5.674203in}{1.086214in}}%
\pgfpathcurveto{\pgfqpoint{5.674203in}{1.097265in}}{\pgfqpoint{5.669812in}{1.107864in}}{\pgfqpoint{5.661999in}{1.115677in}}%
\pgfpathcurveto{\pgfqpoint{5.654185in}{1.123491in}}{\pgfqpoint{5.643586in}{1.127881in}}{\pgfqpoint{5.632536in}{1.127881in}}%
\pgfpathcurveto{\pgfqpoint{5.621486in}{1.127881in}}{\pgfqpoint{5.610887in}{1.123491in}}{\pgfqpoint{5.603073in}{1.115677in}}%
\pgfpathcurveto{\pgfqpoint{5.595260in}{1.107864in}}{\pgfqpoint{5.590869in}{1.097265in}}{\pgfqpoint{5.590869in}{1.086214in}}%
\pgfpathcurveto{\pgfqpoint{5.590869in}{1.075164in}}{\pgfqpoint{5.595260in}{1.064565in}}{\pgfqpoint{5.603073in}{1.056752in}}%
\pgfpathcurveto{\pgfqpoint{5.610887in}{1.048938in}}{\pgfqpoint{5.621486in}{1.044548in}}{\pgfqpoint{5.632536in}{1.044548in}}%
\pgfpathlineto{\pgfqpoint{5.632536in}{1.044548in}}%
\pgfpathclose%
\pgfusepath{stroke,fill}%
\end{pgfscope}%
\begin{pgfscope}%
\pgfpathrectangle{\pgfqpoint{5.292946in}{0.569136in}}{\pgfqpoint{2.177280in}{2.201755in}}%
\pgfusepath{clip}%
\pgfsetbuttcap%
\pgfsetroundjoin%
\definecolor{currentfill}{rgb}{0.121569,0.466667,0.705882}%
\pgfsetfillcolor{currentfill}%
\pgfsetlinewidth{0.481800pt}%
\definecolor{currentstroke}{rgb}{1.000000,1.000000,1.000000}%
\pgfsetstrokecolor{currentstroke}%
\pgfsetdash{}{0pt}%
\pgfpathmoveto{\pgfqpoint{5.718652in}{0.877748in}}%
\pgfpathcurveto{\pgfqpoint{5.729702in}{0.877748in}}{\pgfqpoint{5.740301in}{0.882138in}}{\pgfqpoint{5.748115in}{0.889952in}}%
\pgfpathcurveto{\pgfqpoint{5.755928in}{0.897766in}}{\pgfqpoint{5.760319in}{0.908365in}}{\pgfqpoint{5.760319in}{0.919415in}}%
\pgfpathcurveto{\pgfqpoint{5.760319in}{0.930465in}}{\pgfqpoint{5.755928in}{0.941064in}}{\pgfqpoint{5.748115in}{0.948878in}}%
\pgfpathcurveto{\pgfqpoint{5.740301in}{0.956691in}}{\pgfqpoint{5.729702in}{0.961081in}}{\pgfqpoint{5.718652in}{0.961081in}}%
\pgfpathcurveto{\pgfqpoint{5.707602in}{0.961081in}}{\pgfqpoint{5.697003in}{0.956691in}}{\pgfqpoint{5.689189in}{0.948878in}}%
\pgfpathcurveto{\pgfqpoint{5.681375in}{0.941064in}}{\pgfqpoint{5.676985in}{0.930465in}}{\pgfqpoint{5.676985in}{0.919415in}}%
\pgfpathcurveto{\pgfqpoint{5.676985in}{0.908365in}}{\pgfqpoint{5.681375in}{0.897766in}}{\pgfqpoint{5.689189in}{0.889952in}}%
\pgfpathcurveto{\pgfqpoint{5.697003in}{0.882138in}}{\pgfqpoint{5.707602in}{0.877748in}}{\pgfqpoint{5.718652in}{0.877748in}}%
\pgfpathlineto{\pgfqpoint{5.718652in}{0.877748in}}%
\pgfpathclose%
\pgfusepath{stroke,fill}%
\end{pgfscope}%
\begin{pgfscope}%
\pgfpathrectangle{\pgfqpoint{5.292946in}{0.569136in}}{\pgfqpoint{2.177280in}{2.201755in}}%
\pgfusepath{clip}%
\pgfsetbuttcap%
\pgfsetroundjoin%
\definecolor{currentfill}{rgb}{0.121569,0.466667,0.705882}%
\pgfsetfillcolor{currentfill}%
\pgfsetlinewidth{0.481800pt}%
\definecolor{currentstroke}{rgb}{1.000000,1.000000,1.000000}%
\pgfsetstrokecolor{currentstroke}%
\pgfsetdash{}{0pt}%
\pgfpathmoveto{\pgfqpoint{5.575125in}{0.794348in}}%
\pgfpathcurveto{\pgfqpoint{5.586176in}{0.794348in}}{\pgfqpoint{5.596775in}{0.798739in}}{\pgfqpoint{5.604588in}{0.806552in}}%
\pgfpathcurveto{\pgfqpoint{5.612402in}{0.814366in}}{\pgfqpoint{5.616792in}{0.824965in}}{\pgfqpoint{5.616792in}{0.836015in}}%
\pgfpathcurveto{\pgfqpoint{5.616792in}{0.847065in}}{\pgfqpoint{5.612402in}{0.857664in}}{\pgfqpoint{5.604588in}{0.865478in}}%
\pgfpathcurveto{\pgfqpoint{5.596775in}{0.873291in}}{\pgfqpoint{5.586176in}{0.877682in}}{\pgfqpoint{5.575125in}{0.877682in}}%
\pgfpathcurveto{\pgfqpoint{5.564075in}{0.877682in}}{\pgfqpoint{5.553476in}{0.873291in}}{\pgfqpoint{5.545663in}{0.865478in}}%
\pgfpathcurveto{\pgfqpoint{5.537849in}{0.857664in}}{\pgfqpoint{5.533459in}{0.847065in}}{\pgfqpoint{5.533459in}{0.836015in}}%
\pgfpathcurveto{\pgfqpoint{5.533459in}{0.824965in}}{\pgfqpoint{5.537849in}{0.814366in}}{\pgfqpoint{5.545663in}{0.806552in}}%
\pgfpathcurveto{\pgfqpoint{5.553476in}{0.798739in}}{\pgfqpoint{5.564075in}{0.794348in}}{\pgfqpoint{5.575125in}{0.794348in}}%
\pgfpathlineto{\pgfqpoint{5.575125in}{0.794348in}}%
\pgfpathclose%
\pgfusepath{stroke,fill}%
\end{pgfscope}%
\begin{pgfscope}%
\pgfpathrectangle{\pgfqpoint{5.292946in}{0.569136in}}{\pgfqpoint{2.177280in}{2.201755in}}%
\pgfusepath{clip}%
\pgfsetbuttcap%
\pgfsetroundjoin%
\definecolor{currentfill}{rgb}{0.121569,0.466667,0.705882}%
\pgfsetfillcolor{currentfill}%
\pgfsetlinewidth{0.481800pt}%
\definecolor{currentstroke}{rgb}{1.000000,1.000000,1.000000}%
\pgfsetstrokecolor{currentstroke}%
\pgfsetdash{}{0pt}%
\pgfpathmoveto{\pgfqpoint{5.632536in}{0.710948in}}%
\pgfpathcurveto{\pgfqpoint{5.643586in}{0.710948in}}{\pgfqpoint{5.654185in}{0.715339in}}{\pgfqpoint{5.661999in}{0.723152in}}%
\pgfpathcurveto{\pgfqpoint{5.669812in}{0.730966in}}{\pgfqpoint{5.674203in}{0.741565in}}{\pgfqpoint{5.674203in}{0.752615in}}%
\pgfpathcurveto{\pgfqpoint{5.674203in}{0.763665in}}{\pgfqpoint{5.669812in}{0.774264in}}{\pgfqpoint{5.661999in}{0.782078in}}%
\pgfpathcurveto{\pgfqpoint{5.654185in}{0.789892in}}{\pgfqpoint{5.643586in}{0.794282in}}{\pgfqpoint{5.632536in}{0.794282in}}%
\pgfpathcurveto{\pgfqpoint{5.621486in}{0.794282in}}{\pgfqpoint{5.610887in}{0.789892in}}{\pgfqpoint{5.603073in}{0.782078in}}%
\pgfpathcurveto{\pgfqpoint{5.595260in}{0.774264in}}{\pgfqpoint{5.590869in}{0.763665in}}{\pgfqpoint{5.590869in}{0.752615in}}%
\pgfpathcurveto{\pgfqpoint{5.590869in}{0.741565in}}{\pgfqpoint{5.595260in}{0.730966in}}{\pgfqpoint{5.603073in}{0.723152in}}%
\pgfpathcurveto{\pgfqpoint{5.610887in}{0.715339in}}{\pgfqpoint{5.621486in}{0.710948in}}{\pgfqpoint{5.632536in}{0.710948in}}%
\pgfpathlineto{\pgfqpoint{5.632536in}{0.710948in}}%
\pgfpathclose%
\pgfusepath{stroke,fill}%
\end{pgfscope}%
\begin{pgfscope}%
\pgfpathrectangle{\pgfqpoint{5.292946in}{0.569136in}}{\pgfqpoint{2.177280in}{2.201755in}}%
\pgfusepath{clip}%
\pgfsetbuttcap%
\pgfsetroundjoin%
\definecolor{currentfill}{rgb}{0.121569,0.466667,0.705882}%
\pgfsetfillcolor{currentfill}%
\pgfsetlinewidth{0.481800pt}%
\definecolor{currentstroke}{rgb}{1.000000,1.000000,1.000000}%
\pgfsetstrokecolor{currentstroke}%
\pgfsetdash{}{0pt}%
\pgfpathmoveto{\pgfqpoint{5.575125in}{0.710948in}}%
\pgfpathcurveto{\pgfqpoint{5.586176in}{0.710948in}}{\pgfqpoint{5.596775in}{0.715339in}}{\pgfqpoint{5.604588in}{0.723152in}}%
\pgfpathcurveto{\pgfqpoint{5.612402in}{0.730966in}}{\pgfqpoint{5.616792in}{0.741565in}}{\pgfqpoint{5.616792in}{0.752615in}}%
\pgfpathcurveto{\pgfqpoint{5.616792in}{0.763665in}}{\pgfqpoint{5.612402in}{0.774264in}}{\pgfqpoint{5.604588in}{0.782078in}}%
\pgfpathcurveto{\pgfqpoint{5.596775in}{0.789892in}}{\pgfqpoint{5.586176in}{0.794282in}}{\pgfqpoint{5.575125in}{0.794282in}}%
\pgfpathcurveto{\pgfqpoint{5.564075in}{0.794282in}}{\pgfqpoint{5.553476in}{0.789892in}}{\pgfqpoint{5.545663in}{0.782078in}}%
\pgfpathcurveto{\pgfqpoint{5.537849in}{0.774264in}}{\pgfqpoint{5.533459in}{0.763665in}}{\pgfqpoint{5.533459in}{0.752615in}}%
\pgfpathcurveto{\pgfqpoint{5.533459in}{0.741565in}}{\pgfqpoint{5.537849in}{0.730966in}}{\pgfqpoint{5.545663in}{0.723152in}}%
\pgfpathcurveto{\pgfqpoint{5.553476in}{0.715339in}}{\pgfqpoint{5.564075in}{0.710948in}}{\pgfqpoint{5.575125in}{0.710948in}}%
\pgfpathlineto{\pgfqpoint{5.575125in}{0.710948in}}%
\pgfpathclose%
\pgfusepath{stroke,fill}%
\end{pgfscope}%
\begin{pgfscope}%
\pgfpathrectangle{\pgfqpoint{5.292946in}{0.569136in}}{\pgfqpoint{2.177280in}{2.201755in}}%
\pgfusepath{clip}%
\pgfsetbuttcap%
\pgfsetroundjoin%
\definecolor{currentfill}{rgb}{0.121569,0.466667,0.705882}%
\pgfsetfillcolor{currentfill}%
\pgfsetlinewidth{0.481800pt}%
\definecolor{currentstroke}{rgb}{1.000000,1.000000,1.000000}%
\pgfsetstrokecolor{currentstroke}%
\pgfsetdash{}{0pt}%
\pgfpathmoveto{\pgfqpoint{5.603831in}{0.710948in}}%
\pgfpathcurveto{\pgfqpoint{5.614881in}{0.710948in}}{\pgfqpoint{5.625480in}{0.715339in}}{\pgfqpoint{5.633294in}{0.723152in}}%
\pgfpathcurveto{\pgfqpoint{5.641107in}{0.730966in}}{\pgfqpoint{5.645497in}{0.741565in}}{\pgfqpoint{5.645497in}{0.752615in}}%
\pgfpathcurveto{\pgfqpoint{5.645497in}{0.763665in}}{\pgfqpoint{5.641107in}{0.774264in}}{\pgfqpoint{5.633294in}{0.782078in}}%
\pgfpathcurveto{\pgfqpoint{5.625480in}{0.789892in}}{\pgfqpoint{5.614881in}{0.794282in}}{\pgfqpoint{5.603831in}{0.794282in}}%
\pgfpathcurveto{\pgfqpoint{5.592781in}{0.794282in}}{\pgfqpoint{5.582182in}{0.789892in}}{\pgfqpoint{5.574368in}{0.782078in}}%
\pgfpathcurveto{\pgfqpoint{5.566554in}{0.774264in}}{\pgfqpoint{5.562164in}{0.763665in}}{\pgfqpoint{5.562164in}{0.752615in}}%
\pgfpathcurveto{\pgfqpoint{5.562164in}{0.741565in}}{\pgfqpoint{5.566554in}{0.730966in}}{\pgfqpoint{5.574368in}{0.723152in}}%
\pgfpathcurveto{\pgfqpoint{5.582182in}{0.715339in}}{\pgfqpoint{5.592781in}{0.710948in}}{\pgfqpoint{5.603831in}{0.710948in}}%
\pgfpathlineto{\pgfqpoint{5.603831in}{0.710948in}}%
\pgfpathclose%
\pgfusepath{stroke,fill}%
\end{pgfscope}%
\begin{pgfscope}%
\pgfpathrectangle{\pgfqpoint{5.292946in}{0.569136in}}{\pgfqpoint{2.177280in}{2.201755in}}%
\pgfusepath{clip}%
\pgfsetbuttcap%
\pgfsetroundjoin%
\definecolor{currentfill}{rgb}{0.121569,0.466667,0.705882}%
\pgfsetfillcolor{currentfill}%
\pgfsetlinewidth{0.481800pt}%
\definecolor{currentstroke}{rgb}{1.000000,1.000000,1.000000}%
\pgfsetstrokecolor{currentstroke}%
\pgfsetdash{}{0pt}%
\pgfpathmoveto{\pgfqpoint{5.575125in}{0.710948in}}%
\pgfpathcurveto{\pgfqpoint{5.586176in}{0.710948in}}{\pgfqpoint{5.596775in}{0.715339in}}{\pgfqpoint{5.604588in}{0.723152in}}%
\pgfpathcurveto{\pgfqpoint{5.612402in}{0.730966in}}{\pgfqpoint{5.616792in}{0.741565in}}{\pgfqpoint{5.616792in}{0.752615in}}%
\pgfpathcurveto{\pgfqpoint{5.616792in}{0.763665in}}{\pgfqpoint{5.612402in}{0.774264in}}{\pgfqpoint{5.604588in}{0.782078in}}%
\pgfpathcurveto{\pgfqpoint{5.596775in}{0.789892in}}{\pgfqpoint{5.586176in}{0.794282in}}{\pgfqpoint{5.575125in}{0.794282in}}%
\pgfpathcurveto{\pgfqpoint{5.564075in}{0.794282in}}{\pgfqpoint{5.553476in}{0.789892in}}{\pgfqpoint{5.545663in}{0.782078in}}%
\pgfpathcurveto{\pgfqpoint{5.537849in}{0.774264in}}{\pgfqpoint{5.533459in}{0.763665in}}{\pgfqpoint{5.533459in}{0.752615in}}%
\pgfpathcurveto{\pgfqpoint{5.533459in}{0.741565in}}{\pgfqpoint{5.537849in}{0.730966in}}{\pgfqpoint{5.545663in}{0.723152in}}%
\pgfpathcurveto{\pgfqpoint{5.553476in}{0.715339in}}{\pgfqpoint{5.564075in}{0.710948in}}{\pgfqpoint{5.575125in}{0.710948in}}%
\pgfpathlineto{\pgfqpoint{5.575125in}{0.710948in}}%
\pgfpathclose%
\pgfusepath{stroke,fill}%
\end{pgfscope}%
\begin{pgfscope}%
\pgfpathrectangle{\pgfqpoint{5.292946in}{0.569136in}}{\pgfqpoint{2.177280in}{2.201755in}}%
\pgfusepath{clip}%
\pgfsetbuttcap%
\pgfsetroundjoin%
\definecolor{currentfill}{rgb}{1.000000,0.498039,0.054902}%
\pgfsetfillcolor{currentfill}%
\pgfsetlinewidth{0.481800pt}%
\definecolor{currentstroke}{rgb}{1.000000,1.000000,1.000000}%
\pgfsetstrokecolor{currentstroke}%
\pgfsetdash{}{0pt}%
\pgfpathmoveto{\pgfqpoint{6.522400in}{1.711746in}}%
\pgfpathcurveto{\pgfqpoint{6.533450in}{1.711746in}}{\pgfqpoint{6.544049in}{1.716137in}}{\pgfqpoint{6.551863in}{1.723950in}}%
\pgfpathcurveto{\pgfqpoint{6.559676in}{1.731764in}}{\pgfqpoint{6.564067in}{1.742363in}}{\pgfqpoint{6.564067in}{1.753413in}}%
\pgfpathcurveto{\pgfqpoint{6.564067in}{1.764463in}}{\pgfqpoint{6.559676in}{1.775062in}}{\pgfqpoint{6.551863in}{1.782876in}}%
\pgfpathcurveto{\pgfqpoint{6.544049in}{1.790689in}}{\pgfqpoint{6.533450in}{1.795080in}}{\pgfqpoint{6.522400in}{1.795080in}}%
\pgfpathcurveto{\pgfqpoint{6.511350in}{1.795080in}}{\pgfqpoint{6.500751in}{1.790689in}}{\pgfqpoint{6.492937in}{1.782876in}}%
\pgfpathcurveto{\pgfqpoint{6.485123in}{1.775062in}}{\pgfqpoint{6.480733in}{1.764463in}}{\pgfqpoint{6.480733in}{1.753413in}}%
\pgfpathcurveto{\pgfqpoint{6.480733in}{1.742363in}}{\pgfqpoint{6.485123in}{1.731764in}}{\pgfqpoint{6.492937in}{1.723950in}}%
\pgfpathcurveto{\pgfqpoint{6.500751in}{1.716137in}}{\pgfqpoint{6.511350in}{1.711746in}}{\pgfqpoint{6.522400in}{1.711746in}}%
\pgfpathlineto{\pgfqpoint{6.522400in}{1.711746in}}%
\pgfpathclose%
\pgfusepath{stroke,fill}%
\end{pgfscope}%
\begin{pgfscope}%
\pgfpathrectangle{\pgfqpoint{5.292946in}{0.569136in}}{\pgfqpoint{2.177280in}{2.201755in}}%
\pgfusepath{clip}%
\pgfsetbuttcap%
\pgfsetroundjoin%
\definecolor{currentfill}{rgb}{1.000000,0.498039,0.054902}%
\pgfsetfillcolor{currentfill}%
\pgfsetlinewidth{0.481800pt}%
\definecolor{currentstroke}{rgb}{1.000000,1.000000,1.000000}%
\pgfsetstrokecolor{currentstroke}%
\pgfsetdash{}{0pt}%
\pgfpathmoveto{\pgfqpoint{6.464989in}{1.795146in}}%
\pgfpathcurveto{\pgfqpoint{6.476039in}{1.795146in}}{\pgfqpoint{6.486638in}{1.799536in}}{\pgfqpoint{6.494452in}{1.807350in}}%
\pgfpathcurveto{\pgfqpoint{6.502266in}{1.815164in}}{\pgfqpoint{6.506656in}{1.825763in}}{\pgfqpoint{6.506656in}{1.836813in}}%
\pgfpathcurveto{\pgfqpoint{6.506656in}{1.847863in}}{\pgfqpoint{6.502266in}{1.858462in}}{\pgfqpoint{6.494452in}{1.866276in}}%
\pgfpathcurveto{\pgfqpoint{6.486638in}{1.874089in}}{\pgfqpoint{6.476039in}{1.878479in}}{\pgfqpoint{6.464989in}{1.878479in}}%
\pgfpathcurveto{\pgfqpoint{6.453939in}{1.878479in}}{\pgfqpoint{6.443340in}{1.874089in}}{\pgfqpoint{6.435527in}{1.866276in}}%
\pgfpathcurveto{\pgfqpoint{6.427713in}{1.858462in}}{\pgfqpoint{6.423323in}{1.847863in}}{\pgfqpoint{6.423323in}{1.836813in}}%
\pgfpathcurveto{\pgfqpoint{6.423323in}{1.825763in}}{\pgfqpoint{6.427713in}{1.815164in}}{\pgfqpoint{6.435527in}{1.807350in}}%
\pgfpathcurveto{\pgfqpoint{6.443340in}{1.799536in}}{\pgfqpoint{6.453939in}{1.795146in}}{\pgfqpoint{6.464989in}{1.795146in}}%
\pgfpathlineto{\pgfqpoint{6.464989in}{1.795146in}}%
\pgfpathclose%
\pgfusepath{stroke,fill}%
\end{pgfscope}%
\begin{pgfscope}%
\pgfpathrectangle{\pgfqpoint{5.292946in}{0.569136in}}{\pgfqpoint{2.177280in}{2.201755in}}%
\pgfusepath{clip}%
\pgfsetbuttcap%
\pgfsetroundjoin%
\definecolor{currentfill}{rgb}{1.000000,0.498039,0.054902}%
\pgfsetfillcolor{currentfill}%
\pgfsetlinewidth{0.481800pt}%
\definecolor{currentstroke}{rgb}{1.000000,1.000000,1.000000}%
\pgfsetstrokecolor{currentstroke}%
\pgfsetdash{}{0pt}%
\pgfpathmoveto{\pgfqpoint{6.579810in}{1.795146in}}%
\pgfpathcurveto{\pgfqpoint{6.590861in}{1.795146in}}{\pgfqpoint{6.601460in}{1.799536in}}{\pgfqpoint{6.609273in}{1.807350in}}%
\pgfpathcurveto{\pgfqpoint{6.617087in}{1.815164in}}{\pgfqpoint{6.621477in}{1.825763in}}{\pgfqpoint{6.621477in}{1.836813in}}%
\pgfpathcurveto{\pgfqpoint{6.621477in}{1.847863in}}{\pgfqpoint{6.617087in}{1.858462in}}{\pgfqpoint{6.609273in}{1.866276in}}%
\pgfpathcurveto{\pgfqpoint{6.601460in}{1.874089in}}{\pgfqpoint{6.590861in}{1.878479in}}{\pgfqpoint{6.579810in}{1.878479in}}%
\pgfpathcurveto{\pgfqpoint{6.568760in}{1.878479in}}{\pgfqpoint{6.558161in}{1.874089in}}{\pgfqpoint{6.550348in}{1.866276in}}%
\pgfpathcurveto{\pgfqpoint{6.542534in}{1.858462in}}{\pgfqpoint{6.538144in}{1.847863in}}{\pgfqpoint{6.538144in}{1.836813in}}%
\pgfpathcurveto{\pgfqpoint{6.538144in}{1.825763in}}{\pgfqpoint{6.542534in}{1.815164in}}{\pgfqpoint{6.550348in}{1.807350in}}%
\pgfpathcurveto{\pgfqpoint{6.558161in}{1.799536in}}{\pgfqpoint{6.568760in}{1.795146in}}{\pgfqpoint{6.579810in}{1.795146in}}%
\pgfpathlineto{\pgfqpoint{6.579810in}{1.795146in}}%
\pgfpathclose%
\pgfusepath{stroke,fill}%
\end{pgfscope}%
\begin{pgfscope}%
\pgfpathrectangle{\pgfqpoint{5.292946in}{0.569136in}}{\pgfqpoint{2.177280in}{2.201755in}}%
\pgfusepath{clip}%
\pgfsetbuttcap%
\pgfsetroundjoin%
\definecolor{currentfill}{rgb}{1.000000,0.498039,0.054902}%
\pgfsetfillcolor{currentfill}%
\pgfsetlinewidth{0.481800pt}%
\definecolor{currentstroke}{rgb}{1.000000,1.000000,1.000000}%
\pgfsetstrokecolor{currentstroke}%
\pgfsetdash{}{0pt}%
\pgfpathmoveto{\pgfqpoint{6.321463in}{1.628346in}}%
\pgfpathcurveto{\pgfqpoint{6.332513in}{1.628346in}}{\pgfqpoint{6.343112in}{1.632737in}}{\pgfqpoint{6.350926in}{1.640550in}}%
\pgfpathcurveto{\pgfqpoint{6.358739in}{1.648364in}}{\pgfqpoint{6.363130in}{1.658963in}}{\pgfqpoint{6.363130in}{1.670013in}}%
\pgfpathcurveto{\pgfqpoint{6.363130in}{1.681063in}}{\pgfqpoint{6.358739in}{1.691662in}}{\pgfqpoint{6.350926in}{1.699476in}}%
\pgfpathcurveto{\pgfqpoint{6.343112in}{1.707290in}}{\pgfqpoint{6.332513in}{1.711680in}}{\pgfqpoint{6.321463in}{1.711680in}}%
\pgfpathcurveto{\pgfqpoint{6.310413in}{1.711680in}}{\pgfqpoint{6.299814in}{1.707290in}}{\pgfqpoint{6.292000in}{1.699476in}}%
\pgfpathcurveto{\pgfqpoint{6.284186in}{1.691662in}}{\pgfqpoint{6.279796in}{1.681063in}}{\pgfqpoint{6.279796in}{1.670013in}}%
\pgfpathcurveto{\pgfqpoint{6.279796in}{1.658963in}}{\pgfqpoint{6.284186in}{1.648364in}}{\pgfqpoint{6.292000in}{1.640550in}}%
\pgfpathcurveto{\pgfqpoint{6.299814in}{1.632737in}}{\pgfqpoint{6.310413in}{1.628346in}}{\pgfqpoint{6.321463in}{1.628346in}}%
\pgfpathlineto{\pgfqpoint{6.321463in}{1.628346in}}%
\pgfpathclose%
\pgfusepath{stroke,fill}%
\end{pgfscope}%
\begin{pgfscope}%
\pgfpathrectangle{\pgfqpoint{5.292946in}{0.569136in}}{\pgfqpoint{2.177280in}{2.201755in}}%
\pgfusepath{clip}%
\pgfsetbuttcap%
\pgfsetroundjoin%
\definecolor{currentfill}{rgb}{1.000000,0.498039,0.054902}%
\pgfsetfillcolor{currentfill}%
\pgfsetlinewidth{0.481800pt}%
\definecolor{currentstroke}{rgb}{1.000000,1.000000,1.000000}%
\pgfsetstrokecolor{currentstroke}%
\pgfsetdash{}{0pt}%
\pgfpathmoveto{\pgfqpoint{6.493695in}{1.795146in}}%
\pgfpathcurveto{\pgfqpoint{6.504745in}{1.795146in}}{\pgfqpoint{6.515344in}{1.799536in}}{\pgfqpoint{6.523157in}{1.807350in}}%
\pgfpathcurveto{\pgfqpoint{6.530971in}{1.815164in}}{\pgfqpoint{6.535361in}{1.825763in}}{\pgfqpoint{6.535361in}{1.836813in}}%
\pgfpathcurveto{\pgfqpoint{6.535361in}{1.847863in}}{\pgfqpoint{6.530971in}{1.858462in}}{\pgfqpoint{6.523157in}{1.866276in}}%
\pgfpathcurveto{\pgfqpoint{6.515344in}{1.874089in}}{\pgfqpoint{6.504745in}{1.878479in}}{\pgfqpoint{6.493695in}{1.878479in}}%
\pgfpathcurveto{\pgfqpoint{6.482644in}{1.878479in}}{\pgfqpoint{6.472045in}{1.874089in}}{\pgfqpoint{6.464232in}{1.866276in}}%
\pgfpathcurveto{\pgfqpoint{6.456418in}{1.858462in}}{\pgfqpoint{6.452028in}{1.847863in}}{\pgfqpoint{6.452028in}{1.836813in}}%
\pgfpathcurveto{\pgfqpoint{6.452028in}{1.825763in}}{\pgfqpoint{6.456418in}{1.815164in}}{\pgfqpoint{6.464232in}{1.807350in}}%
\pgfpathcurveto{\pgfqpoint{6.472045in}{1.799536in}}{\pgfqpoint{6.482644in}{1.795146in}}{\pgfqpoint{6.493695in}{1.795146in}}%
\pgfpathlineto{\pgfqpoint{6.493695in}{1.795146in}}%
\pgfpathclose%
\pgfusepath{stroke,fill}%
\end{pgfscope}%
\begin{pgfscope}%
\pgfpathrectangle{\pgfqpoint{5.292946in}{0.569136in}}{\pgfqpoint{2.177280in}{2.201755in}}%
\pgfusepath{clip}%
\pgfsetbuttcap%
\pgfsetroundjoin%
\definecolor{currentfill}{rgb}{1.000000,0.498039,0.054902}%
\pgfsetfillcolor{currentfill}%
\pgfsetlinewidth{0.481800pt}%
\definecolor{currentstroke}{rgb}{1.000000,1.000000,1.000000}%
\pgfsetstrokecolor{currentstroke}%
\pgfsetdash{}{0pt}%
\pgfpathmoveto{\pgfqpoint{6.464989in}{1.628346in}}%
\pgfpathcurveto{\pgfqpoint{6.476039in}{1.628346in}}{\pgfqpoint{6.486638in}{1.632737in}}{\pgfqpoint{6.494452in}{1.640550in}}%
\pgfpathcurveto{\pgfqpoint{6.502266in}{1.648364in}}{\pgfqpoint{6.506656in}{1.658963in}}{\pgfqpoint{6.506656in}{1.670013in}}%
\pgfpathcurveto{\pgfqpoint{6.506656in}{1.681063in}}{\pgfqpoint{6.502266in}{1.691662in}}{\pgfqpoint{6.494452in}{1.699476in}}%
\pgfpathcurveto{\pgfqpoint{6.486638in}{1.707290in}}{\pgfqpoint{6.476039in}{1.711680in}}{\pgfqpoint{6.464989in}{1.711680in}}%
\pgfpathcurveto{\pgfqpoint{6.453939in}{1.711680in}}{\pgfqpoint{6.443340in}{1.707290in}}{\pgfqpoint{6.435527in}{1.699476in}}%
\pgfpathcurveto{\pgfqpoint{6.427713in}{1.691662in}}{\pgfqpoint{6.423323in}{1.681063in}}{\pgfqpoint{6.423323in}{1.670013in}}%
\pgfpathcurveto{\pgfqpoint{6.423323in}{1.658963in}}{\pgfqpoint{6.427713in}{1.648364in}}{\pgfqpoint{6.435527in}{1.640550in}}%
\pgfpathcurveto{\pgfqpoint{6.443340in}{1.632737in}}{\pgfqpoint{6.453939in}{1.628346in}}{\pgfqpoint{6.464989in}{1.628346in}}%
\pgfpathlineto{\pgfqpoint{6.464989in}{1.628346in}}%
\pgfpathclose%
\pgfusepath{stroke,fill}%
\end{pgfscope}%
\begin{pgfscope}%
\pgfpathrectangle{\pgfqpoint{5.292946in}{0.569136in}}{\pgfqpoint{2.177280in}{2.201755in}}%
\pgfusepath{clip}%
\pgfsetbuttcap%
\pgfsetroundjoin%
\definecolor{currentfill}{rgb}{1.000000,0.498039,0.054902}%
\pgfsetfillcolor{currentfill}%
\pgfsetlinewidth{0.481800pt}%
\definecolor{currentstroke}{rgb}{1.000000,1.000000,1.000000}%
\pgfsetstrokecolor{currentstroke}%
\pgfsetdash{}{0pt}%
\pgfpathmoveto{\pgfqpoint{6.522400in}{1.878546in}}%
\pgfpathcurveto{\pgfqpoint{6.533450in}{1.878546in}}{\pgfqpoint{6.544049in}{1.882936in}}{\pgfqpoint{6.551863in}{1.890750in}}%
\pgfpathcurveto{\pgfqpoint{6.559676in}{1.898563in}}{\pgfqpoint{6.564067in}{1.909162in}}{\pgfqpoint{6.564067in}{1.920213in}}%
\pgfpathcurveto{\pgfqpoint{6.564067in}{1.931263in}}{\pgfqpoint{6.559676in}{1.941862in}}{\pgfqpoint{6.551863in}{1.949675in}}%
\pgfpathcurveto{\pgfqpoint{6.544049in}{1.957489in}}{\pgfqpoint{6.533450in}{1.961879in}}{\pgfqpoint{6.522400in}{1.961879in}}%
\pgfpathcurveto{\pgfqpoint{6.511350in}{1.961879in}}{\pgfqpoint{6.500751in}{1.957489in}}{\pgfqpoint{6.492937in}{1.949675in}}%
\pgfpathcurveto{\pgfqpoint{6.485123in}{1.941862in}}{\pgfqpoint{6.480733in}{1.931263in}}{\pgfqpoint{6.480733in}{1.920213in}}%
\pgfpathcurveto{\pgfqpoint{6.480733in}{1.909162in}}{\pgfqpoint{6.485123in}{1.898563in}}{\pgfqpoint{6.492937in}{1.890750in}}%
\pgfpathcurveto{\pgfqpoint{6.500751in}{1.882936in}}{\pgfqpoint{6.511350in}{1.878546in}}{\pgfqpoint{6.522400in}{1.878546in}}%
\pgfpathlineto{\pgfqpoint{6.522400in}{1.878546in}}%
\pgfpathclose%
\pgfusepath{stroke,fill}%
\end{pgfscope}%
\begin{pgfscope}%
\pgfpathrectangle{\pgfqpoint{5.292946in}{0.569136in}}{\pgfqpoint{2.177280in}{2.201755in}}%
\pgfusepath{clip}%
\pgfsetbuttcap%
\pgfsetroundjoin%
\definecolor{currentfill}{rgb}{1.000000,0.498039,0.054902}%
\pgfsetfillcolor{currentfill}%
\pgfsetlinewidth{0.481800pt}%
\definecolor{currentstroke}{rgb}{1.000000,1.000000,1.000000}%
\pgfsetstrokecolor{currentstroke}%
\pgfsetdash{}{0pt}%
\pgfpathmoveto{\pgfqpoint{6.120526in}{1.378147in}}%
\pgfpathcurveto{\pgfqpoint{6.131576in}{1.378147in}}{\pgfqpoint{6.142175in}{1.382537in}}{\pgfqpoint{6.149989in}{1.390351in}}%
\pgfpathcurveto{\pgfqpoint{6.157802in}{1.398164in}}{\pgfqpoint{6.162193in}{1.408764in}}{\pgfqpoint{6.162193in}{1.419814in}}%
\pgfpathcurveto{\pgfqpoint{6.162193in}{1.430864in}}{\pgfqpoint{6.157802in}{1.441463in}}{\pgfqpoint{6.149989in}{1.449276in}}%
\pgfpathcurveto{\pgfqpoint{6.142175in}{1.457090in}}{\pgfqpoint{6.131576in}{1.461480in}}{\pgfqpoint{6.120526in}{1.461480in}}%
\pgfpathcurveto{\pgfqpoint{6.109476in}{1.461480in}}{\pgfqpoint{6.098877in}{1.457090in}}{\pgfqpoint{6.091063in}{1.449276in}}%
\pgfpathcurveto{\pgfqpoint{6.083249in}{1.441463in}}{\pgfqpoint{6.078859in}{1.430864in}}{\pgfqpoint{6.078859in}{1.419814in}}%
\pgfpathcurveto{\pgfqpoint{6.078859in}{1.408764in}}{\pgfqpoint{6.083249in}{1.398164in}}{\pgfqpoint{6.091063in}{1.390351in}}%
\pgfpathcurveto{\pgfqpoint{6.098877in}{1.382537in}}{\pgfqpoint{6.109476in}{1.378147in}}{\pgfqpoint{6.120526in}{1.378147in}}%
\pgfpathlineto{\pgfqpoint{6.120526in}{1.378147in}}%
\pgfpathclose%
\pgfusepath{stroke,fill}%
\end{pgfscope}%
\begin{pgfscope}%
\pgfpathrectangle{\pgfqpoint{5.292946in}{0.569136in}}{\pgfqpoint{2.177280in}{2.201755in}}%
\pgfusepath{clip}%
\pgfsetbuttcap%
\pgfsetroundjoin%
\definecolor{currentfill}{rgb}{1.000000,0.498039,0.054902}%
\pgfsetfillcolor{currentfill}%
\pgfsetlinewidth{0.481800pt}%
\definecolor{currentstroke}{rgb}{1.000000,1.000000,1.000000}%
\pgfsetstrokecolor{currentstroke}%
\pgfsetdash{}{0pt}%
\pgfpathmoveto{\pgfqpoint{6.493695in}{1.628346in}}%
\pgfpathcurveto{\pgfqpoint{6.504745in}{1.628346in}}{\pgfqpoint{6.515344in}{1.632737in}}{\pgfqpoint{6.523157in}{1.640550in}}%
\pgfpathcurveto{\pgfqpoint{6.530971in}{1.648364in}}{\pgfqpoint{6.535361in}{1.658963in}}{\pgfqpoint{6.535361in}{1.670013in}}%
\pgfpathcurveto{\pgfqpoint{6.535361in}{1.681063in}}{\pgfqpoint{6.530971in}{1.691662in}}{\pgfqpoint{6.523157in}{1.699476in}}%
\pgfpathcurveto{\pgfqpoint{6.515344in}{1.707290in}}{\pgfqpoint{6.504745in}{1.711680in}}{\pgfqpoint{6.493695in}{1.711680in}}%
\pgfpathcurveto{\pgfqpoint{6.482644in}{1.711680in}}{\pgfqpoint{6.472045in}{1.707290in}}{\pgfqpoint{6.464232in}{1.699476in}}%
\pgfpathcurveto{\pgfqpoint{6.456418in}{1.691662in}}{\pgfqpoint{6.452028in}{1.681063in}}{\pgfqpoint{6.452028in}{1.670013in}}%
\pgfpathcurveto{\pgfqpoint{6.452028in}{1.658963in}}{\pgfqpoint{6.456418in}{1.648364in}}{\pgfqpoint{6.464232in}{1.640550in}}%
\pgfpathcurveto{\pgfqpoint{6.472045in}{1.632737in}}{\pgfqpoint{6.482644in}{1.628346in}}{\pgfqpoint{6.493695in}{1.628346in}}%
\pgfpathlineto{\pgfqpoint{6.493695in}{1.628346in}}%
\pgfpathclose%
\pgfusepath{stroke,fill}%
\end{pgfscope}%
\begin{pgfscope}%
\pgfpathrectangle{\pgfqpoint{5.292946in}{0.569136in}}{\pgfqpoint{2.177280in}{2.201755in}}%
\pgfusepath{clip}%
\pgfsetbuttcap%
\pgfsetroundjoin%
\definecolor{currentfill}{rgb}{1.000000,0.498039,0.054902}%
\pgfsetfillcolor{currentfill}%
\pgfsetlinewidth{0.481800pt}%
\definecolor{currentstroke}{rgb}{1.000000,1.000000,1.000000}%
\pgfsetstrokecolor{currentstroke}%
\pgfsetdash{}{0pt}%
\pgfpathmoveto{\pgfqpoint{6.292758in}{1.711746in}}%
\pgfpathcurveto{\pgfqpoint{6.303808in}{1.711746in}}{\pgfqpoint{6.314407in}{1.716137in}}{\pgfqpoint{6.322220in}{1.723950in}}%
\pgfpathcurveto{\pgfqpoint{6.330034in}{1.731764in}}{\pgfqpoint{6.334424in}{1.742363in}}{\pgfqpoint{6.334424in}{1.753413in}}%
\pgfpathcurveto{\pgfqpoint{6.334424in}{1.764463in}}{\pgfqpoint{6.330034in}{1.775062in}}{\pgfqpoint{6.322220in}{1.782876in}}%
\pgfpathcurveto{\pgfqpoint{6.314407in}{1.790689in}}{\pgfqpoint{6.303808in}{1.795080in}}{\pgfqpoint{6.292758in}{1.795080in}}%
\pgfpathcurveto{\pgfqpoint{6.281707in}{1.795080in}}{\pgfqpoint{6.271108in}{1.790689in}}{\pgfqpoint{6.263295in}{1.782876in}}%
\pgfpathcurveto{\pgfqpoint{6.255481in}{1.775062in}}{\pgfqpoint{6.251091in}{1.764463in}}{\pgfqpoint{6.251091in}{1.753413in}}%
\pgfpathcurveto{\pgfqpoint{6.251091in}{1.742363in}}{\pgfqpoint{6.255481in}{1.731764in}}{\pgfqpoint{6.263295in}{1.723950in}}%
\pgfpathcurveto{\pgfqpoint{6.271108in}{1.716137in}}{\pgfqpoint{6.281707in}{1.711746in}}{\pgfqpoint{6.292758in}{1.711746in}}%
\pgfpathlineto{\pgfqpoint{6.292758in}{1.711746in}}%
\pgfpathclose%
\pgfusepath{stroke,fill}%
\end{pgfscope}%
\begin{pgfscope}%
\pgfpathrectangle{\pgfqpoint{5.292946in}{0.569136in}}{\pgfqpoint{2.177280in}{2.201755in}}%
\pgfusepath{clip}%
\pgfsetbuttcap%
\pgfsetroundjoin%
\definecolor{currentfill}{rgb}{1.000000,0.498039,0.054902}%
\pgfsetfillcolor{currentfill}%
\pgfsetlinewidth{0.481800pt}%
\definecolor{currentstroke}{rgb}{1.000000,1.000000,1.000000}%
\pgfsetstrokecolor{currentstroke}%
\pgfsetdash{}{0pt}%
\pgfpathmoveto{\pgfqpoint{6.177936in}{1.378147in}}%
\pgfpathcurveto{\pgfqpoint{6.188987in}{1.378147in}}{\pgfqpoint{6.199586in}{1.382537in}}{\pgfqpoint{6.207399in}{1.390351in}}%
\pgfpathcurveto{\pgfqpoint{6.215213in}{1.398164in}}{\pgfqpoint{6.219603in}{1.408764in}}{\pgfqpoint{6.219603in}{1.419814in}}%
\pgfpathcurveto{\pgfqpoint{6.219603in}{1.430864in}}{\pgfqpoint{6.215213in}{1.441463in}}{\pgfqpoint{6.207399in}{1.449276in}}%
\pgfpathcurveto{\pgfqpoint{6.199586in}{1.457090in}}{\pgfqpoint{6.188987in}{1.461480in}}{\pgfqpoint{6.177936in}{1.461480in}}%
\pgfpathcurveto{\pgfqpoint{6.166886in}{1.461480in}}{\pgfqpoint{6.156287in}{1.457090in}}{\pgfqpoint{6.148474in}{1.449276in}}%
\pgfpathcurveto{\pgfqpoint{6.140660in}{1.441463in}}{\pgfqpoint{6.136270in}{1.430864in}}{\pgfqpoint{6.136270in}{1.419814in}}%
\pgfpathcurveto{\pgfqpoint{6.136270in}{1.408764in}}{\pgfqpoint{6.140660in}{1.398164in}}{\pgfqpoint{6.148474in}{1.390351in}}%
\pgfpathcurveto{\pgfqpoint{6.156287in}{1.382537in}}{\pgfqpoint{6.166886in}{1.378147in}}{\pgfqpoint{6.177936in}{1.378147in}}%
\pgfpathlineto{\pgfqpoint{6.177936in}{1.378147in}}%
\pgfpathclose%
\pgfusepath{stroke,fill}%
\end{pgfscope}%
\begin{pgfscope}%
\pgfpathrectangle{\pgfqpoint{5.292946in}{0.569136in}}{\pgfqpoint{2.177280in}{2.201755in}}%
\pgfusepath{clip}%
\pgfsetbuttcap%
\pgfsetroundjoin%
\definecolor{currentfill}{rgb}{1.000000,0.498039,0.054902}%
\pgfsetfillcolor{currentfill}%
\pgfsetlinewidth{0.481800pt}%
\definecolor{currentstroke}{rgb}{1.000000,1.000000,1.000000}%
\pgfsetstrokecolor{currentstroke}%
\pgfsetdash{}{0pt}%
\pgfpathmoveto{\pgfqpoint{6.378873in}{1.795146in}}%
\pgfpathcurveto{\pgfqpoint{6.389924in}{1.795146in}}{\pgfqpoint{6.400523in}{1.799536in}}{\pgfqpoint{6.408336in}{1.807350in}}%
\pgfpathcurveto{\pgfqpoint{6.416150in}{1.815164in}}{\pgfqpoint{6.420540in}{1.825763in}}{\pgfqpoint{6.420540in}{1.836813in}}%
\pgfpathcurveto{\pgfqpoint{6.420540in}{1.847863in}}{\pgfqpoint{6.416150in}{1.858462in}}{\pgfqpoint{6.408336in}{1.866276in}}%
\pgfpathcurveto{\pgfqpoint{6.400523in}{1.874089in}}{\pgfqpoint{6.389924in}{1.878479in}}{\pgfqpoint{6.378873in}{1.878479in}}%
\pgfpathcurveto{\pgfqpoint{6.367823in}{1.878479in}}{\pgfqpoint{6.357224in}{1.874089in}}{\pgfqpoint{6.349411in}{1.866276in}}%
\pgfpathcurveto{\pgfqpoint{6.341597in}{1.858462in}}{\pgfqpoint{6.337207in}{1.847863in}}{\pgfqpoint{6.337207in}{1.836813in}}%
\pgfpathcurveto{\pgfqpoint{6.337207in}{1.825763in}}{\pgfqpoint{6.341597in}{1.815164in}}{\pgfqpoint{6.349411in}{1.807350in}}%
\pgfpathcurveto{\pgfqpoint{6.357224in}{1.799536in}}{\pgfqpoint{6.367823in}{1.795146in}}{\pgfqpoint{6.378873in}{1.795146in}}%
\pgfpathlineto{\pgfqpoint{6.378873in}{1.795146in}}%
\pgfpathclose%
\pgfusepath{stroke,fill}%
\end{pgfscope}%
\begin{pgfscope}%
\pgfpathrectangle{\pgfqpoint{5.292946in}{0.569136in}}{\pgfqpoint{2.177280in}{2.201755in}}%
\pgfusepath{clip}%
\pgfsetbuttcap%
\pgfsetroundjoin%
\definecolor{currentfill}{rgb}{1.000000,0.498039,0.054902}%
\pgfsetfillcolor{currentfill}%
\pgfsetlinewidth{0.481800pt}%
\definecolor{currentstroke}{rgb}{1.000000,1.000000,1.000000}%
\pgfsetstrokecolor{currentstroke}%
\pgfsetdash{}{0pt}%
\pgfpathmoveto{\pgfqpoint{6.321463in}{1.378147in}}%
\pgfpathcurveto{\pgfqpoint{6.332513in}{1.378147in}}{\pgfqpoint{6.343112in}{1.382537in}}{\pgfqpoint{6.350926in}{1.390351in}}%
\pgfpathcurveto{\pgfqpoint{6.358739in}{1.398164in}}{\pgfqpoint{6.363130in}{1.408764in}}{\pgfqpoint{6.363130in}{1.419814in}}%
\pgfpathcurveto{\pgfqpoint{6.363130in}{1.430864in}}{\pgfqpoint{6.358739in}{1.441463in}}{\pgfqpoint{6.350926in}{1.449276in}}%
\pgfpathcurveto{\pgfqpoint{6.343112in}{1.457090in}}{\pgfqpoint{6.332513in}{1.461480in}}{\pgfqpoint{6.321463in}{1.461480in}}%
\pgfpathcurveto{\pgfqpoint{6.310413in}{1.461480in}}{\pgfqpoint{6.299814in}{1.457090in}}{\pgfqpoint{6.292000in}{1.449276in}}%
\pgfpathcurveto{\pgfqpoint{6.284186in}{1.441463in}}{\pgfqpoint{6.279796in}{1.430864in}}{\pgfqpoint{6.279796in}{1.419814in}}%
\pgfpathcurveto{\pgfqpoint{6.279796in}{1.408764in}}{\pgfqpoint{6.284186in}{1.398164in}}{\pgfqpoint{6.292000in}{1.390351in}}%
\pgfpathcurveto{\pgfqpoint{6.299814in}{1.382537in}}{\pgfqpoint{6.310413in}{1.378147in}}{\pgfqpoint{6.321463in}{1.378147in}}%
\pgfpathlineto{\pgfqpoint{6.321463in}{1.378147in}}%
\pgfpathclose%
\pgfusepath{stroke,fill}%
\end{pgfscope}%
\begin{pgfscope}%
\pgfpathrectangle{\pgfqpoint{5.292946in}{0.569136in}}{\pgfqpoint{2.177280in}{2.201755in}}%
\pgfusepath{clip}%
\pgfsetbuttcap%
\pgfsetroundjoin%
\definecolor{currentfill}{rgb}{1.000000,0.498039,0.054902}%
\pgfsetfillcolor{currentfill}%
\pgfsetlinewidth{0.481800pt}%
\definecolor{currentstroke}{rgb}{1.000000,1.000000,1.000000}%
\pgfsetstrokecolor{currentstroke}%
\pgfsetdash{}{0pt}%
\pgfpathmoveto{\pgfqpoint{6.522400in}{1.711746in}}%
\pgfpathcurveto{\pgfqpoint{6.533450in}{1.711746in}}{\pgfqpoint{6.544049in}{1.716137in}}{\pgfqpoint{6.551863in}{1.723950in}}%
\pgfpathcurveto{\pgfqpoint{6.559676in}{1.731764in}}{\pgfqpoint{6.564067in}{1.742363in}}{\pgfqpoint{6.564067in}{1.753413in}}%
\pgfpathcurveto{\pgfqpoint{6.564067in}{1.764463in}}{\pgfqpoint{6.559676in}{1.775062in}}{\pgfqpoint{6.551863in}{1.782876in}}%
\pgfpathcurveto{\pgfqpoint{6.544049in}{1.790689in}}{\pgfqpoint{6.533450in}{1.795080in}}{\pgfqpoint{6.522400in}{1.795080in}}%
\pgfpathcurveto{\pgfqpoint{6.511350in}{1.795080in}}{\pgfqpoint{6.500751in}{1.790689in}}{\pgfqpoint{6.492937in}{1.782876in}}%
\pgfpathcurveto{\pgfqpoint{6.485123in}{1.775062in}}{\pgfqpoint{6.480733in}{1.764463in}}{\pgfqpoint{6.480733in}{1.753413in}}%
\pgfpathcurveto{\pgfqpoint{6.480733in}{1.742363in}}{\pgfqpoint{6.485123in}{1.731764in}}{\pgfqpoint{6.492937in}{1.723950in}}%
\pgfpathcurveto{\pgfqpoint{6.500751in}{1.716137in}}{\pgfqpoint{6.511350in}{1.711746in}}{\pgfqpoint{6.522400in}{1.711746in}}%
\pgfpathlineto{\pgfqpoint{6.522400in}{1.711746in}}%
\pgfpathclose%
\pgfusepath{stroke,fill}%
\end{pgfscope}%
\begin{pgfscope}%
\pgfpathrectangle{\pgfqpoint{5.292946in}{0.569136in}}{\pgfqpoint{2.177280in}{2.201755in}}%
\pgfusepath{clip}%
\pgfsetbuttcap%
\pgfsetroundjoin%
\definecolor{currentfill}{rgb}{1.000000,0.498039,0.054902}%
\pgfsetfillcolor{currentfill}%
\pgfsetlinewidth{0.481800pt}%
\definecolor{currentstroke}{rgb}{1.000000,1.000000,1.000000}%
\pgfsetstrokecolor{currentstroke}%
\pgfsetdash{}{0pt}%
\pgfpathmoveto{\pgfqpoint{6.206642in}{1.628346in}}%
\pgfpathcurveto{\pgfqpoint{6.217692in}{1.628346in}}{\pgfqpoint{6.228291in}{1.632737in}}{\pgfqpoint{6.236104in}{1.640550in}}%
\pgfpathcurveto{\pgfqpoint{6.243918in}{1.648364in}}{\pgfqpoint{6.248308in}{1.658963in}}{\pgfqpoint{6.248308in}{1.670013in}}%
\pgfpathcurveto{\pgfqpoint{6.248308in}{1.681063in}}{\pgfqpoint{6.243918in}{1.691662in}}{\pgfqpoint{6.236104in}{1.699476in}}%
\pgfpathcurveto{\pgfqpoint{6.228291in}{1.707290in}}{\pgfqpoint{6.217692in}{1.711680in}}{\pgfqpoint{6.206642in}{1.711680in}}%
\pgfpathcurveto{\pgfqpoint{6.195592in}{1.711680in}}{\pgfqpoint{6.184993in}{1.707290in}}{\pgfqpoint{6.177179in}{1.699476in}}%
\pgfpathcurveto{\pgfqpoint{6.169365in}{1.691662in}}{\pgfqpoint{6.164975in}{1.681063in}}{\pgfqpoint{6.164975in}{1.670013in}}%
\pgfpathcurveto{\pgfqpoint{6.164975in}{1.658963in}}{\pgfqpoint{6.169365in}{1.648364in}}{\pgfqpoint{6.177179in}{1.640550in}}%
\pgfpathcurveto{\pgfqpoint{6.184993in}{1.632737in}}{\pgfqpoint{6.195592in}{1.628346in}}{\pgfqpoint{6.206642in}{1.628346in}}%
\pgfpathlineto{\pgfqpoint{6.206642in}{1.628346in}}%
\pgfpathclose%
\pgfusepath{stroke,fill}%
\end{pgfscope}%
\begin{pgfscope}%
\pgfpathrectangle{\pgfqpoint{5.292946in}{0.569136in}}{\pgfqpoint{2.177280in}{2.201755in}}%
\pgfusepath{clip}%
\pgfsetbuttcap%
\pgfsetroundjoin%
\definecolor{currentfill}{rgb}{1.000000,0.498039,0.054902}%
\pgfsetfillcolor{currentfill}%
\pgfsetlinewidth{0.481800pt}%
\definecolor{currentstroke}{rgb}{1.000000,1.000000,1.000000}%
\pgfsetstrokecolor{currentstroke}%
\pgfsetdash{}{0pt}%
\pgfpathmoveto{\pgfqpoint{6.436284in}{1.711746in}}%
\pgfpathcurveto{\pgfqpoint{6.447334in}{1.711746in}}{\pgfqpoint{6.457933in}{1.716137in}}{\pgfqpoint{6.465747in}{1.723950in}}%
\pgfpathcurveto{\pgfqpoint{6.473560in}{1.731764in}}{\pgfqpoint{6.477951in}{1.742363in}}{\pgfqpoint{6.477951in}{1.753413in}}%
\pgfpathcurveto{\pgfqpoint{6.477951in}{1.764463in}}{\pgfqpoint{6.473560in}{1.775062in}}{\pgfqpoint{6.465747in}{1.782876in}}%
\pgfpathcurveto{\pgfqpoint{6.457933in}{1.790689in}}{\pgfqpoint{6.447334in}{1.795080in}}{\pgfqpoint{6.436284in}{1.795080in}}%
\pgfpathcurveto{\pgfqpoint{6.425234in}{1.795080in}}{\pgfqpoint{6.414635in}{1.790689in}}{\pgfqpoint{6.406821in}{1.782876in}}%
\pgfpathcurveto{\pgfqpoint{6.399008in}{1.775062in}}{\pgfqpoint{6.394617in}{1.764463in}}{\pgfqpoint{6.394617in}{1.753413in}}%
\pgfpathcurveto{\pgfqpoint{6.394617in}{1.742363in}}{\pgfqpoint{6.399008in}{1.731764in}}{\pgfqpoint{6.406821in}{1.723950in}}%
\pgfpathcurveto{\pgfqpoint{6.414635in}{1.716137in}}{\pgfqpoint{6.425234in}{1.711746in}}{\pgfqpoint{6.436284in}{1.711746in}}%
\pgfpathlineto{\pgfqpoint{6.436284in}{1.711746in}}%
\pgfpathclose%
\pgfusepath{stroke,fill}%
\end{pgfscope}%
\begin{pgfscope}%
\pgfpathrectangle{\pgfqpoint{5.292946in}{0.569136in}}{\pgfqpoint{2.177280in}{2.201755in}}%
\pgfusepath{clip}%
\pgfsetbuttcap%
\pgfsetroundjoin%
\definecolor{currentfill}{rgb}{1.000000,0.498039,0.054902}%
\pgfsetfillcolor{currentfill}%
\pgfsetlinewidth{0.481800pt}%
\definecolor{currentstroke}{rgb}{1.000000,1.000000,1.000000}%
\pgfsetstrokecolor{currentstroke}%
\pgfsetdash{}{0pt}%
\pgfpathmoveto{\pgfqpoint{6.464989in}{1.795146in}}%
\pgfpathcurveto{\pgfqpoint{6.476039in}{1.795146in}}{\pgfqpoint{6.486638in}{1.799536in}}{\pgfqpoint{6.494452in}{1.807350in}}%
\pgfpathcurveto{\pgfqpoint{6.502266in}{1.815164in}}{\pgfqpoint{6.506656in}{1.825763in}}{\pgfqpoint{6.506656in}{1.836813in}}%
\pgfpathcurveto{\pgfqpoint{6.506656in}{1.847863in}}{\pgfqpoint{6.502266in}{1.858462in}}{\pgfqpoint{6.494452in}{1.866276in}}%
\pgfpathcurveto{\pgfqpoint{6.486638in}{1.874089in}}{\pgfqpoint{6.476039in}{1.878479in}}{\pgfqpoint{6.464989in}{1.878479in}}%
\pgfpathcurveto{\pgfqpoint{6.453939in}{1.878479in}}{\pgfqpoint{6.443340in}{1.874089in}}{\pgfqpoint{6.435527in}{1.866276in}}%
\pgfpathcurveto{\pgfqpoint{6.427713in}{1.858462in}}{\pgfqpoint{6.423323in}{1.847863in}}{\pgfqpoint{6.423323in}{1.836813in}}%
\pgfpathcurveto{\pgfqpoint{6.423323in}{1.825763in}}{\pgfqpoint{6.427713in}{1.815164in}}{\pgfqpoint{6.435527in}{1.807350in}}%
\pgfpathcurveto{\pgfqpoint{6.443340in}{1.799536in}}{\pgfqpoint{6.453939in}{1.795146in}}{\pgfqpoint{6.464989in}{1.795146in}}%
\pgfpathlineto{\pgfqpoint{6.464989in}{1.795146in}}%
\pgfpathclose%
\pgfusepath{stroke,fill}%
\end{pgfscope}%
\begin{pgfscope}%
\pgfpathrectangle{\pgfqpoint{5.292946in}{0.569136in}}{\pgfqpoint{2.177280in}{2.201755in}}%
\pgfusepath{clip}%
\pgfsetbuttcap%
\pgfsetroundjoin%
\definecolor{currentfill}{rgb}{1.000000,0.498039,0.054902}%
\pgfsetfillcolor{currentfill}%
\pgfsetlinewidth{0.481800pt}%
\definecolor{currentstroke}{rgb}{1.000000,1.000000,1.000000}%
\pgfsetstrokecolor{currentstroke}%
\pgfsetdash{}{0pt}%
\pgfpathmoveto{\pgfqpoint{6.350168in}{1.378147in}}%
\pgfpathcurveto{\pgfqpoint{6.361218in}{1.378147in}}{\pgfqpoint{6.371817in}{1.382537in}}{\pgfqpoint{6.379631in}{1.390351in}}%
\pgfpathcurveto{\pgfqpoint{6.387445in}{1.398164in}}{\pgfqpoint{6.391835in}{1.408764in}}{\pgfqpoint{6.391835in}{1.419814in}}%
\pgfpathcurveto{\pgfqpoint{6.391835in}{1.430864in}}{\pgfqpoint{6.387445in}{1.441463in}}{\pgfqpoint{6.379631in}{1.449276in}}%
\pgfpathcurveto{\pgfqpoint{6.371817in}{1.457090in}}{\pgfqpoint{6.361218in}{1.461480in}}{\pgfqpoint{6.350168in}{1.461480in}}%
\pgfpathcurveto{\pgfqpoint{6.339118in}{1.461480in}}{\pgfqpoint{6.328519in}{1.457090in}}{\pgfqpoint{6.320705in}{1.449276in}}%
\pgfpathcurveto{\pgfqpoint{6.312892in}{1.441463in}}{\pgfqpoint{6.308501in}{1.430864in}}{\pgfqpoint{6.308501in}{1.419814in}}%
\pgfpathcurveto{\pgfqpoint{6.308501in}{1.408764in}}{\pgfqpoint{6.312892in}{1.398164in}}{\pgfqpoint{6.320705in}{1.390351in}}%
\pgfpathcurveto{\pgfqpoint{6.328519in}{1.382537in}}{\pgfqpoint{6.339118in}{1.378147in}}{\pgfqpoint{6.350168in}{1.378147in}}%
\pgfpathlineto{\pgfqpoint{6.350168in}{1.378147in}}%
\pgfpathclose%
\pgfusepath{stroke,fill}%
\end{pgfscope}%
\begin{pgfscope}%
\pgfpathrectangle{\pgfqpoint{5.292946in}{0.569136in}}{\pgfqpoint{2.177280in}{2.201755in}}%
\pgfusepath{clip}%
\pgfsetbuttcap%
\pgfsetroundjoin%
\definecolor{currentfill}{rgb}{1.000000,0.498039,0.054902}%
\pgfsetfillcolor{currentfill}%
\pgfsetlinewidth{0.481800pt}%
\definecolor{currentstroke}{rgb}{1.000000,1.000000,1.000000}%
\pgfsetstrokecolor{currentstroke}%
\pgfsetdash{}{0pt}%
\pgfpathmoveto{\pgfqpoint{6.464989in}{1.795146in}}%
\pgfpathcurveto{\pgfqpoint{6.476039in}{1.795146in}}{\pgfqpoint{6.486638in}{1.799536in}}{\pgfqpoint{6.494452in}{1.807350in}}%
\pgfpathcurveto{\pgfqpoint{6.502266in}{1.815164in}}{\pgfqpoint{6.506656in}{1.825763in}}{\pgfqpoint{6.506656in}{1.836813in}}%
\pgfpathcurveto{\pgfqpoint{6.506656in}{1.847863in}}{\pgfqpoint{6.502266in}{1.858462in}}{\pgfqpoint{6.494452in}{1.866276in}}%
\pgfpathcurveto{\pgfqpoint{6.486638in}{1.874089in}}{\pgfqpoint{6.476039in}{1.878479in}}{\pgfqpoint{6.464989in}{1.878479in}}%
\pgfpathcurveto{\pgfqpoint{6.453939in}{1.878479in}}{\pgfqpoint{6.443340in}{1.874089in}}{\pgfqpoint{6.435527in}{1.866276in}}%
\pgfpathcurveto{\pgfqpoint{6.427713in}{1.858462in}}{\pgfqpoint{6.423323in}{1.847863in}}{\pgfqpoint{6.423323in}{1.836813in}}%
\pgfpathcurveto{\pgfqpoint{6.423323in}{1.825763in}}{\pgfqpoint{6.427713in}{1.815164in}}{\pgfqpoint{6.435527in}{1.807350in}}%
\pgfpathcurveto{\pgfqpoint{6.443340in}{1.799536in}}{\pgfqpoint{6.453939in}{1.795146in}}{\pgfqpoint{6.464989in}{1.795146in}}%
\pgfpathlineto{\pgfqpoint{6.464989in}{1.795146in}}%
\pgfpathclose%
\pgfusepath{stroke,fill}%
\end{pgfscope}%
\begin{pgfscope}%
\pgfpathrectangle{\pgfqpoint{5.292946in}{0.569136in}}{\pgfqpoint{2.177280in}{2.201755in}}%
\pgfusepath{clip}%
\pgfsetbuttcap%
\pgfsetroundjoin%
\definecolor{currentfill}{rgb}{1.000000,0.498039,0.054902}%
\pgfsetfillcolor{currentfill}%
\pgfsetlinewidth{0.481800pt}%
\definecolor{currentstroke}{rgb}{1.000000,1.000000,1.000000}%
\pgfsetstrokecolor{currentstroke}%
\pgfsetdash{}{0pt}%
\pgfpathmoveto{\pgfqpoint{6.292758in}{1.461547in}}%
\pgfpathcurveto{\pgfqpoint{6.303808in}{1.461547in}}{\pgfqpoint{6.314407in}{1.465937in}}{\pgfqpoint{6.322220in}{1.473751in}}%
\pgfpathcurveto{\pgfqpoint{6.330034in}{1.481564in}}{\pgfqpoint{6.334424in}{1.492163in}}{\pgfqpoint{6.334424in}{1.503213in}}%
\pgfpathcurveto{\pgfqpoint{6.334424in}{1.514264in}}{\pgfqpoint{6.330034in}{1.524863in}}{\pgfqpoint{6.322220in}{1.532676in}}%
\pgfpathcurveto{\pgfqpoint{6.314407in}{1.540490in}}{\pgfqpoint{6.303808in}{1.544880in}}{\pgfqpoint{6.292758in}{1.544880in}}%
\pgfpathcurveto{\pgfqpoint{6.281707in}{1.544880in}}{\pgfqpoint{6.271108in}{1.540490in}}{\pgfqpoint{6.263295in}{1.532676in}}%
\pgfpathcurveto{\pgfqpoint{6.255481in}{1.524863in}}{\pgfqpoint{6.251091in}{1.514264in}}{\pgfqpoint{6.251091in}{1.503213in}}%
\pgfpathcurveto{\pgfqpoint{6.251091in}{1.492163in}}{\pgfqpoint{6.255481in}{1.481564in}}{\pgfqpoint{6.263295in}{1.473751in}}%
\pgfpathcurveto{\pgfqpoint{6.271108in}{1.465937in}}{\pgfqpoint{6.281707in}{1.461547in}}{\pgfqpoint{6.292758in}{1.461547in}}%
\pgfpathlineto{\pgfqpoint{6.292758in}{1.461547in}}%
\pgfpathclose%
\pgfusepath{stroke,fill}%
\end{pgfscope}%
\begin{pgfscope}%
\pgfpathrectangle{\pgfqpoint{5.292946in}{0.569136in}}{\pgfqpoint{2.177280in}{2.201755in}}%
\pgfusepath{clip}%
\pgfsetbuttcap%
\pgfsetroundjoin%
\definecolor{currentfill}{rgb}{1.000000,0.498039,0.054902}%
\pgfsetfillcolor{currentfill}%
\pgfsetlinewidth{0.481800pt}%
\definecolor{currentstroke}{rgb}{1.000000,1.000000,1.000000}%
\pgfsetstrokecolor{currentstroke}%
\pgfsetdash{}{0pt}%
\pgfpathmoveto{\pgfqpoint{6.551105in}{2.045346in}}%
\pgfpathcurveto{\pgfqpoint{6.562155in}{2.045346in}}{\pgfqpoint{6.572754in}{2.049736in}}{\pgfqpoint{6.580568in}{2.057549in}}%
\pgfpathcurveto{\pgfqpoint{6.588382in}{2.065363in}}{\pgfqpoint{6.592772in}{2.075962in}}{\pgfqpoint{6.592772in}{2.087012in}}%
\pgfpathcurveto{\pgfqpoint{6.592772in}{2.098062in}}{\pgfqpoint{6.588382in}{2.108661in}}{\pgfqpoint{6.580568in}{2.116475in}}%
\pgfpathcurveto{\pgfqpoint{6.572754in}{2.124289in}}{\pgfqpoint{6.562155in}{2.128679in}}{\pgfqpoint{6.551105in}{2.128679in}}%
\pgfpathcurveto{\pgfqpoint{6.540055in}{2.128679in}}{\pgfqpoint{6.529456in}{2.124289in}}{\pgfqpoint{6.521642in}{2.116475in}}%
\pgfpathcurveto{\pgfqpoint{6.513829in}{2.108661in}}{\pgfqpoint{6.509438in}{2.098062in}}{\pgfqpoint{6.509438in}{2.087012in}}%
\pgfpathcurveto{\pgfqpoint{6.509438in}{2.075962in}}{\pgfqpoint{6.513829in}{2.065363in}}{\pgfqpoint{6.521642in}{2.057549in}}%
\pgfpathcurveto{\pgfqpoint{6.529456in}{2.049736in}}{\pgfqpoint{6.540055in}{2.045346in}}{\pgfqpoint{6.551105in}{2.045346in}}%
\pgfpathlineto{\pgfqpoint{6.551105in}{2.045346in}}%
\pgfpathclose%
\pgfusepath{stroke,fill}%
\end{pgfscope}%
\begin{pgfscope}%
\pgfpathrectangle{\pgfqpoint{5.292946in}{0.569136in}}{\pgfqpoint{2.177280in}{2.201755in}}%
\pgfusepath{clip}%
\pgfsetbuttcap%
\pgfsetroundjoin%
\definecolor{currentfill}{rgb}{1.000000,0.498039,0.054902}%
\pgfsetfillcolor{currentfill}%
\pgfsetlinewidth{0.481800pt}%
\definecolor{currentstroke}{rgb}{1.000000,1.000000,1.000000}%
\pgfsetstrokecolor{currentstroke}%
\pgfsetdash{}{0pt}%
\pgfpathmoveto{\pgfqpoint{6.321463in}{1.628346in}}%
\pgfpathcurveto{\pgfqpoint{6.332513in}{1.628346in}}{\pgfqpoint{6.343112in}{1.632737in}}{\pgfqpoint{6.350926in}{1.640550in}}%
\pgfpathcurveto{\pgfqpoint{6.358739in}{1.648364in}}{\pgfqpoint{6.363130in}{1.658963in}}{\pgfqpoint{6.363130in}{1.670013in}}%
\pgfpathcurveto{\pgfqpoint{6.363130in}{1.681063in}}{\pgfqpoint{6.358739in}{1.691662in}}{\pgfqpoint{6.350926in}{1.699476in}}%
\pgfpathcurveto{\pgfqpoint{6.343112in}{1.707290in}}{\pgfqpoint{6.332513in}{1.711680in}}{\pgfqpoint{6.321463in}{1.711680in}}%
\pgfpathcurveto{\pgfqpoint{6.310413in}{1.711680in}}{\pgfqpoint{6.299814in}{1.707290in}}{\pgfqpoint{6.292000in}{1.699476in}}%
\pgfpathcurveto{\pgfqpoint{6.284186in}{1.691662in}}{\pgfqpoint{6.279796in}{1.681063in}}{\pgfqpoint{6.279796in}{1.670013in}}%
\pgfpathcurveto{\pgfqpoint{6.279796in}{1.658963in}}{\pgfqpoint{6.284186in}{1.648364in}}{\pgfqpoint{6.292000in}{1.640550in}}%
\pgfpathcurveto{\pgfqpoint{6.299814in}{1.632737in}}{\pgfqpoint{6.310413in}{1.628346in}}{\pgfqpoint{6.321463in}{1.628346in}}%
\pgfpathlineto{\pgfqpoint{6.321463in}{1.628346in}}%
\pgfpathclose%
\pgfusepath{stroke,fill}%
\end{pgfscope}%
\begin{pgfscope}%
\pgfpathrectangle{\pgfqpoint{5.292946in}{0.569136in}}{\pgfqpoint{2.177280in}{2.201755in}}%
\pgfusepath{clip}%
\pgfsetbuttcap%
\pgfsetroundjoin%
\definecolor{currentfill}{rgb}{1.000000,0.498039,0.054902}%
\pgfsetfillcolor{currentfill}%
\pgfsetlinewidth{0.481800pt}%
\definecolor{currentstroke}{rgb}{1.000000,1.000000,1.000000}%
\pgfsetstrokecolor{currentstroke}%
\pgfsetdash{}{0pt}%
\pgfpathmoveto{\pgfqpoint{6.579810in}{1.795146in}}%
\pgfpathcurveto{\pgfqpoint{6.590861in}{1.795146in}}{\pgfqpoint{6.601460in}{1.799536in}}{\pgfqpoint{6.609273in}{1.807350in}}%
\pgfpathcurveto{\pgfqpoint{6.617087in}{1.815164in}}{\pgfqpoint{6.621477in}{1.825763in}}{\pgfqpoint{6.621477in}{1.836813in}}%
\pgfpathcurveto{\pgfqpoint{6.621477in}{1.847863in}}{\pgfqpoint{6.617087in}{1.858462in}}{\pgfqpoint{6.609273in}{1.866276in}}%
\pgfpathcurveto{\pgfqpoint{6.601460in}{1.874089in}}{\pgfqpoint{6.590861in}{1.878479in}}{\pgfqpoint{6.579810in}{1.878479in}}%
\pgfpathcurveto{\pgfqpoint{6.568760in}{1.878479in}}{\pgfqpoint{6.558161in}{1.874089in}}{\pgfqpoint{6.550348in}{1.866276in}}%
\pgfpathcurveto{\pgfqpoint{6.542534in}{1.858462in}}{\pgfqpoint{6.538144in}{1.847863in}}{\pgfqpoint{6.538144in}{1.836813in}}%
\pgfpathcurveto{\pgfqpoint{6.538144in}{1.825763in}}{\pgfqpoint{6.542534in}{1.815164in}}{\pgfqpoint{6.550348in}{1.807350in}}%
\pgfpathcurveto{\pgfqpoint{6.558161in}{1.799536in}}{\pgfqpoint{6.568760in}{1.795146in}}{\pgfqpoint{6.579810in}{1.795146in}}%
\pgfpathlineto{\pgfqpoint{6.579810in}{1.795146in}}%
\pgfpathclose%
\pgfusepath{stroke,fill}%
\end{pgfscope}%
\begin{pgfscope}%
\pgfpathrectangle{\pgfqpoint{5.292946in}{0.569136in}}{\pgfqpoint{2.177280in}{2.201755in}}%
\pgfusepath{clip}%
\pgfsetbuttcap%
\pgfsetroundjoin%
\definecolor{currentfill}{rgb}{1.000000,0.498039,0.054902}%
\pgfsetfillcolor{currentfill}%
\pgfsetlinewidth{0.481800pt}%
\definecolor{currentstroke}{rgb}{1.000000,1.000000,1.000000}%
\pgfsetstrokecolor{currentstroke}%
\pgfsetdash{}{0pt}%
\pgfpathmoveto{\pgfqpoint{6.522400in}{1.544947in}}%
\pgfpathcurveto{\pgfqpoint{6.533450in}{1.544947in}}{\pgfqpoint{6.544049in}{1.549337in}}{\pgfqpoint{6.551863in}{1.557151in}}%
\pgfpathcurveto{\pgfqpoint{6.559676in}{1.564964in}}{\pgfqpoint{6.564067in}{1.575563in}}{\pgfqpoint{6.564067in}{1.586613in}}%
\pgfpathcurveto{\pgfqpoint{6.564067in}{1.597663in}}{\pgfqpoint{6.559676in}{1.608262in}}{\pgfqpoint{6.551863in}{1.616076in}}%
\pgfpathcurveto{\pgfqpoint{6.544049in}{1.623890in}}{\pgfqpoint{6.533450in}{1.628280in}}{\pgfqpoint{6.522400in}{1.628280in}}%
\pgfpathcurveto{\pgfqpoint{6.511350in}{1.628280in}}{\pgfqpoint{6.500751in}{1.623890in}}{\pgfqpoint{6.492937in}{1.616076in}}%
\pgfpathcurveto{\pgfqpoint{6.485123in}{1.608262in}}{\pgfqpoint{6.480733in}{1.597663in}}{\pgfqpoint{6.480733in}{1.586613in}}%
\pgfpathcurveto{\pgfqpoint{6.480733in}{1.575563in}}{\pgfqpoint{6.485123in}{1.564964in}}{\pgfqpoint{6.492937in}{1.557151in}}%
\pgfpathcurveto{\pgfqpoint{6.500751in}{1.549337in}}{\pgfqpoint{6.511350in}{1.544947in}}{\pgfqpoint{6.522400in}{1.544947in}}%
\pgfpathlineto{\pgfqpoint{6.522400in}{1.544947in}}%
\pgfpathclose%
\pgfusepath{stroke,fill}%
\end{pgfscope}%
\begin{pgfscope}%
\pgfpathrectangle{\pgfqpoint{5.292946in}{0.569136in}}{\pgfqpoint{2.177280in}{2.201755in}}%
\pgfusepath{clip}%
\pgfsetbuttcap%
\pgfsetroundjoin%
\definecolor{currentfill}{rgb}{1.000000,0.498039,0.054902}%
\pgfsetfillcolor{currentfill}%
\pgfsetlinewidth{0.481800pt}%
\definecolor{currentstroke}{rgb}{1.000000,1.000000,1.000000}%
\pgfsetstrokecolor{currentstroke}%
\pgfsetdash{}{0pt}%
\pgfpathmoveto{\pgfqpoint{6.407579in}{1.628346in}}%
\pgfpathcurveto{\pgfqpoint{6.418629in}{1.628346in}}{\pgfqpoint{6.429228in}{1.632737in}}{\pgfqpoint{6.437041in}{1.640550in}}%
\pgfpathcurveto{\pgfqpoint{6.444855in}{1.648364in}}{\pgfqpoint{6.449245in}{1.658963in}}{\pgfqpoint{6.449245in}{1.670013in}}%
\pgfpathcurveto{\pgfqpoint{6.449245in}{1.681063in}}{\pgfqpoint{6.444855in}{1.691662in}}{\pgfqpoint{6.437041in}{1.699476in}}%
\pgfpathcurveto{\pgfqpoint{6.429228in}{1.707290in}}{\pgfqpoint{6.418629in}{1.711680in}}{\pgfqpoint{6.407579in}{1.711680in}}%
\pgfpathcurveto{\pgfqpoint{6.396529in}{1.711680in}}{\pgfqpoint{6.385930in}{1.707290in}}{\pgfqpoint{6.378116in}{1.699476in}}%
\pgfpathcurveto{\pgfqpoint{6.370302in}{1.691662in}}{\pgfqpoint{6.365912in}{1.681063in}}{\pgfqpoint{6.365912in}{1.670013in}}%
\pgfpathcurveto{\pgfqpoint{6.365912in}{1.658963in}}{\pgfqpoint{6.370302in}{1.648364in}}{\pgfqpoint{6.378116in}{1.640550in}}%
\pgfpathcurveto{\pgfqpoint{6.385930in}{1.632737in}}{\pgfqpoint{6.396529in}{1.628346in}}{\pgfqpoint{6.407579in}{1.628346in}}%
\pgfpathlineto{\pgfqpoint{6.407579in}{1.628346in}}%
\pgfpathclose%
\pgfusepath{stroke,fill}%
\end{pgfscope}%
\begin{pgfscope}%
\pgfpathrectangle{\pgfqpoint{5.292946in}{0.569136in}}{\pgfqpoint{2.177280in}{2.201755in}}%
\pgfusepath{clip}%
\pgfsetbuttcap%
\pgfsetroundjoin%
\definecolor{currentfill}{rgb}{1.000000,0.498039,0.054902}%
\pgfsetfillcolor{currentfill}%
\pgfsetlinewidth{0.481800pt}%
\definecolor{currentstroke}{rgb}{1.000000,1.000000,1.000000}%
\pgfsetstrokecolor{currentstroke}%
\pgfsetdash{}{0pt}%
\pgfpathmoveto{\pgfqpoint{6.436284in}{1.711746in}}%
\pgfpathcurveto{\pgfqpoint{6.447334in}{1.711746in}}{\pgfqpoint{6.457933in}{1.716137in}}{\pgfqpoint{6.465747in}{1.723950in}}%
\pgfpathcurveto{\pgfqpoint{6.473560in}{1.731764in}}{\pgfqpoint{6.477951in}{1.742363in}}{\pgfqpoint{6.477951in}{1.753413in}}%
\pgfpathcurveto{\pgfqpoint{6.477951in}{1.764463in}}{\pgfqpoint{6.473560in}{1.775062in}}{\pgfqpoint{6.465747in}{1.782876in}}%
\pgfpathcurveto{\pgfqpoint{6.457933in}{1.790689in}}{\pgfqpoint{6.447334in}{1.795080in}}{\pgfqpoint{6.436284in}{1.795080in}}%
\pgfpathcurveto{\pgfqpoint{6.425234in}{1.795080in}}{\pgfqpoint{6.414635in}{1.790689in}}{\pgfqpoint{6.406821in}{1.782876in}}%
\pgfpathcurveto{\pgfqpoint{6.399008in}{1.775062in}}{\pgfqpoint{6.394617in}{1.764463in}}{\pgfqpoint{6.394617in}{1.753413in}}%
\pgfpathcurveto{\pgfqpoint{6.394617in}{1.742363in}}{\pgfqpoint{6.399008in}{1.731764in}}{\pgfqpoint{6.406821in}{1.723950in}}%
\pgfpathcurveto{\pgfqpoint{6.414635in}{1.716137in}}{\pgfqpoint{6.425234in}{1.711746in}}{\pgfqpoint{6.436284in}{1.711746in}}%
\pgfpathlineto{\pgfqpoint{6.436284in}{1.711746in}}%
\pgfpathclose%
\pgfusepath{stroke,fill}%
\end{pgfscope}%
\begin{pgfscope}%
\pgfpathrectangle{\pgfqpoint{5.292946in}{0.569136in}}{\pgfqpoint{2.177280in}{2.201755in}}%
\pgfusepath{clip}%
\pgfsetbuttcap%
\pgfsetroundjoin%
\definecolor{currentfill}{rgb}{1.000000,0.498039,0.054902}%
\pgfsetfillcolor{currentfill}%
\pgfsetlinewidth{0.481800pt}%
\definecolor{currentstroke}{rgb}{1.000000,1.000000,1.000000}%
\pgfsetstrokecolor{currentstroke}%
\pgfsetdash{}{0pt}%
\pgfpathmoveto{\pgfqpoint{6.551105in}{1.711746in}}%
\pgfpathcurveto{\pgfqpoint{6.562155in}{1.711746in}}{\pgfqpoint{6.572754in}{1.716137in}}{\pgfqpoint{6.580568in}{1.723950in}}%
\pgfpathcurveto{\pgfqpoint{6.588382in}{1.731764in}}{\pgfqpoint{6.592772in}{1.742363in}}{\pgfqpoint{6.592772in}{1.753413in}}%
\pgfpathcurveto{\pgfqpoint{6.592772in}{1.764463in}}{\pgfqpoint{6.588382in}{1.775062in}}{\pgfqpoint{6.580568in}{1.782876in}}%
\pgfpathcurveto{\pgfqpoint{6.572754in}{1.790689in}}{\pgfqpoint{6.562155in}{1.795080in}}{\pgfqpoint{6.551105in}{1.795080in}}%
\pgfpathcurveto{\pgfqpoint{6.540055in}{1.795080in}}{\pgfqpoint{6.529456in}{1.790689in}}{\pgfqpoint{6.521642in}{1.782876in}}%
\pgfpathcurveto{\pgfqpoint{6.513829in}{1.775062in}}{\pgfqpoint{6.509438in}{1.764463in}}{\pgfqpoint{6.509438in}{1.753413in}}%
\pgfpathcurveto{\pgfqpoint{6.509438in}{1.742363in}}{\pgfqpoint{6.513829in}{1.731764in}}{\pgfqpoint{6.521642in}{1.723950in}}%
\pgfpathcurveto{\pgfqpoint{6.529456in}{1.716137in}}{\pgfqpoint{6.540055in}{1.711746in}}{\pgfqpoint{6.551105in}{1.711746in}}%
\pgfpathlineto{\pgfqpoint{6.551105in}{1.711746in}}%
\pgfpathclose%
\pgfusepath{stroke,fill}%
\end{pgfscope}%
\begin{pgfscope}%
\pgfpathrectangle{\pgfqpoint{5.292946in}{0.569136in}}{\pgfqpoint{2.177280in}{2.201755in}}%
\pgfusepath{clip}%
\pgfsetbuttcap%
\pgfsetroundjoin%
\definecolor{currentfill}{rgb}{1.000000,0.498039,0.054902}%
\pgfsetfillcolor{currentfill}%
\pgfsetlinewidth{0.481800pt}%
\definecolor{currentstroke}{rgb}{1.000000,1.000000,1.000000}%
\pgfsetstrokecolor{currentstroke}%
\pgfsetdash{}{0pt}%
\pgfpathmoveto{\pgfqpoint{6.608516in}{1.961946in}}%
\pgfpathcurveto{\pgfqpoint{6.619566in}{1.961946in}}{\pgfqpoint{6.630165in}{1.966336in}}{\pgfqpoint{6.637978in}{1.974150in}}%
\pgfpathcurveto{\pgfqpoint{6.645792in}{1.981963in}}{\pgfqpoint{6.650182in}{1.992562in}}{\pgfqpoint{6.650182in}{2.003612in}}%
\pgfpathcurveto{\pgfqpoint{6.650182in}{2.014662in}}{\pgfqpoint{6.645792in}{2.025262in}}{\pgfqpoint{6.637978in}{2.033075in}}%
\pgfpathcurveto{\pgfqpoint{6.630165in}{2.040889in}}{\pgfqpoint{6.619566in}{2.045279in}}{\pgfqpoint{6.608516in}{2.045279in}}%
\pgfpathcurveto{\pgfqpoint{6.597466in}{2.045279in}}{\pgfqpoint{6.586867in}{2.040889in}}{\pgfqpoint{6.579053in}{2.033075in}}%
\pgfpathcurveto{\pgfqpoint{6.571239in}{2.025262in}}{\pgfqpoint{6.566849in}{2.014662in}}{\pgfqpoint{6.566849in}{2.003612in}}%
\pgfpathcurveto{\pgfqpoint{6.566849in}{1.992562in}}{\pgfqpoint{6.571239in}{1.981963in}}{\pgfqpoint{6.579053in}{1.974150in}}%
\pgfpathcurveto{\pgfqpoint{6.586867in}{1.966336in}}{\pgfqpoint{6.597466in}{1.961946in}}{\pgfqpoint{6.608516in}{1.961946in}}%
\pgfpathlineto{\pgfqpoint{6.608516in}{1.961946in}}%
\pgfpathclose%
\pgfusepath{stroke,fill}%
\end{pgfscope}%
\begin{pgfscope}%
\pgfpathrectangle{\pgfqpoint{5.292946in}{0.569136in}}{\pgfqpoint{2.177280in}{2.201755in}}%
\pgfusepath{clip}%
\pgfsetbuttcap%
\pgfsetroundjoin%
\definecolor{currentfill}{rgb}{1.000000,0.498039,0.054902}%
\pgfsetfillcolor{currentfill}%
\pgfsetlinewidth{0.481800pt}%
\definecolor{currentstroke}{rgb}{1.000000,1.000000,1.000000}%
\pgfsetstrokecolor{currentstroke}%
\pgfsetdash{}{0pt}%
\pgfpathmoveto{\pgfqpoint{6.464989in}{1.795146in}}%
\pgfpathcurveto{\pgfqpoint{6.476039in}{1.795146in}}{\pgfqpoint{6.486638in}{1.799536in}}{\pgfqpoint{6.494452in}{1.807350in}}%
\pgfpathcurveto{\pgfqpoint{6.502266in}{1.815164in}}{\pgfqpoint{6.506656in}{1.825763in}}{\pgfqpoint{6.506656in}{1.836813in}}%
\pgfpathcurveto{\pgfqpoint{6.506656in}{1.847863in}}{\pgfqpoint{6.502266in}{1.858462in}}{\pgfqpoint{6.494452in}{1.866276in}}%
\pgfpathcurveto{\pgfqpoint{6.486638in}{1.874089in}}{\pgfqpoint{6.476039in}{1.878479in}}{\pgfqpoint{6.464989in}{1.878479in}}%
\pgfpathcurveto{\pgfqpoint{6.453939in}{1.878479in}}{\pgfqpoint{6.443340in}{1.874089in}}{\pgfqpoint{6.435527in}{1.866276in}}%
\pgfpathcurveto{\pgfqpoint{6.427713in}{1.858462in}}{\pgfqpoint{6.423323in}{1.847863in}}{\pgfqpoint{6.423323in}{1.836813in}}%
\pgfpathcurveto{\pgfqpoint{6.423323in}{1.825763in}}{\pgfqpoint{6.427713in}{1.815164in}}{\pgfqpoint{6.435527in}{1.807350in}}%
\pgfpathcurveto{\pgfqpoint{6.443340in}{1.799536in}}{\pgfqpoint{6.453939in}{1.795146in}}{\pgfqpoint{6.464989in}{1.795146in}}%
\pgfpathlineto{\pgfqpoint{6.464989in}{1.795146in}}%
\pgfpathclose%
\pgfusepath{stroke,fill}%
\end{pgfscope}%
\begin{pgfscope}%
\pgfpathrectangle{\pgfqpoint{5.292946in}{0.569136in}}{\pgfqpoint{2.177280in}{2.201755in}}%
\pgfusepath{clip}%
\pgfsetbuttcap%
\pgfsetroundjoin%
\definecolor{currentfill}{rgb}{1.000000,0.498039,0.054902}%
\pgfsetfillcolor{currentfill}%
\pgfsetlinewidth{0.481800pt}%
\definecolor{currentstroke}{rgb}{1.000000,1.000000,1.000000}%
\pgfsetstrokecolor{currentstroke}%
\pgfsetdash{}{0pt}%
\pgfpathmoveto{\pgfqpoint{6.177936in}{1.378147in}}%
\pgfpathcurveto{\pgfqpoint{6.188987in}{1.378147in}}{\pgfqpoint{6.199586in}{1.382537in}}{\pgfqpoint{6.207399in}{1.390351in}}%
\pgfpathcurveto{\pgfqpoint{6.215213in}{1.398164in}}{\pgfqpoint{6.219603in}{1.408764in}}{\pgfqpoint{6.219603in}{1.419814in}}%
\pgfpathcurveto{\pgfqpoint{6.219603in}{1.430864in}}{\pgfqpoint{6.215213in}{1.441463in}}{\pgfqpoint{6.207399in}{1.449276in}}%
\pgfpathcurveto{\pgfqpoint{6.199586in}{1.457090in}}{\pgfqpoint{6.188987in}{1.461480in}}{\pgfqpoint{6.177936in}{1.461480in}}%
\pgfpathcurveto{\pgfqpoint{6.166886in}{1.461480in}}{\pgfqpoint{6.156287in}{1.457090in}}{\pgfqpoint{6.148474in}{1.449276in}}%
\pgfpathcurveto{\pgfqpoint{6.140660in}{1.441463in}}{\pgfqpoint{6.136270in}{1.430864in}}{\pgfqpoint{6.136270in}{1.419814in}}%
\pgfpathcurveto{\pgfqpoint{6.136270in}{1.408764in}}{\pgfqpoint{6.140660in}{1.398164in}}{\pgfqpoint{6.148474in}{1.390351in}}%
\pgfpathcurveto{\pgfqpoint{6.156287in}{1.382537in}}{\pgfqpoint{6.166886in}{1.378147in}}{\pgfqpoint{6.177936in}{1.378147in}}%
\pgfpathlineto{\pgfqpoint{6.177936in}{1.378147in}}%
\pgfpathclose%
\pgfusepath{stroke,fill}%
\end{pgfscope}%
\begin{pgfscope}%
\pgfpathrectangle{\pgfqpoint{5.292946in}{0.569136in}}{\pgfqpoint{2.177280in}{2.201755in}}%
\pgfusepath{clip}%
\pgfsetbuttcap%
\pgfsetroundjoin%
\definecolor{currentfill}{rgb}{1.000000,0.498039,0.054902}%
\pgfsetfillcolor{currentfill}%
\pgfsetlinewidth{0.481800pt}%
\definecolor{currentstroke}{rgb}{1.000000,1.000000,1.000000}%
\pgfsetstrokecolor{currentstroke}%
\pgfsetdash{}{0pt}%
\pgfpathmoveto{\pgfqpoint{6.264052in}{1.461547in}}%
\pgfpathcurveto{\pgfqpoint{6.275102in}{1.461547in}}{\pgfqpoint{6.285701in}{1.465937in}}{\pgfqpoint{6.293515in}{1.473751in}}%
\pgfpathcurveto{\pgfqpoint{6.301329in}{1.481564in}}{\pgfqpoint{6.305719in}{1.492163in}}{\pgfqpoint{6.305719in}{1.503213in}}%
\pgfpathcurveto{\pgfqpoint{6.305719in}{1.514264in}}{\pgfqpoint{6.301329in}{1.524863in}}{\pgfqpoint{6.293515in}{1.532676in}}%
\pgfpathcurveto{\pgfqpoint{6.285701in}{1.540490in}}{\pgfqpoint{6.275102in}{1.544880in}}{\pgfqpoint{6.264052in}{1.544880in}}%
\pgfpathcurveto{\pgfqpoint{6.253002in}{1.544880in}}{\pgfqpoint{6.242403in}{1.540490in}}{\pgfqpoint{6.234590in}{1.532676in}}%
\pgfpathcurveto{\pgfqpoint{6.226776in}{1.524863in}}{\pgfqpoint{6.222386in}{1.514264in}}{\pgfqpoint{6.222386in}{1.503213in}}%
\pgfpathcurveto{\pgfqpoint{6.222386in}{1.492163in}}{\pgfqpoint{6.226776in}{1.481564in}}{\pgfqpoint{6.234590in}{1.473751in}}%
\pgfpathcurveto{\pgfqpoint{6.242403in}{1.465937in}}{\pgfqpoint{6.253002in}{1.461547in}}{\pgfqpoint{6.264052in}{1.461547in}}%
\pgfpathlineto{\pgfqpoint{6.264052in}{1.461547in}}%
\pgfpathclose%
\pgfusepath{stroke,fill}%
\end{pgfscope}%
\begin{pgfscope}%
\pgfpathrectangle{\pgfqpoint{5.292946in}{0.569136in}}{\pgfqpoint{2.177280in}{2.201755in}}%
\pgfusepath{clip}%
\pgfsetbuttcap%
\pgfsetroundjoin%
\definecolor{currentfill}{rgb}{1.000000,0.498039,0.054902}%
\pgfsetfillcolor{currentfill}%
\pgfsetlinewidth{0.481800pt}%
\definecolor{currentstroke}{rgb}{1.000000,1.000000,1.000000}%
\pgfsetstrokecolor{currentstroke}%
\pgfsetdash{}{0pt}%
\pgfpathmoveto{\pgfqpoint{6.235347in}{1.378147in}}%
\pgfpathcurveto{\pgfqpoint{6.246397in}{1.378147in}}{\pgfqpoint{6.256996in}{1.382537in}}{\pgfqpoint{6.264810in}{1.390351in}}%
\pgfpathcurveto{\pgfqpoint{6.272623in}{1.398164in}}{\pgfqpoint{6.277014in}{1.408764in}}{\pgfqpoint{6.277014in}{1.419814in}}%
\pgfpathcurveto{\pgfqpoint{6.277014in}{1.430864in}}{\pgfqpoint{6.272623in}{1.441463in}}{\pgfqpoint{6.264810in}{1.449276in}}%
\pgfpathcurveto{\pgfqpoint{6.256996in}{1.457090in}}{\pgfqpoint{6.246397in}{1.461480in}}{\pgfqpoint{6.235347in}{1.461480in}}%
\pgfpathcurveto{\pgfqpoint{6.224297in}{1.461480in}}{\pgfqpoint{6.213698in}{1.457090in}}{\pgfqpoint{6.205884in}{1.449276in}}%
\pgfpathcurveto{\pgfqpoint{6.198071in}{1.441463in}}{\pgfqpoint{6.193680in}{1.430864in}}{\pgfqpoint{6.193680in}{1.419814in}}%
\pgfpathcurveto{\pgfqpoint{6.193680in}{1.408764in}}{\pgfqpoint{6.198071in}{1.398164in}}{\pgfqpoint{6.205884in}{1.390351in}}%
\pgfpathcurveto{\pgfqpoint{6.213698in}{1.382537in}}{\pgfqpoint{6.224297in}{1.378147in}}{\pgfqpoint{6.235347in}{1.378147in}}%
\pgfpathlineto{\pgfqpoint{6.235347in}{1.378147in}}%
\pgfpathclose%
\pgfusepath{stroke,fill}%
\end{pgfscope}%
\begin{pgfscope}%
\pgfpathrectangle{\pgfqpoint{5.292946in}{0.569136in}}{\pgfqpoint{2.177280in}{2.201755in}}%
\pgfusepath{clip}%
\pgfsetbuttcap%
\pgfsetroundjoin%
\definecolor{currentfill}{rgb}{1.000000,0.498039,0.054902}%
\pgfsetfillcolor{currentfill}%
\pgfsetlinewidth{0.481800pt}%
\definecolor{currentstroke}{rgb}{1.000000,1.000000,1.000000}%
\pgfsetstrokecolor{currentstroke}%
\pgfsetdash{}{0pt}%
\pgfpathmoveto{\pgfqpoint{6.292758in}{1.544947in}}%
\pgfpathcurveto{\pgfqpoint{6.303808in}{1.544947in}}{\pgfqpoint{6.314407in}{1.549337in}}{\pgfqpoint{6.322220in}{1.557151in}}%
\pgfpathcurveto{\pgfqpoint{6.330034in}{1.564964in}}{\pgfqpoint{6.334424in}{1.575563in}}{\pgfqpoint{6.334424in}{1.586613in}}%
\pgfpathcurveto{\pgfqpoint{6.334424in}{1.597663in}}{\pgfqpoint{6.330034in}{1.608262in}}{\pgfqpoint{6.322220in}{1.616076in}}%
\pgfpathcurveto{\pgfqpoint{6.314407in}{1.623890in}}{\pgfqpoint{6.303808in}{1.628280in}}{\pgfqpoint{6.292758in}{1.628280in}}%
\pgfpathcurveto{\pgfqpoint{6.281707in}{1.628280in}}{\pgfqpoint{6.271108in}{1.623890in}}{\pgfqpoint{6.263295in}{1.616076in}}%
\pgfpathcurveto{\pgfqpoint{6.255481in}{1.608262in}}{\pgfqpoint{6.251091in}{1.597663in}}{\pgfqpoint{6.251091in}{1.586613in}}%
\pgfpathcurveto{\pgfqpoint{6.251091in}{1.575563in}}{\pgfqpoint{6.255481in}{1.564964in}}{\pgfqpoint{6.263295in}{1.557151in}}%
\pgfpathcurveto{\pgfqpoint{6.271108in}{1.549337in}}{\pgfqpoint{6.281707in}{1.544947in}}{\pgfqpoint{6.292758in}{1.544947in}}%
\pgfpathlineto{\pgfqpoint{6.292758in}{1.544947in}}%
\pgfpathclose%
\pgfusepath{stroke,fill}%
\end{pgfscope}%
\begin{pgfscope}%
\pgfpathrectangle{\pgfqpoint{5.292946in}{0.569136in}}{\pgfqpoint{2.177280in}{2.201755in}}%
\pgfusepath{clip}%
\pgfsetbuttcap%
\pgfsetroundjoin%
\definecolor{currentfill}{rgb}{1.000000,0.498039,0.054902}%
\pgfsetfillcolor{currentfill}%
\pgfsetlinewidth{0.481800pt}%
\definecolor{currentstroke}{rgb}{1.000000,1.000000,1.000000}%
\pgfsetstrokecolor{currentstroke}%
\pgfsetdash{}{0pt}%
\pgfpathmoveto{\pgfqpoint{6.637221in}{1.878546in}}%
\pgfpathcurveto{\pgfqpoint{6.648271in}{1.878546in}}{\pgfqpoint{6.658870in}{1.882936in}}{\pgfqpoint{6.666684in}{1.890750in}}%
\pgfpathcurveto{\pgfqpoint{6.674497in}{1.898563in}}{\pgfqpoint{6.678888in}{1.909162in}}{\pgfqpoint{6.678888in}{1.920213in}}%
\pgfpathcurveto{\pgfqpoint{6.678888in}{1.931263in}}{\pgfqpoint{6.674497in}{1.941862in}}{\pgfqpoint{6.666684in}{1.949675in}}%
\pgfpathcurveto{\pgfqpoint{6.658870in}{1.957489in}}{\pgfqpoint{6.648271in}{1.961879in}}{\pgfqpoint{6.637221in}{1.961879in}}%
\pgfpathcurveto{\pgfqpoint{6.626171in}{1.961879in}}{\pgfqpoint{6.615572in}{1.957489in}}{\pgfqpoint{6.607758in}{1.949675in}}%
\pgfpathcurveto{\pgfqpoint{6.599945in}{1.941862in}}{\pgfqpoint{6.595554in}{1.931263in}}{\pgfqpoint{6.595554in}{1.920213in}}%
\pgfpathcurveto{\pgfqpoint{6.595554in}{1.909162in}}{\pgfqpoint{6.599945in}{1.898563in}}{\pgfqpoint{6.607758in}{1.890750in}}%
\pgfpathcurveto{\pgfqpoint{6.615572in}{1.882936in}}{\pgfqpoint{6.626171in}{1.878546in}}{\pgfqpoint{6.637221in}{1.878546in}}%
\pgfpathlineto{\pgfqpoint{6.637221in}{1.878546in}}%
\pgfpathclose%
\pgfusepath{stroke,fill}%
\end{pgfscope}%
\begin{pgfscope}%
\pgfpathrectangle{\pgfqpoint{5.292946in}{0.569136in}}{\pgfqpoint{2.177280in}{2.201755in}}%
\pgfusepath{clip}%
\pgfsetbuttcap%
\pgfsetroundjoin%
\definecolor{currentfill}{rgb}{1.000000,0.498039,0.054902}%
\pgfsetfillcolor{currentfill}%
\pgfsetlinewidth{0.481800pt}%
\definecolor{currentstroke}{rgb}{1.000000,1.000000,1.000000}%
\pgfsetstrokecolor{currentstroke}%
\pgfsetdash{}{0pt}%
\pgfpathmoveto{\pgfqpoint{6.464989in}{1.795146in}}%
\pgfpathcurveto{\pgfqpoint{6.476039in}{1.795146in}}{\pgfqpoint{6.486638in}{1.799536in}}{\pgfqpoint{6.494452in}{1.807350in}}%
\pgfpathcurveto{\pgfqpoint{6.502266in}{1.815164in}}{\pgfqpoint{6.506656in}{1.825763in}}{\pgfqpoint{6.506656in}{1.836813in}}%
\pgfpathcurveto{\pgfqpoint{6.506656in}{1.847863in}}{\pgfqpoint{6.502266in}{1.858462in}}{\pgfqpoint{6.494452in}{1.866276in}}%
\pgfpathcurveto{\pgfqpoint{6.486638in}{1.874089in}}{\pgfqpoint{6.476039in}{1.878479in}}{\pgfqpoint{6.464989in}{1.878479in}}%
\pgfpathcurveto{\pgfqpoint{6.453939in}{1.878479in}}{\pgfqpoint{6.443340in}{1.874089in}}{\pgfqpoint{6.435527in}{1.866276in}}%
\pgfpathcurveto{\pgfqpoint{6.427713in}{1.858462in}}{\pgfqpoint{6.423323in}{1.847863in}}{\pgfqpoint{6.423323in}{1.836813in}}%
\pgfpathcurveto{\pgfqpoint{6.423323in}{1.825763in}}{\pgfqpoint{6.427713in}{1.815164in}}{\pgfqpoint{6.435527in}{1.807350in}}%
\pgfpathcurveto{\pgfqpoint{6.443340in}{1.799536in}}{\pgfqpoint{6.453939in}{1.795146in}}{\pgfqpoint{6.464989in}{1.795146in}}%
\pgfpathlineto{\pgfqpoint{6.464989in}{1.795146in}}%
\pgfpathclose%
\pgfusepath{stroke,fill}%
\end{pgfscope}%
\begin{pgfscope}%
\pgfpathrectangle{\pgfqpoint{5.292946in}{0.569136in}}{\pgfqpoint{2.177280in}{2.201755in}}%
\pgfusepath{clip}%
\pgfsetbuttcap%
\pgfsetroundjoin%
\definecolor{currentfill}{rgb}{1.000000,0.498039,0.054902}%
\pgfsetfillcolor{currentfill}%
\pgfsetlinewidth{0.481800pt}%
\definecolor{currentstroke}{rgb}{1.000000,1.000000,1.000000}%
\pgfsetstrokecolor{currentstroke}%
\pgfsetdash{}{0pt}%
\pgfpathmoveto{\pgfqpoint{6.464989in}{1.878546in}}%
\pgfpathcurveto{\pgfqpoint{6.476039in}{1.878546in}}{\pgfqpoint{6.486638in}{1.882936in}}{\pgfqpoint{6.494452in}{1.890750in}}%
\pgfpathcurveto{\pgfqpoint{6.502266in}{1.898563in}}{\pgfqpoint{6.506656in}{1.909162in}}{\pgfqpoint{6.506656in}{1.920213in}}%
\pgfpathcurveto{\pgfqpoint{6.506656in}{1.931263in}}{\pgfqpoint{6.502266in}{1.941862in}}{\pgfqpoint{6.494452in}{1.949675in}}%
\pgfpathcurveto{\pgfqpoint{6.486638in}{1.957489in}}{\pgfqpoint{6.476039in}{1.961879in}}{\pgfqpoint{6.464989in}{1.961879in}}%
\pgfpathcurveto{\pgfqpoint{6.453939in}{1.961879in}}{\pgfqpoint{6.443340in}{1.957489in}}{\pgfqpoint{6.435527in}{1.949675in}}%
\pgfpathcurveto{\pgfqpoint{6.427713in}{1.941862in}}{\pgfqpoint{6.423323in}{1.931263in}}{\pgfqpoint{6.423323in}{1.920213in}}%
\pgfpathcurveto{\pgfqpoint{6.423323in}{1.909162in}}{\pgfqpoint{6.427713in}{1.898563in}}{\pgfqpoint{6.435527in}{1.890750in}}%
\pgfpathcurveto{\pgfqpoint{6.443340in}{1.882936in}}{\pgfqpoint{6.453939in}{1.878546in}}{\pgfqpoint{6.464989in}{1.878546in}}%
\pgfpathlineto{\pgfqpoint{6.464989in}{1.878546in}}%
\pgfpathclose%
\pgfusepath{stroke,fill}%
\end{pgfscope}%
\begin{pgfscope}%
\pgfpathrectangle{\pgfqpoint{5.292946in}{0.569136in}}{\pgfqpoint{2.177280in}{2.201755in}}%
\pgfusepath{clip}%
\pgfsetbuttcap%
\pgfsetroundjoin%
\definecolor{currentfill}{rgb}{1.000000,0.498039,0.054902}%
\pgfsetfillcolor{currentfill}%
\pgfsetlinewidth{0.481800pt}%
\definecolor{currentstroke}{rgb}{1.000000,1.000000,1.000000}%
\pgfsetstrokecolor{currentstroke}%
\pgfsetdash{}{0pt}%
\pgfpathmoveto{\pgfqpoint{6.522400in}{1.795146in}}%
\pgfpathcurveto{\pgfqpoint{6.533450in}{1.795146in}}{\pgfqpoint{6.544049in}{1.799536in}}{\pgfqpoint{6.551863in}{1.807350in}}%
\pgfpathcurveto{\pgfqpoint{6.559676in}{1.815164in}}{\pgfqpoint{6.564067in}{1.825763in}}{\pgfqpoint{6.564067in}{1.836813in}}%
\pgfpathcurveto{\pgfqpoint{6.564067in}{1.847863in}}{\pgfqpoint{6.559676in}{1.858462in}}{\pgfqpoint{6.551863in}{1.866276in}}%
\pgfpathcurveto{\pgfqpoint{6.544049in}{1.874089in}}{\pgfqpoint{6.533450in}{1.878479in}}{\pgfqpoint{6.522400in}{1.878479in}}%
\pgfpathcurveto{\pgfqpoint{6.511350in}{1.878479in}}{\pgfqpoint{6.500751in}{1.874089in}}{\pgfqpoint{6.492937in}{1.866276in}}%
\pgfpathcurveto{\pgfqpoint{6.485123in}{1.858462in}}{\pgfqpoint{6.480733in}{1.847863in}}{\pgfqpoint{6.480733in}{1.836813in}}%
\pgfpathcurveto{\pgfqpoint{6.480733in}{1.825763in}}{\pgfqpoint{6.485123in}{1.815164in}}{\pgfqpoint{6.492937in}{1.807350in}}%
\pgfpathcurveto{\pgfqpoint{6.500751in}{1.799536in}}{\pgfqpoint{6.511350in}{1.795146in}}{\pgfqpoint{6.522400in}{1.795146in}}%
\pgfpathlineto{\pgfqpoint{6.522400in}{1.795146in}}%
\pgfpathclose%
\pgfusepath{stroke,fill}%
\end{pgfscope}%
\begin{pgfscope}%
\pgfpathrectangle{\pgfqpoint{5.292946in}{0.569136in}}{\pgfqpoint{2.177280in}{2.201755in}}%
\pgfusepath{clip}%
\pgfsetbuttcap%
\pgfsetroundjoin%
\definecolor{currentfill}{rgb}{1.000000,0.498039,0.054902}%
\pgfsetfillcolor{currentfill}%
\pgfsetlinewidth{0.481800pt}%
\definecolor{currentstroke}{rgb}{1.000000,1.000000,1.000000}%
\pgfsetstrokecolor{currentstroke}%
\pgfsetdash{}{0pt}%
\pgfpathmoveto{\pgfqpoint{6.436284in}{1.628346in}}%
\pgfpathcurveto{\pgfqpoint{6.447334in}{1.628346in}}{\pgfqpoint{6.457933in}{1.632737in}}{\pgfqpoint{6.465747in}{1.640550in}}%
\pgfpathcurveto{\pgfqpoint{6.473560in}{1.648364in}}{\pgfqpoint{6.477951in}{1.658963in}}{\pgfqpoint{6.477951in}{1.670013in}}%
\pgfpathcurveto{\pgfqpoint{6.477951in}{1.681063in}}{\pgfqpoint{6.473560in}{1.691662in}}{\pgfqpoint{6.465747in}{1.699476in}}%
\pgfpathcurveto{\pgfqpoint{6.457933in}{1.707290in}}{\pgfqpoint{6.447334in}{1.711680in}}{\pgfqpoint{6.436284in}{1.711680in}}%
\pgfpathcurveto{\pgfqpoint{6.425234in}{1.711680in}}{\pgfqpoint{6.414635in}{1.707290in}}{\pgfqpoint{6.406821in}{1.699476in}}%
\pgfpathcurveto{\pgfqpoint{6.399008in}{1.691662in}}{\pgfqpoint{6.394617in}{1.681063in}}{\pgfqpoint{6.394617in}{1.670013in}}%
\pgfpathcurveto{\pgfqpoint{6.394617in}{1.658963in}}{\pgfqpoint{6.399008in}{1.648364in}}{\pgfqpoint{6.406821in}{1.640550in}}%
\pgfpathcurveto{\pgfqpoint{6.414635in}{1.632737in}}{\pgfqpoint{6.425234in}{1.628346in}}{\pgfqpoint{6.436284in}{1.628346in}}%
\pgfpathlineto{\pgfqpoint{6.436284in}{1.628346in}}%
\pgfpathclose%
\pgfusepath{stroke,fill}%
\end{pgfscope}%
\begin{pgfscope}%
\pgfpathrectangle{\pgfqpoint{5.292946in}{0.569136in}}{\pgfqpoint{2.177280in}{2.201755in}}%
\pgfusepath{clip}%
\pgfsetbuttcap%
\pgfsetroundjoin%
\definecolor{currentfill}{rgb}{1.000000,0.498039,0.054902}%
\pgfsetfillcolor{currentfill}%
\pgfsetlinewidth{0.481800pt}%
\definecolor{currentstroke}{rgb}{1.000000,1.000000,1.000000}%
\pgfsetstrokecolor{currentstroke}%
\pgfsetdash{}{0pt}%
\pgfpathmoveto{\pgfqpoint{6.350168in}{1.628346in}}%
\pgfpathcurveto{\pgfqpoint{6.361218in}{1.628346in}}{\pgfqpoint{6.371817in}{1.632737in}}{\pgfqpoint{6.379631in}{1.640550in}}%
\pgfpathcurveto{\pgfqpoint{6.387445in}{1.648364in}}{\pgfqpoint{6.391835in}{1.658963in}}{\pgfqpoint{6.391835in}{1.670013in}}%
\pgfpathcurveto{\pgfqpoint{6.391835in}{1.681063in}}{\pgfqpoint{6.387445in}{1.691662in}}{\pgfqpoint{6.379631in}{1.699476in}}%
\pgfpathcurveto{\pgfqpoint{6.371817in}{1.707290in}}{\pgfqpoint{6.361218in}{1.711680in}}{\pgfqpoint{6.350168in}{1.711680in}}%
\pgfpathcurveto{\pgfqpoint{6.339118in}{1.711680in}}{\pgfqpoint{6.328519in}{1.707290in}}{\pgfqpoint{6.320705in}{1.699476in}}%
\pgfpathcurveto{\pgfqpoint{6.312892in}{1.691662in}}{\pgfqpoint{6.308501in}{1.681063in}}{\pgfqpoint{6.308501in}{1.670013in}}%
\pgfpathcurveto{\pgfqpoint{6.308501in}{1.658963in}}{\pgfqpoint{6.312892in}{1.648364in}}{\pgfqpoint{6.320705in}{1.640550in}}%
\pgfpathcurveto{\pgfqpoint{6.328519in}{1.632737in}}{\pgfqpoint{6.339118in}{1.628346in}}{\pgfqpoint{6.350168in}{1.628346in}}%
\pgfpathlineto{\pgfqpoint{6.350168in}{1.628346in}}%
\pgfpathclose%
\pgfusepath{stroke,fill}%
\end{pgfscope}%
\begin{pgfscope}%
\pgfpathrectangle{\pgfqpoint{5.292946in}{0.569136in}}{\pgfqpoint{2.177280in}{2.201755in}}%
\pgfusepath{clip}%
\pgfsetbuttcap%
\pgfsetroundjoin%
\definecolor{currentfill}{rgb}{1.000000,0.498039,0.054902}%
\pgfsetfillcolor{currentfill}%
\pgfsetlinewidth{0.481800pt}%
\definecolor{currentstroke}{rgb}{1.000000,1.000000,1.000000}%
\pgfsetstrokecolor{currentstroke}%
\pgfsetdash{}{0pt}%
\pgfpathmoveto{\pgfqpoint{6.321463in}{1.628346in}}%
\pgfpathcurveto{\pgfqpoint{6.332513in}{1.628346in}}{\pgfqpoint{6.343112in}{1.632737in}}{\pgfqpoint{6.350926in}{1.640550in}}%
\pgfpathcurveto{\pgfqpoint{6.358739in}{1.648364in}}{\pgfqpoint{6.363130in}{1.658963in}}{\pgfqpoint{6.363130in}{1.670013in}}%
\pgfpathcurveto{\pgfqpoint{6.363130in}{1.681063in}}{\pgfqpoint{6.358739in}{1.691662in}}{\pgfqpoint{6.350926in}{1.699476in}}%
\pgfpathcurveto{\pgfqpoint{6.343112in}{1.707290in}}{\pgfqpoint{6.332513in}{1.711680in}}{\pgfqpoint{6.321463in}{1.711680in}}%
\pgfpathcurveto{\pgfqpoint{6.310413in}{1.711680in}}{\pgfqpoint{6.299814in}{1.707290in}}{\pgfqpoint{6.292000in}{1.699476in}}%
\pgfpathcurveto{\pgfqpoint{6.284186in}{1.691662in}}{\pgfqpoint{6.279796in}{1.681063in}}{\pgfqpoint{6.279796in}{1.670013in}}%
\pgfpathcurveto{\pgfqpoint{6.279796in}{1.658963in}}{\pgfqpoint{6.284186in}{1.648364in}}{\pgfqpoint{6.292000in}{1.640550in}}%
\pgfpathcurveto{\pgfqpoint{6.299814in}{1.632737in}}{\pgfqpoint{6.310413in}{1.628346in}}{\pgfqpoint{6.321463in}{1.628346in}}%
\pgfpathlineto{\pgfqpoint{6.321463in}{1.628346in}}%
\pgfpathclose%
\pgfusepath{stroke,fill}%
\end{pgfscope}%
\begin{pgfscope}%
\pgfpathrectangle{\pgfqpoint{5.292946in}{0.569136in}}{\pgfqpoint{2.177280in}{2.201755in}}%
\pgfusepath{clip}%
\pgfsetbuttcap%
\pgfsetroundjoin%
\definecolor{currentfill}{rgb}{1.000000,0.498039,0.054902}%
\pgfsetfillcolor{currentfill}%
\pgfsetlinewidth{0.481800pt}%
\definecolor{currentstroke}{rgb}{1.000000,1.000000,1.000000}%
\pgfsetstrokecolor{currentstroke}%
\pgfsetdash{}{0pt}%
\pgfpathmoveto{\pgfqpoint{6.436284in}{1.544947in}}%
\pgfpathcurveto{\pgfqpoint{6.447334in}{1.544947in}}{\pgfqpoint{6.457933in}{1.549337in}}{\pgfqpoint{6.465747in}{1.557151in}}%
\pgfpathcurveto{\pgfqpoint{6.473560in}{1.564964in}}{\pgfqpoint{6.477951in}{1.575563in}}{\pgfqpoint{6.477951in}{1.586613in}}%
\pgfpathcurveto{\pgfqpoint{6.477951in}{1.597663in}}{\pgfqpoint{6.473560in}{1.608262in}}{\pgfqpoint{6.465747in}{1.616076in}}%
\pgfpathcurveto{\pgfqpoint{6.457933in}{1.623890in}}{\pgfqpoint{6.447334in}{1.628280in}}{\pgfqpoint{6.436284in}{1.628280in}}%
\pgfpathcurveto{\pgfqpoint{6.425234in}{1.628280in}}{\pgfqpoint{6.414635in}{1.623890in}}{\pgfqpoint{6.406821in}{1.616076in}}%
\pgfpathcurveto{\pgfqpoint{6.399008in}{1.608262in}}{\pgfqpoint{6.394617in}{1.597663in}}{\pgfqpoint{6.394617in}{1.586613in}}%
\pgfpathcurveto{\pgfqpoint{6.394617in}{1.575563in}}{\pgfqpoint{6.399008in}{1.564964in}}{\pgfqpoint{6.406821in}{1.557151in}}%
\pgfpathcurveto{\pgfqpoint{6.414635in}{1.549337in}}{\pgfqpoint{6.425234in}{1.544947in}}{\pgfqpoint{6.436284in}{1.544947in}}%
\pgfpathlineto{\pgfqpoint{6.436284in}{1.544947in}}%
\pgfpathclose%
\pgfusepath{stroke,fill}%
\end{pgfscope}%
\begin{pgfscope}%
\pgfpathrectangle{\pgfqpoint{5.292946in}{0.569136in}}{\pgfqpoint{2.177280in}{2.201755in}}%
\pgfusepath{clip}%
\pgfsetbuttcap%
\pgfsetroundjoin%
\definecolor{currentfill}{rgb}{1.000000,0.498039,0.054902}%
\pgfsetfillcolor{currentfill}%
\pgfsetlinewidth{0.481800pt}%
\definecolor{currentstroke}{rgb}{1.000000,1.000000,1.000000}%
\pgfsetstrokecolor{currentstroke}%
\pgfsetdash{}{0pt}%
\pgfpathmoveto{\pgfqpoint{6.493695in}{1.711746in}}%
\pgfpathcurveto{\pgfqpoint{6.504745in}{1.711746in}}{\pgfqpoint{6.515344in}{1.716137in}}{\pgfqpoint{6.523157in}{1.723950in}}%
\pgfpathcurveto{\pgfqpoint{6.530971in}{1.731764in}}{\pgfqpoint{6.535361in}{1.742363in}}{\pgfqpoint{6.535361in}{1.753413in}}%
\pgfpathcurveto{\pgfqpoint{6.535361in}{1.764463in}}{\pgfqpoint{6.530971in}{1.775062in}}{\pgfqpoint{6.523157in}{1.782876in}}%
\pgfpathcurveto{\pgfqpoint{6.515344in}{1.790689in}}{\pgfqpoint{6.504745in}{1.795080in}}{\pgfqpoint{6.493695in}{1.795080in}}%
\pgfpathcurveto{\pgfqpoint{6.482644in}{1.795080in}}{\pgfqpoint{6.472045in}{1.790689in}}{\pgfqpoint{6.464232in}{1.782876in}}%
\pgfpathcurveto{\pgfqpoint{6.456418in}{1.775062in}}{\pgfqpoint{6.452028in}{1.764463in}}{\pgfqpoint{6.452028in}{1.753413in}}%
\pgfpathcurveto{\pgfqpoint{6.452028in}{1.742363in}}{\pgfqpoint{6.456418in}{1.731764in}}{\pgfqpoint{6.464232in}{1.723950in}}%
\pgfpathcurveto{\pgfqpoint{6.472045in}{1.716137in}}{\pgfqpoint{6.482644in}{1.711746in}}{\pgfqpoint{6.493695in}{1.711746in}}%
\pgfpathlineto{\pgfqpoint{6.493695in}{1.711746in}}%
\pgfpathclose%
\pgfusepath{stroke,fill}%
\end{pgfscope}%
\begin{pgfscope}%
\pgfpathrectangle{\pgfqpoint{5.292946in}{0.569136in}}{\pgfqpoint{2.177280in}{2.201755in}}%
\pgfusepath{clip}%
\pgfsetbuttcap%
\pgfsetroundjoin%
\definecolor{currentfill}{rgb}{1.000000,0.498039,0.054902}%
\pgfsetfillcolor{currentfill}%
\pgfsetlinewidth{0.481800pt}%
\definecolor{currentstroke}{rgb}{1.000000,1.000000,1.000000}%
\pgfsetstrokecolor{currentstroke}%
\pgfsetdash{}{0pt}%
\pgfpathmoveto{\pgfqpoint{6.321463in}{1.544947in}}%
\pgfpathcurveto{\pgfqpoint{6.332513in}{1.544947in}}{\pgfqpoint{6.343112in}{1.549337in}}{\pgfqpoint{6.350926in}{1.557151in}}%
\pgfpathcurveto{\pgfqpoint{6.358739in}{1.564964in}}{\pgfqpoint{6.363130in}{1.575563in}}{\pgfqpoint{6.363130in}{1.586613in}}%
\pgfpathcurveto{\pgfqpoint{6.363130in}{1.597663in}}{\pgfqpoint{6.358739in}{1.608262in}}{\pgfqpoint{6.350926in}{1.616076in}}%
\pgfpathcurveto{\pgfqpoint{6.343112in}{1.623890in}}{\pgfqpoint{6.332513in}{1.628280in}}{\pgfqpoint{6.321463in}{1.628280in}}%
\pgfpathcurveto{\pgfqpoint{6.310413in}{1.628280in}}{\pgfqpoint{6.299814in}{1.623890in}}{\pgfqpoint{6.292000in}{1.616076in}}%
\pgfpathcurveto{\pgfqpoint{6.284186in}{1.608262in}}{\pgfqpoint{6.279796in}{1.597663in}}{\pgfqpoint{6.279796in}{1.586613in}}%
\pgfpathcurveto{\pgfqpoint{6.279796in}{1.575563in}}{\pgfqpoint{6.284186in}{1.564964in}}{\pgfqpoint{6.292000in}{1.557151in}}%
\pgfpathcurveto{\pgfqpoint{6.299814in}{1.549337in}}{\pgfqpoint{6.310413in}{1.544947in}}{\pgfqpoint{6.321463in}{1.544947in}}%
\pgfpathlineto{\pgfqpoint{6.321463in}{1.544947in}}%
\pgfpathclose%
\pgfusepath{stroke,fill}%
\end{pgfscope}%
\begin{pgfscope}%
\pgfpathrectangle{\pgfqpoint{5.292946in}{0.569136in}}{\pgfqpoint{2.177280in}{2.201755in}}%
\pgfusepath{clip}%
\pgfsetbuttcap%
\pgfsetroundjoin%
\definecolor{currentfill}{rgb}{1.000000,0.498039,0.054902}%
\pgfsetfillcolor{currentfill}%
\pgfsetlinewidth{0.481800pt}%
\definecolor{currentstroke}{rgb}{1.000000,1.000000,1.000000}%
\pgfsetstrokecolor{currentstroke}%
\pgfsetdash{}{0pt}%
\pgfpathmoveto{\pgfqpoint{6.120526in}{1.378147in}}%
\pgfpathcurveto{\pgfqpoint{6.131576in}{1.378147in}}{\pgfqpoint{6.142175in}{1.382537in}}{\pgfqpoint{6.149989in}{1.390351in}}%
\pgfpathcurveto{\pgfqpoint{6.157802in}{1.398164in}}{\pgfqpoint{6.162193in}{1.408764in}}{\pgfqpoint{6.162193in}{1.419814in}}%
\pgfpathcurveto{\pgfqpoint{6.162193in}{1.430864in}}{\pgfqpoint{6.157802in}{1.441463in}}{\pgfqpoint{6.149989in}{1.449276in}}%
\pgfpathcurveto{\pgfqpoint{6.142175in}{1.457090in}}{\pgfqpoint{6.131576in}{1.461480in}}{\pgfqpoint{6.120526in}{1.461480in}}%
\pgfpathcurveto{\pgfqpoint{6.109476in}{1.461480in}}{\pgfqpoint{6.098877in}{1.457090in}}{\pgfqpoint{6.091063in}{1.449276in}}%
\pgfpathcurveto{\pgfqpoint{6.083249in}{1.441463in}}{\pgfqpoint{6.078859in}{1.430864in}}{\pgfqpoint{6.078859in}{1.419814in}}%
\pgfpathcurveto{\pgfqpoint{6.078859in}{1.408764in}}{\pgfqpoint{6.083249in}{1.398164in}}{\pgfqpoint{6.091063in}{1.390351in}}%
\pgfpathcurveto{\pgfqpoint{6.098877in}{1.382537in}}{\pgfqpoint{6.109476in}{1.378147in}}{\pgfqpoint{6.120526in}{1.378147in}}%
\pgfpathlineto{\pgfqpoint{6.120526in}{1.378147in}}%
\pgfpathclose%
\pgfusepath{stroke,fill}%
\end{pgfscope}%
\begin{pgfscope}%
\pgfpathrectangle{\pgfqpoint{5.292946in}{0.569136in}}{\pgfqpoint{2.177280in}{2.201755in}}%
\pgfusepath{clip}%
\pgfsetbuttcap%
\pgfsetroundjoin%
\definecolor{currentfill}{rgb}{1.000000,0.498039,0.054902}%
\pgfsetfillcolor{currentfill}%
\pgfsetlinewidth{0.481800pt}%
\definecolor{currentstroke}{rgb}{1.000000,1.000000,1.000000}%
\pgfsetstrokecolor{currentstroke}%
\pgfsetdash{}{0pt}%
\pgfpathmoveto{\pgfqpoint{6.378873in}{1.628346in}}%
\pgfpathcurveto{\pgfqpoint{6.389924in}{1.628346in}}{\pgfqpoint{6.400523in}{1.632737in}}{\pgfqpoint{6.408336in}{1.640550in}}%
\pgfpathcurveto{\pgfqpoint{6.416150in}{1.648364in}}{\pgfqpoint{6.420540in}{1.658963in}}{\pgfqpoint{6.420540in}{1.670013in}}%
\pgfpathcurveto{\pgfqpoint{6.420540in}{1.681063in}}{\pgfqpoint{6.416150in}{1.691662in}}{\pgfqpoint{6.408336in}{1.699476in}}%
\pgfpathcurveto{\pgfqpoint{6.400523in}{1.707290in}}{\pgfqpoint{6.389924in}{1.711680in}}{\pgfqpoint{6.378873in}{1.711680in}}%
\pgfpathcurveto{\pgfqpoint{6.367823in}{1.711680in}}{\pgfqpoint{6.357224in}{1.707290in}}{\pgfqpoint{6.349411in}{1.699476in}}%
\pgfpathcurveto{\pgfqpoint{6.341597in}{1.691662in}}{\pgfqpoint{6.337207in}{1.681063in}}{\pgfqpoint{6.337207in}{1.670013in}}%
\pgfpathcurveto{\pgfqpoint{6.337207in}{1.658963in}}{\pgfqpoint{6.341597in}{1.648364in}}{\pgfqpoint{6.349411in}{1.640550in}}%
\pgfpathcurveto{\pgfqpoint{6.357224in}{1.632737in}}{\pgfqpoint{6.367823in}{1.628346in}}{\pgfqpoint{6.378873in}{1.628346in}}%
\pgfpathlineto{\pgfqpoint{6.378873in}{1.628346in}}%
\pgfpathclose%
\pgfusepath{stroke,fill}%
\end{pgfscope}%
\begin{pgfscope}%
\pgfpathrectangle{\pgfqpoint{5.292946in}{0.569136in}}{\pgfqpoint{2.177280in}{2.201755in}}%
\pgfusepath{clip}%
\pgfsetbuttcap%
\pgfsetroundjoin%
\definecolor{currentfill}{rgb}{1.000000,0.498039,0.054902}%
\pgfsetfillcolor{currentfill}%
\pgfsetlinewidth{0.481800pt}%
\definecolor{currentstroke}{rgb}{1.000000,1.000000,1.000000}%
\pgfsetstrokecolor{currentstroke}%
\pgfsetdash{}{0pt}%
\pgfpathmoveto{\pgfqpoint{6.378873in}{1.544947in}}%
\pgfpathcurveto{\pgfqpoint{6.389924in}{1.544947in}}{\pgfqpoint{6.400523in}{1.549337in}}{\pgfqpoint{6.408336in}{1.557151in}}%
\pgfpathcurveto{\pgfqpoint{6.416150in}{1.564964in}}{\pgfqpoint{6.420540in}{1.575563in}}{\pgfqpoint{6.420540in}{1.586613in}}%
\pgfpathcurveto{\pgfqpoint{6.420540in}{1.597663in}}{\pgfqpoint{6.416150in}{1.608262in}}{\pgfqpoint{6.408336in}{1.616076in}}%
\pgfpathcurveto{\pgfqpoint{6.400523in}{1.623890in}}{\pgfqpoint{6.389924in}{1.628280in}}{\pgfqpoint{6.378873in}{1.628280in}}%
\pgfpathcurveto{\pgfqpoint{6.367823in}{1.628280in}}{\pgfqpoint{6.357224in}{1.623890in}}{\pgfqpoint{6.349411in}{1.616076in}}%
\pgfpathcurveto{\pgfqpoint{6.341597in}{1.608262in}}{\pgfqpoint{6.337207in}{1.597663in}}{\pgfqpoint{6.337207in}{1.586613in}}%
\pgfpathcurveto{\pgfqpoint{6.337207in}{1.575563in}}{\pgfqpoint{6.341597in}{1.564964in}}{\pgfqpoint{6.349411in}{1.557151in}}%
\pgfpathcurveto{\pgfqpoint{6.357224in}{1.549337in}}{\pgfqpoint{6.367823in}{1.544947in}}{\pgfqpoint{6.378873in}{1.544947in}}%
\pgfpathlineto{\pgfqpoint{6.378873in}{1.544947in}}%
\pgfpathclose%
\pgfusepath{stroke,fill}%
\end{pgfscope}%
\begin{pgfscope}%
\pgfpathrectangle{\pgfqpoint{5.292946in}{0.569136in}}{\pgfqpoint{2.177280in}{2.201755in}}%
\pgfusepath{clip}%
\pgfsetbuttcap%
\pgfsetroundjoin%
\definecolor{currentfill}{rgb}{1.000000,0.498039,0.054902}%
\pgfsetfillcolor{currentfill}%
\pgfsetlinewidth{0.481800pt}%
\definecolor{currentstroke}{rgb}{1.000000,1.000000,1.000000}%
\pgfsetstrokecolor{currentstroke}%
\pgfsetdash{}{0pt}%
\pgfpathmoveto{\pgfqpoint{6.378873in}{1.628346in}}%
\pgfpathcurveto{\pgfqpoint{6.389924in}{1.628346in}}{\pgfqpoint{6.400523in}{1.632737in}}{\pgfqpoint{6.408336in}{1.640550in}}%
\pgfpathcurveto{\pgfqpoint{6.416150in}{1.648364in}}{\pgfqpoint{6.420540in}{1.658963in}}{\pgfqpoint{6.420540in}{1.670013in}}%
\pgfpathcurveto{\pgfqpoint{6.420540in}{1.681063in}}{\pgfqpoint{6.416150in}{1.691662in}}{\pgfqpoint{6.408336in}{1.699476in}}%
\pgfpathcurveto{\pgfqpoint{6.400523in}{1.707290in}}{\pgfqpoint{6.389924in}{1.711680in}}{\pgfqpoint{6.378873in}{1.711680in}}%
\pgfpathcurveto{\pgfqpoint{6.367823in}{1.711680in}}{\pgfqpoint{6.357224in}{1.707290in}}{\pgfqpoint{6.349411in}{1.699476in}}%
\pgfpathcurveto{\pgfqpoint{6.341597in}{1.691662in}}{\pgfqpoint{6.337207in}{1.681063in}}{\pgfqpoint{6.337207in}{1.670013in}}%
\pgfpathcurveto{\pgfqpoint{6.337207in}{1.658963in}}{\pgfqpoint{6.341597in}{1.648364in}}{\pgfqpoint{6.349411in}{1.640550in}}%
\pgfpathcurveto{\pgfqpoint{6.357224in}{1.632737in}}{\pgfqpoint{6.367823in}{1.628346in}}{\pgfqpoint{6.378873in}{1.628346in}}%
\pgfpathlineto{\pgfqpoint{6.378873in}{1.628346in}}%
\pgfpathclose%
\pgfusepath{stroke,fill}%
\end{pgfscope}%
\begin{pgfscope}%
\pgfpathrectangle{\pgfqpoint{5.292946in}{0.569136in}}{\pgfqpoint{2.177280in}{2.201755in}}%
\pgfusepath{clip}%
\pgfsetbuttcap%
\pgfsetroundjoin%
\definecolor{currentfill}{rgb}{1.000000,0.498039,0.054902}%
\pgfsetfillcolor{currentfill}%
\pgfsetlinewidth{0.481800pt}%
\definecolor{currentstroke}{rgb}{1.000000,1.000000,1.000000}%
\pgfsetstrokecolor{currentstroke}%
\pgfsetdash{}{0pt}%
\pgfpathmoveto{\pgfqpoint{6.407579in}{1.628346in}}%
\pgfpathcurveto{\pgfqpoint{6.418629in}{1.628346in}}{\pgfqpoint{6.429228in}{1.632737in}}{\pgfqpoint{6.437041in}{1.640550in}}%
\pgfpathcurveto{\pgfqpoint{6.444855in}{1.648364in}}{\pgfqpoint{6.449245in}{1.658963in}}{\pgfqpoint{6.449245in}{1.670013in}}%
\pgfpathcurveto{\pgfqpoint{6.449245in}{1.681063in}}{\pgfqpoint{6.444855in}{1.691662in}}{\pgfqpoint{6.437041in}{1.699476in}}%
\pgfpathcurveto{\pgfqpoint{6.429228in}{1.707290in}}{\pgfqpoint{6.418629in}{1.711680in}}{\pgfqpoint{6.407579in}{1.711680in}}%
\pgfpathcurveto{\pgfqpoint{6.396529in}{1.711680in}}{\pgfqpoint{6.385930in}{1.707290in}}{\pgfqpoint{6.378116in}{1.699476in}}%
\pgfpathcurveto{\pgfqpoint{6.370302in}{1.691662in}}{\pgfqpoint{6.365912in}{1.681063in}}{\pgfqpoint{6.365912in}{1.670013in}}%
\pgfpathcurveto{\pgfqpoint{6.365912in}{1.658963in}}{\pgfqpoint{6.370302in}{1.648364in}}{\pgfqpoint{6.378116in}{1.640550in}}%
\pgfpathcurveto{\pgfqpoint{6.385930in}{1.632737in}}{\pgfqpoint{6.396529in}{1.628346in}}{\pgfqpoint{6.407579in}{1.628346in}}%
\pgfpathlineto{\pgfqpoint{6.407579in}{1.628346in}}%
\pgfpathclose%
\pgfusepath{stroke,fill}%
\end{pgfscope}%
\begin{pgfscope}%
\pgfpathrectangle{\pgfqpoint{5.292946in}{0.569136in}}{\pgfqpoint{2.177280in}{2.201755in}}%
\pgfusepath{clip}%
\pgfsetbuttcap%
\pgfsetroundjoin%
\definecolor{currentfill}{rgb}{1.000000,0.498039,0.054902}%
\pgfsetfillcolor{currentfill}%
\pgfsetlinewidth{0.481800pt}%
\definecolor{currentstroke}{rgb}{1.000000,1.000000,1.000000}%
\pgfsetstrokecolor{currentstroke}%
\pgfsetdash{}{0pt}%
\pgfpathmoveto{\pgfqpoint{6.034410in}{1.461547in}}%
\pgfpathcurveto{\pgfqpoint{6.045460in}{1.461547in}}{\pgfqpoint{6.056059in}{1.465937in}}{\pgfqpoint{6.063873in}{1.473751in}}%
\pgfpathcurveto{\pgfqpoint{6.071686in}{1.481564in}}{\pgfqpoint{6.076077in}{1.492163in}}{\pgfqpoint{6.076077in}{1.503213in}}%
\pgfpathcurveto{\pgfqpoint{6.076077in}{1.514264in}}{\pgfqpoint{6.071686in}{1.524863in}}{\pgfqpoint{6.063873in}{1.532676in}}%
\pgfpathcurveto{\pgfqpoint{6.056059in}{1.540490in}}{\pgfqpoint{6.045460in}{1.544880in}}{\pgfqpoint{6.034410in}{1.544880in}}%
\pgfpathcurveto{\pgfqpoint{6.023360in}{1.544880in}}{\pgfqpoint{6.012761in}{1.540490in}}{\pgfqpoint{6.004947in}{1.532676in}}%
\pgfpathcurveto{\pgfqpoint{5.997134in}{1.524863in}}{\pgfqpoint{5.992743in}{1.514264in}}{\pgfqpoint{5.992743in}{1.503213in}}%
\pgfpathcurveto{\pgfqpoint{5.992743in}{1.492163in}}{\pgfqpoint{5.997134in}{1.481564in}}{\pgfqpoint{6.004947in}{1.473751in}}%
\pgfpathcurveto{\pgfqpoint{6.012761in}{1.465937in}}{\pgfqpoint{6.023360in}{1.461547in}}{\pgfqpoint{6.034410in}{1.461547in}}%
\pgfpathlineto{\pgfqpoint{6.034410in}{1.461547in}}%
\pgfpathclose%
\pgfusepath{stroke,fill}%
\end{pgfscope}%
\begin{pgfscope}%
\pgfpathrectangle{\pgfqpoint{5.292946in}{0.569136in}}{\pgfqpoint{2.177280in}{2.201755in}}%
\pgfusepath{clip}%
\pgfsetbuttcap%
\pgfsetroundjoin%
\definecolor{currentfill}{rgb}{1.000000,0.498039,0.054902}%
\pgfsetfillcolor{currentfill}%
\pgfsetlinewidth{0.481800pt}%
\definecolor{currentstroke}{rgb}{1.000000,1.000000,1.000000}%
\pgfsetstrokecolor{currentstroke}%
\pgfsetdash{}{0pt}%
\pgfpathmoveto{\pgfqpoint{6.350168in}{1.628346in}}%
\pgfpathcurveto{\pgfqpoint{6.361218in}{1.628346in}}{\pgfqpoint{6.371817in}{1.632737in}}{\pgfqpoint{6.379631in}{1.640550in}}%
\pgfpathcurveto{\pgfqpoint{6.387445in}{1.648364in}}{\pgfqpoint{6.391835in}{1.658963in}}{\pgfqpoint{6.391835in}{1.670013in}}%
\pgfpathcurveto{\pgfqpoint{6.391835in}{1.681063in}}{\pgfqpoint{6.387445in}{1.691662in}}{\pgfqpoint{6.379631in}{1.699476in}}%
\pgfpathcurveto{\pgfqpoint{6.371817in}{1.707290in}}{\pgfqpoint{6.361218in}{1.711680in}}{\pgfqpoint{6.350168in}{1.711680in}}%
\pgfpathcurveto{\pgfqpoint{6.339118in}{1.711680in}}{\pgfqpoint{6.328519in}{1.707290in}}{\pgfqpoint{6.320705in}{1.699476in}}%
\pgfpathcurveto{\pgfqpoint{6.312892in}{1.691662in}}{\pgfqpoint{6.308501in}{1.681063in}}{\pgfqpoint{6.308501in}{1.670013in}}%
\pgfpathcurveto{\pgfqpoint{6.308501in}{1.658963in}}{\pgfqpoint{6.312892in}{1.648364in}}{\pgfqpoint{6.320705in}{1.640550in}}%
\pgfpathcurveto{\pgfqpoint{6.328519in}{1.632737in}}{\pgfqpoint{6.339118in}{1.628346in}}{\pgfqpoint{6.350168in}{1.628346in}}%
\pgfpathlineto{\pgfqpoint{6.350168in}{1.628346in}}%
\pgfpathclose%
\pgfusepath{stroke,fill}%
\end{pgfscope}%
\begin{pgfscope}%
\pgfpathrectangle{\pgfqpoint{5.292946in}{0.569136in}}{\pgfqpoint{2.177280in}{2.201755in}}%
\pgfusepath{clip}%
\pgfsetbuttcap%
\pgfsetroundjoin%
\definecolor{currentfill}{rgb}{0.172549,0.627451,0.172549}%
\pgfsetfillcolor{currentfill}%
\pgfsetlinewidth{0.481800pt}%
\definecolor{currentstroke}{rgb}{1.000000,1.000000,1.000000}%
\pgfsetstrokecolor{currentstroke}%
\pgfsetdash{}{0pt}%
\pgfpathmoveto{\pgfqpoint{6.895569in}{2.629144in}}%
\pgfpathcurveto{\pgfqpoint{6.906619in}{2.629144in}}{\pgfqpoint{6.917218in}{2.633534in}}{\pgfqpoint{6.925031in}{2.641348in}}%
\pgfpathcurveto{\pgfqpoint{6.932845in}{2.649162in}}{\pgfqpoint{6.937235in}{2.659761in}}{\pgfqpoint{6.937235in}{2.670811in}}%
\pgfpathcurveto{\pgfqpoint{6.937235in}{2.681861in}}{\pgfqpoint{6.932845in}{2.692460in}}{\pgfqpoint{6.925031in}{2.700274in}}%
\pgfpathcurveto{\pgfqpoint{6.917218in}{2.708087in}}{\pgfqpoint{6.906619in}{2.712478in}}{\pgfqpoint{6.895569in}{2.712478in}}%
\pgfpathcurveto{\pgfqpoint{6.884518in}{2.712478in}}{\pgfqpoint{6.873919in}{2.708087in}}{\pgfqpoint{6.866106in}{2.700274in}}%
\pgfpathcurveto{\pgfqpoint{6.858292in}{2.692460in}}{\pgfqpoint{6.853902in}{2.681861in}}{\pgfqpoint{6.853902in}{2.670811in}}%
\pgfpathcurveto{\pgfqpoint{6.853902in}{2.659761in}}{\pgfqpoint{6.858292in}{2.649162in}}{\pgfqpoint{6.866106in}{2.641348in}}%
\pgfpathcurveto{\pgfqpoint{6.873919in}{2.633534in}}{\pgfqpoint{6.884518in}{2.629144in}}{\pgfqpoint{6.895569in}{2.629144in}}%
\pgfpathlineto{\pgfqpoint{6.895569in}{2.629144in}}%
\pgfpathclose%
\pgfusepath{stroke,fill}%
\end{pgfscope}%
\begin{pgfscope}%
\pgfpathrectangle{\pgfqpoint{5.292946in}{0.569136in}}{\pgfqpoint{2.177280in}{2.201755in}}%
\pgfusepath{clip}%
\pgfsetbuttcap%
\pgfsetroundjoin%
\definecolor{currentfill}{rgb}{0.172549,0.627451,0.172549}%
\pgfsetfillcolor{currentfill}%
\pgfsetlinewidth{0.481800pt}%
\definecolor{currentstroke}{rgb}{1.000000,1.000000,1.000000}%
\pgfsetstrokecolor{currentstroke}%
\pgfsetdash{}{0pt}%
\pgfpathmoveto{\pgfqpoint{6.637221in}{2.128745in}}%
\pgfpathcurveto{\pgfqpoint{6.648271in}{2.128745in}}{\pgfqpoint{6.658870in}{2.133136in}}{\pgfqpoint{6.666684in}{2.140949in}}%
\pgfpathcurveto{\pgfqpoint{6.674497in}{2.148763in}}{\pgfqpoint{6.678888in}{2.159362in}}{\pgfqpoint{6.678888in}{2.170412in}}%
\pgfpathcurveto{\pgfqpoint{6.678888in}{2.181462in}}{\pgfqpoint{6.674497in}{2.192061in}}{\pgfqpoint{6.666684in}{2.199875in}}%
\pgfpathcurveto{\pgfqpoint{6.658870in}{2.207688in}}{\pgfqpoint{6.648271in}{2.212079in}}{\pgfqpoint{6.637221in}{2.212079in}}%
\pgfpathcurveto{\pgfqpoint{6.626171in}{2.212079in}}{\pgfqpoint{6.615572in}{2.207688in}}{\pgfqpoint{6.607758in}{2.199875in}}%
\pgfpathcurveto{\pgfqpoint{6.599945in}{2.192061in}}{\pgfqpoint{6.595554in}{2.181462in}}{\pgfqpoint{6.595554in}{2.170412in}}%
\pgfpathcurveto{\pgfqpoint{6.595554in}{2.159362in}}{\pgfqpoint{6.599945in}{2.148763in}}{\pgfqpoint{6.607758in}{2.140949in}}%
\pgfpathcurveto{\pgfqpoint{6.615572in}{2.133136in}}{\pgfqpoint{6.626171in}{2.128745in}}{\pgfqpoint{6.637221in}{2.128745in}}%
\pgfpathlineto{\pgfqpoint{6.637221in}{2.128745in}}%
\pgfpathclose%
\pgfusepath{stroke,fill}%
\end{pgfscope}%
\begin{pgfscope}%
\pgfpathrectangle{\pgfqpoint{5.292946in}{0.569136in}}{\pgfqpoint{2.177280in}{2.201755in}}%
\pgfusepath{clip}%
\pgfsetbuttcap%
\pgfsetroundjoin%
\definecolor{currentfill}{rgb}{0.172549,0.627451,0.172549}%
\pgfsetfillcolor{currentfill}%
\pgfsetlinewidth{0.481800pt}%
\definecolor{currentstroke}{rgb}{1.000000,1.000000,1.000000}%
\pgfsetstrokecolor{currentstroke}%
\pgfsetdash{}{0pt}%
\pgfpathmoveto{\pgfqpoint{6.866863in}{2.295545in}}%
\pgfpathcurveto{\pgfqpoint{6.877913in}{2.295545in}}{\pgfqpoint{6.888512in}{2.299935in}}{\pgfqpoint{6.896326in}{2.307749in}}%
\pgfpathcurveto{\pgfqpoint{6.904140in}{2.315562in}}{\pgfqpoint{6.908530in}{2.326161in}}{\pgfqpoint{6.908530in}{2.337212in}}%
\pgfpathcurveto{\pgfqpoint{6.908530in}{2.348262in}}{\pgfqpoint{6.904140in}{2.358861in}}{\pgfqpoint{6.896326in}{2.366674in}}%
\pgfpathcurveto{\pgfqpoint{6.888512in}{2.374488in}}{\pgfqpoint{6.877913in}{2.378878in}}{\pgfqpoint{6.866863in}{2.378878in}}%
\pgfpathcurveto{\pgfqpoint{6.855813in}{2.378878in}}{\pgfqpoint{6.845214in}{2.374488in}}{\pgfqpoint{6.837401in}{2.366674in}}%
\pgfpathcurveto{\pgfqpoint{6.829587in}{2.358861in}}{\pgfqpoint{6.825197in}{2.348262in}}{\pgfqpoint{6.825197in}{2.337212in}}%
\pgfpathcurveto{\pgfqpoint{6.825197in}{2.326161in}}{\pgfqpoint{6.829587in}{2.315562in}}{\pgfqpoint{6.837401in}{2.307749in}}%
\pgfpathcurveto{\pgfqpoint{6.845214in}{2.299935in}}{\pgfqpoint{6.855813in}{2.295545in}}{\pgfqpoint{6.866863in}{2.295545in}}%
\pgfpathlineto{\pgfqpoint{6.866863in}{2.295545in}}%
\pgfpathclose%
\pgfusepath{stroke,fill}%
\end{pgfscope}%
\begin{pgfscope}%
\pgfpathrectangle{\pgfqpoint{5.292946in}{0.569136in}}{\pgfqpoint{2.177280in}{2.201755in}}%
\pgfusepath{clip}%
\pgfsetbuttcap%
\pgfsetroundjoin%
\definecolor{currentfill}{rgb}{0.172549,0.627451,0.172549}%
\pgfsetfillcolor{currentfill}%
\pgfsetlinewidth{0.481800pt}%
\definecolor{currentstroke}{rgb}{1.000000,1.000000,1.000000}%
\pgfsetstrokecolor{currentstroke}%
\pgfsetdash{}{0pt}%
\pgfpathmoveto{\pgfqpoint{6.780747in}{2.045346in}}%
\pgfpathcurveto{\pgfqpoint{6.791798in}{2.045346in}}{\pgfqpoint{6.802397in}{2.049736in}}{\pgfqpoint{6.810210in}{2.057549in}}%
\pgfpathcurveto{\pgfqpoint{6.818024in}{2.065363in}}{\pgfqpoint{6.822414in}{2.075962in}}{\pgfqpoint{6.822414in}{2.087012in}}%
\pgfpathcurveto{\pgfqpoint{6.822414in}{2.098062in}}{\pgfqpoint{6.818024in}{2.108661in}}{\pgfqpoint{6.810210in}{2.116475in}}%
\pgfpathcurveto{\pgfqpoint{6.802397in}{2.124289in}}{\pgfqpoint{6.791798in}{2.128679in}}{\pgfqpoint{6.780747in}{2.128679in}}%
\pgfpathcurveto{\pgfqpoint{6.769697in}{2.128679in}}{\pgfqpoint{6.759098in}{2.124289in}}{\pgfqpoint{6.751285in}{2.116475in}}%
\pgfpathcurveto{\pgfqpoint{6.743471in}{2.108661in}}{\pgfqpoint{6.739081in}{2.098062in}}{\pgfqpoint{6.739081in}{2.087012in}}%
\pgfpathcurveto{\pgfqpoint{6.739081in}{2.075962in}}{\pgfqpoint{6.743471in}{2.065363in}}{\pgfqpoint{6.751285in}{2.057549in}}%
\pgfpathcurveto{\pgfqpoint{6.759098in}{2.049736in}}{\pgfqpoint{6.769697in}{2.045346in}}{\pgfqpoint{6.780747in}{2.045346in}}%
\pgfpathlineto{\pgfqpoint{6.780747in}{2.045346in}}%
\pgfpathclose%
\pgfusepath{stroke,fill}%
\end{pgfscope}%
\begin{pgfscope}%
\pgfpathrectangle{\pgfqpoint{5.292946in}{0.569136in}}{\pgfqpoint{2.177280in}{2.201755in}}%
\pgfusepath{clip}%
\pgfsetbuttcap%
\pgfsetroundjoin%
\definecolor{currentfill}{rgb}{0.172549,0.627451,0.172549}%
\pgfsetfillcolor{currentfill}%
\pgfsetlinewidth{0.481800pt}%
\definecolor{currentstroke}{rgb}{1.000000,1.000000,1.000000}%
\pgfsetstrokecolor{currentstroke}%
\pgfsetdash{}{0pt}%
\pgfpathmoveto{\pgfqpoint{6.838158in}{2.378945in}}%
\pgfpathcurveto{\pgfqpoint{6.849208in}{2.378945in}}{\pgfqpoint{6.859807in}{2.383335in}}{\pgfqpoint{6.867621in}{2.391149in}}%
\pgfpathcurveto{\pgfqpoint{6.875434in}{2.398962in}}{\pgfqpoint{6.879825in}{2.409561in}}{\pgfqpoint{6.879825in}{2.420611in}}%
\pgfpathcurveto{\pgfqpoint{6.879825in}{2.431662in}}{\pgfqpoint{6.875434in}{2.442261in}}{\pgfqpoint{6.867621in}{2.450074in}}%
\pgfpathcurveto{\pgfqpoint{6.859807in}{2.457888in}}{\pgfqpoint{6.849208in}{2.462278in}}{\pgfqpoint{6.838158in}{2.462278in}}%
\pgfpathcurveto{\pgfqpoint{6.827108in}{2.462278in}}{\pgfqpoint{6.816509in}{2.457888in}}{\pgfqpoint{6.808695in}{2.450074in}}%
\pgfpathcurveto{\pgfqpoint{6.800882in}{2.442261in}}{\pgfqpoint{6.796491in}{2.431662in}}{\pgfqpoint{6.796491in}{2.420611in}}%
\pgfpathcurveto{\pgfqpoint{6.796491in}{2.409561in}}{\pgfqpoint{6.800882in}{2.398962in}}{\pgfqpoint{6.808695in}{2.391149in}}%
\pgfpathcurveto{\pgfqpoint{6.816509in}{2.383335in}}{\pgfqpoint{6.827108in}{2.378945in}}{\pgfqpoint{6.838158in}{2.378945in}}%
\pgfpathlineto{\pgfqpoint{6.838158in}{2.378945in}}%
\pgfpathclose%
\pgfusepath{stroke,fill}%
\end{pgfscope}%
\begin{pgfscope}%
\pgfpathrectangle{\pgfqpoint{5.292946in}{0.569136in}}{\pgfqpoint{2.177280in}{2.201755in}}%
\pgfusepath{clip}%
\pgfsetbuttcap%
\pgfsetroundjoin%
\definecolor{currentfill}{rgb}{0.172549,0.627451,0.172549}%
\pgfsetfillcolor{currentfill}%
\pgfsetlinewidth{0.481800pt}%
\definecolor{currentstroke}{rgb}{1.000000,1.000000,1.000000}%
\pgfsetstrokecolor{currentstroke}%
\pgfsetdash{}{0pt}%
\pgfpathmoveto{\pgfqpoint{7.067800in}{2.295545in}}%
\pgfpathcurveto{\pgfqpoint{7.078850in}{2.295545in}}{\pgfqpoint{7.089449in}{2.299935in}}{\pgfqpoint{7.097263in}{2.307749in}}%
\pgfpathcurveto{\pgfqpoint{7.105077in}{2.315562in}}{\pgfqpoint{7.109467in}{2.326161in}}{\pgfqpoint{7.109467in}{2.337212in}}%
\pgfpathcurveto{\pgfqpoint{7.109467in}{2.348262in}}{\pgfqpoint{7.105077in}{2.358861in}}{\pgfqpoint{7.097263in}{2.366674in}}%
\pgfpathcurveto{\pgfqpoint{7.089449in}{2.374488in}}{\pgfqpoint{7.078850in}{2.378878in}}{\pgfqpoint{7.067800in}{2.378878in}}%
\pgfpathcurveto{\pgfqpoint{7.056750in}{2.378878in}}{\pgfqpoint{7.046151in}{2.374488in}}{\pgfqpoint{7.038338in}{2.366674in}}%
\pgfpathcurveto{\pgfqpoint{7.030524in}{2.358861in}}{\pgfqpoint{7.026134in}{2.348262in}}{\pgfqpoint{7.026134in}{2.337212in}}%
\pgfpathcurveto{\pgfqpoint{7.026134in}{2.326161in}}{\pgfqpoint{7.030524in}{2.315562in}}{\pgfqpoint{7.038338in}{2.307749in}}%
\pgfpathcurveto{\pgfqpoint{7.046151in}{2.299935in}}{\pgfqpoint{7.056750in}{2.295545in}}{\pgfqpoint{7.067800in}{2.295545in}}%
\pgfpathlineto{\pgfqpoint{7.067800in}{2.295545in}}%
\pgfpathclose%
\pgfusepath{stroke,fill}%
\end{pgfscope}%
\begin{pgfscope}%
\pgfpathrectangle{\pgfqpoint{5.292946in}{0.569136in}}{\pgfqpoint{2.177280in}{2.201755in}}%
\pgfusepath{clip}%
\pgfsetbuttcap%
\pgfsetroundjoin%
\definecolor{currentfill}{rgb}{0.172549,0.627451,0.172549}%
\pgfsetfillcolor{currentfill}%
\pgfsetlinewidth{0.481800pt}%
\definecolor{currentstroke}{rgb}{1.000000,1.000000,1.000000}%
\pgfsetstrokecolor{currentstroke}%
\pgfsetdash{}{0pt}%
\pgfpathmoveto{\pgfqpoint{6.464989in}{1.961946in}}%
\pgfpathcurveto{\pgfqpoint{6.476039in}{1.961946in}}{\pgfqpoint{6.486638in}{1.966336in}}{\pgfqpoint{6.494452in}{1.974150in}}%
\pgfpathcurveto{\pgfqpoint{6.502266in}{1.981963in}}{\pgfqpoint{6.506656in}{1.992562in}}{\pgfqpoint{6.506656in}{2.003612in}}%
\pgfpathcurveto{\pgfqpoint{6.506656in}{2.014662in}}{\pgfqpoint{6.502266in}{2.025262in}}{\pgfqpoint{6.494452in}{2.033075in}}%
\pgfpathcurveto{\pgfqpoint{6.486638in}{2.040889in}}{\pgfqpoint{6.476039in}{2.045279in}}{\pgfqpoint{6.464989in}{2.045279in}}%
\pgfpathcurveto{\pgfqpoint{6.453939in}{2.045279in}}{\pgfqpoint{6.443340in}{2.040889in}}{\pgfqpoint{6.435527in}{2.033075in}}%
\pgfpathcurveto{\pgfqpoint{6.427713in}{2.025262in}}{\pgfqpoint{6.423323in}{2.014662in}}{\pgfqpoint{6.423323in}{2.003612in}}%
\pgfpathcurveto{\pgfqpoint{6.423323in}{1.992562in}}{\pgfqpoint{6.427713in}{1.981963in}}{\pgfqpoint{6.435527in}{1.974150in}}%
\pgfpathcurveto{\pgfqpoint{6.443340in}{1.966336in}}{\pgfqpoint{6.453939in}{1.961946in}}{\pgfqpoint{6.464989in}{1.961946in}}%
\pgfpathlineto{\pgfqpoint{6.464989in}{1.961946in}}%
\pgfpathclose%
\pgfusepath{stroke,fill}%
\end{pgfscope}%
\begin{pgfscope}%
\pgfpathrectangle{\pgfqpoint{5.292946in}{0.569136in}}{\pgfqpoint{2.177280in}{2.201755in}}%
\pgfusepath{clip}%
\pgfsetbuttcap%
\pgfsetroundjoin%
\definecolor{currentfill}{rgb}{0.172549,0.627451,0.172549}%
\pgfsetfillcolor{currentfill}%
\pgfsetlinewidth{0.481800pt}%
\definecolor{currentstroke}{rgb}{1.000000,1.000000,1.000000}%
\pgfsetstrokecolor{currentstroke}%
\pgfsetdash{}{0pt}%
\pgfpathmoveto{\pgfqpoint{6.981684in}{2.045346in}}%
\pgfpathcurveto{\pgfqpoint{6.992735in}{2.045346in}}{\pgfqpoint{7.003334in}{2.049736in}}{\pgfqpoint{7.011147in}{2.057549in}}%
\pgfpathcurveto{\pgfqpoint{7.018961in}{2.065363in}}{\pgfqpoint{7.023351in}{2.075962in}}{\pgfqpoint{7.023351in}{2.087012in}}%
\pgfpathcurveto{\pgfqpoint{7.023351in}{2.098062in}}{\pgfqpoint{7.018961in}{2.108661in}}{\pgfqpoint{7.011147in}{2.116475in}}%
\pgfpathcurveto{\pgfqpoint{7.003334in}{2.124289in}}{\pgfqpoint{6.992735in}{2.128679in}}{\pgfqpoint{6.981684in}{2.128679in}}%
\pgfpathcurveto{\pgfqpoint{6.970634in}{2.128679in}}{\pgfqpoint{6.960035in}{2.124289in}}{\pgfqpoint{6.952222in}{2.116475in}}%
\pgfpathcurveto{\pgfqpoint{6.944408in}{2.108661in}}{\pgfqpoint{6.940018in}{2.098062in}}{\pgfqpoint{6.940018in}{2.087012in}}%
\pgfpathcurveto{\pgfqpoint{6.940018in}{2.075962in}}{\pgfqpoint{6.944408in}{2.065363in}}{\pgfqpoint{6.952222in}{2.057549in}}%
\pgfpathcurveto{\pgfqpoint{6.960035in}{2.049736in}}{\pgfqpoint{6.970634in}{2.045346in}}{\pgfqpoint{6.981684in}{2.045346in}}%
\pgfpathlineto{\pgfqpoint{6.981684in}{2.045346in}}%
\pgfpathclose%
\pgfusepath{stroke,fill}%
\end{pgfscope}%
\begin{pgfscope}%
\pgfpathrectangle{\pgfqpoint{5.292946in}{0.569136in}}{\pgfqpoint{2.177280in}{2.201755in}}%
\pgfusepath{clip}%
\pgfsetbuttcap%
\pgfsetroundjoin%
\definecolor{currentfill}{rgb}{0.172549,0.627451,0.172549}%
\pgfsetfillcolor{currentfill}%
\pgfsetlinewidth{0.481800pt}%
\definecolor{currentstroke}{rgb}{1.000000,1.000000,1.000000}%
\pgfsetstrokecolor{currentstroke}%
\pgfsetdash{}{0pt}%
\pgfpathmoveto{\pgfqpoint{6.838158in}{2.045346in}}%
\pgfpathcurveto{\pgfqpoint{6.849208in}{2.045346in}}{\pgfqpoint{6.859807in}{2.049736in}}{\pgfqpoint{6.867621in}{2.057549in}}%
\pgfpathcurveto{\pgfqpoint{6.875434in}{2.065363in}}{\pgfqpoint{6.879825in}{2.075962in}}{\pgfqpoint{6.879825in}{2.087012in}}%
\pgfpathcurveto{\pgfqpoint{6.879825in}{2.098062in}}{\pgfqpoint{6.875434in}{2.108661in}}{\pgfqpoint{6.867621in}{2.116475in}}%
\pgfpathcurveto{\pgfqpoint{6.859807in}{2.124289in}}{\pgfqpoint{6.849208in}{2.128679in}}{\pgfqpoint{6.838158in}{2.128679in}}%
\pgfpathcurveto{\pgfqpoint{6.827108in}{2.128679in}}{\pgfqpoint{6.816509in}{2.124289in}}{\pgfqpoint{6.808695in}{2.116475in}}%
\pgfpathcurveto{\pgfqpoint{6.800882in}{2.108661in}}{\pgfqpoint{6.796491in}{2.098062in}}{\pgfqpoint{6.796491in}{2.087012in}}%
\pgfpathcurveto{\pgfqpoint{6.796491in}{2.075962in}}{\pgfqpoint{6.800882in}{2.065363in}}{\pgfqpoint{6.808695in}{2.057549in}}%
\pgfpathcurveto{\pgfqpoint{6.816509in}{2.049736in}}{\pgfqpoint{6.827108in}{2.045346in}}{\pgfqpoint{6.838158in}{2.045346in}}%
\pgfpathlineto{\pgfqpoint{6.838158in}{2.045346in}}%
\pgfpathclose%
\pgfusepath{stroke,fill}%
\end{pgfscope}%
\begin{pgfscope}%
\pgfpathrectangle{\pgfqpoint{5.292946in}{0.569136in}}{\pgfqpoint{2.177280in}{2.201755in}}%
\pgfusepath{clip}%
\pgfsetbuttcap%
\pgfsetroundjoin%
\definecolor{currentfill}{rgb}{0.172549,0.627451,0.172549}%
\pgfsetfillcolor{currentfill}%
\pgfsetlinewidth{0.481800pt}%
\definecolor{currentstroke}{rgb}{1.000000,1.000000,1.000000}%
\pgfsetstrokecolor{currentstroke}%
\pgfsetdash{}{0pt}%
\pgfpathmoveto{\pgfqpoint{6.924274in}{2.629144in}}%
\pgfpathcurveto{\pgfqpoint{6.935324in}{2.629144in}}{\pgfqpoint{6.945923in}{2.633534in}}{\pgfqpoint{6.953737in}{2.641348in}}%
\pgfpathcurveto{\pgfqpoint{6.961550in}{2.649162in}}{\pgfqpoint{6.965941in}{2.659761in}}{\pgfqpoint{6.965941in}{2.670811in}}%
\pgfpathcurveto{\pgfqpoint{6.965941in}{2.681861in}}{\pgfqpoint{6.961550in}{2.692460in}}{\pgfqpoint{6.953737in}{2.700274in}}%
\pgfpathcurveto{\pgfqpoint{6.945923in}{2.708087in}}{\pgfqpoint{6.935324in}{2.712478in}}{\pgfqpoint{6.924274in}{2.712478in}}%
\pgfpathcurveto{\pgfqpoint{6.913224in}{2.712478in}}{\pgfqpoint{6.902625in}{2.708087in}}{\pgfqpoint{6.894811in}{2.700274in}}%
\pgfpathcurveto{\pgfqpoint{6.886997in}{2.692460in}}{\pgfqpoint{6.882607in}{2.681861in}}{\pgfqpoint{6.882607in}{2.670811in}}%
\pgfpathcurveto{\pgfqpoint{6.882607in}{2.659761in}}{\pgfqpoint{6.886997in}{2.649162in}}{\pgfqpoint{6.894811in}{2.641348in}}%
\pgfpathcurveto{\pgfqpoint{6.902625in}{2.633534in}}{\pgfqpoint{6.913224in}{2.629144in}}{\pgfqpoint{6.924274in}{2.629144in}}%
\pgfpathlineto{\pgfqpoint{6.924274in}{2.629144in}}%
\pgfpathclose%
\pgfusepath{stroke,fill}%
\end{pgfscope}%
\begin{pgfscope}%
\pgfpathrectangle{\pgfqpoint{5.292946in}{0.569136in}}{\pgfqpoint{2.177280in}{2.201755in}}%
\pgfusepath{clip}%
\pgfsetbuttcap%
\pgfsetroundjoin%
\definecolor{currentfill}{rgb}{0.172549,0.627451,0.172549}%
\pgfsetfillcolor{currentfill}%
\pgfsetlinewidth{0.481800pt}%
\definecolor{currentstroke}{rgb}{1.000000,1.000000,1.000000}%
\pgfsetstrokecolor{currentstroke}%
\pgfsetdash{}{0pt}%
\pgfpathmoveto{\pgfqpoint{6.637221in}{2.212145in}}%
\pgfpathcurveto{\pgfqpoint{6.648271in}{2.212145in}}{\pgfqpoint{6.658870in}{2.216535in}}{\pgfqpoint{6.666684in}{2.224349in}}%
\pgfpathcurveto{\pgfqpoint{6.674497in}{2.232163in}}{\pgfqpoint{6.678888in}{2.242762in}}{\pgfqpoint{6.678888in}{2.253812in}}%
\pgfpathcurveto{\pgfqpoint{6.678888in}{2.264862in}}{\pgfqpoint{6.674497in}{2.275461in}}{\pgfqpoint{6.666684in}{2.283275in}}%
\pgfpathcurveto{\pgfqpoint{6.658870in}{2.291088in}}{\pgfqpoint{6.648271in}{2.295478in}}{\pgfqpoint{6.637221in}{2.295478in}}%
\pgfpathcurveto{\pgfqpoint{6.626171in}{2.295478in}}{\pgfqpoint{6.615572in}{2.291088in}}{\pgfqpoint{6.607758in}{2.283275in}}%
\pgfpathcurveto{\pgfqpoint{6.599945in}{2.275461in}}{\pgfqpoint{6.595554in}{2.264862in}}{\pgfqpoint{6.595554in}{2.253812in}}%
\pgfpathcurveto{\pgfqpoint{6.595554in}{2.242762in}}{\pgfqpoint{6.599945in}{2.232163in}}{\pgfqpoint{6.607758in}{2.224349in}}%
\pgfpathcurveto{\pgfqpoint{6.615572in}{2.216535in}}{\pgfqpoint{6.626171in}{2.212145in}}{\pgfqpoint{6.637221in}{2.212145in}}%
\pgfpathlineto{\pgfqpoint{6.637221in}{2.212145in}}%
\pgfpathclose%
\pgfusepath{stroke,fill}%
\end{pgfscope}%
\begin{pgfscope}%
\pgfpathrectangle{\pgfqpoint{5.292946in}{0.569136in}}{\pgfqpoint{2.177280in}{2.201755in}}%
\pgfusepath{clip}%
\pgfsetbuttcap%
\pgfsetroundjoin%
\definecolor{currentfill}{rgb}{0.172549,0.627451,0.172549}%
\pgfsetfillcolor{currentfill}%
\pgfsetlinewidth{0.481800pt}%
\definecolor{currentstroke}{rgb}{1.000000,1.000000,1.000000}%
\pgfsetstrokecolor{currentstroke}%
\pgfsetdash{}{0pt}%
\pgfpathmoveto{\pgfqpoint{6.694632in}{2.128745in}}%
\pgfpathcurveto{\pgfqpoint{6.705682in}{2.128745in}}{\pgfqpoint{6.716281in}{2.133136in}}{\pgfqpoint{6.724094in}{2.140949in}}%
\pgfpathcurveto{\pgfqpoint{6.731908in}{2.148763in}}{\pgfqpoint{6.736298in}{2.159362in}}{\pgfqpoint{6.736298in}{2.170412in}}%
\pgfpathcurveto{\pgfqpoint{6.736298in}{2.181462in}}{\pgfqpoint{6.731908in}{2.192061in}}{\pgfqpoint{6.724094in}{2.199875in}}%
\pgfpathcurveto{\pgfqpoint{6.716281in}{2.207688in}}{\pgfqpoint{6.705682in}{2.212079in}}{\pgfqpoint{6.694632in}{2.212079in}}%
\pgfpathcurveto{\pgfqpoint{6.683581in}{2.212079in}}{\pgfqpoint{6.672982in}{2.207688in}}{\pgfqpoint{6.665169in}{2.199875in}}%
\pgfpathcurveto{\pgfqpoint{6.657355in}{2.192061in}}{\pgfqpoint{6.652965in}{2.181462in}}{\pgfqpoint{6.652965in}{2.170412in}}%
\pgfpathcurveto{\pgfqpoint{6.652965in}{2.159362in}}{\pgfqpoint{6.657355in}{2.148763in}}{\pgfqpoint{6.665169in}{2.140949in}}%
\pgfpathcurveto{\pgfqpoint{6.672982in}{2.133136in}}{\pgfqpoint{6.683581in}{2.128745in}}{\pgfqpoint{6.694632in}{2.128745in}}%
\pgfpathlineto{\pgfqpoint{6.694632in}{2.128745in}}%
\pgfpathclose%
\pgfusepath{stroke,fill}%
\end{pgfscope}%
\begin{pgfscope}%
\pgfpathrectangle{\pgfqpoint{5.292946in}{0.569136in}}{\pgfqpoint{2.177280in}{2.201755in}}%
\pgfusepath{clip}%
\pgfsetbuttcap%
\pgfsetroundjoin%
\definecolor{currentfill}{rgb}{0.172549,0.627451,0.172549}%
\pgfsetfillcolor{currentfill}%
\pgfsetlinewidth{0.481800pt}%
\definecolor{currentstroke}{rgb}{1.000000,1.000000,1.000000}%
\pgfsetstrokecolor{currentstroke}%
\pgfsetdash{}{0pt}%
\pgfpathmoveto{\pgfqpoint{6.752042in}{2.295545in}}%
\pgfpathcurveto{\pgfqpoint{6.763092in}{2.295545in}}{\pgfqpoint{6.773691in}{2.299935in}}{\pgfqpoint{6.781505in}{2.307749in}}%
\pgfpathcurveto{\pgfqpoint{6.789319in}{2.315562in}}{\pgfqpoint{6.793709in}{2.326161in}}{\pgfqpoint{6.793709in}{2.337212in}}%
\pgfpathcurveto{\pgfqpoint{6.793709in}{2.348262in}}{\pgfqpoint{6.789319in}{2.358861in}}{\pgfqpoint{6.781505in}{2.366674in}}%
\pgfpathcurveto{\pgfqpoint{6.773691in}{2.374488in}}{\pgfqpoint{6.763092in}{2.378878in}}{\pgfqpoint{6.752042in}{2.378878in}}%
\pgfpathcurveto{\pgfqpoint{6.740992in}{2.378878in}}{\pgfqpoint{6.730393in}{2.374488in}}{\pgfqpoint{6.722579in}{2.366674in}}%
\pgfpathcurveto{\pgfqpoint{6.714766in}{2.358861in}}{\pgfqpoint{6.710375in}{2.348262in}}{\pgfqpoint{6.710375in}{2.337212in}}%
\pgfpathcurveto{\pgfqpoint{6.710375in}{2.326161in}}{\pgfqpoint{6.714766in}{2.315562in}}{\pgfqpoint{6.722579in}{2.307749in}}%
\pgfpathcurveto{\pgfqpoint{6.730393in}{2.299935in}}{\pgfqpoint{6.740992in}{2.295545in}}{\pgfqpoint{6.752042in}{2.295545in}}%
\pgfpathlineto{\pgfqpoint{6.752042in}{2.295545in}}%
\pgfpathclose%
\pgfusepath{stroke,fill}%
\end{pgfscope}%
\begin{pgfscope}%
\pgfpathrectangle{\pgfqpoint{5.292946in}{0.569136in}}{\pgfqpoint{2.177280in}{2.201755in}}%
\pgfusepath{clip}%
\pgfsetbuttcap%
\pgfsetroundjoin%
\definecolor{currentfill}{rgb}{0.172549,0.627451,0.172549}%
\pgfsetfillcolor{currentfill}%
\pgfsetlinewidth{0.481800pt}%
\definecolor{currentstroke}{rgb}{1.000000,1.000000,1.000000}%
\pgfsetstrokecolor{currentstroke}%
\pgfsetdash{}{0pt}%
\pgfpathmoveto{\pgfqpoint{6.608516in}{2.212145in}}%
\pgfpathcurveto{\pgfqpoint{6.619566in}{2.212145in}}{\pgfqpoint{6.630165in}{2.216535in}}{\pgfqpoint{6.637978in}{2.224349in}}%
\pgfpathcurveto{\pgfqpoint{6.645792in}{2.232163in}}{\pgfqpoint{6.650182in}{2.242762in}}{\pgfqpoint{6.650182in}{2.253812in}}%
\pgfpathcurveto{\pgfqpoint{6.650182in}{2.264862in}}{\pgfqpoint{6.645792in}{2.275461in}}{\pgfqpoint{6.637978in}{2.283275in}}%
\pgfpathcurveto{\pgfqpoint{6.630165in}{2.291088in}}{\pgfqpoint{6.619566in}{2.295478in}}{\pgfqpoint{6.608516in}{2.295478in}}%
\pgfpathcurveto{\pgfqpoint{6.597466in}{2.295478in}}{\pgfqpoint{6.586867in}{2.291088in}}{\pgfqpoint{6.579053in}{2.283275in}}%
\pgfpathcurveto{\pgfqpoint{6.571239in}{2.275461in}}{\pgfqpoint{6.566849in}{2.264862in}}{\pgfqpoint{6.566849in}{2.253812in}}%
\pgfpathcurveto{\pgfqpoint{6.566849in}{2.242762in}}{\pgfqpoint{6.571239in}{2.232163in}}{\pgfqpoint{6.579053in}{2.224349in}}%
\pgfpathcurveto{\pgfqpoint{6.586867in}{2.216535in}}{\pgfqpoint{6.597466in}{2.212145in}}{\pgfqpoint{6.608516in}{2.212145in}}%
\pgfpathlineto{\pgfqpoint{6.608516in}{2.212145in}}%
\pgfpathclose%
\pgfusepath{stroke,fill}%
\end{pgfscope}%
\begin{pgfscope}%
\pgfpathrectangle{\pgfqpoint{5.292946in}{0.569136in}}{\pgfqpoint{2.177280in}{2.201755in}}%
\pgfusepath{clip}%
\pgfsetbuttcap%
\pgfsetroundjoin%
\definecolor{currentfill}{rgb}{0.172549,0.627451,0.172549}%
\pgfsetfillcolor{currentfill}%
\pgfsetlinewidth{0.481800pt}%
\definecolor{currentstroke}{rgb}{1.000000,1.000000,1.000000}%
\pgfsetstrokecolor{currentstroke}%
\pgfsetdash{}{0pt}%
\pgfpathmoveto{\pgfqpoint{6.637221in}{2.545744in}}%
\pgfpathcurveto{\pgfqpoint{6.648271in}{2.545744in}}{\pgfqpoint{6.658870in}{2.550135in}}{\pgfqpoint{6.666684in}{2.557948in}}%
\pgfpathcurveto{\pgfqpoint{6.674497in}{2.565762in}}{\pgfqpoint{6.678888in}{2.576361in}}{\pgfqpoint{6.678888in}{2.587411in}}%
\pgfpathcurveto{\pgfqpoint{6.678888in}{2.598461in}}{\pgfqpoint{6.674497in}{2.609060in}}{\pgfqpoint{6.666684in}{2.616874in}}%
\pgfpathcurveto{\pgfqpoint{6.658870in}{2.624687in}}{\pgfqpoint{6.648271in}{2.629078in}}{\pgfqpoint{6.637221in}{2.629078in}}%
\pgfpathcurveto{\pgfqpoint{6.626171in}{2.629078in}}{\pgfqpoint{6.615572in}{2.624687in}}{\pgfqpoint{6.607758in}{2.616874in}}%
\pgfpathcurveto{\pgfqpoint{6.599945in}{2.609060in}}{\pgfqpoint{6.595554in}{2.598461in}}{\pgfqpoint{6.595554in}{2.587411in}}%
\pgfpathcurveto{\pgfqpoint{6.595554in}{2.576361in}}{\pgfqpoint{6.599945in}{2.565762in}}{\pgfqpoint{6.607758in}{2.557948in}}%
\pgfpathcurveto{\pgfqpoint{6.615572in}{2.550135in}}{\pgfqpoint{6.626171in}{2.545744in}}{\pgfqpoint{6.637221in}{2.545744in}}%
\pgfpathlineto{\pgfqpoint{6.637221in}{2.545744in}}%
\pgfpathclose%
\pgfusepath{stroke,fill}%
\end{pgfscope}%
\begin{pgfscope}%
\pgfpathrectangle{\pgfqpoint{5.292946in}{0.569136in}}{\pgfqpoint{2.177280in}{2.201755in}}%
\pgfusepath{clip}%
\pgfsetbuttcap%
\pgfsetroundjoin%
\definecolor{currentfill}{rgb}{0.172549,0.627451,0.172549}%
\pgfsetfillcolor{currentfill}%
\pgfsetlinewidth{0.481800pt}%
\definecolor{currentstroke}{rgb}{1.000000,1.000000,1.000000}%
\pgfsetstrokecolor{currentstroke}%
\pgfsetdash{}{0pt}%
\pgfpathmoveto{\pgfqpoint{6.694632in}{2.462345in}}%
\pgfpathcurveto{\pgfqpoint{6.705682in}{2.462345in}}{\pgfqpoint{6.716281in}{2.466735in}}{\pgfqpoint{6.724094in}{2.474548in}}%
\pgfpathcurveto{\pgfqpoint{6.731908in}{2.482362in}}{\pgfqpoint{6.736298in}{2.492961in}}{\pgfqpoint{6.736298in}{2.504011in}}%
\pgfpathcurveto{\pgfqpoint{6.736298in}{2.515061in}}{\pgfqpoint{6.731908in}{2.525660in}}{\pgfqpoint{6.724094in}{2.533474in}}%
\pgfpathcurveto{\pgfqpoint{6.716281in}{2.541288in}}{\pgfqpoint{6.705682in}{2.545678in}}{\pgfqpoint{6.694632in}{2.545678in}}%
\pgfpathcurveto{\pgfqpoint{6.683581in}{2.545678in}}{\pgfqpoint{6.672982in}{2.541288in}}{\pgfqpoint{6.665169in}{2.533474in}}%
\pgfpathcurveto{\pgfqpoint{6.657355in}{2.525660in}}{\pgfqpoint{6.652965in}{2.515061in}}{\pgfqpoint{6.652965in}{2.504011in}}%
\pgfpathcurveto{\pgfqpoint{6.652965in}{2.492961in}}{\pgfqpoint{6.657355in}{2.482362in}}{\pgfqpoint{6.665169in}{2.474548in}}%
\pgfpathcurveto{\pgfqpoint{6.672982in}{2.466735in}}{\pgfqpoint{6.683581in}{2.462345in}}{\pgfqpoint{6.694632in}{2.462345in}}%
\pgfpathlineto{\pgfqpoint{6.694632in}{2.462345in}}%
\pgfpathclose%
\pgfusepath{stroke,fill}%
\end{pgfscope}%
\begin{pgfscope}%
\pgfpathrectangle{\pgfqpoint{5.292946in}{0.569136in}}{\pgfqpoint{2.177280in}{2.201755in}}%
\pgfusepath{clip}%
\pgfsetbuttcap%
\pgfsetroundjoin%
\definecolor{currentfill}{rgb}{0.172549,0.627451,0.172549}%
\pgfsetfillcolor{currentfill}%
\pgfsetlinewidth{0.481800pt}%
\definecolor{currentstroke}{rgb}{1.000000,1.000000,1.000000}%
\pgfsetstrokecolor{currentstroke}%
\pgfsetdash{}{0pt}%
\pgfpathmoveto{\pgfqpoint{6.752042in}{2.045346in}}%
\pgfpathcurveto{\pgfqpoint{6.763092in}{2.045346in}}{\pgfqpoint{6.773691in}{2.049736in}}{\pgfqpoint{6.781505in}{2.057549in}}%
\pgfpathcurveto{\pgfqpoint{6.789319in}{2.065363in}}{\pgfqpoint{6.793709in}{2.075962in}}{\pgfqpoint{6.793709in}{2.087012in}}%
\pgfpathcurveto{\pgfqpoint{6.793709in}{2.098062in}}{\pgfqpoint{6.789319in}{2.108661in}}{\pgfqpoint{6.781505in}{2.116475in}}%
\pgfpathcurveto{\pgfqpoint{6.773691in}{2.124289in}}{\pgfqpoint{6.763092in}{2.128679in}}{\pgfqpoint{6.752042in}{2.128679in}}%
\pgfpathcurveto{\pgfqpoint{6.740992in}{2.128679in}}{\pgfqpoint{6.730393in}{2.124289in}}{\pgfqpoint{6.722579in}{2.116475in}}%
\pgfpathcurveto{\pgfqpoint{6.714766in}{2.108661in}}{\pgfqpoint{6.710375in}{2.098062in}}{\pgfqpoint{6.710375in}{2.087012in}}%
\pgfpathcurveto{\pgfqpoint{6.710375in}{2.075962in}}{\pgfqpoint{6.714766in}{2.065363in}}{\pgfqpoint{6.722579in}{2.057549in}}%
\pgfpathcurveto{\pgfqpoint{6.730393in}{2.049736in}}{\pgfqpoint{6.740992in}{2.045346in}}{\pgfqpoint{6.752042in}{2.045346in}}%
\pgfpathlineto{\pgfqpoint{6.752042in}{2.045346in}}%
\pgfpathclose%
\pgfusepath{stroke,fill}%
\end{pgfscope}%
\begin{pgfscope}%
\pgfpathrectangle{\pgfqpoint{5.292946in}{0.569136in}}{\pgfqpoint{2.177280in}{2.201755in}}%
\pgfusepath{clip}%
\pgfsetbuttcap%
\pgfsetroundjoin%
\definecolor{currentfill}{rgb}{0.172549,0.627451,0.172549}%
\pgfsetfillcolor{currentfill}%
\pgfsetlinewidth{0.481800pt}%
\definecolor{currentstroke}{rgb}{1.000000,1.000000,1.000000}%
\pgfsetstrokecolor{currentstroke}%
\pgfsetdash{}{0pt}%
\pgfpathmoveto{\pgfqpoint{7.096506in}{2.378945in}}%
\pgfpathcurveto{\pgfqpoint{7.107556in}{2.378945in}}{\pgfqpoint{7.118155in}{2.383335in}}{\pgfqpoint{7.125968in}{2.391149in}}%
\pgfpathcurveto{\pgfqpoint{7.133782in}{2.398962in}}{\pgfqpoint{7.138172in}{2.409561in}}{\pgfqpoint{7.138172in}{2.420611in}}%
\pgfpathcurveto{\pgfqpoint{7.138172in}{2.431662in}}{\pgfqpoint{7.133782in}{2.442261in}}{\pgfqpoint{7.125968in}{2.450074in}}%
\pgfpathcurveto{\pgfqpoint{7.118155in}{2.457888in}}{\pgfqpoint{7.107556in}{2.462278in}}{\pgfqpoint{7.096506in}{2.462278in}}%
\pgfpathcurveto{\pgfqpoint{7.085455in}{2.462278in}}{\pgfqpoint{7.074856in}{2.457888in}}{\pgfqpoint{7.067043in}{2.450074in}}%
\pgfpathcurveto{\pgfqpoint{7.059229in}{2.442261in}}{\pgfqpoint{7.054839in}{2.431662in}}{\pgfqpoint{7.054839in}{2.420611in}}%
\pgfpathcurveto{\pgfqpoint{7.054839in}{2.409561in}}{\pgfqpoint{7.059229in}{2.398962in}}{\pgfqpoint{7.067043in}{2.391149in}}%
\pgfpathcurveto{\pgfqpoint{7.074856in}{2.383335in}}{\pgfqpoint{7.085455in}{2.378945in}}{\pgfqpoint{7.096506in}{2.378945in}}%
\pgfpathlineto{\pgfqpoint{7.096506in}{2.378945in}}%
\pgfpathclose%
\pgfusepath{stroke,fill}%
\end{pgfscope}%
\begin{pgfscope}%
\pgfpathrectangle{\pgfqpoint{5.292946in}{0.569136in}}{\pgfqpoint{2.177280in}{2.201755in}}%
\pgfusepath{clip}%
\pgfsetbuttcap%
\pgfsetroundjoin%
\definecolor{currentfill}{rgb}{0.172549,0.627451,0.172549}%
\pgfsetfillcolor{currentfill}%
\pgfsetlinewidth{0.481800pt}%
\definecolor{currentstroke}{rgb}{1.000000,1.000000,1.000000}%
\pgfsetstrokecolor{currentstroke}%
\pgfsetdash{}{0pt}%
\pgfpathmoveto{\pgfqpoint{7.153916in}{2.462345in}}%
\pgfpathcurveto{\pgfqpoint{7.164966in}{2.462345in}}{\pgfqpoint{7.175565in}{2.466735in}}{\pgfqpoint{7.183379in}{2.474548in}}%
\pgfpathcurveto{\pgfqpoint{7.191193in}{2.482362in}}{\pgfqpoint{7.195583in}{2.492961in}}{\pgfqpoint{7.195583in}{2.504011in}}%
\pgfpathcurveto{\pgfqpoint{7.195583in}{2.515061in}}{\pgfqpoint{7.191193in}{2.525660in}}{\pgfqpoint{7.183379in}{2.533474in}}%
\pgfpathcurveto{\pgfqpoint{7.175565in}{2.541288in}}{\pgfqpoint{7.164966in}{2.545678in}}{\pgfqpoint{7.153916in}{2.545678in}}%
\pgfpathcurveto{\pgfqpoint{7.142866in}{2.545678in}}{\pgfqpoint{7.132267in}{2.541288in}}{\pgfqpoint{7.124453in}{2.533474in}}%
\pgfpathcurveto{\pgfqpoint{7.116640in}{2.525660in}}{\pgfqpoint{7.112249in}{2.515061in}}{\pgfqpoint{7.112249in}{2.504011in}}%
\pgfpathcurveto{\pgfqpoint{7.112249in}{2.492961in}}{\pgfqpoint{7.116640in}{2.482362in}}{\pgfqpoint{7.124453in}{2.474548in}}%
\pgfpathcurveto{\pgfqpoint{7.132267in}{2.466735in}}{\pgfqpoint{7.142866in}{2.462345in}}{\pgfqpoint{7.153916in}{2.462345in}}%
\pgfpathlineto{\pgfqpoint{7.153916in}{2.462345in}}%
\pgfpathclose%
\pgfusepath{stroke,fill}%
\end{pgfscope}%
\begin{pgfscope}%
\pgfpathrectangle{\pgfqpoint{5.292946in}{0.569136in}}{\pgfqpoint{2.177280in}{2.201755in}}%
\pgfusepath{clip}%
\pgfsetbuttcap%
\pgfsetroundjoin%
\definecolor{currentfill}{rgb}{0.172549,0.627451,0.172549}%
\pgfsetfillcolor{currentfill}%
\pgfsetlinewidth{0.481800pt}%
\definecolor{currentstroke}{rgb}{1.000000,1.000000,1.000000}%
\pgfsetstrokecolor{currentstroke}%
\pgfsetdash{}{0pt}%
\pgfpathmoveto{\pgfqpoint{6.608516in}{1.795146in}}%
\pgfpathcurveto{\pgfqpoint{6.619566in}{1.795146in}}{\pgfqpoint{6.630165in}{1.799536in}}{\pgfqpoint{6.637978in}{1.807350in}}%
\pgfpathcurveto{\pgfqpoint{6.645792in}{1.815164in}}{\pgfqpoint{6.650182in}{1.825763in}}{\pgfqpoint{6.650182in}{1.836813in}}%
\pgfpathcurveto{\pgfqpoint{6.650182in}{1.847863in}}{\pgfqpoint{6.645792in}{1.858462in}}{\pgfqpoint{6.637978in}{1.866276in}}%
\pgfpathcurveto{\pgfqpoint{6.630165in}{1.874089in}}{\pgfqpoint{6.619566in}{1.878479in}}{\pgfqpoint{6.608516in}{1.878479in}}%
\pgfpathcurveto{\pgfqpoint{6.597466in}{1.878479in}}{\pgfqpoint{6.586867in}{1.874089in}}{\pgfqpoint{6.579053in}{1.866276in}}%
\pgfpathcurveto{\pgfqpoint{6.571239in}{1.858462in}}{\pgfqpoint{6.566849in}{1.847863in}}{\pgfqpoint{6.566849in}{1.836813in}}%
\pgfpathcurveto{\pgfqpoint{6.566849in}{1.825763in}}{\pgfqpoint{6.571239in}{1.815164in}}{\pgfqpoint{6.579053in}{1.807350in}}%
\pgfpathcurveto{\pgfqpoint{6.586867in}{1.799536in}}{\pgfqpoint{6.597466in}{1.795146in}}{\pgfqpoint{6.608516in}{1.795146in}}%
\pgfpathlineto{\pgfqpoint{6.608516in}{1.795146in}}%
\pgfpathclose%
\pgfusepath{stroke,fill}%
\end{pgfscope}%
\begin{pgfscope}%
\pgfpathrectangle{\pgfqpoint{5.292946in}{0.569136in}}{\pgfqpoint{2.177280in}{2.201755in}}%
\pgfusepath{clip}%
\pgfsetbuttcap%
\pgfsetroundjoin%
\definecolor{currentfill}{rgb}{0.172549,0.627451,0.172549}%
\pgfsetfillcolor{currentfill}%
\pgfsetlinewidth{0.481800pt}%
\definecolor{currentstroke}{rgb}{1.000000,1.000000,1.000000}%
\pgfsetstrokecolor{currentstroke}%
\pgfsetdash{}{0pt}%
\pgfpathmoveto{\pgfqpoint{6.809453in}{2.462345in}}%
\pgfpathcurveto{\pgfqpoint{6.820503in}{2.462345in}}{\pgfqpoint{6.831102in}{2.466735in}}{\pgfqpoint{6.838915in}{2.474548in}}%
\pgfpathcurveto{\pgfqpoint{6.846729in}{2.482362in}}{\pgfqpoint{6.851119in}{2.492961in}}{\pgfqpoint{6.851119in}{2.504011in}}%
\pgfpathcurveto{\pgfqpoint{6.851119in}{2.515061in}}{\pgfqpoint{6.846729in}{2.525660in}}{\pgfqpoint{6.838915in}{2.533474in}}%
\pgfpathcurveto{\pgfqpoint{6.831102in}{2.541288in}}{\pgfqpoint{6.820503in}{2.545678in}}{\pgfqpoint{6.809453in}{2.545678in}}%
\pgfpathcurveto{\pgfqpoint{6.798403in}{2.545678in}}{\pgfqpoint{6.787804in}{2.541288in}}{\pgfqpoint{6.779990in}{2.533474in}}%
\pgfpathcurveto{\pgfqpoint{6.772176in}{2.525660in}}{\pgfqpoint{6.767786in}{2.515061in}}{\pgfqpoint{6.767786in}{2.504011in}}%
\pgfpathcurveto{\pgfqpoint{6.767786in}{2.492961in}}{\pgfqpoint{6.772176in}{2.482362in}}{\pgfqpoint{6.779990in}{2.474548in}}%
\pgfpathcurveto{\pgfqpoint{6.787804in}{2.466735in}}{\pgfqpoint{6.798403in}{2.462345in}}{\pgfqpoint{6.809453in}{2.462345in}}%
\pgfpathlineto{\pgfqpoint{6.809453in}{2.462345in}}%
\pgfpathclose%
\pgfusepath{stroke,fill}%
\end{pgfscope}%
\begin{pgfscope}%
\pgfpathrectangle{\pgfqpoint{5.292946in}{0.569136in}}{\pgfqpoint{2.177280in}{2.201755in}}%
\pgfusepath{clip}%
\pgfsetbuttcap%
\pgfsetroundjoin%
\definecolor{currentfill}{rgb}{0.172549,0.627451,0.172549}%
\pgfsetfillcolor{currentfill}%
\pgfsetlinewidth{0.481800pt}%
\definecolor{currentstroke}{rgb}{1.000000,1.000000,1.000000}%
\pgfsetstrokecolor{currentstroke}%
\pgfsetdash{}{0pt}%
\pgfpathmoveto{\pgfqpoint{6.579810in}{2.212145in}}%
\pgfpathcurveto{\pgfqpoint{6.590861in}{2.212145in}}{\pgfqpoint{6.601460in}{2.216535in}}{\pgfqpoint{6.609273in}{2.224349in}}%
\pgfpathcurveto{\pgfqpoint{6.617087in}{2.232163in}}{\pgfqpoint{6.621477in}{2.242762in}}{\pgfqpoint{6.621477in}{2.253812in}}%
\pgfpathcurveto{\pgfqpoint{6.621477in}{2.264862in}}{\pgfqpoint{6.617087in}{2.275461in}}{\pgfqpoint{6.609273in}{2.283275in}}%
\pgfpathcurveto{\pgfqpoint{6.601460in}{2.291088in}}{\pgfqpoint{6.590861in}{2.295478in}}{\pgfqpoint{6.579810in}{2.295478in}}%
\pgfpathcurveto{\pgfqpoint{6.568760in}{2.295478in}}{\pgfqpoint{6.558161in}{2.291088in}}{\pgfqpoint{6.550348in}{2.283275in}}%
\pgfpathcurveto{\pgfqpoint{6.542534in}{2.275461in}}{\pgfqpoint{6.538144in}{2.264862in}}{\pgfqpoint{6.538144in}{2.253812in}}%
\pgfpathcurveto{\pgfqpoint{6.538144in}{2.242762in}}{\pgfqpoint{6.542534in}{2.232163in}}{\pgfqpoint{6.550348in}{2.224349in}}%
\pgfpathcurveto{\pgfqpoint{6.558161in}{2.216535in}}{\pgfqpoint{6.568760in}{2.212145in}}{\pgfqpoint{6.579810in}{2.212145in}}%
\pgfpathlineto{\pgfqpoint{6.579810in}{2.212145in}}%
\pgfpathclose%
\pgfusepath{stroke,fill}%
\end{pgfscope}%
\begin{pgfscope}%
\pgfpathrectangle{\pgfqpoint{5.292946in}{0.569136in}}{\pgfqpoint{2.177280in}{2.201755in}}%
\pgfusepath{clip}%
\pgfsetbuttcap%
\pgfsetroundjoin%
\definecolor{currentfill}{rgb}{0.172549,0.627451,0.172549}%
\pgfsetfillcolor{currentfill}%
\pgfsetlinewidth{0.481800pt}%
\definecolor{currentstroke}{rgb}{1.000000,1.000000,1.000000}%
\pgfsetstrokecolor{currentstroke}%
\pgfsetdash{}{0pt}%
\pgfpathmoveto{\pgfqpoint{7.096506in}{2.212145in}}%
\pgfpathcurveto{\pgfqpoint{7.107556in}{2.212145in}}{\pgfqpoint{7.118155in}{2.216535in}}{\pgfqpoint{7.125968in}{2.224349in}}%
\pgfpathcurveto{\pgfqpoint{7.133782in}{2.232163in}}{\pgfqpoint{7.138172in}{2.242762in}}{\pgfqpoint{7.138172in}{2.253812in}}%
\pgfpathcurveto{\pgfqpoint{7.138172in}{2.264862in}}{\pgfqpoint{7.133782in}{2.275461in}}{\pgfqpoint{7.125968in}{2.283275in}}%
\pgfpathcurveto{\pgfqpoint{7.118155in}{2.291088in}}{\pgfqpoint{7.107556in}{2.295478in}}{\pgfqpoint{7.096506in}{2.295478in}}%
\pgfpathcurveto{\pgfqpoint{7.085455in}{2.295478in}}{\pgfqpoint{7.074856in}{2.291088in}}{\pgfqpoint{7.067043in}{2.283275in}}%
\pgfpathcurveto{\pgfqpoint{7.059229in}{2.275461in}}{\pgfqpoint{7.054839in}{2.264862in}}{\pgfqpoint{7.054839in}{2.253812in}}%
\pgfpathcurveto{\pgfqpoint{7.054839in}{2.242762in}}{\pgfqpoint{7.059229in}{2.232163in}}{\pgfqpoint{7.067043in}{2.224349in}}%
\pgfpathcurveto{\pgfqpoint{7.074856in}{2.216535in}}{\pgfqpoint{7.085455in}{2.212145in}}{\pgfqpoint{7.096506in}{2.212145in}}%
\pgfpathlineto{\pgfqpoint{7.096506in}{2.212145in}}%
\pgfpathclose%
\pgfusepath{stroke,fill}%
\end{pgfscope}%
\begin{pgfscope}%
\pgfpathrectangle{\pgfqpoint{5.292946in}{0.569136in}}{\pgfqpoint{2.177280in}{2.201755in}}%
\pgfusepath{clip}%
\pgfsetbuttcap%
\pgfsetroundjoin%
\definecolor{currentfill}{rgb}{0.172549,0.627451,0.172549}%
\pgfsetfillcolor{currentfill}%
\pgfsetlinewidth{0.481800pt}%
\definecolor{currentstroke}{rgb}{1.000000,1.000000,1.000000}%
\pgfsetstrokecolor{currentstroke}%
\pgfsetdash{}{0pt}%
\pgfpathmoveto{\pgfqpoint{6.579810in}{2.045346in}}%
\pgfpathcurveto{\pgfqpoint{6.590861in}{2.045346in}}{\pgfqpoint{6.601460in}{2.049736in}}{\pgfqpoint{6.609273in}{2.057549in}}%
\pgfpathcurveto{\pgfqpoint{6.617087in}{2.065363in}}{\pgfqpoint{6.621477in}{2.075962in}}{\pgfqpoint{6.621477in}{2.087012in}}%
\pgfpathcurveto{\pgfqpoint{6.621477in}{2.098062in}}{\pgfqpoint{6.617087in}{2.108661in}}{\pgfqpoint{6.609273in}{2.116475in}}%
\pgfpathcurveto{\pgfqpoint{6.601460in}{2.124289in}}{\pgfqpoint{6.590861in}{2.128679in}}{\pgfqpoint{6.579810in}{2.128679in}}%
\pgfpathcurveto{\pgfqpoint{6.568760in}{2.128679in}}{\pgfqpoint{6.558161in}{2.124289in}}{\pgfqpoint{6.550348in}{2.116475in}}%
\pgfpathcurveto{\pgfqpoint{6.542534in}{2.108661in}}{\pgfqpoint{6.538144in}{2.098062in}}{\pgfqpoint{6.538144in}{2.087012in}}%
\pgfpathcurveto{\pgfqpoint{6.538144in}{2.075962in}}{\pgfqpoint{6.542534in}{2.065363in}}{\pgfqpoint{6.550348in}{2.057549in}}%
\pgfpathcurveto{\pgfqpoint{6.558161in}{2.049736in}}{\pgfqpoint{6.568760in}{2.045346in}}{\pgfqpoint{6.579810in}{2.045346in}}%
\pgfpathlineto{\pgfqpoint{6.579810in}{2.045346in}}%
\pgfpathclose%
\pgfusepath{stroke,fill}%
\end{pgfscope}%
\begin{pgfscope}%
\pgfpathrectangle{\pgfqpoint{5.292946in}{0.569136in}}{\pgfqpoint{2.177280in}{2.201755in}}%
\pgfusepath{clip}%
\pgfsetbuttcap%
\pgfsetroundjoin%
\definecolor{currentfill}{rgb}{0.172549,0.627451,0.172549}%
\pgfsetfillcolor{currentfill}%
\pgfsetlinewidth{0.481800pt}%
\definecolor{currentstroke}{rgb}{1.000000,1.000000,1.000000}%
\pgfsetstrokecolor{currentstroke}%
\pgfsetdash{}{0pt}%
\pgfpathmoveto{\pgfqpoint{6.809453in}{2.295545in}}%
\pgfpathcurveto{\pgfqpoint{6.820503in}{2.295545in}}{\pgfqpoint{6.831102in}{2.299935in}}{\pgfqpoint{6.838915in}{2.307749in}}%
\pgfpathcurveto{\pgfqpoint{6.846729in}{2.315562in}}{\pgfqpoint{6.851119in}{2.326161in}}{\pgfqpoint{6.851119in}{2.337212in}}%
\pgfpathcurveto{\pgfqpoint{6.851119in}{2.348262in}}{\pgfqpoint{6.846729in}{2.358861in}}{\pgfqpoint{6.838915in}{2.366674in}}%
\pgfpathcurveto{\pgfqpoint{6.831102in}{2.374488in}}{\pgfqpoint{6.820503in}{2.378878in}}{\pgfqpoint{6.809453in}{2.378878in}}%
\pgfpathcurveto{\pgfqpoint{6.798403in}{2.378878in}}{\pgfqpoint{6.787804in}{2.374488in}}{\pgfqpoint{6.779990in}{2.366674in}}%
\pgfpathcurveto{\pgfqpoint{6.772176in}{2.358861in}}{\pgfqpoint{6.767786in}{2.348262in}}{\pgfqpoint{6.767786in}{2.337212in}}%
\pgfpathcurveto{\pgfqpoint{6.767786in}{2.326161in}}{\pgfqpoint{6.772176in}{2.315562in}}{\pgfqpoint{6.779990in}{2.307749in}}%
\pgfpathcurveto{\pgfqpoint{6.787804in}{2.299935in}}{\pgfqpoint{6.798403in}{2.295545in}}{\pgfqpoint{6.809453in}{2.295545in}}%
\pgfpathlineto{\pgfqpoint{6.809453in}{2.295545in}}%
\pgfpathclose%
\pgfusepath{stroke,fill}%
\end{pgfscope}%
\begin{pgfscope}%
\pgfpathrectangle{\pgfqpoint{5.292946in}{0.569136in}}{\pgfqpoint{2.177280in}{2.201755in}}%
\pgfusepath{clip}%
\pgfsetbuttcap%
\pgfsetroundjoin%
\definecolor{currentfill}{rgb}{0.172549,0.627451,0.172549}%
\pgfsetfillcolor{currentfill}%
\pgfsetlinewidth{0.481800pt}%
\definecolor{currentstroke}{rgb}{1.000000,1.000000,1.000000}%
\pgfsetstrokecolor{currentstroke}%
\pgfsetdash{}{0pt}%
\pgfpathmoveto{\pgfqpoint{6.895569in}{2.045346in}}%
\pgfpathcurveto{\pgfqpoint{6.906619in}{2.045346in}}{\pgfqpoint{6.917218in}{2.049736in}}{\pgfqpoint{6.925031in}{2.057549in}}%
\pgfpathcurveto{\pgfqpoint{6.932845in}{2.065363in}}{\pgfqpoint{6.937235in}{2.075962in}}{\pgfqpoint{6.937235in}{2.087012in}}%
\pgfpathcurveto{\pgfqpoint{6.937235in}{2.098062in}}{\pgfqpoint{6.932845in}{2.108661in}}{\pgfqpoint{6.925031in}{2.116475in}}%
\pgfpathcurveto{\pgfqpoint{6.917218in}{2.124289in}}{\pgfqpoint{6.906619in}{2.128679in}}{\pgfqpoint{6.895569in}{2.128679in}}%
\pgfpathcurveto{\pgfqpoint{6.884518in}{2.128679in}}{\pgfqpoint{6.873919in}{2.124289in}}{\pgfqpoint{6.866106in}{2.116475in}}%
\pgfpathcurveto{\pgfqpoint{6.858292in}{2.108661in}}{\pgfqpoint{6.853902in}{2.098062in}}{\pgfqpoint{6.853902in}{2.087012in}}%
\pgfpathcurveto{\pgfqpoint{6.853902in}{2.075962in}}{\pgfqpoint{6.858292in}{2.065363in}}{\pgfqpoint{6.866106in}{2.057549in}}%
\pgfpathcurveto{\pgfqpoint{6.873919in}{2.049736in}}{\pgfqpoint{6.884518in}{2.045346in}}{\pgfqpoint{6.895569in}{2.045346in}}%
\pgfpathlineto{\pgfqpoint{6.895569in}{2.045346in}}%
\pgfpathclose%
\pgfusepath{stroke,fill}%
\end{pgfscope}%
\begin{pgfscope}%
\pgfpathrectangle{\pgfqpoint{5.292946in}{0.569136in}}{\pgfqpoint{2.177280in}{2.201755in}}%
\pgfusepath{clip}%
\pgfsetbuttcap%
\pgfsetroundjoin%
\definecolor{currentfill}{rgb}{0.172549,0.627451,0.172549}%
\pgfsetfillcolor{currentfill}%
\pgfsetlinewidth{0.481800pt}%
\definecolor{currentstroke}{rgb}{1.000000,1.000000,1.000000}%
\pgfsetstrokecolor{currentstroke}%
\pgfsetdash{}{0pt}%
\pgfpathmoveto{\pgfqpoint{6.551105in}{2.045346in}}%
\pgfpathcurveto{\pgfqpoint{6.562155in}{2.045346in}}{\pgfqpoint{6.572754in}{2.049736in}}{\pgfqpoint{6.580568in}{2.057549in}}%
\pgfpathcurveto{\pgfqpoint{6.588382in}{2.065363in}}{\pgfqpoint{6.592772in}{2.075962in}}{\pgfqpoint{6.592772in}{2.087012in}}%
\pgfpathcurveto{\pgfqpoint{6.592772in}{2.098062in}}{\pgfqpoint{6.588382in}{2.108661in}}{\pgfqpoint{6.580568in}{2.116475in}}%
\pgfpathcurveto{\pgfqpoint{6.572754in}{2.124289in}}{\pgfqpoint{6.562155in}{2.128679in}}{\pgfqpoint{6.551105in}{2.128679in}}%
\pgfpathcurveto{\pgfqpoint{6.540055in}{2.128679in}}{\pgfqpoint{6.529456in}{2.124289in}}{\pgfqpoint{6.521642in}{2.116475in}}%
\pgfpathcurveto{\pgfqpoint{6.513829in}{2.108661in}}{\pgfqpoint{6.509438in}{2.098062in}}{\pgfqpoint{6.509438in}{2.087012in}}%
\pgfpathcurveto{\pgfqpoint{6.509438in}{2.075962in}}{\pgfqpoint{6.513829in}{2.065363in}}{\pgfqpoint{6.521642in}{2.057549in}}%
\pgfpathcurveto{\pgfqpoint{6.529456in}{2.049736in}}{\pgfqpoint{6.540055in}{2.045346in}}{\pgfqpoint{6.551105in}{2.045346in}}%
\pgfpathlineto{\pgfqpoint{6.551105in}{2.045346in}}%
\pgfpathclose%
\pgfusepath{stroke,fill}%
\end{pgfscope}%
\begin{pgfscope}%
\pgfpathrectangle{\pgfqpoint{5.292946in}{0.569136in}}{\pgfqpoint{2.177280in}{2.201755in}}%
\pgfusepath{clip}%
\pgfsetbuttcap%
\pgfsetroundjoin%
\definecolor{currentfill}{rgb}{0.172549,0.627451,0.172549}%
\pgfsetfillcolor{currentfill}%
\pgfsetlinewidth{0.481800pt}%
\definecolor{currentstroke}{rgb}{1.000000,1.000000,1.000000}%
\pgfsetstrokecolor{currentstroke}%
\pgfsetdash{}{0pt}%
\pgfpathmoveto{\pgfqpoint{6.579810in}{2.045346in}}%
\pgfpathcurveto{\pgfqpoint{6.590861in}{2.045346in}}{\pgfqpoint{6.601460in}{2.049736in}}{\pgfqpoint{6.609273in}{2.057549in}}%
\pgfpathcurveto{\pgfqpoint{6.617087in}{2.065363in}}{\pgfqpoint{6.621477in}{2.075962in}}{\pgfqpoint{6.621477in}{2.087012in}}%
\pgfpathcurveto{\pgfqpoint{6.621477in}{2.098062in}}{\pgfqpoint{6.617087in}{2.108661in}}{\pgfqpoint{6.609273in}{2.116475in}}%
\pgfpathcurveto{\pgfqpoint{6.601460in}{2.124289in}}{\pgfqpoint{6.590861in}{2.128679in}}{\pgfqpoint{6.579810in}{2.128679in}}%
\pgfpathcurveto{\pgfqpoint{6.568760in}{2.128679in}}{\pgfqpoint{6.558161in}{2.124289in}}{\pgfqpoint{6.550348in}{2.116475in}}%
\pgfpathcurveto{\pgfqpoint{6.542534in}{2.108661in}}{\pgfqpoint{6.538144in}{2.098062in}}{\pgfqpoint{6.538144in}{2.087012in}}%
\pgfpathcurveto{\pgfqpoint{6.538144in}{2.075962in}}{\pgfqpoint{6.542534in}{2.065363in}}{\pgfqpoint{6.550348in}{2.057549in}}%
\pgfpathcurveto{\pgfqpoint{6.558161in}{2.049736in}}{\pgfqpoint{6.568760in}{2.045346in}}{\pgfqpoint{6.579810in}{2.045346in}}%
\pgfpathlineto{\pgfqpoint{6.579810in}{2.045346in}}%
\pgfpathclose%
\pgfusepath{stroke,fill}%
\end{pgfscope}%
\begin{pgfscope}%
\pgfpathrectangle{\pgfqpoint{5.292946in}{0.569136in}}{\pgfqpoint{2.177280in}{2.201755in}}%
\pgfusepath{clip}%
\pgfsetbuttcap%
\pgfsetroundjoin%
\definecolor{currentfill}{rgb}{0.172549,0.627451,0.172549}%
\pgfsetfillcolor{currentfill}%
\pgfsetlinewidth{0.481800pt}%
\definecolor{currentstroke}{rgb}{1.000000,1.000000,1.000000}%
\pgfsetstrokecolor{currentstroke}%
\pgfsetdash{}{0pt}%
\pgfpathmoveto{\pgfqpoint{6.780747in}{2.295545in}}%
\pgfpathcurveto{\pgfqpoint{6.791798in}{2.295545in}}{\pgfqpoint{6.802397in}{2.299935in}}{\pgfqpoint{6.810210in}{2.307749in}}%
\pgfpathcurveto{\pgfqpoint{6.818024in}{2.315562in}}{\pgfqpoint{6.822414in}{2.326161in}}{\pgfqpoint{6.822414in}{2.337212in}}%
\pgfpathcurveto{\pgfqpoint{6.822414in}{2.348262in}}{\pgfqpoint{6.818024in}{2.358861in}}{\pgfqpoint{6.810210in}{2.366674in}}%
\pgfpathcurveto{\pgfqpoint{6.802397in}{2.374488in}}{\pgfqpoint{6.791798in}{2.378878in}}{\pgfqpoint{6.780747in}{2.378878in}}%
\pgfpathcurveto{\pgfqpoint{6.769697in}{2.378878in}}{\pgfqpoint{6.759098in}{2.374488in}}{\pgfqpoint{6.751285in}{2.366674in}}%
\pgfpathcurveto{\pgfqpoint{6.743471in}{2.358861in}}{\pgfqpoint{6.739081in}{2.348262in}}{\pgfqpoint{6.739081in}{2.337212in}}%
\pgfpathcurveto{\pgfqpoint{6.739081in}{2.326161in}}{\pgfqpoint{6.743471in}{2.315562in}}{\pgfqpoint{6.751285in}{2.307749in}}%
\pgfpathcurveto{\pgfqpoint{6.759098in}{2.299935in}}{\pgfqpoint{6.769697in}{2.295545in}}{\pgfqpoint{6.780747in}{2.295545in}}%
\pgfpathlineto{\pgfqpoint{6.780747in}{2.295545in}}%
\pgfpathclose%
\pgfusepath{stroke,fill}%
\end{pgfscope}%
\begin{pgfscope}%
\pgfpathrectangle{\pgfqpoint{5.292946in}{0.569136in}}{\pgfqpoint{2.177280in}{2.201755in}}%
\pgfusepath{clip}%
\pgfsetbuttcap%
\pgfsetroundjoin%
\definecolor{currentfill}{rgb}{0.172549,0.627451,0.172549}%
\pgfsetfillcolor{currentfill}%
\pgfsetlinewidth{0.481800pt}%
\definecolor{currentstroke}{rgb}{1.000000,1.000000,1.000000}%
\pgfsetstrokecolor{currentstroke}%
\pgfsetdash{}{0pt}%
\pgfpathmoveto{\pgfqpoint{6.838158in}{1.878546in}}%
\pgfpathcurveto{\pgfqpoint{6.849208in}{1.878546in}}{\pgfqpoint{6.859807in}{1.882936in}}{\pgfqpoint{6.867621in}{1.890750in}}%
\pgfpathcurveto{\pgfqpoint{6.875434in}{1.898563in}}{\pgfqpoint{6.879825in}{1.909162in}}{\pgfqpoint{6.879825in}{1.920213in}}%
\pgfpathcurveto{\pgfqpoint{6.879825in}{1.931263in}}{\pgfqpoint{6.875434in}{1.941862in}}{\pgfqpoint{6.867621in}{1.949675in}}%
\pgfpathcurveto{\pgfqpoint{6.859807in}{1.957489in}}{\pgfqpoint{6.849208in}{1.961879in}}{\pgfqpoint{6.838158in}{1.961879in}}%
\pgfpathcurveto{\pgfqpoint{6.827108in}{1.961879in}}{\pgfqpoint{6.816509in}{1.957489in}}{\pgfqpoint{6.808695in}{1.949675in}}%
\pgfpathcurveto{\pgfqpoint{6.800882in}{1.941862in}}{\pgfqpoint{6.796491in}{1.931263in}}{\pgfqpoint{6.796491in}{1.920213in}}%
\pgfpathcurveto{\pgfqpoint{6.796491in}{1.909162in}}{\pgfqpoint{6.800882in}{1.898563in}}{\pgfqpoint{6.808695in}{1.890750in}}%
\pgfpathcurveto{\pgfqpoint{6.816509in}{1.882936in}}{\pgfqpoint{6.827108in}{1.878546in}}{\pgfqpoint{6.838158in}{1.878546in}}%
\pgfpathlineto{\pgfqpoint{6.838158in}{1.878546in}}%
\pgfpathclose%
\pgfusepath{stroke,fill}%
\end{pgfscope}%
\begin{pgfscope}%
\pgfpathrectangle{\pgfqpoint{5.292946in}{0.569136in}}{\pgfqpoint{2.177280in}{2.201755in}}%
\pgfusepath{clip}%
\pgfsetbuttcap%
\pgfsetroundjoin%
\definecolor{currentfill}{rgb}{0.172549,0.627451,0.172549}%
\pgfsetfillcolor{currentfill}%
\pgfsetlinewidth{0.481800pt}%
\definecolor{currentstroke}{rgb}{1.000000,1.000000,1.000000}%
\pgfsetstrokecolor{currentstroke}%
\pgfsetdash{}{0pt}%
\pgfpathmoveto{\pgfqpoint{6.924274in}{2.128745in}}%
\pgfpathcurveto{\pgfqpoint{6.935324in}{2.128745in}}{\pgfqpoint{6.945923in}{2.133136in}}{\pgfqpoint{6.953737in}{2.140949in}}%
\pgfpathcurveto{\pgfqpoint{6.961550in}{2.148763in}}{\pgfqpoint{6.965941in}{2.159362in}}{\pgfqpoint{6.965941in}{2.170412in}}%
\pgfpathcurveto{\pgfqpoint{6.965941in}{2.181462in}}{\pgfqpoint{6.961550in}{2.192061in}}{\pgfqpoint{6.953737in}{2.199875in}}%
\pgfpathcurveto{\pgfqpoint{6.945923in}{2.207688in}}{\pgfqpoint{6.935324in}{2.212079in}}{\pgfqpoint{6.924274in}{2.212079in}}%
\pgfpathcurveto{\pgfqpoint{6.913224in}{2.212079in}}{\pgfqpoint{6.902625in}{2.207688in}}{\pgfqpoint{6.894811in}{2.199875in}}%
\pgfpathcurveto{\pgfqpoint{6.886997in}{2.192061in}}{\pgfqpoint{6.882607in}{2.181462in}}{\pgfqpoint{6.882607in}{2.170412in}}%
\pgfpathcurveto{\pgfqpoint{6.882607in}{2.159362in}}{\pgfqpoint{6.886997in}{2.148763in}}{\pgfqpoint{6.894811in}{2.140949in}}%
\pgfpathcurveto{\pgfqpoint{6.902625in}{2.133136in}}{\pgfqpoint{6.913224in}{2.128745in}}{\pgfqpoint{6.924274in}{2.128745in}}%
\pgfpathlineto{\pgfqpoint{6.924274in}{2.128745in}}%
\pgfpathclose%
\pgfusepath{stroke,fill}%
\end{pgfscope}%
\begin{pgfscope}%
\pgfpathrectangle{\pgfqpoint{5.292946in}{0.569136in}}{\pgfqpoint{2.177280in}{2.201755in}}%
\pgfusepath{clip}%
\pgfsetbuttcap%
\pgfsetroundjoin%
\definecolor{currentfill}{rgb}{0.172549,0.627451,0.172549}%
\pgfsetfillcolor{currentfill}%
\pgfsetlinewidth{0.481800pt}%
\definecolor{currentstroke}{rgb}{1.000000,1.000000,1.000000}%
\pgfsetstrokecolor{currentstroke}%
\pgfsetdash{}{0pt}%
\pgfpathmoveto{\pgfqpoint{7.010390in}{2.212145in}}%
\pgfpathcurveto{\pgfqpoint{7.021440in}{2.212145in}}{\pgfqpoint{7.032039in}{2.216535in}}{\pgfqpoint{7.039852in}{2.224349in}}%
\pgfpathcurveto{\pgfqpoint{7.047666in}{2.232163in}}{\pgfqpoint{7.052056in}{2.242762in}}{\pgfqpoint{7.052056in}{2.253812in}}%
\pgfpathcurveto{\pgfqpoint{7.052056in}{2.264862in}}{\pgfqpoint{7.047666in}{2.275461in}}{\pgfqpoint{7.039852in}{2.283275in}}%
\pgfpathcurveto{\pgfqpoint{7.032039in}{2.291088in}}{\pgfqpoint{7.021440in}{2.295478in}}{\pgfqpoint{7.010390in}{2.295478in}}%
\pgfpathcurveto{\pgfqpoint{6.999340in}{2.295478in}}{\pgfqpoint{6.988741in}{2.291088in}}{\pgfqpoint{6.980927in}{2.283275in}}%
\pgfpathcurveto{\pgfqpoint{6.973113in}{2.275461in}}{\pgfqpoint{6.968723in}{2.264862in}}{\pgfqpoint{6.968723in}{2.253812in}}%
\pgfpathcurveto{\pgfqpoint{6.968723in}{2.242762in}}{\pgfqpoint{6.973113in}{2.232163in}}{\pgfqpoint{6.980927in}{2.224349in}}%
\pgfpathcurveto{\pgfqpoint{6.988741in}{2.216535in}}{\pgfqpoint{6.999340in}{2.212145in}}{\pgfqpoint{7.010390in}{2.212145in}}%
\pgfpathlineto{\pgfqpoint{7.010390in}{2.212145in}}%
\pgfpathclose%
\pgfusepath{stroke,fill}%
\end{pgfscope}%
\begin{pgfscope}%
\pgfpathrectangle{\pgfqpoint{5.292946in}{0.569136in}}{\pgfqpoint{2.177280in}{2.201755in}}%
\pgfusepath{clip}%
\pgfsetbuttcap%
\pgfsetroundjoin%
\definecolor{currentfill}{rgb}{0.172549,0.627451,0.172549}%
\pgfsetfillcolor{currentfill}%
\pgfsetlinewidth{0.481800pt}%
\definecolor{currentstroke}{rgb}{1.000000,1.000000,1.000000}%
\pgfsetstrokecolor{currentstroke}%
\pgfsetdash{}{0pt}%
\pgfpathmoveto{\pgfqpoint{6.780747in}{2.378945in}}%
\pgfpathcurveto{\pgfqpoint{6.791798in}{2.378945in}}{\pgfqpoint{6.802397in}{2.383335in}}{\pgfqpoint{6.810210in}{2.391149in}}%
\pgfpathcurveto{\pgfqpoint{6.818024in}{2.398962in}}{\pgfqpoint{6.822414in}{2.409561in}}{\pgfqpoint{6.822414in}{2.420611in}}%
\pgfpathcurveto{\pgfqpoint{6.822414in}{2.431662in}}{\pgfqpoint{6.818024in}{2.442261in}}{\pgfqpoint{6.810210in}{2.450074in}}%
\pgfpathcurveto{\pgfqpoint{6.802397in}{2.457888in}}{\pgfqpoint{6.791798in}{2.462278in}}{\pgfqpoint{6.780747in}{2.462278in}}%
\pgfpathcurveto{\pgfqpoint{6.769697in}{2.462278in}}{\pgfqpoint{6.759098in}{2.457888in}}{\pgfqpoint{6.751285in}{2.450074in}}%
\pgfpathcurveto{\pgfqpoint{6.743471in}{2.442261in}}{\pgfqpoint{6.739081in}{2.431662in}}{\pgfqpoint{6.739081in}{2.420611in}}%
\pgfpathcurveto{\pgfqpoint{6.739081in}{2.409561in}}{\pgfqpoint{6.743471in}{2.398962in}}{\pgfqpoint{6.751285in}{2.391149in}}%
\pgfpathcurveto{\pgfqpoint{6.759098in}{2.383335in}}{\pgfqpoint{6.769697in}{2.378945in}}{\pgfqpoint{6.780747in}{2.378945in}}%
\pgfpathlineto{\pgfqpoint{6.780747in}{2.378945in}}%
\pgfpathclose%
\pgfusepath{stroke,fill}%
\end{pgfscope}%
\begin{pgfscope}%
\pgfpathrectangle{\pgfqpoint{5.292946in}{0.569136in}}{\pgfqpoint{2.177280in}{2.201755in}}%
\pgfusepath{clip}%
\pgfsetbuttcap%
\pgfsetroundjoin%
\definecolor{currentfill}{rgb}{0.172549,0.627451,0.172549}%
\pgfsetfillcolor{currentfill}%
\pgfsetlinewidth{0.481800pt}%
\definecolor{currentstroke}{rgb}{1.000000,1.000000,1.000000}%
\pgfsetstrokecolor{currentstroke}%
\pgfsetdash{}{0pt}%
\pgfpathmoveto{\pgfqpoint{6.637221in}{1.795146in}}%
\pgfpathcurveto{\pgfqpoint{6.648271in}{1.795146in}}{\pgfqpoint{6.658870in}{1.799536in}}{\pgfqpoint{6.666684in}{1.807350in}}%
\pgfpathcurveto{\pgfqpoint{6.674497in}{1.815164in}}{\pgfqpoint{6.678888in}{1.825763in}}{\pgfqpoint{6.678888in}{1.836813in}}%
\pgfpathcurveto{\pgfqpoint{6.678888in}{1.847863in}}{\pgfqpoint{6.674497in}{1.858462in}}{\pgfqpoint{6.666684in}{1.866276in}}%
\pgfpathcurveto{\pgfqpoint{6.658870in}{1.874089in}}{\pgfqpoint{6.648271in}{1.878479in}}{\pgfqpoint{6.637221in}{1.878479in}}%
\pgfpathcurveto{\pgfqpoint{6.626171in}{1.878479in}}{\pgfqpoint{6.615572in}{1.874089in}}{\pgfqpoint{6.607758in}{1.866276in}}%
\pgfpathcurveto{\pgfqpoint{6.599945in}{1.858462in}}{\pgfqpoint{6.595554in}{1.847863in}}{\pgfqpoint{6.595554in}{1.836813in}}%
\pgfpathcurveto{\pgfqpoint{6.595554in}{1.825763in}}{\pgfqpoint{6.599945in}{1.815164in}}{\pgfqpoint{6.607758in}{1.807350in}}%
\pgfpathcurveto{\pgfqpoint{6.615572in}{1.799536in}}{\pgfqpoint{6.626171in}{1.795146in}}{\pgfqpoint{6.637221in}{1.795146in}}%
\pgfpathlineto{\pgfqpoint{6.637221in}{1.795146in}}%
\pgfpathclose%
\pgfusepath{stroke,fill}%
\end{pgfscope}%
\begin{pgfscope}%
\pgfpathrectangle{\pgfqpoint{5.292946in}{0.569136in}}{\pgfqpoint{2.177280in}{2.201755in}}%
\pgfusepath{clip}%
\pgfsetbuttcap%
\pgfsetroundjoin%
\definecolor{currentfill}{rgb}{0.172549,0.627451,0.172549}%
\pgfsetfillcolor{currentfill}%
\pgfsetlinewidth{0.481800pt}%
\definecolor{currentstroke}{rgb}{1.000000,1.000000,1.000000}%
\pgfsetstrokecolor{currentstroke}%
\pgfsetdash{}{0pt}%
\pgfpathmoveto{\pgfqpoint{6.780747in}{1.711746in}}%
\pgfpathcurveto{\pgfqpoint{6.791798in}{1.711746in}}{\pgfqpoint{6.802397in}{1.716137in}}{\pgfqpoint{6.810210in}{1.723950in}}%
\pgfpathcurveto{\pgfqpoint{6.818024in}{1.731764in}}{\pgfqpoint{6.822414in}{1.742363in}}{\pgfqpoint{6.822414in}{1.753413in}}%
\pgfpathcurveto{\pgfqpoint{6.822414in}{1.764463in}}{\pgfqpoint{6.818024in}{1.775062in}}{\pgfqpoint{6.810210in}{1.782876in}}%
\pgfpathcurveto{\pgfqpoint{6.802397in}{1.790689in}}{\pgfqpoint{6.791798in}{1.795080in}}{\pgfqpoint{6.780747in}{1.795080in}}%
\pgfpathcurveto{\pgfqpoint{6.769697in}{1.795080in}}{\pgfqpoint{6.759098in}{1.790689in}}{\pgfqpoint{6.751285in}{1.782876in}}%
\pgfpathcurveto{\pgfqpoint{6.743471in}{1.775062in}}{\pgfqpoint{6.739081in}{1.764463in}}{\pgfqpoint{6.739081in}{1.753413in}}%
\pgfpathcurveto{\pgfqpoint{6.739081in}{1.742363in}}{\pgfqpoint{6.743471in}{1.731764in}}{\pgfqpoint{6.751285in}{1.723950in}}%
\pgfpathcurveto{\pgfqpoint{6.759098in}{1.716137in}}{\pgfqpoint{6.769697in}{1.711746in}}{\pgfqpoint{6.780747in}{1.711746in}}%
\pgfpathlineto{\pgfqpoint{6.780747in}{1.711746in}}%
\pgfpathclose%
\pgfusepath{stroke,fill}%
\end{pgfscope}%
\begin{pgfscope}%
\pgfpathrectangle{\pgfqpoint{5.292946in}{0.569136in}}{\pgfqpoint{2.177280in}{2.201755in}}%
\pgfusepath{clip}%
\pgfsetbuttcap%
\pgfsetroundjoin%
\definecolor{currentfill}{rgb}{0.172549,0.627451,0.172549}%
\pgfsetfillcolor{currentfill}%
\pgfsetlinewidth{0.481800pt}%
\definecolor{currentstroke}{rgb}{1.000000,1.000000,1.000000}%
\pgfsetstrokecolor{currentstroke}%
\pgfsetdash{}{0pt}%
\pgfpathmoveto{\pgfqpoint{6.924274in}{2.462345in}}%
\pgfpathcurveto{\pgfqpoint{6.935324in}{2.462345in}}{\pgfqpoint{6.945923in}{2.466735in}}{\pgfqpoint{6.953737in}{2.474548in}}%
\pgfpathcurveto{\pgfqpoint{6.961550in}{2.482362in}}{\pgfqpoint{6.965941in}{2.492961in}}{\pgfqpoint{6.965941in}{2.504011in}}%
\pgfpathcurveto{\pgfqpoint{6.965941in}{2.515061in}}{\pgfqpoint{6.961550in}{2.525660in}}{\pgfqpoint{6.953737in}{2.533474in}}%
\pgfpathcurveto{\pgfqpoint{6.945923in}{2.541288in}}{\pgfqpoint{6.935324in}{2.545678in}}{\pgfqpoint{6.924274in}{2.545678in}}%
\pgfpathcurveto{\pgfqpoint{6.913224in}{2.545678in}}{\pgfqpoint{6.902625in}{2.541288in}}{\pgfqpoint{6.894811in}{2.533474in}}%
\pgfpathcurveto{\pgfqpoint{6.886997in}{2.525660in}}{\pgfqpoint{6.882607in}{2.515061in}}{\pgfqpoint{6.882607in}{2.504011in}}%
\pgfpathcurveto{\pgfqpoint{6.882607in}{2.492961in}}{\pgfqpoint{6.886997in}{2.482362in}}{\pgfqpoint{6.894811in}{2.474548in}}%
\pgfpathcurveto{\pgfqpoint{6.902625in}{2.466735in}}{\pgfqpoint{6.913224in}{2.462345in}}{\pgfqpoint{6.924274in}{2.462345in}}%
\pgfpathlineto{\pgfqpoint{6.924274in}{2.462345in}}%
\pgfpathclose%
\pgfusepath{stroke,fill}%
\end{pgfscope}%
\begin{pgfscope}%
\pgfpathrectangle{\pgfqpoint{5.292946in}{0.569136in}}{\pgfqpoint{2.177280in}{2.201755in}}%
\pgfusepath{clip}%
\pgfsetbuttcap%
\pgfsetroundjoin%
\definecolor{currentfill}{rgb}{0.172549,0.627451,0.172549}%
\pgfsetfillcolor{currentfill}%
\pgfsetlinewidth{0.481800pt}%
\definecolor{currentstroke}{rgb}{1.000000,1.000000,1.000000}%
\pgfsetstrokecolor{currentstroke}%
\pgfsetdash{}{0pt}%
\pgfpathmoveto{\pgfqpoint{6.780747in}{2.545744in}}%
\pgfpathcurveto{\pgfqpoint{6.791798in}{2.545744in}}{\pgfqpoint{6.802397in}{2.550135in}}{\pgfqpoint{6.810210in}{2.557948in}}%
\pgfpathcurveto{\pgfqpoint{6.818024in}{2.565762in}}{\pgfqpoint{6.822414in}{2.576361in}}{\pgfqpoint{6.822414in}{2.587411in}}%
\pgfpathcurveto{\pgfqpoint{6.822414in}{2.598461in}}{\pgfqpoint{6.818024in}{2.609060in}}{\pgfqpoint{6.810210in}{2.616874in}}%
\pgfpathcurveto{\pgfqpoint{6.802397in}{2.624687in}}{\pgfqpoint{6.791798in}{2.629078in}}{\pgfqpoint{6.780747in}{2.629078in}}%
\pgfpathcurveto{\pgfqpoint{6.769697in}{2.629078in}}{\pgfqpoint{6.759098in}{2.624687in}}{\pgfqpoint{6.751285in}{2.616874in}}%
\pgfpathcurveto{\pgfqpoint{6.743471in}{2.609060in}}{\pgfqpoint{6.739081in}{2.598461in}}{\pgfqpoint{6.739081in}{2.587411in}}%
\pgfpathcurveto{\pgfqpoint{6.739081in}{2.576361in}}{\pgfqpoint{6.743471in}{2.565762in}}{\pgfqpoint{6.751285in}{2.557948in}}%
\pgfpathcurveto{\pgfqpoint{6.759098in}{2.550135in}}{\pgfqpoint{6.769697in}{2.545744in}}{\pgfqpoint{6.780747in}{2.545744in}}%
\pgfpathlineto{\pgfqpoint{6.780747in}{2.545744in}}%
\pgfpathclose%
\pgfusepath{stroke,fill}%
\end{pgfscope}%
\begin{pgfscope}%
\pgfpathrectangle{\pgfqpoint{5.292946in}{0.569136in}}{\pgfqpoint{2.177280in}{2.201755in}}%
\pgfusepath{clip}%
\pgfsetbuttcap%
\pgfsetroundjoin%
\definecolor{currentfill}{rgb}{0.172549,0.627451,0.172549}%
\pgfsetfillcolor{currentfill}%
\pgfsetlinewidth{0.481800pt}%
\definecolor{currentstroke}{rgb}{1.000000,1.000000,1.000000}%
\pgfsetstrokecolor{currentstroke}%
\pgfsetdash{}{0pt}%
\pgfpathmoveto{\pgfqpoint{6.752042in}{2.045346in}}%
\pgfpathcurveto{\pgfqpoint{6.763092in}{2.045346in}}{\pgfqpoint{6.773691in}{2.049736in}}{\pgfqpoint{6.781505in}{2.057549in}}%
\pgfpathcurveto{\pgfqpoint{6.789319in}{2.065363in}}{\pgfqpoint{6.793709in}{2.075962in}}{\pgfqpoint{6.793709in}{2.087012in}}%
\pgfpathcurveto{\pgfqpoint{6.793709in}{2.098062in}}{\pgfqpoint{6.789319in}{2.108661in}}{\pgfqpoint{6.781505in}{2.116475in}}%
\pgfpathcurveto{\pgfqpoint{6.773691in}{2.124289in}}{\pgfqpoint{6.763092in}{2.128679in}}{\pgfqpoint{6.752042in}{2.128679in}}%
\pgfpathcurveto{\pgfqpoint{6.740992in}{2.128679in}}{\pgfqpoint{6.730393in}{2.124289in}}{\pgfqpoint{6.722579in}{2.116475in}}%
\pgfpathcurveto{\pgfqpoint{6.714766in}{2.108661in}}{\pgfqpoint{6.710375in}{2.098062in}}{\pgfqpoint{6.710375in}{2.087012in}}%
\pgfpathcurveto{\pgfqpoint{6.710375in}{2.075962in}}{\pgfqpoint{6.714766in}{2.065363in}}{\pgfqpoint{6.722579in}{2.057549in}}%
\pgfpathcurveto{\pgfqpoint{6.730393in}{2.049736in}}{\pgfqpoint{6.740992in}{2.045346in}}{\pgfqpoint{6.752042in}{2.045346in}}%
\pgfpathlineto{\pgfqpoint{6.752042in}{2.045346in}}%
\pgfpathclose%
\pgfusepath{stroke,fill}%
\end{pgfscope}%
\begin{pgfscope}%
\pgfpathrectangle{\pgfqpoint{5.292946in}{0.569136in}}{\pgfqpoint{2.177280in}{2.201755in}}%
\pgfusepath{clip}%
\pgfsetbuttcap%
\pgfsetroundjoin%
\definecolor{currentfill}{rgb}{0.172549,0.627451,0.172549}%
\pgfsetfillcolor{currentfill}%
\pgfsetlinewidth{0.481800pt}%
\definecolor{currentstroke}{rgb}{1.000000,1.000000,1.000000}%
\pgfsetstrokecolor{currentstroke}%
\pgfsetdash{}{0pt}%
\pgfpathmoveto{\pgfqpoint{6.551105in}{2.045346in}}%
\pgfpathcurveto{\pgfqpoint{6.562155in}{2.045346in}}{\pgfqpoint{6.572754in}{2.049736in}}{\pgfqpoint{6.580568in}{2.057549in}}%
\pgfpathcurveto{\pgfqpoint{6.588382in}{2.065363in}}{\pgfqpoint{6.592772in}{2.075962in}}{\pgfqpoint{6.592772in}{2.087012in}}%
\pgfpathcurveto{\pgfqpoint{6.592772in}{2.098062in}}{\pgfqpoint{6.588382in}{2.108661in}}{\pgfqpoint{6.580568in}{2.116475in}}%
\pgfpathcurveto{\pgfqpoint{6.572754in}{2.124289in}}{\pgfqpoint{6.562155in}{2.128679in}}{\pgfqpoint{6.551105in}{2.128679in}}%
\pgfpathcurveto{\pgfqpoint{6.540055in}{2.128679in}}{\pgfqpoint{6.529456in}{2.124289in}}{\pgfqpoint{6.521642in}{2.116475in}}%
\pgfpathcurveto{\pgfqpoint{6.513829in}{2.108661in}}{\pgfqpoint{6.509438in}{2.098062in}}{\pgfqpoint{6.509438in}{2.087012in}}%
\pgfpathcurveto{\pgfqpoint{6.509438in}{2.075962in}}{\pgfqpoint{6.513829in}{2.065363in}}{\pgfqpoint{6.521642in}{2.057549in}}%
\pgfpathcurveto{\pgfqpoint{6.529456in}{2.049736in}}{\pgfqpoint{6.540055in}{2.045346in}}{\pgfqpoint{6.551105in}{2.045346in}}%
\pgfpathlineto{\pgfqpoint{6.551105in}{2.045346in}}%
\pgfpathclose%
\pgfusepath{stroke,fill}%
\end{pgfscope}%
\begin{pgfscope}%
\pgfpathrectangle{\pgfqpoint{5.292946in}{0.569136in}}{\pgfqpoint{2.177280in}{2.201755in}}%
\pgfusepath{clip}%
\pgfsetbuttcap%
\pgfsetroundjoin%
\definecolor{currentfill}{rgb}{0.172549,0.627451,0.172549}%
\pgfsetfillcolor{currentfill}%
\pgfsetlinewidth{0.481800pt}%
\definecolor{currentstroke}{rgb}{1.000000,1.000000,1.000000}%
\pgfsetstrokecolor{currentstroke}%
\pgfsetdash{}{0pt}%
\pgfpathmoveto{\pgfqpoint{6.723337in}{2.295545in}}%
\pgfpathcurveto{\pgfqpoint{6.734387in}{2.295545in}}{\pgfqpoint{6.744986in}{2.299935in}}{\pgfqpoint{6.752800in}{2.307749in}}%
\pgfpathcurveto{\pgfqpoint{6.760613in}{2.315562in}}{\pgfqpoint{6.765004in}{2.326161in}}{\pgfqpoint{6.765004in}{2.337212in}}%
\pgfpathcurveto{\pgfqpoint{6.765004in}{2.348262in}}{\pgfqpoint{6.760613in}{2.358861in}}{\pgfqpoint{6.752800in}{2.366674in}}%
\pgfpathcurveto{\pgfqpoint{6.744986in}{2.374488in}}{\pgfqpoint{6.734387in}{2.378878in}}{\pgfqpoint{6.723337in}{2.378878in}}%
\pgfpathcurveto{\pgfqpoint{6.712287in}{2.378878in}}{\pgfqpoint{6.701688in}{2.374488in}}{\pgfqpoint{6.693874in}{2.366674in}}%
\pgfpathcurveto{\pgfqpoint{6.686060in}{2.358861in}}{\pgfqpoint{6.681670in}{2.348262in}}{\pgfqpoint{6.681670in}{2.337212in}}%
\pgfpathcurveto{\pgfqpoint{6.681670in}{2.326161in}}{\pgfqpoint{6.686060in}{2.315562in}}{\pgfqpoint{6.693874in}{2.307749in}}%
\pgfpathcurveto{\pgfqpoint{6.701688in}{2.299935in}}{\pgfqpoint{6.712287in}{2.295545in}}{\pgfqpoint{6.723337in}{2.295545in}}%
\pgfpathlineto{\pgfqpoint{6.723337in}{2.295545in}}%
\pgfpathclose%
\pgfusepath{stroke,fill}%
\end{pgfscope}%
\begin{pgfscope}%
\pgfpathrectangle{\pgfqpoint{5.292946in}{0.569136in}}{\pgfqpoint{2.177280in}{2.201755in}}%
\pgfusepath{clip}%
\pgfsetbuttcap%
\pgfsetroundjoin%
\definecolor{currentfill}{rgb}{0.172549,0.627451,0.172549}%
\pgfsetfillcolor{currentfill}%
\pgfsetlinewidth{0.481800pt}%
\definecolor{currentstroke}{rgb}{1.000000,1.000000,1.000000}%
\pgfsetstrokecolor{currentstroke}%
\pgfsetdash{}{0pt}%
\pgfpathmoveto{\pgfqpoint{6.780747in}{2.545744in}}%
\pgfpathcurveto{\pgfqpoint{6.791798in}{2.545744in}}{\pgfqpoint{6.802397in}{2.550135in}}{\pgfqpoint{6.810210in}{2.557948in}}%
\pgfpathcurveto{\pgfqpoint{6.818024in}{2.565762in}}{\pgfqpoint{6.822414in}{2.576361in}}{\pgfqpoint{6.822414in}{2.587411in}}%
\pgfpathcurveto{\pgfqpoint{6.822414in}{2.598461in}}{\pgfqpoint{6.818024in}{2.609060in}}{\pgfqpoint{6.810210in}{2.616874in}}%
\pgfpathcurveto{\pgfqpoint{6.802397in}{2.624687in}}{\pgfqpoint{6.791798in}{2.629078in}}{\pgfqpoint{6.780747in}{2.629078in}}%
\pgfpathcurveto{\pgfqpoint{6.769697in}{2.629078in}}{\pgfqpoint{6.759098in}{2.624687in}}{\pgfqpoint{6.751285in}{2.616874in}}%
\pgfpathcurveto{\pgfqpoint{6.743471in}{2.609060in}}{\pgfqpoint{6.739081in}{2.598461in}}{\pgfqpoint{6.739081in}{2.587411in}}%
\pgfpathcurveto{\pgfqpoint{6.739081in}{2.576361in}}{\pgfqpoint{6.743471in}{2.565762in}}{\pgfqpoint{6.751285in}{2.557948in}}%
\pgfpathcurveto{\pgfqpoint{6.759098in}{2.550135in}}{\pgfqpoint{6.769697in}{2.545744in}}{\pgfqpoint{6.780747in}{2.545744in}}%
\pgfpathlineto{\pgfqpoint{6.780747in}{2.545744in}}%
\pgfpathclose%
\pgfusepath{stroke,fill}%
\end{pgfscope}%
\begin{pgfscope}%
\pgfpathrectangle{\pgfqpoint{5.292946in}{0.569136in}}{\pgfqpoint{2.177280in}{2.201755in}}%
\pgfusepath{clip}%
\pgfsetbuttcap%
\pgfsetroundjoin%
\definecolor{currentfill}{rgb}{0.172549,0.627451,0.172549}%
\pgfsetfillcolor{currentfill}%
\pgfsetlinewidth{0.481800pt}%
\definecolor{currentstroke}{rgb}{1.000000,1.000000,1.000000}%
\pgfsetstrokecolor{currentstroke}%
\pgfsetdash{}{0pt}%
\pgfpathmoveto{\pgfqpoint{6.637221in}{2.462345in}}%
\pgfpathcurveto{\pgfqpoint{6.648271in}{2.462345in}}{\pgfqpoint{6.658870in}{2.466735in}}{\pgfqpoint{6.666684in}{2.474548in}}%
\pgfpathcurveto{\pgfqpoint{6.674497in}{2.482362in}}{\pgfqpoint{6.678888in}{2.492961in}}{\pgfqpoint{6.678888in}{2.504011in}}%
\pgfpathcurveto{\pgfqpoint{6.678888in}{2.515061in}}{\pgfqpoint{6.674497in}{2.525660in}}{\pgfqpoint{6.666684in}{2.533474in}}%
\pgfpathcurveto{\pgfqpoint{6.658870in}{2.541288in}}{\pgfqpoint{6.648271in}{2.545678in}}{\pgfqpoint{6.637221in}{2.545678in}}%
\pgfpathcurveto{\pgfqpoint{6.626171in}{2.545678in}}{\pgfqpoint{6.615572in}{2.541288in}}{\pgfqpoint{6.607758in}{2.533474in}}%
\pgfpathcurveto{\pgfqpoint{6.599945in}{2.525660in}}{\pgfqpoint{6.595554in}{2.515061in}}{\pgfqpoint{6.595554in}{2.504011in}}%
\pgfpathcurveto{\pgfqpoint{6.595554in}{2.492961in}}{\pgfqpoint{6.599945in}{2.482362in}}{\pgfqpoint{6.607758in}{2.474548in}}%
\pgfpathcurveto{\pgfqpoint{6.615572in}{2.466735in}}{\pgfqpoint{6.626171in}{2.462345in}}{\pgfqpoint{6.637221in}{2.462345in}}%
\pgfpathlineto{\pgfqpoint{6.637221in}{2.462345in}}%
\pgfpathclose%
\pgfusepath{stroke,fill}%
\end{pgfscope}%
\begin{pgfscope}%
\pgfpathrectangle{\pgfqpoint{5.292946in}{0.569136in}}{\pgfqpoint{2.177280in}{2.201755in}}%
\pgfusepath{clip}%
\pgfsetbuttcap%
\pgfsetroundjoin%
\definecolor{currentfill}{rgb}{0.172549,0.627451,0.172549}%
\pgfsetfillcolor{currentfill}%
\pgfsetlinewidth{0.481800pt}%
\definecolor{currentstroke}{rgb}{1.000000,1.000000,1.000000}%
\pgfsetstrokecolor{currentstroke}%
\pgfsetdash{}{0pt}%
\pgfpathmoveto{\pgfqpoint{6.637221in}{2.128745in}}%
\pgfpathcurveto{\pgfqpoint{6.648271in}{2.128745in}}{\pgfqpoint{6.658870in}{2.133136in}}{\pgfqpoint{6.666684in}{2.140949in}}%
\pgfpathcurveto{\pgfqpoint{6.674497in}{2.148763in}}{\pgfqpoint{6.678888in}{2.159362in}}{\pgfqpoint{6.678888in}{2.170412in}}%
\pgfpathcurveto{\pgfqpoint{6.678888in}{2.181462in}}{\pgfqpoint{6.674497in}{2.192061in}}{\pgfqpoint{6.666684in}{2.199875in}}%
\pgfpathcurveto{\pgfqpoint{6.658870in}{2.207688in}}{\pgfqpoint{6.648271in}{2.212079in}}{\pgfqpoint{6.637221in}{2.212079in}}%
\pgfpathcurveto{\pgfqpoint{6.626171in}{2.212079in}}{\pgfqpoint{6.615572in}{2.207688in}}{\pgfqpoint{6.607758in}{2.199875in}}%
\pgfpathcurveto{\pgfqpoint{6.599945in}{2.192061in}}{\pgfqpoint{6.595554in}{2.181462in}}{\pgfqpoint{6.595554in}{2.170412in}}%
\pgfpathcurveto{\pgfqpoint{6.595554in}{2.159362in}}{\pgfqpoint{6.599945in}{2.148763in}}{\pgfqpoint{6.607758in}{2.140949in}}%
\pgfpathcurveto{\pgfqpoint{6.615572in}{2.133136in}}{\pgfqpoint{6.626171in}{2.128745in}}{\pgfqpoint{6.637221in}{2.128745in}}%
\pgfpathlineto{\pgfqpoint{6.637221in}{2.128745in}}%
\pgfpathclose%
\pgfusepath{stroke,fill}%
\end{pgfscope}%
\begin{pgfscope}%
\pgfpathrectangle{\pgfqpoint{5.292946in}{0.569136in}}{\pgfqpoint{2.177280in}{2.201755in}}%
\pgfusepath{clip}%
\pgfsetbuttcap%
\pgfsetroundjoin%
\definecolor{currentfill}{rgb}{0.172549,0.627451,0.172549}%
\pgfsetfillcolor{currentfill}%
\pgfsetlinewidth{0.481800pt}%
\definecolor{currentstroke}{rgb}{1.000000,1.000000,1.000000}%
\pgfsetstrokecolor{currentstroke}%
\pgfsetdash{}{0pt}%
\pgfpathmoveto{\pgfqpoint{6.866863in}{2.462345in}}%
\pgfpathcurveto{\pgfqpoint{6.877913in}{2.462345in}}{\pgfqpoint{6.888512in}{2.466735in}}{\pgfqpoint{6.896326in}{2.474548in}}%
\pgfpathcurveto{\pgfqpoint{6.904140in}{2.482362in}}{\pgfqpoint{6.908530in}{2.492961in}}{\pgfqpoint{6.908530in}{2.504011in}}%
\pgfpathcurveto{\pgfqpoint{6.908530in}{2.515061in}}{\pgfqpoint{6.904140in}{2.525660in}}{\pgfqpoint{6.896326in}{2.533474in}}%
\pgfpathcurveto{\pgfqpoint{6.888512in}{2.541288in}}{\pgfqpoint{6.877913in}{2.545678in}}{\pgfqpoint{6.866863in}{2.545678in}}%
\pgfpathcurveto{\pgfqpoint{6.855813in}{2.545678in}}{\pgfqpoint{6.845214in}{2.541288in}}{\pgfqpoint{6.837401in}{2.533474in}}%
\pgfpathcurveto{\pgfqpoint{6.829587in}{2.525660in}}{\pgfqpoint{6.825197in}{2.515061in}}{\pgfqpoint{6.825197in}{2.504011in}}%
\pgfpathcurveto{\pgfqpoint{6.825197in}{2.492961in}}{\pgfqpoint{6.829587in}{2.482362in}}{\pgfqpoint{6.837401in}{2.474548in}}%
\pgfpathcurveto{\pgfqpoint{6.845214in}{2.466735in}}{\pgfqpoint{6.855813in}{2.462345in}}{\pgfqpoint{6.866863in}{2.462345in}}%
\pgfpathlineto{\pgfqpoint{6.866863in}{2.462345in}}%
\pgfpathclose%
\pgfusepath{stroke,fill}%
\end{pgfscope}%
\begin{pgfscope}%
\pgfpathrectangle{\pgfqpoint{5.292946in}{0.569136in}}{\pgfqpoint{2.177280in}{2.201755in}}%
\pgfusepath{clip}%
\pgfsetbuttcap%
\pgfsetroundjoin%
\definecolor{currentfill}{rgb}{0.172549,0.627451,0.172549}%
\pgfsetfillcolor{currentfill}%
\pgfsetlinewidth{0.481800pt}%
\definecolor{currentstroke}{rgb}{1.000000,1.000000,1.000000}%
\pgfsetstrokecolor{currentstroke}%
\pgfsetdash{}{0pt}%
\pgfpathmoveto{\pgfqpoint{6.809453in}{2.629144in}}%
\pgfpathcurveto{\pgfqpoint{6.820503in}{2.629144in}}{\pgfqpoint{6.831102in}{2.633534in}}{\pgfqpoint{6.838915in}{2.641348in}}%
\pgfpathcurveto{\pgfqpoint{6.846729in}{2.649162in}}{\pgfqpoint{6.851119in}{2.659761in}}{\pgfqpoint{6.851119in}{2.670811in}}%
\pgfpathcurveto{\pgfqpoint{6.851119in}{2.681861in}}{\pgfqpoint{6.846729in}{2.692460in}}{\pgfqpoint{6.838915in}{2.700274in}}%
\pgfpathcurveto{\pgfqpoint{6.831102in}{2.708087in}}{\pgfqpoint{6.820503in}{2.712478in}}{\pgfqpoint{6.809453in}{2.712478in}}%
\pgfpathcurveto{\pgfqpoint{6.798403in}{2.712478in}}{\pgfqpoint{6.787804in}{2.708087in}}{\pgfqpoint{6.779990in}{2.700274in}}%
\pgfpathcurveto{\pgfqpoint{6.772176in}{2.692460in}}{\pgfqpoint{6.767786in}{2.681861in}}{\pgfqpoint{6.767786in}{2.670811in}}%
\pgfpathcurveto{\pgfqpoint{6.767786in}{2.659761in}}{\pgfqpoint{6.772176in}{2.649162in}}{\pgfqpoint{6.779990in}{2.641348in}}%
\pgfpathcurveto{\pgfqpoint{6.787804in}{2.633534in}}{\pgfqpoint{6.798403in}{2.629144in}}{\pgfqpoint{6.809453in}{2.629144in}}%
\pgfpathlineto{\pgfqpoint{6.809453in}{2.629144in}}%
\pgfpathclose%
\pgfusepath{stroke,fill}%
\end{pgfscope}%
\begin{pgfscope}%
\pgfpathrectangle{\pgfqpoint{5.292946in}{0.569136in}}{\pgfqpoint{2.177280in}{2.201755in}}%
\pgfusepath{clip}%
\pgfsetbuttcap%
\pgfsetroundjoin%
\definecolor{currentfill}{rgb}{0.172549,0.627451,0.172549}%
\pgfsetfillcolor{currentfill}%
\pgfsetlinewidth{0.481800pt}%
\definecolor{currentstroke}{rgb}{1.000000,1.000000,1.000000}%
\pgfsetstrokecolor{currentstroke}%
\pgfsetdash{}{0pt}%
\pgfpathmoveto{\pgfqpoint{6.665926in}{2.462345in}}%
\pgfpathcurveto{\pgfqpoint{6.676976in}{2.462345in}}{\pgfqpoint{6.687575in}{2.466735in}}{\pgfqpoint{6.695389in}{2.474548in}}%
\pgfpathcurveto{\pgfqpoint{6.703203in}{2.482362in}}{\pgfqpoint{6.707593in}{2.492961in}}{\pgfqpoint{6.707593in}{2.504011in}}%
\pgfpathcurveto{\pgfqpoint{6.707593in}{2.515061in}}{\pgfqpoint{6.703203in}{2.525660in}}{\pgfqpoint{6.695389in}{2.533474in}}%
\pgfpathcurveto{\pgfqpoint{6.687575in}{2.541288in}}{\pgfqpoint{6.676976in}{2.545678in}}{\pgfqpoint{6.665926in}{2.545678in}}%
\pgfpathcurveto{\pgfqpoint{6.654876in}{2.545678in}}{\pgfqpoint{6.644277in}{2.541288in}}{\pgfqpoint{6.636464in}{2.533474in}}%
\pgfpathcurveto{\pgfqpoint{6.628650in}{2.525660in}}{\pgfqpoint{6.624260in}{2.515061in}}{\pgfqpoint{6.624260in}{2.504011in}}%
\pgfpathcurveto{\pgfqpoint{6.624260in}{2.492961in}}{\pgfqpoint{6.628650in}{2.482362in}}{\pgfqpoint{6.636464in}{2.474548in}}%
\pgfpathcurveto{\pgfqpoint{6.644277in}{2.466735in}}{\pgfqpoint{6.654876in}{2.462345in}}{\pgfqpoint{6.665926in}{2.462345in}}%
\pgfpathlineto{\pgfqpoint{6.665926in}{2.462345in}}%
\pgfpathclose%
\pgfusepath{stroke,fill}%
\end{pgfscope}%
\begin{pgfscope}%
\pgfpathrectangle{\pgfqpoint{5.292946in}{0.569136in}}{\pgfqpoint{2.177280in}{2.201755in}}%
\pgfusepath{clip}%
\pgfsetbuttcap%
\pgfsetroundjoin%
\definecolor{currentfill}{rgb}{0.172549,0.627451,0.172549}%
\pgfsetfillcolor{currentfill}%
\pgfsetlinewidth{0.481800pt}%
\definecolor{currentstroke}{rgb}{1.000000,1.000000,1.000000}%
\pgfsetstrokecolor{currentstroke}%
\pgfsetdash{}{0pt}%
\pgfpathmoveto{\pgfqpoint{6.608516in}{2.128745in}}%
\pgfpathcurveto{\pgfqpoint{6.619566in}{2.128745in}}{\pgfqpoint{6.630165in}{2.133136in}}{\pgfqpoint{6.637978in}{2.140949in}}%
\pgfpathcurveto{\pgfqpoint{6.645792in}{2.148763in}}{\pgfqpoint{6.650182in}{2.159362in}}{\pgfqpoint{6.650182in}{2.170412in}}%
\pgfpathcurveto{\pgfqpoint{6.650182in}{2.181462in}}{\pgfqpoint{6.645792in}{2.192061in}}{\pgfqpoint{6.637978in}{2.199875in}}%
\pgfpathcurveto{\pgfqpoint{6.630165in}{2.207688in}}{\pgfqpoint{6.619566in}{2.212079in}}{\pgfqpoint{6.608516in}{2.212079in}}%
\pgfpathcurveto{\pgfqpoint{6.597466in}{2.212079in}}{\pgfqpoint{6.586867in}{2.207688in}}{\pgfqpoint{6.579053in}{2.199875in}}%
\pgfpathcurveto{\pgfqpoint{6.571239in}{2.192061in}}{\pgfqpoint{6.566849in}{2.181462in}}{\pgfqpoint{6.566849in}{2.170412in}}%
\pgfpathcurveto{\pgfqpoint{6.566849in}{2.159362in}}{\pgfqpoint{6.571239in}{2.148763in}}{\pgfqpoint{6.579053in}{2.140949in}}%
\pgfpathcurveto{\pgfqpoint{6.586867in}{2.133136in}}{\pgfqpoint{6.597466in}{2.128745in}}{\pgfqpoint{6.608516in}{2.128745in}}%
\pgfpathlineto{\pgfqpoint{6.608516in}{2.128745in}}%
\pgfpathclose%
\pgfusepath{stroke,fill}%
\end{pgfscope}%
\begin{pgfscope}%
\pgfpathrectangle{\pgfqpoint{5.292946in}{0.569136in}}{\pgfqpoint{2.177280in}{2.201755in}}%
\pgfusepath{clip}%
\pgfsetbuttcap%
\pgfsetroundjoin%
\definecolor{currentfill}{rgb}{0.172549,0.627451,0.172549}%
\pgfsetfillcolor{currentfill}%
\pgfsetlinewidth{0.481800pt}%
\definecolor{currentstroke}{rgb}{1.000000,1.000000,1.000000}%
\pgfsetstrokecolor{currentstroke}%
\pgfsetdash{}{0pt}%
\pgfpathmoveto{\pgfqpoint{6.665926in}{2.212145in}}%
\pgfpathcurveto{\pgfqpoint{6.676976in}{2.212145in}}{\pgfqpoint{6.687575in}{2.216535in}}{\pgfqpoint{6.695389in}{2.224349in}}%
\pgfpathcurveto{\pgfqpoint{6.703203in}{2.232163in}}{\pgfqpoint{6.707593in}{2.242762in}}{\pgfqpoint{6.707593in}{2.253812in}}%
\pgfpathcurveto{\pgfqpoint{6.707593in}{2.264862in}}{\pgfqpoint{6.703203in}{2.275461in}}{\pgfqpoint{6.695389in}{2.283275in}}%
\pgfpathcurveto{\pgfqpoint{6.687575in}{2.291088in}}{\pgfqpoint{6.676976in}{2.295478in}}{\pgfqpoint{6.665926in}{2.295478in}}%
\pgfpathcurveto{\pgfqpoint{6.654876in}{2.295478in}}{\pgfqpoint{6.644277in}{2.291088in}}{\pgfqpoint{6.636464in}{2.283275in}}%
\pgfpathcurveto{\pgfqpoint{6.628650in}{2.275461in}}{\pgfqpoint{6.624260in}{2.264862in}}{\pgfqpoint{6.624260in}{2.253812in}}%
\pgfpathcurveto{\pgfqpoint{6.624260in}{2.242762in}}{\pgfqpoint{6.628650in}{2.232163in}}{\pgfqpoint{6.636464in}{2.224349in}}%
\pgfpathcurveto{\pgfqpoint{6.644277in}{2.216535in}}{\pgfqpoint{6.654876in}{2.212145in}}{\pgfqpoint{6.665926in}{2.212145in}}%
\pgfpathlineto{\pgfqpoint{6.665926in}{2.212145in}}%
\pgfpathclose%
\pgfusepath{stroke,fill}%
\end{pgfscope}%
\begin{pgfscope}%
\pgfpathrectangle{\pgfqpoint{5.292946in}{0.569136in}}{\pgfqpoint{2.177280in}{2.201755in}}%
\pgfusepath{clip}%
\pgfsetbuttcap%
\pgfsetroundjoin%
\definecolor{currentfill}{rgb}{0.172549,0.627451,0.172549}%
\pgfsetfillcolor{currentfill}%
\pgfsetlinewidth{0.481800pt}%
\definecolor{currentstroke}{rgb}{1.000000,1.000000,1.000000}%
\pgfsetstrokecolor{currentstroke}%
\pgfsetdash{}{0pt}%
\pgfpathmoveto{\pgfqpoint{6.723337in}{2.462345in}}%
\pgfpathcurveto{\pgfqpoint{6.734387in}{2.462345in}}{\pgfqpoint{6.744986in}{2.466735in}}{\pgfqpoint{6.752800in}{2.474548in}}%
\pgfpathcurveto{\pgfqpoint{6.760613in}{2.482362in}}{\pgfqpoint{6.765004in}{2.492961in}}{\pgfqpoint{6.765004in}{2.504011in}}%
\pgfpathcurveto{\pgfqpoint{6.765004in}{2.515061in}}{\pgfqpoint{6.760613in}{2.525660in}}{\pgfqpoint{6.752800in}{2.533474in}}%
\pgfpathcurveto{\pgfqpoint{6.744986in}{2.541288in}}{\pgfqpoint{6.734387in}{2.545678in}}{\pgfqpoint{6.723337in}{2.545678in}}%
\pgfpathcurveto{\pgfqpoint{6.712287in}{2.545678in}}{\pgfqpoint{6.701688in}{2.541288in}}{\pgfqpoint{6.693874in}{2.533474in}}%
\pgfpathcurveto{\pgfqpoint{6.686060in}{2.525660in}}{\pgfqpoint{6.681670in}{2.515061in}}{\pgfqpoint{6.681670in}{2.504011in}}%
\pgfpathcurveto{\pgfqpoint{6.681670in}{2.492961in}}{\pgfqpoint{6.686060in}{2.482362in}}{\pgfqpoint{6.693874in}{2.474548in}}%
\pgfpathcurveto{\pgfqpoint{6.701688in}{2.466735in}}{\pgfqpoint{6.712287in}{2.462345in}}{\pgfqpoint{6.723337in}{2.462345in}}%
\pgfpathlineto{\pgfqpoint{6.723337in}{2.462345in}}%
\pgfpathclose%
\pgfusepath{stroke,fill}%
\end{pgfscope}%
\begin{pgfscope}%
\pgfpathrectangle{\pgfqpoint{5.292946in}{0.569136in}}{\pgfqpoint{2.177280in}{2.201755in}}%
\pgfusepath{clip}%
\pgfsetbuttcap%
\pgfsetroundjoin%
\definecolor{currentfill}{rgb}{0.172549,0.627451,0.172549}%
\pgfsetfillcolor{currentfill}%
\pgfsetlinewidth{0.481800pt}%
\definecolor{currentstroke}{rgb}{1.000000,1.000000,1.000000}%
\pgfsetstrokecolor{currentstroke}%
\pgfsetdash{}{0pt}%
\pgfpathmoveto{\pgfqpoint{6.637221in}{2.045346in}}%
\pgfpathcurveto{\pgfqpoint{6.648271in}{2.045346in}}{\pgfqpoint{6.658870in}{2.049736in}}{\pgfqpoint{6.666684in}{2.057549in}}%
\pgfpathcurveto{\pgfqpoint{6.674497in}{2.065363in}}{\pgfqpoint{6.678888in}{2.075962in}}{\pgfqpoint{6.678888in}{2.087012in}}%
\pgfpathcurveto{\pgfqpoint{6.678888in}{2.098062in}}{\pgfqpoint{6.674497in}{2.108661in}}{\pgfqpoint{6.666684in}{2.116475in}}%
\pgfpathcurveto{\pgfqpoint{6.658870in}{2.124289in}}{\pgfqpoint{6.648271in}{2.128679in}}{\pgfqpoint{6.637221in}{2.128679in}}%
\pgfpathcurveto{\pgfqpoint{6.626171in}{2.128679in}}{\pgfqpoint{6.615572in}{2.124289in}}{\pgfqpoint{6.607758in}{2.116475in}}%
\pgfpathcurveto{\pgfqpoint{6.599945in}{2.108661in}}{\pgfqpoint{6.595554in}{2.098062in}}{\pgfqpoint{6.595554in}{2.087012in}}%
\pgfpathcurveto{\pgfqpoint{6.595554in}{2.075962in}}{\pgfqpoint{6.599945in}{2.065363in}}{\pgfqpoint{6.607758in}{2.057549in}}%
\pgfpathcurveto{\pgfqpoint{6.615572in}{2.049736in}}{\pgfqpoint{6.626171in}{2.045346in}}{\pgfqpoint{6.637221in}{2.045346in}}%
\pgfpathlineto{\pgfqpoint{6.637221in}{2.045346in}}%
\pgfpathclose%
\pgfusepath{stroke,fill}%
\end{pgfscope}%
\begin{pgfscope}%
\pgfpathrectangle{\pgfqpoint{5.292946in}{0.569136in}}{\pgfqpoint{2.177280in}{2.201755in}}%
\pgfusepath{clip}%
\pgfsetbuttcap%
\pgfsetroundjoin%
\definecolor{currentfill}{rgb}{0.121569,0.466667,0.705882}%
\pgfsetfillcolor{currentfill}%
\pgfsetlinewidth{1.003750pt}%
\definecolor{currentstroke}{rgb}{0.121569,0.466667,0.705882}%
\pgfsetstrokecolor{currentstroke}%
\pgfsetdash{}{0pt}%
\pgfsys@defobject{currentmarker}{\pgfqpoint{-0.041667in}{-0.041667in}}{\pgfqpoint{0.041667in}{0.041667in}}{%
\pgfpathmoveto{\pgfqpoint{0.000000in}{-0.041667in}}%
\pgfpathcurveto{\pgfqpoint{0.011050in}{-0.041667in}}{\pgfqpoint{0.021649in}{-0.037276in}}{\pgfqpoint{0.029463in}{-0.029463in}}%
\pgfpathcurveto{\pgfqpoint{0.037276in}{-0.021649in}}{\pgfqpoint{0.041667in}{-0.011050in}}{\pgfqpoint{0.041667in}{0.000000in}}%
\pgfpathcurveto{\pgfqpoint{0.041667in}{0.011050in}}{\pgfqpoint{0.037276in}{0.021649in}}{\pgfqpoint{0.029463in}{0.029463in}}%
\pgfpathcurveto{\pgfqpoint{0.021649in}{0.037276in}}{\pgfqpoint{0.011050in}{0.041667in}}{\pgfqpoint{0.000000in}{0.041667in}}%
\pgfpathcurveto{\pgfqpoint{-0.011050in}{0.041667in}}{\pgfqpoint{-0.021649in}{0.037276in}}{\pgfqpoint{-0.029463in}{0.029463in}}%
\pgfpathcurveto{\pgfqpoint{-0.037276in}{0.021649in}}{\pgfqpoint{-0.041667in}{0.011050in}}{\pgfqpoint{-0.041667in}{0.000000in}}%
\pgfpathcurveto{\pgfqpoint{-0.041667in}{-0.011050in}}{\pgfqpoint{-0.037276in}{-0.021649in}}{\pgfqpoint{-0.029463in}{-0.029463in}}%
\pgfpathcurveto{\pgfqpoint{-0.021649in}{-0.037276in}}{\pgfqpoint{-0.011050in}{-0.041667in}}{\pgfqpoint{0.000000in}{-0.041667in}}%
\pgfpathlineto{\pgfqpoint{0.000000in}{-0.041667in}}%
\pgfpathclose%
\pgfusepath{stroke,fill}%
}%
\end{pgfscope}%
\begin{pgfscope}%
\pgfpathrectangle{\pgfqpoint{5.292946in}{0.569136in}}{\pgfqpoint{2.177280in}{2.201755in}}%
\pgfusepath{clip}%
\pgfsetbuttcap%
\pgfsetroundjoin%
\definecolor{currentfill}{rgb}{1.000000,0.498039,0.054902}%
\pgfsetfillcolor{currentfill}%
\pgfsetlinewidth{1.003750pt}%
\definecolor{currentstroke}{rgb}{1.000000,0.498039,0.054902}%
\pgfsetstrokecolor{currentstroke}%
\pgfsetdash{}{0pt}%
\pgfsys@defobject{currentmarker}{\pgfqpoint{-0.041667in}{-0.041667in}}{\pgfqpoint{0.041667in}{0.041667in}}{%
\pgfpathmoveto{\pgfqpoint{0.000000in}{-0.041667in}}%
\pgfpathcurveto{\pgfqpoint{0.011050in}{-0.041667in}}{\pgfqpoint{0.021649in}{-0.037276in}}{\pgfqpoint{0.029463in}{-0.029463in}}%
\pgfpathcurveto{\pgfqpoint{0.037276in}{-0.021649in}}{\pgfqpoint{0.041667in}{-0.011050in}}{\pgfqpoint{0.041667in}{0.000000in}}%
\pgfpathcurveto{\pgfqpoint{0.041667in}{0.011050in}}{\pgfqpoint{0.037276in}{0.021649in}}{\pgfqpoint{0.029463in}{0.029463in}}%
\pgfpathcurveto{\pgfqpoint{0.021649in}{0.037276in}}{\pgfqpoint{0.011050in}{0.041667in}}{\pgfqpoint{0.000000in}{0.041667in}}%
\pgfpathcurveto{\pgfqpoint{-0.011050in}{0.041667in}}{\pgfqpoint{-0.021649in}{0.037276in}}{\pgfqpoint{-0.029463in}{0.029463in}}%
\pgfpathcurveto{\pgfqpoint{-0.037276in}{0.021649in}}{\pgfqpoint{-0.041667in}{0.011050in}}{\pgfqpoint{-0.041667in}{0.000000in}}%
\pgfpathcurveto{\pgfqpoint{-0.041667in}{-0.011050in}}{\pgfqpoint{-0.037276in}{-0.021649in}}{\pgfqpoint{-0.029463in}{-0.029463in}}%
\pgfpathcurveto{\pgfqpoint{-0.021649in}{-0.037276in}}{\pgfqpoint{-0.011050in}{-0.041667in}}{\pgfqpoint{0.000000in}{-0.041667in}}%
\pgfpathlineto{\pgfqpoint{0.000000in}{-0.041667in}}%
\pgfpathclose%
\pgfusepath{stroke,fill}%
}%
\end{pgfscope}%
\begin{pgfscope}%
\pgfpathrectangle{\pgfqpoint{5.292946in}{0.569136in}}{\pgfqpoint{2.177280in}{2.201755in}}%
\pgfusepath{clip}%
\pgfsetbuttcap%
\pgfsetroundjoin%
\definecolor{currentfill}{rgb}{0.172549,0.627451,0.172549}%
\pgfsetfillcolor{currentfill}%
\pgfsetlinewidth{1.003750pt}%
\definecolor{currentstroke}{rgb}{0.172549,0.627451,0.172549}%
\pgfsetstrokecolor{currentstroke}%
\pgfsetdash{}{0pt}%
\pgfsys@defobject{currentmarker}{\pgfqpoint{-0.041667in}{-0.041667in}}{\pgfqpoint{0.041667in}{0.041667in}}{%
\pgfpathmoveto{\pgfqpoint{0.000000in}{-0.041667in}}%
\pgfpathcurveto{\pgfqpoint{0.011050in}{-0.041667in}}{\pgfqpoint{0.021649in}{-0.037276in}}{\pgfqpoint{0.029463in}{-0.029463in}}%
\pgfpathcurveto{\pgfqpoint{0.037276in}{-0.021649in}}{\pgfqpoint{0.041667in}{-0.011050in}}{\pgfqpoint{0.041667in}{0.000000in}}%
\pgfpathcurveto{\pgfqpoint{0.041667in}{0.011050in}}{\pgfqpoint{0.037276in}{0.021649in}}{\pgfqpoint{0.029463in}{0.029463in}}%
\pgfpathcurveto{\pgfqpoint{0.021649in}{0.037276in}}{\pgfqpoint{0.011050in}{0.041667in}}{\pgfqpoint{0.000000in}{0.041667in}}%
\pgfpathcurveto{\pgfqpoint{-0.011050in}{0.041667in}}{\pgfqpoint{-0.021649in}{0.037276in}}{\pgfqpoint{-0.029463in}{0.029463in}}%
\pgfpathcurveto{\pgfqpoint{-0.037276in}{0.021649in}}{\pgfqpoint{-0.041667in}{0.011050in}}{\pgfqpoint{-0.041667in}{0.000000in}}%
\pgfpathcurveto{\pgfqpoint{-0.041667in}{-0.011050in}}{\pgfqpoint{-0.037276in}{-0.021649in}}{\pgfqpoint{-0.029463in}{-0.029463in}}%
\pgfpathcurveto{\pgfqpoint{-0.021649in}{-0.037276in}}{\pgfqpoint{-0.011050in}{-0.041667in}}{\pgfqpoint{0.000000in}{-0.041667in}}%
\pgfpathlineto{\pgfqpoint{0.000000in}{-0.041667in}}%
\pgfpathclose%
\pgfusepath{stroke,fill}%
}%
\end{pgfscope}%
\begin{pgfscope}%
\pgfsetbuttcap%
\pgfsetroundjoin%
\definecolor{currentfill}{rgb}{0.000000,0.000000,0.000000}%
\pgfsetfillcolor{currentfill}%
\pgfsetlinewidth{0.803000pt}%
\definecolor{currentstroke}{rgb}{0.000000,0.000000,0.000000}%
\pgfsetstrokecolor{currentstroke}%
\pgfsetdash{}{0pt}%
\pgfsys@defobject{currentmarker}{\pgfqpoint{0.000000in}{-0.048611in}}{\pgfqpoint{0.000000in}{0.000000in}}{%
\pgfpathmoveto{\pgfqpoint{0.000000in}{0.000000in}}%
\pgfpathlineto{\pgfqpoint{0.000000in}{-0.048611in}}%
\pgfusepath{stroke,fill}%
}%
\begin{pgfscope}%
\pgfsys@transformshift{5.747357in}{0.569136in}%
\pgfsys@useobject{currentmarker}{}%
\end{pgfscope}%
\end{pgfscope}%
\begin{pgfscope}%
\definecolor{textcolor}{rgb}{0.000000,0.000000,0.000000}%
\pgfsetstrokecolor{textcolor}%
\pgfsetfillcolor{textcolor}%
\pgftext[x=5.747357in,y=0.471913in,,top]{\color{textcolor}\rmfamily\fontsize{10.000000}{12.000000}\selectfont \(\displaystyle {2}\)}%
\end{pgfscope}%
\begin{pgfscope}%
\pgfsetbuttcap%
\pgfsetroundjoin%
\definecolor{currentfill}{rgb}{0.000000,0.000000,0.000000}%
\pgfsetfillcolor{currentfill}%
\pgfsetlinewidth{0.803000pt}%
\definecolor{currentstroke}{rgb}{0.000000,0.000000,0.000000}%
\pgfsetstrokecolor{currentstroke}%
\pgfsetdash{}{0pt}%
\pgfsys@defobject{currentmarker}{\pgfqpoint{0.000000in}{-0.048611in}}{\pgfqpoint{0.000000in}{0.000000in}}{%
\pgfpathmoveto{\pgfqpoint{0.000000in}{0.000000in}}%
\pgfpathlineto{\pgfqpoint{0.000000in}{-0.048611in}}%
\pgfusepath{stroke,fill}%
}%
\begin{pgfscope}%
\pgfsys@transformshift{6.321463in}{0.569136in}%
\pgfsys@useobject{currentmarker}{}%
\end{pgfscope}%
\end{pgfscope}%
\begin{pgfscope}%
\definecolor{textcolor}{rgb}{0.000000,0.000000,0.000000}%
\pgfsetstrokecolor{textcolor}%
\pgfsetfillcolor{textcolor}%
\pgftext[x=6.321463in,y=0.471913in,,top]{\color{textcolor}\rmfamily\fontsize{10.000000}{12.000000}\selectfont \(\displaystyle {4}\)}%
\end{pgfscope}%
\begin{pgfscope}%
\pgfsetbuttcap%
\pgfsetroundjoin%
\definecolor{currentfill}{rgb}{0.000000,0.000000,0.000000}%
\pgfsetfillcolor{currentfill}%
\pgfsetlinewidth{0.803000pt}%
\definecolor{currentstroke}{rgb}{0.000000,0.000000,0.000000}%
\pgfsetstrokecolor{currentstroke}%
\pgfsetdash{}{0pt}%
\pgfsys@defobject{currentmarker}{\pgfqpoint{0.000000in}{-0.048611in}}{\pgfqpoint{0.000000in}{0.000000in}}{%
\pgfpathmoveto{\pgfqpoint{0.000000in}{0.000000in}}%
\pgfpathlineto{\pgfqpoint{0.000000in}{-0.048611in}}%
\pgfusepath{stroke,fill}%
}%
\begin{pgfscope}%
\pgfsys@transformshift{6.895569in}{0.569136in}%
\pgfsys@useobject{currentmarker}{}%
\end{pgfscope}%
\end{pgfscope}%
\begin{pgfscope}%
\definecolor{textcolor}{rgb}{0.000000,0.000000,0.000000}%
\pgfsetstrokecolor{textcolor}%
\pgfsetfillcolor{textcolor}%
\pgftext[x=6.895569in,y=0.471913in,,top]{\color{textcolor}\rmfamily\fontsize{10.000000}{12.000000}\selectfont \(\displaystyle {6}\)}%
\end{pgfscope}%
\begin{pgfscope}%
\pgfsetbuttcap%
\pgfsetroundjoin%
\definecolor{currentfill}{rgb}{0.000000,0.000000,0.000000}%
\pgfsetfillcolor{currentfill}%
\pgfsetlinewidth{0.803000pt}%
\definecolor{currentstroke}{rgb}{0.000000,0.000000,0.000000}%
\pgfsetstrokecolor{currentstroke}%
\pgfsetdash{}{0pt}%
\pgfsys@defobject{currentmarker}{\pgfqpoint{0.000000in}{-0.048611in}}{\pgfqpoint{0.000000in}{0.000000in}}{%
\pgfpathmoveto{\pgfqpoint{0.000000in}{0.000000in}}%
\pgfpathlineto{\pgfqpoint{0.000000in}{-0.048611in}}%
\pgfusepath{stroke,fill}%
}%
\begin{pgfscope}%
\pgfsys@transformshift{7.469674in}{0.569136in}%
\pgfsys@useobject{currentmarker}{}%
\end{pgfscope}%
\end{pgfscope}%
\begin{pgfscope}%
\definecolor{textcolor}{rgb}{0.000000,0.000000,0.000000}%
\pgfsetstrokecolor{textcolor}%
\pgfsetfillcolor{textcolor}%
\pgftext[x=7.469674in,y=0.471913in,,top]{\color{textcolor}\rmfamily\fontsize{10.000000}{12.000000}\selectfont \(\displaystyle {8}\)}%
\end{pgfscope}%
\begin{pgfscope}%
\definecolor{textcolor}{rgb}{0.000000,0.000000,0.000000}%
\pgfsetstrokecolor{textcolor}%
\pgfsetfillcolor{textcolor}%
\pgftext[x=6.381586in,y=0.292901in,,top]{\color{textcolor}\rmfamily\fontsize{10.000000}{12.000000}\selectfont petal\_length}%
\end{pgfscope}%
\begin{pgfscope}%
\pgfsetbuttcap%
\pgfsetroundjoin%
\definecolor{currentfill}{rgb}{0.000000,0.000000,0.000000}%
\pgfsetfillcolor{currentfill}%
\pgfsetlinewidth{0.803000pt}%
\definecolor{currentstroke}{rgb}{0.000000,0.000000,0.000000}%
\pgfsetstrokecolor{currentstroke}%
\pgfsetdash{}{0pt}%
\pgfsys@defobject{currentmarker}{\pgfqpoint{-0.048611in}{0.000000in}}{\pgfqpoint{-0.000000in}{0.000000in}}{%
\pgfpathmoveto{\pgfqpoint{-0.000000in}{0.000000in}}%
\pgfpathlineto{\pgfqpoint{-0.048611in}{0.000000in}}%
\pgfusepath{stroke,fill}%
}%
\begin{pgfscope}%
\pgfsys@transformshift{5.292946in}{0.585816in}%
\pgfsys@useobject{currentmarker}{}%
\end{pgfscope}%
\end{pgfscope}%
\begin{pgfscope}%
\pgfsetbuttcap%
\pgfsetroundjoin%
\definecolor{currentfill}{rgb}{0.000000,0.000000,0.000000}%
\pgfsetfillcolor{currentfill}%
\pgfsetlinewidth{0.803000pt}%
\definecolor{currentstroke}{rgb}{0.000000,0.000000,0.000000}%
\pgfsetstrokecolor{currentstroke}%
\pgfsetdash{}{0pt}%
\pgfsys@defobject{currentmarker}{\pgfqpoint{-0.048611in}{0.000000in}}{\pgfqpoint{-0.000000in}{0.000000in}}{%
\pgfpathmoveto{\pgfqpoint{-0.000000in}{0.000000in}}%
\pgfpathlineto{\pgfqpoint{-0.048611in}{0.000000in}}%
\pgfusepath{stroke,fill}%
}%
\begin{pgfscope}%
\pgfsys@transformshift{5.292946in}{1.002815in}%
\pgfsys@useobject{currentmarker}{}%
\end{pgfscope}%
\end{pgfscope}%
\begin{pgfscope}%
\pgfsetbuttcap%
\pgfsetroundjoin%
\definecolor{currentfill}{rgb}{0.000000,0.000000,0.000000}%
\pgfsetfillcolor{currentfill}%
\pgfsetlinewidth{0.803000pt}%
\definecolor{currentstroke}{rgb}{0.000000,0.000000,0.000000}%
\pgfsetstrokecolor{currentstroke}%
\pgfsetdash{}{0pt}%
\pgfsys@defobject{currentmarker}{\pgfqpoint{-0.048611in}{0.000000in}}{\pgfqpoint{-0.000000in}{0.000000in}}{%
\pgfpathmoveto{\pgfqpoint{-0.000000in}{0.000000in}}%
\pgfpathlineto{\pgfqpoint{-0.048611in}{0.000000in}}%
\pgfusepath{stroke,fill}%
}%
\begin{pgfscope}%
\pgfsys@transformshift{5.292946in}{1.419814in}%
\pgfsys@useobject{currentmarker}{}%
\end{pgfscope}%
\end{pgfscope}%
\begin{pgfscope}%
\pgfsetbuttcap%
\pgfsetroundjoin%
\definecolor{currentfill}{rgb}{0.000000,0.000000,0.000000}%
\pgfsetfillcolor{currentfill}%
\pgfsetlinewidth{0.803000pt}%
\definecolor{currentstroke}{rgb}{0.000000,0.000000,0.000000}%
\pgfsetstrokecolor{currentstroke}%
\pgfsetdash{}{0pt}%
\pgfsys@defobject{currentmarker}{\pgfqpoint{-0.048611in}{0.000000in}}{\pgfqpoint{-0.000000in}{0.000000in}}{%
\pgfpathmoveto{\pgfqpoint{-0.000000in}{0.000000in}}%
\pgfpathlineto{\pgfqpoint{-0.048611in}{0.000000in}}%
\pgfusepath{stroke,fill}%
}%
\begin{pgfscope}%
\pgfsys@transformshift{5.292946in}{1.836813in}%
\pgfsys@useobject{currentmarker}{}%
\end{pgfscope}%
\end{pgfscope}%
\begin{pgfscope}%
\pgfsetbuttcap%
\pgfsetroundjoin%
\definecolor{currentfill}{rgb}{0.000000,0.000000,0.000000}%
\pgfsetfillcolor{currentfill}%
\pgfsetlinewidth{0.803000pt}%
\definecolor{currentstroke}{rgb}{0.000000,0.000000,0.000000}%
\pgfsetstrokecolor{currentstroke}%
\pgfsetdash{}{0pt}%
\pgfsys@defobject{currentmarker}{\pgfqpoint{-0.048611in}{0.000000in}}{\pgfqpoint{-0.000000in}{0.000000in}}{%
\pgfpathmoveto{\pgfqpoint{-0.000000in}{0.000000in}}%
\pgfpathlineto{\pgfqpoint{-0.048611in}{0.000000in}}%
\pgfusepath{stroke,fill}%
}%
\begin{pgfscope}%
\pgfsys@transformshift{5.292946in}{2.253812in}%
\pgfsys@useobject{currentmarker}{}%
\end{pgfscope}%
\end{pgfscope}%
\begin{pgfscope}%
\pgfsetbuttcap%
\pgfsetroundjoin%
\definecolor{currentfill}{rgb}{0.000000,0.000000,0.000000}%
\pgfsetfillcolor{currentfill}%
\pgfsetlinewidth{0.803000pt}%
\definecolor{currentstroke}{rgb}{0.000000,0.000000,0.000000}%
\pgfsetstrokecolor{currentstroke}%
\pgfsetdash{}{0pt}%
\pgfsys@defobject{currentmarker}{\pgfqpoint{-0.048611in}{0.000000in}}{\pgfqpoint{-0.000000in}{0.000000in}}{%
\pgfpathmoveto{\pgfqpoint{-0.000000in}{0.000000in}}%
\pgfpathlineto{\pgfqpoint{-0.048611in}{0.000000in}}%
\pgfusepath{stroke,fill}%
}%
\begin{pgfscope}%
\pgfsys@transformshift{5.292946in}{2.670811in}%
\pgfsys@useobject{currentmarker}{}%
\end{pgfscope}%
\end{pgfscope}%
\begin{pgfscope}%
\pgfsetrectcap%
\pgfsetmiterjoin%
\pgfsetlinewidth{0.803000pt}%
\definecolor{currentstroke}{rgb}{0.000000,0.000000,0.000000}%
\pgfsetstrokecolor{currentstroke}%
\pgfsetdash{}{0pt}%
\pgfpathmoveto{\pgfqpoint{5.292946in}{0.569136in}}%
\pgfpathlineto{\pgfqpoint{5.292946in}{2.770891in}}%
\pgfusepath{stroke}%
\end{pgfscope}%
\begin{pgfscope}%
\pgfsetrectcap%
\pgfsetmiterjoin%
\pgfsetlinewidth{0.803000pt}%
\definecolor{currentstroke}{rgb}{0.000000,0.000000,0.000000}%
\pgfsetstrokecolor{currentstroke}%
\pgfsetdash{}{0pt}%
\pgfpathmoveto{\pgfqpoint{5.292946in}{0.569136in}}%
\pgfpathlineto{\pgfqpoint{7.470226in}{0.569136in}}%
\pgfusepath{stroke}%
\end{pgfscope}%
\begin{pgfscope}%
\pgfsetbuttcap%
\pgfsetmiterjoin%
\definecolor{currentfill}{rgb}{1.000000,1.000000,1.000000}%
\pgfsetfillcolor{currentfill}%
\pgfsetlinewidth{0.000000pt}%
\definecolor{currentstroke}{rgb}{0.000000,0.000000,0.000000}%
\pgfsetstrokecolor{currentstroke}%
\pgfsetstrokeopacity{0.000000}%
\pgfsetdash{}{0pt}%
\pgfpathmoveto{\pgfqpoint{7.622482in}{0.569136in}}%
\pgfpathlineto{\pgfqpoint{9.799762in}{0.569136in}}%
\pgfpathlineto{\pgfqpoint{9.799762in}{2.770891in}}%
\pgfpathlineto{\pgfqpoint{7.622482in}{2.770891in}}%
\pgfpathlineto{\pgfqpoint{7.622482in}{0.569136in}}%
\pgfpathclose%
\pgfusepath{fill}%
\end{pgfscope}%
\begin{pgfscope}%
\pgfsetbuttcap%
\pgfsetroundjoin%
\definecolor{currentfill}{rgb}{0.000000,0.000000,0.000000}%
\pgfsetfillcolor{currentfill}%
\pgfsetlinewidth{0.803000pt}%
\definecolor{currentstroke}{rgb}{0.000000,0.000000,0.000000}%
\pgfsetstrokecolor{currentstroke}%
\pgfsetdash{}{0pt}%
\pgfsys@defobject{currentmarker}{\pgfqpoint{0.000000in}{-0.048611in}}{\pgfqpoint{0.000000in}{0.000000in}}{%
\pgfpathmoveto{\pgfqpoint{0.000000in}{0.000000in}}%
\pgfpathlineto{\pgfqpoint{0.000000in}{-0.048611in}}%
\pgfusepath{stroke,fill}%
}%
\begin{pgfscope}%
\pgfsys@transformshift{7.751654in}{0.569136in}%
\pgfsys@useobject{currentmarker}{}%
\end{pgfscope}%
\end{pgfscope}%
\begin{pgfscope}%
\definecolor{textcolor}{rgb}{0.000000,0.000000,0.000000}%
\pgfsetstrokecolor{textcolor}%
\pgfsetfillcolor{textcolor}%
\pgftext[x=7.751654in,y=0.471913in,,top]{\color{textcolor}\rmfamily\fontsize{10.000000}{12.000000}\selectfont \(\displaystyle {0}\)}%
\end{pgfscope}%
\begin{pgfscope}%
\pgfsetbuttcap%
\pgfsetroundjoin%
\definecolor{currentfill}{rgb}{0.000000,0.000000,0.000000}%
\pgfsetfillcolor{currentfill}%
\pgfsetlinewidth{0.803000pt}%
\definecolor{currentstroke}{rgb}{0.000000,0.000000,0.000000}%
\pgfsetstrokecolor{currentstroke}%
\pgfsetdash{}{0pt}%
\pgfsys@defobject{currentmarker}{\pgfqpoint{0.000000in}{-0.048611in}}{\pgfqpoint{0.000000in}{0.000000in}}{%
\pgfpathmoveto{\pgfqpoint{0.000000in}{0.000000in}}%
\pgfpathlineto{\pgfqpoint{0.000000in}{-0.048611in}}%
\pgfusepath{stroke,fill}%
}%
\begin{pgfscope}%
\pgfsys@transformshift{8.429193in}{0.569136in}%
\pgfsys@useobject{currentmarker}{}%
\end{pgfscope}%
\end{pgfscope}%
\begin{pgfscope}%
\definecolor{textcolor}{rgb}{0.000000,0.000000,0.000000}%
\pgfsetstrokecolor{textcolor}%
\pgfsetfillcolor{textcolor}%
\pgftext[x=8.429193in,y=0.471913in,,top]{\color{textcolor}\rmfamily\fontsize{10.000000}{12.000000}\selectfont \(\displaystyle {1}\)}%
\end{pgfscope}%
\begin{pgfscope}%
\pgfsetbuttcap%
\pgfsetroundjoin%
\definecolor{currentfill}{rgb}{0.000000,0.000000,0.000000}%
\pgfsetfillcolor{currentfill}%
\pgfsetlinewidth{0.803000pt}%
\definecolor{currentstroke}{rgb}{0.000000,0.000000,0.000000}%
\pgfsetstrokecolor{currentstroke}%
\pgfsetdash{}{0pt}%
\pgfsys@defobject{currentmarker}{\pgfqpoint{0.000000in}{-0.048611in}}{\pgfqpoint{0.000000in}{0.000000in}}{%
\pgfpathmoveto{\pgfqpoint{0.000000in}{0.000000in}}%
\pgfpathlineto{\pgfqpoint{0.000000in}{-0.048611in}}%
\pgfusepath{stroke,fill}%
}%
\begin{pgfscope}%
\pgfsys@transformshift{9.106731in}{0.569136in}%
\pgfsys@useobject{currentmarker}{}%
\end{pgfscope}%
\end{pgfscope}%
\begin{pgfscope}%
\definecolor{textcolor}{rgb}{0.000000,0.000000,0.000000}%
\pgfsetstrokecolor{textcolor}%
\pgfsetfillcolor{textcolor}%
\pgftext[x=9.106731in,y=0.471913in,,top]{\color{textcolor}\rmfamily\fontsize{10.000000}{12.000000}\selectfont \(\displaystyle {2}\)}%
\end{pgfscope}%
\begin{pgfscope}%
\pgfsetbuttcap%
\pgfsetroundjoin%
\definecolor{currentfill}{rgb}{0.000000,0.000000,0.000000}%
\pgfsetfillcolor{currentfill}%
\pgfsetlinewidth{0.803000pt}%
\definecolor{currentstroke}{rgb}{0.000000,0.000000,0.000000}%
\pgfsetstrokecolor{currentstroke}%
\pgfsetdash{}{0pt}%
\pgfsys@defobject{currentmarker}{\pgfqpoint{0.000000in}{-0.048611in}}{\pgfqpoint{0.000000in}{0.000000in}}{%
\pgfpathmoveto{\pgfqpoint{0.000000in}{0.000000in}}%
\pgfpathlineto{\pgfqpoint{0.000000in}{-0.048611in}}%
\pgfusepath{stroke,fill}%
}%
\begin{pgfscope}%
\pgfsys@transformshift{9.784270in}{0.569136in}%
\pgfsys@useobject{currentmarker}{}%
\end{pgfscope}%
\end{pgfscope}%
\begin{pgfscope}%
\definecolor{textcolor}{rgb}{0.000000,0.000000,0.000000}%
\pgfsetstrokecolor{textcolor}%
\pgfsetfillcolor{textcolor}%
\pgftext[x=9.784270in,y=0.471913in,,top]{\color{textcolor}\rmfamily\fontsize{10.000000}{12.000000}\selectfont \(\displaystyle {3}\)}%
\end{pgfscope}%
\begin{pgfscope}%
\definecolor{textcolor}{rgb}{0.000000,0.000000,0.000000}%
\pgfsetstrokecolor{textcolor}%
\pgfsetfillcolor{textcolor}%
\pgftext[x=8.711122in,y=0.292901in,,top]{\color{textcolor}\rmfamily\fontsize{10.000000}{12.000000}\selectfont petal\_width}%
\end{pgfscope}%
\begin{pgfscope}%
\pgfsetbuttcap%
\pgfsetroundjoin%
\definecolor{currentfill}{rgb}{0.000000,0.000000,0.000000}%
\pgfsetfillcolor{currentfill}%
\pgfsetlinewidth{0.803000pt}%
\definecolor{currentstroke}{rgb}{0.000000,0.000000,0.000000}%
\pgfsetstrokecolor{currentstroke}%
\pgfsetdash{}{0pt}%
\pgfsys@defobject{currentmarker}{\pgfqpoint{-0.048611in}{0.000000in}}{\pgfqpoint{-0.000000in}{0.000000in}}{%
\pgfpathmoveto{\pgfqpoint{-0.000000in}{0.000000in}}%
\pgfpathlineto{\pgfqpoint{-0.048611in}{0.000000in}}%
\pgfusepath{stroke,fill}%
}%
\begin{pgfscope}%
\pgfsys@transformshift{7.622482in}{0.585816in}%
\pgfsys@useobject{currentmarker}{}%
\end{pgfscope}%
\end{pgfscope}%
\begin{pgfscope}%
\pgfsetbuttcap%
\pgfsetroundjoin%
\definecolor{currentfill}{rgb}{0.000000,0.000000,0.000000}%
\pgfsetfillcolor{currentfill}%
\pgfsetlinewidth{0.803000pt}%
\definecolor{currentstroke}{rgb}{0.000000,0.000000,0.000000}%
\pgfsetstrokecolor{currentstroke}%
\pgfsetdash{}{0pt}%
\pgfsys@defobject{currentmarker}{\pgfqpoint{-0.048611in}{0.000000in}}{\pgfqpoint{-0.000000in}{0.000000in}}{%
\pgfpathmoveto{\pgfqpoint{-0.000000in}{0.000000in}}%
\pgfpathlineto{\pgfqpoint{-0.048611in}{0.000000in}}%
\pgfusepath{stroke,fill}%
}%
\begin{pgfscope}%
\pgfsys@transformshift{7.622482in}{1.002815in}%
\pgfsys@useobject{currentmarker}{}%
\end{pgfscope}%
\end{pgfscope}%
\begin{pgfscope}%
\pgfsetbuttcap%
\pgfsetroundjoin%
\definecolor{currentfill}{rgb}{0.000000,0.000000,0.000000}%
\pgfsetfillcolor{currentfill}%
\pgfsetlinewidth{0.803000pt}%
\definecolor{currentstroke}{rgb}{0.000000,0.000000,0.000000}%
\pgfsetstrokecolor{currentstroke}%
\pgfsetdash{}{0pt}%
\pgfsys@defobject{currentmarker}{\pgfqpoint{-0.048611in}{0.000000in}}{\pgfqpoint{-0.000000in}{0.000000in}}{%
\pgfpathmoveto{\pgfqpoint{-0.000000in}{0.000000in}}%
\pgfpathlineto{\pgfqpoint{-0.048611in}{0.000000in}}%
\pgfusepath{stroke,fill}%
}%
\begin{pgfscope}%
\pgfsys@transformshift{7.622482in}{1.419814in}%
\pgfsys@useobject{currentmarker}{}%
\end{pgfscope}%
\end{pgfscope}%
\begin{pgfscope}%
\pgfsetbuttcap%
\pgfsetroundjoin%
\definecolor{currentfill}{rgb}{0.000000,0.000000,0.000000}%
\pgfsetfillcolor{currentfill}%
\pgfsetlinewidth{0.803000pt}%
\definecolor{currentstroke}{rgb}{0.000000,0.000000,0.000000}%
\pgfsetstrokecolor{currentstroke}%
\pgfsetdash{}{0pt}%
\pgfsys@defobject{currentmarker}{\pgfqpoint{-0.048611in}{0.000000in}}{\pgfqpoint{-0.000000in}{0.000000in}}{%
\pgfpathmoveto{\pgfqpoint{-0.000000in}{0.000000in}}%
\pgfpathlineto{\pgfqpoint{-0.048611in}{0.000000in}}%
\pgfusepath{stroke,fill}%
}%
\begin{pgfscope}%
\pgfsys@transformshift{7.622482in}{1.836813in}%
\pgfsys@useobject{currentmarker}{}%
\end{pgfscope}%
\end{pgfscope}%
\begin{pgfscope}%
\pgfsetbuttcap%
\pgfsetroundjoin%
\definecolor{currentfill}{rgb}{0.000000,0.000000,0.000000}%
\pgfsetfillcolor{currentfill}%
\pgfsetlinewidth{0.803000pt}%
\definecolor{currentstroke}{rgb}{0.000000,0.000000,0.000000}%
\pgfsetstrokecolor{currentstroke}%
\pgfsetdash{}{0pt}%
\pgfsys@defobject{currentmarker}{\pgfqpoint{-0.048611in}{0.000000in}}{\pgfqpoint{-0.000000in}{0.000000in}}{%
\pgfpathmoveto{\pgfqpoint{-0.000000in}{0.000000in}}%
\pgfpathlineto{\pgfqpoint{-0.048611in}{0.000000in}}%
\pgfusepath{stroke,fill}%
}%
\begin{pgfscope}%
\pgfsys@transformshift{7.622482in}{2.253812in}%
\pgfsys@useobject{currentmarker}{}%
\end{pgfscope}%
\end{pgfscope}%
\begin{pgfscope}%
\pgfsetbuttcap%
\pgfsetroundjoin%
\definecolor{currentfill}{rgb}{0.000000,0.000000,0.000000}%
\pgfsetfillcolor{currentfill}%
\pgfsetlinewidth{0.803000pt}%
\definecolor{currentstroke}{rgb}{0.000000,0.000000,0.000000}%
\pgfsetstrokecolor{currentstroke}%
\pgfsetdash{}{0pt}%
\pgfsys@defobject{currentmarker}{\pgfqpoint{-0.048611in}{0.000000in}}{\pgfqpoint{-0.000000in}{0.000000in}}{%
\pgfpathmoveto{\pgfqpoint{-0.000000in}{0.000000in}}%
\pgfpathlineto{\pgfqpoint{-0.048611in}{0.000000in}}%
\pgfusepath{stroke,fill}%
}%
\begin{pgfscope}%
\pgfsys@transformshift{7.622482in}{2.670811in}%
\pgfsys@useobject{currentmarker}{}%
\end{pgfscope}%
\end{pgfscope}%
\begin{pgfscope}%
\pgfsetrectcap%
\pgfsetmiterjoin%
\pgfsetlinewidth{0.803000pt}%
\definecolor{currentstroke}{rgb}{0.000000,0.000000,0.000000}%
\pgfsetstrokecolor{currentstroke}%
\pgfsetdash{}{0pt}%
\pgfpathmoveto{\pgfqpoint{7.622482in}{0.569136in}}%
\pgfpathlineto{\pgfqpoint{7.622482in}{2.770891in}}%
\pgfusepath{stroke}%
\end{pgfscope}%
\begin{pgfscope}%
\pgfsetrectcap%
\pgfsetmiterjoin%
\pgfsetlinewidth{0.803000pt}%
\definecolor{currentstroke}{rgb}{0.000000,0.000000,0.000000}%
\pgfsetstrokecolor{currentstroke}%
\pgfsetdash{}{0pt}%
\pgfpathmoveto{\pgfqpoint{7.622482in}{0.569136in}}%
\pgfpathlineto{\pgfqpoint{9.799762in}{0.569136in}}%
\pgfusepath{stroke}%
\end{pgfscope}%
\begin{pgfscope}%
\pgfpathrectangle{\pgfqpoint{0.633874in}{7.624184in}}{\pgfqpoint{2.177280in}{2.201755in}}%
\pgfusepath{clip}%
\pgfsetbuttcap%
\pgfsetroundjoin%
\definecolor{currentfill}{rgb}{0.172549,0.627451,0.172549}%
\pgfsetfillcolor{currentfill}%
\pgfsetfillopacity{0.250000}%
\pgfsetlinewidth{1.003750pt}%
\definecolor{currentstroke}{rgb}{0.172549,0.627451,0.172549}%
\pgfsetstrokecolor{currentstroke}%
\pgfsetdash{}{0pt}%
\pgfsys@defobject{currentmarker}{\pgfqpoint{0.817197in}{7.624184in}}{\pgfqpoint{2.712187in}{8.815137in}}{%
\pgfpathmoveto{\pgfqpoint{0.817197in}{7.624755in}}%
\pgfpathlineto{\pgfqpoint{0.817197in}{7.624184in}}%
\pgfpathlineto{\pgfqpoint{0.826720in}{7.624184in}}%
\pgfpathlineto{\pgfqpoint{0.836242in}{7.624184in}}%
\pgfpathlineto{\pgfqpoint{0.845765in}{7.624184in}}%
\pgfpathlineto{\pgfqpoint{0.855287in}{7.624184in}}%
\pgfpathlineto{\pgfqpoint{0.864810in}{7.624184in}}%
\pgfpathlineto{\pgfqpoint{0.874332in}{7.624184in}}%
\pgfpathlineto{\pgfqpoint{0.883855in}{7.624184in}}%
\pgfpathlineto{\pgfqpoint{0.893377in}{7.624184in}}%
\pgfpathlineto{\pgfqpoint{0.902900in}{7.624184in}}%
\pgfpathlineto{\pgfqpoint{0.912423in}{7.624184in}}%
\pgfpathlineto{\pgfqpoint{0.921945in}{7.624184in}}%
\pgfpathlineto{\pgfqpoint{0.931468in}{7.624184in}}%
\pgfpathlineto{\pgfqpoint{0.940990in}{7.624184in}}%
\pgfpathlineto{\pgfqpoint{0.950513in}{7.624184in}}%
\pgfpathlineto{\pgfqpoint{0.960035in}{7.624184in}}%
\pgfpathlineto{\pgfqpoint{0.969558in}{7.624184in}}%
\pgfpathlineto{\pgfqpoint{0.979081in}{7.624184in}}%
\pgfpathlineto{\pgfqpoint{0.988603in}{7.624184in}}%
\pgfpathlineto{\pgfqpoint{0.998126in}{7.624184in}}%
\pgfpathlineto{\pgfqpoint{1.007648in}{7.624184in}}%
\pgfpathlineto{\pgfqpoint{1.017171in}{7.624184in}}%
\pgfpathlineto{\pgfqpoint{1.026693in}{7.624184in}}%
\pgfpathlineto{\pgfqpoint{1.036216in}{7.624184in}}%
\pgfpathlineto{\pgfqpoint{1.045738in}{7.624184in}}%
\pgfpathlineto{\pgfqpoint{1.055261in}{7.624184in}}%
\pgfpathlineto{\pgfqpoint{1.064784in}{7.624184in}}%
\pgfpathlineto{\pgfqpoint{1.074306in}{7.624184in}}%
\pgfpathlineto{\pgfqpoint{1.083829in}{7.624184in}}%
\pgfpathlineto{\pgfqpoint{1.093351in}{7.624184in}}%
\pgfpathlineto{\pgfqpoint{1.102874in}{7.624184in}}%
\pgfpathlineto{\pgfqpoint{1.112396in}{7.624184in}}%
\pgfpathlineto{\pgfqpoint{1.121919in}{7.624184in}}%
\pgfpathlineto{\pgfqpoint{1.131442in}{7.624184in}}%
\pgfpathlineto{\pgfqpoint{1.140964in}{7.624184in}}%
\pgfpathlineto{\pgfqpoint{1.150487in}{7.624184in}}%
\pgfpathlineto{\pgfqpoint{1.160009in}{7.624184in}}%
\pgfpathlineto{\pgfqpoint{1.169532in}{7.624184in}}%
\pgfpathlineto{\pgfqpoint{1.179054in}{7.624184in}}%
\pgfpathlineto{\pgfqpoint{1.188577in}{7.624184in}}%
\pgfpathlineto{\pgfqpoint{1.198099in}{7.624184in}}%
\pgfpathlineto{\pgfqpoint{1.207622in}{7.624184in}}%
\pgfpathlineto{\pgfqpoint{1.217145in}{7.624184in}}%
\pgfpathlineto{\pgfqpoint{1.226667in}{7.624184in}}%
\pgfpathlineto{\pgfqpoint{1.236190in}{7.624184in}}%
\pgfpathlineto{\pgfqpoint{1.245712in}{7.624184in}}%
\pgfpathlineto{\pgfqpoint{1.255235in}{7.624184in}}%
\pgfpathlineto{\pgfqpoint{1.264757in}{7.624184in}}%
\pgfpathlineto{\pgfqpoint{1.274280in}{7.624184in}}%
\pgfpathlineto{\pgfqpoint{1.283803in}{7.624184in}}%
\pgfpathlineto{\pgfqpoint{1.293325in}{7.624184in}}%
\pgfpathlineto{\pgfqpoint{1.302848in}{7.624184in}}%
\pgfpathlineto{\pgfqpoint{1.312370in}{7.624184in}}%
\pgfpathlineto{\pgfqpoint{1.321893in}{7.624184in}}%
\pgfpathlineto{\pgfqpoint{1.331415in}{7.624184in}}%
\pgfpathlineto{\pgfqpoint{1.340938in}{7.624184in}}%
\pgfpathlineto{\pgfqpoint{1.350460in}{7.624184in}}%
\pgfpathlineto{\pgfqpoint{1.359983in}{7.624184in}}%
\pgfpathlineto{\pgfqpoint{1.369506in}{7.624184in}}%
\pgfpathlineto{\pgfqpoint{1.379028in}{7.624184in}}%
\pgfpathlineto{\pgfqpoint{1.388551in}{7.624184in}}%
\pgfpathlineto{\pgfqpoint{1.398073in}{7.624184in}}%
\pgfpathlineto{\pgfqpoint{1.407596in}{7.624184in}}%
\pgfpathlineto{\pgfqpoint{1.417118in}{7.624184in}}%
\pgfpathlineto{\pgfqpoint{1.426641in}{7.624184in}}%
\pgfpathlineto{\pgfqpoint{1.436164in}{7.624184in}}%
\pgfpathlineto{\pgfqpoint{1.445686in}{7.624184in}}%
\pgfpathlineto{\pgfqpoint{1.455209in}{7.624184in}}%
\pgfpathlineto{\pgfqpoint{1.464731in}{7.624184in}}%
\pgfpathlineto{\pgfqpoint{1.474254in}{7.624184in}}%
\pgfpathlineto{\pgfqpoint{1.483776in}{7.624184in}}%
\pgfpathlineto{\pgfqpoint{1.493299in}{7.624184in}}%
\pgfpathlineto{\pgfqpoint{1.502821in}{7.624184in}}%
\pgfpathlineto{\pgfqpoint{1.512344in}{7.624184in}}%
\pgfpathlineto{\pgfqpoint{1.521867in}{7.624184in}}%
\pgfpathlineto{\pgfqpoint{1.531389in}{7.624184in}}%
\pgfpathlineto{\pgfqpoint{1.540912in}{7.624184in}}%
\pgfpathlineto{\pgfqpoint{1.550434in}{7.624184in}}%
\pgfpathlineto{\pgfqpoint{1.559957in}{7.624184in}}%
\pgfpathlineto{\pgfqpoint{1.569479in}{7.624184in}}%
\pgfpathlineto{\pgfqpoint{1.579002in}{7.624184in}}%
\pgfpathlineto{\pgfqpoint{1.588525in}{7.624184in}}%
\pgfpathlineto{\pgfqpoint{1.598047in}{7.624184in}}%
\pgfpathlineto{\pgfqpoint{1.607570in}{7.624184in}}%
\pgfpathlineto{\pgfqpoint{1.617092in}{7.624184in}}%
\pgfpathlineto{\pgfqpoint{1.626615in}{7.624184in}}%
\pgfpathlineto{\pgfqpoint{1.636137in}{7.624184in}}%
\pgfpathlineto{\pgfqpoint{1.645660in}{7.624184in}}%
\pgfpathlineto{\pgfqpoint{1.655182in}{7.624184in}}%
\pgfpathlineto{\pgfqpoint{1.664705in}{7.624184in}}%
\pgfpathlineto{\pgfqpoint{1.674228in}{7.624184in}}%
\pgfpathlineto{\pgfqpoint{1.683750in}{7.624184in}}%
\pgfpathlineto{\pgfqpoint{1.693273in}{7.624184in}}%
\pgfpathlineto{\pgfqpoint{1.702795in}{7.624184in}}%
\pgfpathlineto{\pgfqpoint{1.712318in}{7.624184in}}%
\pgfpathlineto{\pgfqpoint{1.721840in}{7.624184in}}%
\pgfpathlineto{\pgfqpoint{1.731363in}{7.624184in}}%
\pgfpathlineto{\pgfqpoint{1.740886in}{7.624184in}}%
\pgfpathlineto{\pgfqpoint{1.750408in}{7.624184in}}%
\pgfpathlineto{\pgfqpoint{1.759931in}{7.624184in}}%
\pgfpathlineto{\pgfqpoint{1.769453in}{7.624184in}}%
\pgfpathlineto{\pgfqpoint{1.778976in}{7.624184in}}%
\pgfpathlineto{\pgfqpoint{1.788498in}{7.624184in}}%
\pgfpathlineto{\pgfqpoint{1.798021in}{7.624184in}}%
\pgfpathlineto{\pgfqpoint{1.807543in}{7.624184in}}%
\pgfpathlineto{\pgfqpoint{1.817066in}{7.624184in}}%
\pgfpathlineto{\pgfqpoint{1.826589in}{7.624184in}}%
\pgfpathlineto{\pgfqpoint{1.836111in}{7.624184in}}%
\pgfpathlineto{\pgfqpoint{1.845634in}{7.624184in}}%
\pgfpathlineto{\pgfqpoint{1.855156in}{7.624184in}}%
\pgfpathlineto{\pgfqpoint{1.864679in}{7.624184in}}%
\pgfpathlineto{\pgfqpoint{1.874201in}{7.624184in}}%
\pgfpathlineto{\pgfqpoint{1.883724in}{7.624184in}}%
\pgfpathlineto{\pgfqpoint{1.893247in}{7.624184in}}%
\pgfpathlineto{\pgfqpoint{1.902769in}{7.624184in}}%
\pgfpathlineto{\pgfqpoint{1.912292in}{7.624184in}}%
\pgfpathlineto{\pgfqpoint{1.921814in}{7.624184in}}%
\pgfpathlineto{\pgfqpoint{1.931337in}{7.624184in}}%
\pgfpathlineto{\pgfqpoint{1.940859in}{7.624184in}}%
\pgfpathlineto{\pgfqpoint{1.950382in}{7.624184in}}%
\pgfpathlineto{\pgfqpoint{1.959904in}{7.624184in}}%
\pgfpathlineto{\pgfqpoint{1.969427in}{7.624184in}}%
\pgfpathlineto{\pgfqpoint{1.978950in}{7.624184in}}%
\pgfpathlineto{\pgfqpoint{1.988472in}{7.624184in}}%
\pgfpathlineto{\pgfqpoint{1.997995in}{7.624184in}}%
\pgfpathlineto{\pgfqpoint{2.007517in}{7.624184in}}%
\pgfpathlineto{\pgfqpoint{2.017040in}{7.624184in}}%
\pgfpathlineto{\pgfqpoint{2.026562in}{7.624184in}}%
\pgfpathlineto{\pgfqpoint{2.036085in}{7.624184in}}%
\pgfpathlineto{\pgfqpoint{2.045608in}{7.624184in}}%
\pgfpathlineto{\pgfqpoint{2.055130in}{7.624184in}}%
\pgfpathlineto{\pgfqpoint{2.064653in}{7.624184in}}%
\pgfpathlineto{\pgfqpoint{2.074175in}{7.624184in}}%
\pgfpathlineto{\pgfqpoint{2.083698in}{7.624184in}}%
\pgfpathlineto{\pgfqpoint{2.093220in}{7.624184in}}%
\pgfpathlineto{\pgfqpoint{2.102743in}{7.624184in}}%
\pgfpathlineto{\pgfqpoint{2.112265in}{7.624184in}}%
\pgfpathlineto{\pgfqpoint{2.121788in}{7.624184in}}%
\pgfpathlineto{\pgfqpoint{2.131311in}{7.624184in}}%
\pgfpathlineto{\pgfqpoint{2.140833in}{7.624184in}}%
\pgfpathlineto{\pgfqpoint{2.150356in}{7.624184in}}%
\pgfpathlineto{\pgfqpoint{2.159878in}{7.624184in}}%
\pgfpathlineto{\pgfqpoint{2.169401in}{7.624184in}}%
\pgfpathlineto{\pgfqpoint{2.178923in}{7.624184in}}%
\pgfpathlineto{\pgfqpoint{2.188446in}{7.624184in}}%
\pgfpathlineto{\pgfqpoint{2.197969in}{7.624184in}}%
\pgfpathlineto{\pgfqpoint{2.207491in}{7.624184in}}%
\pgfpathlineto{\pgfqpoint{2.217014in}{7.624184in}}%
\pgfpathlineto{\pgfqpoint{2.226536in}{7.624184in}}%
\pgfpathlineto{\pgfqpoint{2.236059in}{7.624184in}}%
\pgfpathlineto{\pgfqpoint{2.245581in}{7.624184in}}%
\pgfpathlineto{\pgfqpoint{2.255104in}{7.624184in}}%
\pgfpathlineto{\pgfqpoint{2.264626in}{7.624184in}}%
\pgfpathlineto{\pgfqpoint{2.274149in}{7.624184in}}%
\pgfpathlineto{\pgfqpoint{2.283672in}{7.624184in}}%
\pgfpathlineto{\pgfqpoint{2.293194in}{7.624184in}}%
\pgfpathlineto{\pgfqpoint{2.302717in}{7.624184in}}%
\pgfpathlineto{\pgfqpoint{2.312239in}{7.624184in}}%
\pgfpathlineto{\pgfqpoint{2.321762in}{7.624184in}}%
\pgfpathlineto{\pgfqpoint{2.331284in}{7.624184in}}%
\pgfpathlineto{\pgfqpoint{2.340807in}{7.624184in}}%
\pgfpathlineto{\pgfqpoint{2.350330in}{7.624184in}}%
\pgfpathlineto{\pgfqpoint{2.359852in}{7.624184in}}%
\pgfpathlineto{\pgfqpoint{2.369375in}{7.624184in}}%
\pgfpathlineto{\pgfqpoint{2.378897in}{7.624184in}}%
\pgfpathlineto{\pgfqpoint{2.388420in}{7.624184in}}%
\pgfpathlineto{\pgfqpoint{2.397942in}{7.624184in}}%
\pgfpathlineto{\pgfqpoint{2.407465in}{7.624184in}}%
\pgfpathlineto{\pgfqpoint{2.416987in}{7.624184in}}%
\pgfpathlineto{\pgfqpoint{2.426510in}{7.624184in}}%
\pgfpathlineto{\pgfqpoint{2.436033in}{7.624184in}}%
\pgfpathlineto{\pgfqpoint{2.445555in}{7.624184in}}%
\pgfpathlineto{\pgfqpoint{2.455078in}{7.624184in}}%
\pgfpathlineto{\pgfqpoint{2.464600in}{7.624184in}}%
\pgfpathlineto{\pgfqpoint{2.474123in}{7.624184in}}%
\pgfpathlineto{\pgfqpoint{2.483645in}{7.624184in}}%
\pgfpathlineto{\pgfqpoint{2.493168in}{7.624184in}}%
\pgfpathlineto{\pgfqpoint{2.502691in}{7.624184in}}%
\pgfpathlineto{\pgfqpoint{2.512213in}{7.624184in}}%
\pgfpathlineto{\pgfqpoint{2.521736in}{7.624184in}}%
\pgfpathlineto{\pgfqpoint{2.531258in}{7.624184in}}%
\pgfpathlineto{\pgfqpoint{2.540781in}{7.624184in}}%
\pgfpathlineto{\pgfqpoint{2.550303in}{7.624184in}}%
\pgfpathlineto{\pgfqpoint{2.559826in}{7.624184in}}%
\pgfpathlineto{\pgfqpoint{2.569348in}{7.624184in}}%
\pgfpathlineto{\pgfqpoint{2.578871in}{7.624184in}}%
\pgfpathlineto{\pgfqpoint{2.588394in}{7.624184in}}%
\pgfpathlineto{\pgfqpoint{2.597916in}{7.624184in}}%
\pgfpathlineto{\pgfqpoint{2.607439in}{7.624184in}}%
\pgfpathlineto{\pgfqpoint{2.616961in}{7.624184in}}%
\pgfpathlineto{\pgfqpoint{2.626484in}{7.624184in}}%
\pgfpathlineto{\pgfqpoint{2.636006in}{7.624184in}}%
\pgfpathlineto{\pgfqpoint{2.645529in}{7.624184in}}%
\pgfpathlineto{\pgfqpoint{2.655052in}{7.624184in}}%
\pgfpathlineto{\pgfqpoint{2.664574in}{7.624184in}}%
\pgfpathlineto{\pgfqpoint{2.674097in}{7.624184in}}%
\pgfpathlineto{\pgfqpoint{2.683619in}{7.624184in}}%
\pgfpathlineto{\pgfqpoint{2.693142in}{7.624184in}}%
\pgfpathlineto{\pgfqpoint{2.702664in}{7.624184in}}%
\pgfpathlineto{\pgfqpoint{2.712187in}{7.624184in}}%
\pgfpathlineto{\pgfqpoint{2.712187in}{7.625000in}}%
\pgfpathlineto{\pgfqpoint{2.712187in}{7.625000in}}%
\pgfpathlineto{\pgfqpoint{2.702664in}{7.625244in}}%
\pgfpathlineto{\pgfqpoint{2.693142in}{7.625551in}}%
\pgfpathlineto{\pgfqpoint{2.683619in}{7.625937in}}%
\pgfpathlineto{\pgfqpoint{2.674097in}{7.626418in}}%
\pgfpathlineto{\pgfqpoint{2.664574in}{7.627016in}}%
\pgfpathlineto{\pgfqpoint{2.655052in}{7.627751in}}%
\pgfpathlineto{\pgfqpoint{2.645529in}{7.628652in}}%
\pgfpathlineto{\pgfqpoint{2.636006in}{7.629748in}}%
\pgfpathlineto{\pgfqpoint{2.626484in}{7.631072in}}%
\pgfpathlineto{\pgfqpoint{2.616961in}{7.632662in}}%
\pgfpathlineto{\pgfqpoint{2.607439in}{7.634558in}}%
\pgfpathlineto{\pgfqpoint{2.597916in}{7.636806in}}%
\pgfpathlineto{\pgfqpoint{2.588394in}{7.639452in}}%
\pgfpathlineto{\pgfqpoint{2.578871in}{7.642549in}}%
\pgfpathlineto{\pgfqpoint{2.569348in}{7.646147in}}%
\pgfpathlineto{\pgfqpoint{2.559826in}{7.650300in}}%
\pgfpathlineto{\pgfqpoint{2.550303in}{7.655064in}}%
\pgfpathlineto{\pgfqpoint{2.540781in}{7.660490in}}%
\pgfpathlineto{\pgfqpoint{2.531258in}{7.666629in}}%
\pgfpathlineto{\pgfqpoint{2.521736in}{7.673528in}}%
\pgfpathlineto{\pgfqpoint{2.512213in}{7.681228in}}%
\pgfpathlineto{\pgfqpoint{2.502691in}{7.689763in}}%
\pgfpathlineto{\pgfqpoint{2.493168in}{7.699157in}}%
\pgfpathlineto{\pgfqpoint{2.483645in}{7.709424in}}%
\pgfpathlineto{\pgfqpoint{2.474123in}{7.720565in}}%
\pgfpathlineto{\pgfqpoint{2.464600in}{7.732568in}}%
\pgfpathlineto{\pgfqpoint{2.455078in}{7.745407in}}%
\pgfpathlineto{\pgfqpoint{2.445555in}{7.759037in}}%
\pgfpathlineto{\pgfqpoint{2.436033in}{7.773401in}}%
\pgfpathlineto{\pgfqpoint{2.426510in}{7.788423in}}%
\pgfpathlineto{\pgfqpoint{2.416987in}{7.804015in}}%
\pgfpathlineto{\pgfqpoint{2.407465in}{7.820074in}}%
\pgfpathlineto{\pgfqpoint{2.397942in}{7.836485in}}%
\pgfpathlineto{\pgfqpoint{2.388420in}{7.853122in}}%
\pgfpathlineto{\pgfqpoint{2.378897in}{7.869856in}}%
\pgfpathlineto{\pgfqpoint{2.369375in}{7.886551in}}%
\pgfpathlineto{\pgfqpoint{2.359852in}{7.903075in}}%
\pgfpathlineto{\pgfqpoint{2.350330in}{7.919295in}}%
\pgfpathlineto{\pgfqpoint{2.340807in}{7.935090in}}%
\pgfpathlineto{\pgfqpoint{2.331284in}{7.950349in}}%
\pgfpathlineto{\pgfqpoint{2.321762in}{7.964973in}}%
\pgfpathlineto{\pgfqpoint{2.312239in}{7.978884in}}%
\pgfpathlineto{\pgfqpoint{2.302717in}{7.992022in}}%
\pgfpathlineto{\pgfqpoint{2.293194in}{8.004349in}}%
\pgfpathlineto{\pgfqpoint{2.283672in}{8.015847in}}%
\pgfpathlineto{\pgfqpoint{2.274149in}{8.026525in}}%
\pgfpathlineto{\pgfqpoint{2.264626in}{8.036410in}}%
\pgfpathlineto{\pgfqpoint{2.255104in}{8.045551in}}%
\pgfpathlineto{\pgfqpoint{2.245581in}{8.054016in}}%
\pgfpathlineto{\pgfqpoint{2.236059in}{8.061891in}}%
\pgfpathlineto{\pgfqpoint{2.226536in}{8.069272in}}%
\pgfpathlineto{\pgfqpoint{2.217014in}{8.076268in}}%
\pgfpathlineto{\pgfqpoint{2.207491in}{8.082994in}}%
\pgfpathlineto{\pgfqpoint{2.197969in}{8.089569in}}%
\pgfpathlineto{\pgfqpoint{2.188446in}{8.096116in}}%
\pgfpathlineto{\pgfqpoint{2.178923in}{8.102751in}}%
\pgfpathlineto{\pgfqpoint{2.169401in}{8.109592in}}%
\pgfpathlineto{\pgfqpoint{2.159878in}{8.116747in}}%
\pgfpathlineto{\pgfqpoint{2.150356in}{8.124318in}}%
\pgfpathlineto{\pgfqpoint{2.140833in}{8.132401in}}%
\pgfpathlineto{\pgfqpoint{2.131311in}{8.141080in}}%
\pgfpathlineto{\pgfqpoint{2.121788in}{8.150433in}}%
\pgfpathlineto{\pgfqpoint{2.112265in}{8.160526in}}%
\pgfpathlineto{\pgfqpoint{2.102743in}{8.171419in}}%
\pgfpathlineto{\pgfqpoint{2.093220in}{8.183162in}}%
\pgfpathlineto{\pgfqpoint{2.083698in}{8.195796in}}%
\pgfpathlineto{\pgfqpoint{2.074175in}{8.209356in}}%
\pgfpathlineto{\pgfqpoint{2.064653in}{8.223868in}}%
\pgfpathlineto{\pgfqpoint{2.055130in}{8.239348in}}%
\pgfpathlineto{\pgfqpoint{2.045608in}{8.255806in}}%
\pgfpathlineto{\pgfqpoint{2.036085in}{8.273243in}}%
\pgfpathlineto{\pgfqpoint{2.026562in}{8.291650in}}%
\pgfpathlineto{\pgfqpoint{2.017040in}{8.311009in}}%
\pgfpathlineto{\pgfqpoint{2.007517in}{8.331292in}}%
\pgfpathlineto{\pgfqpoint{1.997995in}{8.352459in}}%
\pgfpathlineto{\pgfqpoint{1.988472in}{8.374461in}}%
\pgfpathlineto{\pgfqpoint{1.978950in}{8.397235in}}%
\pgfpathlineto{\pgfqpoint{1.969427in}{8.420706in}}%
\pgfpathlineto{\pgfqpoint{1.959904in}{8.444787in}}%
\pgfpathlineto{\pgfqpoint{1.950382in}{8.469375in}}%
\pgfpathlineto{\pgfqpoint{1.940859in}{8.494358in}}%
\pgfpathlineto{\pgfqpoint{1.931337in}{8.519606in}}%
\pgfpathlineto{\pgfqpoint{1.921814in}{8.544979in}}%
\pgfpathlineto{\pgfqpoint{1.912292in}{8.570321in}}%
\pgfpathlineto{\pgfqpoint{1.902769in}{8.595465in}}%
\pgfpathlineto{\pgfqpoint{1.893247in}{8.620231in}}%
\pgfpathlineto{\pgfqpoint{1.883724in}{8.644426in}}%
\pgfpathlineto{\pgfqpoint{1.874201in}{8.667851in}}%
\pgfpathlineto{\pgfqpoint{1.864679in}{8.690296in}}%
\pgfpathlineto{\pgfqpoint{1.855156in}{8.711546in}}%
\pgfpathlineto{\pgfqpoint{1.845634in}{8.731383in}}%
\pgfpathlineto{\pgfqpoint{1.836111in}{8.749591in}}%
\pgfpathlineto{\pgfqpoint{1.826589in}{8.765956in}}%
\pgfpathlineto{\pgfqpoint{1.817066in}{8.780274in}}%
\pgfpathlineto{\pgfqpoint{1.807543in}{8.792353in}}%
\pgfpathlineto{\pgfqpoint{1.798021in}{8.802017in}}%
\pgfpathlineto{\pgfqpoint{1.788498in}{8.809116in}}%
\pgfpathlineto{\pgfqpoint{1.778976in}{8.813521in}}%
\pgfpathlineto{\pgfqpoint{1.769453in}{8.815137in}}%
\pgfpathlineto{\pgfqpoint{1.759931in}{8.813900in}}%
\pgfpathlineto{\pgfqpoint{1.750408in}{8.809783in}}%
\pgfpathlineto{\pgfqpoint{1.740886in}{8.802795in}}%
\pgfpathlineto{\pgfqpoint{1.731363in}{8.792982in}}%
\pgfpathlineto{\pgfqpoint{1.721840in}{8.780429in}}%
\pgfpathlineto{\pgfqpoint{1.712318in}{8.765252in}}%
\pgfpathlineto{\pgfqpoint{1.702795in}{8.747601in}}%
\pgfpathlineto{\pgfqpoint{1.693273in}{8.727651in}}%
\pgfpathlineto{\pgfqpoint{1.683750in}{8.705602in}}%
\pgfpathlineto{\pgfqpoint{1.674228in}{8.681668in}}%
\pgfpathlineto{\pgfqpoint{1.664705in}{8.656074in}}%
\pgfpathlineto{\pgfqpoint{1.655182in}{8.629052in}}%
\pgfpathlineto{\pgfqpoint{1.645660in}{8.600830in}}%
\pgfpathlineto{\pgfqpoint{1.636137in}{8.571630in}}%
\pgfpathlineto{\pgfqpoint{1.626615in}{8.541663in}}%
\pgfpathlineto{\pgfqpoint{1.617092in}{8.511123in}}%
\pgfpathlineto{\pgfqpoint{1.607570in}{8.480187in}}%
\pgfpathlineto{\pgfqpoint{1.598047in}{8.449010in}}%
\pgfpathlineto{\pgfqpoint{1.588525in}{8.417725in}}%
\pgfpathlineto{\pgfqpoint{1.579002in}{8.386445in}}%
\pgfpathlineto{\pgfqpoint{1.569479in}{8.355262in}}%
\pgfpathlineto{\pgfqpoint{1.559957in}{8.324250in}}%
\pgfpathlineto{\pgfqpoint{1.550434in}{8.293467in}}%
\pgfpathlineto{\pgfqpoint{1.540912in}{8.262959in}}%
\pgfpathlineto{\pgfqpoint{1.531389in}{8.232763in}}%
\pgfpathlineto{\pgfqpoint{1.521867in}{8.202910in}}%
\pgfpathlineto{\pgfqpoint{1.512344in}{8.173431in}}%
\pgfpathlineto{\pgfqpoint{1.502821in}{8.144355in}}%
\pgfpathlineto{\pgfqpoint{1.493299in}{8.115717in}}%
\pgfpathlineto{\pgfqpoint{1.483776in}{8.087556in}}%
\pgfpathlineto{\pgfqpoint{1.474254in}{8.059919in}}%
\pgfpathlineto{\pgfqpoint{1.464731in}{8.032858in}}%
\pgfpathlineto{\pgfqpoint{1.455209in}{8.006436in}}%
\pgfpathlineto{\pgfqpoint{1.445686in}{7.980721in}}%
\pgfpathlineto{\pgfqpoint{1.436164in}{7.955784in}}%
\pgfpathlineto{\pgfqpoint{1.426641in}{7.931702in}}%
\pgfpathlineto{\pgfqpoint{1.417118in}{7.908554in}}%
\pgfpathlineto{\pgfqpoint{1.407596in}{7.886416in}}%
\pgfpathlineto{\pgfqpoint{1.398073in}{7.865361in}}%
\pgfpathlineto{\pgfqpoint{1.388551in}{7.845457in}}%
\pgfpathlineto{\pgfqpoint{1.379028in}{7.826763in}}%
\pgfpathlineto{\pgfqpoint{1.369506in}{7.809325in}}%
\pgfpathlineto{\pgfqpoint{1.359983in}{7.793182in}}%
\pgfpathlineto{\pgfqpoint{1.350460in}{7.778353in}}%
\pgfpathlineto{\pgfqpoint{1.340938in}{7.764847in}}%
\pgfpathlineto{\pgfqpoint{1.331415in}{7.752655in}}%
\pgfpathlineto{\pgfqpoint{1.321893in}{7.741753in}}%
\pgfpathlineto{\pgfqpoint{1.312370in}{7.732102in}}%
\pgfpathlineto{\pgfqpoint{1.302848in}{7.723650in}}%
\pgfpathlineto{\pgfqpoint{1.293325in}{7.716329in}}%
\pgfpathlineto{\pgfqpoint{1.283803in}{7.710062in}}%
\pgfpathlineto{\pgfqpoint{1.274280in}{7.704760in}}%
\pgfpathlineto{\pgfqpoint{1.264757in}{7.700328in}}%
\pgfpathlineto{\pgfqpoint{1.255235in}{7.696665in}}%
\pgfpathlineto{\pgfqpoint{1.245712in}{7.693665in}}%
\pgfpathlineto{\pgfqpoint{1.236190in}{7.691224in}}%
\pgfpathlineto{\pgfqpoint{1.226667in}{7.689237in}}%
\pgfpathlineto{\pgfqpoint{1.217145in}{7.687602in}}%
\pgfpathlineto{\pgfqpoint{1.207622in}{7.686225in}}%
\pgfpathlineto{\pgfqpoint{1.198099in}{7.685016in}}%
\pgfpathlineto{\pgfqpoint{1.188577in}{7.683895in}}%
\pgfpathlineto{\pgfqpoint{1.179054in}{7.682793in}}%
\pgfpathlineto{\pgfqpoint{1.169532in}{7.681649in}}%
\pgfpathlineto{\pgfqpoint{1.160009in}{7.680414in}}%
\pgfpathlineto{\pgfqpoint{1.150487in}{7.679051in}}%
\pgfpathlineto{\pgfqpoint{1.140964in}{7.677535in}}%
\pgfpathlineto{\pgfqpoint{1.131442in}{7.675850in}}%
\pgfpathlineto{\pgfqpoint{1.121919in}{7.673990in}}%
\pgfpathlineto{\pgfqpoint{1.112396in}{7.671959in}}%
\pgfpathlineto{\pgfqpoint{1.102874in}{7.669769in}}%
\pgfpathlineto{\pgfqpoint{1.093351in}{7.667437in}}%
\pgfpathlineto{\pgfqpoint{1.083829in}{7.664988in}}%
\pgfpathlineto{\pgfqpoint{1.074306in}{7.662449in}}%
\pgfpathlineto{\pgfqpoint{1.064784in}{7.659849in}}%
\pgfpathlineto{\pgfqpoint{1.055261in}{7.657219in}}%
\pgfpathlineto{\pgfqpoint{1.045738in}{7.654589in}}%
\pgfpathlineto{\pgfqpoint{1.036216in}{7.651991in}}%
\pgfpathlineto{\pgfqpoint{1.026693in}{7.649451in}}%
\pgfpathlineto{\pgfqpoint{1.017171in}{7.646993in}}%
\pgfpathlineto{\pgfqpoint{1.007648in}{7.644641in}}%
\pgfpathlineto{\pgfqpoint{0.998126in}{7.642410in}}%
\pgfpathlineto{\pgfqpoint{0.988603in}{7.640315in}}%
\pgfpathlineto{\pgfqpoint{0.979081in}{7.638367in}}%
\pgfpathlineto{\pgfqpoint{0.969558in}{7.636572in}}%
\pgfpathlineto{\pgfqpoint{0.960035in}{7.634932in}}%
\pgfpathlineto{\pgfqpoint{0.950513in}{7.633447in}}%
\pgfpathlineto{\pgfqpoint{0.940990in}{7.632114in}}%
\pgfpathlineto{\pgfqpoint{0.931468in}{7.630927in}}%
\pgfpathlineto{\pgfqpoint{0.921945in}{7.629880in}}%
\pgfpathlineto{\pgfqpoint{0.912423in}{7.628963in}}%
\pgfpathlineto{\pgfqpoint{0.902900in}{7.628167in}}%
\pgfpathlineto{\pgfqpoint{0.893377in}{7.627481in}}%
\pgfpathlineto{\pgfqpoint{0.883855in}{7.626896in}}%
\pgfpathlineto{\pgfqpoint{0.874332in}{7.626399in}}%
\pgfpathlineto{\pgfqpoint{0.864810in}{7.625981in}}%
\pgfpathlineto{\pgfqpoint{0.855287in}{7.625632in}}%
\pgfpathlineto{\pgfqpoint{0.845765in}{7.625343in}}%
\pgfpathlineto{\pgfqpoint{0.836242in}{7.625106in}}%
\pgfpathlineto{\pgfqpoint{0.826720in}{7.624912in}}%
\pgfpathlineto{\pgfqpoint{0.817197in}{7.624755in}}%
\pgfpathlineto{\pgfqpoint{0.817197in}{7.624755in}}%
\pgfpathclose%
\pgfusepath{stroke,fill}%
}%
\begin{pgfscope}%
\pgfsys@transformshift{0.000000in}{0.000000in}%
\pgfsys@useobject{currentmarker}{}%
\end{pgfscope}%
\end{pgfscope}%
\begin{pgfscope}%
\pgfpathrectangle{\pgfqpoint{0.633874in}{7.624184in}}{\pgfqpoint{2.177280in}{2.201755in}}%
\pgfusepath{clip}%
\pgfsetbuttcap%
\pgfsetroundjoin%
\definecolor{currentfill}{rgb}{1.000000,0.498039,0.054902}%
\pgfsetfillcolor{currentfill}%
\pgfsetfillopacity{0.250000}%
\pgfsetlinewidth{1.003750pt}%
\definecolor{currentstroke}{rgb}{1.000000,0.498039,0.054902}%
\pgfsetstrokecolor{currentstroke}%
\pgfsetdash{}{0pt}%
\pgfsys@defobject{currentmarker}{\pgfqpoint{0.882788in}{7.624184in}}{\pgfqpoint{2.287147in}{8.956540in}}{%
\pgfpathmoveto{\pgfqpoint{0.882788in}{7.625295in}}%
\pgfpathlineto{\pgfqpoint{0.882788in}{7.624184in}}%
\pgfpathlineto{\pgfqpoint{0.889845in}{7.624184in}}%
\pgfpathlineto{\pgfqpoint{0.896902in}{7.624184in}}%
\pgfpathlineto{\pgfqpoint{0.903959in}{7.624184in}}%
\pgfpathlineto{\pgfqpoint{0.911017in}{7.624184in}}%
\pgfpathlineto{\pgfqpoint{0.918074in}{7.624184in}}%
\pgfpathlineto{\pgfqpoint{0.925131in}{7.624184in}}%
\pgfpathlineto{\pgfqpoint{0.932188in}{7.624184in}}%
\pgfpathlineto{\pgfqpoint{0.939245in}{7.624184in}}%
\pgfpathlineto{\pgfqpoint{0.946302in}{7.624184in}}%
\pgfpathlineto{\pgfqpoint{0.953359in}{7.624184in}}%
\pgfpathlineto{\pgfqpoint{0.960416in}{7.624184in}}%
\pgfpathlineto{\pgfqpoint{0.967473in}{7.624184in}}%
\pgfpathlineto{\pgfqpoint{0.974530in}{7.624184in}}%
\pgfpathlineto{\pgfqpoint{0.981587in}{7.624184in}}%
\pgfpathlineto{\pgfqpoint{0.988644in}{7.624184in}}%
\pgfpathlineto{\pgfqpoint{0.995702in}{7.624184in}}%
\pgfpathlineto{\pgfqpoint{1.002759in}{7.624184in}}%
\pgfpathlineto{\pgfqpoint{1.009816in}{7.624184in}}%
\pgfpathlineto{\pgfqpoint{1.016873in}{7.624184in}}%
\pgfpathlineto{\pgfqpoint{1.023930in}{7.624184in}}%
\pgfpathlineto{\pgfqpoint{1.030987in}{7.624184in}}%
\pgfpathlineto{\pgfqpoint{1.038044in}{7.624184in}}%
\pgfpathlineto{\pgfqpoint{1.045101in}{7.624184in}}%
\pgfpathlineto{\pgfqpoint{1.052158in}{7.624184in}}%
\pgfpathlineto{\pgfqpoint{1.059215in}{7.624184in}}%
\pgfpathlineto{\pgfqpoint{1.066272in}{7.624184in}}%
\pgfpathlineto{\pgfqpoint{1.073329in}{7.624184in}}%
\pgfpathlineto{\pgfqpoint{1.080387in}{7.624184in}}%
\pgfpathlineto{\pgfqpoint{1.087444in}{7.624184in}}%
\pgfpathlineto{\pgfqpoint{1.094501in}{7.624184in}}%
\pgfpathlineto{\pgfqpoint{1.101558in}{7.624184in}}%
\pgfpathlineto{\pgfqpoint{1.108615in}{7.624184in}}%
\pgfpathlineto{\pgfqpoint{1.115672in}{7.624184in}}%
\pgfpathlineto{\pgfqpoint{1.122729in}{7.624184in}}%
\pgfpathlineto{\pgfqpoint{1.129786in}{7.624184in}}%
\pgfpathlineto{\pgfqpoint{1.136843in}{7.624184in}}%
\pgfpathlineto{\pgfqpoint{1.143900in}{7.624184in}}%
\pgfpathlineto{\pgfqpoint{1.150957in}{7.624184in}}%
\pgfpathlineto{\pgfqpoint{1.158014in}{7.624184in}}%
\pgfpathlineto{\pgfqpoint{1.165071in}{7.624184in}}%
\pgfpathlineto{\pgfqpoint{1.172129in}{7.624184in}}%
\pgfpathlineto{\pgfqpoint{1.179186in}{7.624184in}}%
\pgfpathlineto{\pgfqpoint{1.186243in}{7.624184in}}%
\pgfpathlineto{\pgfqpoint{1.193300in}{7.624184in}}%
\pgfpathlineto{\pgfqpoint{1.200357in}{7.624184in}}%
\pgfpathlineto{\pgfqpoint{1.207414in}{7.624184in}}%
\pgfpathlineto{\pgfqpoint{1.214471in}{7.624184in}}%
\pgfpathlineto{\pgfqpoint{1.221528in}{7.624184in}}%
\pgfpathlineto{\pgfqpoint{1.228585in}{7.624184in}}%
\pgfpathlineto{\pgfqpoint{1.235642in}{7.624184in}}%
\pgfpathlineto{\pgfqpoint{1.242699in}{7.624184in}}%
\pgfpathlineto{\pgfqpoint{1.249756in}{7.624184in}}%
\pgfpathlineto{\pgfqpoint{1.256814in}{7.624184in}}%
\pgfpathlineto{\pgfqpoint{1.263871in}{7.624184in}}%
\pgfpathlineto{\pgfqpoint{1.270928in}{7.624184in}}%
\pgfpathlineto{\pgfqpoint{1.277985in}{7.624184in}}%
\pgfpathlineto{\pgfqpoint{1.285042in}{7.624184in}}%
\pgfpathlineto{\pgfqpoint{1.292099in}{7.624184in}}%
\pgfpathlineto{\pgfqpoint{1.299156in}{7.624184in}}%
\pgfpathlineto{\pgfqpoint{1.306213in}{7.624184in}}%
\pgfpathlineto{\pgfqpoint{1.313270in}{7.624184in}}%
\pgfpathlineto{\pgfqpoint{1.320327in}{7.624184in}}%
\pgfpathlineto{\pgfqpoint{1.327384in}{7.624184in}}%
\pgfpathlineto{\pgfqpoint{1.334441in}{7.624184in}}%
\pgfpathlineto{\pgfqpoint{1.341498in}{7.624184in}}%
\pgfpathlineto{\pgfqpoint{1.348556in}{7.624184in}}%
\pgfpathlineto{\pgfqpoint{1.355613in}{7.624184in}}%
\pgfpathlineto{\pgfqpoint{1.362670in}{7.624184in}}%
\pgfpathlineto{\pgfqpoint{1.369727in}{7.624184in}}%
\pgfpathlineto{\pgfqpoint{1.376784in}{7.624184in}}%
\pgfpathlineto{\pgfqpoint{1.383841in}{7.624184in}}%
\pgfpathlineto{\pgfqpoint{1.390898in}{7.624184in}}%
\pgfpathlineto{\pgfqpoint{1.397955in}{7.624184in}}%
\pgfpathlineto{\pgfqpoint{1.405012in}{7.624184in}}%
\pgfpathlineto{\pgfqpoint{1.412069in}{7.624184in}}%
\pgfpathlineto{\pgfqpoint{1.419126in}{7.624184in}}%
\pgfpathlineto{\pgfqpoint{1.426183in}{7.624184in}}%
\pgfpathlineto{\pgfqpoint{1.433241in}{7.624184in}}%
\pgfpathlineto{\pgfqpoint{1.440298in}{7.624184in}}%
\pgfpathlineto{\pgfqpoint{1.447355in}{7.624184in}}%
\pgfpathlineto{\pgfqpoint{1.454412in}{7.624184in}}%
\pgfpathlineto{\pgfqpoint{1.461469in}{7.624184in}}%
\pgfpathlineto{\pgfqpoint{1.468526in}{7.624184in}}%
\pgfpathlineto{\pgfqpoint{1.475583in}{7.624184in}}%
\pgfpathlineto{\pgfqpoint{1.482640in}{7.624184in}}%
\pgfpathlineto{\pgfqpoint{1.489697in}{7.624184in}}%
\pgfpathlineto{\pgfqpoint{1.496754in}{7.624184in}}%
\pgfpathlineto{\pgfqpoint{1.503811in}{7.624184in}}%
\pgfpathlineto{\pgfqpoint{1.510868in}{7.624184in}}%
\pgfpathlineto{\pgfqpoint{1.517926in}{7.624184in}}%
\pgfpathlineto{\pgfqpoint{1.524983in}{7.624184in}}%
\pgfpathlineto{\pgfqpoint{1.532040in}{7.624184in}}%
\pgfpathlineto{\pgfqpoint{1.539097in}{7.624184in}}%
\pgfpathlineto{\pgfqpoint{1.546154in}{7.624184in}}%
\pgfpathlineto{\pgfqpoint{1.553211in}{7.624184in}}%
\pgfpathlineto{\pgfqpoint{1.560268in}{7.624184in}}%
\pgfpathlineto{\pgfqpoint{1.567325in}{7.624184in}}%
\pgfpathlineto{\pgfqpoint{1.574382in}{7.624184in}}%
\pgfpathlineto{\pgfqpoint{1.581439in}{7.624184in}}%
\pgfpathlineto{\pgfqpoint{1.588496in}{7.624184in}}%
\pgfpathlineto{\pgfqpoint{1.595553in}{7.624184in}}%
\pgfpathlineto{\pgfqpoint{1.602610in}{7.624184in}}%
\pgfpathlineto{\pgfqpoint{1.609668in}{7.624184in}}%
\pgfpathlineto{\pgfqpoint{1.616725in}{7.624184in}}%
\pgfpathlineto{\pgfqpoint{1.623782in}{7.624184in}}%
\pgfpathlineto{\pgfqpoint{1.630839in}{7.624184in}}%
\pgfpathlineto{\pgfqpoint{1.637896in}{7.624184in}}%
\pgfpathlineto{\pgfqpoint{1.644953in}{7.624184in}}%
\pgfpathlineto{\pgfqpoint{1.652010in}{7.624184in}}%
\pgfpathlineto{\pgfqpoint{1.659067in}{7.624184in}}%
\pgfpathlineto{\pgfqpoint{1.666124in}{7.624184in}}%
\pgfpathlineto{\pgfqpoint{1.673181in}{7.624184in}}%
\pgfpathlineto{\pgfqpoint{1.680238in}{7.624184in}}%
\pgfpathlineto{\pgfqpoint{1.687295in}{7.624184in}}%
\pgfpathlineto{\pgfqpoint{1.694353in}{7.624184in}}%
\pgfpathlineto{\pgfqpoint{1.701410in}{7.624184in}}%
\pgfpathlineto{\pgfqpoint{1.708467in}{7.624184in}}%
\pgfpathlineto{\pgfqpoint{1.715524in}{7.624184in}}%
\pgfpathlineto{\pgfqpoint{1.722581in}{7.624184in}}%
\pgfpathlineto{\pgfqpoint{1.729638in}{7.624184in}}%
\pgfpathlineto{\pgfqpoint{1.736695in}{7.624184in}}%
\pgfpathlineto{\pgfqpoint{1.743752in}{7.624184in}}%
\pgfpathlineto{\pgfqpoint{1.750809in}{7.624184in}}%
\pgfpathlineto{\pgfqpoint{1.757866in}{7.624184in}}%
\pgfpathlineto{\pgfqpoint{1.764923in}{7.624184in}}%
\pgfpathlineto{\pgfqpoint{1.771980in}{7.624184in}}%
\pgfpathlineto{\pgfqpoint{1.779037in}{7.624184in}}%
\pgfpathlineto{\pgfqpoint{1.786095in}{7.624184in}}%
\pgfpathlineto{\pgfqpoint{1.793152in}{7.624184in}}%
\pgfpathlineto{\pgfqpoint{1.800209in}{7.624184in}}%
\pgfpathlineto{\pgfqpoint{1.807266in}{7.624184in}}%
\pgfpathlineto{\pgfqpoint{1.814323in}{7.624184in}}%
\pgfpathlineto{\pgfqpoint{1.821380in}{7.624184in}}%
\pgfpathlineto{\pgfqpoint{1.828437in}{7.624184in}}%
\pgfpathlineto{\pgfqpoint{1.835494in}{7.624184in}}%
\pgfpathlineto{\pgfqpoint{1.842551in}{7.624184in}}%
\pgfpathlineto{\pgfqpoint{1.849608in}{7.624184in}}%
\pgfpathlineto{\pgfqpoint{1.856665in}{7.624184in}}%
\pgfpathlineto{\pgfqpoint{1.863722in}{7.624184in}}%
\pgfpathlineto{\pgfqpoint{1.870780in}{7.624184in}}%
\pgfpathlineto{\pgfqpoint{1.877837in}{7.624184in}}%
\pgfpathlineto{\pgfqpoint{1.884894in}{7.624184in}}%
\pgfpathlineto{\pgfqpoint{1.891951in}{7.624184in}}%
\pgfpathlineto{\pgfqpoint{1.899008in}{7.624184in}}%
\pgfpathlineto{\pgfqpoint{1.906065in}{7.624184in}}%
\pgfpathlineto{\pgfqpoint{1.913122in}{7.624184in}}%
\pgfpathlineto{\pgfqpoint{1.920179in}{7.624184in}}%
\pgfpathlineto{\pgfqpoint{1.927236in}{7.624184in}}%
\pgfpathlineto{\pgfqpoint{1.934293in}{7.624184in}}%
\pgfpathlineto{\pgfqpoint{1.941350in}{7.624184in}}%
\pgfpathlineto{\pgfqpoint{1.948407in}{7.624184in}}%
\pgfpathlineto{\pgfqpoint{1.955465in}{7.624184in}}%
\pgfpathlineto{\pgfqpoint{1.962522in}{7.624184in}}%
\pgfpathlineto{\pgfqpoint{1.969579in}{7.624184in}}%
\pgfpathlineto{\pgfqpoint{1.976636in}{7.624184in}}%
\pgfpathlineto{\pgfqpoint{1.983693in}{7.624184in}}%
\pgfpathlineto{\pgfqpoint{1.990750in}{7.624184in}}%
\pgfpathlineto{\pgfqpoint{1.997807in}{7.624184in}}%
\pgfpathlineto{\pgfqpoint{2.004864in}{7.624184in}}%
\pgfpathlineto{\pgfqpoint{2.011921in}{7.624184in}}%
\pgfpathlineto{\pgfqpoint{2.018978in}{7.624184in}}%
\pgfpathlineto{\pgfqpoint{2.026035in}{7.624184in}}%
\pgfpathlineto{\pgfqpoint{2.033092in}{7.624184in}}%
\pgfpathlineto{\pgfqpoint{2.040149in}{7.624184in}}%
\pgfpathlineto{\pgfqpoint{2.047207in}{7.624184in}}%
\pgfpathlineto{\pgfqpoint{2.054264in}{7.624184in}}%
\pgfpathlineto{\pgfqpoint{2.061321in}{7.624184in}}%
\pgfpathlineto{\pgfqpoint{2.068378in}{7.624184in}}%
\pgfpathlineto{\pgfqpoint{2.075435in}{7.624184in}}%
\pgfpathlineto{\pgfqpoint{2.082492in}{7.624184in}}%
\pgfpathlineto{\pgfqpoint{2.089549in}{7.624184in}}%
\pgfpathlineto{\pgfqpoint{2.096606in}{7.624184in}}%
\pgfpathlineto{\pgfqpoint{2.103663in}{7.624184in}}%
\pgfpathlineto{\pgfqpoint{2.110720in}{7.624184in}}%
\pgfpathlineto{\pgfqpoint{2.117777in}{7.624184in}}%
\pgfpathlineto{\pgfqpoint{2.124834in}{7.624184in}}%
\pgfpathlineto{\pgfqpoint{2.131892in}{7.624184in}}%
\pgfpathlineto{\pgfqpoint{2.138949in}{7.624184in}}%
\pgfpathlineto{\pgfqpoint{2.146006in}{7.624184in}}%
\pgfpathlineto{\pgfqpoint{2.153063in}{7.624184in}}%
\pgfpathlineto{\pgfqpoint{2.160120in}{7.624184in}}%
\pgfpathlineto{\pgfqpoint{2.167177in}{7.624184in}}%
\pgfpathlineto{\pgfqpoint{2.174234in}{7.624184in}}%
\pgfpathlineto{\pgfqpoint{2.181291in}{7.624184in}}%
\pgfpathlineto{\pgfqpoint{2.188348in}{7.624184in}}%
\pgfpathlineto{\pgfqpoint{2.195405in}{7.624184in}}%
\pgfpathlineto{\pgfqpoint{2.202462in}{7.624184in}}%
\pgfpathlineto{\pgfqpoint{2.209519in}{7.624184in}}%
\pgfpathlineto{\pgfqpoint{2.216577in}{7.624184in}}%
\pgfpathlineto{\pgfqpoint{2.223634in}{7.624184in}}%
\pgfpathlineto{\pgfqpoint{2.230691in}{7.624184in}}%
\pgfpathlineto{\pgfqpoint{2.237748in}{7.624184in}}%
\pgfpathlineto{\pgfqpoint{2.244805in}{7.624184in}}%
\pgfpathlineto{\pgfqpoint{2.251862in}{7.624184in}}%
\pgfpathlineto{\pgfqpoint{2.258919in}{7.624184in}}%
\pgfpathlineto{\pgfqpoint{2.265976in}{7.624184in}}%
\pgfpathlineto{\pgfqpoint{2.273033in}{7.624184in}}%
\pgfpathlineto{\pgfqpoint{2.280090in}{7.624184in}}%
\pgfpathlineto{\pgfqpoint{2.287147in}{7.624184in}}%
\pgfpathlineto{\pgfqpoint{2.287147in}{7.625130in}}%
\pgfpathlineto{\pgfqpoint{2.287147in}{7.625130in}}%
\pgfpathlineto{\pgfqpoint{2.280090in}{7.625378in}}%
\pgfpathlineto{\pgfqpoint{2.273033in}{7.625684in}}%
\pgfpathlineto{\pgfqpoint{2.265976in}{7.626059in}}%
\pgfpathlineto{\pgfqpoint{2.258919in}{7.626515in}}%
\pgfpathlineto{\pgfqpoint{2.251862in}{7.627067in}}%
\pgfpathlineto{\pgfqpoint{2.244805in}{7.627734in}}%
\pgfpathlineto{\pgfqpoint{2.237748in}{7.628532in}}%
\pgfpathlineto{\pgfqpoint{2.230691in}{7.629483in}}%
\pgfpathlineto{\pgfqpoint{2.223634in}{7.630612in}}%
\pgfpathlineto{\pgfqpoint{2.216577in}{7.631942in}}%
\pgfpathlineto{\pgfqpoint{2.209519in}{7.633503in}}%
\pgfpathlineto{\pgfqpoint{2.202462in}{7.635325in}}%
\pgfpathlineto{\pgfqpoint{2.195405in}{7.637440in}}%
\pgfpathlineto{\pgfqpoint{2.188348in}{7.639883in}}%
\pgfpathlineto{\pgfqpoint{2.181291in}{7.642689in}}%
\pgfpathlineto{\pgfqpoint{2.174234in}{7.645897in}}%
\pgfpathlineto{\pgfqpoint{2.167177in}{7.649546in}}%
\pgfpathlineto{\pgfqpoint{2.160120in}{7.653675in}}%
\pgfpathlineto{\pgfqpoint{2.153063in}{7.658323in}}%
\pgfpathlineto{\pgfqpoint{2.146006in}{7.663531in}}%
\pgfpathlineto{\pgfqpoint{2.138949in}{7.669337in}}%
\pgfpathlineto{\pgfqpoint{2.131892in}{7.675779in}}%
\pgfpathlineto{\pgfqpoint{2.124834in}{7.682891in}}%
\pgfpathlineto{\pgfqpoint{2.117777in}{7.690707in}}%
\pgfpathlineto{\pgfqpoint{2.110720in}{7.699254in}}%
\pgfpathlineto{\pgfqpoint{2.103663in}{7.708557in}}%
\pgfpathlineto{\pgfqpoint{2.096606in}{7.718638in}}%
\pgfpathlineto{\pgfqpoint{2.089549in}{7.729510in}}%
\pgfpathlineto{\pgfqpoint{2.082492in}{7.741182in}}%
\pgfpathlineto{\pgfqpoint{2.075435in}{7.753657in}}%
\pgfpathlineto{\pgfqpoint{2.068378in}{7.766929in}}%
\pgfpathlineto{\pgfqpoint{2.061321in}{7.780987in}}%
\pgfpathlineto{\pgfqpoint{2.054264in}{7.795813in}}%
\pgfpathlineto{\pgfqpoint{2.047207in}{7.811378in}}%
\pgfpathlineto{\pgfqpoint{2.040149in}{7.827648in}}%
\pgfpathlineto{\pgfqpoint{2.033092in}{7.844581in}}%
\pgfpathlineto{\pgfqpoint{2.026035in}{7.862127in}}%
\pgfpathlineto{\pgfqpoint{2.018978in}{7.880230in}}%
\pgfpathlineto{\pgfqpoint{2.011921in}{7.898827in}}%
\pgfpathlineto{\pgfqpoint{2.004864in}{7.917848in}}%
\pgfpathlineto{\pgfqpoint{1.997807in}{7.937219in}}%
\pgfpathlineto{\pgfqpoint{1.990750in}{7.956862in}}%
\pgfpathlineto{\pgfqpoint{1.983693in}{7.976695in}}%
\pgfpathlineto{\pgfqpoint{1.976636in}{7.996635in}}%
\pgfpathlineto{\pgfqpoint{1.969579in}{8.016598in}}%
\pgfpathlineto{\pgfqpoint{1.962522in}{8.036500in}}%
\pgfpathlineto{\pgfqpoint{1.955465in}{8.056261in}}%
\pgfpathlineto{\pgfqpoint{1.948407in}{8.075803in}}%
\pgfpathlineto{\pgfqpoint{1.941350in}{8.095055in}}%
\pgfpathlineto{\pgfqpoint{1.934293in}{8.113952in}}%
\pgfpathlineto{\pgfqpoint{1.927236in}{8.132438in}}%
\pgfpathlineto{\pgfqpoint{1.920179in}{8.150464in}}%
\pgfpathlineto{\pgfqpoint{1.913122in}{8.167996in}}%
\pgfpathlineto{\pgfqpoint{1.906065in}{8.185007in}}%
\pgfpathlineto{\pgfqpoint{1.899008in}{8.201486in}}%
\pgfpathlineto{\pgfqpoint{1.891951in}{8.217434in}}%
\pgfpathlineto{\pgfqpoint{1.884894in}{8.232862in}}%
\pgfpathlineto{\pgfqpoint{1.877837in}{8.247796in}}%
\pgfpathlineto{\pgfqpoint{1.870780in}{8.262273in}}%
\pgfpathlineto{\pgfqpoint{1.863722in}{8.276340in}}%
\pgfpathlineto{\pgfqpoint{1.856665in}{8.290055in}}%
\pgfpathlineto{\pgfqpoint{1.849608in}{8.303482in}}%
\pgfpathlineto{\pgfqpoint{1.842551in}{8.316692in}}%
\pgfpathlineto{\pgfqpoint{1.835494in}{8.329760in}}%
\pgfpathlineto{\pgfqpoint{1.828437in}{8.342762in}}%
\pgfpathlineto{\pgfqpoint{1.821380in}{8.355774in}}%
\pgfpathlineto{\pgfqpoint{1.814323in}{8.368870in}}%
\pgfpathlineto{\pgfqpoint{1.807266in}{8.382118in}}%
\pgfpathlineto{\pgfqpoint{1.800209in}{8.395580in}}%
\pgfpathlineto{\pgfqpoint{1.793152in}{8.409308in}}%
\pgfpathlineto{\pgfqpoint{1.786095in}{8.423346in}}%
\pgfpathlineto{\pgfqpoint{1.779037in}{8.437726in}}%
\pgfpathlineto{\pgfqpoint{1.771980in}{8.452467in}}%
\pgfpathlineto{\pgfqpoint{1.764923in}{8.467578in}}%
\pgfpathlineto{\pgfqpoint{1.757866in}{8.483053in}}%
\pgfpathlineto{\pgfqpoint{1.750809in}{8.498877in}}%
\pgfpathlineto{\pgfqpoint{1.743752in}{8.515022in}}%
\pgfpathlineto{\pgfqpoint{1.736695in}{8.531451in}}%
\pgfpathlineto{\pgfqpoint{1.729638in}{8.548118in}}%
\pgfpathlineto{\pgfqpoint{1.722581in}{8.564972in}}%
\pgfpathlineto{\pgfqpoint{1.715524in}{8.581955in}}%
\pgfpathlineto{\pgfqpoint{1.708467in}{8.599008in}}%
\pgfpathlineto{\pgfqpoint{1.701410in}{8.616072in}}%
\pgfpathlineto{\pgfqpoint{1.694353in}{8.633087in}}%
\pgfpathlineto{\pgfqpoint{1.687295in}{8.649998in}}%
\pgfpathlineto{\pgfqpoint{1.680238in}{8.666755in}}%
\pgfpathlineto{\pgfqpoint{1.673181in}{8.683313in}}%
\pgfpathlineto{\pgfqpoint{1.666124in}{8.699635in}}%
\pgfpathlineto{\pgfqpoint{1.659067in}{8.715691in}}%
\pgfpathlineto{\pgfqpoint{1.652010in}{8.731459in}}%
\pgfpathlineto{\pgfqpoint{1.644953in}{8.746924in}}%
\pgfpathlineto{\pgfqpoint{1.637896in}{8.762078in}}%
\pgfpathlineto{\pgfqpoint{1.630839in}{8.776917in}}%
\pgfpathlineto{\pgfqpoint{1.623782in}{8.791441in}}%
\pgfpathlineto{\pgfqpoint{1.616725in}{8.805650in}}%
\pgfpathlineto{\pgfqpoint{1.609668in}{8.819543in}}%
\pgfpathlineto{\pgfqpoint{1.602610in}{8.833115in}}%
\pgfpathlineto{\pgfqpoint{1.595553in}{8.846351in}}%
\pgfpathlineto{\pgfqpoint{1.588496in}{8.859231in}}%
\pgfpathlineto{\pgfqpoint{1.581439in}{8.871717in}}%
\pgfpathlineto{\pgfqpoint{1.574382in}{8.883761in}}%
\pgfpathlineto{\pgfqpoint{1.567325in}{8.895295in}}%
\pgfpathlineto{\pgfqpoint{1.560268in}{8.906234in}}%
\pgfpathlineto{\pgfqpoint{1.553211in}{8.916476in}}%
\pgfpathlineto{\pgfqpoint{1.546154in}{8.925898in}}%
\pgfpathlineto{\pgfqpoint{1.539097in}{8.934363in}}%
\pgfpathlineto{\pgfqpoint{1.532040in}{8.941715in}}%
\pgfpathlineto{\pgfqpoint{1.524983in}{8.947788in}}%
\pgfpathlineto{\pgfqpoint{1.517926in}{8.952404in}}%
\pgfpathlineto{\pgfqpoint{1.510868in}{8.955382in}}%
\pgfpathlineto{\pgfqpoint{1.503811in}{8.956540in}}%
\pgfpathlineto{\pgfqpoint{1.496754in}{8.955701in}}%
\pgfpathlineto{\pgfqpoint{1.489697in}{8.952696in}}%
\pgfpathlineto{\pgfqpoint{1.482640in}{8.947374in}}%
\pgfpathlineto{\pgfqpoint{1.475583in}{8.939606in}}%
\pgfpathlineto{\pgfqpoint{1.468526in}{8.929286in}}%
\pgfpathlineto{\pgfqpoint{1.461469in}{8.916343in}}%
\pgfpathlineto{\pgfqpoint{1.454412in}{8.900738in}}%
\pgfpathlineto{\pgfqpoint{1.447355in}{8.882472in}}%
\pgfpathlineto{\pgfqpoint{1.440298in}{8.861584in}}%
\pgfpathlineto{\pgfqpoint{1.433241in}{8.838156in}}%
\pgfpathlineto{\pgfqpoint{1.426183in}{8.812311in}}%
\pgfpathlineto{\pgfqpoint{1.419126in}{8.784209in}}%
\pgfpathlineto{\pgfqpoint{1.412069in}{8.754051in}}%
\pgfpathlineto{\pgfqpoint{1.405012in}{8.722067in}}%
\pgfpathlineto{\pgfqpoint{1.397955in}{8.688518in}}%
\pgfpathlineto{\pgfqpoint{1.390898in}{8.653688in}}%
\pgfpathlineto{\pgfqpoint{1.383841in}{8.617876in}}%
\pgfpathlineto{\pgfqpoint{1.376784in}{8.581393in}}%
\pgfpathlineto{\pgfqpoint{1.369727in}{8.544552in}}%
\pgfpathlineto{\pgfqpoint{1.362670in}{8.507662in}}%
\pgfpathlineto{\pgfqpoint{1.355613in}{8.471022in}}%
\pgfpathlineto{\pgfqpoint{1.348556in}{8.434912in}}%
\pgfpathlineto{\pgfqpoint{1.341498in}{8.399591in}}%
\pgfpathlineto{\pgfqpoint{1.334441in}{8.365288in}}%
\pgfpathlineto{\pgfqpoint{1.327384in}{8.332201in}}%
\pgfpathlineto{\pgfqpoint{1.320327in}{8.300489in}}%
\pgfpathlineto{\pgfqpoint{1.313270in}{8.270278in}}%
\pgfpathlineto{\pgfqpoint{1.306213in}{8.241651in}}%
\pgfpathlineto{\pgfqpoint{1.299156in}{8.214654in}}%
\pgfpathlineto{\pgfqpoint{1.292099in}{8.189298in}}%
\pgfpathlineto{\pgfqpoint{1.285042in}{8.165555in}}%
\pgfpathlineto{\pgfqpoint{1.277985in}{8.143367in}}%
\pgfpathlineto{\pgfqpoint{1.270928in}{8.122649in}}%
\pgfpathlineto{\pgfqpoint{1.263871in}{8.103290in}}%
\pgfpathlineto{\pgfqpoint{1.256814in}{8.085162in}}%
\pgfpathlineto{\pgfqpoint{1.249756in}{8.068124in}}%
\pgfpathlineto{\pgfqpoint{1.242699in}{8.052025in}}%
\pgfpathlineto{\pgfqpoint{1.235642in}{8.036713in}}%
\pgfpathlineto{\pgfqpoint{1.228585in}{8.022035in}}%
\pgfpathlineto{\pgfqpoint{1.221528in}{8.007847in}}%
\pgfpathlineto{\pgfqpoint{1.214471in}{7.994014in}}%
\pgfpathlineto{\pgfqpoint{1.207414in}{7.980414in}}%
\pgfpathlineto{\pgfqpoint{1.200357in}{7.966943in}}%
\pgfpathlineto{\pgfqpoint{1.193300in}{7.953513in}}%
\pgfpathlineto{\pgfqpoint{1.186243in}{7.940056in}}%
\pgfpathlineto{\pgfqpoint{1.179186in}{7.926527in}}%
\pgfpathlineto{\pgfqpoint{1.172129in}{7.912896in}}%
\pgfpathlineto{\pgfqpoint{1.165071in}{7.899156in}}%
\pgfpathlineto{\pgfqpoint{1.158014in}{7.885315in}}%
\pgfpathlineto{\pgfqpoint{1.150957in}{7.871398in}}%
\pgfpathlineto{\pgfqpoint{1.143900in}{7.857444in}}%
\pgfpathlineto{\pgfqpoint{1.136843in}{7.843502in}}%
\pgfpathlineto{\pgfqpoint{1.129786in}{7.829632in}}%
\pgfpathlineto{\pgfqpoint{1.122729in}{7.815899in}}%
\pgfpathlineto{\pgfqpoint{1.115672in}{7.802369in}}%
\pgfpathlineto{\pgfqpoint{1.108615in}{7.789113in}}%
\pgfpathlineto{\pgfqpoint{1.101558in}{7.776197in}}%
\pgfpathlineto{\pgfqpoint{1.094501in}{7.763687in}}%
\pgfpathlineto{\pgfqpoint{1.087444in}{7.751642in}}%
\pgfpathlineto{\pgfqpoint{1.080387in}{7.740113in}}%
\pgfpathlineto{\pgfqpoint{1.073329in}{7.729145in}}%
\pgfpathlineto{\pgfqpoint{1.066272in}{7.718776in}}%
\pgfpathlineto{\pgfqpoint{1.059215in}{7.709032in}}%
\pgfpathlineto{\pgfqpoint{1.052158in}{7.699931in}}%
\pgfpathlineto{\pgfqpoint{1.045101in}{7.691483in}}%
\pgfpathlineto{\pgfqpoint{1.038044in}{7.683689in}}%
\pgfpathlineto{\pgfqpoint{1.030987in}{7.676543in}}%
\pgfpathlineto{\pgfqpoint{1.023930in}{7.670029in}}%
\pgfpathlineto{\pgfqpoint{1.016873in}{7.664128in}}%
\pgfpathlineto{\pgfqpoint{1.009816in}{7.658815in}}%
\pgfpathlineto{\pgfqpoint{1.002759in}{7.654059in}}%
\pgfpathlineto{\pgfqpoint{0.995702in}{7.649827in}}%
\pgfpathlineto{\pgfqpoint{0.988644in}{7.646084in}}%
\pgfpathlineto{\pgfqpoint{0.981587in}{7.642793in}}%
\pgfpathlineto{\pgfqpoint{0.974530in}{7.639917in}}%
\pgfpathlineto{\pgfqpoint{0.967473in}{7.637418in}}%
\pgfpathlineto{\pgfqpoint{0.960416in}{7.635259in}}%
\pgfpathlineto{\pgfqpoint{0.953359in}{7.633405in}}%
\pgfpathlineto{\pgfqpoint{0.946302in}{7.631822in}}%
\pgfpathlineto{\pgfqpoint{0.939245in}{7.630478in}}%
\pgfpathlineto{\pgfqpoint{0.932188in}{7.629344in}}%
\pgfpathlineto{\pgfqpoint{0.925131in}{7.628393in}}%
\pgfpathlineto{\pgfqpoint{0.918074in}{7.627599in}}%
\pgfpathlineto{\pgfqpoint{0.911017in}{7.626940in}}%
\pgfpathlineto{\pgfqpoint{0.903959in}{7.626397in}}%
\pgfpathlineto{\pgfqpoint{0.896902in}{7.625952in}}%
\pgfpathlineto{\pgfqpoint{0.889845in}{7.625589in}}%
\pgfpathlineto{\pgfqpoint{0.882788in}{7.625295in}}%
\pgfpathlineto{\pgfqpoint{0.882788in}{7.625295in}}%
\pgfpathclose%
\pgfusepath{stroke,fill}%
}%
\begin{pgfscope}%
\pgfsys@transformshift{0.000000in}{0.000000in}%
\pgfsys@useobject{currentmarker}{}%
\end{pgfscope}%
\end{pgfscope}%
\begin{pgfscope}%
\pgfpathrectangle{\pgfqpoint{0.633874in}{7.624184in}}{\pgfqpoint{2.177280in}{2.201755in}}%
\pgfusepath{clip}%
\pgfsetbuttcap%
\pgfsetroundjoin%
\definecolor{currentfill}{rgb}{0.121569,0.466667,0.705882}%
\pgfsetfillcolor{currentfill}%
\pgfsetfillopacity{0.250000}%
\pgfsetlinewidth{1.003750pt}%
\definecolor{currentstroke}{rgb}{0.121569,0.466667,0.705882}%
\pgfsetstrokecolor{currentstroke}%
\pgfsetdash{}{0pt}%
\pgfsys@defobject{currentmarker}{\pgfqpoint{0.732841in}{7.624184in}}{\pgfqpoint{1.718198in}{9.721093in}}{%
\pgfpathmoveto{\pgfqpoint{0.732841in}{7.625625in}}%
\pgfpathlineto{\pgfqpoint{0.732841in}{7.624184in}}%
\pgfpathlineto{\pgfqpoint{0.737793in}{7.624184in}}%
\pgfpathlineto{\pgfqpoint{0.742744in}{7.624184in}}%
\pgfpathlineto{\pgfqpoint{0.747696in}{7.624184in}}%
\pgfpathlineto{\pgfqpoint{0.752647in}{7.624184in}}%
\pgfpathlineto{\pgfqpoint{0.757599in}{7.624184in}}%
\pgfpathlineto{\pgfqpoint{0.762550in}{7.624184in}}%
\pgfpathlineto{\pgfqpoint{0.767502in}{7.624184in}}%
\pgfpathlineto{\pgfqpoint{0.772454in}{7.624184in}}%
\pgfpathlineto{\pgfqpoint{0.777405in}{7.624184in}}%
\pgfpathlineto{\pgfqpoint{0.782357in}{7.624184in}}%
\pgfpathlineto{\pgfqpoint{0.787308in}{7.624184in}}%
\pgfpathlineto{\pgfqpoint{0.792260in}{7.624184in}}%
\pgfpathlineto{\pgfqpoint{0.797211in}{7.624184in}}%
\pgfpathlineto{\pgfqpoint{0.802163in}{7.624184in}}%
\pgfpathlineto{\pgfqpoint{0.807114in}{7.624184in}}%
\pgfpathlineto{\pgfqpoint{0.812066in}{7.624184in}}%
\pgfpathlineto{\pgfqpoint{0.817017in}{7.624184in}}%
\pgfpathlineto{\pgfqpoint{0.821969in}{7.624184in}}%
\pgfpathlineto{\pgfqpoint{0.826920in}{7.624184in}}%
\pgfpathlineto{\pgfqpoint{0.831872in}{7.624184in}}%
\pgfpathlineto{\pgfqpoint{0.836824in}{7.624184in}}%
\pgfpathlineto{\pgfqpoint{0.841775in}{7.624184in}}%
\pgfpathlineto{\pgfqpoint{0.846727in}{7.624184in}}%
\pgfpathlineto{\pgfqpoint{0.851678in}{7.624184in}}%
\pgfpathlineto{\pgfqpoint{0.856630in}{7.624184in}}%
\pgfpathlineto{\pgfqpoint{0.861581in}{7.624184in}}%
\pgfpathlineto{\pgfqpoint{0.866533in}{7.624184in}}%
\pgfpathlineto{\pgfqpoint{0.871484in}{7.624184in}}%
\pgfpathlineto{\pgfqpoint{0.876436in}{7.624184in}}%
\pgfpathlineto{\pgfqpoint{0.881387in}{7.624184in}}%
\pgfpathlineto{\pgfqpoint{0.886339in}{7.624184in}}%
\pgfpathlineto{\pgfqpoint{0.891290in}{7.624184in}}%
\pgfpathlineto{\pgfqpoint{0.896242in}{7.624184in}}%
\pgfpathlineto{\pgfqpoint{0.901194in}{7.624184in}}%
\pgfpathlineto{\pgfqpoint{0.906145in}{7.624184in}}%
\pgfpathlineto{\pgfqpoint{0.911097in}{7.624184in}}%
\pgfpathlineto{\pgfqpoint{0.916048in}{7.624184in}}%
\pgfpathlineto{\pgfqpoint{0.921000in}{7.624184in}}%
\pgfpathlineto{\pgfqpoint{0.925951in}{7.624184in}}%
\pgfpathlineto{\pgfqpoint{0.930903in}{7.624184in}}%
\pgfpathlineto{\pgfqpoint{0.935854in}{7.624184in}}%
\pgfpathlineto{\pgfqpoint{0.940806in}{7.624184in}}%
\pgfpathlineto{\pgfqpoint{0.945757in}{7.624184in}}%
\pgfpathlineto{\pgfqpoint{0.950709in}{7.624184in}}%
\pgfpathlineto{\pgfqpoint{0.955661in}{7.624184in}}%
\pgfpathlineto{\pgfqpoint{0.960612in}{7.624184in}}%
\pgfpathlineto{\pgfqpoint{0.965564in}{7.624184in}}%
\pgfpathlineto{\pgfqpoint{0.970515in}{7.624184in}}%
\pgfpathlineto{\pgfqpoint{0.975467in}{7.624184in}}%
\pgfpathlineto{\pgfqpoint{0.980418in}{7.624184in}}%
\pgfpathlineto{\pgfqpoint{0.985370in}{7.624184in}}%
\pgfpathlineto{\pgfqpoint{0.990321in}{7.624184in}}%
\pgfpathlineto{\pgfqpoint{0.995273in}{7.624184in}}%
\pgfpathlineto{\pgfqpoint{1.000224in}{7.624184in}}%
\pgfpathlineto{\pgfqpoint{1.005176in}{7.624184in}}%
\pgfpathlineto{\pgfqpoint{1.010127in}{7.624184in}}%
\pgfpathlineto{\pgfqpoint{1.015079in}{7.624184in}}%
\pgfpathlineto{\pgfqpoint{1.020031in}{7.624184in}}%
\pgfpathlineto{\pgfqpoint{1.024982in}{7.624184in}}%
\pgfpathlineto{\pgfqpoint{1.029934in}{7.624184in}}%
\pgfpathlineto{\pgfqpoint{1.034885in}{7.624184in}}%
\pgfpathlineto{\pgfqpoint{1.039837in}{7.624184in}}%
\pgfpathlineto{\pgfqpoint{1.044788in}{7.624184in}}%
\pgfpathlineto{\pgfqpoint{1.049740in}{7.624184in}}%
\pgfpathlineto{\pgfqpoint{1.054691in}{7.624184in}}%
\pgfpathlineto{\pgfqpoint{1.059643in}{7.624184in}}%
\pgfpathlineto{\pgfqpoint{1.064594in}{7.624184in}}%
\pgfpathlineto{\pgfqpoint{1.069546in}{7.624184in}}%
\pgfpathlineto{\pgfqpoint{1.074497in}{7.624184in}}%
\pgfpathlineto{\pgfqpoint{1.079449in}{7.624184in}}%
\pgfpathlineto{\pgfqpoint{1.084401in}{7.624184in}}%
\pgfpathlineto{\pgfqpoint{1.089352in}{7.624184in}}%
\pgfpathlineto{\pgfqpoint{1.094304in}{7.624184in}}%
\pgfpathlineto{\pgfqpoint{1.099255in}{7.624184in}}%
\pgfpathlineto{\pgfqpoint{1.104207in}{7.624184in}}%
\pgfpathlineto{\pgfqpoint{1.109158in}{7.624184in}}%
\pgfpathlineto{\pgfqpoint{1.114110in}{7.624184in}}%
\pgfpathlineto{\pgfqpoint{1.119061in}{7.624184in}}%
\pgfpathlineto{\pgfqpoint{1.124013in}{7.624184in}}%
\pgfpathlineto{\pgfqpoint{1.128964in}{7.624184in}}%
\pgfpathlineto{\pgfqpoint{1.133916in}{7.624184in}}%
\pgfpathlineto{\pgfqpoint{1.138867in}{7.624184in}}%
\pgfpathlineto{\pgfqpoint{1.143819in}{7.624184in}}%
\pgfpathlineto{\pgfqpoint{1.148771in}{7.624184in}}%
\pgfpathlineto{\pgfqpoint{1.153722in}{7.624184in}}%
\pgfpathlineto{\pgfqpoint{1.158674in}{7.624184in}}%
\pgfpathlineto{\pgfqpoint{1.163625in}{7.624184in}}%
\pgfpathlineto{\pgfqpoint{1.168577in}{7.624184in}}%
\pgfpathlineto{\pgfqpoint{1.173528in}{7.624184in}}%
\pgfpathlineto{\pgfqpoint{1.178480in}{7.624184in}}%
\pgfpathlineto{\pgfqpoint{1.183431in}{7.624184in}}%
\pgfpathlineto{\pgfqpoint{1.188383in}{7.624184in}}%
\pgfpathlineto{\pgfqpoint{1.193334in}{7.624184in}}%
\pgfpathlineto{\pgfqpoint{1.198286in}{7.624184in}}%
\pgfpathlineto{\pgfqpoint{1.203238in}{7.624184in}}%
\pgfpathlineto{\pgfqpoint{1.208189in}{7.624184in}}%
\pgfpathlineto{\pgfqpoint{1.213141in}{7.624184in}}%
\pgfpathlineto{\pgfqpoint{1.218092in}{7.624184in}}%
\pgfpathlineto{\pgfqpoint{1.223044in}{7.624184in}}%
\pgfpathlineto{\pgfqpoint{1.227995in}{7.624184in}}%
\pgfpathlineto{\pgfqpoint{1.232947in}{7.624184in}}%
\pgfpathlineto{\pgfqpoint{1.237898in}{7.624184in}}%
\pgfpathlineto{\pgfqpoint{1.242850in}{7.624184in}}%
\pgfpathlineto{\pgfqpoint{1.247801in}{7.624184in}}%
\pgfpathlineto{\pgfqpoint{1.252753in}{7.624184in}}%
\pgfpathlineto{\pgfqpoint{1.257704in}{7.624184in}}%
\pgfpathlineto{\pgfqpoint{1.262656in}{7.624184in}}%
\pgfpathlineto{\pgfqpoint{1.267608in}{7.624184in}}%
\pgfpathlineto{\pgfqpoint{1.272559in}{7.624184in}}%
\pgfpathlineto{\pgfqpoint{1.277511in}{7.624184in}}%
\pgfpathlineto{\pgfqpoint{1.282462in}{7.624184in}}%
\pgfpathlineto{\pgfqpoint{1.287414in}{7.624184in}}%
\pgfpathlineto{\pgfqpoint{1.292365in}{7.624184in}}%
\pgfpathlineto{\pgfqpoint{1.297317in}{7.624184in}}%
\pgfpathlineto{\pgfqpoint{1.302268in}{7.624184in}}%
\pgfpathlineto{\pgfqpoint{1.307220in}{7.624184in}}%
\pgfpathlineto{\pgfqpoint{1.312171in}{7.624184in}}%
\pgfpathlineto{\pgfqpoint{1.317123in}{7.624184in}}%
\pgfpathlineto{\pgfqpoint{1.322074in}{7.624184in}}%
\pgfpathlineto{\pgfqpoint{1.327026in}{7.624184in}}%
\pgfpathlineto{\pgfqpoint{1.331978in}{7.624184in}}%
\pgfpathlineto{\pgfqpoint{1.336929in}{7.624184in}}%
\pgfpathlineto{\pgfqpoint{1.341881in}{7.624184in}}%
\pgfpathlineto{\pgfqpoint{1.346832in}{7.624184in}}%
\pgfpathlineto{\pgfqpoint{1.351784in}{7.624184in}}%
\pgfpathlineto{\pgfqpoint{1.356735in}{7.624184in}}%
\pgfpathlineto{\pgfqpoint{1.361687in}{7.624184in}}%
\pgfpathlineto{\pgfqpoint{1.366638in}{7.624184in}}%
\pgfpathlineto{\pgfqpoint{1.371590in}{7.624184in}}%
\pgfpathlineto{\pgfqpoint{1.376541in}{7.624184in}}%
\pgfpathlineto{\pgfqpoint{1.381493in}{7.624184in}}%
\pgfpathlineto{\pgfqpoint{1.386444in}{7.624184in}}%
\pgfpathlineto{\pgfqpoint{1.391396in}{7.624184in}}%
\pgfpathlineto{\pgfqpoint{1.396348in}{7.624184in}}%
\pgfpathlineto{\pgfqpoint{1.401299in}{7.624184in}}%
\pgfpathlineto{\pgfqpoint{1.406251in}{7.624184in}}%
\pgfpathlineto{\pgfqpoint{1.411202in}{7.624184in}}%
\pgfpathlineto{\pgfqpoint{1.416154in}{7.624184in}}%
\pgfpathlineto{\pgfqpoint{1.421105in}{7.624184in}}%
\pgfpathlineto{\pgfqpoint{1.426057in}{7.624184in}}%
\pgfpathlineto{\pgfqpoint{1.431008in}{7.624184in}}%
\pgfpathlineto{\pgfqpoint{1.435960in}{7.624184in}}%
\pgfpathlineto{\pgfqpoint{1.440911in}{7.624184in}}%
\pgfpathlineto{\pgfqpoint{1.445863in}{7.624184in}}%
\pgfpathlineto{\pgfqpoint{1.450815in}{7.624184in}}%
\pgfpathlineto{\pgfqpoint{1.455766in}{7.624184in}}%
\pgfpathlineto{\pgfqpoint{1.460718in}{7.624184in}}%
\pgfpathlineto{\pgfqpoint{1.465669in}{7.624184in}}%
\pgfpathlineto{\pgfqpoint{1.470621in}{7.624184in}}%
\pgfpathlineto{\pgfqpoint{1.475572in}{7.624184in}}%
\pgfpathlineto{\pgfqpoint{1.480524in}{7.624184in}}%
\pgfpathlineto{\pgfqpoint{1.485475in}{7.624184in}}%
\pgfpathlineto{\pgfqpoint{1.490427in}{7.624184in}}%
\pgfpathlineto{\pgfqpoint{1.495378in}{7.624184in}}%
\pgfpathlineto{\pgfqpoint{1.500330in}{7.624184in}}%
\pgfpathlineto{\pgfqpoint{1.505281in}{7.624184in}}%
\pgfpathlineto{\pgfqpoint{1.510233in}{7.624184in}}%
\pgfpathlineto{\pgfqpoint{1.515185in}{7.624184in}}%
\pgfpathlineto{\pgfqpoint{1.520136in}{7.624184in}}%
\pgfpathlineto{\pgfqpoint{1.525088in}{7.624184in}}%
\pgfpathlineto{\pgfqpoint{1.530039in}{7.624184in}}%
\pgfpathlineto{\pgfqpoint{1.534991in}{7.624184in}}%
\pgfpathlineto{\pgfqpoint{1.539942in}{7.624184in}}%
\pgfpathlineto{\pgfqpoint{1.544894in}{7.624184in}}%
\pgfpathlineto{\pgfqpoint{1.549845in}{7.624184in}}%
\pgfpathlineto{\pgfqpoint{1.554797in}{7.624184in}}%
\pgfpathlineto{\pgfqpoint{1.559748in}{7.624184in}}%
\pgfpathlineto{\pgfqpoint{1.564700in}{7.624184in}}%
\pgfpathlineto{\pgfqpoint{1.569651in}{7.624184in}}%
\pgfpathlineto{\pgfqpoint{1.574603in}{7.624184in}}%
\pgfpathlineto{\pgfqpoint{1.579555in}{7.624184in}}%
\pgfpathlineto{\pgfqpoint{1.584506in}{7.624184in}}%
\pgfpathlineto{\pgfqpoint{1.589458in}{7.624184in}}%
\pgfpathlineto{\pgfqpoint{1.594409in}{7.624184in}}%
\pgfpathlineto{\pgfqpoint{1.599361in}{7.624184in}}%
\pgfpathlineto{\pgfqpoint{1.604312in}{7.624184in}}%
\pgfpathlineto{\pgfqpoint{1.609264in}{7.624184in}}%
\pgfpathlineto{\pgfqpoint{1.614215in}{7.624184in}}%
\pgfpathlineto{\pgfqpoint{1.619167in}{7.624184in}}%
\pgfpathlineto{\pgfqpoint{1.624118in}{7.624184in}}%
\pgfpathlineto{\pgfqpoint{1.629070in}{7.624184in}}%
\pgfpathlineto{\pgfqpoint{1.634022in}{7.624184in}}%
\pgfpathlineto{\pgfqpoint{1.638973in}{7.624184in}}%
\pgfpathlineto{\pgfqpoint{1.643925in}{7.624184in}}%
\pgfpathlineto{\pgfqpoint{1.648876in}{7.624184in}}%
\pgfpathlineto{\pgfqpoint{1.653828in}{7.624184in}}%
\pgfpathlineto{\pgfqpoint{1.658779in}{7.624184in}}%
\pgfpathlineto{\pgfqpoint{1.663731in}{7.624184in}}%
\pgfpathlineto{\pgfqpoint{1.668682in}{7.624184in}}%
\pgfpathlineto{\pgfqpoint{1.673634in}{7.624184in}}%
\pgfpathlineto{\pgfqpoint{1.678585in}{7.624184in}}%
\pgfpathlineto{\pgfqpoint{1.683537in}{7.624184in}}%
\pgfpathlineto{\pgfqpoint{1.688488in}{7.624184in}}%
\pgfpathlineto{\pgfqpoint{1.693440in}{7.624184in}}%
\pgfpathlineto{\pgfqpoint{1.698392in}{7.624184in}}%
\pgfpathlineto{\pgfqpoint{1.703343in}{7.624184in}}%
\pgfpathlineto{\pgfqpoint{1.708295in}{7.624184in}}%
\pgfpathlineto{\pgfqpoint{1.713246in}{7.624184in}}%
\pgfpathlineto{\pgfqpoint{1.718198in}{7.624184in}}%
\pgfpathlineto{\pgfqpoint{1.718198in}{7.625480in}}%
\pgfpathlineto{\pgfqpoint{1.718198in}{7.625480in}}%
\pgfpathlineto{\pgfqpoint{1.713246in}{7.625829in}}%
\pgfpathlineto{\pgfqpoint{1.708295in}{7.626259in}}%
\pgfpathlineto{\pgfqpoint{1.703343in}{7.626787in}}%
\pgfpathlineto{\pgfqpoint{1.698392in}{7.627433in}}%
\pgfpathlineto{\pgfqpoint{1.693440in}{7.628216in}}%
\pgfpathlineto{\pgfqpoint{1.688488in}{7.629160in}}%
\pgfpathlineto{\pgfqpoint{1.683537in}{7.630293in}}%
\pgfpathlineto{\pgfqpoint{1.678585in}{7.631642in}}%
\pgfpathlineto{\pgfqpoint{1.673634in}{7.633241in}}%
\pgfpathlineto{\pgfqpoint{1.668682in}{7.635124in}}%
\pgfpathlineto{\pgfqpoint{1.663731in}{7.637327in}}%
\pgfpathlineto{\pgfqpoint{1.658779in}{7.639890in}}%
\pgfpathlineto{\pgfqpoint{1.653828in}{7.642853in}}%
\pgfpathlineto{\pgfqpoint{1.648876in}{7.646258in}}%
\pgfpathlineto{\pgfqpoint{1.643925in}{7.650147in}}%
\pgfpathlineto{\pgfqpoint{1.638973in}{7.654561in}}%
\pgfpathlineto{\pgfqpoint{1.634022in}{7.659541in}}%
\pgfpathlineto{\pgfqpoint{1.629070in}{7.665125in}}%
\pgfpathlineto{\pgfqpoint{1.624118in}{7.671348in}}%
\pgfpathlineto{\pgfqpoint{1.619167in}{7.678239in}}%
\pgfpathlineto{\pgfqpoint{1.614215in}{7.685825in}}%
\pgfpathlineto{\pgfqpoint{1.609264in}{7.694124in}}%
\pgfpathlineto{\pgfqpoint{1.604312in}{7.703146in}}%
\pgfpathlineto{\pgfqpoint{1.599361in}{7.712897in}}%
\pgfpathlineto{\pgfqpoint{1.594409in}{7.723369in}}%
\pgfpathlineto{\pgfqpoint{1.589458in}{7.734550in}}%
\pgfpathlineto{\pgfqpoint{1.584506in}{7.746415in}}%
\pgfpathlineto{\pgfqpoint{1.579555in}{7.758934in}}%
\pgfpathlineto{\pgfqpoint{1.574603in}{7.772065in}}%
\pgfpathlineto{\pgfqpoint{1.569651in}{7.785762in}}%
\pgfpathlineto{\pgfqpoint{1.564700in}{7.799971in}}%
\pgfpathlineto{\pgfqpoint{1.559748in}{7.814636in}}%
\pgfpathlineto{\pgfqpoint{1.554797in}{7.829695in}}%
\pgfpathlineto{\pgfqpoint{1.549845in}{7.845088in}}%
\pgfpathlineto{\pgfqpoint{1.544894in}{7.860758in}}%
\pgfpathlineto{\pgfqpoint{1.539942in}{7.876651in}}%
\pgfpathlineto{\pgfqpoint{1.534991in}{7.892718in}}%
\pgfpathlineto{\pgfqpoint{1.530039in}{7.908920in}}%
\pgfpathlineto{\pgfqpoint{1.525088in}{7.925231in}}%
\pgfpathlineto{\pgfqpoint{1.520136in}{7.941632in}}%
\pgfpathlineto{\pgfqpoint{1.515185in}{7.958121in}}%
\pgfpathlineto{\pgfqpoint{1.510233in}{7.974709in}}%
\pgfpathlineto{\pgfqpoint{1.505281in}{7.991419in}}%
\pgfpathlineto{\pgfqpoint{1.500330in}{8.008288in}}%
\pgfpathlineto{\pgfqpoint{1.495378in}{8.025363in}}%
\pgfpathlineto{\pgfqpoint{1.490427in}{8.042703in}}%
\pgfpathlineto{\pgfqpoint{1.485475in}{8.060372in}}%
\pgfpathlineto{\pgfqpoint{1.480524in}{8.078439in}}%
\pgfpathlineto{\pgfqpoint{1.475572in}{8.096971in}}%
\pgfpathlineto{\pgfqpoint{1.470621in}{8.116036in}}%
\pgfpathlineto{\pgfqpoint{1.465669in}{8.135694in}}%
\pgfpathlineto{\pgfqpoint{1.460718in}{8.155996in}}%
\pgfpathlineto{\pgfqpoint{1.455766in}{8.176979in}}%
\pgfpathlineto{\pgfqpoint{1.450815in}{8.198669in}}%
\pgfpathlineto{\pgfqpoint{1.445863in}{8.221074in}}%
\pgfpathlineto{\pgfqpoint{1.440911in}{8.244186in}}%
\pgfpathlineto{\pgfqpoint{1.435960in}{8.267980in}}%
\pgfpathlineto{\pgfqpoint{1.431008in}{8.292417in}}%
\pgfpathlineto{\pgfqpoint{1.426057in}{8.317445in}}%
\pgfpathlineto{\pgfqpoint{1.421105in}{8.343002in}}%
\pgfpathlineto{\pgfqpoint{1.416154in}{8.369020in}}%
\pgfpathlineto{\pgfqpoint{1.411202in}{8.395432in}}%
\pgfpathlineto{\pgfqpoint{1.406251in}{8.422171in}}%
\pgfpathlineto{\pgfqpoint{1.401299in}{8.449181in}}%
\pgfpathlineto{\pgfqpoint{1.396348in}{8.476420in}}%
\pgfpathlineto{\pgfqpoint{1.391396in}{8.503862in}}%
\pgfpathlineto{\pgfqpoint{1.386444in}{8.531503in}}%
\pgfpathlineto{\pgfqpoint{1.381493in}{8.559365in}}%
\pgfpathlineto{\pgfqpoint{1.376541in}{8.587494in}}%
\pgfpathlineto{\pgfqpoint{1.371590in}{8.615961in}}%
\pgfpathlineto{\pgfqpoint{1.366638in}{8.644864in}}%
\pgfpathlineto{\pgfqpoint{1.361687in}{8.674320in}}%
\pgfpathlineto{\pgfqpoint{1.356735in}{8.704465in}}%
\pgfpathlineto{\pgfqpoint{1.351784in}{8.735445in}}%
\pgfpathlineto{\pgfqpoint{1.346832in}{8.767414in}}%
\pgfpathlineto{\pgfqpoint{1.341881in}{8.800517in}}%
\pgfpathlineto{\pgfqpoint{1.336929in}{8.834894in}}%
\pgfpathlineto{\pgfqpoint{1.331978in}{8.870658in}}%
\pgfpathlineto{\pgfqpoint{1.327026in}{8.907900in}}%
\pgfpathlineto{\pgfqpoint{1.322074in}{8.946668in}}%
\pgfpathlineto{\pgfqpoint{1.317123in}{8.986969in}}%
\pgfpathlineto{\pgfqpoint{1.312171in}{9.028760in}}%
\pgfpathlineto{\pgfqpoint{1.307220in}{9.071943in}}%
\pgfpathlineto{\pgfqpoint{1.302268in}{9.116363in}}%
\pgfpathlineto{\pgfqpoint{1.297317in}{9.161806in}}%
\pgfpathlineto{\pgfqpoint{1.292365in}{9.208004in}}%
\pgfpathlineto{\pgfqpoint{1.287414in}{9.254635in}}%
\pgfpathlineto{\pgfqpoint{1.282462in}{9.301328in}}%
\pgfpathlineto{\pgfqpoint{1.277511in}{9.347674in}}%
\pgfpathlineto{\pgfqpoint{1.272559in}{9.393232in}}%
\pgfpathlineto{\pgfqpoint{1.267608in}{9.437537in}}%
\pgfpathlineto{\pgfqpoint{1.262656in}{9.480117in}}%
\pgfpathlineto{\pgfqpoint{1.257704in}{9.520499in}}%
\pgfpathlineto{\pgfqpoint{1.252753in}{9.558220in}}%
\pgfpathlineto{\pgfqpoint{1.247801in}{9.592846in}}%
\pgfpathlineto{\pgfqpoint{1.242850in}{9.623972in}}%
\pgfpathlineto{\pgfqpoint{1.237898in}{9.651239in}}%
\pgfpathlineto{\pgfqpoint{1.232947in}{9.674341in}}%
\pgfpathlineto{\pgfqpoint{1.227995in}{9.693029in}}%
\pgfpathlineto{\pgfqpoint{1.223044in}{9.707118in}}%
\pgfpathlineto{\pgfqpoint{1.218092in}{9.716490in}}%
\pgfpathlineto{\pgfqpoint{1.213141in}{9.721093in}}%
\pgfpathlineto{\pgfqpoint{1.208189in}{9.720946in}}%
\pgfpathlineto{\pgfqpoint{1.203238in}{9.716127in}}%
\pgfpathlineto{\pgfqpoint{1.198286in}{9.706780in}}%
\pgfpathlineto{\pgfqpoint{1.193334in}{9.693100in}}%
\pgfpathlineto{\pgfqpoint{1.188383in}{9.675332in}}%
\pgfpathlineto{\pgfqpoint{1.183431in}{9.653764in}}%
\pgfpathlineto{\pgfqpoint{1.178480in}{9.628715in}}%
\pgfpathlineto{\pgfqpoint{1.173528in}{9.600529in}}%
\pgfpathlineto{\pgfqpoint{1.168577in}{9.569569in}}%
\pgfpathlineto{\pgfqpoint{1.163625in}{9.536205in}}%
\pgfpathlineto{\pgfqpoint{1.158674in}{9.500809in}}%
\pgfpathlineto{\pgfqpoint{1.153722in}{9.463748in}}%
\pgfpathlineto{\pgfqpoint{1.148771in}{9.425377in}}%
\pgfpathlineto{\pgfqpoint{1.143819in}{9.386033in}}%
\pgfpathlineto{\pgfqpoint{1.138867in}{9.346034in}}%
\pgfpathlineto{\pgfqpoint{1.133916in}{9.305670in}}%
\pgfpathlineto{\pgfqpoint{1.128964in}{9.265206in}}%
\pgfpathlineto{\pgfqpoint{1.124013in}{9.224879in}}%
\pgfpathlineto{\pgfqpoint{1.119061in}{9.184892in}}%
\pgfpathlineto{\pgfqpoint{1.114110in}{9.145423in}}%
\pgfpathlineto{\pgfqpoint{1.109158in}{9.106618in}}%
\pgfpathlineto{\pgfqpoint{1.104207in}{9.068594in}}%
\pgfpathlineto{\pgfqpoint{1.099255in}{9.031444in}}%
\pgfpathlineto{\pgfqpoint{1.094304in}{8.995234in}}%
\pgfpathlineto{\pgfqpoint{1.089352in}{8.960006in}}%
\pgfpathlineto{\pgfqpoint{1.084401in}{8.925785in}}%
\pgfpathlineto{\pgfqpoint{1.079449in}{8.892574in}}%
\pgfpathlineto{\pgfqpoint{1.074497in}{8.860361in}}%
\pgfpathlineto{\pgfqpoint{1.069546in}{8.829119in}}%
\pgfpathlineto{\pgfqpoint{1.064594in}{8.798811in}}%
\pgfpathlineto{\pgfqpoint{1.059643in}{8.769387in}}%
\pgfpathlineto{\pgfqpoint{1.054691in}{8.740791in}}%
\pgfpathlineto{\pgfqpoint{1.049740in}{8.712958in}}%
\pgfpathlineto{\pgfqpoint{1.044788in}{8.685817in}}%
\pgfpathlineto{\pgfqpoint{1.039837in}{8.659296in}}%
\pgfpathlineto{\pgfqpoint{1.034885in}{8.633316in}}%
\pgfpathlineto{\pgfqpoint{1.029934in}{8.607796in}}%
\pgfpathlineto{\pgfqpoint{1.024982in}{8.582657in}}%
\pgfpathlineto{\pgfqpoint{1.020031in}{8.557816in}}%
\pgfpathlineto{\pgfqpoint{1.015079in}{8.533191in}}%
\pgfpathlineto{\pgfqpoint{1.010127in}{8.508704in}}%
\pgfpathlineto{\pgfqpoint{1.005176in}{8.484277in}}%
\pgfpathlineto{\pgfqpoint{1.000224in}{8.459837in}}%
\pgfpathlineto{\pgfqpoint{0.995273in}{8.435315in}}%
\pgfpathlineto{\pgfqpoint{0.990321in}{8.410650in}}%
\pgfpathlineto{\pgfqpoint{0.985370in}{8.385788in}}%
\pgfpathlineto{\pgfqpoint{0.980418in}{8.360683in}}%
\pgfpathlineto{\pgfqpoint{0.975467in}{8.335300in}}%
\pgfpathlineto{\pgfqpoint{0.970515in}{8.309618in}}%
\pgfpathlineto{\pgfqpoint{0.965564in}{8.283624in}}%
\pgfpathlineto{\pgfqpoint{0.960612in}{8.257323in}}%
\pgfpathlineto{\pgfqpoint{0.955661in}{8.230732in}}%
\pgfpathlineto{\pgfqpoint{0.950709in}{8.203883in}}%
\pgfpathlineto{\pgfqpoint{0.945757in}{8.176821in}}%
\pgfpathlineto{\pgfqpoint{0.940806in}{8.149606in}}%
\pgfpathlineto{\pgfqpoint{0.935854in}{8.122311in}}%
\pgfpathlineto{\pgfqpoint{0.930903in}{8.095020in}}%
\pgfpathlineto{\pgfqpoint{0.925951in}{8.067826in}}%
\pgfpathlineto{\pgfqpoint{0.921000in}{8.040830in}}%
\pgfpathlineto{\pgfqpoint{0.916048in}{8.014139in}}%
\pgfpathlineto{\pgfqpoint{0.911097in}{7.987864in}}%
\pgfpathlineto{\pgfqpoint{0.906145in}{7.962115in}}%
\pgfpathlineto{\pgfqpoint{0.901194in}{7.937000in}}%
\pgfpathlineto{\pgfqpoint{0.896242in}{7.912623in}}%
\pgfpathlineto{\pgfqpoint{0.891290in}{7.889083in}}%
\pgfpathlineto{\pgfqpoint{0.886339in}{7.866467in}}%
\pgfpathlineto{\pgfqpoint{0.881387in}{7.844855in}}%
\pgfpathlineto{\pgfqpoint{0.876436in}{7.824312in}}%
\pgfpathlineto{\pgfqpoint{0.871484in}{7.804891in}}%
\pgfpathlineto{\pgfqpoint{0.866533in}{7.786634in}}%
\pgfpathlineto{\pgfqpoint{0.861581in}{7.769564in}}%
\pgfpathlineto{\pgfqpoint{0.856630in}{7.753695in}}%
\pgfpathlineto{\pgfqpoint{0.851678in}{7.739026in}}%
\pgfpathlineto{\pgfqpoint{0.846727in}{7.725542in}}%
\pgfpathlineto{\pgfqpoint{0.841775in}{7.713219in}}%
\pgfpathlineto{\pgfqpoint{0.836824in}{7.702020in}}%
\pgfpathlineto{\pgfqpoint{0.831872in}{7.691903in}}%
\pgfpathlineto{\pgfqpoint{0.826920in}{7.682814in}}%
\pgfpathlineto{\pgfqpoint{0.821969in}{7.674696in}}%
\pgfpathlineto{\pgfqpoint{0.817017in}{7.667487in}}%
\pgfpathlineto{\pgfqpoint{0.812066in}{7.661123in}}%
\pgfpathlineto{\pgfqpoint{0.807114in}{7.655536in}}%
\pgfpathlineto{\pgfqpoint{0.802163in}{7.650660in}}%
\pgfpathlineto{\pgfqpoint{0.797211in}{7.646430in}}%
\pgfpathlineto{\pgfqpoint{0.792260in}{7.642780in}}%
\pgfpathlineto{\pgfqpoint{0.787308in}{7.639649in}}%
\pgfpathlineto{\pgfqpoint{0.782357in}{7.636980in}}%
\pgfpathlineto{\pgfqpoint{0.777405in}{7.634716in}}%
\pgfpathlineto{\pgfqpoint{0.772454in}{7.632808in}}%
\pgfpathlineto{\pgfqpoint{0.767502in}{7.631209in}}%
\pgfpathlineto{\pgfqpoint{0.762550in}{7.629876in}}%
\pgfpathlineto{\pgfqpoint{0.757599in}{7.628772in}}%
\pgfpathlineto{\pgfqpoint{0.752647in}{7.627863in}}%
\pgfpathlineto{\pgfqpoint{0.747696in}{7.627118in}}%
\pgfpathlineto{\pgfqpoint{0.742744in}{7.626512in}}%
\pgfpathlineto{\pgfqpoint{0.737793in}{7.626021in}}%
\pgfpathlineto{\pgfqpoint{0.732841in}{7.625625in}}%
\pgfpathlineto{\pgfqpoint{0.732841in}{7.625625in}}%
\pgfpathclose%
\pgfusepath{stroke,fill}%
}%
\begin{pgfscope}%
\pgfsys@transformshift{0.000000in}{0.000000in}%
\pgfsys@useobject{currentmarker}{}%
\end{pgfscope}%
\end{pgfscope}%
\begin{pgfscope}%
\pgfpathrectangle{\pgfqpoint{2.963410in}{5.272501in}}{\pgfqpoint{2.177280in}{2.201755in}}%
\pgfusepath{clip}%
\pgfsetbuttcap%
\pgfsetroundjoin%
\definecolor{currentfill}{rgb}{0.172549,0.627451,0.172549}%
\pgfsetfillcolor{currentfill}%
\pgfsetfillopacity{0.250000}%
\pgfsetlinewidth{1.003750pt}%
\definecolor{currentstroke}{rgb}{0.172549,0.627451,0.172549}%
\pgfsetstrokecolor{currentstroke}%
\pgfsetdash{}{0pt}%
\pgfsys@defobject{currentmarker}{\pgfqpoint{3.173478in}{5.272501in}}{\pgfqpoint{4.641425in}{7.369411in}}{%
\pgfpathmoveto{\pgfqpoint{3.173478in}{5.273515in}}%
\pgfpathlineto{\pgfqpoint{3.173478in}{5.272501in}}%
\pgfpathlineto{\pgfqpoint{3.180855in}{5.272501in}}%
\pgfpathlineto{\pgfqpoint{3.188231in}{5.272501in}}%
\pgfpathlineto{\pgfqpoint{3.195608in}{5.272501in}}%
\pgfpathlineto{\pgfqpoint{3.202985in}{5.272501in}}%
\pgfpathlineto{\pgfqpoint{3.210361in}{5.272501in}}%
\pgfpathlineto{\pgfqpoint{3.217738in}{5.272501in}}%
\pgfpathlineto{\pgfqpoint{3.225114in}{5.272501in}}%
\pgfpathlineto{\pgfqpoint{3.232491in}{5.272501in}}%
\pgfpathlineto{\pgfqpoint{3.239868in}{5.272501in}}%
\pgfpathlineto{\pgfqpoint{3.247244in}{5.272501in}}%
\pgfpathlineto{\pgfqpoint{3.254621in}{5.272501in}}%
\pgfpathlineto{\pgfqpoint{3.261998in}{5.272501in}}%
\pgfpathlineto{\pgfqpoint{3.269374in}{5.272501in}}%
\pgfpathlineto{\pgfqpoint{3.276751in}{5.272501in}}%
\pgfpathlineto{\pgfqpoint{3.284127in}{5.272501in}}%
\pgfpathlineto{\pgfqpoint{3.291504in}{5.272501in}}%
\pgfpathlineto{\pgfqpoint{3.298881in}{5.272501in}}%
\pgfpathlineto{\pgfqpoint{3.306257in}{5.272501in}}%
\pgfpathlineto{\pgfqpoint{3.313634in}{5.272501in}}%
\pgfpathlineto{\pgfqpoint{3.321011in}{5.272501in}}%
\pgfpathlineto{\pgfqpoint{3.328387in}{5.272501in}}%
\pgfpathlineto{\pgfqpoint{3.335764in}{5.272501in}}%
\pgfpathlineto{\pgfqpoint{3.343140in}{5.272501in}}%
\pgfpathlineto{\pgfqpoint{3.350517in}{5.272501in}}%
\pgfpathlineto{\pgfqpoint{3.357894in}{5.272501in}}%
\pgfpathlineto{\pgfqpoint{3.365270in}{5.272501in}}%
\pgfpathlineto{\pgfqpoint{3.372647in}{5.272501in}}%
\pgfpathlineto{\pgfqpoint{3.380023in}{5.272501in}}%
\pgfpathlineto{\pgfqpoint{3.387400in}{5.272501in}}%
\pgfpathlineto{\pgfqpoint{3.394777in}{5.272501in}}%
\pgfpathlineto{\pgfqpoint{3.402153in}{5.272501in}}%
\pgfpathlineto{\pgfqpoint{3.409530in}{5.272501in}}%
\pgfpathlineto{\pgfqpoint{3.416907in}{5.272501in}}%
\pgfpathlineto{\pgfqpoint{3.424283in}{5.272501in}}%
\pgfpathlineto{\pgfqpoint{3.431660in}{5.272501in}}%
\pgfpathlineto{\pgfqpoint{3.439036in}{5.272501in}}%
\pgfpathlineto{\pgfqpoint{3.446413in}{5.272501in}}%
\pgfpathlineto{\pgfqpoint{3.453790in}{5.272501in}}%
\pgfpathlineto{\pgfqpoint{3.461166in}{5.272501in}}%
\pgfpathlineto{\pgfqpoint{3.468543in}{5.272501in}}%
\pgfpathlineto{\pgfqpoint{3.475920in}{5.272501in}}%
\pgfpathlineto{\pgfqpoint{3.483296in}{5.272501in}}%
\pgfpathlineto{\pgfqpoint{3.490673in}{5.272501in}}%
\pgfpathlineto{\pgfqpoint{3.498049in}{5.272501in}}%
\pgfpathlineto{\pgfqpoint{3.505426in}{5.272501in}}%
\pgfpathlineto{\pgfqpoint{3.512803in}{5.272501in}}%
\pgfpathlineto{\pgfqpoint{3.520179in}{5.272501in}}%
\pgfpathlineto{\pgfqpoint{3.527556in}{5.272501in}}%
\pgfpathlineto{\pgfqpoint{3.534932in}{5.272501in}}%
\pgfpathlineto{\pgfqpoint{3.542309in}{5.272501in}}%
\pgfpathlineto{\pgfqpoint{3.549686in}{5.272501in}}%
\pgfpathlineto{\pgfqpoint{3.557062in}{5.272501in}}%
\pgfpathlineto{\pgfqpoint{3.564439in}{5.272501in}}%
\pgfpathlineto{\pgfqpoint{3.571816in}{5.272501in}}%
\pgfpathlineto{\pgfqpoint{3.579192in}{5.272501in}}%
\pgfpathlineto{\pgfqpoint{3.586569in}{5.272501in}}%
\pgfpathlineto{\pgfqpoint{3.593945in}{5.272501in}}%
\pgfpathlineto{\pgfqpoint{3.601322in}{5.272501in}}%
\pgfpathlineto{\pgfqpoint{3.608699in}{5.272501in}}%
\pgfpathlineto{\pgfqpoint{3.616075in}{5.272501in}}%
\pgfpathlineto{\pgfqpoint{3.623452in}{5.272501in}}%
\pgfpathlineto{\pgfqpoint{3.630829in}{5.272501in}}%
\pgfpathlineto{\pgfqpoint{3.638205in}{5.272501in}}%
\pgfpathlineto{\pgfqpoint{3.645582in}{5.272501in}}%
\pgfpathlineto{\pgfqpoint{3.652958in}{5.272501in}}%
\pgfpathlineto{\pgfqpoint{3.660335in}{5.272501in}}%
\pgfpathlineto{\pgfqpoint{3.667712in}{5.272501in}}%
\pgfpathlineto{\pgfqpoint{3.675088in}{5.272501in}}%
\pgfpathlineto{\pgfqpoint{3.682465in}{5.272501in}}%
\pgfpathlineto{\pgfqpoint{3.689841in}{5.272501in}}%
\pgfpathlineto{\pgfqpoint{3.697218in}{5.272501in}}%
\pgfpathlineto{\pgfqpoint{3.704595in}{5.272501in}}%
\pgfpathlineto{\pgfqpoint{3.711971in}{5.272501in}}%
\pgfpathlineto{\pgfqpoint{3.719348in}{5.272501in}}%
\pgfpathlineto{\pgfqpoint{3.726725in}{5.272501in}}%
\pgfpathlineto{\pgfqpoint{3.734101in}{5.272501in}}%
\pgfpathlineto{\pgfqpoint{3.741478in}{5.272501in}}%
\pgfpathlineto{\pgfqpoint{3.748854in}{5.272501in}}%
\pgfpathlineto{\pgfqpoint{3.756231in}{5.272501in}}%
\pgfpathlineto{\pgfqpoint{3.763608in}{5.272501in}}%
\pgfpathlineto{\pgfqpoint{3.770984in}{5.272501in}}%
\pgfpathlineto{\pgfqpoint{3.778361in}{5.272501in}}%
\pgfpathlineto{\pgfqpoint{3.785738in}{5.272501in}}%
\pgfpathlineto{\pgfqpoint{3.793114in}{5.272501in}}%
\pgfpathlineto{\pgfqpoint{3.800491in}{5.272501in}}%
\pgfpathlineto{\pgfqpoint{3.807867in}{5.272501in}}%
\pgfpathlineto{\pgfqpoint{3.815244in}{5.272501in}}%
\pgfpathlineto{\pgfqpoint{3.822621in}{5.272501in}}%
\pgfpathlineto{\pgfqpoint{3.829997in}{5.272501in}}%
\pgfpathlineto{\pgfqpoint{3.837374in}{5.272501in}}%
\pgfpathlineto{\pgfqpoint{3.844750in}{5.272501in}}%
\pgfpathlineto{\pgfqpoint{3.852127in}{5.272501in}}%
\pgfpathlineto{\pgfqpoint{3.859504in}{5.272501in}}%
\pgfpathlineto{\pgfqpoint{3.866880in}{5.272501in}}%
\pgfpathlineto{\pgfqpoint{3.874257in}{5.272501in}}%
\pgfpathlineto{\pgfqpoint{3.881634in}{5.272501in}}%
\pgfpathlineto{\pgfqpoint{3.889010in}{5.272501in}}%
\pgfpathlineto{\pgfqpoint{3.896387in}{5.272501in}}%
\pgfpathlineto{\pgfqpoint{3.903763in}{5.272501in}}%
\pgfpathlineto{\pgfqpoint{3.911140in}{5.272501in}}%
\pgfpathlineto{\pgfqpoint{3.918517in}{5.272501in}}%
\pgfpathlineto{\pgfqpoint{3.925893in}{5.272501in}}%
\pgfpathlineto{\pgfqpoint{3.933270in}{5.272501in}}%
\pgfpathlineto{\pgfqpoint{3.940647in}{5.272501in}}%
\pgfpathlineto{\pgfqpoint{3.948023in}{5.272501in}}%
\pgfpathlineto{\pgfqpoint{3.955400in}{5.272501in}}%
\pgfpathlineto{\pgfqpoint{3.962776in}{5.272501in}}%
\pgfpathlineto{\pgfqpoint{3.970153in}{5.272501in}}%
\pgfpathlineto{\pgfqpoint{3.977530in}{5.272501in}}%
\pgfpathlineto{\pgfqpoint{3.984906in}{5.272501in}}%
\pgfpathlineto{\pgfqpoint{3.992283in}{5.272501in}}%
\pgfpathlineto{\pgfqpoint{3.999659in}{5.272501in}}%
\pgfpathlineto{\pgfqpoint{4.007036in}{5.272501in}}%
\pgfpathlineto{\pgfqpoint{4.014413in}{5.272501in}}%
\pgfpathlineto{\pgfqpoint{4.021789in}{5.272501in}}%
\pgfpathlineto{\pgfqpoint{4.029166in}{5.272501in}}%
\pgfpathlineto{\pgfqpoint{4.036543in}{5.272501in}}%
\pgfpathlineto{\pgfqpoint{4.043919in}{5.272501in}}%
\pgfpathlineto{\pgfqpoint{4.051296in}{5.272501in}}%
\pgfpathlineto{\pgfqpoint{4.058672in}{5.272501in}}%
\pgfpathlineto{\pgfqpoint{4.066049in}{5.272501in}}%
\pgfpathlineto{\pgfqpoint{4.073426in}{5.272501in}}%
\pgfpathlineto{\pgfqpoint{4.080802in}{5.272501in}}%
\pgfpathlineto{\pgfqpoint{4.088179in}{5.272501in}}%
\pgfpathlineto{\pgfqpoint{4.095556in}{5.272501in}}%
\pgfpathlineto{\pgfqpoint{4.102932in}{5.272501in}}%
\pgfpathlineto{\pgfqpoint{4.110309in}{5.272501in}}%
\pgfpathlineto{\pgfqpoint{4.117685in}{5.272501in}}%
\pgfpathlineto{\pgfqpoint{4.125062in}{5.272501in}}%
\pgfpathlineto{\pgfqpoint{4.132439in}{5.272501in}}%
\pgfpathlineto{\pgfqpoint{4.139815in}{5.272501in}}%
\pgfpathlineto{\pgfqpoint{4.147192in}{5.272501in}}%
\pgfpathlineto{\pgfqpoint{4.154568in}{5.272501in}}%
\pgfpathlineto{\pgfqpoint{4.161945in}{5.272501in}}%
\pgfpathlineto{\pgfqpoint{4.169322in}{5.272501in}}%
\pgfpathlineto{\pgfqpoint{4.176698in}{5.272501in}}%
\pgfpathlineto{\pgfqpoint{4.184075in}{5.272501in}}%
\pgfpathlineto{\pgfqpoint{4.191452in}{5.272501in}}%
\pgfpathlineto{\pgfqpoint{4.198828in}{5.272501in}}%
\pgfpathlineto{\pgfqpoint{4.206205in}{5.272501in}}%
\pgfpathlineto{\pgfqpoint{4.213581in}{5.272501in}}%
\pgfpathlineto{\pgfqpoint{4.220958in}{5.272501in}}%
\pgfpathlineto{\pgfqpoint{4.228335in}{5.272501in}}%
\pgfpathlineto{\pgfqpoint{4.235711in}{5.272501in}}%
\pgfpathlineto{\pgfqpoint{4.243088in}{5.272501in}}%
\pgfpathlineto{\pgfqpoint{4.250465in}{5.272501in}}%
\pgfpathlineto{\pgfqpoint{4.257841in}{5.272501in}}%
\pgfpathlineto{\pgfqpoint{4.265218in}{5.272501in}}%
\pgfpathlineto{\pgfqpoint{4.272594in}{5.272501in}}%
\pgfpathlineto{\pgfqpoint{4.279971in}{5.272501in}}%
\pgfpathlineto{\pgfqpoint{4.287348in}{5.272501in}}%
\pgfpathlineto{\pgfqpoint{4.294724in}{5.272501in}}%
\pgfpathlineto{\pgfqpoint{4.302101in}{5.272501in}}%
\pgfpathlineto{\pgfqpoint{4.309477in}{5.272501in}}%
\pgfpathlineto{\pgfqpoint{4.316854in}{5.272501in}}%
\pgfpathlineto{\pgfqpoint{4.324231in}{5.272501in}}%
\pgfpathlineto{\pgfqpoint{4.331607in}{5.272501in}}%
\pgfpathlineto{\pgfqpoint{4.338984in}{5.272501in}}%
\pgfpathlineto{\pgfqpoint{4.346361in}{5.272501in}}%
\pgfpathlineto{\pgfqpoint{4.353737in}{5.272501in}}%
\pgfpathlineto{\pgfqpoint{4.361114in}{5.272501in}}%
\pgfpathlineto{\pgfqpoint{4.368490in}{5.272501in}}%
\pgfpathlineto{\pgfqpoint{4.375867in}{5.272501in}}%
\pgfpathlineto{\pgfqpoint{4.383244in}{5.272501in}}%
\pgfpathlineto{\pgfqpoint{4.390620in}{5.272501in}}%
\pgfpathlineto{\pgfqpoint{4.397997in}{5.272501in}}%
\pgfpathlineto{\pgfqpoint{4.405374in}{5.272501in}}%
\pgfpathlineto{\pgfqpoint{4.412750in}{5.272501in}}%
\pgfpathlineto{\pgfqpoint{4.420127in}{5.272501in}}%
\pgfpathlineto{\pgfqpoint{4.427503in}{5.272501in}}%
\pgfpathlineto{\pgfqpoint{4.434880in}{5.272501in}}%
\pgfpathlineto{\pgfqpoint{4.442257in}{5.272501in}}%
\pgfpathlineto{\pgfqpoint{4.449633in}{5.272501in}}%
\pgfpathlineto{\pgfqpoint{4.457010in}{5.272501in}}%
\pgfpathlineto{\pgfqpoint{4.464386in}{5.272501in}}%
\pgfpathlineto{\pgfqpoint{4.471763in}{5.272501in}}%
\pgfpathlineto{\pgfqpoint{4.479140in}{5.272501in}}%
\pgfpathlineto{\pgfqpoint{4.486516in}{5.272501in}}%
\pgfpathlineto{\pgfqpoint{4.493893in}{5.272501in}}%
\pgfpathlineto{\pgfqpoint{4.501270in}{5.272501in}}%
\pgfpathlineto{\pgfqpoint{4.508646in}{5.272501in}}%
\pgfpathlineto{\pgfqpoint{4.516023in}{5.272501in}}%
\pgfpathlineto{\pgfqpoint{4.523399in}{5.272501in}}%
\pgfpathlineto{\pgfqpoint{4.530776in}{5.272501in}}%
\pgfpathlineto{\pgfqpoint{4.538153in}{5.272501in}}%
\pgfpathlineto{\pgfqpoint{4.545529in}{5.272501in}}%
\pgfpathlineto{\pgfqpoint{4.552906in}{5.272501in}}%
\pgfpathlineto{\pgfqpoint{4.560283in}{5.272501in}}%
\pgfpathlineto{\pgfqpoint{4.567659in}{5.272501in}}%
\pgfpathlineto{\pgfqpoint{4.575036in}{5.272501in}}%
\pgfpathlineto{\pgfqpoint{4.582412in}{5.272501in}}%
\pgfpathlineto{\pgfqpoint{4.589789in}{5.272501in}}%
\pgfpathlineto{\pgfqpoint{4.597166in}{5.272501in}}%
\pgfpathlineto{\pgfqpoint{4.604542in}{5.272501in}}%
\pgfpathlineto{\pgfqpoint{4.611919in}{5.272501in}}%
\pgfpathlineto{\pgfqpoint{4.619295in}{5.272501in}}%
\pgfpathlineto{\pgfqpoint{4.626672in}{5.272501in}}%
\pgfpathlineto{\pgfqpoint{4.634049in}{5.272501in}}%
\pgfpathlineto{\pgfqpoint{4.641425in}{5.272501in}}%
\pgfpathlineto{\pgfqpoint{4.641425in}{5.274533in}}%
\pgfpathlineto{\pgfqpoint{4.641425in}{5.274533in}}%
\pgfpathlineto{\pgfqpoint{4.634049in}{5.275113in}}%
\pgfpathlineto{\pgfqpoint{4.626672in}{5.275833in}}%
\pgfpathlineto{\pgfqpoint{4.619295in}{5.276723in}}%
\pgfpathlineto{\pgfqpoint{4.611919in}{5.277812in}}%
\pgfpathlineto{\pgfqpoint{4.604542in}{5.279135in}}%
\pgfpathlineto{\pgfqpoint{4.597166in}{5.280729in}}%
\pgfpathlineto{\pgfqpoint{4.589789in}{5.282634in}}%
\pgfpathlineto{\pgfqpoint{4.582412in}{5.284891in}}%
\pgfpathlineto{\pgfqpoint{4.575036in}{5.287546in}}%
\pgfpathlineto{\pgfqpoint{4.567659in}{5.290641in}}%
\pgfpathlineto{\pgfqpoint{4.560283in}{5.294219in}}%
\pgfpathlineto{\pgfqpoint{4.552906in}{5.298322in}}%
\pgfpathlineto{\pgfqpoint{4.545529in}{5.302986in}}%
\pgfpathlineto{\pgfqpoint{4.538153in}{5.308243in}}%
\pgfpathlineto{\pgfqpoint{4.530776in}{5.314116in}}%
\pgfpathlineto{\pgfqpoint{4.523399in}{5.320621in}}%
\pgfpathlineto{\pgfqpoint{4.516023in}{5.327761in}}%
\pgfpathlineto{\pgfqpoint{4.508646in}{5.335527in}}%
\pgfpathlineto{\pgfqpoint{4.501270in}{5.343897in}}%
\pgfpathlineto{\pgfqpoint{4.493893in}{5.352834in}}%
\pgfpathlineto{\pgfqpoint{4.486516in}{5.362284in}}%
\pgfpathlineto{\pgfqpoint{4.479140in}{5.372182in}}%
\pgfpathlineto{\pgfqpoint{4.471763in}{5.382444in}}%
\pgfpathlineto{\pgfqpoint{4.464386in}{5.392975in}}%
\pgfpathlineto{\pgfqpoint{4.457010in}{5.403668in}}%
\pgfpathlineto{\pgfqpoint{4.449633in}{5.414409in}}%
\pgfpathlineto{\pgfqpoint{4.442257in}{5.425075in}}%
\pgfpathlineto{\pgfqpoint{4.434880in}{5.435545in}}%
\pgfpathlineto{\pgfqpoint{4.427503in}{5.445695in}}%
\pgfpathlineto{\pgfqpoint{4.420127in}{5.455411in}}%
\pgfpathlineto{\pgfqpoint{4.412750in}{5.464585in}}%
\pgfpathlineto{\pgfqpoint{4.405374in}{5.473126in}}%
\pgfpathlineto{\pgfqpoint{4.397997in}{5.480957in}}%
\pgfpathlineto{\pgfqpoint{4.390620in}{5.488025in}}%
\pgfpathlineto{\pgfqpoint{4.383244in}{5.494298in}}%
\pgfpathlineto{\pgfqpoint{4.375867in}{5.499768in}}%
\pgfpathlineto{\pgfqpoint{4.368490in}{5.504457in}}%
\pgfpathlineto{\pgfqpoint{4.361114in}{5.508409in}}%
\pgfpathlineto{\pgfqpoint{4.353737in}{5.511697in}}%
\pgfpathlineto{\pgfqpoint{4.346361in}{5.514419in}}%
\pgfpathlineto{\pgfqpoint{4.338984in}{5.516697in}}%
\pgfpathlineto{\pgfqpoint{4.331607in}{5.518671in}}%
\pgfpathlineto{\pgfqpoint{4.324231in}{5.520503in}}%
\pgfpathlineto{\pgfqpoint{4.316854in}{5.522368in}}%
\pgfpathlineto{\pgfqpoint{4.309477in}{5.524453in}}%
\pgfpathlineto{\pgfqpoint{4.302101in}{5.526954in}}%
\pgfpathlineto{\pgfqpoint{4.294724in}{5.530069in}}%
\pgfpathlineto{\pgfqpoint{4.287348in}{5.534001in}}%
\pgfpathlineto{\pgfqpoint{4.279971in}{5.538948in}}%
\pgfpathlineto{\pgfqpoint{4.272594in}{5.545104in}}%
\pgfpathlineto{\pgfqpoint{4.265218in}{5.552653in}}%
\pgfpathlineto{\pgfqpoint{4.257841in}{5.561768in}}%
\pgfpathlineto{\pgfqpoint{4.250465in}{5.572611in}}%
\pgfpathlineto{\pgfqpoint{4.243088in}{5.585325in}}%
\pgfpathlineto{\pgfqpoint{4.235711in}{5.600035in}}%
\pgfpathlineto{\pgfqpoint{4.228335in}{5.616849in}}%
\pgfpathlineto{\pgfqpoint{4.220958in}{5.635851in}}%
\pgfpathlineto{\pgfqpoint{4.213581in}{5.657107in}}%
\pgfpathlineto{\pgfqpoint{4.206205in}{5.680657in}}%
\pgfpathlineto{\pgfqpoint{4.198828in}{5.706520in}}%
\pgfpathlineto{\pgfqpoint{4.191452in}{5.734694in}}%
\pgfpathlineto{\pgfqpoint{4.184075in}{5.765154in}}%
\pgfpathlineto{\pgfqpoint{4.176698in}{5.797856in}}%
\pgfpathlineto{\pgfqpoint{4.169322in}{5.832736in}}%
\pgfpathlineto{\pgfqpoint{4.161945in}{5.869716in}}%
\pgfpathlineto{\pgfqpoint{4.154568in}{5.908703in}}%
\pgfpathlineto{\pgfqpoint{4.147192in}{5.949592in}}%
\pgfpathlineto{\pgfqpoint{4.139815in}{5.992269in}}%
\pgfpathlineto{\pgfqpoint{4.132439in}{6.036616in}}%
\pgfpathlineto{\pgfqpoint{4.125062in}{6.082509in}}%
\pgfpathlineto{\pgfqpoint{4.117685in}{6.129823in}}%
\pgfpathlineto{\pgfqpoint{4.110309in}{6.178436in}}%
\pgfpathlineto{\pgfqpoint{4.102932in}{6.228223in}}%
\pgfpathlineto{\pgfqpoint{4.095556in}{6.279064in}}%
\pgfpathlineto{\pgfqpoint{4.088179in}{6.330840in}}%
\pgfpathlineto{\pgfqpoint{4.080802in}{6.383429in}}%
\pgfpathlineto{\pgfqpoint{4.073426in}{6.436712in}}%
\pgfpathlineto{\pgfqpoint{4.066049in}{6.490560in}}%
\pgfpathlineto{\pgfqpoint{4.058672in}{6.544839in}}%
\pgfpathlineto{\pgfqpoint{4.051296in}{6.599404in}}%
\pgfpathlineto{\pgfqpoint{4.043919in}{6.654091in}}%
\pgfpathlineto{\pgfqpoint{4.036543in}{6.708719in}}%
\pgfpathlineto{\pgfqpoint{4.029166in}{6.763085in}}%
\pgfpathlineto{\pgfqpoint{4.021789in}{6.816959in}}%
\pgfpathlineto{\pgfqpoint{4.014413in}{6.870087in}}%
\pgfpathlineto{\pgfqpoint{4.007036in}{6.922188in}}%
\pgfpathlineto{\pgfqpoint{3.999659in}{6.972956in}}%
\pgfpathlineto{\pgfqpoint{3.992283in}{7.022068in}}%
\pgfpathlineto{\pgfqpoint{3.984906in}{7.069180in}}%
\pgfpathlineto{\pgfqpoint{3.977530in}{7.113944in}}%
\pgfpathlineto{\pgfqpoint{3.970153in}{7.156010in}}%
\pgfpathlineto{\pgfqpoint{3.962776in}{7.195036in}}%
\pgfpathlineto{\pgfqpoint{3.955400in}{7.230699in}}%
\pgfpathlineto{\pgfqpoint{3.948023in}{7.262706in}}%
\pgfpathlineto{\pgfqpoint{3.940647in}{7.290800in}}%
\pgfpathlineto{\pgfqpoint{3.933270in}{7.314772in}}%
\pgfpathlineto{\pgfqpoint{3.925893in}{7.334466in}}%
\pgfpathlineto{\pgfqpoint{3.918517in}{7.349783in}}%
\pgfpathlineto{\pgfqpoint{3.911140in}{7.360687in}}%
\pgfpathlineto{\pgfqpoint{3.903763in}{7.367202in}}%
\pgfpathlineto{\pgfqpoint{3.896387in}{7.369411in}}%
\pgfpathlineto{\pgfqpoint{3.889010in}{7.367448in}}%
\pgfpathlineto{\pgfqpoint{3.881634in}{7.361497in}}%
\pgfpathlineto{\pgfqpoint{3.874257in}{7.351777in}}%
\pgfpathlineto{\pgfqpoint{3.866880in}{7.338537in}}%
\pgfpathlineto{\pgfqpoint{3.859504in}{7.322041in}}%
\pgfpathlineto{\pgfqpoint{3.852127in}{7.302564in}}%
\pgfpathlineto{\pgfqpoint{3.844750in}{7.280375in}}%
\pgfpathlineto{\pgfqpoint{3.837374in}{7.255734in}}%
\pgfpathlineto{\pgfqpoint{3.829997in}{7.228884in}}%
\pgfpathlineto{\pgfqpoint{3.822621in}{7.200046in}}%
\pgfpathlineto{\pgfqpoint{3.815244in}{7.169418in}}%
\pgfpathlineto{\pgfqpoint{3.807867in}{7.137171in}}%
\pgfpathlineto{\pgfqpoint{3.800491in}{7.103453in}}%
\pgfpathlineto{\pgfqpoint{3.793114in}{7.068392in}}%
\pgfpathlineto{\pgfqpoint{3.785738in}{7.032098in}}%
\pgfpathlineto{\pgfqpoint{3.778361in}{6.994670in}}%
\pgfpathlineto{\pgfqpoint{3.770984in}{6.956197in}}%
\pgfpathlineto{\pgfqpoint{3.763608in}{6.916766in}}%
\pgfpathlineto{\pgfqpoint{3.756231in}{6.876462in}}%
\pgfpathlineto{\pgfqpoint{3.748854in}{6.835376in}}%
\pgfpathlineto{\pgfqpoint{3.741478in}{6.793600in}}%
\pgfpathlineto{\pgfqpoint{3.734101in}{6.751230in}}%
\pgfpathlineto{\pgfqpoint{3.726725in}{6.708367in}}%
\pgfpathlineto{\pgfqpoint{3.719348in}{6.665112in}}%
\pgfpathlineto{\pgfqpoint{3.711971in}{6.621566in}}%
\pgfpathlineto{\pgfqpoint{3.704595in}{6.577826in}}%
\pgfpathlineto{\pgfqpoint{3.697218in}{6.533982in}}%
\pgfpathlineto{\pgfqpoint{3.689841in}{6.490115in}}%
\pgfpathlineto{\pgfqpoint{3.682465in}{6.446298in}}%
\pgfpathlineto{\pgfqpoint{3.675088in}{6.402589in}}%
\pgfpathlineto{\pgfqpoint{3.667712in}{6.359039in}}%
\pgfpathlineto{\pgfqpoint{3.660335in}{6.315687in}}%
\pgfpathlineto{\pgfqpoint{3.652958in}{6.272566in}}%
\pgfpathlineto{\pgfqpoint{3.645582in}{6.229705in}}%
\pgfpathlineto{\pgfqpoint{3.638205in}{6.187132in}}%
\pgfpathlineto{\pgfqpoint{3.630829in}{6.144881in}}%
\pgfpathlineto{\pgfqpoint{3.623452in}{6.102990in}}%
\pgfpathlineto{\pgfqpoint{3.616075in}{6.061510in}}%
\pgfpathlineto{\pgfqpoint{3.608699in}{6.020505in}}%
\pgfpathlineto{\pgfqpoint{3.601322in}{5.980054in}}%
\pgfpathlineto{\pgfqpoint{3.593945in}{5.940251in}}%
\pgfpathlineto{\pgfqpoint{3.586569in}{5.901206in}}%
\pgfpathlineto{\pgfqpoint{3.579192in}{5.863041in}}%
\pgfpathlineto{\pgfqpoint{3.571816in}{5.825887in}}%
\pgfpathlineto{\pgfqpoint{3.564439in}{5.789880in}}%
\pgfpathlineto{\pgfqpoint{3.557062in}{5.755157in}}%
\pgfpathlineto{\pgfqpoint{3.549686in}{5.721850in}}%
\pgfpathlineto{\pgfqpoint{3.542309in}{5.690079in}}%
\pgfpathlineto{\pgfqpoint{3.534932in}{5.659951in}}%
\pgfpathlineto{\pgfqpoint{3.527556in}{5.631549in}}%
\pgfpathlineto{\pgfqpoint{3.520179in}{5.604935in}}%
\pgfpathlineto{\pgfqpoint{3.512803in}{5.580143in}}%
\pgfpathlineto{\pgfqpoint{3.505426in}{5.557176in}}%
\pgfpathlineto{\pgfqpoint{3.498049in}{5.536013in}}%
\pgfpathlineto{\pgfqpoint{3.490673in}{5.516602in}}%
\pgfpathlineto{\pgfqpoint{3.483296in}{5.498867in}}%
\pgfpathlineto{\pgfqpoint{3.475920in}{5.482709in}}%
\pgfpathlineto{\pgfqpoint{3.468543in}{5.468011in}}%
\pgfpathlineto{\pgfqpoint{3.461166in}{5.454643in}}%
\pgfpathlineto{\pgfqpoint{3.453790in}{5.442467in}}%
\pgfpathlineto{\pgfqpoint{3.446413in}{5.431341in}}%
\pgfpathlineto{\pgfqpoint{3.439036in}{5.421123in}}%
\pgfpathlineto{\pgfqpoint{3.431660in}{5.411680in}}%
\pgfpathlineto{\pgfqpoint{3.424283in}{5.402887in}}%
\pgfpathlineto{\pgfqpoint{3.416907in}{5.394631in}}%
\pgfpathlineto{\pgfqpoint{3.409530in}{5.386816in}}%
\pgfpathlineto{\pgfqpoint{3.402153in}{5.379361in}}%
\pgfpathlineto{\pgfqpoint{3.394777in}{5.372202in}}%
\pgfpathlineto{\pgfqpoint{3.387400in}{5.365292in}}%
\pgfpathlineto{\pgfqpoint{3.380023in}{5.358599in}}%
\pgfpathlineto{\pgfqpoint{3.372647in}{5.352107in}}%
\pgfpathlineto{\pgfqpoint{3.365270in}{5.345809in}}%
\pgfpathlineto{\pgfqpoint{3.357894in}{5.339711in}}%
\pgfpathlineto{\pgfqpoint{3.350517in}{5.333824in}}%
\pgfpathlineto{\pgfqpoint{3.343140in}{5.328166in}}%
\pgfpathlineto{\pgfqpoint{3.335764in}{5.322756in}}%
\pgfpathlineto{\pgfqpoint{3.328387in}{5.317614in}}%
\pgfpathlineto{\pgfqpoint{3.321011in}{5.312759in}}%
\pgfpathlineto{\pgfqpoint{3.313634in}{5.308208in}}%
\pgfpathlineto{\pgfqpoint{3.306257in}{5.303973in}}%
\pgfpathlineto{\pgfqpoint{3.298881in}{5.300064in}}%
\pgfpathlineto{\pgfqpoint{3.291504in}{5.296483in}}%
\pgfpathlineto{\pgfqpoint{3.284127in}{5.293230in}}%
\pgfpathlineto{\pgfqpoint{3.276751in}{5.290298in}}%
\pgfpathlineto{\pgfqpoint{3.269374in}{5.287678in}}%
\pgfpathlineto{\pgfqpoint{3.261998in}{5.285355in}}%
\pgfpathlineto{\pgfqpoint{3.254621in}{5.283313in}}%
\pgfpathlineto{\pgfqpoint{3.247244in}{5.281532in}}%
\pgfpathlineto{\pgfqpoint{3.239868in}{5.279992in}}%
\pgfpathlineto{\pgfqpoint{3.232491in}{5.278672in}}%
\pgfpathlineto{\pgfqpoint{3.225114in}{5.277548in}}%
\pgfpathlineto{\pgfqpoint{3.217738in}{5.276600in}}%
\pgfpathlineto{\pgfqpoint{3.210361in}{5.275807in}}%
\pgfpathlineto{\pgfqpoint{3.202985in}{5.275148in}}%
\pgfpathlineto{\pgfqpoint{3.195608in}{5.274606in}}%
\pgfpathlineto{\pgfqpoint{3.188231in}{5.274163in}}%
\pgfpathlineto{\pgfqpoint{3.180855in}{5.273804in}}%
\pgfpathlineto{\pgfqpoint{3.173478in}{5.273515in}}%
\pgfpathlineto{\pgfqpoint{3.173478in}{5.273515in}}%
\pgfpathclose%
\pgfusepath{stroke,fill}%
}%
\begin{pgfscope}%
\pgfsys@transformshift{0.000000in}{0.000000in}%
\pgfsys@useobject{currentmarker}{}%
\end{pgfscope}%
\end{pgfscope}%
\begin{pgfscope}%
\pgfpathrectangle{\pgfqpoint{2.963410in}{5.272501in}}{\pgfqpoint{2.177280in}{2.201755in}}%
\pgfusepath{clip}%
\pgfsetbuttcap%
\pgfsetroundjoin%
\definecolor{currentfill}{rgb}{1.000000,0.498039,0.054902}%
\pgfsetfillcolor{currentfill}%
\pgfsetfillopacity{0.250000}%
\pgfsetlinewidth{1.003750pt}%
\definecolor{currentstroke}{rgb}{1.000000,0.498039,0.054902}%
\pgfsetstrokecolor{currentstroke}%
\pgfsetdash{}{0pt}%
\pgfsys@defobject{currentmarker}{\pgfqpoint{3.062377in}{5.272501in}}{\pgfqpoint{4.398075in}{7.314730in}}{%
\pgfpathmoveto{\pgfqpoint{3.062377in}{5.273554in}}%
\pgfpathlineto{\pgfqpoint{3.062377in}{5.272501in}}%
\pgfpathlineto{\pgfqpoint{3.069089in}{5.272501in}}%
\pgfpathlineto{\pgfqpoint{3.075801in}{5.272501in}}%
\pgfpathlineto{\pgfqpoint{3.082513in}{5.272501in}}%
\pgfpathlineto{\pgfqpoint{3.089225in}{5.272501in}}%
\pgfpathlineto{\pgfqpoint{3.095937in}{5.272501in}}%
\pgfpathlineto{\pgfqpoint{3.102650in}{5.272501in}}%
\pgfpathlineto{\pgfqpoint{3.109362in}{5.272501in}}%
\pgfpathlineto{\pgfqpoint{3.116074in}{5.272501in}}%
\pgfpathlineto{\pgfqpoint{3.122786in}{5.272501in}}%
\pgfpathlineto{\pgfqpoint{3.129498in}{5.272501in}}%
\pgfpathlineto{\pgfqpoint{3.136210in}{5.272501in}}%
\pgfpathlineto{\pgfqpoint{3.142922in}{5.272501in}}%
\pgfpathlineto{\pgfqpoint{3.149634in}{5.272501in}}%
\pgfpathlineto{\pgfqpoint{3.156346in}{5.272501in}}%
\pgfpathlineto{\pgfqpoint{3.163058in}{5.272501in}}%
\pgfpathlineto{\pgfqpoint{3.169770in}{5.272501in}}%
\pgfpathlineto{\pgfqpoint{3.176482in}{5.272501in}}%
\pgfpathlineto{\pgfqpoint{3.183194in}{5.272501in}}%
\pgfpathlineto{\pgfqpoint{3.189906in}{5.272501in}}%
\pgfpathlineto{\pgfqpoint{3.196618in}{5.272501in}}%
\pgfpathlineto{\pgfqpoint{3.203330in}{5.272501in}}%
\pgfpathlineto{\pgfqpoint{3.210042in}{5.272501in}}%
\pgfpathlineto{\pgfqpoint{3.216754in}{5.272501in}}%
\pgfpathlineto{\pgfqpoint{3.223466in}{5.272501in}}%
\pgfpathlineto{\pgfqpoint{3.230178in}{5.272501in}}%
\pgfpathlineto{\pgfqpoint{3.236890in}{5.272501in}}%
\pgfpathlineto{\pgfqpoint{3.243603in}{5.272501in}}%
\pgfpathlineto{\pgfqpoint{3.250315in}{5.272501in}}%
\pgfpathlineto{\pgfqpoint{3.257027in}{5.272501in}}%
\pgfpathlineto{\pgfqpoint{3.263739in}{5.272501in}}%
\pgfpathlineto{\pgfqpoint{3.270451in}{5.272501in}}%
\pgfpathlineto{\pgfqpoint{3.277163in}{5.272501in}}%
\pgfpathlineto{\pgfqpoint{3.283875in}{5.272501in}}%
\pgfpathlineto{\pgfqpoint{3.290587in}{5.272501in}}%
\pgfpathlineto{\pgfqpoint{3.297299in}{5.272501in}}%
\pgfpathlineto{\pgfqpoint{3.304011in}{5.272501in}}%
\pgfpathlineto{\pgfqpoint{3.310723in}{5.272501in}}%
\pgfpathlineto{\pgfqpoint{3.317435in}{5.272501in}}%
\pgfpathlineto{\pgfqpoint{3.324147in}{5.272501in}}%
\pgfpathlineto{\pgfqpoint{3.330859in}{5.272501in}}%
\pgfpathlineto{\pgfqpoint{3.337571in}{5.272501in}}%
\pgfpathlineto{\pgfqpoint{3.344283in}{5.272501in}}%
\pgfpathlineto{\pgfqpoint{3.350995in}{5.272501in}}%
\pgfpathlineto{\pgfqpoint{3.357707in}{5.272501in}}%
\pgfpathlineto{\pgfqpoint{3.364419in}{5.272501in}}%
\pgfpathlineto{\pgfqpoint{3.371131in}{5.272501in}}%
\pgfpathlineto{\pgfqpoint{3.377843in}{5.272501in}}%
\pgfpathlineto{\pgfqpoint{3.384556in}{5.272501in}}%
\pgfpathlineto{\pgfqpoint{3.391268in}{5.272501in}}%
\pgfpathlineto{\pgfqpoint{3.397980in}{5.272501in}}%
\pgfpathlineto{\pgfqpoint{3.404692in}{5.272501in}}%
\pgfpathlineto{\pgfqpoint{3.411404in}{5.272501in}}%
\pgfpathlineto{\pgfqpoint{3.418116in}{5.272501in}}%
\pgfpathlineto{\pgfqpoint{3.424828in}{5.272501in}}%
\pgfpathlineto{\pgfqpoint{3.431540in}{5.272501in}}%
\pgfpathlineto{\pgfqpoint{3.438252in}{5.272501in}}%
\pgfpathlineto{\pgfqpoint{3.444964in}{5.272501in}}%
\pgfpathlineto{\pgfqpoint{3.451676in}{5.272501in}}%
\pgfpathlineto{\pgfqpoint{3.458388in}{5.272501in}}%
\pgfpathlineto{\pgfqpoint{3.465100in}{5.272501in}}%
\pgfpathlineto{\pgfqpoint{3.471812in}{5.272501in}}%
\pgfpathlineto{\pgfqpoint{3.478524in}{5.272501in}}%
\pgfpathlineto{\pgfqpoint{3.485236in}{5.272501in}}%
\pgfpathlineto{\pgfqpoint{3.491948in}{5.272501in}}%
\pgfpathlineto{\pgfqpoint{3.498660in}{5.272501in}}%
\pgfpathlineto{\pgfqpoint{3.505372in}{5.272501in}}%
\pgfpathlineto{\pgfqpoint{3.512084in}{5.272501in}}%
\pgfpathlineto{\pgfqpoint{3.518796in}{5.272501in}}%
\pgfpathlineto{\pgfqpoint{3.525508in}{5.272501in}}%
\pgfpathlineto{\pgfqpoint{3.532221in}{5.272501in}}%
\pgfpathlineto{\pgfqpoint{3.538933in}{5.272501in}}%
\pgfpathlineto{\pgfqpoint{3.545645in}{5.272501in}}%
\pgfpathlineto{\pgfqpoint{3.552357in}{5.272501in}}%
\pgfpathlineto{\pgfqpoint{3.559069in}{5.272501in}}%
\pgfpathlineto{\pgfqpoint{3.565781in}{5.272501in}}%
\pgfpathlineto{\pgfqpoint{3.572493in}{5.272501in}}%
\pgfpathlineto{\pgfqpoint{3.579205in}{5.272501in}}%
\pgfpathlineto{\pgfqpoint{3.585917in}{5.272501in}}%
\pgfpathlineto{\pgfqpoint{3.592629in}{5.272501in}}%
\pgfpathlineto{\pgfqpoint{3.599341in}{5.272501in}}%
\pgfpathlineto{\pgfqpoint{3.606053in}{5.272501in}}%
\pgfpathlineto{\pgfqpoint{3.612765in}{5.272501in}}%
\pgfpathlineto{\pgfqpoint{3.619477in}{5.272501in}}%
\pgfpathlineto{\pgfqpoint{3.626189in}{5.272501in}}%
\pgfpathlineto{\pgfqpoint{3.632901in}{5.272501in}}%
\pgfpathlineto{\pgfqpoint{3.639613in}{5.272501in}}%
\pgfpathlineto{\pgfqpoint{3.646325in}{5.272501in}}%
\pgfpathlineto{\pgfqpoint{3.653037in}{5.272501in}}%
\pgfpathlineto{\pgfqpoint{3.659749in}{5.272501in}}%
\pgfpathlineto{\pgfqpoint{3.666461in}{5.272501in}}%
\pgfpathlineto{\pgfqpoint{3.673174in}{5.272501in}}%
\pgfpathlineto{\pgfqpoint{3.679886in}{5.272501in}}%
\pgfpathlineto{\pgfqpoint{3.686598in}{5.272501in}}%
\pgfpathlineto{\pgfqpoint{3.693310in}{5.272501in}}%
\pgfpathlineto{\pgfqpoint{3.700022in}{5.272501in}}%
\pgfpathlineto{\pgfqpoint{3.706734in}{5.272501in}}%
\pgfpathlineto{\pgfqpoint{3.713446in}{5.272501in}}%
\pgfpathlineto{\pgfqpoint{3.720158in}{5.272501in}}%
\pgfpathlineto{\pgfqpoint{3.726870in}{5.272501in}}%
\pgfpathlineto{\pgfqpoint{3.733582in}{5.272501in}}%
\pgfpathlineto{\pgfqpoint{3.740294in}{5.272501in}}%
\pgfpathlineto{\pgfqpoint{3.747006in}{5.272501in}}%
\pgfpathlineto{\pgfqpoint{3.753718in}{5.272501in}}%
\pgfpathlineto{\pgfqpoint{3.760430in}{5.272501in}}%
\pgfpathlineto{\pgfqpoint{3.767142in}{5.272501in}}%
\pgfpathlineto{\pgfqpoint{3.773854in}{5.272501in}}%
\pgfpathlineto{\pgfqpoint{3.780566in}{5.272501in}}%
\pgfpathlineto{\pgfqpoint{3.787278in}{5.272501in}}%
\pgfpathlineto{\pgfqpoint{3.793990in}{5.272501in}}%
\pgfpathlineto{\pgfqpoint{3.800702in}{5.272501in}}%
\pgfpathlineto{\pgfqpoint{3.807414in}{5.272501in}}%
\pgfpathlineto{\pgfqpoint{3.814127in}{5.272501in}}%
\pgfpathlineto{\pgfqpoint{3.820839in}{5.272501in}}%
\pgfpathlineto{\pgfqpoint{3.827551in}{5.272501in}}%
\pgfpathlineto{\pgfqpoint{3.834263in}{5.272501in}}%
\pgfpathlineto{\pgfqpoint{3.840975in}{5.272501in}}%
\pgfpathlineto{\pgfqpoint{3.847687in}{5.272501in}}%
\pgfpathlineto{\pgfqpoint{3.854399in}{5.272501in}}%
\pgfpathlineto{\pgfqpoint{3.861111in}{5.272501in}}%
\pgfpathlineto{\pgfqpoint{3.867823in}{5.272501in}}%
\pgfpathlineto{\pgfqpoint{3.874535in}{5.272501in}}%
\pgfpathlineto{\pgfqpoint{3.881247in}{5.272501in}}%
\pgfpathlineto{\pgfqpoint{3.887959in}{5.272501in}}%
\pgfpathlineto{\pgfqpoint{3.894671in}{5.272501in}}%
\pgfpathlineto{\pgfqpoint{3.901383in}{5.272501in}}%
\pgfpathlineto{\pgfqpoint{3.908095in}{5.272501in}}%
\pgfpathlineto{\pgfqpoint{3.914807in}{5.272501in}}%
\pgfpathlineto{\pgfqpoint{3.921519in}{5.272501in}}%
\pgfpathlineto{\pgfqpoint{3.928231in}{5.272501in}}%
\pgfpathlineto{\pgfqpoint{3.934943in}{5.272501in}}%
\pgfpathlineto{\pgfqpoint{3.941655in}{5.272501in}}%
\pgfpathlineto{\pgfqpoint{3.948367in}{5.272501in}}%
\pgfpathlineto{\pgfqpoint{3.955079in}{5.272501in}}%
\pgfpathlineto{\pgfqpoint{3.961792in}{5.272501in}}%
\pgfpathlineto{\pgfqpoint{3.968504in}{5.272501in}}%
\pgfpathlineto{\pgfqpoint{3.975216in}{5.272501in}}%
\pgfpathlineto{\pgfqpoint{3.981928in}{5.272501in}}%
\pgfpathlineto{\pgfqpoint{3.988640in}{5.272501in}}%
\pgfpathlineto{\pgfqpoint{3.995352in}{5.272501in}}%
\pgfpathlineto{\pgfqpoint{4.002064in}{5.272501in}}%
\pgfpathlineto{\pgfqpoint{4.008776in}{5.272501in}}%
\pgfpathlineto{\pgfqpoint{4.015488in}{5.272501in}}%
\pgfpathlineto{\pgfqpoint{4.022200in}{5.272501in}}%
\pgfpathlineto{\pgfqpoint{4.028912in}{5.272501in}}%
\pgfpathlineto{\pgfqpoint{4.035624in}{5.272501in}}%
\pgfpathlineto{\pgfqpoint{4.042336in}{5.272501in}}%
\pgfpathlineto{\pgfqpoint{4.049048in}{5.272501in}}%
\pgfpathlineto{\pgfqpoint{4.055760in}{5.272501in}}%
\pgfpathlineto{\pgfqpoint{4.062472in}{5.272501in}}%
\pgfpathlineto{\pgfqpoint{4.069184in}{5.272501in}}%
\pgfpathlineto{\pgfqpoint{4.075896in}{5.272501in}}%
\pgfpathlineto{\pgfqpoint{4.082608in}{5.272501in}}%
\pgfpathlineto{\pgfqpoint{4.089320in}{5.272501in}}%
\pgfpathlineto{\pgfqpoint{4.096032in}{5.272501in}}%
\pgfpathlineto{\pgfqpoint{4.102745in}{5.272501in}}%
\pgfpathlineto{\pgfqpoint{4.109457in}{5.272501in}}%
\pgfpathlineto{\pgfqpoint{4.116169in}{5.272501in}}%
\pgfpathlineto{\pgfqpoint{4.122881in}{5.272501in}}%
\pgfpathlineto{\pgfqpoint{4.129593in}{5.272501in}}%
\pgfpathlineto{\pgfqpoint{4.136305in}{5.272501in}}%
\pgfpathlineto{\pgfqpoint{4.143017in}{5.272501in}}%
\pgfpathlineto{\pgfqpoint{4.149729in}{5.272501in}}%
\pgfpathlineto{\pgfqpoint{4.156441in}{5.272501in}}%
\pgfpathlineto{\pgfqpoint{4.163153in}{5.272501in}}%
\pgfpathlineto{\pgfqpoint{4.169865in}{5.272501in}}%
\pgfpathlineto{\pgfqpoint{4.176577in}{5.272501in}}%
\pgfpathlineto{\pgfqpoint{4.183289in}{5.272501in}}%
\pgfpathlineto{\pgfqpoint{4.190001in}{5.272501in}}%
\pgfpathlineto{\pgfqpoint{4.196713in}{5.272501in}}%
\pgfpathlineto{\pgfqpoint{4.203425in}{5.272501in}}%
\pgfpathlineto{\pgfqpoint{4.210137in}{5.272501in}}%
\pgfpathlineto{\pgfqpoint{4.216849in}{5.272501in}}%
\pgfpathlineto{\pgfqpoint{4.223561in}{5.272501in}}%
\pgfpathlineto{\pgfqpoint{4.230273in}{5.272501in}}%
\pgfpathlineto{\pgfqpoint{4.236985in}{5.272501in}}%
\pgfpathlineto{\pgfqpoint{4.243698in}{5.272501in}}%
\pgfpathlineto{\pgfqpoint{4.250410in}{5.272501in}}%
\pgfpathlineto{\pgfqpoint{4.257122in}{5.272501in}}%
\pgfpathlineto{\pgfqpoint{4.263834in}{5.272501in}}%
\pgfpathlineto{\pgfqpoint{4.270546in}{5.272501in}}%
\pgfpathlineto{\pgfqpoint{4.277258in}{5.272501in}}%
\pgfpathlineto{\pgfqpoint{4.283970in}{5.272501in}}%
\pgfpathlineto{\pgfqpoint{4.290682in}{5.272501in}}%
\pgfpathlineto{\pgfqpoint{4.297394in}{5.272501in}}%
\pgfpathlineto{\pgfqpoint{4.304106in}{5.272501in}}%
\pgfpathlineto{\pgfqpoint{4.310818in}{5.272501in}}%
\pgfpathlineto{\pgfqpoint{4.317530in}{5.272501in}}%
\pgfpathlineto{\pgfqpoint{4.324242in}{5.272501in}}%
\pgfpathlineto{\pgfqpoint{4.330954in}{5.272501in}}%
\pgfpathlineto{\pgfqpoint{4.337666in}{5.272501in}}%
\pgfpathlineto{\pgfqpoint{4.344378in}{5.272501in}}%
\pgfpathlineto{\pgfqpoint{4.351090in}{5.272501in}}%
\pgfpathlineto{\pgfqpoint{4.357802in}{5.272501in}}%
\pgfpathlineto{\pgfqpoint{4.364514in}{5.272501in}}%
\pgfpathlineto{\pgfqpoint{4.371226in}{5.272501in}}%
\pgfpathlineto{\pgfqpoint{4.377938in}{5.272501in}}%
\pgfpathlineto{\pgfqpoint{4.384650in}{5.272501in}}%
\pgfpathlineto{\pgfqpoint{4.391363in}{5.272501in}}%
\pgfpathlineto{\pgfqpoint{4.398075in}{5.272501in}}%
\pgfpathlineto{\pgfqpoint{4.398075in}{5.273661in}}%
\pgfpathlineto{\pgfqpoint{4.398075in}{5.273661in}}%
\pgfpathlineto{\pgfqpoint{4.391363in}{5.273978in}}%
\pgfpathlineto{\pgfqpoint{4.384650in}{5.274370in}}%
\pgfpathlineto{\pgfqpoint{4.377938in}{5.274853in}}%
\pgfpathlineto{\pgfqpoint{4.371226in}{5.275443in}}%
\pgfpathlineto{\pgfqpoint{4.364514in}{5.276160in}}%
\pgfpathlineto{\pgfqpoint{4.357802in}{5.277026in}}%
\pgfpathlineto{\pgfqpoint{4.351090in}{5.278067in}}%
\pgfpathlineto{\pgfqpoint{4.344378in}{5.279308in}}%
\pgfpathlineto{\pgfqpoint{4.337666in}{5.280781in}}%
\pgfpathlineto{\pgfqpoint{4.330954in}{5.282517in}}%
\pgfpathlineto{\pgfqpoint{4.324242in}{5.284552in}}%
\pgfpathlineto{\pgfqpoint{4.317530in}{5.286923in}}%
\pgfpathlineto{\pgfqpoint{4.310818in}{5.289671in}}%
\pgfpathlineto{\pgfqpoint{4.304106in}{5.292835in}}%
\pgfpathlineto{\pgfqpoint{4.297394in}{5.296459in}}%
\pgfpathlineto{\pgfqpoint{4.290682in}{5.300587in}}%
\pgfpathlineto{\pgfqpoint{4.283970in}{5.305264in}}%
\pgfpathlineto{\pgfqpoint{4.277258in}{5.310533in}}%
\pgfpathlineto{\pgfqpoint{4.270546in}{5.316441in}}%
\pgfpathlineto{\pgfqpoint{4.263834in}{5.323029in}}%
\pgfpathlineto{\pgfqpoint{4.257122in}{5.330341in}}%
\pgfpathlineto{\pgfqpoint{4.250410in}{5.338418in}}%
\pgfpathlineto{\pgfqpoint{4.243698in}{5.347298in}}%
\pgfpathlineto{\pgfqpoint{4.236985in}{5.357018in}}%
\pgfpathlineto{\pgfqpoint{4.230273in}{5.367613in}}%
\pgfpathlineto{\pgfqpoint{4.223561in}{5.379115in}}%
\pgfpathlineto{\pgfqpoint{4.216849in}{5.391553in}}%
\pgfpathlineto{\pgfqpoint{4.210137in}{5.404957in}}%
\pgfpathlineto{\pgfqpoint{4.203425in}{5.419351in}}%
\pgfpathlineto{\pgfqpoint{4.196713in}{5.434761in}}%
\pgfpathlineto{\pgfqpoint{4.190001in}{5.451211in}}%
\pgfpathlineto{\pgfqpoint{4.183289in}{5.468722in}}%
\pgfpathlineto{\pgfqpoint{4.176577in}{5.487319in}}%
\pgfpathlineto{\pgfqpoint{4.169865in}{5.507024in}}%
\pgfpathlineto{\pgfqpoint{4.163153in}{5.527862in}}%
\pgfpathlineto{\pgfqpoint{4.156441in}{5.549858in}}%
\pgfpathlineto{\pgfqpoint{4.149729in}{5.573040in}}%
\pgfpathlineto{\pgfqpoint{4.143017in}{5.597436in}}%
\pgfpathlineto{\pgfqpoint{4.136305in}{5.623076in}}%
\pgfpathlineto{\pgfqpoint{4.129593in}{5.649995in}}%
\pgfpathlineto{\pgfqpoint{4.122881in}{5.678226in}}%
\pgfpathlineto{\pgfqpoint{4.116169in}{5.707805in}}%
\pgfpathlineto{\pgfqpoint{4.109457in}{5.738770in}}%
\pgfpathlineto{\pgfqpoint{4.102745in}{5.771158in}}%
\pgfpathlineto{\pgfqpoint{4.096032in}{5.805004in}}%
\pgfpathlineto{\pgfqpoint{4.089320in}{5.840344in}}%
\pgfpathlineto{\pgfqpoint{4.082608in}{5.877210in}}%
\pgfpathlineto{\pgfqpoint{4.075896in}{5.915627in}}%
\pgfpathlineto{\pgfqpoint{4.069184in}{5.955617in}}%
\pgfpathlineto{\pgfqpoint{4.062472in}{5.997191in}}%
\pgfpathlineto{\pgfqpoint{4.055760in}{6.040352in}}%
\pgfpathlineto{\pgfqpoint{4.049048in}{6.085087in}}%
\pgfpathlineto{\pgfqpoint{4.042336in}{6.131371in}}%
\pgfpathlineto{\pgfqpoint{4.035624in}{6.179159in}}%
\pgfpathlineto{\pgfqpoint{4.028912in}{6.228388in}}%
\pgfpathlineto{\pgfqpoint{4.022200in}{6.278971in}}%
\pgfpathlineto{\pgfqpoint{4.015488in}{6.330798in}}%
\pgfpathlineto{\pgfqpoint{4.008776in}{6.383733in}}%
\pgfpathlineto{\pgfqpoint{4.002064in}{6.437614in}}%
\pgfpathlineto{\pgfqpoint{3.995352in}{6.492250in}}%
\pgfpathlineto{\pgfqpoint{3.988640in}{6.547424in}}%
\pgfpathlineto{\pgfqpoint{3.981928in}{6.602891in}}%
\pgfpathlineto{\pgfqpoint{3.975216in}{6.658381in}}%
\pgfpathlineto{\pgfqpoint{3.968504in}{6.713603in}}%
\pgfpathlineto{\pgfqpoint{3.961792in}{6.768245in}}%
\pgfpathlineto{\pgfqpoint{3.955079in}{6.821981in}}%
\pgfpathlineto{\pgfqpoint{3.948367in}{6.874472in}}%
\pgfpathlineto{\pgfqpoint{3.941655in}{6.925375in}}%
\pgfpathlineto{\pgfqpoint{3.934943in}{6.974349in}}%
\pgfpathlineto{\pgfqpoint{3.928231in}{7.021057in}}%
\pgfpathlineto{\pgfqpoint{3.921519in}{7.065174in}}%
\pgfpathlineto{\pgfqpoint{3.914807in}{7.106397in}}%
\pgfpathlineto{\pgfqpoint{3.908095in}{7.144442in}}%
\pgfpathlineto{\pgfqpoint{3.901383in}{7.179059in}}%
\pgfpathlineto{\pgfqpoint{3.894671in}{7.210028in}}%
\pgfpathlineto{\pgfqpoint{3.887959in}{7.237169in}}%
\pgfpathlineto{\pgfqpoint{3.881247in}{7.260342in}}%
\pgfpathlineto{\pgfqpoint{3.874535in}{7.279447in}}%
\pgfpathlineto{\pgfqpoint{3.867823in}{7.294430in}}%
\pgfpathlineto{\pgfqpoint{3.861111in}{7.305278in}}%
\pgfpathlineto{\pgfqpoint{3.854399in}{7.312022in}}%
\pgfpathlineto{\pgfqpoint{3.847687in}{7.314730in}}%
\pgfpathlineto{\pgfqpoint{3.840975in}{7.313512in}}%
\pgfpathlineto{\pgfqpoint{3.834263in}{7.308508in}}%
\pgfpathlineto{\pgfqpoint{3.827551in}{7.299893in}}%
\pgfpathlineto{\pgfqpoint{3.820839in}{7.287868in}}%
\pgfpathlineto{\pgfqpoint{3.814127in}{7.272655in}}%
\pgfpathlineto{\pgfqpoint{3.807414in}{7.254497in}}%
\pgfpathlineto{\pgfqpoint{3.800702in}{7.233651in}}%
\pgfpathlineto{\pgfqpoint{3.793990in}{7.210387in}}%
\pgfpathlineto{\pgfqpoint{3.787278in}{7.184980in}}%
\pgfpathlineto{\pgfqpoint{3.780566in}{7.157708in}}%
\pgfpathlineto{\pgfqpoint{3.773854in}{7.128851in}}%
\pgfpathlineto{\pgfqpoint{3.767142in}{7.098686in}}%
\pgfpathlineto{\pgfqpoint{3.760430in}{7.067484in}}%
\pgfpathlineto{\pgfqpoint{3.753718in}{7.035508in}}%
\pgfpathlineto{\pgfqpoint{3.747006in}{7.003011in}}%
\pgfpathlineto{\pgfqpoint{3.740294in}{6.970232in}}%
\pgfpathlineto{\pgfqpoint{3.733582in}{6.937397in}}%
\pgfpathlineto{\pgfqpoint{3.726870in}{6.904716in}}%
\pgfpathlineto{\pgfqpoint{3.720158in}{6.872377in}}%
\pgfpathlineto{\pgfqpoint{3.713446in}{6.840553in}}%
\pgfpathlineto{\pgfqpoint{3.706734in}{6.809393in}}%
\pgfpathlineto{\pgfqpoint{3.700022in}{6.779026in}}%
\pgfpathlineto{\pgfqpoint{3.693310in}{6.749555in}}%
\pgfpathlineto{\pgfqpoint{3.686598in}{6.721065in}}%
\pgfpathlineto{\pgfqpoint{3.679886in}{6.693611in}}%
\pgfpathlineto{\pgfqpoint{3.673174in}{6.667231in}}%
\pgfpathlineto{\pgfqpoint{3.666461in}{6.641935in}}%
\pgfpathlineto{\pgfqpoint{3.659749in}{6.617715in}}%
\pgfpathlineto{\pgfqpoint{3.653037in}{6.594539in}}%
\pgfpathlineto{\pgfqpoint{3.646325in}{6.572358in}}%
\pgfpathlineto{\pgfqpoint{3.639613in}{6.551106in}}%
\pgfpathlineto{\pgfqpoint{3.632901in}{6.530700in}}%
\pgfpathlineto{\pgfqpoint{3.626189in}{6.511045in}}%
\pgfpathlineto{\pgfqpoint{3.619477in}{6.492034in}}%
\pgfpathlineto{\pgfqpoint{3.612765in}{6.473556in}}%
\pgfpathlineto{\pgfqpoint{3.606053in}{6.455488in}}%
\pgfpathlineto{\pgfqpoint{3.599341in}{6.437709in}}%
\pgfpathlineto{\pgfqpoint{3.592629in}{6.420094in}}%
\pgfpathlineto{\pgfqpoint{3.585917in}{6.402519in}}%
\pgfpathlineto{\pgfqpoint{3.579205in}{6.384866in}}%
\pgfpathlineto{\pgfqpoint{3.572493in}{6.367018in}}%
\pgfpathlineto{\pgfqpoint{3.565781in}{6.348866in}}%
\pgfpathlineto{\pgfqpoint{3.559069in}{6.330311in}}%
\pgfpathlineto{\pgfqpoint{3.552357in}{6.311261in}}%
\pgfpathlineto{\pgfqpoint{3.545645in}{6.291635in}}%
\pgfpathlineto{\pgfqpoint{3.538933in}{6.271365in}}%
\pgfpathlineto{\pgfqpoint{3.532221in}{6.250393in}}%
\pgfpathlineto{\pgfqpoint{3.525508in}{6.228677in}}%
\pgfpathlineto{\pgfqpoint{3.518796in}{6.206188in}}%
\pgfpathlineto{\pgfqpoint{3.512084in}{6.182911in}}%
\pgfpathlineto{\pgfqpoint{3.505372in}{6.158849in}}%
\pgfpathlineto{\pgfqpoint{3.498660in}{6.134017in}}%
\pgfpathlineto{\pgfqpoint{3.491948in}{6.108447in}}%
\pgfpathlineto{\pgfqpoint{3.485236in}{6.082186in}}%
\pgfpathlineto{\pgfqpoint{3.478524in}{6.055297in}}%
\pgfpathlineto{\pgfqpoint{3.471812in}{6.027855in}}%
\pgfpathlineto{\pgfqpoint{3.465100in}{5.999948in}}%
\pgfpathlineto{\pgfqpoint{3.458388in}{5.971677in}}%
\pgfpathlineto{\pgfqpoint{3.451676in}{5.943152in}}%
\pgfpathlineto{\pgfqpoint{3.444964in}{5.914489in}}%
\pgfpathlineto{\pgfqpoint{3.438252in}{5.885811in}}%
\pgfpathlineto{\pgfqpoint{3.431540in}{5.857243in}}%
\pgfpathlineto{\pgfqpoint{3.424828in}{5.828910in}}%
\pgfpathlineto{\pgfqpoint{3.418116in}{5.800935in}}%
\pgfpathlineto{\pgfqpoint{3.411404in}{5.773436in}}%
\pgfpathlineto{\pgfqpoint{3.404692in}{5.746521in}}%
\pgfpathlineto{\pgfqpoint{3.397980in}{5.720292in}}%
\pgfpathlineto{\pgfqpoint{3.391268in}{5.694835in}}%
\pgfpathlineto{\pgfqpoint{3.384556in}{5.670227in}}%
\pgfpathlineto{\pgfqpoint{3.377843in}{5.646527in}}%
\pgfpathlineto{\pgfqpoint{3.371131in}{5.623780in}}%
\pgfpathlineto{\pgfqpoint{3.364419in}{5.602018in}}%
\pgfpathlineto{\pgfqpoint{3.357707in}{5.581256in}}%
\pgfpathlineto{\pgfqpoint{3.350995in}{5.561496in}}%
\pgfpathlineto{\pgfqpoint{3.344283in}{5.542728in}}%
\pgfpathlineto{\pgfqpoint{3.337571in}{5.524931in}}%
\pgfpathlineto{\pgfqpoint{3.330859in}{5.508074in}}%
\pgfpathlineto{\pgfqpoint{3.324147in}{5.492119in}}%
\pgfpathlineto{\pgfqpoint{3.317435in}{5.477026in}}%
\pgfpathlineto{\pgfqpoint{3.310723in}{5.462747in}}%
\pgfpathlineto{\pgfqpoint{3.304011in}{5.449238in}}%
\pgfpathlineto{\pgfqpoint{3.297299in}{5.436451in}}%
\pgfpathlineto{\pgfqpoint{3.290587in}{5.424345in}}%
\pgfpathlineto{\pgfqpoint{3.283875in}{5.412878in}}%
\pgfpathlineto{\pgfqpoint{3.277163in}{5.402013in}}%
\pgfpathlineto{\pgfqpoint{3.270451in}{5.391721in}}%
\pgfpathlineto{\pgfqpoint{3.263739in}{5.381972in}}%
\pgfpathlineto{\pgfqpoint{3.257027in}{5.372746in}}%
\pgfpathlineto{\pgfqpoint{3.250315in}{5.364023in}}%
\pgfpathlineto{\pgfqpoint{3.243603in}{5.355791in}}%
\pgfpathlineto{\pgfqpoint{3.236890in}{5.348036in}}%
\pgfpathlineto{\pgfqpoint{3.230178in}{5.340752in}}%
\pgfpathlineto{\pgfqpoint{3.223466in}{5.333929in}}%
\pgfpathlineto{\pgfqpoint{3.216754in}{5.327561in}}%
\pgfpathlineto{\pgfqpoint{3.210042in}{5.321641in}}%
\pgfpathlineto{\pgfqpoint{3.203330in}{5.316160in}}%
\pgfpathlineto{\pgfqpoint{3.196618in}{5.311111in}}%
\pgfpathlineto{\pgfqpoint{3.189906in}{5.306481in}}%
\pgfpathlineto{\pgfqpoint{3.183194in}{5.302258in}}%
\pgfpathlineto{\pgfqpoint{3.176482in}{5.298428in}}%
\pgfpathlineto{\pgfqpoint{3.169770in}{5.294973in}}%
\pgfpathlineto{\pgfqpoint{3.163058in}{5.291876in}}%
\pgfpathlineto{\pgfqpoint{3.156346in}{5.289114in}}%
\pgfpathlineto{\pgfqpoint{3.149634in}{5.286669in}}%
\pgfpathlineto{\pgfqpoint{3.142922in}{5.284516in}}%
\pgfpathlineto{\pgfqpoint{3.136210in}{5.282632in}}%
\pgfpathlineto{\pgfqpoint{3.129498in}{5.280995in}}%
\pgfpathlineto{\pgfqpoint{3.122786in}{5.279582in}}%
\pgfpathlineto{\pgfqpoint{3.116074in}{5.278369in}}%
\pgfpathlineto{\pgfqpoint{3.109362in}{5.277336in}}%
\pgfpathlineto{\pgfqpoint{3.102650in}{5.276461in}}%
\pgfpathlineto{\pgfqpoint{3.095937in}{5.275725in}}%
\pgfpathlineto{\pgfqpoint{3.089225in}{5.275110in}}%
\pgfpathlineto{\pgfqpoint{3.082513in}{5.274600in}}%
\pgfpathlineto{\pgfqpoint{3.075801in}{5.274179in}}%
\pgfpathlineto{\pgfqpoint{3.069089in}{5.273835in}}%
\pgfpathlineto{\pgfqpoint{3.062377in}{5.273554in}}%
\pgfpathlineto{\pgfqpoint{3.062377in}{5.273554in}}%
\pgfpathclose%
\pgfusepath{stroke,fill}%
}%
\begin{pgfscope}%
\pgfsys@transformshift{0.000000in}{0.000000in}%
\pgfsys@useobject{currentmarker}{}%
\end{pgfscope}%
\end{pgfscope}%
\begin{pgfscope}%
\pgfpathrectangle{\pgfqpoint{2.963410in}{5.272501in}}{\pgfqpoint{2.177280in}{2.201755in}}%
\pgfusepath{clip}%
\pgfsetbuttcap%
\pgfsetroundjoin%
\definecolor{currentfill}{rgb}{0.121569,0.466667,0.705882}%
\pgfsetfillcolor{currentfill}%
\pgfsetfillopacity{0.250000}%
\pgfsetlinewidth{1.003750pt}%
\definecolor{currentstroke}{rgb}{0.121569,0.466667,0.705882}%
\pgfsetstrokecolor{currentstroke}%
\pgfsetdash{}{0pt}%
\pgfsys@defobject{currentmarker}{\pgfqpoint{3.186707in}{5.272501in}}{\pgfqpoint{5.041723in}{6.994174in}}{%
\pgfpathmoveto{\pgfqpoint{3.186707in}{5.273363in}}%
\pgfpathlineto{\pgfqpoint{3.186707in}{5.272501in}}%
\pgfpathlineto{\pgfqpoint{3.196029in}{5.272501in}}%
\pgfpathlineto{\pgfqpoint{3.205351in}{5.272501in}}%
\pgfpathlineto{\pgfqpoint{3.214672in}{5.272501in}}%
\pgfpathlineto{\pgfqpoint{3.223994in}{5.272501in}}%
\pgfpathlineto{\pgfqpoint{3.233316in}{5.272501in}}%
\pgfpathlineto{\pgfqpoint{3.242637in}{5.272501in}}%
\pgfpathlineto{\pgfqpoint{3.251959in}{5.272501in}}%
\pgfpathlineto{\pgfqpoint{3.261281in}{5.272501in}}%
\pgfpathlineto{\pgfqpoint{3.270603in}{5.272501in}}%
\pgfpathlineto{\pgfqpoint{3.279924in}{5.272501in}}%
\pgfpathlineto{\pgfqpoint{3.289246in}{5.272501in}}%
\pgfpathlineto{\pgfqpoint{3.298568in}{5.272501in}}%
\pgfpathlineto{\pgfqpoint{3.307889in}{5.272501in}}%
\pgfpathlineto{\pgfqpoint{3.317211in}{5.272501in}}%
\pgfpathlineto{\pgfqpoint{3.326533in}{5.272501in}}%
\pgfpathlineto{\pgfqpoint{3.335854in}{5.272501in}}%
\pgfpathlineto{\pgfqpoint{3.345176in}{5.272501in}}%
\pgfpathlineto{\pgfqpoint{3.354498in}{5.272501in}}%
\pgfpathlineto{\pgfqpoint{3.363819in}{5.272501in}}%
\pgfpathlineto{\pgfqpoint{3.373141in}{5.272501in}}%
\pgfpathlineto{\pgfqpoint{3.382463in}{5.272501in}}%
\pgfpathlineto{\pgfqpoint{3.391784in}{5.272501in}}%
\pgfpathlineto{\pgfqpoint{3.401106in}{5.272501in}}%
\pgfpathlineto{\pgfqpoint{3.410428in}{5.272501in}}%
\pgfpathlineto{\pgfqpoint{3.419750in}{5.272501in}}%
\pgfpathlineto{\pgfqpoint{3.429071in}{5.272501in}}%
\pgfpathlineto{\pgfqpoint{3.438393in}{5.272501in}}%
\pgfpathlineto{\pgfqpoint{3.447715in}{5.272501in}}%
\pgfpathlineto{\pgfqpoint{3.457036in}{5.272501in}}%
\pgfpathlineto{\pgfqpoint{3.466358in}{5.272501in}}%
\pgfpathlineto{\pgfqpoint{3.475680in}{5.272501in}}%
\pgfpathlineto{\pgfqpoint{3.485001in}{5.272501in}}%
\pgfpathlineto{\pgfqpoint{3.494323in}{5.272501in}}%
\pgfpathlineto{\pgfqpoint{3.503645in}{5.272501in}}%
\pgfpathlineto{\pgfqpoint{3.512966in}{5.272501in}}%
\pgfpathlineto{\pgfqpoint{3.522288in}{5.272501in}}%
\pgfpathlineto{\pgfqpoint{3.531610in}{5.272501in}}%
\pgfpathlineto{\pgfqpoint{3.540931in}{5.272501in}}%
\pgfpathlineto{\pgfqpoint{3.550253in}{5.272501in}}%
\pgfpathlineto{\pgfqpoint{3.559575in}{5.272501in}}%
\pgfpathlineto{\pgfqpoint{3.568897in}{5.272501in}}%
\pgfpathlineto{\pgfqpoint{3.578218in}{5.272501in}}%
\pgfpathlineto{\pgfqpoint{3.587540in}{5.272501in}}%
\pgfpathlineto{\pgfqpoint{3.596862in}{5.272501in}}%
\pgfpathlineto{\pgfqpoint{3.606183in}{5.272501in}}%
\pgfpathlineto{\pgfqpoint{3.615505in}{5.272501in}}%
\pgfpathlineto{\pgfqpoint{3.624827in}{5.272501in}}%
\pgfpathlineto{\pgfqpoint{3.634148in}{5.272501in}}%
\pgfpathlineto{\pgfqpoint{3.643470in}{5.272501in}}%
\pgfpathlineto{\pgfqpoint{3.652792in}{5.272501in}}%
\pgfpathlineto{\pgfqpoint{3.662113in}{5.272501in}}%
\pgfpathlineto{\pgfqpoint{3.671435in}{5.272501in}}%
\pgfpathlineto{\pgfqpoint{3.680757in}{5.272501in}}%
\pgfpathlineto{\pgfqpoint{3.690078in}{5.272501in}}%
\pgfpathlineto{\pgfqpoint{3.699400in}{5.272501in}}%
\pgfpathlineto{\pgfqpoint{3.708722in}{5.272501in}}%
\pgfpathlineto{\pgfqpoint{3.718043in}{5.272501in}}%
\pgfpathlineto{\pgfqpoint{3.727365in}{5.272501in}}%
\pgfpathlineto{\pgfqpoint{3.736687in}{5.272501in}}%
\pgfpathlineto{\pgfqpoint{3.746009in}{5.272501in}}%
\pgfpathlineto{\pgfqpoint{3.755330in}{5.272501in}}%
\pgfpathlineto{\pgfqpoint{3.764652in}{5.272501in}}%
\pgfpathlineto{\pgfqpoint{3.773974in}{5.272501in}}%
\pgfpathlineto{\pgfqpoint{3.783295in}{5.272501in}}%
\pgfpathlineto{\pgfqpoint{3.792617in}{5.272501in}}%
\pgfpathlineto{\pgfqpoint{3.801939in}{5.272501in}}%
\pgfpathlineto{\pgfqpoint{3.811260in}{5.272501in}}%
\pgfpathlineto{\pgfqpoint{3.820582in}{5.272501in}}%
\pgfpathlineto{\pgfqpoint{3.829904in}{5.272501in}}%
\pgfpathlineto{\pgfqpoint{3.839225in}{5.272501in}}%
\pgfpathlineto{\pgfqpoint{3.848547in}{5.272501in}}%
\pgfpathlineto{\pgfqpoint{3.857869in}{5.272501in}}%
\pgfpathlineto{\pgfqpoint{3.867190in}{5.272501in}}%
\pgfpathlineto{\pgfqpoint{3.876512in}{5.272501in}}%
\pgfpathlineto{\pgfqpoint{3.885834in}{5.272501in}}%
\pgfpathlineto{\pgfqpoint{3.895156in}{5.272501in}}%
\pgfpathlineto{\pgfqpoint{3.904477in}{5.272501in}}%
\pgfpathlineto{\pgfqpoint{3.913799in}{5.272501in}}%
\pgfpathlineto{\pgfqpoint{3.923121in}{5.272501in}}%
\pgfpathlineto{\pgfqpoint{3.932442in}{5.272501in}}%
\pgfpathlineto{\pgfqpoint{3.941764in}{5.272501in}}%
\pgfpathlineto{\pgfqpoint{3.951086in}{5.272501in}}%
\pgfpathlineto{\pgfqpoint{3.960407in}{5.272501in}}%
\pgfpathlineto{\pgfqpoint{3.969729in}{5.272501in}}%
\pgfpathlineto{\pgfqpoint{3.979051in}{5.272501in}}%
\pgfpathlineto{\pgfqpoint{3.988372in}{5.272501in}}%
\pgfpathlineto{\pgfqpoint{3.997694in}{5.272501in}}%
\pgfpathlineto{\pgfqpoint{4.007016in}{5.272501in}}%
\pgfpathlineto{\pgfqpoint{4.016337in}{5.272501in}}%
\pgfpathlineto{\pgfqpoint{4.025659in}{5.272501in}}%
\pgfpathlineto{\pgfqpoint{4.034981in}{5.272501in}}%
\pgfpathlineto{\pgfqpoint{4.044303in}{5.272501in}}%
\pgfpathlineto{\pgfqpoint{4.053624in}{5.272501in}}%
\pgfpathlineto{\pgfqpoint{4.062946in}{5.272501in}}%
\pgfpathlineto{\pgfqpoint{4.072268in}{5.272501in}}%
\pgfpathlineto{\pgfqpoint{4.081589in}{5.272501in}}%
\pgfpathlineto{\pgfqpoint{4.090911in}{5.272501in}}%
\pgfpathlineto{\pgfqpoint{4.100233in}{5.272501in}}%
\pgfpathlineto{\pgfqpoint{4.109554in}{5.272501in}}%
\pgfpathlineto{\pgfqpoint{4.118876in}{5.272501in}}%
\pgfpathlineto{\pgfqpoint{4.128198in}{5.272501in}}%
\pgfpathlineto{\pgfqpoint{4.137519in}{5.272501in}}%
\pgfpathlineto{\pgfqpoint{4.146841in}{5.272501in}}%
\pgfpathlineto{\pgfqpoint{4.156163in}{5.272501in}}%
\pgfpathlineto{\pgfqpoint{4.165484in}{5.272501in}}%
\pgfpathlineto{\pgfqpoint{4.174806in}{5.272501in}}%
\pgfpathlineto{\pgfqpoint{4.184128in}{5.272501in}}%
\pgfpathlineto{\pgfqpoint{4.193449in}{5.272501in}}%
\pgfpathlineto{\pgfqpoint{4.202771in}{5.272501in}}%
\pgfpathlineto{\pgfqpoint{4.212093in}{5.272501in}}%
\pgfpathlineto{\pgfqpoint{4.221415in}{5.272501in}}%
\pgfpathlineto{\pgfqpoint{4.230736in}{5.272501in}}%
\pgfpathlineto{\pgfqpoint{4.240058in}{5.272501in}}%
\pgfpathlineto{\pgfqpoint{4.249380in}{5.272501in}}%
\pgfpathlineto{\pgfqpoint{4.258701in}{5.272501in}}%
\pgfpathlineto{\pgfqpoint{4.268023in}{5.272501in}}%
\pgfpathlineto{\pgfqpoint{4.277345in}{5.272501in}}%
\pgfpathlineto{\pgfqpoint{4.286666in}{5.272501in}}%
\pgfpathlineto{\pgfqpoint{4.295988in}{5.272501in}}%
\pgfpathlineto{\pgfqpoint{4.305310in}{5.272501in}}%
\pgfpathlineto{\pgfqpoint{4.314631in}{5.272501in}}%
\pgfpathlineto{\pgfqpoint{4.323953in}{5.272501in}}%
\pgfpathlineto{\pgfqpoint{4.333275in}{5.272501in}}%
\pgfpathlineto{\pgfqpoint{4.342596in}{5.272501in}}%
\pgfpathlineto{\pgfqpoint{4.351918in}{5.272501in}}%
\pgfpathlineto{\pgfqpoint{4.361240in}{5.272501in}}%
\pgfpathlineto{\pgfqpoint{4.370562in}{5.272501in}}%
\pgfpathlineto{\pgfqpoint{4.379883in}{5.272501in}}%
\pgfpathlineto{\pgfqpoint{4.389205in}{5.272501in}}%
\pgfpathlineto{\pgfqpoint{4.398527in}{5.272501in}}%
\pgfpathlineto{\pgfqpoint{4.407848in}{5.272501in}}%
\pgfpathlineto{\pgfqpoint{4.417170in}{5.272501in}}%
\pgfpathlineto{\pgfqpoint{4.426492in}{5.272501in}}%
\pgfpathlineto{\pgfqpoint{4.435813in}{5.272501in}}%
\pgfpathlineto{\pgfqpoint{4.445135in}{5.272501in}}%
\pgfpathlineto{\pgfqpoint{4.454457in}{5.272501in}}%
\pgfpathlineto{\pgfqpoint{4.463778in}{5.272501in}}%
\pgfpathlineto{\pgfqpoint{4.473100in}{5.272501in}}%
\pgfpathlineto{\pgfqpoint{4.482422in}{5.272501in}}%
\pgfpathlineto{\pgfqpoint{4.491743in}{5.272501in}}%
\pgfpathlineto{\pgfqpoint{4.501065in}{5.272501in}}%
\pgfpathlineto{\pgfqpoint{4.510387in}{5.272501in}}%
\pgfpathlineto{\pgfqpoint{4.519709in}{5.272501in}}%
\pgfpathlineto{\pgfqpoint{4.529030in}{5.272501in}}%
\pgfpathlineto{\pgfqpoint{4.538352in}{5.272501in}}%
\pgfpathlineto{\pgfqpoint{4.547674in}{5.272501in}}%
\pgfpathlineto{\pgfqpoint{4.556995in}{5.272501in}}%
\pgfpathlineto{\pgfqpoint{4.566317in}{5.272501in}}%
\pgfpathlineto{\pgfqpoint{4.575639in}{5.272501in}}%
\pgfpathlineto{\pgfqpoint{4.584960in}{5.272501in}}%
\pgfpathlineto{\pgfqpoint{4.594282in}{5.272501in}}%
\pgfpathlineto{\pgfqpoint{4.603604in}{5.272501in}}%
\pgfpathlineto{\pgfqpoint{4.612925in}{5.272501in}}%
\pgfpathlineto{\pgfqpoint{4.622247in}{5.272501in}}%
\pgfpathlineto{\pgfqpoint{4.631569in}{5.272501in}}%
\pgfpathlineto{\pgfqpoint{4.640890in}{5.272501in}}%
\pgfpathlineto{\pgfqpoint{4.650212in}{5.272501in}}%
\pgfpathlineto{\pgfqpoint{4.659534in}{5.272501in}}%
\pgfpathlineto{\pgfqpoint{4.668855in}{5.272501in}}%
\pgfpathlineto{\pgfqpoint{4.678177in}{5.272501in}}%
\pgfpathlineto{\pgfqpoint{4.687499in}{5.272501in}}%
\pgfpathlineto{\pgfqpoint{4.696821in}{5.272501in}}%
\pgfpathlineto{\pgfqpoint{4.706142in}{5.272501in}}%
\pgfpathlineto{\pgfqpoint{4.715464in}{5.272501in}}%
\pgfpathlineto{\pgfqpoint{4.724786in}{5.272501in}}%
\pgfpathlineto{\pgfqpoint{4.734107in}{5.272501in}}%
\pgfpathlineto{\pgfqpoint{4.743429in}{5.272501in}}%
\pgfpathlineto{\pgfqpoint{4.752751in}{5.272501in}}%
\pgfpathlineto{\pgfqpoint{4.762072in}{5.272501in}}%
\pgfpathlineto{\pgfqpoint{4.771394in}{5.272501in}}%
\pgfpathlineto{\pgfqpoint{4.780716in}{5.272501in}}%
\pgfpathlineto{\pgfqpoint{4.790037in}{5.272501in}}%
\pgfpathlineto{\pgfqpoint{4.799359in}{5.272501in}}%
\pgfpathlineto{\pgfqpoint{4.808681in}{5.272501in}}%
\pgfpathlineto{\pgfqpoint{4.818002in}{5.272501in}}%
\pgfpathlineto{\pgfqpoint{4.827324in}{5.272501in}}%
\pgfpathlineto{\pgfqpoint{4.836646in}{5.272501in}}%
\pgfpathlineto{\pgfqpoint{4.845968in}{5.272501in}}%
\pgfpathlineto{\pgfqpoint{4.855289in}{5.272501in}}%
\pgfpathlineto{\pgfqpoint{4.864611in}{5.272501in}}%
\pgfpathlineto{\pgfqpoint{4.873933in}{5.272501in}}%
\pgfpathlineto{\pgfqpoint{4.883254in}{5.272501in}}%
\pgfpathlineto{\pgfqpoint{4.892576in}{5.272501in}}%
\pgfpathlineto{\pgfqpoint{4.901898in}{5.272501in}}%
\pgfpathlineto{\pgfqpoint{4.911219in}{5.272501in}}%
\pgfpathlineto{\pgfqpoint{4.920541in}{5.272501in}}%
\pgfpathlineto{\pgfqpoint{4.929863in}{5.272501in}}%
\pgfpathlineto{\pgfqpoint{4.939184in}{5.272501in}}%
\pgfpathlineto{\pgfqpoint{4.948506in}{5.272501in}}%
\pgfpathlineto{\pgfqpoint{4.957828in}{5.272501in}}%
\pgfpathlineto{\pgfqpoint{4.967149in}{5.272501in}}%
\pgfpathlineto{\pgfqpoint{4.976471in}{5.272501in}}%
\pgfpathlineto{\pgfqpoint{4.985793in}{5.272501in}}%
\pgfpathlineto{\pgfqpoint{4.995115in}{5.272501in}}%
\pgfpathlineto{\pgfqpoint{5.004436in}{5.272501in}}%
\pgfpathlineto{\pgfqpoint{5.013758in}{5.272501in}}%
\pgfpathlineto{\pgfqpoint{5.023080in}{5.272501in}}%
\pgfpathlineto{\pgfqpoint{5.032401in}{5.272501in}}%
\pgfpathlineto{\pgfqpoint{5.041723in}{5.272501in}}%
\pgfpathlineto{\pgfqpoint{5.041723in}{5.273378in}}%
\pgfpathlineto{\pgfqpoint{5.041723in}{5.273378in}}%
\pgfpathlineto{\pgfqpoint{5.032401in}{5.273650in}}%
\pgfpathlineto{\pgfqpoint{5.023080in}{5.273996in}}%
\pgfpathlineto{\pgfqpoint{5.013758in}{5.274430in}}%
\pgfpathlineto{\pgfqpoint{5.004436in}{5.274970in}}%
\pgfpathlineto{\pgfqpoint{4.995115in}{5.275636in}}%
\pgfpathlineto{\pgfqpoint{4.985793in}{5.276450in}}%
\pgfpathlineto{\pgfqpoint{4.976471in}{5.277437in}}%
\pgfpathlineto{\pgfqpoint{4.967149in}{5.278622in}}%
\pgfpathlineto{\pgfqpoint{4.957828in}{5.280033in}}%
\pgfpathlineto{\pgfqpoint{4.948506in}{5.281698in}}%
\pgfpathlineto{\pgfqpoint{4.939184in}{5.283646in}}%
\pgfpathlineto{\pgfqpoint{4.929863in}{5.285903in}}%
\pgfpathlineto{\pgfqpoint{4.920541in}{5.288497in}}%
\pgfpathlineto{\pgfqpoint{4.911219in}{5.291452in}}%
\pgfpathlineto{\pgfqpoint{4.901898in}{5.294789in}}%
\pgfpathlineto{\pgfqpoint{4.892576in}{5.298525in}}%
\pgfpathlineto{\pgfqpoint{4.883254in}{5.302674in}}%
\pgfpathlineto{\pgfqpoint{4.873933in}{5.307241in}}%
\pgfpathlineto{\pgfqpoint{4.864611in}{5.312228in}}%
\pgfpathlineto{\pgfqpoint{4.855289in}{5.317632in}}%
\pgfpathlineto{\pgfqpoint{4.845968in}{5.323440in}}%
\pgfpathlineto{\pgfqpoint{4.836646in}{5.329637in}}%
\pgfpathlineto{\pgfqpoint{4.827324in}{5.336201in}}%
\pgfpathlineto{\pgfqpoint{4.818002in}{5.343108in}}%
\pgfpathlineto{\pgfqpoint{4.808681in}{5.350330in}}%
\pgfpathlineto{\pgfqpoint{4.799359in}{5.357837in}}%
\pgfpathlineto{\pgfqpoint{4.790037in}{5.365600in}}%
\pgfpathlineto{\pgfqpoint{4.780716in}{5.373593in}}%
\pgfpathlineto{\pgfqpoint{4.771394in}{5.381789in}}%
\pgfpathlineto{\pgfqpoint{4.762072in}{5.390172in}}%
\pgfpathlineto{\pgfqpoint{4.752751in}{5.398726in}}%
\pgfpathlineto{\pgfqpoint{4.743429in}{5.407445in}}%
\pgfpathlineto{\pgfqpoint{4.734107in}{5.416330in}}%
\pgfpathlineto{\pgfqpoint{4.724786in}{5.425391in}}%
\pgfpathlineto{\pgfqpoint{4.715464in}{5.434646in}}%
\pgfpathlineto{\pgfqpoint{4.706142in}{5.444121in}}%
\pgfpathlineto{\pgfqpoint{4.696821in}{5.453850in}}%
\pgfpathlineto{\pgfqpoint{4.687499in}{5.463874in}}%
\pgfpathlineto{\pgfqpoint{4.678177in}{5.474242in}}%
\pgfpathlineto{\pgfqpoint{4.668855in}{5.485008in}}%
\pgfpathlineto{\pgfqpoint{4.659534in}{5.496233in}}%
\pgfpathlineto{\pgfqpoint{4.650212in}{5.507981in}}%
\pgfpathlineto{\pgfqpoint{4.640890in}{5.520320in}}%
\pgfpathlineto{\pgfqpoint{4.631569in}{5.533321in}}%
\pgfpathlineto{\pgfqpoint{4.622247in}{5.547058in}}%
\pgfpathlineto{\pgfqpoint{4.612925in}{5.561603in}}%
\pgfpathlineto{\pgfqpoint{4.603604in}{5.577031in}}%
\pgfpathlineto{\pgfqpoint{4.594282in}{5.593412in}}%
\pgfpathlineto{\pgfqpoint{4.584960in}{5.610815in}}%
\pgfpathlineto{\pgfqpoint{4.575639in}{5.629302in}}%
\pgfpathlineto{\pgfqpoint{4.566317in}{5.648928in}}%
\pgfpathlineto{\pgfqpoint{4.556995in}{5.669738in}}%
\pgfpathlineto{\pgfqpoint{4.547674in}{5.691767in}}%
\pgfpathlineto{\pgfqpoint{4.538352in}{5.715033in}}%
\pgfpathlineto{\pgfqpoint{4.529030in}{5.739539in}}%
\pgfpathlineto{\pgfqpoint{4.519709in}{5.765273in}}%
\pgfpathlineto{\pgfqpoint{4.510387in}{5.792203in}}%
\pgfpathlineto{\pgfqpoint{4.501065in}{5.820280in}}%
\pgfpathlineto{\pgfqpoint{4.491743in}{5.849437in}}%
\pgfpathlineto{\pgfqpoint{4.482422in}{5.879593in}}%
\pgfpathlineto{\pgfqpoint{4.473100in}{5.910654in}}%
\pgfpathlineto{\pgfqpoint{4.463778in}{5.942519in}}%
\pgfpathlineto{\pgfqpoint{4.454457in}{5.975081in}}%
\pgfpathlineto{\pgfqpoint{4.445135in}{6.008235in}}%
\pgfpathlineto{\pgfqpoint{4.435813in}{6.041881in}}%
\pgfpathlineto{\pgfqpoint{4.426492in}{6.075932in}}%
\pgfpathlineto{\pgfqpoint{4.417170in}{6.110314in}}%
\pgfpathlineto{\pgfqpoint{4.407848in}{6.144972in}}%
\pgfpathlineto{\pgfqpoint{4.398527in}{6.179875in}}%
\pgfpathlineto{\pgfqpoint{4.389205in}{6.215009in}}%
\pgfpathlineto{\pgfqpoint{4.379883in}{6.250384in}}%
\pgfpathlineto{\pgfqpoint{4.370562in}{6.286025in}}%
\pgfpathlineto{\pgfqpoint{4.361240in}{6.321970in}}%
\pgfpathlineto{\pgfqpoint{4.351918in}{6.358264in}}%
\pgfpathlineto{\pgfqpoint{4.342596in}{6.394945in}}%
\pgfpathlineto{\pgfqpoint{4.333275in}{6.432043in}}%
\pgfpathlineto{\pgfqpoint{4.323953in}{6.469565in}}%
\pgfpathlineto{\pgfqpoint{4.314631in}{6.507485in}}%
\pgfpathlineto{\pgfqpoint{4.305310in}{6.545741in}}%
\pgfpathlineto{\pgfqpoint{4.295988in}{6.584220in}}%
\pgfpathlineto{\pgfqpoint{4.286666in}{6.622761in}}%
\pgfpathlineto{\pgfqpoint{4.277345in}{6.661148in}}%
\pgfpathlineto{\pgfqpoint{4.268023in}{6.699111in}}%
\pgfpathlineto{\pgfqpoint{4.258701in}{6.736334in}}%
\pgfpathlineto{\pgfqpoint{4.249380in}{6.772458in}}%
\pgfpathlineto{\pgfqpoint{4.240058in}{6.807095in}}%
\pgfpathlineto{\pgfqpoint{4.230736in}{6.839841in}}%
\pgfpathlineto{\pgfqpoint{4.221415in}{6.870288in}}%
\pgfpathlineto{\pgfqpoint{4.212093in}{6.898044in}}%
\pgfpathlineto{\pgfqpoint{4.202771in}{6.922749in}}%
\pgfpathlineto{\pgfqpoint{4.193449in}{6.944088in}}%
\pgfpathlineto{\pgfqpoint{4.184128in}{6.961811in}}%
\pgfpathlineto{\pgfqpoint{4.174806in}{6.975737in}}%
\pgfpathlineto{\pgfqpoint{4.165484in}{6.985767in}}%
\pgfpathlineto{\pgfqpoint{4.156163in}{6.991889in}}%
\pgfpathlineto{\pgfqpoint{4.146841in}{6.994174in}}%
\pgfpathlineto{\pgfqpoint{4.137519in}{6.992774in}}%
\pgfpathlineto{\pgfqpoint{4.128198in}{6.987910in}}%
\pgfpathlineto{\pgfqpoint{4.118876in}{6.979866in}}%
\pgfpathlineto{\pgfqpoint{4.109554in}{6.968965in}}%
\pgfpathlineto{\pgfqpoint{4.100233in}{6.955559in}}%
\pgfpathlineto{\pgfqpoint{4.090911in}{6.940004in}}%
\pgfpathlineto{\pgfqpoint{4.081589in}{6.922645in}}%
\pgfpathlineto{\pgfqpoint{4.072268in}{6.903798in}}%
\pgfpathlineto{\pgfqpoint{4.062946in}{6.883732in}}%
\pgfpathlineto{\pgfqpoint{4.053624in}{6.862658in}}%
\pgfpathlineto{\pgfqpoint{4.044303in}{6.840718in}}%
\pgfpathlineto{\pgfqpoint{4.034981in}{6.817985in}}%
\pgfpathlineto{\pgfqpoint{4.025659in}{6.794455in}}%
\pgfpathlineto{\pgfqpoint{4.016337in}{6.770055in}}%
\pgfpathlineto{\pgfqpoint{4.007016in}{6.744653in}}%
\pgfpathlineto{\pgfqpoint{3.997694in}{6.718061in}}%
\pgfpathlineto{\pgfqpoint{3.988372in}{6.690060in}}%
\pgfpathlineto{\pgfqpoint{3.979051in}{6.660406in}}%
\pgfpathlineto{\pgfqpoint{3.969729in}{6.628854in}}%
\pgfpathlineto{\pgfqpoint{3.960407in}{6.595173in}}%
\pgfpathlineto{\pgfqpoint{3.951086in}{6.559162in}}%
\pgfpathlineto{\pgfqpoint{3.941764in}{6.520668in}}%
\pgfpathlineto{\pgfqpoint{3.932442in}{6.479595in}}%
\pgfpathlineto{\pgfqpoint{3.923121in}{6.435918in}}%
\pgfpathlineto{\pgfqpoint{3.913799in}{6.389687in}}%
\pgfpathlineto{\pgfqpoint{3.904477in}{6.341031in}}%
\pgfpathlineto{\pgfqpoint{3.895156in}{6.290158in}}%
\pgfpathlineto{\pgfqpoint{3.885834in}{6.237346in}}%
\pgfpathlineto{\pgfqpoint{3.876512in}{6.182944in}}%
\pgfpathlineto{\pgfqpoint{3.867190in}{6.127353in}}%
\pgfpathlineto{\pgfqpoint{3.857869in}{6.071020in}}%
\pgfpathlineto{\pgfqpoint{3.848547in}{6.014421in}}%
\pgfpathlineto{\pgfqpoint{3.839225in}{5.958045in}}%
\pgfpathlineto{\pgfqpoint{3.829904in}{5.902381in}}%
\pgfpathlineto{\pgfqpoint{3.820582in}{5.847906in}}%
\pgfpathlineto{\pgfqpoint{3.811260in}{5.795067in}}%
\pgfpathlineto{\pgfqpoint{3.801939in}{5.744272in}}%
\pgfpathlineto{\pgfqpoint{3.792617in}{5.695880in}}%
\pgfpathlineto{\pgfqpoint{3.783295in}{5.650196in}}%
\pgfpathlineto{\pgfqpoint{3.773974in}{5.607465in}}%
\pgfpathlineto{\pgfqpoint{3.764652in}{5.567869in}}%
\pgfpathlineto{\pgfqpoint{3.755330in}{5.531530in}}%
\pgfpathlineto{\pgfqpoint{3.746009in}{5.498510in}}%
\pgfpathlineto{\pgfqpoint{3.736687in}{5.468815in}}%
\pgfpathlineto{\pgfqpoint{3.727365in}{5.442402in}}%
\pgfpathlineto{\pgfqpoint{3.718043in}{5.419184in}}%
\pgfpathlineto{\pgfqpoint{3.708722in}{5.399039in}}%
\pgfpathlineto{\pgfqpoint{3.699400in}{5.381812in}}%
\pgfpathlineto{\pgfqpoint{3.690078in}{5.367328in}}%
\pgfpathlineto{\pgfqpoint{3.680757in}{5.355394in}}%
\pgfpathlineto{\pgfqpoint{3.671435in}{5.345804in}}%
\pgfpathlineto{\pgfqpoint{3.662113in}{5.338348in}}%
\pgfpathlineto{\pgfqpoint{3.652792in}{5.332813in}}%
\pgfpathlineto{\pgfqpoint{3.643470in}{5.328986in}}%
\pgfpathlineto{\pgfqpoint{3.634148in}{5.326657in}}%
\pgfpathlineto{\pgfqpoint{3.624827in}{5.325620in}}%
\pgfpathlineto{\pgfqpoint{3.615505in}{5.325675in}}%
\pgfpathlineto{\pgfqpoint{3.606183in}{5.326630in}}%
\pgfpathlineto{\pgfqpoint{3.596862in}{5.328298in}}%
\pgfpathlineto{\pgfqpoint{3.587540in}{5.330502in}}%
\pgfpathlineto{\pgfqpoint{3.578218in}{5.333072in}}%
\pgfpathlineto{\pgfqpoint{3.568897in}{5.335849in}}%
\pgfpathlineto{\pgfqpoint{3.559575in}{5.338683in}}%
\pgfpathlineto{\pgfqpoint{3.550253in}{5.341437in}}%
\pgfpathlineto{\pgfqpoint{3.540931in}{5.343986in}}%
\pgfpathlineto{\pgfqpoint{3.531610in}{5.346220in}}%
\pgfpathlineto{\pgfqpoint{3.522288in}{5.348046in}}%
\pgfpathlineto{\pgfqpoint{3.512966in}{5.349386in}}%
\pgfpathlineto{\pgfqpoint{3.503645in}{5.350183in}}%
\pgfpathlineto{\pgfqpoint{3.494323in}{5.350395in}}%
\pgfpathlineto{\pgfqpoint{3.485001in}{5.350003in}}%
\pgfpathlineto{\pgfqpoint{3.475680in}{5.349004in}}%
\pgfpathlineto{\pgfqpoint{3.466358in}{5.347415in}}%
\pgfpathlineto{\pgfqpoint{3.457036in}{5.345266in}}%
\pgfpathlineto{\pgfqpoint{3.447715in}{5.342605in}}%
\pgfpathlineto{\pgfqpoint{3.438393in}{5.339490in}}%
\pgfpathlineto{\pgfqpoint{3.429071in}{5.335989in}}%
\pgfpathlineto{\pgfqpoint{3.419750in}{5.332178in}}%
\pgfpathlineto{\pgfqpoint{3.410428in}{5.328135in}}%
\pgfpathlineto{\pgfqpoint{3.401106in}{5.323938in}}%
\pgfpathlineto{\pgfqpoint{3.391784in}{5.319667in}}%
\pgfpathlineto{\pgfqpoint{3.382463in}{5.315394in}}%
\pgfpathlineto{\pgfqpoint{3.373141in}{5.311186in}}%
\pgfpathlineto{\pgfqpoint{3.363819in}{5.307104in}}%
\pgfpathlineto{\pgfqpoint{3.354498in}{5.303197in}}%
\pgfpathlineto{\pgfqpoint{3.345176in}{5.299507in}}%
\pgfpathlineto{\pgfqpoint{3.335854in}{5.296064in}}%
\pgfpathlineto{\pgfqpoint{3.326533in}{5.292891in}}%
\pgfpathlineto{\pgfqpoint{3.317211in}{5.289999in}}%
\pgfpathlineto{\pgfqpoint{3.307889in}{5.287394in}}%
\pgfpathlineto{\pgfqpoint{3.298568in}{5.285072in}}%
\pgfpathlineto{\pgfqpoint{3.289246in}{5.283024in}}%
\pgfpathlineto{\pgfqpoint{3.279924in}{5.281237in}}%
\pgfpathlineto{\pgfqpoint{3.270603in}{5.279694in}}%
\pgfpathlineto{\pgfqpoint{3.261281in}{5.278375in}}%
\pgfpathlineto{\pgfqpoint{3.251959in}{5.277258in}}%
\pgfpathlineto{\pgfqpoint{3.242637in}{5.276321in}}%
\pgfpathlineto{\pgfqpoint{3.233316in}{5.275544in}}%
\pgfpathlineto{\pgfqpoint{3.223994in}{5.274905in}}%
\pgfpathlineto{\pgfqpoint{3.214672in}{5.274384in}}%
\pgfpathlineto{\pgfqpoint{3.205351in}{5.273964in}}%
\pgfpathlineto{\pgfqpoint{3.196029in}{5.273628in}}%
\pgfpathlineto{\pgfqpoint{3.186707in}{5.273363in}}%
\pgfpathlineto{\pgfqpoint{3.186707in}{5.273363in}}%
\pgfpathclose%
\pgfusepath{stroke,fill}%
}%
\begin{pgfscope}%
\pgfsys@transformshift{0.000000in}{0.000000in}%
\pgfsys@useobject{currentmarker}{}%
\end{pgfscope}%
\end{pgfscope}%
\begin{pgfscope}%
\pgfpathrectangle{\pgfqpoint{5.292946in}{2.920818in}}{\pgfqpoint{2.177280in}{2.201755in}}%
\pgfusepath{clip}%
\pgfsetbuttcap%
\pgfsetroundjoin%
\definecolor{currentfill}{rgb}{0.172549,0.627451,0.172549}%
\pgfsetfillcolor{currentfill}%
\pgfsetfillopacity{0.250000}%
\pgfsetlinewidth{1.003750pt}%
\definecolor{currentstroke}{rgb}{0.172549,0.627451,0.172549}%
\pgfsetstrokecolor{currentstroke}%
\pgfsetdash{}{0pt}%
\pgfsys@defobject{currentmarker}{\pgfqpoint{6.247646in}{2.920818in}}{\pgfqpoint{7.371259in}{3.482494in}}{%
\pgfpathmoveto{\pgfqpoint{6.247646in}{2.921139in}}%
\pgfpathlineto{\pgfqpoint{6.247646in}{2.920818in}}%
\pgfpathlineto{\pgfqpoint{6.253293in}{2.920818in}}%
\pgfpathlineto{\pgfqpoint{6.258939in}{2.920818in}}%
\pgfpathlineto{\pgfqpoint{6.264585in}{2.920818in}}%
\pgfpathlineto{\pgfqpoint{6.270232in}{2.920818in}}%
\pgfpathlineto{\pgfqpoint{6.275878in}{2.920818in}}%
\pgfpathlineto{\pgfqpoint{6.281524in}{2.920818in}}%
\pgfpathlineto{\pgfqpoint{6.287171in}{2.920818in}}%
\pgfpathlineto{\pgfqpoint{6.292817in}{2.920818in}}%
\pgfpathlineto{\pgfqpoint{6.298463in}{2.920818in}}%
\pgfpathlineto{\pgfqpoint{6.304109in}{2.920818in}}%
\pgfpathlineto{\pgfqpoint{6.309756in}{2.920818in}}%
\pgfpathlineto{\pgfqpoint{6.315402in}{2.920818in}}%
\pgfpathlineto{\pgfqpoint{6.321048in}{2.920818in}}%
\pgfpathlineto{\pgfqpoint{6.326695in}{2.920818in}}%
\pgfpathlineto{\pgfqpoint{6.332341in}{2.920818in}}%
\pgfpathlineto{\pgfqpoint{6.337987in}{2.920818in}}%
\pgfpathlineto{\pgfqpoint{6.343633in}{2.920818in}}%
\pgfpathlineto{\pgfqpoint{6.349280in}{2.920818in}}%
\pgfpathlineto{\pgfqpoint{6.354926in}{2.920818in}}%
\pgfpathlineto{\pgfqpoint{6.360572in}{2.920818in}}%
\pgfpathlineto{\pgfqpoint{6.366219in}{2.920818in}}%
\pgfpathlineto{\pgfqpoint{6.371865in}{2.920818in}}%
\pgfpathlineto{\pgfqpoint{6.377511in}{2.920818in}}%
\pgfpathlineto{\pgfqpoint{6.383158in}{2.920818in}}%
\pgfpathlineto{\pgfqpoint{6.388804in}{2.920818in}}%
\pgfpathlineto{\pgfqpoint{6.394450in}{2.920818in}}%
\pgfpathlineto{\pgfqpoint{6.400096in}{2.920818in}}%
\pgfpathlineto{\pgfqpoint{6.405743in}{2.920818in}}%
\pgfpathlineto{\pgfqpoint{6.411389in}{2.920818in}}%
\pgfpathlineto{\pgfqpoint{6.417035in}{2.920818in}}%
\pgfpathlineto{\pgfqpoint{6.422682in}{2.920818in}}%
\pgfpathlineto{\pgfqpoint{6.428328in}{2.920818in}}%
\pgfpathlineto{\pgfqpoint{6.433974in}{2.920818in}}%
\pgfpathlineto{\pgfqpoint{6.439620in}{2.920818in}}%
\pgfpathlineto{\pgfqpoint{6.445267in}{2.920818in}}%
\pgfpathlineto{\pgfqpoint{6.450913in}{2.920818in}}%
\pgfpathlineto{\pgfqpoint{6.456559in}{2.920818in}}%
\pgfpathlineto{\pgfqpoint{6.462206in}{2.920818in}}%
\pgfpathlineto{\pgfqpoint{6.467852in}{2.920818in}}%
\pgfpathlineto{\pgfqpoint{6.473498in}{2.920818in}}%
\pgfpathlineto{\pgfqpoint{6.479145in}{2.920818in}}%
\pgfpathlineto{\pgfqpoint{6.484791in}{2.920818in}}%
\pgfpathlineto{\pgfqpoint{6.490437in}{2.920818in}}%
\pgfpathlineto{\pgfqpoint{6.496083in}{2.920818in}}%
\pgfpathlineto{\pgfqpoint{6.501730in}{2.920818in}}%
\pgfpathlineto{\pgfqpoint{6.507376in}{2.920818in}}%
\pgfpathlineto{\pgfqpoint{6.513022in}{2.920818in}}%
\pgfpathlineto{\pgfqpoint{6.518669in}{2.920818in}}%
\pgfpathlineto{\pgfqpoint{6.524315in}{2.920818in}}%
\pgfpathlineto{\pgfqpoint{6.529961in}{2.920818in}}%
\pgfpathlineto{\pgfqpoint{6.535607in}{2.920818in}}%
\pgfpathlineto{\pgfqpoint{6.541254in}{2.920818in}}%
\pgfpathlineto{\pgfqpoint{6.546900in}{2.920818in}}%
\pgfpathlineto{\pgfqpoint{6.552546in}{2.920818in}}%
\pgfpathlineto{\pgfqpoint{6.558193in}{2.920818in}}%
\pgfpathlineto{\pgfqpoint{6.563839in}{2.920818in}}%
\pgfpathlineto{\pgfqpoint{6.569485in}{2.920818in}}%
\pgfpathlineto{\pgfqpoint{6.575132in}{2.920818in}}%
\pgfpathlineto{\pgfqpoint{6.580778in}{2.920818in}}%
\pgfpathlineto{\pgfqpoint{6.586424in}{2.920818in}}%
\pgfpathlineto{\pgfqpoint{6.592070in}{2.920818in}}%
\pgfpathlineto{\pgfqpoint{6.597717in}{2.920818in}}%
\pgfpathlineto{\pgfqpoint{6.603363in}{2.920818in}}%
\pgfpathlineto{\pgfqpoint{6.609009in}{2.920818in}}%
\pgfpathlineto{\pgfqpoint{6.614656in}{2.920818in}}%
\pgfpathlineto{\pgfqpoint{6.620302in}{2.920818in}}%
\pgfpathlineto{\pgfqpoint{6.625948in}{2.920818in}}%
\pgfpathlineto{\pgfqpoint{6.631594in}{2.920818in}}%
\pgfpathlineto{\pgfqpoint{6.637241in}{2.920818in}}%
\pgfpathlineto{\pgfqpoint{6.642887in}{2.920818in}}%
\pgfpathlineto{\pgfqpoint{6.648533in}{2.920818in}}%
\pgfpathlineto{\pgfqpoint{6.654180in}{2.920818in}}%
\pgfpathlineto{\pgfqpoint{6.659826in}{2.920818in}}%
\pgfpathlineto{\pgfqpoint{6.665472in}{2.920818in}}%
\pgfpathlineto{\pgfqpoint{6.671119in}{2.920818in}}%
\pgfpathlineto{\pgfqpoint{6.676765in}{2.920818in}}%
\pgfpathlineto{\pgfqpoint{6.682411in}{2.920818in}}%
\pgfpathlineto{\pgfqpoint{6.688057in}{2.920818in}}%
\pgfpathlineto{\pgfqpoint{6.693704in}{2.920818in}}%
\pgfpathlineto{\pgfqpoint{6.699350in}{2.920818in}}%
\pgfpathlineto{\pgfqpoint{6.704996in}{2.920818in}}%
\pgfpathlineto{\pgfqpoint{6.710643in}{2.920818in}}%
\pgfpathlineto{\pgfqpoint{6.716289in}{2.920818in}}%
\pgfpathlineto{\pgfqpoint{6.721935in}{2.920818in}}%
\pgfpathlineto{\pgfqpoint{6.727581in}{2.920818in}}%
\pgfpathlineto{\pgfqpoint{6.733228in}{2.920818in}}%
\pgfpathlineto{\pgfqpoint{6.738874in}{2.920818in}}%
\pgfpathlineto{\pgfqpoint{6.744520in}{2.920818in}}%
\pgfpathlineto{\pgfqpoint{6.750167in}{2.920818in}}%
\pgfpathlineto{\pgfqpoint{6.755813in}{2.920818in}}%
\pgfpathlineto{\pgfqpoint{6.761459in}{2.920818in}}%
\pgfpathlineto{\pgfqpoint{6.767106in}{2.920818in}}%
\pgfpathlineto{\pgfqpoint{6.772752in}{2.920818in}}%
\pgfpathlineto{\pgfqpoint{6.778398in}{2.920818in}}%
\pgfpathlineto{\pgfqpoint{6.784044in}{2.920818in}}%
\pgfpathlineto{\pgfqpoint{6.789691in}{2.920818in}}%
\pgfpathlineto{\pgfqpoint{6.795337in}{2.920818in}}%
\pgfpathlineto{\pgfqpoint{6.800983in}{2.920818in}}%
\pgfpathlineto{\pgfqpoint{6.806630in}{2.920818in}}%
\pgfpathlineto{\pgfqpoint{6.812276in}{2.920818in}}%
\pgfpathlineto{\pgfqpoint{6.817922in}{2.920818in}}%
\pgfpathlineto{\pgfqpoint{6.823568in}{2.920818in}}%
\pgfpathlineto{\pgfqpoint{6.829215in}{2.920818in}}%
\pgfpathlineto{\pgfqpoint{6.834861in}{2.920818in}}%
\pgfpathlineto{\pgfqpoint{6.840507in}{2.920818in}}%
\pgfpathlineto{\pgfqpoint{6.846154in}{2.920818in}}%
\pgfpathlineto{\pgfqpoint{6.851800in}{2.920818in}}%
\pgfpathlineto{\pgfqpoint{6.857446in}{2.920818in}}%
\pgfpathlineto{\pgfqpoint{6.863093in}{2.920818in}}%
\pgfpathlineto{\pgfqpoint{6.868739in}{2.920818in}}%
\pgfpathlineto{\pgfqpoint{6.874385in}{2.920818in}}%
\pgfpathlineto{\pgfqpoint{6.880031in}{2.920818in}}%
\pgfpathlineto{\pgfqpoint{6.885678in}{2.920818in}}%
\pgfpathlineto{\pgfqpoint{6.891324in}{2.920818in}}%
\pgfpathlineto{\pgfqpoint{6.896970in}{2.920818in}}%
\pgfpathlineto{\pgfqpoint{6.902617in}{2.920818in}}%
\pgfpathlineto{\pgfqpoint{6.908263in}{2.920818in}}%
\pgfpathlineto{\pgfqpoint{6.913909in}{2.920818in}}%
\pgfpathlineto{\pgfqpoint{6.919555in}{2.920818in}}%
\pgfpathlineto{\pgfqpoint{6.925202in}{2.920818in}}%
\pgfpathlineto{\pgfqpoint{6.930848in}{2.920818in}}%
\pgfpathlineto{\pgfqpoint{6.936494in}{2.920818in}}%
\pgfpathlineto{\pgfqpoint{6.942141in}{2.920818in}}%
\pgfpathlineto{\pgfqpoint{6.947787in}{2.920818in}}%
\pgfpathlineto{\pgfqpoint{6.953433in}{2.920818in}}%
\pgfpathlineto{\pgfqpoint{6.959080in}{2.920818in}}%
\pgfpathlineto{\pgfqpoint{6.964726in}{2.920818in}}%
\pgfpathlineto{\pgfqpoint{6.970372in}{2.920818in}}%
\pgfpathlineto{\pgfqpoint{6.976018in}{2.920818in}}%
\pgfpathlineto{\pgfqpoint{6.981665in}{2.920818in}}%
\pgfpathlineto{\pgfqpoint{6.987311in}{2.920818in}}%
\pgfpathlineto{\pgfqpoint{6.992957in}{2.920818in}}%
\pgfpathlineto{\pgfqpoint{6.998604in}{2.920818in}}%
\pgfpathlineto{\pgfqpoint{7.004250in}{2.920818in}}%
\pgfpathlineto{\pgfqpoint{7.009896in}{2.920818in}}%
\pgfpathlineto{\pgfqpoint{7.015542in}{2.920818in}}%
\pgfpathlineto{\pgfqpoint{7.021189in}{2.920818in}}%
\pgfpathlineto{\pgfqpoint{7.026835in}{2.920818in}}%
\pgfpathlineto{\pgfqpoint{7.032481in}{2.920818in}}%
\pgfpathlineto{\pgfqpoint{7.038128in}{2.920818in}}%
\pgfpathlineto{\pgfqpoint{7.043774in}{2.920818in}}%
\pgfpathlineto{\pgfqpoint{7.049420in}{2.920818in}}%
\pgfpathlineto{\pgfqpoint{7.055067in}{2.920818in}}%
\pgfpathlineto{\pgfqpoint{7.060713in}{2.920818in}}%
\pgfpathlineto{\pgfqpoint{7.066359in}{2.920818in}}%
\pgfpathlineto{\pgfqpoint{7.072005in}{2.920818in}}%
\pgfpathlineto{\pgfqpoint{7.077652in}{2.920818in}}%
\pgfpathlineto{\pgfqpoint{7.083298in}{2.920818in}}%
\pgfpathlineto{\pgfqpoint{7.088944in}{2.920818in}}%
\pgfpathlineto{\pgfqpoint{7.094591in}{2.920818in}}%
\pgfpathlineto{\pgfqpoint{7.100237in}{2.920818in}}%
\pgfpathlineto{\pgfqpoint{7.105883in}{2.920818in}}%
\pgfpathlineto{\pgfqpoint{7.111529in}{2.920818in}}%
\pgfpathlineto{\pgfqpoint{7.117176in}{2.920818in}}%
\pgfpathlineto{\pgfqpoint{7.122822in}{2.920818in}}%
\pgfpathlineto{\pgfqpoint{7.128468in}{2.920818in}}%
\pgfpathlineto{\pgfqpoint{7.134115in}{2.920818in}}%
\pgfpathlineto{\pgfqpoint{7.139761in}{2.920818in}}%
\pgfpathlineto{\pgfqpoint{7.145407in}{2.920818in}}%
\pgfpathlineto{\pgfqpoint{7.151054in}{2.920818in}}%
\pgfpathlineto{\pgfqpoint{7.156700in}{2.920818in}}%
\pgfpathlineto{\pgfqpoint{7.162346in}{2.920818in}}%
\pgfpathlineto{\pgfqpoint{7.167992in}{2.920818in}}%
\pgfpathlineto{\pgfqpoint{7.173639in}{2.920818in}}%
\pgfpathlineto{\pgfqpoint{7.179285in}{2.920818in}}%
\pgfpathlineto{\pgfqpoint{7.184931in}{2.920818in}}%
\pgfpathlineto{\pgfqpoint{7.190578in}{2.920818in}}%
\pgfpathlineto{\pgfqpoint{7.196224in}{2.920818in}}%
\pgfpathlineto{\pgfqpoint{7.201870in}{2.920818in}}%
\pgfpathlineto{\pgfqpoint{7.207516in}{2.920818in}}%
\pgfpathlineto{\pgfqpoint{7.213163in}{2.920818in}}%
\pgfpathlineto{\pgfqpoint{7.218809in}{2.920818in}}%
\pgfpathlineto{\pgfqpoint{7.224455in}{2.920818in}}%
\pgfpathlineto{\pgfqpoint{7.230102in}{2.920818in}}%
\pgfpathlineto{\pgfqpoint{7.235748in}{2.920818in}}%
\pgfpathlineto{\pgfqpoint{7.241394in}{2.920818in}}%
\pgfpathlineto{\pgfqpoint{7.247041in}{2.920818in}}%
\pgfpathlineto{\pgfqpoint{7.252687in}{2.920818in}}%
\pgfpathlineto{\pgfqpoint{7.258333in}{2.920818in}}%
\pgfpathlineto{\pgfqpoint{7.263979in}{2.920818in}}%
\pgfpathlineto{\pgfqpoint{7.269626in}{2.920818in}}%
\pgfpathlineto{\pgfqpoint{7.275272in}{2.920818in}}%
\pgfpathlineto{\pgfqpoint{7.280918in}{2.920818in}}%
\pgfpathlineto{\pgfqpoint{7.286565in}{2.920818in}}%
\pgfpathlineto{\pgfqpoint{7.292211in}{2.920818in}}%
\pgfpathlineto{\pgfqpoint{7.297857in}{2.920818in}}%
\pgfpathlineto{\pgfqpoint{7.303503in}{2.920818in}}%
\pgfpathlineto{\pgfqpoint{7.309150in}{2.920818in}}%
\pgfpathlineto{\pgfqpoint{7.314796in}{2.920818in}}%
\pgfpathlineto{\pgfqpoint{7.320442in}{2.920818in}}%
\pgfpathlineto{\pgfqpoint{7.326089in}{2.920818in}}%
\pgfpathlineto{\pgfqpoint{7.331735in}{2.920818in}}%
\pgfpathlineto{\pgfqpoint{7.337381in}{2.920818in}}%
\pgfpathlineto{\pgfqpoint{7.343028in}{2.920818in}}%
\pgfpathlineto{\pgfqpoint{7.348674in}{2.920818in}}%
\pgfpathlineto{\pgfqpoint{7.354320in}{2.920818in}}%
\pgfpathlineto{\pgfqpoint{7.359966in}{2.920818in}}%
\pgfpathlineto{\pgfqpoint{7.365613in}{2.920818in}}%
\pgfpathlineto{\pgfqpoint{7.371259in}{2.920818in}}%
\pgfpathlineto{\pgfqpoint{7.371259in}{2.921174in}}%
\pgfpathlineto{\pgfqpoint{7.371259in}{2.921174in}}%
\pgfpathlineto{\pgfqpoint{7.365613in}{2.921270in}}%
\pgfpathlineto{\pgfqpoint{7.359966in}{2.921389in}}%
\pgfpathlineto{\pgfqpoint{7.354320in}{2.921535in}}%
\pgfpathlineto{\pgfqpoint{7.348674in}{2.921714in}}%
\pgfpathlineto{\pgfqpoint{7.343028in}{2.921932in}}%
\pgfpathlineto{\pgfqpoint{7.337381in}{2.922195in}}%
\pgfpathlineto{\pgfqpoint{7.331735in}{2.922511in}}%
\pgfpathlineto{\pgfqpoint{7.326089in}{2.922888in}}%
\pgfpathlineto{\pgfqpoint{7.320442in}{2.923336in}}%
\pgfpathlineto{\pgfqpoint{7.314796in}{2.923864in}}%
\pgfpathlineto{\pgfqpoint{7.309150in}{2.924484in}}%
\pgfpathlineto{\pgfqpoint{7.303503in}{2.925208in}}%
\pgfpathlineto{\pgfqpoint{7.297857in}{2.926046in}}%
\pgfpathlineto{\pgfqpoint{7.292211in}{2.927013in}}%
\pgfpathlineto{\pgfqpoint{7.286565in}{2.928121in}}%
\pgfpathlineto{\pgfqpoint{7.280918in}{2.929383in}}%
\pgfpathlineto{\pgfqpoint{7.275272in}{2.930813in}}%
\pgfpathlineto{\pgfqpoint{7.269626in}{2.932424in}}%
\pgfpathlineto{\pgfqpoint{7.263979in}{2.934227in}}%
\pgfpathlineto{\pgfqpoint{7.258333in}{2.936234in}}%
\pgfpathlineto{\pgfqpoint{7.252687in}{2.938455in}}%
\pgfpathlineto{\pgfqpoint{7.247041in}{2.940899in}}%
\pgfpathlineto{\pgfqpoint{7.241394in}{2.943572in}}%
\pgfpathlineto{\pgfqpoint{7.235748in}{2.946478in}}%
\pgfpathlineto{\pgfqpoint{7.230102in}{2.949620in}}%
\pgfpathlineto{\pgfqpoint{7.224455in}{2.952995in}}%
\pgfpathlineto{\pgfqpoint{7.218809in}{2.956599in}}%
\pgfpathlineto{\pgfqpoint{7.213163in}{2.960426in}}%
\pgfpathlineto{\pgfqpoint{7.207516in}{2.964463in}}%
\pgfpathlineto{\pgfqpoint{7.201870in}{2.968696in}}%
\pgfpathlineto{\pgfqpoint{7.196224in}{2.973108in}}%
\pgfpathlineto{\pgfqpoint{7.190578in}{2.977677in}}%
\pgfpathlineto{\pgfqpoint{7.184931in}{2.982378in}}%
\pgfpathlineto{\pgfqpoint{7.179285in}{2.987184in}}%
\pgfpathlineto{\pgfqpoint{7.173639in}{2.992066in}}%
\pgfpathlineto{\pgfqpoint{7.167992in}{2.996992in}}%
\pgfpathlineto{\pgfqpoint{7.162346in}{3.001930in}}%
\pgfpathlineto{\pgfqpoint{7.156700in}{3.006845in}}%
\pgfpathlineto{\pgfqpoint{7.151054in}{3.011707in}}%
\pgfpathlineto{\pgfqpoint{7.145407in}{3.016482in}}%
\pgfpathlineto{\pgfqpoint{7.139761in}{3.021140in}}%
\pgfpathlineto{\pgfqpoint{7.134115in}{3.025656in}}%
\pgfpathlineto{\pgfqpoint{7.128468in}{3.030004in}}%
\pgfpathlineto{\pgfqpoint{7.122822in}{3.034166in}}%
\pgfpathlineto{\pgfqpoint{7.117176in}{3.038128in}}%
\pgfpathlineto{\pgfqpoint{7.111529in}{3.041880in}}%
\pgfpathlineto{\pgfqpoint{7.105883in}{3.045422in}}%
\pgfpathlineto{\pgfqpoint{7.100237in}{3.048757in}}%
\pgfpathlineto{\pgfqpoint{7.094591in}{3.051896in}}%
\pgfpathlineto{\pgfqpoint{7.088944in}{3.054857in}}%
\pgfpathlineto{\pgfqpoint{7.083298in}{3.057664in}}%
\pgfpathlineto{\pgfqpoint{7.077652in}{3.060347in}}%
\pgfpathlineto{\pgfqpoint{7.072005in}{3.062943in}}%
\pgfpathlineto{\pgfqpoint{7.066359in}{3.065493in}}%
\pgfpathlineto{\pgfqpoint{7.060713in}{3.068043in}}%
\pgfpathlineto{\pgfqpoint{7.055067in}{3.070642in}}%
\pgfpathlineto{\pgfqpoint{7.049420in}{3.073341in}}%
\pgfpathlineto{\pgfqpoint{7.043774in}{3.076192in}}%
\pgfpathlineto{\pgfqpoint{7.038128in}{3.079249in}}%
\pgfpathlineto{\pgfqpoint{7.032481in}{3.082564in}}%
\pgfpathlineto{\pgfqpoint{7.026835in}{3.086185in}}%
\pgfpathlineto{\pgfqpoint{7.021189in}{3.090160in}}%
\pgfpathlineto{\pgfqpoint{7.015542in}{3.094531in}}%
\pgfpathlineto{\pgfqpoint{7.009896in}{3.099337in}}%
\pgfpathlineto{\pgfqpoint{7.004250in}{3.104610in}}%
\pgfpathlineto{\pgfqpoint{6.998604in}{3.110377in}}%
\pgfpathlineto{\pgfqpoint{6.992957in}{3.116656in}}%
\pgfpathlineto{\pgfqpoint{6.987311in}{3.123464in}}%
\pgfpathlineto{\pgfqpoint{6.981665in}{3.130805in}}%
\pgfpathlineto{\pgfqpoint{6.976018in}{3.138681in}}%
\pgfpathlineto{\pgfqpoint{6.970372in}{3.147086in}}%
\pgfpathlineto{\pgfqpoint{6.964726in}{3.156009in}}%
\pgfpathlineto{\pgfqpoint{6.959080in}{3.165432in}}%
\pgfpathlineto{\pgfqpoint{6.953433in}{3.175333in}}%
\pgfpathlineto{\pgfqpoint{6.947787in}{3.185685in}}%
\pgfpathlineto{\pgfqpoint{6.942141in}{3.196458in}}%
\pgfpathlineto{\pgfqpoint{6.936494in}{3.207617in}}%
\pgfpathlineto{\pgfqpoint{6.930848in}{3.219126in}}%
\pgfpathlineto{\pgfqpoint{6.925202in}{3.230944in}}%
\pgfpathlineto{\pgfqpoint{6.919555in}{3.243029in}}%
\pgfpathlineto{\pgfqpoint{6.913909in}{3.255337in}}%
\pgfpathlineto{\pgfqpoint{6.908263in}{3.267819in}}%
\pgfpathlineto{\pgfqpoint{6.902617in}{3.280428in}}%
\pgfpathlineto{\pgfqpoint{6.896970in}{3.293113in}}%
\pgfpathlineto{\pgfqpoint{6.891324in}{3.305820in}}%
\pgfpathlineto{\pgfqpoint{6.885678in}{3.318493in}}%
\pgfpathlineto{\pgfqpoint{6.880031in}{3.331076in}}%
\pgfpathlineto{\pgfqpoint{6.874385in}{3.343508in}}%
\pgfpathlineto{\pgfqpoint{6.868739in}{3.355728in}}%
\pgfpathlineto{\pgfqpoint{6.863093in}{3.367675in}}%
\pgfpathlineto{\pgfqpoint{6.857446in}{3.379283in}}%
\pgfpathlineto{\pgfqpoint{6.851800in}{3.390491in}}%
\pgfpathlineto{\pgfqpoint{6.846154in}{3.401234in}}%
\pgfpathlineto{\pgfqpoint{6.840507in}{3.411453in}}%
\pgfpathlineto{\pgfqpoint{6.834861in}{3.421091in}}%
\pgfpathlineto{\pgfqpoint{6.829215in}{3.430096in}}%
\pgfpathlineto{\pgfqpoint{6.823568in}{3.438422in}}%
\pgfpathlineto{\pgfqpoint{6.817922in}{3.446030in}}%
\pgfpathlineto{\pgfqpoint{6.812276in}{3.452892in}}%
\pgfpathlineto{\pgfqpoint{6.806630in}{3.458991in}}%
\pgfpathlineto{\pgfqpoint{6.800983in}{3.464319in}}%
\pgfpathlineto{\pgfqpoint{6.795337in}{3.468882in}}%
\pgfpathlineto{\pgfqpoint{6.789691in}{3.472699in}}%
\pgfpathlineto{\pgfqpoint{6.784044in}{3.475801in}}%
\pgfpathlineto{\pgfqpoint{6.778398in}{3.478231in}}%
\pgfpathlineto{\pgfqpoint{6.772752in}{3.480044in}}%
\pgfpathlineto{\pgfqpoint{6.767106in}{3.481303in}}%
\pgfpathlineto{\pgfqpoint{6.761459in}{3.482079in}}%
\pgfpathlineto{\pgfqpoint{6.755813in}{3.482450in}}%
\pgfpathlineto{\pgfqpoint{6.750167in}{3.482494in}}%
\pgfpathlineto{\pgfqpoint{6.744520in}{3.482291in}}%
\pgfpathlineto{\pgfqpoint{6.738874in}{3.481916in}}%
\pgfpathlineto{\pgfqpoint{6.733228in}{3.481437in}}%
\pgfpathlineto{\pgfqpoint{6.727581in}{3.480914in}}%
\pgfpathlineto{\pgfqpoint{6.721935in}{3.480395in}}%
\pgfpathlineto{\pgfqpoint{6.716289in}{3.479912in}}%
\pgfpathlineto{\pgfqpoint{6.710643in}{3.479482in}}%
\pgfpathlineto{\pgfqpoint{6.704996in}{3.479104in}}%
\pgfpathlineto{\pgfqpoint{6.699350in}{3.478758in}}%
\pgfpathlineto{\pgfqpoint{6.693704in}{3.478409in}}%
\pgfpathlineto{\pgfqpoint{6.688057in}{3.478000in}}%
\pgfpathlineto{\pgfqpoint{6.682411in}{3.477462in}}%
\pgfpathlineto{\pgfqpoint{6.676765in}{3.476710in}}%
\pgfpathlineto{\pgfqpoint{6.671119in}{3.475649in}}%
\pgfpathlineto{\pgfqpoint{6.665472in}{3.474174in}}%
\pgfpathlineto{\pgfqpoint{6.659826in}{3.472178in}}%
\pgfpathlineto{\pgfqpoint{6.654180in}{3.469550in}}%
\pgfpathlineto{\pgfqpoint{6.648533in}{3.466183in}}%
\pgfpathlineto{\pgfqpoint{6.642887in}{3.461976in}}%
\pgfpathlineto{\pgfqpoint{6.637241in}{3.456840in}}%
\pgfpathlineto{\pgfqpoint{6.631594in}{3.450695in}}%
\pgfpathlineto{\pgfqpoint{6.625948in}{3.443480in}}%
\pgfpathlineto{\pgfqpoint{6.620302in}{3.435151in}}%
\pgfpathlineto{\pgfqpoint{6.614656in}{3.425684in}}%
\pgfpathlineto{\pgfqpoint{6.609009in}{3.415074in}}%
\pgfpathlineto{\pgfqpoint{6.603363in}{3.403340in}}%
\pgfpathlineto{\pgfqpoint{6.597717in}{3.390519in}}%
\pgfpathlineto{\pgfqpoint{6.592070in}{3.376670in}}%
\pgfpathlineto{\pgfqpoint{6.586424in}{3.361869in}}%
\pgfpathlineto{\pgfqpoint{6.580778in}{3.346209in}}%
\pgfpathlineto{\pgfqpoint{6.575132in}{3.329797in}}%
\pgfpathlineto{\pgfqpoint{6.569485in}{3.312752in}}%
\pgfpathlineto{\pgfqpoint{6.563839in}{3.295200in}}%
\pgfpathlineto{\pgfqpoint{6.558193in}{3.277273in}}%
\pgfpathlineto{\pgfqpoint{6.552546in}{3.259108in}}%
\pgfpathlineto{\pgfqpoint{6.546900in}{3.240836in}}%
\pgfpathlineto{\pgfqpoint{6.541254in}{3.222590in}}%
\pgfpathlineto{\pgfqpoint{6.535607in}{3.204495in}}%
\pgfpathlineto{\pgfqpoint{6.529961in}{3.186667in}}%
\pgfpathlineto{\pgfqpoint{6.524315in}{3.169213in}}%
\pgfpathlineto{\pgfqpoint{6.518669in}{3.152228in}}%
\pgfpathlineto{\pgfqpoint{6.513022in}{3.135797in}}%
\pgfpathlineto{\pgfqpoint{6.507376in}{3.119988in}}%
\pgfpathlineto{\pgfqpoint{6.501730in}{3.104858in}}%
\pgfpathlineto{\pgfqpoint{6.496083in}{3.090451in}}%
\pgfpathlineto{\pgfqpoint{6.490437in}{3.076796in}}%
\pgfpathlineto{\pgfqpoint{6.484791in}{3.063912in}}%
\pgfpathlineto{\pgfqpoint{6.479145in}{3.051806in}}%
\pgfpathlineto{\pgfqpoint{6.473498in}{3.040474in}}%
\pgfpathlineto{\pgfqpoint{6.467852in}{3.029906in}}%
\pgfpathlineto{\pgfqpoint{6.462206in}{3.020082in}}%
\pgfpathlineto{\pgfqpoint{6.456559in}{3.010977in}}%
\pgfpathlineto{\pgfqpoint{6.450913in}{3.002561in}}%
\pgfpathlineto{\pgfqpoint{6.445267in}{2.994800in}}%
\pgfpathlineto{\pgfqpoint{6.439620in}{2.987659in}}%
\pgfpathlineto{\pgfqpoint{6.433974in}{2.981101in}}%
\pgfpathlineto{\pgfqpoint{6.428328in}{2.975089in}}%
\pgfpathlineto{\pgfqpoint{6.422682in}{2.969585in}}%
\pgfpathlineto{\pgfqpoint{6.417035in}{2.964555in}}%
\pgfpathlineto{\pgfqpoint{6.411389in}{2.959964in}}%
\pgfpathlineto{\pgfqpoint{6.405743in}{2.955778in}}%
\pgfpathlineto{\pgfqpoint{6.400096in}{2.951968in}}%
\pgfpathlineto{\pgfqpoint{6.394450in}{2.948505in}}%
\pgfpathlineto{\pgfqpoint{6.388804in}{2.945363in}}%
\pgfpathlineto{\pgfqpoint{6.383158in}{2.942516in}}%
\pgfpathlineto{\pgfqpoint{6.377511in}{2.939942in}}%
\pgfpathlineto{\pgfqpoint{6.371865in}{2.937620in}}%
\pgfpathlineto{\pgfqpoint{6.366219in}{2.935530in}}%
\pgfpathlineto{\pgfqpoint{6.360572in}{2.933655in}}%
\pgfpathlineto{\pgfqpoint{6.354926in}{2.931978in}}%
\pgfpathlineto{\pgfqpoint{6.349280in}{2.930482in}}%
\pgfpathlineto{\pgfqpoint{6.343633in}{2.929152in}}%
\pgfpathlineto{\pgfqpoint{6.337987in}{2.927975in}}%
\pgfpathlineto{\pgfqpoint{6.332341in}{2.926938in}}%
\pgfpathlineto{\pgfqpoint{6.326695in}{2.926027in}}%
\pgfpathlineto{\pgfqpoint{6.321048in}{2.925232in}}%
\pgfpathlineto{\pgfqpoint{6.315402in}{2.924540in}}%
\pgfpathlineto{\pgfqpoint{6.309756in}{2.923941in}}%
\pgfpathlineto{\pgfqpoint{6.304109in}{2.923426in}}%
\pgfpathlineto{\pgfqpoint{6.298463in}{2.922984in}}%
\pgfpathlineto{\pgfqpoint{6.292817in}{2.922608in}}%
\pgfpathlineto{\pgfqpoint{6.287171in}{2.922290in}}%
\pgfpathlineto{\pgfqpoint{6.281524in}{2.922022in}}%
\pgfpathlineto{\pgfqpoint{6.275878in}{2.921797in}}%
\pgfpathlineto{\pgfqpoint{6.270232in}{2.921610in}}%
\pgfpathlineto{\pgfqpoint{6.264585in}{2.921455in}}%
\pgfpathlineto{\pgfqpoint{6.258939in}{2.921328in}}%
\pgfpathlineto{\pgfqpoint{6.253293in}{2.921223in}}%
\pgfpathlineto{\pgfqpoint{6.247646in}{2.921139in}}%
\pgfpathlineto{\pgfqpoint{6.247646in}{2.921139in}}%
\pgfpathclose%
\pgfusepath{stroke,fill}%
}%
\begin{pgfscope}%
\pgfsys@transformshift{0.000000in}{0.000000in}%
\pgfsys@useobject{currentmarker}{}%
\end{pgfscope}%
\end{pgfscope}%
\begin{pgfscope}%
\pgfpathrectangle{\pgfqpoint{5.292946in}{2.920818in}}{\pgfqpoint{2.177280in}{2.201755in}}%
\pgfusepath{clip}%
\pgfsetbuttcap%
\pgfsetroundjoin%
\definecolor{currentfill}{rgb}{1.000000,0.498039,0.054902}%
\pgfsetfillcolor{currentfill}%
\pgfsetfillopacity{0.250000}%
\pgfsetlinewidth{1.003750pt}%
\definecolor{currentstroke}{rgb}{1.000000,0.498039,0.054902}%
\pgfsetstrokecolor{currentstroke}%
\pgfsetdash{}{0pt}%
\pgfsys@defobject{currentmarker}{\pgfqpoint{5.849353in}{2.920818in}}{\pgfqpoint{6.822278in}{3.623558in}}{%
\pgfpathmoveto{\pgfqpoint{5.849353in}{2.921185in}}%
\pgfpathlineto{\pgfqpoint{5.849353in}{2.920818in}}%
\pgfpathlineto{\pgfqpoint{5.854242in}{2.920818in}}%
\pgfpathlineto{\pgfqpoint{5.859131in}{2.920818in}}%
\pgfpathlineto{\pgfqpoint{5.864021in}{2.920818in}}%
\pgfpathlineto{\pgfqpoint{5.868910in}{2.920818in}}%
\pgfpathlineto{\pgfqpoint{5.873799in}{2.920818in}}%
\pgfpathlineto{\pgfqpoint{5.878688in}{2.920818in}}%
\pgfpathlineto{\pgfqpoint{5.883577in}{2.920818in}}%
\pgfpathlineto{\pgfqpoint{5.888466in}{2.920818in}}%
\pgfpathlineto{\pgfqpoint{5.893355in}{2.920818in}}%
\pgfpathlineto{\pgfqpoint{5.898244in}{2.920818in}}%
\pgfpathlineto{\pgfqpoint{5.903133in}{2.920818in}}%
\pgfpathlineto{\pgfqpoint{5.908022in}{2.920818in}}%
\pgfpathlineto{\pgfqpoint{5.912911in}{2.920818in}}%
\pgfpathlineto{\pgfqpoint{5.917800in}{2.920818in}}%
\pgfpathlineto{\pgfqpoint{5.922689in}{2.920818in}}%
\pgfpathlineto{\pgfqpoint{5.927578in}{2.920818in}}%
\pgfpathlineto{\pgfqpoint{5.932467in}{2.920818in}}%
\pgfpathlineto{\pgfqpoint{5.937357in}{2.920818in}}%
\pgfpathlineto{\pgfqpoint{5.942246in}{2.920818in}}%
\pgfpathlineto{\pgfqpoint{5.947135in}{2.920818in}}%
\pgfpathlineto{\pgfqpoint{5.952024in}{2.920818in}}%
\pgfpathlineto{\pgfqpoint{5.956913in}{2.920818in}}%
\pgfpathlineto{\pgfqpoint{5.961802in}{2.920818in}}%
\pgfpathlineto{\pgfqpoint{5.966691in}{2.920818in}}%
\pgfpathlineto{\pgfqpoint{5.971580in}{2.920818in}}%
\pgfpathlineto{\pgfqpoint{5.976469in}{2.920818in}}%
\pgfpathlineto{\pgfqpoint{5.981358in}{2.920818in}}%
\pgfpathlineto{\pgfqpoint{5.986247in}{2.920818in}}%
\pgfpathlineto{\pgfqpoint{5.991136in}{2.920818in}}%
\pgfpathlineto{\pgfqpoint{5.996025in}{2.920818in}}%
\pgfpathlineto{\pgfqpoint{6.000914in}{2.920818in}}%
\pgfpathlineto{\pgfqpoint{6.005803in}{2.920818in}}%
\pgfpathlineto{\pgfqpoint{6.010693in}{2.920818in}}%
\pgfpathlineto{\pgfqpoint{6.015582in}{2.920818in}}%
\pgfpathlineto{\pgfqpoint{6.020471in}{2.920818in}}%
\pgfpathlineto{\pgfqpoint{6.025360in}{2.920818in}}%
\pgfpathlineto{\pgfqpoint{6.030249in}{2.920818in}}%
\pgfpathlineto{\pgfqpoint{6.035138in}{2.920818in}}%
\pgfpathlineto{\pgfqpoint{6.040027in}{2.920818in}}%
\pgfpathlineto{\pgfqpoint{6.044916in}{2.920818in}}%
\pgfpathlineto{\pgfqpoint{6.049805in}{2.920818in}}%
\pgfpathlineto{\pgfqpoint{6.054694in}{2.920818in}}%
\pgfpathlineto{\pgfqpoint{6.059583in}{2.920818in}}%
\pgfpathlineto{\pgfqpoint{6.064472in}{2.920818in}}%
\pgfpathlineto{\pgfqpoint{6.069361in}{2.920818in}}%
\pgfpathlineto{\pgfqpoint{6.074250in}{2.920818in}}%
\pgfpathlineto{\pgfqpoint{6.079139in}{2.920818in}}%
\pgfpathlineto{\pgfqpoint{6.084029in}{2.920818in}}%
\pgfpathlineto{\pgfqpoint{6.088918in}{2.920818in}}%
\pgfpathlineto{\pgfqpoint{6.093807in}{2.920818in}}%
\pgfpathlineto{\pgfqpoint{6.098696in}{2.920818in}}%
\pgfpathlineto{\pgfqpoint{6.103585in}{2.920818in}}%
\pgfpathlineto{\pgfqpoint{6.108474in}{2.920818in}}%
\pgfpathlineto{\pgfqpoint{6.113363in}{2.920818in}}%
\pgfpathlineto{\pgfqpoint{6.118252in}{2.920818in}}%
\pgfpathlineto{\pgfqpoint{6.123141in}{2.920818in}}%
\pgfpathlineto{\pgfqpoint{6.128030in}{2.920818in}}%
\pgfpathlineto{\pgfqpoint{6.132919in}{2.920818in}}%
\pgfpathlineto{\pgfqpoint{6.137808in}{2.920818in}}%
\pgfpathlineto{\pgfqpoint{6.142697in}{2.920818in}}%
\pgfpathlineto{\pgfqpoint{6.147586in}{2.920818in}}%
\pgfpathlineto{\pgfqpoint{6.152475in}{2.920818in}}%
\pgfpathlineto{\pgfqpoint{6.157365in}{2.920818in}}%
\pgfpathlineto{\pgfqpoint{6.162254in}{2.920818in}}%
\pgfpathlineto{\pgfqpoint{6.167143in}{2.920818in}}%
\pgfpathlineto{\pgfqpoint{6.172032in}{2.920818in}}%
\pgfpathlineto{\pgfqpoint{6.176921in}{2.920818in}}%
\pgfpathlineto{\pgfqpoint{6.181810in}{2.920818in}}%
\pgfpathlineto{\pgfqpoint{6.186699in}{2.920818in}}%
\pgfpathlineto{\pgfqpoint{6.191588in}{2.920818in}}%
\pgfpathlineto{\pgfqpoint{6.196477in}{2.920818in}}%
\pgfpathlineto{\pgfqpoint{6.201366in}{2.920818in}}%
\pgfpathlineto{\pgfqpoint{6.206255in}{2.920818in}}%
\pgfpathlineto{\pgfqpoint{6.211144in}{2.920818in}}%
\pgfpathlineto{\pgfqpoint{6.216033in}{2.920818in}}%
\pgfpathlineto{\pgfqpoint{6.220922in}{2.920818in}}%
\pgfpathlineto{\pgfqpoint{6.225811in}{2.920818in}}%
\pgfpathlineto{\pgfqpoint{6.230701in}{2.920818in}}%
\pgfpathlineto{\pgfqpoint{6.235590in}{2.920818in}}%
\pgfpathlineto{\pgfqpoint{6.240479in}{2.920818in}}%
\pgfpathlineto{\pgfqpoint{6.245368in}{2.920818in}}%
\pgfpathlineto{\pgfqpoint{6.250257in}{2.920818in}}%
\pgfpathlineto{\pgfqpoint{6.255146in}{2.920818in}}%
\pgfpathlineto{\pgfqpoint{6.260035in}{2.920818in}}%
\pgfpathlineto{\pgfqpoint{6.264924in}{2.920818in}}%
\pgfpathlineto{\pgfqpoint{6.269813in}{2.920818in}}%
\pgfpathlineto{\pgfqpoint{6.274702in}{2.920818in}}%
\pgfpathlineto{\pgfqpoint{6.279591in}{2.920818in}}%
\pgfpathlineto{\pgfqpoint{6.284480in}{2.920818in}}%
\pgfpathlineto{\pgfqpoint{6.289369in}{2.920818in}}%
\pgfpathlineto{\pgfqpoint{6.294258in}{2.920818in}}%
\pgfpathlineto{\pgfqpoint{6.299148in}{2.920818in}}%
\pgfpathlineto{\pgfqpoint{6.304037in}{2.920818in}}%
\pgfpathlineto{\pgfqpoint{6.308926in}{2.920818in}}%
\pgfpathlineto{\pgfqpoint{6.313815in}{2.920818in}}%
\pgfpathlineto{\pgfqpoint{6.318704in}{2.920818in}}%
\pgfpathlineto{\pgfqpoint{6.323593in}{2.920818in}}%
\pgfpathlineto{\pgfqpoint{6.328482in}{2.920818in}}%
\pgfpathlineto{\pgfqpoint{6.333371in}{2.920818in}}%
\pgfpathlineto{\pgfqpoint{6.338260in}{2.920818in}}%
\pgfpathlineto{\pgfqpoint{6.343149in}{2.920818in}}%
\pgfpathlineto{\pgfqpoint{6.348038in}{2.920818in}}%
\pgfpathlineto{\pgfqpoint{6.352927in}{2.920818in}}%
\pgfpathlineto{\pgfqpoint{6.357816in}{2.920818in}}%
\pgfpathlineto{\pgfqpoint{6.362705in}{2.920818in}}%
\pgfpathlineto{\pgfqpoint{6.367594in}{2.920818in}}%
\pgfpathlineto{\pgfqpoint{6.372484in}{2.920818in}}%
\pgfpathlineto{\pgfqpoint{6.377373in}{2.920818in}}%
\pgfpathlineto{\pgfqpoint{6.382262in}{2.920818in}}%
\pgfpathlineto{\pgfqpoint{6.387151in}{2.920818in}}%
\pgfpathlineto{\pgfqpoint{6.392040in}{2.920818in}}%
\pgfpathlineto{\pgfqpoint{6.396929in}{2.920818in}}%
\pgfpathlineto{\pgfqpoint{6.401818in}{2.920818in}}%
\pgfpathlineto{\pgfqpoint{6.406707in}{2.920818in}}%
\pgfpathlineto{\pgfqpoint{6.411596in}{2.920818in}}%
\pgfpathlineto{\pgfqpoint{6.416485in}{2.920818in}}%
\pgfpathlineto{\pgfqpoint{6.421374in}{2.920818in}}%
\pgfpathlineto{\pgfqpoint{6.426263in}{2.920818in}}%
\pgfpathlineto{\pgfqpoint{6.431152in}{2.920818in}}%
\pgfpathlineto{\pgfqpoint{6.436041in}{2.920818in}}%
\pgfpathlineto{\pgfqpoint{6.440930in}{2.920818in}}%
\pgfpathlineto{\pgfqpoint{6.445820in}{2.920818in}}%
\pgfpathlineto{\pgfqpoint{6.450709in}{2.920818in}}%
\pgfpathlineto{\pgfqpoint{6.455598in}{2.920818in}}%
\pgfpathlineto{\pgfqpoint{6.460487in}{2.920818in}}%
\pgfpathlineto{\pgfqpoint{6.465376in}{2.920818in}}%
\pgfpathlineto{\pgfqpoint{6.470265in}{2.920818in}}%
\pgfpathlineto{\pgfqpoint{6.475154in}{2.920818in}}%
\pgfpathlineto{\pgfqpoint{6.480043in}{2.920818in}}%
\pgfpathlineto{\pgfqpoint{6.484932in}{2.920818in}}%
\pgfpathlineto{\pgfqpoint{6.489821in}{2.920818in}}%
\pgfpathlineto{\pgfqpoint{6.494710in}{2.920818in}}%
\pgfpathlineto{\pgfqpoint{6.499599in}{2.920818in}}%
\pgfpathlineto{\pgfqpoint{6.504488in}{2.920818in}}%
\pgfpathlineto{\pgfqpoint{6.509377in}{2.920818in}}%
\pgfpathlineto{\pgfqpoint{6.514266in}{2.920818in}}%
\pgfpathlineto{\pgfqpoint{6.519156in}{2.920818in}}%
\pgfpathlineto{\pgfqpoint{6.524045in}{2.920818in}}%
\pgfpathlineto{\pgfqpoint{6.528934in}{2.920818in}}%
\pgfpathlineto{\pgfqpoint{6.533823in}{2.920818in}}%
\pgfpathlineto{\pgfqpoint{6.538712in}{2.920818in}}%
\pgfpathlineto{\pgfqpoint{6.543601in}{2.920818in}}%
\pgfpathlineto{\pgfqpoint{6.548490in}{2.920818in}}%
\pgfpathlineto{\pgfqpoint{6.553379in}{2.920818in}}%
\pgfpathlineto{\pgfqpoint{6.558268in}{2.920818in}}%
\pgfpathlineto{\pgfqpoint{6.563157in}{2.920818in}}%
\pgfpathlineto{\pgfqpoint{6.568046in}{2.920818in}}%
\pgfpathlineto{\pgfqpoint{6.572935in}{2.920818in}}%
\pgfpathlineto{\pgfqpoint{6.577824in}{2.920818in}}%
\pgfpathlineto{\pgfqpoint{6.582713in}{2.920818in}}%
\pgfpathlineto{\pgfqpoint{6.587602in}{2.920818in}}%
\pgfpathlineto{\pgfqpoint{6.592492in}{2.920818in}}%
\pgfpathlineto{\pgfqpoint{6.597381in}{2.920818in}}%
\pgfpathlineto{\pgfqpoint{6.602270in}{2.920818in}}%
\pgfpathlineto{\pgfqpoint{6.607159in}{2.920818in}}%
\pgfpathlineto{\pgfqpoint{6.612048in}{2.920818in}}%
\pgfpathlineto{\pgfqpoint{6.616937in}{2.920818in}}%
\pgfpathlineto{\pgfqpoint{6.621826in}{2.920818in}}%
\pgfpathlineto{\pgfqpoint{6.626715in}{2.920818in}}%
\pgfpathlineto{\pgfqpoint{6.631604in}{2.920818in}}%
\pgfpathlineto{\pgfqpoint{6.636493in}{2.920818in}}%
\pgfpathlineto{\pgfqpoint{6.641382in}{2.920818in}}%
\pgfpathlineto{\pgfqpoint{6.646271in}{2.920818in}}%
\pgfpathlineto{\pgfqpoint{6.651160in}{2.920818in}}%
\pgfpathlineto{\pgfqpoint{6.656049in}{2.920818in}}%
\pgfpathlineto{\pgfqpoint{6.660938in}{2.920818in}}%
\pgfpathlineto{\pgfqpoint{6.665828in}{2.920818in}}%
\pgfpathlineto{\pgfqpoint{6.670717in}{2.920818in}}%
\pgfpathlineto{\pgfqpoint{6.675606in}{2.920818in}}%
\pgfpathlineto{\pgfqpoint{6.680495in}{2.920818in}}%
\pgfpathlineto{\pgfqpoint{6.685384in}{2.920818in}}%
\pgfpathlineto{\pgfqpoint{6.690273in}{2.920818in}}%
\pgfpathlineto{\pgfqpoint{6.695162in}{2.920818in}}%
\pgfpathlineto{\pgfqpoint{6.700051in}{2.920818in}}%
\pgfpathlineto{\pgfqpoint{6.704940in}{2.920818in}}%
\pgfpathlineto{\pgfqpoint{6.709829in}{2.920818in}}%
\pgfpathlineto{\pgfqpoint{6.714718in}{2.920818in}}%
\pgfpathlineto{\pgfqpoint{6.719607in}{2.920818in}}%
\pgfpathlineto{\pgfqpoint{6.724496in}{2.920818in}}%
\pgfpathlineto{\pgfqpoint{6.729385in}{2.920818in}}%
\pgfpathlineto{\pgfqpoint{6.734274in}{2.920818in}}%
\pgfpathlineto{\pgfqpoint{6.739164in}{2.920818in}}%
\pgfpathlineto{\pgfqpoint{6.744053in}{2.920818in}}%
\pgfpathlineto{\pgfqpoint{6.748942in}{2.920818in}}%
\pgfpathlineto{\pgfqpoint{6.753831in}{2.920818in}}%
\pgfpathlineto{\pgfqpoint{6.758720in}{2.920818in}}%
\pgfpathlineto{\pgfqpoint{6.763609in}{2.920818in}}%
\pgfpathlineto{\pgfqpoint{6.768498in}{2.920818in}}%
\pgfpathlineto{\pgfqpoint{6.773387in}{2.920818in}}%
\pgfpathlineto{\pgfqpoint{6.778276in}{2.920818in}}%
\pgfpathlineto{\pgfqpoint{6.783165in}{2.920818in}}%
\pgfpathlineto{\pgfqpoint{6.788054in}{2.920818in}}%
\pgfpathlineto{\pgfqpoint{6.792943in}{2.920818in}}%
\pgfpathlineto{\pgfqpoint{6.797832in}{2.920818in}}%
\pgfpathlineto{\pgfqpoint{6.802721in}{2.920818in}}%
\pgfpathlineto{\pgfqpoint{6.807610in}{2.920818in}}%
\pgfpathlineto{\pgfqpoint{6.812500in}{2.920818in}}%
\pgfpathlineto{\pgfqpoint{6.817389in}{2.920818in}}%
\pgfpathlineto{\pgfqpoint{6.822278in}{2.920818in}}%
\pgfpathlineto{\pgfqpoint{6.822278in}{2.921296in}}%
\pgfpathlineto{\pgfqpoint{6.822278in}{2.921296in}}%
\pgfpathlineto{\pgfqpoint{6.817389in}{2.921430in}}%
\pgfpathlineto{\pgfqpoint{6.812500in}{2.921597in}}%
\pgfpathlineto{\pgfqpoint{6.807610in}{2.921804in}}%
\pgfpathlineto{\pgfqpoint{6.802721in}{2.922059in}}%
\pgfpathlineto{\pgfqpoint{6.797832in}{2.922371in}}%
\pgfpathlineto{\pgfqpoint{6.792943in}{2.922752in}}%
\pgfpathlineto{\pgfqpoint{6.788054in}{2.923212in}}%
\pgfpathlineto{\pgfqpoint{6.783165in}{2.923765in}}%
\pgfpathlineto{\pgfqpoint{6.778276in}{2.924428in}}%
\pgfpathlineto{\pgfqpoint{6.773387in}{2.925215in}}%
\pgfpathlineto{\pgfqpoint{6.768498in}{2.926146in}}%
\pgfpathlineto{\pgfqpoint{6.763609in}{2.927241in}}%
\pgfpathlineto{\pgfqpoint{6.758720in}{2.928520in}}%
\pgfpathlineto{\pgfqpoint{6.753831in}{2.930008in}}%
\pgfpathlineto{\pgfqpoint{6.748942in}{2.931727in}}%
\pgfpathlineto{\pgfqpoint{6.744053in}{2.933704in}}%
\pgfpathlineto{\pgfqpoint{6.739164in}{2.935966in}}%
\pgfpathlineto{\pgfqpoint{6.734274in}{2.938538in}}%
\pgfpathlineto{\pgfqpoint{6.729385in}{2.941450in}}%
\pgfpathlineto{\pgfqpoint{6.724496in}{2.944728in}}%
\pgfpathlineto{\pgfqpoint{6.719607in}{2.948401in}}%
\pgfpathlineto{\pgfqpoint{6.714718in}{2.952496in}}%
\pgfpathlineto{\pgfqpoint{6.709829in}{2.957040in}}%
\pgfpathlineto{\pgfqpoint{6.704940in}{2.962059in}}%
\pgfpathlineto{\pgfqpoint{6.700051in}{2.967578in}}%
\pgfpathlineto{\pgfqpoint{6.695162in}{2.973620in}}%
\pgfpathlineto{\pgfqpoint{6.690273in}{2.980206in}}%
\pgfpathlineto{\pgfqpoint{6.685384in}{2.987356in}}%
\pgfpathlineto{\pgfqpoint{6.680495in}{2.995088in}}%
\pgfpathlineto{\pgfqpoint{6.675606in}{3.003418in}}%
\pgfpathlineto{\pgfqpoint{6.670717in}{3.012359in}}%
\pgfpathlineto{\pgfqpoint{6.665828in}{3.021923in}}%
\pgfpathlineto{\pgfqpoint{6.660938in}{3.032118in}}%
\pgfpathlineto{\pgfqpoint{6.656049in}{3.042951in}}%
\pgfpathlineto{\pgfqpoint{6.651160in}{3.054426in}}%
\pgfpathlineto{\pgfqpoint{6.646271in}{3.066546in}}%
\pgfpathlineto{\pgfqpoint{6.641382in}{3.079309in}}%
\pgfpathlineto{\pgfqpoint{6.636493in}{3.092712in}}%
\pgfpathlineto{\pgfqpoint{6.631604in}{3.106749in}}%
\pgfpathlineto{\pgfqpoint{6.626715in}{3.121412in}}%
\pgfpathlineto{\pgfqpoint{6.621826in}{3.136686in}}%
\pgfpathlineto{\pgfqpoint{6.616937in}{3.152557in}}%
\pgfpathlineto{\pgfqpoint{6.612048in}{3.169004in}}%
\pgfpathlineto{\pgfqpoint{6.607159in}{3.186002in}}%
\pgfpathlineto{\pgfqpoint{6.602270in}{3.203521in}}%
\pgfpathlineto{\pgfqpoint{6.597381in}{3.221527in}}%
\pgfpathlineto{\pgfqpoint{6.592492in}{3.239977in}}%
\pgfpathlineto{\pgfqpoint{6.587602in}{3.258826in}}%
\pgfpathlineto{\pgfqpoint{6.582713in}{3.278019in}}%
\pgfpathlineto{\pgfqpoint{6.577824in}{3.297496in}}%
\pgfpathlineto{\pgfqpoint{6.572935in}{3.317189in}}%
\pgfpathlineto{\pgfqpoint{6.568046in}{3.337023in}}%
\pgfpathlineto{\pgfqpoint{6.563157in}{3.356917in}}%
\pgfpathlineto{\pgfqpoint{6.558268in}{3.376783in}}%
\pgfpathlineto{\pgfqpoint{6.553379in}{3.396527in}}%
\pgfpathlineto{\pgfqpoint{6.548490in}{3.416050in}}%
\pgfpathlineto{\pgfqpoint{6.543601in}{3.435250in}}%
\pgfpathlineto{\pgfqpoint{6.538712in}{3.454020in}}%
\pgfpathlineto{\pgfqpoint{6.533823in}{3.472252in}}%
\pgfpathlineto{\pgfqpoint{6.528934in}{3.489839in}}%
\pgfpathlineto{\pgfqpoint{6.524045in}{3.506674in}}%
\pgfpathlineto{\pgfqpoint{6.519156in}{3.522655in}}%
\pgfpathlineto{\pgfqpoint{6.514266in}{3.537685in}}%
\pgfpathlineto{\pgfqpoint{6.509377in}{3.551674in}}%
\pgfpathlineto{\pgfqpoint{6.504488in}{3.564541in}}%
\pgfpathlineto{\pgfqpoint{6.499599in}{3.576217in}}%
\pgfpathlineto{\pgfqpoint{6.494710in}{3.586646in}}%
\pgfpathlineto{\pgfqpoint{6.489821in}{3.595787in}}%
\pgfpathlineto{\pgfqpoint{6.484932in}{3.603612in}}%
\pgfpathlineto{\pgfqpoint{6.480043in}{3.610112in}}%
\pgfpathlineto{\pgfqpoint{6.475154in}{3.615294in}}%
\pgfpathlineto{\pgfqpoint{6.470265in}{3.619181in}}%
\pgfpathlineto{\pgfqpoint{6.465376in}{3.621815in}}%
\pgfpathlineto{\pgfqpoint{6.460487in}{3.623250in}}%
\pgfpathlineto{\pgfqpoint{6.455598in}{3.623558in}}%
\pgfpathlineto{\pgfqpoint{6.450709in}{3.622820in}}%
\pgfpathlineto{\pgfqpoint{6.445820in}{3.621129in}}%
\pgfpathlineto{\pgfqpoint{6.440930in}{3.618584in}}%
\pgfpathlineto{\pgfqpoint{6.436041in}{3.615290in}}%
\pgfpathlineto{\pgfqpoint{6.431152in}{3.611354in}}%
\pgfpathlineto{\pgfqpoint{6.426263in}{3.606879in}}%
\pgfpathlineto{\pgfqpoint{6.421374in}{3.601967in}}%
\pgfpathlineto{\pgfqpoint{6.416485in}{3.596708in}}%
\pgfpathlineto{\pgfqpoint{6.411596in}{3.591186in}}%
\pgfpathlineto{\pgfqpoint{6.406707in}{3.585470in}}%
\pgfpathlineto{\pgfqpoint{6.401818in}{3.579616in}}%
\pgfpathlineto{\pgfqpoint{6.396929in}{3.573665in}}%
\pgfpathlineto{\pgfqpoint{6.392040in}{3.567640in}}%
\pgfpathlineto{\pgfqpoint{6.387151in}{3.561549in}}%
\pgfpathlineto{\pgfqpoint{6.382262in}{3.555384in}}%
\pgfpathlineto{\pgfqpoint{6.377373in}{3.549124in}}%
\pgfpathlineto{\pgfqpoint{6.372484in}{3.542731in}}%
\pgfpathlineto{\pgfqpoint{6.367594in}{3.536159in}}%
\pgfpathlineto{\pgfqpoint{6.362705in}{3.529355in}}%
\pgfpathlineto{\pgfqpoint{6.357816in}{3.522258in}}%
\pgfpathlineto{\pgfqpoint{6.352927in}{3.514806in}}%
\pgfpathlineto{\pgfqpoint{6.348038in}{3.506938in}}%
\pgfpathlineto{\pgfqpoint{6.343149in}{3.498596in}}%
\pgfpathlineto{\pgfqpoint{6.338260in}{3.489731in}}%
\pgfpathlineto{\pgfqpoint{6.333371in}{3.480299in}}%
\pgfpathlineto{\pgfqpoint{6.328482in}{3.470273in}}%
\pgfpathlineto{\pgfqpoint{6.323593in}{3.459636in}}%
\pgfpathlineto{\pgfqpoint{6.318704in}{3.448387in}}%
\pgfpathlineto{\pgfqpoint{6.313815in}{3.436540in}}%
\pgfpathlineto{\pgfqpoint{6.308926in}{3.424124in}}%
\pgfpathlineto{\pgfqpoint{6.304037in}{3.411184in}}%
\pgfpathlineto{\pgfqpoint{6.299148in}{3.397778in}}%
\pgfpathlineto{\pgfqpoint{6.294258in}{3.383979in}}%
\pgfpathlineto{\pgfqpoint{6.289369in}{3.369866in}}%
\pgfpathlineto{\pgfqpoint{6.284480in}{3.355529in}}%
\pgfpathlineto{\pgfqpoint{6.279591in}{3.341065in}}%
\pgfpathlineto{\pgfqpoint{6.274702in}{3.326572in}}%
\pgfpathlineto{\pgfqpoint{6.269813in}{3.312148in}}%
\pgfpathlineto{\pgfqpoint{6.264924in}{3.297889in}}%
\pgfpathlineto{\pgfqpoint{6.260035in}{3.283886in}}%
\pgfpathlineto{\pgfqpoint{6.255146in}{3.270225in}}%
\pgfpathlineto{\pgfqpoint{6.250257in}{3.256979in}}%
\pgfpathlineto{\pgfqpoint{6.245368in}{3.244212in}}%
\pgfpathlineto{\pgfqpoint{6.240479in}{3.231978in}}%
\pgfpathlineto{\pgfqpoint{6.235590in}{3.220314in}}%
\pgfpathlineto{\pgfqpoint{6.230701in}{3.209248in}}%
\pgfpathlineto{\pgfqpoint{6.225811in}{3.198793in}}%
\pgfpathlineto{\pgfqpoint{6.220922in}{3.188950in}}%
\pgfpathlineto{\pgfqpoint{6.216033in}{3.179708in}}%
\pgfpathlineto{\pgfqpoint{6.211144in}{3.171046in}}%
\pgfpathlineto{\pgfqpoint{6.206255in}{3.162934in}}%
\pgfpathlineto{\pgfqpoint{6.201366in}{3.155334in}}%
\pgfpathlineto{\pgfqpoint{6.196477in}{3.148202in}}%
\pgfpathlineto{\pgfqpoint{6.191588in}{3.141491in}}%
\pgfpathlineto{\pgfqpoint{6.186699in}{3.135149in}}%
\pgfpathlineto{\pgfqpoint{6.181810in}{3.129126in}}%
\pgfpathlineto{\pgfqpoint{6.176921in}{3.123371in}}%
\pgfpathlineto{\pgfqpoint{6.172032in}{3.117834in}}%
\pgfpathlineto{\pgfqpoint{6.167143in}{3.112471in}}%
\pgfpathlineto{\pgfqpoint{6.162254in}{3.107238in}}%
\pgfpathlineto{\pgfqpoint{6.157365in}{3.102100in}}%
\pgfpathlineto{\pgfqpoint{6.152475in}{3.097023in}}%
\pgfpathlineto{\pgfqpoint{6.147586in}{3.091983in}}%
\pgfpathlineto{\pgfqpoint{6.142697in}{3.086959in}}%
\pgfpathlineto{\pgfqpoint{6.137808in}{3.081935in}}%
\pgfpathlineto{\pgfqpoint{6.132919in}{3.076903in}}%
\pgfpathlineto{\pgfqpoint{6.128030in}{3.071859in}}%
\pgfpathlineto{\pgfqpoint{6.123141in}{3.066802in}}%
\pgfpathlineto{\pgfqpoint{6.118252in}{3.061736in}}%
\pgfpathlineto{\pgfqpoint{6.113363in}{3.056669in}}%
\pgfpathlineto{\pgfqpoint{6.108474in}{3.051611in}}%
\pgfpathlineto{\pgfqpoint{6.103585in}{3.046572in}}%
\pgfpathlineto{\pgfqpoint{6.098696in}{3.041566in}}%
\pgfpathlineto{\pgfqpoint{6.093807in}{3.036605in}}%
\pgfpathlineto{\pgfqpoint{6.088918in}{3.031702in}}%
\pgfpathlineto{\pgfqpoint{6.084029in}{3.026870in}}%
\pgfpathlineto{\pgfqpoint{6.079139in}{3.022118in}}%
\pgfpathlineto{\pgfqpoint{6.074250in}{3.017455in}}%
\pgfpathlineto{\pgfqpoint{6.069361in}{3.012891in}}%
\pgfpathlineto{\pgfqpoint{6.064472in}{3.008429in}}%
\pgfpathlineto{\pgfqpoint{6.059583in}{3.004076in}}%
\pgfpathlineto{\pgfqpoint{6.054694in}{2.999831in}}%
\pgfpathlineto{\pgfqpoint{6.049805in}{2.995697in}}%
\pgfpathlineto{\pgfqpoint{6.044916in}{2.991674in}}%
\pgfpathlineto{\pgfqpoint{6.040027in}{2.987760in}}%
\pgfpathlineto{\pgfqpoint{6.035138in}{2.983953in}}%
\pgfpathlineto{\pgfqpoint{6.030249in}{2.980253in}}%
\pgfpathlineto{\pgfqpoint{6.025360in}{2.976656in}}%
\pgfpathlineto{\pgfqpoint{6.020471in}{2.973161in}}%
\pgfpathlineto{\pgfqpoint{6.015582in}{2.969769in}}%
\pgfpathlineto{\pgfqpoint{6.010693in}{2.966478in}}%
\pgfpathlineto{\pgfqpoint{6.005803in}{2.963290in}}%
\pgfpathlineto{\pgfqpoint{6.000914in}{2.960207in}}%
\pgfpathlineto{\pgfqpoint{5.996025in}{2.957230in}}%
\pgfpathlineto{\pgfqpoint{5.991136in}{2.954364in}}%
\pgfpathlineto{\pgfqpoint{5.986247in}{2.951612in}}%
\pgfpathlineto{\pgfqpoint{5.981358in}{2.948978in}}%
\pgfpathlineto{\pgfqpoint{5.976469in}{2.946467in}}%
\pgfpathlineto{\pgfqpoint{5.971580in}{2.944081in}}%
\pgfpathlineto{\pgfqpoint{5.966691in}{2.941826in}}%
\pgfpathlineto{\pgfqpoint{5.961802in}{2.939704in}}%
\pgfpathlineto{\pgfqpoint{5.956913in}{2.937717in}}%
\pgfpathlineto{\pgfqpoint{5.952024in}{2.935866in}}%
\pgfpathlineto{\pgfqpoint{5.947135in}{2.934151in}}%
\pgfpathlineto{\pgfqpoint{5.942246in}{2.932572in}}%
\pgfpathlineto{\pgfqpoint{5.937357in}{2.931126in}}%
\pgfpathlineto{\pgfqpoint{5.932467in}{2.929811in}}%
\pgfpathlineto{\pgfqpoint{5.927578in}{2.928621in}}%
\pgfpathlineto{\pgfqpoint{5.922689in}{2.927551in}}%
\pgfpathlineto{\pgfqpoint{5.917800in}{2.926596in}}%
\pgfpathlineto{\pgfqpoint{5.912911in}{2.925749in}}%
\pgfpathlineto{\pgfqpoint{5.908022in}{2.925002in}}%
\pgfpathlineto{\pgfqpoint{5.903133in}{2.924347in}}%
\pgfpathlineto{\pgfqpoint{5.898244in}{2.923778in}}%
\pgfpathlineto{\pgfqpoint{5.893355in}{2.923286in}}%
\pgfpathlineto{\pgfqpoint{5.888466in}{2.922864in}}%
\pgfpathlineto{\pgfqpoint{5.883577in}{2.922504in}}%
\pgfpathlineto{\pgfqpoint{5.878688in}{2.922199in}}%
\pgfpathlineto{\pgfqpoint{5.873799in}{2.921942in}}%
\pgfpathlineto{\pgfqpoint{5.868910in}{2.921728in}}%
\pgfpathlineto{\pgfqpoint{5.864021in}{2.921550in}}%
\pgfpathlineto{\pgfqpoint{5.859131in}{2.921403in}}%
\pgfpathlineto{\pgfqpoint{5.854242in}{2.921283in}}%
\pgfpathlineto{\pgfqpoint{5.849353in}{2.921185in}}%
\pgfpathlineto{\pgfqpoint{5.849353in}{2.921185in}}%
\pgfpathclose%
\pgfusepath{stroke,fill}%
}%
\begin{pgfscope}%
\pgfsys@transformshift{0.000000in}{0.000000in}%
\pgfsys@useobject{currentmarker}{}%
\end{pgfscope}%
\end{pgfscope}%
\begin{pgfscope}%
\pgfpathrectangle{\pgfqpoint{5.292946in}{2.920818in}}{\pgfqpoint{2.177280in}{2.201755in}}%
\pgfusepath{clip}%
\pgfsetbuttcap%
\pgfsetroundjoin%
\definecolor{currentfill}{rgb}{0.121569,0.466667,0.705882}%
\pgfsetfillcolor{currentfill}%
\pgfsetfillopacity{0.250000}%
\pgfsetlinewidth{1.003750pt}%
\definecolor{currentstroke}{rgb}{0.121569,0.466667,0.705882}%
\pgfsetstrokecolor{currentstroke}%
\pgfsetdash{}{0pt}%
\pgfsys@defobject{currentmarker}{\pgfqpoint{5.391913in}{2.920818in}}{\pgfqpoint{5.787043in}{5.017728in}}{%
\pgfpathmoveto{\pgfqpoint{5.391913in}{2.921811in}}%
\pgfpathlineto{\pgfqpoint{5.391913in}{2.920818in}}%
\pgfpathlineto{\pgfqpoint{5.393899in}{2.920818in}}%
\pgfpathlineto{\pgfqpoint{5.395884in}{2.920818in}}%
\pgfpathlineto{\pgfqpoint{5.397870in}{2.920818in}}%
\pgfpathlineto{\pgfqpoint{5.399856in}{2.920818in}}%
\pgfpathlineto{\pgfqpoint{5.401841in}{2.920818in}}%
\pgfpathlineto{\pgfqpoint{5.403827in}{2.920818in}}%
\pgfpathlineto{\pgfqpoint{5.405812in}{2.920818in}}%
\pgfpathlineto{\pgfqpoint{5.407798in}{2.920818in}}%
\pgfpathlineto{\pgfqpoint{5.409783in}{2.920818in}}%
\pgfpathlineto{\pgfqpoint{5.411769in}{2.920818in}}%
\pgfpathlineto{\pgfqpoint{5.413755in}{2.920818in}}%
\pgfpathlineto{\pgfqpoint{5.415740in}{2.920818in}}%
\pgfpathlineto{\pgfqpoint{5.417726in}{2.920818in}}%
\pgfpathlineto{\pgfqpoint{5.419711in}{2.920818in}}%
\pgfpathlineto{\pgfqpoint{5.421697in}{2.920818in}}%
\pgfpathlineto{\pgfqpoint{5.423683in}{2.920818in}}%
\pgfpathlineto{\pgfqpoint{5.425668in}{2.920818in}}%
\pgfpathlineto{\pgfqpoint{5.427654in}{2.920818in}}%
\pgfpathlineto{\pgfqpoint{5.429639in}{2.920818in}}%
\pgfpathlineto{\pgfqpoint{5.431625in}{2.920818in}}%
\pgfpathlineto{\pgfqpoint{5.433610in}{2.920818in}}%
\pgfpathlineto{\pgfqpoint{5.435596in}{2.920818in}}%
\pgfpathlineto{\pgfqpoint{5.437582in}{2.920818in}}%
\pgfpathlineto{\pgfqpoint{5.439567in}{2.920818in}}%
\pgfpathlineto{\pgfqpoint{5.441553in}{2.920818in}}%
\pgfpathlineto{\pgfqpoint{5.443538in}{2.920818in}}%
\pgfpathlineto{\pgfqpoint{5.445524in}{2.920818in}}%
\pgfpathlineto{\pgfqpoint{5.447509in}{2.920818in}}%
\pgfpathlineto{\pgfqpoint{5.449495in}{2.920818in}}%
\pgfpathlineto{\pgfqpoint{5.451481in}{2.920818in}}%
\pgfpathlineto{\pgfqpoint{5.453466in}{2.920818in}}%
\pgfpathlineto{\pgfqpoint{5.455452in}{2.920818in}}%
\pgfpathlineto{\pgfqpoint{5.457437in}{2.920818in}}%
\pgfpathlineto{\pgfqpoint{5.459423in}{2.920818in}}%
\pgfpathlineto{\pgfqpoint{5.461408in}{2.920818in}}%
\pgfpathlineto{\pgfqpoint{5.463394in}{2.920818in}}%
\pgfpathlineto{\pgfqpoint{5.465380in}{2.920818in}}%
\pgfpathlineto{\pgfqpoint{5.467365in}{2.920818in}}%
\pgfpathlineto{\pgfqpoint{5.469351in}{2.920818in}}%
\pgfpathlineto{\pgfqpoint{5.471336in}{2.920818in}}%
\pgfpathlineto{\pgfqpoint{5.473322in}{2.920818in}}%
\pgfpathlineto{\pgfqpoint{5.475307in}{2.920818in}}%
\pgfpathlineto{\pgfqpoint{5.477293in}{2.920818in}}%
\pgfpathlineto{\pgfqpoint{5.479279in}{2.920818in}}%
\pgfpathlineto{\pgfqpoint{5.481264in}{2.920818in}}%
\pgfpathlineto{\pgfqpoint{5.483250in}{2.920818in}}%
\pgfpathlineto{\pgfqpoint{5.485235in}{2.920818in}}%
\pgfpathlineto{\pgfqpoint{5.487221in}{2.920818in}}%
\pgfpathlineto{\pgfqpoint{5.489207in}{2.920818in}}%
\pgfpathlineto{\pgfqpoint{5.491192in}{2.920818in}}%
\pgfpathlineto{\pgfqpoint{5.493178in}{2.920818in}}%
\pgfpathlineto{\pgfqpoint{5.495163in}{2.920818in}}%
\pgfpathlineto{\pgfqpoint{5.497149in}{2.920818in}}%
\pgfpathlineto{\pgfqpoint{5.499134in}{2.920818in}}%
\pgfpathlineto{\pgfqpoint{5.501120in}{2.920818in}}%
\pgfpathlineto{\pgfqpoint{5.503106in}{2.920818in}}%
\pgfpathlineto{\pgfqpoint{5.505091in}{2.920818in}}%
\pgfpathlineto{\pgfqpoint{5.507077in}{2.920818in}}%
\pgfpathlineto{\pgfqpoint{5.509062in}{2.920818in}}%
\pgfpathlineto{\pgfqpoint{5.511048in}{2.920818in}}%
\pgfpathlineto{\pgfqpoint{5.513033in}{2.920818in}}%
\pgfpathlineto{\pgfqpoint{5.515019in}{2.920818in}}%
\pgfpathlineto{\pgfqpoint{5.517005in}{2.920818in}}%
\pgfpathlineto{\pgfqpoint{5.518990in}{2.920818in}}%
\pgfpathlineto{\pgfqpoint{5.520976in}{2.920818in}}%
\pgfpathlineto{\pgfqpoint{5.522961in}{2.920818in}}%
\pgfpathlineto{\pgfqpoint{5.524947in}{2.920818in}}%
\pgfpathlineto{\pgfqpoint{5.526932in}{2.920818in}}%
\pgfpathlineto{\pgfqpoint{5.528918in}{2.920818in}}%
\pgfpathlineto{\pgfqpoint{5.530904in}{2.920818in}}%
\pgfpathlineto{\pgfqpoint{5.532889in}{2.920818in}}%
\pgfpathlineto{\pgfqpoint{5.534875in}{2.920818in}}%
\pgfpathlineto{\pgfqpoint{5.536860in}{2.920818in}}%
\pgfpathlineto{\pgfqpoint{5.538846in}{2.920818in}}%
\pgfpathlineto{\pgfqpoint{5.540831in}{2.920818in}}%
\pgfpathlineto{\pgfqpoint{5.542817in}{2.920818in}}%
\pgfpathlineto{\pgfqpoint{5.544803in}{2.920818in}}%
\pgfpathlineto{\pgfqpoint{5.546788in}{2.920818in}}%
\pgfpathlineto{\pgfqpoint{5.548774in}{2.920818in}}%
\pgfpathlineto{\pgfqpoint{5.550759in}{2.920818in}}%
\pgfpathlineto{\pgfqpoint{5.552745in}{2.920818in}}%
\pgfpathlineto{\pgfqpoint{5.554731in}{2.920818in}}%
\pgfpathlineto{\pgfqpoint{5.556716in}{2.920818in}}%
\pgfpathlineto{\pgfqpoint{5.558702in}{2.920818in}}%
\pgfpathlineto{\pgfqpoint{5.560687in}{2.920818in}}%
\pgfpathlineto{\pgfqpoint{5.562673in}{2.920818in}}%
\pgfpathlineto{\pgfqpoint{5.564658in}{2.920818in}}%
\pgfpathlineto{\pgfqpoint{5.566644in}{2.920818in}}%
\pgfpathlineto{\pgfqpoint{5.568630in}{2.920818in}}%
\pgfpathlineto{\pgfqpoint{5.570615in}{2.920818in}}%
\pgfpathlineto{\pgfqpoint{5.572601in}{2.920818in}}%
\pgfpathlineto{\pgfqpoint{5.574586in}{2.920818in}}%
\pgfpathlineto{\pgfqpoint{5.576572in}{2.920818in}}%
\pgfpathlineto{\pgfqpoint{5.578557in}{2.920818in}}%
\pgfpathlineto{\pgfqpoint{5.580543in}{2.920818in}}%
\pgfpathlineto{\pgfqpoint{5.582529in}{2.920818in}}%
\pgfpathlineto{\pgfqpoint{5.584514in}{2.920818in}}%
\pgfpathlineto{\pgfqpoint{5.586500in}{2.920818in}}%
\pgfpathlineto{\pgfqpoint{5.588485in}{2.920818in}}%
\pgfpathlineto{\pgfqpoint{5.590471in}{2.920818in}}%
\pgfpathlineto{\pgfqpoint{5.592456in}{2.920818in}}%
\pgfpathlineto{\pgfqpoint{5.594442in}{2.920818in}}%
\pgfpathlineto{\pgfqpoint{5.596428in}{2.920818in}}%
\pgfpathlineto{\pgfqpoint{5.598413in}{2.920818in}}%
\pgfpathlineto{\pgfqpoint{5.600399in}{2.920818in}}%
\pgfpathlineto{\pgfqpoint{5.602384in}{2.920818in}}%
\pgfpathlineto{\pgfqpoint{5.604370in}{2.920818in}}%
\pgfpathlineto{\pgfqpoint{5.606355in}{2.920818in}}%
\pgfpathlineto{\pgfqpoint{5.608341in}{2.920818in}}%
\pgfpathlineto{\pgfqpoint{5.610327in}{2.920818in}}%
\pgfpathlineto{\pgfqpoint{5.612312in}{2.920818in}}%
\pgfpathlineto{\pgfqpoint{5.614298in}{2.920818in}}%
\pgfpathlineto{\pgfqpoint{5.616283in}{2.920818in}}%
\pgfpathlineto{\pgfqpoint{5.618269in}{2.920818in}}%
\pgfpathlineto{\pgfqpoint{5.620254in}{2.920818in}}%
\pgfpathlineto{\pgfqpoint{5.622240in}{2.920818in}}%
\pgfpathlineto{\pgfqpoint{5.624226in}{2.920818in}}%
\pgfpathlineto{\pgfqpoint{5.626211in}{2.920818in}}%
\pgfpathlineto{\pgfqpoint{5.628197in}{2.920818in}}%
\pgfpathlineto{\pgfqpoint{5.630182in}{2.920818in}}%
\pgfpathlineto{\pgfqpoint{5.632168in}{2.920818in}}%
\pgfpathlineto{\pgfqpoint{5.634154in}{2.920818in}}%
\pgfpathlineto{\pgfqpoint{5.636139in}{2.920818in}}%
\pgfpathlineto{\pgfqpoint{5.638125in}{2.920818in}}%
\pgfpathlineto{\pgfqpoint{5.640110in}{2.920818in}}%
\pgfpathlineto{\pgfqpoint{5.642096in}{2.920818in}}%
\pgfpathlineto{\pgfqpoint{5.644081in}{2.920818in}}%
\pgfpathlineto{\pgfqpoint{5.646067in}{2.920818in}}%
\pgfpathlineto{\pgfqpoint{5.648053in}{2.920818in}}%
\pgfpathlineto{\pgfqpoint{5.650038in}{2.920818in}}%
\pgfpathlineto{\pgfqpoint{5.652024in}{2.920818in}}%
\pgfpathlineto{\pgfqpoint{5.654009in}{2.920818in}}%
\pgfpathlineto{\pgfqpoint{5.655995in}{2.920818in}}%
\pgfpathlineto{\pgfqpoint{5.657980in}{2.920818in}}%
\pgfpathlineto{\pgfqpoint{5.659966in}{2.920818in}}%
\pgfpathlineto{\pgfqpoint{5.661952in}{2.920818in}}%
\pgfpathlineto{\pgfqpoint{5.663937in}{2.920818in}}%
\pgfpathlineto{\pgfqpoint{5.665923in}{2.920818in}}%
\pgfpathlineto{\pgfqpoint{5.667908in}{2.920818in}}%
\pgfpathlineto{\pgfqpoint{5.669894in}{2.920818in}}%
\pgfpathlineto{\pgfqpoint{5.671879in}{2.920818in}}%
\pgfpathlineto{\pgfqpoint{5.673865in}{2.920818in}}%
\pgfpathlineto{\pgfqpoint{5.675851in}{2.920818in}}%
\pgfpathlineto{\pgfqpoint{5.677836in}{2.920818in}}%
\pgfpathlineto{\pgfqpoint{5.679822in}{2.920818in}}%
\pgfpathlineto{\pgfqpoint{5.681807in}{2.920818in}}%
\pgfpathlineto{\pgfqpoint{5.683793in}{2.920818in}}%
\pgfpathlineto{\pgfqpoint{5.685778in}{2.920818in}}%
\pgfpathlineto{\pgfqpoint{5.687764in}{2.920818in}}%
\pgfpathlineto{\pgfqpoint{5.689750in}{2.920818in}}%
\pgfpathlineto{\pgfqpoint{5.691735in}{2.920818in}}%
\pgfpathlineto{\pgfqpoint{5.693721in}{2.920818in}}%
\pgfpathlineto{\pgfqpoint{5.695706in}{2.920818in}}%
\pgfpathlineto{\pgfqpoint{5.697692in}{2.920818in}}%
\pgfpathlineto{\pgfqpoint{5.699678in}{2.920818in}}%
\pgfpathlineto{\pgfqpoint{5.701663in}{2.920818in}}%
\pgfpathlineto{\pgfqpoint{5.703649in}{2.920818in}}%
\pgfpathlineto{\pgfqpoint{5.705634in}{2.920818in}}%
\pgfpathlineto{\pgfqpoint{5.707620in}{2.920818in}}%
\pgfpathlineto{\pgfqpoint{5.709605in}{2.920818in}}%
\pgfpathlineto{\pgfqpoint{5.711591in}{2.920818in}}%
\pgfpathlineto{\pgfqpoint{5.713577in}{2.920818in}}%
\pgfpathlineto{\pgfqpoint{5.715562in}{2.920818in}}%
\pgfpathlineto{\pgfqpoint{5.717548in}{2.920818in}}%
\pgfpathlineto{\pgfqpoint{5.719533in}{2.920818in}}%
\pgfpathlineto{\pgfqpoint{5.721519in}{2.920818in}}%
\pgfpathlineto{\pgfqpoint{5.723504in}{2.920818in}}%
\pgfpathlineto{\pgfqpoint{5.725490in}{2.920818in}}%
\pgfpathlineto{\pgfqpoint{5.727476in}{2.920818in}}%
\pgfpathlineto{\pgfqpoint{5.729461in}{2.920818in}}%
\pgfpathlineto{\pgfqpoint{5.731447in}{2.920818in}}%
\pgfpathlineto{\pgfqpoint{5.733432in}{2.920818in}}%
\pgfpathlineto{\pgfqpoint{5.735418in}{2.920818in}}%
\pgfpathlineto{\pgfqpoint{5.737403in}{2.920818in}}%
\pgfpathlineto{\pgfqpoint{5.739389in}{2.920818in}}%
\pgfpathlineto{\pgfqpoint{5.741375in}{2.920818in}}%
\pgfpathlineto{\pgfqpoint{5.743360in}{2.920818in}}%
\pgfpathlineto{\pgfqpoint{5.745346in}{2.920818in}}%
\pgfpathlineto{\pgfqpoint{5.747331in}{2.920818in}}%
\pgfpathlineto{\pgfqpoint{5.749317in}{2.920818in}}%
\pgfpathlineto{\pgfqpoint{5.751302in}{2.920818in}}%
\pgfpathlineto{\pgfqpoint{5.753288in}{2.920818in}}%
\pgfpathlineto{\pgfqpoint{5.755274in}{2.920818in}}%
\pgfpathlineto{\pgfqpoint{5.757259in}{2.920818in}}%
\pgfpathlineto{\pgfqpoint{5.759245in}{2.920818in}}%
\pgfpathlineto{\pgfqpoint{5.761230in}{2.920818in}}%
\pgfpathlineto{\pgfqpoint{5.763216in}{2.920818in}}%
\pgfpathlineto{\pgfqpoint{5.765202in}{2.920818in}}%
\pgfpathlineto{\pgfqpoint{5.767187in}{2.920818in}}%
\pgfpathlineto{\pgfqpoint{5.769173in}{2.920818in}}%
\pgfpathlineto{\pgfqpoint{5.771158in}{2.920818in}}%
\pgfpathlineto{\pgfqpoint{5.773144in}{2.920818in}}%
\pgfpathlineto{\pgfqpoint{5.775129in}{2.920818in}}%
\pgfpathlineto{\pgfqpoint{5.777115in}{2.920818in}}%
\pgfpathlineto{\pgfqpoint{5.779101in}{2.920818in}}%
\pgfpathlineto{\pgfqpoint{5.781086in}{2.920818in}}%
\pgfpathlineto{\pgfqpoint{5.783072in}{2.920818in}}%
\pgfpathlineto{\pgfqpoint{5.785057in}{2.920818in}}%
\pgfpathlineto{\pgfqpoint{5.787043in}{2.920818in}}%
\pgfpathlineto{\pgfqpoint{5.787043in}{2.922783in}}%
\pgfpathlineto{\pgfqpoint{5.787043in}{2.922783in}}%
\pgfpathlineto{\pgfqpoint{5.785057in}{2.923360in}}%
\pgfpathlineto{\pgfqpoint{5.783072in}{2.924081in}}%
\pgfpathlineto{\pgfqpoint{5.781086in}{2.924976in}}%
\pgfpathlineto{\pgfqpoint{5.779101in}{2.926077in}}%
\pgfpathlineto{\pgfqpoint{5.777115in}{2.927418in}}%
\pgfpathlineto{\pgfqpoint{5.775129in}{2.929039in}}%
\pgfpathlineto{\pgfqpoint{5.773144in}{2.930980in}}%
\pgfpathlineto{\pgfqpoint{5.771158in}{2.933285in}}%
\pgfpathlineto{\pgfqpoint{5.769173in}{2.935998in}}%
\pgfpathlineto{\pgfqpoint{5.767187in}{2.939161in}}%
\pgfpathlineto{\pgfqpoint{5.765202in}{2.942817in}}%
\pgfpathlineto{\pgfqpoint{5.763216in}{2.947003in}}%
\pgfpathlineto{\pgfqpoint{5.761230in}{2.951750in}}%
\pgfpathlineto{\pgfqpoint{5.759245in}{2.957083in}}%
\pgfpathlineto{\pgfqpoint{5.757259in}{2.963017in}}%
\pgfpathlineto{\pgfqpoint{5.755274in}{2.969553in}}%
\pgfpathlineto{\pgfqpoint{5.753288in}{2.976679in}}%
\pgfpathlineto{\pgfqpoint{5.751302in}{2.984370in}}%
\pgfpathlineto{\pgfqpoint{5.749317in}{2.992581in}}%
\pgfpathlineto{\pgfqpoint{5.747331in}{3.001252in}}%
\pgfpathlineto{\pgfqpoint{5.745346in}{3.010305in}}%
\pgfpathlineto{\pgfqpoint{5.743360in}{3.019645in}}%
\pgfpathlineto{\pgfqpoint{5.741375in}{3.029164in}}%
\pgfpathlineto{\pgfqpoint{5.739389in}{3.038742in}}%
\pgfpathlineto{\pgfqpoint{5.737403in}{3.048248in}}%
\pgfpathlineto{\pgfqpoint{5.735418in}{3.057551in}}%
\pgfpathlineto{\pgfqpoint{5.733432in}{3.066517in}}%
\pgfpathlineto{\pgfqpoint{5.731447in}{3.075019in}}%
\pgfpathlineto{\pgfqpoint{5.729461in}{3.082944in}}%
\pgfpathlineto{\pgfqpoint{5.727476in}{3.090193in}}%
\pgfpathlineto{\pgfqpoint{5.725490in}{3.096693in}}%
\pgfpathlineto{\pgfqpoint{5.723504in}{3.102397in}}%
\pgfpathlineto{\pgfqpoint{5.721519in}{3.107293in}}%
\pgfpathlineto{\pgfqpoint{5.719533in}{3.111403in}}%
\pgfpathlineto{\pgfqpoint{5.717548in}{3.114788in}}%
\pgfpathlineto{\pgfqpoint{5.715562in}{3.117546in}}%
\pgfpathlineto{\pgfqpoint{5.713577in}{3.119813in}}%
\pgfpathlineto{\pgfqpoint{5.711591in}{3.121760in}}%
\pgfpathlineto{\pgfqpoint{5.709605in}{3.123591in}}%
\pgfpathlineto{\pgfqpoint{5.707620in}{3.125534in}}%
\pgfpathlineto{\pgfqpoint{5.705634in}{3.127836in}}%
\pgfpathlineto{\pgfqpoint{5.703649in}{3.130759in}}%
\pgfpathlineto{\pgfqpoint{5.701663in}{3.134565in}}%
\pgfpathlineto{\pgfqpoint{5.699678in}{3.139517in}}%
\pgfpathlineto{\pgfqpoint{5.697692in}{3.145860in}}%
\pgfpathlineto{\pgfqpoint{5.695706in}{3.153825in}}%
\pgfpathlineto{\pgfqpoint{5.693721in}{3.163611in}}%
\pgfpathlineto{\pgfqpoint{5.691735in}{3.175389in}}%
\pgfpathlineto{\pgfqpoint{5.689750in}{3.189290in}}%
\pgfpathlineto{\pgfqpoint{5.687764in}{3.205406in}}%
\pgfpathlineto{\pgfqpoint{5.685778in}{3.223790in}}%
\pgfpathlineto{\pgfqpoint{5.683793in}{3.244453in}}%
\pgfpathlineto{\pgfqpoint{5.681807in}{3.267368in}}%
\pgfpathlineto{\pgfqpoint{5.679822in}{3.292473in}}%
\pgfpathlineto{\pgfqpoint{5.677836in}{3.319678in}}%
\pgfpathlineto{\pgfqpoint{5.675851in}{3.348866in}}%
\pgfpathlineto{\pgfqpoint{5.673865in}{3.379905in}}%
\pgfpathlineto{\pgfqpoint{5.671879in}{3.412649in}}%
\pgfpathlineto{\pgfqpoint{5.669894in}{3.446952in}}%
\pgfpathlineto{\pgfqpoint{5.667908in}{3.482665in}}%
\pgfpathlineto{\pgfqpoint{5.665923in}{3.519649in}}%
\pgfpathlineto{\pgfqpoint{5.663937in}{3.557777in}}%
\pgfpathlineto{\pgfqpoint{5.661952in}{3.596939in}}%
\pgfpathlineto{\pgfqpoint{5.659966in}{3.637039in}}%
\pgfpathlineto{\pgfqpoint{5.657980in}{3.678005in}}%
\pgfpathlineto{\pgfqpoint{5.655995in}{3.719782in}}%
\pgfpathlineto{\pgfqpoint{5.654009in}{3.762332in}}%
\pgfpathlineto{\pgfqpoint{5.652024in}{3.805634in}}%
\pgfpathlineto{\pgfqpoint{5.650038in}{3.849679in}}%
\pgfpathlineto{\pgfqpoint{5.648053in}{3.894465in}}%
\pgfpathlineto{\pgfqpoint{5.646067in}{3.939996in}}%
\pgfpathlineto{\pgfqpoint{5.644081in}{3.986272in}}%
\pgfpathlineto{\pgfqpoint{5.642096in}{4.033287in}}%
\pgfpathlineto{\pgfqpoint{5.640110in}{4.081026in}}%
\pgfpathlineto{\pgfqpoint{5.638125in}{4.129455in}}%
\pgfpathlineto{\pgfqpoint{5.636139in}{4.178519in}}%
\pgfpathlineto{\pgfqpoint{5.634154in}{4.228141in}}%
\pgfpathlineto{\pgfqpoint{5.632168in}{4.278214in}}%
\pgfpathlineto{\pgfqpoint{5.630182in}{4.328602in}}%
\pgfpathlineto{\pgfqpoint{5.628197in}{4.379138in}}%
\pgfpathlineto{\pgfqpoint{5.626211in}{4.429624in}}%
\pgfpathlineto{\pgfqpoint{5.624226in}{4.479833in}}%
\pgfpathlineto{\pgfqpoint{5.622240in}{4.529511in}}%
\pgfpathlineto{\pgfqpoint{5.620254in}{4.578381in}}%
\pgfpathlineto{\pgfqpoint{5.618269in}{4.626147in}}%
\pgfpathlineto{\pgfqpoint{5.616283in}{4.672502in}}%
\pgfpathlineto{\pgfqpoint{5.614298in}{4.717128in}}%
\pgfpathlineto{\pgfqpoint{5.612312in}{4.759711in}}%
\pgfpathlineto{\pgfqpoint{5.610327in}{4.799940in}}%
\pgfpathlineto{\pgfqpoint{5.608341in}{4.837515in}}%
\pgfpathlineto{\pgfqpoint{5.606355in}{4.872153in}}%
\pgfpathlineto{\pgfqpoint{5.604370in}{4.903595in}}%
\pgfpathlineto{\pgfqpoint{5.602384in}{4.931602in}}%
\pgfpathlineto{\pgfqpoint{5.600399in}{4.955964in}}%
\pgfpathlineto{\pgfqpoint{5.598413in}{4.976500in}}%
\pgfpathlineto{\pgfqpoint{5.596428in}{4.993055in}}%
\pgfpathlineto{\pgfqpoint{5.594442in}{5.005505in}}%
\pgfpathlineto{\pgfqpoint{5.592456in}{5.013753in}}%
\pgfpathlineto{\pgfqpoint{5.590471in}{5.017728in}}%
\pgfpathlineto{\pgfqpoint{5.588485in}{5.017387in}}%
\pgfpathlineto{\pgfqpoint{5.586500in}{5.012713in}}%
\pgfpathlineto{\pgfqpoint{5.584514in}{5.003714in}}%
\pgfpathlineto{\pgfqpoint{5.582529in}{4.990422in}}%
\pgfpathlineto{\pgfqpoint{5.580543in}{4.972898in}}%
\pgfpathlineto{\pgfqpoint{5.578557in}{4.951227in}}%
\pgfpathlineto{\pgfqpoint{5.576572in}{4.925524in}}%
\pgfpathlineto{\pgfqpoint{5.574586in}{4.895927in}}%
\pgfpathlineto{\pgfqpoint{5.572601in}{4.862604in}}%
\pgfpathlineto{\pgfqpoint{5.570615in}{4.825749in}}%
\pgfpathlineto{\pgfqpoint{5.568630in}{4.785581in}}%
\pgfpathlineto{\pgfqpoint{5.566644in}{4.742339in}}%
\pgfpathlineto{\pgfqpoint{5.564658in}{4.696285in}}%
\pgfpathlineto{\pgfqpoint{5.562673in}{4.647696in}}%
\pgfpathlineto{\pgfqpoint{5.560687in}{4.596861in}}%
\pgfpathlineto{\pgfqpoint{5.558702in}{4.544079in}}%
\pgfpathlineto{\pgfqpoint{5.556716in}{4.489651in}}%
\pgfpathlineto{\pgfqpoint{5.554731in}{4.433882in}}%
\pgfpathlineto{\pgfqpoint{5.552745in}{4.377071in}}%
\pgfpathlineto{\pgfqpoint{5.550759in}{4.319514in}}%
\pgfpathlineto{\pgfqpoint{5.548774in}{4.261496in}}%
\pgfpathlineto{\pgfqpoint{5.546788in}{4.203296in}}%
\pgfpathlineto{\pgfqpoint{5.544803in}{4.145180in}}%
\pgfpathlineto{\pgfqpoint{5.542817in}{4.087402in}}%
\pgfpathlineto{\pgfqpoint{5.540831in}{4.030205in}}%
\pgfpathlineto{\pgfqpoint{5.538846in}{3.973818in}}%
\pgfpathlineto{\pgfqpoint{5.536860in}{3.918458in}}%
\pgfpathlineto{\pgfqpoint{5.534875in}{3.864326in}}%
\pgfpathlineto{\pgfqpoint{5.532889in}{3.811612in}}%
\pgfpathlineto{\pgfqpoint{5.530904in}{3.760487in}}%
\pgfpathlineto{\pgfqpoint{5.528918in}{3.711106in}}%
\pgfpathlineto{\pgfqpoint{5.526932in}{3.663608in}}%
\pgfpathlineto{\pgfqpoint{5.524947in}{3.618110in}}%
\pgfpathlineto{\pgfqpoint{5.522961in}{3.574712in}}%
\pgfpathlineto{\pgfqpoint{5.520976in}{3.533490in}}%
\pgfpathlineto{\pgfqpoint{5.518990in}{3.494500in}}%
\pgfpathlineto{\pgfqpoint{5.517005in}{3.457777in}}%
\pgfpathlineto{\pgfqpoint{5.515019in}{3.423331in}}%
\pgfpathlineto{\pgfqpoint{5.513033in}{3.391155in}}%
\pgfpathlineto{\pgfqpoint{5.511048in}{3.361220in}}%
\pgfpathlineto{\pgfqpoint{5.509062in}{3.333479in}}%
\pgfpathlineto{\pgfqpoint{5.507077in}{3.307868in}}%
\pgfpathlineto{\pgfqpoint{5.505091in}{3.284311in}}%
\pgfpathlineto{\pgfqpoint{5.503106in}{3.262716in}}%
\pgfpathlineto{\pgfqpoint{5.501120in}{3.242982in}}%
\pgfpathlineto{\pgfqpoint{5.499134in}{3.225001in}}%
\pgfpathlineto{\pgfqpoint{5.497149in}{3.208657in}}%
\pgfpathlineto{\pgfqpoint{5.495163in}{3.193830in}}%
\pgfpathlineto{\pgfqpoint{5.493178in}{3.180397in}}%
\pgfpathlineto{\pgfqpoint{5.491192in}{3.168233in}}%
\pgfpathlineto{\pgfqpoint{5.489207in}{3.157214in}}%
\pgfpathlineto{\pgfqpoint{5.487221in}{3.147215in}}%
\pgfpathlineto{\pgfqpoint{5.485235in}{3.138113in}}%
\pgfpathlineto{\pgfqpoint{5.483250in}{3.129790in}}%
\pgfpathlineto{\pgfqpoint{5.481264in}{3.122128in}}%
\pgfpathlineto{\pgfqpoint{5.479279in}{3.115016in}}%
\pgfpathlineto{\pgfqpoint{5.477293in}{3.108347in}}%
\pgfpathlineto{\pgfqpoint{5.475307in}{3.102020in}}%
\pgfpathlineto{\pgfqpoint{5.473322in}{3.095942in}}%
\pgfpathlineto{\pgfqpoint{5.471336in}{3.090027in}}%
\pgfpathlineto{\pgfqpoint{5.469351in}{3.084199in}}%
\pgfpathlineto{\pgfqpoint{5.467365in}{3.078392in}}%
\pgfpathlineto{\pgfqpoint{5.465380in}{3.072548in}}%
\pgfpathlineto{\pgfqpoint{5.463394in}{3.066623in}}%
\pgfpathlineto{\pgfqpoint{5.461408in}{3.060585in}}%
\pgfpathlineto{\pgfqpoint{5.459423in}{3.054413in}}%
\pgfpathlineto{\pgfqpoint{5.457437in}{3.048097in}}%
\pgfpathlineto{\pgfqpoint{5.455452in}{3.041639in}}%
\pgfpathlineto{\pgfqpoint{5.453466in}{3.035055in}}%
\pgfpathlineto{\pgfqpoint{5.451481in}{3.028365in}}%
\pgfpathlineto{\pgfqpoint{5.449495in}{3.021603in}}%
\pgfpathlineto{\pgfqpoint{5.447509in}{3.014807in}}%
\pgfpathlineto{\pgfqpoint{5.445524in}{3.008021in}}%
\pgfpathlineto{\pgfqpoint{5.443538in}{3.001292in}}%
\pgfpathlineto{\pgfqpoint{5.441553in}{2.994669in}}%
\pgfpathlineto{\pgfqpoint{5.439567in}{2.988202in}}%
\pgfpathlineto{\pgfqpoint{5.437582in}{2.981936in}}%
\pgfpathlineto{\pgfqpoint{5.435596in}{2.975916in}}%
\pgfpathlineto{\pgfqpoint{5.433610in}{2.970178in}}%
\pgfpathlineto{\pgfqpoint{5.431625in}{2.964757in}}%
\pgfpathlineto{\pgfqpoint{5.429639in}{2.959677in}}%
\pgfpathlineto{\pgfqpoint{5.427654in}{2.954958in}}%
\pgfpathlineto{\pgfqpoint{5.425668in}{2.950612in}}%
\pgfpathlineto{\pgfqpoint{5.423683in}{2.946642in}}%
\pgfpathlineto{\pgfqpoint{5.421697in}{2.943048in}}%
\pgfpathlineto{\pgfqpoint{5.419711in}{2.939822in}}%
\pgfpathlineto{\pgfqpoint{5.417726in}{2.936950in}}%
\pgfpathlineto{\pgfqpoint{5.415740in}{2.934416in}}%
\pgfpathlineto{\pgfqpoint{5.413755in}{2.932198in}}%
\pgfpathlineto{\pgfqpoint{5.411769in}{2.930274in}}%
\pgfpathlineto{\pgfqpoint{5.409783in}{2.928619in}}%
\pgfpathlineto{\pgfqpoint{5.407798in}{2.927207in}}%
\pgfpathlineto{\pgfqpoint{5.405812in}{2.926012in}}%
\pgfpathlineto{\pgfqpoint{5.403827in}{2.925011in}}%
\pgfpathlineto{\pgfqpoint{5.401841in}{2.924177in}}%
\pgfpathlineto{\pgfqpoint{5.399856in}{2.923490in}}%
\pgfpathlineto{\pgfqpoint{5.397870in}{2.922927in}}%
\pgfpathlineto{\pgfqpoint{5.395884in}{2.922471in}}%
\pgfpathlineto{\pgfqpoint{5.393899in}{2.922104in}}%
\pgfpathlineto{\pgfqpoint{5.391913in}{2.921811in}}%
\pgfpathlineto{\pgfqpoint{5.391913in}{2.921811in}}%
\pgfpathclose%
\pgfusepath{stroke,fill}%
}%
\begin{pgfscope}%
\pgfsys@transformshift{0.000000in}{0.000000in}%
\pgfsys@useobject{currentmarker}{}%
\end{pgfscope}%
\end{pgfscope}%
\begin{pgfscope}%
\pgfpathrectangle{\pgfqpoint{7.622482in}{0.569136in}}{\pgfqpoint{2.177280in}{2.201755in}}%
\pgfusepath{clip}%
\pgfsetbuttcap%
\pgfsetroundjoin%
\definecolor{currentfill}{rgb}{0.172549,0.627451,0.172549}%
\pgfsetfillcolor{currentfill}%
\pgfsetfillopacity{0.250000}%
\pgfsetlinewidth{1.003750pt}%
\definecolor{currentstroke}{rgb}{0.172549,0.627451,0.172549}%
\pgfsetstrokecolor{currentstroke}%
\pgfsetdash{}{0pt}%
\pgfsys@defobject{currentmarker}{\pgfqpoint{8.444914in}{0.569136in}}{\pgfqpoint{9.700795in}{1.089260in}}{%
\pgfpathmoveto{\pgfqpoint{8.444914in}{0.569470in}}%
\pgfpathlineto{\pgfqpoint{8.444914in}{0.569136in}}%
\pgfpathlineto{\pgfqpoint{8.451225in}{0.569136in}}%
\pgfpathlineto{\pgfqpoint{8.457536in}{0.569136in}}%
\pgfpathlineto{\pgfqpoint{8.463847in}{0.569136in}}%
\pgfpathlineto{\pgfqpoint{8.470158in}{0.569136in}}%
\pgfpathlineto{\pgfqpoint{8.476469in}{0.569136in}}%
\pgfpathlineto{\pgfqpoint{8.482780in}{0.569136in}}%
\pgfpathlineto{\pgfqpoint{8.489091in}{0.569136in}}%
\pgfpathlineto{\pgfqpoint{8.495402in}{0.569136in}}%
\pgfpathlineto{\pgfqpoint{8.501713in}{0.569136in}}%
\pgfpathlineto{\pgfqpoint{8.508024in}{0.569136in}}%
\pgfpathlineto{\pgfqpoint{8.514335in}{0.569136in}}%
\pgfpathlineto{\pgfqpoint{8.520645in}{0.569136in}}%
\pgfpathlineto{\pgfqpoint{8.526956in}{0.569136in}}%
\pgfpathlineto{\pgfqpoint{8.533267in}{0.569136in}}%
\pgfpathlineto{\pgfqpoint{8.539578in}{0.569136in}}%
\pgfpathlineto{\pgfqpoint{8.545889in}{0.569136in}}%
\pgfpathlineto{\pgfqpoint{8.552200in}{0.569136in}}%
\pgfpathlineto{\pgfqpoint{8.558511in}{0.569136in}}%
\pgfpathlineto{\pgfqpoint{8.564822in}{0.569136in}}%
\pgfpathlineto{\pgfqpoint{8.571133in}{0.569136in}}%
\pgfpathlineto{\pgfqpoint{8.577444in}{0.569136in}}%
\pgfpathlineto{\pgfqpoint{8.583755in}{0.569136in}}%
\pgfpathlineto{\pgfqpoint{8.590066in}{0.569136in}}%
\pgfpathlineto{\pgfqpoint{8.596377in}{0.569136in}}%
\pgfpathlineto{\pgfqpoint{8.602688in}{0.569136in}}%
\pgfpathlineto{\pgfqpoint{8.608999in}{0.569136in}}%
\pgfpathlineto{\pgfqpoint{8.615310in}{0.569136in}}%
\pgfpathlineto{\pgfqpoint{8.621621in}{0.569136in}}%
\pgfpathlineto{\pgfqpoint{8.627932in}{0.569136in}}%
\pgfpathlineto{\pgfqpoint{8.634243in}{0.569136in}}%
\pgfpathlineto{\pgfqpoint{8.640554in}{0.569136in}}%
\pgfpathlineto{\pgfqpoint{8.646865in}{0.569136in}}%
\pgfpathlineto{\pgfqpoint{8.653176in}{0.569136in}}%
\pgfpathlineto{\pgfqpoint{8.659487in}{0.569136in}}%
\pgfpathlineto{\pgfqpoint{8.665798in}{0.569136in}}%
\pgfpathlineto{\pgfqpoint{8.672109in}{0.569136in}}%
\pgfpathlineto{\pgfqpoint{8.678419in}{0.569136in}}%
\pgfpathlineto{\pgfqpoint{8.684730in}{0.569136in}}%
\pgfpathlineto{\pgfqpoint{8.691041in}{0.569136in}}%
\pgfpathlineto{\pgfqpoint{8.697352in}{0.569136in}}%
\pgfpathlineto{\pgfqpoint{8.703663in}{0.569136in}}%
\pgfpathlineto{\pgfqpoint{8.709974in}{0.569136in}}%
\pgfpathlineto{\pgfqpoint{8.716285in}{0.569136in}}%
\pgfpathlineto{\pgfqpoint{8.722596in}{0.569136in}}%
\pgfpathlineto{\pgfqpoint{8.728907in}{0.569136in}}%
\pgfpathlineto{\pgfqpoint{8.735218in}{0.569136in}}%
\pgfpathlineto{\pgfqpoint{8.741529in}{0.569136in}}%
\pgfpathlineto{\pgfqpoint{8.747840in}{0.569136in}}%
\pgfpathlineto{\pgfqpoint{8.754151in}{0.569136in}}%
\pgfpathlineto{\pgfqpoint{8.760462in}{0.569136in}}%
\pgfpathlineto{\pgfqpoint{8.766773in}{0.569136in}}%
\pgfpathlineto{\pgfqpoint{8.773084in}{0.569136in}}%
\pgfpathlineto{\pgfqpoint{8.779395in}{0.569136in}}%
\pgfpathlineto{\pgfqpoint{8.785706in}{0.569136in}}%
\pgfpathlineto{\pgfqpoint{8.792017in}{0.569136in}}%
\pgfpathlineto{\pgfqpoint{8.798328in}{0.569136in}}%
\pgfpathlineto{\pgfqpoint{8.804639in}{0.569136in}}%
\pgfpathlineto{\pgfqpoint{8.810950in}{0.569136in}}%
\pgfpathlineto{\pgfqpoint{8.817261in}{0.569136in}}%
\pgfpathlineto{\pgfqpoint{8.823572in}{0.569136in}}%
\pgfpathlineto{\pgfqpoint{8.829883in}{0.569136in}}%
\pgfpathlineto{\pgfqpoint{8.836193in}{0.569136in}}%
\pgfpathlineto{\pgfqpoint{8.842504in}{0.569136in}}%
\pgfpathlineto{\pgfqpoint{8.848815in}{0.569136in}}%
\pgfpathlineto{\pgfqpoint{8.855126in}{0.569136in}}%
\pgfpathlineto{\pgfqpoint{8.861437in}{0.569136in}}%
\pgfpathlineto{\pgfqpoint{8.867748in}{0.569136in}}%
\pgfpathlineto{\pgfqpoint{8.874059in}{0.569136in}}%
\pgfpathlineto{\pgfqpoint{8.880370in}{0.569136in}}%
\pgfpathlineto{\pgfqpoint{8.886681in}{0.569136in}}%
\pgfpathlineto{\pgfqpoint{8.892992in}{0.569136in}}%
\pgfpathlineto{\pgfqpoint{8.899303in}{0.569136in}}%
\pgfpathlineto{\pgfqpoint{8.905614in}{0.569136in}}%
\pgfpathlineto{\pgfqpoint{8.911925in}{0.569136in}}%
\pgfpathlineto{\pgfqpoint{8.918236in}{0.569136in}}%
\pgfpathlineto{\pgfqpoint{8.924547in}{0.569136in}}%
\pgfpathlineto{\pgfqpoint{8.930858in}{0.569136in}}%
\pgfpathlineto{\pgfqpoint{8.937169in}{0.569136in}}%
\pgfpathlineto{\pgfqpoint{8.943480in}{0.569136in}}%
\pgfpathlineto{\pgfqpoint{8.949791in}{0.569136in}}%
\pgfpathlineto{\pgfqpoint{8.956102in}{0.569136in}}%
\pgfpathlineto{\pgfqpoint{8.962413in}{0.569136in}}%
\pgfpathlineto{\pgfqpoint{8.968724in}{0.569136in}}%
\pgfpathlineto{\pgfqpoint{8.975035in}{0.569136in}}%
\pgfpathlineto{\pgfqpoint{8.981346in}{0.569136in}}%
\pgfpathlineto{\pgfqpoint{8.987657in}{0.569136in}}%
\pgfpathlineto{\pgfqpoint{8.993967in}{0.569136in}}%
\pgfpathlineto{\pgfqpoint{9.000278in}{0.569136in}}%
\pgfpathlineto{\pgfqpoint{9.006589in}{0.569136in}}%
\pgfpathlineto{\pgfqpoint{9.012900in}{0.569136in}}%
\pgfpathlineto{\pgfqpoint{9.019211in}{0.569136in}}%
\pgfpathlineto{\pgfqpoint{9.025522in}{0.569136in}}%
\pgfpathlineto{\pgfqpoint{9.031833in}{0.569136in}}%
\pgfpathlineto{\pgfqpoint{9.038144in}{0.569136in}}%
\pgfpathlineto{\pgfqpoint{9.044455in}{0.569136in}}%
\pgfpathlineto{\pgfqpoint{9.050766in}{0.569136in}}%
\pgfpathlineto{\pgfqpoint{9.057077in}{0.569136in}}%
\pgfpathlineto{\pgfqpoint{9.063388in}{0.569136in}}%
\pgfpathlineto{\pgfqpoint{9.069699in}{0.569136in}}%
\pgfpathlineto{\pgfqpoint{9.076010in}{0.569136in}}%
\pgfpathlineto{\pgfqpoint{9.082321in}{0.569136in}}%
\pgfpathlineto{\pgfqpoint{9.088632in}{0.569136in}}%
\pgfpathlineto{\pgfqpoint{9.094943in}{0.569136in}}%
\pgfpathlineto{\pgfqpoint{9.101254in}{0.569136in}}%
\pgfpathlineto{\pgfqpoint{9.107565in}{0.569136in}}%
\pgfpathlineto{\pgfqpoint{9.113876in}{0.569136in}}%
\pgfpathlineto{\pgfqpoint{9.120187in}{0.569136in}}%
\pgfpathlineto{\pgfqpoint{9.126498in}{0.569136in}}%
\pgfpathlineto{\pgfqpoint{9.132809in}{0.569136in}}%
\pgfpathlineto{\pgfqpoint{9.139120in}{0.569136in}}%
\pgfpathlineto{\pgfqpoint{9.145431in}{0.569136in}}%
\pgfpathlineto{\pgfqpoint{9.151742in}{0.569136in}}%
\pgfpathlineto{\pgfqpoint{9.158052in}{0.569136in}}%
\pgfpathlineto{\pgfqpoint{9.164363in}{0.569136in}}%
\pgfpathlineto{\pgfqpoint{9.170674in}{0.569136in}}%
\pgfpathlineto{\pgfqpoint{9.176985in}{0.569136in}}%
\pgfpathlineto{\pgfqpoint{9.183296in}{0.569136in}}%
\pgfpathlineto{\pgfqpoint{9.189607in}{0.569136in}}%
\pgfpathlineto{\pgfqpoint{9.195918in}{0.569136in}}%
\pgfpathlineto{\pgfqpoint{9.202229in}{0.569136in}}%
\pgfpathlineto{\pgfqpoint{9.208540in}{0.569136in}}%
\pgfpathlineto{\pgfqpoint{9.214851in}{0.569136in}}%
\pgfpathlineto{\pgfqpoint{9.221162in}{0.569136in}}%
\pgfpathlineto{\pgfqpoint{9.227473in}{0.569136in}}%
\pgfpathlineto{\pgfqpoint{9.233784in}{0.569136in}}%
\pgfpathlineto{\pgfqpoint{9.240095in}{0.569136in}}%
\pgfpathlineto{\pgfqpoint{9.246406in}{0.569136in}}%
\pgfpathlineto{\pgfqpoint{9.252717in}{0.569136in}}%
\pgfpathlineto{\pgfqpoint{9.259028in}{0.569136in}}%
\pgfpathlineto{\pgfqpoint{9.265339in}{0.569136in}}%
\pgfpathlineto{\pgfqpoint{9.271650in}{0.569136in}}%
\pgfpathlineto{\pgfqpoint{9.277961in}{0.569136in}}%
\pgfpathlineto{\pgfqpoint{9.284272in}{0.569136in}}%
\pgfpathlineto{\pgfqpoint{9.290583in}{0.569136in}}%
\pgfpathlineto{\pgfqpoint{9.296894in}{0.569136in}}%
\pgfpathlineto{\pgfqpoint{9.303205in}{0.569136in}}%
\pgfpathlineto{\pgfqpoint{9.309516in}{0.569136in}}%
\pgfpathlineto{\pgfqpoint{9.315826in}{0.569136in}}%
\pgfpathlineto{\pgfqpoint{9.322137in}{0.569136in}}%
\pgfpathlineto{\pgfqpoint{9.328448in}{0.569136in}}%
\pgfpathlineto{\pgfqpoint{9.334759in}{0.569136in}}%
\pgfpathlineto{\pgfqpoint{9.341070in}{0.569136in}}%
\pgfpathlineto{\pgfqpoint{9.347381in}{0.569136in}}%
\pgfpathlineto{\pgfqpoint{9.353692in}{0.569136in}}%
\pgfpathlineto{\pgfqpoint{9.360003in}{0.569136in}}%
\pgfpathlineto{\pgfqpoint{9.366314in}{0.569136in}}%
\pgfpathlineto{\pgfqpoint{9.372625in}{0.569136in}}%
\pgfpathlineto{\pgfqpoint{9.378936in}{0.569136in}}%
\pgfpathlineto{\pgfqpoint{9.385247in}{0.569136in}}%
\pgfpathlineto{\pgfqpoint{9.391558in}{0.569136in}}%
\pgfpathlineto{\pgfqpoint{9.397869in}{0.569136in}}%
\pgfpathlineto{\pgfqpoint{9.404180in}{0.569136in}}%
\pgfpathlineto{\pgfqpoint{9.410491in}{0.569136in}}%
\pgfpathlineto{\pgfqpoint{9.416802in}{0.569136in}}%
\pgfpathlineto{\pgfqpoint{9.423113in}{0.569136in}}%
\pgfpathlineto{\pgfqpoint{9.429424in}{0.569136in}}%
\pgfpathlineto{\pgfqpoint{9.435735in}{0.569136in}}%
\pgfpathlineto{\pgfqpoint{9.442046in}{0.569136in}}%
\pgfpathlineto{\pgfqpoint{9.448357in}{0.569136in}}%
\pgfpathlineto{\pgfqpoint{9.454668in}{0.569136in}}%
\pgfpathlineto{\pgfqpoint{9.460979in}{0.569136in}}%
\pgfpathlineto{\pgfqpoint{9.467290in}{0.569136in}}%
\pgfpathlineto{\pgfqpoint{9.473600in}{0.569136in}}%
\pgfpathlineto{\pgfqpoint{9.479911in}{0.569136in}}%
\pgfpathlineto{\pgfqpoint{9.486222in}{0.569136in}}%
\pgfpathlineto{\pgfqpoint{9.492533in}{0.569136in}}%
\pgfpathlineto{\pgfqpoint{9.498844in}{0.569136in}}%
\pgfpathlineto{\pgfqpoint{9.505155in}{0.569136in}}%
\pgfpathlineto{\pgfqpoint{9.511466in}{0.569136in}}%
\pgfpathlineto{\pgfqpoint{9.517777in}{0.569136in}}%
\pgfpathlineto{\pgfqpoint{9.524088in}{0.569136in}}%
\pgfpathlineto{\pgfqpoint{9.530399in}{0.569136in}}%
\pgfpathlineto{\pgfqpoint{9.536710in}{0.569136in}}%
\pgfpathlineto{\pgfqpoint{9.543021in}{0.569136in}}%
\pgfpathlineto{\pgfqpoint{9.549332in}{0.569136in}}%
\pgfpathlineto{\pgfqpoint{9.555643in}{0.569136in}}%
\pgfpathlineto{\pgfqpoint{9.561954in}{0.569136in}}%
\pgfpathlineto{\pgfqpoint{9.568265in}{0.569136in}}%
\pgfpathlineto{\pgfqpoint{9.574576in}{0.569136in}}%
\pgfpathlineto{\pgfqpoint{9.580887in}{0.569136in}}%
\pgfpathlineto{\pgfqpoint{9.587198in}{0.569136in}}%
\pgfpathlineto{\pgfqpoint{9.593509in}{0.569136in}}%
\pgfpathlineto{\pgfqpoint{9.599820in}{0.569136in}}%
\pgfpathlineto{\pgfqpoint{9.606131in}{0.569136in}}%
\pgfpathlineto{\pgfqpoint{9.612442in}{0.569136in}}%
\pgfpathlineto{\pgfqpoint{9.618753in}{0.569136in}}%
\pgfpathlineto{\pgfqpoint{9.625064in}{0.569136in}}%
\pgfpathlineto{\pgfqpoint{9.631374in}{0.569136in}}%
\pgfpathlineto{\pgfqpoint{9.637685in}{0.569136in}}%
\pgfpathlineto{\pgfqpoint{9.643996in}{0.569136in}}%
\pgfpathlineto{\pgfqpoint{9.650307in}{0.569136in}}%
\pgfpathlineto{\pgfqpoint{9.656618in}{0.569136in}}%
\pgfpathlineto{\pgfqpoint{9.662929in}{0.569136in}}%
\pgfpathlineto{\pgfqpoint{9.669240in}{0.569136in}}%
\pgfpathlineto{\pgfqpoint{9.675551in}{0.569136in}}%
\pgfpathlineto{\pgfqpoint{9.681862in}{0.569136in}}%
\pgfpathlineto{\pgfqpoint{9.688173in}{0.569136in}}%
\pgfpathlineto{\pgfqpoint{9.694484in}{0.569136in}}%
\pgfpathlineto{\pgfqpoint{9.700795in}{0.569136in}}%
\pgfpathlineto{\pgfqpoint{9.700795in}{0.570082in}}%
\pgfpathlineto{\pgfqpoint{9.700795in}{0.570082in}}%
\pgfpathlineto{\pgfqpoint{9.694484in}{0.570320in}}%
\pgfpathlineto{\pgfqpoint{9.688173in}{0.570610in}}%
\pgfpathlineto{\pgfqpoint{9.681862in}{0.570962in}}%
\pgfpathlineto{\pgfqpoint{9.675551in}{0.571386in}}%
\pgfpathlineto{\pgfqpoint{9.669240in}{0.571894in}}%
\pgfpathlineto{\pgfqpoint{9.662929in}{0.572500in}}%
\pgfpathlineto{\pgfqpoint{9.656618in}{0.573218in}}%
\pgfpathlineto{\pgfqpoint{9.650307in}{0.574064in}}%
\pgfpathlineto{\pgfqpoint{9.643996in}{0.575055in}}%
\pgfpathlineto{\pgfqpoint{9.637685in}{0.576211in}}%
\pgfpathlineto{\pgfqpoint{9.631374in}{0.577552in}}%
\pgfpathlineto{\pgfqpoint{9.625064in}{0.579096in}}%
\pgfpathlineto{\pgfqpoint{9.618753in}{0.580868in}}%
\pgfpathlineto{\pgfqpoint{9.612442in}{0.582887in}}%
\pgfpathlineto{\pgfqpoint{9.606131in}{0.585178in}}%
\pgfpathlineto{\pgfqpoint{9.599820in}{0.587763in}}%
\pgfpathlineto{\pgfqpoint{9.593509in}{0.590663in}}%
\pgfpathlineto{\pgfqpoint{9.587198in}{0.593901in}}%
\pgfpathlineto{\pgfqpoint{9.580887in}{0.597496in}}%
\pgfpathlineto{\pgfqpoint{9.574576in}{0.601468in}}%
\pgfpathlineto{\pgfqpoint{9.568265in}{0.605834in}}%
\pgfpathlineto{\pgfqpoint{9.561954in}{0.610610in}}%
\pgfpathlineto{\pgfqpoint{9.555643in}{0.615808in}}%
\pgfpathlineto{\pgfqpoint{9.549332in}{0.621438in}}%
\pgfpathlineto{\pgfqpoint{9.543021in}{0.627508in}}%
\pgfpathlineto{\pgfqpoint{9.536710in}{0.634021in}}%
\pgfpathlineto{\pgfqpoint{9.530399in}{0.640979in}}%
\pgfpathlineto{\pgfqpoint{9.524088in}{0.648378in}}%
\pgfpathlineto{\pgfqpoint{9.517777in}{0.656213in}}%
\pgfpathlineto{\pgfqpoint{9.511466in}{0.664474in}}%
\pgfpathlineto{\pgfqpoint{9.505155in}{0.673150in}}%
\pgfpathlineto{\pgfqpoint{9.498844in}{0.682225in}}%
\pgfpathlineto{\pgfqpoint{9.492533in}{0.691680in}}%
\pgfpathlineto{\pgfqpoint{9.486222in}{0.701494in}}%
\pgfpathlineto{\pgfqpoint{9.479911in}{0.711643in}}%
\pgfpathlineto{\pgfqpoint{9.473600in}{0.722101in}}%
\pgfpathlineto{\pgfqpoint{9.467290in}{0.732838in}}%
\pgfpathlineto{\pgfqpoint{9.460979in}{0.743824in}}%
\pgfpathlineto{\pgfqpoint{9.454668in}{0.755024in}}%
\pgfpathlineto{\pgfqpoint{9.448357in}{0.766403in}}%
\pgfpathlineto{\pgfqpoint{9.442046in}{0.777924in}}%
\pgfpathlineto{\pgfqpoint{9.435735in}{0.789546in}}%
\pgfpathlineto{\pgfqpoint{9.429424in}{0.801228in}}%
\pgfpathlineto{\pgfqpoint{9.423113in}{0.812925in}}%
\pgfpathlineto{\pgfqpoint{9.416802in}{0.824592in}}%
\pgfpathlineto{\pgfqpoint{9.410491in}{0.836181in}}%
\pgfpathlineto{\pgfqpoint{9.404180in}{0.847643in}}%
\pgfpathlineto{\pgfqpoint{9.397869in}{0.858929in}}%
\pgfpathlineto{\pgfqpoint{9.391558in}{0.869987in}}%
\pgfpathlineto{\pgfqpoint{9.385247in}{0.880769in}}%
\pgfpathlineto{\pgfqpoint{9.378936in}{0.891224in}}%
\pgfpathlineto{\pgfqpoint{9.372625in}{0.901304in}}%
\pgfpathlineto{\pgfqpoint{9.366314in}{0.910965in}}%
\pgfpathlineto{\pgfqpoint{9.360003in}{0.920164in}}%
\pgfpathlineto{\pgfqpoint{9.353692in}{0.928865in}}%
\pgfpathlineto{\pgfqpoint{9.347381in}{0.937035in}}%
\pgfpathlineto{\pgfqpoint{9.341070in}{0.944648in}}%
\pgfpathlineto{\pgfqpoint{9.334759in}{0.951688in}}%
\pgfpathlineto{\pgfqpoint{9.328448in}{0.958143in}}%
\pgfpathlineto{\pgfqpoint{9.322137in}{0.964011in}}%
\pgfpathlineto{\pgfqpoint{9.315826in}{0.969299in}}%
\pgfpathlineto{\pgfqpoint{9.309516in}{0.974023in}}%
\pgfpathlineto{\pgfqpoint{9.303205in}{0.978205in}}%
\pgfpathlineto{\pgfqpoint{9.296894in}{0.981878in}}%
\pgfpathlineto{\pgfqpoint{9.290583in}{0.985081in}}%
\pgfpathlineto{\pgfqpoint{9.284272in}{0.987860in}}%
\pgfpathlineto{\pgfqpoint{9.277961in}{0.990264in}}%
\pgfpathlineto{\pgfqpoint{9.271650in}{0.992349in}}%
\pgfpathlineto{\pgfqpoint{9.265339in}{0.994171in}}%
\pgfpathlineto{\pgfqpoint{9.259028in}{0.995788in}}%
\pgfpathlineto{\pgfqpoint{9.252717in}{0.997259in}}%
\pgfpathlineto{\pgfqpoint{9.246406in}{0.998638in}}%
\pgfpathlineto{\pgfqpoint{9.240095in}{0.999980in}}%
\pgfpathlineto{\pgfqpoint{9.233784in}{1.001332in}}%
\pgfpathlineto{\pgfqpoint{9.227473in}{1.002738in}}%
\pgfpathlineto{\pgfqpoint{9.221162in}{1.004236in}}%
\pgfpathlineto{\pgfqpoint{9.214851in}{1.005859in}}%
\pgfpathlineto{\pgfqpoint{9.208540in}{1.007631in}}%
\pgfpathlineto{\pgfqpoint{9.202229in}{1.009572in}}%
\pgfpathlineto{\pgfqpoint{9.195918in}{1.011696in}}%
\pgfpathlineto{\pgfqpoint{9.189607in}{1.014010in}}%
\pgfpathlineto{\pgfqpoint{9.183296in}{1.016517in}}%
\pgfpathlineto{\pgfqpoint{9.176985in}{1.019214in}}%
\pgfpathlineto{\pgfqpoint{9.170674in}{1.022097in}}%
\pgfpathlineto{\pgfqpoint{9.164363in}{1.025157in}}%
\pgfpathlineto{\pgfqpoint{9.158052in}{1.028382in}}%
\pgfpathlineto{\pgfqpoint{9.151742in}{1.031761in}}%
\pgfpathlineto{\pgfqpoint{9.145431in}{1.035277in}}%
\pgfpathlineto{\pgfqpoint{9.139120in}{1.038914in}}%
\pgfpathlineto{\pgfqpoint{9.132809in}{1.042655in}}%
\pgfpathlineto{\pgfqpoint{9.126498in}{1.046480in}}%
\pgfpathlineto{\pgfqpoint{9.120187in}{1.050367in}}%
\pgfpathlineto{\pgfqpoint{9.113876in}{1.054291in}}%
\pgfpathlineto{\pgfqpoint{9.107565in}{1.058224in}}%
\pgfpathlineto{\pgfqpoint{9.101254in}{1.062135in}}%
\pgfpathlineto{\pgfqpoint{9.094943in}{1.065987in}}%
\pgfpathlineto{\pgfqpoint{9.088632in}{1.069738in}}%
\pgfpathlineto{\pgfqpoint{9.082321in}{1.073341in}}%
\pgfpathlineto{\pgfqpoint{9.076010in}{1.076743in}}%
\pgfpathlineto{\pgfqpoint{9.069699in}{1.079884in}}%
\pgfpathlineto{\pgfqpoint{9.063388in}{1.082698in}}%
\pgfpathlineto{\pgfqpoint{9.057077in}{1.085115in}}%
\pgfpathlineto{\pgfqpoint{9.050766in}{1.087059in}}%
\pgfpathlineto{\pgfqpoint{9.044455in}{1.088452in}}%
\pgfpathlineto{\pgfqpoint{9.038144in}{1.089212in}}%
\pgfpathlineto{\pgfqpoint{9.031833in}{1.089260in}}%
\pgfpathlineto{\pgfqpoint{9.025522in}{1.088516in}}%
\pgfpathlineto{\pgfqpoint{9.019211in}{1.086905in}}%
\pgfpathlineto{\pgfqpoint{9.012900in}{1.084359in}}%
\pgfpathlineto{\pgfqpoint{9.006589in}{1.080818in}}%
\pgfpathlineto{\pgfqpoint{9.000278in}{1.076235in}}%
\pgfpathlineto{\pgfqpoint{8.993967in}{1.070572in}}%
\pgfpathlineto{\pgfqpoint{8.987657in}{1.063808in}}%
\pgfpathlineto{\pgfqpoint{8.981346in}{1.055940in}}%
\pgfpathlineto{\pgfqpoint{8.975035in}{1.046978in}}%
\pgfpathlineto{\pgfqpoint{8.968724in}{1.036953in}}%
\pgfpathlineto{\pgfqpoint{8.962413in}{1.025913in}}%
\pgfpathlineto{\pgfqpoint{8.956102in}{1.013922in}}%
\pgfpathlineto{\pgfqpoint{8.949791in}{1.001063in}}%
\pgfpathlineto{\pgfqpoint{8.943480in}{0.987431in}}%
\pgfpathlineto{\pgfqpoint{8.937169in}{0.973136in}}%
\pgfpathlineto{\pgfqpoint{8.930858in}{0.958297in}}%
\pgfpathlineto{\pgfqpoint{8.924547in}{0.943042in}}%
\pgfpathlineto{\pgfqpoint{8.918236in}{0.927504in}}%
\pgfpathlineto{\pgfqpoint{8.911925in}{0.911817in}}%
\pgfpathlineto{\pgfqpoint{8.905614in}{0.896115in}}%
\pgfpathlineto{\pgfqpoint{8.899303in}{0.880528in}}%
\pgfpathlineto{\pgfqpoint{8.892992in}{0.865179in}}%
\pgfpathlineto{\pgfqpoint{8.886681in}{0.850183in}}%
\pgfpathlineto{\pgfqpoint{8.880370in}{0.835642in}}%
\pgfpathlineto{\pgfqpoint{8.874059in}{0.821645in}}%
\pgfpathlineto{\pgfqpoint{8.867748in}{0.808268in}}%
\pgfpathlineto{\pgfqpoint{8.861437in}{0.795571in}}%
\pgfpathlineto{\pgfqpoint{8.855126in}{0.783598in}}%
\pgfpathlineto{\pgfqpoint{8.848815in}{0.772375in}}%
\pgfpathlineto{\pgfqpoint{8.842504in}{0.761916in}}%
\pgfpathlineto{\pgfqpoint{8.836193in}{0.752220in}}%
\pgfpathlineto{\pgfqpoint{8.829883in}{0.743269in}}%
\pgfpathlineto{\pgfqpoint{8.823572in}{0.735037in}}%
\pgfpathlineto{\pgfqpoint{8.817261in}{0.727485in}}%
\pgfpathlineto{\pgfqpoint{8.810950in}{0.720566in}}%
\pgfpathlineto{\pgfqpoint{8.804639in}{0.714228in}}%
\pgfpathlineto{\pgfqpoint{8.798328in}{0.708412in}}%
\pgfpathlineto{\pgfqpoint{8.792017in}{0.703055in}}%
\pgfpathlineto{\pgfqpoint{8.785706in}{0.698097in}}%
\pgfpathlineto{\pgfqpoint{8.779395in}{0.693474in}}%
\pgfpathlineto{\pgfqpoint{8.773084in}{0.689127in}}%
\pgfpathlineto{\pgfqpoint{8.766773in}{0.684999in}}%
\pgfpathlineto{\pgfqpoint{8.760462in}{0.681037in}}%
\pgfpathlineto{\pgfqpoint{8.754151in}{0.677195in}}%
\pgfpathlineto{\pgfqpoint{8.747840in}{0.673430in}}%
\pgfpathlineto{\pgfqpoint{8.741529in}{0.669708in}}%
\pgfpathlineto{\pgfqpoint{8.735218in}{0.666001in}}%
\pgfpathlineto{\pgfqpoint{8.728907in}{0.662286in}}%
\pgfpathlineto{\pgfqpoint{8.722596in}{0.658548in}}%
\pgfpathlineto{\pgfqpoint{8.716285in}{0.654777in}}%
\pgfpathlineto{\pgfqpoint{8.709974in}{0.650970in}}%
\pgfpathlineto{\pgfqpoint{8.703663in}{0.647128in}}%
\pgfpathlineto{\pgfqpoint{8.697352in}{0.643257in}}%
\pgfpathlineto{\pgfqpoint{8.691041in}{0.639366in}}%
\pgfpathlineto{\pgfqpoint{8.684730in}{0.635467in}}%
\pgfpathlineto{\pgfqpoint{8.678419in}{0.631577in}}%
\pgfpathlineto{\pgfqpoint{8.672109in}{0.627712in}}%
\pgfpathlineto{\pgfqpoint{8.665798in}{0.623889in}}%
\pgfpathlineto{\pgfqpoint{8.659487in}{0.620127in}}%
\pgfpathlineto{\pgfqpoint{8.653176in}{0.616444in}}%
\pgfpathlineto{\pgfqpoint{8.646865in}{0.612857in}}%
\pgfpathlineto{\pgfqpoint{8.640554in}{0.609383in}}%
\pgfpathlineto{\pgfqpoint{8.634243in}{0.606035in}}%
\pgfpathlineto{\pgfqpoint{8.627932in}{0.602828in}}%
\pgfpathlineto{\pgfqpoint{8.621621in}{0.599772in}}%
\pgfpathlineto{\pgfqpoint{8.615310in}{0.596876in}}%
\pgfpathlineto{\pgfqpoint{8.608999in}{0.594147in}}%
\pgfpathlineto{\pgfqpoint{8.602688in}{0.591590in}}%
\pgfpathlineto{\pgfqpoint{8.596377in}{0.589206in}}%
\pgfpathlineto{\pgfqpoint{8.590066in}{0.586996in}}%
\pgfpathlineto{\pgfqpoint{8.583755in}{0.584960in}}%
\pgfpathlineto{\pgfqpoint{8.577444in}{0.583092in}}%
\pgfpathlineto{\pgfqpoint{8.571133in}{0.581390in}}%
\pgfpathlineto{\pgfqpoint{8.564822in}{0.579846in}}%
\pgfpathlineto{\pgfqpoint{8.558511in}{0.578454in}}%
\pgfpathlineto{\pgfqpoint{8.552200in}{0.577205in}}%
\pgfpathlineto{\pgfqpoint{8.545889in}{0.576091in}}%
\pgfpathlineto{\pgfqpoint{8.539578in}{0.575103in}}%
\pgfpathlineto{\pgfqpoint{8.533267in}{0.574230in}}%
\pgfpathlineto{\pgfqpoint{8.526956in}{0.573465in}}%
\pgfpathlineto{\pgfqpoint{8.520645in}{0.572797in}}%
\pgfpathlineto{\pgfqpoint{8.514335in}{0.572217in}}%
\pgfpathlineto{\pgfqpoint{8.508024in}{0.571717in}}%
\pgfpathlineto{\pgfqpoint{8.501713in}{0.571287in}}%
\pgfpathlineto{\pgfqpoint{8.495402in}{0.570920in}}%
\pgfpathlineto{\pgfqpoint{8.489091in}{0.570608in}}%
\pgfpathlineto{\pgfqpoint{8.482780in}{0.570345in}}%
\pgfpathlineto{\pgfqpoint{8.476469in}{0.570124in}}%
\pgfpathlineto{\pgfqpoint{8.470158in}{0.569939in}}%
\pgfpathlineto{\pgfqpoint{8.463847in}{0.569786in}}%
\pgfpathlineto{\pgfqpoint{8.457536in}{0.569659in}}%
\pgfpathlineto{\pgfqpoint{8.451225in}{0.569555in}}%
\pgfpathlineto{\pgfqpoint{8.444914in}{0.569470in}}%
\pgfpathlineto{\pgfqpoint{8.444914in}{0.569470in}}%
\pgfpathclose%
\pgfusepath{stroke,fill}%
}%
\begin{pgfscope}%
\pgfsys@transformshift{0.000000in}{0.000000in}%
\pgfsys@useobject{currentmarker}{}%
\end{pgfscope}%
\end{pgfscope}%
\begin{pgfscope}%
\pgfpathrectangle{\pgfqpoint{7.622482in}{0.569136in}}{\pgfqpoint{2.177280in}{2.201755in}}%
\pgfusepath{clip}%
\pgfsetbuttcap%
\pgfsetroundjoin%
\definecolor{currentfill}{rgb}{1.000000,0.498039,0.054902}%
\pgfsetfillcolor{currentfill}%
\pgfsetfillopacity{0.250000}%
\pgfsetlinewidth{1.003750pt}%
\definecolor{currentstroke}{rgb}{1.000000,0.498039,0.054902}%
\pgfsetstrokecolor{currentstroke}%
\pgfsetdash{}{0pt}%
\pgfsys@defobject{currentmarker}{\pgfqpoint{8.245377in}{0.569136in}}{\pgfqpoint{9.155040in}{1.346729in}}{%
\pgfpathmoveto{\pgfqpoint{8.245377in}{0.572018in}}%
\pgfpathlineto{\pgfqpoint{8.245377in}{0.569136in}}%
\pgfpathlineto{\pgfqpoint{8.249948in}{0.569136in}}%
\pgfpathlineto{\pgfqpoint{8.254519in}{0.569136in}}%
\pgfpathlineto{\pgfqpoint{8.259090in}{0.569136in}}%
\pgfpathlineto{\pgfqpoint{8.263661in}{0.569136in}}%
\pgfpathlineto{\pgfqpoint{8.268232in}{0.569136in}}%
\pgfpathlineto{\pgfqpoint{8.272804in}{0.569136in}}%
\pgfpathlineto{\pgfqpoint{8.277375in}{0.569136in}}%
\pgfpathlineto{\pgfqpoint{8.281946in}{0.569136in}}%
\pgfpathlineto{\pgfqpoint{8.286517in}{0.569136in}}%
\pgfpathlineto{\pgfqpoint{8.291088in}{0.569136in}}%
\pgfpathlineto{\pgfqpoint{8.295659in}{0.569136in}}%
\pgfpathlineto{\pgfqpoint{8.300231in}{0.569136in}}%
\pgfpathlineto{\pgfqpoint{8.304802in}{0.569136in}}%
\pgfpathlineto{\pgfqpoint{8.309373in}{0.569136in}}%
\pgfpathlineto{\pgfqpoint{8.313944in}{0.569136in}}%
\pgfpathlineto{\pgfqpoint{8.318515in}{0.569136in}}%
\pgfpathlineto{\pgfqpoint{8.323087in}{0.569136in}}%
\pgfpathlineto{\pgfqpoint{8.327658in}{0.569136in}}%
\pgfpathlineto{\pgfqpoint{8.332229in}{0.569136in}}%
\pgfpathlineto{\pgfqpoint{8.336800in}{0.569136in}}%
\pgfpathlineto{\pgfqpoint{8.341371in}{0.569136in}}%
\pgfpathlineto{\pgfqpoint{8.345942in}{0.569136in}}%
\pgfpathlineto{\pgfqpoint{8.350514in}{0.569136in}}%
\pgfpathlineto{\pgfqpoint{8.355085in}{0.569136in}}%
\pgfpathlineto{\pgfqpoint{8.359656in}{0.569136in}}%
\pgfpathlineto{\pgfqpoint{8.364227in}{0.569136in}}%
\pgfpathlineto{\pgfqpoint{8.368798in}{0.569136in}}%
\pgfpathlineto{\pgfqpoint{8.373369in}{0.569136in}}%
\pgfpathlineto{\pgfqpoint{8.377941in}{0.569136in}}%
\pgfpathlineto{\pgfqpoint{8.382512in}{0.569136in}}%
\pgfpathlineto{\pgfqpoint{8.387083in}{0.569136in}}%
\pgfpathlineto{\pgfqpoint{8.391654in}{0.569136in}}%
\pgfpathlineto{\pgfqpoint{8.396225in}{0.569136in}}%
\pgfpathlineto{\pgfqpoint{8.400796in}{0.569136in}}%
\pgfpathlineto{\pgfqpoint{8.405368in}{0.569136in}}%
\pgfpathlineto{\pgfqpoint{8.409939in}{0.569136in}}%
\pgfpathlineto{\pgfqpoint{8.414510in}{0.569136in}}%
\pgfpathlineto{\pgfqpoint{8.419081in}{0.569136in}}%
\pgfpathlineto{\pgfqpoint{8.423652in}{0.569136in}}%
\pgfpathlineto{\pgfqpoint{8.428223in}{0.569136in}}%
\pgfpathlineto{\pgfqpoint{8.432795in}{0.569136in}}%
\pgfpathlineto{\pgfqpoint{8.437366in}{0.569136in}}%
\pgfpathlineto{\pgfqpoint{8.441937in}{0.569136in}}%
\pgfpathlineto{\pgfqpoint{8.446508in}{0.569136in}}%
\pgfpathlineto{\pgfqpoint{8.451079in}{0.569136in}}%
\pgfpathlineto{\pgfqpoint{8.455651in}{0.569136in}}%
\pgfpathlineto{\pgfqpoint{8.460222in}{0.569136in}}%
\pgfpathlineto{\pgfqpoint{8.464793in}{0.569136in}}%
\pgfpathlineto{\pgfqpoint{8.469364in}{0.569136in}}%
\pgfpathlineto{\pgfqpoint{8.473935in}{0.569136in}}%
\pgfpathlineto{\pgfqpoint{8.478506in}{0.569136in}}%
\pgfpathlineto{\pgfqpoint{8.483078in}{0.569136in}}%
\pgfpathlineto{\pgfqpoint{8.487649in}{0.569136in}}%
\pgfpathlineto{\pgfqpoint{8.492220in}{0.569136in}}%
\pgfpathlineto{\pgfqpoint{8.496791in}{0.569136in}}%
\pgfpathlineto{\pgfqpoint{8.501362in}{0.569136in}}%
\pgfpathlineto{\pgfqpoint{8.505933in}{0.569136in}}%
\pgfpathlineto{\pgfqpoint{8.510505in}{0.569136in}}%
\pgfpathlineto{\pgfqpoint{8.515076in}{0.569136in}}%
\pgfpathlineto{\pgfqpoint{8.519647in}{0.569136in}}%
\pgfpathlineto{\pgfqpoint{8.524218in}{0.569136in}}%
\pgfpathlineto{\pgfqpoint{8.528789in}{0.569136in}}%
\pgfpathlineto{\pgfqpoint{8.533360in}{0.569136in}}%
\pgfpathlineto{\pgfqpoint{8.537932in}{0.569136in}}%
\pgfpathlineto{\pgfqpoint{8.542503in}{0.569136in}}%
\pgfpathlineto{\pgfqpoint{8.547074in}{0.569136in}}%
\pgfpathlineto{\pgfqpoint{8.551645in}{0.569136in}}%
\pgfpathlineto{\pgfqpoint{8.556216in}{0.569136in}}%
\pgfpathlineto{\pgfqpoint{8.560787in}{0.569136in}}%
\pgfpathlineto{\pgfqpoint{8.565359in}{0.569136in}}%
\pgfpathlineto{\pgfqpoint{8.569930in}{0.569136in}}%
\pgfpathlineto{\pgfqpoint{8.574501in}{0.569136in}}%
\pgfpathlineto{\pgfqpoint{8.579072in}{0.569136in}}%
\pgfpathlineto{\pgfqpoint{8.583643in}{0.569136in}}%
\pgfpathlineto{\pgfqpoint{8.588215in}{0.569136in}}%
\pgfpathlineto{\pgfqpoint{8.592786in}{0.569136in}}%
\pgfpathlineto{\pgfqpoint{8.597357in}{0.569136in}}%
\pgfpathlineto{\pgfqpoint{8.601928in}{0.569136in}}%
\pgfpathlineto{\pgfqpoint{8.606499in}{0.569136in}}%
\pgfpathlineto{\pgfqpoint{8.611070in}{0.569136in}}%
\pgfpathlineto{\pgfqpoint{8.615642in}{0.569136in}}%
\pgfpathlineto{\pgfqpoint{8.620213in}{0.569136in}}%
\pgfpathlineto{\pgfqpoint{8.624784in}{0.569136in}}%
\pgfpathlineto{\pgfqpoint{8.629355in}{0.569136in}}%
\pgfpathlineto{\pgfqpoint{8.633926in}{0.569136in}}%
\pgfpathlineto{\pgfqpoint{8.638497in}{0.569136in}}%
\pgfpathlineto{\pgfqpoint{8.643069in}{0.569136in}}%
\pgfpathlineto{\pgfqpoint{8.647640in}{0.569136in}}%
\pgfpathlineto{\pgfqpoint{8.652211in}{0.569136in}}%
\pgfpathlineto{\pgfqpoint{8.656782in}{0.569136in}}%
\pgfpathlineto{\pgfqpoint{8.661353in}{0.569136in}}%
\pgfpathlineto{\pgfqpoint{8.665924in}{0.569136in}}%
\pgfpathlineto{\pgfqpoint{8.670496in}{0.569136in}}%
\pgfpathlineto{\pgfqpoint{8.675067in}{0.569136in}}%
\pgfpathlineto{\pgfqpoint{8.679638in}{0.569136in}}%
\pgfpathlineto{\pgfqpoint{8.684209in}{0.569136in}}%
\pgfpathlineto{\pgfqpoint{8.688780in}{0.569136in}}%
\pgfpathlineto{\pgfqpoint{8.693351in}{0.569136in}}%
\pgfpathlineto{\pgfqpoint{8.697923in}{0.569136in}}%
\pgfpathlineto{\pgfqpoint{8.702494in}{0.569136in}}%
\pgfpathlineto{\pgfqpoint{8.707065in}{0.569136in}}%
\pgfpathlineto{\pgfqpoint{8.711636in}{0.569136in}}%
\pgfpathlineto{\pgfqpoint{8.716207in}{0.569136in}}%
\pgfpathlineto{\pgfqpoint{8.720779in}{0.569136in}}%
\pgfpathlineto{\pgfqpoint{8.725350in}{0.569136in}}%
\pgfpathlineto{\pgfqpoint{8.729921in}{0.569136in}}%
\pgfpathlineto{\pgfqpoint{8.734492in}{0.569136in}}%
\pgfpathlineto{\pgfqpoint{8.739063in}{0.569136in}}%
\pgfpathlineto{\pgfqpoint{8.743634in}{0.569136in}}%
\pgfpathlineto{\pgfqpoint{8.748206in}{0.569136in}}%
\pgfpathlineto{\pgfqpoint{8.752777in}{0.569136in}}%
\pgfpathlineto{\pgfqpoint{8.757348in}{0.569136in}}%
\pgfpathlineto{\pgfqpoint{8.761919in}{0.569136in}}%
\pgfpathlineto{\pgfqpoint{8.766490in}{0.569136in}}%
\pgfpathlineto{\pgfqpoint{8.771061in}{0.569136in}}%
\pgfpathlineto{\pgfqpoint{8.775633in}{0.569136in}}%
\pgfpathlineto{\pgfqpoint{8.780204in}{0.569136in}}%
\pgfpathlineto{\pgfqpoint{8.784775in}{0.569136in}}%
\pgfpathlineto{\pgfqpoint{8.789346in}{0.569136in}}%
\pgfpathlineto{\pgfqpoint{8.793917in}{0.569136in}}%
\pgfpathlineto{\pgfqpoint{8.798488in}{0.569136in}}%
\pgfpathlineto{\pgfqpoint{8.803060in}{0.569136in}}%
\pgfpathlineto{\pgfqpoint{8.807631in}{0.569136in}}%
\pgfpathlineto{\pgfqpoint{8.812202in}{0.569136in}}%
\pgfpathlineto{\pgfqpoint{8.816773in}{0.569136in}}%
\pgfpathlineto{\pgfqpoint{8.821344in}{0.569136in}}%
\pgfpathlineto{\pgfqpoint{8.825915in}{0.569136in}}%
\pgfpathlineto{\pgfqpoint{8.830487in}{0.569136in}}%
\pgfpathlineto{\pgfqpoint{8.835058in}{0.569136in}}%
\pgfpathlineto{\pgfqpoint{8.839629in}{0.569136in}}%
\pgfpathlineto{\pgfqpoint{8.844200in}{0.569136in}}%
\pgfpathlineto{\pgfqpoint{8.848771in}{0.569136in}}%
\pgfpathlineto{\pgfqpoint{8.853342in}{0.569136in}}%
\pgfpathlineto{\pgfqpoint{8.857914in}{0.569136in}}%
\pgfpathlineto{\pgfqpoint{8.862485in}{0.569136in}}%
\pgfpathlineto{\pgfqpoint{8.867056in}{0.569136in}}%
\pgfpathlineto{\pgfqpoint{8.871627in}{0.569136in}}%
\pgfpathlineto{\pgfqpoint{8.876198in}{0.569136in}}%
\pgfpathlineto{\pgfqpoint{8.880770in}{0.569136in}}%
\pgfpathlineto{\pgfqpoint{8.885341in}{0.569136in}}%
\pgfpathlineto{\pgfqpoint{8.889912in}{0.569136in}}%
\pgfpathlineto{\pgfqpoint{8.894483in}{0.569136in}}%
\pgfpathlineto{\pgfqpoint{8.899054in}{0.569136in}}%
\pgfpathlineto{\pgfqpoint{8.903625in}{0.569136in}}%
\pgfpathlineto{\pgfqpoint{8.908197in}{0.569136in}}%
\pgfpathlineto{\pgfqpoint{8.912768in}{0.569136in}}%
\pgfpathlineto{\pgfqpoint{8.917339in}{0.569136in}}%
\pgfpathlineto{\pgfqpoint{8.921910in}{0.569136in}}%
\pgfpathlineto{\pgfqpoint{8.926481in}{0.569136in}}%
\pgfpathlineto{\pgfqpoint{8.931052in}{0.569136in}}%
\pgfpathlineto{\pgfqpoint{8.935624in}{0.569136in}}%
\pgfpathlineto{\pgfqpoint{8.940195in}{0.569136in}}%
\pgfpathlineto{\pgfqpoint{8.944766in}{0.569136in}}%
\pgfpathlineto{\pgfqpoint{8.949337in}{0.569136in}}%
\pgfpathlineto{\pgfqpoint{8.953908in}{0.569136in}}%
\pgfpathlineto{\pgfqpoint{8.958479in}{0.569136in}}%
\pgfpathlineto{\pgfqpoint{8.963051in}{0.569136in}}%
\pgfpathlineto{\pgfqpoint{8.967622in}{0.569136in}}%
\pgfpathlineto{\pgfqpoint{8.972193in}{0.569136in}}%
\pgfpathlineto{\pgfqpoint{8.976764in}{0.569136in}}%
\pgfpathlineto{\pgfqpoint{8.981335in}{0.569136in}}%
\pgfpathlineto{\pgfqpoint{8.985906in}{0.569136in}}%
\pgfpathlineto{\pgfqpoint{8.990478in}{0.569136in}}%
\pgfpathlineto{\pgfqpoint{8.995049in}{0.569136in}}%
\pgfpathlineto{\pgfqpoint{8.999620in}{0.569136in}}%
\pgfpathlineto{\pgfqpoint{9.004191in}{0.569136in}}%
\pgfpathlineto{\pgfqpoint{9.008762in}{0.569136in}}%
\pgfpathlineto{\pgfqpoint{9.013334in}{0.569136in}}%
\pgfpathlineto{\pgfqpoint{9.017905in}{0.569136in}}%
\pgfpathlineto{\pgfqpoint{9.022476in}{0.569136in}}%
\pgfpathlineto{\pgfqpoint{9.027047in}{0.569136in}}%
\pgfpathlineto{\pgfqpoint{9.031618in}{0.569136in}}%
\pgfpathlineto{\pgfqpoint{9.036189in}{0.569136in}}%
\pgfpathlineto{\pgfqpoint{9.040761in}{0.569136in}}%
\pgfpathlineto{\pgfqpoint{9.045332in}{0.569136in}}%
\pgfpathlineto{\pgfqpoint{9.049903in}{0.569136in}}%
\pgfpathlineto{\pgfqpoint{9.054474in}{0.569136in}}%
\pgfpathlineto{\pgfqpoint{9.059045in}{0.569136in}}%
\pgfpathlineto{\pgfqpoint{9.063616in}{0.569136in}}%
\pgfpathlineto{\pgfqpoint{9.068188in}{0.569136in}}%
\pgfpathlineto{\pgfqpoint{9.072759in}{0.569136in}}%
\pgfpathlineto{\pgfqpoint{9.077330in}{0.569136in}}%
\pgfpathlineto{\pgfqpoint{9.081901in}{0.569136in}}%
\pgfpathlineto{\pgfqpoint{9.086472in}{0.569136in}}%
\pgfpathlineto{\pgfqpoint{9.091043in}{0.569136in}}%
\pgfpathlineto{\pgfqpoint{9.095615in}{0.569136in}}%
\pgfpathlineto{\pgfqpoint{9.100186in}{0.569136in}}%
\pgfpathlineto{\pgfqpoint{9.104757in}{0.569136in}}%
\pgfpathlineto{\pgfqpoint{9.109328in}{0.569136in}}%
\pgfpathlineto{\pgfqpoint{9.113899in}{0.569136in}}%
\pgfpathlineto{\pgfqpoint{9.118470in}{0.569136in}}%
\pgfpathlineto{\pgfqpoint{9.123042in}{0.569136in}}%
\pgfpathlineto{\pgfqpoint{9.127613in}{0.569136in}}%
\pgfpathlineto{\pgfqpoint{9.132184in}{0.569136in}}%
\pgfpathlineto{\pgfqpoint{9.136755in}{0.569136in}}%
\pgfpathlineto{\pgfqpoint{9.141326in}{0.569136in}}%
\pgfpathlineto{\pgfqpoint{9.145898in}{0.569136in}}%
\pgfpathlineto{\pgfqpoint{9.150469in}{0.569136in}}%
\pgfpathlineto{\pgfqpoint{9.155040in}{0.569136in}}%
\pgfpathlineto{\pgfqpoint{9.155040in}{0.569552in}}%
\pgfpathlineto{\pgfqpoint{9.155040in}{0.569552in}}%
\pgfpathlineto{\pgfqpoint{9.150469in}{0.569656in}}%
\pgfpathlineto{\pgfqpoint{9.145898in}{0.569782in}}%
\pgfpathlineto{\pgfqpoint{9.141326in}{0.569935in}}%
\pgfpathlineto{\pgfqpoint{9.136755in}{0.570118in}}%
\pgfpathlineto{\pgfqpoint{9.132184in}{0.570337in}}%
\pgfpathlineto{\pgfqpoint{9.127613in}{0.570597in}}%
\pgfpathlineto{\pgfqpoint{9.123042in}{0.570904in}}%
\pgfpathlineto{\pgfqpoint{9.118470in}{0.571264in}}%
\pgfpathlineto{\pgfqpoint{9.113899in}{0.571683in}}%
\pgfpathlineto{\pgfqpoint{9.109328in}{0.572170in}}%
\pgfpathlineto{\pgfqpoint{9.104757in}{0.572729in}}%
\pgfpathlineto{\pgfqpoint{9.100186in}{0.573371in}}%
\pgfpathlineto{\pgfqpoint{9.095615in}{0.574100in}}%
\pgfpathlineto{\pgfqpoint{9.091043in}{0.574926in}}%
\pgfpathlineto{\pgfqpoint{9.086472in}{0.575854in}}%
\pgfpathlineto{\pgfqpoint{9.081901in}{0.576892in}}%
\pgfpathlineto{\pgfqpoint{9.077330in}{0.578046in}}%
\pgfpathlineto{\pgfqpoint{9.072759in}{0.579320in}}%
\pgfpathlineto{\pgfqpoint{9.068188in}{0.580720in}}%
\pgfpathlineto{\pgfqpoint{9.063616in}{0.582249in}}%
\pgfpathlineto{\pgfqpoint{9.059045in}{0.583908in}}%
\pgfpathlineto{\pgfqpoint{9.054474in}{0.585699in}}%
\pgfpathlineto{\pgfqpoint{9.049903in}{0.587622in}}%
\pgfpathlineto{\pgfqpoint{9.045332in}{0.589675in}}%
\pgfpathlineto{\pgfqpoint{9.040761in}{0.591854in}}%
\pgfpathlineto{\pgfqpoint{9.036189in}{0.594157in}}%
\pgfpathlineto{\pgfqpoint{9.031618in}{0.596577in}}%
\pgfpathlineto{\pgfqpoint{9.027047in}{0.599108in}}%
\pgfpathlineto{\pgfqpoint{9.022476in}{0.601744in}}%
\pgfpathlineto{\pgfqpoint{9.017905in}{0.604478in}}%
\pgfpathlineto{\pgfqpoint{9.013334in}{0.607302in}}%
\pgfpathlineto{\pgfqpoint{9.008762in}{0.610210in}}%
\pgfpathlineto{\pgfqpoint{9.004191in}{0.613196in}}%
\pgfpathlineto{\pgfqpoint{8.999620in}{0.616255in}}%
\pgfpathlineto{\pgfqpoint{8.995049in}{0.619385in}}%
\pgfpathlineto{\pgfqpoint{8.990478in}{0.622584in}}%
\pgfpathlineto{\pgfqpoint{8.985906in}{0.625854in}}%
\pgfpathlineto{\pgfqpoint{8.981335in}{0.629200in}}%
\pgfpathlineto{\pgfqpoint{8.976764in}{0.632629in}}%
\pgfpathlineto{\pgfqpoint{8.972193in}{0.636153in}}%
\pgfpathlineto{\pgfqpoint{8.967622in}{0.639785in}}%
\pgfpathlineto{\pgfqpoint{8.963051in}{0.643544in}}%
\pgfpathlineto{\pgfqpoint{8.958479in}{0.647453in}}%
\pgfpathlineto{\pgfqpoint{8.953908in}{0.651537in}}%
\pgfpathlineto{\pgfqpoint{8.949337in}{0.655826in}}%
\pgfpathlineto{\pgfqpoint{8.944766in}{0.660352in}}%
\pgfpathlineto{\pgfqpoint{8.940195in}{0.665152in}}%
\pgfpathlineto{\pgfqpoint{8.935624in}{0.670264in}}%
\pgfpathlineto{\pgfqpoint{8.931052in}{0.675728in}}%
\pgfpathlineto{\pgfqpoint{8.926481in}{0.681587in}}%
\pgfpathlineto{\pgfqpoint{8.921910in}{0.687884in}}%
\pgfpathlineto{\pgfqpoint{8.917339in}{0.694664in}}%
\pgfpathlineto{\pgfqpoint{8.912768in}{0.701968in}}%
\pgfpathlineto{\pgfqpoint{8.908197in}{0.709839in}}%
\pgfpathlineto{\pgfqpoint{8.903625in}{0.718318in}}%
\pgfpathlineto{\pgfqpoint{8.899054in}{0.727441in}}%
\pgfpathlineto{\pgfqpoint{8.894483in}{0.737241in}}%
\pgfpathlineto{\pgfqpoint{8.889912in}{0.747746in}}%
\pgfpathlineto{\pgfqpoint{8.885341in}{0.758980in}}%
\pgfpathlineto{\pgfqpoint{8.880770in}{0.770958in}}%
\pgfpathlineto{\pgfqpoint{8.876198in}{0.783688in}}%
\pgfpathlineto{\pgfqpoint{8.871627in}{0.797171in}}%
\pgfpathlineto{\pgfqpoint{8.867056in}{0.811398in}}%
\pgfpathlineto{\pgfqpoint{8.862485in}{0.826350in}}%
\pgfpathlineto{\pgfqpoint{8.857914in}{0.841998in}}%
\pgfpathlineto{\pgfqpoint{8.853342in}{0.858303in}}%
\pgfpathlineto{\pgfqpoint{8.848771in}{0.875214in}}%
\pgfpathlineto{\pgfqpoint{8.844200in}{0.892670in}}%
\pgfpathlineto{\pgfqpoint{8.839629in}{0.910601in}}%
\pgfpathlineto{\pgfqpoint{8.835058in}{0.928927in}}%
\pgfpathlineto{\pgfqpoint{8.830487in}{0.947559in}}%
\pgfpathlineto{\pgfqpoint{8.825915in}{0.966401in}}%
\pgfpathlineto{\pgfqpoint{8.821344in}{0.985352in}}%
\pgfpathlineto{\pgfqpoint{8.816773in}{1.004307in}}%
\pgfpathlineto{\pgfqpoint{8.812202in}{1.023159in}}%
\pgfpathlineto{\pgfqpoint{8.807631in}{1.041802in}}%
\pgfpathlineto{\pgfqpoint{8.803060in}{1.060133in}}%
\pgfpathlineto{\pgfqpoint{8.798488in}{1.078054in}}%
\pgfpathlineto{\pgfqpoint{8.793917in}{1.095474in}}%
\pgfpathlineto{\pgfqpoint{8.789346in}{1.112312in}}%
\pgfpathlineto{\pgfqpoint{8.784775in}{1.128501in}}%
\pgfpathlineto{\pgfqpoint{8.780204in}{1.143983in}}%
\pgfpathlineto{\pgfqpoint{8.775633in}{1.158719in}}%
\pgfpathlineto{\pgfqpoint{8.771061in}{1.172682in}}%
\pgfpathlineto{\pgfqpoint{8.766490in}{1.185863in}}%
\pgfpathlineto{\pgfqpoint{8.761919in}{1.198266in}}%
\pgfpathlineto{\pgfqpoint{8.757348in}{1.209911in}}%
\pgfpathlineto{\pgfqpoint{8.752777in}{1.220833in}}%
\pgfpathlineto{\pgfqpoint{8.748206in}{1.231074in}}%
\pgfpathlineto{\pgfqpoint{8.743634in}{1.240687in}}%
\pgfpathlineto{\pgfqpoint{8.739063in}{1.249732in}}%
\pgfpathlineto{\pgfqpoint{8.734492in}{1.258270in}}%
\pgfpathlineto{\pgfqpoint{8.729921in}{1.266364in}}%
\pgfpathlineto{\pgfqpoint{8.725350in}{1.274071in}}%
\pgfpathlineto{\pgfqpoint{8.720779in}{1.281444in}}%
\pgfpathlineto{\pgfqpoint{8.716207in}{1.288526in}}%
\pgfpathlineto{\pgfqpoint{8.711636in}{1.295345in}}%
\pgfpathlineto{\pgfqpoint{8.707065in}{1.301918in}}%
\pgfpathlineto{\pgfqpoint{8.702494in}{1.308246in}}%
\pgfpathlineto{\pgfqpoint{8.697923in}{1.314311in}}%
\pgfpathlineto{\pgfqpoint{8.693351in}{1.320078in}}%
\pgfpathlineto{\pgfqpoint{8.688780in}{1.325499in}}%
\pgfpathlineto{\pgfqpoint{8.684209in}{1.330505in}}%
\pgfpathlineto{\pgfqpoint{8.679638in}{1.335017in}}%
\pgfpathlineto{\pgfqpoint{8.675067in}{1.338943in}}%
\pgfpathlineto{\pgfqpoint{8.670496in}{1.342183in}}%
\pgfpathlineto{\pgfqpoint{8.665924in}{1.344631in}}%
\pgfpathlineto{\pgfqpoint{8.661353in}{1.346182in}}%
\pgfpathlineto{\pgfqpoint{8.656782in}{1.346729in}}%
\pgfpathlineto{\pgfqpoint{8.652211in}{1.346176in}}%
\pgfpathlineto{\pgfqpoint{8.647640in}{1.344433in}}%
\pgfpathlineto{\pgfqpoint{8.643069in}{1.341427in}}%
\pgfpathlineto{\pgfqpoint{8.638497in}{1.337099in}}%
\pgfpathlineto{\pgfqpoint{8.633926in}{1.331409in}}%
\pgfpathlineto{\pgfqpoint{8.629355in}{1.324340in}}%
\pgfpathlineto{\pgfqpoint{8.624784in}{1.315896in}}%
\pgfpathlineto{\pgfqpoint{8.620213in}{1.306105in}}%
\pgfpathlineto{\pgfqpoint{8.615642in}{1.295018in}}%
\pgfpathlineto{\pgfqpoint{8.611070in}{1.282709in}}%
\pgfpathlineto{\pgfqpoint{8.606499in}{1.269271in}}%
\pgfpathlineto{\pgfqpoint{8.601928in}{1.254821in}}%
\pgfpathlineto{\pgfqpoint{8.597357in}{1.239488in}}%
\pgfpathlineto{\pgfqpoint{8.592786in}{1.223417in}}%
\pgfpathlineto{\pgfqpoint{8.588215in}{1.206765in}}%
\pgfpathlineto{\pgfqpoint{8.583643in}{1.189696in}}%
\pgfpathlineto{\pgfqpoint{8.579072in}{1.172378in}}%
\pgfpathlineto{\pgfqpoint{8.574501in}{1.154981in}}%
\pgfpathlineto{\pgfqpoint{8.569930in}{1.137670in}}%
\pgfpathlineto{\pgfqpoint{8.565359in}{1.120607in}}%
\pgfpathlineto{\pgfqpoint{8.560787in}{1.103945in}}%
\pgfpathlineto{\pgfqpoint{8.556216in}{1.087825in}}%
\pgfpathlineto{\pgfqpoint{8.551645in}{1.072376in}}%
\pgfpathlineto{\pgfqpoint{8.547074in}{1.057709in}}%
\pgfpathlineto{\pgfqpoint{8.542503in}{1.043919in}}%
\pgfpathlineto{\pgfqpoint{8.537932in}{1.031083in}}%
\pgfpathlineto{\pgfqpoint{8.533360in}{1.019258in}}%
\pgfpathlineto{\pgfqpoint{8.528789in}{1.008479in}}%
\pgfpathlineto{\pgfqpoint{8.524218in}{0.998763in}}%
\pgfpathlineto{\pgfqpoint{8.519647in}{0.990104in}}%
\pgfpathlineto{\pgfqpoint{8.515076in}{0.982478in}}%
\pgfpathlineto{\pgfqpoint{8.510505in}{0.975839in}}%
\pgfpathlineto{\pgfqpoint{8.505933in}{0.970124in}}%
\pgfpathlineto{\pgfqpoint{8.501362in}{0.965252in}}%
\pgfpathlineto{\pgfqpoint{8.496791in}{0.961127in}}%
\pgfpathlineto{\pgfqpoint{8.492220in}{0.957637in}}%
\pgfpathlineto{\pgfqpoint{8.487649in}{0.954660in}}%
\pgfpathlineto{\pgfqpoint{8.483078in}{0.952065in}}%
\pgfpathlineto{\pgfqpoint{8.478506in}{0.949713in}}%
\pgfpathlineto{\pgfqpoint{8.473935in}{0.947462in}}%
\pgfpathlineto{\pgfqpoint{8.469364in}{0.945169in}}%
\pgfpathlineto{\pgfqpoint{8.464793in}{0.942694in}}%
\pgfpathlineto{\pgfqpoint{8.460222in}{0.939900in}}%
\pgfpathlineto{\pgfqpoint{8.455651in}{0.936660in}}%
\pgfpathlineto{\pgfqpoint{8.451079in}{0.932859in}}%
\pgfpathlineto{\pgfqpoint{8.446508in}{0.928395in}}%
\pgfpathlineto{\pgfqpoint{8.441937in}{0.923180in}}%
\pgfpathlineto{\pgfqpoint{8.437366in}{0.917148in}}%
\pgfpathlineto{\pgfqpoint{8.432795in}{0.910250in}}%
\pgfpathlineto{\pgfqpoint{8.428223in}{0.902457in}}%
\pgfpathlineto{\pgfqpoint{8.423652in}{0.893764in}}%
\pgfpathlineto{\pgfqpoint{8.419081in}{0.884184in}}%
\pgfpathlineto{\pgfqpoint{8.414510in}{0.873751in}}%
\pgfpathlineto{\pgfqpoint{8.409939in}{0.862519in}}%
\pgfpathlineto{\pgfqpoint{8.405368in}{0.850559in}}%
\pgfpathlineto{\pgfqpoint{8.400796in}{0.837957in}}%
\pgfpathlineto{\pgfqpoint{8.396225in}{0.824812in}}%
\pgfpathlineto{\pgfqpoint{8.391654in}{0.811232in}}%
\pgfpathlineto{\pgfqpoint{8.387083in}{0.797335in}}%
\pgfpathlineto{\pgfqpoint{8.382512in}{0.783240in}}%
\pgfpathlineto{\pgfqpoint{8.377941in}{0.769069in}}%
\pgfpathlineto{\pgfqpoint{8.373369in}{0.754941in}}%
\pgfpathlineto{\pgfqpoint{8.368798in}{0.740973in}}%
\pgfpathlineto{\pgfqpoint{8.364227in}{0.727273in}}%
\pgfpathlineto{\pgfqpoint{8.359656in}{0.713941in}}%
\pgfpathlineto{\pgfqpoint{8.355085in}{0.701066in}}%
\pgfpathlineto{\pgfqpoint{8.350514in}{0.688726in}}%
\pgfpathlineto{\pgfqpoint{8.345942in}{0.676985in}}%
\pgfpathlineto{\pgfqpoint{8.341371in}{0.665895in}}%
\pgfpathlineto{\pgfqpoint{8.336800in}{0.655496in}}%
\pgfpathlineto{\pgfqpoint{8.332229in}{0.645811in}}%
\pgfpathlineto{\pgfqpoint{8.327658in}{0.636856in}}%
\pgfpathlineto{\pgfqpoint{8.323087in}{0.628631in}}%
\pgfpathlineto{\pgfqpoint{8.318515in}{0.621128in}}%
\pgfpathlineto{\pgfqpoint{8.313944in}{0.614329in}}%
\pgfpathlineto{\pgfqpoint{8.309373in}{0.608210in}}%
\pgfpathlineto{\pgfqpoint{8.304802in}{0.602738in}}%
\pgfpathlineto{\pgfqpoint{8.300231in}{0.597877in}}%
\pgfpathlineto{\pgfqpoint{8.295659in}{0.593587in}}%
\pgfpathlineto{\pgfqpoint{8.291088in}{0.589825in}}%
\pgfpathlineto{\pgfqpoint{8.286517in}{0.586547in}}%
\pgfpathlineto{\pgfqpoint{8.281946in}{0.583709in}}%
\pgfpathlineto{\pgfqpoint{8.277375in}{0.581267in}}%
\pgfpathlineto{\pgfqpoint{8.272804in}{0.579179in}}%
\pgfpathlineto{\pgfqpoint{8.268232in}{0.577406in}}%
\pgfpathlineto{\pgfqpoint{8.263661in}{0.575908in}}%
\pgfpathlineto{\pgfqpoint{8.259090in}{0.574651in}}%
\pgfpathlineto{\pgfqpoint{8.254519in}{0.573603in}}%
\pgfpathlineto{\pgfqpoint{8.249948in}{0.572734in}}%
\pgfpathlineto{\pgfqpoint{8.245377in}{0.572018in}}%
\pgfpathlineto{\pgfqpoint{8.245377in}{0.572018in}}%
\pgfpathclose%
\pgfusepath{stroke,fill}%
}%
\begin{pgfscope}%
\pgfsys@transformshift{0.000000in}{0.000000in}%
\pgfsys@useobject{currentmarker}{}%
\end{pgfscope}%
\end{pgfscope}%
\begin{pgfscope}%
\pgfpathrectangle{\pgfqpoint{7.622482in}{0.569136in}}{\pgfqpoint{2.177280in}{2.201755in}}%
\pgfusepath{clip}%
\pgfsetbuttcap%
\pgfsetroundjoin%
\definecolor{currentfill}{rgb}{0.121569,0.466667,0.705882}%
\pgfsetfillcolor{currentfill}%
\pgfsetfillopacity{0.250000}%
\pgfsetlinewidth{1.003750pt}%
\definecolor{currentstroke}{rgb}{0.121569,0.466667,0.705882}%
\pgfsetstrokecolor{currentstroke}%
\pgfsetdash{}{0pt}%
\pgfsys@defobject{currentmarker}{\pgfqpoint{7.721449in}{0.569136in}}{\pgfqpoint{8.256136in}{2.666045in}}{%
\pgfpathmoveto{\pgfqpoint{7.721449in}{0.572972in}}%
\pgfpathlineto{\pgfqpoint{7.721449in}{0.569136in}}%
\pgfpathlineto{\pgfqpoint{7.724136in}{0.569136in}}%
\pgfpathlineto{\pgfqpoint{7.726823in}{0.569136in}}%
\pgfpathlineto{\pgfqpoint{7.729510in}{0.569136in}}%
\pgfpathlineto{\pgfqpoint{7.732197in}{0.569136in}}%
\pgfpathlineto{\pgfqpoint{7.734884in}{0.569136in}}%
\pgfpathlineto{\pgfqpoint{7.737571in}{0.569136in}}%
\pgfpathlineto{\pgfqpoint{7.740257in}{0.569136in}}%
\pgfpathlineto{\pgfqpoint{7.742944in}{0.569136in}}%
\pgfpathlineto{\pgfqpoint{7.745631in}{0.569136in}}%
\pgfpathlineto{\pgfqpoint{7.748318in}{0.569136in}}%
\pgfpathlineto{\pgfqpoint{7.751005in}{0.569136in}}%
\pgfpathlineto{\pgfqpoint{7.753692in}{0.569136in}}%
\pgfpathlineto{\pgfqpoint{7.756379in}{0.569136in}}%
\pgfpathlineto{\pgfqpoint{7.759065in}{0.569136in}}%
\pgfpathlineto{\pgfqpoint{7.761752in}{0.569136in}}%
\pgfpathlineto{\pgfqpoint{7.764439in}{0.569136in}}%
\pgfpathlineto{\pgfqpoint{7.767126in}{0.569136in}}%
\pgfpathlineto{\pgfqpoint{7.769813in}{0.569136in}}%
\pgfpathlineto{\pgfqpoint{7.772500in}{0.569136in}}%
\pgfpathlineto{\pgfqpoint{7.775187in}{0.569136in}}%
\pgfpathlineto{\pgfqpoint{7.777874in}{0.569136in}}%
\pgfpathlineto{\pgfqpoint{7.780560in}{0.569136in}}%
\pgfpathlineto{\pgfqpoint{7.783247in}{0.569136in}}%
\pgfpathlineto{\pgfqpoint{7.785934in}{0.569136in}}%
\pgfpathlineto{\pgfqpoint{7.788621in}{0.569136in}}%
\pgfpathlineto{\pgfqpoint{7.791308in}{0.569136in}}%
\pgfpathlineto{\pgfqpoint{7.793995in}{0.569136in}}%
\pgfpathlineto{\pgfqpoint{7.796682in}{0.569136in}}%
\pgfpathlineto{\pgfqpoint{7.799368in}{0.569136in}}%
\pgfpathlineto{\pgfqpoint{7.802055in}{0.569136in}}%
\pgfpathlineto{\pgfqpoint{7.804742in}{0.569136in}}%
\pgfpathlineto{\pgfqpoint{7.807429in}{0.569136in}}%
\pgfpathlineto{\pgfqpoint{7.810116in}{0.569136in}}%
\pgfpathlineto{\pgfqpoint{7.812803in}{0.569136in}}%
\pgfpathlineto{\pgfqpoint{7.815490in}{0.569136in}}%
\pgfpathlineto{\pgfqpoint{7.818177in}{0.569136in}}%
\pgfpathlineto{\pgfqpoint{7.820863in}{0.569136in}}%
\pgfpathlineto{\pgfqpoint{7.823550in}{0.569136in}}%
\pgfpathlineto{\pgfqpoint{7.826237in}{0.569136in}}%
\pgfpathlineto{\pgfqpoint{7.828924in}{0.569136in}}%
\pgfpathlineto{\pgfqpoint{7.831611in}{0.569136in}}%
\pgfpathlineto{\pgfqpoint{7.834298in}{0.569136in}}%
\pgfpathlineto{\pgfqpoint{7.836985in}{0.569136in}}%
\pgfpathlineto{\pgfqpoint{7.839671in}{0.569136in}}%
\pgfpathlineto{\pgfqpoint{7.842358in}{0.569136in}}%
\pgfpathlineto{\pgfqpoint{7.845045in}{0.569136in}}%
\pgfpathlineto{\pgfqpoint{7.847732in}{0.569136in}}%
\pgfpathlineto{\pgfqpoint{7.850419in}{0.569136in}}%
\pgfpathlineto{\pgfqpoint{7.853106in}{0.569136in}}%
\pgfpathlineto{\pgfqpoint{7.855793in}{0.569136in}}%
\pgfpathlineto{\pgfqpoint{7.858480in}{0.569136in}}%
\pgfpathlineto{\pgfqpoint{7.861166in}{0.569136in}}%
\pgfpathlineto{\pgfqpoint{7.863853in}{0.569136in}}%
\pgfpathlineto{\pgfqpoint{7.866540in}{0.569136in}}%
\pgfpathlineto{\pgfqpoint{7.869227in}{0.569136in}}%
\pgfpathlineto{\pgfqpoint{7.871914in}{0.569136in}}%
\pgfpathlineto{\pgfqpoint{7.874601in}{0.569136in}}%
\pgfpathlineto{\pgfqpoint{7.877288in}{0.569136in}}%
\pgfpathlineto{\pgfqpoint{7.879974in}{0.569136in}}%
\pgfpathlineto{\pgfqpoint{7.882661in}{0.569136in}}%
\pgfpathlineto{\pgfqpoint{7.885348in}{0.569136in}}%
\pgfpathlineto{\pgfqpoint{7.888035in}{0.569136in}}%
\pgfpathlineto{\pgfqpoint{7.890722in}{0.569136in}}%
\pgfpathlineto{\pgfqpoint{7.893409in}{0.569136in}}%
\pgfpathlineto{\pgfqpoint{7.896096in}{0.569136in}}%
\pgfpathlineto{\pgfqpoint{7.898783in}{0.569136in}}%
\pgfpathlineto{\pgfqpoint{7.901469in}{0.569136in}}%
\pgfpathlineto{\pgfqpoint{7.904156in}{0.569136in}}%
\pgfpathlineto{\pgfqpoint{7.906843in}{0.569136in}}%
\pgfpathlineto{\pgfqpoint{7.909530in}{0.569136in}}%
\pgfpathlineto{\pgfqpoint{7.912217in}{0.569136in}}%
\pgfpathlineto{\pgfqpoint{7.914904in}{0.569136in}}%
\pgfpathlineto{\pgfqpoint{7.917591in}{0.569136in}}%
\pgfpathlineto{\pgfqpoint{7.920277in}{0.569136in}}%
\pgfpathlineto{\pgfqpoint{7.922964in}{0.569136in}}%
\pgfpathlineto{\pgfqpoint{7.925651in}{0.569136in}}%
\pgfpathlineto{\pgfqpoint{7.928338in}{0.569136in}}%
\pgfpathlineto{\pgfqpoint{7.931025in}{0.569136in}}%
\pgfpathlineto{\pgfqpoint{7.933712in}{0.569136in}}%
\pgfpathlineto{\pgfqpoint{7.936399in}{0.569136in}}%
\pgfpathlineto{\pgfqpoint{7.939086in}{0.569136in}}%
\pgfpathlineto{\pgfqpoint{7.941772in}{0.569136in}}%
\pgfpathlineto{\pgfqpoint{7.944459in}{0.569136in}}%
\pgfpathlineto{\pgfqpoint{7.947146in}{0.569136in}}%
\pgfpathlineto{\pgfqpoint{7.949833in}{0.569136in}}%
\pgfpathlineto{\pgfqpoint{7.952520in}{0.569136in}}%
\pgfpathlineto{\pgfqpoint{7.955207in}{0.569136in}}%
\pgfpathlineto{\pgfqpoint{7.957894in}{0.569136in}}%
\pgfpathlineto{\pgfqpoint{7.960581in}{0.569136in}}%
\pgfpathlineto{\pgfqpoint{7.963267in}{0.569136in}}%
\pgfpathlineto{\pgfqpoint{7.965954in}{0.569136in}}%
\pgfpathlineto{\pgfqpoint{7.968641in}{0.569136in}}%
\pgfpathlineto{\pgfqpoint{7.971328in}{0.569136in}}%
\pgfpathlineto{\pgfqpoint{7.974015in}{0.569136in}}%
\pgfpathlineto{\pgfqpoint{7.976702in}{0.569136in}}%
\pgfpathlineto{\pgfqpoint{7.979389in}{0.569136in}}%
\pgfpathlineto{\pgfqpoint{7.982075in}{0.569136in}}%
\pgfpathlineto{\pgfqpoint{7.984762in}{0.569136in}}%
\pgfpathlineto{\pgfqpoint{7.987449in}{0.569136in}}%
\pgfpathlineto{\pgfqpoint{7.990136in}{0.569136in}}%
\pgfpathlineto{\pgfqpoint{7.992823in}{0.569136in}}%
\pgfpathlineto{\pgfqpoint{7.995510in}{0.569136in}}%
\pgfpathlineto{\pgfqpoint{7.998197in}{0.569136in}}%
\pgfpathlineto{\pgfqpoint{8.000884in}{0.569136in}}%
\pgfpathlineto{\pgfqpoint{8.003570in}{0.569136in}}%
\pgfpathlineto{\pgfqpoint{8.006257in}{0.569136in}}%
\pgfpathlineto{\pgfqpoint{8.008944in}{0.569136in}}%
\pgfpathlineto{\pgfqpoint{8.011631in}{0.569136in}}%
\pgfpathlineto{\pgfqpoint{8.014318in}{0.569136in}}%
\pgfpathlineto{\pgfqpoint{8.017005in}{0.569136in}}%
\pgfpathlineto{\pgfqpoint{8.019692in}{0.569136in}}%
\pgfpathlineto{\pgfqpoint{8.022378in}{0.569136in}}%
\pgfpathlineto{\pgfqpoint{8.025065in}{0.569136in}}%
\pgfpathlineto{\pgfqpoint{8.027752in}{0.569136in}}%
\pgfpathlineto{\pgfqpoint{8.030439in}{0.569136in}}%
\pgfpathlineto{\pgfqpoint{8.033126in}{0.569136in}}%
\pgfpathlineto{\pgfqpoint{8.035813in}{0.569136in}}%
\pgfpathlineto{\pgfqpoint{8.038500in}{0.569136in}}%
\pgfpathlineto{\pgfqpoint{8.041187in}{0.569136in}}%
\pgfpathlineto{\pgfqpoint{8.043873in}{0.569136in}}%
\pgfpathlineto{\pgfqpoint{8.046560in}{0.569136in}}%
\pgfpathlineto{\pgfqpoint{8.049247in}{0.569136in}}%
\pgfpathlineto{\pgfqpoint{8.051934in}{0.569136in}}%
\pgfpathlineto{\pgfqpoint{8.054621in}{0.569136in}}%
\pgfpathlineto{\pgfqpoint{8.057308in}{0.569136in}}%
\pgfpathlineto{\pgfqpoint{8.059995in}{0.569136in}}%
\pgfpathlineto{\pgfqpoint{8.062681in}{0.569136in}}%
\pgfpathlineto{\pgfqpoint{8.065368in}{0.569136in}}%
\pgfpathlineto{\pgfqpoint{8.068055in}{0.569136in}}%
\pgfpathlineto{\pgfqpoint{8.070742in}{0.569136in}}%
\pgfpathlineto{\pgfqpoint{8.073429in}{0.569136in}}%
\pgfpathlineto{\pgfqpoint{8.076116in}{0.569136in}}%
\pgfpathlineto{\pgfqpoint{8.078803in}{0.569136in}}%
\pgfpathlineto{\pgfqpoint{8.081490in}{0.569136in}}%
\pgfpathlineto{\pgfqpoint{8.084176in}{0.569136in}}%
\pgfpathlineto{\pgfqpoint{8.086863in}{0.569136in}}%
\pgfpathlineto{\pgfqpoint{8.089550in}{0.569136in}}%
\pgfpathlineto{\pgfqpoint{8.092237in}{0.569136in}}%
\pgfpathlineto{\pgfqpoint{8.094924in}{0.569136in}}%
\pgfpathlineto{\pgfqpoint{8.097611in}{0.569136in}}%
\pgfpathlineto{\pgfqpoint{8.100298in}{0.569136in}}%
\pgfpathlineto{\pgfqpoint{8.102984in}{0.569136in}}%
\pgfpathlineto{\pgfqpoint{8.105671in}{0.569136in}}%
\pgfpathlineto{\pgfqpoint{8.108358in}{0.569136in}}%
\pgfpathlineto{\pgfqpoint{8.111045in}{0.569136in}}%
\pgfpathlineto{\pgfqpoint{8.113732in}{0.569136in}}%
\pgfpathlineto{\pgfqpoint{8.116419in}{0.569136in}}%
\pgfpathlineto{\pgfqpoint{8.119106in}{0.569136in}}%
\pgfpathlineto{\pgfqpoint{8.121793in}{0.569136in}}%
\pgfpathlineto{\pgfqpoint{8.124479in}{0.569136in}}%
\pgfpathlineto{\pgfqpoint{8.127166in}{0.569136in}}%
\pgfpathlineto{\pgfqpoint{8.129853in}{0.569136in}}%
\pgfpathlineto{\pgfqpoint{8.132540in}{0.569136in}}%
\pgfpathlineto{\pgfqpoint{8.135227in}{0.569136in}}%
\pgfpathlineto{\pgfqpoint{8.137914in}{0.569136in}}%
\pgfpathlineto{\pgfqpoint{8.140601in}{0.569136in}}%
\pgfpathlineto{\pgfqpoint{8.143287in}{0.569136in}}%
\pgfpathlineto{\pgfqpoint{8.145974in}{0.569136in}}%
\pgfpathlineto{\pgfqpoint{8.148661in}{0.569136in}}%
\pgfpathlineto{\pgfqpoint{8.151348in}{0.569136in}}%
\pgfpathlineto{\pgfqpoint{8.154035in}{0.569136in}}%
\pgfpathlineto{\pgfqpoint{8.156722in}{0.569136in}}%
\pgfpathlineto{\pgfqpoint{8.159409in}{0.569136in}}%
\pgfpathlineto{\pgfqpoint{8.162096in}{0.569136in}}%
\pgfpathlineto{\pgfqpoint{8.164782in}{0.569136in}}%
\pgfpathlineto{\pgfqpoint{8.167469in}{0.569136in}}%
\pgfpathlineto{\pgfqpoint{8.170156in}{0.569136in}}%
\pgfpathlineto{\pgfqpoint{8.172843in}{0.569136in}}%
\pgfpathlineto{\pgfqpoint{8.175530in}{0.569136in}}%
\pgfpathlineto{\pgfqpoint{8.178217in}{0.569136in}}%
\pgfpathlineto{\pgfqpoint{8.180904in}{0.569136in}}%
\pgfpathlineto{\pgfqpoint{8.183590in}{0.569136in}}%
\pgfpathlineto{\pgfqpoint{8.186277in}{0.569136in}}%
\pgfpathlineto{\pgfqpoint{8.188964in}{0.569136in}}%
\pgfpathlineto{\pgfqpoint{8.191651in}{0.569136in}}%
\pgfpathlineto{\pgfqpoint{8.194338in}{0.569136in}}%
\pgfpathlineto{\pgfqpoint{8.197025in}{0.569136in}}%
\pgfpathlineto{\pgfqpoint{8.199712in}{0.569136in}}%
\pgfpathlineto{\pgfqpoint{8.202399in}{0.569136in}}%
\pgfpathlineto{\pgfqpoint{8.205085in}{0.569136in}}%
\pgfpathlineto{\pgfqpoint{8.207772in}{0.569136in}}%
\pgfpathlineto{\pgfqpoint{8.210459in}{0.569136in}}%
\pgfpathlineto{\pgfqpoint{8.213146in}{0.569136in}}%
\pgfpathlineto{\pgfqpoint{8.215833in}{0.569136in}}%
\pgfpathlineto{\pgfqpoint{8.218520in}{0.569136in}}%
\pgfpathlineto{\pgfqpoint{8.221207in}{0.569136in}}%
\pgfpathlineto{\pgfqpoint{8.223893in}{0.569136in}}%
\pgfpathlineto{\pgfqpoint{8.226580in}{0.569136in}}%
\pgfpathlineto{\pgfqpoint{8.229267in}{0.569136in}}%
\pgfpathlineto{\pgfqpoint{8.231954in}{0.569136in}}%
\pgfpathlineto{\pgfqpoint{8.234641in}{0.569136in}}%
\pgfpathlineto{\pgfqpoint{8.237328in}{0.569136in}}%
\pgfpathlineto{\pgfqpoint{8.240015in}{0.569136in}}%
\pgfpathlineto{\pgfqpoint{8.242702in}{0.569136in}}%
\pgfpathlineto{\pgfqpoint{8.245388in}{0.569136in}}%
\pgfpathlineto{\pgfqpoint{8.248075in}{0.569136in}}%
\pgfpathlineto{\pgfqpoint{8.250762in}{0.569136in}}%
\pgfpathlineto{\pgfqpoint{8.253449in}{0.569136in}}%
\pgfpathlineto{\pgfqpoint{8.256136in}{0.569136in}}%
\pgfpathlineto{\pgfqpoint{8.256136in}{0.569902in}}%
\pgfpathlineto{\pgfqpoint{8.256136in}{0.569902in}}%
\pgfpathlineto{\pgfqpoint{8.253449in}{0.570113in}}%
\pgfpathlineto{\pgfqpoint{8.250762in}{0.570375in}}%
\pgfpathlineto{\pgfqpoint{8.248075in}{0.570695in}}%
\pgfpathlineto{\pgfqpoint{8.245388in}{0.571085in}}%
\pgfpathlineto{\pgfqpoint{8.242702in}{0.571556in}}%
\pgfpathlineto{\pgfqpoint{8.240015in}{0.572121in}}%
\pgfpathlineto{\pgfqpoint{8.237328in}{0.572793in}}%
\pgfpathlineto{\pgfqpoint{8.234641in}{0.573585in}}%
\pgfpathlineto{\pgfqpoint{8.231954in}{0.574514in}}%
\pgfpathlineto{\pgfqpoint{8.229267in}{0.575592in}}%
\pgfpathlineto{\pgfqpoint{8.226580in}{0.576835in}}%
\pgfpathlineto{\pgfqpoint{8.223893in}{0.578255in}}%
\pgfpathlineto{\pgfqpoint{8.221207in}{0.579864in}}%
\pgfpathlineto{\pgfqpoint{8.218520in}{0.581673in}}%
\pgfpathlineto{\pgfqpoint{8.215833in}{0.583690in}}%
\pgfpathlineto{\pgfqpoint{8.213146in}{0.585918in}}%
\pgfpathlineto{\pgfqpoint{8.210459in}{0.588359in}}%
\pgfpathlineto{\pgfqpoint{8.207772in}{0.591009in}}%
\pgfpathlineto{\pgfqpoint{8.205085in}{0.593860in}}%
\pgfpathlineto{\pgfqpoint{8.202399in}{0.596899in}}%
\pgfpathlineto{\pgfqpoint{8.199712in}{0.600107in}}%
\pgfpathlineto{\pgfqpoint{8.197025in}{0.603461in}}%
\pgfpathlineto{\pgfqpoint{8.194338in}{0.606932in}}%
\pgfpathlineto{\pgfqpoint{8.191651in}{0.610487in}}%
\pgfpathlineto{\pgfqpoint{8.188964in}{0.614089in}}%
\pgfpathlineto{\pgfqpoint{8.186277in}{0.617697in}}%
\pgfpathlineto{\pgfqpoint{8.183590in}{0.621270in}}%
\pgfpathlineto{\pgfqpoint{8.180904in}{0.624764in}}%
\pgfpathlineto{\pgfqpoint{8.178217in}{0.628137in}}%
\pgfpathlineto{\pgfqpoint{8.175530in}{0.631348in}}%
\pgfpathlineto{\pgfqpoint{8.172843in}{0.634361in}}%
\pgfpathlineto{\pgfqpoint{8.170156in}{0.637142in}}%
\pgfpathlineto{\pgfqpoint{8.167469in}{0.639666in}}%
\pgfpathlineto{\pgfqpoint{8.164782in}{0.641914in}}%
\pgfpathlineto{\pgfqpoint{8.162096in}{0.643874in}}%
\pgfpathlineto{\pgfqpoint{8.159409in}{0.645544in}}%
\pgfpathlineto{\pgfqpoint{8.156722in}{0.646932in}}%
\pgfpathlineto{\pgfqpoint{8.154035in}{0.648051in}}%
\pgfpathlineto{\pgfqpoint{8.151348in}{0.648926in}}%
\pgfpathlineto{\pgfqpoint{8.148661in}{0.649588in}}%
\pgfpathlineto{\pgfqpoint{8.145974in}{0.650075in}}%
\pgfpathlineto{\pgfqpoint{8.143287in}{0.650430in}}%
\pgfpathlineto{\pgfqpoint{8.140601in}{0.650701in}}%
\pgfpathlineto{\pgfqpoint{8.137914in}{0.650939in}}%
\pgfpathlineto{\pgfqpoint{8.135227in}{0.651195in}}%
\pgfpathlineto{\pgfqpoint{8.132540in}{0.651520in}}%
\pgfpathlineto{\pgfqpoint{8.129853in}{0.651966in}}%
\pgfpathlineto{\pgfqpoint{8.127166in}{0.652582in}}%
\pgfpathlineto{\pgfqpoint{8.124479in}{0.653415in}}%
\pgfpathlineto{\pgfqpoint{8.121793in}{0.654511in}}%
\pgfpathlineto{\pgfqpoint{8.119106in}{0.655912in}}%
\pgfpathlineto{\pgfqpoint{8.116419in}{0.657662in}}%
\pgfpathlineto{\pgfqpoint{8.113732in}{0.659803in}}%
\pgfpathlineto{\pgfqpoint{8.111045in}{0.662379in}}%
\pgfpathlineto{\pgfqpoint{8.108358in}{0.665434in}}%
\pgfpathlineto{\pgfqpoint{8.105671in}{0.669020in}}%
\pgfpathlineto{\pgfqpoint{8.102984in}{0.673189in}}%
\pgfpathlineto{\pgfqpoint{8.100298in}{0.678001in}}%
\pgfpathlineto{\pgfqpoint{8.097611in}{0.683520in}}%
\pgfpathlineto{\pgfqpoint{8.094924in}{0.689816in}}%
\pgfpathlineto{\pgfqpoint{8.092237in}{0.696962in}}%
\pgfpathlineto{\pgfqpoint{8.089550in}{0.705033in}}%
\pgfpathlineto{\pgfqpoint{8.086863in}{0.714107in}}%
\pgfpathlineto{\pgfqpoint{8.084176in}{0.724253in}}%
\pgfpathlineto{\pgfqpoint{8.081490in}{0.735537in}}%
\pgfpathlineto{\pgfqpoint{8.078803in}{0.748012in}}%
\pgfpathlineto{\pgfqpoint{8.076116in}{0.761712in}}%
\pgfpathlineto{\pgfqpoint{8.073429in}{0.776654in}}%
\pgfpathlineto{\pgfqpoint{8.070742in}{0.792826in}}%
\pgfpathlineto{\pgfqpoint{8.068055in}{0.810190in}}%
\pgfpathlineto{\pgfqpoint{8.065368in}{0.828671in}}%
\pgfpathlineto{\pgfqpoint{8.062681in}{0.848164in}}%
\pgfpathlineto{\pgfqpoint{8.059995in}{0.868526in}}%
\pgfpathlineto{\pgfqpoint{8.057308in}{0.889581in}}%
\pgfpathlineto{\pgfqpoint{8.054621in}{0.911121in}}%
\pgfpathlineto{\pgfqpoint{8.051934in}{0.932910in}}%
\pgfpathlineto{\pgfqpoint{8.049247in}{0.954691in}}%
\pgfpathlineto{\pgfqpoint{8.046560in}{0.976194in}}%
\pgfpathlineto{\pgfqpoint{8.043873in}{0.997141in}}%
\pgfpathlineto{\pgfqpoint{8.041187in}{1.017261in}}%
\pgfpathlineto{\pgfqpoint{8.038500in}{1.036295in}}%
\pgfpathlineto{\pgfqpoint{8.035813in}{1.054008in}}%
\pgfpathlineto{\pgfqpoint{8.033126in}{1.070199in}}%
\pgfpathlineto{\pgfqpoint{8.030439in}{1.084712in}}%
\pgfpathlineto{\pgfqpoint{8.027752in}{1.097435in}}%
\pgfpathlineto{\pgfqpoint{8.025065in}{1.108315in}}%
\pgfpathlineto{\pgfqpoint{8.022378in}{1.117354in}}%
\pgfpathlineto{\pgfqpoint{8.019692in}{1.124615in}}%
\pgfpathlineto{\pgfqpoint{8.017005in}{1.130214in}}%
\pgfpathlineto{\pgfqpoint{8.014318in}{1.134321in}}%
\pgfpathlineto{\pgfqpoint{8.011631in}{1.137154in}}%
\pgfpathlineto{\pgfqpoint{8.008944in}{1.138963in}}%
\pgfpathlineto{\pgfqpoint{8.006257in}{1.140031in}}%
\pgfpathlineto{\pgfqpoint{8.003570in}{1.140655in}}%
\pgfpathlineto{\pgfqpoint{8.000884in}{1.141138in}}%
\pgfpathlineto{\pgfqpoint{7.998197in}{1.141782in}}%
\pgfpathlineto{\pgfqpoint{7.995510in}{1.142871in}}%
\pgfpathlineto{\pgfqpoint{7.992823in}{1.144671in}}%
\pgfpathlineto{\pgfqpoint{7.990136in}{1.147419in}}%
\pgfpathlineto{\pgfqpoint{7.987449in}{1.151324in}}%
\pgfpathlineto{\pgfqpoint{7.984762in}{1.156561in}}%
\pgfpathlineto{\pgfqpoint{7.982075in}{1.163278in}}%
\pgfpathlineto{\pgfqpoint{7.979389in}{1.171601in}}%
\pgfpathlineto{\pgfqpoint{7.976702in}{1.181634in}}%
\pgfpathlineto{\pgfqpoint{7.974015in}{1.193477in}}%
\pgfpathlineto{\pgfqpoint{7.971328in}{1.207225in}}%
\pgfpathlineto{\pgfqpoint{7.968641in}{1.222986in}}%
\pgfpathlineto{\pgfqpoint{7.965954in}{1.240885in}}%
\pgfpathlineto{\pgfqpoint{7.963267in}{1.261070in}}%
\pgfpathlineto{\pgfqpoint{7.960581in}{1.283720in}}%
\pgfpathlineto{\pgfqpoint{7.957894in}{1.309042in}}%
\pgfpathlineto{\pgfqpoint{7.955207in}{1.337274in}}%
\pgfpathlineto{\pgfqpoint{7.952520in}{1.368672in}}%
\pgfpathlineto{\pgfqpoint{7.949833in}{1.403507in}}%
\pgfpathlineto{\pgfqpoint{7.947146in}{1.442043in}}%
\pgfpathlineto{\pgfqpoint{7.944459in}{1.484524in}}%
\pgfpathlineto{\pgfqpoint{7.941772in}{1.531154in}}%
\pgfpathlineto{\pgfqpoint{7.939086in}{1.582068in}}%
\pgfpathlineto{\pgfqpoint{7.936399in}{1.637318in}}%
\pgfpathlineto{\pgfqpoint{7.933712in}{1.696846in}}%
\pgfpathlineto{\pgfqpoint{7.931025in}{1.760462in}}%
\pgfpathlineto{\pgfqpoint{7.928338in}{1.827833in}}%
\pgfpathlineto{\pgfqpoint{7.925651in}{1.898468in}}%
\pgfpathlineto{\pgfqpoint{7.922964in}{1.971715in}}%
\pgfpathlineto{\pgfqpoint{7.920277in}{2.046761in}}%
\pgfpathlineto{\pgfqpoint{7.917591in}{2.122640in}}%
\pgfpathlineto{\pgfqpoint{7.914904in}{2.198254in}}%
\pgfpathlineto{\pgfqpoint{7.912217in}{2.272393in}}%
\pgfpathlineto{\pgfqpoint{7.909530in}{2.343769in}}%
\pgfpathlineto{\pgfqpoint{7.906843in}{2.411052in}}%
\pgfpathlineto{\pgfqpoint{7.904156in}{2.472908in}}%
\pgfpathlineto{\pgfqpoint{7.901469in}{2.528046in}}%
\pgfpathlineto{\pgfqpoint{7.898783in}{2.575261in}}%
\pgfpathlineto{\pgfqpoint{7.896096in}{2.613476in}}%
\pgfpathlineto{\pgfqpoint{7.893409in}{2.641780in}}%
\pgfpathlineto{\pgfqpoint{7.890722in}{2.659464in}}%
\pgfpathlineto{\pgfqpoint{7.888035in}{2.666045in}}%
\pgfpathlineto{\pgfqpoint{7.885348in}{2.661288in}}%
\pgfpathlineto{\pgfqpoint{7.882661in}{2.645209in}}%
\pgfpathlineto{\pgfqpoint{7.879974in}{2.618081in}}%
\pgfpathlineto{\pgfqpoint{7.877288in}{2.580418in}}%
\pgfpathlineto{\pgfqpoint{7.874601in}{2.532960in}}%
\pgfpathlineto{\pgfqpoint{7.871914in}{2.476641in}}%
\pgfpathlineto{\pgfqpoint{7.869227in}{2.412559in}}%
\pgfpathlineto{\pgfqpoint{7.866540in}{2.341935in}}%
\pgfpathlineto{\pgfqpoint{7.863853in}{2.266067in}}%
\pgfpathlineto{\pgfqpoint{7.861166in}{2.186293in}}%
\pgfpathlineto{\pgfqpoint{7.858480in}{2.103942in}}%
\pgfpathlineto{\pgfqpoint{7.855793in}{2.020295in}}%
\pgfpathlineto{\pgfqpoint{7.853106in}{1.936552in}}%
\pgfpathlineto{\pgfqpoint{7.850419in}{1.853798in}}%
\pgfpathlineto{\pgfqpoint{7.847732in}{1.772980in}}%
\pgfpathlineto{\pgfqpoint{7.845045in}{1.694891in}}%
\pgfpathlineto{\pgfqpoint{7.842358in}{1.620160in}}%
\pgfpathlineto{\pgfqpoint{7.839671in}{1.549249in}}%
\pgfpathlineto{\pgfqpoint{7.836985in}{1.482462in}}%
\pgfpathlineto{\pgfqpoint{7.834298in}{1.419948in}}%
\pgfpathlineto{\pgfqpoint{7.831611in}{1.361723in}}%
\pgfpathlineto{\pgfqpoint{7.828924in}{1.307688in}}%
\pgfpathlineto{\pgfqpoint{7.826237in}{1.257646in}}%
\pgfpathlineto{\pgfqpoint{7.823550in}{1.211330in}}%
\pgfpathlineto{\pgfqpoint{7.820863in}{1.168424in}}%
\pgfpathlineto{\pgfqpoint{7.818177in}{1.128584in}}%
\pgfpathlineto{\pgfqpoint{7.815490in}{1.091458in}}%
\pgfpathlineto{\pgfqpoint{7.812803in}{1.056705in}}%
\pgfpathlineto{\pgfqpoint{7.810116in}{1.024006in}}%
\pgfpathlineto{\pgfqpoint{7.807429in}{0.993077in}}%
\pgfpathlineto{\pgfqpoint{7.804742in}{0.963673in}}%
\pgfpathlineto{\pgfqpoint{7.802055in}{0.935599in}}%
\pgfpathlineto{\pgfqpoint{7.799368in}{0.908701in}}%
\pgfpathlineto{\pgfqpoint{7.796682in}{0.882875in}}%
\pgfpathlineto{\pgfqpoint{7.793995in}{0.858052in}}%
\pgfpathlineto{\pgfqpoint{7.791308in}{0.834203in}}%
\pgfpathlineto{\pgfqpoint{7.788621in}{0.811326in}}%
\pgfpathlineto{\pgfqpoint{7.785934in}{0.789440in}}%
\pgfpathlineto{\pgfqpoint{7.783247in}{0.768580in}}%
\pgfpathlineto{\pgfqpoint{7.780560in}{0.748788in}}%
\pgfpathlineto{\pgfqpoint{7.777874in}{0.730107in}}%
\pgfpathlineto{\pgfqpoint{7.775187in}{0.712580in}}%
\pgfpathlineto{\pgfqpoint{7.772500in}{0.696237in}}%
\pgfpathlineto{\pgfqpoint{7.769813in}{0.681101in}}%
\pgfpathlineto{\pgfqpoint{7.767126in}{0.667179in}}%
\pgfpathlineto{\pgfqpoint{7.764439in}{0.654467in}}%
\pgfpathlineto{\pgfqpoint{7.761752in}{0.642945in}}%
\pgfpathlineto{\pgfqpoint{7.759065in}{0.632578in}}%
\pgfpathlineto{\pgfqpoint{7.756379in}{0.623322in}}%
\pgfpathlineto{\pgfqpoint{7.753692in}{0.615120in}}%
\pgfpathlineto{\pgfqpoint{7.751005in}{0.607908in}}%
\pgfpathlineto{\pgfqpoint{7.748318in}{0.601614in}}%
\pgfpathlineto{\pgfqpoint{7.745631in}{0.596164in}}%
\pgfpathlineto{\pgfqpoint{7.742944in}{0.591481in}}%
\pgfpathlineto{\pgfqpoint{7.740257in}{0.587488in}}%
\pgfpathlineto{\pgfqpoint{7.737571in}{0.584108in}}%
\pgfpathlineto{\pgfqpoint{7.734884in}{0.581270in}}%
\pgfpathlineto{\pgfqpoint{7.732197in}{0.578904in}}%
\pgfpathlineto{\pgfqpoint{7.729510in}{0.576947in}}%
\pgfpathlineto{\pgfqpoint{7.726823in}{0.575340in}}%
\pgfpathlineto{\pgfqpoint{7.724136in}{0.574031in}}%
\pgfpathlineto{\pgfqpoint{7.721449in}{0.572972in}}%
\pgfpathlineto{\pgfqpoint{7.721449in}{0.572972in}}%
\pgfpathclose%
\pgfusepath{stroke,fill}%
}%
\begin{pgfscope}%
\pgfsys@transformshift{0.000000in}{0.000000in}%
\pgfsys@useobject{currentmarker}{}%
\end{pgfscope}%
\end{pgfscope}%
\begin{pgfscope}%
\definecolor{textcolor}{rgb}{0.000000,0.000000,0.000000}%
\pgfsetstrokecolor{textcolor}%
\pgfsetfillcolor{textcolor}%
\pgftext[x=10.259645in,y=5.255787in,left,base]{\color{textcolor}\rmfamily\fontsize{10.000000}{12.000000}\selectfont species}%
\end{pgfscope}%
\begin{pgfscope}%
\pgfsetbuttcap%
\pgfsetroundjoin%
\definecolor{currentfill}{rgb}{0.121569,0.466667,0.705882}%
\pgfsetfillcolor{currentfill}%
\pgfsetlinewidth{1.003750pt}%
\definecolor{currentstroke}{rgb}{0.121569,0.466667,0.705882}%
\pgfsetstrokecolor{currentstroke}%
\pgfsetdash{}{0pt}%
\pgfsys@defobject{currentmarker}{\pgfqpoint{-0.041667in}{-0.041667in}}{\pgfqpoint{0.041667in}{0.041667in}}{%
\pgfpathmoveto{\pgfqpoint{0.000000in}{-0.041667in}}%
\pgfpathcurveto{\pgfqpoint{0.011050in}{-0.041667in}}{\pgfqpoint{0.021649in}{-0.037276in}}{\pgfqpoint{0.029463in}{-0.029463in}}%
\pgfpathcurveto{\pgfqpoint{0.037276in}{-0.021649in}}{\pgfqpoint{0.041667in}{-0.011050in}}{\pgfqpoint{0.041667in}{0.000000in}}%
\pgfpathcurveto{\pgfqpoint{0.041667in}{0.011050in}}{\pgfqpoint{0.037276in}{0.021649in}}{\pgfqpoint{0.029463in}{0.029463in}}%
\pgfpathcurveto{\pgfqpoint{0.021649in}{0.037276in}}{\pgfqpoint{0.011050in}{0.041667in}}{\pgfqpoint{0.000000in}{0.041667in}}%
\pgfpathcurveto{\pgfqpoint{-0.011050in}{0.041667in}}{\pgfqpoint{-0.021649in}{0.037276in}}{\pgfqpoint{-0.029463in}{0.029463in}}%
\pgfpathcurveto{\pgfqpoint{-0.037276in}{0.021649in}}{\pgfqpoint{-0.041667in}{0.011050in}}{\pgfqpoint{-0.041667in}{0.000000in}}%
\pgfpathcurveto{\pgfqpoint{-0.041667in}{-0.011050in}}{\pgfqpoint{-0.037276in}{-0.021649in}}{\pgfqpoint{-0.029463in}{-0.029463in}}%
\pgfpathcurveto{\pgfqpoint{-0.021649in}{-0.037276in}}{\pgfqpoint{-0.011050in}{-0.041667in}}{\pgfqpoint{0.000000in}{-0.041667in}}%
\pgfpathlineto{\pgfqpoint{0.000000in}{-0.041667in}}%
\pgfpathclose%
\pgfusepath{stroke,fill}%
}%
\begin{pgfscope}%
\pgfsys@transformshift{10.125000in}{5.098572in}%
\pgfsys@useobject{currentmarker}{}%
\end{pgfscope}%
\end{pgfscope}%
\begin{pgfscope}%
\definecolor{textcolor}{rgb}{0.000000,0.000000,0.000000}%
\pgfsetstrokecolor{textcolor}%
\pgfsetfillcolor{textcolor}%
\pgftext[x=10.375000in,y=5.062114in,left,base]{\color{textcolor}\rmfamily\fontsize{10.000000}{12.000000}\selectfont setosa}%
\end{pgfscope}%
\begin{pgfscope}%
\pgfsetbuttcap%
\pgfsetroundjoin%
\definecolor{currentfill}{rgb}{1.000000,0.498039,0.054902}%
\pgfsetfillcolor{currentfill}%
\pgfsetlinewidth{1.003750pt}%
\definecolor{currentstroke}{rgb}{1.000000,0.498039,0.054902}%
\pgfsetstrokecolor{currentstroke}%
\pgfsetdash{}{0pt}%
\pgfsys@defobject{currentmarker}{\pgfqpoint{-0.041667in}{-0.041667in}}{\pgfqpoint{0.041667in}{0.041667in}}{%
\pgfpathmoveto{\pgfqpoint{0.000000in}{-0.041667in}}%
\pgfpathcurveto{\pgfqpoint{0.011050in}{-0.041667in}}{\pgfqpoint{0.021649in}{-0.037276in}}{\pgfqpoint{0.029463in}{-0.029463in}}%
\pgfpathcurveto{\pgfqpoint{0.037276in}{-0.021649in}}{\pgfqpoint{0.041667in}{-0.011050in}}{\pgfqpoint{0.041667in}{0.000000in}}%
\pgfpathcurveto{\pgfqpoint{0.041667in}{0.011050in}}{\pgfqpoint{0.037276in}{0.021649in}}{\pgfqpoint{0.029463in}{0.029463in}}%
\pgfpathcurveto{\pgfqpoint{0.021649in}{0.037276in}}{\pgfqpoint{0.011050in}{0.041667in}}{\pgfqpoint{0.000000in}{0.041667in}}%
\pgfpathcurveto{\pgfqpoint{-0.011050in}{0.041667in}}{\pgfqpoint{-0.021649in}{0.037276in}}{\pgfqpoint{-0.029463in}{0.029463in}}%
\pgfpathcurveto{\pgfqpoint{-0.037276in}{0.021649in}}{\pgfqpoint{-0.041667in}{0.011050in}}{\pgfqpoint{-0.041667in}{0.000000in}}%
\pgfpathcurveto{\pgfqpoint{-0.041667in}{-0.011050in}}{\pgfqpoint{-0.037276in}{-0.021649in}}{\pgfqpoint{-0.029463in}{-0.029463in}}%
\pgfpathcurveto{\pgfqpoint{-0.021649in}{-0.037276in}}{\pgfqpoint{-0.011050in}{-0.041667in}}{\pgfqpoint{0.000000in}{-0.041667in}}%
\pgfpathlineto{\pgfqpoint{0.000000in}{-0.041667in}}%
\pgfpathclose%
\pgfusepath{stroke,fill}%
}%
\begin{pgfscope}%
\pgfsys@transformshift{10.125000in}{4.904900in}%
\pgfsys@useobject{currentmarker}{}%
\end{pgfscope}%
\end{pgfscope}%
\begin{pgfscope}%
\definecolor{textcolor}{rgb}{0.000000,0.000000,0.000000}%
\pgfsetstrokecolor{textcolor}%
\pgfsetfillcolor{textcolor}%
\pgftext[x=10.375000in,y=4.868441in,left,base]{\color{textcolor}\rmfamily\fontsize{10.000000}{12.000000}\selectfont versicolor}%
\end{pgfscope}%
\begin{pgfscope}%
\pgfsetbuttcap%
\pgfsetroundjoin%
\definecolor{currentfill}{rgb}{0.172549,0.627451,0.172549}%
\pgfsetfillcolor{currentfill}%
\pgfsetlinewidth{1.003750pt}%
\definecolor{currentstroke}{rgb}{0.172549,0.627451,0.172549}%
\pgfsetstrokecolor{currentstroke}%
\pgfsetdash{}{0pt}%
\pgfsys@defobject{currentmarker}{\pgfqpoint{-0.041667in}{-0.041667in}}{\pgfqpoint{0.041667in}{0.041667in}}{%
\pgfpathmoveto{\pgfqpoint{0.000000in}{-0.041667in}}%
\pgfpathcurveto{\pgfqpoint{0.011050in}{-0.041667in}}{\pgfqpoint{0.021649in}{-0.037276in}}{\pgfqpoint{0.029463in}{-0.029463in}}%
\pgfpathcurveto{\pgfqpoint{0.037276in}{-0.021649in}}{\pgfqpoint{0.041667in}{-0.011050in}}{\pgfqpoint{0.041667in}{0.000000in}}%
\pgfpathcurveto{\pgfqpoint{0.041667in}{0.011050in}}{\pgfqpoint{0.037276in}{0.021649in}}{\pgfqpoint{0.029463in}{0.029463in}}%
\pgfpathcurveto{\pgfqpoint{0.021649in}{0.037276in}}{\pgfqpoint{0.011050in}{0.041667in}}{\pgfqpoint{0.000000in}{0.041667in}}%
\pgfpathcurveto{\pgfqpoint{-0.011050in}{0.041667in}}{\pgfqpoint{-0.021649in}{0.037276in}}{\pgfqpoint{-0.029463in}{0.029463in}}%
\pgfpathcurveto{\pgfqpoint{-0.037276in}{0.021649in}}{\pgfqpoint{-0.041667in}{0.011050in}}{\pgfqpoint{-0.041667in}{0.000000in}}%
\pgfpathcurveto{\pgfqpoint{-0.041667in}{-0.011050in}}{\pgfqpoint{-0.037276in}{-0.021649in}}{\pgfqpoint{-0.029463in}{-0.029463in}}%
\pgfpathcurveto{\pgfqpoint{-0.021649in}{-0.037276in}}{\pgfqpoint{-0.011050in}{-0.041667in}}{\pgfqpoint{0.000000in}{-0.041667in}}%
\pgfpathlineto{\pgfqpoint{0.000000in}{-0.041667in}}%
\pgfpathclose%
\pgfusepath{stroke,fill}%
}%
\begin{pgfscope}%
\pgfsys@transformshift{10.125000in}{4.711227in}%
\pgfsys@useobject{currentmarker}{}%
\end{pgfscope}%
\end{pgfscope}%
\begin{pgfscope}%
\definecolor{textcolor}{rgb}{0.000000,0.000000,0.000000}%
\pgfsetstrokecolor{textcolor}%
\pgfsetfillcolor{textcolor}%
\pgftext[x=10.375000in,y=4.674769in,left,base]{\color{textcolor}\rmfamily\fontsize{10.000000}{12.000000}\selectfont virginica}%
\end{pgfscope}%
\end{pgfpicture}%
\makeatother%
\endgroup%
}
  \end{center}
  \caption{Pairplot of iris data.}
  \label{fig:pairplot}
\end{figure}

In addition to visualizing the data, \ac{eda} also involves calculating numerical summaries of the data. For example, correlation and covariance are measures of the relationship between variables and help to understand the strength and direction of the relationship between variables, as shown in figure \ref{fig:cor-cov}.

\begin{figure}
  \begin{center}
    % I cannot get this to work
    % \inputpgf{assets/eda}{cor-cov.pgf}
  \end{center}
  \caption{Correlation and covariance of iris data.}
  \label{fig:cor-cov}
\end{figure}

Cross-correlation is another useful measure in \ac{eda} that determines how two signals are related, particularly in time-series data. Cross-correlation helps to determine the lag between two signals and understand whether there is any association between them, as shown in figure \ref{fig:cross-cor}.

\begin{figure}
  \begin{center}
    \resizebox{0.8\textwidth}{!}{%% Creator: Matplotlib, PGF backend
%%
%% To include the figure in your LaTeX document, write
%%   \input{<filename>.pgf}
%%
%% Make sure the required packages are loaded in your preamble
%%   \usepackage{pgf}
%%
%% Also ensure that all the required font packages are loaded; for instance,
%% the lmodern package is sometimes necessary when using math font.
%%   \usepackage{lmodern}
%%
%% Figures using additional raster images can only be included by \input if
%% they are in the same directory as the main LaTeX file. For loading figures
%% from other directories you can use the `import` package
%%   \usepackage{import}
%%
%% and then include the figures with
%%   \import{<path to file>}{<filename>.pgf}
%%
%% Matplotlib used the following preamble
%%
\begingroup%
\makeatletter%
\begin{pgfpicture}%
\pgfpathrectangle{\pgfpointorigin}{\pgfqpoint{6.000000in}{3.000000in}}%
\pgfusepath{use as bounding box, clip}%
\begin{pgfscope}%
\pgfsetbuttcap%
\pgfsetmiterjoin%
\definecolor{currentfill}{rgb}{1.000000,1.000000,1.000000}%
\pgfsetfillcolor{currentfill}%
\pgfsetlinewidth{0.000000pt}%
\definecolor{currentstroke}{rgb}{1.000000,1.000000,1.000000}%
\pgfsetstrokecolor{currentstroke}%
\pgfsetdash{}{0pt}%
\pgfpathmoveto{\pgfqpoint{0.000000in}{0.000000in}}%
\pgfpathlineto{\pgfqpoint{6.000000in}{0.000000in}}%
\pgfpathlineto{\pgfqpoint{6.000000in}{3.000000in}}%
\pgfpathlineto{\pgfqpoint{0.000000in}{3.000000in}}%
\pgfpathlineto{\pgfqpoint{0.000000in}{0.000000in}}%
\pgfpathclose%
\pgfusepath{fill}%
\end{pgfscope}%
\begin{pgfscope}%
\pgfsetbuttcap%
\pgfsetmiterjoin%
\definecolor{currentfill}{rgb}{1.000000,1.000000,1.000000}%
\pgfsetfillcolor{currentfill}%
\pgfsetlinewidth{0.000000pt}%
\definecolor{currentstroke}{rgb}{0.000000,0.000000,0.000000}%
\pgfsetstrokecolor{currentstroke}%
\pgfsetstrokeopacity{0.000000}%
\pgfsetdash{}{0pt}%
\pgfpathmoveto{\pgfqpoint{0.750000in}{0.330000in}}%
\pgfpathlineto{\pgfqpoint{2.863636in}{0.330000in}}%
\pgfpathlineto{\pgfqpoint{2.863636in}{2.640000in}}%
\pgfpathlineto{\pgfqpoint{0.750000in}{2.640000in}}%
\pgfpathlineto{\pgfqpoint{0.750000in}{0.330000in}}%
\pgfpathclose%
\pgfusepath{fill}%
\end{pgfscope}%
\begin{pgfscope}%
\pgfpathrectangle{\pgfqpoint{0.750000in}{0.330000in}}{\pgfqpoint{2.113636in}{2.310000in}}%
\pgfusepath{clip}%
\pgfsetrectcap%
\pgfsetroundjoin%
\pgfsetlinewidth{0.803000pt}%
\definecolor{currentstroke}{rgb}{0.690196,0.690196,0.690196}%
\pgfsetstrokecolor{currentstroke}%
\pgfsetdash{}{0pt}%
\pgfpathmoveto{\pgfqpoint{0.846074in}{0.330000in}}%
\pgfpathlineto{\pgfqpoint{0.846074in}{2.640000in}}%
\pgfusepath{stroke}%
\end{pgfscope}%
\begin{pgfscope}%
\pgfsetbuttcap%
\pgfsetroundjoin%
\definecolor{currentfill}{rgb}{0.000000,0.000000,0.000000}%
\pgfsetfillcolor{currentfill}%
\pgfsetlinewidth{0.803000pt}%
\definecolor{currentstroke}{rgb}{0.000000,0.000000,0.000000}%
\pgfsetstrokecolor{currentstroke}%
\pgfsetdash{}{0pt}%
\pgfsys@defobject{currentmarker}{\pgfqpoint{0.000000in}{-0.048611in}}{\pgfqpoint{0.000000in}{0.000000in}}{%
\pgfpathmoveto{\pgfqpoint{0.000000in}{0.000000in}}%
\pgfpathlineto{\pgfqpoint{0.000000in}{-0.048611in}}%
\pgfusepath{stroke,fill}%
}%
\begin{pgfscope}%
\pgfsys@transformshift{0.846074in}{0.330000in}%
\pgfsys@useobject{currentmarker}{}%
\end{pgfscope}%
\end{pgfscope}%
\begin{pgfscope}%
\definecolor{textcolor}{rgb}{0.000000,0.000000,0.000000}%
\pgfsetstrokecolor{textcolor}%
\pgfsetfillcolor{textcolor}%
\pgftext[x=0.846074in,y=0.232778in,,top]{\color{textcolor}\rmfamily\fontsize{10.000000}{12.000000}\selectfont \(\displaystyle {\ensuremath{-}50}\)}%
\end{pgfscope}%
\begin{pgfscope}%
\pgfpathrectangle{\pgfqpoint{0.750000in}{0.330000in}}{\pgfqpoint{2.113636in}{2.310000in}}%
\pgfusepath{clip}%
\pgfsetrectcap%
\pgfsetroundjoin%
\pgfsetlinewidth{0.803000pt}%
\definecolor{currentstroke}{rgb}{0.690196,0.690196,0.690196}%
\pgfsetstrokecolor{currentstroke}%
\pgfsetdash{}{0pt}%
\pgfpathmoveto{\pgfqpoint{1.326446in}{0.330000in}}%
\pgfpathlineto{\pgfqpoint{1.326446in}{2.640000in}}%
\pgfusepath{stroke}%
\end{pgfscope}%
\begin{pgfscope}%
\pgfsetbuttcap%
\pgfsetroundjoin%
\definecolor{currentfill}{rgb}{0.000000,0.000000,0.000000}%
\pgfsetfillcolor{currentfill}%
\pgfsetlinewidth{0.803000pt}%
\definecolor{currentstroke}{rgb}{0.000000,0.000000,0.000000}%
\pgfsetstrokecolor{currentstroke}%
\pgfsetdash{}{0pt}%
\pgfsys@defobject{currentmarker}{\pgfqpoint{0.000000in}{-0.048611in}}{\pgfqpoint{0.000000in}{0.000000in}}{%
\pgfpathmoveto{\pgfqpoint{0.000000in}{0.000000in}}%
\pgfpathlineto{\pgfqpoint{0.000000in}{-0.048611in}}%
\pgfusepath{stroke,fill}%
}%
\begin{pgfscope}%
\pgfsys@transformshift{1.326446in}{0.330000in}%
\pgfsys@useobject{currentmarker}{}%
\end{pgfscope}%
\end{pgfscope}%
\begin{pgfscope}%
\definecolor{textcolor}{rgb}{0.000000,0.000000,0.000000}%
\pgfsetstrokecolor{textcolor}%
\pgfsetfillcolor{textcolor}%
\pgftext[x=1.326446in,y=0.232778in,,top]{\color{textcolor}\rmfamily\fontsize{10.000000}{12.000000}\selectfont \(\displaystyle {\ensuremath{-}25}\)}%
\end{pgfscope}%
\begin{pgfscope}%
\pgfpathrectangle{\pgfqpoint{0.750000in}{0.330000in}}{\pgfqpoint{2.113636in}{2.310000in}}%
\pgfusepath{clip}%
\pgfsetrectcap%
\pgfsetroundjoin%
\pgfsetlinewidth{0.803000pt}%
\definecolor{currentstroke}{rgb}{0.690196,0.690196,0.690196}%
\pgfsetstrokecolor{currentstroke}%
\pgfsetdash{}{0pt}%
\pgfpathmoveto{\pgfqpoint{1.806818in}{0.330000in}}%
\pgfpathlineto{\pgfqpoint{1.806818in}{2.640000in}}%
\pgfusepath{stroke}%
\end{pgfscope}%
\begin{pgfscope}%
\pgfsetbuttcap%
\pgfsetroundjoin%
\definecolor{currentfill}{rgb}{0.000000,0.000000,0.000000}%
\pgfsetfillcolor{currentfill}%
\pgfsetlinewidth{0.803000pt}%
\definecolor{currentstroke}{rgb}{0.000000,0.000000,0.000000}%
\pgfsetstrokecolor{currentstroke}%
\pgfsetdash{}{0pt}%
\pgfsys@defobject{currentmarker}{\pgfqpoint{0.000000in}{-0.048611in}}{\pgfqpoint{0.000000in}{0.000000in}}{%
\pgfpathmoveto{\pgfqpoint{0.000000in}{0.000000in}}%
\pgfpathlineto{\pgfqpoint{0.000000in}{-0.048611in}}%
\pgfusepath{stroke,fill}%
}%
\begin{pgfscope}%
\pgfsys@transformshift{1.806818in}{0.330000in}%
\pgfsys@useobject{currentmarker}{}%
\end{pgfscope}%
\end{pgfscope}%
\begin{pgfscope}%
\definecolor{textcolor}{rgb}{0.000000,0.000000,0.000000}%
\pgfsetstrokecolor{textcolor}%
\pgfsetfillcolor{textcolor}%
\pgftext[x=1.806818in,y=0.232778in,,top]{\color{textcolor}\rmfamily\fontsize{10.000000}{12.000000}\selectfont \(\displaystyle {0}\)}%
\end{pgfscope}%
\begin{pgfscope}%
\pgfpathrectangle{\pgfqpoint{0.750000in}{0.330000in}}{\pgfqpoint{2.113636in}{2.310000in}}%
\pgfusepath{clip}%
\pgfsetrectcap%
\pgfsetroundjoin%
\pgfsetlinewidth{0.803000pt}%
\definecolor{currentstroke}{rgb}{0.690196,0.690196,0.690196}%
\pgfsetstrokecolor{currentstroke}%
\pgfsetdash{}{0pt}%
\pgfpathmoveto{\pgfqpoint{2.287190in}{0.330000in}}%
\pgfpathlineto{\pgfqpoint{2.287190in}{2.640000in}}%
\pgfusepath{stroke}%
\end{pgfscope}%
\begin{pgfscope}%
\pgfsetbuttcap%
\pgfsetroundjoin%
\definecolor{currentfill}{rgb}{0.000000,0.000000,0.000000}%
\pgfsetfillcolor{currentfill}%
\pgfsetlinewidth{0.803000pt}%
\definecolor{currentstroke}{rgb}{0.000000,0.000000,0.000000}%
\pgfsetstrokecolor{currentstroke}%
\pgfsetdash{}{0pt}%
\pgfsys@defobject{currentmarker}{\pgfqpoint{0.000000in}{-0.048611in}}{\pgfqpoint{0.000000in}{0.000000in}}{%
\pgfpathmoveto{\pgfqpoint{0.000000in}{0.000000in}}%
\pgfpathlineto{\pgfqpoint{0.000000in}{-0.048611in}}%
\pgfusepath{stroke,fill}%
}%
\begin{pgfscope}%
\pgfsys@transformshift{2.287190in}{0.330000in}%
\pgfsys@useobject{currentmarker}{}%
\end{pgfscope}%
\end{pgfscope}%
\begin{pgfscope}%
\definecolor{textcolor}{rgb}{0.000000,0.000000,0.000000}%
\pgfsetstrokecolor{textcolor}%
\pgfsetfillcolor{textcolor}%
\pgftext[x=2.287190in,y=0.232778in,,top]{\color{textcolor}\rmfamily\fontsize{10.000000}{12.000000}\selectfont \(\displaystyle {25}\)}%
\end{pgfscope}%
\begin{pgfscope}%
\pgfpathrectangle{\pgfqpoint{0.750000in}{0.330000in}}{\pgfqpoint{2.113636in}{2.310000in}}%
\pgfusepath{clip}%
\pgfsetrectcap%
\pgfsetroundjoin%
\pgfsetlinewidth{0.803000pt}%
\definecolor{currentstroke}{rgb}{0.690196,0.690196,0.690196}%
\pgfsetstrokecolor{currentstroke}%
\pgfsetdash{}{0pt}%
\pgfpathmoveto{\pgfqpoint{2.767562in}{0.330000in}}%
\pgfpathlineto{\pgfqpoint{2.767562in}{2.640000in}}%
\pgfusepath{stroke}%
\end{pgfscope}%
\begin{pgfscope}%
\pgfsetbuttcap%
\pgfsetroundjoin%
\definecolor{currentfill}{rgb}{0.000000,0.000000,0.000000}%
\pgfsetfillcolor{currentfill}%
\pgfsetlinewidth{0.803000pt}%
\definecolor{currentstroke}{rgb}{0.000000,0.000000,0.000000}%
\pgfsetstrokecolor{currentstroke}%
\pgfsetdash{}{0pt}%
\pgfsys@defobject{currentmarker}{\pgfqpoint{0.000000in}{-0.048611in}}{\pgfqpoint{0.000000in}{0.000000in}}{%
\pgfpathmoveto{\pgfqpoint{0.000000in}{0.000000in}}%
\pgfpathlineto{\pgfqpoint{0.000000in}{-0.048611in}}%
\pgfusepath{stroke,fill}%
}%
\begin{pgfscope}%
\pgfsys@transformshift{2.767562in}{0.330000in}%
\pgfsys@useobject{currentmarker}{}%
\end{pgfscope}%
\end{pgfscope}%
\begin{pgfscope}%
\definecolor{textcolor}{rgb}{0.000000,0.000000,0.000000}%
\pgfsetstrokecolor{textcolor}%
\pgfsetfillcolor{textcolor}%
\pgftext[x=2.767562in,y=0.232778in,,top]{\color{textcolor}\rmfamily\fontsize{10.000000}{12.000000}\selectfont \(\displaystyle {50}\)}%
\end{pgfscope}%
\begin{pgfscope}%
\pgfpathrectangle{\pgfqpoint{0.750000in}{0.330000in}}{\pgfqpoint{2.113636in}{2.310000in}}%
\pgfusepath{clip}%
\pgfsetrectcap%
\pgfsetroundjoin%
\pgfsetlinewidth{0.803000pt}%
\definecolor{currentstroke}{rgb}{0.690196,0.690196,0.690196}%
\pgfsetstrokecolor{currentstroke}%
\pgfsetdash{}{0pt}%
\pgfpathmoveto{\pgfqpoint{0.750000in}{0.435000in}}%
\pgfpathlineto{\pgfqpoint{2.863636in}{0.435000in}}%
\pgfusepath{stroke}%
\end{pgfscope}%
\begin{pgfscope}%
\pgfsetbuttcap%
\pgfsetroundjoin%
\definecolor{currentfill}{rgb}{0.000000,0.000000,0.000000}%
\pgfsetfillcolor{currentfill}%
\pgfsetlinewidth{0.803000pt}%
\definecolor{currentstroke}{rgb}{0.000000,0.000000,0.000000}%
\pgfsetstrokecolor{currentstroke}%
\pgfsetdash{}{0pt}%
\pgfsys@defobject{currentmarker}{\pgfqpoint{-0.048611in}{0.000000in}}{\pgfqpoint{-0.000000in}{0.000000in}}{%
\pgfpathmoveto{\pgfqpoint{-0.000000in}{0.000000in}}%
\pgfpathlineto{\pgfqpoint{-0.048611in}{0.000000in}}%
\pgfusepath{stroke,fill}%
}%
\begin{pgfscope}%
\pgfsys@transformshift{0.750000in}{0.435000in}%
\pgfsys@useobject{currentmarker}{}%
\end{pgfscope}%
\end{pgfscope}%
\begin{pgfscope}%
\definecolor{textcolor}{rgb}{0.000000,0.000000,0.000000}%
\pgfsetstrokecolor{textcolor}%
\pgfsetfillcolor{textcolor}%
\pgftext[x=0.475308in, y=0.386775in, left, base]{\color{textcolor}\rmfamily\fontsize{10.000000}{12.000000}\selectfont \(\displaystyle {0.0}\)}%
\end{pgfscope}%
\begin{pgfscope}%
\pgfpathrectangle{\pgfqpoint{0.750000in}{0.330000in}}{\pgfqpoint{2.113636in}{2.310000in}}%
\pgfusepath{clip}%
\pgfsetrectcap%
\pgfsetroundjoin%
\pgfsetlinewidth{0.803000pt}%
\definecolor{currentstroke}{rgb}{0.690196,0.690196,0.690196}%
\pgfsetstrokecolor{currentstroke}%
\pgfsetdash{}{0pt}%
\pgfpathmoveto{\pgfqpoint{0.750000in}{0.864442in}}%
\pgfpathlineto{\pgfqpoint{2.863636in}{0.864442in}}%
\pgfusepath{stroke}%
\end{pgfscope}%
\begin{pgfscope}%
\pgfsetbuttcap%
\pgfsetroundjoin%
\definecolor{currentfill}{rgb}{0.000000,0.000000,0.000000}%
\pgfsetfillcolor{currentfill}%
\pgfsetlinewidth{0.803000pt}%
\definecolor{currentstroke}{rgb}{0.000000,0.000000,0.000000}%
\pgfsetstrokecolor{currentstroke}%
\pgfsetdash{}{0pt}%
\pgfsys@defobject{currentmarker}{\pgfqpoint{-0.048611in}{0.000000in}}{\pgfqpoint{-0.000000in}{0.000000in}}{%
\pgfpathmoveto{\pgfqpoint{-0.000000in}{0.000000in}}%
\pgfpathlineto{\pgfqpoint{-0.048611in}{0.000000in}}%
\pgfusepath{stroke,fill}%
}%
\begin{pgfscope}%
\pgfsys@transformshift{0.750000in}{0.864442in}%
\pgfsys@useobject{currentmarker}{}%
\end{pgfscope}%
\end{pgfscope}%
\begin{pgfscope}%
\definecolor{textcolor}{rgb}{0.000000,0.000000,0.000000}%
\pgfsetstrokecolor{textcolor}%
\pgfsetfillcolor{textcolor}%
\pgftext[x=0.475308in, y=0.816217in, left, base]{\color{textcolor}\rmfamily\fontsize{10.000000}{12.000000}\selectfont \(\displaystyle {0.2}\)}%
\end{pgfscope}%
\begin{pgfscope}%
\pgfpathrectangle{\pgfqpoint{0.750000in}{0.330000in}}{\pgfqpoint{2.113636in}{2.310000in}}%
\pgfusepath{clip}%
\pgfsetrectcap%
\pgfsetroundjoin%
\pgfsetlinewidth{0.803000pt}%
\definecolor{currentstroke}{rgb}{0.690196,0.690196,0.690196}%
\pgfsetstrokecolor{currentstroke}%
\pgfsetdash{}{0pt}%
\pgfpathmoveto{\pgfqpoint{0.750000in}{1.293884in}}%
\pgfpathlineto{\pgfqpoint{2.863636in}{1.293884in}}%
\pgfusepath{stroke}%
\end{pgfscope}%
\begin{pgfscope}%
\pgfsetbuttcap%
\pgfsetroundjoin%
\definecolor{currentfill}{rgb}{0.000000,0.000000,0.000000}%
\pgfsetfillcolor{currentfill}%
\pgfsetlinewidth{0.803000pt}%
\definecolor{currentstroke}{rgb}{0.000000,0.000000,0.000000}%
\pgfsetstrokecolor{currentstroke}%
\pgfsetdash{}{0pt}%
\pgfsys@defobject{currentmarker}{\pgfqpoint{-0.048611in}{0.000000in}}{\pgfqpoint{-0.000000in}{0.000000in}}{%
\pgfpathmoveto{\pgfqpoint{-0.000000in}{0.000000in}}%
\pgfpathlineto{\pgfqpoint{-0.048611in}{0.000000in}}%
\pgfusepath{stroke,fill}%
}%
\begin{pgfscope}%
\pgfsys@transformshift{0.750000in}{1.293884in}%
\pgfsys@useobject{currentmarker}{}%
\end{pgfscope}%
\end{pgfscope}%
\begin{pgfscope}%
\definecolor{textcolor}{rgb}{0.000000,0.000000,0.000000}%
\pgfsetstrokecolor{textcolor}%
\pgfsetfillcolor{textcolor}%
\pgftext[x=0.475308in, y=1.245659in, left, base]{\color{textcolor}\rmfamily\fontsize{10.000000}{12.000000}\selectfont \(\displaystyle {0.4}\)}%
\end{pgfscope}%
\begin{pgfscope}%
\pgfpathrectangle{\pgfqpoint{0.750000in}{0.330000in}}{\pgfqpoint{2.113636in}{2.310000in}}%
\pgfusepath{clip}%
\pgfsetrectcap%
\pgfsetroundjoin%
\pgfsetlinewidth{0.803000pt}%
\definecolor{currentstroke}{rgb}{0.690196,0.690196,0.690196}%
\pgfsetstrokecolor{currentstroke}%
\pgfsetdash{}{0pt}%
\pgfpathmoveto{\pgfqpoint{0.750000in}{1.723326in}}%
\pgfpathlineto{\pgfqpoint{2.863636in}{1.723326in}}%
\pgfusepath{stroke}%
\end{pgfscope}%
\begin{pgfscope}%
\pgfsetbuttcap%
\pgfsetroundjoin%
\definecolor{currentfill}{rgb}{0.000000,0.000000,0.000000}%
\pgfsetfillcolor{currentfill}%
\pgfsetlinewidth{0.803000pt}%
\definecolor{currentstroke}{rgb}{0.000000,0.000000,0.000000}%
\pgfsetstrokecolor{currentstroke}%
\pgfsetdash{}{0pt}%
\pgfsys@defobject{currentmarker}{\pgfqpoint{-0.048611in}{0.000000in}}{\pgfqpoint{-0.000000in}{0.000000in}}{%
\pgfpathmoveto{\pgfqpoint{-0.000000in}{0.000000in}}%
\pgfpathlineto{\pgfqpoint{-0.048611in}{0.000000in}}%
\pgfusepath{stroke,fill}%
}%
\begin{pgfscope}%
\pgfsys@transformshift{0.750000in}{1.723326in}%
\pgfsys@useobject{currentmarker}{}%
\end{pgfscope}%
\end{pgfscope}%
\begin{pgfscope}%
\definecolor{textcolor}{rgb}{0.000000,0.000000,0.000000}%
\pgfsetstrokecolor{textcolor}%
\pgfsetfillcolor{textcolor}%
\pgftext[x=0.475308in, y=1.675101in, left, base]{\color{textcolor}\rmfamily\fontsize{10.000000}{12.000000}\selectfont \(\displaystyle {0.6}\)}%
\end{pgfscope}%
\begin{pgfscope}%
\pgfpathrectangle{\pgfqpoint{0.750000in}{0.330000in}}{\pgfqpoint{2.113636in}{2.310000in}}%
\pgfusepath{clip}%
\pgfsetrectcap%
\pgfsetroundjoin%
\pgfsetlinewidth{0.803000pt}%
\definecolor{currentstroke}{rgb}{0.690196,0.690196,0.690196}%
\pgfsetstrokecolor{currentstroke}%
\pgfsetdash{}{0pt}%
\pgfpathmoveto{\pgfqpoint{0.750000in}{2.152768in}}%
\pgfpathlineto{\pgfqpoint{2.863636in}{2.152768in}}%
\pgfusepath{stroke}%
\end{pgfscope}%
\begin{pgfscope}%
\pgfsetbuttcap%
\pgfsetroundjoin%
\definecolor{currentfill}{rgb}{0.000000,0.000000,0.000000}%
\pgfsetfillcolor{currentfill}%
\pgfsetlinewidth{0.803000pt}%
\definecolor{currentstroke}{rgb}{0.000000,0.000000,0.000000}%
\pgfsetstrokecolor{currentstroke}%
\pgfsetdash{}{0pt}%
\pgfsys@defobject{currentmarker}{\pgfqpoint{-0.048611in}{0.000000in}}{\pgfqpoint{-0.000000in}{0.000000in}}{%
\pgfpathmoveto{\pgfqpoint{-0.000000in}{0.000000in}}%
\pgfpathlineto{\pgfqpoint{-0.048611in}{0.000000in}}%
\pgfusepath{stroke,fill}%
}%
\begin{pgfscope}%
\pgfsys@transformshift{0.750000in}{2.152768in}%
\pgfsys@useobject{currentmarker}{}%
\end{pgfscope}%
\end{pgfscope}%
\begin{pgfscope}%
\definecolor{textcolor}{rgb}{0.000000,0.000000,0.000000}%
\pgfsetstrokecolor{textcolor}%
\pgfsetfillcolor{textcolor}%
\pgftext[x=0.475308in, y=2.104543in, left, base]{\color{textcolor}\rmfamily\fontsize{10.000000}{12.000000}\selectfont \(\displaystyle {0.8}\)}%
\end{pgfscope}%
\begin{pgfscope}%
\pgfpathrectangle{\pgfqpoint{0.750000in}{0.330000in}}{\pgfqpoint{2.113636in}{2.310000in}}%
\pgfusepath{clip}%
\pgfsetrectcap%
\pgfsetroundjoin%
\pgfsetlinewidth{0.803000pt}%
\definecolor{currentstroke}{rgb}{0.690196,0.690196,0.690196}%
\pgfsetstrokecolor{currentstroke}%
\pgfsetdash{}{0pt}%
\pgfpathmoveto{\pgfqpoint{0.750000in}{2.582210in}}%
\pgfpathlineto{\pgfqpoint{2.863636in}{2.582210in}}%
\pgfusepath{stroke}%
\end{pgfscope}%
\begin{pgfscope}%
\pgfsetbuttcap%
\pgfsetroundjoin%
\definecolor{currentfill}{rgb}{0.000000,0.000000,0.000000}%
\pgfsetfillcolor{currentfill}%
\pgfsetlinewidth{0.803000pt}%
\definecolor{currentstroke}{rgb}{0.000000,0.000000,0.000000}%
\pgfsetstrokecolor{currentstroke}%
\pgfsetdash{}{0pt}%
\pgfsys@defobject{currentmarker}{\pgfqpoint{-0.048611in}{0.000000in}}{\pgfqpoint{-0.000000in}{0.000000in}}{%
\pgfpathmoveto{\pgfqpoint{-0.000000in}{0.000000in}}%
\pgfpathlineto{\pgfqpoint{-0.048611in}{0.000000in}}%
\pgfusepath{stroke,fill}%
}%
\begin{pgfscope}%
\pgfsys@transformshift{0.750000in}{2.582210in}%
\pgfsys@useobject{currentmarker}{}%
\end{pgfscope}%
\end{pgfscope}%
\begin{pgfscope}%
\definecolor{textcolor}{rgb}{0.000000,0.000000,0.000000}%
\pgfsetstrokecolor{textcolor}%
\pgfsetfillcolor{textcolor}%
\pgftext[x=0.475308in, y=2.533985in, left, base]{\color{textcolor}\rmfamily\fontsize{10.000000}{12.000000}\selectfont \(\displaystyle {1.0}\)}%
\end{pgfscope}%
\begin{pgfscope}%
\pgfpathrectangle{\pgfqpoint{0.750000in}{0.330000in}}{\pgfqpoint{2.113636in}{2.310000in}}%
\pgfusepath{clip}%
\pgfsetbuttcap%
\pgfsetroundjoin%
\pgfsetlinewidth{2.007500pt}%
\definecolor{currentstroke}{rgb}{0.121569,0.466667,0.705882}%
\pgfsetstrokecolor{currentstroke}%
\pgfsetdash{}{0pt}%
\pgfpathmoveto{\pgfqpoint{0.846074in}{0.435000in}}%
\pgfpathlineto{\pgfqpoint{0.846074in}{1.671923in}}%
\pgfusepath{stroke}%
\end{pgfscope}%
\begin{pgfscope}%
\pgfpathrectangle{\pgfqpoint{0.750000in}{0.330000in}}{\pgfqpoint{2.113636in}{2.310000in}}%
\pgfusepath{clip}%
\pgfsetbuttcap%
\pgfsetroundjoin%
\pgfsetlinewidth{2.007500pt}%
\definecolor{currentstroke}{rgb}{0.121569,0.466667,0.705882}%
\pgfsetstrokecolor{currentstroke}%
\pgfsetdash{}{0pt}%
\pgfpathmoveto{\pgfqpoint{0.865289in}{0.435000in}}%
\pgfpathlineto{\pgfqpoint{0.865289in}{1.689715in}}%
\pgfusepath{stroke}%
\end{pgfscope}%
\begin{pgfscope}%
\pgfpathrectangle{\pgfqpoint{0.750000in}{0.330000in}}{\pgfqpoint{2.113636in}{2.310000in}}%
\pgfusepath{clip}%
\pgfsetbuttcap%
\pgfsetroundjoin%
\pgfsetlinewidth{2.007500pt}%
\definecolor{currentstroke}{rgb}{0.121569,0.466667,0.705882}%
\pgfsetstrokecolor{currentstroke}%
\pgfsetdash{}{0pt}%
\pgfpathmoveto{\pgfqpoint{0.884504in}{0.435000in}}%
\pgfpathlineto{\pgfqpoint{0.884504in}{1.703438in}}%
\pgfusepath{stroke}%
\end{pgfscope}%
\begin{pgfscope}%
\pgfpathrectangle{\pgfqpoint{0.750000in}{0.330000in}}{\pgfqpoint{2.113636in}{2.310000in}}%
\pgfusepath{clip}%
\pgfsetbuttcap%
\pgfsetroundjoin%
\pgfsetlinewidth{2.007500pt}%
\definecolor{currentstroke}{rgb}{0.121569,0.466667,0.705882}%
\pgfsetstrokecolor{currentstroke}%
\pgfsetdash{}{0pt}%
\pgfpathmoveto{\pgfqpoint{0.903719in}{0.435000in}}%
\pgfpathlineto{\pgfqpoint{0.903719in}{1.721300in}}%
\pgfusepath{stroke}%
\end{pgfscope}%
\begin{pgfscope}%
\pgfpathrectangle{\pgfqpoint{0.750000in}{0.330000in}}{\pgfqpoint{2.113636in}{2.310000in}}%
\pgfusepath{clip}%
\pgfsetbuttcap%
\pgfsetroundjoin%
\pgfsetlinewidth{2.007500pt}%
\definecolor{currentstroke}{rgb}{0.121569,0.466667,0.705882}%
\pgfsetstrokecolor{currentstroke}%
\pgfsetdash{}{0pt}%
\pgfpathmoveto{\pgfqpoint{0.922934in}{0.435000in}}%
\pgfpathlineto{\pgfqpoint{0.922934in}{1.739516in}}%
\pgfusepath{stroke}%
\end{pgfscope}%
\begin{pgfscope}%
\pgfpathrectangle{\pgfqpoint{0.750000in}{0.330000in}}{\pgfqpoint{2.113636in}{2.310000in}}%
\pgfusepath{clip}%
\pgfsetbuttcap%
\pgfsetroundjoin%
\pgfsetlinewidth{2.007500pt}%
\definecolor{currentstroke}{rgb}{0.121569,0.466667,0.705882}%
\pgfsetstrokecolor{currentstroke}%
\pgfsetdash{}{0pt}%
\pgfpathmoveto{\pgfqpoint{0.942149in}{0.435000in}}%
\pgfpathlineto{\pgfqpoint{0.942149in}{1.754865in}}%
\pgfusepath{stroke}%
\end{pgfscope}%
\begin{pgfscope}%
\pgfpathrectangle{\pgfqpoint{0.750000in}{0.330000in}}{\pgfqpoint{2.113636in}{2.310000in}}%
\pgfusepath{clip}%
\pgfsetbuttcap%
\pgfsetroundjoin%
\pgfsetlinewidth{2.007500pt}%
\definecolor{currentstroke}{rgb}{0.121569,0.466667,0.705882}%
\pgfsetstrokecolor{currentstroke}%
\pgfsetdash{}{0pt}%
\pgfpathmoveto{\pgfqpoint{0.961364in}{0.435000in}}%
\pgfpathlineto{\pgfqpoint{0.961364in}{1.775500in}}%
\pgfusepath{stroke}%
\end{pgfscope}%
\begin{pgfscope}%
\pgfpathrectangle{\pgfqpoint{0.750000in}{0.330000in}}{\pgfqpoint{2.113636in}{2.310000in}}%
\pgfusepath{clip}%
\pgfsetbuttcap%
\pgfsetroundjoin%
\pgfsetlinewidth{2.007500pt}%
\definecolor{currentstroke}{rgb}{0.121569,0.466667,0.705882}%
\pgfsetstrokecolor{currentstroke}%
\pgfsetdash{}{0pt}%
\pgfpathmoveto{\pgfqpoint{0.980579in}{0.435000in}}%
\pgfpathlineto{\pgfqpoint{0.980579in}{1.786230in}}%
\pgfusepath{stroke}%
\end{pgfscope}%
\begin{pgfscope}%
\pgfpathrectangle{\pgfqpoint{0.750000in}{0.330000in}}{\pgfqpoint{2.113636in}{2.310000in}}%
\pgfusepath{clip}%
\pgfsetbuttcap%
\pgfsetroundjoin%
\pgfsetlinewidth{2.007500pt}%
\definecolor{currentstroke}{rgb}{0.121569,0.466667,0.705882}%
\pgfsetstrokecolor{currentstroke}%
\pgfsetdash{}{0pt}%
\pgfpathmoveto{\pgfqpoint{0.999793in}{0.435000in}}%
\pgfpathlineto{\pgfqpoint{0.999793in}{1.805326in}}%
\pgfusepath{stroke}%
\end{pgfscope}%
\begin{pgfscope}%
\pgfpathrectangle{\pgfqpoint{0.750000in}{0.330000in}}{\pgfqpoint{2.113636in}{2.310000in}}%
\pgfusepath{clip}%
\pgfsetbuttcap%
\pgfsetroundjoin%
\pgfsetlinewidth{2.007500pt}%
\definecolor{currentstroke}{rgb}{0.121569,0.466667,0.705882}%
\pgfsetstrokecolor{currentstroke}%
\pgfsetdash{}{0pt}%
\pgfpathmoveto{\pgfqpoint{1.019008in}{0.435000in}}%
\pgfpathlineto{\pgfqpoint{1.019008in}{1.819190in}}%
\pgfusepath{stroke}%
\end{pgfscope}%
\begin{pgfscope}%
\pgfpathrectangle{\pgfqpoint{0.750000in}{0.330000in}}{\pgfqpoint{2.113636in}{2.310000in}}%
\pgfusepath{clip}%
\pgfsetbuttcap%
\pgfsetroundjoin%
\pgfsetlinewidth{2.007500pt}%
\definecolor{currentstroke}{rgb}{0.121569,0.466667,0.705882}%
\pgfsetstrokecolor{currentstroke}%
\pgfsetdash{}{0pt}%
\pgfpathmoveto{\pgfqpoint{1.038223in}{0.435000in}}%
\pgfpathlineto{\pgfqpoint{1.038223in}{1.838710in}}%
\pgfusepath{stroke}%
\end{pgfscope}%
\begin{pgfscope}%
\pgfpathrectangle{\pgfqpoint{0.750000in}{0.330000in}}{\pgfqpoint{2.113636in}{2.310000in}}%
\pgfusepath{clip}%
\pgfsetbuttcap%
\pgfsetroundjoin%
\pgfsetlinewidth{2.007500pt}%
\definecolor{currentstroke}{rgb}{0.121569,0.466667,0.705882}%
\pgfsetstrokecolor{currentstroke}%
\pgfsetdash{}{0pt}%
\pgfpathmoveto{\pgfqpoint{1.057438in}{0.435000in}}%
\pgfpathlineto{\pgfqpoint{1.057438in}{1.855654in}}%
\pgfusepath{stroke}%
\end{pgfscope}%
\begin{pgfscope}%
\pgfpathrectangle{\pgfqpoint{0.750000in}{0.330000in}}{\pgfqpoint{2.113636in}{2.310000in}}%
\pgfusepath{clip}%
\pgfsetbuttcap%
\pgfsetroundjoin%
\pgfsetlinewidth{2.007500pt}%
\definecolor{currentstroke}{rgb}{0.121569,0.466667,0.705882}%
\pgfsetstrokecolor{currentstroke}%
\pgfsetdash{}{0pt}%
\pgfpathmoveto{\pgfqpoint{1.076653in}{0.435000in}}%
\pgfpathlineto{\pgfqpoint{1.076653in}{1.869887in}}%
\pgfusepath{stroke}%
\end{pgfscope}%
\begin{pgfscope}%
\pgfpathrectangle{\pgfqpoint{0.750000in}{0.330000in}}{\pgfqpoint{2.113636in}{2.310000in}}%
\pgfusepath{clip}%
\pgfsetbuttcap%
\pgfsetroundjoin%
\pgfsetlinewidth{2.007500pt}%
\definecolor{currentstroke}{rgb}{0.121569,0.466667,0.705882}%
\pgfsetstrokecolor{currentstroke}%
\pgfsetdash{}{0pt}%
\pgfpathmoveto{\pgfqpoint{1.095868in}{0.435000in}}%
\pgfpathlineto{\pgfqpoint{1.095868in}{1.886100in}}%
\pgfusepath{stroke}%
\end{pgfscope}%
\begin{pgfscope}%
\pgfpathrectangle{\pgfqpoint{0.750000in}{0.330000in}}{\pgfqpoint{2.113636in}{2.310000in}}%
\pgfusepath{clip}%
\pgfsetbuttcap%
\pgfsetroundjoin%
\pgfsetlinewidth{2.007500pt}%
\definecolor{currentstroke}{rgb}{0.121569,0.466667,0.705882}%
\pgfsetstrokecolor{currentstroke}%
\pgfsetdash{}{0pt}%
\pgfpathmoveto{\pgfqpoint{1.115083in}{0.435000in}}%
\pgfpathlineto{\pgfqpoint{1.115083in}{1.902690in}}%
\pgfusepath{stroke}%
\end{pgfscope}%
\begin{pgfscope}%
\pgfpathrectangle{\pgfqpoint{0.750000in}{0.330000in}}{\pgfqpoint{2.113636in}{2.310000in}}%
\pgfusepath{clip}%
\pgfsetbuttcap%
\pgfsetroundjoin%
\pgfsetlinewidth{2.007500pt}%
\definecolor{currentstroke}{rgb}{0.121569,0.466667,0.705882}%
\pgfsetstrokecolor{currentstroke}%
\pgfsetdash{}{0pt}%
\pgfpathmoveto{\pgfqpoint{1.134298in}{0.435000in}}%
\pgfpathlineto{\pgfqpoint{1.134298in}{1.917834in}}%
\pgfusepath{stroke}%
\end{pgfscope}%
\begin{pgfscope}%
\pgfpathrectangle{\pgfqpoint{0.750000in}{0.330000in}}{\pgfqpoint{2.113636in}{2.310000in}}%
\pgfusepath{clip}%
\pgfsetbuttcap%
\pgfsetroundjoin%
\pgfsetlinewidth{2.007500pt}%
\definecolor{currentstroke}{rgb}{0.121569,0.466667,0.705882}%
\pgfsetstrokecolor{currentstroke}%
\pgfsetdash{}{0pt}%
\pgfpathmoveto{\pgfqpoint{1.153512in}{0.435000in}}%
\pgfpathlineto{\pgfqpoint{1.153512in}{1.936239in}}%
\pgfusepath{stroke}%
\end{pgfscope}%
\begin{pgfscope}%
\pgfpathrectangle{\pgfqpoint{0.750000in}{0.330000in}}{\pgfqpoint{2.113636in}{2.310000in}}%
\pgfusepath{clip}%
\pgfsetbuttcap%
\pgfsetroundjoin%
\pgfsetlinewidth{2.007500pt}%
\definecolor{currentstroke}{rgb}{0.121569,0.466667,0.705882}%
\pgfsetstrokecolor{currentstroke}%
\pgfsetdash{}{0pt}%
\pgfpathmoveto{\pgfqpoint{1.172727in}{0.435000in}}%
\pgfpathlineto{\pgfqpoint{1.172727in}{1.954282in}}%
\pgfusepath{stroke}%
\end{pgfscope}%
\begin{pgfscope}%
\pgfpathrectangle{\pgfqpoint{0.750000in}{0.330000in}}{\pgfqpoint{2.113636in}{2.310000in}}%
\pgfusepath{clip}%
\pgfsetbuttcap%
\pgfsetroundjoin%
\pgfsetlinewidth{2.007500pt}%
\definecolor{currentstroke}{rgb}{0.121569,0.466667,0.705882}%
\pgfsetstrokecolor{currentstroke}%
\pgfsetdash{}{0pt}%
\pgfpathmoveto{\pgfqpoint{1.191942in}{0.435000in}}%
\pgfpathlineto{\pgfqpoint{1.191942in}{1.976920in}}%
\pgfusepath{stroke}%
\end{pgfscope}%
\begin{pgfscope}%
\pgfpathrectangle{\pgfqpoint{0.750000in}{0.330000in}}{\pgfqpoint{2.113636in}{2.310000in}}%
\pgfusepath{clip}%
\pgfsetbuttcap%
\pgfsetroundjoin%
\pgfsetlinewidth{2.007500pt}%
\definecolor{currentstroke}{rgb}{0.121569,0.466667,0.705882}%
\pgfsetstrokecolor{currentstroke}%
\pgfsetdash{}{0pt}%
\pgfpathmoveto{\pgfqpoint{1.211157in}{0.435000in}}%
\pgfpathlineto{\pgfqpoint{1.211157in}{1.994893in}}%
\pgfusepath{stroke}%
\end{pgfscope}%
\begin{pgfscope}%
\pgfpathrectangle{\pgfqpoint{0.750000in}{0.330000in}}{\pgfqpoint{2.113636in}{2.310000in}}%
\pgfusepath{clip}%
\pgfsetbuttcap%
\pgfsetroundjoin%
\pgfsetlinewidth{2.007500pt}%
\definecolor{currentstroke}{rgb}{0.121569,0.466667,0.705882}%
\pgfsetstrokecolor{currentstroke}%
\pgfsetdash{}{0pt}%
\pgfpathmoveto{\pgfqpoint{1.230372in}{0.435000in}}%
\pgfpathlineto{\pgfqpoint{1.230372in}{2.010367in}}%
\pgfusepath{stroke}%
\end{pgfscope}%
\begin{pgfscope}%
\pgfpathrectangle{\pgfqpoint{0.750000in}{0.330000in}}{\pgfqpoint{2.113636in}{2.310000in}}%
\pgfusepath{clip}%
\pgfsetbuttcap%
\pgfsetroundjoin%
\pgfsetlinewidth{2.007500pt}%
\definecolor{currentstroke}{rgb}{0.121569,0.466667,0.705882}%
\pgfsetstrokecolor{currentstroke}%
\pgfsetdash{}{0pt}%
\pgfpathmoveto{\pgfqpoint{1.249587in}{0.435000in}}%
\pgfpathlineto{\pgfqpoint{1.249587in}{2.025645in}}%
\pgfusepath{stroke}%
\end{pgfscope}%
\begin{pgfscope}%
\pgfpathrectangle{\pgfqpoint{0.750000in}{0.330000in}}{\pgfqpoint{2.113636in}{2.310000in}}%
\pgfusepath{clip}%
\pgfsetbuttcap%
\pgfsetroundjoin%
\pgfsetlinewidth{2.007500pt}%
\definecolor{currentstroke}{rgb}{0.121569,0.466667,0.705882}%
\pgfsetstrokecolor{currentstroke}%
\pgfsetdash{}{0pt}%
\pgfpathmoveto{\pgfqpoint{1.268802in}{0.435000in}}%
\pgfpathlineto{\pgfqpoint{1.268802in}{2.041772in}}%
\pgfusepath{stroke}%
\end{pgfscope}%
\begin{pgfscope}%
\pgfpathrectangle{\pgfqpoint{0.750000in}{0.330000in}}{\pgfqpoint{2.113636in}{2.310000in}}%
\pgfusepath{clip}%
\pgfsetbuttcap%
\pgfsetroundjoin%
\pgfsetlinewidth{2.007500pt}%
\definecolor{currentstroke}{rgb}{0.121569,0.466667,0.705882}%
\pgfsetstrokecolor{currentstroke}%
\pgfsetdash{}{0pt}%
\pgfpathmoveto{\pgfqpoint{1.288017in}{0.435000in}}%
\pgfpathlineto{\pgfqpoint{1.288017in}{2.060703in}}%
\pgfusepath{stroke}%
\end{pgfscope}%
\begin{pgfscope}%
\pgfpathrectangle{\pgfqpoint{0.750000in}{0.330000in}}{\pgfqpoint{2.113636in}{2.310000in}}%
\pgfusepath{clip}%
\pgfsetbuttcap%
\pgfsetroundjoin%
\pgfsetlinewidth{2.007500pt}%
\definecolor{currentstroke}{rgb}{0.121569,0.466667,0.705882}%
\pgfsetstrokecolor{currentstroke}%
\pgfsetdash{}{0pt}%
\pgfpathmoveto{\pgfqpoint{1.307231in}{0.435000in}}%
\pgfpathlineto{\pgfqpoint{1.307231in}{2.078400in}}%
\pgfusepath{stroke}%
\end{pgfscope}%
\begin{pgfscope}%
\pgfpathrectangle{\pgfqpoint{0.750000in}{0.330000in}}{\pgfqpoint{2.113636in}{2.310000in}}%
\pgfusepath{clip}%
\pgfsetbuttcap%
\pgfsetroundjoin%
\pgfsetlinewidth{2.007500pt}%
\definecolor{currentstroke}{rgb}{0.121569,0.466667,0.705882}%
\pgfsetstrokecolor{currentstroke}%
\pgfsetdash{}{0pt}%
\pgfpathmoveto{\pgfqpoint{1.326446in}{0.435000in}}%
\pgfpathlineto{\pgfqpoint{1.326446in}{2.091157in}}%
\pgfusepath{stroke}%
\end{pgfscope}%
\begin{pgfscope}%
\pgfpathrectangle{\pgfqpoint{0.750000in}{0.330000in}}{\pgfqpoint{2.113636in}{2.310000in}}%
\pgfusepath{clip}%
\pgfsetbuttcap%
\pgfsetroundjoin%
\pgfsetlinewidth{2.007500pt}%
\definecolor{currentstroke}{rgb}{0.121569,0.466667,0.705882}%
\pgfsetstrokecolor{currentstroke}%
\pgfsetdash{}{0pt}%
\pgfpathmoveto{\pgfqpoint{1.345661in}{0.435000in}}%
\pgfpathlineto{\pgfqpoint{1.345661in}{2.110747in}}%
\pgfusepath{stroke}%
\end{pgfscope}%
\begin{pgfscope}%
\pgfpathrectangle{\pgfqpoint{0.750000in}{0.330000in}}{\pgfqpoint{2.113636in}{2.310000in}}%
\pgfusepath{clip}%
\pgfsetbuttcap%
\pgfsetroundjoin%
\pgfsetlinewidth{2.007500pt}%
\definecolor{currentstroke}{rgb}{0.121569,0.466667,0.705882}%
\pgfsetstrokecolor{currentstroke}%
\pgfsetdash{}{0pt}%
\pgfpathmoveto{\pgfqpoint{1.364876in}{0.435000in}}%
\pgfpathlineto{\pgfqpoint{1.364876in}{2.128853in}}%
\pgfusepath{stroke}%
\end{pgfscope}%
\begin{pgfscope}%
\pgfpathrectangle{\pgfqpoint{0.750000in}{0.330000in}}{\pgfqpoint{2.113636in}{2.310000in}}%
\pgfusepath{clip}%
\pgfsetbuttcap%
\pgfsetroundjoin%
\pgfsetlinewidth{2.007500pt}%
\definecolor{currentstroke}{rgb}{0.121569,0.466667,0.705882}%
\pgfsetstrokecolor{currentstroke}%
\pgfsetdash{}{0pt}%
\pgfpathmoveto{\pgfqpoint{1.384091in}{0.435000in}}%
\pgfpathlineto{\pgfqpoint{1.384091in}{2.143707in}}%
\pgfusepath{stroke}%
\end{pgfscope}%
\begin{pgfscope}%
\pgfpathrectangle{\pgfqpoint{0.750000in}{0.330000in}}{\pgfqpoint{2.113636in}{2.310000in}}%
\pgfusepath{clip}%
\pgfsetbuttcap%
\pgfsetroundjoin%
\pgfsetlinewidth{2.007500pt}%
\definecolor{currentstroke}{rgb}{0.121569,0.466667,0.705882}%
\pgfsetstrokecolor{currentstroke}%
\pgfsetdash{}{0pt}%
\pgfpathmoveto{\pgfqpoint{1.403306in}{0.435000in}}%
\pgfpathlineto{\pgfqpoint{1.403306in}{2.160933in}}%
\pgfusepath{stroke}%
\end{pgfscope}%
\begin{pgfscope}%
\pgfpathrectangle{\pgfqpoint{0.750000in}{0.330000in}}{\pgfqpoint{2.113636in}{2.310000in}}%
\pgfusepath{clip}%
\pgfsetbuttcap%
\pgfsetroundjoin%
\pgfsetlinewidth{2.007500pt}%
\definecolor{currentstroke}{rgb}{0.121569,0.466667,0.705882}%
\pgfsetstrokecolor{currentstroke}%
\pgfsetdash{}{0pt}%
\pgfpathmoveto{\pgfqpoint{1.422521in}{0.435000in}}%
\pgfpathlineto{\pgfqpoint{1.422521in}{2.180414in}}%
\pgfusepath{stroke}%
\end{pgfscope}%
\begin{pgfscope}%
\pgfpathrectangle{\pgfqpoint{0.750000in}{0.330000in}}{\pgfqpoint{2.113636in}{2.310000in}}%
\pgfusepath{clip}%
\pgfsetbuttcap%
\pgfsetroundjoin%
\pgfsetlinewidth{2.007500pt}%
\definecolor{currentstroke}{rgb}{0.121569,0.466667,0.705882}%
\pgfsetstrokecolor{currentstroke}%
\pgfsetdash{}{0pt}%
\pgfpathmoveto{\pgfqpoint{1.441736in}{0.435000in}}%
\pgfpathlineto{\pgfqpoint{1.441736in}{2.201356in}}%
\pgfusepath{stroke}%
\end{pgfscope}%
\begin{pgfscope}%
\pgfpathrectangle{\pgfqpoint{0.750000in}{0.330000in}}{\pgfqpoint{2.113636in}{2.310000in}}%
\pgfusepath{clip}%
\pgfsetbuttcap%
\pgfsetroundjoin%
\pgfsetlinewidth{2.007500pt}%
\definecolor{currentstroke}{rgb}{0.121569,0.466667,0.705882}%
\pgfsetstrokecolor{currentstroke}%
\pgfsetdash{}{0pt}%
\pgfpathmoveto{\pgfqpoint{1.460950in}{0.435000in}}%
\pgfpathlineto{\pgfqpoint{1.460950in}{2.221410in}}%
\pgfusepath{stroke}%
\end{pgfscope}%
\begin{pgfscope}%
\pgfpathrectangle{\pgfqpoint{0.750000in}{0.330000in}}{\pgfqpoint{2.113636in}{2.310000in}}%
\pgfusepath{clip}%
\pgfsetbuttcap%
\pgfsetroundjoin%
\pgfsetlinewidth{2.007500pt}%
\definecolor{currentstroke}{rgb}{0.121569,0.466667,0.705882}%
\pgfsetstrokecolor{currentstroke}%
\pgfsetdash{}{0pt}%
\pgfpathmoveto{\pgfqpoint{1.480165in}{0.435000in}}%
\pgfpathlineto{\pgfqpoint{1.480165in}{2.237363in}}%
\pgfusepath{stroke}%
\end{pgfscope}%
\begin{pgfscope}%
\pgfpathrectangle{\pgfqpoint{0.750000in}{0.330000in}}{\pgfqpoint{2.113636in}{2.310000in}}%
\pgfusepath{clip}%
\pgfsetbuttcap%
\pgfsetroundjoin%
\pgfsetlinewidth{2.007500pt}%
\definecolor{currentstroke}{rgb}{0.121569,0.466667,0.705882}%
\pgfsetstrokecolor{currentstroke}%
\pgfsetdash{}{0pt}%
\pgfpathmoveto{\pgfqpoint{1.499380in}{0.435000in}}%
\pgfpathlineto{\pgfqpoint{1.499380in}{2.254165in}}%
\pgfusepath{stroke}%
\end{pgfscope}%
\begin{pgfscope}%
\pgfpathrectangle{\pgfqpoint{0.750000in}{0.330000in}}{\pgfqpoint{2.113636in}{2.310000in}}%
\pgfusepath{clip}%
\pgfsetbuttcap%
\pgfsetroundjoin%
\pgfsetlinewidth{2.007500pt}%
\definecolor{currentstroke}{rgb}{0.121569,0.466667,0.705882}%
\pgfsetstrokecolor{currentstroke}%
\pgfsetdash{}{0pt}%
\pgfpathmoveto{\pgfqpoint{1.518595in}{0.435000in}}%
\pgfpathlineto{\pgfqpoint{1.518595in}{2.277746in}}%
\pgfusepath{stroke}%
\end{pgfscope}%
\begin{pgfscope}%
\pgfpathrectangle{\pgfqpoint{0.750000in}{0.330000in}}{\pgfqpoint{2.113636in}{2.310000in}}%
\pgfusepath{clip}%
\pgfsetbuttcap%
\pgfsetroundjoin%
\pgfsetlinewidth{2.007500pt}%
\definecolor{currentstroke}{rgb}{0.121569,0.466667,0.705882}%
\pgfsetstrokecolor{currentstroke}%
\pgfsetdash{}{0pt}%
\pgfpathmoveto{\pgfqpoint{1.537810in}{0.435000in}}%
\pgfpathlineto{\pgfqpoint{1.537810in}{2.297934in}}%
\pgfusepath{stroke}%
\end{pgfscope}%
\begin{pgfscope}%
\pgfpathrectangle{\pgfqpoint{0.750000in}{0.330000in}}{\pgfqpoint{2.113636in}{2.310000in}}%
\pgfusepath{clip}%
\pgfsetbuttcap%
\pgfsetroundjoin%
\pgfsetlinewidth{2.007500pt}%
\definecolor{currentstroke}{rgb}{0.121569,0.466667,0.705882}%
\pgfsetstrokecolor{currentstroke}%
\pgfsetdash{}{0pt}%
\pgfpathmoveto{\pgfqpoint{1.557025in}{0.435000in}}%
\pgfpathlineto{\pgfqpoint{1.557025in}{2.316731in}}%
\pgfusepath{stroke}%
\end{pgfscope}%
\begin{pgfscope}%
\pgfpathrectangle{\pgfqpoint{0.750000in}{0.330000in}}{\pgfqpoint{2.113636in}{2.310000in}}%
\pgfusepath{clip}%
\pgfsetbuttcap%
\pgfsetroundjoin%
\pgfsetlinewidth{2.007500pt}%
\definecolor{currentstroke}{rgb}{0.121569,0.466667,0.705882}%
\pgfsetstrokecolor{currentstroke}%
\pgfsetdash{}{0pt}%
\pgfpathmoveto{\pgfqpoint{1.576240in}{0.435000in}}%
\pgfpathlineto{\pgfqpoint{1.576240in}{2.328741in}}%
\pgfusepath{stroke}%
\end{pgfscope}%
\begin{pgfscope}%
\pgfpathrectangle{\pgfqpoint{0.750000in}{0.330000in}}{\pgfqpoint{2.113636in}{2.310000in}}%
\pgfusepath{clip}%
\pgfsetbuttcap%
\pgfsetroundjoin%
\pgfsetlinewidth{2.007500pt}%
\definecolor{currentstroke}{rgb}{0.121569,0.466667,0.705882}%
\pgfsetstrokecolor{currentstroke}%
\pgfsetdash{}{0pt}%
\pgfpathmoveto{\pgfqpoint{1.595455in}{0.435000in}}%
\pgfpathlineto{\pgfqpoint{1.595455in}{2.340257in}}%
\pgfusepath{stroke}%
\end{pgfscope}%
\begin{pgfscope}%
\pgfpathrectangle{\pgfqpoint{0.750000in}{0.330000in}}{\pgfqpoint{2.113636in}{2.310000in}}%
\pgfusepath{clip}%
\pgfsetbuttcap%
\pgfsetroundjoin%
\pgfsetlinewidth{2.007500pt}%
\definecolor{currentstroke}{rgb}{0.121569,0.466667,0.705882}%
\pgfsetstrokecolor{currentstroke}%
\pgfsetdash{}{0pt}%
\pgfpathmoveto{\pgfqpoint{1.614669in}{0.435000in}}%
\pgfpathlineto{\pgfqpoint{1.614669in}{2.359455in}}%
\pgfusepath{stroke}%
\end{pgfscope}%
\begin{pgfscope}%
\pgfpathrectangle{\pgfqpoint{0.750000in}{0.330000in}}{\pgfqpoint{2.113636in}{2.310000in}}%
\pgfusepath{clip}%
\pgfsetbuttcap%
\pgfsetroundjoin%
\pgfsetlinewidth{2.007500pt}%
\definecolor{currentstroke}{rgb}{0.121569,0.466667,0.705882}%
\pgfsetstrokecolor{currentstroke}%
\pgfsetdash{}{0pt}%
\pgfpathmoveto{\pgfqpoint{1.633884in}{0.435000in}}%
\pgfpathlineto{\pgfqpoint{1.633884in}{2.378519in}}%
\pgfusepath{stroke}%
\end{pgfscope}%
\begin{pgfscope}%
\pgfpathrectangle{\pgfqpoint{0.750000in}{0.330000in}}{\pgfqpoint{2.113636in}{2.310000in}}%
\pgfusepath{clip}%
\pgfsetbuttcap%
\pgfsetroundjoin%
\pgfsetlinewidth{2.007500pt}%
\definecolor{currentstroke}{rgb}{0.121569,0.466667,0.705882}%
\pgfsetstrokecolor{currentstroke}%
\pgfsetdash{}{0pt}%
\pgfpathmoveto{\pgfqpoint{1.653099in}{0.435000in}}%
\pgfpathlineto{\pgfqpoint{1.653099in}{2.393404in}}%
\pgfusepath{stroke}%
\end{pgfscope}%
\begin{pgfscope}%
\pgfpathrectangle{\pgfqpoint{0.750000in}{0.330000in}}{\pgfqpoint{2.113636in}{2.310000in}}%
\pgfusepath{clip}%
\pgfsetbuttcap%
\pgfsetroundjoin%
\pgfsetlinewidth{2.007500pt}%
\definecolor{currentstroke}{rgb}{0.121569,0.466667,0.705882}%
\pgfsetstrokecolor{currentstroke}%
\pgfsetdash{}{0pt}%
\pgfpathmoveto{\pgfqpoint{1.672314in}{0.435000in}}%
\pgfpathlineto{\pgfqpoint{1.672314in}{2.408125in}}%
\pgfusepath{stroke}%
\end{pgfscope}%
\begin{pgfscope}%
\pgfpathrectangle{\pgfqpoint{0.750000in}{0.330000in}}{\pgfqpoint{2.113636in}{2.310000in}}%
\pgfusepath{clip}%
\pgfsetbuttcap%
\pgfsetroundjoin%
\pgfsetlinewidth{2.007500pt}%
\definecolor{currentstroke}{rgb}{0.121569,0.466667,0.705882}%
\pgfsetstrokecolor{currentstroke}%
\pgfsetdash{}{0pt}%
\pgfpathmoveto{\pgfqpoint{1.691529in}{0.435000in}}%
\pgfpathlineto{\pgfqpoint{1.691529in}{2.423725in}}%
\pgfusepath{stroke}%
\end{pgfscope}%
\begin{pgfscope}%
\pgfpathrectangle{\pgfqpoint{0.750000in}{0.330000in}}{\pgfqpoint{2.113636in}{2.310000in}}%
\pgfusepath{clip}%
\pgfsetbuttcap%
\pgfsetroundjoin%
\pgfsetlinewidth{2.007500pt}%
\definecolor{currentstroke}{rgb}{0.121569,0.466667,0.705882}%
\pgfsetstrokecolor{currentstroke}%
\pgfsetdash{}{0pt}%
\pgfpathmoveto{\pgfqpoint{1.710744in}{0.435000in}}%
\pgfpathlineto{\pgfqpoint{1.710744in}{2.446183in}}%
\pgfusepath{stroke}%
\end{pgfscope}%
\begin{pgfscope}%
\pgfpathrectangle{\pgfqpoint{0.750000in}{0.330000in}}{\pgfqpoint{2.113636in}{2.310000in}}%
\pgfusepath{clip}%
\pgfsetbuttcap%
\pgfsetroundjoin%
\pgfsetlinewidth{2.007500pt}%
\definecolor{currentstroke}{rgb}{0.121569,0.466667,0.705882}%
\pgfsetstrokecolor{currentstroke}%
\pgfsetdash{}{0pt}%
\pgfpathmoveto{\pgfqpoint{1.729959in}{0.435000in}}%
\pgfpathlineto{\pgfqpoint{1.729959in}{2.460848in}}%
\pgfusepath{stroke}%
\end{pgfscope}%
\begin{pgfscope}%
\pgfpathrectangle{\pgfqpoint{0.750000in}{0.330000in}}{\pgfqpoint{2.113636in}{2.310000in}}%
\pgfusepath{clip}%
\pgfsetbuttcap%
\pgfsetroundjoin%
\pgfsetlinewidth{2.007500pt}%
\definecolor{currentstroke}{rgb}{0.121569,0.466667,0.705882}%
\pgfsetstrokecolor{currentstroke}%
\pgfsetdash{}{0pt}%
\pgfpathmoveto{\pgfqpoint{1.749174in}{0.435000in}}%
\pgfpathlineto{\pgfqpoint{1.749174in}{2.474932in}}%
\pgfusepath{stroke}%
\end{pgfscope}%
\begin{pgfscope}%
\pgfpathrectangle{\pgfqpoint{0.750000in}{0.330000in}}{\pgfqpoint{2.113636in}{2.310000in}}%
\pgfusepath{clip}%
\pgfsetbuttcap%
\pgfsetroundjoin%
\pgfsetlinewidth{2.007500pt}%
\definecolor{currentstroke}{rgb}{0.121569,0.466667,0.705882}%
\pgfsetstrokecolor{currentstroke}%
\pgfsetdash{}{0pt}%
\pgfpathmoveto{\pgfqpoint{1.768388in}{0.435000in}}%
\pgfpathlineto{\pgfqpoint{1.768388in}{2.493400in}}%
\pgfusepath{stroke}%
\end{pgfscope}%
\begin{pgfscope}%
\pgfpathrectangle{\pgfqpoint{0.750000in}{0.330000in}}{\pgfqpoint{2.113636in}{2.310000in}}%
\pgfusepath{clip}%
\pgfsetbuttcap%
\pgfsetroundjoin%
\pgfsetlinewidth{2.007500pt}%
\definecolor{currentstroke}{rgb}{0.121569,0.466667,0.705882}%
\pgfsetstrokecolor{currentstroke}%
\pgfsetdash{}{0pt}%
\pgfpathmoveto{\pgfqpoint{1.787603in}{0.435000in}}%
\pgfpathlineto{\pgfqpoint{1.787603in}{2.508198in}}%
\pgfusepath{stroke}%
\end{pgfscope}%
\begin{pgfscope}%
\pgfpathrectangle{\pgfqpoint{0.750000in}{0.330000in}}{\pgfqpoint{2.113636in}{2.310000in}}%
\pgfusepath{clip}%
\pgfsetbuttcap%
\pgfsetroundjoin%
\pgfsetlinewidth{2.007500pt}%
\definecolor{currentstroke}{rgb}{0.121569,0.466667,0.705882}%
\pgfsetstrokecolor{currentstroke}%
\pgfsetdash{}{0pt}%
\pgfpathmoveto{\pgfqpoint{1.806818in}{0.435000in}}%
\pgfpathlineto{\pgfqpoint{1.806818in}{2.535000in}}%
\pgfusepath{stroke}%
\end{pgfscope}%
\begin{pgfscope}%
\pgfpathrectangle{\pgfqpoint{0.750000in}{0.330000in}}{\pgfqpoint{2.113636in}{2.310000in}}%
\pgfusepath{clip}%
\pgfsetbuttcap%
\pgfsetroundjoin%
\pgfsetlinewidth{2.007500pt}%
\definecolor{currentstroke}{rgb}{0.121569,0.466667,0.705882}%
\pgfsetstrokecolor{currentstroke}%
\pgfsetdash{}{0pt}%
\pgfpathmoveto{\pgfqpoint{1.826033in}{0.435000in}}%
\pgfpathlineto{\pgfqpoint{1.826033in}{2.513571in}}%
\pgfusepath{stroke}%
\end{pgfscope}%
\begin{pgfscope}%
\pgfpathrectangle{\pgfqpoint{0.750000in}{0.330000in}}{\pgfqpoint{2.113636in}{2.310000in}}%
\pgfusepath{clip}%
\pgfsetbuttcap%
\pgfsetroundjoin%
\pgfsetlinewidth{2.007500pt}%
\definecolor{currentstroke}{rgb}{0.121569,0.466667,0.705882}%
\pgfsetstrokecolor{currentstroke}%
\pgfsetdash{}{0pt}%
\pgfpathmoveto{\pgfqpoint{1.845248in}{0.435000in}}%
\pgfpathlineto{\pgfqpoint{1.845248in}{2.501679in}}%
\pgfusepath{stroke}%
\end{pgfscope}%
\begin{pgfscope}%
\pgfpathrectangle{\pgfqpoint{0.750000in}{0.330000in}}{\pgfqpoint{2.113636in}{2.310000in}}%
\pgfusepath{clip}%
\pgfsetbuttcap%
\pgfsetroundjoin%
\pgfsetlinewidth{2.007500pt}%
\definecolor{currentstroke}{rgb}{0.121569,0.466667,0.705882}%
\pgfsetstrokecolor{currentstroke}%
\pgfsetdash{}{0pt}%
\pgfpathmoveto{\pgfqpoint{1.864463in}{0.435000in}}%
\pgfpathlineto{\pgfqpoint{1.864463in}{2.487940in}}%
\pgfusepath{stroke}%
\end{pgfscope}%
\begin{pgfscope}%
\pgfpathrectangle{\pgfqpoint{0.750000in}{0.330000in}}{\pgfqpoint{2.113636in}{2.310000in}}%
\pgfusepath{clip}%
\pgfsetbuttcap%
\pgfsetroundjoin%
\pgfsetlinewidth{2.007500pt}%
\definecolor{currentstroke}{rgb}{0.121569,0.466667,0.705882}%
\pgfsetstrokecolor{currentstroke}%
\pgfsetdash{}{0pt}%
\pgfpathmoveto{\pgfqpoint{1.883678in}{0.435000in}}%
\pgfpathlineto{\pgfqpoint{1.883678in}{2.482120in}}%
\pgfusepath{stroke}%
\end{pgfscope}%
\begin{pgfscope}%
\pgfpathrectangle{\pgfqpoint{0.750000in}{0.330000in}}{\pgfqpoint{2.113636in}{2.310000in}}%
\pgfusepath{clip}%
\pgfsetbuttcap%
\pgfsetroundjoin%
\pgfsetlinewidth{2.007500pt}%
\definecolor{currentstroke}{rgb}{0.121569,0.466667,0.705882}%
\pgfsetstrokecolor{currentstroke}%
\pgfsetdash{}{0pt}%
\pgfpathmoveto{\pgfqpoint{1.902893in}{0.435000in}}%
\pgfpathlineto{\pgfqpoint{1.902893in}{2.469583in}}%
\pgfusepath{stroke}%
\end{pgfscope}%
\begin{pgfscope}%
\pgfpathrectangle{\pgfqpoint{0.750000in}{0.330000in}}{\pgfqpoint{2.113636in}{2.310000in}}%
\pgfusepath{clip}%
\pgfsetbuttcap%
\pgfsetroundjoin%
\pgfsetlinewidth{2.007500pt}%
\definecolor{currentstroke}{rgb}{0.121569,0.466667,0.705882}%
\pgfsetstrokecolor{currentstroke}%
\pgfsetdash{}{0pt}%
\pgfpathmoveto{\pgfqpoint{1.922107in}{0.435000in}}%
\pgfpathlineto{\pgfqpoint{1.922107in}{2.455137in}}%
\pgfusepath{stroke}%
\end{pgfscope}%
\begin{pgfscope}%
\pgfpathrectangle{\pgfqpoint{0.750000in}{0.330000in}}{\pgfqpoint{2.113636in}{2.310000in}}%
\pgfusepath{clip}%
\pgfsetbuttcap%
\pgfsetroundjoin%
\pgfsetlinewidth{2.007500pt}%
\definecolor{currentstroke}{rgb}{0.121569,0.466667,0.705882}%
\pgfsetstrokecolor{currentstroke}%
\pgfsetdash{}{0pt}%
\pgfpathmoveto{\pgfqpoint{1.941322in}{0.435000in}}%
\pgfpathlineto{\pgfqpoint{1.941322in}{2.443779in}}%
\pgfusepath{stroke}%
\end{pgfscope}%
\begin{pgfscope}%
\pgfpathrectangle{\pgfqpoint{0.750000in}{0.330000in}}{\pgfqpoint{2.113636in}{2.310000in}}%
\pgfusepath{clip}%
\pgfsetbuttcap%
\pgfsetroundjoin%
\pgfsetlinewidth{2.007500pt}%
\definecolor{currentstroke}{rgb}{0.121569,0.466667,0.705882}%
\pgfsetstrokecolor{currentstroke}%
\pgfsetdash{}{0pt}%
\pgfpathmoveto{\pgfqpoint{1.960537in}{0.435000in}}%
\pgfpathlineto{\pgfqpoint{1.960537in}{2.434785in}}%
\pgfusepath{stroke}%
\end{pgfscope}%
\begin{pgfscope}%
\pgfpathrectangle{\pgfqpoint{0.750000in}{0.330000in}}{\pgfqpoint{2.113636in}{2.310000in}}%
\pgfusepath{clip}%
\pgfsetbuttcap%
\pgfsetroundjoin%
\pgfsetlinewidth{2.007500pt}%
\definecolor{currentstroke}{rgb}{0.121569,0.466667,0.705882}%
\pgfsetstrokecolor{currentstroke}%
\pgfsetdash{}{0pt}%
\pgfpathmoveto{\pgfqpoint{1.979752in}{0.435000in}}%
\pgfpathlineto{\pgfqpoint{1.979752in}{2.419955in}}%
\pgfusepath{stroke}%
\end{pgfscope}%
\begin{pgfscope}%
\pgfpathrectangle{\pgfqpoint{0.750000in}{0.330000in}}{\pgfqpoint{2.113636in}{2.310000in}}%
\pgfusepath{clip}%
\pgfsetbuttcap%
\pgfsetroundjoin%
\pgfsetlinewidth{2.007500pt}%
\definecolor{currentstroke}{rgb}{0.121569,0.466667,0.705882}%
\pgfsetstrokecolor{currentstroke}%
\pgfsetdash{}{0pt}%
\pgfpathmoveto{\pgfqpoint{1.998967in}{0.435000in}}%
\pgfpathlineto{\pgfqpoint{1.998967in}{2.409798in}}%
\pgfusepath{stroke}%
\end{pgfscope}%
\begin{pgfscope}%
\pgfpathrectangle{\pgfqpoint{0.750000in}{0.330000in}}{\pgfqpoint{2.113636in}{2.310000in}}%
\pgfusepath{clip}%
\pgfsetbuttcap%
\pgfsetroundjoin%
\pgfsetlinewidth{2.007500pt}%
\definecolor{currentstroke}{rgb}{0.121569,0.466667,0.705882}%
\pgfsetstrokecolor{currentstroke}%
\pgfsetdash{}{0pt}%
\pgfpathmoveto{\pgfqpoint{2.018182in}{0.435000in}}%
\pgfpathlineto{\pgfqpoint{2.018182in}{2.394567in}}%
\pgfusepath{stroke}%
\end{pgfscope}%
\begin{pgfscope}%
\pgfpathrectangle{\pgfqpoint{0.750000in}{0.330000in}}{\pgfqpoint{2.113636in}{2.310000in}}%
\pgfusepath{clip}%
\pgfsetbuttcap%
\pgfsetroundjoin%
\pgfsetlinewidth{2.007500pt}%
\definecolor{currentstroke}{rgb}{0.121569,0.466667,0.705882}%
\pgfsetstrokecolor{currentstroke}%
\pgfsetdash{}{0pt}%
\pgfpathmoveto{\pgfqpoint{2.037397in}{0.435000in}}%
\pgfpathlineto{\pgfqpoint{2.037397in}{2.385510in}}%
\pgfusepath{stroke}%
\end{pgfscope}%
\begin{pgfscope}%
\pgfpathrectangle{\pgfqpoint{0.750000in}{0.330000in}}{\pgfqpoint{2.113636in}{2.310000in}}%
\pgfusepath{clip}%
\pgfsetbuttcap%
\pgfsetroundjoin%
\pgfsetlinewidth{2.007500pt}%
\definecolor{currentstroke}{rgb}{0.121569,0.466667,0.705882}%
\pgfsetstrokecolor{currentstroke}%
\pgfsetdash{}{0pt}%
\pgfpathmoveto{\pgfqpoint{2.056612in}{0.435000in}}%
\pgfpathlineto{\pgfqpoint{2.056612in}{2.374725in}}%
\pgfusepath{stroke}%
\end{pgfscope}%
\begin{pgfscope}%
\pgfpathrectangle{\pgfqpoint{0.750000in}{0.330000in}}{\pgfqpoint{2.113636in}{2.310000in}}%
\pgfusepath{clip}%
\pgfsetbuttcap%
\pgfsetroundjoin%
\pgfsetlinewidth{2.007500pt}%
\definecolor{currentstroke}{rgb}{0.121569,0.466667,0.705882}%
\pgfsetstrokecolor{currentstroke}%
\pgfsetdash{}{0pt}%
\pgfpathmoveto{\pgfqpoint{2.075826in}{0.435000in}}%
\pgfpathlineto{\pgfqpoint{2.075826in}{2.362471in}}%
\pgfusepath{stroke}%
\end{pgfscope}%
\begin{pgfscope}%
\pgfpathrectangle{\pgfqpoint{0.750000in}{0.330000in}}{\pgfqpoint{2.113636in}{2.310000in}}%
\pgfusepath{clip}%
\pgfsetbuttcap%
\pgfsetroundjoin%
\pgfsetlinewidth{2.007500pt}%
\definecolor{currentstroke}{rgb}{0.121569,0.466667,0.705882}%
\pgfsetstrokecolor{currentstroke}%
\pgfsetdash{}{0pt}%
\pgfpathmoveto{\pgfqpoint{2.095041in}{0.435000in}}%
\pgfpathlineto{\pgfqpoint{2.095041in}{2.348505in}}%
\pgfusepath{stroke}%
\end{pgfscope}%
\begin{pgfscope}%
\pgfpathrectangle{\pgfqpoint{0.750000in}{0.330000in}}{\pgfqpoint{2.113636in}{2.310000in}}%
\pgfusepath{clip}%
\pgfsetbuttcap%
\pgfsetroundjoin%
\pgfsetlinewidth{2.007500pt}%
\definecolor{currentstroke}{rgb}{0.121569,0.466667,0.705882}%
\pgfsetstrokecolor{currentstroke}%
\pgfsetdash{}{0pt}%
\pgfpathmoveto{\pgfqpoint{2.114256in}{0.435000in}}%
\pgfpathlineto{\pgfqpoint{2.114256in}{2.337044in}}%
\pgfusepath{stroke}%
\end{pgfscope}%
\begin{pgfscope}%
\pgfpathrectangle{\pgfqpoint{0.750000in}{0.330000in}}{\pgfqpoint{2.113636in}{2.310000in}}%
\pgfusepath{clip}%
\pgfsetbuttcap%
\pgfsetroundjoin%
\pgfsetlinewidth{2.007500pt}%
\definecolor{currentstroke}{rgb}{0.121569,0.466667,0.705882}%
\pgfsetstrokecolor{currentstroke}%
\pgfsetdash{}{0pt}%
\pgfpathmoveto{\pgfqpoint{2.133471in}{0.435000in}}%
\pgfpathlineto{\pgfqpoint{2.133471in}{2.325756in}}%
\pgfusepath{stroke}%
\end{pgfscope}%
\begin{pgfscope}%
\pgfpathrectangle{\pgfqpoint{0.750000in}{0.330000in}}{\pgfqpoint{2.113636in}{2.310000in}}%
\pgfusepath{clip}%
\pgfsetbuttcap%
\pgfsetroundjoin%
\pgfsetlinewidth{2.007500pt}%
\definecolor{currentstroke}{rgb}{0.121569,0.466667,0.705882}%
\pgfsetstrokecolor{currentstroke}%
\pgfsetdash{}{0pt}%
\pgfpathmoveto{\pgfqpoint{2.152686in}{0.435000in}}%
\pgfpathlineto{\pgfqpoint{2.152686in}{2.315938in}}%
\pgfusepath{stroke}%
\end{pgfscope}%
\begin{pgfscope}%
\pgfpathrectangle{\pgfqpoint{0.750000in}{0.330000in}}{\pgfqpoint{2.113636in}{2.310000in}}%
\pgfusepath{clip}%
\pgfsetbuttcap%
\pgfsetroundjoin%
\pgfsetlinewidth{2.007500pt}%
\definecolor{currentstroke}{rgb}{0.121569,0.466667,0.705882}%
\pgfsetstrokecolor{currentstroke}%
\pgfsetdash{}{0pt}%
\pgfpathmoveto{\pgfqpoint{2.171901in}{0.435000in}}%
\pgfpathlineto{\pgfqpoint{2.171901in}{2.300204in}}%
\pgfusepath{stroke}%
\end{pgfscope}%
\begin{pgfscope}%
\pgfpathrectangle{\pgfqpoint{0.750000in}{0.330000in}}{\pgfqpoint{2.113636in}{2.310000in}}%
\pgfusepath{clip}%
\pgfsetbuttcap%
\pgfsetroundjoin%
\pgfsetlinewidth{2.007500pt}%
\definecolor{currentstroke}{rgb}{0.121569,0.466667,0.705882}%
\pgfsetstrokecolor{currentstroke}%
\pgfsetdash{}{0pt}%
\pgfpathmoveto{\pgfqpoint{2.191116in}{0.435000in}}%
\pgfpathlineto{\pgfqpoint{2.191116in}{2.287353in}}%
\pgfusepath{stroke}%
\end{pgfscope}%
\begin{pgfscope}%
\pgfpathrectangle{\pgfqpoint{0.750000in}{0.330000in}}{\pgfqpoint{2.113636in}{2.310000in}}%
\pgfusepath{clip}%
\pgfsetbuttcap%
\pgfsetroundjoin%
\pgfsetlinewidth{2.007500pt}%
\definecolor{currentstroke}{rgb}{0.121569,0.466667,0.705882}%
\pgfsetstrokecolor{currentstroke}%
\pgfsetdash{}{0pt}%
\pgfpathmoveto{\pgfqpoint{2.210331in}{0.435000in}}%
\pgfpathlineto{\pgfqpoint{2.210331in}{2.275892in}}%
\pgfusepath{stroke}%
\end{pgfscope}%
\begin{pgfscope}%
\pgfpathrectangle{\pgfqpoint{0.750000in}{0.330000in}}{\pgfqpoint{2.113636in}{2.310000in}}%
\pgfusepath{clip}%
\pgfsetbuttcap%
\pgfsetroundjoin%
\pgfsetlinewidth{2.007500pt}%
\definecolor{currentstroke}{rgb}{0.121569,0.466667,0.705882}%
\pgfsetstrokecolor{currentstroke}%
\pgfsetdash{}{0pt}%
\pgfpathmoveto{\pgfqpoint{2.229545in}{0.435000in}}%
\pgfpathlineto{\pgfqpoint{2.229545in}{2.264235in}}%
\pgfusepath{stroke}%
\end{pgfscope}%
\begin{pgfscope}%
\pgfpathrectangle{\pgfqpoint{0.750000in}{0.330000in}}{\pgfqpoint{2.113636in}{2.310000in}}%
\pgfusepath{clip}%
\pgfsetbuttcap%
\pgfsetroundjoin%
\pgfsetlinewidth{2.007500pt}%
\definecolor{currentstroke}{rgb}{0.121569,0.466667,0.705882}%
\pgfsetstrokecolor{currentstroke}%
\pgfsetdash{}{0pt}%
\pgfpathmoveto{\pgfqpoint{2.248760in}{0.435000in}}%
\pgfpathlineto{\pgfqpoint{2.248760in}{2.251699in}}%
\pgfusepath{stroke}%
\end{pgfscope}%
\begin{pgfscope}%
\pgfpathrectangle{\pgfqpoint{0.750000in}{0.330000in}}{\pgfqpoint{2.113636in}{2.310000in}}%
\pgfusepath{clip}%
\pgfsetbuttcap%
\pgfsetroundjoin%
\pgfsetlinewidth{2.007500pt}%
\definecolor{currentstroke}{rgb}{0.121569,0.466667,0.705882}%
\pgfsetstrokecolor{currentstroke}%
\pgfsetdash{}{0pt}%
\pgfpathmoveto{\pgfqpoint{2.267975in}{0.435000in}}%
\pgfpathlineto{\pgfqpoint{2.267975in}{2.239374in}}%
\pgfusepath{stroke}%
\end{pgfscope}%
\begin{pgfscope}%
\pgfpathrectangle{\pgfqpoint{0.750000in}{0.330000in}}{\pgfqpoint{2.113636in}{2.310000in}}%
\pgfusepath{clip}%
\pgfsetbuttcap%
\pgfsetroundjoin%
\pgfsetlinewidth{2.007500pt}%
\definecolor{currentstroke}{rgb}{0.121569,0.466667,0.705882}%
\pgfsetstrokecolor{currentstroke}%
\pgfsetdash{}{0pt}%
\pgfpathmoveto{\pgfqpoint{2.287190in}{0.435000in}}%
\pgfpathlineto{\pgfqpoint{2.287190in}{2.229712in}}%
\pgfusepath{stroke}%
\end{pgfscope}%
\begin{pgfscope}%
\pgfpathrectangle{\pgfqpoint{0.750000in}{0.330000in}}{\pgfqpoint{2.113636in}{2.310000in}}%
\pgfusepath{clip}%
\pgfsetbuttcap%
\pgfsetroundjoin%
\pgfsetlinewidth{2.007500pt}%
\definecolor{currentstroke}{rgb}{0.121569,0.466667,0.705882}%
\pgfsetstrokecolor{currentstroke}%
\pgfsetdash{}{0pt}%
\pgfpathmoveto{\pgfqpoint{2.306405in}{0.435000in}}%
\pgfpathlineto{\pgfqpoint{2.306405in}{2.218794in}}%
\pgfusepath{stroke}%
\end{pgfscope}%
\begin{pgfscope}%
\pgfpathrectangle{\pgfqpoint{0.750000in}{0.330000in}}{\pgfqpoint{2.113636in}{2.310000in}}%
\pgfusepath{clip}%
\pgfsetbuttcap%
\pgfsetroundjoin%
\pgfsetlinewidth{2.007500pt}%
\definecolor{currentstroke}{rgb}{0.121569,0.466667,0.705882}%
\pgfsetstrokecolor{currentstroke}%
\pgfsetdash{}{0pt}%
\pgfpathmoveto{\pgfqpoint{2.325620in}{0.435000in}}%
\pgfpathlineto{\pgfqpoint{2.325620in}{2.205872in}}%
\pgfusepath{stroke}%
\end{pgfscope}%
\begin{pgfscope}%
\pgfpathrectangle{\pgfqpoint{0.750000in}{0.330000in}}{\pgfqpoint{2.113636in}{2.310000in}}%
\pgfusepath{clip}%
\pgfsetbuttcap%
\pgfsetroundjoin%
\pgfsetlinewidth{2.007500pt}%
\definecolor{currentstroke}{rgb}{0.121569,0.466667,0.705882}%
\pgfsetstrokecolor{currentstroke}%
\pgfsetdash{}{0pt}%
\pgfpathmoveto{\pgfqpoint{2.344835in}{0.435000in}}%
\pgfpathlineto{\pgfqpoint{2.344835in}{2.195888in}}%
\pgfusepath{stroke}%
\end{pgfscope}%
\begin{pgfscope}%
\pgfpathrectangle{\pgfqpoint{0.750000in}{0.330000in}}{\pgfqpoint{2.113636in}{2.310000in}}%
\pgfusepath{clip}%
\pgfsetbuttcap%
\pgfsetroundjoin%
\pgfsetlinewidth{2.007500pt}%
\definecolor{currentstroke}{rgb}{0.121569,0.466667,0.705882}%
\pgfsetstrokecolor{currentstroke}%
\pgfsetdash{}{0pt}%
\pgfpathmoveto{\pgfqpoint{2.364050in}{0.435000in}}%
\pgfpathlineto{\pgfqpoint{2.364050in}{2.185370in}}%
\pgfusepath{stroke}%
\end{pgfscope}%
\begin{pgfscope}%
\pgfpathrectangle{\pgfqpoint{0.750000in}{0.330000in}}{\pgfqpoint{2.113636in}{2.310000in}}%
\pgfusepath{clip}%
\pgfsetbuttcap%
\pgfsetroundjoin%
\pgfsetlinewidth{2.007500pt}%
\definecolor{currentstroke}{rgb}{0.121569,0.466667,0.705882}%
\pgfsetstrokecolor{currentstroke}%
\pgfsetdash{}{0pt}%
\pgfpathmoveto{\pgfqpoint{2.383264in}{0.435000in}}%
\pgfpathlineto{\pgfqpoint{2.383264in}{2.175669in}}%
\pgfusepath{stroke}%
\end{pgfscope}%
\begin{pgfscope}%
\pgfpathrectangle{\pgfqpoint{0.750000in}{0.330000in}}{\pgfqpoint{2.113636in}{2.310000in}}%
\pgfusepath{clip}%
\pgfsetbuttcap%
\pgfsetroundjoin%
\pgfsetlinewidth{2.007500pt}%
\definecolor{currentstroke}{rgb}{0.121569,0.466667,0.705882}%
\pgfsetstrokecolor{currentstroke}%
\pgfsetdash{}{0pt}%
\pgfpathmoveto{\pgfqpoint{2.402479in}{0.435000in}}%
\pgfpathlineto{\pgfqpoint{2.402479in}{2.168019in}}%
\pgfusepath{stroke}%
\end{pgfscope}%
\begin{pgfscope}%
\pgfpathrectangle{\pgfqpoint{0.750000in}{0.330000in}}{\pgfqpoint{2.113636in}{2.310000in}}%
\pgfusepath{clip}%
\pgfsetbuttcap%
\pgfsetroundjoin%
\pgfsetlinewidth{2.007500pt}%
\definecolor{currentstroke}{rgb}{0.121569,0.466667,0.705882}%
\pgfsetstrokecolor{currentstroke}%
\pgfsetdash{}{0pt}%
\pgfpathmoveto{\pgfqpoint{2.421694in}{0.435000in}}%
\pgfpathlineto{\pgfqpoint{2.421694in}{2.157846in}}%
\pgfusepath{stroke}%
\end{pgfscope}%
\begin{pgfscope}%
\pgfpathrectangle{\pgfqpoint{0.750000in}{0.330000in}}{\pgfqpoint{2.113636in}{2.310000in}}%
\pgfusepath{clip}%
\pgfsetbuttcap%
\pgfsetroundjoin%
\pgfsetlinewidth{2.007500pt}%
\definecolor{currentstroke}{rgb}{0.121569,0.466667,0.705882}%
\pgfsetstrokecolor{currentstroke}%
\pgfsetdash{}{0pt}%
\pgfpathmoveto{\pgfqpoint{2.440909in}{0.435000in}}%
\pgfpathlineto{\pgfqpoint{2.440909in}{2.140023in}}%
\pgfusepath{stroke}%
\end{pgfscope}%
\begin{pgfscope}%
\pgfpathrectangle{\pgfqpoint{0.750000in}{0.330000in}}{\pgfqpoint{2.113636in}{2.310000in}}%
\pgfusepath{clip}%
\pgfsetbuttcap%
\pgfsetroundjoin%
\pgfsetlinewidth{2.007500pt}%
\definecolor{currentstroke}{rgb}{0.121569,0.466667,0.705882}%
\pgfsetstrokecolor{currentstroke}%
\pgfsetdash{}{0pt}%
\pgfpathmoveto{\pgfqpoint{2.460124in}{0.435000in}}%
\pgfpathlineto{\pgfqpoint{2.460124in}{2.129662in}}%
\pgfusepath{stroke}%
\end{pgfscope}%
\begin{pgfscope}%
\pgfpathrectangle{\pgfqpoint{0.750000in}{0.330000in}}{\pgfqpoint{2.113636in}{2.310000in}}%
\pgfusepath{clip}%
\pgfsetbuttcap%
\pgfsetroundjoin%
\pgfsetlinewidth{2.007500pt}%
\definecolor{currentstroke}{rgb}{0.121569,0.466667,0.705882}%
\pgfsetstrokecolor{currentstroke}%
\pgfsetdash{}{0pt}%
\pgfpathmoveto{\pgfqpoint{2.479339in}{0.435000in}}%
\pgfpathlineto{\pgfqpoint{2.479339in}{2.117118in}}%
\pgfusepath{stroke}%
\end{pgfscope}%
\begin{pgfscope}%
\pgfpathrectangle{\pgfqpoint{0.750000in}{0.330000in}}{\pgfqpoint{2.113636in}{2.310000in}}%
\pgfusepath{clip}%
\pgfsetbuttcap%
\pgfsetroundjoin%
\pgfsetlinewidth{2.007500pt}%
\definecolor{currentstroke}{rgb}{0.121569,0.466667,0.705882}%
\pgfsetstrokecolor{currentstroke}%
\pgfsetdash{}{0pt}%
\pgfpathmoveto{\pgfqpoint{2.498554in}{0.435000in}}%
\pgfpathlineto{\pgfqpoint{2.498554in}{2.105901in}}%
\pgfusepath{stroke}%
\end{pgfscope}%
\begin{pgfscope}%
\pgfpathrectangle{\pgfqpoint{0.750000in}{0.330000in}}{\pgfqpoint{2.113636in}{2.310000in}}%
\pgfusepath{clip}%
\pgfsetbuttcap%
\pgfsetroundjoin%
\pgfsetlinewidth{2.007500pt}%
\definecolor{currentstroke}{rgb}{0.121569,0.466667,0.705882}%
\pgfsetstrokecolor{currentstroke}%
\pgfsetdash{}{0pt}%
\pgfpathmoveto{\pgfqpoint{2.517769in}{0.435000in}}%
\pgfpathlineto{\pgfqpoint{2.517769in}{2.093890in}}%
\pgfusepath{stroke}%
\end{pgfscope}%
\begin{pgfscope}%
\pgfpathrectangle{\pgfqpoint{0.750000in}{0.330000in}}{\pgfqpoint{2.113636in}{2.310000in}}%
\pgfusepath{clip}%
\pgfsetbuttcap%
\pgfsetroundjoin%
\pgfsetlinewidth{2.007500pt}%
\definecolor{currentstroke}{rgb}{0.121569,0.466667,0.705882}%
\pgfsetstrokecolor{currentstroke}%
\pgfsetdash{}{0pt}%
\pgfpathmoveto{\pgfqpoint{2.536983in}{0.435000in}}%
\pgfpathlineto{\pgfqpoint{2.536983in}{2.082587in}}%
\pgfusepath{stroke}%
\end{pgfscope}%
\begin{pgfscope}%
\pgfpathrectangle{\pgfqpoint{0.750000in}{0.330000in}}{\pgfqpoint{2.113636in}{2.310000in}}%
\pgfusepath{clip}%
\pgfsetbuttcap%
\pgfsetroundjoin%
\pgfsetlinewidth{2.007500pt}%
\definecolor{currentstroke}{rgb}{0.121569,0.466667,0.705882}%
\pgfsetstrokecolor{currentstroke}%
\pgfsetdash{}{0pt}%
\pgfpathmoveto{\pgfqpoint{2.556198in}{0.435000in}}%
\pgfpathlineto{\pgfqpoint{2.556198in}{2.076326in}}%
\pgfusepath{stroke}%
\end{pgfscope}%
\begin{pgfscope}%
\pgfpathrectangle{\pgfqpoint{0.750000in}{0.330000in}}{\pgfqpoint{2.113636in}{2.310000in}}%
\pgfusepath{clip}%
\pgfsetbuttcap%
\pgfsetroundjoin%
\pgfsetlinewidth{2.007500pt}%
\definecolor{currentstroke}{rgb}{0.121569,0.466667,0.705882}%
\pgfsetstrokecolor{currentstroke}%
\pgfsetdash{}{0pt}%
\pgfpathmoveto{\pgfqpoint{2.575413in}{0.435000in}}%
\pgfpathlineto{\pgfqpoint{2.575413in}{2.063177in}}%
\pgfusepath{stroke}%
\end{pgfscope}%
\begin{pgfscope}%
\pgfpathrectangle{\pgfqpoint{0.750000in}{0.330000in}}{\pgfqpoint{2.113636in}{2.310000in}}%
\pgfusepath{clip}%
\pgfsetbuttcap%
\pgfsetroundjoin%
\pgfsetlinewidth{2.007500pt}%
\definecolor{currentstroke}{rgb}{0.121569,0.466667,0.705882}%
\pgfsetstrokecolor{currentstroke}%
\pgfsetdash{}{0pt}%
\pgfpathmoveto{\pgfqpoint{2.594628in}{0.435000in}}%
\pgfpathlineto{\pgfqpoint{2.594628in}{2.047153in}}%
\pgfusepath{stroke}%
\end{pgfscope}%
\begin{pgfscope}%
\pgfpathrectangle{\pgfqpoint{0.750000in}{0.330000in}}{\pgfqpoint{2.113636in}{2.310000in}}%
\pgfusepath{clip}%
\pgfsetbuttcap%
\pgfsetroundjoin%
\pgfsetlinewidth{2.007500pt}%
\definecolor{currentstroke}{rgb}{0.121569,0.466667,0.705882}%
\pgfsetstrokecolor{currentstroke}%
\pgfsetdash{}{0pt}%
\pgfpathmoveto{\pgfqpoint{2.613843in}{0.435000in}}%
\pgfpathlineto{\pgfqpoint{2.613843in}{2.036108in}}%
\pgfusepath{stroke}%
\end{pgfscope}%
\begin{pgfscope}%
\pgfpathrectangle{\pgfqpoint{0.750000in}{0.330000in}}{\pgfqpoint{2.113636in}{2.310000in}}%
\pgfusepath{clip}%
\pgfsetbuttcap%
\pgfsetroundjoin%
\pgfsetlinewidth{2.007500pt}%
\definecolor{currentstroke}{rgb}{0.121569,0.466667,0.705882}%
\pgfsetstrokecolor{currentstroke}%
\pgfsetdash{}{0pt}%
\pgfpathmoveto{\pgfqpoint{2.633058in}{0.435000in}}%
\pgfpathlineto{\pgfqpoint{2.633058in}{2.028968in}}%
\pgfusepath{stroke}%
\end{pgfscope}%
\begin{pgfscope}%
\pgfpathrectangle{\pgfqpoint{0.750000in}{0.330000in}}{\pgfqpoint{2.113636in}{2.310000in}}%
\pgfusepath{clip}%
\pgfsetbuttcap%
\pgfsetroundjoin%
\pgfsetlinewidth{2.007500pt}%
\definecolor{currentstroke}{rgb}{0.121569,0.466667,0.705882}%
\pgfsetstrokecolor{currentstroke}%
\pgfsetdash{}{0pt}%
\pgfpathmoveto{\pgfqpoint{2.652273in}{0.435000in}}%
\pgfpathlineto{\pgfqpoint{2.652273in}{2.021922in}}%
\pgfusepath{stroke}%
\end{pgfscope}%
\begin{pgfscope}%
\pgfpathrectangle{\pgfqpoint{0.750000in}{0.330000in}}{\pgfqpoint{2.113636in}{2.310000in}}%
\pgfusepath{clip}%
\pgfsetbuttcap%
\pgfsetroundjoin%
\pgfsetlinewidth{2.007500pt}%
\definecolor{currentstroke}{rgb}{0.121569,0.466667,0.705882}%
\pgfsetstrokecolor{currentstroke}%
\pgfsetdash{}{0pt}%
\pgfpathmoveto{\pgfqpoint{2.671488in}{0.435000in}}%
\pgfpathlineto{\pgfqpoint{2.671488in}{2.006927in}}%
\pgfusepath{stroke}%
\end{pgfscope}%
\begin{pgfscope}%
\pgfpathrectangle{\pgfqpoint{0.750000in}{0.330000in}}{\pgfqpoint{2.113636in}{2.310000in}}%
\pgfusepath{clip}%
\pgfsetbuttcap%
\pgfsetroundjoin%
\pgfsetlinewidth{2.007500pt}%
\definecolor{currentstroke}{rgb}{0.121569,0.466667,0.705882}%
\pgfsetstrokecolor{currentstroke}%
\pgfsetdash{}{0pt}%
\pgfpathmoveto{\pgfqpoint{2.690702in}{0.435000in}}%
\pgfpathlineto{\pgfqpoint{2.690702in}{1.998805in}}%
\pgfusepath{stroke}%
\end{pgfscope}%
\begin{pgfscope}%
\pgfpathrectangle{\pgfqpoint{0.750000in}{0.330000in}}{\pgfqpoint{2.113636in}{2.310000in}}%
\pgfusepath{clip}%
\pgfsetbuttcap%
\pgfsetroundjoin%
\pgfsetlinewidth{2.007500pt}%
\definecolor{currentstroke}{rgb}{0.121569,0.466667,0.705882}%
\pgfsetstrokecolor{currentstroke}%
\pgfsetdash{}{0pt}%
\pgfpathmoveto{\pgfqpoint{2.709917in}{0.435000in}}%
\pgfpathlineto{\pgfqpoint{2.709917in}{1.985820in}}%
\pgfusepath{stroke}%
\end{pgfscope}%
\begin{pgfscope}%
\pgfpathrectangle{\pgfqpoint{0.750000in}{0.330000in}}{\pgfqpoint{2.113636in}{2.310000in}}%
\pgfusepath{clip}%
\pgfsetbuttcap%
\pgfsetroundjoin%
\pgfsetlinewidth{2.007500pt}%
\definecolor{currentstroke}{rgb}{0.121569,0.466667,0.705882}%
\pgfsetstrokecolor{currentstroke}%
\pgfsetdash{}{0pt}%
\pgfpathmoveto{\pgfqpoint{2.729132in}{0.435000in}}%
\pgfpathlineto{\pgfqpoint{2.729132in}{1.973417in}}%
\pgfusepath{stroke}%
\end{pgfscope}%
\begin{pgfscope}%
\pgfpathrectangle{\pgfqpoint{0.750000in}{0.330000in}}{\pgfqpoint{2.113636in}{2.310000in}}%
\pgfusepath{clip}%
\pgfsetbuttcap%
\pgfsetroundjoin%
\pgfsetlinewidth{2.007500pt}%
\definecolor{currentstroke}{rgb}{0.121569,0.466667,0.705882}%
\pgfsetstrokecolor{currentstroke}%
\pgfsetdash{}{0pt}%
\pgfpathmoveto{\pgfqpoint{2.748347in}{0.435000in}}%
\pgfpathlineto{\pgfqpoint{2.748347in}{1.962216in}}%
\pgfusepath{stroke}%
\end{pgfscope}%
\begin{pgfscope}%
\pgfpathrectangle{\pgfqpoint{0.750000in}{0.330000in}}{\pgfqpoint{2.113636in}{2.310000in}}%
\pgfusepath{clip}%
\pgfsetbuttcap%
\pgfsetroundjoin%
\pgfsetlinewidth{2.007500pt}%
\definecolor{currentstroke}{rgb}{0.121569,0.466667,0.705882}%
\pgfsetstrokecolor{currentstroke}%
\pgfsetdash{}{0pt}%
\pgfpathmoveto{\pgfqpoint{2.767562in}{0.435000in}}%
\pgfpathlineto{\pgfqpoint{2.767562in}{1.948320in}}%
\pgfusepath{stroke}%
\end{pgfscope}%
\begin{pgfscope}%
\pgfpathrectangle{\pgfqpoint{0.750000in}{0.330000in}}{\pgfqpoint{2.113636in}{2.310000in}}%
\pgfusepath{clip}%
\pgfsetrectcap%
\pgfsetroundjoin%
\pgfsetlinewidth{2.007500pt}%
\definecolor{currentstroke}{rgb}{0.121569,0.466667,0.705882}%
\pgfsetstrokecolor{currentstroke}%
\pgfsetdash{}{0pt}%
\pgfpathmoveto{\pgfqpoint{0.750000in}{0.435000in}}%
\pgfpathlineto{\pgfqpoint{2.863636in}{0.435000in}}%
\pgfusepath{stroke}%
\end{pgfscope}%
\begin{pgfscope}%
\pgfsetrectcap%
\pgfsetmiterjoin%
\pgfsetlinewidth{0.803000pt}%
\definecolor{currentstroke}{rgb}{0.000000,0.000000,0.000000}%
\pgfsetstrokecolor{currentstroke}%
\pgfsetdash{}{0pt}%
\pgfpathmoveto{\pgfqpoint{0.750000in}{0.330000in}}%
\pgfpathlineto{\pgfqpoint{0.750000in}{2.640000in}}%
\pgfusepath{stroke}%
\end{pgfscope}%
\begin{pgfscope}%
\pgfsetrectcap%
\pgfsetmiterjoin%
\pgfsetlinewidth{0.803000pt}%
\definecolor{currentstroke}{rgb}{0.000000,0.000000,0.000000}%
\pgfsetstrokecolor{currentstroke}%
\pgfsetdash{}{0pt}%
\pgfpathmoveto{\pgfqpoint{2.863636in}{0.330000in}}%
\pgfpathlineto{\pgfqpoint{2.863636in}{2.640000in}}%
\pgfusepath{stroke}%
\end{pgfscope}%
\begin{pgfscope}%
\pgfsetrectcap%
\pgfsetmiterjoin%
\pgfsetlinewidth{0.803000pt}%
\definecolor{currentstroke}{rgb}{0.000000,0.000000,0.000000}%
\pgfsetstrokecolor{currentstroke}%
\pgfsetdash{}{0pt}%
\pgfpathmoveto{\pgfqpoint{0.750000in}{0.330000in}}%
\pgfpathlineto{\pgfqpoint{2.863636in}{0.330000in}}%
\pgfusepath{stroke}%
\end{pgfscope}%
\begin{pgfscope}%
\pgfsetrectcap%
\pgfsetmiterjoin%
\pgfsetlinewidth{0.803000pt}%
\definecolor{currentstroke}{rgb}{0.000000,0.000000,0.000000}%
\pgfsetstrokecolor{currentstroke}%
\pgfsetdash{}{0pt}%
\pgfpathmoveto{\pgfqpoint{0.750000in}{2.640000in}}%
\pgfpathlineto{\pgfqpoint{2.863636in}{2.640000in}}%
\pgfusepath{stroke}%
\end{pgfscope}%
\begin{pgfscope}%
\definecolor{textcolor}{rgb}{0.000000,0.000000,0.000000}%
\pgfsetstrokecolor{textcolor}%
\pgfsetfillcolor{textcolor}%
\pgftext[x=1.806818in,y=2.723333in,,base]{\color{textcolor}\rmfamily\fontsize{12.000000}{14.400000}\selectfont Cross-correlation between sepal length and width}%
\end{pgfscope}%
\begin{pgfscope}%
\pgfsetbuttcap%
\pgfsetmiterjoin%
\definecolor{currentfill}{rgb}{1.000000,1.000000,1.000000}%
\pgfsetfillcolor{currentfill}%
\pgfsetlinewidth{0.000000pt}%
\definecolor{currentstroke}{rgb}{0.000000,0.000000,0.000000}%
\pgfsetstrokecolor{currentstroke}%
\pgfsetstrokeopacity{0.000000}%
\pgfsetdash{}{0pt}%
\pgfpathmoveto{\pgfqpoint{3.286364in}{0.330000in}}%
\pgfpathlineto{\pgfqpoint{5.400000in}{0.330000in}}%
\pgfpathlineto{\pgfqpoint{5.400000in}{2.640000in}}%
\pgfpathlineto{\pgfqpoint{3.286364in}{2.640000in}}%
\pgfpathlineto{\pgfqpoint{3.286364in}{0.330000in}}%
\pgfpathclose%
\pgfusepath{fill}%
\end{pgfscope}%
\begin{pgfscope}%
\pgfpathrectangle{\pgfqpoint{3.286364in}{0.330000in}}{\pgfqpoint{2.113636in}{2.310000in}}%
\pgfusepath{clip}%
\pgfsetrectcap%
\pgfsetroundjoin%
\pgfsetlinewidth{0.803000pt}%
\definecolor{currentstroke}{rgb}{0.690196,0.690196,0.690196}%
\pgfsetstrokecolor{currentstroke}%
\pgfsetdash{}{0pt}%
\pgfpathmoveto{\pgfqpoint{3.382438in}{0.330000in}}%
\pgfpathlineto{\pgfqpoint{3.382438in}{2.640000in}}%
\pgfusepath{stroke}%
\end{pgfscope}%
\begin{pgfscope}%
\pgfsetbuttcap%
\pgfsetroundjoin%
\definecolor{currentfill}{rgb}{0.000000,0.000000,0.000000}%
\pgfsetfillcolor{currentfill}%
\pgfsetlinewidth{0.803000pt}%
\definecolor{currentstroke}{rgb}{0.000000,0.000000,0.000000}%
\pgfsetstrokecolor{currentstroke}%
\pgfsetdash{}{0pt}%
\pgfsys@defobject{currentmarker}{\pgfqpoint{0.000000in}{-0.048611in}}{\pgfqpoint{0.000000in}{0.000000in}}{%
\pgfpathmoveto{\pgfqpoint{0.000000in}{0.000000in}}%
\pgfpathlineto{\pgfqpoint{0.000000in}{-0.048611in}}%
\pgfusepath{stroke,fill}%
}%
\begin{pgfscope}%
\pgfsys@transformshift{3.382438in}{0.330000in}%
\pgfsys@useobject{currentmarker}{}%
\end{pgfscope}%
\end{pgfscope}%
\begin{pgfscope}%
\definecolor{textcolor}{rgb}{0.000000,0.000000,0.000000}%
\pgfsetstrokecolor{textcolor}%
\pgfsetfillcolor{textcolor}%
\pgftext[x=3.382438in,y=0.232778in,,top]{\color{textcolor}\rmfamily\fontsize{10.000000}{12.000000}\selectfont \(\displaystyle {\ensuremath{-}50}\)}%
\end{pgfscope}%
\begin{pgfscope}%
\pgfpathrectangle{\pgfqpoint{3.286364in}{0.330000in}}{\pgfqpoint{2.113636in}{2.310000in}}%
\pgfusepath{clip}%
\pgfsetrectcap%
\pgfsetroundjoin%
\pgfsetlinewidth{0.803000pt}%
\definecolor{currentstroke}{rgb}{0.690196,0.690196,0.690196}%
\pgfsetstrokecolor{currentstroke}%
\pgfsetdash{}{0pt}%
\pgfpathmoveto{\pgfqpoint{3.862810in}{0.330000in}}%
\pgfpathlineto{\pgfqpoint{3.862810in}{2.640000in}}%
\pgfusepath{stroke}%
\end{pgfscope}%
\begin{pgfscope}%
\pgfsetbuttcap%
\pgfsetroundjoin%
\definecolor{currentfill}{rgb}{0.000000,0.000000,0.000000}%
\pgfsetfillcolor{currentfill}%
\pgfsetlinewidth{0.803000pt}%
\definecolor{currentstroke}{rgb}{0.000000,0.000000,0.000000}%
\pgfsetstrokecolor{currentstroke}%
\pgfsetdash{}{0pt}%
\pgfsys@defobject{currentmarker}{\pgfqpoint{0.000000in}{-0.048611in}}{\pgfqpoint{0.000000in}{0.000000in}}{%
\pgfpathmoveto{\pgfqpoint{0.000000in}{0.000000in}}%
\pgfpathlineto{\pgfqpoint{0.000000in}{-0.048611in}}%
\pgfusepath{stroke,fill}%
}%
\begin{pgfscope}%
\pgfsys@transformshift{3.862810in}{0.330000in}%
\pgfsys@useobject{currentmarker}{}%
\end{pgfscope}%
\end{pgfscope}%
\begin{pgfscope}%
\definecolor{textcolor}{rgb}{0.000000,0.000000,0.000000}%
\pgfsetstrokecolor{textcolor}%
\pgfsetfillcolor{textcolor}%
\pgftext[x=3.862810in,y=0.232778in,,top]{\color{textcolor}\rmfamily\fontsize{10.000000}{12.000000}\selectfont \(\displaystyle {\ensuremath{-}25}\)}%
\end{pgfscope}%
\begin{pgfscope}%
\pgfpathrectangle{\pgfqpoint{3.286364in}{0.330000in}}{\pgfqpoint{2.113636in}{2.310000in}}%
\pgfusepath{clip}%
\pgfsetrectcap%
\pgfsetroundjoin%
\pgfsetlinewidth{0.803000pt}%
\definecolor{currentstroke}{rgb}{0.690196,0.690196,0.690196}%
\pgfsetstrokecolor{currentstroke}%
\pgfsetdash{}{0pt}%
\pgfpathmoveto{\pgfqpoint{4.343182in}{0.330000in}}%
\pgfpathlineto{\pgfqpoint{4.343182in}{2.640000in}}%
\pgfusepath{stroke}%
\end{pgfscope}%
\begin{pgfscope}%
\pgfsetbuttcap%
\pgfsetroundjoin%
\definecolor{currentfill}{rgb}{0.000000,0.000000,0.000000}%
\pgfsetfillcolor{currentfill}%
\pgfsetlinewidth{0.803000pt}%
\definecolor{currentstroke}{rgb}{0.000000,0.000000,0.000000}%
\pgfsetstrokecolor{currentstroke}%
\pgfsetdash{}{0pt}%
\pgfsys@defobject{currentmarker}{\pgfqpoint{0.000000in}{-0.048611in}}{\pgfqpoint{0.000000in}{0.000000in}}{%
\pgfpathmoveto{\pgfqpoint{0.000000in}{0.000000in}}%
\pgfpathlineto{\pgfqpoint{0.000000in}{-0.048611in}}%
\pgfusepath{stroke,fill}%
}%
\begin{pgfscope}%
\pgfsys@transformshift{4.343182in}{0.330000in}%
\pgfsys@useobject{currentmarker}{}%
\end{pgfscope}%
\end{pgfscope}%
\begin{pgfscope}%
\definecolor{textcolor}{rgb}{0.000000,0.000000,0.000000}%
\pgfsetstrokecolor{textcolor}%
\pgfsetfillcolor{textcolor}%
\pgftext[x=4.343182in,y=0.232778in,,top]{\color{textcolor}\rmfamily\fontsize{10.000000}{12.000000}\selectfont \(\displaystyle {0}\)}%
\end{pgfscope}%
\begin{pgfscope}%
\pgfpathrectangle{\pgfqpoint{3.286364in}{0.330000in}}{\pgfqpoint{2.113636in}{2.310000in}}%
\pgfusepath{clip}%
\pgfsetrectcap%
\pgfsetroundjoin%
\pgfsetlinewidth{0.803000pt}%
\definecolor{currentstroke}{rgb}{0.690196,0.690196,0.690196}%
\pgfsetstrokecolor{currentstroke}%
\pgfsetdash{}{0pt}%
\pgfpathmoveto{\pgfqpoint{4.823554in}{0.330000in}}%
\pgfpathlineto{\pgfqpoint{4.823554in}{2.640000in}}%
\pgfusepath{stroke}%
\end{pgfscope}%
\begin{pgfscope}%
\pgfsetbuttcap%
\pgfsetroundjoin%
\definecolor{currentfill}{rgb}{0.000000,0.000000,0.000000}%
\pgfsetfillcolor{currentfill}%
\pgfsetlinewidth{0.803000pt}%
\definecolor{currentstroke}{rgb}{0.000000,0.000000,0.000000}%
\pgfsetstrokecolor{currentstroke}%
\pgfsetdash{}{0pt}%
\pgfsys@defobject{currentmarker}{\pgfqpoint{0.000000in}{-0.048611in}}{\pgfqpoint{0.000000in}{0.000000in}}{%
\pgfpathmoveto{\pgfqpoint{0.000000in}{0.000000in}}%
\pgfpathlineto{\pgfqpoint{0.000000in}{-0.048611in}}%
\pgfusepath{stroke,fill}%
}%
\begin{pgfscope}%
\pgfsys@transformshift{4.823554in}{0.330000in}%
\pgfsys@useobject{currentmarker}{}%
\end{pgfscope}%
\end{pgfscope}%
\begin{pgfscope}%
\definecolor{textcolor}{rgb}{0.000000,0.000000,0.000000}%
\pgfsetstrokecolor{textcolor}%
\pgfsetfillcolor{textcolor}%
\pgftext[x=4.823554in,y=0.232778in,,top]{\color{textcolor}\rmfamily\fontsize{10.000000}{12.000000}\selectfont \(\displaystyle {25}\)}%
\end{pgfscope}%
\begin{pgfscope}%
\pgfpathrectangle{\pgfqpoint{3.286364in}{0.330000in}}{\pgfqpoint{2.113636in}{2.310000in}}%
\pgfusepath{clip}%
\pgfsetrectcap%
\pgfsetroundjoin%
\pgfsetlinewidth{0.803000pt}%
\definecolor{currentstroke}{rgb}{0.690196,0.690196,0.690196}%
\pgfsetstrokecolor{currentstroke}%
\pgfsetdash{}{0pt}%
\pgfpathmoveto{\pgfqpoint{5.303926in}{0.330000in}}%
\pgfpathlineto{\pgfqpoint{5.303926in}{2.640000in}}%
\pgfusepath{stroke}%
\end{pgfscope}%
\begin{pgfscope}%
\pgfsetbuttcap%
\pgfsetroundjoin%
\definecolor{currentfill}{rgb}{0.000000,0.000000,0.000000}%
\pgfsetfillcolor{currentfill}%
\pgfsetlinewidth{0.803000pt}%
\definecolor{currentstroke}{rgb}{0.000000,0.000000,0.000000}%
\pgfsetstrokecolor{currentstroke}%
\pgfsetdash{}{0pt}%
\pgfsys@defobject{currentmarker}{\pgfqpoint{0.000000in}{-0.048611in}}{\pgfqpoint{0.000000in}{0.000000in}}{%
\pgfpathmoveto{\pgfqpoint{0.000000in}{0.000000in}}%
\pgfpathlineto{\pgfqpoint{0.000000in}{-0.048611in}}%
\pgfusepath{stroke,fill}%
}%
\begin{pgfscope}%
\pgfsys@transformshift{5.303926in}{0.330000in}%
\pgfsys@useobject{currentmarker}{}%
\end{pgfscope}%
\end{pgfscope}%
\begin{pgfscope}%
\definecolor{textcolor}{rgb}{0.000000,0.000000,0.000000}%
\pgfsetstrokecolor{textcolor}%
\pgfsetfillcolor{textcolor}%
\pgftext[x=5.303926in,y=0.232778in,,top]{\color{textcolor}\rmfamily\fontsize{10.000000}{12.000000}\selectfont \(\displaystyle {50}\)}%
\end{pgfscope}%
\begin{pgfscope}%
\pgfpathrectangle{\pgfqpoint{3.286364in}{0.330000in}}{\pgfqpoint{2.113636in}{2.310000in}}%
\pgfusepath{clip}%
\pgfsetrectcap%
\pgfsetroundjoin%
\pgfsetlinewidth{0.803000pt}%
\definecolor{currentstroke}{rgb}{0.690196,0.690196,0.690196}%
\pgfsetstrokecolor{currentstroke}%
\pgfsetdash{}{0pt}%
\pgfpathmoveto{\pgfqpoint{3.286364in}{0.435000in}}%
\pgfpathlineto{\pgfqpoint{5.400000in}{0.435000in}}%
\pgfusepath{stroke}%
\end{pgfscope}%
\begin{pgfscope}%
\pgfsetbuttcap%
\pgfsetroundjoin%
\definecolor{currentfill}{rgb}{0.000000,0.000000,0.000000}%
\pgfsetfillcolor{currentfill}%
\pgfsetlinewidth{0.803000pt}%
\definecolor{currentstroke}{rgb}{0.000000,0.000000,0.000000}%
\pgfsetstrokecolor{currentstroke}%
\pgfsetdash{}{0pt}%
\pgfsys@defobject{currentmarker}{\pgfqpoint{-0.048611in}{0.000000in}}{\pgfqpoint{-0.000000in}{0.000000in}}{%
\pgfpathmoveto{\pgfqpoint{-0.000000in}{0.000000in}}%
\pgfpathlineto{\pgfqpoint{-0.048611in}{0.000000in}}%
\pgfusepath{stroke,fill}%
}%
\begin{pgfscope}%
\pgfsys@transformshift{3.286364in}{0.435000in}%
\pgfsys@useobject{currentmarker}{}%
\end{pgfscope}%
\end{pgfscope}%
\begin{pgfscope}%
\definecolor{textcolor}{rgb}{0.000000,0.000000,0.000000}%
\pgfsetstrokecolor{textcolor}%
\pgfsetfillcolor{textcolor}%
\pgftext[x=3.011672in, y=0.386775in, left, base]{\color{textcolor}\rmfamily\fontsize{10.000000}{12.000000}\selectfont \(\displaystyle {0.0}\)}%
\end{pgfscope}%
\begin{pgfscope}%
\pgfpathrectangle{\pgfqpoint{3.286364in}{0.330000in}}{\pgfqpoint{2.113636in}{2.310000in}}%
\pgfusepath{clip}%
\pgfsetrectcap%
\pgfsetroundjoin%
\pgfsetlinewidth{0.803000pt}%
\definecolor{currentstroke}{rgb}{0.690196,0.690196,0.690196}%
\pgfsetstrokecolor{currentstroke}%
\pgfsetdash{}{0pt}%
\pgfpathmoveto{\pgfqpoint{3.286364in}{0.862025in}}%
\pgfpathlineto{\pgfqpoint{5.400000in}{0.862025in}}%
\pgfusepath{stroke}%
\end{pgfscope}%
\begin{pgfscope}%
\pgfsetbuttcap%
\pgfsetroundjoin%
\definecolor{currentfill}{rgb}{0.000000,0.000000,0.000000}%
\pgfsetfillcolor{currentfill}%
\pgfsetlinewidth{0.803000pt}%
\definecolor{currentstroke}{rgb}{0.000000,0.000000,0.000000}%
\pgfsetstrokecolor{currentstroke}%
\pgfsetdash{}{0pt}%
\pgfsys@defobject{currentmarker}{\pgfqpoint{-0.048611in}{0.000000in}}{\pgfqpoint{-0.000000in}{0.000000in}}{%
\pgfpathmoveto{\pgfqpoint{-0.000000in}{0.000000in}}%
\pgfpathlineto{\pgfqpoint{-0.048611in}{0.000000in}}%
\pgfusepath{stroke,fill}%
}%
\begin{pgfscope}%
\pgfsys@transformshift{3.286364in}{0.862025in}%
\pgfsys@useobject{currentmarker}{}%
\end{pgfscope}%
\end{pgfscope}%
\begin{pgfscope}%
\definecolor{textcolor}{rgb}{0.000000,0.000000,0.000000}%
\pgfsetstrokecolor{textcolor}%
\pgfsetfillcolor{textcolor}%
\pgftext[x=3.011672in, y=0.813799in, left, base]{\color{textcolor}\rmfamily\fontsize{10.000000}{12.000000}\selectfont \(\displaystyle {0.2}\)}%
\end{pgfscope}%
\begin{pgfscope}%
\pgfpathrectangle{\pgfqpoint{3.286364in}{0.330000in}}{\pgfqpoint{2.113636in}{2.310000in}}%
\pgfusepath{clip}%
\pgfsetrectcap%
\pgfsetroundjoin%
\pgfsetlinewidth{0.803000pt}%
\definecolor{currentstroke}{rgb}{0.690196,0.690196,0.690196}%
\pgfsetstrokecolor{currentstroke}%
\pgfsetdash{}{0pt}%
\pgfpathmoveto{\pgfqpoint{3.286364in}{1.289049in}}%
\pgfpathlineto{\pgfqpoint{5.400000in}{1.289049in}}%
\pgfusepath{stroke}%
\end{pgfscope}%
\begin{pgfscope}%
\pgfsetbuttcap%
\pgfsetroundjoin%
\definecolor{currentfill}{rgb}{0.000000,0.000000,0.000000}%
\pgfsetfillcolor{currentfill}%
\pgfsetlinewidth{0.803000pt}%
\definecolor{currentstroke}{rgb}{0.000000,0.000000,0.000000}%
\pgfsetstrokecolor{currentstroke}%
\pgfsetdash{}{0pt}%
\pgfsys@defobject{currentmarker}{\pgfqpoint{-0.048611in}{0.000000in}}{\pgfqpoint{-0.000000in}{0.000000in}}{%
\pgfpathmoveto{\pgfqpoint{-0.000000in}{0.000000in}}%
\pgfpathlineto{\pgfqpoint{-0.048611in}{0.000000in}}%
\pgfusepath{stroke,fill}%
}%
\begin{pgfscope}%
\pgfsys@transformshift{3.286364in}{1.289049in}%
\pgfsys@useobject{currentmarker}{}%
\end{pgfscope}%
\end{pgfscope}%
\begin{pgfscope}%
\definecolor{textcolor}{rgb}{0.000000,0.000000,0.000000}%
\pgfsetstrokecolor{textcolor}%
\pgfsetfillcolor{textcolor}%
\pgftext[x=3.011672in, y=1.240824in, left, base]{\color{textcolor}\rmfamily\fontsize{10.000000}{12.000000}\selectfont \(\displaystyle {0.4}\)}%
\end{pgfscope}%
\begin{pgfscope}%
\pgfpathrectangle{\pgfqpoint{3.286364in}{0.330000in}}{\pgfqpoint{2.113636in}{2.310000in}}%
\pgfusepath{clip}%
\pgfsetrectcap%
\pgfsetroundjoin%
\pgfsetlinewidth{0.803000pt}%
\definecolor{currentstroke}{rgb}{0.690196,0.690196,0.690196}%
\pgfsetstrokecolor{currentstroke}%
\pgfsetdash{}{0pt}%
\pgfpathmoveto{\pgfqpoint{3.286364in}{1.716074in}}%
\pgfpathlineto{\pgfqpoint{5.400000in}{1.716074in}}%
\pgfusepath{stroke}%
\end{pgfscope}%
\begin{pgfscope}%
\pgfsetbuttcap%
\pgfsetroundjoin%
\definecolor{currentfill}{rgb}{0.000000,0.000000,0.000000}%
\pgfsetfillcolor{currentfill}%
\pgfsetlinewidth{0.803000pt}%
\definecolor{currentstroke}{rgb}{0.000000,0.000000,0.000000}%
\pgfsetstrokecolor{currentstroke}%
\pgfsetdash{}{0pt}%
\pgfsys@defobject{currentmarker}{\pgfqpoint{-0.048611in}{0.000000in}}{\pgfqpoint{-0.000000in}{0.000000in}}{%
\pgfpathmoveto{\pgfqpoint{-0.000000in}{0.000000in}}%
\pgfpathlineto{\pgfqpoint{-0.048611in}{0.000000in}}%
\pgfusepath{stroke,fill}%
}%
\begin{pgfscope}%
\pgfsys@transformshift{3.286364in}{1.716074in}%
\pgfsys@useobject{currentmarker}{}%
\end{pgfscope}%
\end{pgfscope}%
\begin{pgfscope}%
\definecolor{textcolor}{rgb}{0.000000,0.000000,0.000000}%
\pgfsetstrokecolor{textcolor}%
\pgfsetfillcolor{textcolor}%
\pgftext[x=3.011672in, y=1.667849in, left, base]{\color{textcolor}\rmfamily\fontsize{10.000000}{12.000000}\selectfont \(\displaystyle {0.6}\)}%
\end{pgfscope}%
\begin{pgfscope}%
\pgfpathrectangle{\pgfqpoint{3.286364in}{0.330000in}}{\pgfqpoint{2.113636in}{2.310000in}}%
\pgfusepath{clip}%
\pgfsetrectcap%
\pgfsetroundjoin%
\pgfsetlinewidth{0.803000pt}%
\definecolor{currentstroke}{rgb}{0.690196,0.690196,0.690196}%
\pgfsetstrokecolor{currentstroke}%
\pgfsetdash{}{0pt}%
\pgfpathmoveto{\pgfqpoint{3.286364in}{2.143099in}}%
\pgfpathlineto{\pgfqpoint{5.400000in}{2.143099in}}%
\pgfusepath{stroke}%
\end{pgfscope}%
\begin{pgfscope}%
\pgfsetbuttcap%
\pgfsetroundjoin%
\definecolor{currentfill}{rgb}{0.000000,0.000000,0.000000}%
\pgfsetfillcolor{currentfill}%
\pgfsetlinewidth{0.803000pt}%
\definecolor{currentstroke}{rgb}{0.000000,0.000000,0.000000}%
\pgfsetstrokecolor{currentstroke}%
\pgfsetdash{}{0pt}%
\pgfsys@defobject{currentmarker}{\pgfqpoint{-0.048611in}{0.000000in}}{\pgfqpoint{-0.000000in}{0.000000in}}{%
\pgfpathmoveto{\pgfqpoint{-0.000000in}{0.000000in}}%
\pgfpathlineto{\pgfqpoint{-0.048611in}{0.000000in}}%
\pgfusepath{stroke,fill}%
}%
\begin{pgfscope}%
\pgfsys@transformshift{3.286364in}{2.143099in}%
\pgfsys@useobject{currentmarker}{}%
\end{pgfscope}%
\end{pgfscope}%
\begin{pgfscope}%
\definecolor{textcolor}{rgb}{0.000000,0.000000,0.000000}%
\pgfsetstrokecolor{textcolor}%
\pgfsetfillcolor{textcolor}%
\pgftext[x=3.011672in, y=2.094873in, left, base]{\color{textcolor}\rmfamily\fontsize{10.000000}{12.000000}\selectfont \(\displaystyle {0.8}\)}%
\end{pgfscope}%
\begin{pgfscope}%
\pgfpathrectangle{\pgfqpoint{3.286364in}{0.330000in}}{\pgfqpoint{2.113636in}{2.310000in}}%
\pgfusepath{clip}%
\pgfsetrectcap%
\pgfsetroundjoin%
\pgfsetlinewidth{0.803000pt}%
\definecolor{currentstroke}{rgb}{0.690196,0.690196,0.690196}%
\pgfsetstrokecolor{currentstroke}%
\pgfsetdash{}{0pt}%
\pgfpathmoveto{\pgfqpoint{3.286364in}{2.570123in}}%
\pgfpathlineto{\pgfqpoint{5.400000in}{2.570123in}}%
\pgfusepath{stroke}%
\end{pgfscope}%
\begin{pgfscope}%
\pgfsetbuttcap%
\pgfsetroundjoin%
\definecolor{currentfill}{rgb}{0.000000,0.000000,0.000000}%
\pgfsetfillcolor{currentfill}%
\pgfsetlinewidth{0.803000pt}%
\definecolor{currentstroke}{rgb}{0.000000,0.000000,0.000000}%
\pgfsetstrokecolor{currentstroke}%
\pgfsetdash{}{0pt}%
\pgfsys@defobject{currentmarker}{\pgfqpoint{-0.048611in}{0.000000in}}{\pgfqpoint{-0.000000in}{0.000000in}}{%
\pgfpathmoveto{\pgfqpoint{-0.000000in}{0.000000in}}%
\pgfpathlineto{\pgfqpoint{-0.048611in}{0.000000in}}%
\pgfusepath{stroke,fill}%
}%
\begin{pgfscope}%
\pgfsys@transformshift{3.286364in}{2.570123in}%
\pgfsys@useobject{currentmarker}{}%
\end{pgfscope}%
\end{pgfscope}%
\begin{pgfscope}%
\definecolor{textcolor}{rgb}{0.000000,0.000000,0.000000}%
\pgfsetstrokecolor{textcolor}%
\pgfsetfillcolor{textcolor}%
\pgftext[x=3.011672in, y=2.521898in, left, base]{\color{textcolor}\rmfamily\fontsize{10.000000}{12.000000}\selectfont \(\displaystyle {1.0}\)}%
\end{pgfscope}%
\begin{pgfscope}%
\pgfpathrectangle{\pgfqpoint{3.286364in}{0.330000in}}{\pgfqpoint{2.113636in}{2.310000in}}%
\pgfusepath{clip}%
\pgfsetbuttcap%
\pgfsetroundjoin%
\pgfsetlinewidth{2.007500pt}%
\definecolor{currentstroke}{rgb}{0.121569,0.466667,0.705882}%
\pgfsetstrokecolor{currentstroke}%
\pgfsetdash{}{0pt}%
\pgfpathmoveto{\pgfqpoint{3.382438in}{0.435000in}}%
\pgfpathlineto{\pgfqpoint{3.382438in}{1.709870in}}%
\pgfusepath{stroke}%
\end{pgfscope}%
\begin{pgfscope}%
\pgfpathrectangle{\pgfqpoint{3.286364in}{0.330000in}}{\pgfqpoint{2.113636in}{2.310000in}}%
\pgfusepath{clip}%
\pgfsetbuttcap%
\pgfsetroundjoin%
\pgfsetlinewidth{2.007500pt}%
\definecolor{currentstroke}{rgb}{0.121569,0.466667,0.705882}%
\pgfsetstrokecolor{currentstroke}%
\pgfsetdash{}{0pt}%
\pgfpathmoveto{\pgfqpoint{3.401653in}{0.435000in}}%
\pgfpathlineto{\pgfqpoint{3.401653in}{1.727339in}}%
\pgfusepath{stroke}%
\end{pgfscope}%
\begin{pgfscope}%
\pgfpathrectangle{\pgfqpoint{3.286364in}{0.330000in}}{\pgfqpoint{2.113636in}{2.310000in}}%
\pgfusepath{clip}%
\pgfsetbuttcap%
\pgfsetroundjoin%
\pgfsetlinewidth{2.007500pt}%
\definecolor{currentstroke}{rgb}{0.121569,0.466667,0.705882}%
\pgfsetstrokecolor{currentstroke}%
\pgfsetdash{}{0pt}%
\pgfpathmoveto{\pgfqpoint{3.420868in}{0.435000in}}%
\pgfpathlineto{\pgfqpoint{3.420868in}{1.742175in}}%
\pgfusepath{stroke}%
\end{pgfscope}%
\begin{pgfscope}%
\pgfpathrectangle{\pgfqpoint{3.286364in}{0.330000in}}{\pgfqpoint{2.113636in}{2.310000in}}%
\pgfusepath{clip}%
\pgfsetbuttcap%
\pgfsetroundjoin%
\pgfsetlinewidth{2.007500pt}%
\definecolor{currentstroke}{rgb}{0.121569,0.466667,0.705882}%
\pgfsetstrokecolor{currentstroke}%
\pgfsetdash{}{0pt}%
\pgfpathmoveto{\pgfqpoint{3.440083in}{0.435000in}}%
\pgfpathlineto{\pgfqpoint{3.440083in}{1.754957in}}%
\pgfusepath{stroke}%
\end{pgfscope}%
\begin{pgfscope}%
\pgfpathrectangle{\pgfqpoint{3.286364in}{0.330000in}}{\pgfqpoint{2.113636in}{2.310000in}}%
\pgfusepath{clip}%
\pgfsetbuttcap%
\pgfsetroundjoin%
\pgfsetlinewidth{2.007500pt}%
\definecolor{currentstroke}{rgb}{0.121569,0.466667,0.705882}%
\pgfsetstrokecolor{currentstroke}%
\pgfsetdash{}{0pt}%
\pgfpathmoveto{\pgfqpoint{3.459298in}{0.435000in}}%
\pgfpathlineto{\pgfqpoint{3.459298in}{1.774408in}}%
\pgfusepath{stroke}%
\end{pgfscope}%
\begin{pgfscope}%
\pgfpathrectangle{\pgfqpoint{3.286364in}{0.330000in}}{\pgfqpoint{2.113636in}{2.310000in}}%
\pgfusepath{clip}%
\pgfsetbuttcap%
\pgfsetroundjoin%
\pgfsetlinewidth{2.007500pt}%
\definecolor{currentstroke}{rgb}{0.121569,0.466667,0.705882}%
\pgfsetstrokecolor{currentstroke}%
\pgfsetdash{}{0pt}%
\pgfpathmoveto{\pgfqpoint{3.478512in}{0.435000in}}%
\pgfpathlineto{\pgfqpoint{3.478512in}{1.792506in}}%
\pgfusepath{stroke}%
\end{pgfscope}%
\begin{pgfscope}%
\pgfpathrectangle{\pgfqpoint{3.286364in}{0.330000in}}{\pgfqpoint{2.113636in}{2.310000in}}%
\pgfusepath{clip}%
\pgfsetbuttcap%
\pgfsetroundjoin%
\pgfsetlinewidth{2.007500pt}%
\definecolor{currentstroke}{rgb}{0.121569,0.466667,0.705882}%
\pgfsetstrokecolor{currentstroke}%
\pgfsetdash{}{0pt}%
\pgfpathmoveto{\pgfqpoint{3.497727in}{0.435000in}}%
\pgfpathlineto{\pgfqpoint{3.497727in}{1.815436in}}%
\pgfusepath{stroke}%
\end{pgfscope}%
\begin{pgfscope}%
\pgfpathrectangle{\pgfqpoint{3.286364in}{0.330000in}}{\pgfqpoint{2.113636in}{2.310000in}}%
\pgfusepath{clip}%
\pgfsetbuttcap%
\pgfsetroundjoin%
\pgfsetlinewidth{2.007500pt}%
\definecolor{currentstroke}{rgb}{0.121569,0.466667,0.705882}%
\pgfsetstrokecolor{currentstroke}%
\pgfsetdash{}{0pt}%
\pgfpathmoveto{\pgfqpoint{3.516942in}{0.435000in}}%
\pgfpathlineto{\pgfqpoint{3.516942in}{1.819955in}}%
\pgfusepath{stroke}%
\end{pgfscope}%
\begin{pgfscope}%
\pgfpathrectangle{\pgfqpoint{3.286364in}{0.330000in}}{\pgfqpoint{2.113636in}{2.310000in}}%
\pgfusepath{clip}%
\pgfsetbuttcap%
\pgfsetroundjoin%
\pgfsetlinewidth{2.007500pt}%
\definecolor{currentstroke}{rgb}{0.121569,0.466667,0.705882}%
\pgfsetstrokecolor{currentstroke}%
\pgfsetdash{}{0pt}%
\pgfpathmoveto{\pgfqpoint{3.536157in}{0.435000in}}%
\pgfpathlineto{\pgfqpoint{3.536157in}{1.840420in}}%
\pgfusepath{stroke}%
\end{pgfscope}%
\begin{pgfscope}%
\pgfpathrectangle{\pgfqpoint{3.286364in}{0.330000in}}{\pgfqpoint{2.113636in}{2.310000in}}%
\pgfusepath{clip}%
\pgfsetbuttcap%
\pgfsetroundjoin%
\pgfsetlinewidth{2.007500pt}%
\definecolor{currentstroke}{rgb}{0.121569,0.466667,0.705882}%
\pgfsetstrokecolor{currentstroke}%
\pgfsetdash{}{0pt}%
\pgfpathmoveto{\pgfqpoint{3.555372in}{0.435000in}}%
\pgfpathlineto{\pgfqpoint{3.555372in}{1.865477in}}%
\pgfusepath{stroke}%
\end{pgfscope}%
\begin{pgfscope}%
\pgfpathrectangle{\pgfqpoint{3.286364in}{0.330000in}}{\pgfqpoint{2.113636in}{2.310000in}}%
\pgfusepath{clip}%
\pgfsetbuttcap%
\pgfsetroundjoin%
\pgfsetlinewidth{2.007500pt}%
\definecolor{currentstroke}{rgb}{0.121569,0.466667,0.705882}%
\pgfsetstrokecolor{currentstroke}%
\pgfsetdash{}{0pt}%
\pgfpathmoveto{\pgfqpoint{3.574587in}{0.435000in}}%
\pgfpathlineto{\pgfqpoint{3.574587in}{1.883817in}}%
\pgfusepath{stroke}%
\end{pgfscope}%
\begin{pgfscope}%
\pgfpathrectangle{\pgfqpoint{3.286364in}{0.330000in}}{\pgfqpoint{2.113636in}{2.310000in}}%
\pgfusepath{clip}%
\pgfsetbuttcap%
\pgfsetroundjoin%
\pgfsetlinewidth{2.007500pt}%
\definecolor{currentstroke}{rgb}{0.121569,0.466667,0.705882}%
\pgfsetstrokecolor{currentstroke}%
\pgfsetdash{}{0pt}%
\pgfpathmoveto{\pgfqpoint{3.593802in}{0.435000in}}%
\pgfpathlineto{\pgfqpoint{3.593802in}{1.899063in}}%
\pgfusepath{stroke}%
\end{pgfscope}%
\begin{pgfscope}%
\pgfpathrectangle{\pgfqpoint{3.286364in}{0.330000in}}{\pgfqpoint{2.113636in}{2.310000in}}%
\pgfusepath{clip}%
\pgfsetbuttcap%
\pgfsetroundjoin%
\pgfsetlinewidth{2.007500pt}%
\definecolor{currentstroke}{rgb}{0.121569,0.466667,0.705882}%
\pgfsetstrokecolor{currentstroke}%
\pgfsetdash{}{0pt}%
\pgfpathmoveto{\pgfqpoint{3.613017in}{0.435000in}}%
\pgfpathlineto{\pgfqpoint{3.613017in}{1.908994in}}%
\pgfusepath{stroke}%
\end{pgfscope}%
\begin{pgfscope}%
\pgfpathrectangle{\pgfqpoint{3.286364in}{0.330000in}}{\pgfqpoint{2.113636in}{2.310000in}}%
\pgfusepath{clip}%
\pgfsetbuttcap%
\pgfsetroundjoin%
\pgfsetlinewidth{2.007500pt}%
\definecolor{currentstroke}{rgb}{0.121569,0.466667,0.705882}%
\pgfsetstrokecolor{currentstroke}%
\pgfsetdash{}{0pt}%
\pgfpathmoveto{\pgfqpoint{3.632231in}{0.435000in}}%
\pgfpathlineto{\pgfqpoint{3.632231in}{1.927068in}}%
\pgfusepath{stroke}%
\end{pgfscope}%
\begin{pgfscope}%
\pgfpathrectangle{\pgfqpoint{3.286364in}{0.330000in}}{\pgfqpoint{2.113636in}{2.310000in}}%
\pgfusepath{clip}%
\pgfsetbuttcap%
\pgfsetroundjoin%
\pgfsetlinewidth{2.007500pt}%
\definecolor{currentstroke}{rgb}{0.121569,0.466667,0.705882}%
\pgfsetstrokecolor{currentstroke}%
\pgfsetdash{}{0pt}%
\pgfpathmoveto{\pgfqpoint{3.651446in}{0.435000in}}%
\pgfpathlineto{\pgfqpoint{3.651446in}{1.945673in}}%
\pgfusepath{stroke}%
\end{pgfscope}%
\begin{pgfscope}%
\pgfpathrectangle{\pgfqpoint{3.286364in}{0.330000in}}{\pgfqpoint{2.113636in}{2.310000in}}%
\pgfusepath{clip}%
\pgfsetbuttcap%
\pgfsetroundjoin%
\pgfsetlinewidth{2.007500pt}%
\definecolor{currentstroke}{rgb}{0.121569,0.466667,0.705882}%
\pgfsetstrokecolor{currentstroke}%
\pgfsetdash{}{0pt}%
\pgfpathmoveto{\pgfqpoint{3.670661in}{0.435000in}}%
\pgfpathlineto{\pgfqpoint{3.670661in}{1.960944in}}%
\pgfusepath{stroke}%
\end{pgfscope}%
\begin{pgfscope}%
\pgfpathrectangle{\pgfqpoint{3.286364in}{0.330000in}}{\pgfqpoint{2.113636in}{2.310000in}}%
\pgfusepath{clip}%
\pgfsetbuttcap%
\pgfsetroundjoin%
\pgfsetlinewidth{2.007500pt}%
\definecolor{currentstroke}{rgb}{0.121569,0.466667,0.705882}%
\pgfsetstrokecolor{currentstroke}%
\pgfsetdash{}{0pt}%
\pgfpathmoveto{\pgfqpoint{3.689876in}{0.435000in}}%
\pgfpathlineto{\pgfqpoint{3.689876in}{1.973218in}}%
\pgfusepath{stroke}%
\end{pgfscope}%
\begin{pgfscope}%
\pgfpathrectangle{\pgfqpoint{3.286364in}{0.330000in}}{\pgfqpoint{2.113636in}{2.310000in}}%
\pgfusepath{clip}%
\pgfsetbuttcap%
\pgfsetroundjoin%
\pgfsetlinewidth{2.007500pt}%
\definecolor{currentstroke}{rgb}{0.121569,0.466667,0.705882}%
\pgfsetstrokecolor{currentstroke}%
\pgfsetdash{}{0pt}%
\pgfpathmoveto{\pgfqpoint{3.709091in}{0.435000in}}%
\pgfpathlineto{\pgfqpoint{3.709091in}{1.989915in}}%
\pgfusepath{stroke}%
\end{pgfscope}%
\begin{pgfscope}%
\pgfpathrectangle{\pgfqpoint{3.286364in}{0.330000in}}{\pgfqpoint{2.113636in}{2.310000in}}%
\pgfusepath{clip}%
\pgfsetbuttcap%
\pgfsetroundjoin%
\pgfsetlinewidth{2.007500pt}%
\definecolor{currentstroke}{rgb}{0.121569,0.466667,0.705882}%
\pgfsetstrokecolor{currentstroke}%
\pgfsetdash{}{0pt}%
\pgfpathmoveto{\pgfqpoint{3.728306in}{0.435000in}}%
\pgfpathlineto{\pgfqpoint{3.728306in}{2.014367in}}%
\pgfusepath{stroke}%
\end{pgfscope}%
\begin{pgfscope}%
\pgfpathrectangle{\pgfqpoint{3.286364in}{0.330000in}}{\pgfqpoint{2.113636in}{2.310000in}}%
\pgfusepath{clip}%
\pgfsetbuttcap%
\pgfsetroundjoin%
\pgfsetlinewidth{2.007500pt}%
\definecolor{currentstroke}{rgb}{0.121569,0.466667,0.705882}%
\pgfsetstrokecolor{currentstroke}%
\pgfsetdash{}{0pt}%
\pgfpathmoveto{\pgfqpoint{3.747521in}{0.435000in}}%
\pgfpathlineto{\pgfqpoint{3.747521in}{2.033190in}}%
\pgfusepath{stroke}%
\end{pgfscope}%
\begin{pgfscope}%
\pgfpathrectangle{\pgfqpoint{3.286364in}{0.330000in}}{\pgfqpoint{2.113636in}{2.310000in}}%
\pgfusepath{clip}%
\pgfsetbuttcap%
\pgfsetroundjoin%
\pgfsetlinewidth{2.007500pt}%
\definecolor{currentstroke}{rgb}{0.121569,0.466667,0.705882}%
\pgfsetstrokecolor{currentstroke}%
\pgfsetdash{}{0pt}%
\pgfpathmoveto{\pgfqpoint{3.766736in}{0.435000in}}%
\pgfpathlineto{\pgfqpoint{3.766736in}{2.040004in}}%
\pgfusepath{stroke}%
\end{pgfscope}%
\begin{pgfscope}%
\pgfpathrectangle{\pgfqpoint{3.286364in}{0.330000in}}{\pgfqpoint{2.113636in}{2.310000in}}%
\pgfusepath{clip}%
\pgfsetbuttcap%
\pgfsetroundjoin%
\pgfsetlinewidth{2.007500pt}%
\definecolor{currentstroke}{rgb}{0.121569,0.466667,0.705882}%
\pgfsetstrokecolor{currentstroke}%
\pgfsetdash{}{0pt}%
\pgfpathmoveto{\pgfqpoint{3.785950in}{0.435000in}}%
\pgfpathlineto{\pgfqpoint{3.785950in}{2.049959in}}%
\pgfusepath{stroke}%
\end{pgfscope}%
\begin{pgfscope}%
\pgfpathrectangle{\pgfqpoint{3.286364in}{0.330000in}}{\pgfqpoint{2.113636in}{2.310000in}}%
\pgfusepath{clip}%
\pgfsetbuttcap%
\pgfsetroundjoin%
\pgfsetlinewidth{2.007500pt}%
\definecolor{currentstroke}{rgb}{0.121569,0.466667,0.705882}%
\pgfsetstrokecolor{currentstroke}%
\pgfsetdash{}{0pt}%
\pgfpathmoveto{\pgfqpoint{3.805165in}{0.435000in}}%
\pgfpathlineto{\pgfqpoint{3.805165in}{2.071633in}}%
\pgfusepath{stroke}%
\end{pgfscope}%
\begin{pgfscope}%
\pgfpathrectangle{\pgfqpoint{3.286364in}{0.330000in}}{\pgfqpoint{2.113636in}{2.310000in}}%
\pgfusepath{clip}%
\pgfsetbuttcap%
\pgfsetroundjoin%
\pgfsetlinewidth{2.007500pt}%
\definecolor{currentstroke}{rgb}{0.121569,0.466667,0.705882}%
\pgfsetstrokecolor{currentstroke}%
\pgfsetdash{}{0pt}%
\pgfpathmoveto{\pgfqpoint{3.824380in}{0.435000in}}%
\pgfpathlineto{\pgfqpoint{3.824380in}{2.096206in}}%
\pgfusepath{stroke}%
\end{pgfscope}%
\begin{pgfscope}%
\pgfpathrectangle{\pgfqpoint{3.286364in}{0.330000in}}{\pgfqpoint{2.113636in}{2.310000in}}%
\pgfusepath{clip}%
\pgfsetbuttcap%
\pgfsetroundjoin%
\pgfsetlinewidth{2.007500pt}%
\definecolor{currentstroke}{rgb}{0.121569,0.466667,0.705882}%
\pgfsetstrokecolor{currentstroke}%
\pgfsetdash{}{0pt}%
\pgfpathmoveto{\pgfqpoint{3.843595in}{0.435000in}}%
\pgfpathlineto{\pgfqpoint{3.843595in}{2.107345in}}%
\pgfusepath{stroke}%
\end{pgfscope}%
\begin{pgfscope}%
\pgfpathrectangle{\pgfqpoint{3.286364in}{0.330000in}}{\pgfqpoint{2.113636in}{2.310000in}}%
\pgfusepath{clip}%
\pgfsetbuttcap%
\pgfsetroundjoin%
\pgfsetlinewidth{2.007500pt}%
\definecolor{currentstroke}{rgb}{0.121569,0.466667,0.705882}%
\pgfsetstrokecolor{currentstroke}%
\pgfsetdash{}{0pt}%
\pgfpathmoveto{\pgfqpoint{3.862810in}{0.435000in}}%
\pgfpathlineto{\pgfqpoint{3.862810in}{2.116527in}}%
\pgfusepath{stroke}%
\end{pgfscope}%
\begin{pgfscope}%
\pgfpathrectangle{\pgfqpoint{3.286364in}{0.330000in}}{\pgfqpoint{2.113636in}{2.310000in}}%
\pgfusepath{clip}%
\pgfsetbuttcap%
\pgfsetroundjoin%
\pgfsetlinewidth{2.007500pt}%
\definecolor{currentstroke}{rgb}{0.121569,0.466667,0.705882}%
\pgfsetstrokecolor{currentstroke}%
\pgfsetdash{}{0pt}%
\pgfpathmoveto{\pgfqpoint{3.882025in}{0.435000in}}%
\pgfpathlineto{\pgfqpoint{3.882025in}{2.137549in}}%
\pgfusepath{stroke}%
\end{pgfscope}%
\begin{pgfscope}%
\pgfpathrectangle{\pgfqpoint{3.286364in}{0.330000in}}{\pgfqpoint{2.113636in}{2.310000in}}%
\pgfusepath{clip}%
\pgfsetbuttcap%
\pgfsetroundjoin%
\pgfsetlinewidth{2.007500pt}%
\definecolor{currentstroke}{rgb}{0.121569,0.466667,0.705882}%
\pgfsetstrokecolor{currentstroke}%
\pgfsetdash{}{0pt}%
\pgfpathmoveto{\pgfqpoint{3.901240in}{0.435000in}}%
\pgfpathlineto{\pgfqpoint{3.901240in}{2.152409in}}%
\pgfusepath{stroke}%
\end{pgfscope}%
\begin{pgfscope}%
\pgfpathrectangle{\pgfqpoint{3.286364in}{0.330000in}}{\pgfqpoint{2.113636in}{2.310000in}}%
\pgfusepath{clip}%
\pgfsetbuttcap%
\pgfsetroundjoin%
\pgfsetlinewidth{2.007500pt}%
\definecolor{currentstroke}{rgb}{0.121569,0.466667,0.705882}%
\pgfsetstrokecolor{currentstroke}%
\pgfsetdash{}{0pt}%
\pgfpathmoveto{\pgfqpoint{3.920455in}{0.435000in}}%
\pgfpathlineto{\pgfqpoint{3.920455in}{2.160310in}}%
\pgfusepath{stroke}%
\end{pgfscope}%
\begin{pgfscope}%
\pgfpathrectangle{\pgfqpoint{3.286364in}{0.330000in}}{\pgfqpoint{2.113636in}{2.310000in}}%
\pgfusepath{clip}%
\pgfsetbuttcap%
\pgfsetroundjoin%
\pgfsetlinewidth{2.007500pt}%
\definecolor{currentstroke}{rgb}{0.121569,0.466667,0.705882}%
\pgfsetstrokecolor{currentstroke}%
\pgfsetdash{}{0pt}%
\pgfpathmoveto{\pgfqpoint{3.939669in}{0.435000in}}%
\pgfpathlineto{\pgfqpoint{3.939669in}{2.174638in}}%
\pgfusepath{stroke}%
\end{pgfscope}%
\begin{pgfscope}%
\pgfpathrectangle{\pgfqpoint{3.286364in}{0.330000in}}{\pgfqpoint{2.113636in}{2.310000in}}%
\pgfusepath{clip}%
\pgfsetbuttcap%
\pgfsetroundjoin%
\pgfsetlinewidth{2.007500pt}%
\definecolor{currentstroke}{rgb}{0.121569,0.466667,0.705882}%
\pgfsetstrokecolor{currentstroke}%
\pgfsetdash{}{0pt}%
\pgfpathmoveto{\pgfqpoint{3.958884in}{0.435000in}}%
\pgfpathlineto{\pgfqpoint{3.958884in}{2.197230in}}%
\pgfusepath{stroke}%
\end{pgfscope}%
\begin{pgfscope}%
\pgfpathrectangle{\pgfqpoint{3.286364in}{0.330000in}}{\pgfqpoint{2.113636in}{2.310000in}}%
\pgfusepath{clip}%
\pgfsetbuttcap%
\pgfsetroundjoin%
\pgfsetlinewidth{2.007500pt}%
\definecolor{currentstroke}{rgb}{0.121569,0.466667,0.705882}%
\pgfsetstrokecolor{currentstroke}%
\pgfsetdash{}{0pt}%
\pgfpathmoveto{\pgfqpoint{3.978099in}{0.435000in}}%
\pgfpathlineto{\pgfqpoint{3.978099in}{2.217020in}}%
\pgfusepath{stroke}%
\end{pgfscope}%
\begin{pgfscope}%
\pgfpathrectangle{\pgfqpoint{3.286364in}{0.330000in}}{\pgfqpoint{2.113636in}{2.310000in}}%
\pgfusepath{clip}%
\pgfsetbuttcap%
\pgfsetroundjoin%
\pgfsetlinewidth{2.007500pt}%
\definecolor{currentstroke}{rgb}{0.121569,0.466667,0.705882}%
\pgfsetstrokecolor{currentstroke}%
\pgfsetdash{}{0pt}%
\pgfpathmoveto{\pgfqpoint{3.997314in}{0.435000in}}%
\pgfpathlineto{\pgfqpoint{3.997314in}{2.241279in}}%
\pgfusepath{stroke}%
\end{pgfscope}%
\begin{pgfscope}%
\pgfpathrectangle{\pgfqpoint{3.286364in}{0.330000in}}{\pgfqpoint{2.113636in}{2.310000in}}%
\pgfusepath{clip}%
\pgfsetbuttcap%
\pgfsetroundjoin%
\pgfsetlinewidth{2.007500pt}%
\definecolor{currentstroke}{rgb}{0.121569,0.466667,0.705882}%
\pgfsetstrokecolor{currentstroke}%
\pgfsetdash{}{0pt}%
\pgfpathmoveto{\pgfqpoint{4.016529in}{0.435000in}}%
\pgfpathlineto{\pgfqpoint{4.016529in}{2.250606in}}%
\pgfusepath{stroke}%
\end{pgfscope}%
\begin{pgfscope}%
\pgfpathrectangle{\pgfqpoint{3.286364in}{0.330000in}}{\pgfqpoint{2.113636in}{2.310000in}}%
\pgfusepath{clip}%
\pgfsetbuttcap%
\pgfsetroundjoin%
\pgfsetlinewidth{2.007500pt}%
\definecolor{currentstroke}{rgb}{0.121569,0.466667,0.705882}%
\pgfsetstrokecolor{currentstroke}%
\pgfsetdash{}{0pt}%
\pgfpathmoveto{\pgfqpoint{4.035744in}{0.435000in}}%
\pgfpathlineto{\pgfqpoint{4.035744in}{2.262325in}}%
\pgfusepath{stroke}%
\end{pgfscope}%
\begin{pgfscope}%
\pgfpathrectangle{\pgfqpoint{3.286364in}{0.330000in}}{\pgfqpoint{2.113636in}{2.310000in}}%
\pgfusepath{clip}%
\pgfsetbuttcap%
\pgfsetroundjoin%
\pgfsetlinewidth{2.007500pt}%
\definecolor{currentstroke}{rgb}{0.121569,0.466667,0.705882}%
\pgfsetstrokecolor{currentstroke}%
\pgfsetdash{}{0pt}%
\pgfpathmoveto{\pgfqpoint{4.054959in}{0.435000in}}%
\pgfpathlineto{\pgfqpoint{4.054959in}{2.286366in}}%
\pgfusepath{stroke}%
\end{pgfscope}%
\begin{pgfscope}%
\pgfpathrectangle{\pgfqpoint{3.286364in}{0.330000in}}{\pgfqpoint{2.113636in}{2.310000in}}%
\pgfusepath{clip}%
\pgfsetbuttcap%
\pgfsetroundjoin%
\pgfsetlinewidth{2.007500pt}%
\definecolor{currentstroke}{rgb}{0.121569,0.466667,0.705882}%
\pgfsetstrokecolor{currentstroke}%
\pgfsetdash{}{0pt}%
\pgfpathmoveto{\pgfqpoint{4.074174in}{0.435000in}}%
\pgfpathlineto{\pgfqpoint{4.074174in}{2.308451in}}%
\pgfusepath{stroke}%
\end{pgfscope}%
\begin{pgfscope}%
\pgfpathrectangle{\pgfqpoint{3.286364in}{0.330000in}}{\pgfqpoint{2.113636in}{2.310000in}}%
\pgfusepath{clip}%
\pgfsetbuttcap%
\pgfsetroundjoin%
\pgfsetlinewidth{2.007500pt}%
\definecolor{currentstroke}{rgb}{0.121569,0.466667,0.705882}%
\pgfsetstrokecolor{currentstroke}%
\pgfsetdash{}{0pt}%
\pgfpathmoveto{\pgfqpoint{4.093388in}{0.435000in}}%
\pgfpathlineto{\pgfqpoint{4.093388in}{2.326114in}}%
\pgfusepath{stroke}%
\end{pgfscope}%
\begin{pgfscope}%
\pgfpathrectangle{\pgfqpoint{3.286364in}{0.330000in}}{\pgfqpoint{2.113636in}{2.310000in}}%
\pgfusepath{clip}%
\pgfsetbuttcap%
\pgfsetroundjoin%
\pgfsetlinewidth{2.007500pt}%
\definecolor{currentstroke}{rgb}{0.121569,0.466667,0.705882}%
\pgfsetstrokecolor{currentstroke}%
\pgfsetdash{}{0pt}%
\pgfpathmoveto{\pgfqpoint{4.112603in}{0.435000in}}%
\pgfpathlineto{\pgfqpoint{4.112603in}{2.333749in}}%
\pgfusepath{stroke}%
\end{pgfscope}%
\begin{pgfscope}%
\pgfpathrectangle{\pgfqpoint{3.286364in}{0.330000in}}{\pgfqpoint{2.113636in}{2.310000in}}%
\pgfusepath{clip}%
\pgfsetbuttcap%
\pgfsetroundjoin%
\pgfsetlinewidth{2.007500pt}%
\definecolor{currentstroke}{rgb}{0.121569,0.466667,0.705882}%
\pgfsetstrokecolor{currentstroke}%
\pgfsetdash{}{0pt}%
\pgfpathmoveto{\pgfqpoint{4.131818in}{0.435000in}}%
\pgfpathlineto{\pgfqpoint{4.131818in}{2.346314in}}%
\pgfusepath{stroke}%
\end{pgfscope}%
\begin{pgfscope}%
\pgfpathrectangle{\pgfqpoint{3.286364in}{0.330000in}}{\pgfqpoint{2.113636in}{2.310000in}}%
\pgfusepath{clip}%
\pgfsetbuttcap%
\pgfsetroundjoin%
\pgfsetlinewidth{2.007500pt}%
\definecolor{currentstroke}{rgb}{0.121569,0.466667,0.705882}%
\pgfsetstrokecolor{currentstroke}%
\pgfsetdash{}{0pt}%
\pgfpathmoveto{\pgfqpoint{4.151033in}{0.435000in}}%
\pgfpathlineto{\pgfqpoint{4.151033in}{2.366441in}}%
\pgfusepath{stroke}%
\end{pgfscope}%
\begin{pgfscope}%
\pgfpathrectangle{\pgfqpoint{3.286364in}{0.330000in}}{\pgfqpoint{2.113636in}{2.310000in}}%
\pgfusepath{clip}%
\pgfsetbuttcap%
\pgfsetroundjoin%
\pgfsetlinewidth{2.007500pt}%
\definecolor{currentstroke}{rgb}{0.121569,0.466667,0.705882}%
\pgfsetstrokecolor{currentstroke}%
\pgfsetdash{}{0pt}%
\pgfpathmoveto{\pgfqpoint{4.170248in}{0.435000in}}%
\pgfpathlineto{\pgfqpoint{4.170248in}{2.379827in}}%
\pgfusepath{stroke}%
\end{pgfscope}%
\begin{pgfscope}%
\pgfpathrectangle{\pgfqpoint{3.286364in}{0.330000in}}{\pgfqpoint{2.113636in}{2.310000in}}%
\pgfusepath{clip}%
\pgfsetbuttcap%
\pgfsetroundjoin%
\pgfsetlinewidth{2.007500pt}%
\definecolor{currentstroke}{rgb}{0.121569,0.466667,0.705882}%
\pgfsetstrokecolor{currentstroke}%
\pgfsetdash{}{0pt}%
\pgfpathmoveto{\pgfqpoint{4.189463in}{0.435000in}}%
\pgfpathlineto{\pgfqpoint{4.189463in}{2.391812in}}%
\pgfusepath{stroke}%
\end{pgfscope}%
\begin{pgfscope}%
\pgfpathrectangle{\pgfqpoint{3.286364in}{0.330000in}}{\pgfqpoint{2.113636in}{2.310000in}}%
\pgfusepath{clip}%
\pgfsetbuttcap%
\pgfsetroundjoin%
\pgfsetlinewidth{2.007500pt}%
\definecolor{currentstroke}{rgb}{0.121569,0.466667,0.705882}%
\pgfsetstrokecolor{currentstroke}%
\pgfsetdash{}{0pt}%
\pgfpathmoveto{\pgfqpoint{4.208678in}{0.435000in}}%
\pgfpathlineto{\pgfqpoint{4.208678in}{2.403797in}}%
\pgfusepath{stroke}%
\end{pgfscope}%
\begin{pgfscope}%
\pgfpathrectangle{\pgfqpoint{3.286364in}{0.330000in}}{\pgfqpoint{2.113636in}{2.310000in}}%
\pgfusepath{clip}%
\pgfsetbuttcap%
\pgfsetroundjoin%
\pgfsetlinewidth{2.007500pt}%
\definecolor{currentstroke}{rgb}{0.121569,0.466667,0.705882}%
\pgfsetstrokecolor{currentstroke}%
\pgfsetdash{}{0pt}%
\pgfpathmoveto{\pgfqpoint{4.227893in}{0.435000in}}%
\pgfpathlineto{\pgfqpoint{4.227893in}{2.422426in}}%
\pgfusepath{stroke}%
\end{pgfscope}%
\begin{pgfscope}%
\pgfpathrectangle{\pgfqpoint{3.286364in}{0.330000in}}{\pgfqpoint{2.113636in}{2.310000in}}%
\pgfusepath{clip}%
\pgfsetbuttcap%
\pgfsetroundjoin%
\pgfsetlinewidth{2.007500pt}%
\definecolor{currentstroke}{rgb}{0.121569,0.466667,0.705882}%
\pgfsetstrokecolor{currentstroke}%
\pgfsetdash{}{0pt}%
\pgfpathmoveto{\pgfqpoint{4.247107in}{0.435000in}}%
\pgfpathlineto{\pgfqpoint{4.247107in}{2.443279in}}%
\pgfusepath{stroke}%
\end{pgfscope}%
\begin{pgfscope}%
\pgfpathrectangle{\pgfqpoint{3.286364in}{0.330000in}}{\pgfqpoint{2.113636in}{2.310000in}}%
\pgfusepath{clip}%
\pgfsetbuttcap%
\pgfsetroundjoin%
\pgfsetlinewidth{2.007500pt}%
\definecolor{currentstroke}{rgb}{0.121569,0.466667,0.705882}%
\pgfsetstrokecolor{currentstroke}%
\pgfsetdash{}{0pt}%
\pgfpathmoveto{\pgfqpoint{4.266322in}{0.435000in}}%
\pgfpathlineto{\pgfqpoint{4.266322in}{2.454732in}}%
\pgfusepath{stroke}%
\end{pgfscope}%
\begin{pgfscope}%
\pgfpathrectangle{\pgfqpoint{3.286364in}{0.330000in}}{\pgfqpoint{2.113636in}{2.310000in}}%
\pgfusepath{clip}%
\pgfsetbuttcap%
\pgfsetroundjoin%
\pgfsetlinewidth{2.007500pt}%
\definecolor{currentstroke}{rgb}{0.121569,0.466667,0.705882}%
\pgfsetstrokecolor{currentstroke}%
\pgfsetdash{}{0pt}%
\pgfpathmoveto{\pgfqpoint{4.285537in}{0.435000in}}%
\pgfpathlineto{\pgfqpoint{4.285537in}{2.464421in}}%
\pgfusepath{stroke}%
\end{pgfscope}%
\begin{pgfscope}%
\pgfpathrectangle{\pgfqpoint{3.286364in}{0.330000in}}{\pgfqpoint{2.113636in}{2.310000in}}%
\pgfusepath{clip}%
\pgfsetbuttcap%
\pgfsetroundjoin%
\pgfsetlinewidth{2.007500pt}%
\definecolor{currentstroke}{rgb}{0.121569,0.466667,0.705882}%
\pgfsetstrokecolor{currentstroke}%
\pgfsetdash{}{0pt}%
\pgfpathmoveto{\pgfqpoint{4.304752in}{0.435000in}}%
\pgfpathlineto{\pgfqpoint{4.304752in}{2.483461in}}%
\pgfusepath{stroke}%
\end{pgfscope}%
\begin{pgfscope}%
\pgfpathrectangle{\pgfqpoint{3.286364in}{0.330000in}}{\pgfqpoint{2.113636in}{2.310000in}}%
\pgfusepath{clip}%
\pgfsetbuttcap%
\pgfsetroundjoin%
\pgfsetlinewidth{2.007500pt}%
\definecolor{currentstroke}{rgb}{0.121569,0.466667,0.705882}%
\pgfsetstrokecolor{currentstroke}%
\pgfsetdash{}{0pt}%
\pgfpathmoveto{\pgfqpoint{4.323967in}{0.435000in}}%
\pgfpathlineto{\pgfqpoint{4.323967in}{2.502380in}}%
\pgfusepath{stroke}%
\end{pgfscope}%
\begin{pgfscope}%
\pgfpathrectangle{\pgfqpoint{3.286364in}{0.330000in}}{\pgfqpoint{2.113636in}{2.310000in}}%
\pgfusepath{clip}%
\pgfsetbuttcap%
\pgfsetroundjoin%
\pgfsetlinewidth{2.007500pt}%
\definecolor{currentstroke}{rgb}{0.121569,0.466667,0.705882}%
\pgfsetstrokecolor{currentstroke}%
\pgfsetdash{}{0pt}%
\pgfpathmoveto{\pgfqpoint{4.343182in}{0.435000in}}%
\pgfpathlineto{\pgfqpoint{4.343182in}{2.535000in}}%
\pgfusepath{stroke}%
\end{pgfscope}%
\begin{pgfscope}%
\pgfpathrectangle{\pgfqpoint{3.286364in}{0.330000in}}{\pgfqpoint{2.113636in}{2.310000in}}%
\pgfusepath{clip}%
\pgfsetbuttcap%
\pgfsetroundjoin%
\pgfsetlinewidth{2.007500pt}%
\definecolor{currentstroke}{rgb}{0.121569,0.466667,0.705882}%
\pgfsetstrokecolor{currentstroke}%
\pgfsetdash{}{0pt}%
\pgfpathmoveto{\pgfqpoint{4.362397in}{0.435000in}}%
\pgfpathlineto{\pgfqpoint{4.362397in}{2.493803in}}%
\pgfusepath{stroke}%
\end{pgfscope}%
\begin{pgfscope}%
\pgfpathrectangle{\pgfqpoint{3.286364in}{0.330000in}}{\pgfqpoint{2.113636in}{2.310000in}}%
\pgfusepath{clip}%
\pgfsetbuttcap%
\pgfsetroundjoin%
\pgfsetlinewidth{2.007500pt}%
\definecolor{currentstroke}{rgb}{0.121569,0.466667,0.705882}%
\pgfsetstrokecolor{currentstroke}%
\pgfsetdash{}{0pt}%
\pgfpathmoveto{\pgfqpoint{4.381612in}{0.435000in}}%
\pgfpathlineto{\pgfqpoint{4.381612in}{2.473071in}}%
\pgfusepath{stroke}%
\end{pgfscope}%
\begin{pgfscope}%
\pgfpathrectangle{\pgfqpoint{3.286364in}{0.330000in}}{\pgfqpoint{2.113636in}{2.310000in}}%
\pgfusepath{clip}%
\pgfsetbuttcap%
\pgfsetroundjoin%
\pgfsetlinewidth{2.007500pt}%
\definecolor{currentstroke}{rgb}{0.121569,0.466667,0.705882}%
\pgfsetstrokecolor{currentstroke}%
\pgfsetdash{}{0pt}%
\pgfpathmoveto{\pgfqpoint{4.400826in}{0.435000in}}%
\pgfpathlineto{\pgfqpoint{4.400826in}{2.449730in}}%
\pgfusepath{stroke}%
\end{pgfscope}%
\begin{pgfscope}%
\pgfpathrectangle{\pgfqpoint{3.286364in}{0.330000in}}{\pgfqpoint{2.113636in}{2.310000in}}%
\pgfusepath{clip}%
\pgfsetbuttcap%
\pgfsetroundjoin%
\pgfsetlinewidth{2.007500pt}%
\definecolor{currentstroke}{rgb}{0.121569,0.466667,0.705882}%
\pgfsetstrokecolor{currentstroke}%
\pgfsetdash{}{0pt}%
\pgfpathmoveto{\pgfqpoint{4.420041in}{0.435000in}}%
\pgfpathlineto{\pgfqpoint{4.420041in}{2.432067in}}%
\pgfusepath{stroke}%
\end{pgfscope}%
\begin{pgfscope}%
\pgfpathrectangle{\pgfqpoint{3.286364in}{0.330000in}}{\pgfqpoint{2.113636in}{2.310000in}}%
\pgfusepath{clip}%
\pgfsetbuttcap%
\pgfsetroundjoin%
\pgfsetlinewidth{2.007500pt}%
\definecolor{currentstroke}{rgb}{0.121569,0.466667,0.705882}%
\pgfsetstrokecolor{currentstroke}%
\pgfsetdash{}{0pt}%
\pgfpathmoveto{\pgfqpoint{4.439256in}{0.435000in}}%
\pgfpathlineto{\pgfqpoint{4.439256in}{2.404787in}}%
\pgfusepath{stroke}%
\end{pgfscope}%
\begin{pgfscope}%
\pgfpathrectangle{\pgfqpoint{3.286364in}{0.330000in}}{\pgfqpoint{2.113636in}{2.310000in}}%
\pgfusepath{clip}%
\pgfsetbuttcap%
\pgfsetroundjoin%
\pgfsetlinewidth{2.007500pt}%
\definecolor{currentstroke}{rgb}{0.121569,0.466667,0.705882}%
\pgfsetstrokecolor{currentstroke}%
\pgfsetdash{}{0pt}%
\pgfpathmoveto{\pgfqpoint{4.458471in}{0.435000in}}%
\pgfpathlineto{\pgfqpoint{4.458471in}{2.367794in}}%
\pgfusepath{stroke}%
\end{pgfscope}%
\begin{pgfscope}%
\pgfpathrectangle{\pgfqpoint{3.286364in}{0.330000in}}{\pgfqpoint{2.113636in}{2.310000in}}%
\pgfusepath{clip}%
\pgfsetbuttcap%
\pgfsetroundjoin%
\pgfsetlinewidth{2.007500pt}%
\definecolor{currentstroke}{rgb}{0.121569,0.466667,0.705882}%
\pgfsetstrokecolor{currentstroke}%
\pgfsetdash{}{0pt}%
\pgfpathmoveto{\pgfqpoint{4.477686in}{0.435000in}}%
\pgfpathlineto{\pgfqpoint{4.477686in}{2.355592in}}%
\pgfusepath{stroke}%
\end{pgfscope}%
\begin{pgfscope}%
\pgfpathrectangle{\pgfqpoint{3.286364in}{0.330000in}}{\pgfqpoint{2.113636in}{2.310000in}}%
\pgfusepath{clip}%
\pgfsetbuttcap%
\pgfsetroundjoin%
\pgfsetlinewidth{2.007500pt}%
\definecolor{currentstroke}{rgb}{0.121569,0.466667,0.705882}%
\pgfsetstrokecolor{currentstroke}%
\pgfsetdash{}{0pt}%
\pgfpathmoveto{\pgfqpoint{4.496901in}{0.435000in}}%
\pgfpathlineto{\pgfqpoint{4.496901in}{2.336842in}}%
\pgfusepath{stroke}%
\end{pgfscope}%
\begin{pgfscope}%
\pgfpathrectangle{\pgfqpoint{3.286364in}{0.330000in}}{\pgfqpoint{2.113636in}{2.310000in}}%
\pgfusepath{clip}%
\pgfsetbuttcap%
\pgfsetroundjoin%
\pgfsetlinewidth{2.007500pt}%
\definecolor{currentstroke}{rgb}{0.121569,0.466667,0.705882}%
\pgfsetstrokecolor{currentstroke}%
\pgfsetdash{}{0pt}%
\pgfpathmoveto{\pgfqpoint{4.516116in}{0.435000in}}%
\pgfpathlineto{\pgfqpoint{4.516116in}{2.306180in}}%
\pgfusepath{stroke}%
\end{pgfscope}%
\begin{pgfscope}%
\pgfpathrectangle{\pgfqpoint{3.286364in}{0.330000in}}{\pgfqpoint{2.113636in}{2.310000in}}%
\pgfusepath{clip}%
\pgfsetbuttcap%
\pgfsetroundjoin%
\pgfsetlinewidth{2.007500pt}%
\definecolor{currentstroke}{rgb}{0.121569,0.466667,0.705882}%
\pgfsetstrokecolor{currentstroke}%
\pgfsetdash{}{0pt}%
\pgfpathmoveto{\pgfqpoint{4.535331in}{0.435000in}}%
\pgfpathlineto{\pgfqpoint{4.535331in}{2.274164in}}%
\pgfusepath{stroke}%
\end{pgfscope}%
\begin{pgfscope}%
\pgfpathrectangle{\pgfqpoint{3.286364in}{0.330000in}}{\pgfqpoint{2.113636in}{2.310000in}}%
\pgfusepath{clip}%
\pgfsetbuttcap%
\pgfsetroundjoin%
\pgfsetlinewidth{2.007500pt}%
\definecolor{currentstroke}{rgb}{0.121569,0.466667,0.705882}%
\pgfsetstrokecolor{currentstroke}%
\pgfsetdash{}{0pt}%
\pgfpathmoveto{\pgfqpoint{4.554545in}{0.435000in}}%
\pgfpathlineto{\pgfqpoint{4.554545in}{2.259691in}}%
\pgfusepath{stroke}%
\end{pgfscope}%
\begin{pgfscope}%
\pgfpathrectangle{\pgfqpoint{3.286364in}{0.330000in}}{\pgfqpoint{2.113636in}{2.310000in}}%
\pgfusepath{clip}%
\pgfsetbuttcap%
\pgfsetroundjoin%
\pgfsetlinewidth{2.007500pt}%
\definecolor{currentstroke}{rgb}{0.121569,0.466667,0.705882}%
\pgfsetstrokecolor{currentstroke}%
\pgfsetdash{}{0pt}%
\pgfpathmoveto{\pgfqpoint{4.573760in}{0.435000in}}%
\pgfpathlineto{\pgfqpoint{4.573760in}{2.240530in}}%
\pgfusepath{stroke}%
\end{pgfscope}%
\begin{pgfscope}%
\pgfpathrectangle{\pgfqpoint{3.286364in}{0.330000in}}{\pgfqpoint{2.113636in}{2.310000in}}%
\pgfusepath{clip}%
\pgfsetbuttcap%
\pgfsetroundjoin%
\pgfsetlinewidth{2.007500pt}%
\definecolor{currentstroke}{rgb}{0.121569,0.466667,0.705882}%
\pgfsetstrokecolor{currentstroke}%
\pgfsetdash{}{0pt}%
\pgfpathmoveto{\pgfqpoint{4.592975in}{0.435000in}}%
\pgfpathlineto{\pgfqpoint{4.592975in}{2.227458in}}%
\pgfusepath{stroke}%
\end{pgfscope}%
\begin{pgfscope}%
\pgfpathrectangle{\pgfqpoint{3.286364in}{0.330000in}}{\pgfqpoint{2.113636in}{2.310000in}}%
\pgfusepath{clip}%
\pgfsetbuttcap%
\pgfsetroundjoin%
\pgfsetlinewidth{2.007500pt}%
\definecolor{currentstroke}{rgb}{0.121569,0.466667,0.705882}%
\pgfsetstrokecolor{currentstroke}%
\pgfsetdash{}{0pt}%
\pgfpathmoveto{\pgfqpoint{4.612190in}{0.435000in}}%
\pgfpathlineto{\pgfqpoint{4.612190in}{2.195708in}}%
\pgfusepath{stroke}%
\end{pgfscope}%
\begin{pgfscope}%
\pgfpathrectangle{\pgfqpoint{3.286364in}{0.330000in}}{\pgfqpoint{2.113636in}{2.310000in}}%
\pgfusepath{clip}%
\pgfsetbuttcap%
\pgfsetroundjoin%
\pgfsetlinewidth{2.007500pt}%
\definecolor{currentstroke}{rgb}{0.121569,0.466667,0.705882}%
\pgfsetstrokecolor{currentstroke}%
\pgfsetdash{}{0pt}%
\pgfpathmoveto{\pgfqpoint{4.631405in}{0.435000in}}%
\pgfpathlineto{\pgfqpoint{4.631405in}{2.174058in}}%
\pgfusepath{stroke}%
\end{pgfscope}%
\begin{pgfscope}%
\pgfpathrectangle{\pgfqpoint{3.286364in}{0.330000in}}{\pgfqpoint{2.113636in}{2.310000in}}%
\pgfusepath{clip}%
\pgfsetbuttcap%
\pgfsetroundjoin%
\pgfsetlinewidth{2.007500pt}%
\definecolor{currentstroke}{rgb}{0.121569,0.466667,0.705882}%
\pgfsetstrokecolor{currentstroke}%
\pgfsetdash{}{0pt}%
\pgfpathmoveto{\pgfqpoint{4.650620in}{0.435000in}}%
\pgfpathlineto{\pgfqpoint{4.650620in}{2.162774in}}%
\pgfusepath{stroke}%
\end{pgfscope}%
\begin{pgfscope}%
\pgfpathrectangle{\pgfqpoint{3.286364in}{0.330000in}}{\pgfqpoint{2.113636in}{2.310000in}}%
\pgfusepath{clip}%
\pgfsetbuttcap%
\pgfsetroundjoin%
\pgfsetlinewidth{2.007500pt}%
\definecolor{currentstroke}{rgb}{0.121569,0.466667,0.705882}%
\pgfsetstrokecolor{currentstroke}%
\pgfsetdash{}{0pt}%
\pgfpathmoveto{\pgfqpoint{4.669835in}{0.435000in}}%
\pgfpathlineto{\pgfqpoint{4.669835in}{2.149968in}}%
\pgfusepath{stroke}%
\end{pgfscope}%
\begin{pgfscope}%
\pgfpathrectangle{\pgfqpoint{3.286364in}{0.330000in}}{\pgfqpoint{2.113636in}{2.310000in}}%
\pgfusepath{clip}%
\pgfsetbuttcap%
\pgfsetroundjoin%
\pgfsetlinewidth{2.007500pt}%
\definecolor{currentstroke}{rgb}{0.121569,0.466667,0.705882}%
\pgfsetstrokecolor{currentstroke}%
\pgfsetdash{}{0pt}%
\pgfpathmoveto{\pgfqpoint{4.689050in}{0.435000in}}%
\pgfpathlineto{\pgfqpoint{4.689050in}{2.124332in}}%
\pgfusepath{stroke}%
\end{pgfscope}%
\begin{pgfscope}%
\pgfpathrectangle{\pgfqpoint{3.286364in}{0.330000in}}{\pgfqpoint{2.113636in}{2.310000in}}%
\pgfusepath{clip}%
\pgfsetbuttcap%
\pgfsetroundjoin%
\pgfsetlinewidth{2.007500pt}%
\definecolor{currentstroke}{rgb}{0.121569,0.466667,0.705882}%
\pgfsetstrokecolor{currentstroke}%
\pgfsetdash{}{0pt}%
\pgfpathmoveto{\pgfqpoint{4.708264in}{0.435000in}}%
\pgfpathlineto{\pgfqpoint{4.708264in}{2.098526in}}%
\pgfusepath{stroke}%
\end{pgfscope}%
\begin{pgfscope}%
\pgfpathrectangle{\pgfqpoint{3.286364in}{0.330000in}}{\pgfqpoint{2.113636in}{2.310000in}}%
\pgfusepath{clip}%
\pgfsetbuttcap%
\pgfsetroundjoin%
\pgfsetlinewidth{2.007500pt}%
\definecolor{currentstroke}{rgb}{0.121569,0.466667,0.705882}%
\pgfsetstrokecolor{currentstroke}%
\pgfsetdash{}{0pt}%
\pgfpathmoveto{\pgfqpoint{4.727479in}{0.435000in}}%
\pgfpathlineto{\pgfqpoint{4.727479in}{2.082772in}}%
\pgfusepath{stroke}%
\end{pgfscope}%
\begin{pgfscope}%
\pgfpathrectangle{\pgfqpoint{3.286364in}{0.330000in}}{\pgfqpoint{2.113636in}{2.310000in}}%
\pgfusepath{clip}%
\pgfsetbuttcap%
\pgfsetroundjoin%
\pgfsetlinewidth{2.007500pt}%
\definecolor{currentstroke}{rgb}{0.121569,0.466667,0.705882}%
\pgfsetstrokecolor{currentstroke}%
\pgfsetdash{}{0pt}%
\pgfpathmoveto{\pgfqpoint{4.746694in}{0.435000in}}%
\pgfpathlineto{\pgfqpoint{4.746694in}{2.059382in}}%
\pgfusepath{stroke}%
\end{pgfscope}%
\begin{pgfscope}%
\pgfpathrectangle{\pgfqpoint{3.286364in}{0.330000in}}{\pgfqpoint{2.113636in}{2.310000in}}%
\pgfusepath{clip}%
\pgfsetbuttcap%
\pgfsetroundjoin%
\pgfsetlinewidth{2.007500pt}%
\definecolor{currentstroke}{rgb}{0.121569,0.466667,0.705882}%
\pgfsetstrokecolor{currentstroke}%
\pgfsetdash{}{0pt}%
\pgfpathmoveto{\pgfqpoint{4.765909in}{0.435000in}}%
\pgfpathlineto{\pgfqpoint{4.765909in}{2.044474in}}%
\pgfusepath{stroke}%
\end{pgfscope}%
\begin{pgfscope}%
\pgfpathrectangle{\pgfqpoint{3.286364in}{0.330000in}}{\pgfqpoint{2.113636in}{2.310000in}}%
\pgfusepath{clip}%
\pgfsetbuttcap%
\pgfsetroundjoin%
\pgfsetlinewidth{2.007500pt}%
\definecolor{currentstroke}{rgb}{0.121569,0.466667,0.705882}%
\pgfsetstrokecolor{currentstroke}%
\pgfsetdash{}{0pt}%
\pgfpathmoveto{\pgfqpoint{4.785124in}{0.435000in}}%
\pgfpathlineto{\pgfqpoint{4.785124in}{2.016445in}}%
\pgfusepath{stroke}%
\end{pgfscope}%
\begin{pgfscope}%
\pgfpathrectangle{\pgfqpoint{3.286364in}{0.330000in}}{\pgfqpoint{2.113636in}{2.310000in}}%
\pgfusepath{clip}%
\pgfsetbuttcap%
\pgfsetroundjoin%
\pgfsetlinewidth{2.007500pt}%
\definecolor{currentstroke}{rgb}{0.121569,0.466667,0.705882}%
\pgfsetstrokecolor{currentstroke}%
\pgfsetdash{}{0pt}%
\pgfpathmoveto{\pgfqpoint{4.804339in}{0.435000in}}%
\pgfpathlineto{\pgfqpoint{4.804339in}{1.998130in}}%
\pgfusepath{stroke}%
\end{pgfscope}%
\begin{pgfscope}%
\pgfpathrectangle{\pgfqpoint{3.286364in}{0.330000in}}{\pgfqpoint{2.113636in}{2.310000in}}%
\pgfusepath{clip}%
\pgfsetbuttcap%
\pgfsetroundjoin%
\pgfsetlinewidth{2.007500pt}%
\definecolor{currentstroke}{rgb}{0.121569,0.466667,0.705882}%
\pgfsetstrokecolor{currentstroke}%
\pgfsetdash{}{0pt}%
\pgfpathmoveto{\pgfqpoint{4.823554in}{0.435000in}}%
\pgfpathlineto{\pgfqpoint{4.823554in}{1.983343in}}%
\pgfusepath{stroke}%
\end{pgfscope}%
\begin{pgfscope}%
\pgfpathrectangle{\pgfqpoint{3.286364in}{0.330000in}}{\pgfqpoint{2.113636in}{2.310000in}}%
\pgfusepath{clip}%
\pgfsetbuttcap%
\pgfsetroundjoin%
\pgfsetlinewidth{2.007500pt}%
\definecolor{currentstroke}{rgb}{0.121569,0.466667,0.705882}%
\pgfsetstrokecolor{currentstroke}%
\pgfsetdash{}{0pt}%
\pgfpathmoveto{\pgfqpoint{4.842769in}{0.435000in}}%
\pgfpathlineto{\pgfqpoint{4.842769in}{1.957972in}}%
\pgfusepath{stroke}%
\end{pgfscope}%
\begin{pgfscope}%
\pgfpathrectangle{\pgfqpoint{3.286364in}{0.330000in}}{\pgfqpoint{2.113636in}{2.310000in}}%
\pgfusepath{clip}%
\pgfsetbuttcap%
\pgfsetroundjoin%
\pgfsetlinewidth{2.007500pt}%
\definecolor{currentstroke}{rgb}{0.121569,0.466667,0.705882}%
\pgfsetstrokecolor{currentstroke}%
\pgfsetdash{}{0pt}%
\pgfpathmoveto{\pgfqpoint{4.861983in}{0.435000in}}%
\pgfpathlineto{\pgfqpoint{4.861983in}{1.941082in}}%
\pgfusepath{stroke}%
\end{pgfscope}%
\begin{pgfscope}%
\pgfpathrectangle{\pgfqpoint{3.286364in}{0.330000in}}{\pgfqpoint{2.113636in}{2.310000in}}%
\pgfusepath{clip}%
\pgfsetbuttcap%
\pgfsetroundjoin%
\pgfsetlinewidth{2.007500pt}%
\definecolor{currentstroke}{rgb}{0.121569,0.466667,0.705882}%
\pgfsetstrokecolor{currentstroke}%
\pgfsetdash{}{0pt}%
\pgfpathmoveto{\pgfqpoint{4.881198in}{0.435000in}}%
\pgfpathlineto{\pgfqpoint{4.881198in}{1.916388in}}%
\pgfusepath{stroke}%
\end{pgfscope}%
\begin{pgfscope}%
\pgfpathrectangle{\pgfqpoint{3.286364in}{0.330000in}}{\pgfqpoint{2.113636in}{2.310000in}}%
\pgfusepath{clip}%
\pgfsetbuttcap%
\pgfsetroundjoin%
\pgfsetlinewidth{2.007500pt}%
\definecolor{currentstroke}{rgb}{0.121569,0.466667,0.705882}%
\pgfsetstrokecolor{currentstroke}%
\pgfsetdash{}{0pt}%
\pgfpathmoveto{\pgfqpoint{4.900413in}{0.435000in}}%
\pgfpathlineto{\pgfqpoint{4.900413in}{1.897396in}}%
\pgfusepath{stroke}%
\end{pgfscope}%
\begin{pgfscope}%
\pgfpathrectangle{\pgfqpoint{3.286364in}{0.330000in}}{\pgfqpoint{2.113636in}{2.310000in}}%
\pgfusepath{clip}%
\pgfsetbuttcap%
\pgfsetroundjoin%
\pgfsetlinewidth{2.007500pt}%
\definecolor{currentstroke}{rgb}{0.121569,0.466667,0.705882}%
\pgfsetstrokecolor{currentstroke}%
\pgfsetdash{}{0pt}%
\pgfpathmoveto{\pgfqpoint{4.919628in}{0.435000in}}%
\pgfpathlineto{\pgfqpoint{4.919628in}{1.878356in}}%
\pgfusepath{stroke}%
\end{pgfscope}%
\begin{pgfscope}%
\pgfpathrectangle{\pgfqpoint{3.286364in}{0.330000in}}{\pgfqpoint{2.113636in}{2.310000in}}%
\pgfusepath{clip}%
\pgfsetbuttcap%
\pgfsetroundjoin%
\pgfsetlinewidth{2.007500pt}%
\definecolor{currentstroke}{rgb}{0.121569,0.466667,0.705882}%
\pgfsetstrokecolor{currentstroke}%
\pgfsetdash{}{0pt}%
\pgfpathmoveto{\pgfqpoint{4.938843in}{0.435000in}}%
\pgfpathlineto{\pgfqpoint{4.938843in}{1.865622in}}%
\pgfusepath{stroke}%
\end{pgfscope}%
\begin{pgfscope}%
\pgfpathrectangle{\pgfqpoint{3.286364in}{0.330000in}}{\pgfqpoint{2.113636in}{2.310000in}}%
\pgfusepath{clip}%
\pgfsetbuttcap%
\pgfsetroundjoin%
\pgfsetlinewidth{2.007500pt}%
\definecolor{currentstroke}{rgb}{0.121569,0.466667,0.705882}%
\pgfsetstrokecolor{currentstroke}%
\pgfsetdash{}{0pt}%
\pgfpathmoveto{\pgfqpoint{4.958058in}{0.435000in}}%
\pgfpathlineto{\pgfqpoint{4.958058in}{1.841000in}}%
\pgfusepath{stroke}%
\end{pgfscope}%
\begin{pgfscope}%
\pgfpathrectangle{\pgfqpoint{3.286364in}{0.330000in}}{\pgfqpoint{2.113636in}{2.310000in}}%
\pgfusepath{clip}%
\pgfsetbuttcap%
\pgfsetroundjoin%
\pgfsetlinewidth{2.007500pt}%
\definecolor{currentstroke}{rgb}{0.121569,0.466667,0.705882}%
\pgfsetstrokecolor{currentstroke}%
\pgfsetdash{}{0pt}%
\pgfpathmoveto{\pgfqpoint{4.977273in}{0.435000in}}%
\pgfpathlineto{\pgfqpoint{4.977273in}{1.816765in}}%
\pgfusepath{stroke}%
\end{pgfscope}%
\begin{pgfscope}%
\pgfpathrectangle{\pgfqpoint{3.286364in}{0.330000in}}{\pgfqpoint{2.113636in}{2.310000in}}%
\pgfusepath{clip}%
\pgfsetbuttcap%
\pgfsetroundjoin%
\pgfsetlinewidth{2.007500pt}%
\definecolor{currentstroke}{rgb}{0.121569,0.466667,0.705882}%
\pgfsetstrokecolor{currentstroke}%
\pgfsetdash{}{0pt}%
\pgfpathmoveto{\pgfqpoint{4.996488in}{0.435000in}}%
\pgfpathlineto{\pgfqpoint{4.996488in}{1.798909in}}%
\pgfusepath{stroke}%
\end{pgfscope}%
\begin{pgfscope}%
\pgfpathrectangle{\pgfqpoint{3.286364in}{0.330000in}}{\pgfqpoint{2.113636in}{2.310000in}}%
\pgfusepath{clip}%
\pgfsetbuttcap%
\pgfsetroundjoin%
\pgfsetlinewidth{2.007500pt}%
\definecolor{currentstroke}{rgb}{0.121569,0.466667,0.705882}%
\pgfsetstrokecolor{currentstroke}%
\pgfsetdash{}{0pt}%
\pgfpathmoveto{\pgfqpoint{5.015702in}{0.435000in}}%
\pgfpathlineto{\pgfqpoint{5.015702in}{1.775133in}}%
\pgfusepath{stroke}%
\end{pgfscope}%
\begin{pgfscope}%
\pgfpathrectangle{\pgfqpoint{3.286364in}{0.330000in}}{\pgfqpoint{2.113636in}{2.310000in}}%
\pgfusepath{clip}%
\pgfsetbuttcap%
\pgfsetroundjoin%
\pgfsetlinewidth{2.007500pt}%
\definecolor{currentstroke}{rgb}{0.121569,0.466667,0.705882}%
\pgfsetstrokecolor{currentstroke}%
\pgfsetdash{}{0pt}%
\pgfpathmoveto{\pgfqpoint{5.034917in}{0.435000in}}%
\pgfpathlineto{\pgfqpoint{5.034917in}{1.739251in}}%
\pgfusepath{stroke}%
\end{pgfscope}%
\begin{pgfscope}%
\pgfpathrectangle{\pgfqpoint{3.286364in}{0.330000in}}{\pgfqpoint{2.113636in}{2.310000in}}%
\pgfusepath{clip}%
\pgfsetbuttcap%
\pgfsetroundjoin%
\pgfsetlinewidth{2.007500pt}%
\definecolor{currentstroke}{rgb}{0.121569,0.466667,0.705882}%
\pgfsetstrokecolor{currentstroke}%
\pgfsetdash{}{0pt}%
\pgfpathmoveto{\pgfqpoint{5.054132in}{0.435000in}}%
\pgfpathlineto{\pgfqpoint{5.054132in}{1.718013in}}%
\pgfusepath{stroke}%
\end{pgfscope}%
\begin{pgfscope}%
\pgfpathrectangle{\pgfqpoint{3.286364in}{0.330000in}}{\pgfqpoint{2.113636in}{2.310000in}}%
\pgfusepath{clip}%
\pgfsetbuttcap%
\pgfsetroundjoin%
\pgfsetlinewidth{2.007500pt}%
\definecolor{currentstroke}{rgb}{0.121569,0.466667,0.705882}%
\pgfsetstrokecolor{currentstroke}%
\pgfsetdash{}{0pt}%
\pgfpathmoveto{\pgfqpoint{5.073347in}{0.435000in}}%
\pgfpathlineto{\pgfqpoint{5.073347in}{1.690008in}}%
\pgfusepath{stroke}%
\end{pgfscope}%
\begin{pgfscope}%
\pgfpathrectangle{\pgfqpoint{3.286364in}{0.330000in}}{\pgfqpoint{2.113636in}{2.310000in}}%
\pgfusepath{clip}%
\pgfsetbuttcap%
\pgfsetroundjoin%
\pgfsetlinewidth{2.007500pt}%
\definecolor{currentstroke}{rgb}{0.121569,0.466667,0.705882}%
\pgfsetstrokecolor{currentstroke}%
\pgfsetdash{}{0pt}%
\pgfpathmoveto{\pgfqpoint{5.092562in}{0.435000in}}%
\pgfpathlineto{\pgfqpoint{5.092562in}{1.681334in}}%
\pgfusepath{stroke}%
\end{pgfscope}%
\begin{pgfscope}%
\pgfpathrectangle{\pgfqpoint{3.286364in}{0.330000in}}{\pgfqpoint{2.113636in}{2.310000in}}%
\pgfusepath{clip}%
\pgfsetbuttcap%
\pgfsetroundjoin%
\pgfsetlinewidth{2.007500pt}%
\definecolor{currentstroke}{rgb}{0.121569,0.466667,0.705882}%
\pgfsetstrokecolor{currentstroke}%
\pgfsetdash{}{0pt}%
\pgfpathmoveto{\pgfqpoint{5.111777in}{0.435000in}}%
\pgfpathlineto{\pgfqpoint{5.111777in}{1.657751in}}%
\pgfusepath{stroke}%
\end{pgfscope}%
\begin{pgfscope}%
\pgfpathrectangle{\pgfqpoint{3.286364in}{0.330000in}}{\pgfqpoint{2.113636in}{2.310000in}}%
\pgfusepath{clip}%
\pgfsetbuttcap%
\pgfsetroundjoin%
\pgfsetlinewidth{2.007500pt}%
\definecolor{currentstroke}{rgb}{0.121569,0.466667,0.705882}%
\pgfsetstrokecolor{currentstroke}%
\pgfsetdash{}{0pt}%
\pgfpathmoveto{\pgfqpoint{5.130992in}{0.435000in}}%
\pgfpathlineto{\pgfqpoint{5.130992in}{1.628272in}}%
\pgfusepath{stroke}%
\end{pgfscope}%
\begin{pgfscope}%
\pgfpathrectangle{\pgfqpoint{3.286364in}{0.330000in}}{\pgfqpoint{2.113636in}{2.310000in}}%
\pgfusepath{clip}%
\pgfsetbuttcap%
\pgfsetroundjoin%
\pgfsetlinewidth{2.007500pt}%
\definecolor{currentstroke}{rgb}{0.121569,0.466667,0.705882}%
\pgfsetstrokecolor{currentstroke}%
\pgfsetdash{}{0pt}%
\pgfpathmoveto{\pgfqpoint{5.150207in}{0.435000in}}%
\pgfpathlineto{\pgfqpoint{5.150207in}{1.606816in}}%
\pgfusepath{stroke}%
\end{pgfscope}%
\begin{pgfscope}%
\pgfpathrectangle{\pgfqpoint{3.286364in}{0.330000in}}{\pgfqpoint{2.113636in}{2.310000in}}%
\pgfusepath{clip}%
\pgfsetbuttcap%
\pgfsetroundjoin%
\pgfsetlinewidth{2.007500pt}%
\definecolor{currentstroke}{rgb}{0.121569,0.466667,0.705882}%
\pgfsetstrokecolor{currentstroke}%
\pgfsetdash{}{0pt}%
\pgfpathmoveto{\pgfqpoint{5.169421in}{0.435000in}}%
\pgfpathlineto{\pgfqpoint{5.169421in}{1.588863in}}%
\pgfusepath{stroke}%
\end{pgfscope}%
\begin{pgfscope}%
\pgfpathrectangle{\pgfqpoint{3.286364in}{0.330000in}}{\pgfqpoint{2.113636in}{2.310000in}}%
\pgfusepath{clip}%
\pgfsetbuttcap%
\pgfsetroundjoin%
\pgfsetlinewidth{2.007500pt}%
\definecolor{currentstroke}{rgb}{0.121569,0.466667,0.705882}%
\pgfsetstrokecolor{currentstroke}%
\pgfsetdash{}{0pt}%
\pgfpathmoveto{\pgfqpoint{5.188636in}{0.435000in}}%
\pgfpathlineto{\pgfqpoint{5.188636in}{1.578715in}}%
\pgfusepath{stroke}%
\end{pgfscope}%
\begin{pgfscope}%
\pgfpathrectangle{\pgfqpoint{3.286364in}{0.330000in}}{\pgfqpoint{2.113636in}{2.310000in}}%
\pgfusepath{clip}%
\pgfsetbuttcap%
\pgfsetroundjoin%
\pgfsetlinewidth{2.007500pt}%
\definecolor{currentstroke}{rgb}{0.121569,0.466667,0.705882}%
\pgfsetstrokecolor{currentstroke}%
\pgfsetdash{}{0pt}%
\pgfpathmoveto{\pgfqpoint{5.207851in}{0.435000in}}%
\pgfpathlineto{\pgfqpoint{5.207851in}{1.554649in}}%
\pgfusepath{stroke}%
\end{pgfscope}%
\begin{pgfscope}%
\pgfpathrectangle{\pgfqpoint{3.286364in}{0.330000in}}{\pgfqpoint{2.113636in}{2.310000in}}%
\pgfusepath{clip}%
\pgfsetbuttcap%
\pgfsetroundjoin%
\pgfsetlinewidth{2.007500pt}%
\definecolor{currentstroke}{rgb}{0.121569,0.466667,0.705882}%
\pgfsetstrokecolor{currentstroke}%
\pgfsetdash{}{0pt}%
\pgfpathmoveto{\pgfqpoint{5.227066in}{0.435000in}}%
\pgfpathlineto{\pgfqpoint{5.227066in}{1.536116in}}%
\pgfusepath{stroke}%
\end{pgfscope}%
\begin{pgfscope}%
\pgfpathrectangle{\pgfqpoint{3.286364in}{0.330000in}}{\pgfqpoint{2.113636in}{2.310000in}}%
\pgfusepath{clip}%
\pgfsetbuttcap%
\pgfsetroundjoin%
\pgfsetlinewidth{2.007500pt}%
\definecolor{currentstroke}{rgb}{0.121569,0.466667,0.705882}%
\pgfsetstrokecolor{currentstroke}%
\pgfsetdash{}{0pt}%
\pgfpathmoveto{\pgfqpoint{5.246281in}{0.435000in}}%
\pgfpathlineto{\pgfqpoint{5.246281in}{1.517028in}}%
\pgfusepath{stroke}%
\end{pgfscope}%
\begin{pgfscope}%
\pgfpathrectangle{\pgfqpoint{3.286364in}{0.330000in}}{\pgfqpoint{2.113636in}{2.310000in}}%
\pgfusepath{clip}%
\pgfsetbuttcap%
\pgfsetroundjoin%
\pgfsetlinewidth{2.007500pt}%
\definecolor{currentstroke}{rgb}{0.121569,0.466667,0.705882}%
\pgfsetstrokecolor{currentstroke}%
\pgfsetdash{}{0pt}%
\pgfpathmoveto{\pgfqpoint{5.265496in}{0.435000in}}%
\pgfpathlineto{\pgfqpoint{5.265496in}{1.495088in}}%
\pgfusepath{stroke}%
\end{pgfscope}%
\begin{pgfscope}%
\pgfpathrectangle{\pgfqpoint{3.286364in}{0.330000in}}{\pgfqpoint{2.113636in}{2.310000in}}%
\pgfusepath{clip}%
\pgfsetbuttcap%
\pgfsetroundjoin%
\pgfsetlinewidth{2.007500pt}%
\definecolor{currentstroke}{rgb}{0.121569,0.466667,0.705882}%
\pgfsetstrokecolor{currentstroke}%
\pgfsetdash{}{0pt}%
\pgfpathmoveto{\pgfqpoint{5.284711in}{0.435000in}}%
\pgfpathlineto{\pgfqpoint{5.284711in}{1.475613in}}%
\pgfusepath{stroke}%
\end{pgfscope}%
\begin{pgfscope}%
\pgfpathrectangle{\pgfqpoint{3.286364in}{0.330000in}}{\pgfqpoint{2.113636in}{2.310000in}}%
\pgfusepath{clip}%
\pgfsetbuttcap%
\pgfsetroundjoin%
\pgfsetlinewidth{2.007500pt}%
\definecolor{currentstroke}{rgb}{0.121569,0.466667,0.705882}%
\pgfsetstrokecolor{currentstroke}%
\pgfsetdash{}{0pt}%
\pgfpathmoveto{\pgfqpoint{5.303926in}{0.435000in}}%
\pgfpathlineto{\pgfqpoint{5.303926in}{1.449058in}}%
\pgfusepath{stroke}%
\end{pgfscope}%
\begin{pgfscope}%
\pgfpathrectangle{\pgfqpoint{3.286364in}{0.330000in}}{\pgfqpoint{2.113636in}{2.310000in}}%
\pgfusepath{clip}%
\pgfsetrectcap%
\pgfsetroundjoin%
\pgfsetlinewidth{2.007500pt}%
\definecolor{currentstroke}{rgb}{0.121569,0.466667,0.705882}%
\pgfsetstrokecolor{currentstroke}%
\pgfsetdash{}{0pt}%
\pgfpathmoveto{\pgfqpoint{3.286364in}{0.435000in}}%
\pgfpathlineto{\pgfqpoint{5.400000in}{0.435000in}}%
\pgfusepath{stroke}%
\end{pgfscope}%
\begin{pgfscope}%
\pgfsetrectcap%
\pgfsetmiterjoin%
\pgfsetlinewidth{0.803000pt}%
\definecolor{currentstroke}{rgb}{0.000000,0.000000,0.000000}%
\pgfsetstrokecolor{currentstroke}%
\pgfsetdash{}{0pt}%
\pgfpathmoveto{\pgfqpoint{3.286364in}{0.330000in}}%
\pgfpathlineto{\pgfqpoint{3.286364in}{2.640000in}}%
\pgfusepath{stroke}%
\end{pgfscope}%
\begin{pgfscope}%
\pgfsetrectcap%
\pgfsetmiterjoin%
\pgfsetlinewidth{0.803000pt}%
\definecolor{currentstroke}{rgb}{0.000000,0.000000,0.000000}%
\pgfsetstrokecolor{currentstroke}%
\pgfsetdash{}{0pt}%
\pgfpathmoveto{\pgfqpoint{5.400000in}{0.330000in}}%
\pgfpathlineto{\pgfqpoint{5.400000in}{2.640000in}}%
\pgfusepath{stroke}%
\end{pgfscope}%
\begin{pgfscope}%
\pgfsetrectcap%
\pgfsetmiterjoin%
\pgfsetlinewidth{0.803000pt}%
\definecolor{currentstroke}{rgb}{0.000000,0.000000,0.000000}%
\pgfsetstrokecolor{currentstroke}%
\pgfsetdash{}{0pt}%
\pgfpathmoveto{\pgfqpoint{3.286364in}{0.330000in}}%
\pgfpathlineto{\pgfqpoint{5.400000in}{0.330000in}}%
\pgfusepath{stroke}%
\end{pgfscope}%
\begin{pgfscope}%
\pgfsetrectcap%
\pgfsetmiterjoin%
\pgfsetlinewidth{0.803000pt}%
\definecolor{currentstroke}{rgb}{0.000000,0.000000,0.000000}%
\pgfsetstrokecolor{currentstroke}%
\pgfsetdash{}{0pt}%
\pgfpathmoveto{\pgfqpoint{3.286364in}{2.640000in}}%
\pgfpathlineto{\pgfqpoint{5.400000in}{2.640000in}}%
\pgfusepath{stroke}%
\end{pgfscope}%
\begin{pgfscope}%
\definecolor{textcolor}{rgb}{0.000000,0.000000,0.000000}%
\pgfsetstrokecolor{textcolor}%
\pgfsetfillcolor{textcolor}%
\pgftext[x=4.343182in,y=2.723333in,,base]{\color{textcolor}\rmfamily\fontsize{12.000000}{14.400000}\selectfont Cross-correlation between petal length and width}%
\end{pgfscope}%
\end{pgfpicture}%
\makeatother%
\endgroup%
}
  \end{center}
  \caption{Cross-correlation of iris data.}
  \label{fig:cross-cor}
\end{figure}

In conclusion, \ac{eda} is a critical step in the data analysis process, and involves visualizing and summarizing the data in order to gain insight and make informed decisions about the data. Whether you are using histograms, boxplots, pairplots, correlation, covariance, or cross-correlation, it is important to understand the information that each of these tools can provide, and how they can help you to better understand your data.
