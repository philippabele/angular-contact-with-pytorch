\chapter{EDA on Bearings Dataset}
\label{sec:bearings-eda}


\section{Introduction to EDA Chapter}

In this chapter, we discuss \ac{eda}, an indispensable step in data analysis. \ac{eda} focuses on gaining a comprehensive understanding of the data using descriptive statistics and graphical methods before the application of sophisticated statistical techniques or machine learning algorithms.

The purpose of our EDA is to delve into a dataset that includes measurements related to the performance of shaft bearings. These measurements include \(Fr\) (radial force), \(n\) (rotational speed), and \(Lifetime\) (expected bearing lifetime). Through our \ac{eda}, we aim to understand the distributions of these variables and their interactions.

Such insights derived from \ac{eda} can be valuable for further data processing, hypothesis testing, feature selection, and machine learning model selection. In our specific context, understanding the factors that affect the lifetime of a bearing can help make informed decisions related to their operation and maintenance.

Our EDA process will be systematic and iterative, revising our understanding as new insights emerge. We will walk through an initial data inspection, followed by univariate and multivariate analyses, ultimately leading to feature engineering and hypothesis generation.


\section{Initial Data Inspection}

The first step in our \ac{eda} is the initial inspection of the dataset. This involves loading the data, examining a few records, identifying the type of each variable, checking for missing values, and noting the range of values each variable can take. A selection of data points from the dataset is presented in Table \ref{table:bearings-dataset} for initial observation.

\begin{table}[ht]
  \centering
  \caption{Sample of the Dataset}
  \label{table:bearings-dataset}
  \begin{tabular}{|c|c|c|c|}
  \hline
  Line & Fr & n & Lifetime \\
  \hline
  1 & 200 & 100 & 88,445,568.46 \\
  2 & 200 & 200 & 44,222,784.23 \\
  ... & ... & ... & ... \\
  34 & 200 & 3400 & 2,601,340.249 \\
  35 & 200 & 3500 & 2,527,016.242 \\
  36 & 300 & 100 & 22,893,132.09 \\
  37 & 300 & 200 & 11,446,566.04 \\
  ... & ... & ... & ... \\
  69 & 300 & 3400 & 673,327.4144 \\
  70 & 300 & 3500 & 654,089.4882 \\
  71 & 400 & 100 & 8,774,911.776 \\
  72 & 400 & 200 & 4,387,455.888 \\
  ... & ... & ... & ... \\
  1364 & 4000 & 3400 & 119.7927427 \\
  1365 & 4000 & 3500 & 116.3700929 \\
  \hline
  \end{tabular}
\end{table}

Our dataset is a structured one, with \(Fr\) and \(n\) recorded as integer values and \(Lifetime\) as a floating-point number. \(Fr\) represents the radial force applied on the bearing in Newton (N) and ranges from 200 to 4000 N. \(n\) represents the rotational speed in \ac{rpm} and spans values from 100 to 3500 rpm. \(Lifetime\) is the expected lifetime of the bearing in hours (h) and varies from 88,445,568.46 h to 116.37 h. A significant observation made at this point is the step-wise increment pattern seen in \(Fr\) and \(n\), and the corresponding decrement in \(Lifetime\). Each unique combination of \(Fr\) and \(n\) represents different operational conditions for the bearing, and their influence on the bearing's expected \(Lifetime\) is our primary point of interest.

No missing values are present in the dataset, which allows us to proceed with the analysis without the need for imputation or deletion strategies.


\section{Univariate Analysis}

Univariate analysis provides deeper insight into each feature independently. Its primary goal is to examine the distribution of each feature, detect potential outliers, and determine whether any transformations might be necessary.

As previously noted, the \(Fr\) and \(n\) variables show a structured, uniform distribution due to the step-wise pattern of data recording. This setup ensures a comprehensive coverage of various operational conditions.

The main focus of our univariate analysis is on the \(Lifetime\) variable. A notable observation is that \(Lifetime\) follows an exponential decrease pattern as \(Fr\) and \(n\) increase. This behavior suggests a potential non-linear relationship between \(Lifetime\) and the operational parameters \(Fr\) and \(n\).

To better visualize the distribution of \(Lifetime\), we employed histograms and boxplots (Figure \ref{fig:bearings-histogram} and Figure \ref{fig:bearings-boxplot}). Since \(Lifetime\) features a long tail towards lower values, we transformed the data using a log function. This transformation helped us to bucket the \(Lifetime\) data into more meaningful intervals for visualization, revealing a clearer distribution pattern.

\begin{figure}[ht]
    \centering
    \includegraphics[width=\textwidth]{assets/bearings-eda/histogram.png}
    \caption{Histogram of Lifetime}
    \label{fig:bearings-histogram}
\end{figure}

\begin{figure}[ht]
    \centering
    \includegraphics[width=\textwidth]{assets/bearings-eda/boxplot.png}
    \caption{Boxplot of Lifetime}
    \label{fig:bearings-boxplot}
\end{figure}

While the \(Lifetime\) data features a wide range of values, this is to be expected given the nature of the measured attribute. The variation can be attributed to the range of operating conditions (\(Fr\) and \(n\)) represented in the dataset. In this context, extreme values in \(Lifetime\) are not necessarily outliers in the traditional sense, but rather indicate the broad spectrum of potential lifetimes for the shaft bearings under different conditions.

Overall, the univariate analysis offers valuable insights into each feature's characteristics. It also paves the way for a more nuanced understanding of interactions between features in the subsequent multivariate analysis.


\section{Multivariate Analysis}

In the process of \ac{eda}, multivariate analysis plays a critical role in examining the interrelationships between different features. Simultaneously, the insights gleaned during this process can inform the creation of new features, enhancing our dataset and offering further explanatory power for our hypotheses.

To understand the relationships between the variables \(Fr\), \(n\), and \(Lifetime\), we used various visualization techniques, including 3D plots, pair plots, and correlation matrices.

The 3D plot (Figure \ref{fig:bearings-3dplot}), with axes representing \(Fr\), \(n\), and the logarithm of \(Lifetime\), helped us to visually interpret the interactions between these variables. We used the logarithm of \(Lifetime\) to better visualize the effects, as this transformation does not change the relations but makes it easier to observe the trends. We noticed that the \(Lifetime\) showed a pronounced decrease as either \(Fr\) or \(n\) increased, suggesting an inverse relationship. Moreover, the decrease in \(Lifetime\) was steeper with respect to \(Fr\) compared to \(n\), indicating that \(Fr\) has a more significant impact on \(Lifetime\).

\begin{figure}[h]
    \centering
    \includegraphics[width=0.8\textwidth]{assets/bearings-eda/3dplot-log.png}
    \caption{3D plot of Fr, n, and Lifetime}
    \label{fig:bearings-3dplot}
\end{figure}

In addition to the 3D plot, we utilized a pair plot (Figure \ref{fig:bearings-pairplot}) to analyze pairwise relationships between the variables. Similar to the 3D plot, the pair plot further affirmed the observation that the lifetime decreases more sharply when plotted against \(Fr\) compared to \(n\). This could be attributed to the difference in data range between \(Fr\) and \(n\), but also to the greater impact of \(Fr\) on the lifetime.

\begin{figure}[h]
    \centering
    \includegraphics[width=0.8\textwidth]{assets/bearings-eda/pairplot.png}
    \caption{Pairplot of variables}
    \label{fig:bearings-pairplot}
\end{figure}

Lastly, we examined the correlation matrix (Figure \ref{fig:bearings-corrmat}) to quantify the relationships between the variables. We observed a stronger negative correlation of \(Fr\) with the logarithm of \(Lifetime\) compared to \(n\). This negative correlation implies that as \(Fr\) increases, \(Lifetime\) decreases, and vice versa. Moreover, the stronger correlation with the logarithm of \(Lifetime\) further suggests an exponential relationship between these variables.

\begin{figure}[h]
    \centering
    \includegraphics[width=0.8\textwidth]{assets/bearings-eda/correlation-matrix.png}
    \caption{Correlation matrix of variables}
    \label{fig:bearings-corrmat}
\end{figure}

The insights gained from the multivariate analysis regarding the impact of \(Fr\) and \(n\) on \(Lifetime\) are valuable. They provide a basis for the subsequent steps in model development, and enable a more informed and refined approach to predictive analysis.


\section{Hypothesis Generation}

During our \ac{eda}, we observed an inverse relationship between \(Lifetime\) and both \(Fr\) and \(n\). This relationship suggested an exponential decay form, as depicted by the 3D plot. From this observation, we hypothesized that the relationship could be represented by a function of the form:

\begin{equation}
\label{eq:hypothesis}
Lifetime \approx c \cdot Fr^{-a} \cdot n^{-b}
\end{equation}

where \(a\) and \(b\) represent the relative influence of \(Fr\) and \(n\) on \(Lifetime\). This exponential function indicates a multiplicative interaction between \(Fr\) and \(n\), each with an inverse relationship with \(Lifetime\). The constant \(c\) scales the function to the correct value range.

To further refine our hypothesis, we performed a sensitivity analysis. By investigating the response of the \(Lifetime\) to a change in the \(Fr\) and \(n\) values, we found that when \(Fr\) doubles, the \(Lifetime\) reduces to approximately \(\frac{1}{10.08}\) of the original. Conversely, when \(n\) doubles, the \(Lifetime\) becomes half. 

Using these observations, we can mathematically derive the weights \(a\) and \(b\). If doubling \(Fr\) (replacing \(Fr\) with \(2Fr\)) results in \(Lifetime\) being roughly one-tenth of the original, it implies that \(2^{-a} = \frac{1}{10.08}\). By solving this equation, we find \(a = -\log_2(\frac{1}{10.08}) \approx \frac{10}{3}\). This value represents the weight of \(Fr\) in our model function.

Similarly, if doubling \(n\) (replacing \(n\) with \(2n\)) results in \(Lifetime\) being half of the original, it implies that \(2^{-b} = 0.5\). Solving this equation, we find \(b = -\log_2(0.5) = 1\).

Using these values for \(a\) and \(b\), we then calculated \(c\) by comparing the actual \(Lifetime\) values with \(Fr^{-a} \cdot n^{-b}\) for the data point with the highest \(Lifetime\). This yielded a constant value for \(c\).

\begin{equation}
\label{eq:constant_c}
c \approx \frac{88445568.46}{200^{-10/3} \cdot 100^{-1}} \approx 4.1378625769 \cdot 10^{17}
\end{equation}

Our refined empirical hypothesis for the function form thus became:

\begin{equation}
\label{eq:refined_hypothesis}
Lifetime \approx \frac{4.1378625767 \cdot 10^{17}}{Fr^{10/3} \cdot n}
\end{equation}

Interestingly, there is a striking resemblance between our derived formula and the ISO standard formula for the lifetime of roller bearings under dynamic load in section \ref{sec:lifetime_calculation}. In both formulas, we see that the lifetime is inversely proportional to \(n\) and the load to the power of \(\frac{10}{3}\). In the ISO standard formula, there is also a factor \(C\), which represents the dynamic load rating. This value is not present in the dataset, and it's possible that it has been constant for all samples or already factored into the constant in our empirical formula. The similarity in the structure of these formulas suggests that our dataset may represent the lifetimes of roller bearings under varying conditions of force and rotational speed.

This hypothesis serves as a benchmark and starting point for our model. To assess the fit of our function to the dataset, we calculated the relative error between the \(Lifetime\) and the function's output. The error primarily consists of rounding errors, as our refined formula closely approximates the original dataset.

The relative error was calculated as follows:

\begin{equation}
\text{relative error} = \left(\frac{\text{actual Lifetime}}{\text{predicted Lifetime}}\right) - 1
\end{equation}

This calculation provides a normalized measure of how the predicted Lifetime deviates from the actual Lifetime in the dataset.

The mean of the relative error is \(4.56 \cdot 10^{-12}\), very close to zero, and the standard deviation is \(1.34 \cdot 10^{-10}\), indicating that the error is tightly clustered around the mean. Figure (\ref{fig:bearings-error}) visually supports this, showing the relative error on the z-axis. Despite the plot's limited precision, it's evident that the values represent a very narrow range between \(-4.72 \cdot 10^{-10}\) and \(4.56 \cdot 10^{-10}\).

\begin{figure}[ht]
    \centering
    \includegraphics[width=\textwidth]{assets/bearings-eda/3dplot-error.png}
    \caption{3D plot showing the relative error between the actual \(Lifetime\) and the predicted values from our function in \(10^{-10}\).}
    \label{fig:bearings-error}
\end{figure}

The statistical and visual analysis of the relative error confirms that the approximation of the constant \(c\) in our function is highly precise. The minor variations are primarily attributed to the rounding of the \(Lifetime\) values in the dataset. The extreme precision and predictability of the \(Lifetime\) based on \(Fr\) and \(n\) alone suggest that the dataset may have been artificially generated rather than derived from real-world measurements. In practice, other factors would likely introduce more variability and uncertainty into the \(Lifetime\) of the bearings.

Having established this empirical function and assessed its fit to the data, we now have a solid foundation for further modeling. Subsequent stages of our analysis will build on this foundation, making necessary adjustments and improvements to the model.
