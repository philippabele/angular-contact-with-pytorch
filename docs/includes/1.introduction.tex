\chapter{Introduction}

\section{Background}

Ball bearings are fundamental components in a wide array of mechanical systems, from small household appliances to large industrial machines. They are designed to reduce rotational friction and support radial and axial loads. 

However, in spite of their robustness, ball bearings are susceptible to wear and tear, and ultimately, failure. The failure of a ball bearing can have detrimental effects on the functionality of the entire system in which it is used. It is, therefore, imperative to monitor the state of ball bearings and predict their \ac{rul}.

Traditional methods for monitoring and predicting the lifetime of ball bearings mainly revolve around physical inspection and data from sensors that measure vibrations or temperature. These methods, although useful, can be time-consuming and may not always accurately predict failures.

On the other hand, \ac{ml} offers an alternative approach. Through \ac{ml} algorithms, it is possible to analyze large amounts of data and predict outcomes with high accuracy. Specifically, this document explores the use of PyTorch, a powerful open-source machine learning library, for predicting the lifetime of ball bearings.


\section{Objectives}

This study serves as a foundation for individuals seeking to gain a preliminary understanding of applying machine learning techniques to estimate the lifetime of ball bearings. The objectives of this document are multi-faceted, ranging from understanding the dataset's characteristics to evaluating the performance of the developed model. More specifically, this study aims to:

\begin{itemize}
    \item Analyze the dataset and discern the relationships between its features and the target variable through exploratory data analysis.
    \item Establish a deep learning model using PyTorch, focusing on its development and optimization, to predict the lifetime of ball bearings.
    \item Assess the developed model's performance to gauge the feasibility and effectiveness of using machine learning in predictive maintenance applications.
\end{itemize}

In addressing these objectives, the study provides insights into the potential and limitations of machine learning, particularly deep learning with PyTorch, as a tool for predictive maintenance of mechanical components such as ball bearings.


\section{Scope of the Document}

This study primarily focuses on developing a machine learning model to predict the lifetime of ball bearings using PyTorch. The dataset comprises various features alongside the lifetime of ball bearings. The origin of this dataset is unclear, and it is essential to conduct exploratory data analysis to understand its characteristics, which may also provide insights into its origins. A critical aspect of this study is analyzing the relationships between features and understanding how the model learns these relationships for prediction.

While Chapter 2 of this study offers a brief overview of machine learning, PyTorch, and ball bearings, it does not delve into the intricate details of these topics. The purpose of providing this overview is to ensure that readers have enough background information to understand the subsequent content. The document proceeds systematically through several stages of model development:
\begin{itemize}
    \item Chapter 3 conducts an exploratory data analysis to understand the dataset's characteristics and the relationships between the features and the lifetime of ball bearings.
    \item Chapter 4 deals with establishing the machine learning model, focusing on aspects such as architecture, loss function, and validation.
    \item Chapter 5 covers hyperparameter optimization techniques to enhance the model's performance.
\end{itemize}

The scope of this document is limited to the development and optimization of a machine learning model for predicting ball bearing lifetime. Any findings or possibilities that extend beyond this scope, such as potential applications or new methodologies, will be considered as recommendations for future research and are not explored in detail in this study.
